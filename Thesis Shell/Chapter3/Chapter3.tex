\chapter{Taking waterfalls}\label{chap:4}

\section{Introduction}\label{sec:intro4}

In this chapter, I address expropration of waterfalls, starting from a brief overview on the legislation relating to expropriation generally. I then go on to give a chronological presentation of expropration for hydropower development. The story begins in the late 19th century, when the first statutory authorities for such expropration began to emerge. Initially, these authorities were very narrow, however, and they did not cover expropriation of waterfalls, only additional land and rights that waterfall owners might need to develop their resource. 

Later, in the early 20th century, the public began to expropriate waterfalls for hydropower, but this was only authorised on narrowly defined conditions, to enable the state to provide a new public service: electricity supply. Private companies could not expropriate  waterfalls unless they were already majority owners in the local area, a situation that did not change until the passage of the \cite{wra00}. 

Even though expropriation for hydropower had clear public characterisitcs, it was increasingly often met with resistance by local people and environmental groups, particularly as the state begun to pursue large-scale projects after WW2. I discuss the case law that develop in this period, particularly in relation to procedural rules. I conclude that the public nature of hydropower development led the courts to defer very broadly to the discretion of the executive and the legislature, also in relation to the content and scope of provisions in administrative law. 

I then go on to present expropriation of waterfalls under the present wide authority which emerged alongside the market-based reform of the early 1990s. I track the history of this new authority for expropraition, which for the first time also opens up for private and commercial takings of watefalls for hydropower. I note that this change in the law was not given much attention by the executive, and that the legislature did not address it at all. In fact, the legislation did not even explicitly make clear that private expropriation would result, it merely set up a framework that empowered the executive to decide, using directives, the class of legal persons that could be given a license to expropriate waterfalls. Such a directive was subsequently issued, leading to the present situation when waterfalls may be taken for profit, by any commercial company. 

I conclude the chapter by an investigation of how the procedural rules play out in the context of for-profit takings. I anchor the presentaiton in the recent case of {\it Måland}, which upheld the principles of {\it Alta} in the context of a for-profit taking that would 
deprive local owners of waterfalls that they wished to develop commercially in their own small-scale hydropower project. The result, I argue, is that owner's standing under administrative law is extraordinarily weak, particularly compared to the magnitude of their interests. Moreover, I argue that the case of {\it Måland} illustrated how the Supmpre Court adheres to a narrow perspective on the meaning of property protection, as an issue that begins and ends with the question of what the owners should be awarded in compensation for their rights. This, I argue, fails to do justice to the issue at stake, which, more than anything else, concerns the democratic legitimacy of the Norwegian regulatory system for managing water resources. As this system has been depoliticised, expert-based and market-oriented, the continued annihilation of local property rights threatens to render it entirely subservient to the interests of the most powerful market actors. 

%
%
%
%
%g
%
%In this chapter, we address some recent demands made by local people, to the effect that they should be allowed to partake more actively in decision-making processes regarding local waterfalls. We focus on the legal status of such demands under Norwegian law, and we do so by considering the recent Supreme Court Case of \emph{Ola Måland and others v. Jørpeland Kraft AS}.\footnote{Ola M{\aa}land and others v. J{\o}rpeland Kraft AS, Rt 2011 s. 1393. I mention that I represented the local owners in this case, as a trainee lawyer in the district and regional courts, and as the responsible lawyer before the Supreme Court.} In this case, the local owners protested the legality of a license that granted the developer, Jørpeland Kraft AS, a right to divert water away from their waterfalls, thereby reducing the potential for local hydro-power. The owners argued that consultation had been insufficient, and they also contended that the assessment of the case made by the water authorities had been inadequate, and that the decision has been based on an erroneous account of the facts. They won the case in the district Court, Stavanger Tingrett, but lost under appeal to the regional Court, Gulating Lagmannsrett. The Norwegian Supreme Court also found in favor of the developer, and the argument they gave to support this conclusion goes far in suggesting that while the commercial interests of local owners need to be compensated, the presence of such interests do not entitle local communities to a greater say in decision-making processes. Importantly, the Court held that the presence of local interests does not necessitate the adoption of different administrative practices, and that the procedures developed during the decades of direct, non-commercial, administration of the energy sector, could still be followed.
%
%The case is significant, since the traditional hydro-electric scheme in Norway typically involves expropriation, often interfering with the property rights of hundreds of local individuals. Traditionally, owners of waterfalls would be compensated according to a standardized mathematical method that was based on the assumption that they had no interest in hydro-power themselves.\footnote{The method consists in calculating the number of \emph{natural horsepowers} in the waterfall, and then multiplying this number by a price pr. natural horsepower, determined by the discretion of the Court, but in practice based almost solely on what has been awarded in previous cases. For a description of the traditional method, we point to \cite{falk} Chapter 7, page 521-522 (in Norwegian).} In a landmark case from 2008, however, the Norwegian Supreme Court commented, in an \emph{obiter dicta}, that the traditional method for calculating compensation for waterfalls was no longer appropriate, at least not in cases when it can be demonstrated that the original owners could have exploited the resource themselves, if the expropriation had not taken place.\footnote{The case of Agder Energi Produksjon AS vs. Magne Møllen, Rt. 2008 s. 81. The local owner lost the case, the reason being that the Supreme Court held that compensation should not be based on the present day value of the waterfall, but the value it had when the original transferral of rights took place, in the 1960's. The \emph{obiter dicta} has been used as an authority for subsequent Supreme Court decisions, however, see, for instance, Rt. 2010 s. 1056 and Rt. 2011 s. 1683. It has received quite a lot of scholarly attention as well, see \cite{Tf1,Tf2,Tf3}.} This decision has had a profound impact on the level of compensation awarded for waterfall rights, leading to payments that can will typically be ten to a hundred times higher than that which would have been awarded according to the traditional method.\footnote{So far, in cases that have come before the court, there has been about a twenty-fold increase in compensation, see \cite{Tf1}, but the new method will, when applied to certain kinds of projects (cheap to build, and involving little or not regulation of the water-flow), result in compensation having to be paid that amounts to at least a hundred times more than what could be expected if the traditional method had been applied.} More generally, it also served to shift the balance of power in favor of local owners and their communities, who increasingly expect to have their voices heard, and to get a more direct say over how their energy resources are managed and exploited.
%
%After \emph{Måland}, however, it has become unclear to what extent the presence of local, commercial interests will continue to influence the Norwegian energy sector, and if we will see more active participation by local people in the future. In fact, recent statements made by the Norwegian water authorities seem to suggest that this is becoming increasingly unlikely, as a shift in policy seems to have taken place, whereby local, small scale projects, are now to be given lower priority than large scale projects undertaken by established energy companies.\footnote{These statements were not linked to \emph{Måland}, but were made in a more general context, ostensibly motivated by the desire to increase the efficiency of the administrative process, see http://www.nve.no/no/Konsesjoner/Vannkraft/Smaakraft/ where the new policy was announced. It also received some attention from the press, see, for instance, http://www.tu.no/energi/2012/01/18/nve-varsler-flere-smakraft-avslag.}
%
%Taking a broader view on Norwegian law, we believe that recent experiences regarding hydro-power provides an interesting case to study, and one that will shed light on how property rights function in a social, economic and political context. It seems, in particular, that the view of property rights to waterfalls adopted by the Supreme Court rests on a narrow interpretation, seeing such rights merely as bestowing financial interests on certain individuals. This was clearly felt in \emph{Måland}, and we think the case also serves to illuminate certain consequences of such a view, suggesting, in particular, that it can have detrimental social and political consequences, and can very easily lead to perceived injustices. We also think it is pertinent to ask if the narrow view of property which seems to have been adopted for waterfalls in Norway is adequate with respect to human rights law, or if the right to property should also be considered a right to participate, and a right to be heard, in decision-making processes.
%
%In the following, we first give the reader some further background on Norwegian hydro-power, and then 
%we present \emph{Måland} in some detail, focusing on giving the reader an impression of current administrative practices, by describing how they played out in this particular case, and by detailing how they came to result in a decision that the original owners felt to be fundamentally unjust. We also address the legal arguments given by the opposing sides and the arguments relied upon by the national courts. We conclude by presenting some overreaching issues that we believe the case raises, regarding both the social context of property rights, the content of property as human right, and the question of whether or not the protection awarded under Norwegian law currently meets the standard set by the European Convention of Human Rights, as interpreted by the Court in Strasbourg. 

\section{Norwegian expropriation law: A brief overview}

As mentioned in Chapter \ref{chap:2}, Section \ref{sec:...}, the right to property is enthrenched in section 105 of the Norwegian Constitution. There, it is made clear that when property is taken for public use, full compensation is to be paid to the owner. As I have already mentioned, the public use requirement is understood very widely, or not regarded as a requirement at all. However, it is a rule of unwritten constitutional law that exporpriation, as it affects the rights of individuals, can only be carried out when it is positively authorised by the law. The executive does not have a general right to deprive private owners of property, but must rey on rules, usually provided for in acts of parliament, that authorises compulsory acquisition of property on spcific terms.  

Historically, there was no general act relating to expropriation, and a range of different acts provided the necessary authority to exporpriate for specific purposes, such as roads, public buildings, and hydropower projects. Today, many of these authorities are found in the \cite{ea59}.\footnote{Act no 3 of 23 October 1959 Relating to Expropriation of Real Property.} Still, many specific authorities remain, such as the aforementioned section 16 of the \cite{wra17}, which gives an automatic right to expropriate to the holder of a watercourse regulation license.

However, following the \cite{wra00}, the general authority used to exporpriate waterfalls has been included in the \cite{ea59}, in section 2. This section gives general conditions for when expropriation is authorised, and item number 51 states that ``hydropower production'' is a legitimate purpose. In addition, it is made clear that the authorising authority is the King in Council. However, the Act also makes clear that this authority can be delegated further, to ministries, or, in special circumstances, also to other state bodies that the King in Council may instruct.\footnote{See \cite[5]{ea59}.} 

Furthermore, section 2 makes clear that compensation is to be paid as determined by an appraisement court.\footnote{For further details of this procedure, I refer to Chapter \ref{chap:5})}. Lastly, the section states that a license to expropriate should only be granted when it is determined that the benefit of doing so ``undoubtedly'' outweighs the harm.\footnote{See \cite[2]{ea59}.} Notice that this sets expropriation licenses apart from the various hydropower licenses discussed in Sections \ref{wra00}-\ref{ea90} of Chapter \ref{chap:3}. In relation to the latter, in particular, the benefit is required to outweigh the harm, but it need not be ascertained that this is so ``undoubtedly''.

In section 2, no mention is made of {\it who} may legitimately expropriate private property. However, in section 3, it is made clear that unless the Kind in Council decides otherwise, only state or municipality bodies may be granted this power. This is formulated as a limiting principle, but in effect it serves as a general authorisation for the executive to decide, without parliamentary involvement, what class of legal persons may be granted expropriation licenses pursuant to section 2. For many of the purposes mentioned there, directives have been issued that extend the class of possible beneficiaries to any legal person, including companies operating for profit. Such a directive has been issued, in particular, for the authority to expropriate in favour of hydropower production.\footnote{See Directive no 391 of 06 April 2001.}

In addition to providing a general authority for expropriation, the \cite{ea59} also contains several procedural rules. These are collected in Chapter 3 of the Act. Section 11 gives minimal requirements for what an application for an expropriation license must include, stating that it should make clear who will be affected, how the property is to be used, and what the purpose of acquisition is. In addition, the section requires the applicant to specify what land will actually be acquired, and to include information about the type of land in question and the current use that is made of it.

An obligation to give notice to affected owners follows from the second paragraph of section 12. The starting point is that every owner is to be given individual notice, although this obligation falls away when it is ` unreasonable difficult'' to fulfill. In this is found to be the case, it is sufficient that the documents of the case are made available at a suitable place in the local area. A public announcement must also be made and included in ..., as well as in two widely read local newspapers.

The first paragraph of section 12 sets out a general obligation on part of the licensing authority to ensure that the facts of the case are clarified to the ``greatest extent possible''.\footnote{The Norwegian expression is ``best råd er'', which literally means ``best possible way''.} This formulation seems to establish a principle that is stronger than the general duty to duly assess cases involving rights and responsibilities of individuals, a duty that follows from general administrative law. However, the exact meaning of the phrase  ``greatest possible extent'' can be hard to pin down. In fact, administrative practice from several fields, including the hydropower sector, suggests that special attention is rarely devoted to the expropriation issue, particularly not when the expropriation license is granted to implement a state-sanctioned plan for the use of the land in question. I return to this issue in more depth in Section \ref{sec:paa67} when I discuss expropriation in light of administrative law, and in Section \ref{sec:ola} when I relate the discussion specifically to the case of waterfall expropriation.

The last rule of section 12, expressed in the third paragraph, states that a decision to grant an expropriation license must be accompanied by reasons that are submitted to the parties, in accordance with general administrative law. This rule is largely superfluous, as the obligation to give reasons would in most cases also follow independently from administrative law, c.f., Section \ref{sec:paa67}.

According to section 15, the costs incurred by owners in relation to a pending application for expropriation against them is to be covered by the applicant. The exact formulation is that the applicant is obliged to cover the costs that ``the rules in this chapter carry with them''. That is, the applicant is obliged to cover the costs that are related to the owners' rights pursuant to Chapter 3 of the \cite{ea59}. What this actually means is unclear, and in practice an applicant will be denied costs if the competent authority takes the view that they are unreasonable or disproportionate to his interests in the case.\footnote{If the case progresses to an appraisment dispute, the competent authority to decide on costs is the appraisement court. Otherwise, the decision is left with the executive.}

Particularly problematic are cases for which there is no clear division between those aspects of the case that relate to expropriation and those that relate to other licenses or land use planning more generally. This is the situation, for instance, in relation to hydropower development. In such cases, it is unusual for local owners to get any significant coverage of costs relating to the application processing. Legal expenses, for instance, are rarely covered unless they are incurred in relation to a subsequent appraisement dispute. This can be a problem for owners that wish to resist expropriation. Obviously, it is crucial for them to voice convincing objections already at the application processing stage.

In addition to the procedural rules in the \cite{ea59}, many rules of administrative law apply in expropriation cases. In the next section, I give an overview of the most important ones.

\subsection{The Public Administration Act}\label{sec:paa67}

After WW2, public administration in Norway underwent a reform whereby administrative bodies came to be placed more directly under centralized political control. At the same time, the established system based on legal expertise and strict adherence to the letter of the law was replaced by a form of management that actively sought to pursue political goals. As a result, the ambit of administrative decision-making power widened significantly. Many new administrative bodies were set up, and many were empowered greatly by statutory authorities that only specified their purpose and competence in broad strokes. The new style of legislation that developed often left great room for the exercise of administrative discretion, which, the argument went, was reasonable all the while the level of direct control exercised by the central government had been increased.\footnote{See generally \cite{grønli..}.}

However, these reform processes eventually led to concerns about the lack of formal safeguards for those individuals and groups of people that were targeted by administrative decisions. such safeguards would have been largely superfluous so long as the competence of administrative bodies was narrowly drawn up and expressly limited by the authorising statute. But as administrative bodies became increasingly sharpened as instruments of political decision-making, critical voices began to warn against the dangers of an unrestrained public administration free to implement political decisions without much scrutiny and public debate. Particularly worrying was the fact that administrative bodies were often empowered to implement such decisions directly against specific individuals, without having to bring out their general consequences, or justify them as matters of general policy.

In response to these worries, a general statute was proposed that would set out some minimum standards of due process for administrative decision-making process. This proposal eventually became the \cite{paa67}.\footnote{Act no 86 of 10 February 1967 Relating to Procedure in Cases Concerning the Public Administration.} This Act sets out the fundamental procedural principles that the executive is meant to follow when preparing to make administrative decisions. Some rules apply to any such decision, but a particularly important class of rules apply specifically to so-called {\it individual decisions}, which affect the rights and responsibilities of one or more specific persons.\footcite[2]{paa67} These persons are then referred to as parties to the decision. Clearly, owners of property covered by an expropriation license fall into this category, so that owners are parties to the individual decision to grant such a license.

Many of the rules in the \cite{paa67} mirror those of the \cite{ea59}, although they tend to include more detailed, albeit less strict, formulations. Section 16 stipulates that advance notice is required to all those affected by an individual decision. As was the case for expropriation, a possible exception is granted if it is practically unfeasible to reach the parties. However, section 16 also specifies in more depth what the notice must contain. In the second paragraph, it stated that ``the advance notification shall explain the nature of the case, and otherwise contain such information as is considered necessary to enable the party to protect his interests in a proper manner''. Hence, it is not enough simply to inform the party, the Act also explicitly stipulates that the notice has to meet a minimum quality standard. In relation to expropriation of waterfalls this takes on special significance, since, as I discussed in Chapter \ref{chap:3}, it is established practice in such cases fro the exporpriating party to send out this notice, with no involvement of the water authorities. 

In section 17, the duty to clarify cases is expressed, mirroring the rules in section 12 of the \cite{ea59}. The formulation is similarly imprecise, as it declare that cases are to be ``clarified as thoroughly as possible'' before a decision is made. Importantly, however, section 17 also includes specific rules that oblige the authorities to inform parties about information they retrieve during their assessment of the case, and to submit such information for comments in so far as the party must be assumed to have an interest in it.\footnote{See the 2 and 3 paragraphs of \cite[17]{paa67}.}

In section 24, the duty to give grounds for the decision is expressed. It applies to most individual decision, with some narrowly defined exceptions concerning cases when no party can be assumed to be dissatisfied, or when giving grounds would involve disclosing information to which the party is not entitled. Moreover, the King is authorised to limit the duty to give grounds when ``special circumstances so require''. All these exceptions are unusual, and hardly ever apply to hydropower cases.

In section 25, requirements are given concerning the content of grounds given for decisions. It is stipulated that the grounds should mention the relevant rules authorising the decision, the factual assessment the underlies it, as well as the main considerations that have been decisive for the use of discretionary power. 

In some cases, the complication of the matter at hand may be such that some parties are ill-equipped to look after their interests, even if the safeguards mentioned above are respected. This can be the case, for instance, in hydropower cases, as many waterfall owners must be expected to not possess the technical, commercial, and legal knowledge necessary to realize the meaning and value of their ownership. In section 11, a more recent amendment of the \cite{paa67}, a general rule of guidance is given, stipulating that the administration is obliged to provide guidance to parties so that they may look after their interests in the ``best possible way''. Again, the formulation is vague, and it is explicitly stated that the level of guidance must be adapted to the circumstances and the capacity that the agency has for offering such assistance. However, in the second paragraph it is stated that the agency must assess, on their own motion, the parties' need for guidance.

As demonstrated in this and the previous section, expropriation law and general administrative law imposes a range of procedural rules that must be followed when deciding on an application for a license to expropriate. In principle these apply also when waterfalls are expropriated, but as there are special rules that regulate the procedure followed in hydropower cases, the question becomes how these rules relate to each other. Also, the practical question is to what extent the water authorities interpret these rules in concrete cases, and whether they actually observe the more general rules regarding expropriation alongside the rules that target the licensing applications under water law. I return to this issue by giving an in-depth study of a concrete dispute in Section \ref{sec:ola}. First, I elaborate a little on the expropriation rules found in the law relating to hydropower, and its relationship, at the theoretical level, with the rules discussed above.

\section{Special rules for waterfalls}\label{sec:special}

As I mentioned in Chapter \ref{chap:3}, Section \ref{sec:wra17}, section 16 of the \cite{wra17} established an automatic right to expropriate rights needed to undertake a watercourse regulation. This is not understood to include a right to expropriate waterfalls needed for the hydropower development. However, in a recent decision, it was made clear that it does include a right to transfer water away from a river course for development somewhere else. This is a {\it de facto} license to expropriate a waterfall, as the water disappears form the river in which the owners have rights. It is also recognized as such in terms of compensation, which is paid for the waterfalls in cases like this, as they loose their value. However, it is still not considered as expropriation of the waterfalls themselves, but only of the right to deprive them of value.

Hence, section 16 \cite{wra17} entitles the license holder to a form of regulatory taking of the waterfalls of the owners in the river system where the water disappears. The status of such a takings after \cite{måland11} is in effect half-way between regulation and expropriation. The right to compensation is recognized, but the procedural and substantive rules that otherwise apply to expropriation of waterfalls do not apply. In particular, section 16 alone is sufficient authority for this kind of taking. The question arises about the extent to which the rules in the \cite{ea59} applies in such cases. 

First, this questions arises because the rules there are often regarded as expressing general principles of expropriation law. Second, it arises specifically in relation to section 30, number 2, which states that the rules apply to expropriation pursuant to the \cite{wra17}, in so far as they are ``suitable'' and do not ``contradict'' the special rules given in that Act. As we will see in Section \ref{sec:ola}, the established practice is to regard the procedural rules in the \cite{wra17} as exhaustive, and in keeping with the procedural rules in the \cite{ea59}. In addition, the strengthened assessment requirement in section 2, which stipulates that expropriation must``undoubtedly'' be of more benefit than harm, is not considered to have any independent significance alongside the assessment criterion of section 8 in the \cite{wra17}, which does not include any such formulation.

However, there is no doubt that the rules of the \cite{paa67} apply to takings of water rights pursuant to the \cite{wra17}. Moreover, there is no doubt that when a separate expropriation license is sought for waterfalls, these rules, as well as the rules in \cite{ea59} both apply. In practice, they nevertheless play a minimal role when the water authorities assess cases, as the assessment is unified, and the focus remains on balancing environmental and energy interest.

Hence, the broader question is how the practices adopted by the water authorities hold up against the requirements of administrative and expropriation law. This issue was not specifically addressed in \cite{måland11}, but the Supreme Court made some comments that can be taken to imply that they find no fault with current practices. I will shed more light on that they consist in over the course of the following sections. 

First, it is important to note that the reform of the energy sector means that expropriation of waterfalls takes place in a different context today than it did when many of the current practices developed. The value of precedent set during the time of the energy monopoly may be of limited value. In any event, it needs to be understood as a reflection of its time. This means that it is natural to structuring the presentation of expropriation practices chronologically, by dealing first with the period prior to the reform implemented by the \cite{ea90}. I now pursue this approach.

\section{Taking waterfalls for progress}\label{sec:twp}

Historically, Norwegian law admitted no general authority for the state to expropriate waterfalls, neither on its own behalf nor on behalf of private parties. However, there were a range of special provisions that authorized the state to appropriate water for specific purposes, but the criteria were typically quite narrow. For instance, the Water Resources Act 1887 authorized expropriation for the purpose of drinking water, but not for use in industry. Moreover, the purpose of expropriation was largely understood to be binding also on the future use, so that the taker would not gain unrestricted control over the rights he acquired, but were obliged to use them in accordance with the authorised purpose.

While there were no circumstances in which private parties could expropriate waterfalls for industrial development, but private owners of waterfalls could obtain licenses to expropriate surrounding land needed to exploit waterfalls that they already owned. In addition, a new right had been granted through the Water Resources Act 1887, giving the owners of waterfalls a right to engage in various industrial exploits, even if these would damage other landowners, for instance through deprivation of water or flooding. These rules are highly similar to many of the rules found in the so-called mill acts from the US, that I discussed in Chapter \ref{chap:2}, Section \ref{sec:mill}. Some of them could even theoretically have the effect of a {\it de facto} expropriation of waterfalls, but such cases were relatively rare.

An important reason for this was that expropriation law in general was based on the principle that eminent domain should not be exercised when the interests of the expropriating party were of the same kind as the interests of the owner. This applied regardless of whether or not the owner, subjectively speaking, were likely to pursue those interests in an optimal way. The principle was guiding for expropriation law until the early 20th century, and it applied regardless of whether the taker was public or private. It meant, for instance, that expropriation of waterfalls for the purpose of hydropower was ruled out already as a matter of principle. In particular, as the regulatory system of the day made private hydropower development possible, no owner could be deprived of rights to 
a waterfall by any hydropower developer, private or public. 

However, as the industrial advances meant that the interest in hydropower exploded in the late 19th century, the state increasingly came to see it as a political priority to secure that waterfalls were used in the public interest. The most important expression of this came in the form of the licensing acts presented in Chapter \ref{chap:3}, Sections \ref{sec:wra17} and \ref{sec:ica17}. However, during the same time, parliament passed legislation that authorised expropriation of waterfalls to the benefit of public bodies for the purpose of hydropower development.\footnote{Legislation that made it possible to expropriate waterfalls to the benefit of the municipalities was introduced in 1911, and a similar authority that authorised expropriation in favour of the state appeared in 1917, see \cite[29]{amundsen..}}

In 1940, these authorities were consolidated and integrated in the general water resources legislation, through the Water Systems Act 1940 (replaced by the \cite{wra00}). Still, the authority to expropriate waterfalls applied only to the state and to the municipalities, for the latter on the explicit condition that the purpose of expropriation was for ``general electricity supply in the district''. 

Hence, the required public purpose of expropriation was explicitly stipulated in the authority, and expropriation licenses could not be granted to private or commercial entities. Expropriation during this time therefore had a clear public character; in so far as the letter of the law was respected, little doubt could be raised that expropriation was indeed taking place in the public interest. Moreover, the public had to benefit directly, and locally. Economic development in itself was not regarded as a sufficiently public purpose to justify expropriation.

In addition, the fact that the energy sector was organized as a regional monopoly under direct political control meant that it was hard to contend, as a matter of fact, that expropriation was not an expression of the public will. At the same time, however, there were severe political conflicts over hydropower, conflicts that could then be meaningfully addressed within the framework of a politically managed electricity sector. Flaws in this system emerged, however, particularly as the state began to aggressively pursue hydropower development for economic development. 

This still took the form of a public undertaking, but as the scale of development grew massively following WW2, hydropower became increasingly politically sensitive. The democratic legitimacy of development with respect to the local communities that were affected also often seemed very weak. The decision-making authority was completely centralized, the benefit would tend to accrue in urban areas, but the negative effects were almost entirely contained in the local rural communities.

In practice, during this time the limitation to general electricity supply also became less important in practice. In particular, the interpretation of the supply requirement was relaxed significantly over the years, especially following the development of the national electricity grid and the liberalization of the energy sector in the early 1990s. It was no longer obvious, from a technical point of view, when exactly a hydropower development could be said to qualify as making a contribution to the local electricity supply. The electricity was not necessarily used locally, but, indirectly, also the local supply situation might be said to improve.

However, the rule that private parties could not expropriate waterfalls was enforced, and it remained in place until the executive passed the directive mentioned in Section \ref{sec:ea59}, in the year 2000. Only then, some 90 years after the introduction of a general expropriation right for the state and the municipalities, did it become possible for arbitrary commercial entities to acquire waterfalls compulsorily for hydropower.

In light of this, the vast majority of cases dealing with waterfall expropriation under Norwegian law can not be looked at as pure economic development takings. Certainly, the desire for economic development played a crucial part in motivating state and municipality development of hydropower. But their activities in this regard were not themselves commercial in nature. Rather, supplying electricity was regarded as a public service, one that would in turn stimulate commercial activity in other areas of the economy. However, the issue of the extent to which state could legitimately interfere with the rights of waterfall owners still arose. It was often contested, in particular, who the true beneficiaries were, particularly in relation to large-scale developments that would benefit communities far removed from those in which the water resources were found. In addition, particularly in the early 20th century, there was a general feeling of unease about how far the state could go in regulating and monopolizing the hydropower sector without thereby depriving the owners of waterfalls of constitutionally protected rights.

This debate culminated in the conflict surrounding the rule of reversion that was introduced by the licensing acts passed between 1906 and 1917. As mentioned, the rule of reversion meant that in order to sell a waterfall to a private development, the owners and the purchaser had to apply for a license that was only ever granted on the condition that after some number of years, at most 60, the state would acquire the waterfalls without paying compensation. The question that arose was whether this was merely a regulation of the permitted use of waterfalls, or whether it should be regarded as expropriation, so that compensation would be payable pursuant to section 105 of the Constitution.

The conflict over this issue became fierce, with many influential conservatives, including legal scholars, attacking the rule of reversion as a ploy by the state to confiscate Norwegian waterfalls without paying compensation to the owners. However, in a 4-3 decision, the Supreme Court eventually held that section 105 did not apply, since reversion was merely a licensing condition, not an act of expropriation. No owner was compelled to hand over his property to the state, or to sell it to a private party so that the state would eventually acquire it.

One of the judges voting with the majority summed up their view by commenting that he would not regard it as expropriation if the state were to forbid sale of waterfalls to private parties altogether. Why then, he asked, should it be regarded as expropriation if such a sale was allowed to take place only on specific conditions? Against this, the minority argued that the licensing requirement as such was so severe that it had to be regarded as a {\it de facto} expropriation that entitled the owners to compensation. Moreover,  as the purpose was clearly to ensure that waterfalls were eventually brought under state ownership, the minority did not think is was appropriate to consider reversion merely as a regulation of use.

After the decision by the Supreme Court in the reversion case, the legal foundation for the hydropower monopoly solidified. The development of this monopoly happened gradually, however, and expropriation on a large scale did not take place until after WW2. At this time, the state became to involve itself greater in hydropower projects, and it typically pursued very large-scale projects. This caused a new period of controversy, mainly motivated by envirnomental concerns. However, the interest of local people also featured strongly in this debate. Moreover, as the regulatory system was beyond reproach at this point, the local interest were typically aligned with the environmental interests. Large-scale hydropower projects, in particular, would tend to cause nuisance, or even significant loss, to traditional forms of agriculture. Therefore, in a situation when local owners could not themselves benefit significantly from hydropower, their rational response was to oppose it.

The patterns of conflict that emerged during this time converged in the case of \cite{alta8.}. In this case, there was an added complication: The local people all lacked formal title. This was because the development that was being planned would take place in the northernmost part of Norway, in the native land of the Sami people. Norway has a history of discrimination against the Sami, and as their culture is largely nomadic, their land rights were never formalized in the law. As a result, almost the entire northern region of Finnmark is owned by the state. Despite lacking title to the land, the Sami have continued to struggle for their rights to use the land, particularly for their nomadic form of reindeer farming, with an extensive additional reliance on fishing and hunting.

The plans to develop large-scale hydropower in a Sami area therefore raised particularly strong criticism. Particularly significant was the fact that the opposition to the plans brought together environmental groups and groups fighting for aboriginal rights. A broad political mass movement was mobilized in opposition to the plans, eventually resulting in several serious cases of civil disobedience, including what might today well be classified as ``terrorism''.\footnote{In particular, there were several instances when local people blew up equipment that was meant to be used to construct the hydropower plant. In one famous episode, the person behind the bombing miscalculated, resulting in the loss of his own arm. Apart from this, however, the protests were relatively peaceful.}
The case was also dealt with by the court, as the Sami interests claimed, primarily on the basis of administrative law, that the development licenses that had been granted for the development were invalid. 

At first sight, the case is not particularly relevant to the question of expropriation. However, as the Norwegian regulatory system focuses on the development issue, with little or no separate attention paid to the issue of expropriation, the case has in fact been highly significant to the owners of waterfalls. It effectively serves as the primary measuring stick with which the executive and the courts assess the level of substantive and procedural protection that local people are entitled to.

Due to the controversy surrounding the case, it was admitted directly from the District Court to the Supreme Court in plenum. The presiding judge commented that as far as he knew it was the longest and most extensive civil case that the Court had ever heard.\footcite[254]{alta82} In an opinion totaling138 pages, the Court argues that the decision to grant the license is valid. The opinion deals mostly with procedural rules. The substantive arguments, and arguments relating to international law, were not subjected to much scrutiny, as the Court express strong confidence in the opinion that no objection could be raised against the licenses on such grounds. 

However, in addition to arguing against the decision on this basis, the opponents of the development had pointed out a very wide range of purported shortcomings of the decision-making process. This aspect of the case was considered in great depth by the courts, leading also to a further elucidation of the procedural rules of the \cite{wra17} and the application of general administrative law to hydropower cases.

It was clear that the original licensing application did not meet the requirements stipulated in section 5 of the \cite{wra17}. Essentially, the original application contained little more than the technical details about the planned development, with little or no identification or assessment of deleterious effects on other interests, neither private nor public. This shortcoming, moreover, had been acknowledge by the water authorities themselves, who had nevertheless initiated a public hearing, citing an electricity deficit in the northern part of Norway. 

The Supreme Court concluded that this was ``clearly unfortunate''.\footcite[265]{alta82} However, several reports and assessments had subsequently been provided by the water authorities, to fill the gaps left open by the initial application. The Supreme Court held that this might well serve to make the initial mistakes irrelevant to the validity of the licenses, as it was the licensing process as a whole that should be assessed against the procedural rules. Hence, shortcomings of specific stages in the assessment would not be given weight it they did not imbue the process with a dubious character overall.\footcite[265]{alta82}

The question then turned to the question of whether the process as a whole fulfilled procedural requirements. This turned largely on 
the extent to which the various assessments that had been made in the case adequately served to clarify the case, in accordance with section .... of the \cite{wra17} and section 16 of the \cite{paa67}. 

In this regard, the local people objecting the development pointed to a range of negative effects that they believed had not been considered, or had not been considered in enough depth. In relation to nomadic reindeer interests, for instance, it was argued that the water authorities had failed to adequately consider the indirect consequences of development. These effects were described as ``catastrophic'' by an expert testimony presented to the Court. By contrast, the water authorities had based their decision on assessments that did not place much weight on indirect consequences, citing the difficult involved in attempting to quantify such effects. 

After considering the reports and assessments in some depth, the Supreme Court did not find fault with the procedure in this regard. Importantly, the Court stresses that the water authorities were aware of the possibility of indirect negative consequences, but simply chose, as a matter of expert discretion, not to place much weight on such consequences. This, moreover, was construed as an expression of disagreement with those claiming (later) that the effects would be catastrophic. As a result, the grounds for claiming procedural error disappeared, as the lack of attention directed at indirect consequences was held to reflect administrative discretion that could not be made subject to judicial review.

More generally, the Court's opinion on this point reflects how indeterminate the distinction between administrative discretion and procedure can become. Importantly, the Court makes statements of principle in this regard, that serve to limit the scope of judicial review under procedural rules in hydropower cases. In particular, the Court concludes that many of the relevant procedural rules in such cases by their very nature tend to be largely ``discretionary' '. As the licensing decision itself is a discretionary one, the argument goes, it is appropriate to admit to the executive a wide discretionary authority to decide for themselves also how to interpret many of the admittedly rather vague procedural requirements of administrative law. By contrast, the view taken by the appellants, based on the idea that the content and scope of such rules is a purely judicial question, is described by the Court as ``overly formalistic''.

The Court makes a second statement of principle, namely that the scope of assessment required for the purposes of reaching a licensing decision is not in any event as extensive as the level of assessment that is required in a subsequent appraisement dispute. 
Hence, the meaning of the obligation to clarify cases to the best possible extent is put into perspective: Assessments of deleterious effects may be omitted at the executive's discretion even in circumstances when such assessments are practically relevant to the level of compensation payable and {\it will} in fact be provided at a later stage.

More concretely, in relation to the negative effects on fishing, the {\it Alta} Court conceded that the assessments could have been better, but pointed out that the purpose of assessment was only to answer yes or no to development, not to give a detailed presentation of its effects.\footcite[330]{alta82}. Crucially, the Court goes on to note that in so far as mistakes are uncovered as a result of insufficient assessment, this will influence the compensation payments and can also motivate subsequent regulation.\footcite[330]{alta82} In effect, the risk of error is downplayed by making reference to the purely fiscal interest of owners and the regulatory authority of the state. This echoes the dichotomy mentioned in Chapter \ref{chap:3}, whereby there is a tendency in Norwegian law to perceive the interests of affected citizens in purely fiscal terms, while the state is assumed to be the sole protector of social and environmental values attached to property.

In relation to some negative effects of the {\it Alta} development, it was made apparent that they had not been considered at all. In addition, it was clear that erroneous information had been assembled in relation to some issues, particularly regarding alternative ways to meet the need for electricity in Finnmark and Norway as a whole, as well as the extent of this need. The Supreme Court agreed that this was a flaw, but held that it did not imply invalidity of the license. In this regard, a third statement of principle was made. It was stated in particular,  that the duty to consider alternative ways of achieving the public purpose underlying the development license was very limited. 

Some alternatives should be mentioned, but they did not have to be made subject to assessment. This, in turn, was used by the Court to argue that the errors in the information provided about alternative were unlikely to have affected the outcome of the case.\footcite[346]{alta82} This was so despite the fact that alternatives {\it had} in fact been assessed in some depth. Moreover, erroneous information about alternatives had been handed over to the Storting, who had approved the plans on three separate occasions, but always under reference to the precarious electricity situation in Finnmark.

In effect, the Court established a principle whereby the nature of possible alternatives is considered a marginal issue in relation to assessment of licensing applications. Even errors in the data provided about this issue will normally not be given much weight. In {\it Alta}, however, this seems to have been at odds with how the Storting approached the case. There is little doubt, in particular, that the eventual political assessment of the plans depended heavily on the perceived electricity crisis in Finnmark, as well as the electricity supply situation in Norway more generally and the perceived inadequacies of alternative solutions.

In relation to the supply situation in Norway, the state's council in {\it Alta} suggests explicitly that as these aspects were considered relevant mainly at the political stage of the decision-making, they were largely irrelevant to the legal issues that had been raised.\footcite[341]{alta82} But this is hardly reassuring, particularly not as the decision to grant the license was very much a political one. The information gathered by the water authorities, therefore, would be put to use in a largely political decision-making context. In light of this, it seems that the procedural rules were, if anything, {\it more} important to observe in so far as they pertained to the quality of the factual basis that would be used in subsequent political assessments.

The Supreme Court clearly did not approach the matter from this angle, but how exactly it reasoned in this regard is not clear from the opinion. In fact, it is very noticeable how briefly the Court comments on this compared to other aspects of the case. On the one hand, it goes into great detail about purported weaknesses of the licensing procedure that seem relatively minor comparatively speaking. As a result, the Court's assessment that these aspects are not in any event relevant to the outcome of the case do not come as a surprise. On the other hand, in relation to the duty to assess alternatives and the level of necessity, the Court says nothing expect that the duty is very limited. For the details, which demonstrate factual inadequacies in the basis provided to the political decision-makers, the Court only refers briefly to the arguments presented by council for the state. These arguments, based on the contention that the inadequacies were not significant, is accepted with no further discussion.\footcite[346]{alta82}

The dismissive attitude towards the duty to correctly assess alternatives is no doubt a controversial aspect of the {\it Alta}-decision, and it has also later been criticized by legal scholars. Today, the principle becomes particularly problematic. Alternatives are no longer limited to other public projects that can potentially provide the same public service at a smaller environmental and social cost. Instead, the most important alternatives now typically consist in owner-led projects proposed in commercial competition to the applicant's commercial project. In so far as the duty to assess these alternatives is construed as loosely as the duty to assess alternatives was construed in {\it Alta}, it will hardly be reassuring for those owners of waterfalls that oppose commercial development projects based on their own hydropower plans. 

In the next two sections, we will see that so far, no adjustments have been made to the way the water authorities approach this issue. Moreover, the dismissive attitude to this question in {\it Alta} has been upheld in a recent Supreme Court decision involving a concrete owner-led alternative regarding which the NVE had provided manifestly erroneous information to the Ministry.

\section{Taking waterfalls for profit}

As I mentioned in the previous section, private companies could not expropriate waterfalls in Norway prior to the passage of the \cite{wra00}. Moreover, the public purpose requirement was enforced strictly by the authorising statute, particularly in cases when the development was undertaken by municipality companies. I also mentioned how the hydropower sector developed after WW2 from a sector dominated by local municipality companies, to a sector dominated by the state. This, in turn, was accompanied by increased conflicts and doubts regarding the legitimacy of the established licensing procedures, particularly the highly centralized nature of the decision-making process. 

Even so, the debate at this time was still very much anchored in a system that presupposed political management of the hydropower sector as a public service provider. Importantly, the conflicts rarely, if ever, involved significant commercial interests on the part of the local waterfall owners. Many critics voiced arguments to the effect that the fiscal interest of the state motivated wanton destruction of nature and local patterns of land use, including commercial uses. But in financial terms, these interest were typically negligible compared to the scale of the hydropower development. 

As a result, controversies relating to the legitimacy of interference involved only the waterfall rights at their periphery. More focused conflicts involving waterfalls specifically arose in relation to the question of compensation, but the issues typically discussed in this regard were also of relatively minor structural importance, although they could of course be important enough for the individuals directly affected.

In Chapter \ref{chap:3}, I presented the reform of the energy sector of the early 1990s, after which hydropower development has been regarded as a commercial pursuit. Following the regulatory reform, a new general statute dealing with water resources was also proposed, eventually leading to the passage of the \cite{wra00}. This Act provided the first every authority for the state to allow developers to take waterfalls compulsorily for profit. Moreover, it made possible the later executive directive by which waterfalls could be expropriated and handed over to {\it any} legal person, including private companies.

The combination of the legal and regulatory reforms mean that today, takings of waterfalls for hydropower are takings for profit. But this change in the function of expropriation received little attention when these reforms were introduced. When the \cite{wra00} was proposed, the increased scope of expropriation was not singled out for political consideration by the MoPE when it handed the case over to the Storting. In the legislative proposal handed over to the Storting, the new expropriation authority for waterfalls is described merely as a ``simplification'' of older law. 

As the discussion above shows, this is a hardly accurate. However, the commission appointed by the Ministry to prepare the Act also adopted a very low-key approach to expropriation. The commission mentioned that its proposals would imply increased scope for expropriation, but it did  not discussed the desirability of this in any depth. In particular, it did not related its proposals in this regard to the recent market -based reform of the energy sector.  The report from the commission, totaling almost 500 pages, devote only three pages to the proposed ``simplification'' of the expropriation authority.\footcite[235-237]{nou94}

First, the commission notes that a range of different authorities for expropriation co-exist in the law, with many of them positing strict and concrete public interest requirements as a precondition for granting a license. This, the commission argues, is not a very ``pedagogical'' way of providing expropriation authorities. Moreover, the commission notes that it runs the risk of omitting important purposes for which expropriation should be possible. Hence, the commission proposes to replace all older authorities by a sweeping authority that makes expropriation possible for any project that involves ``measures in water courses''.  

The commission comments that their formulation might seem wide, but remark that this is not a problem since the executive can simply deny giving an expropriation license in so far as they regard expropriation as undesirable. The commission does not reflect on the  constitutional consequences of such a perspective, neither in relation to property rights nor in relation to the balance of power between the legislature and the executive. The commission does offer a very brief presentation of the rationale behind dropping the local supply restriction for municipal expropriation, by remarking that these rules complicate the law and might make desirable expropriations impossible.\footcite[235]{nou94}  But the commission do not clarify what kind of desirable expropriations it thinks might be left out. In particular, it does not relate its proposals to the recent market-based reform of the energy sector. Hence, the obvious practical consequences of their proposal, namely that expropriation will be made available as a profit-making mechanism for commercial companies, is not discussed or critically assessed.

The issue of {\it who} should be permitted to benefit from an expropriation license is also dealt with very superficially. In this regard, the commission structures their presentation around the so-called redemption rule that was introduced in \cite{wra40}. Recall that this rule made it possible for a majority owner of a waterfall to compulsorily acquire minority rights, if this was necessary to facilitate hydropower development. Hence, it was a rule that provided only a limited opportunity for private takings, restricted to local owners themselves or external developers that had been able to strike a deal with a locally based majority. 

The main justification given by the commission for introducing a general private takings authority is that the special redemption rule had not been much used in practice. Why this is an argument in favour of extending private expropriation rights is not made clear. Indeed, it seems just as natural to regard it as an argument {\it against} doing so. Why extend the possibility for private expropriation if the demand for such expropriation has been limited in past? Presumably, the commission thought there would be a demand for such expropriation in the future, but this is not stated explicitly, nor is the appropriateness of meeting such a demand discussed. As to the requirement that private takers must already control a majority of the waterfall rights in the local area, the commission only remarks that it regards such a restriction as old-fashioned.\footcite[236]{nou94} No discussion is offered regarding the consequences for local waterfall owners at a time when the energy sector was also being reformed according to market principles.

Since the passage of the \cite{wra00}, it has become clear that the new authority for expropriation has been one of the most practically significant, and controversial, aspects of the Act. During the last 14 years, an unprecedented number of cases has raised the issue of legitimacy of expropriation of waterfalls. Today, practically all cases of expropriation imply that local owners are deprived of the development potential in favour of a commercial actor seeking development of the same kind. According to the law before the \cite{wra00}, expropriation of this kind would not be easy to justify against the relevant authorities. In so far as the beneficiary was a private company, it would not be possible at all. 

In \cite{sauda08}, this issue came into focus, as the waterfall owners protested a license that granted a private company the right to expropriate their waterfalls. Here, the owner's principal argument was that the executive directive granting such rights to private parties was in fact invalid, since it had not been sanctioned by the Storting. Formally speaking, this argument was very weak, since the \cite{ea59} had been amended in such a way that the executive was in fact authorized to determine the class of legal persons that could be granted an expropriation license to pursue hydropower. However, the owners argued that the executive had not appropriately informed the Storting that this would be the consequence of the amendment, which had been passed as a formality following the adoption of the \cite{wra00}. 

The owners pointed to interviews with two of the members of the parliamentary committee that had prepared the case for the Storting, noting that neither of them could recollect that they were even aware that a right to expropriate for private developers would result from the Act they had passed. This, as noted earlier, was not conveyed to them by the executive. Moreover, it was not explicitly stated anywhere in the Act that had been passed. Rather, it followed implicitly from three different sections in two separate Acts that the executive would be empowered to issue such a directive. This, the owners argued, meant that the purported authority was not in fact constitutionally valid.

This argument was rejected, but the level of compensation paid for the waterfall rights was dramatically increased compared to earlier practice, c.f., Chapter \ ref{chap:5}. Because of this, the development company appealed the decision to the Supreme Court, with the owners lodging a counter-appeal regarding the question of legitimacy. The Supreme Court decided not to hear the case, however, as it had recently addressed the compensation question from a similar angle in the paradigmatic case of \cite{møllen08}.

In addition to raising the issue of constitutional legitimacy of the new expropriation authority, the owners in \cite{sauda08} also raised several procedural objections against the expropriation license. This line of argument also proved unsuccessful, but it foreshadows the later case of \cite{måland13}, where the owners were initially successful in arguing that the procedures developed in the takings for progress era were no longer appropriate. This decision was overturned on appeal, however, a decision that was in turn upheld by the Supreme Court, in a decision relying on the precedent set by \cite{alta82}. This case is very well suited to bringing out how administrative practices relating to expropriation function in the context of commercial development projects were the owners have competing plans. It also illustrate common grievances raised by local owners, as well as serving to highlight the response to these grievances by the water authorities and the judiciary.

\section{Ola Måland v Jørpeland Kraft AS}

The expropriating party in \cite{måland13} is Jørpeland Kraft AS, a company jointly owned by Scana Steel Stavanger AS, who owns 1/3 of the shares, and Lyse Kraft AS, who is the majority shareholder holding the remaining shares. The former is a steelworks company located in the small town of Jørpeland in Rogaland county, southwestern Norway. Historically, this company has been a major employer in Jørpeland, which is located by the sea, next to a mountainous area. The main source of energy for the steel industry in Norway has been hydro-power, and Scana Steel Stavanger AS is no exception. The company uses energy harnessed from the rivers in the area, and while the primary river runs through the town of Jørpeland itself, it is supplemented by water from other rivers in the area that are diverted so that they can be exploited more efficiently along with the water from the Jørpeland river.

Recently, Norwegian steel companies have become less profitable, due in great part to increased foreign competition and a significant increase in cost of operation associated with this type of industry in Norway, particularly salary costs.\footnote{For a reference on this, see \emph{Information Booklet about Norwegian Trade and Industry}, published by the Ministry of Trade and Industry in 2005.} This has led to many such companies shifting their attention away from labor-intensive steel production, and focusing instead on producing electricity, selling it directly on the national grid. Jørpeland Kraft AS was established as part of such a move being made with regards to the energy resources in Jørpeland, and the role played by Lyse Kraft AS is an important one. As we mentioned, Norwegian law favors companies where the majority of the shares are held by public bodies, and Lyse Kraft AS, being publicly owned, with the city of Stavanger as the main shareholder, is therefore a valuable partner. Moreover, Lyse Kraft AS, while being a commercial company, is also responsible for the electricity grid in the region. It was established as a merger between several local monopoly companies in the Stavanger region which were reorganized following liberalizaion of the sector in the early 1990's. As discussed in Section \ref{context}, there is little doubt that old monopolists still enjoy considerable power and influence.\footnote{In fact, Lyse Kraft AS is good example suggesting that their power might in some cases have \emph{increased}. Since liberalization, the restraints imposed both by the non-commercial nature of former monopolists, and the local, political, anchoring of such companies, have disappeared.} This is another reason why they can serve as valuable partners for private companies wishing to make a profit from Norwegian hydro-power.

With attention shifting from harnessing rivers for the purpose of industrial production to the purpose of producing electricity to sell on the national grid, the main variables that determines the profitability of the undertaking also changes. On the cost side, what matters becomes only the cost of producing the electricity itself, and this is typically determined, for the most part, by the investments required for the original construction works.\footnote{For an overview of the considerations made when assessing the commercial value of small scale hydro-power, we point to \cite{kartlegging}. In fact, due to the importance that small scale hydro-power has assumed in recent years, investigating models for investing in such projects has become an active field of research in Norway, see for instance \cite{investment}.} Running and maintaining a hydro-power station tends to be comparatively inexpensive. On the income side, what matters is the price of energy on the electricity market, a market that is no longer anchored in the local conditions of supply and demand.

Importantly, as long as energy production is the sole focus, the business no longer depends in any significant way on the local labor force, and as a result, it is typical that large scale exploitation becomes much more profitable, compared to the medium or small scale power plants typically needed to facilitate local industrial exploits. Hence, it was in keeping with a general trend in Norway when Jørpeland Kraft AS, following their shift in commercial strategy, proposed to undertake measures to increase their energy output. This could be achieved relatively cheaply, by further constructions aimed at channeling water from nearby waterfalls into dams that were already built to collect the water from the Jørpeland river.

One relatively small waterfall from which Jørpeland Kraft AS suggested to extract water was owned by Ola Måland and five other local farmers. This waterfall is not located in Jørpeland kommune, and does not reach the sea at Jørpeland, but runs through the neighboring municipality of Hjelmeland, on the other side of a mountain range, until it eventually reaches the sea at Tau, another neighboring municipality. The plans to divert this water would deprive original owners of water along some 15 km of riverbed, all the way from the mountains on the border between Hjelmeland and Jørpeland, to the sea at Tau. Far from all the water would be removed, but the water-flow would be greatly reduced in the upper part of the river known as "Sagåna", the rights to which is held jointly by Ola Måland and five other local farmers from Hjelmeland. 

The water in question stems from the \emph{Brokavatn}, located 646 meters above sea level, where altitude soon drops rapidly so that hydro-power is a particularly well-suited form of exploitation for this water. Plans were already in place for making such use of it, from about the altitude of Brokavatn, to the valley in which the original owners' farms are located, at about 80 meters above sea level. In fact, a rough estimate of the potential was originally made by the NVE, and estimated to yield gross annual production of 7.49 GWh per annum, about five times more than the water from Brokavatn would contribute to the project proposed by Jørpeland Kraft AS. This estimate was not made in relation to the case, but as part of a national project to survey the remaining energy potential in Norwegian rivers.\footnote{The survey was carried out in 2004, and its results are summarized in \cite{kartlegging}.} %\noo{More recent calculations, made by several different experts, acting both on behalf of Jørpeland Kraft AS and original owners, suggests that the water which would be lost would in fact be crucial to the commercial potential of hydro-power for the original owners. Having the water available would take such a project from being somewhat marginal to being a highly profitable endeavor. The owners were not aware of this at the time when the case was being prepared by the water authorities, nor where they informed of this as part of the process.} 

Despite holding the relevant property rights, and despite having considerable commercial interests that would be effected, original owners were not identified as significant stakeholders in the project. Rather, the approach to the case was the traditional one, with focus being directed at the environmental impact, with relevant interests groups being called upon to comment on consequences in this regard, and quite some public debate arising with respect to the balancing of commercial interests and the desire to preserve wildlife and nature.

Nevertheless, one of the owners, Arne Ritland, commented on the proposed project, in an informal letter sent directly to Scana Steel Stavanger AS. In this letter he inquired for further information, and he protested the transferral of water from Brokavatn. He also mentioned the possibility that an alternative hydro-power project could be undertaken by original owners, but he did not go into any details regarding this, stating only that such a locally owned hydro-power plant had previously been in operation in the area. The plant he was referring to dates back to the time before we had a national grid, and was only directed at local supply of electricity. It has since been shut down.

Arne Ritland received a reply stating that more information on the project and its consequences would soon be provided, and he did not pursue the matter further at this time. Meanwhile, Scana Steel Stavanger AS submitted his letter to the water authorities, who in turn presented it to the NVE as a formal comment directed at the application. This prompted Jørpeland Kraft AS to undertake their own survey of alternative hydro-power in Sagåna, and the conclusion, but not the report itself, was sent to the water authorities. The original owners were not informed, and they were not asked to comment on it, or even told that such an investigation of the commercial potential in their waterfalls was being conducted by the expropriating party, as a response to Ritland's letter.

Despite being presented with the issue, the water authorities did not take steps to investigate the commercial potential of local hydro power on their own accord. Moreover, the conclusion presented by Jørpeland Kraft AS did not go into details, but merely stated that if the local owners decided to build two hydro-power plants in Sagåna, then one of them, in the upper part of the river, close to Brokavatn, would not be profitable, neither with nor without the water in question. The other project, on the other hand, in the lower part, could still be carried out profitably even after the transferral. No mention was made as to what the original owners actually stood to loose, nor was there any argument given as to why it made sense to build two separate small-scale power plants in Sagåna. In their final report, the NVE handed these findings over to the Ministry, but did not inform the original owners. 

In addition to the report made by Jørpeland Kraft AS themselves, Hjelmeland kommune, the local municipality government, also commented on the possibility of local hydro-power. In their statement to the NVE, they directed attention to the data in the NVE's own national survey, which suggested that a single hydro-power plant in Sagåna would be a highly profitable undertaking. On this basis, they protested the transferral, arguing that original owners should be given the possibility of undertaking such a project. This statement was not communicated to the original owners, and in their final report it was dismissed by the NVE, who stated that the most energy efficient use of the water would be to transfer it and harness it at Jørpeland.

In addition to the statement made by Ritland, one other property owner, Ola Måland, commented on transferral. He did so without having any knowledge of the commercial potential the water held for him and his co-owners, and without having been informed of the statement made by Hjelmeland Kommune. On this basis, he expressed his support for the transferral, citing that the risk of flooding in Sagåna would be reduced. He also phrased his letter in such a way as to suggest he was speaking on behalf of other owners, but he was the only person to sign it. In the final report to the Ministry, the NVE, in their own conclusion, use this as an argument in favor of transferral, stating that the original owners were in favor of it, and that the opinion of Hjelmeland Kommune should therefore not be given any weight. They neglect to mention Arne Ritland's statement in this regard, and earlier in the report, where his statement is referred to along with many others, Ritland is referred to as a private individual, while Ola Måland is referred to as a property owner, and taken to speak on behalf of the others. The report made by the NVE, while it was not communicated to the affected local owners, it was sent to many other stakeholders, including Hjelmeland Kommune. In light of NVE's conclusions, they changed their original position, informing the Ministry that they would not press any further for local hydro-power, since this was not what the original owners wanted themselves. 

While the case was being prepared by the water authorities, the original owners had begun to consider the potential for hydro-power on their own accord, and in late 2006, when the case reached the Ministry, they where not aware that a decision was imminent. Rather, they were under the impression that they would receive further information before the case went further. Still, as they came to realize the commercial value of the water from Brokavatn in their own project, they approached the NVE, inquiring about the status of the plans proposed by Jørpeland Kraft AS. They were subsequently informed that an opinion in support of transferral had already been offered to the Ministry, and that a final decision would soon be made. This communication took place in late November 2006, summarized in minutes from meetings between local owners, dated 21 and 29 of November. On 15 of December 2006, the King in Council granted a concession for Jørpeland Kraft AS to transfer the water from Brokavatn to Jørpeland.

At this point, it was becoming increasingly clear to the original owners that the water from Brokavatn would be crucial to the commercial potential of their own project, and they also retrieved expert opinions suggesting that the NVE was wrong in concluding that transferral would be the most efficient use of the water. In light of this, they decided to question the legality of the transferral, arguing that the decision was invalid.

The license given to Jørpeland Kraft AS was challenged by the original owners on the grounds that the expropriation was materially unjustified, and that the administrative process leading up to the permission to expropriate did not fulfill procedural requirements. The local court, Stavanger Tingrett, held that the original owners were right in protesting the transfer, with the court emphasizing that the preparatory steps taken in cases such as these needed to provide adequate guarantee that the authorities had also considered the fact that the waterfalls could have been exploited commercially by the original owners themselves.\footnote{Stavanger Tingrett 20.05.2009, case nr. 07-185495SKJ-STAV.}

This view was rejected by the regional court, Gulating Lagmannsrett, which held that sufficient steps had been taken to clarify the commercial interests of the owners, and, moreover, that established practice regarding the preparation and evaluation of such cases -- dating from a time when it was not feasible for original owners to undertake hydro-power schemes -- still provided adequate protection.\footnote{Gulating Lagmannsrett 10.01.2011, case nr. 09-138108ASD-GULA/AVD2.} The Supreme Court also held in favor of Jørpeland Kraft AS, and they went even further in stating that established practice was beyond reproach.

In the following section, we present the main legal arguments relied on by the parties, as well as a summary of how the three national courts approached the case, and how they argued for their respective decisions.

\subsection{The legal arguments, and the view taken by the national courts}\label{view}

The original owners had several arguments in support of their claim that the concession was invalid. Firstly, they argued that procedural mistakes had been made in preparing the case; secondly, they argued that according to Norwegian expropriation law, it was not permissible to expropriate in a situation such as this, when the loss of energy and commercial potential would outweigh the gain to those same interests, which, ostensibly, were the only interests identified in favor of transferral. It seemed to the original owners that expropriation in this case would only serve to benefit the commercial interests of Jørpeland Kraft AS, and that it would do so to the detriment of both local and public interests. For this reason, the owners held that the concession should be regarded as an abuse of power, a manifestly ill-founded decision which could not be upheld.\footnote{There are at least two different ways in which to argue such a point under Norwegian law. One is with respect to water law and general administrative law, whereby clearly ill-founded decisions can be overturned by the courts, even when they involve discretion on part of the executive, which is otherwise not subject to review by the courts. Secondly, an argument can be made with respect to the Norwegian Constitution, Section 105, which gives property a protected status. The former is usually more effective, but in both cases, quite a severe transgression will have to be established before courts consider it within their competence to overturn discretionary decisions. A scholarly examination of these two sets of provisions are given in \cite{Efvl} and \cite{flei} respectively (both in Norwegian).} The owners argued, moreover, that the government had not fulfilled its duty to consider the case with due care, and that the assessment made with respect to the interests of the local community at Hjelmeland, and the local owners residing there, was not adequate. Particular attention was directed at the fact that local owners had not been informed about the progress of the case, and had not been told of, or asked to comment on, those preparatory steps that were being made explicitly with regards to assessing their interests. 

In addition, owners also argued that irrespectively of how the matter stood with respect to national law, the expropriation was unlawful because it would be in breach of the provisions in the ECHR TP1-1 regarding the protection of property.\footnote{European Convention of Human Rights Article 1 of Protocol 1.}\noo{An argument was also made to the effect that expropriation would be in breach of provisions in the EEA agreement regarding unlawful state support for the commercial interests of specific companies.}

Jørpeland Kraft AS protested all these objections to the expropriation, arguing that it was the responsibility of the owners themselves to provide information about possible objections against the project, and that the process had therefore been in accordance with the law. Unfortunate misunderstandings, if any, should be attributed to the fact that original owners had neglected their responsibilities in this regard. Moreover, Jørpeland Kraft AS argued that it was not for the courts to subject the assessment of public and private interests to any further scrutiny, since this was a matter for the government to decide. 

Indeed, according to Norwegian national law, it is traditionally held that unless the exercise of power it clearly unjustified, the courts do not have the authority to overturn decisions based on discretion, unless it can be demonstrated that the government has made procedural mistakes. While this view has become somewhat more relaxed in recent years, with a standard of \emph{reasonableness} increasingly being imposed by courts in similar cases, the inadmissibility of court interference in administrative discretionary decisions is still very much a part of Norwegian national law.\footnote{See \cite{Efvl}, in particular, chapters 24 and 29.}

Finally, Jørpeland Kraft AS argued that there was no issue of human rights at stake in the case. While they argued for this by stating that as the procedural rules had been followed and that the material decision was beyond reproach, they also went far in suggesting that as the owners would be compensated financially by the courts for whatever loss they would incur, no human rights issues could possibly arise in the case. \noo{ They also rejected the view that the case could be seen as an instance of illegitimate state support for Jørpeland Kraft, but failed to provide specific arguments in this regard.}

The matter went before Stavanger Tingrett who gave their judgment on 20 May 2009. In the following, we offer a presentation of the reasons given by this court, leading to the conclusion that the expropriation was unlawful and that the transferral could not be carried out. 

Stavanger Tingrett agreed with the original owners that the decision to grant concession was based on an erroneous account of the relevant facts, and they concluded that it was evident, from the NVE's own figures, that allowing the applicants to use the water from Brokavatn in their own hydro-electric scheme would be the most efficient way of harnessing the potential for hydroelectric production, directly contradicting what the NVE stated in their report. Moreover, they noted that these were the same estimates as those referred to by  Hjelmeland Kommune in their initial objection, and found it to be in breach of procedural rules that this was not considered further by the authorities.

The Court substantiated their decision by giving direct quotes from the report made by the NVE. For instance, in the report, on p. 199, it says, as quoted by Stavanger Tingrett (my translation):
%\begin{quote}Hjelmeland kommune ser helst at kraftressursene i vassdraget blir utnyttet av lokale %grunneiere. 
%Dette står i kontrast til uttalelsen fra grunneierne selv som ønsker at overføring blir gjennomført, 
%slik at flom og erosjonsskader kan bli noe redusert. NVE mener at den beste utnyttelsen med tanke 
%på kraftproduksjon vil være å tillate overføringen da en slik løsning vil innebære at vannet utnittes i 
%størst fallhøyde. Når dette samtidig er grunneiernes eget ønske har vi ikke tillagt Hjelmeland 
%kommunes synspunkt på dette noen vekt
%\end{quote}
%Our own translation follows below: 
\begin{quote}
Hjelmeland kommune would like the hydro-electric potential in the waterfall to be exploited by 
local property owners. This stands in contrast to the statement given by the property owners 
themselves, who wish that the transfer of water takes place, so that damage due to flooding can be 
somewhat reduced. NVE thinks that the best use of the water with respect to hydro-electric 
production is to allow a transfer, since this means that the water can be exploited over the greatest
distance in elevation. When this is also the property owners' own wish, we will not attribute any 
weight to the views of Hjelmeland kommune.
\end{quote}

Stavanger Tingrett concluded that as this was a factually erroneous account of the situation, the decision made to allow transferral of the water could not be upheld. Summing up, the Court offered the following assessment of the case (my translation):

\begin{quote}
It is the opinion of the court, having considered how the case was prepared by the authorities, that the factual basis for the decision made by the government suffers from several significant mistakes and is also incomplete.
\end{quote}

In light of this, Stavanger Tingrett concluded that the decision to grant concession for transfer of water was invalid. As to the legal basis of this, the court relied on the recognized principle of Norwegian public law that while the exercise of discretionary powers is usually not subject to review by court, a decision based on factual mistakes is nevertheless invalid if it can be shown that the mistakes in question were such that they could have affected the outcome. This is not provided for explicitly in statue, but it is one of the core unwritten legal principles of Norwegian public law.\footnote{See \cite{Efvl}}

Concerning the second requirement, that the factual mistakes could have affected the outcome, Stavanger Tingerett found that it was clearly fulfilled in this case since, in fact, the hydro-power suggested by original owners was, based on data available to the government at the time of decision, an objectively speaking \emph{better} use of the resource, even with respect to public interest. In any event, the requirement with regards to factual and procedural mistakes is only that the mistakes \emph{could} have affected the outcome; in the presence of mistakes, the burden of proof is shifted over to the party seeking to defend the decision.

Since Stavanger Tingrett agreed with the original owners that the decision was invalid due to being based on incorrect facts, there was no need to consider further the claims regarding the legitimacy of the decision with respect to human rights law. Stavanger Tingrett did conclude, however, making a more overreaching assessment of the case, that the procedure followed in preparing the case had not taken sufficient regard of owners' interests, and that this was the likely cause of the mistakes that had been made. The Court also argued that the standard of protection for interest of original owners had to interpreted as being more strict now that local hydro-power was an option available to original owners. 

\noo{In this regard, t also seems that Stavanger Tingett found some additional support in its interpretation of Norwegian law that was based on human rights concerns, especially the fact that expropriation, in circumstances such as those of this case, appeared to be a major interference in the rights of owners, and that established practice developed under a different regulatory regime was therefore no longer able to provide adequate protection.}

Jøpeland Kraft AS appealed the decision, and the case then went before the regional court, Gulating Lagmannsrett. They overruled the decision made by Stavanger Tingrett. In their argument, they do not rely on direct assessment of the report made by NVE, nor do they mention the expert statements retrieved by the opposing sides. Instead, they base their decision on general considerations concerning the need for efficient procedures in cases such as these. Such reasoning provides the apparent grounds for making the following rather crucial observation concerning the facts:

\begin{quote}... It was not a mistake to take Ola Måland's statement into consideration, as he was, and still is, a significant property owner. NVE's statement to the effect that granting the concession will facilitate 
a more effective use of the water seems appropriate, as it refers to a current hydro-electric plant that 
exploits a waterfall of 13.5 meters.
\end{quote}

Nowhere in their decision do they mention the statement made by Hjelmeland kommune, nor do they comment on the fact that alternative hydro-power, as suggested by the NVE itself, and pointed to in this statement, amounts to exploiting the waterfall over a difference in altitude of some 550 meters. In fact, the hydroelectric plant that they do mention has nothing to do with Ola Måland and the other owners, but exploits the same water further downstream. It was brought up in the testimony made by a representative from NVE, who, when pressed on the matter, claimed that the reasonable way to interpret the paragraph that Stavanger Tingrett quoted, and to which Gulating Lagmannsrett implicitly refer, was to see it as a statement regarding this hydro- electric plant. In light of the statement provided by Hjelmeland kommune, to which the report explicitly refers, this appears to be a manifestly ill-founded interpretation. But the regional court adopted it, without further comment.

As far as the legal basis of their decision is concerned, it seems that Gulating Lagmannsrett holds, quite generally, that the practice adopted by the water authorities in cases like these still provide adequate protection for original owners, and that it is not for the courts to subject it to critical review. As mentioned, they seem to base their stance in this regard on an overreaching appeal to the need for efficient procedures to deal with cases such as these.

The decision was appealed by Ola Måland and other, and the Norwegian Supreme Court decided to consider the juridical aspects of the case. The appeal concerning the assessment of the facts made by Gulating Lagmannsrett would not be considered, but was to be taken as correct. Since Gulating Lagmannsrett decided to regard as inessential several facts that were seemingly apparent, even from the report made by NVE itself, the appellants presented these facts to the Supreme Court and argued that Stavanger Tingrett was right regarding their consequences. \noo{In addition to this, written statements were retrieved from the Øystein Grundt, the public officer from the NVE that had been responsible for the preparation of the case, and Harald Sollie, }

The Supreme Court ruled in favor of Jørpeland Kraft AS. They comment on the relevant facts on 
p. 9 of their decision. There, they mention that Jørpeland Kraft AS had considered the possibility that a hydro-electric scheme could be undertaken by local property owners. As we mentioned in Section \ref{sum}, a statement was provided to the NVE by Jørpeland Kraft AS themselves -- the parties who stood to benefit from the transferral -- addressing one possible project that was deemed not to be commercially viable. Recall that in the same statement another project was also identified -- in the same river, using the same water -- that they claimed was such a good project that it could be carried out even after the transferral. As we mentioned, the statement does not say anything about what the property owners stand to loose when the water from Brokavatn disappears, and the Supreme Court is also silent on this. Nor do they mention that the statement was never handed over to the applicants, and that the details of the calculations were never handed over to, or considered by, the NVE. In fact, the full report first appeared during the hearing at Gulating Lagmannsrett, but this fact was not considered relevant by the Supreme Court.

Moreover, the Supreme Court remains silent on the fact that the conclusion concerning efficiency of exploitation contradicts both the NVE's own assessment, the statement made by Hjelmeland Kommune, and also all subsequent assessments made both on behalf of the applicants and on behalf of Jørpeland Kraft AS. We mention that all of the above were presented to all national courts, including the Supreme Court.

As to the legal questions raised by the case, the Supreme Court makes a more detailed argument than the regional court, culminating in the conclusion that established practice still provides adequate protection. Interestingly, the Supreme Court base their arguments in this regard on the premise that the case does \emph{not} involve expropriation of waterfalls. A similar sentiment is expressed by Gulating Lagmannsrett, and it was also argued for by Jørpeland Kraft AS, but the true force of this point of view did not become apparent until the case reached the Supreme Court. 

The Court first concludes that a legal basis for the concession to transfer the water is to be found in the Watercourse Regulation Act, Section 16. Moreover, they conclude that while this provision alone does not provide a right to expropriate the waterfall, it does give the applicant a right to divert the water away from it. While the Supreme Court notes that this amounts to an interference in property rights, they take it as an argument in favor of regarding the rules in the Watercourse Regulation Act as the primary source of guidance concerning what should be considered when preparing such cases. The hold, in particular, that the provisions in the Expropriation Act applies only so far as they supplement, and are not in conflict with, the rules of the Watercourse Regulation Act and established practice with respect to the provisions in this Act. Moreover, the main reason they give for this is that the diversion of water is \emph{not} to be considered as an expropriation of a waterfall.

There is, as we mentioned, no rule in the Watercourse Regulation Act which states that the authorities are required to consider specifically the question of how the regulation affects the interests of property owners. Such a rule is found in the Expropriation Act, Section 2, but according to the Supreme Court, it does not apply in cases where water is being diverted away from a river. This is so, according to the Supreme Court, because transferral of water is not regarded as a case of expropriation of a right to the waterfall, but merely an expropriation of a right to deprive the waterfall of water.

This is significant in two ways. First, it is important with respect to the legal status of owners who are affected by projects involving transferral of water. In Norwegian law after Måland, it seems that established practice with respect to the assessment of such cases, focusing on environmental aspects and the positions taken by various interest groups, is beyond reproach already because such cases do not involve expropriation of waterfalls. However, considering that the Norwegian water authorities seem to follow these practices generally, and not just in cases where water is transferred, it remains to be seen if this is a practically significant difference in the level of protection. Is the conclusion regarding the admissibility of current administrative practices supposed to apply only to those cases when water is subject to transferral? If it is, then it leads to the peculiar situation that the level of protection for owners depend solely on the way in which the developer propose to gain control over the water. The difference appears completely arbitrary, however, at least from the point of view of owners. But of course, it will soon cease to be arbitrary for developers, who must be expected to favor gutter projects, collecting water from many small rivers and diverting it, since this mode of exploitation makes it easier to acquire necessary rights. On the other hand, if the Supreme Court is to be understood as saying that traditional practices are adequate in general, the consequences of the decision seem fairly dramatic for local owners. It appears that it is not possible, in cases involving expropriation of waterfalls, to solicit any kind of judicial review, not even in circumstances when the factual basis of the decision is manifestly erroneous, and not even if this appears to be the consequence of the authorities neglecting to keep local owners informed about the assessments made regarding their interests.

To illustrate that a lack of consultation is a general problem, and not confined to the particular case of \emph{Måland}, we will conclude by offering a quote from Harald Solli, director of the Section for Concessions at the Ministry of Petroleum and Energy, who submitted written evidence to the Supreme Court regarding the practices followed in cases involving expropriation of waterfalls. Below, we give one of several exchanges that seem to indicate that under current practices, local owners are left in a rather precarious position (my translation).

\begin{quote}
Q: In cases such as this, should owners affected by a loss of small scale hydro-power potential be kept informed about the factual basis on which the authorities plan to base their decision? I am thinking especially about those cases in which the authorities make an assessment regarding the potential for small scale hydro-power on affected properties. \\
A: Affected owners must look after their own interests. The assessments made by the NVE in their report is a public document, and it can be accessed online through the home page of the NVE.
\end{quote}

By their reasoning in \emph{Måland}, it appears that the Supreme Court gave this dismissive attitude towards local owners a stamp of approval. In light of this, we believe the study of the law in a socio-legal setting becomes all the more relevant. For while this attitude might be a reflection of correct national law, as decided in the final instance by the Supreme Court, it seems pertinent to ask if it is \emph{reasonable} law. Also, it seems that one must ask if a case can not be made with respect to human rights, by arguing that the protection awarded is insufficient in this regard. This point, while it was raised by the original owners in \emph{Måland}, did not receive any separate treatment in the Supreme Court. In the following section, we briefly describe some more questions we think the case raises and which we will address further in subsequent chapters.

Following \emph{Måland}, it seems we must conclude that the development which has taken place in the energy sector, and has lead to small scale hydro-power becoming profitable and possible for local owners to carry out themselves, does not imply that original owners are entitled to increased participation in decision-making processes under national law. Even if this is the view held by the Norwegian judiciary, we should of course not overlook the possibility that the water authorities themselves will eventually adopt new practices regarding the assessment of such cases. So far, however, it seems that they stick quite closely to the established routine. 

Since the outcome in Norwegian Courts was that established practices were not found to be in breach of principles of Norwegian expropriation law, it seems reasonable to ask instead about the sustainability of these practices. In fact, the case of \emph{Måland} seems to illustrate precisely why the current system is inadequate, and how it can lead to decisions that appear ill-founded and leave the affected communities feeling marginalized. The likelihood of \emph{factual mistakes}, in particular, seems to increase greatly when the involvement of the local population is not ensured in the preparatory stages.

More importantly, it seems that decisions reached following a traditional process can easily lead to takings for which it is difficult to see any legitimate reason why the project proposed by the developer would be a better form of exploitation than allowing the local owners to carry out their own projects. Indeed, in the case of \emph{Måland}, it seemed that small-scale hydro-power would be a better way of harnessing the water in question, even in the sense that it would be more efficient, and would provide the public with more electricity at a lower cost. More generally, unless the issue of alternative exploitation in small scale hydro-power is considered during the assessment made by the water authorities, one risks making decisions that are not in the public interest at all. 

Even worse, it can send out the signal that expropriation of owners' rights is undertaken solely in order to benefit the commercial interests of the energy company applying for a development license. At this point, it seems appropriate to recall the concerns expressed by US Justice O'Connor in the case of {\it Kelo}.

There, a major point of contention was whether or not her grim predictions about the fallout of the decision did indeed reflect a realistic analysis of the fallout of the decision. Surely, anyone who agrees with Justice O'Connor that the powerful will usurp the power of eminent domain to the detriment of the poor, would also agree with here conclusion that it is perverse. However, whether her pessimism is warranted by empirical fact seems less clear. In this context, we believe the case of Norwegian waterfalls can serve an important broader purpose, as a means towards shedding more light on the hypothesis that a loose interpretation of the public interest requirement will indeed lead to a transfer of property from those with fewer resources to those with more. 

The \emph{Måland} case, and the current tensions regarding expropriation for the benefit of Norwegian hydro-power, seems to suggest that her concern should indeed be taken seriously. Also, the Norwegian experience seems to show that we need to be clear about the fact that property has a social and political function that goes beyond the financial interests of individuals. For the Norwegian case at least, it seems particularly relevant to ask if local people, by virtue of their right to property and their original attachment to the land, have a legitimate expectation \emph{both} that their commercial interests should be protected, \emph{and} that they should be granted a say in decision-making processes. Financial protection does not necessarily imply social protection, and the right to participate and be heard might be both more significant, and harder won, than the right to be compensated according to whatever the powers that be come to regard as the market value of the property in question.

Another perspective, which we will also pursue further in subsequent chapter, is the question of how property rights relates to the overreaching goal of sustainable development of natural resources. Rather than seeing property rights as a means towards securing sustainable development, it seems more common to see it as an impediment. This, indeed, has shaped much of the Norwegian discourse regarding environmental law and policy, including that which relates to waterfalls.\footnote{For example, such a skeptical view of property rights appear to provide an overriding perspective in \cite{backer1} (in Norwegian), which is a widely used textbook on environmental law in Norway.} 

Moreover, a typical justification given for interference in property is that an equitable and responsible management of natural resources requires it. It seems to us, however, that an egalitarian system of private ownership of resources -- as we find in Norway for the case of waterfalls -- could itself serve as a sustainable basis for management of these resources. It seems plausible for us to suggest that private property rights is one of the most robust ways in which local communities can be given a degree of self-determination concerning how to manage local resources. This is typically considered desirable also from the point of view of sustainability, but perhaps even more importantly, when property is in the hands of the many rather than the few, is it not also reasonable to expect that the state will be able to more effectively and rationally exercise its regulatory powers? 

Otherwise, the danger is that the government is being intimidated by large commercial enterprises, perhaps partly owned by the State itself, that command political influence and might not take lightly to what they perceive as undue political interference in their business practices. Such a position might be tenable if you are one of the worlds leading energy companies, but hardly if you are a farmer. 

I think the case of \emph{Måland} suggests that we should investigate these questions in more depth. It seems, in particular, that we must ask about the extent to which commercial companies have succeeded in usurping the notions of sustainable development and public interest, putting the power of these ideas to use in order to secure control over resources and to enlist governmental support, and favorable treatment, for their own commercial undertakings. The extent to which such a mechanism influences the Norwegian energy sector, and the possible implications this might have, both legally and socially, remains to be worked out.

In subsequent chapters, two questions arising from this will receive particular focus. First, we will aim to clarify the importance of the conflict between large scale hydro-power and small scale development by surveying recent and current hydro-power projects in Norway, not in any depth, but by taking note of whether the issue arose. Secondly, we will aim to shed light on the importance of small scale hydro-power to the communities in which local owners reside. As we mentioned, they are usually farmers, and most often in areas were farming is becoming increasingly unprofitable. From the socio-legal point of view it seems highly relevant to ask who the people who loose their resources are, and in what social context we find them. Moreover, while it is clear that hydro-power has become an important source of income in many small and relatively impoverished farming communities, the exact implications of this development, financially and socially, remains to be mapped out.

Following this, it seems natural to return to the legal question of the legitimacy of interference, not from the point of view of national law, but from the point of view of property as a human right. Importantly, it seems to us that property has a clear social dimension, and that mapping out the socio-legal function of specific property rights should inform the judgment we make regarding the level of protection to which owners are entitled. Also, while property is an individual right, it can also be a communal one, and, as such, it can serve to empower local communities that would otherwise be marginalized. The protection of an egalitarian structure of ownership, then, does not appear to be subsumed by, or even conceptually the same as, protecting against individual transgressions. We believe that the case of Norwegian waterfalls demonstrates that this should be kept in mind when analyzing the legitimacy of interference in property for the benefit of commercial undertakings.

\noo{current ownership structure of waterfalls is therefore not simply a question of protecting the commercial interests of individuals who happen to own valuable resources, but also a question of protecting the local communities where these resources are found, giving them the possibility of influencing the way in which the resources are to be harnessed. It seems, however, that local people are often in danger of being seen as an hindrance, both to sustainable development and economic growth, because the commercial companies, along with the environmental interests groups, have claimed this stage as their own. Such, it seems, is the case for Norwegian waterfall. Despite an explosion of interest in small scale hydro-power in recent years, there still seems to be little room left for local communities in the Norwegian discourse concerning hydro-power. It will be an important aim of our work in following chapters to map our in more detail how this influences the law and the administrative policies that are adopted.
}

\section{Conclusion}\label{conc}

In this Chapter, I set out to show that the law relating to expropriation of waterfalls in Norway is based on a tradition that assumes owners to be profit-maximizing while the state is welfare-seeking. Hence, the question of striking a balance between private and public interests is approached under the presumption that private property rights embody mainly private values, while public values are pursued through regulation that ensures public ownership and control. I observed how this perspective shaped the law of expropriation of waterfalls, so that expropriation could only take place for narrowly defined public purposes and only to the benefit of public bodies.

I noted, however, how the increasing centralization of the energy sector and the increasing scale of projects following WW2 led to increased worry about the legitimacy of interference in property and nature on behalf of public hydropower interests. I concluded that the ensuing conflicts, while severe, largely failed to make an impact on the law relating to hydropower, as the public nature of this sector, and the level of political control exercised over it, meant that courts shunned away from adopting a strict view on legitimacy. This did not only apply to the question of authority to expropriate, which was hardly raised at all in the period between the reversion controversy of the early 20th century and the market-reform of the early 1990s.  It also applied to the procedural rules, which the Supreme Court adopting an explicit stance that these rules were themselves largely ``discretionary'' in nature, so that it would fall under the authority of the executive to determine their scope and application in concrete cases.

i noted how this perspective has been maintained by the courts and the executive even after the market-reform meant that expropriation largely lost its public characteristics. I argued that today, expropriation of waterfalls for hydropower development can no longer be looked at as an aspect of providing a public service, but must be regarded as takings for profit, typical economic development takings. I discussed how the law came to be changed on this point, with a dramatically widened expropriation authority introduced in conjunction with the \cite{wra00}. I observed how the issue of expropriation was not considered politically, with the reform in the legislative basis taking place without the active involvement or consideration by the members of the Norwegian Storting.

I concluded with a description of the fallout from this, as expressed concretely in the case of \cite{måland11}. The case serves to illustrate how administrative practices developed and sanctioned during the monopoly days are now applied in a context of competing commercial interests, meaning that expropriation becomes an important tool that the powerful market players now use to gain the upper hand in competition with locally based companies or smaller companies that rely on cooperation with owners. I noted, in particular, that the law is entirely unprepared for dealing with this dynamic. Still, in the case of \cite{måland11}, the Supreme Court explicitly denied that established practices were in need of revision. Moreover, it refused to reconsider the established interpretation of the scope of procedural rules in hydropower cases, rejecting arguments to the effect that these must now be understood to provide protection for waterfall owners that matches the protection offered to other affected parties.

In the next Chapter, I will consider an aspect of the law were the Supreme Court {\it has} taken the view that a revision of established practices is in order, namely in relation to the question of compensation. I note, however, that the Court's emphasis on the compensation issue serves to reinforces the idea that private property rights pertain mainly to financial entitlements. As I have already argued, this perspective hardly does justice to the role of private ownership of waterfalls in Norway. I will return to this in the last Chapter of the thesis, where I consider land consolidation as an alternative to expropriation. However, as demonstrated in the present Chapter, Norwegian courts do not seem to recognize the shortcomings of the current system. Until they do, or are directed to do so by political bodies or international tribunals, it is unlikely that expropriation law will evolve much from its current fixation on the compensation issue. 

