\chapter{Taking waterfalls}\label{chap:4}

\section{Introduction}\label{intro}

In most jurisdictions that recognize private ownership rights, the State retains a right to interfere with property rights to the extent needed in order to provide public services and further public interests. However, in many cases when the State infringes on private property rights in this way it also recognizes an obligation to compensate the owner. This is certainly the case when interference takes the form of an outright expropriation and the property rights themselves are transferred from the original owner to the State or one of its agents. In many jurisdictions, the obligation to compensate the owner in such cases follows already from constitutional provisions protecting the right to property. Also, the owner's right to compensation tends to follow from international law to protect human rights.\footnote{The principle of a fundamental right to property is widely recognized, and is encoded in Article 17 of the Universal Declaration of Human Rights. A particularly relevant formulation, in a European context, is the one found in the European Declaration of Human Rights (ECHR) Protocol 1, Article 1. Its importance stems from the fact that the ECHR is accompanied by a special judicial body, the European Court of Human Rights (ECtHR) in Strasbourg, and has also been widely incorporated into the national law of the ratifying countries, including the UK and Norway.} 

One important question that arise is how compensation should be calculated. An often adopted starting point is to say that the owner should be compensated for his financial loss. In some cases, specific policy reasons might be given in favor of deviating from this, for instance if considerations based on social justice dictates that less than full compensation should be paid, but still, as an overreaching principle, compensation for financial loss has widespread status as a general rule. This is reflected, for instance, in case-law from the European Court of Human Rights.\footnote{In particular, the Court in Strasbourg adopts the view that interference should be \emph{proportional}, which, in most cases, entails compensating the owner for his full financial loss, usually taken to be reflected by the market value of his property, see \cite{AllenCom} for an in depth analysis of this principle, showing also its controversial aspects in light of the history of the ECHR.} While the principle protects property owners, it is also often cited as a reason to reject claims of compensation based on a hypothetical voluntary agreement between property owners and the public. Such an approach might correctly reflect the \emph{willingness} of the public to pay, but the value to be compensated is the loss, or, as it is often said, the \emph{value to the owner}. This, in particular, is typically taken to mean that the public should not be required to pay extra to reflect the fact that the scheme underlying expropriation serves an important social function that the public might be willing to pay for. In this way, the State ensures that public interests are not being made available as a means for property owners to make excess financial profits, beyond what they could otherwise expect from their property.

Under modern planning regimes, however, the scope of interests allowed to benefit from expropriation has been significantly widened, and in many cases the State explicitly allows for public interest schemes to merge with commercial undertakings. In this way, it is nowadays commonplace that companies operating for profit come to benefit financially from schemes involving expropriation. Increasingly, expropriation is also made available as a tool for purely commercial schemes, the rationale behind this being that there are indirect benefits to the public -- increased tax revenues, generation of jobs etc -- that justify the use of compulsion. This kind of expropriation often proves controversial, however, as illustrated by the uproar in the US following the case of \emph{Kelo}, where the drug company Pfizer was allowed to expropriate land for the construction of research facilities.\footnote{\emph{Kelo v City of New London}, 545 U.S. 469 (2005)} While a majority the US Supreme Court found that the expropriation was constitutional, it was widely felt to be inappropriate, both by members of the general public and the legal profession.\footnote{See, for instance \cite{notimminent}, which also demonstrates that this kind of expropriation is relatively uncommon in the US, thus reflecting the pervading opinion that it is unsound, if not illegal.}

While expropriation for commercial schemes raises important questions about legitimacy of interference, it raises even more pressing questions in relation to established principles for calculating compensation. Crucially, the traditional justification for the "value to the owner"-principle appears to be severely undercut by the fact that certain actors are already, as a result of the regulatory regime, placed in a position to profit financially from schemes that are ostensibly carried out in the public interests. This begs the question: What are the policy reasons for maintaining a principle that excludes property owners from a share in this profit? As was noted by the Law Commission in a discussion paper from 2001, traditional arguments for the appropriateness of limiting the right to compensation often fail to do justice to such cases, and the adequacy of existing rules and principles needs to be considered in this light.\footnote{See \cite{lcdisc} and \cite{newuk,kelouk}, all addressing the problem from the point of view of UK law.\noo{The same issue was raised ....}} Moreover, the question arises as to whether or not established practices are in keeping with human rights law. For instance, can owners claim a share of the commercial value of their property on the basis that they are entitled to it under the European Convention of Human Rights, even if established national practices would lead to compensation payments that do not reflect this value?

So far, a theoretical framework for discussing questions such as these, arising specifically with respect to awarding compensation for commercial schemes, appears to be largely missing, and not much scholarly work has been devoted to it.\footnote{Is this really true? I could not find that much...} In this paper, we make a contribution in this regard, offering a case study of Norwegian waterfalls, which are often expropriated for the development of commercial hydro-power projects.\footnote{Hydro-power is an important source of electricity in Norway, with some 95 \% of annual domestic electricity consumption due to hydro-power.} 

In the early 1990s, the Norwegian energy sector was liberalized, making it possible for owners of waterfalls to exploit these commercially in small-scale hydro-power projects. Soon, a market also developed for waterfalls, where the prices paid would reflect the value of the waterfall for commercial, small-scale, hydro-power. As a result of this, the law relating to compensation for expropriation of waterfalls has recently undergone significant changes, with courts now increasingly looking to the new market for guidance when determining the right level of compensation following expropriation. The new market is still embryonic, however, with most large companies opting instead to use expropriation to acquire necessary rights. As such, the market-based approach often cannot offer more than a limited perspective on the true commercial value of waterfalls, and this is the root of many contemporary controversies in Norwegian hydro-power law.

It seems that the case of Norwegian waterfalls is ideal for shedding light on many of the general questions that arise for compensation following expropriation that benefits commercial schemes and it is particularly interesting to see how the Norwegian Supreme Court have dealt with these questions in a series of recent decisions. While the sentiment has changed in favor of regarding the commercial potential for hydro-power as a relevant factor when valuating waterfalls, the influence of the "value to the owner" principle is still clearly felt in the Court's reasoning. However, in several cases, the logical conclusion offered by such reasoning has appeared so offensive to the lower courts that they have instead opted for a retreat to a traditional, theoretical method for calculating compensation, that is \emph{not} based on a standard "value to the owner" approach.

In a recent decision, the Supreme Court gave this approach its stamp of approval, providing thus a further indication that there is a pressing need for academic assessment of the current state of affairs in Norway.\footnote{\emph{Bjørnarå and Others v Otra Kraf DA and Otteraaens Brugseierforening} (Otra II), Rt. 2013 s. 612}

The structure of our paper is as follows. In Section \ref{sec:noscheme}, we discuss the standard version of the "value to the owner" principle, focusing on those aspects that are expressed through the so-called "no-scheme" rule, often referred to as the Pointe Gourde rule in common law.\footnote{Following the precedent set by \emph{Pointe Gourde Quarrying and Transport Co v Sub-Intendent of Crown Lands} [1947] AC 565,
PC, 572, per Lord MacDermott. We remark, however, that the case was a clarification, and, seemingly, a widening of the principle, but by no means the first application of such a rule in common law, see \cite{lcdisc} \noo{Appendix D of Law Commission Report No 286, 2003 for an historical presentation}.} We approach the rule building on a conceptual distinction between commercial and non-commercial aspects of a scheme benefiting from expropriation, and we argue that this is appropriate, and even necessary, if we are to do justice to the rule in light of property as a human right. Then we move on to present a brief overview of Norwegian law relating to the no-scheme rule, arguing that the need for a conceptual distinction along the lines we propose is clearly felt in the contemporary Norwegian debate, which has at times been heated.

In Section \ref{sec:trad}, we present the traditional method for calculating compensation for expropriation of Norwegian waterfalls, showing that it deviates completely from the "value to the owner" principle, relying instead on a theoretical assessment of the value of the developer's scheme. We briefly discuss the historic context of this rule and we note the lack of attention paid to it in the general debate on compensation. We argue, however, that seeing the rule only as an anomaly is inappropriate. Rather, the rule serves as an illustration of one possible, albeit largely misguided, approach to a challenge that is becoming increasingly relevant in general, namely that of finding principles for valuation that are suitable for schemes with a substantial commercial component. 

In Section \ref{sec:new}, we discuss the new method that has been adopted by the courts after the liberalization of the Norwegian energy sector. While the method has yet to mature, it appears to follow a standard no-scheme approach. This raises some problems, however, and we discuss various recent cases that illustrate this. We then return to the issues raised in Section \ref{sec:noscheme}, and we argue that the case of Norwegian waterfalls illustrates clearly that a standard "value to the owner"-theory of compensation is inadequate in modern regulatory settings. 

The Norwegian case suggests, we believe, that there is a pressing need for regulation and legislation based on a distinction between commercial and non-commercial aspects of schemes benefiting from expropriation, and on the \emph{kind} of value that is inherent in a scheme benefiting from expropriation. While \emph{public value} is hardly to be regarded as value to the owner, it seems that \emph{commercial value}, to avoid discrimination, must largely be regarded as such for the purpose of compensation, irrespectively of \emph{who} the State happens to prefer as developer of the scheme. It appears, in particular, that especially the \emph{negative} aspect of the no-scheme rule, providing for a decrease in compensation, is hard or impossible to justify with respect to the commercial aspects of undertakings benefiting from expropriation. In Section \ref{sec:conc}, we conclude and point to possible directions for future work.

\section{The no-scheme rule}\label{sec:noscheme}

In most jurisdictions, a fundamental principle relating to compensation following expropriation is that the owner's loss should be calculated without taking into account changes in the value of his property that are due to the expropriation itself, or the scheme underlying it. In a recent Law Commission consultation paper, this principle is referred to as the \emph{no-scheme} rule, a terminology we will also adopt here, noting that while the exact details of the rule might differ between jurisdictions, the underlying principle appears to play a crucial role both in civil and common law traditions for regulating compensation following expropriation.\footnote{Need a good reference for this...}

While the no-scheme rule is easy enough to comprehend when it is stated in general terms, it raises many difficult questions when it is to be applied in concrete cases. What the rule asks of the valuer, in particular, is quite daunting; he is forced to consider the counterfactual "no-scheme world", and he must calculate the value of the property based on the workings of this imaginary world. One crucial question that arises, and which has traditionally proved to be highly contentious, is the question of what exactly this world looks like.

In the first instance, it might be tempting to take the view that this is a "question of fact for the arbitrator in each case", as expressed by the Privy Council in \emph{Fraser}, an influential Canadian case from 1917.\footnote{\emph{Fraser v City of Fraserville}, [1917] AC 187, p. 194} However, as the history of the no-scheme rule has shown, this point of view is not tenable.\footnote{For an history of the rule in UK law, clearly illustrating the difficulty in interpreting it and applying it to concrete cases, we point to Appendix D of Law Commission Report No 286, 2003} A more recent description of the rule, and the problems associated with it, was given by Lord Nicholls in the recent case of \emph{Waters}, who summarized the state of the law relating to compensation for expropriation as follows.\footnote{\emph{Waters and other v Welsh National Assembly} [2004] UKHL 19}

\begin{quote}
Unhappily the law in this country on this important subject is fraught with complexity and obscurity. To understand the present state of the law it is necessary to go back 150 years to the Lands Clauses Consolidation Act 1845. From there a path must be traced, not always easily, through piecemeal development of the law by judicial exposition and statutory provision. Some of the more recent statutory provisions defy ready comprehension. Difficulties and uncertainties abound. One of the most intractable problems concerns the 'Pointe Gourde principle' or, as it is sometimes known, the 'no scheme rule'. On this appeal your Lordships' House has the daunting task of considering the content and application of this principle.
\end{quote}

\noo{
\begin{quote}
The extreme complexity of the issues that I have had to consider, the
uncertainty in the law, the obscurity of the statutory provisions, and
the difficulties of looking back over a long period of time in order to
decide what would have happened in the no-scheme world
demonstrate, in my view, that legislation is badly needed in order to
produce a simpler and clearer compensation regime. I believe that
fairness, both to claimants and to acquiring authorities, requires
this
\end{quote}
}
In the case of \emph{Waters}, the House of Lords seems to have made it an explicit aim to offer a clarification of the no-scheme rule and how to interpret it, and their judgment went into much more detail than warranted by the concrete case at hand, which seems to have been fairly straightforward. Even if it was not needed for the result, the House of Lords addressed many of the issues raised by the Law Commission in their report, focusing particularly on resolving the tension which was identified there between the principle relied on in \emph{Pointe Gourde} and the reasoning adopted in the so-called \emph{Indian} case from 1939.\footnote{\emph{Vyricherla Narayana Gajapatiraju v Revenue Divisional
Officer, Vizagapatam} [1939] AC 302.} In the \emph{Indian} case, the scheme was given a very narrow interpretation, with Lord Romer interpreting the scope as follows.
\begin{quote}
The only difference that the scheme has made is that the acquiring
authority, who before the scheme were possible purchasers only, have
become purchasers who are under a pressing need to acquire the
land; and that is a circumstance that is never allowed to enhance the
value.
\end{quote}
This, however, did not entail that the purchaser's demand for the property was to be disregarded, since, as Lord Romer puts it:

\begin{quote}
[...] The fact is that the only possible purchaser of a potentiality is
usually quite willing to pay for it […]
\end{quote}

In \emph{Pointe Gourde}, a different stance seemed to be taken in this regard. The case concerned a quarry that was expropriated for the construction of a US naval base in Trinidad. The quarry had value to the owner as a going concern, and the valuer had found that if the quarry had not been forcibly acquired, it could have o supplied the US navel base on a commercial basis, leading to its value being enhanced. This value, however, fell to be disregarded, with Lord MacDermott describing the no-scheme rule as follows.

\begin{quote}
It is well settled that compensation for the compulsory acquisition of
land cannot include an increase in value, which is entirely due to the
scheme underlying the acquisition
\end{quote}

Seemingly, then, the two cases are at odds with each other as far the interpretation of the no-scheme rule goes. In \emph{Waters}, both Lord Nicholls and Lord Scott addressed this tension in great detail, and offered a reconciliatory interpretation, which seems to narrow the no-scheme rule compared to how it has most commonly been understood following \emph{Pointe Gourde}. Moreover, the House of Lords also noted the need for reform and legislation, with Lord Scott going as far as referring to what he described as the present "highly unsatisfactory state of the law".

To understand how a seemingly simple principle could come to prove so troubling in practice, it is helpful to keep in mind that following extensive planning legislation, especially following the Second World War, development of property tends to be contingent on governmental licenses and subject to extensive oversight. Moreover, the power to expropriate is often granted as a result of comprehensive regulation of the property-use in an area, often following public plans that are wider and encompass more than the particular project that will benefit from such a power. As a result, it became increasingly difficult to ascertain what is meant by the scheme; does it include the whole planning history leading to expropriation, does it only refer to the power to expropriate, or is it something in between?

When attempting to address this issue, there any many pitfalls, and the policy reasons suggest that a fine balancing act must be made. If given a wide interpretation, the property owner might easily come to feel that he is not compensated for his true loss, but rather an imaginary one. Moreover, the no-scheme world that the valuer must consider can end up being far removed from the actual one, forcing him to go back many years, perhaps decades, to establish what would have been the status of the property in question if the sequence of planning steps eventually leading to expropriation had not taken place. This can leave the property owner in a perilous situation, and make the outcome seem so arbitrary as to run amiss with respect to human rights law and constitutional provisions protecting private property. On the other hand, if the scheme is interpreted too narrowly, it runs the risk of endangering important public schemes by compelling the public to pay extortionate amounts for an increase in value that is entirely due to their own non-commercial investments and plans for the area in which the property is found.

It is important to keep in mind here, as noted by the Law Commission, that the no-scheme rule serves two very different policy aims.\footnote{Ibid ....} It should be noted, in particular, that the rule has an important \emph{positive} dimension: property owners are not only compensated for the direct loss of their property, but also for the possible depreciation of their property's value following the decision to carry out a scheme which requires expropriation. Seemingly, this is easy enough to justify; it would easily appear unreasonable, and possibly in breach of human rights law, if compensation payment was reduced as a result of the threat of compulsion.

However, under the extensive planning regimes common today, it is not clear where to draw the line: When is the regulation leading up to the scheme to be regarded as reflecting general public control over property use, and when is it to be regarded as a measure specifically aimed at compelling private owners to give up their property? As we will see when we consider the role of the no-scheme rule in Norwegian law, this question can easily become highly controversial, especially when it is linked with the more general question of whether or not the State should be liable to pay compensation for regulation that adversely affects the potential for future development. In jurisdictions that do not recognize owners' right to such compensation, like Norway and the UK, it is easily argued that the positive aspect of the no-scheme rule must be limited correspondingly. Why would a depreciation of value following regulation imply compensation when the property is eventually expropriated, but not otherwise?

In addition to its positive dimension, the no-scheme rule also has an important \emph{negative} dimension, which is the dimension with which \emph{Waters} was mostly concerned, and which was expressed in \emph{Pointe Gourde} by saying that one should disregard an increase in value that was "entirely due to the scheme". The negative dimension has attracted even more interest and controversy than the positive dimension, especially in the UK, and this is understandable all the while the negative aspect of the rule can easily come to be perceived as unfair by property owners. However, on a traditional understanding of the public purpose of expropriation, the negative aspect of the rule is also seemingly easy to justify. In \emph{Waters}, Lord Nicholls describes the policy reasons behind it as follows:

\begin{quote}
When granting a power to acquire land compulsorily for a particular purpose Parliament cannot have intended thereby to increase the value of the subject land. Parliament cannot have intended that the acquiring authority should pay as compensation a larger amount than the owner could reasonably have obtained for his land in the absence of the power. For the same reason there should also be disregarded the 'special want' of an acquiring authority for a particular site which arises from the authority having been authorised to acquire it.
\end{quote}

This appears like a reasonable line of argument. Notice, however, that Lord Nicholls completely avoids using the word "scheme" here, and does not use the absence of the scheme as the yardstick by which parliament must have intended that compensation should be based. Rather, Lord Nicholls speaks of what the owner could reasonably have obtained in \emph{the absence of the power} to acquire the land compulsory. In this way, he seems to prescribe a rather narrow interpretation of the negative dimension of the no-scheme rule.\footnote{I mention that this interpretation of \emph{Waters} is also argued for in \cite{newuk}.} It is the power to expropriate that should not give rise to an increased value, and nothing is said at this stage about the scheme that benefits from it. It would appear, therefore, that there is nothing in principle that prevents the property from being compensated on the basis of its value in a scheme that differs from the scheme underlying expropriation only in that it does not have such powers. Indeed, this subtle distinction appears to have been rather crucial for the remainder of Lord Nicholls' reasoning, where he attempts to reconcile the principle adopted in the \emph{Indian} case with the \emph{Pointe Gourde} case.

It will lead us to far astray to go into further details about the interpretation of the no-scheme rule in UK law and the possible implications of \emph{Waters}. Rather, we would like to turn our attention to the recent UK Supreme Court case of \emph{Bocardo}.\footnote{\emph{Star Energy Weald Basin Limited and another (Respondents) v Bocardo SA (Appellant) [2010] UKSC 35}} This case was decided under dissent, and it suggests that the clarification offered in \emph{Waters} might not have been as conclusive as one had hoped. This worry arises, as we will see, particularly in those cases when expropriation benefits commercial schemes.\noo{ and for which the conceptual framework surrounding expropriation is, in our opinion, in need of refinement.}

\emph{Bocardo} was such a case. In short, it concerned a reservoir of petroleum that extended beneath the appellant's estate, and could not be exploited without carrying out works beneath their land. The first question that arose was whether or not extraction of the petroleum amounted to an infringement on property rights, which was answered in the affirmative. The second question that arose was what principle of compensation should be adopted to compensate the owner. The Supreme Court, following some deliberation, found that the general rules applied, and that the case should be decided on the basis of an application of the no-scheme rule. Here, however, opinions differed as to the correct interpretation of the law, as well as how the facts should be held against the law. The crucial point of disagreement arose with respect to whether or not the special suitability, or \emph{key value}, of the appellant's land for the purpose of petroleum exploitation was to be regarded as \emph{pre-existing} with respect to the petroleum scheme.

In \emph{Waters}, the House of Lords had cited and expressed support for the following passage, taken from Mann LJ's judgment in \emph{Batchelor}.\footnote{\emph{Batchelor v Kent County Council} 59 P \& CR 357 p. 361}

\begin{quote}
If a premium value is 'entirely due to the scheme underlying the acquisition' then it must be disregarded. If it was pre-existent to the acquisition it must in my judgment be regarded. To ignore the pre-existent value would be to expropriate it without compensation and would be to contravene the fundamental principle of equivalence (see \emph{Horn v Sunderland Corporation}).
\end{quote}

Relying on this distinction between the potentialities that are "pre-existing" and those that are due to the scheme, the minority in \emph{Bocardo}, led by Lord Clarke, made the following observation.

\begin{quote}
Anyone who had obtained a licence to search, bore for and get the petroleum under Bocardo’s
land would have had precisely the same need to obtain a wayleave to obtain access
to it if it was not to commit a trespass. So it was not the respondents’ scheme that
gave the relevant strata beneath Bocardo’s land its peculiar and unusual value. It
was the geographical position that its land occupies above the apex of the
reservoir, coupled with the fact that it was only by drilling through Bocardo’s land
that any licence holder could obtain access to that part of the reservoir that gives it
its key value.
\end{quote}

This, however, was rejected by the majority, led by Lord Brown, who interpreted the no-scheme rule quite differently in this respect, and who made the following comments regarding the issue of whether or not the value of the appellants land for petroleum extraction existed prior to the scheme.

\begin{quote}To my mind it is impossible to characterise the key value in the ancillary
right being granted here as “pre-existent” to the scheme. There is, of course,
always the chance that a statutory body with compulsory purchase powers may
need to acquire land or rights over land to accomplish a statutory purpose for
which these powers have been accorded to them. But that does not mean that upon
the materialisation of such a scheme, the “key” value of the land or rights which
now are required is to be regarded as “pre-existent”.
\end{quote}

While the case was resolved in keeping with this view, the dissent suggests that the clarification in \emph{Waters} has not resolved all issues, and that special questions arise with respect to the question of what potentials for development should be taken into account when evaluating a property. Crucially, the question raised in \emph{Bocardo} does \emph{not} relate to the scope of the scheme -- it was obvious that the scheme was the entire project aiming to extract petroleum from the reserve. However, even when the scheme was unambiguously circumscribed, significant questions arose as to what "value to the owner" actually meant. 

In fact, it seems to us that \emph{Bocardo} serves to take the debate regarding compensation and the no-scheme rule one step further, and in a somewhat different direction compared to the debate revolving around the "classical" problem of determining the extent of the scheme. In some sense, it seems that the question raised by \emph{Bocardo} goes deeper, and to the very core of the idea underlying the negative aspect of the no-scheme rule. When is it appropriate to say that some particular value is \emph{due to} the scheme?

This asks us to establish a causal link between scheme and value, and as \emph{Bocardo} illustrates, it is by no means obvious what should be taken to constitute evidence for such a link. Moreover, it seems that the answer can depend largely on the point of view with which you \emph{choose} to analyze the matter at hand. For instance, when Lord Clarke went on to point out that the State, as owners of the Petroleum following nationalization in 1937, could have given the right to extract it to \emph{someone else}, he was certainly not incorrect.\footnote{References.} Moreover, it seems that this fact does in some sense break the causal link between scheme and value, although weakly so, since the difference between all schemes so conceived would only relate to \emph{who} the developers are, not the nature of the schemes as such. Consider, however, a scheme that was conceived of slightly differently, and assumed to suffer precisely from such an \emph{absence of the power to expropriate} as Lord Nicholls referred to in \emph{Waters}. Would it not follow that this scheme would also have \emph{precisely the same need to obtain a wayleave}, as Lord Clarke puts it, and that those behind it might now also be \emph{quite willing to pay}, as Lord Romer expressed it in the \emph{Indian} case?

On the other hand, it is also possible to take the point of view adopted by Lord Brown, which, albeit less clear in its formulation, we interpret to be roughly the following: Since the relevant strata did not, as a matter of fact, have any value except such value as it derived from its key value to a petroleum-scheme requiring access, its value was causally dependent on the existence of \emph{some} such scheme, and could thus not be regarded as pre-existent to \emph{any} such scheme, including the actual scheme, for which power to expropriate \emph{was} in fact granted.

Clearly, the outcome of \emph{Bocardo} turned largely also on the specific question of how to appropriately compensate property that has "key value" with respect to the development of other property. It seems, in particular, that the strata, in its absence of any inherent value, more easily fell to be disregarded, and that Lord Brown's arguments in particular relies on establishing such a lack of intrinsic value. However, the question of when a particular aspect of value is to be regarded as pre-existent tend to arise in many other cases as well, and can be expected to arise particularly often with respect to commercial schemes. An extreme case obtains when we consider expropriation of \emph{natural resources}. Surely, if what was subject to expropriation in \emph{Bocardo} had been the petroleum itself, and not a right to access it, then even Lord Brown would have concluded that its value was pre-existent? This seems likely indeed, and then it appears to be good law in the UK after \emph{Waters} that it should also be compensated, irrespectively of whether or not the expropriating party is the only potential buyer.

However, in several of the "classical" cases that are cited as the foundation for the original no-scheme rule, the opposite outcome has been reached in very similar circumstances. This is true, in particular, for both \emph{Cedars} (1914) and \emph{Fraser} (1917), two important Canadian cases concerning expropriation for hydro-power, cited both by the Law Commission and the House of Lords in \emph{Waters}.\footnote{\emph{Cedars Rapids Manufacturing and Power Co v Lacoste}, [1914] AC 569 and \emph{Fraser v City of Fraserville} [1917] AC 187.} In \emph{Fraser}, it was the waterfalls themselves that were subject to expropriation, yet the Privy Council still found that the value of the potential for hydro-power exploitation of these falls should be disregarded when compensating them, following a standard "value to the owner" approach. Reasoning along the same lines is, as we will see later, prevalent in Norwegian law, although with some significant caveats suggesting the problematic nature of this line of reasoning. Indeed, it would appear most problematic also in light of \emph{Waters}, raising the question of the current status of the Canadian cases.

In any event, they can serve as great examples of the type of situation where the need for a distinction between commercial and non-commercial aspects arise most forcefully. It seems, in particular, that there can be no doubt that the energy inherent in water pre-exists any scheme seeking to harness it. Moreover, it seems clear that energy has value, and so, the conclusion would have to be that also the value of a waterfall pre-exists any scheme for hydro-power exploitation. However, we can then refine our approach by asking: what \emph{kind} of value is it? This, indeed, might be the solution to our troubles. 

For it seems that any value resulting in compensation to the owner must by the nature of things either be \emph{personal}, related to claims for disturbance etc, or else \emph{commercial}, namely such a kind of value that can be realized by a company or an individual operating for profit -- possibly the owner himself, possibly some buyer of his property. A different kind of value altogether is the \emph{public value}, which can not be realized for profit by \emph{anyone}. 

The distinction between commercial and public value is, obviously, down to a political decision, and it can hardly be regarded as permanent. Moreover, it can often be difficult to assess where the line is to be drawn, especially in cases when public/private partnerships cooperate to provide public services. Nevertheless, it seems perfectly legitimate to make this distinction, and it seems like it can be very helpful in many cases. For instance, even if the public value of hydro-power pre-exists the hydro-power scheme, this does \emph{not} mean that there is any pre-existent commercial value in hydro-power. That, in particular, depends entirely on whether or not the public has settled on a regulatory regime that allows commercial exploitation. On the other hand, once a decision to allow commercial exploitation has been made, it seems quite reasonable to apply the "pre-existence" test used in \emph{Bocardo}: An owner should always be compensated for the value of any pre-existent \emph{commercial} value that his property has.\footnote{Certainly, a clarification along these line would not resolve all issues. It would not, for instance, offer any conclusive guidance with respect to the specific issues related to "key value" raised in \emph{Bocardo}.} 

To conclude, we would like to remark that unlike problems relating to the scope of the scheme, the question of what commercial value can be said to pre-exist a scheme might turn rather more on facts than on law. It seems, in particular, that \emph{this} is a question that it is not so easy, or even desirable, to attempt to resolve by legislation or a fixed set of principles. It seems quite clear, in particular, that in order to answer the question of what should be counted as a pre-existing commercial value, one must take a broad look at the prevailing regulatory regime. Moreover, one must expect that the correct assessment of this question will depend on the context of regulation, in particular the extent to which the State \emph{allows} the disputed value to be commercially realized. The law relating to compensation should be such that it can tolerate significant changes in these parameters, and it seems therefore that the important legal question in this regard is to provide a sound conceptual foundation for making a sound assessment across a range of different scenarios. Moreover, it seems that the courts, in light also of human rights law, has an important supervisory role to play in this regard.

In the next section, we will address the no-scheme rule in Norwegian law, and as we will see, the distinction between commercial and public value is rarely made in general compensation law. This seems unfortunate, and as we will see, it makes the special rules adopted for waterfalls appear as something of an enigma in Norwegian expropriation law.

\subsection{The no-scheme rule in Norwegian law}\label{sec:nonor}

Before 1973, the Norwegian law relating to compensation for expropriation of property was based on case-law. The courts would interpret Section 105 in the Norwegian constitution which demands that "full" compensation is to be paid. A no-scheme rule was typically applied, such that when assessing the value of the property, the element of compulsion was disregarded, and changes in value that could be attributed to the underlying scheme tended not to be taken into account. As in the UK, difficult questions would arise for comprehensive schemes based on public plans for the use of the property, and it proved difficult to identify any clear rule concerning the distinction between the scheme itself and the regulation of property-use that preceded it.\footnote{References.}

Following the Second World War, there was an increasing trend that effects of regulation \emph{would} be taken into account, but only with regards to the \emph{positive} aspect of the no-scheme rule. That is, the scheme was taken into account in so far as it could be used to argue against alternative development, but not in such a way that it could lead to an increase in the value of the property. In some cases, even the regulation directly preceding, and providing the basis for, the use of compulsion, would be taken into account.\footnote{Such as in Rt. 1970 s. 1028, where a property which had been used by a local business owner was expropriated to implement a public plan that regulated the property for use as a public road with parking spaces. The owner was not compensated for the loss of business revenue, since, according to the majority in the Supreme Court, the regulation of the property had to be taken into account (the decision was given under dissent). The case was somewhat special, however, since the business value could be realized by the owner only if he had been given the opportunity to rebuild his store, following a fire. Moreover, he was already ensured compensation based on the value of the property as a plot for housing, for independent reasons.}

The general picture, however, was that a no-scheme rule applied to underlying regulation of property use as long as there was a causal link between this regulation and the subsequent expropriation.\footnote{References. \noo{Husaas-komiteen}} To apply this in concrete cases often proved problematic, however, as illustrated by the many conflicting opinions voiced about the current law during the preparation of the original Compensation Act from 1973.\footnote{See, for instance, the historical overview given in NOU 2003:29}

In the 1973 Act, a radical rule was put in place to resolve all outstanding issues: valuation should be based on the  \emph{existing use} of the property at the time of expropriation.\footnote{So the Norwegian law mirrored the rule introduced in the UK Town and Country Planning Act 1947 which was later replaced by the current Land Compensation Act 1961.} The rule went further than the no-scheme rule in that it prescribed that compensation should disregard \emph{all} kinds of hypothetical development of the property, notwithstanding their status with respect to existing plans and regulations. But it also involved a break with it, since, on the face of it, the implication would be that any kind of regulation predating the scheme would be taken into account when it provided the basis for the "existing use".

However, in Section 4, nr. 3 of the Act, this aspect of the "existing use" principle was limited by a \emph{separate} provision implementing the \emph{negative} aspect of the no-scheme rule; the value of existing use due to public regulation underlying the expropriation should be \emph{deducted} from the compensation payment. As such, the Compensation Act 1973 implemented a system whereby the positive part of the no-scheme rule would be given a very narrow interpretation -- any scheme or regulation limiting current use was to be be taken into account -- while the negative aspect was explicitly provided for in statute -- the value of existing use that could be attributed to public regulation underlying the scheme was to be deducted.

This new rule was quite controversial, and to make the system more flexible, the 1973 Act included a rule that allowed the Lands Tribunal to increase compensation, on a discretionary basis, by taking into account that value of comparable properties in the district where expropriation took place. Still, property owners felt that the new Act went too far in depriving them of the right to compensation, and the matter came before the Supreme Court in \emph{Kløfta}.\footnote{Rt. 1976 s. 1} After deliberating in plenum, the Court presented a revisionary interpretation of the new Act, essentially judging the intention behind the main rule as being incompatible with the protection of property encoded in the Norwegian Constitution, Section 105. The main step taken by the Court was to regard the discretionary increase of compensation as a \emph{mandatory} step, one that had to be carried out whenever certain conditions were fulfilled. What exactly these conditions amounted to, and how they should be interpreted, was not conclusively resolved, however. Indeed, the confusion that arose after \emph{Kløfta} led to a heated academic debate in Norway, and a long line of Supreme Court cases has since attempted to clarify the current state of the law. \footnote{References.}

In 1984, taking into account the ruling in \emph{Kløfta}, a new Compensation Act was passed, which is still in force today.\footnote{Act No. 17 of 06. April 1984 relating to Compensation following Expropriation of Real Property} According to Section 4 of the Compensation Act, compensation is to be calculated as the highest of either the value of the property as it could be put to use by the owner, his \emph{value of use}, or the \emph{market value}, the payment he could expect to receive from a typical willing buyer. In both cases, a no-scheme rule applies: in Section 5, Paragraph 3, it is stated that when calculating the market value, changes in value due to the scheme is to be disregarded, while in Section 6, it is stated that the value of use should be based on \emph{foreseeable} use of the property. In practice, this has been interpreted as referring to such use as it is reasonable to expect would have occurred in the absence of the scheme.\footnote{References.} 

However, the spirit of the 1973 Act is still clearly felt in Norwegian law, and the no-scheme rule is thought of somewhat differently than in the UK. Firstly, it is common to distinguish more sharply between the positive and the negative aspects of the rule, and unlike in the UK, much, if not most, attention has been devoted to the former aspect, when the scheme is used to justify decreased levels of compensation. Secondly, and specifically as it relates to the positive aspect of the rule, the tendency has been to give "the scheme" a narrow interpretation, regarding public plans and regulation as binding for valuation, even when they are intimately related to the undertaking for which a right to expropriate is granted. Simultaneously, if the plan leads to an increased value that \emph{can not be realized by the current owner}, a "value to the owner" principle typically applies directly, such that this value is not compensated, irrespectively of what the scheme is taken to be.\footnote{References.}

As we mentioned, much attention in Norway has been directed at the positive aspect of the no-scheme rule, and there are some important exceptions to the main principle of regarding public plans as binding for the evaluation. The main exception is that a plan tends to be disregarded when it has no other purpose than to facilitate the scheme for which expropriation is needed. The important precedent in this regard is the influential Supreme Court case of \emph{Lena}.\footnote{Rt. 1996 s. 521.} However, the line of reasoning adopted by the Supreme Court in this case has so far been called on almost exclusively in cases when compulsory acquisition takes place to implement plans for public buildings or other kinds of public installations, like playgrounds or parking spaces.

In case of expropriation taking place to implement such plans, if a valuation were to be made on the basis of the use prescribed by the plan itself, one would expect the market value of the property to come out as nil or close to nil, such that one would be forced to return to existing use as the basis for valuation.\footnote{In fact, a logical continuation of the line of reasoning prescribed by the main rule would suggest that even existing use would be inadmissible as a basis for compensation since continuation of this use would not be in accordance with the plan, and hence unforeseeable. However, this has not, as far as we are aware, ever been argued.} It is not hard to understand how this could come to be felt as unfair to property owners. Moreover, it seems that it would easily come to run counter to Section 105 of the Norwegian Constitution. In some sense, it would represent a watering down of this provision, allowing the State to deprive property owners of value by using unfavorable regulation as an explicit means to later acquire properties cheaply by use of expropriation.

On the other hand, it can be argued that the question raised by cases such as these is not really a question of compensation for expropriation, but rather a question of whether or not property owners should have a claim of compensation for losses incurred due to \emph{regulation}. This is not generally granted under Norwegian law, and so the special rules that entitles the owner to such compensation in cases of regulation leading to expropriation can be seen as insufficiently justified within the broader context of Norwegian planning law. Indeed, some scholars have voiced this opinion forcefully, and the current state of the law is unclear at best, with the special rule introduced in \emph{Lena} being hard to apply to other cases, and giving rise to further Supreme Court decisions attempting to map out in more detail when exactly they come into play. Moreover, attempts to reform the law on this point have so far stranded in controversy.\footnote{NOU 2003:29 and further references.}

It seems to us, however, that while this debate is interesting, the truly fundamental questions about the current state of Norwegian law do not arise in this regard, but with regards to the other principle we mentioned, namely that no compensation is offered for value that \emph{the owner can not realize}.\footnote{By "realize" here, we mean realizable either as the owner's "value of use" or else realizable by selling the property at "market value", as prescribed by the Compensation Act.} This rule seemingly applies without reservation, and, as in the UK context, it appears like a logical consequence of the \emph{value to the owner} principle. However, as we argued for in the general discussion on the no-scheme rule in Section \ref{sec:noscheme}, it seems pertinent to distinguish between the \emph{subjective} and \emph{objective} aspect of this principle. In particular, if the owner can not realize the value because the value is not of a \emph{kind} that is available for commercial realization, for instance because it only represents non-commercial value to the public, then the rule appears easy to defend along traditional lines. On the other hand, if the owner can not realize the value because the State desires that \emph{someone else} be allowed to realize it, then the principle appears highly problematic.

While the general debate on Norwegian compensation law has completely neglected to consider this aspect, it features extensively, albeit implicitly, in one particular branch, namely the law relating to compensation for waterfalls. In the remainder of this paper, we turn to this particular area of Norwegian law, offering a detailed analysis of the problems that arise, and how they severely challenge the traditional "value to the owner" reasoning about compensation for commercial undertakings. 

\section{The traditional method for compensating waterfalls}\label{sec:trad}

In the early 1900s, Norwegian hydro-power was not subjected to much regulation, and waterfalls, having recently been discovered as an important supply of cheap electricity for industrial exploits, were rapidly falling into the hand of foreign speculators. In response to this, Norwegian politicians introduced legislation to secure national interests, the main provision being that concession from the state was made a requirement for anyone who wanted to acquire a waterfall.\footnote{References.} As a result, the market for waterfalls in Norway dwindled and the State assumed control of hydro-power exploitation. Unlike private investors, the State would tend to expropriate waterfalls rather than acquire them through voluntary agreements, and the question arose as to how the original owner should be compensated. This question, if resolved by a standard no-scheme approach, could easily prove shockingly unfair to owners of waterfalls. Presumably, since waterfalls could not be exploited for any significant commercial gain except through hydro-power exploitation, disregarding the hydro-power scheme when calculating compensation could lead to nil or close to nil being awarded to the owner. But this was not seen as an acceptable outcome, and instead the Norwegian courts introduced a special method to compensate waterfalls that gave the owner a \emph{share in the value of the hydro-power scheme} for which expropriation was taking place.

Norway did not at this time have any legislation specifically aimed at regulating compensation following expropriation, and when formulating the special rules for compensation of waterfalls, the Norwegian courts seems to have relied on an analogical application of the gross valuation techniques introduced in the Industrial Concession Act 1917 and the Watercourse Regulation Act 1917.\footnote{Act No. 17 of 14 December 1917 relating to Regulations of Watercourses and Act No. 16 of 14 December 1917 relating to Acquisition of Waterfalls, Mines and other Real Property}. Neither of these acts were aimed at compensating owners, but they relied on methods for assessing the potential and significance of hydro-power projects with respect to the question of whether or not a special concession from the State was required.\footnote{To acquire the waterfall and the right to regulate the water-flow respectively.} In effect, by relying on the methods of valuation introduced there, the compensation mechanism that was introduced deviated completely from the "value to the owner" principle. On the other hand, it also closely mimicked the manner in which owners of waterfalls would be compensated on the market in the early days, prior to the introduction of our concession laws, when speculators would pay for waterfalls on the basis of what they assumed to get out of them in subsequent hydro-power projects.\footnote{References.}

In the Supreme Court case of \emph{Hellandsfoss}, some 80 years after it was first introduced, the traditional method for compensation was still in use, and the Court described it as follows, starting from the observation that the general principles that were later encoded in the Compensation Act 1984 were of little use for determining the right level of compensation for waterfalls (my translation).\footnote{Rt. 1997 s. 1594.} 
\begin{quote}
The principle set out in the Compensation Act, Section 5, is that compensation should be determined on the basis of an estimation of what ordinary buyers would pay for the property in a voluntary sale, taking into account such use of the property as could reasonably be anticipated. For waterfalls, however, this often offers little guidance, and the value of waterfall rights have traditionally been determined based on the number of natural horsepowers in the fall, which are then multiplied by a price per unit. The unit price is determined after an overall assessment of the waterfall, including the cost of the scheme, its location, and levels of compensation paid for similar types of waterfalls in the past. The number of natural horsepowers is calculated by the formula "natural horsepower = $13.33 \ \times \ Qreg \ \times \ H$", where $Qreg$ is the regulated water flow and $Hbr$ is the height of the waterfall.
\end{quote}

In this formula, $Qreg$ represents a quantity of water, measured in cubic metres per second (m3/sec), while $H$ is the height of the waterfall measured in meters. Horsepower is an old-fashioned measure of effect, and in the standard account of the traditional method, it is said that the number of natural horsepowers in a waterfall is a measure of gross effect in the waterfall, giving us the amount of “raw” water-power in the waterfall.\footnote{See \cite{Falk}(in Norwegian)} This, however, is flat out false for most waterfalls, and it has always been more accurate to regard the number of natural horsepowers as a measure of the \emph{level of regulation} involved in a given planned hydro-power project. This, indeed, is what the concept was actually introduced to measure, and it is how it is used in the Industrial Concession Act 1917 and the Watercourse Regulation Act 1917. Historically, however, it made  sense to conflate the energy-potential of the waterfall with the level of planned regulation, since regulation of water-flow used to be crucial for the development of efficient hydro-power generation. This has changed, however, and the traditional method, when applied as a tool to assess the energy-potential of a waterfall, is horribly outdated. In the following subsection, we give a detailed presentation showing this.

\subsection{Not so natural: The physics and the law behind the notion of a "natural horsepower"}\label{subsec:notnat}

Let us first remark that horsepower is no longer used as a measure of effect in the energy business. Today, it is general practice to use kilowatts (kW) instead, at least as long as there are not any lawyers present.\footnote{1 Kilowatt(Kw) = 1000 Watt(W)} In the following, we will give the reader quite a detailed presentation of the concept of effect and the link between the two units horsepower and kilowatt. We shall try to give a rudimentary explanation of the physical facts which underlie it, leading to the conclusion that the number of natural horsepowers in a waterfall no longer has any relevance to its value in hydro-power production.

\noo{
8    “Erstatning for erverv av fallretter” (Compensation for Acquisition of Waterfall Rights) by Ulf 
        Larsen Karoline Lund and Stein Erik Stinesen in Tidsskrift for Eiendomsrett (Journal of Property 
         Righs) Nu 4 2006
9      See paragraph 3.5 in Larsen/Lund/Stinson (above)
10    See for instance p 262 in “Vassdrag og Energirett” (Law of Waterfalls and Energy) by
        Falkanger/Haagensen (Ed.) 2002. Referred to as “type” because there are several definitions of Q,
        see paragraph 1.4.
}

The notion of an \emph{Effect} ($E$) is defined in physics by means of the more elementary concepts of \emph{Work} ($Wr$), \emph{Time} ($t$), \emph{Distance} ($d$) and \emph{Force} ($F$). The relationship between them, in particular, satisfies the following equation.

\begin{equation}\label{eq:effect}
E = Wr/t = (F \times d)/t
\end{equation}

The formal notion of Work ($Wr$), in turn, is defined as follows.

\begin{equation}\label{eq:work}
Wr = F \times d
\end{equation}

The last non-trivial concept needed to define effect is the notion of Force ($F$), which is fundamental in physics. It can be explained by what is needed to change the speed of an object, or cause it to move, c.f. the First Law of Newton. The unit most commonly used to denote this is \emph{Newton}. One Newton is defined as the force which is needed to increase the speed of a mass of 1 Kilogram (Kg) by 1 Meter (m) per sec in one second. 

It is the force of gravity which is harnessed to produce an electric effect, and in turn, to harness the power of water. The force of gravity is $9,81$ Newton per Kg mass. That is, if 1 Kg falls freely to the ground it will accelerate by $9,81 \ m$ per sec in a second (in reality somewhat less because of air resistance). The power of gravity works continuously. Therefore, the mass will accelerate by $9,81$ Newton as long as it is falling. This means that for every second the mass is falling the speed will increase by $9,81 \ m/sec$.

Historically, the unit of force was often defined as the force needed to pull 1 Kg to the ground. This force was often called 1 “Kilo” and named “Kilopond” (Kp) to distinguish it from the concept of mass which was also, in everyday usage, often referred to as “Kilo” (but should actually be called "Kilogram”). When Newton is used as a unit for force, the corresponding unit of work becomes Newton $\times$ meter ($Nm$), also called Joule ($J$). Looking to the definition of work in Equation \ref{eq:work}, we can then make the following calculation: When a mass of 1 Kg falls 1 Meter, the work done amounts to $9,81 \ Nm$. Similarly, when Kilopond is used as a unit for force, you get Kilopond $\times$ meter ($Kpm$) as a unit for work. The work which is done when 1 Kg mass is falling 1 meter can then be described by the following formula:

\begin{equation}\label{eq:work}
Wr = 1 \ Kp \times 1 \ M = 1 \ Kpm 
\end{equation}

When we look at formula \ref{eq:effect} for effect, if we use Newton as the unit of force, we get \emph{Watt} ($W$) as the corresponding unit of effect, so that we get an effect of $9,81 \ Nm/Sec$, or $9,81 \ W$, when 1 Kg mass falls $1$ meter in a second. Similarly, if we use $Kp$ as the unit of force, we get an Effect of $1 \ Kpm/Sec$. Somewhat curiously, $1 \ Kpm/Sec$ does not equal $1$ Horsepower ($Hp$). Instead, the choice was made to define horsepower as follows
\begin{equation}\label{eq:hp}
1 \ Hp = 75 \ Kpm/Sec
\end{equation}
Consequently, we get the following relation between $Hp$ and $W$ as a measure of effect:

\begin{equation}\label{eq:hpw}
1 \ Hp = 75 \ Kpm /Sec = 9,81 \times 75 \ W = 735,75 \ W = 0,736 \ kW
\end{equation}

This might seem like an unduly technical exercise for a law paper. However, it is needed to understand that there is nothing magical about a Natural Horsepower ($nat.Hp$), and that horsepower, which is no longer used by the energy business as a measure of effect (or the general public, save for car enthusiasts), can easily be replaced by Watts. This, indeed would make more sense in this day and age, and open the method up to be more readily scrutinized. Indeed, moving from a quantity of $x \ nat.Hp$ to the same amount of effect, measured in nat.kW is easy; the latter is obtained from the former when multiplying by $0.736$, i.e., such that 
\begin{equation}\label{eq:natkw}
x \ nat.HP = 0.736 \times x \ nat.kW
\end{equation}

We are in a position to properly explain the formula for natural horsepowers in a waterfall. It measures the effect in the waterfall in Hp, given certain information and certain assumptions about the features of the waterfall under consideration. When we go through this in detail, however, we come to realize that the horsepowers arrived at are not so natural after all, since they are based on assumptions that are largely irrelevant for modern hydro-electric schemes, and that have been upheld in law seemingly for no other reason than the fact that they have been habitually used by lawyers and judges with no regards to, or understanding of, their underlying \emph{meaning}.

In light of what we have already seen, it is now easy to explain the constant factor of "13.33", used in the formula for natural horsepowers: Since one cubic meter ($m3$) of water ($1000 \ l$) has a mass of $1000 \ kg$, it is pulled towards the center of the earth at a force of $1000 \ Kp$. If we lift, or allow to fall, $1 \ m3$ of water by $1 m$ in one second, we work with an effect, or, in the case of falling, \emph{release} an effect, of $1000 \ Kpm /sec$. From the formulas in Equations (\ref{eq:effect}-\ref{eq:hp}), we then see that we work with, or release, an effect which measured in Hp amounts to the following.

\begin{equation}\label{eq:whp}
1000 \ Kpm /Sec : (75 \ Kpm /Sec) /Hp = 13,33 \ Hp
\end{equation}

That is, $13,33$ is simply the effect, measured in $Hp$, of $1 m3$ of water falling $1 m$ in $1 sec$. So if we have an amount of water measured in $Q \ m3$, we must then multiply $13,33$ by $Q$ to get the effect of this amount of water falling $1 \ m/sec$. Then, if we have a waterfall that is $H$ meters high, we can multiply by $H$ to find the effect of $Q \ m3$ of water falling $H m$ in $1 sec$. If the amount of water flowing through a river is Q \ m3/sec, it means that we have the amount of water $Q$ available every second. This will in turn ensure that as long as the water supply continues to be $Q \ m3/sec$ -- that is, as long as the water-supply is \emph{constant} -- the effect of the work being done by the water will also be constant and it will be given by the formula $13,33 \ x \ Q \ x \ H$.

Energy is defined as the capacity to do work. There are several units for energy, but when energy takes the form of electricity, such as in a hydro-power plant, it is standard to use the unit kWh (kW $\times$ Hour). Notice that this is consistent with the definition of work as $Wr = E \times t$ ("Energy times Time"). If you have a certain effect available over time, the amount of energy you acquire is measured by multiplying the effect by the amount of time that the effect is operative. When mechanical energy is transformed into electrical energy in a power station, the effect of the generator is multiplied by the time the generator is operative with the same effect. To get the result in $kWh$, all you must do is to ensure that you measure your time in hours and your effect in kilowatts, and then multiply the two together. 

However, in practice, the effect harnessed in a hydro-power station changes when there are changed in the water flow, so you get the true amount of energy produced only when you make this calculation sufficiently often, by multiplying a given effect with the number of hours (maybe just minutes) for which the station is operative at this particular effect. The sum of these chunks of energy you get over the year will be the amount of annual production, and this, today, is what the energy business use as a yardstick when \emph{they} assess the value of a waterfall. 

With modern technology, the energy output is registered by fine-tuned electrical equipment, maybe every 15 minutes or so, and hence the annual amount of energy generated can easily be registered and measured, even if there are significant fluctuations in the available water, leading to the generator operating at different levels of effect. This is a significant observation, since it means that the assumption inherent in the natural horsepower formula, namely that the water-flow, and hence the effect, remains constant, is no longer tenable, and gives a completely erroneous account of the energy-potential in a waterfall. 

Certainly, the amount of energy generated in a power plant could be measured in other units than kWh, e.g. in terms of the amount of horsepower-hours per year. However, an energy producer gets paid for the amount of energy he can deliver, \emph{not} the effect he can maintain in his station constantly, and as a result it simply would not correspond to reality if one would attempt to measure the energy by \emph{natural} horsepower-hours. This would only be correct if the owner of the hydro-power plant \emph{chose} to produce electricity at a constant effect all year round, which he would never do.\footnote{In addition, and pulling somewhat in the opposite direction, come the fact that it is not a realizable effect that is derived from the formula of natural horsepower, but only a gross estimate. The effect that we find is calculated based on an assumption of ideal circumstances, i.e. without any loss of energy in the production step. In reality, there will always be some loss of energy both in the pipes and in the turbine/generator. That is the reason why natural horsepower is often described as the “raw” or as the “gross effect”. But a far more important mechanism is the gross simplifications involved in moving from the physical fact that the flow of water in a river varies quite a lot during a year, to one fixed amount, Qreg, assumed to be available constantly.}
This makes it practically meaningless to talk about the number of natural horsepower in a waterfall as a measure of its potential. Effect is not something we have in a waterfall, but something we get when a certain amount of water is falling. 

From this observation also follows that the amount of water, $Q = Qreg$, that is to be put to use in the formula for natural horsepower must be chosen by an application of \emph{law}. Indeed, what to choose for $Qreg$ is a legal question of great significance, and as long as $Qreg$ is conceived of as a constant, as in the traditional method, it measure the degree of \emph{regulation of water-flow}, not the potential for energy generation.

So how is $Qreg$ typically defined? This is actually a very tricky question, although apparently, the amount of water to be used in the calculation is actually prescribed by statute. In practice, however, the statutory definition can lead to such offensive results, when applied in the context of compensation, that the courts, or, as it were, the experts presenting these calculations on behalf of the expropriating party, usually adopts a \emph{different} definition. 

The Watercourse Regulation Act 1917, Section 2 reads as follows.

\begin{quote}
Section 2:  Waterfall regulation for the production of electric energy which increases hydro power:
\begin{itemize}
\item[a)]            by at least 500 natural horsepower in one or several waterfalls which can be developed 
                 collectedly, or

\item [b)]             by at least 3.000 natural horsepowers in the whole watercourse, or

\item [c)]              which alone or together with earlier regulations significantly affects the environmental
                 conditions or other public interests, can only be exploited by the State or  
               a developer who obtains permission from the King.

\end{itemize}

If regulation of a watercourse increases the water-power in the river by at least 20.000 natural horsepowers, or if there are essential conflicting interests, then the case should be submitted to Parliament before license is given, unless the Department finds it unnecessary.

The increase of the hydro-power according to the first and second point is calculated on the basis of the increase of the low water-flow of the watercourse, which the regulation is supposed to cause beyond the water-flow which is considered foreseeable for 350 days a year. When making the calculation it is to be assumed that the regulation is operated in such a way that the water-flow during the low water periods becomes as even and regular as possible.  
\end{quote}

In the third paragraph, the definition of $Qreg$ is provided, when it states that the \emph{increase of the hydro-power}, measured in natural horsepower, is to be calculated based on the water-flow which it is foreseeable that will be present for at least 350 days a year. That is, $Qreg$ is to be taken as the maximum amount of water that one can expect to be present for at least 350 days of the year after regulation minus the water that could be expected for 350 days without regulation, which is the quantity referred to as the \emph{low water-flow}.

Regulation of a watercourse can involve building a reservoir and/or installations that transfer water from one river to another. Then, if there is excess water, for instance due to flooding, water can be stored in the dam for later use, while if there is drought, the stored water can be released. In this way it becomes possible to even out the water-flow over the year. This again means that the water which is guaranteed to be present for at least 350 days a year will typically increase. In light of the definition of $Qreg$, it is clear that the definition of natural horsepower depends crucially on the level of regulation involved in the planned hydro-power project. But the definition only takes into account the gross effect resulting from the \emph{increase} in low water-flow following regulation. It follows that if the planned project does not involve regulation, which is common today, especially for small-scale hydro-power, the number $Qreg$ will by necessity be $0$ and the waterfall will be deemed not to posses any natural horsepowers at all.

In fact, if the traditional method for calculating compensation had remained true to the wording of the Watercourse Regulation Act, things could sometimes have been even worse for the owner. This we notice, in particular, when we  consult Section 10 of the Water Resources Act 2000.\footnote{Act No. 82 of 24 November 2000 relating to River Systems and Groundwater} Here, the NVE is given the power to compel the owner of a hydro-power scheme to ensure that a certain quantity of water is always allowed to pass through the intake of the hydro-power plant. Moreover, there is nothing to prevent the NVE from demanding that this minimum water-flow is set \emph{higher} than the low water-flow, and this, indeed, is often the case, especially in cases when the low water-flow only amounts to a small fraction of the average water-flow, and environmental concerns arise with respect to wildlife and fisheries. Then, indeed, it appears that the minimum water-flow, required to be left untouched, should be subtracted from the regulated water-flow when calculating $Qreg$.\footnote{In fact, this was done in the case of \emph{Sauda}, LG-2007-176723 (Gulating Lagmannsrett, a regional Court of Appeal} Intuitively, this even appear reasonable; The minimum water-flow can not, as a matter of fact, be harnessed for energy production.

However, for hydro-power projects that do not involve regulation, this would then lead to the regulated low water-flow being \emph{less than} the low water-flow. Subtracting then the latter from the first, as required by Section 2 of the Watercourse Regulation Act 2000, would lead to a \emph{negative} number for $Qreg$, and a corresponding \emph{negative} number of natural horsepowers attributed to the waterfall. Logically speaking, then, the traditional method would entail awarding a negative sum as compensation, compelling the owner to pay the expropriating party for taking over his waterfall!

In practice, of course, the traditional method has never been applied in this way, and more generally, the definition in Section 2 of the Watercourse Regulation Act has tended to be completely disregarded by valuers using the traditional method for calculating compensation. Instead, the definition has been changed for this purpose, such that the low water-flow prior to regulation is not deducted from the low water-flow after regulation.

Even after this modification, the number of natural horsepowers give a drastically skewed picture of the potential of the waterfall, especially for projects that do not involve regulation. It is not unusual, in particular, especially not for waterfalls suitable for small-scale hydro-power, that the low water-flow amounts to only about 3-5 \% of the average water-supply. In modern hydro-power projects, one would expect 70-80 \% of this water-flow to be harnessed for energy production even in the absence of any regulation. So in these cases, the traditional method of compensation is effectively based on compensating the owner for only a small fraction of the energy that can actually be extracted from his waterfall.

This observation, which follows from elementary facts about physics and contemporary hydro-power production, was not noted or discussed in connection with the principles used for compensation before the early 2000s, when the issue was raised following the growing interest in small-scale hydro-power. However, the Norwegian government has certainly been aware of these facts, as illustrated for instance by the following passage, taken from a report presented to parliament in 1991-1992.\footnote{Ot.prp. No 50 (1991-1992) p 19, discussing the notion of natural horsepower in connection to the uses made of that term in other parts of the law.}

\begin{quote}
The Department of Oil and Energy have considered moving a proposition for changing the hydrological definitions in the Industrial Concession Act 1917 and the Watercourse Regulation Act 1917. Today the act uses a calculation method based on an increase in regulated water-flow, i.e. that of natural horsepower.[.......] The hydrological definitions of these acts, supposed to indicate how much electricity can be generated, were made on the basis of technical and operative conditions differing very much from contemporary circumstances. In implementing the definitions referred to above one has tried to adapt to the new technological realities of the present day. Therefore, in practice, a calculation based on current production is used instead. From several quarters, particularly the Association of Waterfall Regulators, there has been raised a strong wish to authorize this practice by altering the definitions of the relevant laws. The Department of Oil and Energy agree, but have not as yet made a sufficient elucidation of the issues to be able to move a proposition of alteration of these acts.
\end{quote}

Within the ranks of the water authorities, it has actually been well-known for decades that the notion of a natural horsepower fails to give an adequate picture of the potential that a waterfall has for hydro-power. The development of new technology had made this apparent already in the 1950's, when it was also raised as an issue, specifically with respect to compensation following expropriation, by a director at the NVE, who commented, in 1957, that he failed to see how the traditional method could be an adequate means for valuating waterfalls.\footnote{See \cite{....}. The director even went as far as to illustrate a different method, which would also be outdated given today's regulatory regime, but which would reflect contemporary \emph{actual} valuations, used by the NVE itself.}

Considering the physics behind the traditional method is enough to reveal that it fails to give rise to valuations that reflect the value of waterfalls, under any reasonable set of assumptions about the correct general compensation principles one should adopt. Firstly, the traditional method, by relying on data from the expropriating party's project, deviates from the "value to the owner" principle. Secondly, and even more importantly, it amounts to compensating the owner based only on the level of regulation, which is not only mostly irrelevant to the value, but is also the one aspect of the scheme which can not readily be traced to properties of the waterfall, but depends rather crucial on the investment decisions made by the expropriating party. While a case can be made that any extra power harnessed by regulation should \emph{also} be compensated, for instance if it can be established that the expropriating party is not the only one who could have regulated the waterfall, it seems rather perverse to \emph{only} compensate the owner based on this parameter.

However, while the idea of compensating the owner of waterfalls by a price per natural horsepower is fundamentally flawed already at the level of physics, there are even more serious concerns that arise when one begins to consider the way in which the \emph{unit price} has been determined, and the effects this has had on the level of compensation payments. In case-law based on the traditional method, it is often said that the price set per natural horsepower is set according to "market price" for waterfalls, but for the most part, what this means is that the court looks to prices awarded in earlier compensation cases, not to prices obtained in voluntary sales.

This, in turn, gives rise to a price level that is entirely artificial, reflecting, more than anything else, the power balance between buyer and seller in the courtroom, and not any genuine market value. Indeed, while the prices paid did see some increase during the 80 years that the traditional method was used, this hardly reflected the general increase in value of hydro-power, nor did it reflect the general level of inflation.\footnote{References needed.} Moreover, and particularly worrying, while the price-level was determined by the courts, there were also some cases of voluntary agreements that used the same method, and could thus be used to justify is status as a genuine market-based valuation principle. In fact, as late as in 2002, a waterfall belonging to local landowners in the rural community of Måren, in Western Norway, was sold for the sum of kr 45 000 (roughly £ 5000), based on traditional calculations. The waterfall has now been exploited in a small-scale hydro-power plant belonging to the large energy company BKK, with annual energy output of 21 GWh.\footnote{$http://www.bkk.no/om_oss/anlegg-utbygging/Kraftverk_og_vassdrag/andre-vassdrag/article29899.ece$} For comparison, we mention that in the case of \emph{Sauda}, based on the new method for calculating compensation, the owners received a compensation which totaled about 1 kr/kWh annual production.\footnote{LG-2007-176723 (I acted as council for some of the owners in this case).} Applied to the Måren case, this would take the compensation from kr 45 000 to kr 21 000 000, that is, almost 500 times more.\footnote{In fact, the Måren waterfalls were cheaper to exploit, so in reality, one would expect that the new method applied to Måren would yield even greater compensation per kWh. We also remark that the value awarded in \emph{Sauda} was market-value, not value of use, since it was assumed that the owners would have to cooperate with a so-called "professional" energy company to develop hydro-power. This, in effect, halved the compensation awarded.}

The case of Måren is somewhat extreme, but in no way unique.\footnote{I should assemble a list probably....} Moreover, it illustrates an important point, namely that when the traditional method was used, and described as the "market value" of waterfalls by the courts, this became a self-fulfilling prophecy in many cases. The prices paid in voluntary transactions were influences by the practice adopted by the courts far more than the other way around. This, indeed, appears to be a general danger in cases when expropriation is widely used for the purpose of commercial development. Then, it seems, prices paid can easily be kept artificially low by developers turning to the use of expropriation as soon as they threaten to rise, and relying on the "market value" thus established when arguing in court for the appropriateness of those compensation levels that so benefits them commercially. That this mechanism can be severe is nicely illustrated by the case of Norwegian waterfalls, and how to prevent it is, in our opinion, a main challenge that is likely to arise in any regulatory system that aims to make extensive use of expropriation to further economic development.

\section{The new method}\label{sec:new}

Following the liberalization of the Norwegian energy sector in the 1990s, the traditional method came under increasing pressure. It was argued to be unjust by owners who felt that they were being deprived of a valuable commercial assets, and it was held to be illogical by engineers working on developing small-scale hydro-power.\footnote{References} Eventually, legal professionals followed suit, and came to the realization that established rules based on market value could now be applied. Indeed, a new market for waterfalls had begun to develop at this point, following the increased interest in small-scale hydro-power and the formation of new companies specializing in cooperating with local owners. For transactions of rights to waterfalls taking place in this market, the traditional method of valuation was not used, and waterfalls were rarely sold at all, but rather leased to the development company for an annual fee. Typically, this fee was calculated by fixing a percentage of the energy produced during the year, and compensating the owners of the waterfall by multiplying this with the market price for electricity obtained throughout the year, possibly deducting production specific taxes, but with no deduction of other cost. In effect, owners would get a fee corresponding to a set percentage of annual gross income in the hydro-power plant.\footnote{References}

Usually, the fee entitles the owners to 10-20 \% of the income from sale of electricity, depending on the cost of the project. Moreover, it is common that the owners are entitled to up to 50 \% of the income derived from so-called \emph{green certificates}, a support mechanism for new renewable energy projects, corresponding to the Renewables Obligation in the UK.\footnote{See http://www.ofgem.gov.uk/Sustainability/Environment/RenewablObl/ for further details.} Essentially, and somewhat simplified, the scheme allows the energy producer to collect a premium on his sale of electricity, which, owning to its "green" status, is valued more highly by buyers (usually electricity suppliers), who are required to ensure that a certain proportion of the energy they offer to their customers (usually consumers, like you and me) is considered green. In Norway, such a scheme has been talked about for years, but was only put in force in 2012.\footnote{http://www.regjeringen.no/en/dep/oed/Subject/energy-in-norway/electricity-certificates.html?id=517462} Currently, energy producers can claim a premium of about 2 pp per KWh per year, meaning that about a third of the annual income for new renewable energy projects comes from the sale of green certificates.\footnote{While the premium must be expected to go down somewhat as the certificate market matures and more energy producers acquire "green" status, it will certainly remain an important source of extra income for renewable energy producers also in the future.}

In light of the fact that the agreements on the new market are based on leases that tie compensation to the fate of the particular hydro-power project that is being undertaken, several questions arise when attempting to value waterfalls by looking to this market. If a given project has been identified as providing the basis for valuation, the task is difficult, but mainly a question of factual assessment. The valuer have to determine first what the annual production would be and also determine the costs of carrying out the project. Then, on this basis, he must move on to determine what the annual fee would be, and then, in order to complete the process, he must stipulate what the price of electricity, and of green certificates, is likely to be for the next 20 years or so. On the basis of this information, it becomes possible to determine the annual income to the owner of the waterfall over a period of 20 years, and then one would also have a basis upon which to calculate a reasonable present-day value of the scheme to the owner of the waterfall. 

Indeed, this is the model that has been used in the cases that have been before the courts and where the traditional method has not been adopted. The first such case was \emph{Møllen}, and while the Supreme Court rejected the method as it was applied in this case, because it was found that the date of valuation was to be based on prices obtained for waterfalls in the 1960s, they commented that they supported the adoption of the new method in cases when \emph{alternative} small-scale development was deemed a \emph{foreseeable} use of the waterfall in the absence of the scheme.\footnote{Rt. 2008 s. 82.} 
Since \emph{Møllen} the new method has been used in many cases before the lower courts and the Lands Tribunal.\footnote{See for instance \cite{tf1,tf2,tf3}, a series of academic papers discussing the new method (in Norwegian).}

Unsurprisingly, the new method tends to lead to a rather protracted process of valuation, mostly dominated by experts. Moreover, given all the uncertain elements of the calculation, it is typical that the opposing parties produce expert witnesses that diverge significantly in their valuations. While this is problematic enough, the fundamental \emph{legal} challenge arises with respect to the choice made about what scheme the compensation should be based on. This becomes especially tricky if one attempts to follow a standard no-scheme approach. In the following, we summarize the main issues that arise.

\begin{enumerate}
\item In the absence of the hydro-power scheme benefiting from expropriation, is it foreseeable that the waterfall would nevertheless be used in a hydro-power project?
\item If the answer to Question (1) is yes, what would such a foreseeable project look like?
\item Is it foreseeable that an alternative project would get planning permission?
\item Does the no-scheme rule imply that the project benefiting from expropriation cannot be regarded as the foreseeable use for the purpose of compensation?
\item Can the fact that the scheme underlying expropriation obtained planning permission be taken as evidence to support that alternative uses of the waterfall would not be given planning permission?
\item How should compensation be calculated if it is determined that no alternative project is foreseeable? 
\end{enumerate}

In some cases, for instance when the project benefiting from expropriation is not commercially viable but is carried out for public purposes with the help of special State funding, the answer to Question (1) might be no. However in most cases, the question will be answered in the affirmative, since the scheme benefiting from expropriation already serves as an indication that the waterfall can be harnessed for energy. However, here the no-scheme rule comes into play and creates severe difficulty once we reach Question (2). For what kind of scheme can be assumed foreseeable all the while we are obliged to disregard the scheme underlying expropriation? In most cases that have come before the courts so far, the owners have claimed that alternative development in small-scale hydro-power should serve as the basis for compensation, and such cases the problem of the no-scheme rule appears to have been circumvented. This, however, is not necessarily the case. It appears, in particular, that the answer to Question (3), asking about the likelihood of planning permission, might again depend on how you view the no-scheme rule. It seems, in particular, that anyone who answers Question (5) in the affirmative, looking to the planning permission actually given to the expropriating scheme for evidence, might be inclined to say that the alternative project could not expect to get planning permission, and that this is so \emph{because} planning permission was granted to the expropriating party (or this line of reasoning might be sugarcoated by pointing to whatever underlying reasons the authorities had for considering the scheme underlying expropriation the optimal use of the waterfall).

Then the question arises: Is someone who reasons like this at odds with the no-scheme rule? It would seem so, but remember the earlier discussion on the no-scheme rule in Norwegian law, where we noted that the rule has tended to be applied much more narrowly along its positive dimension. Following up on this, it can be argued, in keeping with the general tendency in how the law is applied in Norway, that while the expropriation scheme is to be disregarded for the purpose of compensation valuation, the regulation underlying the scheme -- or at least the rationale behind this regulation -- is nevertheless to be taken into account. If this point of view is taken, then the conclusion can easily become that alternative development is to be regarded as unforeseeable, and that the reason why this must be the case is precisely the fact that the expropriation scheme received planning permission. Indeed, this line of reasoning was given a stamp of approval in the recent Supreme Court case of \emph{Otra II}. Here, the presiding judge made the following remarks, quoting Gulating Lagmannsrett (the regional Court of Appeal), expressing his support, and adding a few comments of his own.

\begin{quote}
"[....] The Court of Appeal finds it difficult to distinguish this case from other cases when it has been established that alternative development is not foreseeable. It does not seem relevant whether this is the case because the alternative is not commercially viable or because the alternative must yield to a different exploitation of the waterfall" 
I agree with the Court of Appeal, and I would like to add the following: As the survey of the general principles have shown, it is assumed, both in the Expropriation Act, Sections 5 and 6, and in case-law, that only the value of a foreseeable alternative should be compensated. This starting point means that it would be in breach of the general arrangement if a waterfall that can not be used in foreseeable small-scale hydro-power was to be compensated as if it could be put to such use.
\end{quote}

Having used the planning permission granted to the expropriating party as evidence that alternative development was unforeseeable, the Court needed to answer Question (6) by coming up with some alternative way of compensating the owners.  To do so, the Court also had to answer Question (4), however, asking whether or not the scheme underlying expropriation could be taken into account at this stage. Moreover, one would think, given how the expropriation scheme was used as an argument when answering Question (5) in the negative, that it \emph{could} be taken into account. This, however, was answered in the negative by the Supreme Court in \emph{Otra II}, where the presiding judge reasoned as follows.

\begin{quote}
Based on the arguments presented to the Supreme Court, I find it safe to assume that there does not today exist any market for the sale and leasing of waterfalls for which alternative development is not foreseeable, but where the waterfalls can be used in more complex hydro-power schemes. The appellants have not been able to produce documents or prices to document the existence of such a market
\end{quote}

Let us first remark that it is hard to imagine how a market such as that asked for here could ever develop, all the while alternative buyers, by the courts own reasoning, are excluded from being taken into account. In this case, if there was to be a market, it would have to be down either to the regular benevolence of the expropriating parties in cases like these, or to the government compelling them to enter friendly negotiations. In the Norwegian context, both seem unlikely. More important, however, is the implicit assumption that in order to value the waterfall according to its potential for hydro-power production, a market needs to be identified. It is \emph{not} considered sufficient that the scheme for which expropriation takes place is itself a hydro-power project, on the basis of which the value of the waterfall could be assessed following exactly the same steps as in the new method.

In fact, the Supreme Court's reasoning in \emph{Otra II} serve as an excellent example of the type of reasoning that makes the no-scheme rule highly problematic for cases of expropriation that benefit commercial schemes. Indeed, it seems to follow that the scheme itself should not provide a basis for calculating the compensation, but then, on other hand, it also appears that there really is no other way to calculate it, seeing as the State, by giving permission to carry out the scheme, have effectively acted in such a way as to make an alternative use \emph{of the same kind} seem inherently unforeseeable. This is not so much of a problem when the use that the owner could make of the property has a different character than the scheme, since in this case, if we disregard the scheme, this other use might seem foreseeable. However, when the alternative use of the property is of \emph{exactly the same kind} as the use made of it by the scheme, it does seem counterintuitive, as noted by the Court of Appeal in \emph{Otra II}, to regard it as foreseeable, all the while the scheme will tend to appear the more rational form of exploitation.

It seems, however, that when taken to its logical conclusion, this line of reasoning, based on the no-scheme rule, leads to an offensive results; the commercial value of the property is not to be compensated because the optimal commercial use of the property is the use that the expropriating party aims to make of it in the scheme underlying expropriation. Note that the conclusion is not just that this optimal value, inherent in the scheme, should not be compensated. No, the conclusion in \emph{Otra II} was that \emph{no} compensation could be estimated for any use of the same \emph{kind}, since such use was not foreseeable, owing to the absence of a market.

It is certainly possible to argue that this decision represent a misguided application of the no-scheme rule. In effect, the Supreme Court allowed the planning permission given to the expropriating party to act as evidence that alternative development was unforeseeable, while it used the no-scheme rule to argue that the hydro-power scheme for which this planning permission was given could not itself form basis for compensation payment based on market value.  On the other hand, it seems that even if we disregard the scheme completely, it is unnatural to base the compensation payment on the value of a hydro-power scheme that is less beneficial, both commercially and in terms of resource efficiency, than the scheme for which expropriation takes place. Such a scheme would not, one must presume, \emph{actually} have been carried out, regardless of the questions of whether or not it would have been given planning permission. But it does seem particularly difficult, intuitively speaking, to predict what use would have been made of the waterfall if these were the facts: more or less exactly the same scheme as that underlying expropriation would be implemented, but by some from of voluntary agreement with the owners, not by means of expropriation. 

In \emph{Otra II}, however, this line of thought was also rejected, although this was, in part, due to the point not having been argued before the Court of Appeal. But then the question arose as to how exactly compensation should be calculated. The answer, following up on the "value to the owner" principle so forcefully adopted elsewhere in the judgment, would appear to be that no compensation should be paid at all, save perhaps for loss of fishing rights and the like. This, however, was \emph{not} the conclusion reached by the Supreme Court. Instead, the Court states that a return to the traditional method is in order. However, they do not apply it in the traditional way. Rather, they casually sanction the replacement of regulated low water-flow in the definition of $Qreg$ by the average water-flow, thus moving away from compensation based on the level of regulation, to compensation based on average effect in the project. Moreover, they also sanctioned the use of a significantly increased unit price compared to earlier times.

What to make of this? It seems hard indeed to make sense of indeed, since effectively, by relying on the traditional method, the Supreme Court contradicts its own conclusion that compensations should be based on market value. Instead, they rely on a method that, in effect, is based on an attempt to quantify the value of the waterfall as it is being used by the expropriating party in his project. However, by relying on a technical method that has been completely outdated, and have lived its own life in the courts, it becomes difficult to assess the outcome properly, at least for a non-expert. It seems, in particular, that the Supreme Court prefers the obscurity of the traditional method, and its status as an established principle, over the possibly radical conclusion that, in cases such as this, it simply is not tenable to adopt the "value to the owner" principle, as least not as construed in Norwegian law.

Indeed, it is simple enough to be critical of the Supreme Court based on the fact that they regarded alternative development as unforeseeable in this case, when the planning permission granted to the expropriating scheme itself appears to have provided the decisive bit of evidence. Still, it is not possible to escape the fact that this reflects a general tendency in Norwegian law, and so, even if it appears unreasonable, it might very well be a correct application of national law. Moreover, it could very well have been that alternative development was unforeseeable for \emph{some other reason}, for instance because the only commercially viable exploitation was the scheme planned by the expropriating party. In this case, the problem of how to compensate the owners in the absence of an alternative form of exploitation would still arise. It is this question, in particular, which seems entirely unsatisfactorily resolved under an application of a "value to the owner" principle.

This is witnessed by \emph{Otra II}, and, in fact, it appears that the Supreme Courts decision \emph{not} to follow their own reasoning to its logical consequence is the main lesson to be learned from the case. For all intents and purposes, the Supreme Court \emph{rejects} the "value to the owner" principle, but they obscure this by wrapping it up in the traditional method, which is deeply flawed. However, the problem it attempts to solve appears significant, and it pertains directly to the question discussed more generally in Section \ref{sec:noscheme}, namely how to apply the "value to the owner" principle with respect to commercial schemes. It seems that even the fiercest supporters of limiting owners' right to compensation tend to find it too offensive to apply this principle when it leaves the owners with no form of compensation for giving up property to multi-million, purely commercial undertakings. Indeed, such a practice would surely also be in breach of the human rights law. It seems, in particular, that the subjective aspect of the "value to the owner" principle is impossible to maintain. Indeed, if the commercial value falls to be disregarded for no other reason than the fact that the State happens to have granted planning permission to the expropriating party rather than the owner, this is not only dubious with respect to human rights protecting property, but also appears to be a case of \emph{discrimination}, e.g., as prohibited by ECHR Article 14.

The problem does not arise when the buyer sees value in the property that is of a different \emph{kind} than that realizable by \emph{any} private owner. In this case, the rule simply states that the owner should not be able to demand that "public value" is transformed into commercial value just for him. This appears like a reasonable principle. But when there is commercial value already present on the "public" side of the transaction, it seems completely unwarranted that the public should be allowed to transfer this value from the owner to someone else without compensation. Thus, it seems that more accurately and acceptably, the "value to the owner" principle should be thought of as a "commercial value" principle. It seems, in particular, that the principle need to be stripped of any suggestion that a preferential financial position is to be awarded to whoever benefits from expropriation.\footnote{Exceptions might be possible to imagine, but, one would think, only when they can be construed as falling under the "public value" banner in some way.}

It seems unfortunate that this aspect has not been made explicit, and the difficulties that arise in the absence of this nuance seems nicely illustrated by the case of Norwegian waterfalls. Still, as the case of \emph{Otra II} seems to indicate, an interpretation of the "value to the owner" principle along less offensive lines is in reality already in place with regards to Norwegian hydro-power. Here, it seems that "value to the owner" has in fact \emph{never} been applied in the traditional way. Hopefully, rather than obscuring this fact by relying on an unsatisfactory and artificial method for calculating the compensation, the future will see further developments that recognize the need for new principles. It should be recognized, in particular, that as the law has been applied for the last 80 years, despite its grave flaws and injustices, there has always been an implicit recognition in Norwegian law that the owners of waterfalls are \emph{entitled to their share} of the commercial benefits of hydro-power. 

In fact, in the recent Supreme Court case of \emph{Kløvtveit}, a further illustration of this is found. The conclusion here was also that alternative development was not foreseeable. However, unlike in \emph{Otra II}, the Court of Appeal had compensated the owners based on the fact that they regarded it foreseeable that in the absence of the scheme, the waterfalls would have been exploited in exactly the same way, except that it would have happened in the form of \emph{cooperation} between the owners and the expropriating party. By this line of reasoning, the Court effectively seems to have adopted a more rational "commercial value" principle, to replace the traditional method. 

Indeed, is it not always the case, at least under objective standards of assessment, that when alternative development is unforeseeable, then a rational alternative buyer -- assumed to operate in a world where there are no "powers of compulsion", to paraphrase Lord Nicholls in \emph{Waters} -- would look precisely to the likely possibility of cooperating with the expropriating party? This, on the other hand, would \emph{never} be a safe assumption to make for non-commercial aspects which, in the absence of commercial potential would not give an alternative buyer financial incentive to do so.

We mention that \emph{Kløvtveit} was discussed in \emph{Otra II}, and that the presiding judge made some reflections, focusing on what he regarded to be "practical problems" with cooperation. However, this was not crucial to the decision, since the cooperation model was not argued for by council. In light of this, one can only hope that \emph{Kløvtveit}, rather than \emph{Otra II}, will become the influential precedent for future cases.

\section{Conclusion and future work}\label{sec:conc}

