\chapter{Taking Waterfalls}\label{chap:4}

\section{Introduction}\label{sec:intro4}

In this chapter, I address expropriation of waterfalls in more depth, particularly the administrative practices that have evolved in relation to such expropriation. My main aim is to shed light on how these practices impact on the position of owners and local communities.

In Norway, the water authorities tend to consider expropriation of riparian rights as a natural component of hydropower development. In particular, a license to expropriate from local owners is typically considered a more or less automatic consequence of a development license. Moreover, as discussed in the previous chapter, the administrative licensing assessment tends to focus on the environmental consequences of development, not how interference in property affects owners and local communities.

As a result, a {\it presumption} has
developed, whereby the administrative decision-makers consider a license to undertake large-scale development as an indication that an expropriation order should also be granted.\footnote{The leader of the hydropower licensing division of the NVE made an explicit statement to this effect in \cite{flatby08}.} Importantly, this presumption still remains in place, even though the regulatory and economic context of riparian expropriation has changed dramatically as a result of the liberalisation of the electricity sector in the early 1990s. 

In this chapter, I give a detailed presentation of the relevant statutory rules, the history of the law in this area, and current administrative practices. I then illustrate how the water authorities and the courts perceive and apply the law in this area, by carefully presenting the recent Supreme Court case of {\it Jørpeland}.\footnote{See \cite{jorpeland11}.}

The structure of the chapter is as follows. I begin in Section \ref{sec:explaw} by giving a brief overview of expropriation law generally, as well as special statutory rules relating to hydropower. In Section \ref{sec:twp}, I present the historical context to these rules. In short, I argue that because hydropower development was carried out by public utilities, expropriation for hydropower development enjoyed a high degree of political legitimacy. In addition, the lack of an open market meant that owners could not benefit commercially from developing hydropower themselves. Hence, their financial loss following expropriation was limited. In fact, expropriation (or voluntary sale) of riparian rights was usually the best an owner could hope for in terms of benefiting financially from hydropower.

%The story begins in the late 19th century, when the first statutory authorities for such expropriation began to emerge. Initially, these authorities were very narrow, however, and they did not cover expropriation of waterfalls, only additional land and rights that waterfall owners might need to develop their resource. 
%
%Later, in the early 20th century, the public sector began to expropriate waterfalls for hydropower, but this was only authorised on narrowly defined conditions, to enable the state to provide a new public service: electricity supply. Private companies could not expropriate waterfalls unless they were already majority owners in the local area, a situation that did not change until the passage of the \cite{wra00}. 
%
%Even before this, when hydropower development was still seen as a public service, it was sometimes met with resistance by local people and environmental groups, particularly as the state begun to pursue large-scale projects after World War Two. I discuss the case law that developed in this regard, particularly in relation to procedural rules. I conclude that the public-sector characteristics of hydropower development led the courts to defer very broadly to the discretion of the executive and the legislature, also in relation to the content and scope of provisions in administrative law. 

As discussed in the previous chapter, this changed following liberalisation of the electricity sector in the early 1990s. Ten years later, a new  expropriation authority was also introduced, in the \cite{wra00}. For the first time in Norwegian history, waterfalls and rivers could now be expropriated for purely commercial gain, also by private companies. In Section \ref{sec:twpp}, I place this change in the law in an historical context, before presenting the expropriation framework currently in place. I note that apart from the increased scope of expropriation, the framework developed prior to liberalisation remains largely in place.

%I note that this change in the law was not given much attention by the executive committee that prepared the act. It was described merely as a ``simplification'' of existing rules. Moreover, the legislature did not address the change at all when the Act passed through parliament. In fact, the wording of the \cite{wra00} does not explicitly make clear that private expropriation is now permitted. However, the Act empowers the executive to decide, using directives, what class of legal persons can be given a license to expropriate waterfalls. Such a directive has been issued, with little or no debate, granting the possibility to benefit from an expropriation for hydropower development to ``anyone''.
%I present the law relating to expropriation of waterfalls in some depth, before I go on to consider more concretely how the procedure plays out in the context of for-profit takings. Here I anchor the presentation in the recent Supreme Court case of {\it Jørpeland}, where the issue of procedural legitimacy arose after a commercial company was granted permission to deprive local owners of waterfall rights that the owners wished to make use of in their own hydropower project.\footcite{jorpeland11}

In Section \ref{sec:jorpeland}, I use the case of {\it Jørpeland} to show how the expropriation framework currently in place leaves owners particularly marginalised. Moreover, I note that their standing is very weak under administrative law, as a result of how the expropriation issue is overshadowed by the licensing question.

I then argue that the Supreme Court adheres to a narrow perspective on the meaning of property protection, taking it to be an issue that begins and ends with the question of compensation. In my opinion, this fails to do justice to the most important issue that arises when waterfalls are taken for profit, namely the question of democratic legitimacy. 

I conclude the chapter by elaborating on this theme, connecting it also with the theoretical discussion in Part I of the thesis. This sets the stage for the final chapter, where I consider land consolidation as a democracy-enhancing alternative to expropriation in hydropower cases.

%As discussed in Chapter \ref{chap:4}, the hydropower sector has now become depoliticised, expert-dominated and market-oriented. In light of this, my overarching argument is that the continued annihilation of local property rights threatens to render the state's functions in this sector subservient to the interests of the most powerful market actors, not the interests of the Norwegian people.

\section{Norwegian Expropriation Law: A Brief Overview}\label{sec:explaw}

As mentioned in Chapter \ref{chap:2}, the right to property is entrenched in section 105 of the Norwegian Constitution. There it is made clear that when property is taken for public use, full compensation is to be paid to the owner. The public use requirement is understood very broadly. According to some legal scholars in Norway, it places no limit at all on the state's authority to expropriate.\footnote{See \cite[249]{aall04}.}

However, it is a rule of unwritten constitutional law that administrative decisions which affect the rights of individuals can only be carried out when they are positively authorised by law.\footnote{See generally \cite{hogberg11}.} Moreover, the Constitution is not understood as providing an authority for the state to expropriate, it merely expresses the presupposition that expropriation is possible.\footnote{See, e.g., \cite[6]{fleischer86}.} Hence, when applying eminent domain, the government needs to justify this on the basis of specific authorising provisions. 

Historically, there was no general act relating to expropriation, and a range of different acts provided the necessary authority to expropriate for specific purposes such as roads, public buildings, and schools.\footnote{See \cite[11-12]{nut54}.} Today, many of these authorities have been collected, broadened, and included in the \cite{ea59}.\footnote{Act no 3 of 23 October 1959 Relating to Expropriation of Real Property.} Still, some specific authorities remain, such as section 16 of the \cite{wra17}, which authorises expropriation for watercourse regulation.

Following the introduction of the \cite{wra00}, the general authority used to expropriate waterfalls has been included in the general act on expropriation.\footcite[2 no 51]{ea59}. Here it is stated that expropriation may take place in order to facilitate ``hydropower production''. In addition, it is made clear that expropriation can only be authorised if the benefits undoubtedly outweigh the harms. 

This sets expropriation orders apart from the various hydropower licenses discussed in Sections \ref{sec:wra00}-\ref{sec:ea} of Chapter \ref{chap:3}. For an applicant to obtain development licenses, it is sufficient to show that the benefits outweigh the harms, it need not be ascertained that this is {\it undoubtedly} the case. However, the practical significance of this difference is limited. According to the Supreme Court, the additional requirement in expropriation cases means only that it should be clear that the benefit is greater, it does not imply that the benefit has to be qualitatively more significant than in the licensing cases.\footnote{See \cite{lovenskiold09}.}

The authorising authority is the King in Council. However, this authority can be delegated to ministries or other state bodies that the King in Council can instruct.\footnote{See \cite[5]{ea59}.} The compensation to the owner is determined following a judicial procedure administered by the so-called appraisement courts.\footnote{\cite[2]{ea59}.} This is the name given to the regular civil courts when they hear appraisement cases, observing the special procedure set out in the \cite{aa17}. The appraisal procedure emphasises the importance of factual assessment and lay discretion (the appraisal court typically sits with four lay judges).\footnote{See \cite[11-12]{aa17}.} In addition, there are special rules regarding costs, indicating that the expropriating party is usually required to pay for the procedure, include the owners' legal expenses.\footnote{See \cite[54]{aa17}.} In other regards, the appraisal procedure resembles a typical adversarial process before a civil court.\footnote{See generally \cite{dyrkolbotn15}.} 

The \cite{ea59} states that unless the Kind in Council decides otherwise, expropriation orders may only be granted to state or municipality bodies. This is formulated as a limiting principle, but in effect it serves as a general authorisation for the executive to decide, without parliamentary involvement, that a larger class of legal persons may be granted expropriation licenses. 

For many purposes, directives have been issued that extend the class of possible beneficiaries to any legal person, including companies operating for profit. In 2001, such a directive was issued for the authority to expropriate in favour of hydropower production.\footnote{See Directive no 391 of 06 April 2001.} 

In addition to providing a general authority for expropriation, the \cite{ea59} also contains several procedural rules. These are collected in Chapter 3 of the Act. Here the Act sets out minimal requirements for what an application for an expropriation license must include, stating that it should make clear who will be affected, how the property is to be used, and what the purpose of acquisition is.\footnote{See \cite[11]{ea59}.} In addition, the Act requires the applicant to specify exactly what property they require, and to include information about the type of property in question and the current use that is made of it.

The owners must be notified, and the starting point is that every owner should be given individual notice, although this obligation is loosened when it is ``unreasonable difficult'' to fulfil\footnote{See \cite[12]{ea59}, para 2.} In such cases, it is sufficient that the documents of the case are made available at a suitable place in the local area. In addition, a public announcement must then  be made in the official notification publication of the government, as well as in two widely read local newspapers.\footnote{See \cite[12]{ea59}.}

The licensing authority is required to ensure that the facts of the case are clarified to the ``greatest extent possible''.\footnote{The Norwegian expression is ``best råd er'', which literally means ``best possible way''. See \cite[12]{ea59}, para 2.} This formulation seems very strict, but is also highly non-specific. In practice, the level of scrutiny given to the expropriation question under Norwegian law varies greatly depending on sector-specific administrative practices.  

Established practice from several fields, including the hydropower sector, suggests that when expropriation takes place to implement a public plan or a licensed development, little attention is devoted to expropriation as a special issue.\footnote{For zoning plans, see \cite{namsos98,bo99}. For hydropower, see \cite{jorpeland11}.}

A decision to grant an expropriation license must be justified, and the parties should be informed of the reasons for the decision.\footnote{See \cite[12]{ea59}, para 3.} This expropriation-specific rule is largely superfluous, however, as the obligation to give reasons would in most cases follow independently from general administrative law, c.f., Section \ref{sec:paa67}.

The applicant must cover costs incurred by owners in relation to a pending application for expropriation.\footnote{See \cite[15]{ea59}.} The exact formulation is that the applicant is obliged to cover the costs that ``the rules in this chapter carry with them''. That is, the applicant is obliged to cover the costs that are related to the owners' rights pursuant to Chapter 3 of the \cite{ea59}. In practice, an owner will be denied costs if the competent authority takes the view that they are unreasonable or disproportionate to their interests in the case.\footnote{If the case progresses to an appraisement dispute, the competent authority to decide on costs is the appraisement court. Otherwise, the decision is left with the executive. See \cite[15]{ea59}.}

%Particularly problematic are cases for which there is no clear division between those aspects of the case that relate to expropriation and those that relate to other licenses or land use planning more generally. This is the situation, for instance, in relation to hydropower development. In such cases, it is unusual for local owners to get any significant coverage of costs relating to the application processing. Legal expenses, for instance, are rarely covered unless they are incurred in relation to a subsequent appraisement dispute. This can be a problem for owners that wish to resist expropriation. Obviously, it is crucial for them to voice convincing objections already at the application processing stage.

In addition to the procedural rules in the \cite{ea59}, many rules of administrative law apply in expropriation cases. In the next section, I give a brief overview of administrative law in Norway, including the most relevant rules of the \cite{paa67}.

\subsection{The Public Administration Act}\label{sec:paa67}

Starting in the late 19th century, the importance of public administration gradually increased in Norway.\footnote{See \cite[8-12]{nut58}.} This development gained momentum after the Second World War, when administrative bodies also came to be placed more directly under centralised political control. At the same time, the traditional administrative ideal based on strict adherence to the letter of the law was replaced by a form of management that actively sought to pursue political goals.\footnote{See generally \cite{gronlie00}.} The ambit of administrative decision-making power widened significantly. Many new administrative bodies were set up, while many of those already established were empowered greatly as new statutory rules were introduced that specified the competence of administrative bodies in broader and broader strokes.

As administrative bodies became increasingly powerful, concerns arose regarding the relative lack of procedural safeguards to protect the individuals affected by administrative decisions. This concern was also fuelled by the fact that as the importance of state regulation increased, so did the power of the administrative branch to make decisions that would directly affect the rights and obligations of specific individuals.\footnote{See \cite[12-16]{nut58}.}

In response to this, minimum standards of administrative due process were encoded in the \cite{paa67}.\footnote{Act no 86 of 10 February 1967 Relating to Procedure in Cases Concerning the Public Administration.} This Act sets out the fundamental procedural principles that government bodies must follow when preparing to make administrative decisions. Some rules apply to any such decision, but a particularly important class of rules apply specifically to so-called {\it individual decisions}, namely those that affect the rights and responsibilities of one or more specific legal persons.\footcite[2]{paa67} Clearly, owners of property targeted by an expropriation application fall into this category, so the \cite{paa67} applies in expropriation cases.

Many of the rules in the \cite{paa67} mirror those of the \cite{ea59}. However, the \cite{paa67} tends to include broader and more detailed formulations. For instance, the duty to give advance notice is accompanied by more information about what kind of information such a notice must contain.\footnote{See \cite[16]{paa67}. Just like the \cite{ea59}, the rule in the \cite{paa67} makes clear that individual notices might not be required if the parties are difficult to reach.} In particular, it is said that ``the advance notification shall explain the nature of the case, and otherwise contain such information as is considered necessary to enable the party to protect their interests in a proper manner''.\footcite[16]{paa67} 

Hence, it is not enough simply to inform the party that a case is under way, the Act also stipulates that the notice has to meet a minimum standard of quality. In relation to expropriation of waterfalls this becomes potentially significant, especially in light of the practice I discussed in Chapter \ref{chap:3}, whereby applicants send out these notices themselves. One may ask, in particular, what owners are supposed to think when they receive a letter from a commercial company stating that unless a friendly settlement can be reached, their waterfalls and rivers will be expropriated.\footnote{Such a formulation is typical, used for instance by the expropriating party in \cite{sauda09}. In general, according to my own experience, a generic letter is sent by the developer to those private individuals who may be affected, with no individuation based on their interests in the case (e.g., based on whether they stand to loose a small-scale hydropower potential or  are affected in some (minor) ways by building works). Clearly, this approach can discourage riparian owners from engaging in the administrative process in a manner commensurate with the fact that they own the natural resource in question.}

The duty to assess cases also follows from the \cite{paa67}, mirroring the rules of the \cite{ea59}. The formulation is similarly imprecise, as it is declared that cases must be ``clarified as thoroughly as possible'' before a decision is made.\footcite[17]{paa67} Importantly, the \cite{paa67} includes specific rules that oblige the authorities to inform parties about information they retrieve during their assessment of the case, and to actively solicit further comments from the parties.\footnote{See paras 2 and 3 of \cite[17]{paa67}.}

The duty to justify and give reasons for administrative decision is also expressed in the \cite{paa67}. The duty applies to most individual decision, with some narrowly defined exceptions concerning cases when no party can be assumed to be dissatisfied, or when giving grounds would involve disclosing privileged information.\footnote{See \cite[24]{paa67}. Moreover, the King is authorised to limit the duty to give grounds when ``special circumstances so require''.} As to the content of the reasons given, the authorities should mention the relevant rules authorising the decision, outline the factual assessment, and describe the main considerations that have been decisive for the use of discretionary power.\footcite[25]{paa67} In case law, the duty to give reasons has some practical significance, since the Supreme Court has declared that insufficient reasons can be taken as an indication that the decision itself suffers from a shortcoming.\footnote{See \cite{isene81,hauge00}.} In hydropower cases, however, the duty to give reasons is understood to pertain to the licensing question as a whole, so that the authorities are not obligated to give individuated reasons to riparian owners, pertaining specifically to the expropriation question.\footnote{See \cite{sauda09,jorpeland11} (discussed in more depth in Section \ref{sec:jorpeland}).}

Sometimes, the parties to an administrative decision are ill-equipped to look after their interests, even if the safeguards mentioned above are respected. This situation often occurs in hydropower cases, as riparian owners often lack the technical, commercial, and legal knowledge necessary to understand the value of their property and their own legal position as owners. The \cite{paa67} establishes a general duty to provide guidance, to ensure that the parties are able to look after their interests in the ``best possible way''.\footcite[11]{paa67} However, it is explicitly stated that the level of guidance must be adapted to the circumstances and the capacity that the government agency has for offering such assistance. At the same time, it is made clear that the decision-making agency must assess, on their own motion, the parties' need for guidance.

To summarise, both the law of expropriation and general administrative law impose a range of procedural rules that ordinarily apply to expropriation cases. In principle, these apply also when rivers and waterfalls are expropriated. In practice, however, they are completely overshadowed by the special rules that regulate the licensing procedure in such cases. I return to this issue in more depth in Section \ref{sec:jorpeland}. First, I elaborate on statutory rules that specifically target expropriation for hydropower, within the context of the relevant licensing procedures.

\section{Taking Waterfalls by Obtaining a Regulation License}\label{sec:special}

As I mentioned in Chapter \ref{chap:3}, Section \ref{sec:wra17}, the \cite{wra17} establishes an automatic right to expropriate rights needed to undertake a watercourse regulation. This is not understood to include a right to expropriate rivers and waterfalls needed for the hydropower development. However, it includes a right to transfer water away from a river for development somewhere else. 

This is of course a {\it de facto} license to expropriate riparian rights, since the water as such is taken by the expropriating party. Moreover, it has always been treated as expropriation of riparian rights in relation to the compensation issue.\footnote{See \cite{jorpeland11}.} Formally, however, the interference is not considered a riparian expropriation, but rather seen as an expropriation of a right to deprive rivers of water, a sort of easement whereby the developer acquires the right to interfere with the rights of riparian owners in source rivers.

In theory, the rules in the \cite{ea59} and the \cite{paa67} still apply in such cases. Indeed, the rules in the \cite{paa67} express general principles of administrative law, pertaining to all kinds of individual decisions, including both expropriation and  licensing decisions. The \cite{ea59}, for its part, explicitly states that it applies to property interferences authorised under the \cite{wra17}.\footnote{See \cite[30]{ea59}.} However, it is also stated that the rules in the \cite{ea59} only apply in so far as they are ``suitable'' and do not ``contradict'' sector-specific rules.\footcite[30]{ea59} This points to the potential caveat that while a range of procedural rules apply in theory, they may be ignored in practice, in so far as they are deemed ``unsuitable'' by some competent state body.

This is practically significant in hydropower cases. In particular, the established practice among the water authorities is to regard the procedural rules in the \cite{wra17} as exhaustive.\footnote{This was made clear through the case of \cite{jorpeland11}, where this practice also got a stamp of approval from the Supreme Court.} In addition, the material assessment requirement in the \cite{ea59} is not considered to have any independent significance alongside the assessment criterion in the \cite{wra17}.\footnote{Again, see \cite{jorpeland11}.} This is so even though case law on the former assessment criterion emphasises the interests of affected property owners in a way that case law on the licensing issue does not.\footnote{In addition, the formulation in \cite[2]{ea59} contains the additional qualification that the benefit of interference must ``undoubtedly'' outweigh the harm, meaning that this clearly must be the case (pertaining to the evidence, not the weight of the benefit compared to the harm), see \cite{lovenskiold09}. No corresponding requirement is included in the \cite[8]{wra17}. Instead, the formulation there is that a license should ``normally'' not be given unless the benefits outweigh the harms. See also \cite[325-236]{haagensen02} (arguing that the ``normally'' qualification is without practical significance).}

%However, there is no doubt that the rules of the \cite{paa67} apply to takings of water rights pursuant to the \cite{wra17}. Moreover, there is no doubt that when a separate expropriation license is sought for waterfalls, these rules, as well as the rules in \cite{ea59} both apply. In practice, they nevertheless play a minimal role when the water authorities assess cases, as the assessment is unified, and the focus remains on balancing environmental and energy interest.

As a consequence of how the law is understood on this point, it is very hard for owners to challenge the legality of a decision to allow expropriation of their riparian rights, especially when expropriation takes place pursuant to the \cite{wra17}.\footnote{It follows from the discussion in Chapter \ref{chap:4} that large-scale development projects almost always involve a license pursuant to the \cite{wra17} (or such that the rules from this Act, including s 16 on expropriation apply pursuant to the \cite{wra00}).} Moreover, even if section 16 of the \cite{wra17} does not apply, the water authorities tend to approach the affected owners in a similar way. In particular, the practices observed with regard to the issue of property interference is largely the same in all cases when the administrative branch classifies the license application as pertaining to a large-scale project.\footnote{See \cite{flatby08}.}

For such projects, the water authorities rely on a presumption that the conditions to permit expropriation are fulfilled whenever a development license may be granted.\footnote{See \cite{flatby08}.} Hence, in order to defend themselves, owners must proceed in a roundabout manner by addressing the licensing question as such. In practice, there is little or no room for arguing on the basis of rules that protect private property.    Moreover, in order to argue that the expropriation is unlawful on procedural grounds, the owners must effectively demonstrate that the water authorities dealt with the case in contravention of sector-specific rules and practices pertaining primarily to the licensing question. This is a daunting task, particularly in light of case law developed during the period of monopoly regulation. This body of case law suggests that the courts will largely defer to the administrative branch, even when it comes to interpreting the relevant procedural rules.\footnote{The deferential stance was expressed most clearly in the {\it Alta} case discussed in Section \ref{sec:twp} below.}

As a consequence, procedural objections pertaining to the administrative assessment of existing property interests are unlikely to be successful. I am not aware of any case where such an argument has succeeded. In Section \ref{sec:jorpeland}, I will further demonstrate the present situation by tracking in detail the extent to which the Supreme Court is prepared to tolerate procedural shortcomings pertaining to the expropriation issue in hydropower cases.%\footnote{The justification for this, as I will show, is that the {\it Alta} case still serves as the primary precedent when assessing procedural complaints in hydropower cases. See, e.g., \cite{sauda09,jorpeland11}. As I will show in Section \ref{sec:twp}, the {\it Alta} case had nothing to do with expropriation and property rights. Rather, it arose from environmental concerns (and concerns about property-less indigenous people). In particular, the procedural objections that were raised in {\it Alta} were specifically related to the assessment of environmental consequences, an assessment that is usually carried out quite thoroughly  in hydropower cases (even more than usual in the {\it Alta} case, which was already very controversial).} 

First, I give a chronological presentation of how the law on expropriation of waterfalls has developed as part of the legal framework for management of hydropower. I begin with the period prior to the reform implemented by the \cite{ea90}.

\section{Taking Waterfalls for Progress}\label{sec:twp}

Historically, Norwegian law did not contain a general authority for expropriation of riparian rights.\footnote{See \cite[29]{amundsen28}.} In the \cite{wra88}, a range of provisions authorised appropriation of water rights and land for specific purposes, but the criteria were narrow.\footnote{See \cite[69-85]{dahl88}. In addition, the purpose of expropriation was largely understood to be binding also on future use, so that the taker would not gain unrestricted control over the rights they acquired. Rather, they were obliged to use these rights to pursue the specific public purpose for which expropriation was authorised. See, e.g., \cite[133-140]{rygh12}.} Rivers and waterfalls as such could never be made subject to expropriation, and expropriation of other water rights could only be permitted in so far as the affected owners were not thereby deprived of any water power that they could reasonably make use of themselves.\footnote{See \cite[58,60]{dahl88}.}

Specifically, expropriation for hydropower development was not permitted, except to the benefit of riparian owners who needed to acquire surrounding land in order to exploit their existing water rights.\footnote{See the \cite[15-16]{wra88}. See also the commentary in \cite[60-65]{dahl88}.} At the same time, riparian owners could apply for licenses to engage in various industrial exploits, in some cases also when this would prove damaging to other landowners, for instance through deprivation of water or flooding.\footnote{See \cite[14]{wra88}. See also the commentary in \cite[54-60]{dahl88}.} These rules are similar to many of the rules found in contemporaneous mill acts from the US, c.f., the discussion in Chapter \ref{chap:2}, Section \ref{sec:hop}. As in the US, the kinds of takings in question here could be classified as economic development takings. However, the source of the economic development potential was never taken from the owners in these cases. Rather, the takings only targeted additional rights that were needed in order for the existing owners to realise the full potential of their own resources.

In fact, an important principle of expropriation law at this time was that no property could be taken if the taker's interest in that property was the same as that of its current owner.\footnote{See \cite[168-170]{dahl88}.} This applied regardless of whether or not the owners, subjectively speaking, were likely to pursue those interests optimally. Hence, expropriation of water power was ruled out already as a matter of principle. In particular, as the regulatory system of the day made private hydropower development possible, a private riparian owner was regarded as possessing a hydropower interest. As a result, such owners could not be deprived of their rights by a taker whose interest was also to undertake hydropower development.

Following industrial advances, the interest in hydropower exploded in the late 19th century.\footnote{See \cite[58-59]{falkanger87}. See also the discussion in Chapter \ref{chap:3} Section \ref{sec:x}.} As a result, the state increasingly came to see it as a political priority to sensibly regulate the hydropower sector. As discussed in Chapter \ref{chap:3}, the most important expressions of this came in the form of two new licensing acts, namely the \cite{wra17} (Section \ref{sec:wra17} and the \cite{ica17} (Section \ref{sec:ica17}).

%Recall that the \cite{ica17} set up a licensing framework that would make it less attractive for speculators to purchase waterfalls, while also ensuring that the waterfalls purchased came under state ownership after a concession period. The \cite{wra17}, on the other hand, established a novel principle, namely that the right to regulate the flow of water in a river system did not belong to the riparian owners, but the state. This did not imply any difference in the right to use rivers for hydropower generation. This right still belonged to the local landowners. Hence, to undertake hydropower development involving watercourse regulation, both the riparian rights of local landowners as well as the regulation right of the state would be required.

Following up on this, parliament soon passed legislation that authorised expropriation of riparian rights for the benefit of public bodies, also when the purpose was hydropower development.\footnote{Legislation that made it possible to expropriate waterfalls to the benefit of the municipalities was introduced in 1911, and a similar authority that authorised expropriation in favour of the state appeared in 1917, see \cite[29]{amundsen28}.} In 1940, these authorities were consolidated and integrated in the general water resources legislation, through the \cite{wra40}.\footnote{This act has since largely been replaced by the \cite{wra00}.} According to this act, the authority to expropriate waterfalls could be granted only to the state and the municipalities. Moreover, the municipalities could only expropriate waterfalls when the purpose was to provide electricity to the local district.\footnote{See the \cite[148]{wra40}. See also the commentary in \cite[201-210]{sorensen41}.}

Private parties could only expropriate in exceptional circumstances, when they already owned more than 50 \% of the riparian rights that sought to exploit.\footnote{See the \cite[55]{wra40}. See also the commentary in \cite[70-74]{sorensen41}. I remark that this was a novel rule in the 1940 Act, which contradicted earlier theories about the legitimacy of allowing expropriation for private benefit.} Moreover, whenever expropriation took place, it was felt that benefit sharing with local owners was required. Hence, special rules were introduced to ensure that takers would have to pay {\it more} than full compensation (typically a 25 \% premium, but in some cases the owner was also given a right to opt for compensation in the form of a proportion of the electricity output of the plant).\footnote{See \cite[70-91,184,210]{sorensen41}.}

As I showed in Chapter \ref{chap:4}, the electricity supply in Norway just after the passage of the \cite{wra40} was already well developed, with 80 \% of the population having access to electricity. Moreover, in the rural areas the supply often came from one among a vast number of small, local, power plants. In light of the progress already made and the highly decentralised structure of the hydroelectric sector at this time, one might have expected expropriation to remain a relatively rare occurrence.

However, the use of expropriation to facilitate hydropower development increased greatly after the war, as the state itself became engaged much more actively with hydropower development, also for commercially oriented industrial purposes.\footnote{See \cite[59-71]{thue96}. See also \cite{skjold06}.}

Hence, despite the spirit and wording of the \cite{wra40}, this was the time when expropriation of rivers and waterfalls became a measure to facilitate economic development. At first, this would still take place on non-commercial, politically governed, terms. But the increased prevalence of expropriation seen during this time had little to do with a pressing need to supply electricity to the people. Rather, it was a consequence of an increased political demand for industrial hydropower, combined with the fact that the hydropower sector was reorganised and brought under increasingly centralised political control.\footnote{See \cite[69-71]{thue96}.}

As I mentioned in the previous chapter, many local, privately owned, hydropower plants were shut down during this period, as a result of an explicit policy meant to create government monopolists.\footnote{See Chapter \ref{chap:3} Section \ref{sec:x}.} Moreover, as a result of centralisation, a growing share of the financial benefits from development would also accrue to urban areas, as local development companies were replaced by state companies and companies dominated by prosperous city municipalities.\footnote{In 2007, as the result of a gradual centralisation process, the 15 largest hydropower companies in Norway, which are largely controlled by the state and some city municipalities, owned roughly 80\% of Norwegian hydropower, measured in terms of annual output. In 2006, the public owners of hydropower in Norway benefited from receiving more than NOK 9 billion in dividends. See \cite[28]{otprp61}.} The interpretation of the supply requirement in the \cite{wra40} was also significantly relaxed, especially following the development of the national electricity grid. It was no longer obvious, from a technical point of view, when exactly a hydropower development could be said to qualify as making a contribution to the local electricity supply. The electricity was not necessarily used locally. Indirectly, however, one could still argue that the local supply situation would improve whenever more electricity was supplied to the national grid.

\noo{ While the public spiritedness of hydropower development was arguably reduced, the rule that private parties could not expropriate riparian rights was still respected. It remained in place until 2000, when the law was changed to make it possible for any legal person to expropriate Norwegian rivers and waterfalls in order to develop hydroopower.\footnote{This change in the law was effected by an executive directive, not an explicit act of the legislature, as discussed in Section \ref{sec:twpp}.}

In light of this, the vast majority of cases dealing with waterfall expropriation under Norwegian law can not be looked at as takings for profit, even though they increasingly became economic development takings. Certainly, the desire for economic development played a crucial part in motivating state and municipality development projects in post-war Norway. But their activities in this regard were not themselves commercial in nature. Rather, supplying electricity was regarded as a public service, one that would in turn stimulate commercial activity in other areas of the economy.\footnote{See generally \cite{thue06b,skjold06}.}
}

%In the following subsection, I present the case law that developed during the post-war period. In light of how the courts have chosen to approach recent controversies, this body of case law is still highly relevant, even though the context of interference has changed as a result of liberalisation.

\subsection{The Supreme Court on the Rule of Reversion}\label{sec:prelib}

The period before liberalisation was not free from conflict regarding the legitimacy of measures undertaken to facilitate hydropower. Already the first major assertion of state control, embodied in the licensing acts of the early 20th century, resulted in significant controversy. At this time, there was a feeling of unease regarding the extent to which the state could regulate the hydropower sector without offending against the property clause in the Constitution.

This debate culminated in the conflict surrounding the rule of reversion that was introduced by the licensing acts passed between 1906 and 1917. As mentioned, the rule of reversion meant that in order to purchase riparian rights from private owners, the purchaser had to agree to a licensing condition stating that eventually, after at most 60 years, the state would acquire the waterfalls without paying compensation.

The question that arose was whether this should be regarded as a form of expropriation. If so, compensation would have to be paid pursuant to section 105 of the Constitution. This question resulted in fierce conflict, with some influential legal scholars attacking the rule as a ploy by the state to confiscate Norwegian rivers without compensating owners.\footnote{See \cite{morgenstierne14}.} However, in a 4-3 decision, the Supreme Court held that section 105 did not apply, since reversion was a licensing condition, not an independent act of property deprivation.\footcite{johansen18} No owner was compelled to hand over their rights to the state.\footnote{See \cite[406]{johansen18}.} Moreover, no owner was compelled to sell their rights. Rather, their willingness to do so was a precondition for the rule of reversion to apply.

One of the judges summed up the majority reasoning by commenting that he would not regard it as expropriation if the state were to forbid sale of riparian rights to private parties altogether.\footnote{See \cite[407]{johansen18}.} Why then, he asked, should it be regarded as expropriation if such a sale was allowed to take place only on specific conditions? Against this, the minority argued that reversion as a licensing requirement was so severe that it had to be regarded as a {\it de facto} expropriation, known as a regulatory taking in US terminology.\footnote{See \cite[412-413]{johansen18}. For a brief discussion on regulatory takings, see Chapter \cite{chap:1} Section \ref{sec:x}.} In addition, the minority argued that as the purpose of the reversion rule was to ensure that water rights were eventually brought under state ownership, this rule itself could not be understood merely as an act of regulation. According to the minority, the rule also invoked the power of eminent domain.\footnote{See \cite[415-416]{johansen18}.} By contrast, the majority chose to regard the eventual transfer to state ownership as a secondary purpose only, which could justifiably be pursued on the basis of the state's regulatory power alone.\footnote{See \cite[407]{johansen18}.}

After the Supreme Court upheld the rule of reversion, the legal foundation for tight state regulation of the hydropower sector solidified. As discussed in the previous chapter, the state pursued increasingly complex hydropower projects after the Second World War. At this time, technological and economic advances also made it feasible to divert large amounts of water over great distances (typically through tunnels), to collect water from several different rivers in a common reservoir for joint exploitation. Such projects became known as ``gutter'' projects, and they grew greatly in scope during the post-War years. Since the relevant licensing procedure was covered by the \cite{wra17}, the practical importance of the expropriation authority in section 16 of this act also increased dramatically.\footnote{See \cite[11]{innst59}. This was a proposition to parliament regarding an amendment of the \cite{wra17}. The amendment proposed to remove an earlier rule that applied only to diversion regulations, whereby a license to divert water from a river should {\it normally} only be granted when the riparian owners in the source river agreed to the measure. This rule made licenses harder to obtain in the diversion cases. However, following the department's recommendation, the rule was removed in 1959. The department argued that the rule had an ``unfortunate effect'' on the administrative procedure in large-scale diversion cases, noting also the vastly increasing complexity and scale of typical diversion regulations. The minority in the parliamentary committee recommended against the amendment, noting that it would ``greatly increase'' the authority to expropriate waterfalls, contrasting with the expropriation rules in the \cite{wra40}, see \cite[14]{innst59}. The majority countered this argument by maintaining that the regulatory power of the state would be used to prevent any abuse of power, and that the practical significance of the amendment would be limited to ensuring a ``more rational'' procedural approach to large-scale applications, see \cite[14]{innst59}.}

As discussed in Section \ref{sec:x} of the previous chapter, the opposition to hydropower grew proportionally to the scale and complexity of typical development projects. The critical focus was often on environmental effects, but the interests of local people also featured in these debates. 
Moreover, local interest were often aligned with the environmental interests. In a situation when local owners could not themselves benefit commercially  from hydropower, their response was often to oppose it. 

The controversies regarding large-scale development culminated in the case of {\it Alta}, still arguably the most important Supreme Court precedent in the area of hydropower law.

\subsection{The {\it Alta} Controversy}\label{sec:alta}

The {\it Alta} case went before the Supreme Court in 1982 after a long period of high-intensity conflict going back to the mid-seventies.\footnote{See \cite{alta82}. For commentaries, see \cite{eckhoff82,boe83,hagvar88}.} In {\it Alta}, the affected local population largely lacked formal title to the property they sought to defend. This was because the development in question would take place in the northernmost part of Norway, in the native land of the Sami people.\footnote{For Sami law generally, see \cite{skogvang02}.}

Norway has a history of discrimination against the Sami, and as their culture is largely nomadic, their land rights were never formalised in private law.\footnote{See \cite[149-156]{ravna12s}} As a result, land and natural resources in the county of Finnmark are largely owned by the state, at least in the sense of the state appearing as the nominal {\it in rem} owner.\footnote{In the past 30 years, partly as a response to the controversy of the {\it Alta} case, there has been a gradual change in attitude, whereby the rights of the Sami people receives greater legal recognition. In 2007, formal title to most of the land in the county of Finnark was transferred to a special state agency which is regulated by a special statute that obliges it to manage the land with due regard to customary and prescriptive rights of aboriginal groups and local people. See generally \cite{bull07}.}

Due to the sensitive context of interference, the {\it Alta} plans met with particularly strong criticism, both from environmental groups and groups fighting for aboriginal rights. A broad political movement was mobilised in opposition to the plans, eventually resulting in several serious cases of civil disobedience.\footnote{This included hunger strikes and attempts at sabotage, see \cite[80-83]{nilsen08}. For the Alta controversy generally, see \cite{altawiki,hjorthol06}.} The case also came before the courts, as the local population and environmental groups claimed, primarily on the basis of administrative law, that the development licenses that had been granted were invalid.\footnote{See \cite{eckhoff82}.}

The {\it Alta} case did not involve expropriation of the right to harness hydropower. Hence, one might think that the case has limited relevance to the issues addressed in this thesis. However, because of the priority given to the licensing procedure over specific expropriation procedures, the principles expressed in {\it Alta} also largely determine the legal position of waterfall owners whose rights to hydropower are expropriated.\footnote{See \cite{sauda09,jorpeland11}.}

{\it Alta} was admitted to the Supreme Court in plenum, directly on appeal from the district court.\footnote{This is a special arrangement available in cases that raise important questions of principle, cf., \cite[30-2]{cda05} and \cite[5]{ca15}.} The presiding judge commented that as far as he knew, it was the longest and most extensive civil case that the Court had ever heard.\footcite[254]{alta82} In an opinion totalling 138 pages, the Court considers a long range of objections against the development licenses, all of which are either rejected or held to provide insufficient reasons to declare the licenses invalid.

The opponents of the {\it Alta} development also argued on the basis of human rights and international law.\footnote{First, on the basis of articles 1 and 27 of the \cite{fnp}. Second, on the basis of \cite{ilo107} (later replaced by \cite{ilo169}). Third, on the basis of P1(1) of the \cite{echr}.} As noted by Eckhoff, these arguments raised subtle legal questions about how to apply the relevant principles of international law to a concrete dispute over hydropower development.\footnote{See \cite[351-352]{eckhoff82}. One of the most important international instruments, namely ILO Convention No 107, was not ratified by Norway at the time of {\it Alta} (Norway later ratified its replacement, ILO Convention No 169). However, it was argued that it had the status of customary international law. See generally \cite{eide80}.}
However, the Court refused to consider such  questions, finding that the negative effect of the hydroelectric plant was not so severe as to raise  human rights issues.\footnote{See \cite[299-300]{alta82}. See also \cite[351-352]{eckhoff82}.}

Instead, the Supreme Court approached the case on the basis of administrative law, focusing on the  procedural rules of the \cite{wra17}. In this regard, the opponents of the {\it Alta} development had pointed to a large number of purported shortcomings of the decision-making process. 

First, it had been argued that the original licensing application did not meet the requirements stipulated in section 5 of the \cite{wra17}. Essentially, the original application contained little more than technical details about the planned development, with hardly any identification or assessment of deleterious effects.\footnote{See \cite[264-265]{alta82}.} This shortcoming had been openly acknowledge by the water authorities themselves, who had nevertheless initiated a public hearing.\footnote{See \cite[265]{alta82}.}

The Supreme Court concluded that this was ``clearly unfortunate''.\footcite[265]{alta82} However, several reports and assessments had subsequently been provided, to fill the gaps left open by the initial application. For this reason, the Supreme Court held that the initial mistakes were irrelevant, since it was the licensing process as a whole that should be assessed.\footnote{See \cite[265-266]{alta82}.} Shortcomings at specific stages in the assessment would not be given weight unless they could be seen to imbue the process with a dubious character overall.\footcite[265]{alta82}

The Court then moved on to assess whether the process as a whole fulfilled procedural requirements, particularly those laid down in sections 5 and 6 of the \cite{wra17}. In addition, it had to be considered whether the assessment of the licensing criteria in section 8 of the \cite{wra17} had been sufficiently detailed.\footnote{Recall section 16 of the \cite{paa67}, requiring assessments to be as detailed as ``possible''.}

In this regard, those who objected to the licenses pointed to a range of negative effects that they believed had not been considered, or had not been considered in enough depth. In relation to nomadic reindeer interests, it was argued that the water authorities had failed to adequately consider the indirect negative consequences of hydroelectric development on reindeer farming in the local area.\footnote{See \cite[176-179]{alta82}.} These effects were described as ``practically catastrophic'' by some expert witnesses.\footnote{See \cite[278]{alta82}.} By contrast, the water authorities had not devoted much attention to the possibility of indirect consequences, citing the difficulty (described as an ``impossibility'') of attempting to quantify such effects.\footnote{See \cite[277]{alta82}.}

After considering the reports and assessments in some depth, the Supreme Court did not find fault with the procedure in this regard. Importantly, the Court stressed that the water authorities had been well aware of the possibility of indirect negative consequences. The water authorities had simply chosen, as a matter of expert discretion, not to place much weight on such consequences.\footnote{See \cite[279]{alta82}.} This, according to the Supreme Court, could be regarded as an expression of disagreement with those claiming that the effects would be catastrophic.\footnote{See \cite[278]{alta82}.} As a result, the grounds for claiming procedural error disappeared, as the lack of attention directed at indirect consequences was held to reflect an (implicit) factual assessment to the effect that they where not particularly severe.

The structure of the argument used here is more interesting than the factual question. The argument structure, in particular, serves to recast a lack of assessment, a possible procedural error, as an exercise of factual discretion.

The {\it Alta} Court also made some apparent statements of principle in this regard. In particular, the Court held that since the licensing decision itself is discretionary, it is appropriate to grant the executive some margin of appreciation also with regard to the question of how to interpret vague requirements of administrative law.\footnote{See \cite[262-264]{alta82}.}

%By contrast, the appellants argued on the basis that the content and scope of procedural rules is a question for the judiciary. This line of argument was described by the {\it Alta} Court as ``very formal''.\footnote{See \cite[262]{alta82}.}

The Court made a second decision of principle when it supported the state's contention that the administrative licensing assessment did not have to be as thorough as that required in a subsequent appraisement dispute.\footnote{See \cite[279|330]{alta82}.} Hence, the meaning of the obligation to clarify cases to the best possible extent is put into perspective: assessments of deleterious effects may sometimes be omitted at the decision-maker's discretion, also in circumstances when such assessments will be needed later to clarify the owners' actual loss for the purpose of calculating compensation.

In relation to the negative effects on fishing, the Court conceded that the assessments could have been better, but went on to point out that the purpose of assessment was only to answer yes or no to development, not to give a detailed presentation of its effects.\footcite[330]{alta82} Crucially, the Court noted that if additional negative effects were uncovered after the licences had been granted, this could be addressed through compensation payments and future regulatory measures.\footcite[330]{alta82}

In effect, the risk of factual error is downplayed by making reference to the owners' compensation right and the regulatory power of the state. This echoes the dichotomy mentioned in Chapter \ref{chap:3}, whereby there is a tendency in Norwegian law to perceive the interests of local people as revolving around financial entitlements.

In {\it Alta}, the Court agreed that erroneous information had been provided in relation to some issues, particularly regarding alternative ways to meet the need for electricity in Finnmark.\footnote{See \cite[346-357]{alta82}.} However, the Supreme Court did not regard the factual errors in this regard as relevant to the licensing decision.\footnote{See \cite[346]{alta82}.}

Here a third clarification of principle took place. The Court held, in particular, that the duty to consider alternatives -- different ways in which the public purpose could be satisfied -- is very limited in hydropower cases.\footnote{See \cite[346]{alta82}.} This position of principle, in turn, was the key building block that the Court used to argue that errors and inadequacies in the information provided about alternatives were irrelevant.\footcite[346]{alta82} 

The Court's perspective in this regard was at odds with how parliament has actually approached the case. There was little doubt that the favourable political assessment of the {\it Alta} development depended heavily on the perceived electricity crisis in Finnmark and the supply situation in Norway generally, as well as the perceived inadequacies of alternative solutions.\footnote{See \cite[338-347]{alta82}.}

In relation to this question, the legal counsel acting for the state in {\it Alta} openly argued that as these aspects came into focus only at the political stage of the decision-making, they were largely irrelevant.\footcite[341]{alta82} This line of argument is rather striking, since the decision to grant the license was made by parliament, which had indeed dealt with the case on three separate occasions.\footnote{See \cite[342]{alta82}.}

The Supreme Court did not address the state's arguments in this regard explicitly. However, it is worth noting how briefly the Court comments on the issue of factual errors concerning alternatives.\footnote{See also the surprise expressed in \cite[349-351]{eckhoff82}.} By contrast, the Court goes into painstaking detail regarding issues that seem to have been far less important to the political decision-makers.

%In relation to the duty to assess alternatives, the Court says nothing expect that the duty is very limited. For the details, which demonstrate factual inadequacies in the material given to the political decision-makers, the Court only refers briefly to the state's arguments. These arguments, based on the contention that the inadequacies were not significant, is accepted with no further discussion.\footcite[346]{alta82}

The dismissive attitude towards the duty to correctly assess alternatives is a controversial aspect of the {\it Alta}-decision.\footnote{See \cite[311]{haagensen02}. For criticism of the Supreme Court on this point, see \cite[580-584]{backer86}.} More generally, the decision in {\it Alta} has met with criticism from commentators arguing that the decision shows the extent to which the courts in Norway tend to identify themselves with other organs of state.\footnote{See \cite[64]{graver88} (commenting also that ``government prestige'' was at stake).} 

Due to the perceived weaknesses of the decision, some argued that {\it Alta} would have a limited impact as a precedent.\footnote{See\cite[580-584]{backer86}.} This has not proved to be the case. Indeed, {\it Alta} continues to receive favourable citations by the Supreme Court. It appears to enjoy significant influence, not only in hydropower law, but in administrative law more generally.\footnote{See \cite{ambassade09,jorpeland11}.}

It should be mentioned, however, that after the {\it Alta} decision, the legal position of the Sami people have greatly improved.\footnote{See generally \cite{gauslaa07}. Gauslaa presents the emergence of {\it Sami law}, a collection of rules and principles serving to protect established land use patterns and the Sami way of life while also giving the Sami people a better opportunity to partake in decision-making processes that affect them as group.} Moreover, the controversy surrounding {\it Alta} has been regarded as a catalyst for change in this regard.\footnote{See \cite[156]{ravna12s}.} Hence, it is unlikely that the courts today would be as quick as the {\it Alta} court to dismiss arguments based on aboriginal rights.\footnote{See \cite[180]{gauslaa07}.}

With regard to the position of local owners, on the other hand, the {\it Alta} decision is regarded to express key principles that still apply.\footnote{See \cite{jorpeland11}. See also \cite[312]{haagensen02}.} That said, administrative practices have changed since {\it Alta}, arguably also in direct response to the criticism that was directed at this decision.\footnote{See \cite[122-123]{backer10}.} Today, the assessment of licensing applications typically involve a more extensive assessment of environmental effects and possible alternatives that are less environmentally intrusive.\footnote{For an overview of current procedures, see Chapter \ref{chap:3} Section \ref{sec:x}. See also \cite[625-659]{backer86}.} However, as I will discuss in more depth below, the position of local owners during the assessment stages appears unaltered by the increased intensity of assessment regarding the environment. 

In particular, the key administrative procedures in hydropower cases have not been adapted to account for the liberalisation of the electricity sector. Today, key characteristics both of the typical taker and the typical owner are very different than they were at the time of {\it Alta}. First, as discussed in the previous chapter, takings of waterfalls now have the effect of depriving original owners of a resource that they could potentially develop themselves. Second, as I discuss in the next section, takings of waterfalls have become pure takings for profit.

\section{Taking Waterfalls for Profit}\label{sec:twpp}
\noo{ 
As I mentioned in the previous section, private companies could not expropriate waterfalls in Norway prior to the passage of the \cite{wra00}. Moreover, the public purpose requirement was formulated quite strictly, particularly in cases when the development was undertaken by municipality companies. I also mentioned how the hydropower sector developed after the Second World War, from a sector dominated by private and municipality companies, to a sector dominated by the state. This development, in turn, was accompanied by increased conflicts and doubts regarding the legitimacy of the established licensing procedures, particularly the highly centralised nature of the decision-making process. 

Even so, the debate at this time was still very much anchored in a system that presupposed political management of the hydropower sector as a public service provider. Importantly, the conflicts rarely, if ever, involved significant commercial interests on the part of the local riparian owners. Many critics argued that the fiscal interest of the state could not be used to justify destruction of nature and local patterns of land use. But in financial terms, the value of what was destroyed was typically negligible compared to the value of the hydropower development.

As a result, controversies relating to the legitimacy of interference involved riparian rights only at their periphery. More focused conflicts involving such rights arose in relation to the question of compensation, but the issues typically discussed in this regard were also of relatively minor structural importance.

In Chapter \ref{chap:3}, I presented the reform of the energy sector of the early 1990s, after which hydropower development has been regarded as a commercial pursuit. Following the regulatory reform, a new general statute dealing with water resources was also proposed, eventually leading to the passage of the \cite{wra00}. This Act provided the first authority for the state to allow developers to take waterfalls compulsorily for profit. Moreover, it made possible the later executive directive by which waterfalls could be expropriated and handed over to {\it any} legal person, including private companies.
}

After the legal and regulatory reforms of the 1990s, takings of waterfalls for hydropower have become takings for profit. However, this change in the function of expropriation received little attention when these reforms were introduced. Moreover, when the \cite{wra00} was proposed, the new expropriation authority was not singled out for political consideration. In fact, the increased scope of expropriation was not mentioned at all when the Ministry presented their proposal to parliament. Rather, the new expropriation authority was described merely as a ``simplification'' of existing law.\footcite[223-225]{otprp39}

This was grossly inaccurate. For the first time in Norwegian history, private commercial interests would be able to expropriate waterfalls. The original proposal to this effect stemmed from the report handed to the Ministry by a commission appointed to prepare a new act relating to water resources. The commission report mentions briefly that the proposed expropriation authority would imply increased scope for expropriation. However, it does not discuss the desirability of this in any depth.\footcite[235-237]{nou94} 

The report totals almost 500 pages, but devotes only three of those pages to discussing the new expropriation authority. Here the committee notes that a range of different authorities for expropriation has long co-existed in the law, with many of them positing strict and concrete public interest requirements as a precondition for granting a license. This, the commission argues, is not a very ``pedagogical'' way of providing expropriation authorities.\footcite[235]{nou94} Moreover, the commission notes that it runs the risk of omitting important purposes for which expropriation should be possible. Hence, the commission proposes to replace all older authorities by a sweeping authority that makes expropriation possible for any project that involves ``measures in watercourses''.\footcite[235-236]{nou94}

The commission comments that their formulation might seem wide, but remark that this is not a problem since the executive can simply refuse to issue an expropriation order when they regard expropriation as undesirable.\footcite[235]{nou94} The commission does not reflect on the  constitutional consequences of such a perspective, neither in relation to property rights nor in relation to the balance of power between the legislature, the executive and the courts. Instead, the commission offers a brief presentation of the rationale behind dropping the local supply restriction for municipal expropriation. They comment that these rules complicate the law and might make desirable expropriations impossible.\footcite[235]{nou94} Nothing is said to clarify what kind of desirable expropriations the committee think might be left out. 

Importantly, the committee do not relate their proposals to the recent market-based reform of the energy sector. Hence, the obvious practical consequence of their proposal, namely that expropriation of waterfalls would be made available as a profit-making mechanism, is not discussed or critically assessed.

The issue of {\it who} should be permitted to benefit from an expropriation license is also dealt with very superficially. In this regard, the commission structure their presentation around the so-called {\it redemption} rule of the \cite{wra40}. As mentioned briefly in Section \ref{sec:twp}, this rule made it possible for the majority owners of a waterfall to compulsorily acquire minority rights, if this was necessary to facilitate hydropower development. Hence, it was a rule that provided only a limited opportunity for private takings, restricted to owners themselves or external developers that had been able to reach a deal with a locally based majority.

The main justification given by the commission for introducing a general private takings authority is that the special redemption rule had not been much used in practice.\footcite[236]{nou94} Why this is an argument in favour of opening up for private expropriation in general is not made clear. Indeed, it seems just as natural to regard it as an argument {\it against} doing so. Why extend the possibility for private expropriation if the demand for such expropriation has been limited? 

Presumably, the commission thought there would be a demand for such expropriation in the future, but this is not stated explicitly, nor is the appropriateness of it discussed. As to the requirement that private takers must already control a majority of the waterfall rights in the local area, the commission only remarks that it regards such a restriction as old-fashioned.\footcite[236]{nou94} No discussion is offered regarding the consequences for local owners.

Since the passage of the \cite{wra00}, it has become clear that the new authority for expropriation is a highly significant and controversial aspect of the act. Today, practically all cases of waterfall expropriation imply that local owners are deprived of a small-scale development potential in favour of a commercial company.

\subsection{\it Sauda}

In {\it Sauda}, a case before the court of appeal,  the riparian owners formally protested a license that granted a private company the right to expropriate their rivers and waterfalls.\footnote{See \cite{sauda09}.} The owners' principal argument was that the executive could not grant such a right to a private party, since this had not been sanctioned by parliament. 

This argument appeared weak, since the \cite{ea59} had been amended to ensure that the executive would be authorised to decide what legal persons could expropriate for hydropower purposes. However, the owners argued that the executive had not appropriately informed parliament that this would be the consequence of the amendment. In particular, the amendment itself had been passed as a mere formality following the adoption of the \cite{wra00}. 

The owners presented the written testimony of two members of the parliamentary committee that had prepared the Act. Neither of them could recollect that they had been aware that the Act would make private expropriation possible. This was not conveyed to them by the executive. Moreover, it was not explicitly stated anywhere in the Act itself. Rather, it followed implicitly from three different sections in two separate acts. In the entire collection of preparatory documents, the change was discussed only once, in the report from the committee to the Ministry. 

On this basis, the owners argued that the purported expropriation authority was not constitutionally valid, since parliament had not intended it. Unsurprisingly, this argument was rejected. According to the court, it had to be assumed that parliament understood the consequences of their own legislative actions. 

However, while the owners lost the validity dispute, the level of compensation they received was dramatically increased compared to earlier practice. Because of this, the development company appealed the decision to the Supreme Court, with the owners lodging a counter-appeal regarding the question of legitimacy. The Supreme Court decided not to hear the case. 

Indeed, it had recently addressed the compensation question in the case of {\it Uleberg}.\footnote{See \cite{uleberg08}.} Here a new principle of market-value compensation was introduced, for those cases when small-scale development by owners was deemed to have been ``foreseeable'' in the absence of expropriation.

If this requirement is met, the owners can now expect at least 10-20 times more in compensation than they would get under the traditional approach, which was based on a  theoretical estimation of the value of the riparian rights.\footnote{For a more in-depth presentation of compensation issues, I refer to \cite[71-76]{dyrkolbotn15}.}

In addition to raising the compensation issue and the issue of constitutional legitimacy, {\it Sauda} raised procedural questions. The owners argued that mistakes had been made and that the administrative expropriation decision was therefore invalid. The court did not agree. 

The procedural arguments made here foreshadow the later case of {\it Jørpeland}. Here the owners were initially successful, as the district court held that the decision to expropriation was invalid due to procedural mistakes. In particular, the district court held that the practices  developed during the monopoly era were no longer appropriate.\footnote{See the decision by Stavanger Tingrett in \cite{jorpeland09}.} 

This decision was overturned on appeal, a decision that was in turn upheld by the Supreme Court. Eventually, the case was decided by an application of the precedent set by {\it Alta}, demonstrating the continued importance of this case in the context of expropriation for commercial development.\footnote{See the decision by Gulating Lagmannsrett in \cite{jorpeland11a}. See also the Supreme Court decision in \cite{jorpeland11} ({\it Jørpeland}).} In the next section, I will consider the {\it Jørpeland} case in depth, to bring out how administrative practices relating to expropriation for hydropower can work in practice.

\section{{\it Ola Måland v Jørpeland Kraft AS}}\label{sec:jorpeland}

The expropriating party was a public-private commercial partnership, Jørpeland Kraft AS. This limited liability company is jointly owned by Scana Steel Stavanger AS, with 1/3 of the shares, and Lyse Kraft AS, with the remaining shares.\footnote{See \cite[2]{jorpeland09}.} The former is a private steelworks company located in the small town of Jørpeland in Rogaland county, south-western Norway. Historically, this company was a major employer in Jørpeland, which is located by the sea, next to a mountainous area.

The main source of energy for the steel industry in Norway is hydropower and Scana Steel Stavanger AS was no exception. The company used energy harnessed from the rivers in the area, particularly the river which reaches the sea near Jørpeland. Moreover, the water from this river is supplemented by water from other rivers in the area that are diverted so that it can be exploited along with the water from the Jørpeland river.

Recently, Norwegian steel companies have become less profitable, due largely to increased foreign competition and a significant increase in costs of operation associated with this type of industry in Norway.\footnote{Salary costs, in particular, have become prohibitive. See, e.g., \emph{Information Booklet about Norwegian Trade and Industry}, published by the Ministry of Trade and Industry in 2005.} This has led to many such companies shifting their attention away from labour-intensive steel production, focusing instead on producing electricity, selling it directly on the national grid. Jørpeland Kraft AS was established as part of such a move to exploit the energy resources in Jørpeland.

The role played by the majority shareholder, Lyse Kraft AS, is important in this regard. Indeed, as I discussed in Chapter \ref{chap:3}, Norwegian law favours companies where the majority of the shares are held by the state or the municipalities. Lyse Kraft AS operates for profit, organised as a limited liability company, but it is publicly owned, with the city of Stavanger as the main shareholder. Hence, it is a very valuable partner to Scana Steel. In addition to being under public ownership, Lyse Kraft AS is responsible for the electricity grid in the region, so is well-positioned to access the electricity market.

%Lyse Kraft AS was established as a merger between several former monopoly companies in the Stavanger region. These were all reorganised following liberalisation of the sector in the early 1990's. As discussed in Chapter \ref{chap:3}, old monopolists still enjoy considerable power and influence, particularly with regards to state bodies that are tasked with regulating the industry. This is another important reason why they can serve as valuable partners for private companies wishing to make a profit from Norwegian rivers and waterfalls.

As attention shifts from harnessing rivers for the purpose of industrial production to the purpose of producing electricity to sell on the national grid (and, increasingly, to export abroad), new variables determine the degree of profitability. On the cost side, what matters most is the one-time investment required to construct the hydropower plant.\noo{\footnote{For an overview of the considerations made when assessing the commercial value of hydropower, I point to \cite{jensen07}. In fact, due to the importance that small-scale hydropower has assumed in recent years, investigating models for investing in such projects has become an active field of research in Norway, see for instance \cite{investment}.}} Maintaining and operating a hydropower station tends to be comparatively inexpensive. On the income side, what matters is the price of energy on the electricity market, a market that is no longer anchored in local conditions of supply and demand.

Importantly, as long as energy production is the sole focus, the business no longer depends in any significant way on the local labour force. Hence, large-scale exploitation becomes much more profitable than the medium or small-scale power plants that would otherwise be suitable for local industrial exploits. Indeed, it was in keeping with a general trend in Norway when Jørpeland Kraft AS, following  their new commercial strategy, proposed to undertake measures to increase their energy output. This could be achieved relatively cheaply, by further constructions aimed at diverting water away from nearby rivers into dams that were already built to collect the water from the Jørpeland river.

\subsection{The Facts of the Case}

One relatively small river from which Jørpeland Kraft AS suggested to extract water was not located in Jørpeland. Rather, it runs through the neighbouring municipality of Hjelmeland, on the other side of a mountain range, until it eventually reaches the sea at Tau, another neighbouring municipality. 

The plans to divert the river would deprive the riparian owners of water along some 15 km of riverbed, all the way from the mountains on the border between Hjelmeland and Jørpeland, to the sea at Tau. Not all the water would be removed, but the flow of water would be greatly reduced in the upper part of the river known as {\it Sagåna}, the rights to which is held jointly by Ola Måland and five other local farmers from Hjelmeland.

The water in question comes from a lake called \emph{Brokavatn}, located 646 meters above sea level, where altitude soon drops rapidly, making the river suitable for hydropower development. Plans were already in place for such a project, which would use the water from just below the altitude of Brokavatn, to the valley in which the original owners' farms are located, about 80 meters above sea level. 

A rough estimate of the potential of this project was made by the NVE itself, stating that the energy yield would be 7.49 GWh per annum.\footnote{See \cite[16]{jorpeland09}.} This is about five times more energy than the water from Brokavatn would contribute to the project proposed by Jørpeland Kraft AS.\footnote{See \cite[19]{jorpeland09}.}

Importantly, the estimate was not made in relation to the expropriation case, but as part of a national project to survey the remaining energy potential in Norwegian rivers.\footnote{The survey was carried out in 2004 and its results are summarised in \cite{jensen04}.} Ola Måland and the other owners of the river were not identified as significant stakeholders and were not notified of the assessment that had been made. Moreover, even after Jørpeland Kraft AS had submitted a formal application for permission to divert the water, the owners were not notified by the water authorities.\footnote{See \cite[16]{jorpeland09}. However, a generic orientation letter was apparently sent by Jørpeland Kraft AS, a letter that the owners themselves could not remember having received. See \cite[5|8]{jorpeland11a}.}

Moreover, the procedural approach to the case was the traditional one, with an assessment directed at evaluating the environmental impact. Many interest groups were called on to comment on environmental consequences, and public debate arose with respect to the balancing of commercial interests and the desire to preserve wildlife and nature.\footnote{See \cite[19]{jorpeland09}.}

One of the local owners, Arne Ritland, also commented on the proposed project. He did this in an informal letter sent directly to Scana Steel Stavanger AS.\footnote{See \cite[17]{jorpeland09}.} In this letter, he inquired for further information and protested the proposed diversion of water from Brokavatn. He also mentioned the possibility that an alternative hydropower project could be undertaken by original owners, but he did not go into any details, stating only that a locally owned hydropower plant had previously been in operation in the area. 

The plant he was referring to dates back to the time before there was a national grid. It ensured a local supply of electricity, but has since been shut down, in keeping with the general trend mentioned in Chapter \ref{chap:3}.

Arne Ritland received a reply from Scana Steel Stavanger AS, which stated that more information on the project and its consequences would soon be provided. Ritland did not pursue the matter further at this time. Meanwhile, Scana Steel Stavanger AS submitted his letter to the NVE, who in turn presented it as a comment directed at the application.\footnote{\cite[18]{jorpeland09}.}

This prompted the majority owner of Jørpeland Kraft AS, Lyse Kraft AS, to undertake their own survey of alternative hydropower in Sagåna.\footnote{See \cite[19]{jorpeland09}.} The conclusions were sent to the water authorities, but the owners were not informed that such an investigation was being conducted.\footnote{See \cite[23]{jorpeland09}.} Moreover, the water authorities did not take steps to investigate the commercial potential of local hydropower on their own accord. Instead, they referred to the conclusion presented by Jørpeland Kraft AS, stating that if the local owners decided to build two hydropower plants in Sagåna, then one of them, in the upper part of the river, would not be profitable, neither with nor without the contested water. The other project, in the lower part, could apparently still be carried out, even after the diversion.\footnote{See \cite[23]{jorpeland09}.}

No mention was made of what the original owners stood to loose, nor was there any argument given as to why it made sense to build two separate small-scale power plants in Sagåna. Nevertheless, the NVE handed the expropriating party's findings over to the Ministry, without conducting their own assessment and without informing the original owners.\footnote{See \cite[22-23]{jorpeland09}.}

In addition to the report made by Jørpeland Kraft AS, the municipality government of Hjelmeland also commented on the possibility of local hydropower. In their statement to the NVE, they directed attention to the data in the NVE's own national survey, which suggested that a single hydropower plant in Sagåna would be a highly beneficial undertaking.\footnote{See \cite[19]{jorpeland09}.} On this basis, they protested the diversion, arguing that original owners should be given the possibility of undertaking such a project.

This statement was not communicated to the original owners, and in their final report the NVE dismissed it by stating that the most efficient use of the water would be to transfer it and harness it at Jørpeland.\footnote{See \cite[19]{jorpeland09}.}

In addition to the statement made by Ritland, one other property owner, Ola Måland, commented on the plans.\footnote{See \cite[17]{jorpeland09}.} He did so without having any knowledge of the commercial potential of the waterfall and without having been informed of the statement made by the municipality of Hjelmeland. Therefore, Måland expressed his support for Jørpeland Kraft's plans, citing that the risk of flooding in Sagåna would be reduced.\footnote{He later joined the other owners in opposition to the expropriation.} He also phrased his letter in such a way that it could be interpreted as a statement on behalf of the owners as a group.\footnote{See \cite[17]{jorpeland09}.} However, Måland was the only person who signed.

In the final report to the Ministry, the NVE refer to Måland's letter and state that the original owners are in favour of the plans.\footnote{See \cite[19]{jorpeland09}.} For this reason, the NVE concludes that the opinion of the municipality of Hjelmeland should not be given any weight.\footnote{See \cite[19]{jorpeland09}.} The NVE neglects to mention that Arne Ritland's statement strongly opposed expropriation. Moreover, earlier in the report, where all incoming statements are reported, Ritland is referred to as a private individual, while Ola Måland is referred to as a property owner who speaks on behalf of the owners as a group.

The report made by the NVE was not communicated to the affected local owners at all, so the owners had no chance of correcting mistakes. However, the report was sent to many other stakeholders, including the municipality of Hjelmeland.\footnote{See \cite[24]{jorpeland09}.} In light of the report, the municipality changed their original position and informed the Ministry that they would not press for local hydropower, since this was not what the affected owners (i.e., Ola Måland) wanted.\footnote{See \cite[24]{jorpeland09}.}

This happened without the owners' knowledge. However, while the case was being prepared by the water authorities, the original owners had begun to seriously consider the potential for hydropower on their own accord. In late 2006, Jørpeland Kraft's application reached the Ministry and a decision was imminent. At the same time, the owners were under the impression that they would receive further information before the case progressed to the assessment stage.

As the owners had now come to realise the commercial value of the water from Brokavatn, they approached the NVE, inquiring about the status of the plans proposed by Jørpeland Kraft AS. They were subsequently informed that an opinion in support of the transfer had already been delivered to the Ministry. This communication took place in late November 2006, summarised in minutes from meetings between local owners, dated 21 and 29 November.\footnote{Available upon request.} On 15 December 2006, the King in Council granted a concession for Jørpeland Kraft AS to transfer the water from Brokavatn to Jørpeland.\footnote{See \cite[3]{jorpeland09}.}

At this point, it had become clear to the original owners that the water from Brokavatn would be crucial to the commercial potential of their own project. They also retrieved expert opinions that strongly indicated that the NVE was wrong when they concluded that diverting the water would be the most efficient use of the water.\footnote{See \cite[23]{jorpeland09}.} In light of this, the owners decided to question the legality of the licence (with the corresponding permission to expropriate). They argued, in particular, that the administrative decision to grant the license was invalid.

\noo{ According to the owners, the expropriation was materially unjustified. Moreover, they contended that the administrative process had not fulfilled procedural requirements. 

The district court, Stavanger tingrett, held in favour of the owners.\footnote{See \cite{jorpeland09}.} The court emphasised that the authorities were obliged to properly take into consideration the fact that the riparian rights in Sagåna could have been exploited commercially by the owners.\footnote{See \cite[14]{jorpeland09}.} Moreover, the court noted that in order to ensure a proper assessment, the authorities should have ensured that the owners were kept informed, so that they would have an opportunity to comment in a timely fashion.\footnote{See \cite[22]{jorpeland09}.} The court also noted that the administrative decision was based on factual mistakes, regarding both the owners' opinion of the plans and the energy potential of diversion plans compared to an owner-led project in Sagåna.\footnote{See \cite[25]{jorpeland09}.} This, the court held, meant that the concession had to be declared invalid.

This view was rejected by the court of appeal, Gulating lagmannsrett, which held that sufficient steps had been taken to clarify the commercial interests of the owners.\footnote{See \cite{jorpeland11a}.} Moreover, the court held that established practices regarding the preparation and evaluation of such cases -- dating from a time when it was not feasible for ordinary owners to undertake hydropower schemes -- still provided adequate protection.\footnote{See \cite[8]{jorpeland11s}.} The owners appealed the decision to the Supreme Court, who decided to hear the case. However, the appeal was eventually rejected, and the Supreme Court went even further than the court of appeal when they declared that established administrative practices were beyond reproach.\footnote{See \cite{jorpeland11}.}
}

In the following section, I present the main legal arguments relied on by the parties, as well as a summary of how the three national courts judged the case.

\subsection{Legal Arguments}\label{view}

First, the owners argued that procedural mistakes had been made by the water authorities when preparing the case.\footnote{See \cite[12]{jorpeland09}.} This, in turn, had resulted in factual mistakes forming the basis of the decision to grant the development license. Since the outcome might have been different if these mistakes had not been made, the owners concluded that the development license could not be upheld.

Second, the owners argued that expropriation of their rights would result in the loss of an economic development potential.\footnote{See \cite[5]{jorpeland11a}.} Moreover, they argued that the economic loss would be greater than the gain even for the public, since the owners were in a position to make more efficient use of the water rights in question. Therefore, allowing expropriation would only serve to benefit the commercial interests of Jørpeland Kraft AS, to the detriment of both local and public interests. On this basis, the owners contended that the decision to grant the licence was a manifestly ill-founded decision that could not be upheld.

Third, the owners argued that the government had not fulfilled its duty to consider the case with due care.\footnote{See \cite[12]{jorpeland09}.} In particular, the assessment of local community interests and the interests of local owners had not been satisfactory. Particular attention was directed at the fact that local owners had not been informed about the progress of the case, and had not been told of assessments pertaining to their interests.

Fourth, the owners argued that irrespective of how the matter stood with respect to national law, the expropriation was unlawful because it would be in breach of the provisions in P1(1) of the ECHR regarding the protection of property.\footnote{See \cite[07-08]{jorpeland09}.}

Jørpeland Kraft AS protested, arguing that there were no factual errors in the report from the NVE.\footnote{See \cite[16]{jorpeland11}.} The relevant assessments were those pertaining to the licensing question as a whole, and it would be incorrect to focus on specific elements.\footnote{See \cite[2]{jorpeland11a)}.} Moreover, it was likely that Måland had discussed the diversion of water with other affected owners, and that they had all agreed to support it.\footnote{See \cite[2]{jorpeland11a}.} Furthermore, it was argued that all the procedural rules of the \cite{wra17} had been observed. Other procedural rules might be relevant, but only if they are compatible with the rules in the \cite{wra17}.\footnote{See \cite[16]{jorpeland11}.} Jørpeland Kraft AS also argued that it was not for the courts to subject the assessment of public and private interests to any further scrutiny, since this was a matter for the administrative branch.\footnote{See \cite[2]{jorpeland11a}.} Finally, Jørpeland Kraft AS argued that diverting the water did not represent a breach of the owners' human rights.\footnote{See \cite[2]{jorpeland11a}.} They argued for this by pointing to the fact that the procedural rules had been followed and that the material decision was beyond reproach. Moreover, Jørpeland Kraft AS argued that since the owners would be compensated financially for whatever loss they incurred, it was clear that no human rights issues were at stake.\footnote{See \cite[2]{jorpeland11a}.}

The matter went before the district court in the city of Stavanger, which decided in favour of the owners on 20 May 2009.\footnote{See \cite{jorpeland09}.} In the following, I offer a presentation of the reasons given by this court, leading to the conclusion that the expropriation was unlawful and that the diversion of Brokavatn could not be carried out.

\subsection{The District Court}

The district court of Stavanger agreed with the original owners that the decision to grant the license was based on an erroneous account of the relevant facts.\footnote{See \cite[25]{jorpeland11}.} Moreover, the court concluded that it was evident that allowing the applicants to use the water from Brokavatn in their own hydroelectric scheme would be the most efficient way of harnessing the hydropower potential.\footnote{See \cite[22-23]{jorpeland09}.} This, the court noted, directly contradicted what the NVE had stated in their report.\footnote{See \cite[23]{jorpeland09}.}

The court backed up its decision by giving several direct quotes from the report made by the NVE. For instance, it quoted how the NVE had commented on the statement from the municipality of Hjelmeland:
%\begin{quote}Hjelmeland kommune ser helst at kraftressursene i vassdraget blir utnyttet av lokale %grunneiere. 
%Dette står i kontrast til uttalelsen fra grunneierne selv som ønsker at overføring blir gjennomført, 
%slik at flom og erosjonsskader kan bli noe redusert. NVE mener at den beste utnyttelsen med tanke 
%på kraftproduksjon vil være å tillate overføringen da en slik løsning vil innebære at vannet utnittes i 
%størst fallhøyde. Når dette samtidig er grunneiernes eget ønske har vi ikke tillagt Hjelmeland 
%kommunes synspunkt på dette noen vekt
%\end{quote}
%Our own translation follows below: 
\begin{quote}
The municipality of Hjelmeland would like the hydroelectric potential in the waterfall to be exploited by local property owners. This contrasts with the statement submitted by the property owners 
themselves, who wish that the transfer of water takes place, so that damage due to flooding can be 
somewhat reduced. NVE thinks that the best use of the water with respect to hydroelectric 
production is to allow a transfer, since this means that the water can be exploited over the greatest
distance in elevation. When this is also the property owners' own wish, we will not attribute any 
weight to the views of the municipality of Hjelmeland.\footnote{See \cite[12]{jorpeland09}.}
\end{quote}

The district court concluded that as this was a factually erroneous account of the situation, the decision made to allow transferral of the water could not be upheld. In light of this, the district court concluded that the decision to grant concession for diversion of water was invalid. Here the court relied on a well-established principle of administrative law: while the exercise of discretionary powers is usually not subject to review by court, a decision based on factual mistakes is invalid if it can be shown that the mistakes in question were such that they could have affected the outcome.\footnote{See \cite[407-410]{eckhoff14}. For the requirement that the mistakes must have been such that they could have affected the outcome, see \cite[41]{paa67}} Since the small-scale alternative would in fact represent a more effective use of the water in question, the court was not in doubt that this principle applied here.\footnote{See \cite[25]{jorpeland09}.}

Since the district court held that the license to allow diversion was invalid because it was based on factual mistakes, there was no need to consider claims regarding the legitimacy of the diversion with respect to human rights law. However, the district court did comment that the traditional procedure used to deal with diversion cases was inadequate and had to be supplemented by looking to the procedural rules in the \cite{ea59} and the \cite{paa67}.\footnote{See \cite[21]{jorpeland09}.} 

Moreover, the court made a crucial statement about expropriation of riparian rights in general, regarding the duty of the water authorities to properly assess whether or not an expropriation license should be granted.\footnote{The duty is a general principle of administrative law, expressed both in \cite[12]{ea59} and \cite[16]{paa67}.} This duty, the court held, included a duty to properly consider negative effects on small-scale development potentials.\footnote{See \cite[22]{jorpeland09}.} According to the court, this was the natural consequence of the increasing interest in small-scale development. If this principle had become part of Norwegian hydropower law, it would have had significant implications for the water authorities, directly confronting their traditional lack of interest in the expropriation question. However, it was not to be, as the court's decision was overturned on appeal.

\subsection{The Court of Appeal}

The court of appeal approached the case very different than the district court. Specifically, its decision does not rely on any close assessment of the facts and the report made by the NVE. Instead, the court of appeal largely base its decision on the opinion that the rules in the \cite{wra17} exhaustively regulate the administrative procedure.\footnote{See \cite[7]{jorpeland11a}.} According to the court of appeal, the procedural rules in the \cite{ea59} and the \cite{paa67} do not apply to diversions of water authorised under section 16 of the \cite{wra17}.\footnote{See \cite[7]{jorpeland11a}.} 

The reason given for this that the more specific rules of the \cite{wra17} have priority under the so-called {\it lex specialis} principle, which applies in case of conflict between different sets of rules.\footnote{See \cite[7]{jorpeland11a}.} Apparently, the court must have thought that there was such a conflict between general administrative law and the rules that apply specifically in hydropower cases.\footnote{The decision is not entirely clear on this point, however, as the court also makes a sweeping remark to the effect that the rules in the \cite{wra17} conform to all ``basic and general'' procedural demands of administrative law.} With regard to the procedural rules of the \cite{wra17}, the court does not go into much detail, but concludes that the assessment of the water authorities met all general requirements and was clearly adequate. Regarding the fact, the court is very brief, commenting only the following:

\begin{quote} It was not a mistake to take Ola Måland's statement into consideration, as he was, and still is, a significant property owner. NVE's statement to the effect that granting the concession will facilitate a more effective use of the water seems appropriate, as it refers to a current hydroelectric plant that exploits a waterfall of 13.5 meters.\footnote{See \cite[8]{jorpeland11a}.}
\end{quote}

Here the court mentions a hydroelectric plant further downstream which would be only very marginally affected by the transfer of water away from {\it Brokavatn}. The owners of this plant were not parties to the dispute (their interests were obviously quite negligible). Moreover, the downstream plant had nothing to do with the statement made by the municipality of Hjelmeland, which precipitated NVE's comments on the effectiveness of small-scale hydropower.\footnote{This is clear from decision of the  district court, which quotes from the statement submitted by the municipality, see \cite[16]{jorpeland09}.} The court of appeal never mentions this statement, nor the fact the small-scale alternative suggested there involved exploiting the contested water over a difference in altitude of about 550 meters.

Instead, the court of appeal points out that the NVE was well aware of the possibility of developing small-scale hydropower, was well-informed about such development, and had considered it during their assessment.\footnote{See \cite[9]{jorpeland11a}.} The court of appeal admits that the NVE's written assessment on this point was brief, but argues that this must be understood as a natural response to the lack of input from local owners.\footnote{See \cite[9]{jorpeland11a}.}

The owners appealed the court of appeal's decision to the Supreme Court, which decided to hear the  juridical aspects of the case.\footnote{See \cite[8]{jorpeland11}. Specifically, the Supreme Court would not engage in any independent factfinding, but only consider legal questions, including how the law should be applied to the facts.}

\subsection{The Supreme Court}

The Supreme Court approached the case in much the same way as the court of appeal. Regarding the facts, the Court emphasises that the majority owner of Jørpeland Kraft AS had considered the possibility that a hydroelectric scheme could be undertaken by local property owners.\footnote{See \cite[53]{jorpeland11}.} As mentioned, this resulted in a report based on the premise that the owners could have built two separate small-scale plants in the same river. The conclusion is that one of these would be unprofitable regardless of the diversion, while the other one could still be carried out.\footnote{See \cite[23]{jorpeland09}.} However, the report does not explain why anyone would want to build two consecutive small-scale plants in the same river, an approach that diverges from all other expert reports retrieved about small-scale potentials.\footnote{See \cite[16|23]{jorpeland09}.}

In any event, the most relevant question would obviously be what the owners stood to loose when the water from Brokavatn disappeared. Both the report and the Supreme Court remained silent on this. Moreover, the Court does not mention that the report was never handed over to the applicants, nor that the details of the calculations were never independently considered by the NVE. Just like the court of appeal, the Supreme Court also neglects to mention that small-scale development would be a more efficient use of the water, according to the national survey of small-scale potentials carried out by the NVE itself.\footnote{See \cite[16]{jorpeland09}.} Furthermore, no mention is made of the fact that the NVE claims that the opposite is true in the report to the Ministry, contradicting also the statement made by the municipality of Hjelmeland.

Regarding the legal questions raised by the case, the Supreme Court rejects the view that the procedural rules in the \cite{ea59} and the \cite{paa67} do not apply to the case.\footnote{See \cite[32-34]{jorpeland11}.} However, the Court holds that these procedural rules do not imply a more extensive duty to assess the expropriation question, compared to established practices in hydropower cases.\footnote{See \cite[51-52]{jorpeland11} (citing also the {\it Alta} case, \cite{alta82}).} 

There is no rule in the \cite{wra17} which states that the authorities are required to consider specifically the question of how the regulation affects the interests of property owners. Moreover, administrative practice suggests that this is not normally done, except to perhaps to a limited extent when the issue is raised during the hearing.\footnote{See \cite{stokker10}. This is the water authorities' own guideline for the assessment of large-scale applications. The previous version of this guideline (which also fails to mention the interests of owners) was presented to the Supreme Court. The Court also refers to it explicitly when it comments that existing practices are beyond reproach. See \cite[51]{jorpeland11}.} However, a rule explicitly demanding this is found in section 2 of the \cite{ea59}. This is not regarded as a procedural rule, however, as it pertains to the material considerations that the administrative branch is required to carry out in expropriation cases. 

Indeed, according to the Supreme Court, the rule does not apply at all when expropriation takes place on the basis of section 16 of the \cite{wra17}.\footnote{See \cite[30]{jorpeland11}.} This is the conclusion despite the fact that section 30 of the \cite{ea59} explicitly states that the provisions of that act apply to expropriations pursuant to the \cite{wra17}, in so far as they are compatible with the rules therein. It would appear to follow, by implication, that the Supreme Court does {\it not} think that directing more attention at owners' interests, as prescribed by section 2 of the \cite{ea59}, is compatible with the \cite{wra17}. 

This is a clear rejection of the principled position taken by the district court, whereby the water authorities should generally be obliged to consider small-scale alternatives before allowing expropriation. According to the Supreme Court, no special procedural obligations arise at all in such cases, compared to cases when the water power already belongs to the applicant. Expropriation is to remain a non-issue during the licensing process pursuant to the \cite{wra17}. 

Formally, this implication of {\it Jørpeland} only applies to expropriations carried out on the basis of section 16 of that act. However, in practice, there is reason to believe that it will have a uniform impact in all cases involving large-scale hydropower development. Indeed, the water authorities themselves do not appear to make any significant distinction between such cases based on whether or not a separate license to expropriate waterfalls is formally required.\footnote{See \cite{flatby08}.}

It also bears noting that the facts in {\it Jørpeland} appear to suggest that the procedural shortcomings underlying that case were much more obvious than the shortcomings complained of in {\it Alta} (although the scale of the underlying conflict was much greater in {\it Alta}). To further illustrate the extent to which {\it Jørpeland} signals a dismissive attitude towards owners and local communities, I will conclude by offering a quote from Harald Solli, director of the hydropower licensing section at the Ministry. Sollie submitted written evidence to the Supreme Court regarding the practices observed in cases involving expropriation of water power. Below, I quote two exchanges that demonstrate how current practices leave local owners in a precarious position.

\begin{quote}
Q: In cases pursuant to the \cite{wra17}, is it common for the water authorities to send prior written notices to the private owners that may be affected by a loss of small-scale hydropower potential? \\
A: The procedural rules that apply to cases pursuant to the \cite{wra17} are found in section 6. To give such a written notice to private owners is not required. As far as I am aware, it is also not done, but I have no first-hand knowledge of this, since the NVE is responsible for the case at this stage. \\
Q: In cases such as this, should owners affected by the loss of a small-scale hydropower potential be kept informed about the factual basis on which the authorities plan to make their decision? I am thinking especially about cases when the authorities do in fact provide an assessment of the potential for small-scale hydropower on private properties. \\
A: Affected owners must look after their own interests. The assessments made by the NVE in their report is a public document, and it can be accessed through the homepage of the NVE.
\end{quote}

By their reasoning in \emph{Jørpeland}, it appears that the Supreme Court gave this dismissive attitude towards local owners a stamp of approval. In light of this, I believe the study of the law in a socio-legal setting becomes all the more relevant. For while the dismissive attitude might be a part of the national legal order, it seems pertinent to ask if it is a reasonable attitude to take towards local owners of valuable natural resources. Also, one may ask if a case can be made with respect to human rights, by arguing that the protection awarded is insufficient with regard to P1(1). This point was raised in \emph{Jørpeland}, but did not receive any attention from the Supreme Court.\footnote{The {\it Jørpeland} case resulted in a complaint to the ECtHR which has yet to be considered by the Court.}

\noo{ \subsection{A Predatory Taking?}

Following \emph{Jørpeland}, I conclude that the liberalisation of the energy sector does not imply that original owners are entitled to increased protection or participation during decision-making processes regarding the use of rivers and waterfalls. This, at least, is the view held by the Norwegian judiciary. Of course, one should not overlook the possibility that the water authorities themselves will eventually adopt new practices regarding the assessment of such cases. So far, however, it seems that they stick quite closely to the established routine.

Hence, it seems reasonable to ask about the sustainability of these practices. In fact, I believe the case of \emph{Jørpeland} illustrates why the current system is inadequate, and how it can lead to decisions that appear ill-founded and leave the affected communities marginalised. The likelihood of factual mistakes, in particular, might increase greatly when the involvement of the local population in the decision-making process is not ensured.

More importantly, it seems that when decisions are made following the traditional procedure, it can often be hard or impossible to see any legitimate reason why the project proposed by the developer would be a better form of exploitation than allowing the local owners to carry out their own projects. In the case of \emph{Jørpeland}, it seemed that small-scale hydropower would be a better way of harnessing the water in question, even in the sense that it would be more efficient, providing the public with more electricity at a lower cost. More generally, unless the issue of alternative exploitation in small-scale hydropower is considered during the assessment stage, one risks making decisions that are not in the public interest at all. 

Additionally, this can send out the signal that expropriation of owners' rights is undertaken solely in order to benefit the commercial interests of the energy company applying for a development license. At this point, it seems appropriate to recall the concerns expressed by US Justice O'Connor in the case of {\it Kelo}, regarding the disproportionate influence of powerful commercial actors in takings cases.\footnote{As discussed in Chapter 1.} In relation to these concerns, a major point of contention is  whether or not Justice O'Connor's grim predictions about the fallout of the {\it Kelo} decision did indeed reflect a realistic analysis. 

Surely, anyone who agrees with Justice O'Connor that the powerful will usurp the power of eminent domain to the detriment of the poor, would also agree with her conclusion that it is perverse. However, whether her pessimism is warranted by empirical fact seems less clear. In this context, I believe the case of Norwegian hydropower serves a broader purpose, since it demonstrates that a loose interpretation of the public interest requirement can  indeed lead to transfers of property from those with fewer resources to those with more.

However, we also need to be clear about the fact that property has a social and political function that goes beyond the financial interests of individuals. For the Norwegian case at least, it seems particularly relevant to ask if local people, by virtue of their right to property and their original attachment to the land, have a legitimate expectation \emph{both} that their commercial interests should be protected, \emph{and} that they should be granted a say in decision-making processes. 

Protection of individual financial interests does not necessarily imply social protection, and the right to participate might be both more significant, and harder won, than the right to be compensated according to whatever the courts regard as the market value of the property in question.

\noo{ Another perspective, which I will pursue in more detail in the next chapter, is the question of how property rights relate to the overreaching goal of sustainable development of natural resources. Rather than seeing property rights as a means towards securing sustainable development, it is presently very common to see it as an impediment. This, indeed, has shaped much of the Norwegian discourse regarding environmental law and policy, including the law relating to hydropower.\footnote{For example, such a sceptical view of property rights appear to provide an overarching perspective on the law of sustainability in \cite{backer12} (a widely used textbook on environmental law in Norway, by one of the most influential jurists over the past 25 years).}

Moreover, a typical justification given for interference in property is that an equitable and responsible management of natural resources requires it. It seems, however, that an egalitarian system of private ownership of resources -- as we find in Norway -- could itself serve as a sustainable basis for management of these resources. In particular, private property rights seems like a potentially robust way in which local communities can be given a degree of self-determination concerning how to manage local resources. 

This is in itself considered desirable from the point of view of sustainability, but often, it appears that the favoured approach to securing it is through administrative arrangements that is supposed to empower local people. The voice that locals get, under such an arrangement, is heard at the discretion of the administrative branch. Hence, it is easily drowned in the context of large-scale commercial development, particularly when a narrative is established whereby this development is carried out for the ``common good''. I believe the case of Norwegian hydropower illustrates this point, and also shows that there is every reason to remain sceptical when commercial interests claim that they embody the public interest.

If property becomes a privilege for the few, rather than an obligation for the many, there is reason to expect that the level of vigilance will be diminished in this regard. In this way, takings for profit serve a double purpose in that they establish a culture of property which threatens to transform it from an egalitarian institution focused on virtue, to an elitist institution focused on entitlement.
The repercussions of this, I believe, extend well beyond the local communities that are adversely affected in the first instance. For example, it seems to me that an egalitarian property regime is one of the paramount guarantees we have that the state will be able to effectively and rationally exercise its regulatory powers, without bias. 

In a system where property no longer acts as an equaliser, but rather as a marker of political and commercial power, one must expect that the government itself will be more easily intimidated by commercial interests. This becomes particularly likely if such interests permeate the political system itself. The case of hydropower in Norway illustrates how this can happen, as a result of naive policies of state-ownership, aiming to protect the public interest, but effectively serving to render the public will subservient to the imperatives of the market. 

Moreover, a capitalist elite which commands significant regulatory power may not take lightly to what they perceive as undue political interference in their business practices. A company which is partly owned by the state, operates the electricity grid on behalf of the public, and is accustomed to expropriating property without meeting with much resistance, is likely to act with great confidence and boldness also when it faces political opposition. Much more so, one would presume, than a group of peasants, the original owners of waterfalls.

I think the case of \emph{Jørpeland} suggests that these perspectives are all relevant when considering takings for profits and their consequences. In the next chapter, I will turn to the question of how to ensure commercial development via compulsion by a different route, {\it without} incurring the negative effects associated with expropriation in such circumstances. Here I believe the fundamental challenge is one of setting up a decision-making structure that can ensure a balance between the various stakeholders. Importantly, the decision-making structure itself should recognise that local owners are those who have the most to loose, and gain, from undertakings involving their property. 

Indeed, I believe that the appropriate form of democratic decision-making involving property interests needs to reflect both the obligations and rights associated with those interests. The owners must not be conceptualised as an impediment to development, but as a potential driving force. Moreover, they cannot be conceived of merely as interested bystanders, but must be recognised as {\it the} primary actor in any collective action that seeks to transform their property so that it may serve a more valuable social function. No doubt, to achieve such a transformation may require compulsion. But it need not require outright expropriation, as I will attempt to demonstrate in the next chapter. The overriding concern is to develop strategies whereby the state can impose new property uses without undermining existing distributions of property rights and obligations among its citizens.
}
}
\section{Conclusion}\label{conc}

\noo{ In this Chapter, I have argued that the law relating to expropriation of waterfalls in Norway is based on a tradition that sees owners as profit-maximising and the state as welfare-seeking. }
In this Chapter, I have studied the law and practices relating to the taking of riparian rights under Norwegian law. I observed that the question of striking a balance between private and public interests is approached under the presumption that private property rights embody mainly private values, while public values are pursued through regulation that ensures public ownership and control. I tracked how this perspective shaped the law of expropriation of waterfalls, so that expropriation could only take place for narrowly defined public purposes and only to the benefit of public bodies.

I noted, however, how the increasing centralisation of the energy sector and the increasing scale of projects following the Second World War led to increased worry about the legitimacy of interference in property and the natural environment. I concluded that the ensuing conflicts failed to make much of an impact on the law relating to hydropower, which was still organised as a public service at this time. 
The degree of political control over the sector, which was perceived to be great, meant that courts shunned away from adopting a strict view on legitimacy. 

However, this did not mainly apply to the question of the authority to expropriate, which was hardly raised at all in the period between the reversion controversy of the early 20th century and the market-reform of the early 1990s. It applied mainly to general procedural rules. Here the Supreme Court adopted a stance whereby these rules were themselves considered largely ``discretionary'' in nature. Hence, it would fall under the authority of the executive to determine their scope and application in concrete cases.

I noted how this perspective has been maintained by the courts and the executive even after liberalisation. I argued that today, expropriation for hydropower development must be regarded as takings for profit, typical examples of economic development takings. I discussed how the law came to be changed on this point, with a dramatically widened expropriation authority introduced in conjunction with the \cite{wra00}.

I concluded with a description of the fallout from this, as expressed concretely in the case of {\it Jørpeland}. This case served to illustrate that administrative practices developed and sanctioned during the monopoly days are now applied uncritically in the context of competing commercial interests. As a result,  expropriation has become an important tool that the powerful market players can use to gain the upper hand in competition with locally based companies or smaller companies that rely on cooperation with owners. I noted that the law as it stands is unprepared for dealing with this dynamic. 

Still, in the case of {\it Jørpeland}, the Supreme Court explicitly denied that established practices were in need of revision. Moreover, the Court refused to reconsider the established interpretation of the scope of procedural rules in hydropower cases, rejecting arguments to the effect that these must now be understood to provide protection for waterfall owners that matches the protection offered to other affected parties.

\noo{ In the next Chapter, I will consider an aspect of the law were the Supreme Court {\it has} taken the view that a revision of established practices is in order, namely in relation to the question of compensation. I note, however, that the Court's emphasis on the compensation issue serves to reinforces the idea that private property rights pertain mainly to financial entitlements. As I have already argued, this perspective hardly does justice to the role of private ownership of waterfalls in Norway. }

In the next chapter, which is the last chapter of the thesis, I consider land consolidation as an alternative to expropriation. I will argue that it has great potential for successfully addressing holdouts among local owners without giving rise to many of the problems associated with using expropriation to facilitate commercial development of hydropower. However, as demonstrated in the present chapter, Norwegian courts do not seem willing to recognise the shortcomings of the current system. Until they do, or are directed to do so by political bodies or international tribunals, it is unlikely that expropriation law will evolve much from its current fixation on the compensation issue and its unshaken belief in public-private commercial partnerships as arbiters of the common good.
