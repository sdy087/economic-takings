\chapter{Taking Waterfalls}\label{chap:5}

\section{Introduction}\label{sec:5:1}

The Norwegian water authorities have extensive powers to take waterfalls for hydropower development. However, they rarely need to reflect on this power, not even when they use it. The reason is that expropriation tends to be an {\it automatic} consequence of a development license; those who obtain a license to develop a large-scale hydropower plant almost always obtain also a license to expropriate the private property rights they require for this purpose.

In some cases, this follows from section 16 of the \cite{wra17}, which gives license holders a right to expropriate all property rights needed for the development in question.\footnote{As mentioned briefly in the previous chapter.} However, even outside the scope of these provisions, the same approach to expropriation tends to be adopted. Specifically, the authorities adhere to the presumption that whenever a license to undertake large-scale development should be granted, then so should a license to expropriate.\footnote{The leader of the hydropower licensing division of the NVE expressed this presumption in  \cite{flatby08} (noting also that the same presumption is not applied for small-scale projects, e.g., when some owners wish to expropriate from neighbours who oppose development).}

The expropriation presumption has remained in place even though the regulatory and economic context of riparian expropriation has changed dramatically after the liberalisation of the electricity sector.
This is significant, especially due to how licensing cases are processed. As discussed in the previous chapter, the administrative licensing assessment tends to focus on the environmental consequences of development, with little attention devoted to how the loss of property rights affects the owners and their local communities. This is so even though a license to develop is in effect also a license to expropriate.

How did this system come about, and where does it leave local owners whose waterfalls are targeted by large-scale proposals? This chapter addresses these two questions in depth. 

First, the history of the law is presented. This will serve to demonstrate that the current state of affairs developed gradually from a pre-industrial starting point where expropriation of waterfalls was generally not permitted Further to this, the chapter discusses more recent changes, specifically the changes in the expropriation regime that were implemented following liberalisation of the electricity sector. 

To make expropriation available as a tool for commercial companies, the earlier rules had to be modified. Specifically, public interest requirements had to be relaxed and limitations on private-to-private transfers had to be abrogated. The manner in which this was achieved, with only minimal parliamentary involvement, is in itself worth noting when addressing the legitimacy of current practices.

After the historical assessment, the chapter illustrates how the water authorities and the courts interpret and apply the rules currently in place. Specifically, I give a detailed presentation of the recent Supreme Court case of {\it Jørpeland}.\footnote{See \cite{jorpeland11} (I mention that I acted as legal counsel for the owners in this case).} This case demonstrates that the standing of owners is very weak under administrative law, a result of how the expropriation issue is overshadowed by the licensing question.

%More generally, {\it Jørpeland} and other recent cases suggest that the Supreme Court adhere to a very narrow perspective on the meaning of property protection, taking it to be an issue that begins and ends with the question of compensation. In this regard, owners were initially able to make some progress towards a more equitably level of compensation, but as this chapter shows, the early progress made on this point is likely to be reversed following the Supreme Court decision in the case of {\it Otra II}.\footnote{See \cite{otra13}.}

I conclude this chapter with a more overarching assessment based on the theoretical framework presented in Part I of the thesis, to shed further light on the legitimacy of rules and practices surrounding takings of waterfalls in Norway. I argue that the current system is likely to systematically result in takings that fail the Gray test presented in Chapter \ref{chap:2}. This sets the stage for the final chapter, where I consider land consolidation as a legitimacy-enhancing alternative to expropriation for hydropower development.

\section{Norwegian Expropriation Law: A Brief Overview}\label{sec:5:2}

As mentioned in Chapter \ref{chap:2}, the right to property is entrenched in section 105 of the Norwegian Constitution. There it is made clear that when property is taken for public use, full compensation is to be paid to the owner. The formulation bears a striking resemblance to the formulation of the US takings clause in the fifth amendment. However, there is no active public use debate in Norway. The meaning of public use is hardly ever discussed by the courts, and according to legal scholars, the public use formulation places no limit at all on the state's authority to expropriate.\footnote{See \cite[249]{aall04}. For a comment to more or less the same effect, made by the court of appeal, see \cite{sauda09}.}

However, it is a rule of unwritten constitutional law that administrative decisions which affect the rights of individuals can only be carried out when they are positively authorised by law.\footnote{See generally \cite{hogberg11}.} Moreover, the Constitution is not understood as providing an authority for the state to expropriate. It merely expresses the presupposition that expropriation is possible.\footnote{See, e.g., \cite[6]{fleischer86}.} Hence, when applying eminent domain, the government needs to justify this on the basis of a more specific authorising provision. 

Historically, there was no general act relating to expropriation and a range of different acts authorised the executive to expropriate for specific purposes such as roads, public buildings, and schools.\footnote{See \cite[11-12]{nut54}.} Today, many of these authorities have been broadened and included in the \cite{ea59}.\footnote{Act no 3 of 23 October 1959 Relating to Expropriation of Real Property.} After an amendment in 2001, this act includes an authority for the government to authorise expropriation of property and use rights in order to facilitate hydropower production.\footcite[2 no 51]{ea59} This is understood to include the authority to expropriate waterfalls.\footnote{See, e.g., 
\cite{sauda08}.}

According to the \cite{ea59}, expropriation can only be authorised if the benefits undoubtedly outweigh the harms, as determined following a discretionary assessment.\footnote{See \cite[2]{ea59}.} Formally, the authorising authority is the King in Council. However, this authority can be delegated to ministries or other state bodies that the King in Council can instruct.\footnote{See \cite[5]{ea59}.} The compensation to the owner is determined following a judicial procedure administered by the so-called appraisal courts.\footnote{\cite[2]{ea59}.} This is the name given to the regular civil courts when they hear appraisal cases, observing the special procedure set out in the \cite{aa17}. The appraisal procedure emphasises the importance of factual assessment and lay discretion (the appraisal court typically sits with four lay judges).\footnote{See \cite[11-12]{aa17}.} In addition, there are special rules regarding costs, indicating that the expropriating party is usually required to pay for the procedure, including the owners' legal expenses.\footnote{See \cite[54]{aa17}.} In other regards, the appraisal procedure resembles a typical adversarial process before a civil court.\footnote{See generally \cite{dyrkolbotn15}.} 

The \cite{ea59} states that unless the Kind in Council decides otherwise, expropriation orders may only be granted to state or municipality bodies. This is formulated as a limiting principle, but in effect it serves as a general authorisation for the executive to decide, without parliamentary involvement, that a larger class of legal persons may be granted expropriation licenses. 

For many purposes, directives have been issued that extend the class of possible beneficiaries to any legal person, including companies operating for profit. In 2001, such a directive was issued for the authority to expropriate in favour of hydropower production.\footnote{See Directive no 391 of 06 April 2001.} 

In addition to providing a general authority for expropriation, the \cite{ea59} also contains several procedural rules. These are collected in Chapter 3 of the Act. Here the Act sets out minimal requirements for what an application for an expropriation license must include: it should make clear who will be affected, how the property is to be used, and what the purpose of acquisition is.\footnote{See \cite[11]{ea59}.} In addition, the Act requires the applicant to specify exactly what property they require, and to include information about the type of property in question and the current use that is made of it.

The owners must be notified, and the starting point is that every owner should be given individual notice, although this obligation is relaxed when it is ``unreasonable difficult'' to fulfil\footnote{See \cite[12]{ea59}, para 2.} In such cases, it is sufficient that the documents of the case are made available at a suitable place in the local area. In addition, a public announcement must then  be made in the official notification publication of the government, as well as in two widely read local newspapers.\footnote{See \cite[12]{ea59}.}

The licensing authority is required to ensure that the facts of the case are clarified to the ``greatest extent possible''.\footnote{The Norwegian expression is ``best råd er'', which literally means ``best possible way''. See \cite[12]{ea59}, para 2.} This formulation seems very strict, but is also highly non-specific. In practice, the level of scrutiny given to the expropriation question under Norwegian law varies greatly depending on sector-specific administrative practices.\footnote{See \cite[380-381]{dyrkolbotn15}.} Moreover, established practice from several fields, including the hydropower sector, suggests that when expropriation takes place to implement a public plan or a licensed development, little attention is devoted to expropriation as a special issue.\footnote{For zoning plans, see \cite{namsos98,bo99}. For hydropower, see \cite{jorpeland11}.}

The applicant must cover costs incurred by owners in relation to a pending application for expropriation.\footnote{See \cite[15]{ea59}.} The exact formulation is that the applicant is obliged to cover the costs that ``the rules in this chapter carry with them''. That is, the applicant is obliged to cover the costs that are related to the owners' rights pursuant to Chapter 3 of the \cite{ea59}. In practice, an owner will be denied costs if the competent authority takes the view that they are unreasonable or disproportionate to their interests in the case.\footnote{If the case progresses to an appraisal dispute, the competent authority to decide on costs is the appraisal court. Otherwise, the decision is left with the executive. See \cite[15]{ea59}.} Finally, the decision to grant an expropriation license must be justified, and the parties should be informed of the reasons for the decision.\footnote{See \cite[12]{ea59}, para 3.}

In addition to the procedural rules in the \cite{ea59}, the rules of the \cite{paa67} also apply in expropriation cases. These rules largely stipulate the same requirements as those discussed above, so I omit a detailed presentation. Instead, I go on to present sector-specific rules pertaining to takings for hydropower. %However, I mention that it has been controversial whether or not these rules provide any basis at all for scrutiny of established practices adopted by the water authorities. Specifically, it has been argued that the licensing procedures spelled out in the \cite{wra00} and the \cite{wra17} are exhaustive in hydropower cases.\footnote{See \cite{jorpeland11a}.} In the case of {\it Jørpeland}, the Supreme Court held that general rules of administrative law did apply in theory, but went quite far in suggesting that they would have limited significance in practice, as sector-specific rules and practices would take priority.\footnote{See \cite{jorpeland11}.}

\section{Taking Waterfalls by Obtaining a Development License}\label{sec:5:3}

As mentioned in the introduction, section 16 of the \cite{wra17} establishes an automatic right to expropriate rights needed to implement a licensed watercourse regulation. This does not include a right to expropriate rivers and waterfalls needed for the hydropower development as such. However, it includes a right to transfer water away from a river for development somewhere else. This has the {\it de facto} effect of a waterfall expropriation, since the water is transferred away from the source river.

This kind of expropriation has always been treated as waterfall expropriation in relation to the compensation issue.\footnote{See \cite{jorpeland11}.} Formally, however, the interference is not considered an expropriation of real property, but rather an expropriation of a right to remove the water, a sort of easement whereby the developer acquires the right to interfere with the rights of riparian owners in source rivers.

In theory, the rules in the \cite{ea59} and the \cite{paa67} still apply when the right to expropriate follows automatically from a development license. Indeed, the rules in the \cite{paa67} express general principles of administrative law, pertaining to all kinds of individual decisions, including both expropriation and licensing decisions. The \cite{ea59}, for its part, explicitly states that it applies to property interferences authorised under the \cite{wra17}.\footnote{See \cite[30]{ea59}.} However, it is also stated that the rules in the \cite{ea59} only apply in so far as they are ``suitable'' and do not ``contradict'' sector-specific rules.\footcite[30]{ea59} This points to the potential caveat that while a range of procedural rules apply in theory, there is a risk that they will be ignored in practice, if they are deemed unsuitable by the licensing authorities.

This is practically significant in hydropower cases. Specifically, the water authorities regard the \cite{wra17} as providing an exhaustive legislative basis for the licensing procedure.\footnote{This was made clear through the case of \cite{jorpeland11}, where this practice also got a stamp of approval from the Supreme Court.} This also means that the material assessment requirement in the \cite{ea59} is not considered to have any independent significance alongside the assessment criterion in the \cite{wra17}.\footnote{See \cite[30]{jorpeland11}.}

This is so even though case law on the former assessment criterion emphasises the interests of affected property owners in a way that case law and administrative practice on the licensing issue does not.\footnote{In addition, the formulation in \cite[2]{ea59} contains the additional qualification that the benefit of interference must ``undoubtedly'' outweigh the harm, meaning that this clearly must be the case (pertaining to the evidence, not the weight of the benefit compared to the harm), see \cite{lovenskiold09}. No corresponding requirement is included in the \cite[8]{wra17}. Instead, the formulation there is that a license should ``normally'' not be given unless the benefits outweigh the harms. See also \cite[325-236]{haagensen02} (arguing that the ``normally'' qualification is without practical significance).} As a consequence of how the law is understood on this point, it is very hard for owners to challenge the legality of a decision to allow expropriation of their riparian rights, especially when expropriation takes place pursuant to the \cite{wra17}.\footnote{It follows from the discussion in Chapter \ref{chap:4} that large-scale development projects almost always involve a license pursuant to the \cite{wra17} (or such that the rules from this act, including section 16 on expropriation apply pursuant to the \cite{wra00}).} Moreover, even if section 16 of the \cite{wra17} does not apply, the water authorities rely on the presumption that an expropriation license should be granted whenever a large-scale development license is granted.\footnote{See \cite{flatby08}.}

Hence, in order to defend themselves, owners must proceed in a roundabout manner by addressing the licensing question, for instance by arguing that large-scale development will be environmentally unsound or by presenting a detailed development plan of their own in the hope of convincing the water authorities of its merits. This is a daunting task, particularly in light of the continued influence of case law and administrative practices developed during the period of monopoly regulation of the electricity sector.\footnote{See Sections \ref{sec:5:4:2} and \ref{sec:5:4:3} below.}

To shed further light on how the current situation came about, I will now present the history of the law in this area. As will become clear, the current state of affairs was not inevitable, but rather the result of a series of reforms that gradually undermined property as an anchor for active community participation in hydropower development.

\section{Taking Waterfalls for Progress}\label{sec:4}

%Historically, Norwegian law did not permit expropriation of waterfalls for hydropower development.\footnote{See \cite[29]{amundsen28}.} 
In the now repealed \cite{wra88}, several provisions authorised appropriation of water-rights and land for various water-related purposes, but the criteria were very narrow.\footnote{See \cite[69-85]{dahl88}. In addition, the purpose of expropriation was largely understood to be binding also on future use, so that the taker would not gain unrestricted control over the rights they acquired. Rather, they were obliged to use these rights to pursue the specific public purpose for which expropriation was authorised. See, e.g., \cite[133-140]{rygh12}.} Waterfall rights as such could never be expropriated, and expropriation of other rights pertaining to the use of water could only be permitted in so far as the affected owners were not thereby deprived of any water-power that they could reasonably make use of themselves.\footnote{See \cite[58|60]{dahl88}.}

Specifically, expropriation for hydropower development was only permitted when it benefited waterfall owners who needed to acquire surrounding land in order to make better use of their own property rights.\footnote{See the \cite[15-16]{wra88}. See also the commentary in \cite[60-65]{dahl88}.} Moreover, riparian owners could apply for licenses to engage in various industrial exploits, in some cases also when this would prove damaging to other landowners, for instance through deprivation of water or flooding.\footnote{See \cite[14]{wra88}. See also the commentary in \cite[54-60]{dahl88}.} These rules are similar to many of the rules found in contemporaneous mill acts from the US, discussed in Chapter \ref{chap:2}. As in the US, these rules could be classified as giving rise to economic development takings. However, the source of the economic development potential as such was not supposed to be taken from the owners under these rules.\footnote{See \cite[168-170]{dahl88}.} Rather, takings were only warranted with respect to additional rights that existing owners needed to realise the full potential of their own resources.

In fact, an important principle of Norwegian expropriation law at this time was that no property could be taken if the taker's interest in that property was part of the current owner's bundle of interests associated with the property.\footnote{See \cite[168-170]{dahl88}.} This applied regardless of whether or not the owners, subjectively speaking, were likely to pursue the interest in question in an optimal way. On the basis of this principle, expropriation of waterfalls for hydropower development was not permissible. The reason was simple: the right to develop hydropower was considered part of the owners' bundle of interests. Hence, it could not be taken from them, as a matter of principle.

By contrast, if ancillary land was needed by someone wishing to make optimal use of {\it their} waterfall rights, expropriation was possible. In these cases, the takers did not seek to take the owners' rights as much as to negate them, in order to fully enjoy their own. More generally, expropriation at this time was considered a way to resolve conflicts between rights, not a way to redistribute them.\footnote{See \cite[168-170]{dahl88}.}

Following industrial advances, the interest in hydropower exploded in the late 19th century.\footnote{See \cite[58-59]{falkanger87}.} As a result, the state increasingly came to see it as a political priority to regulate the hydropower sector, especially to prevent foreign speculators and industrialists from acquiring ownership of Norwegian resources.\footnote{See \cite[58-59]{falkanger87}.} As discussed in Chapter \ref{chap:3}, the most important expressions of this came in the form of two new licensing acts, namely the \cite{wra17} (Section \ref{sec:wra17} and the \cite{ica17} (Section \ref{sec:ica17}).

Following up on this, parliament soon passed legislation that authorised expropriation of riparian rights for the benefit of public bodies, also when the purpose was hydropower development.\footnote{Legislation that made it possible to expropriate waterfalls to the benefit of the municipalities was introduced in 1911, and a similar authority that authorised expropriation in favour of the state appeared in 1917, see \cite[29]{amundsen28}.} In 1940, these authorities were consolidated and integrated in the general water resources legislation, through the \cite{wra40}.\footnote{This act has since largely been replaced by the \cite{wra00}.} According to this act, the authority to expropriate waterfalls could be granted only to the state and the municipalities. Moreover, the municipalities could only expropriate waterfalls when the purpose was to provide electricity to the local district.\footnote{See the \cite[148]{wra40}. See also the commentary in \cite[201-210]{sorensen41}.} Private parties could not expropriate except in exceptional circumstances, when they already owned more than 50 \% of the riparian rights they sought to exploit.\footnote{See the \cite[55]{wra40}. See also the commentary in \cite[70-74]{sorensen41}. I remark that this was a novel rule in the 1940 Act, which contradicted earlier theories about the legitimacy of allowing expropriation for private benefit.} 

In all cases of waterfall expropriation, it was felt that benefit sharing with local owners was required. Hence, special rules were introduced to ensure that takers would have to pay {\it more} than full compensation (typically a 25 \% premium, but in some cases the owner was also given a right to opt for compensation in the form of a proportion of the electricity output of the plant).\footnote{See \cite[70-91,184,210]{sorensen41}. For more on compensation, see below in Section \ref{sec:5:4:1}.}

As I showed in Chapter \ref{chap:4}, the electricity supply in Norway just after the passage of the \cite{wra40} was already well developed, with 80 \% of the population having access to electricity. Moreover, in the rural areas the supply often came from one among a vast number of small power plants. In light of the progress already made and the highly decentralised structure of the hydroelectric sector at this time, one might have expected expropriation to remain a relatively rare occurrence.

However, the prevalence of expropriation to facilitate hydropower development increased greatly after the war, as the state itself became engaged much more actively with hydropower development, also for commercially oriented industrial purposes.\footnote{See \cite[59-71]{thue96}. See also \cite{skjold06}.} Hence, despite the spirit and wording of the \cite{wra40}, this was the time when expropriation of rivers and waterfalls became a measure to facilitate large-scale transfer of resources. 

This development had little do with supplying electricity to the people. Rather, it arose from increased political demand for hydropower to support the metallurgical and electrochemical industries, combined with the fact that the hydropower sector was reorganised and brought under increasingly centralised political control.\footnote{See \cite[69-71]{thue96}.}

Following this, a growing share of the financial benefits from development would accrue to urban areas, as local development companies were replaced by state companies and companies dominated by prosperous city municipalities.\footnote{In 2007, as the result of a gradual centralisation process, the 15 largest hydropower companies in Norway, which are largely controlled by the state and some city municipalities, owned roughly 80\% of Norwegian hydropower, measured in terms of annual output. In 2006, the public owners of hydropower in Norway benefited from receiving more than NOK 9 billion in dividends. See \cite[28]{otprp61}.} In addition, a highly idiosyncratic compensation method was adopted in expropriation cases, resulting in a situation where waterfalls could be acquired very cheaply from local owners.

\subsection{The Natural Horsepower Method}\label{sec:5:4:1}

In Section \ref{sec:wra17}, I presented the notion of a natural horsepower, used to determine when a development project requires development licenses pursuant to the \cite{wra17} and the \cite{ica17}. As mentioned, the natural horsepower of a development scheme is a gross measure of the stable electric effect harnessed following the development. Specifically, it measures the electric output that can be maintained for at least 350 days each year, a figure that is sensitive to fluctuating water levels. In practice, a power plant can often generate electricity at a much higher level, by using more water whenever it is available.\footnote{See \cite{sofienlund07}.}

For this reason, the number of natural horsepower in a development project is an unreliable measure of the total amount of energy harnessed in a year. As a result, it is also an unreliable starting point for assessing the value of a development project. Today, energy producers get paid for all the energy they produce, not just that which they can guarantee in advance.\footnote{See \cite[83-84]{uleberg08}.} Prior to the establishment of a national grid, this was different. Without a grid, fluctuations in electric output would not be evened out by supply from other parts of the country. Hence, the importance of maintaining a stable supply was much greater. Indeed, energy producers would often get paid based on the amount of electric effect they could deliver stably over the year, not the total amount of energy harnessed.\footnote{See \cite[83]{uleberg08}.}

Hence, early in the 20th century, the notion of natural horsepower could be used to provide a sensible measure of how much a developer would be willing to pay for access to riparian rights.\footnote{See \cite[83]{uleberg08}.} Indeed, the notion was used in this way on the market for waterfalls that existed prior to state regulation. The price of a waterfall, specifically, was typically calculated on the basis of the price that the developer was willing to pay per natural horsepower that the planned development would yield.\footnote{See \cite[83]{uleberg08}.} The total payment offered to the owners, consequently, would be found by multiplying the natural horsepower of the development with the price offered per natural horsepower.

This method was duly adopted by appraisal courts to fix the level of compensation following expropriation.\footnote{\cite[521]{vislie02}.} Moreover, when the notion of natural horsepower fell into disuse among energy producers, because it no longer reflected the actual value of development projects, the courts did not modify their compensation practices. They stuck with the natural horsepower method, which was now applied on a customary basis, not as a way of calculating realistic economic values.\footnote{See, e.g., \cite[1599]{hellandsfoss99} (the Supreme Court comments that the method is used customarily because the market provides ``little guidance'').}

Over time, the price level became more and more unrealistic as a measure of the value of waterfalls as a natural resource. After the liberalisation of the hydropower sector in the early 1990s, the discrepancy became extreme. To give an example: in 1999, the appraisal court of appeal awarded a one time payment of NOK 722 068 in compensation for a waterfall that yields 152 GWh per annum.\footnote{See \cite{hellandsfoss99}.} By comparison, in the case of {\it Sauda} from 2009, where a market-based valuation method was used, the owners of the {\it Maldal} river were awarded NOK 1 149 044 in compensation as a {\it yearly payment} for a waterfall that would yield 36.5 GWh per annum.\footnote{See \cite{sauda09}.} If we assume an interest rate of 4 \%, this corresponds to a one time payment of NOK 28 726 100. This, in turn, corresponds to NOK 787 017 per 1 GWh produced annually. That is, the owners of {\it Maldal} were paid in the excess of 150 times more for their waterfall than the owners of {\it Hellandsfoss}.

The mismatch between economic values and compensation payments had been noted long before the liberalisation of the electricity sector. In fact, the existence of a major discrepancy had been noted as early as in the 1950s, by the head of the water directorate himself. In an article published in an internal newsletter in 1956, the director commented that the natural horsepower method did not result in compensation payments that reflected the true economic value of waterfalls as a natural resource.\footnote{See \cite{rogstad56}.} Moreover, he speculated that the method could be sustained only through exploiting the lack of knowledge about hydropower development among rural populations.\footnote{See \cite{rogstad56}.}

One might think that the continued use of the natural horsepower method, in a situation when the water authorities themselves were aware of its shortcomings, would result in controversy. However, at this time, the local owners of waterfalls did not attack the method in court. Active resistance on this point would not be seen until much later, after the liberalisation of the electricity sector, as discussed in Sections \ref{sec:5:5:2} and \ref{sec:5:5:3} below. However, conflicts arose with respect to other aspects of the regulatory framework regarding hydropower, as discussed in the following section.

\subsection{Increased Scale of Development and Increased Tension}\label{sec:5:4:2}

As discussed in the previous chapter, the state pursued increasingly complex hydropower projects after the Second World War. At this time, technological and economic advances also made it more feasible to divert water over great distances, to collect several different rivers in a common reservoir for joint exploitation. Such projects became known as ``gutter'' projects, and they grew greatly in scope during the post-War years. Since the relevant licensing procedure was covered by the \cite{wra17}, the practical importance of the expropriation authority in section 16 of this act also increased dramatically.\footnote{See \cite[11]{innst59}. This was a proposition to parliament regarding an amendment of the \cite{wra17}. The amendment proposed to remove an earlier rule that applied only to diversion regulations, whereby a license to divert water from a river should {\it normally} only be granted when the riparian owners in the source river agreed to the measure. This rule made licenses harder to obtain in the diversion cases. However, following the department's recommendation, the rule was removed in 1959. The department argued that the rule had an ``unfortunate effect'' on the administrative procedure in large-scale diversion cases, noting also the vastly increasing complexity and scale of typical diversion regulations. The minority in the parliamentary committee recommended against the amendment, noting that it would ``greatly increase'' the authority to expropriate waterfalls, contrasting with the expropriation rules in the \cite{wra40}, see \cite[14]{innst59}. The majority countered this argument by maintaining that the regulatory power of the state would be used to prevent any abuse of power, and that the practical significance of the amendment would be limited to ensuring a ``more rational'' procedural approach to large-scale applications, see \cite[14]{innst59}.}

As mentioned in the previous chapter, the opposition to hydropower grew proportionally to the scale and complexity of typical development projects.\footnote{See generally \cite[64-65]{nilsen08}.} The critical focus was often on environmental effects, but the interests of local people also featured in these debates. Moreover, local interest were often aligned with the environmental interests.\footnote{See \cite[72-73]{nilsen08}.}

%In the first cases that reached the Supreme Court from this era, the question of legitimacy was not raised in full breadth. Instead, the early cases concerned specific legal points, such as the issue of whether (informal) agreements and understandings between owners, municipalities and the central government were binding on future decision-making processes regarding development.\footnote{See \cite{aura61,mardola73}.} In addition, the question arose as to what extent additional compensation should be paid for `damages' and `inconveniences' caused by large-scale development, in addition to the compensation calculated using the natural horsepower method. Finally, questions arose over the status of non-waterfall owners who owned land that was still crucial to the development, for instance because it would be flooded or used to construct regulation installations. Should such owners receive compensation based on the value of their rights for development purposes, or should they only receive compensation based on the value of their current property uses, as had been the norm before?

Some controversies led to legal conflicts that came before the Supreme Court. Here the claims of owners and local communities were consistently rejected. First, the Court held that landowners who did not own waterfalls were not entitled to compensation based on the value of their rights as an asset for hydropower development.\footnote{See \cite[332-333]{tokke63}.} Instead, they would only receive damages based on the value of their current use of the property. Second, it was held that when compensation was awarded to waterfall owners according to the natural horsepower method, then this would preclude additional compensation for harms and nuisances associated with large-scale watercourse regulation.\footnote{See \cite{vikfalli71,driva82}.}

In the case of {\it Aura}, the owners argued that they had originally agreed to sell their water rights to a private developer, on the understanding that a specific development project would take place, not involving diversion of water.\footnote{See \cite[1284]{aura61} (the original transaction took place in 1906-1910, when there was still a market for sale of waterfalls to private speculators and developers).} Hence, the owners thought that the purchaser of their water rights had not acquired a right to divert the water away from the river. Still, when the government later acquired the water rights in question, they decided to embark on a more intrusive project that {\it would} involve water diversion. For this reason, the owners argued that they were entitled to additional compensation. 

The claim was rejected by the Supreme Court, which held that insufficient evidence had been provided to establish that the sale of the water rights was made conditional on a specific type of development.\footnote{See \cite[1285-1286]{aura61}.} Moreover, it was held -- on the basis of the facts -- that the sale of the water rights had {\it not} been restricted to only cover the waterfall (i.e., the right to harness hydropower from the river in question). According to the Supreme Court, the fact that the rights in question had been referred to as ``water rights'' meant that the right to divert away the water was also included.\footnote{See \cite[1284-1285]{aura61}.}

In the later case of {\it Mardøla}, the situation was similar, with the crucial difference being that these local owners had not sold ``water rights''; their contract explicitly stated that what had been sold was the waterfalls.\footnote{See \cite[112]{mardola73} (the voluntary sale dated back to the early 20th century, when the market for waterfall was still unregulated).} However, the government interpreted this to mean that they had a right to divert water away from the river, without paying any additional compensation.\footnote{See \cite[112]{mardola73}.} This contradicted the premise of {\it Aura}, where the decision to allow a diversion was premised on the fact that {\it not only} the waterfalls had been acquired by the developer. Still, in {\it Mardøla}, the Supreme Court cites {\it Aura} as the primary authority in favour of a {\it general rule} by which the sale of a ``waterfall'' also includes the right to divert water away from the river.\footnote{See \cite[112]{mardola73}.} No explanation is provided by the Court to reconcile this with what was actually said in {\it Aura}.\footnote{Arguably, the Court's finding on this point has since been overruled by \cite{jorpeland11}. Here a waterfall right was defined explicitly as a right to exploit the hydropower in a river along its present trajectory. This definition was provided in order to avoid the conclusion that a diversion of water by someone other than the waterfall owner (in the source river) amounts to a waterfall expropriation. It bears noting that if the Court had concluded in keeping with the precedent set by {\it Mardøla}, by holding that the diversion right is part of the waterfall bundle, it would have shed serious doubt on the legitimacy of the established practice of allowing diversions under section 16 of the \cite{wra17}, with no prior acquisition of the waterfall rights in source rivers.}

%In {\it Mardøla}, both the owners and the local municipality had explicitly agreed to support the central government on the understanding that a specific development plan would be adopted. Later, this plan was abandoned in favour of a project that was deemed by some local owners to be both less beneficial and more intrusive. Hence, both the owners and the municipality argued that the resulting development license was invalid. The Supreme Court conceded that prior statements made by the water authorities had been striking, serving to create a clear expectation among the locals for a specific development plan.\footnote{See \cite[111]{mardola73}.}

%However, the Court chose to rely on what it described as a ``general presumption'' against the position that the central government is bound in its decision-making by prior statements.\footnote{See \cite[110]{mardola73}.} According to the Supreme Court, the statements made by the water directorate in {\it Mardøla} were not clearly endorsed by the Ministry and the Parliament, and could therefore not be regarded as binding on the final decision.\footnote{See \cite[111]{mardola73}.} %However, the Court did not address the question at all from a procedural angle, by inquiring into the legitimacy of creating what appeared to be a legitimate expectation on part of the owners and the municipality.

%This finding seems reasonable enough. However, one rather crucial question was not addressed at all: is it legitimate procedure for the water authorities to make ``striking statements'' of the kind offered in {\it Mardøla}, when this serves to silence opposition and induce support for development during the assessment stages? The Supreme Court apparently did not wish to consider this question, raising the possibility that it felt the answer would not have been to its liking.

The case of {\it Mardøla} illustrates the increasing tension that arose regarding hydropower in the 1970s, and arguably also a tendency on part of the Supreme Court to side with large-scale development interests. Indeed, the development in {\it Mardøla} stirred up a high level of controversy that also resulted in civil disobedience and criminal prosecution of environmental activists.\footnote{See \cite{mar71}.} %The case also illustrates how the central government attempted to minimise tensions by entering into dialogue with local authorities and owners. Crucially, this dialogue was not premised on a legal framework that ensured local participation, and, despite appearances, did not necessarily result in any new entitlements for local people.\footnote{However, the increased tension during this time did sometimes lead to additional benefits being bestowed on local power groups. These new measures were typically relatively minor, and they were directed primarily at municipalities and regional government bodies rather than local owners. See generally \cite[75-76]{nilsen08}.}

\noo{ In effect, the {\it Mardøla} Court sanctioned an approach whereby the water authorities could limit local opposition by expressing commitments early on, which could then simply be ignored at a later stage of the decision-making process. At this later stage, it would be too late for the local population to launch an effective opposition, e.g., it would be too late for them to aligning themselves with environmental activists. More generally,}

%Moreover, despite occasional concessions being made to local and regional government institutions, controversies continued to arise. 
The culmination of the increasing tension surrounding hydropower at this time came with the case of {\it Alta}, where the question of procedural legitimacy was raised in full breadth. To this day, the {\it Alta} case remains the most important Supreme Court precedent in the area of hydropower law.

\subsection{The {\it Alta} Controversy}\label{sec:5:4:3}

The {\it Alta} case went before the Supreme Court in 1982 after a long period of high-intensity conflict going back to the mid-seventies.\footnote{See \cite{alta82}. For commentaries, see \cite{eckhoff82,boe83,hagvar88}.} In {\it Alta}, the affected local population largely lacked formal title to the property they sought to defend. This was because the development in question would take place in the northernmost part of Norway, in the native land of the Sami people.\footnote{For Sami law generally, see \cite{skogvang02}.}

Norway has a history of discrimination against the Sami, and as their culture is largely nomadic, their land rights were never formalised in private law.\footnote{See \cite[149-156]{ravna12s}} As a result, land and natural resources in the county of Finnmark are largely owned by the state, at least in the sense of the state appearing as the nominal {\it in rem} owner.\footnote{In the past 30 years, partly as a response to the controversy of the {\it Alta} case, there has been a gradual change in attitude, whereby the rights of the Sami people receives greater legal recognition. In 2007, formal title to most of the land in the county of Finnark was transferred to a special state agency which is regulated by a special statute that obliges it to manage the land with due regard to customary and prescriptive rights of aboriginal groups and local people. See generally \cite{bull07}.}

Due to the sensitive context of interference, the {\it Alta} plans met with particularly strong criticism from local people, as well as environmental groups and groups fighting for aboriginal rights. A broad political movement was mobilised in opposition to the plans, eventually resulting in several serious cases of civil disobedience.\footnote{This included hunger strikes and attempts at sabotage, see \cite[80-83]{nilsen08}. For the Alta controversy generally, see \cite{altawiki,hjorthol06}.} The case also came before the courts, as the local population and environmental groups claimed, primarily on the basis of administrative law, that the development licenses that had been granted were invalid.\footnote{See \cite{eckhoff82}.}

The {\it Alta} case did not involve expropriation of waterfalls. However, because of the priority given to the licensing procedure over specific expropriation procedures, the principles expressed in {\it Alta} also largely determine the legal position of waterfall owners whose rights to hydropower are expropriated.\footnote{See \cite{sauda09,jorpeland11}.} Moreover, the case involved expropriation of other property rights as well as special usufructuary rights held by the Sami people.

{\it Alta} was admitted to the Supreme Court in plenum, directly on appeal from the district court.\footnote{This is a special arrangement available in cases that raise important questions of principle, see \cite[30-2]{cda05} and \cite[5]{ca15}.} The presiding judge commented that as far as he knew, it was the longest and most extensive civil case that the Court had ever heard.\footcite[254]{alta82} In an opinion totalling 138 pages, the Court considers a long range of objections against the development licenses, all of which are either rejected or held to provide insufficient reasons to declare the licenses invalid.

\noo{The opponents of the {\it Alta} development also argued on the basis of human rights and international law.\footnote{First, on the basis of articles 1 and 27 of the \cite{fnp}. Second, on the basis of \cite{ilo107} (later replaced by \cite{ilo169}). Third, on the basis of P1(1) of the \cite{echr}.} As noted by Eckhoff, these arguments raised subtle legal questions about how to apply the relevant principles of international law to a concrete dispute over hydropower development.\footnote{See \cite[351-352]{eckhoff82}. One of the most important international instruments, namely ILO Convention No 107, was not ratified by Norway at the time of {\it Alta} (Norway later ratified its replacement, ILO Convention No 169). However, it was argued that it had the status of customary international law. See generally \cite{eide80}.} However, the Court refused to consider such  questions, finding that the negative effect of the hydroelectric plant was not so severe as to raise  human rights issues.\footnote{See \cite[299-300]{alta82}. See also \cite[351-352]{eckhoff82}.}}

First, the Court summarily rejects arguments based on indigenous and human rights law on the basis that the interference in question would not be sufficiently severe to raise any issues in this regard. After concluding in this way, the Supreme Court goes on to approach the case on the basis of administrative law instead.\footnote{See \cite[351-352]{eckhoff82}. It also bears noting that the most important international instrument protecting indigenous rights, ILO Convention No 107, was not ratified by Norway at the time of {\it Alta} (Norway later ratified its replacement, ILO Convention No 169). Still, it was argued that it had the status of customary international law, an argument not considered in any depth by the Supreme Court. See generally \cite{eide80}.} The focus was solely on the procedural rules of the \cite{wra17}. In this regard, the opponents of the {\it Alta} development had pointed to a large number of purported shortcomings of the decision-making process. 

First, it had been argued that the original licensing application did not meet the requirements stipulated in section 5 of the \cite{wra17}. Essentially, the original application contained little more than technical details about the planned development, with hardly any identification or assessment of deleterious effects.\footnote{See \cite[264-265]{alta82}.} This shortcoming had been openly acknowledge by the water authorities themselves, who had nevertheless initiated a public hearing.\footnote{See \cite[265]{alta82}.}

The Supreme Court concluded that this was ``clearly unfortunate''.\footcite[265]{alta82} However, several reports and assessments had subsequently been provided, to fill the gaps left open by the initial application. For this reason, the Supreme Court held that the initial mistakes were irrelevant, since it was the licensing process as a whole that should be assessed.\footnote{See \cite[265-266]{alta82}.} Shortcomings at specific stages in the assessment would not be given weight unless they could be seen to imbue the process with a dubious character overall.\footcite[265]{alta82}

The Court then moved on to assess whether the process as a whole fulfilled the procedural requirements of sections 5 and 6 in the \cite{wra17}. In addition, the Court considered whether the assessment of the licensing criteria in section 8 of the \cite{wra17} had been sufficiently detailed.

In addition to assessing a large amount of information regarding the situation in {\it Alta} and how it had been assessed by the water authorities, the {\it Alta} Court also made some important statements of principle. In particular, the Court held that since a licensing decision itself is discretionary, it is appropriate to grant the executive some margin of appreciation also with regard to the question of how to interpret vague requirements of administrative law.\footnote{See \cite[262-264]{alta82}.}

The Court made a second decision of principle when it supported the state's contention that the administrative licensing assessment did not have to be as thorough as that required in a subsequent appraisal dispute.\footnote{See \cite[279|330]{alta82}.} This also serves to downplay the risk of factual error; if mistakes are made with regard to the owners' losses at the assessment stage, these mistakes can be corrected later by a correct compensation award.

In fact, the {\it Alta} Court agreed that the license had been based on erroneous information about some issues, particularly regarding alternative ways to meet the need for electricity in Finnmark.\footnote{See \cite[346-357]{alta82}.} However, the Supreme Court did not regard the factual errors in this regard as relevant to the licensing decision.\footnote{See \cite[346]{alta82}.} 

Here a third clarification of principle took place. The Court held, in particular, that the duty to consider alternatives -- different ways in which the public purpose could be satisfied -- is very limited in hydropower cases.\footnote{See \cite[346]{alta82}.} On this basis, the Court argues that factual errors and inadequate information regarding alternatives is less relevant.\footcite[346]{alta82} Apparently, since this information is not required in the first place, if the authorities get it wrong, it does not as easily count as a breach of procedure.

The Court's perspective on alternatives appears to have been at odds with how parliament had actually approached the licensing question, on three separate occasions.\footnote{See \cite[342]{alta82}.} Indeed, there was little doubt that the favourable political assessment of the {\it Alta} development depended strongly on the perceived electricity crisis in Finnmark and the supply situation in Norway generally, as well as the perceived inadequacies of alternative solutions.\footnote{See \cite[338-347]{alta82}.}

Hence, it is quite remarkable how little attention the Court directs towards the factual errors and the inadequate information that had been provided concerning alternatives.\footnote{See also the surprise expressed in \cite[349-351]{eckhoff82}.} By contrast, the Court goes into painstaking detail regarding issues that seem to have been far less important to the political decision-makers.

The dismissive attitude towards the duty to correctly assess alternatives is a controversial aspect of the {\it Alta}-decision.\footnote{See \cite[311]{haagensen02}. For criticism of the Supreme Court on this point, see \cite[580-584]{backer86}.} On this point especially, the decision has met with criticism from commentators arguing that the decision shows the extent to which the courts in Norway tend to identify themselves with other organs of state.\footnote{See, e.g., \cite[64]{graver88} (commenting also that ``government prestige'' was at stake).} Some have taken a more positive approach by arguing that {\it Alta} would be unlikely to become a leading precedent, especially with regard to the duty to assess alternatives.\footnote{See \cite[580-584]{backer86}.} But this has been proven wrong. Indeed, {\it Alta} continues to receive favourable citations by the Supreme Court, both in relation to hydropower as well as with regard to administrative law more generally.\footnote{See \cite{ambassade09,jorpeland11}.}

It should be mentioned, however, that after the {\it Alta} decision, the legal position of the Sami people has improved quite significantly.\footnote{See generally \cite{gauslaa07}. Gauslaa presents the emergence of {\it Sami law}, a collection of rules and principles serving to protect established land use patterns and the Sami way of life while also giving the Sami people a better opportunity to partake in decision-making processes that affect them as group.} Moreover, the controversy surrounding {\it Alta} has been regarded as a catalyst for change in this regard.\footnote{See \cite[156]{ravna12s}.} Hence, it is unlikely that the courts today would be as quick as the {\it Alta} court to dismiss arguments based on aboriginal rights.\footnote{See \cite[180]{gauslaa07}.}

However, with regard to local owners more generally, the {\it Alta} decision is considered to express key principles that still apply.\footnote{See \cite{jorpeland11}. See also \cite[312]{haagensen02}.} 
At the same time, the context surrounding takings for hydropower development have changed significantly since {\it Alta}. First, as discussed in the previous chapter, takings of waterfalls now occur in a very different economic context. Moreover, as discussed in the next section, the legal context has also changed, giving rise to a situation where expropriations of waterfalls have become pure takings for profit.

\section{Taking Waterfalls for Profit}\label{sec:5:5}

Following the introduction of the \cite{wra00}, the legislative authority to expropriate waterfalls  was expanded and incorporated in the \cite{ea59}. This change in the law was not singled out for political consideration. In fact, the increased scope of expropriation was not mentioned at all when the Ministry presented their proposals to parliament. Rather, the new expropriation authority was described merely as a ``simplification'' of existing law.\footcite[223-225]{otprp39}

The original proposal stemmed from the report handed to the Ministry by a commission appointed to prepare a new act relating to water resources. The report totals almost 500 pages, but devotes only three of those pages to discussing the new expropriation authority.\footnote{See \cite[235-237]{nou94}.} Here the committee notes that a range of different authorities for expropriation has long co-existed in the law, with many of them positing strict and specific public interest requirements as a precondition for granting a license. This, the commission argues, is not a very ``pedagogical'' way of providing expropriation authorities.\footcite[235]{nou94} Moreover, the commission notes that it runs the risk of omitting important purposes for which expropriation should be possible. Hence, the commission proposes to replace all older authorities by a sweeping authority that will make expropriation possible for the purpose of facilitating ``measures in watercourses''.\footcite[235-236]{nou94}

The commission comments that their formulation might seem wide, but remarks that this is not a problem since the executive can simply refuse to issue an expropriation order when they regard expropriation as undesirable.\footcite[235]{nou94} The commission does not reflect on the consequences of such a perspective, neither in relation to property rights nor in relation to the balance of power between the legislature, the executive and the courts.

\noo{Instead, the commission offers a brief presentation of the rationale behind dropping the local supply restriction for municipal expropriation. They comment that these rules complicate the law and might make desirable expropriations impossible.\footcite[235]{nou94} Nothing is said to clarify what kind of desirable expropriations the committee think might be left out. 

The committee do not relate their proposals to the recent liberalisation of the energy sector. Hence, the obvious practical consequence of their proposal, namely that expropriation of waterfalls would be made available as a profit-making tool for commercial companies, is not discussed or critically assessed. The issue of {\it who} should be permitted to benefit from an expropriation license is also dealt with only superficially. In this regard, the commission structure their presentation around the redemption rule of the \cite{wra40}. As mentioned briefly in Section \ref{sec:twp}, this rule made it possible for the majority owners of a waterfall to compulsorily acquire minority rights, if this was necessary to facilitate hydropower development. Hence, it was a rule that provided only a limited opportunity for private takings, restricted to owners themselves or external developers that had been able to reach a deal with a locally based majority.

The main justification given by the commission for introducing a general private takings authority is that the special redemption rule had not been much used.\footcite[236]{nou94} Why this is an argument in favour of opening up for private expropriation in general is not made clear. It seems just as natural to regard it as an argument {\it against} doing so. Why extend the possibility for private expropriation if the demand for such expropriation has been limited?

Presumably, the commission thought there would be a demand for private expropriation in the future, but this is not stated explicitly, nor is the appropriateness of it discussed. As to the requirement that private takers must already control a majority of the waterfall rights in the local area, the commission only remarks that it regards such a restriction as old-fashioned.\footcite[236]{nou94} No discussion is offered regarding the consequences for local communities, if it is dropped.}

However, the later case of {\it Sauda} shed light on this question, as it emerged that the new authority would, for the first time in Norwegian history, make it possible for private commercial interests to openly expropriate waterfalls.\footnote{In cases involving diversion of water, a {\it de facto} right to expropriate could be granted to private actors already under section 16 of the \cite{wra17}. See the discussion in Section \ref{sec:5:4:2} above.}

%when it emerged that the members of parliament themselves had not been aware that the new legislation would result in private takings of waterfalls.

%t emerged that the members of parliamentary committee preparing the act had 

%Since the passage of the \cite{wra00}, it has become clear that the new authority for expropriation is a particularly controversial aspect of the act. This is because cases of waterfall expropriation today tend to imply that local owners are deprived of a small-scale development potential in favour of a commercial company. This has resulted in a new body of case law developing on takings for hydropower, as discussed in the next few sections.

\subsection{{\it Sauda}}\label{sec:5:5:1}

In {\it Sauda}, a case before the court of appeal, the riparian owners formally protested a license that granted a private company the right to expropriate their rivers and waterfalls.\footnote{See \cite{sauda07} (the decision from the district appraisal court) and \cite{sauda09} (the decision from the appraisal court of appeal).} In the district court, the owners' principal argument was that the executive could not grant such a right to a private party, since the legislation authorising private expropriation of waterfalls had not been properly authorised by parliament.\footnote{See \cite{sauda07}.}

This argument appeared weak, since the \cite{ea59} contains a provision which implies that the executive is authorised to decide what legal persons can benefit from expropriation licenses under that act.\footnote{See section 3 of the \cite{ea59}. As mentioned in Section \ref{5:2} above, the provision makes its impact in a rather roundabout way, by stating that no one except state bodies may be granted a permission to expropriate, {\it unless} the King in Council has decided otherwise.} However, the owners argued that the executive had not appropriately informed parliament that private takings of waterfalls would result from including the waterfall expropriation authority in the \cite{ea59}. It was pointed out, in particular, that the crucial amendment to the \cite{ea59} had been passed as a mere formality following the adoption of the \cite{wra00}, on the basis of the Ministry's description of the new expropriation authority as a ``simplification'' of existing law.

To back up their constitutional argument, the owners presented the written testimony of two members of the parliamentary committee that had prepared and voted for the \cite{wra00} and the associated amendments of the \cite{ea59}.\footnote{Presented to the Court in \cite{sauda07} (available from the author on request).} Neither of them could recollect that they had been aware that their actions would make private expropriation possible. Plainly, they had not been told, and had not realised on their own, that this would be the result. Their ignorance was not in fact very surprising, since the crucial change in the law was only apparent as an implicit consequence of the combined effect of three different sections in two separate acts.\footnote{The \cite[51]{wra00} and \cite[2][3]{ea59} respectively.} In the entire collection of preparatory documents, the change was discussed only once, and then only very briefly, in the report from the committee to the Ministry.

On this basis, the owners argued that the purported expropriation authority was not constitutionally valid, since parliament had not intended it. Unsurprisingly, this argument was rejected by the district court.\footnote{See \cite{sauda07}.} It had to be assumed that members of parliament understood the consequences of their own legislative actions. 

Interestingly, however, when the case went before the court of appeal, the issue was not discussed at all, since the court decided to rely on a {\it different} authority as the legislative basis for allowing the expropriation to go ahead. Specifically, since the expropriation in question involved a diversion of water, the court of appeal held that the taker did not in fact require the expropriation license it had been granted pursuant to the \cite{ea59}. Section 16 of the \cite{wra17} would suffice.\footnote{See \cite{sauda09}. See also the discussion in Section \ref{sec:5:2} above.} Hence, the constitutional question could be laid to rest.

In addition to the constitutional complaint, the owners in {\it Sauda} also raised procedural objections. They argued, in particular, that the expropriation question had been insufficiently assessed by the water authorities.\footnote{See \cite{sauda09}.} The court did not agree, and the procedural arguments at stake here foreshadow the later Supreme Court case of {\it Jørpeland}, discussed in more depth in Section \ref{sec:5:6}.

While the owners in {\it Sauda} lost the validity dispute, the level of compensation they received was dramatically increased compared to earlier practice. Because of this, the development company appealed the decision to the Supreme Court, with the owners lodging a counter-appeal regarding the question of legitimacy. The Supreme Court decided not to hear the case, possibly because it had recently considered the compensation issue in the case of {\it Uleberg}, discussed in the next section.

\subsection{{\it Uleberg}}\label{sec:5:5:2}

Just before the {\it Sauda} case was decided by the court of appeal, the Supreme Court had addressed the compensation question in the case of {\it Uleberg}.\footnote{See \cite{uleberg08}.} Here the Supreme Court agreed in principle that the natural horsepower method was not binding on the appraisal courts. Specifically, the Court held that market value compensation could be awarded just in case small-scale development by owners would have been ``foreseeable'' in the absence of expropriation.\footnote{See \cite[81]{uleberg08}.} In {\it Uleberg}, this was not the case. Specifically, the Supreme Court found that the relevant date of valuation was in 1968, when the waterfall rights in question had been transferred to the developer by a voluntary agreement.\footnote{See \cite[70]{uleberg08}.}

This agreement stated that the final payment to the owners should be fixed by the appraisal courts at the time when the development took place. Both the appraisal court and the appraisal court of appeal took this to mean that the valuation should be based on the value of the waterfall at the time when the compensation was awarded. However, the Supreme Court disagreed, holding instead that the intended reading was that the valuation should be based on the value of the waterfall at the date when the voluntary agreement was made (with interest paid for the delay).\footnote{See \cite[71]{uleberg08}.}

Furthermore, the Supreme Court then stated, without any substantive argument, that since this was the date of valuation, the natural horsepower method should be used.\footnote{See \cite[62]{uleberg08}.} Presumably, this was based on the opinion that it was obvious that owner-led development would have been `unforeseeable' at this time. The exact meaning of the foreseeability requirement has since become a much contested issue, resulting in several Supreme Court cases pertaining specifically to the compensation question.

\subsection{Recent Developments on Compensation}\label{sec:5:5:3}

Since {\it Uleberg}, there have been many controversial cases involving expropriation of waterfalls.\footnote{See generally \cite{larsen06,larsen08,larsen12}.} In most of these, the issue of compensation has occupied center stage. With respect to this issue, owners initially appeared to be gaining significant ground, as the appraisal courts started to apply a market-based method quite systematically, resulting in dramatically increased compensation payments.\footnote{See the discussion on the natural horsepower method above, in Section \ref{sec:nathp}.}

The large energy companies consistently resisted this development, typically by arguing that small-scale hydropower was unforeseeable and therefore not compensable according to the principle expressed in {\it Uleberg}.\footnote{See, e.g., \cite{klovtveit11,otra10,otra13}.} Moreover, the large energy companies would tend to argue that a license to undertake large-scale development was by itself conclusive evidence in support of the claim that small-scale development was unforeseeable.\footnote{See, e.g., \cite[17]{otra10}.} The large-scale development license showed, according to the large energy companies, that a license to undertake small-scale development could not be regarded as foreseeable.

This line of argument clearly conflicts with the so-called no-scheme principle, whereby compensation for expropriated property is to be based on the situation such as it would have been in the absence of the expropriation scheme.\footnote{This principle is also referred to as the ``Pointe Gourde'' principle in common law, and is sometimes known as the ``elimination rule''  in Europe. See generally \cite[20-22]{sluysman15}. For a more detailed presentation of the version that applies in Norway, including a comparison with England and Wales, as well as an assessment of the special issues that arise when the principle is applied to economic development takings, see \cite{dyrkolbotn15}.} In the absence of plans for large-scale development, it is often quite clear that the owners would have succeeded in obtaining a license to undertake small-scale hydropower. However, the large-scale energy companies maintained that small-scale hydropower should be considered unforeseeable even in these cases, since large-scale development was the preferred option for the licensing authorities.

In most early cases before the lower courts, this argument failed. Moreover, in the case of {\it Otra I}, it appeared as though it was rejected also by the Supreme Court.\footnote{See \cite[31-48]{otra10}.} However, the Court did not focus specifically on the no-scheme principle and how it should be applied in hydropower cases. Moreover, the taker in that case succeeded in having the appraisal court of appeal's decision overturned on the basis that inadequate reasons had been provided to justify the amount of compensation awarded to owners.\footnote{See \cite[52]{otra10}.} The court of appeal therefore had to hear the case again. This time, the taker was able to successfully argue that small-scale hydropower was unforeseeable. Hence, the court of appeal used the natural horsepower method to calculate compensation.\footnote{In fact, the court used a slightly modified version of the method, first developed in \cite{sauda09}, serving to make the discrepancy between market value and compensation slightly less pronounced. See \cite{otra12}.} 

The owners duly appealed the decision to the Supreme Court, which agreed to consider the case for a second time.\footnote{See \cite{otra13}.} But this time, the Supreme Court endorsed the understanding of the no-scheme principle of the large energy companies. Specifically, the Court refused to censor the appraisal court of appeal's assessment of foreseeability, even though it was based explicitly on the premise that the expropriation project was preferable from the point of view of the licensing authorities.\footnote{See \cite[53-54]{otra13}.}

If the precedent set by {\it Otra II} stands, market value compensation will generally not be awarded in future cases where waterfalls are expropriated in favour of large-scale schemes.\footnote{The precedent has already been used to deny small-scale compensation in the case of \cite{smibelg15} (appeal to the Supreme Court denied).} However, it should be noted that the Supreme Court has been very vague on how exactly it understands the no-scheme principle in these cases. Instead of tackling this issue directly, the Court has chosen to rely largely on deference to the foreseeability determinations carried out by the appraisal courts.

This is clearly illustrated by the earlier case of {\it Kløvtveit}.\footnote{See \cite{klovtveit11}.} Here the Supreme Court agreed with the appraisal court of appeal that it might in principle be foreseeable that the owners, in the absence of expropriation, could have cooperated with the taker to implement the expropriation project. This too contradicts the no-scheme principle, but unlike the reasoning of {\it Otra II}, it also provides an alternative route to market value compensation, on the basis of a valuation of the expropriation project itself. In effect, it points to an approach that promises to deliver a form of {\it benefit sharing} between owners and takers. 

For this reason, {\it Kløvtveit} is an interesting decision. However, it seems quite unlikely that it will become an important precedent for the future. Its importance was undermined already by {\it Otra II}, when the presiding judge explicitly denied that cooperation between the taker and the owners was a realistic scenario in that case.\footnote{See \cite[69-71]{otra13}.} Moreover, {\it Kløvtveit} itself was eventually sent back to the appraisal court of appeal, because the Supreme Court held that the date of valuation had been incorrectly determined.\footnote{See \cite[35-39]{klovtveit11}.} On the second hearing in the appraisal court of appeal, cooperation between owners and taker was regarded as unforeseeable, so market value compensation was denied.\footnote{See \cite{klovtveit13}.}

In fact, no case heard by the Supreme Court so far has concluded with compensation based on market values. In the end, the natural horsepower method has always been used. This, no doubt, sends a clear signal to the appraisal courts. In the future, it seems likely that we will see a resurgence of the natural horsepower method and a return to compensation awards amounting to tiny fractions of the actual values that are taken from local owners.

In light of this development, the broader issue of legitimacy becomes increasingly important. The financial entitlements of owners and communities, which seemed to be more strongly protected after {\it Uleberg}, are again at risk of being undermined. Moreover, as the social function theory of property indicates, the issue of legitimacy goes well beyond the individual financial entitlements of owners. It also pertains to the status of the local communities, the duties of owners in this regard, sustainable management, and the democratic legitimacy of decision-making regarding natural resources. These aspects have not received any attention from Norwegian courts so far. However, as I have already mentioned, the case of {\it Jørpeland} saw the procedural legitimacy of hydropower takings come to the forefront, for the first time since the case of {\it Alta}. In addition to clarifying legal points in this regard, the case also sheds light on the practices adopted by the water authorities in expropriation cases. Hence, it provides an excellent opportunity for a closer inquiry into the legitimacy question.

\section{A detailed case study: {\it Ola Måland v Jørpeland Kraft AS}}\label{sec:5:6}

The expropriating party was a public-private commercial partnership, Jørpeland Kraft AS. Originally, this limited liability company was jointly owned by Scana Steel Stavanger AS, with 1/3 of the shares, and Lyse Kraft AS, with the remainder.\footnote{See \cite[2]{jorpeland09}.} Lyse Kraft AS is a publicly owned energy company with the city municipality of Stavanger being the dominating shareholder. Scana Steel Stavanger AS, on the other hand, was a subsidiary of the publicly traded Scana Steel Industrier ASA. The largest shareholder of the parent company is a private individual, a leading business person and one of the richest people in the city of Stavanger.\footnote{See \cite{birkevold09} (the business man in question is John Arild Ertvaag).}

Scana Steel had long operated a steel mill in the small town of Jørpeland, belonging to the municipality of Stranda, in Rogaland county, south-west Norway. The source of energy was a relatively small hydropower plant harnessing energy from the river that reaches the sea near Jørpeland.\footnote{See \cite{aadland09}.} At the height of activity, the mill had about 1200 employees and was an important local institution.\footnote{See \cite[11]{meland82}.} However, after going bankrupt and being reorganised in 1977, the importance of the steel mill declined significantly.\footnote{See \cite[8-15]{meland82}.} After a second bankruptcy in 2015, Scana Steel Stavanger AS was wound up. The mill was reorganised yet again, and the number of employees was reduced from around 100 to around 30.\footnote{See \cite{jossang15}.}

In parallel with the decline of the steel mill, the hydropower plant in Jørpeland was rebuilt and expanded, not to supply energy for local industry, but to sell electricity on the national grid.\footnote{See \cite{aadland09}.} Jørpeland Kraft AS was charged with undertaking this development, which was thereby decoupled from the steel mill operations. In 2011, the same year when {\it Måland} came before the Supreme Court, Scana Steel Stavanger AS sold their shares in Jørpeland Kraft AS to the German investment company Aquilla Capital.\footnote{See \cite{sandvik11}.} In light of this, the story of Jørpeland also illustrates broader trends in the history of hydropower in Norway, reflecting the discussion contained in Chapter \ref{chap:4}.

The river that gave rise to controversy in {\it Måland} was not located in the same municipality as Jørpeland, but in a different valley across a mountain range, in the municipality of Hjelmeland. The contested license in {\it Måland} gave Jørpeland Kraft AS the right to divert the water from this river for electricity production at Jørpeland. In the following, I present the facts of the case in more detail, before considering the legal questions that were addressed by the courts.

\subsection{The Facts of the Case}\label{sec:5:6:1}

One relatively small river from which Jørpeland Kraft AS suggested to extract water was not located in Jørpeland. Rather, it runs through the neighbouring municipality of Hjelmeland, on the other side of a mountain range, until it eventually reaches the sea at Tau, another neighbouring municipality. 

The plans to divert the river would deprive the riparian owners of water along some 15 kilometres of riverbed, all the way from the mountains on the border between Hjelmeland and Jørpeland, to the sea at Tau. Not all the water would be removed, but the flow of water would be greatly reduced in the upper part of the river known as {\it Sagåna}, the rights to which is held jointly by Ola Måland and five other local farmers from Hjelmeland.

The water in question comes from a lake called \emph{Brokavatn}, located 646 meters above sea level, where altitude soon drops rapidly, making the river suitable for hydropower development. Plans were already in place for such a project, which would use the water from just below the altitude of Brokavatn, to the valley in which the original owners' farms are located, about 80 meters above sea level. 

A rough estimate of the potential of this project was made by the NVE itself, stating that the energy yield would be 7.49 GWh per annum.\footnote{See \cite[16]{jorpeland09}.} This is about five times more energy than the water from Brokavatn would contribute to the project proposed by Jørpeland Kraft AS.\footnote{See \cite[19]{jorpeland09}.}

Importantly, the estimate was not made in relation to the expropriation case, but as part of a national project to survey the remaining energy potential in Norwegian rivers.\footnote{The survey was carried out in 2004 and its results are summarised in \cite{jensen04}.} Ola Måland and the other owners of the river were not identified as significant stakeholders and were not notified of the assessment that had been made. Moreover, even after Jørpeland Kraft AS had submitted a formal application for permission to divert the water, the owners were not notified by the water authorities.\footnote{See \cite[16]{jorpeland09}. However, a generic orientation letter was apparently sent by Jørpeland Kraft AS, a letter that the owners themselves could not remember having received. See \cite[5|8]{jorpeland11a}.}

Moreover, the procedural approach to the case was the traditional one, with an assessment directed at evaluating the environmental impact. Many interest groups were called on to comment on environmental consequences, and public debate arose with respect to the balancing of commercial interests and the desire to preserve wildlife and nature.\footnote{See \cite[19]{jorpeland09}.}

One of the local owners, Arne Ritland, also commented on the proposed project. He did this in an informal letter sent directly to Scana Steel Stavanger AS.\footnote{See \cite[17]{jorpeland09}.} In this letter, he inquired for further information and protested the proposed diversion of water from Brokavatn. He also mentioned the possibility that an alternative hydropower project could be undertaken by original owners, but he did not go into any details, stating only that a locally owned hydropower plant had previously been in operation in the area.\footnote{The plant he was referring to dates back to the time before there was a national grid. It ensured a local supply of electricity, but has since been shut down, in keeping with the general trend mentioned in Chapter \ref{chap:4}, Section \ref{sec:4:4}.}

Arne Ritland received a reply from Scana Steel Stavanger AS, which stated that more information on the project and its consequences would soon be provided. Ritland did not pursue the matter further at this time. Meanwhile, Scana Steel Stavanger AS submitted his letter to the NVE, who in turn presented it as a comment directed at the application.\footnote{\cite[18]{jorpeland09}.}

This prompted the majority owner of Jørpeland Kraft AS, Lyse Kraft AS, to undertake their own survey of alternative hydropower in Sagåna.\footnote{See \cite[19]{jorpeland09}.} The conclusions were sent to the water authorities, but the owners were not informed that such an investigation was being conducted.\footnote{See \cite[23]{jorpeland09}.} Moreover, the water authorities did not take steps to investigate the commercial potential of local hydropower on their own accord. Instead, they referred to the conclusion presented by Jørpeland Kraft AS, stating that if the local owners decided to build two hydropower plants in Sagåna, then one of them, in the upper part of the river, would not be profitable, neither with nor without the contested water. The other project, in the lower part, could apparently still be carried out, even after the diversion.\footnote{See \cite[23]{jorpeland09}.}

No mention was made of what the original owners stood to loose, nor was there any argument given as to why it made sense to build two separate small-scale power plants in Sagåna. Nevertheless, the NVE handed the expropriating party's findings over to the Ministry, without conducting their own assessment and without informing the original owners.\footnote{See \cite[22-23]{jorpeland09}.}

In addition to the report made by Jørpeland Kraft AS, the municipality government of Hjelmeland also commented on the possibility of local hydropower. In their statement to the NVE, they directed attention to the data in the NVE's own national survey, which suggested that a single hydropower plant in Sagåna would be a highly beneficial undertaking.\footnote{See \cite[19]{jorpeland09}.} On this basis, they protested the diversion, arguing that original owners should be given the possibility of undertaking such a project.

This statement was not communicated to the original owners, and in their final report the NVE dismissed it by stating that the most efficient use of the water would be to transfer it and harness it at Jørpeland.\footnote{See \cite[19]{jorpeland09}.}

In addition to the statement made by Ritland, one other property owner, Ola Måland, commented on the plans.\footnote{See \cite[17]{jorpeland09}.} He did so without having any knowledge of the commercial potential of the waterfall and without having been informed of the statement made by the municipality of Hjelmeland. On this basis, Måland expressed his support for Jørpeland Kraft's plans, citing that the risk of flooding in Sagåna would be reduced.\footnote{He later joined the other owners in opposition to the expropriation.} He also phrased his letter in such a way that it could be interpreted as a statement on behalf of the owners as a group.\footnote{See \cite[17]{jorpeland09}.} However, Måland was the only person who signed.

In the final report to the Ministry, the NVE refers to Måland's letter and state that the original owners are in favour of the plans.\footnote{See \cite[19]{jorpeland09}.} For this reason, the NVE concludes that the opinion of the municipality of Hjelmeland should not be given any weight.\footnote{See \cite[19]{jorpeland09}.} The NVE neglects to mention that Arne Ritland's statement strongly opposed the development. %Moreover, earlier in the report, where all incoming statements are reported, Ritland is referred to as a private individual, while Ola Måland is referred to as a property owner who speaks on behalf of the owners as a group.

The report made by the NVE was not communicated to the affected local owners at all, so the owners had no chance of correcting the mistakes that had been made. However, the report was sent to many other stakeholders, including the municipality of Hjelmeland.\footnote{See \cite[24]{jorpeland09}.} In light of the report, the municipality changed their original position and informed the Ministry that they would not press for local hydropower, since this was not what the affected owners (i.e., Ola Måland) wanted.\footnote{See \cite[24]{jorpeland09}.}

This happened without the owners' knowledge. However, while the case was being prepared by the water authorities, the original owners had begun to seriously consider the potential for hydropower on their own accord. In late 2006, Jørpeland Kraft's application reached the Ministry and a decision was imminent. At the same time, the owners were under the impression that they would receive further information before the case progressed to the assessment stage.

All the owners, including Ola Måland, had now come to realise the commercial value of the water from Brokavatn. Hence, they approached the NVE, inquiring about the status of the plans proposed by Jørpeland Kraft AS. They were subsequently informed that an opinion in support of the transfer had already been delivered to the Ministry. This communication took place in late November 2006, summarised in minutes from meetings between local owners, dated 21 and 29 November.\footnote{Presented to the courts, available upon request.} On 15 December 2006, the King in Council granted a concession for Jørpeland Kraft AS to transfer the water from Brokavatn to Jørpeland.\footnote{See \cite[3]{jorpeland09}.}

At this point, it had become clear to the original owners that the water from Brokavatn would be crucial to the commercial potential of their own project. They also retrieved expert opinions that strongly indicated that the NVE was wrong to conclude that diversion to Jørpeland would be the most efficient use of the water.\footnote{See \cite[23]{jorpeland09}.} In light of this, the owners decided to question the legality of the licence (with the corresponding permission to expropriate). They argued, in particular, that the administrative decision to grant the license was invalid.

In the following section, I present the main legal arguments relied on by the parties, as well as a summary of how the three national courts judged the case.

\subsection{Legal Arguments}\label{sec:5:6:2}

First, the owners argued that procedural mistakes had been made by the water authorities when preparing the case.\footnote{See \cite[12]{jorpeland09}.} This, in turn, had resulted in factual mistakes forming the basis of the decision to grant the development license. Since the outcome might have been different if these mistakes had not been made, the owners concluded that the development license could not be upheld.

Second, the owners argued that expropriation of their rights would result in a disproportionate loss of an economic development potential.\footnote{See \cite[5]{jorpeland11a}.} Moreover, they argued that the economic loss would clearly be greater than the gain also from the point of view of the public, since the owners were in a position to make more efficient use of the contested water. Therefore, allowing expropriation would only serve to benefit the commercial interests of Jørpeland Kraft AS, to the detriment of both local and public interests.

Third, the owners argued that the government had not fulfilled its duty to consider the case with due care.\footnote{See \cite[12]{jorpeland09}.} In particular, the assessment of local community interests and the interests of local owners had not been satisfactory. Particular attention was directed at the fact that local owners had not been informed about the progress of the case, and had not been told of assessments pertaining to their interests.

Fourth, the owners argued that irrespective of how the matter stood with respect to national law, the expropriation was unlawful because it would be in breach of the provisions in P1(1) of the ECHR regarding the protection of property.\footnote{See \cite[07-08]{jorpeland09}.}

Jørpeland Kraft AS protested, arguing first that the report from the NVE was not based on factually erroneous information.\footnote{See \cite[16]{jorpeland11}.} With respect to the apparent mistakes that had been made, Jørpeland Kraft AS argued that these did not in any event undermine the quality of the report as a whole.\footnote{See \cite[2]{jorpeland11a}.} Moreover, it was argued that Måland had probably discussed the diversion of water with other affected owners, and that they had all probably agreed to support it.\footnote{See \cite[2]{jorpeland11a}.} Furthermore, according to Jørpeland Kraft AS, all the procedural rules of the \cite{wra17} had been observed. Other procedural rules might be relevant, but only if they were compatible with the rules in the \cite{wra17}.\footnote{See \cite[16]{jorpeland11}.} It was also argued that it was not for the courts to subject the assessment of public and private interests to any further scrutiny, since this was a matter for the administrative branch.\footnote{See \cite[2]{jorpeland11a}.} Finally, Jørpeland Kraft AS argued that diverting the water did not represent a breach of the owners' human rights.\footnote{See \cite[2]{jorpeland11a}.} They argued for this by pointing to the fact that the procedural rules had been followed and that the material decision was discretionary. Moreover, Jørpeland Kraft AS argued that since the owners would be compensated financially for whatever loss they incurred, it was clear that no human rights issues were at stake in the case.\footnote{See \cite[2]{jorpeland11a}.}

\subsection{The Lower Courts}\label{sec:5:6:3}

The matter went before the district court in the city of Stavanger, which decided in favour of the owners on 20 May 2009.\footnote{See \cite{jorpeland09}.} The district court agreed with the local  owners that the decision to grant the license was based on an erroneous account of the relevant facts.\footnote{See \cite[25]{jorpeland11}.} Moreover, the court concluded that it was evident that allowing the applicants to use the water from Brokavatn in their own hydroelectric scheme would be the most efficient way of harnessing the hydropower potential.\footnote{See \cite[22-23]{jorpeland09}.} This, the court noted, directly contradicted what the NVE had stated in their report.\footnote{See \cite[23]{jorpeland09}.}

The court backed up its conclusion on the facts by giving several direct quotes from the report made by the NVE. On the legal side, they relied on a well-established principle of administrative law: while the exercise of discretionary powers is usually not subject to review by court, a decision based on factual mistakes is invalid if it can be shown that the mistakes in question were such that they could have affected the outcome.\footnote{See \cite[407-410]{eckhoff14}. For the requirement that the mistakes must have been such that they could have affected the outcome, see \cite[41]{paa67}} Since the small-scale alternative would in fact represent a more effective use of the water in question, the court was not in doubt that this principle applied here.\footnote{See \cite[25]{jorpeland09}.}

Since the district court held that the license to allow diversion was invalid because it was based on factual mistakes, there was no need to consider claims regarding the legitimacy of the diversion with respect to human rights law. However, the district court did comment that the traditional procedure used to deal with diversion cases was inadequate and had to be supplemented by looking to the procedural rules in the \cite{ea59} and the \cite{paa67}.\footnote{See \cite[21]{jorpeland09}.} 

Moreover, the court made a crucial statement about expropriation of riparian rights in general, regarding the duty of the water authorities to properly assess whether or not an expropriation license should be granted.\footnote{The duty is a general principle of administrative law, expressed both in \cite[12]{ea59} and \cite[16]{paa67}.} This duty, the court held, included a duty to properly consider negative effects on small-scale development potentials.\footnote{See \cite[22]{jorpeland09}.} According to the court, this was the natural consequence of the increasing interest in small-scale development. 

If the principle expressed by the district court on this point had been allowed to stand, it would have had significant implications for the water authorities more generally, as it would directly confront their traditional lack of interest in the expropriation question. However, it was not to be, as the court's decision was overturned on appeal.

The court of appeal approached the case very different than the district court. Specifically, its decision did not rely on any close assessment of the facts against the report made by the NVE. Instead, the court of appeal largely based its decision on the opinion that the rules in the \cite{wra17} exhaustively regulate the administrative procedure in watercourse regulation cases.\footnote{See \cite[7]{jorpeland11a}.} According to the court of appeal, the procedural rules in the \cite{ea59} and the \cite{paa67} do not apply at all to diversions of water authorised under section 16 of the \cite{wra17}.\footnote{See \cite[7]{jorpeland11a}.} 

This finding was based on the argument that the more specific rules of the \cite{wra17} have priority under the so-called {\it lex specialis} principle, which applies in case of conflict between different sets of rules, giving priority to those that are more specific.\footnote{See \cite[7]{jorpeland11a}.} Apparently, the court thought that the procedural rules of the \cite{ea59} and the \cite{paa67} conflicts with the rules that apply specifically in hydropower cases.\footnote{The court also made a sweeping remark to the effect that the rules in the \cite{wra17} conform to all ``basic and general'' procedural demands of administrative law. This, however, seems to be a reference to core unwritten principles, not specific provisions included in the \cite{paa67} and the \cite{ea59}.} With regard to the procedural rules of the \cite{wra17}, the concludes that the assessment made by the water authorities met all general requirements and was clearly adequate. Regarding the factual basis for the license, the court did not comment on most of the evidence presented to them. Moreover, the Court did not address those quoted segments of the report from the NVE that had formed the basis for the district court's decision.

Specifically, the court of appeal never mentions the objection to the transfer made by the municipality of Hjelmeland, nor does it mention the fact the small-scale alternative proposed by the municipality would use the contested water more effectively. Instead, the court of appeal points out that the NVE was well aware of the possibility of developing small-scale hydropower, was well-informed about such development, and had considered it during their assessment.\footnote{See \cite[9]{jorpeland11a}.} The court of appeal notes that the NVE's written assessment on this point was brief, but argues that this must be understood as a natural response to what the court of appeal describes as a lack of input from local owners.\footnote{See \cite[9]{jorpeland11a}.}

The owners appealed the court of appeal's decision to the Supreme Court, which decided to hear the  juridical aspects of the case.\footnote{See \cite[8]{jorpeland11}. Specifically, the Supreme Court would not engage in any independent fact-finding, but only consider legal questions, including how the law should be applied to the facts.}

\subsection{The Supreme Court}\label{sec:5:6:4}

The Supreme Court approached the case in much the same way as the court of appeal. Regarding the facts, the Court emphasised that the majority owner of Jørpeland Kraft AS had considered the possibility that a hydroelectric scheme could be undertaken by local property owners.\footnote{See \cite[53]{jorpeland11}.} As mentioned, Lyse Kraft AS had indeed made a report on this, concluding that one small-scale plant would be unprofitable regardless of the diversion, while another one, further down in the river, could still be carried out.\footnote{See \cite[23]{jorpeland09}.} As mentioned earlier, the report did not explain why anyone would want to build two consecutive small-scale plants in the same river.\footnote{See \cite[16|23]{jorpeland09}.}

In any event, the most relevant question would be what the owners stood to loose when the water from Brokavatn was diverted away from Sagåna. Both the report and the Supreme Court remained silent on this point. Moreover, the Court does not mention that the report was never handed over to the applicants, nor that the details of the calculations were never independently considered by the NVE. Just like the court of appeal, the Supreme Court also neglects to mention that small-scale development would be a more efficient use of the water, according to the national survey of small-scale potentials carried out by the NVE itself.\footnote{See \cite[16]{jorpeland09}.} Furthermore, no mention is made of the fact that the NVE claims that the opposite is true in the report to the Ministry, contradicting also the statement made by the municipality of Hjelmeland.

Regarding the legal questions raised by the case, the Supreme Court rejects the view that the procedural rules in the \cite{ea59} and the \cite{paa67} do not apply to the case.\footnote{See \cite[32-34]{jorpeland11}.} However, the Court holds that these procedural rules do not imply a more extensive duty to assess the expropriation question, compared to established practices in hydropower cases.\footnote{See \cite[51-52]{jorpeland11} (citing also the {\it Alta} case, \cite{alta82}).}

Moreover, there is no rule in the \cite{wra17} which states that the authorities are required to consider specifically the question of how the regulation affects the interests of property owners. This is also not considered in practice, except perhaps to some extent if the issue is explicitly and forcefully raised during the hearing.\footnote{See \cite{stokker10}. This is the water authorities' own guideline for the assessment of large-scale applications. The previous version of this guideline (which also fails to mention the interests of owners) was presented to the Supreme Court. The Court also refers to it explicitly when it comments that existing practices are beyond reproach. See \cite[51]{jorpeland11}.} However, a rule explicitly demanding this is found in section 2 of the \cite{ea59}. This is not regarded as a procedural rule, as it pertains to the material considerations that the administrative branch is required to carry out in expropriation cases.

Indeed, according to the Supreme Court, the rule does not apply at all when expropriation takes place on the basis of section 16 of the \cite{wra17}.\footnote{See \cite[30]{jorpeland11}.} This is the conclusion despite the fact that section 30 of the \cite{ea59} explicitly states that the provisions of that act apply to expropriations pursuant to the \cite{wra17}, in so far as they are compatible with the rules therein. It would appear to follow, by implication, that the Supreme Court does {\it not} think that directing more attention at owners' interests, as prescribed by section 2 of the \cite{ea59}, is compatible with the \cite{wra17}.

This is a clear rejection of the principled position taken by the district court, whereby the water authorities should generally be obliged to consider small-scale alternatives before allowing expropriation. According to the Supreme Court, no special procedural obligations arise at all in such cases, which can still processed exactly as they would have been if all the waterfalls already belonged to the applicant. In short, expropriation is allowed to remain a non-issue during the licensing process pursuant to the \cite{wra17}.

Formally, this implication of {\it Jørpeland} only applies to expropriations carried out on the basis of section 16 of that act. However, in practice, there is reason to believe that the impact will be the same for all cases involving large-scale hydropower development. Indeed, the water authorities themselves do not appear to make any significant distinction between large-scale applications based on whether or not a separate license to expropriate waterfalls is formally required.\footnote{See \cite{flatby08}.}

\noo{ %It also bears noting that the facts in {\it Jørpeland} appear to suggest that the procedural shortcomings underlying that case were much more obvious than the shortcomings complained of in {\it Alta} (although the scale of the underlying conflict was much greater in {\it Alta}).
To further illustrate the extent to which {\it Jørpeland} signals a dismissive attitude towards owners and local communities, I will conclude by offering a quote from Harald Solli, director of the hydropower licensing section at the Ministry. Sollie submitted written evidence to the Supreme Court regarding the practices observed in cases involving expropriation of water power. Below, I quote two exchanges that demonstrate how current practices leave local owners in a precarious position.

\begin{quote}
Q: In cases pursuant to the \cite{wra17}, is it common for the water authorities to send prior written notices to the private owners that may be affected by a loss of a small-scale hydropower potential? \\
A: The procedural rules that apply to cases pursuant to the \cite{wra17} are found in section 6. To give such a written notice to private owners is not required. As far as I am aware, it is also not done, but I have no first-hand knowledge of this, since the NVE is responsible for the case at this stage. \\
Q: In cases such as this, should owners affected by the loss of a small-scale hydropower potential be kept informed about the factual basis on which the authorities plan to make their decision? I am thinking especially about cases when the authorities do in fact provide an assessment of the potential for small-scale hydropower on private properties. \\
A: Affected owners must look after their own interests. The assessments made by the NVE in their report is a public document, and it can be accessed through the homepage of the NVE.
\end{quote}

By their reasoning in \emph{Jørpeland}, it appears that the Supreme Court gave this dismissive attitude towards local owners a stamp of approval. In light of this, I believe the study of the law in a socio-legal setting becomes all the more relevant. For while the dismissive attitude might be a part of the national legal order, it seems pertinent to ask if it is a reasonable attitude to take towards local owners of valuable natural resources. Also, one may ask if a case can be made with respect to human rights, by arguing that the protection awarded is insufficient with regard to P1(1). This point was raised in \emph{Jørpeland}, but did not receive any attention from the Supreme Court.\footnote{The {\it Jørpeland} case resulted in a complaint to the ECtHR which has yet to be considered by the Court.}
}

\section{Predation?}\label{sec:5:7}

How should takings of waterfalls be assessed according to the normative theory developed in the first part of this theory? In Chapter \ref{chap:3}, I presented the Gray test, a set of key assessment points for determining whether a taking violates important property norms.\footnote{See Chapter \ref{chap:3}, Section \ref{sec:3:5}.} In the following, I briefly assess takings of waterfalls against the criteria of the Gray test, to shed further light on the normative status of current practices observed in Norway.

\subsubsection{The Balance of Power}\label{sec:5:7:1}

In light of the presentation so far, it is safe to conclude that typical large-scale waterfall expropriations in Norway are marked by a severe imbalance of power between the taker and the owners. The economic and political power of local communities is clearly very limited compared to that of the large energy companies. Moreover, it is interesting to observe that this imbalance is accentuated by procedural arrangements and practices presided over by the water authorities. As demonstrated by the case of {\it Jørpeland}, the formal position of owners and local communities under administrative law is very weak in hydropower cases. Hence, in addition to shedding doubt on the legitimacy of current practices in Norway, assessing waterfall takings against the balance of power criterion also underscores that this criterion is related to administrative law.

Ideally, procedural rules should function so as to maintain an appropriate balance of power between the different actors involved in an administrative dispute. At least, the rich and powerful should not be allowed to dominate decision-making processes within the polity, at the expense of those most intimately affected by the decisions reached. If the administrative branch fails in this regard, or acts in such a way that existing imbalances are worsened, this is surely a cause for additional criticism with respect to the balance of power criterion. I believe the case study so far shows that the framework for management of Norwegian hydropower is deserving of such criticism.

\subsubsection{The Net Effect on the Parties}\label{sec:5:7:2}

The immediate financial effect that a taking for hydropower has on the owners depends on how the compensation is calculated. As discussed in Section \ref{sec:5:4:3}, the law on this point has been in turmoil in recent years. In the late 2000s, there were signs that a commercially realistic valuation method might become dominant, leading in turn to a dramatic increase in compensation compared to earlier practice based on the natural horsepower method. But this trend now appears to have been reversed, as the energy companies have successfully argued that a license for large-scale development counts as proof that owner-led projects would not in any case have been `foreseeable' (because the necessary licenses would not have been granted). For this reason, the argument goes, owners suffer no actual loss when their resources are taken from them.\footnote{See \cite{otra13}.}

The local owners are in an even weaker position when it comes to indirect financial effects, as well as social and political effects, such as harms done to the cohesion and prosperity of the local community. In this regard, losses are not only under-compensated, they are typically not acknowledged at all, neither by the executive nor by the courts. The effects that go unnoticed range from the concrete, such as losses incurred because the expropriation proceedings drag out in time, to the abstract, such as the damage that is done to democracy when owners and local municipality governments are replaced by energy companies as the primary resource managers in the local district.\footnote{In \cite{smibelg15}, the owner submitted an application for small-scale hydropower in 2005 which the water authorities refused to process on account of a pending large-scale application. In 2015, compensation was awarded based on the natural horsepower method, with no compensation for, or even acknowledgement of, the owners' loss in the 10 year period where the water authorities refused to process their applications.}

\subsubsection{Initiative}\label{sec:5:7:3}

It follows from the regulatory framework that almost all cases involving expropriation for hydropower development originate from applications submitted by commercial companies.\footnote{See especially Chapter \ref{chap:4}, Section \ref{sec:4:4} and Chapter \ref{chap:5}, Section \ref{sec:5:3}.} The energy company draws up the plans and initiates the expropriation proceedings, by submitting a request for a development license to the water authorities. The main purpose, which is usually acknowledged by both the applicant and the water authorities, is to make money. Hence, it is usually hard or impossible to argue that takings of waterfalls for hydropower development in Norway are motivated by any direct public interests. %Indeed, applying the initiative test will suffice to conclude that the primary motive is profit-making. 

Exceptions to this are possible, in so far as the energy companies themselves embody public service functions. In some cases one might argue that they do, but such arguments are becoming increasingly unconvincing due to the fact that most energy companies have been reorganised as for-profit enterprises whose activities are largely unconstrained by institutions of local government.\footnote{See, e.g., the EFTA Court's description of the industry, \cite{efta07}.}

\subsubsection{Location}\label{sec:5:7:4}

Compared to notorious US cases such as {\it Kelo} and {\it Poletown}, the stakes for the owners appear lower in hydropower cases from Norway.\footnote{See \cite{poletown81,kelo05}.} However, as mentioned in Chapter 4, riparian rights are often of great importance to Norwegian farming communities and the subsistence of its members.\footnote{See especially Chapter \ref{chap:4}, Sections \ref{sec:4:2}, \ref{sec:4:4} and \ref{sec:4:5}.} Indeed, the taking of riparian rights from a local community might well contribute significantly to depopulation, although indirectly rather than by physical displacement.\footnote{Today, it is very unlikely that the Norwegian government would sanction physical displacement of people from their homes in order to facilitate hydropower development. However, this state of affairs cannot be taken for granted; the current political attitude on this point appears to have arisen in large part due to extensive and forceful anti-development activism during the 1970s, especially in relation to the {\it Alta} case (which initially involved plans to physically displace a local Sami community). See \cite{altawiki}.} Moreover, in many rural communities, small-scale hydropower appears to be the only growth industry, as farming is becoming increasingly unprofitable and communities are threatened by stagnation and decline.\footnote{For an example, I refer again to the case of the Gloppen municipality, discussed in Chapter \ref{chap:4}, Section \ref{sec:4:4}.}

Hence, the location criterion suggests that takings of waterfalls merit heightened critical scrutiny, especially due to the importance of the property that is taken to the subsistence of the local communities forced to give it up.

\subsubsection{Social Merit}\label{sec:5:7:5}

There is no shortage of electric energy in Norway, and electricity prices are very low compared to the rest of Europe.\footnote{See Chapter \ref{chap:4}, Section \ref{sec:4:1}.} Indeed, development projects such as {\it Jørpeland} are not motivated by any particular need to supply more energy to the Norwegian people or local industry, but openly pursued as commercial endeavours.\footnote{See Chapter \ref{chap:5}, Section \ref{sec:5:6}.} Hence, they do not appear to have any particular social merit.

On the contrary, waterfall takings can contribute to creating social ills. In south-western Norway, for instance, where {\it Jørpeland} is located, the average income for a sheep farmer corresponds to about half of the minimum wage that farmers are required to pay to full-time farm workers.\footnote{According to the Norwegian Bioresearch Institute, the average sheep farmer could expect to earn NOK 65 per hour from working at their farm in 2012. See \cite[50]{smesdal14}. The minimum wage for unskilled farm workers during the same time was NOK 123.15 per hour. The minimum wage for 16-17 year old vacation workers was NOK 83.75 per hour. See \cite{tariff12}.} During harvesting season, sheep farmers wishing to hire 16 year old vacation workers are required to pay the kids about 30 \% more per hour than they themselves can expect to earn from running their own farms. In short, sheep farming communities in western Norway, such as that affected by the taking in {\it Jørpeland}, are struggling.

In this context, it seems that the social harm created by expropriation, whereby disadvantaged rural communities are deprived of the opportunity to manage their own water resources, should be a pressing concern. Because of the traditional approach to hydropower, focusing solely on environmental harms, the social merit of maintaining local ownership and control over resources receives little or no attention from the water authorities in expropriation cases. This in itself suggests that typical cases of waterfall expropriation in Norway will tend to fail the social merit test.

\subsubsection{Environmental Impact}\label{sec:5:7:6}

It is clear that hydropower development can have negative environmental impacts. Hence, it is important that the value of development is appropriately balanced against environmental interests. To ensure this is a core aspect of the regulatory system. As discussed in Chapter 4, local initiatives for small-scale hydropower are now typically scrutinized quite intensely in this regard, particularly after reforms in recent years. By contrast, large-scale projects appear increasingly likely to receive preferential treatment. %Since 2000, only one such project has been denied a license by the water authorities.\footnote{....}
 
Moreover, the large companies are clearly in a better position to exert pressure on the regulator and to invest in lobbying in order to overcome regulatory hurdles. The fact that large-scale solutions continued to receive priority, despite it being official government policy for a decade that no more large-scale plants should be built, is an indication of the severity of this effect. Hence, while the debate continues regarding the comparative environmental merits of different kinds of hydropower, it appears safe to conclude that the dynamics of power on display in relation to environmental issues raise further doubts about the legitimacy of waterfall takings.

\subsubsection{Regulatory Impact}\label{sec:5:7:7}

When waterfall rights are expropriated, they also become a separate commodity, divorced from the surrounding land rights. They are also typically removed from the sphere of municipal control on land use, falling instead under the regulatory jurisdiction of the centralised water authorities. Hence, the regulatory context shifts from one emphasising holistic resource management and local community needs to one which focuses mainly on facilitating hydropower development.

Moreover, the fact that the takers of waterfalls are powerful actors might make it harder for regulators to do their job. After expropriation, the parties who stand to loose from increased regulation are the state-supported energy companies. They are therefore likely to oppose stricter standards, and to do so in a manner that is much more forceful than any lobbying one might expect from local community owners of waterfalls. Hence, takings in this sector appear likely to cause systemic imbalances and a push for less intrusive government control, or government regulation on terms dictated by the major market players. A sign of this effect can be found in recent controversial decisions made with regard to the national grid, where the interests of the electricity industry appears to have completely overshadowed broad public opposition against further environmental intrusions in valuable nature areas in the west of Norway.

\subsubsection{Impact on Non-Owners}\label{sec:5:7:8}

Non-owners can exercise some influence during the licensing procedure. However, this requires them to be organised or aligned with special interest groups. Organisations, rather than individuals, are entitled to the greatest protection under the \cite{wra17}. The non-owners most directly affected by hydropower development are usually local residents, from the same community as the waterfall owners. These owners have little chance of being heard in the process, except if they find that their interests are aligned with those of more powerful stakeholders, such as national or regional environmental groups. In general, the means available for local non-owners to partake in the decision-making do not appear commensurate with the local stakes in hydropower cases.

The transfer of property to a large-scale owner, moreover, changes the dynamic of interaction between owners and non-owners. Formally, the transfer of riparian rights away from the jurisdiction of municipal governments is particularly significant, since it significantly reduces the level of (local) democratic control over the use of the water resource. In addition, one should again consider the informal effect of transferring property away from local community members to large corporations. Unlike local owners, corporations that take waterfalls appear highly unlikely to interact with local non-owners on equal terms.

\subsubsection{Democratic Merit}\label{sec:5:7:9}

Following \emph{Jørpeland}, it seems that owners' right to participate in decision-making processes regarding the use of their rivers and waterfalls is extremely limited under Norwegian law. The regulatory system effectively negates private property rights by making expropriation an automatic consequence of any large-scale development license granted to any non-owner. The original justification for this might be found in the idea that the regulatory power of the state should take precedence over private proprietary entitlements. However, after the liberalisation of the energy sector, this idea has transformed completely into a practice of systematic prioritisation of powerful commercial interests at the expense of local communities. This has happened despite the political commitment to end large-scale development, which remained official government policy for over a decade.

In light of this, it is especially hard to see any democratic merit in the practice of taking waterfalls for profit, to the benefit of large-scale development companies. Overall, it seems clear that according to the Gray test, current rules and practices regarding takings for hydropower render such takings highly suspect with regard to the question of legitimacy.

\section{Conclusion}\label{sec:5:8}

This chapter has explored expropriation of waterfalls, focusing on the legitimacy issue. The presentation has focused on making the case that property rights have effectively been rendered subservient to the management framework set up by the sector-oriented water resource legislation. Specifically, the chapter tracks a transformation of the regulatory framework whereby the licensing authority is now used by the government to exercise {\it de facto} proprietary control over water resources, unconstrained by the fact that these resources remain in private ownership.

As such, this chapter has shed further light on the tension identified at the beginning of the previous chapter, between hydropower as private property and hydropower as a national asset. Importantly, however, the flavour of this particular ``national asset'' is strongly influenced by the liberalisation of the electricity sector. In particular, this chapter has made the case that takings of waterfalls today are pure takings for profit. Moreover, the government itself does not even feel the need to argue otherwise, since expropriation simply follows automatically from large-scale development licenses.

The chapter used the case of {\it Jørpeland} to shed light on the effect that this can have in practice, showing how owners desiring to carry out alternative projects can be completely marginalised in the decision-making process, regardless of the merits of their proposals. In light of this, based on a combination of concrete and general observations about the Norwegian system, I concluded that this system fails to deliver on legitimacy in the sense of the word explored in Part I of this thesis. 

This chapter has thus identified a legitimacy problem, raising the question of how to resolve it. This is addressed in the next chapter, where I consider the institution of land consolidation and how it is used as an alternative to expropriation in Norway.% which local owners have made very active use of in cases when {\it they} wish to impose economic development on recalcitrant neighbours. Arguably, the consolidation approach represents a solution in line with Ostrom's design principles for local self-governance and common pool resource management. However, the system does have some idiosyncratic features which, if anything, increases the degree of control that stakeholders other than the owners can exercise over the resource in question. This and more is explored in depth in the next chapter of this thesis.