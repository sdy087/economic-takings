\chapter{Taking waterfalls}\label{chap:4}

\section{Introduction}\label{sec:intro4}

In this chapter, I address expropration of waterfalls. 

In this chapter, we address some recent demands made by local people, to the effect that they should be allowed to partake more actively in decision-making processes regarding local waterfalls. We focus on the legal status of such demands under Norwegian law, and we do so by considering the recent Supreme Court Case of \emph{Ola Måland and others v. Jørpeland Kraft AS}.\footnote{Ola M{\aa}land and others v. J{\o}rpeland Kraft AS, Rt 2011 s. 1393. I mention that I represented the local owners in this case, as a trainee lawyer in the district and regional courts, and as the responsible lawyer before the Supreme Court.} In this case, the local owners protested the legality of a license that granted the developer, Jørpeland Kraft AS, a right to divert water away from their waterfalls, thereby reducing the potential for local hydro-power. The owners argued that consultation had been insufficient, and they also contended that the assessment of the case made by the water authorities had been inadequate, and that the decision has been based on an erroneous account of the facts. They won the case in the district Court, Stavanger Tingrett, but lost under appeal to the regional Court, Gulating Lagmannsrett. The Norwegian Supreme Court also found in favor of the developer, and the argument they gave to support this conclusion goes far in suggesting that while the commercial interests of local owners need to be compensated, the presence of such interests do not entitle local communities to a greater say in decision-making processes. Importantly, the Court held that the presence of local interests does not necessitate the adoption of different administrative practices, and that the procedures developed during the decades of direct, non-commercial, administration of the energy sector, could still be followed.

The case is significant, since the traditional hydro-electric scheme in Norway typically involves expropriation, often interfering with the property rights of hundreds of local individuals. Traditionally, owners of waterfalls would be compensated according to a standardized mathematical method that was based on the assumption that they had no interest in hydro-power themselves.\footnote{The method consists in calculating the number of \emph{natural horsepowers} in the waterfall, and then multiplying this number by a price pr. natural horsepower, determined by the discretion of the Court, but in practice based almost solely on what has been awarded in previous cases. For a description of the traditional method, we point to \cite{falk} Chapter 7, page 521-522 (in Norwegian).} In a landmark case from 2008, however, the Norwegian Supreme Court commented, in an \emph{obiter dicta}, that the traditional method for calculating compensation for waterfalls was no longer appropriate, at least not in cases when it can be demonstrated that the original owners could have exploited the resource themselves, if the expropriation had not taken place.\footnote{The case of Agder Energi Produksjon AS vs. Magne Møllen, Rt. 2008 s. 81. The local owner lost the case, the reason being that the Supreme Court held that compensation should not be based on the present day value of the waterfall, but the value it had when the original transferral of rights took place, in the 1960's. The \emph{obiter dicta} has been used as an authority for subsequent Supreme Court decisions, however, see, for instance, Rt. 2010 s. 1056 and Rt. 2011 s. 1683. It has received quite a lot of scholarly attention as well, see \cite{Tf1,Tf2,Tf3}.} This decision has had a profound impact on the level of compensation awarded for waterfall rights, leading to payments that can will typically be ten to a hundred times higher than that which would have been awarded according to the traditional method.\footnote{So far, in cases that have come before the court, there has been about a twenty-fold increase in compensation, see \cite{Tf1}, but the new method will, when applied to certain kinds of projects (cheap to build, and involving little or not regulation of the water-flow), result in compensation having to be paid that amounts to at least a hundred times more than what could be expected if the traditional method had been applied.} More generally, it also served to shift the balance of power in favor of local owners and their communities, who increasingly expect to have their voices heard, and to get a more direct say over how their energy resources are managed and exploited.

After \emph{Måland}, however, it has become unclear to what extent the presence of local, commercial interests will continue to influence the Norwegian energy sector, and if we will see more active participation by local people in the future. In fact, recent statements made by the Norwegian water authorities seem to suggest that this is becoming increasingly unlikely, as a shift in policy seems to have taken place, whereby local, small scale projects, are now to be given lower priority than large scale projects undertaken by established energy companies.\footnote{These statements were not linked to \emph{Måland}, but were made in a more general context, ostensibly motivated by the desire to increase the efficiency of the administrative process, see http://www.nve.no/no/Konsesjoner/Vannkraft/Smaakraft/ where the new policy was announced. It also received some attention from the press, see, for instance, http://www.tu.no/energi/2012/01/18/nve-varsler-flere-smakraft-avslag.}

Taking a broader view on Norwegian law, we believe that recent experiences regarding hydro-power provides an interesting case to study, and one that will shed light on how property rights function in a social, economic and political context. It seems, in particular, that the view of property rights to waterfalls adopted by the Supreme Court rests on a narrow interpretation, seeing such rights merely as bestowing financial interests on certain individuals. This was clearly felt in \emph{Måland}, and we think the case also serves to illuminate certain consequences of such a view, suggesting, in particular, that it can have detrimental social and political consequences, and can very easily lead to perceived injustices. We also think it is pertinent to ask if the narrow view of property which seems to have been adopted for waterfalls in Norway is adequate with respect to human rights law, or if the right to property should also be considered a right to participate, and a right to be heard, in decision-making processes.

In the following, we first give the reader some further background on Norwegian hydro-power, and then 
we present \emph{Måland} in some detail, focusing on giving the reader an impression of current administrative practices, by describing how they played out in this particular case, and by detailing how they came to result in a decision that the original owners felt to be fundamentally unjust. We also address the legal arguments given by the opposing sides and the arguments relied upon by the national courts. We conclude by presenting some overreaching issues that we believe the case raises, regarding both the social context of property rights, the content of property as human right, and the question of whether or not the protection awarded under Norwegian law currently meets the standard set by the European Convention of Human Rights, as interpreted by the Court in Strasbourg. 

\section{Norwegian expropriation law: A brief overview}

As mentioned in Chapter \ref{chap:2}, Section \ref{sec:...}, the right to property is enthrenched in section 105 of the Norwegian Constitution. There, it is made clear that when property is taken for public use, full compensation is to be paid to the owner. As I have already mentioned, the public use requirement is understood very widely, or not regarded as a requirement at all. However, it is a rule of unwritten constitutional law that exporpriation, as it affects the rights of individuals, can only be carried out when it is positively authorised by the law. The executive does not have a general right to deprive private owners of property, but must rey on rules, usually provided for in acts of parliament, that authorises compulsory acquisition of property on spcific terms.  

Historically, there was no general act relating to expropriation, and a range of different acts provided the necessary authority to exporpriate for specific purposes, such as roads, public buildings, and hydropower projects. Today, many of these authorities are found in the \cite{ea59}.\footnote{Act no 3 of 23 October 1959 Relating to Expropriation of Real Property.} Still, many specific authorities remain, such as the aforementioned section 16 of the \cite{wra17}, which gives an automatic right to expropriate to the holder of a watercourse regulation license.

However, following the \cite{wra00}, the general authority used to exporpriate waterfalls has been included in the \cite{ea59}, in section 2. This section gives general conditions for when expropriation is authorised, and item number 51 states that ``hydropower production'' is a legitimate purpose. In addition, it is made clear that the authorising authority is the King in Council. However, the Act also makes clear that this authority can be delegated further, to ministries, or, in special circumstances, also to other state bodies that the King in Council may instruct.\footnote{See \cite[5]{ea59}.} 

Furthermore, section 2 makes clear that compensation is to be paid as determined by an appraisement court.\footnote{For further details of this procedure, I refer to Chapter \ref{chap:5})}. Lastly, the section states that a license to expropriate should only be granted when it is determined that the benefit of doing so ``undoubtedly'' outweighs the harm.\footnote{See \cite[2]{ea59}.} Notice that this sets expropriation licenses apart from the various hydropower licenses discussed in Sections \ref{wra00}-\ref{ea90} of Chapter \ref{chap:3}. In relation to the latter, in particular, the benefit is required to outweigh the harm, but it need not be ascertained that this is so ``undoubtedly''.

In section 2, no mention is made of {\it who} may legitimately expropriate private property. However, in section 3, it is made clear that unless the Kind in Council decides otherwise, only state or municipality bodies may be granted this power. This is formulated as a limiting principle, but in effect it serves as a general authorisation for the executive to decide, without parliamentary involvement, what class of legal persons may be granted expropriation licenses pursuant to section 2. For many of the purposes mentioned there, directives have been issued that extend the class of possible beneficiaries to any legal person, including companies operating for profit. Such a directive has been issued, in particular, for the authority to expropriate in favour of hydropower production.\footnote{See Directive no 391 of 06 April 2001.}

In addition to providing a general authority for expropriation, the \cite{ea59} also contains several procedural rules. These are collected in Chapter 3 of the Act. Section 11 gives minimal requirements for what an application for an expropriation license must include, stating that it should make clear who will be affected, how the property is to be used, and what the purpose of acquisition is. In addition, the section requires the applicant to specify what land will actually be acquired, and to include information about the type of land in question and the current use that is made of it.

An obligation to give notice to affected owners follows from the second paragraph of section 12. The starting point is that every owner is to be given individual notice, although this obligation falls away when it is ` unreasonable difficult'' to fulfill. In this is found to be the case, it is sufficient that the documents of the case are made available at a suitable place in the local area. A public announcement must also be made and included in ..., as well as in two widely read local newspapers.

The first paragraph of section 12 sets out a general obligation on part of the licensing authority to ensure that the facts of the case are clarified to the ``greatest extent possible''.\footnote{The Norwegian expression is ``best råd er'', which literally means ``best possible way''.} This formulation seems to establish a principle that is stronger than the general duty to duly assess cases involving rights and responsibilities of individuals, a duty that follows from general administrative law. However, the exact meaning of the phrase  ``greatest possible extent'' can be hard to pin down. In fact, administrative practice from several fields, including the hydropower sector, suggests that special attention is rarely devoted to the expropriation issue, particularly not when the expropriation license is granted to implement a state-sanctioned plan for the use of the land in question. I return to this issue in more depth in Section \ref{sec:paa67} when I discuss expropriation in light of administrative law, and in Section \ref{sec:ola} when I relate the discussion specifically to the case of waterfall expropriation.

The last rule of section 12, expressed in the third paragraph, states that a decision to grant an expropriation license must be accompanied by reasons that are submitted to the parties, in accordance with general administrative law. This rule is largely superfluous, as the obligation to give reasons would in most cases also follow independently from administrative law, c.f., Section \ref{sec:paa67}.

According to section 15, the costs incurred by owners in relation to a pending application for expropriation against them is to be covered by the applicant. The exact formulation is that the applicant is obliged to cover the costs that ``the rules in this chapter carry with them''. That is, the applicant is obliged to cover the costs that are related to the owners' rights pursuant to Chapter 3 of the \cite{ea59}. What this actually means is unclear, and in practice an applicant will be denied costs if the competent authority takes the view that they are unreasonable or disproportionate to his interests in the case.\footnote{If the case progresses to an appraisment dispute, the competent authority to decide on costs is the appraisement court. Otherwise, the decision is left with the executive.}

Particularly problematic are cases for which there is no clear division between those aspects of the case that relate to expropriation and those that relate to other licenses or land use planning more generally. This is the situation, for instance, in relation to hydropower development. In such cases, it is unusual for local owners to get any significant coverage of costs relating to the application processing. Legal expenses, for instance, are rarely covered unless they are incurred in relation to a subsequent appraisement dispute. This can be a problem for owners that wish to resist expropriation. Obviously, it is crucial for them to voice convincing objections already at the application processing stage.

In addition to the procedural rules in the \cite{ea59}, many rules of administrative law apply in expropriation cases. In the next section, I give an overview of the most important ones.

\subsection{The Public Administration Act}\label{sec:paa67}

After WW2, public administration in Norway underwent a reform whereby administrative bodies came to be placed more directly under centralized political control. At the same time, the established system based on legal expertise and strict adherence to the letter of the law was replaced by a form of management that actively sought to pursue political goals. As a result, the ambit of administrative decision-making power widened significantly. Many new administrative bodies were set up, and many were empowered greatly by statutory authorities that only specified their purpose and competence in broad strokes. The new style of legislation that developed often left great room for the exercise of administrative discretion, which, the argument went, was reasonable all the while the level of direct control exercised by the central government had been increased.\footnote{See generally \cite{grønli..}.}

However, these reform processes eventually led to concerns about the lack of formal safeguards for those individuals and groups of people that were targeted by administrative decisions. such safeguards would have been largely superfluous so long as the competence of administrative bodies was narrowly drawn up and expressly limited by the authorising statute. But as administrative bodies became increasingly sharpened as instruments of political decision-making, critical voices began to warn against the dangers of an unrestrained public administration free to implement political decisions without much scrutiny and public debate. Particularly worrying was the fact that administrative bodies were often empowered to implement such decisions directly against specific individuals, without having to bring out their general consequences, or justify them as matters of general policy.

In response to these worries, a general statute was proposed that would set out some minimum standards of due process for administrative decision-making process. This proposal eventually became the \cite{paa67}.\footnote{Act no 86 of 10 February 1967 Relating to Procedure in Cases Concerning the Public Administration.} This Act sets out the fundamental procedural principles that the executive is meant to follow when preparing to make administrative decisions. Some rules apply to any such decision, but a particularly important class of rules apply specifically to so-called {\it individual decisions}, which affect the rights and responsibilities of one or more specific persons.\footcite[2]{paa67} These persons are then referred to as parties to the decision. Clearly, owners of property covered by an expropriation license fall into this category, so that owners are parties to the individual decision to grant such a license.

Many of the rules in the \cite{paa67} mirror those of the \cite{ea59}, although they tend to include more detailed, albeit less strict, formulations. Section 16 stipulates that advance notice is required to all those affected by an individual decision. As was the case for expropriation, a possible exception is granted if it is practically unfeasible to reach the parties. However, section 16 also specifies in more depth what the notice must contain. In the second paragraph, it stated that ``the advance notification shall explain the nature of the case, and otherwise contain such information as is considered necessary to enable the party to protect his interests in a proper manner''. Hence, it is not enough simply to inform the party, the Act also explicitly stipulates that the notice has to meet a minimum quality standard. In relation to expropriation of waterfalls this takes on special significance, since, as I discussed in Chapter \ref{chap:3}, it is established practice in such cases fro the exporpriating party to send out this notice, with no involvement of the water authorities. 

In section 17, the duty to clarify cases is expressed, mirroring the rules in section 12 of the \cite{ea59}. The formulation is similarly imprecise, as it declare that cases are to be ``clarified as thoroughly as possible'' before a decision is made. Importantly, however, section 17 also includes specific rules that oblige the authorities to inform parties about information they retrieve during their assessment of the case, and to submit such information for comments in so far as the party must be assumed to have an interest in it.\footnote{See the 2 and 3 paragraphs of \cite[17]{paa67}.}

In section 24, the duty to give grounds for the decision is expressed. It applies to most individual decision, with some narrowly defined exceptions concerning cases when no party can be assumed to be dissatisfied, or when giving grounds would involve disclosing information to which the party is not entitled. Moreover, the King is authorised to limit the duty to give grounds when ``special circumstances so require''. All these exceptions are unusual, and hardly ever apply to hydropower cases.

In section 25, requirements are given concerning the content of grounds given for decisions. It is stipulated that the grounds should mention the relevant rules authorising the decision, the factual assessment the underlies it, as well as the main considerations that have been decisive for the use of discretionary power. 

In some cases, the complication of the matter at hand may be such that some parties are ill-equipped to look after their interests, even if the safeguards mentioned above are respected. This can be the case, for instance, in hydropower cases, as many waterfall owners must be expected to not possess the technical, commercial, and legal knowledge necessary to realize the meaning and value of their ownership. In section 11, a more recent amendment of the \cite{paa67}, a general rule of guidance is given, stipulating that the administration is obliged to provide guidance to parties so that they may look after their interests in the ``best possible way''. Again, the formulation is vague, and it is explicitly stated that the level of guidance must be adapted to the circumstances and the capacity that the agency has for offering such assistance. However, in the second paragraph it is stated that the agency must assess, on their own motion, the parties' need for guidance.

As demonstrated in this and the previous section, expropriation law and general administrative law imposes a range of procedural rules that must be followed when deciding on an application for a license to expropriate. In principle these apply also when waterfalls are expropriated, but as there are special rules that regulate the procedure followed in hydropower cases, the question becomes how these rules relate to each other. Also, the practical question is to what extent the water authorities interpret these rules in concrete cases, and whether they actually observe the more general rules regarding expropriation alongside the rules that target the licensing applications under water law. I return to this issue by giving an in-depth study of a concrete dispute in Section \ref{sec:ola}. First, I elaborate a little on the expropriation rules found in the law relating to hydropower, and its relationship, at the theoretical level, with the rules discussed above.

\section{Special rules for waterfalls}\label{sec:special}

As I mentioned in Chapter \ref{chap:3}, Section \ref{sec:wra17}, section 16 of the \cite{wra17} established an automatic right to expropriate rights needed to undertake a watercourse regulation. This is not understood to include a right to expropriate waterfalls needed for the hydropower development. However, in a recent decision, it was made clear that it does include a right to transfer water away from a river course for development somewhere else. This is a {\it de facto} license to expropriate a waterfall, as the water disappears form the river in which the owners have rights. It is also recognized as such in terms of compensation, which is paid for the waterfalls in cases like this, as they loose their value. However, it is still not considered as expropriation of the waterfalls themselves, but only of the right to deprive them of value.

Hence, section 16 \cite{wra17} entitles the license holder to a form of regulatory taking of the waterfalls of the owners in the river system where the water disappears. The status of such a takings after \cite{måland11} is in effect half-way between regulation and expropriation. The right to compensation is recognized, but the procedural and substantive rules that otherwise apply to expropriation of waterfalls do not apply. In particular, section 16 alone is sufficient authority for this kind of taking. The question arises about the extent to which the rules in the \cite{ea59} applies in such cases. 

First, this questions arises because the rules there are often regarded as expressing general principles of expropriation law. Second, it arises specifically in relation to section 30, number 2, which states that the rules apply to expropriation pursuant to the \cite{wra17}, in so far as they are ``suitable'' and do not ``contradict'' the special rules given in that Act. As we will see in Section \ref{sec:ola}, the established practice is to regard the procedural rules in the \cite{wra17} as exhaustive, and in keeping with the procedural rules in the \cite{ea59}. In addition, the strengthened assessment requirement in section 2, which stipulates that expropriation must``undoubtedly'' be of more benefit than harm, is not considered to have any independent significance alongside the assessment criterion of section 8 in the \cite{wra17}, which does not include any such formulation.

However, there is no doubt that the rules of the \cite{paa67} apply to takings of water rights pursuant to the \cite{wra17}. Moreover, there is no doubt that when a separate expropriation license is sought for waterfalls, these rules, as well as the rules in \cite{ea59} both apply. In practice, they nevertheless play a minimal role when the water authorities assess cases, as the assessment is unified, and the focus remains on balancing environmental and energy interest.

Hence, the broader question is how the practices adopted by the water authorities hold up against the requirements of administrative and expropriation law. This issue was not specifically addressed in \cite{måland11}, but the Supreme Court made some comments that can be taken to imply that they find no fault with current practices. I will shed more light on that they consist in over the course of the following sections. 

First, it is important to note that the reform of the energy sector means that expropriation of waterfalls takes place in a different context today than it did when many of the current practices developed. The value of precedent set during the time of the energy monopoly may be of limited value. In any event, it needs to be understood as a reflection of its time. This means that it is natural to structuring the presentation of expropriation practices chronologically, by dealing first with the period prior to the reform implemented by the \cite{ea90}. I now pursue this approach.

\section{Taking waterfalls for progress}\label{sec:twp}

Historically, Norwegian law admitted no general authority for the state to expropriate waterfalls, neither on its own behalf nor on behalf of private parties. However, there were a range of special provisions that authorized the state to appropriate water for specific purposes, but the criteria were typically quite narrow. For instance, the Water Resources Act 1887 authorized expropriation for the purpose of drinking water, but not for use in industry. Moreover, the purpose of expropriation was largely understood to be binding also on the future use, so that the taker would not gain unrestricted control over the rights he acquired, but were obliged to use them in accordance with the authorised purpose.

While there were no circumstances in which private parties could expropriate waterfalls for industrial development, but private owners of waterfalls could obtain licenses to expropriate surrounding land needed to exploit waterfalls that they already owned. In addition, a new right had been granted through the Water Resources Act 1887, giving the owners of waterfalls a right to engage in various industrial exploits, even if these would damage other landowners, for instance through deprivation of water or flooding. These rules are highly similar to many of the rules found in the so-called mill acts from the US, that I discussed in Chapter \ref{chap:2}, Section \ref{sec:mill}. Some of them could even theoretically have the effect of a {\it de facto} expropriation of waterfalls, but such cases were relatively rare.

An important reason for this was that expropriation law in general was based on the principle that eminent domain should not be exercised when the interests of the expropriating party were of the same kind as the interests of the owner. This applied regardless of whether or not the owner, subjectively speaking, were likely to pursue those interests in an optimal way. The principle was guiding for expropriation law until the early 20th century, and it applied regardless of whether the taker was public or private. It meant, for instance, that expropriation of waterfalls for the purpose of hydropower was ruled out already as a matter of principle. In particular, as the regulatory system of the day made private hydropower development possible, no owner could be deprived of rights to 
a waterfall by any hydropower developer, private or public. 

However, as the industrial advances meant that the interest in hydropower exploded in the late 19th century, the state increasingly came to see it as a political priority to secure that waterfalls were used in the public interest. The most important expression of this came in the form of the licensing acts presented in Chapter \ref{chap:3}, Sections \ref{sec:wra17} and \ref{sec:ica17}. However, during the same time, parliament passed legislation that authorised expropriation of waterfalls to the benefit of public bodies for the purpose of hydropower development.\footnote{Legislation that made it possible to expropriate waterfalls to the benefit of the municipalities was introduced in 1911, and a similar authority that authorised expropriation in favour of the state appeared in 1917, see \cite[29]{amundsen..}}

In 1940, these authorities were consolidated and integrated in the general water resources legislation, through the Water Systems Act 1940 (replaced by the \cite{wra00}). Still, the authority to expropriate waterfalls applied only to the state and to the municipalities, for the latter on the explicit condition that the purpose of expropriation was for ``general electricity supply in the district''. 

Hence, the required public purpose of expropriation was explicitly stipulated in the authority, and expropriation licenses could not be granted to private or commercial entities. Expropriation during this time therefore had a clear public character; in so far as the letter of the law was respected, little doubt could be raised that expropriation was indeed taking place in the public interest. Moreover, the public had to benefit directly, and locally. Economic development in itself was not regarded as a sufficiently public purpose to justify expropriation.

In addition, the fact that the energy sector was organized as a regional monopoly under direct political control meant that it was hard to contend, as a matter of fact, that expropriation was not an expression of the public will. At the same time, however, there were severe political conflicts over hydropower, conflicts that could then be meaningfully addressed within the framework of a politically managed electricity sector. Flaws in this system emerged, however, particularly as the state began to aggressively pursue hydropower development for economic development. 

This still took the form of a public undertaking, but as the scale of development grew massively following WW2, hydropower became increasingly politically sensitive. The democratic legitimacy of development with respect to the local communities that were affected also often seemed very weak. The decision-making authority was completely centralized, the benefit would tend to accrue in urban areas, but the negative effects were almost entirely contained in the local rural communities.

In practice, during this time the limitation to general electricity supply also became less important in practice. In particular, the interpretation of the supply requirement was relaxed significantly over the years, especially following the development of the national electricity grid and the liberalization of the energy sector in the early 1990s. It was no longer obvious, from a technical point of view, when exactly a hydropower development could be said to qualify as making a contribution to the local electricity supply. The electricity was not necessarily used locally, but, indirectly, also the local supply situation might be said to improve.

However, the rule that private parties could not expropriate waterfalls was enforced, and it remained in place until the executive passed the directive mentioned in Section \ref{sec:ea59}, in the year 2000. Only then, some 90 years after the introduction of a general expropriation right for the state and the municipalities, did it become possible for arbitrary commercial entities to acquire waterfalls compulsorily for hydropower.

In light of this, the vast majority of cases dealing with waterfall expropriation under Norwegian law can not be looked at as pure economic development takings. Certainly, the desire for economic development played a crucial part in motivating state and municipality development of hydropower. But their activities in this regard were not themselves commercial in nature. Rather, supplying electricity was regarded as a public service, one that would in turn stimulate commercial activity in other areas of the economy. However, the issue of the extent to which state could legitimately interfere with the rights of waterfall owners still arose. It was often contested, in particular, who the true beneficiaries were, particularly in relation to large-scale developments that would benefit communities far removed from those in which the water resources were found. In addition, particularly in the early 20th century, there was a general feeling of unease about how far the state could go in regulating and monopolizing the hydropower sector without thereby depriving the owners of waterfalls of constitutionally protected rights.

This debate culminated in the conflict surrounding the rule of reversion that was introduced by the licensing acts passed between 1906 and 1917. As mentioned, the rule of reversion meant that in order to sell a waterfall to a private development, the owners and the purchaser had to apply for a license that was only ever granted on the condition that after some number of years, at most 60, the state would acquire the waterfalls without paying compensation. The question that arose was whether this was merely a regulation of the permitted use of waterfalls, or whether it should be regarded as expropriation, so that compensation would be payable pursuant to section 105 of the Constitution.

The conflict over this issue became fierce, with many influential conservatives, including legal scholars, attacking the rule of reversion as a ploy by the state to confiscate Norwegian waterfalls without paying compensation to the owners. However, in a 4-3 decision, the Supreme Court eventually held that section 105 did not apply, since reversion was merely a licensing condition, not an act of expropriation. No owner was compelled to hand over his property to the state, or to sell it to a private party so that the state would eventually acquire it.

One of the judges voting with the majority summed up their view by commenting that he would not regard it as expropriation if the state were to forbid sale of waterfalls to private parties altogether. Why then, he asked, should it be regarded as expropriation if such a sale was allowed to take place only on specific conditions? Against this, the minority argued that the licensing requirement as such was so severe that it had to be regarded as a {\it de facto} expropriation that entitled the owners to compensation. Moreover,  as the purpose was clearly to ensure that waterfalls were eventually brought under state ownership, the minority did not think is was appropriate to consider reversion merely as a regulation of use.

After the decision by the Supreme Court in the reversion case, the legal foundation for the hydropower monopoly solidified. The development of this monopoly happened gradually, however, and expropriation on a large scale did not take place until after WW2. At this time, the state became to involve itself greater in hydropower projects, and it typically pursued very large-scale projects. This caused a new period of controversy, mainly motivated by envirnomental concerns. However, the interest of local people also featured strongly in this debate. Moreover, as the regulatory system was beyond reproach at this point, the local interest were typically aligned with the environmental interests. Large-scale hydropower projects, in particular, would tend to cause nuisance, or even significant loss, to traditional forms of agriculture. Therefore, in a situation when local owners could not themselves benefit significantly from hydropower, their rational response was to oppose it.

The patterns of conflict that emerged during this time converged in the case of \cite{alta8.}. In this case, there was an added complication: The local people all lacked formal title. This was because the development that was being planned would take place in the northernmost part of Norway, in the native land of the Sami people. Norway has a history of discrimination against the Sami, and as their culture is largely nomadic, their land rights were never formalized in the law. As a result, almost the entire northern region of Finnmark is owned by the state. Despite lacking title to the land, the Sami have continued to struggle for their rights to use the land, particularly for their nomadic form of reindeer farming, with an extensive additional reliance on fishing and hunting.

The plans to develop large-scale hydropower in a Sami area therefore raised particularly strong criticism. Particularly significant was the fact that the opposition to the plans brought together environmental groups and groups fighting for aboriginal rights. A broad political mass movement was mobilized in opposition to the plans, eventually resulting in several serious cases of civil disobedience, including what might today well be classified as ``terrorism''.\footnote{In particular, there were several instances when local people blew up equipment that was meant to be used to construct the hydropower plant. In one famous episode, the person behind the bombing miscalculated, resulting in the loss of his own arm. Apart from this, however, the protests were relatively peaceful.}
The case was also dealt with by the court, as the Sami interests claimed, primarily on the basis of administrative law, that the development licenses that had been granted for the development were invalid. 

At first sight, the case is not particularly relevant to the question of expropriation. However, as the Norwegian regulatory system focuses on the development issue, with little or no separate attention paid to the issue of expropriation, the case has in fact been highly significant to the owners of waterfalls. It effectively serves as the primary measuring stick with which the executive and the courts assess the level of substantive and procedural protection that local people are entitled to.

Due to the controversy surrounding the case, it was admitted directly from the District Court to the Supreme Court in plenum. The presiding judge commented that as far as he knew it was the longest and most extensive civil case that the Court had ever heard.\footcite[254]{alta82} In an opinion totaling138 pages, the Court argues that the decision to grant the license is valid. The opinion deals mostly with procedural rules. The substantive arguments, and arguments relating to international law, were not subjected to much scrutiny, as the Court express strong confidence in the opinion that no objection could be raised against the licenses on such grounds. 

However, in addition to arguing against the decision on this basis, the opponents of the development had pointed out a very wide range of purported shortcomings of the decision-making process. This aspect of the case was considered in great depth by the courts, leading also to a further elucidation of the procedural rules of the \cite{wra17} and the application of general administrative law to hydropower cases.

It was clear that the original licensing application did not meet the requirements stipulated in section 5 of the \cite{wra17}. Essentially, the original application contained little more than the technical details about the planned development, with little or no identification or assessment of deleterious effects on other interests, neither private nor public. This shortcoming, moreover, had been acknowledge by the water authorities themselves, who had nevertheless initiated a public hearing, citing an electricity deficit in the northern part of Norway. 

The Supreme Court concluded that this was ``clearly unfortunate''.\footcite[265]{alta82} However, several reports and assessments had subsequently been provided by the water authorities, to fill the gaps left open by the initial application. The Supreme Court held that this might well serve to make the initial mistakes irrelevant to the validity of the licenses, as it was the licensing process as a whole that should be assessed against the procedural rules. Hence, shortcomings of specific stages in the assessment would not be given weight it they did not imbue the process with a dubious character overall.\footcite[265]{alta82}

The question then turned to the question of whether the process as a whole fulfilled procedural requirements. This turned largely on 
the extent to which the various assessments that had been made in the case adequately served to clarify the case, in accordance with section .... of the \cite{wra17} and section 16 of the \cite{paa67}. 

In this regard, the local people objecting the development pointed to a range of negative effects that they believed had not been considered, or had not been considered in enough depth. In relation to nomadic reindeer interests, for instance, it was argued that the water authorities had failed to adequately consider the indirect consequences of development. These effects were described as ``catastrophic'' by an expert testimony presented to the Court. By contrast, the water authorities had based their decision on assessments that did not place much weight on indirect consequences, citing the difficult involved in attempting to quantify such effects. 

After considering the reports and assessments in some depth, the Supreme Court did not find fault with the procedure in this regard. Importantly, the Court stresses that the water authorities were aware of the possibility of indirect negative consequences, but simply chose, as a matter of expert discretion, not to place much weight on such consequences. This, moreover, was construed as an expression of disagreement with those claiming (later) that the effects would be catastrophic. As a result, the grounds for claiming procedural error disappeared, as the lack of attention directed at indirect consequences was held to reflect administrative discretion that could not be made subject to judicial review.

More generally, the Court's opinion on this point reflects how indeterminate the distinction between administrative discretion and procedure can become. Importantly, the Court makes statements of principle in this regard, that serve to limit the scope of judicial review under procedural rules in hydropower cases. In particular, the Court concludes that many of the relevant procedural rules in such cases by their very nature tend to be largely ``discretionary' '. As the licensing decision itself is a discretionary one, the argument goes, it is appropriate to admit to the executive a wide discretionary authority to decide for themselves also how to interpret many of the admittedly rather vague procedural requirements of administrative law. By contrast, the view taken by the appellants, based on the idea that the content and scope of such rules is a purely judicial question, is described by the Court as ``overly formalistic''.

The Court makes a second statement of principle, namely that the scope of assessment required for the purposes of reaching a licensing decision is not in any event as extensive as the level of assessment that is required in a subsequent appraisement dispute. 
Hence, the meaning of the obligation to clarify cases to the best possible extent is put into perspective: Assessments of deleterious effects may be omitted at the executive's discretion even in circumstances when such assessments are practically relevant to the level of compensation payable and {\it will} in fact be provided at a later stage.

More concretely, in relation to the negative effects on fishing, the {\it Alta} Court conceded that the assessments could have been better, but pointed out that the purpose of assessment was only to answer yes or no to development, not to give a detailed presentation of its effects.\footcite[330]{alta82}. Crucially, the Court goes on to note that in so far as mistakes are uncovered as a result of insufficient assessment, this will influence the compensation payments and can also motivate subsequent regulation.\footcite[330]{alta82} In effect, the risk of error is downplayed by making reference to the purely fiscal interest of owners and the regulatory authority of the state. This echoes the dichotomy mentioned in Chapter \ref{chap:3}, whereby there is a tendency in Norwegian law to perceive the interests of affected citizens in purely fiscal terms, while the state is assumed to be the sole protector of social and environmental values attached to property.

In relation to some negative effects of the {\it Alta} development, it was made apparent that they had not been considered at all. In addition, it was clear that erroneous information had been assembled in relation to some issues, particularly regarding alternative ways to meet the need for electricity in Finnmark and Norway as a whole, as well as the extent of this need. The Supreme Court agreed that this was a flaw, but held that it did not imply invalidity of the license. In this regard, a third statement of principle was made. It was stated in particular,  that the duty to consider alternative ways of achieving the public purpose underlying the development license was very limited. 

Some alternatives should be mentioned, but they did not have to be made subject to assessment. This, in turn, was used by the Court to argue that the errors in the information provided about alternative were unlikely to have affected the outcome of the case.\footcite[346]{alta82} This was so despite the fact that alternatives {\it had} in fact been assessed in some depth. Moreover, erroneous information about alternatives had been handed over to the Storting, who had approved the plans on three separate occasions, but always under reference to the precarious electricity situation in Finnmark.

In effect, the Court established a principle whereby the nature of possible alternatives is considered a marginal issue in relation to assessment of licensing applications. Even errors in the data provided about this issue will normally not be given much weight. In {\it Alta}, however, this seems to have been at odds with how the Storting approached the case. There is little doubt, in particular, that the eventual political assessment of the plans depended heavily on the perceived electricity crisis in Finnmark, as well as the electricity supply situation in Norway more generally and the perceived inadequacies of alternative solutions.

In relation to the supply situation in Norway, the state's council in {\it Alta} suggests explicitly that as these aspects were considered relevant mainly at the political stage of the decision-making, they were largely irrelevant to the legal issues that had been raised.\footcite[341]{alta82} But this is hardly reassuring, particularly not as the decision to grant the license was very much a political one. The information gathered by the water authorities, therefore, would be put to use in a largely political decision-making context. In light of this, it seems that the procedural rules were, if anything, {\it more} important to observe in so far as they pertained to the quality of the factual basis that would be used in subsequent political assessments.

The Supreme Court clearly did not approach the matter from this angle, but how exactly it reasoned in this regard is not clear from the opinion. In fact, it is very noticeable how briefly the Court comments on this compared to other aspects of the case. On the one hand, it goes into great detail about purported weaknesses of the licensing procedure that seem relatively minor comparatively speaking. As a result, the Court's assessment that these aspects are not in any event relevant to the outcome of the case do not come as a surprise. On the other hand, in relation to the duty to assess alternatives and the level of necessity, the Court says nothing expect that the duty is very limited. For the details, which demonstrate factual inadequacies in the basis provided to the political decision-makers, the Court only refers briefly to the arguments presented by council for the state. These arguments, based on the contention that the inadequacies were not significant, is accepted with no further discussion.\footcite[346]{alta82}

The dismissive attitude towards the duty to correctly assess alternatives is no doubt a controversial aspect of the {\it Alta}-decision, and it has also later been criticized by legal scholars. Today, the principle becomes particularly problematic. Alternatives are no longer limited to other public projects that can potentially provide the same public service at a smaller environmental and social cost. Instead, the most important alternatives now typically consist in owner-led projects proposed in commercial competition to the applicant's commercial project. In so far as the duty to assess these alternatives is construed as loosely as the duty to assess alternatives was construed in {\it Alta}, it will hardly be reassuring for those owners of waterfalls that oppose commercial development projects based on their own hydropower plans. 

In the next two sections, we will see that so far, no adjustments have been made to the way the water authorities approach this issue. Moreover, the dismissive attitude to this question in {\it Alta} has been upheld in a recent Supreme Court decision involving a concrete owner-led alternative regarding which the NVE had provided manifestly erroneous information to the Ministry.

\section{Taking waterfalls for profit}

As I mentioned in the previous section, private companies could not expropriate waterfalls in Norway prior to the passage of the \cite{wra00}. Moreover, the public purpose requirement was enforced strictly by the authorising statute, particularly in cases when the development was undertaken by municipality companies. I also mentioned how the hydropower sector developed after WW2 from a sector dominated by local municipality companies, to a sector dominated by the state. This, in turn, was accompanied by increased conflicts and doubts regarding the legitimacy of the established licensing procedures, particularly the highly centralized nature of the decision-making process. 

Even so, the debate at this time was still very much anchored in a system that presupposed political management of the hydropower sector as a public service provider. Importantly, the conflicts rarely, if ever, involved significant commercial interests on the part of the local waterfall owners. Many critics voiced arguments to the effect that the fiscal interest of the state motivated wanton destruction of nature and local patterns of land use, including commercial uses. But in financial terms, these interest were typically negligible compared to the scale of the hydropower development. 

As a result, controversies relating to the legitimacy of interference involved only the waterfall rights at their periphery. More focused conflicts involving waterfalls specifically arose in relation to the question of compensation, but the issues typically discussed in this regard were also of relatively minor structural importance, although they could of course be important enough for the individuals directly affected.

In Chapter \ref{chap:3}, I presented the reform of the energy sector of the early 1990s, after which hydropower development has been regarded as a commercial pursuit. Following the regulatory reform, a new general statute dealing with water resources was also proposed, eventually leading to the passage of the \cite{wra00}. This Act provided the first every authority for the state to allow developers to take waterfalls compulsorily for profit. Moreover, it made possible the later executive directive by which waterfalls could be expropriated and handed over to {\it any} legal person, including private companies.

The combination of the legal and regulatory reforms mean that today, takings of waterfalls for hydropower are takings for profit. But this change in the function of expropriation received little attention when these reforms were introduced. When the \cite{wra00} was proposed, the increased scope of expropriation was not singled out for political consideration by the MoPE when it handed the case over to the Storting. In the legislative proposal handed over to the Storting, the new expropriation authority for waterfalls is described merely as a ``simplification'' of older law. 

As the discussion above shows, this is a hardly accurate. However, the commission appointed by the Ministry to prepare the Act also adopted a very low-key approach to expropriation. The commission mentioned that its proposals would imply increased scope for expropriation, but it did  not discussed the desirability of this in any depth. In particular, it did not related its proposals in this regard to the recent market -based reform of the energy sector.  The report from the commission, totaling almost 500 pages, devote only three pages to the proposed ``simplification'' of the expropriation authority.\footcite[235-237]{nou94}

First, the commission notes that a range of different authorities for expropriation co-exist in the law, with many of them positing strict and concrete public interest requirements as a precondition for granting a license. This, the commission argues, is not a very ``pedagogical'' way of providing expropriation authorities. Moreover, the commission notes that it runs the risk of omitting important purposes for which expropriation should be possible. Hence, the commission proposes to replace all older authorities by a sweeping authority that makes expropriation possible for any project that involves ``measures in water courses''.  

The commission comments that their formulation might seem wide, but remark that this is not a problem since the executive can simply deny giving an expropriation license in so far as they regard expropriation as undesirable. The commission does not reflect on the  constitutional consequences of such a perspective, neither in relation to property rights nor in relation to the balance of power between the legislature and the executive. The commission does offer a very brief presentation of the rationale behind dropping the local supply restriction for municipal expropriation, by remarking that these rules complicate the law and might make desirable expropriations impossible.\footcite[235]{nou94}  But the commission do not clarify what kind of desirable expropriations it thinks might be left out. In particular, it does not relate its proposals to the recent market-based reform of the energy sector. Hence, the obvious practical consequences of their proposal, namely that expropriation will be made available as a profit-making mechanism for commercial companies, is not discussed or critically assessed.

The issue of {\it who} should be permitted to benefit from an expropriation license is also dealt with very superficially. In this regard, the commission structures their presentation around the so-called redemption rule that was introduced in \cite{wra40}. Recall that this rule made it possible for a majority owner of a waterfall to compulsorily acquire minority rights, if this was necessary to facilitate hydropower development. Hence, it was a rule that provided only a limited opportunity for private takings, restricted to local owners themselves or external developers that had been able to strike a deal with a locally based majority. 

The main justification given by the commission for introducing a general private takings authority is that the special redemption rule had not been much used in practice. Why this is an argument in favour of extending private expropriation rights is not made clear. Indeed, it seems just as natural to regard it as an argument {\it against} doing so. Why extend the possibility for private expropriation if the demand for such expropriation has been limited in past? Presumably, the commission thought there would be a demand for such expropriation in the future, but this is not stated explicitly, nor is the appropriateness of meeting such a demand discussed. As to the requirement that private takers must already control a majority of the waterfall rights in the local area, the commission only remarks that it regards such a restriction as old-fashioned.\footcite[236]{nou94} No discussion is offered regarding the consequences for local waterfall owners at a time when the energy sector was also being reformed according to market principles.

Since the passage of the \cite{wra00}, it has become clear that the new authority for expropriation has been one of the most practically significant, and controversial, aspects of the Act. During the last 14 years, an unprecedented number of cases has raised the issue of legitimacy of expropriation of waterfalls. Today, practically all cases of expropriation imply that local owners are deprived of the development potential in favour of a commercial actor seeking development of the same kind. According to the law before the \cite{wra00}, expropriation of this kind would not be easy to justify against the relevant authorities. In so far as the beneficiary was a private company, it would not be possible at all. 

In \cite{sauda08}, this issue came into focus, as the waterfall owners protested a license that granted a private company the right to expropriate their waterfalls. Here, the owner's principal argument was that the executive directive granting such rights to private parties was in fact invalid, since it had not been sanctioned by the Storting. Formally speaking, this argument was very weak, since the \cite{ea59} had been amended in such a way that the executive was in fact authorized to determine the class of legal persons that could be granted an expropriation license to pursue hydropower. However, the owners argued that the executive had not appropriately informed the Storting that this would be the consequence of the amendment, which had been passed as a formality following the adoption of the \cite{wra00}. 

The owners pointed to interviews with two of the members of the parliamentary committee that had prepared the case for the Storting, noting that neither of them could recollect that they were even aware that a right to expropriate for private developers would result from the Act they had passed. This, as noted earlier, was not conveyed to them by the executive. Moreover, it was not explicitly stated anywhere in the Act that had been passed. Rather, it followed implicitly from three different sections in two separate Acts that the executive would be empowered to issue such a directive. This, the owners argued, meant that the purported authority was not in fact constitutionally valid.

This argument was rejected, but the level of compensation paid for the waterfall rights was dramatically increased compared to earlier practice, c.f., Chapter \ ref{chap:5}. Because of this, the development company appealed the decision to the Supreme Court, with the owners lodging a counter-appeal regarding the question of legitimacy. The Supreme Court decided not to hear the case, however, as it had recently addressed the compensation question from a similar angle in the paradigmatic case of \cite{møllen08}.

In addition to raising the issue of constitutional legitimacy of the new expropriation authority, the owners in \cite{sauda08} also raised several procedural objections against the expropriation license. This line of argument also proved unsuccessful, but it foreshadows the later case of \cite{måland13}, where the owners were initially successful in arguing that the procedures developed in the takings for progress era were no longer appropriate. This decision was overturned on appeal, however, a decision that was in turn upheld by the Supreme Court, in a decision relying on the precedent set by \cite{alta82}. This case is very well suited to bringing out how administrative practices relating to expropriation function in the context of commercial development projects were the owners have competing plans. It also illustrate common grievances raised by local owners, as well as serving to highlight the response to these grievances by the water authorities and the judiciary.

\section{Ola Måland v Jørpeland Kraft AS}

The expropriating party in \cite{måland13} is Jørpeland Kraft AS, a company jointly owned by Scana Steel Stavanger AS, who owns 1/3 of the shares, and Lyse Kraft AS, who is the majority shareholder holding the remaining shares. The former is a steelworks company located in the small town of Jørpeland in Rogaland county, southwestern Norway. Historically, this company has been a major employer in Jørpeland, which is located by the sea, next to a mountainous area. The main source of energy for the steel industry in Norway has been hydro-power, and Scana Steel Stavanger AS is no exception. The company uses energy harnessed from the rivers in the area, and while the primary river runs through the town of Jørpeland itself, it is supplemented by water from other rivers in the area that are diverted so that they can be exploited more efficiently along with the water from the Jørpeland river.

Recently, Norwegian steel companies have become less profitable, due in great part to increased foreign competition and a significant increase in cost of operation associated with this type of industry in Norway, particularly salary costs.\footnote{For a reference on this, see \emph{Information Booklet about Norwegian Trade and Industry}, published by the Ministry of Trade and Industry in 2005.} This has led to many such companies shifting their attention away from labor-intensive steel production, and focusing instead on producing electricity, selling it directly on the national grid. Jørpeland Kraft AS was established as part of such a move being made with regards to the energy resources in Jørpeland, and the role played by Lyse Kraft AS is an important one. As we mentioned, Norwegian law favors companies where the majority of the shares are held by public bodies, and Lyse Kraft AS, being publicly owned, with the city of Stavanger as the main shareholder, is therefore a valuable partner. Moreover, Lyse Kraft AS, while being a commercial company, is also responsible for the electricity grid in the region. It was established as a merger between several local monopoly companies in the Stavanger region which were reorganized following liberalizaion of the sector in the early 1990's. As discussed in Section \ref{context}, there is little doubt that old monopolists still enjoy considerable power and influence.\footnote{In fact, Lyse Kraft AS is good example suggesting that their power might in some cases have \emph{increased}. Since liberalization, the restraints imposed both by the non-commercial nature of former monopolists, and the local, political, anchoring of such companies, have disappeared.} This is another reason why they can serve as valuable partners for private companies wishing to make a profit from Norwegian hydro-power.

With attention shifting from harnessing rivers for the purpose of industrial production to the purpose of producing electricity to sell on the national grid, the main variables that determines the profitability of the undertaking also changes. On the cost side, what matters becomes only the cost of producing the electricity itself, and this is typically determined, for the most part, by the investments required for the original construction works.\footnote{For an overview of the considerations made when assessing the commercial value of small scale hydro-power, we point to \cite{kartlegging}. In fact, due to the importance that small scale hydro-power has assumed in recent years, investigating models for investing in such projects has become an active field of research in Norway, see for instance \cite{investment}.} Running and maintaining a hydro-power station tends to be comparatively inexpensive. On the income side, what matters is the price of energy on the electricity market, a market that is no longer anchored in the local conditions of supply and demand.

Importantly, as long as energy production is the sole focus, the business no longer depends in any significant way on the local labor force, and as a result, it is typical that large scale exploitation becomes much more profitable, compared to the medium or small scale power plants typically needed to facilitate local industrial exploits. Hence, it was in keeping with a general trend in Norway when Jørpeland Kraft AS, following their shift in commercial strategy, proposed to undertake measures to increase their energy output. This could be achieved relatively cheaply, by further constructions aimed at channeling water from nearby waterfalls into dams that were already built to collect the water from the Jørpeland river.

One relatively small waterfall from which Jørpeland Kraft AS suggested to extract water was owned by Ola Måland and five other local farmers. This waterfall is not located in Jørpeland kommune, and does not reach the sea at Jørpeland, but runs through the neighboring municipality of Hjelmeland, on the other side of a mountain range, until it eventually reaches the sea at Tau, another neighboring municipality. The plans to divert this water would deprive original owners of water along some 15 km of riverbed, all the way from the mountains on the border between Hjelmeland and Jørpeland, to the sea at Tau. Far from all the water would be removed, but the water-flow would be greatly reduced in the upper part of the river known as "Sagåna", the rights to which is held jointly by Ola Måland and five other local farmers from Hjelmeland. 

The water in question stems from the \emph{Brokavatn}, located 646 meters above sea level, where altitude soon drops rapidly so that hydro-power is a particularly well-suited form of exploitation for this water. Plans were already in place for making such use of it, from about the altitude of Brokavatn, to the valley in which the original owners' farms are located, at about 80 meters above sea level. In fact, a rough estimate of the potential was originally made by the NVE, and estimated to yield gross annual production of 7.49 GWh per annum, about five times more than the water from Brokavatn would contribute to the project proposed by Jørpeland Kraft AS. This estimate was not made in relation to the case, but as part of a national project to survey the remaining energy potential in Norwegian rivers.\footnote{The survey was carried out in 2004, and its results are summarized in \cite{kartlegging}.} %\noo{More recent calculations, made by several different experts, acting both on behalf of Jørpeland Kraft AS and original owners, suggests that the water which would be lost would in fact be crucial to the commercial potential of hydro-power for the original owners. Having the water available would take such a project from being somewhat marginal to being a highly profitable endeavor. The owners were not aware of this at the time when the case was being prepared by the water authorities, nor where they informed of this as part of the process.} 

Despite holding the relevant property rights, and despite having considerable commercial interests that would be effected, original owners were not identified as significant stakeholders in the project. Rather, the approach to the case was the traditional one, with focus being directed at the environmental impact, with relevant interests groups being called upon to comment on consequences in this regard, and quite some public debate arising with respect to the balancing of commercial interests and the desire to preserve wildlife and nature.

Nevertheless, one of the owners, Arne Ritland, commented on the proposed project, in an informal letter sent directly to Scana Steel Stavanger AS. In this letter he inquired for further information, and he protested the transferral of water from Brokavatn. He also mentioned the possibility that an alternative hydro-power project could be undertaken by original owners, but he did not go into any details regarding this, stating only that such a locally owned hydro-power plant had previously been in operation in the area. The plant he was referring to dates back to the time before we had a national grid, and was only directed at local supply of electricity. It has since been shut down.

Arne Ritland received a reply stating that more information on the project and its consequences would soon be provided, and he did not pursue the matter further at this time. Meanwhile, Scana Steel Stavanger AS submitted his letter to the water authorities, who in turn presented it to the NVE as a formal comment directed at the application. This prompted Jørpeland Kraft AS to undertake their own survey of alternative hydro-power in Sagåna, and the conclusion, but not the report itself, was sent to the water authorities. The original owners were not informed, and they were not asked to comment on it, or even told that such an investigation of the commercial potential in their waterfalls was being conducted by the expropriating party, as a response to Ritland's letter.

Despite being presented with the issue, the water authorities did not take steps to investigate the commercial potential of local hydro power on their own accord. Moreover, the conclusion presented by Jørpeland Kraft AS did not go into details, but merely stated that if the local owners decided to build two hydro-power plants in Sagåna, then one of them, in the upper part of the river, close to Brokavatn, would not be profitable, neither with nor without the water in question. The other project, on the other hand, in the lower part, could still be carried out profitably even after the transferral. No mention was made as to what the original owners actually stood to loose, nor was there any argument given as to why it made sense to build two separate small-scale power plants in Sagåna. In their final report, the NVE handed these findings over to the Ministry, but did not inform the original owners. 

In addition to the report made by Jørpeland Kraft AS themselves, Hjelmeland kommune, the local municipality government, also commented on the possibility of local hydro-power. In their statement to the NVE, they directed attention to the data in the NVE's own national survey, which suggested that a single hydro-power plant in Sagåna would be a highly profitable undertaking. On this basis, they protested the transferral, arguing that original owners should be given the possibility of undertaking such a project. This statement was not communicated to the original owners, and in their final report it was dismissed by the NVE, who stated that the most energy efficient use of the water would be to transfer it and harness it at Jørpeland.

In addition to the statement made by Ritland, one other property owner, Ola Måland, commented on transferral. He did so without having any knowledge of the commercial potential the water held for him and his co-owners, and without having been informed of the statement made by Hjelmeland Kommune. On this basis, he expressed his support for the transferral, citing that the risk of flooding in Sagåna would be reduced. He also phrased his letter in such a way as to suggest he was speaking on behalf of other owners, but he was the only person to sign it. In the final report to the Ministry, the NVE, in their own conclusion, use this as an argument in favor of transferral, stating that the original owners were in favor of it, and that the opinion of Hjelmeland Kommune should therefore not be given any weight. They neglect to mention Arne Ritland's statement in this regard, and earlier in the report, where his statement is referred to along with many others, Ritland is referred to as a private individual, while Ola Måland is referred to as a property owner, and taken to speak on behalf of the others. The report made by the NVE, while it was not communicated to the affected local owners, it was sent to many other stakeholders, including Hjelmeland Kommune. In light of NVE's conclusions, they changed their original position, informing the Ministry that they would not press any further for local hydro-power, since this was not what the original owners wanted themselves. 

While the case was being prepared by the water authorities, the original owners had begun to consider the potential for hydro-power on their own accord, and in late 2006, when the case reached the Ministry, they where not aware that a decision was imminent. Rather, they were under the impression that they would receive further information before the case went further. Still, as they came to realize the commercial value of the water from Brokavatn in their own project, they approached the NVE, inquiring about the status of the plans proposed by Jørpeland Kraft AS. They were subsequently informed that an opinion in support of transferral had already been offered to the Ministry, and that a final decision would soon be made. This communication took place in late November 2006, summarized in minutes from meetings between local owners, dated 21 and 29 of November. On 15 of December 2006, the King in Council granted a concession for Jørpeland Kraft AS to transfer the water from Brokavatn to Jørpeland.

At this point, it was becoming increasingly clear to the original owners that the water from Brokavatn would be crucial to the commercial potential of their own project, and they also retrieved expert opinions suggesting that the NVE was wrong in concluding that transferral would be the most efficient use of the water. In light of this, they decided to question the legality of the transferral, arguing that the decision was invalid.

The license given to Jørpeland Kraft AS was challenged by the original owners on the grounds that the expropriation was materially unjustified, and that the administrative process leading up to the permission to expropriate did not fulfill procedural requirements. The local court, Stavanger Tingrett, held that the original owners were right in protesting the transfer, with the court emphasizing that the preparatory steps taken in cases such as these needed to provide adequate guarantee that the authorities had also considered the fact that the waterfalls could have been exploited commercially by the original owners themselves.\footnote{Stavanger Tingrett 20.05.2009, case nr. 07-185495SKJ-STAV.}

This view was rejected by the regional court, Gulating Lagmannsrett, which held that sufficient steps had been taken to clarify the commercial interests of the owners, and, moreover, that established practice regarding the preparation and evaluation of such cases -- dating from a time when it was not feasible for original owners to undertake hydro-power schemes -- still provided adequate protection.\footnote{Gulating Lagmannsrett 10.01.2011, case nr. 09-138108ASD-GULA/AVD2.} The Supreme Court also held in favor of Jørpeland Kraft AS, and they went even further in stating that established practice was beyond reproach.

In the following section, we present the main legal arguments relied on by the parties, as well as a summary of how the three national courts approached the case, and how they argued for their respective decisions.

\subsection{The legal arguments, and the view taken by the national courts}\label{view}

The original owners had several arguments in support of their claim that the concession was invalid. Firstly, they argued that procedural mistakes had been made in preparing the case; secondly, they argued that according to Norwegian expropriation law, it was not permissible to expropriate in a situation such as this, when the loss of energy and commercial potential would outweigh the gain to those same interests, which, ostensibly, were the only interests identified in favor of transferral. It seemed to the original owners that expropriation in this case would only serve to benefit the commercial interests of Jørpeland Kraft AS, and that it would do so to the detriment of both local and public interests. For this reason, the owners held that the concession should be regarded as an abuse of power, a manifestly ill-founded decision which could not be upheld.\footnote{There are at least two different ways in which to argue such a point under Norwegian law. One is with respect to water law and general administrative law, whereby clearly ill-founded decisions can be overturned by the courts, even when they involve discretion on part of the executive, which is otherwise not subject to review by the courts. Secondly, an argument can be made with respect to the Norwegian Constitution, Section 105, which gives property a protected status. The former is usually more effective, but in both cases, quite a severe transgression will have to be established before courts consider it within their competence to overturn discretionary decisions. A scholarly examination of these two sets of provisions are given in \cite{Efvl} and \cite{flei} respectively (both in Norwegian).} The owners argued, moreover, that the government had not fulfilled its duty to consider the case with due care, and that the assessment made with respect to the interests of the local community at Hjelmeland, and the local owners residing there, was not adequate. Particular attention was directed at the fact that local owners had not been informed about the progress of the case, and had not been told of, or asked to comment on, those preparatory steps that were being made explicitly with regards to assessing their interests. 

In addition, owners also argued that irrespectively of how the matter stood with respect to national law, the expropriation was unlawful because it would be in breach of the provisions in the ECHR TP1-1 regarding the protection of property.\footnote{European Convention of Human Rights Article 1 of Protocol 1.}\noo{An argument was also made to the effect that expropriation would be in breach of provisions in the EEA agreement regarding unlawful state support for the commercial interests of specific companies.}

Jørpeland Kraft AS protested all these objections to the expropriation, arguing that it was the responsibility of the owners themselves to provide information about possible objections against the project, and that the process had therefore been in accordance with the law. Unfortunate misunderstandings, if any, should be attributed to the fact that original owners had neglected their responsibilities in this regard. Moreover, Jørpeland Kraft AS argued that it was not for the courts to subject the assessment of public and private interests to any further scrutiny, since this was a matter for the government to decide. 

Indeed, according to Norwegian national law, it is traditionally held that unless the exercise of power it clearly unjustified, the courts do not have the authority to overturn decisions based on discretion, unless it can be demonstrated that the government has made procedural mistakes. While this view has become somewhat more relaxed in recent years, with a standard of \emph{reasonableness} increasingly being imposed by courts in similar cases, the inadmissibility of court interference in administrative discretionary decisions is still very much a part of Norwegian national law.\footnote{See \cite{Efvl}, in particular, chapters 24 and 29.}

Finally, Jørpeland Kraft AS argued that there was no issue of human rights at stake in the case. While they argued for this by stating that as the procedural rules had been followed and that the material decision was beyond reproach, they also went far in suggesting that as the owners would be compensated financially by the courts for whatever loss they would incur, no human rights issues could possibly arise in the case. \noo{ They also rejected the view that the case could be seen as an instance of illegitimate state support for Jørpeland Kraft, but failed to provide specific arguments in this regard.}

The matter went before Stavanger Tingrett who gave their judgment on 20 May 2009. In the following, we offer a presentation of the reasons given by this court, leading to the conclusion that the expropriation was unlawful and that the transferral could not be carried out. 

Stavanger Tingrett agreed with the original owners that the decision to grant concession was based on an erroneous account of the relevant facts, and they concluded that it was evident, from the NVE's own figures, that allowing the applicants to use the water from Brokavatn in their own hydro-electric scheme would be the most efficient way of harnessing the potential for hydroelectric production, directly contradicting what the NVE stated in their report. Moreover, they noted that these were the same estimates as those referred to by  Hjelmeland Kommune in their initial objection, and found it to be in breach of procedural rules that this was not considered further by the authorities.

The Court substantiated their decision by giving direct quotes from the report made by the NVE. For instance, in the report, on p. 199, it says, as quoted by Stavanger Tingrett (my translation):
%\begin{quote}Hjelmeland kommune ser helst at kraftressursene i vassdraget blir utnyttet av lokale %grunneiere. 
%Dette står i kontrast til uttalelsen fra grunneierne selv som ønsker at overføring blir gjennomført, 
%slik at flom og erosjonsskader kan bli noe redusert. NVE mener at den beste utnyttelsen med tanke 
%på kraftproduksjon vil være å tillate overføringen da en slik løsning vil innebære at vannet utnittes i 
%størst fallhøyde. Når dette samtidig er grunneiernes eget ønske har vi ikke tillagt Hjelmeland 
%kommunes synspunkt på dette noen vekt
%\end{quote}
%Our own translation follows below: 
\begin{quote}
Hjelmeland kommune would like the hydro-electric potential in the waterfall to be exploited by 
local property owners. This stands in contrast to the statement given by the property owners 
themselves, who wish that the transfer of water takes place, so that damage due to flooding can be 
somewhat reduced. NVE thinks that the best use of the water with respect to hydro-electric 
production is to allow a transfer, since this means that the water can be exploited over the greatest
distance in elevation. When this is also the property owners' own wish, we will not attribute any 
weight to the views of Hjelmeland kommune.
\end{quote}

Stavanger Tingrett concluded that as this was a factually erroneous account of the situation, the decision made to allow transferral of the water could not be upheld. Summing up, the Court offered the following assessment of the case (my translation):

\begin{quote}
It is the opinion of the court, having considered how the case was prepared by the authorities, that the factual basis for the decision made by the government suffers from several significant mistakes and is also incomplete.
\end{quote}

In light of this, Stavanger Tingrett concluded that the decision to grant concession for transfer of water was invalid. As to the legal basis of this, the court relied on the recognized principle of Norwegian public law that while the exercise of discretionary powers is usually not subject to review by court, a decision based on factual mistakes is nevertheless invalid if it can be shown that the mistakes in question were such that they could have affected the outcome. This is not provided for explicitly in statue, but it is one of the core unwritten legal principles of Norwegian public law.\footnote{See \cite{Efvl}}

Concerning the second requirement, that the factual mistakes could have affected the outcome, Stavanger Tingerett found that it was clearly fulfilled in this case since, in fact, the hydro-power suggested by original owners was, based on data available to the government at the time of decision, an objectively speaking \emph{better} use of the resource, even with respect to public interest. In any event, the requirement with regards to factual and procedural mistakes is only that the mistakes \emph{could} have affected the outcome; in the presence of mistakes, the burden of proof is shifted over to the party seeking to defend the decision.

Since Stavanger Tingrett agreed with the original owners that the decision was invalid due to being based on incorrect facts, there was no need to consider further the claims regarding the legitimacy of the decision with respect to human rights law. Stavanger Tingrett did conclude, however, making a more overreaching assessment of the case, that the procedure followed in preparing the case had not taken sufficient regard of owners' interests, and that this was the likely cause of the mistakes that had been made. The Court also argued that the standard of protection for interest of original owners had to interpreted as being more strict now that local hydro-power was an option available to original owners. 

\noo{In this regard, t also seems that Stavanger Tingett found some additional support in its interpretation of Norwegian law that was based on human rights concerns, especially the fact that expropriation, in circumstances such as those of this case, appeared to be a major interference in the rights of owners, and that established practice developed under a different regulatory regime was therefore no longer able to provide adequate protection.}

Jøpeland Kraft AS appealed the decision, and the case then went before the regional court, Gulating Lagmannsrett. They overruled the decision made by Stavanger Tingrett. In their argument, they do not rely on direct assessment of the report made by NVE, nor do they mention the expert statements retrieved by the opposing sides. Instead, they base their decision on general considerations concerning the need for efficient procedures in cases such as these. Such reasoning provides the apparent grounds for making the following rather crucial observation concerning the facts:

\begin{quote}... It was not a mistake to take Ola Måland's statement into consideration, as he was, and still is, a significant property owner. NVE's statement to the effect that granting the concession will facilitate 
a more effective use of the water seems appropriate, as it refers to a current hydro-electric plant that 
exploits a waterfall of 13.5 meters.
\end{quote}

Nowhere in their decision do they mention the statement made by Hjelmeland kommune, nor do they comment on the fact that alternative hydro-power, as suggested by the NVE itself, and pointed to in this statement, amounts to exploiting the waterfall over a difference in altitude of some 550 meters. In fact, the hydroelectric plant that they do mention has nothing to do with Ola Måland and the other owners, but exploits the same water further downstream. It was brought up in the testimony made by a representative from NVE, who, when pressed on the matter, claimed that the reasonable way to interpret the paragraph that Stavanger Tingrett quoted, and to which Gulating Lagmannsrett implicitly refer, was to see it as a statement regarding this hydro- electric plant. In light of the statement provided by Hjelmeland kommune, to which the report explicitly refers, this appears to be a manifestly ill-founded interpretation. But the regional court adopted it, without further comment.

As far as the legal basis of their decision is concerned, it seems that Gulating Lagmannsrett holds, quite generally, that the practice adopted by the water authorities in cases like these still provide adequate protection for original owners, and that it is not for the courts to subject it to critical review. As mentioned, they seem to base their stance in this regard on an overreaching appeal to the need for efficient procedures to deal with cases such as these.

The decision was appealed by Ola Måland and other, and the Norwegian Supreme Court decided to consider the juridical aspects of the case. The appeal concerning the assessment of the facts made by Gulating Lagmannsrett would not be considered, but was to be taken as correct. Since Gulating Lagmannsrett decided to regard as inessential several facts that were seemingly apparent, even from the report made by NVE itself, the appellants presented these facts to the Supreme Court and argued that Stavanger Tingrett was right regarding their consequences. \noo{In addition to this, written statements were retrieved from the Øystein Grundt, the public officer from the NVE that had been responsible for the preparation of the case, and Harald Sollie, }

The Supreme Court ruled in favor of Jørpeland Kraft AS. They comment on the relevant facts on 
p. 9 of their decision. There, they mention that Jørpeland Kraft AS had considered the possibility that a hydro-electric scheme could be undertaken by local property owners. As we mentioned in Section \ref{sum}, a statement was provided to the NVE by Jørpeland Kraft AS themselves -- the parties who stood to benefit from the transferral -- addressing one possible project that was deemed not to be commercially viable. Recall that in the same statement another project was also identified -- in the same river, using the same water -- that they claimed was such a good project that it could be carried out even after the transferral. As we mentioned, the statement does not say anything about what the property owners stand to loose when the water from Brokavatn disappears, and the Supreme Court is also silent on this. Nor do they mention that the statement was never handed over to the applicants, and that the details of the calculations were never handed over to, or considered by, the NVE. In fact, the full report first appeared during the hearing at Gulating Lagmannsrett, but this fact was not considered relevant by the Supreme Court.

Moreover, the Supreme Court remains silent on the fact that the conclusion concerning efficiency of exploitation contradicts both the NVE's own assessment, the statement made by Hjelmeland Kommune, and also all subsequent assessments made both on behalf of the applicants and on behalf of Jørpeland Kraft AS. We mention that all of the above were presented to all national courts, including the Supreme Court.

As to the legal questions raised by the case, the Supreme Court makes a more detailed argument than the regional court, culminating in the conclusion that established practice still provides adequate protection. Interestingly, the Supreme Court base their arguments in this regard on the premise that the case does \emph{not} involve expropriation of waterfalls. A similar sentiment is expressed by Gulating Lagmannsrett, and it was also argued for by Jørpeland Kraft AS, but the true force of this point of view did not become apparent until the case reached the Supreme Court. 

The Court first concludes that a legal basis for the concession to transfer the water is to be found in the Watercourse Regulation Act, Section 16. Moreover, they conclude that while this provision alone does not provide a right to expropriate the waterfall, it does give the applicant a right to divert the water away from it. While the Supreme Court notes that this amounts to an interference in property rights, they take it as an argument in favor of regarding the rules in the Watercourse Regulation Act as the primary source of guidance concerning what should be considered when preparing such cases. The hold, in particular, that the provisions in the Expropriation Act applies only so far as they supplement, and are not in conflict with, the rules of the Watercourse Regulation Act and established practice with respect to the provisions in this Act. Moreover, the main reason they give for this is that the diversion of water is \emph{not} to be considered as an expropriation of a waterfall.

There is, as we mentioned, no rule in the Watercourse Regulation Act which states that the authorities are required to consider specifically the question of how the regulation affects the interests of property owners. Such a rule is found in the Expropriation Act, Section 2, but according to the Supreme Court, it does not apply in cases where water is being diverted away from a river. This is so, according to the Supreme Court, because transferral of water is not regarded as a case of expropriation of a right to the waterfall, but merely an expropriation of a right to deprive the waterfall of water.

This is significant in two ways. First, it is important with respect to the legal status of owners who are affected by projects involving transferral of water. In Norwegian law after Måland, it seems that established practice with respect to the assessment of such cases, focusing on environmental aspects and the positions taken by various interest groups, is beyond reproach already because such cases do not involve expropriation of waterfalls. However, considering that the Norwegian water authorities seem to follow these practices generally, and not just in cases where water is transferred, it remains to be seen if this is a practically significant difference in the level of protection. Is the conclusion regarding the admissibility of current administrative practices supposed to apply only to those cases when water is subject to transferral? If it is, then it leads to the peculiar situation that the level of protection for owners depend solely on the way in which the developer propose to gain control over the water. The difference appears completely arbitrary, however, at least from the point of view of owners. But of course, it will soon cease to be arbitrary for developers, who must be expected to favor gutter projects, collecting water from many small rivers and diverting it, since this mode of exploitation makes it easier to acquire necessary rights. On the other hand, if the Supreme Court is to be understood as saying that traditional practices are adequate in general, the consequences of the decision seem fairly dramatic for local owners. It appears that it is not possible, in cases involving expropriation of waterfalls, to solicit any kind of judicial review, not even in circumstances when the factual basis of the decision is manifestly erroneous, and not even if this appears to be the consequence of the authorities neglecting to keep local owners informed about the assessments made regarding their interests.

To illustrate that a lack of consultation is a general problem, and not confined to the particular case of \emph{Måland}, we will conclude by offering a quote from Harald Solli, director of the Section for Concessions at the Ministry of Petroleum and Energy, who submitted written evidence to the Supreme Court regarding the practices followed in cases involving expropriation of waterfalls. Below, we give one of several exchanges that seem to indicate that under current practices, local owners are left in a rather precarious position (my translation).

\begin{quote}
Q: In cases such as this, should owners affected by a loss of small scale hydro-power potential be kept informed about the factual basis on which the authorities plan to base their decision? I am thinking especially about those cases in which the authorities make an assessment regarding the potential for small scale hydro-power on affected properties. \\
A: Affected owners must look after their own interests. The assessments made by the NVE in their report is a public document, and it can be accessed online through the home page of the NVE.
\end{quote}

By their reasoning in \emph{Måland}, it appears that the Supreme Court gave this dismissive attitude towards local owners a stamp of approval. In light of this, we believe the study of the law in a socio-legal setting becomes all the more relevant. For while this attitude might be a reflection of correct national law, as decided in the final instance by the Supreme Court, it seems pertinent to ask if it is \emph{reasonable} law. Also, it seems that one must ask if a case can not be made with respect to human rights, by arguing that the protection awarded is insufficient in this regard. This point, while it was raised by the original owners in \emph{Måland}, did not receive any separate treatment in the Supreme Court. In the following section, we briefly describe some more questions we think the case raises and which we will address further in subsequent chapters.

Following \emph{Måland}, it seems we must conclude that the development which has taken place in the energy sector, and has lead to small scale hydro-power becoming profitable and possible for local owners to carry out themselves, does not imply that original owners are entitled to increased participation in decision-making processes under national law. Even if this is the view held by the Norwegian judiciary, we should of course not overlook the possibility that the water authorities themselves will eventually adopt new practices regarding the assessment of such cases. So far, however, it seems that they stick quite closely to the established routine. 

Since the outcome in Norwegian Courts was that established practices were not found to be in breach of principles of Norwegian expropriation law, it seems reasonable to ask instead about the sustainability of these practices. In fact, the case of \emph{Måland} seems to illustrate precisely why the current system is inadequate, and how it can lead to decisions that appear ill-founded and leave the affected communities feeling marginalized. The likelihood of \emph{factual mistakes}, in particular, seems to increase greatly when the involvement of the local population is not ensured in the preparatory stages.

More importantly, it seems that decisions reached following a traditional process can easily lead to takings for which it is difficult to see any legitimate reason why the project proposed by the developer would be a better form of exploitation than allowing the local owners to carry out their own projects. Indeed, in the case of \emph{Måland}, it seemed that small-scale hydro-power would be a better way of harnessing the water in question, even in the sense that it would be more efficient, and would provide the public with more electricity at a lower cost. More generally, unless the issue of alternative exploitation in small scale hydro-power is considered during the assessment made by the water authorities, one risks making decisions that are not in the public interest at all. 

Even worse, it can send out the signal that expropriation of owners' rights is undertaken solely in order to benefit the commercial interests of the energy company applying for a development license. At this point, it seems appropriate to recall the concerns expressed by US Justice O'Connor in the case of {\it Kelo}.

There, a major point of contention was whether or not her grim predictions about the fallout of the decision did indeed reflect a realistic analysis of the fallout of the decision. Surely, anyone who agrees with Justice O'Connor that the powerful will usurp the power of eminent domain to the detriment of the poor, would also agree with here conclusion that it is perverse. However, whether her pessimism is warranted by empirical fact seems less clear. In this context, we believe the case of Norwegian waterfalls can serve an important broader purpose, as a means towards shedding more light on the hypothesis that a loose interpretation of the public interest requirement will indeed lead to a transfer of property from those with fewer resources to those with more. 

The \emph{Måland} case, and the current tensions regarding expropriation for the benefit of Norwegian hydro-power, seems to suggest that her concern should indeed be taken seriously. Also, the Norwegian experience seems to show that we need to be clear about the fact that property has a social and political function that goes beyond the financial interests of individuals. For the Norwegian case at least, it seems particularly relevant to ask if local people, by virtue of their right to property and their original attachment to the land, have a legitimate expectation \emph{both} that their commercial interests should be protected, \emph{and} that they should be granted a say in decision-making processes. Financial protection does not necessarily imply social protection, and the right to participate and be heard might be both more significant, and harder won, than the right to be compensated according to whatever the powers that be come to regard as the market value of the property in question.

Another perspective, which we will also pursue further in subsequent chapter, is the question of how property rights relates to the overreaching goal of sustainable development of natural resources. Rather than seeing property rights as a means towards securing sustainable development, it seems more common to see it as an impediment. This, indeed, has shaped much of the Norwegian discourse regarding environmental law and policy, including that which relates to waterfalls.\footnote{For example, such a skeptical view of property rights appear to provide an overriding perspective in \cite{backer1} (in Norwegian), which is a widely used textbook on environmental law in Norway.} 

Moreover, a typical justification given for interference in property is that an equitable and responsible management of natural resources requires it. It seems to us, however, that an egalitarian system of private ownership of resources -- as we find in Norway for the case of waterfalls -- could itself serve as a sustainable basis for management of these resources. It seems plausible for us to suggest that private property rights is one of the most robust ways in which local communities can be given a degree of self-determination concerning how to manage local resources. This is typically considered desirable also from the point of view of sustainability, but perhaps even more importantly, when property is in the hands of the many rather than the few, is it not also reasonable to expect that the state will be able to more effectively and rationally exercise its regulatory powers? 

Otherwise, the danger is that the government is being intimidated by large commercial enterprises, perhaps partly owned by the State itself, that command political influence and might not take lightly to what they perceive as undue political interference in their business practices. Such a position might be tenable if you are one of the worlds leading energy companies, but hardly if you are a farmer. 

I think the case of \emph{Måland} suggests that we should investigate these questions in more depth. It seems, in particular, that we must ask about the extent to which commercial companies have succeeded in usurping the notions of sustainable development and public interest, putting the power of these ideas to use in order to secure control over resources and to enlist governmental support, and favorable treatment, for their own commercial undertakings. The extent to which such a mechanism influences the Norwegian energy sector, and the possible implications this might have, both legally and socially, remains to be worked out.

In subsequent chapters, two questions arising from this will receive particular focus. First, we will aim to clarify the importance of the conflict between large scale hydro-power and small scale development by surveying recent and current hydro-power projects in Norway, not in any depth, but by taking note of whether the issue arose. Secondly, we will aim to shed light on the importance of small scale hydro-power to the communities in which local owners reside. As we mentioned, they are usually farmers, and most often in areas were farming is becoming increasingly unprofitable. From the socio-legal point of view it seems highly relevant to ask who the people who loose their resources are, and in what social context we find them. Moreover, while it is clear that hydro-power has become an important source of income in many small and relatively impoverished farming communities, the exact implications of this development, financially and socially, remains to be mapped out.

Following this, it seems natural to return to the legal question of the legitimacy of interference, not from the point of view of national law, but from the point of view of property as a human right. Importantly, it seems to us that property has a clear social dimension, and that mapping out the socio-legal function of specific property rights should inform the judgment we make regarding the level of protection to which owners are entitled. Also, while property is an individual right, it can also be a communal one, and, as such, it can serve to empower local communities that would otherwise be marginalized. The protection of an egalitarian structure of ownership, then, does not appear to be subsumed by, or even conceptually the same as, protecting against individual transgressions. We believe that the case of Norwegian waterfalls demonstrates that this should be kept in mind when analyzing the legitimacy of interference in property for the benefit of commercial undertakings.

\noo{current ownership structure of waterfalls is therefore not simply a question of protecting the commercial interests of individuals who happen to own valuable resources, but also a question of protecting the local communities where these resources are found, giving them the possibility of influencing the way in which the resources are to be harnessed. It seems, however, that local people are often in danger of being seen as an hindrance, both to sustainable development and economic growth, because the commercial companies, along with the environmental interests groups, have claimed this stage as their own. Such, it seems, is the case for Norwegian waterfall. Despite an explosion of interest in small scale hydro-power in recent years, there still seems to be little room left for local communities in the Norwegian discourse concerning hydro-power. It will be an important aim of our work in following chapters to map our in more detail how this influences the law and the administrative policies that are adopted.
}

\section{Conclusion}\label{conc}

In this Chapter, I set out to show that the law relating to expropriation of waterfalls in Norway is based on a tradition that assumes owners to be profit-maximizing while the state is welfare-seeking. Hence, the question of striking a balance between private and public interests is approached under the presumption that private property rights embody mainly private values, while public values are pursued through regulation that ensures public ownership and control. I observed how this perspective shaped the law of expropriation of waterfalls, so that expropriation could only take place for narrowly defined public purposes and only to the benefit of public bodies.

I noted, however, how the increasing centralization of the energy sector and the increasing scale of projects following WW2 led to increased worry about the legitimacy of interference in property and nature on behalf of public hydropower interests. I concluded that the ensuing conflicts, while severe, largely failed to make an impact on the law relating to hydropower, as the public nature of this sector, and the level of political control exercised over it, meant that courts shunned away from adopting a strict view on legitimacy. This did not only apply to the question of authority to expropriate, which was hardly raised at all in the period between the reversion controversy of the early 20th century and the market-reform of the early 1990s.  It also applied to the procedural rules, which the Supreme Court adopting an explicit stance that these rules were themselves largely ``discretionary'' in nature, so that it would fall under the authority of the executive to determine their scope and application in concrete cases.

I noted how this perspective has been maintained by the courts and the executive even after the market-reform meant that expropriation largely lost its public characteristics. I argued that today, expropriation of waterfalls for hydropower development can no longer be looked at as an aspect of providing a public service, but must be regarded as takings for profit, typical economic development takings. I discussed how the law came to be changed on this point, with a dramatically widened expropriation authority introduced in conjunction with the \cite{wra00}. I observed how the issue of expropriation was not considered politically, with the reform in the legislative basis taking place without the active involvement or consideration by the members of the Norwegian Storting.

I concluded with a description of the fallout from this, as expressed concretely in the case of \cite{måland11}. The case serves to illustrate how administrative practices developed and sanctioned during the monopoly days are now applied in a context of competing commercial interests, meaning that expropriation becomes an important tool that the powerful market players now use to gain the upper hand in competition with locally based companies or smaller companies that rely on cooperation with owners. I noted, in particular, that the law is entirely unprepared for dealing with this dynamic. Still, in the case of \cite{måland11}, the Supreme Court explicitly denied that established practices were in need of revision. Moreover, it refused to reconsider the established interpretation of the scope of procedural rules in hydropower cases, rejecting arguments to the effect that these must now be understood to provide protection for waterfall owners that matches the protection offered to other affected parties.

At the same time, I noted how the Supreme Court {\it did} agree that a revision of established compensation practices is in order. I noted, however, that the Court's emphasis on the compensation issue serves to reinforces the idea that private property rights pertain mainly to financial entitlements. In addition, the great complexity and many special rules in this area of the law means that powerful market actors can use a whole battery of different strategies to bring compensation payments down, despite the revised starting point sanctioned by the Supreme Court. 

More importantly, I argued that the compensation-based perspective hardly does justice to the role of private ownership of waterfalls in Norway. It does not, in particular, take into account the social function of this patter of resource ownership. Importantly, owners are not empowered to participate in decision-making processes in a meaningful way -- the compensation-perspective renders them passive participant who might -- if they are lucky or able to secure good council -- benefit from a financial windfall, with few strings and little meaning attached to it. The waterfall owners might well feel like they have won the lottery, and their neighbours and the general population might well resent them for it. In this way, expropriation weakens property as a social institution, beyond the damage done by specific acts of expropriation.

To address this concern I will now go on to consider {\it alternatives} to expropriation. The system of {\it land consolidation}, in particular, is an ancient institution in Norway that sees property as part of the solution towards sustainable development, rather than as part of the problem. It is now being extensively used also in relation to hydropower development, particularly when local owners themselves carry it out. I discuss its merits in the next, and final, chapter of this thesis.

As demonstrated in the present Chapter, Norwegian courts do not seem to recognize the shortcomings of the current system. Until they do, or are directed to do so by political bodies or international tribunals, it is unlikely that expropriation law will evolve much from its current fixation on the compensation issue. 








In this chapter, I have given a presentation of expropriation of waterfalls for hydropower 

As we have shown, \emph{Måland} serves to illustrate many of the current tensions and issues surrounding expropriation of waterfalls in Norway. It also serves to clarify the extent to which local owners are 
marginalized under the regulatory practices currently in place, and shows that the regulatory system does not clearly separate the question of how to judge an application to undertake development from the question of whether or not expropriation should take place. Moreover, the case seems to suggest that this will tend to lead to the emphasis being on issues that have to do with development, while issues relating to expropriation, and owners' interests, will be overlooked. Summing up, the case seems to show that the current regulatory system in Norway functions in such a way that it is bound to give rise to conflicts between local interests and the interests of commercial companies and the State.

The case also sheds new light on the legitimacy of using expropriation in order to benefit commercial interests. In this way, it takes on broader significance, by lending empirical support to the prediction offered by Justice O'Connor with respect to \emph{Kelo}, regarding the fallout of a loose interpretation of the public interest requirement for expropriation.

In our opinion, this contributes to making Norwegian waterfalls an interesting case study on expropriation,  and one that warrants further consideration with respect to human rights. In subsequent chapters, we will offer such an analysis, by addressing the question of whether or not local owners and communities can claim that they are entitled to greater protection than that which is currently provided under Norwegian law.

\section{Conclusion}

Old stuff:

\section{Introduction}\label{intro}

In most jurisdictions that recognize private ownership rights, the state retains a right to interfere with property rights to the extent needed in order to provide public services and further public interests. However, in many cases when the State infringes on private property rights in this way it also recognizes an obligation to compensate the owner. This is certainly the case when interference takes the form of an outright expropriation and the property rights themselves are transferred from the original owner to the State or one of its agents. In many jurisdictions, the obligation to compensate the owner in such cases follows already from constitutional provisions protecting the right to property. Also, the owner's right to compensation tends to follow from international law to protect human rights.\footnote{The principle of a fundamental right to property is widely recognized, and is encoded in Article 17 of the Universal Declaration of Human Rights. A particularly relevant formulation, in a European context, is the one found in the European Declaration of Human Rights (ECHR) Protocol 1, Article 1. Its importance stems from the fact that the ECHR is accompanied by a special judicial body, the European Court of Human Rights (ECtHR) in Strasbourg, and has also been widely incorporated into the national law of the ratifying countries, including the UK and Norway.} 

One important question that arise is how compensation should be calculated. An often adopted starting point is to say that the owner should be compensated for his financial loss. In some cases, specific policy reasons might be given in favor of deviating from this, for instance if considerations based on social justice dictates that less than full compensation should be paid, but still, as an overreaching principle, compensation for financial loss has widespread status as a general rule. This is reflected, for instance, in case-law from the European Court of Human Rights.\footnote{In particular, the Court in Strasbourg adopts the view that interference should be \emph{proportional}, which, in most cases, entails compensating the owner for his full financial loss, usually taken to be reflected by the market value of his property, see \cite{AllenCom} for an in depth analysis of this principle, showing also its controversial aspects in light of the history of the ECHR.} While the principle protects property owners, it is also often cited as a reason to reject claims of compensation based on a hypothetical voluntary agreement between property owners and the public. Such an approach might correctly reflect the \emph{willingness} of the public to pay, but the value to be compensated is the loss, or, as it is often said, the \emph{value to the owner}. This, in particular, is typically taken to mean that the public should not be required to pay extra to reflect the fact that the scheme underlying expropriation serves an important social function that the public might be willing to pay for. In this way, the State ensures that public interests are not being made available as a means for property owners to make excess financial profits, beyond what they could otherwise expect from their property.

Under modern planning regimes, however, the scope of interests allowed to benefit from expropriation has been significantly widened, and in many cases the State explicitly allows for public interest schemes to merge with commercial undertakings. In this way, it is nowadays commonplace that companies operating for profit come to benefit financially from schemes involving expropriation. Increasingly, expropriation is also made available as a tool for purely commercial schemes, the rationale behind this being that there are indirect benefits to the public -- increased tax revenues, generation of jobs etc -- that justify the use of compulsion. This kind of expropriation often proves controversial, however, as illustrated by the uproar in the US following the case of \emph{Kelo}, where the drug company Pfizer was allowed to expropriate land for the construction of research facilities.\footnote{\emph{Kelo v City of New London}, 545 U.S. 469 (2005)} While a majority the US Supreme Court found that the expropriation was constitutional, it was widely felt to be inappropriate, both by members of the general public and the legal profession.\footnote{See, for instance \cite{notimminent}, which also demonstrates that this kind of expropriation is relatively uncommon in the US, thus reflecting the pervading opinion that it is unsound, if not illegal.}

While expropriation for commercial schemes raises important questions about legitimacy of interference, it raises even more pressing questions in relation to established principles for calculating compensation. Crucially, the traditional justification for the "value to the owner"-principle appears to be severely undercut by the fact that certain actors are already, as a result of the regulatory regime, placed in a position to profit financially from schemes that are ostensibly carried out in the public interests. This begs the question: What are the policy reasons for maintaining a principle that excludes property owners from a share in this profit? As was noted by the Law Commission in a discussion paper from 2001, traditional arguments for the appropriateness of limiting the right to compensation often fail to do justice to such cases, and the adequacy of existing rules and principles needs to be considered in this light.\footnote{See \cite{lcdisc} and \cite{newuk,kelouk}, all addressing the problem from the point of view of UK law.\noo{The same issue was raised ....}} Moreover, the question arises as to whether or not established practices are in keeping with human rights law. For instance, can owners claim a share of the commercial value of their property on the basis that they are entitled to it under the European Convention of Human Rights, even if established national practices would lead to compensation payments that do not reflect this value?

So far, a theoretical framework for discussing questions such as these, arising specifically with respect to awarding compensation for commercial schemes, appears to be largely missing, and not much scholarly work has been devoted to it.\footnote{Is this really true? I could not find that much...} In this paper, we make a contribution in this regard, offering a case study of Norwegian waterfalls, which are often expropriated for the development of commercial hydro-power projects.\footnote{Hydro-power is an important source of electricity in Norway, with some 95 \% of annual domestic electricity consumption due to hydro-power.} 

In the early 1990s, the Norwegian energy sector was liberalized, making it possible for owners of waterfalls to exploit these commercially in small-scale hydro-power projects. Soon, a market also developed for waterfalls, where the prices paid would reflect the value of the waterfall for commercial, small-scale, hydro-power. As a result of this, the law relating to compensation for expropriation of waterfalls has recently undergone significant changes, with courts now increasingly looking to the new market for guidance when determining the right level of compensation following expropriation. The new market is still embryonic, however, with most large companies opting instead to use expropriation to acquire necessary rights. As such, the market-based approach often cannot offer more than a limited perspective on the true commercial value of waterfalls, and this is the root of many contemporary controversies in Norwegian hydro-power law.

It seems that the case of Norwegian waterfalls is ideal for shedding light on many of the general questions that arise for compensation following expropriation that benefits commercial schemes and it is particularly interesting to see how the Norwegian Supreme Court have dealt with these questions in a series of recent decisions. While the sentiment has changed in favor of regarding the commercial potential for hydro-power as a relevant factor when valuating waterfalls, the influence of the "value to the owner" principle is still clearly felt in the Court's reasoning. However, in several cases, the logical conclusion offered by such reasoning has appeared so offensive to the lower courts that they have instead opted for a retreat to a traditional, theoretical method for calculating compensation, that is \emph{not} based on a standard "value to the owner" approach.

In a recent decision, the Supreme Court gave this approach its stamp of approval, providing thus a further indication that there is a pressing need for academic assessment of the current state of affairs in Norway.\footnote{\emph{Bjørnarå and Others v Otra Kraf DA and Otteraaens Brugseierforening} (Otra II), Rt. 2013 s. 612}

The structure of our paper is as follows. In Section \ref{sec:noscheme}, we discuss the standard version of the "value to the owner" principle, focusing on those aspects that are expressed through the so-called "no-scheme" rule, often referred to as the Pointe Gourde rule in common law.\footnote{Following the precedent set by \emph{Pointe Gourde Quarrying and Transport Co v Sub-Intendent of Crown Lands} [1947] AC 565,
PC, 572, per Lord MacDermott. We remark, however, that the case was a clarification, and, seemingly, a widening of the principle, but by no means the first application of such a rule in common law, see \cite{lcdisc} \noo{Appendix D of Law Commission Report No 286, 2003 for an historical presentation}.} We approach the rule building on a conceptual distinction between commercial and non-commercial aspects of a scheme benefiting from expropriation, and we argue that this is appropriate, and even necessary, if we are to do justice to the rule in light of property as a human right. Then we move on to present a brief overview of Norwegian law relating to the no-scheme rule, arguing that the need for a conceptual distinction along the lines we propose is clearly felt in the contemporary Norwegian debate, which has at times been heated.

In Section \ref{sec:trad}, we present the traditional method for calculating compensation for expropriation of Norwegian waterfalls, showing that it deviates completely from the "value to the owner" principle, relying instead on a theoretical assessment of the value of the developer's scheme. We briefly discuss the historic context of this rule and we note the lack of attention paid to it in the general debate on compensation. We argue, however, that seeing the rule only as an anomaly is inappropriate. Rather, the rule serves as an illustration of one possible, albeit largely misguided, approach to a challenge that is becoming increasingly relevant in general, namely that of finding principles for valuation that are suitable for schemes with a substantial commercial component. 

In Section \ref{sec:new}, we discuss the new method that has been adopted by the courts after the liberalization of the Norwegian energy sector. While the method has yet to mature, it appears to follow a standard no-scheme approach. This raises some problems, however, and we discuss various recent cases that illustrate this. We then return to the issues raised in Section \ref{sec:noscheme}, and we argue that the case of Norwegian waterfalls illustrates clearly that a standard "value to the owner"-theory of compensation is inadequate in modern regulatory settings. 

The Norwegian case suggests, we believe, that there is a pressing need for regulation and legislation based on a distinction between commercial and non-commercial aspects of schemes benefiting from expropriation, and on the \emph{kind} of value that is inherent in a scheme benefiting from expropriation. While \emph{public value} is hardly to be regarded as value to the owner, it seems that \emph{commercial value}, to avoid discrimination, must largely be regarded as such for the purpose of compensation, irrespectively of \emph{who} the State happens to prefer as developer of the scheme. It appears, in particular, that especially the \emph{negative} aspect of the no-scheme rule, providing for a decrease in compensation, is hard or impossible to justify with respect to the commercial aspects of undertakings benefiting from expropriation. In Section \ref{sec:conc}, we conclude and point to possible directions for future work.

\section{The no-scheme rule}\label{sec:noscheme}

In most jurisdictions, a fundamental principle relating to compensation following expropriation is that the owner's loss should be calculated without taking into account changes in the value of his property that are due to the expropriation itself, or the scheme underlying it. In a recent Law Commission consultation paper, this principle is referred to as the \emph{no-scheme} rule, a terminology we will also adopt here, noting that while the exact details of the rule might differ between jurisdictions, the underlying principle appears to play a crucial role both in civil and common law traditions for regulating compensation following expropriation.\footnote{Need a good reference for this...}

While the no-scheme rule is easy enough to comprehend when it is stated in general terms, it raises many difficult questions when it is to be applied in concrete cases. What the rule asks of the valuer, in particular, is quite daunting; he is forced to consider the counterfactual "no-scheme world", and he must calculate the value of the property based on the workings of this imaginary world. One crucial question that arises, and which has traditionally proved to be highly contentious, is the question of what exactly this world looks like.

In the first instance, it might be tempting to take the view that this is a "question of fact for the arbitrator in each case", as expressed by the Privy Council in \emph{Fraser}, an influential Canadian case from 1917.\footnote{\emph{Fraser v City of Fraserville}, [1917] AC 187, p. 194} However, as the history of the no-scheme rule has shown, this point of view is not tenable.\footnote{For an history of the rule in UK law, clearly illustrating the difficulty in interpreting it and applying it to concrete cases, we point to Appendix D of Law Commission Report No 286, 2003} A more recent description of the rule, and the problems associated with it, was given by Lord Nicholls in the recent case of \emph{Waters}, who summarized the state of the law relating to compensation for expropriation as follows.\footnote{\emph{Waters and other v Welsh National Assembly} [2004] UKHL 19}

\begin{quote}
Unhappily the law in this country on this important subject is fraught with complexity and obscurity. To understand the present state of the law it is necessary to go back 150 years to the Lands Clauses Consolidation Act 1845. From there a path must be traced, not always easily, through piecemeal development of the law by judicial exposition and statutory provision. Some of the more recent statutory provisions defy ready comprehension. Difficulties and uncertainties abound. One of the most intractable problems concerns the 'Pointe Gourde principle' or, as it is sometimes known, the 'no scheme rule'. On this appeal your Lordships' House has the daunting task of considering the content and application of this principle.
\end{quote}

\noo{
\begin{quote}
The extreme complexity of the issues that I have had to consider, the
uncertainty in the law, the obscurity of the statutory provisions, and
the difficulties of looking back over a long period of time in order to
decide what would have happened in the no-scheme world
demonstrate, in my view, that legislation is badly needed in order to
produce a simpler and clearer compensation regime. I believe that
fairness, both to claimants and to acquiring authorities, requires
this
\end{quote}
}
In the case of \emph{Waters}, the House of Lords seems to have made it an explicit aim to offer a clarification of the no-scheme rule and how to interpret it, and their judgment went into much more detail than warranted by the concrete case at hand, which seems to have been fairly straightforward. Even if it was not needed for the result, the House of Lords addressed many of the issues raised by the Law Commission in their report, focusing particularly on resolving the tension which was identified there between the principle relied on in \emph{Pointe Gourde} and the reasoning adopted in the so-called \emph{Indian} case from 1939.\footnote{\emph{Vyricherla Narayana Gajapatiraju v Revenue Divisional
Officer, Vizagapatam} [1939] AC 302.} In the \emph{Indian} case, the scheme was given a very narrow interpretation, with Lord Romer interpreting the scope as follows.
\begin{quote}
The only difference that the scheme has made is that the acquiring
authority, who before the scheme were possible purchasers only, have
become purchasers who are under a pressing need to acquire the
land; and that is a circumstance that is never allowed to enhance the
value.
\end{quote}
This, however, did not entail that the purchaser's demand for the property was to be disregarded, since, as Lord Romer puts it:

\begin{quote}
[...] The fact is that the only possible purchaser of a potentiality is
usually quite willing to pay for it […]
\end{quote}

In \emph{Pointe Gourde}, a different stance seemed to be taken in this regard. The case concerned a quarry that was expropriated for the construction of a US naval base in Trinidad. The quarry had value to the owner as a going concern, and the valuer had found that if the quarry had not been forcibly acquired, it could have o supplied the US navel base on a commercial basis, leading to its value being enhanced. This value, however, fell to be disregarded, with Lord MacDermott describing the no-scheme rule as follows.

\begin{quote}
It is well settled that compensation for the compulsory acquisition of
land cannot include an increase in value, which is entirely due to the
scheme underlying the acquisition
\end{quote}

Seemingly, then, the two cases are at odds with each other as far the interpretation of the no-scheme rule goes. In \emph{Waters}, both Lord Nicholls and Lord Scott addressed this tension in great detail, and offered a reconciliatory interpretation, which seems to narrow the no-scheme rule compared to how it has most commonly been understood following \emph{Pointe Gourde}. Moreover, the House of Lords also noted the need for reform and legislation, with Lord Scott going as far as referring to what he described as the present "highly unsatisfactory state of the law".

To understand how a seemingly simple principle could come to prove so troubling in practice, it is helpful to keep in mind that following extensive planning legislation, especially following the Second World War, development of property tends to be contingent on governmental licenses and subject to extensive oversight. Moreover, the power to expropriate is often granted as a result of comprehensive regulation of the property-use in an area, often following public plans that are wider and encompass more than the particular project that will benefit from such a power. As a result, it became increasingly difficult to ascertain what is meant by the scheme; does it include the whole planning history leading to expropriation, does it only refer to the power to expropriate, or is it something in between?

When attempting to address this issue, there any many pitfalls, and the policy reasons suggest that a fine balancing act must be made. If given a wide interpretation, the property owner might easily come to feel that he is not compensated for his true loss, but rather an imaginary one. Moreover, the no-scheme world that the valuer must consider can end up being far removed from the actual one, forcing him to go back many years, perhaps decades, to establish what would have been the status of the property in question if the sequence of planning steps eventually leading to expropriation had not taken place. This can leave the property owner in a perilous situation, and make the outcome seem so arbitrary as to run amiss with respect to human rights law and constitutional provisions protecting private property. On the other hand, if the scheme is interpreted too narrowly, it runs the risk of endangering important public schemes by compelling the public to pay extortionate amounts for an increase in value that is entirely due to their own non-commercial investments and plans for the area in which the property is found.

It is important to keep in mind here, as noted by the Law Commission, that the no-scheme rule serves two very different policy aims.\footnote{Ibid ....} It should be noted, in particular, that the rule has an important \emph{positive} dimension: property owners are not only compensated for the direct loss of their property, but also for the possible depreciation of their property's value following the decision to carry out a scheme which requires expropriation. Seemingly, this is easy enough to justify; it would easily appear unreasonable, and possibly in breach of human rights law, if compensation payment was reduced as a result of the threat of compulsion.

However, under the extensive planning regimes common today, it is not clear where to draw the line: When is the regulation leading up to the scheme to be regarded as reflecting general public control over property use, and when is it to be regarded as a measure specifically aimed at compelling private owners to give up their property? As we will see when we consider the role of the no-scheme rule in Norwegian law, this question can easily become highly controversial, especially when it is linked with the more general question of whether or not the State should be liable to pay compensation for regulation that adversely affects the potential for future development. In jurisdictions that do not recognize owners' right to such compensation, like Norway and the UK, it is easily argued that the positive aspect of the no-scheme rule must be limited correspondingly. Why would a depreciation of value following regulation imply compensation when the property is eventually expropriated, but not otherwise?

In addition to its positive dimension, the no-scheme rule also has an important \emph{negative} dimension, which is the dimension with which \emph{Waters} was mostly concerned, and which was expressed in \emph{Pointe Gourde} by saying that one should disregard an increase in value that was "entirely due to the scheme". The negative dimension has attracted even more interest and controversy than the positive dimension, especially in the UK, and this is understandable all the while the negative aspect of the rule can easily come to be perceived as unfair by property owners. However, on a traditional understanding of the public purpose of expropriation, the negative aspect of the rule is also seemingly easy to justify. In \emph{Waters}, Lord Nicholls describes the policy reasons behind it as follows:

\begin{quote}
When granting a power to acquire land compulsorily for a particular purpose Parliament cannot have intended thereby to increase the value of the subject land. Parliament cannot have intended that the acquiring authority should pay as compensation a larger amount than the owner could reasonably have obtained for his land in the absence of the power. For the same reason there should also be disregarded the 'special want' of an acquiring authority for a particular site which arises from the authority having been authorised to acquire it.
\end{quote}

This appears like a reasonable line of argument. Notice, however, that Lord Nicholls completely avoids using the word "scheme" here, and does not use the absence of the scheme as the yardstick by which parliament must have intended that compensation should be based. Rather, Lord Nicholls speaks of what the owner could reasonably have obtained in \emph{the absence of the power} to acquire the land compulsory. In this way, he seems to prescribe a rather narrow interpretation of the negative dimension of the no-scheme rule.\footnote{I mention that this interpretation of \emph{Waters} is also argued for in \cite{newuk}.} It is the power to expropriate that should not give rise to an increased value, and nothing is said at this stage about the scheme that benefits from it. It would appear, therefore, that there is nothing in principle that prevents the property from being compensated on the basis of its value in a scheme that differs from the scheme underlying expropriation only in that it does not have such powers. Indeed, this subtle distinction appears to have been rather crucial for the remainder of Lord Nicholls' reasoning, where he attempts to reconcile the principle adopted in the \emph{Indian} case with the \emph{Pointe Gourde} case.

It will lead us to far astray to go into further details about the interpretation of the no-scheme rule in UK law and the possible implications of \emph{Waters}. Rather, we would like to turn our attention to the recent UK Supreme Court case of \emph{Bocardo}.\footnote{\emph{Star Energy Weald Basin Limited and another (Respondents) v Bocardo SA (Appellant) [2010] UKSC 35}} This case was decided under dissent, and it suggests that the clarification offered in \emph{Waters} might not have been as conclusive as one had hoped. This worry arises, as we will see, particularly in those cases when expropriation benefits commercial schemes.\noo{ and for which the conceptual framework surrounding expropriation is, in our opinion, in need of refinement.}

\emph{Bocardo} was such a case. In short, it concerned a reservoir of petroleum that extended beneath the appellant's estate, and could not be exploited without carrying out works beneath their land. The first question that arose was whether or not extraction of the petroleum amounted to an infringement on property rights, which was answered in the affirmative. The second question that arose was what principle of compensation should be adopted to compensate the owner. The Supreme Court, following some deliberation, found that the general rules applied, and that the case should be decided on the basis of an application of the no-scheme rule. Here, however, opinions differed as to the correct interpretation of the law, as well as how the facts should be held against the law. The crucial point of disagreement arose with respect to whether or not the special suitability, or \emph{key value}, of the appellant's land for the purpose of petroleum exploitation was to be regarded as \emph{pre-existing} with respect to the petroleum scheme.

In \emph{Waters}, the House of Lords had cited and expressed support for the following passage, taken from Mann LJ's judgment in \emph{Batchelor}.\footnote{\emph{Batchelor v Kent County Council} 59 P \& CR 357 p. 361}

\begin{quote}
If a premium value is 'entirely due to the scheme underlying the acquisition' then it must be disregarded. If it was pre-existent to the acquisition it must in my judgment be regarded. To ignore the pre-existent value would be to expropriate it without compensation and would be to contravene the fundamental principle of equivalence (see \emph{Horn v Sunderland Corporation}).
\end{quote}

Relying on this distinction between the potentialities that are "pre-existing" and those that are due to the scheme, the minority in \emph{Bocardo}, led by Lord Clarke, made the following observation.

\begin{quote}
Anyone who had obtained a licence to search, bore for and get the petroleum under Bocardo’s
land would have had precisely the same need to obtain a wayleave to obtain access
to it if it was not to commit a trespass. So it was not the respondents’ scheme that
gave the relevant strata beneath Bocardo’s land its peculiar and unusual value. It
was the geographical position that its land occupies above the apex of the
reservoir, coupled with the fact that it was only by drilling through Bocardo’s land
that any licence holder could obtain access to that part of the reservoir that gives it
its key value.
\end{quote}

This, however, was rejected by the majority, led by Lord Brown, who interpreted the no-scheme rule quite differently in this respect, and who made the following comments regarding the issue of whether or not the value of the appellants land for petroleum extraction existed prior to the scheme.

\begin{quote}To my mind it is impossible to characterise the key value in the ancillary
right being granted here as “pre-existent” to the scheme. There is, of course,
always the chance that a statutory body with compulsory purchase powers may
need to acquire land or rights over land to accomplish a statutory purpose for
which these powers have been accorded to them. But that does not mean that upon
the materialisation of such a scheme, the “key” value of the land or rights which
now are required is to be regarded as “pre-existent”.
\end{quote}

While the case was resolved in keeping with this view, the dissent suggests that the clarification in \emph{Waters} has not resolved all issues, and that special questions arise with respect to the question of what potentials for development should be taken into account when evaluating a property. Crucially, the question raised in \emph{Bocardo} does \emph{not} relate to the scope of the scheme -- it was obvious that the scheme was the entire project aiming to extract petroleum from the reserve. However, even when the scheme was unambiguously circumscribed, significant questions arose as to what "value to the owner" actually meant. 

In fact, it seems to us that \emph{Bocardo} serves to take the debate regarding compensation and the no-scheme rule one step further, and in a somewhat different direction compared to the debate revolving around the "classical" problem of determining the extent of the scheme. In some sense, it seems that the question raised by \emph{Bocardo} goes deeper, and to the very core of the idea underlying the negative aspect of the no-scheme rule. When is it appropriate to say that some particular value is \emph{due to} the scheme?

This asks us to establish a causal link between scheme and value, and as \emph{Bocardo} illustrates, it is by no means obvious what should be taken to constitute evidence for such a link. Moreover, it seems that the answer can depend largely on the point of view with which you \emph{choose} to analyze the matter at hand. For instance, when Lord Clarke went on to point out that the State, as owners of the Petroleum following nationalization in 1937, could have given the right to extract it to \emph{someone else}, he was certainly not incorrect.\footnote{References.} Moreover, it seems that this fact does in some sense break the causal link between scheme and value, although weakly so, since the difference between all schemes so conceived would only relate to \emph{who} the developers are, not the nature of the schemes as such. Consider, however, a scheme that was conceived of slightly differently, and assumed to suffer precisely from such an \emph{absence of the power to expropriate} as Lord Nicholls referred to in \emph{Waters}. Would it not follow that this scheme would also have \emph{precisely the same need to obtain a wayleave}, as Lord Clarke puts it, and that those behind it might now also be \emph{quite willing to pay}, as Lord Romer expressed it in the \emph{Indian} case?

On the other hand, it is also possible to take the point of view adopted by Lord Brown, which, albeit less clear in its formulation, we interpret to be roughly the following: Since the relevant strata did not, as a matter of fact, have any value except such value as it derived from its key value to a petroleum-scheme requiring access, its value was causally dependent on the existence of \emph{some} such scheme, and could thus not be regarded as pre-existent to \emph{any} such scheme, including the actual scheme, for which power to expropriate \emph{was} in fact granted.

Clearly, the outcome of \emph{Bocardo} turned largely also on the specific question of how to appropriately compensate property that has "key value" with respect to the development of other property. It seems, in particular, that the strata, in its absence of any inherent value, more easily fell to be disregarded, and that Lord Brown's arguments in particular relies on establishing such a lack of intrinsic value. However, the question of when a particular aspect of value is to be regarded as pre-existent tend to arise in many other cases as well, and can be expected to arise particularly often with respect to commercial schemes. An extreme case obtains when we consider expropriation of \emph{natural resources}. Surely, if what was subject to expropriation in \emph{Bocardo} had been the petroleum itself, and not a right to access it, then even Lord Brown would have concluded that its value was pre-existent? This seems likely indeed, and then it appears to be good law in the UK after \emph{Waters} that it should also be compensated, irrespectively of whether or not the expropriating party is the only potential buyer.

However, in several of the "classical" cases that are cited as the foundation for the original no-scheme rule, the opposite outcome has been reached in very similar circumstances. This is true, in particular, for both \emph{Cedars} (1914) and \emph{Fraser} (1917), two important Canadian cases concerning expropriation for hydro-power, cited both by the Law Commission and the House of Lords in \emph{Waters}.\footnote{\emph{Cedars Rapids Manufacturing and Power Co v Lacoste}, [1914] AC 569 and \emph{Fraser v City of Fraserville} [1917] AC 187.} In \emph{Fraser}, it was the waterfalls themselves that were subject to expropriation, yet the Privy Council still found that the value of the potential for hydro-power exploitation of these falls should be disregarded when compensating them, following a standard "value to the owner" approach. Reasoning along the same lines is, as we will see later, prevalent in Norwegian law, although with some significant caveats suggesting the problematic nature of this line of reasoning. Indeed, it would appear most problematic also in light of \emph{Waters}, raising the question of the current status of the Canadian cases.

In any event, they can serve as great examples of the type of situation where the need for a distinction between commercial and non-commercial aspects arise most forcefully. It seems, in particular, that there can be no doubt that the energy inherent in water pre-exists any scheme seeking to harness it. Moreover, it seems clear that energy has value, and so, the conclusion would have to be that also the value of a waterfall pre-exists any scheme for hydro-power exploitation. However, we can then refine our approach by asking: what \emph{kind} of value is it? This, indeed, might be the solution to our troubles. 

For it seems that any value resulting in compensation to the owner must by the nature of things either be \emph{personal}, related to claims for disturbance etc, or else \emph{commercial}, namely such a kind of value that can be realized by a company or an individual operating for profit -- possibly the owner himself, possibly some buyer of his property. A different kind of value altogether is the \emph{public value}, which can not be realized for profit by \emph{anyone}. 

The distinction between commercial and public value is, obviously, down to a political decision, and it can hardly be regarded as permanent. Moreover, it can often be difficult to assess where the line is to be drawn, especially in cases when public/private partnerships cooperate to provide public services. Nevertheless, it seems perfectly legitimate to make this distinction, and it seems like it can be very helpful in many cases. For instance, even if the public value of hydro-power pre-exists the hydro-power scheme, this does \emph{not} mean that there is any pre-existent commercial value in hydro-power. That, in particular, depends entirely on whether or not the public has settled on a regulatory regime that allows commercial exploitation. On the other hand, once a decision to allow commercial exploitation has been made, it seems quite reasonable to apply the "pre-existence" test used in \emph{Bocardo}: An owner should always be compensated for the value of any pre-existent \emph{commercial} value that his property has.\footnote{Certainly, a clarification along these line would not resolve all issues. It would not, for instance, offer any conclusive guidance with respect to the specific issues related to "key value" raised in \emph{Bocardo}.} 

To conclude, we would like to remark that unlike problems relating to the scope of the scheme, the question of what commercial value can be said to pre-exist a scheme might turn rather more on facts than on law. It seems, in particular, that \emph{this} is a question that it is not so easy, or even desirable, to attempt to resolve by legislation or a fixed set of principles. It seems quite clear, in particular, that in order to answer the question of what should be counted as a pre-existing commercial value, one must take a broad look at the prevailing regulatory regime. Moreover, one must expect that the correct assessment of this question will depend on the context of regulation, in particular the extent to which the State \emph{allows} the disputed value to be commercially realized. The law relating to compensation should be such that it can tolerate significant changes in these parameters, and it seems therefore that the important legal question in this regard is to provide a sound conceptual foundation for making a sound assessment across a range of different scenarios. Moreover, it seems that the courts, in light also of human rights law, has an important supervisory role to play in this regard.

In the next section, we will address the no-scheme rule in Norwegian law, and as we will see, the distinction between commercial and public value is rarely made in general compensation law. This seems unfortunate, and as we will see, it makes the special rules adopted for waterfalls appear as something of an enigma in Norwegian expropriation law.

\subsection{The no-scheme rule in Norwegian law}\label{sec:nonor}

Before 1973, the Norwegian law relating to compensation for expropriation of property was based on case-law. The courts would interpret Section 105 in the Norwegian constitution which demands that "full" compensation is to be paid. A no-scheme rule was typically applied, such that when assessing the value of the property, the element of compulsion was disregarded, and changes in value that could be attributed to the underlying scheme tended not to be taken into account. As in the UK, difficult questions would arise for comprehensive schemes based on public plans for the use of the property, and it proved difficult to identify any clear rule concerning the distinction between the scheme itself and the regulation of property-use that preceded it.\footnote{References.}

Following the Second World War, there was an increasing trend that effects of regulation \emph{would} be taken into account, but only with regards to the \emph{positive} aspect of the no-scheme rule. That is, the scheme was taken into account in so far as it could be used to argue against alternative development, but not in such a way that it could lead to an increase in the value of the property. In some cases, even the regulation directly preceding, and providing the basis for, the use of compulsion, would be taken into account.\footnote{Such as in Rt. 1970 s. 1028, where a property which had been used by a local business owner was expropriated to implement a public plan that regulated the property for use as a public road with parking spaces. The owner was not compensated for the loss of business revenue, since, according to the majority in the Supreme Court, the regulation of the property had to be taken into account (the decision was given under dissent). The case was somewhat special, however, since the business value could be realized by the owner only if he had been given the opportunity to rebuild his store, following a fire. Moreover, he was already ensured compensation based on the value of the property as a plot for housing, for independent reasons.}

The general picture, however, was that a no-scheme rule applied to underlying regulation of property use as long as there was a causal link between this regulation and the subsequent expropriation.\footnote{References. \noo{Husaas-komiteen}} To apply this in concrete cases often proved problematic, however, as illustrated by the many conflicting opinions voiced about the current law during the preparation of the original Compensation Act from 1973.\footnote{See, for instance, the historical overview given in NOU 2003:29}

In the 1973 Act, a radical rule was put in place to resolve all outstanding issues: valuation should be based on the  \emph{existing use} of the property at the time of expropriation.\footnote{So the Norwegian law mirrored the rule introduced in the UK Town and Country Planning Act 1947 which was later replaced by the current Land Compensation Act 1961.} The rule went further than the no-scheme rule in that it prescribed that compensation should disregard \emph{all} kinds of hypothetical development of the property, notwithstanding their status with respect to existing plans and regulations. But it also involved a break with it, since, on the face of it, the implication would be that any kind of regulation predating the scheme would be taken into account when it provided the basis for the "existing use".

However, in Section 4, nr. 3 of the Act, this aspect of the "existing use" principle was limited by a \emph{separate} provision implementing the \emph{negative} aspect of the no-scheme rule; the value of existing use due to public regulation underlying the expropriation should be \emph{deducted} from the compensation payment. As such, the Compensation Act 1973 implemented a system whereby the positive part of the no-scheme rule would be given a very narrow interpretation -- any scheme or regulation limiting current use was to be be taken into account -- while the negative aspect was explicitly provided for in statute -- the value of existing use that could be attributed to public regulation underlying the scheme was to be deducted.

This new rule was quite controversial, and to make the system more flexible, the 1973 Act included a rule that allowed the Lands Tribunal to increase compensation, on a discretionary basis, by taking into account that value of comparable properties in the district where expropriation took place. Still, property owners felt that the new Act went too far in depriving them of the right to compensation, and the matter came before the Supreme Court in \emph{Kløfta}.\footnote{Rt. 1976 s. 1} After deliberating in plenum, the Court presented a revisionary interpretation of the new Act, essentially judging the intention behind the main rule as being incompatible with the protection of property encoded in the Norwegian Constitution, Section 105. The main step taken by the Court was to regard the discretionary increase of compensation as a \emph{mandatory} step, one that had to be carried out whenever certain conditions were fulfilled. What exactly these conditions amounted to, and how they should be interpreted, was not conclusively resolved, however. Indeed, the confusion that arose after \emph{Kløfta} led to a heated academic debate in Norway, and a long line of Supreme Court cases has since attempted to clarify the current state of the law. \footnote{References.}

In 1984, taking into account the ruling in \emph{Kløfta}, a new Compensation Act was passed, which is still in force today.\footnote{Act No. 17 of 06. April 1984 relating to Compensation following Expropriation of Real Property} According to Section 4 of the Compensation Act, compensation is to be calculated as the highest of either the value of the property as it could be put to use by the owner, his \emph{value of use}, or the \emph{market value}, the payment he could expect to receive from a typical willing buyer. In both cases, a no-scheme rule applies: in Section 5, Paragraph 3, it is stated that when calculating the market value, changes in value due to the scheme is to be disregarded, while in Section 6, it is stated that the value of use should be based on \emph{foreseeable} use of the property. In practice, this has been interpreted as referring to such use as it is reasonable to expect would have occurred in the absence of the scheme.\footnote{References.} 

However, the spirit of the 1973 Act is still clearly felt in Norwegian law, and the no-scheme rule is thought of somewhat differently than in the UK. Firstly, it is common to distinguish more sharply between the positive and the negative aspects of the rule, and unlike in the UK, much, if not most, attention has been devoted to the former aspect, when the scheme is used to justify decreased levels of compensation. Secondly, and specifically as it relates to the positive aspect of the rule, the tendency has been to give "the scheme" a narrow interpretation, regarding public plans and regulation as binding for valuation, even when they are intimately related to the undertaking for which a right to expropriate is granted. Simultaneously, if the plan leads to an increased value that \emph{can not be realized by the current owner}, a "value to the owner" principle typically applies directly, such that this value is not compensated, irrespectively of what the scheme is taken to be.\footnote{References.}

As we mentioned, much attention in Norway has been directed at the positive aspect of the no-scheme rule, and there are some important exceptions to the main principle of regarding public plans as binding for the evaluation. The main exception is that a plan tends to be disregarded when it has no other purpose than to facilitate the scheme for which expropriation is needed. The important precedent in this regard is the influential Supreme Court case of \emph{Lena}.\footnote{Rt. 1996 s. 521.} However, the line of reasoning adopted by the Supreme Court in this case has so far been called on almost exclusively in cases when compulsory acquisition takes place to implement plans for public buildings or other kinds of public installations, like playgrounds or parking spaces.

In case of expropriation taking place to implement such plans, if a valuation were to be made on the basis of the use prescribed by the plan itself, one would expect the market value of the property to come out as nil or close to nil, such that one would be forced to return to existing use as the basis for valuation.\footnote{In fact, a logical continuation of the line of reasoning prescribed by the main rule would suggest that even existing use would be inadmissible as a basis for compensation since continuation of this use would not be in accordance with the plan, and hence unforeseeable. However, this has not, as far as we are aware, ever been argued.} It is not hard to understand how this could come to be felt as unfair to property owners. Moreover, it seems that it would easily come to run counter to Section 105 of the Norwegian Constitution. In some sense, it would represent a watering down of this provision, allowing the State to deprive property owners of value by using unfavorable regulation as an explicit means to later acquire properties cheaply by use of expropriation.

On the other hand, it can be argued that the question raised by cases such as these is not really a question of compensation for expropriation, but rather a question of whether or not property owners should have a claim of compensation for losses incurred due to \emph{regulation}. This is not generally granted under Norwegian law, and so the special rules that entitles the owner to such compensation in cases of regulation leading to expropriation can be seen as insufficiently justified within the broader context of Norwegian planning law. Indeed, some scholars have voiced this opinion forcefully, and the current state of the law is unclear at best, with the special rule introduced in \emph{Lena} being hard to apply to other cases, and giving rise to further Supreme Court decisions attempting to map out in more detail when exactly they come into play. Moreover, attempts to reform the law on this point have so far stranded in controversy.\footnote{NOU 2003:29 and further references.}

It seems to us, however, that while this debate is interesting, the truly fundamental questions about the current state of Norwegian law do not arise in this regard, but with regards to the other principle we mentioned, namely that no compensation is offered for value that \emph{the owner can not realize}.\footnote{By "realize" here, we mean realizable either as the owner's "value of use" or else realizable by selling the property at "market value", as prescribed by the Compensation Act.} This rule seemingly applies without reservation, and, as in the UK context, it appears like a logical consequence of the \emph{value to the owner} principle. However, as we argued for in the general discussion on the no-scheme rule in Section \ref{sec:noscheme}, it seems pertinent to distinguish between the \emph{subjective} and \emph{objective} aspect of this principle. In particular, if the owner can not realize the value because the value is not of a \emph{kind} that is available for commercial realization, for instance because it only represents non-commercial value to the public, then the rule appears easy to defend along traditional lines. On the other hand, if the owner can not realize the value because the State desires that \emph{someone else} be allowed to realize it, then the principle appears highly problematic.

While the general debate on Norwegian compensation law has completely neglected to consider this aspect, it features extensively, albeit implicitly, in one particular branch, namely the law relating to compensation for waterfalls. In the remainder of this paper, we turn to this particular area of Norwegian law, offering a detailed analysis of the problems that arise, and how they severely challenge the traditional "value to the owner" reasoning about compensation for commercial undertakings. 

\section{The traditional method for compensating waterfalls}\label{sec:trad}

In the early 1900s, Norwegian hydro-power was not subjected to much regulation, and waterfalls, having recently been discovered as an important supply of cheap electricity for industrial exploits, were rapidly falling into the hand of foreign speculators. In response to this, Norwegian politicians introduced legislation to secure national interests, the main provision being that concession from the state was made a requirement for anyone who wanted to acquire a waterfall.\footnote{References.} As a result, the market for waterfalls in Norway dwindled and the State assumed control of hydro-power exploitation. Unlike private investors, the State would tend to expropriate waterfalls rather than acquire them through voluntary agreements, and the question arose as to how the original owner should be compensated. This question, if resolved by a standard no-scheme approach, could easily prove shockingly unfair to owners of waterfalls. Presumably, since waterfalls could not be exploited for any significant commercial gain except through hydro-power exploitation, disregarding the hydro-power scheme when calculating compensation could lead to nil or close to nil being awarded to the owner. But this was not seen as an acceptable outcome, and instead the Norwegian courts introduced a special method to compensate waterfalls that gave the owner a \emph{share in the value of the hydro-power scheme} for which expropriation was taking place.

Norway did not at this time have any legislation specifically aimed at regulating compensation following expropriation, and when formulating the special rules for compensation of waterfalls, the Norwegian courts seems to have relied on an analogical application of the gross valuation techniques introduced in the Industrial Concession Act 1917 and the Watercourse Regulation Act 1917.\footnote{Act No. 17 of 14 December 1917 relating to Regulations of Watercourses and Act No. 16 of 14 December 1917 relating to Acquisition of Waterfalls, Mines and other Real Property}. Neither of these acts were aimed at compensating owners, but they relied on methods for assessing the potential and significance of hydro-power projects with respect to the question of whether or not a special concession from the State was required.\footnote{To acquire the waterfall and the right to regulate the water-flow respectively.} In effect, by relying on the methods of valuation introduced there, the compensation mechanism that was introduced deviated completely from the "value to the owner" principle. On the other hand, it also closely mimicked the manner in which owners of waterfalls would be compensated on the market in the early days, prior to the introduction of our concession laws, when speculators would pay for waterfalls on the basis of what they assumed to get out of them in subsequent hydro-power projects.\footnote{References.}

In the Supreme Court case of \emph{Hellandsfoss}, some 80 years after it was first introduced, the traditional method for compensation was still in use, and the Court described it as follows, starting from the observation that the general principles that were later encoded in the Compensation Act 1984 were of little use for determining the right level of compensation for waterfalls (my translation).\footnote{Rt. 1997 s. 1594.} 
\begin{quote}
The principle set out in the Compensation Act, Section 5, is that compensation should be determined on the basis of an estimation of what ordinary buyers would pay for the property in a voluntary sale, taking into account such use of the property as could reasonably be anticipated. For waterfalls, however, this often offers little guidance, and the value of waterfall rights have traditionally been determined based on the number of natural horsepowers in the fall, which are then multiplied by a price per unit. The unit price is determined after an overall assessment of the waterfall, including the cost of the scheme, its location, and levels of compensation paid for similar types of waterfalls in the past. The number of natural horsepowers is calculated by the formula "natural horsepower = $13.33 \ \times \ Qreg \ \times \ H$", where $Qreg$ is the regulated water flow and $Hbr$ is the height of the waterfall.
\end{quote}

In this formula, $Qreg$ represents a quantity of water, measured in cubic metres per second (m3/sec), while $H$ is the height of the waterfall measured in meters. Horsepower is an old-fashioned measure of effect, and in the standard account of the traditional method, it is said that the number of natural horsepowers in a waterfall is a measure of gross effect in the waterfall, giving us the amount of “raw” water-power in the waterfall.\footnote{See \cite{Falk}(in Norwegian)} This, however, is flat out false for most waterfalls, and it has always been more accurate to regard the number of natural horsepowers as a measure of the \emph{level of regulation} involved in a given planned hydro-power project. This, indeed, is what the concept was actually introduced to measure, and it is how it is used in the Industrial Concession Act 1917 and the Watercourse Regulation Act 1917. Historically, however, it made  sense to conflate the energy-potential of the waterfall with the level of planned regulation, since regulation of water-flow used to be crucial for the development of efficient hydro-power generation. This has changed, however, and the traditional method, when applied as a tool to assess the energy-potential of a waterfall, is horribly outdated. In the following subsection, we give a detailed presentation showing this.

\subsection{Not so natural: The physics and the law behind the notion of a "natural horsepower"}\label{subsec:notnat}

Let us first remark that horsepower is no longer used as a measure of effect in the energy business. Today, it is general practice to use kilowatts (kW) instead, at least as long as there are not any lawyers present.\footnote{1 Kilowatt(Kw) = 1000 Watt(W)} In the following, we will give the reader quite a detailed presentation of the concept of effect and the link between the two units horsepower and kilowatt. We shall try to give a rudimentary explanation of the physical facts which underlie it, leading to the conclusion that the number of natural horsepowers in a waterfall no longer has any relevance to its value in hydro-power production.

\noo{
8    “Erstatning for erverv av fallretter” (Compensation for Acquisition of Waterfall Rights) by Ulf 
        Larsen Karoline Lund and Stein Erik Stinesen in Tidsskrift for Eiendomsrett (Journal of Property 
         Righs) Nu 4 2006
9      See paragraph 3.5 in Larsen/Lund/Stinson (above)
10    See for instance p 262 in “Vassdrag og Energirett” (Law of Waterfalls and Energy) by
        Falkanger/Haagensen (Ed.) 2002. Referred to as “type” because there are several definitions of Q,
        see paragraph 1.4.
}

The notion of an \emph{Effect} ($E$) is defined in physics by means of the more elementary concepts of \emph{Work} ($Wr$), \emph{Time} ($t$), \emph{Distance} ($d$) and \emph{Force} ($F$). The relationship between them, in particular, satisfies the following equation.

\begin{equation}\label{eq:effect}
E = Wr/t = (F \times d)/t
\end{equation}

The formal notion of Work ($Wr$), in turn, is defined as follows.

\begin{equation}\label{eq:work}
Wr = F \times d
\end{equation}

The last non-trivial concept needed to define effect is the notion of Force ($F$), which is fundamental in physics. It can be explained by what is needed to change the speed of an object, or cause it to move, c.f. the First Law of Newton. The unit most commonly used to denote this is \emph{Newton}. One Newton is defined as the force which is needed to increase the speed of a mass of 1 Kilogram (Kg) by 1 Meter (m) per sec in one second. 

It is the force of gravity which is harnessed to produce an electric effect, and in turn, to harness the power of water. The force of gravity is $9,81$ Newton per Kg mass. That is, if 1 Kg falls freely to the ground it will accelerate by $9,81 \ m$ per sec in a second (in reality somewhat less because of air resistance). The power of gravity works continuously. Therefore, the mass will accelerate by $9,81$ Newton as long as it is falling. This means that for every second the mass is falling the speed will increase by $9,81 \ m/sec$.

Historically, the unit of force was often defined as the force needed to pull 1 Kg to the ground. This force was often called 1 “Kilo” and named “Kilopond” (Kp) to distinguish it from the concept of mass which was also, in everyday usage, often referred to as “Kilo” (but should actually be called "Kilogram”). When Newton is used as a unit for force, the corresponding unit of work becomes Newton $\times$ meter ($Nm$), also called Joule ($J$). Looking to the definition of work in Equation \ref{eq:work}, we can then make the following calculation: When a mass of 1 Kg falls 1 Meter, the work done amounts to $9,81 \ Nm$. Similarly, when Kilopond is used as a unit for force, you get Kilopond $\times$ meter ($Kpm$) as a unit for work. The work which is done when 1 Kg mass is falling 1 meter can then be described by the following formula:

\begin{equation}\label{eq:work}
Wr = 1 \ Kp \times 1 \ M = 1 \ Kpm 
\end{equation}

When we look at formula \ref{eq:effect} for effect, if we use Newton as the unit of force, we get \emph{Watt} ($W$) as the corresponding unit of effect, so that we get an effect of $9,81 \ Nm/Sec$, or $9,81 \ W$, when 1 Kg mass falls $1$ meter in a second. Similarly, if we use $Kp$ as the unit of force, we get an Effect of $1 \ Kpm/Sec$. Somewhat curiously, $1 \ Kpm/Sec$ does not equal $1$ Horsepower ($Hp$). Instead, the choice was made to define horsepower as follows
\begin{equation}\label{eq:hp}
1 \ Hp = 75 \ Kpm/Sec
\end{equation}
Consequently, we get the following relation between $Hp$ and $W$ as a measure of effect:

\begin{equation}\label{eq:hpw}
1 \ Hp = 75 \ Kpm /Sec = 9,81 \times 75 \ W = 735,75 \ W = 0,736 \ kW
\end{equation}

This might seem like an unduly technical exercise for a law paper. However, it is needed to understand that there is nothing magical about a Natural Horsepower ($nat.Hp$), and that horsepower, which is no longer used by the energy business as a measure of effect (or the general public, save for car enthusiasts), can easily be replaced by Watts. This, indeed would make more sense in this day and age, and open the method up to be more readily scrutinized. Indeed, moving from a quantity of $x \ nat.Hp$ to the same amount of effect, measured in nat.kW is easy; the latter is obtained from the former when multiplying by $0.736$, i.e., such that 
\begin{equation}\label{eq:natkw}
x \ nat.HP = 0.736 \times x \ nat.kW
\end{equation}

We are in a position to properly explain the formula for natural horsepowers in a waterfall. It measures the effect in the waterfall in Hp, given certain information and certain assumptions about the features of the waterfall under consideration. When we go through this in detail, however, we come to realize that the horsepowers arrived at are not so natural after all, since they are based on assumptions that are largely irrelevant for modern hydro-electric schemes, and that have been upheld in law seemingly for no other reason than the fact that they have been habitually used by lawyers and judges with no regards to, or understanding of, their underlying \emph{meaning}.

In light of what we have already seen, it is now easy to explain the constant factor of "13.33", used in the formula for natural horsepowers: Since one cubic meter ($m3$) of water ($1000 \ l$) has a mass of $1000 \ kg$, it is pulled towards the center of the earth at a force of $1000 \ Kp$. If we lift, or allow to fall, $1 \ m3$ of water by $1 m$ in one second, we work with an effect, or, in the case of falling, \emph{release} an effect, of $1000 \ Kpm /sec$. From the formulas in Equations (\ref{eq:effect}-\ref{eq:hp}), we then see that we work with, or release, an effect which measured in Hp amounts to the following.

\begin{equation}\label{eq:whp}
1000 \ Kpm /Sec : (75 \ Kpm /Sec) /Hp = 13,33 \ Hp
\end{equation}

That is, $13,33$ is simply the effect, measured in $Hp$, of $1 m3$ of water falling $1 m$ in $1 sec$. So if we have an amount of water measured in $Q \ m3$, we must then multiply $13,33$ by $Q$ to get the effect of this amount of water falling $1 \ m/sec$. Then, if we have a waterfall that is $H$ meters high, we can multiply by $H$ to find the effect of $Q \ m3$ of water falling $H m$ in $1 sec$. If the amount of water flowing through a river is Q \ m3/sec, it means that we have the amount of water $Q$ available every second. This will in turn ensure that as long as the water supply continues to be $Q \ m3/sec$ -- that is, as long as the water-supply is \emph{constant} -- the effect of the work being done by the water will also be constant and it will be given by the formula $13,33 \ x \ Q \ x \ H$.

Energy is defined as the capacity to do work. There are several units for energy, but when energy takes the form of electricity, such as in a hydro-power plant, it is standard to use the unit kWh (kW $\times$ Hour). Notice that this is consistent with the definition of work as $Wr = E \times t$ ("Energy times Time"). If you have a certain effect available over time, the amount of energy you acquire is measured by multiplying the effect by the amount of time that the effect is operative. When mechanical energy is transformed into electrical energy in a power station, the effect of the generator is multiplied by the time the generator is operative with the same effect. To get the result in $kWh$, all you must do is to ensure that you measure your time in hours and your effect in kilowatts, and then multiply the two together. 

However, in practice, the effect harnessed in a hydro-power station changes when there are changed in the water flow, so you get the true amount of energy produced only when you make this calculation sufficiently often, by multiplying a given effect with the number of hours (maybe just minutes) for which the station is operative at this particular effect. The sum of these chunks of energy you get over the year will be the amount of annual production, and this, today, is what the energy business use as a yardstick when \emph{they} assess the value of a waterfall. 

With modern technology, the energy output is registered by fine-tuned electrical equipment, maybe every 15 minutes or so, and hence the annual amount of energy generated can easily be registered and measured, even if there are significant fluctuations in the available water, leading to the generator operating at different levels of effect. This is a significant observation, since it means that the assumption inherent in the natural horsepower formula, namely that the water-flow, and hence the effect, remains constant, is no longer tenable, and gives a completely erroneous account of the energy-potential in a waterfall. 

Certainly, the amount of energy generated in a power plant could be measured in other units than kWh, e.g. in terms of the amount of horsepower-hours per year. However, an energy producer gets paid for the amount of energy he can deliver, \emph{not} the effect he can maintain in his station constantly, and as a result it simply would not correspond to reality if one would attempt to measure the energy by \emph{natural} horsepower-hours. This would only be correct if the owner of the hydro-power plant \emph{chose} to produce electricity at a constant effect all year round, which he would never do.\footnote{In addition, and pulling somewhat in the opposite direction, come the fact that it is not a realizable effect that is derived from the formula of natural horsepower, but only a gross estimate. The effect that we find is calculated based on an assumption of ideal circumstances, i.e. without any loss of energy in the production step. In reality, there will always be some loss of energy both in the pipes and in the turbine/generator. That is the reason why natural horsepower is often described as the “raw” or as the “gross effect”. But a far more important mechanism is the gross simplifications involved in moving from the physical fact that the flow of water in a river varies quite a lot during a year, to one fixed amount, Qreg, assumed to be available constantly.}
This makes it practically meaningless to talk about the number of natural horsepower in a waterfall as a measure of its potential. Effect is not something we have in a waterfall, but something we get when a certain amount of water is falling. 

From this observation also follows that the amount of water, $Q = Qreg$, that is to be put to use in the formula for natural horsepower must be chosen by an application of \emph{law}. Indeed, what to choose for $Qreg$ is a legal question of great significance, and as long as $Qreg$ is conceived of as a constant, as in the traditional method, it measure the degree of \emph{regulation of water-flow}, not the potential for energy generation.

So how is $Qreg$ typically defined? This is actually a very tricky question, although apparently, the amount of water to be used in the calculation is actually prescribed by statute. In practice, however, the statutory definition can lead to such offensive results, when applied in the context of compensation, that the courts, or, as it were, the experts presenting these calculations on behalf of the expropriating party, usually adopts a \emph{different} definition. 

The Watercourse Regulation Act 1917, Section 2 reads as follows.

\begin{quote}
Section 2:  Waterfall regulation for the production of electric energy which increases hydro power:
\begin{itemize}
\item[a)]            by at least 500 natural horsepower in one or several waterfalls which can be developed 
                 collectedly, or

\item [b)]             by at least 3.000 natural horsepowers in the whole watercourse, or

\item [c)]              which alone or together with earlier regulations significantly affects the environmental
                 conditions or other public interests, can only be exploited by the State or  
               a developer who obtains permission from the King.

\end{itemize}

If regulation of a watercourse increases the water-power in the river by at least 20.000 natural horsepowers, or if there are essential conflicting interests, then the case should be submitted to Parliament before license is given, unless the Department finds it unnecessary.

The increase of the hydro-power according to the first and second point is calculated on the basis of the increase of the low water-flow of the watercourse, which the regulation is supposed to cause beyond the water-flow which is considered foreseeable for 350 days a year. When making the calculation it is to be assumed that the regulation is operated in such a way that the water-flow during the low water periods becomes as even and regular as possible.  
\end{quote}

In the third paragraph, the definition of $Qreg$ is provided, when it states that the \emph{increase of the hydro-power}, measured in natural horsepower, is to be calculated based on the water-flow which it is foreseeable that will be present for at least 350 days a year. That is, $Qreg$ is to be taken as the maximum amount of water that one can expect to be present for at least 350 days of the year after regulation minus the water that could be expected for 350 days without regulation, which is the quantity referred to as the \emph{low water-flow}.

Regulation of a watercourse can involve building a reservoir and/or installations that transfer water from one river to another. Then, if there is excess water, for instance due to flooding, water can be stored in the dam for later use, while if there is drought, the stored water can be released. In this way it becomes possible to even out the water-flow over the year. This again means that the water which is guaranteed to be present for at least 350 days a year will typically increase. In light of the definition of $Qreg$, it is clear that the definition of natural horsepower depends crucially on the level of regulation involved in the planned hydro-power project. But the definition only takes into account the gross effect resulting from the \emph{increase} in low water-flow following regulation. It follows that if the planned project does not involve regulation, which is common today, especially for small-scale hydro-power, the number $Qreg$ will by necessity be $0$ and the waterfall will be deemed not to posses any natural horsepowers at all.

In fact, if the traditional method for calculating compensation had remained true to the wording of the Watercourse Regulation Act, things could sometimes have been even worse for the owner. This we notice, in particular, when we  consult Section 10 of the Water Resources Act 2000.\footnote{Act No. 82 of 24 November 2000 relating to River Systems and Groundwater} Here, the NVE is given the power to compel the owner of a hydro-power scheme to ensure that a certain quantity of water is always allowed to pass through the intake of the hydro-power plant. Moreover, there is nothing to prevent the NVE from demanding that this minimum water-flow is set \emph{higher} than the low water-flow, and this, indeed, is often the case, especially in cases when the low water-flow only amounts to a small fraction of the average water-flow, and environmental concerns arise with respect to wildlife and fisheries. Then, indeed, it appears that the minimum water-flow, required to be left untouched, should be subtracted from the regulated water-flow when calculating $Qreg$.\footnote{In fact, this was done in the case of \emph{Sauda}, LG-2007-176723 (Gulating Lagmannsrett, a regional Court of Appeal} Intuitively, this even appear reasonable; The minimum water-flow can not, as a matter of fact, be harnessed for energy production.

However, for hydro-power projects that do not involve regulation, this would then lead to the regulated low water-flow being \emph{less than} the low water-flow. Subtracting then the latter from the first, as required by Section 2 of the Watercourse Regulation Act 2000, would lead to a \emph{negative} number for $Qreg$, and a corresponding \emph{negative} number of natural horsepowers attributed to the waterfall. Logically speaking, then, the traditional method would entail awarding a negative sum as compensation, compelling the owner to pay the expropriating party for taking over his waterfall!

In practice, of course, the traditional method has never been applied in this way, and more generally, the definition in Section 2 of the Watercourse Regulation Act has tended to be completely disregarded by valuers using the traditional method for calculating compensation. Instead, the definition has been changed for this purpose, such that the low water-flow prior to regulation is not deducted from the low water-flow after regulation.

Even after this modification, the number of natural horsepowers give a drastically skewed picture of the potential of the waterfall, especially for projects that do not involve regulation. It is not unusual, in particular, especially not for waterfalls suitable for small-scale hydro-power, that the low water-flow amounts to only about 3-5 \% of the average water-supply. In modern hydro-power projects, one would expect 70-80 \% of this water-flow to be harnessed for energy production even in the absence of any regulation. So in these cases, the traditional method of compensation is effectively based on compensating the owner for only a small fraction of the energy that can actually be extracted from his waterfall.

This observation, which follows from elementary facts about physics and contemporary hydro-power production, was not noted or discussed in connection with the principles used for compensation before the early 2000s, when the issue was raised following the growing interest in small-scale hydro-power. However, the Norwegian government has certainly been aware of these facts, as illustrated for instance by the following passage, taken from a report presented to parliament in 1991-1992.\footnote{Ot.prp. No 50 (1991-1992) p 19, discussing the notion of natural horsepower in connection to the uses made of that term in other parts of the law.}

\begin{quote}
The Department of Oil and Energy have considered moving a proposition for changing the hydrological definitions in the Industrial Concession Act 1917 and the Watercourse Regulation Act 1917. Today the act uses a calculation method based on an increase in regulated water-flow, i.e. that of natural horsepower.[.......] The hydrological definitions of these acts, supposed to indicate how much electricity can be generated, were made on the basis of technical and operative conditions differing very much from contemporary circumstances. In implementing the definitions referred to above one has tried to adapt to the new technological realities of the present day. Therefore, in practice, a calculation based on current production is used instead. From several quarters, particularly the Association of Waterfall Regulators, there has been raised a strong wish to authorize this practice by altering the definitions of the relevant laws. The Department of Oil and Energy agree, but have not as yet made a sufficient elucidation of the issues to be able to move a proposition of alteration of these acts.
\end{quote}

Within the ranks of the water authorities, it has actually been well-known for decades that the notion of a natural horsepower fails to give an adequate picture of the potential that a waterfall has for hydro-power. The development of new technology had made this apparent already in the 1950's, when it was also raised as an issue, specifically with respect to compensation following expropriation, by a director at the NVE, who commented, in 1957, that he failed to see how the traditional method could be an adequate means for valuating waterfalls.\footnote{See \cite{....}. The director even went as far as to illustrate a different method, which would also be outdated given today's regulatory regime, but which would reflect contemporary \emph{actual} valuations, used by the NVE itself.}

Considering the physics behind the traditional method is enough to reveal that it fails to give rise to valuations that reflect the value of waterfalls, under any reasonable set of assumptions about the correct general compensation principles one should adopt. Firstly, the traditional method, by relying on data from the expropriating party's project, deviates from the "value to the owner" principle. Secondly, and even more importantly, it amounts to compensating the owner based only on the level of regulation, which is not only mostly irrelevant to the value, but is also the one aspect of the scheme which can not readily be traced to properties of the waterfall, but depends rather crucial on the investment decisions made by the expropriating party. While a case can be made that any extra power harnessed by regulation should \emph{also} be compensated, for instance if it can be established that the expropriating party is not the only one who could have regulated the waterfall, it seems rather perverse to \emph{only} compensate the owner based on this parameter.

However, while the idea of compensating the owner of waterfalls by a price per natural horsepower is fundamentally flawed already at the level of physics, there are even more serious concerns that arise when one begins to consider the way in which the \emph{unit price} has been determined, and the effects this has had on the level of compensation payments. In case-law based on the traditional method, it is often said that the price set per natural horsepower is set according to "market price" for waterfalls, but for the most part, what this means is that the court looks to prices awarded in earlier compensation cases, not to prices obtained in voluntary sales.

This, in turn, gives rise to a price level that is entirely artificial, reflecting, more than anything else, the power balance between buyer and seller in the courtroom, and not any genuine market value. Indeed, while the prices paid did see some increase during the 80 years that the traditional method was used, this hardly reflected the general increase in value of hydro-power, nor did it reflect the general level of inflation.\footnote{References needed.} Moreover, and particularly worrying, while the price-level was determined by the courts, there were also some cases of voluntary agreements that used the same method, and could thus be used to justify is status as a genuine market-based valuation principle. In fact, as late as in 2002, a waterfall belonging to local landowners in the rural community of Måren, in Western Norway, was sold for the sum of kr 45 000 (roughly £ 5000), based on traditional calculations. The waterfall has now been exploited in a small-scale hydro-power plant belonging to the large energy company BKK, with annual energy output of 21 GWh.\footnote{$http://www.bkk.no/om_oss/anlegg-utbygging/Kraftverk_og_vassdrag/andre-vassdrag/article29899.ece$} For comparison, we mention that in the case of \emph{Sauda}, based on the new method for calculating compensation, the owners received a compensation which totaled about 1 kr/kWh annual production.\footnote{LG-2007-176723 (I acted as council for some of the owners in this case).} Applied to the Måren case, this would take the compensation from kr 45 000 to kr 21 000 000, that is, almost 500 times more.\footnote{In fact, the Måren waterfalls were cheaper to exploit, so in reality, one would expect that the new method applied to Måren would yield even greater compensation per kWh. We also remark that the value awarded in \emph{Sauda} was market-value, not value of use, since it was assumed that the owners would have to cooperate with a so-called "professional" energy company to develop hydro-power. This, in effect, halved the compensation awarded.}

The case of Måren is somewhat extreme, but in no way unique.\footnote{I should assemble a list probably....} Moreover, it illustrates an important point, namely that when the traditional method was used, and described as the "market value" of waterfalls by the courts, this became a self-fulfilling prophecy in many cases. The prices paid in voluntary transactions were influences by the practice adopted by the courts far more than the other way around. This, indeed, appears to be a general danger in cases when expropriation is widely used for the purpose of commercial development. Then, it seems, prices paid can easily be kept artificially low by developers turning to the use of expropriation as soon as they threaten to rise, and relying on the "market value" thus established when arguing in court for the appropriateness of those compensation levels that so benefits them commercially. That this mechanism can be severe is nicely illustrated by the case of Norwegian waterfalls, and how to prevent it is, in our opinion, a main challenge that is likely to arise in any regulatory system that aims to make extensive use of expropriation to further economic development.

\section{The new method}\label{sec:new}

Following the liberalization of the Norwegian energy sector in the 1990s, the traditional method came under increasing pressure. It was argued to be unjust by owners who felt that they were being deprived of a valuable commercial assets, and it was held to be illogical by engineers working on developing small-scale hydro-power.\footnote{References} Eventually, legal professionals followed suit, and came to the realization that established rules based on market value could now be applied. Indeed, a new market for waterfalls had begun to develop at this point, following the increased interest in small-scale hydro-power and the formation of new companies specializing in cooperating with local owners. For transactions of rights to waterfalls taking place in this market, the traditional method of valuation was not used, and waterfalls were rarely sold at all, but rather leased to the development company for an annual fee. Typically, this fee was calculated by fixing a percentage of the energy produced during the year, and compensating the owners of the waterfall by multiplying this with the market price for electricity obtained throughout the year, possibly deducting production specific taxes, but with no deduction of other cost. In effect, owners would get a fee corresponding to a set percentage of annual gross income in the hydro-power plant.\footnote{References}

Usually, the fee entitles the owners to 10-20 \% of the income from sale of electricity, depending on the cost of the project. Moreover, it is common that the owners are entitled to up to 50 \% of the income derived from so-called \emph{green certificates}, a support mechanism for new renewable energy projects, corresponding to the Renewables Obligation in the UK.\footnote{See http://www.ofgem.gov.uk/Sustainability/Environment/RenewablObl/ for further details.} Essentially, and somewhat simplified, the scheme allows the energy producer to collect a premium on his sale of electricity, which, owning to its "green" status, is valued more highly by buyers (usually electricity suppliers), who are required to ensure that a certain proportion of the energy they offer to their customers (usually consumers, like you and me) is considered green. In Norway, such a scheme has been talked about for years, but was only put in force in 2012.\footnote{http://www.regjeringen.no/en/dep/oed/Subject/energy-in-norway/electricity-certificates.html?id=517462} Currently, energy producers can claim a premium of about 2 pp per KWh per year, meaning that about a third of the annual income for new renewable energy projects comes from the sale of green certificates.\footnote{While the premium must be expected to go down somewhat as the certificate market matures and more energy producers acquire "green" status, it will certainly remain an important source of extra income for renewable energy producers also in the future.}

In light of the fact that the agreements on the new market are based on leases that tie compensation to the fate of the particular hydro-power project that is being undertaken, several questions arise when attempting to value waterfalls by looking to this market. If a given project has been identified as providing the basis for valuation, the task is difficult, but mainly a question of factual assessment. The valuer have to determine first what the annual production would be and also determine the costs of carrying out the project. Then, on this basis, he must move on to determine what the annual fee would be, and then, in order to complete the process, he must stipulate what the price of electricity, and of green certificates, is likely to be for the next 20 years or so. On the basis of this information, it becomes possible to determine the annual income to the owner of the waterfall over a period of 20 years, and then one would also have a basis upon which to calculate a reasonable present-day value of the scheme to the owner of the waterfall. 

Indeed, this is the model that has been used in the cases that have been before the courts and where the traditional method has not been adopted. The first such case was \emph{Møllen}, and while the Supreme Court rejected the method as it was applied in this case, because it was found that the date of valuation was to be based on prices obtained for waterfalls in the 1960s, they commented that they supported the adoption of the new method in cases when \emph{alternative} small-scale development was deemed a \emph{foreseeable} use of the waterfall in the absence of the scheme.\footnote{Rt. 2008 s. 82.} 
Since \emph{Møllen} the new method has been used in many cases before the lower courts and the Lands Tribunal.\footnote{See for instance \cite{tf1,tf2,tf3}, a series of academic papers discussing the new method (in Norwegian).}

Unsurprisingly, the new method tends to lead to a rather protracted process of valuation, mostly dominated by experts. Moreover, given all the uncertain elements of the calculation, it is typical that the opposing parties produce expert witnesses that diverge significantly in their valuations. While this is problematic enough, the fundamental \emph{legal} challenge arises with respect to the choice made about what scheme the compensation should be based on. This becomes especially tricky if one attempts to follow a standard no-scheme approach. In the following, we summarize the main issues that arise.

\begin{enumerate}
\item In the absence of the hydro-power scheme benefiting from expropriation, is it foreseeable that the waterfall would nevertheless be used in a hydro-power project?
\item If the answer to Question (1) is yes, what would such a foreseeable project look like?
\item Is it foreseeable that an alternative project would get planning permission?
\item Does the no-scheme rule imply that the project benefiting from expropriation cannot be regarded as the foreseeable use for the purpose of compensation?
\item Can the fact that the scheme underlying expropriation obtained planning permission be taken as evidence to support that alternative uses of the waterfall would not be given planning permission?
\item How should compensation be calculated if it is determined that no alternative project is foreseeable? 
\end{enumerate}

In some cases, for instance when the project benefiting from expropriation is not commercially viable but is carried out for public purposes with the help of special State funding, the answer to Question (1) might be no. However in most cases, the question will be answered in the affirmative, since the scheme benefiting from expropriation already serves as an indication that the waterfall can be harnessed for energy. However, here the no-scheme rule comes into play and creates severe difficulty once we reach Question (2). For what kind of scheme can be assumed foreseeable all the while we are obliged to disregard the scheme underlying expropriation? In most cases that have come before the courts so far, the owners have claimed that alternative development in small-scale hydro-power should serve as the basis for compensation, and such cases the problem of the no-scheme rule appears to have been circumvented. This, however, is not necessarily the case. It appears, in particular, that the answer to Question (3), asking about the likelihood of planning permission, might again depend on how you view the no-scheme rule. It seems, in particular, that anyone who answers Question (5) in the affirmative, looking to the planning permission actually given to the expropriating scheme for evidence, might be inclined to say that the alternative project could not expect to get planning permission, and that this is so \emph{because} planning permission was granted to the expropriating party (or this line of reasoning might be sugarcoated by pointing to whatever underlying reasons the authorities had for considering the scheme underlying expropriation the optimal use of the waterfall).

Then the question arises: Is someone who reasons like this at odds with the no-scheme rule? It would seem so, but remember the earlier discussion on the no-scheme rule in Norwegian law, where we noted that the rule has tended to be applied much more narrowly along its positive dimension. Following up on this, it can be argued, in keeping with the general tendency in how the law is applied in Norway, that while the expropriation scheme is to be disregarded for the purpose of compensation valuation, the regulation underlying the scheme -- or at least the rationale behind this regulation -- is nevertheless to be taken into account. If this point of view is taken, then the conclusion can easily become that alternative development is to be regarded as unforeseeable, and that the reason why this must be the case is precisely the fact that the expropriation scheme received planning permission. Indeed, this line of reasoning was given a stamp of approval in the recent Supreme Court case of \emph{Otra II}. Here, the presiding judge made the following remarks, quoting Gulating Lagmannsrett (the regional Court of Appeal), expressing his support, and adding a few comments of his own.

\begin{quote}
"[....] The Court of Appeal finds it difficult to distinguish this case from other cases when it has been established that alternative development is not foreseeable. It does not seem relevant whether this is the case because the alternative is not commercially viable or because the alternative must yield to a different exploitation of the waterfall" 
I agree with the Court of Appeal, and I would like to add the following: As the survey of the general principles have shown, it is assumed, both in the Expropriation Act, Sections 5 and 6, and in case-law, that only the value of a foreseeable alternative should be compensated. This starting point means that it would be in breach of the general arrangement if a waterfall that can not be used in foreseeable small-scale hydro-power was to be compensated as if it could be put to such use.
\end{quote}

Having used the planning permission granted to the expropriating party as evidence that alternative development was unforeseeable, the Court needed to answer Question (6) by coming up with some alternative way of compensating the owners.  To do so, the Court also had to answer Question (4), however, asking whether or not the scheme underlying expropriation could be taken into account at this stage. Moreover, one would think, given how the expropriation scheme was used as an argument when answering Question (5) in the negative, that it \emph{could} be taken into account. This, however, was answered in the negative by the Supreme Court in \emph{Otra II}, where the presiding judge reasoned as follows.

\begin{quote}
Based on the arguments presented to the Supreme Court, I find it safe to assume that there does not today exist any market for the sale and leasing of waterfalls for which alternative development is not foreseeable, but where the waterfalls can be used in more complex hydro-power schemes. The appellants have not been able to produce documents or prices to document the existence of such a market
\end{quote}

Let us first remark that it is hard to imagine how a market such as that asked for here could ever develop, all the while alternative buyers, by the courts own reasoning, are excluded from being taken into account. In this case, if there was to be a market, it would have to be down either to the regular benevolence of the expropriating parties in cases like these, or to the government compelling them to enter friendly negotiations. In the Norwegian context, both seem unlikely. More important, however, is the implicit assumption that in order to value the waterfall according to its potential for hydro-power production, a market needs to be identified. It is \emph{not} considered sufficient that the scheme for which expropriation takes place is itself a hydro-power project, on the basis of which the value of the waterfall could be assessed following exactly the same steps as in the new method.

In fact, the Supreme Court's reasoning in \emph{Otra II} serve as an excellent example of the type of reasoning that makes the no-scheme rule highly problematic for cases of expropriation that benefit commercial schemes. Indeed, it seems to follow that the scheme itself should not provide a basis for calculating the compensation, but then, on other hand, it also appears that there really is no other way to calculate it, seeing as the State, by giving permission to carry out the scheme, have effectively acted in such a way as to make an alternative use \emph{of the same kind} seem inherently unforeseeable. This is not so much of a problem when the use that the owner could make of the property has a different character than the scheme, since in this case, if we disregard the scheme, this other use might seem foreseeable. However, when the alternative use of the property is of \emph{exactly the same kind} as the use made of it by the scheme, it does seem counterintuitive, as noted by the Court of Appeal in \emph{Otra II}, to regard it as foreseeable, all the while the scheme will tend to appear the more rational form of exploitation.

It seems, however, that when taken to its logical conclusion, this line of reasoning, based on the no-scheme rule, leads to an offensive results; the commercial value of the property is not to be compensated because the optimal commercial use of the property is the use that the expropriating party aims to make of it in the scheme underlying expropriation. Note that the conclusion is not just that this optimal value, inherent in the scheme, should not be compensated. No, the conclusion in \emph{Otra II} was that \emph{no} compensation could be estimated for any use of the same \emph{kind}, since such use was not foreseeable, owing to the absence of a market.

It is certainly possible to argue that this decision represent a misguided application of the no-scheme rule. In effect, the Supreme Court allowed the planning permission given to the expropriating party to act as evidence that alternative development was unforeseeable, while it used the no-scheme rule to argue that the hydro-power scheme for which this planning permission was given could not itself form basis for compensation payment based on market value.  On the other hand, it seems that even if we disregard the scheme completely, it is unnatural to base the compensation payment on the value of a hydro-power scheme that is less beneficial, both commercially and in terms of resource efficiency, than the scheme for which expropriation takes place. Such a scheme would not, one must presume, \emph{actually} have been carried out, regardless of the questions of whether or not it would have been given planning permission. But it does seem particularly difficult, intuitively speaking, to predict what use would have been made of the waterfall if these were the facts: more or less exactly the same scheme as that underlying expropriation would be implemented, but by some from of voluntary agreement with the owners, not by means of expropriation. 

In \emph{Otra II}, however, this line of thought was also rejected, although this was, in part, due to the point not having been argued before the Court of Appeal. But then the question arose as to how exactly compensation should be calculated. The answer, following up on the "value to the owner" principle so forcefully adopted elsewhere in the judgment, would appear to be that no compensation should be paid at all, save perhaps for loss of fishing rights and the like. This, however, was \emph{not} the conclusion reached by the Supreme Court. Instead, the Court states that a return to the traditional method is in order. However, they do not apply it in the traditional way. Rather, they casually sanction the replacement of regulated low water-flow in the definition of $Qreg$ by the average water-flow, thus moving away from compensation based on the level of regulation, to compensation based on average effect in the project. Moreover, they also sanctioned the use of a significantly increased unit price compared to earlier times.

What to make of this? It seems hard indeed to make sense of indeed, since effectively, by relying on the traditional method, the Supreme Court contradicts its own conclusion that compensations should be based on market value. Instead, they rely on a method that, in effect, is based on an attempt to quantify the value of the waterfall as it is being used by the expropriating party in his project. However, by relying on a technical method that has been completely outdated, and have lived its own life in the courts, it becomes difficult to assess the outcome properly, at least for a non-expert. It seems, in particular, that the Supreme Court prefers the obscurity of the traditional method, and its status as an established principle, over the possibly radical conclusion that, in cases such as this, it simply is not tenable to adopt the "value to the owner" principle, as least not as construed in Norwegian law.

Indeed, it is simple enough to be critical of the Supreme Court based on the fact that they regarded alternative development as unforeseeable in this case, when the planning permission granted to the expropriating scheme itself appears to have provided the decisive bit of evidence. Still, it is not possible to escape the fact that this reflects a general tendency in Norwegian law, and so, even if it appears unreasonable, it might very well be a correct application of national law. Moreover, it could very well have been that alternative development was unforeseeable for \emph{some other reason}, for instance because the only commercially viable exploitation was the scheme planned by the expropriating party. In this case, the problem of how to compensate the owners in the absence of an alternative form of exploitation would still arise. It is this question, in particular, which seems entirely unsatisfactorily resolved under an application of a "value to the owner" principle.

This is witnessed by \emph{Otra II}, and, in fact, it appears that the Supreme Courts decision \emph{not} to follow their own reasoning to its logical consequence is the main lesson to be learned from the case. For all intents and purposes, the Supreme Court \emph{rejects} the "value to the owner" principle, but they obscure this by wrapping it up in the traditional method, which is deeply flawed. However, the problem it attempts to solve appears significant, and it pertains directly to the question discussed more generally in Section \ref{sec:noscheme}, namely how to apply the "value to the owner" principle with respect to commercial schemes. It seems that even the fiercest supporters of limiting owners' right to compensation tend to find it too offensive to apply this principle when it leaves the owners with no form of compensation for giving up property to multi-million, purely commercial undertakings. Indeed, such a practice would surely also be in breach of the human rights law. It seems, in particular, that the subjective aspect of the "value to the owner" principle is impossible to maintain. Indeed, if the commercial value falls to be disregarded for no other reason than the fact that the State happens to have granted planning permission to the expropriating party rather than the owner, this is not only dubious with respect to human rights protecting property, but also appears to be a case of \emph{discrimination}, e.g., as prohibited by ECHR Article 14.

The problem does not arise when the buyer sees value in the property that is of a different \emph{kind} than that realizable by \emph{any} private owner. In this case, the rule simply states that the owner should not be able to demand that "public value" is transformed into commercial value just for him. This appears like a reasonable principle. But when there is commercial value already present on the "public" side of the transaction, it seems completely unwarranted that the public should be allowed to transfer this value from the owner to someone else without compensation. Thus, it seems that more accurately and acceptably, the "value to the owner" principle should be thought of as a "commercial value" principle. It seems, in particular, that the principle need to be stripped of any suggestion that a preferential financial position is to be awarded to whoever benefits from expropriation.\footnote{Exceptions might be possible to imagine, but, one would think, only when they can be construed as falling under the "public value" banner in some way.}

It seems unfortunate that this aspect has not been made explicit, and the difficulties that arise in the absence of this nuance seems nicely illustrated by the case of Norwegian waterfalls. Still, as the case of \emph{Otra II} seems to indicate, an interpretation of the "value to the owner" principle along less offensive lines is in reality already in place with regards to Norwegian hydro-power. Here, it seems that "value to the owner" has in fact \emph{never} been applied in the traditional way. Hopefully, rather than obscuring this fact by relying on an unsatisfactory and artificial method for calculating the compensation, the future will see further developments that recognize the need for new principles. It should be recognized, in particular, that as the law has been applied for the last 80 years, despite its grave flaws and injustices, there has always been an implicit recognition in Norwegian law that the owners of waterfalls are \emph{entitled to their share} of the commercial benefits of hydro-power. 

In fact, in the recent Supreme Court case of \emph{Kløvtveit}, a further illustration of this is found. The conclusion here was also that alternative development was not foreseeable. However, unlike in \emph{Otra II}, the Court of Appeal had compensated the owners based on the fact that they regarded it foreseeable that in the absence of the scheme, the waterfalls would have been exploited in exactly the same way, except that it would have happened in the form of \emph{cooperation} between the owners and the expropriating party. By this line of reasoning, the Court effectively seems to have adopted a more rational "commercial value" principle, to replace the traditional method. 

Indeed, is it not always the case, at least under objective standards of assessment, that when alternative development is unforeseeable, then a rational alternative buyer -- assumed to operate in a world where there are no "powers of compulsion", to paraphrase Lord Nicholls in \emph{Waters} -- would look precisely to the likely possibility of cooperating with the expropriating party? This, on the other hand, would \emph{never} be a safe assumption to make for non-commercial aspects which, in the absence of commercial potential would not give an alternative buyer financial incentive to do so.

We mention that \emph{Kløvtveit} was discussed in \emph{Otra II}, and that the presiding judge made some reflections, focusing on what he regarded to be "practical problems" with cooperation. However, this was not crucial to the decision, since the cooperation model was not argued for by council. In light of this, one can only hope that \emph{Kløvtveit}, rather than \emph{Otra II}, will become the influential precedent for future cases.

\section{Conclusion and future work}\label{sec:conc}

