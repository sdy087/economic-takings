%input macros (i.e. write your own macros file called MacroFile1.tex)
\include{Macros/MacroFile1}
%\includeonly{Chapter5/endelig_5}
%NOTE: if you want to work on just one Chapter, you can take out the `%' sign on the previous line and compile the thesis accordingly. The above command, for instance, will give you just the first Chapter. The bonus of doing it this way is that your cross references and page numbers will remain as they are in the full file.

\documentclass[a4paper,twoside,openright,10pt]{thesisPSnPDF}

\usepackage[utf8]{inputenc}
\usepackage{babel}
%\usepackage{nomencl}
%\DeclareUnicodeCharacter{00A0}{ }
%\makenomenclature

%\newcommand{\isr}[1]{#1}
\newcommand{\dni}{\DNI}
\newcommand{\noo}[1]{}
\newcommand{\sjur}[1]{SJUR: #1}
\newcommand{\nathp}[1]{NatHp(#1)}

\def\signed #1{{\leavevmode\unskip\nobreak\hfil\penalty50\hskip2em
  \hbox{}\nobreak\hfil(#1)%
  \parfillskip=0pt \finalhyphendemerits=0 \endgraf}}

\newsavebox\mybox
\newenvironment{aquote}[1]
  {\savebox\mybox{#1}\begin{quote}}
  {\signed{\usebox\mybox}\end{quote}}

\usepackage{titlesec}
\titleformat{\chapter}[hang]
  {\normalfont\huge\bfseries\centering}{\thechapter}{20pt}{\Huge}

\addbibresource{thesis.bib}

% turn of those nasty overfull and underfull hboxes
\hbadness=10000
\hfuzz=50pt

% Put all the style files you want in the directory StyleFiles and usepackage like this:
%\usepackage{StyleFiles/watermark}

%The following indexes are to ensure the table of cases functions properly. You can leave this to one side for now, though it is worth learning early on how to make the table of cases. It is pretty easy; but it'd be a shame if it got to near submission and you couldn't figure out how to do it. 
% NB: I haven't provided for Northern Irish cases here
\makeindex[name=casesgb, title={England and Wales}, columns=1,intoc]
\makeindex[name=casessc, title={Scotland}, columns=1,intoc]
\makeindex[name=casesus, title={The United States}, columns=1,intoc]
\makeindex[name=casesnz, title={New Zealand}, columns=1,intoc]
\makeindex[name=casesau, title={Australia}, columns=1,intoc]
\makeindex[name=casesca, title={Canada}, columns=1,intoc]
\makeindex[name=legis, title={United Kingdom}, columns=1,intoc]
\makeindex[name=casesother, title={Norway}, columns=1,intoc]
\makeindex[name=casesechr, title={European Court of Human Rights}, columns=1,intoc]
\makeindex[name=casesjunk, title = {Other Jurisdictions}, columns = 1, intoc]
\makeindex[name=noleg, title = {Norway}, columns = 1, intoc]
\makeindex[name=prepno, title = {Norway}, columns = 1, intoc]
\makeindex[name=intleg, title = {Other Law Instruments}, columns = 1, intoc]
\DeclareIndexAssociation{gbcases}{casesgb}% ENGLAND
\DeclareIndexAssociation{sccases}{casessc}% SCOTLAND
\DeclareIndexAssociation{aucases}{casesau}% AUSTRALIA
\DeclareIndexAssociation{cacases}{casesca}% CANADA
\DeclareIndexAssociation{nzcases}{caseszn}% NEW ZEALAND
\DeclareIndexAssociation{uscases}{casesus}% UNITED STATES
\DeclareIndexAssociation{eucases}{casesother}% EU
\DeclareIndexAssociation{echrcases}{casesechr}% ECHR
\DeclareIndexAssociation{pilcases}{casesother}%
\DeclareIndexAssociation{othercases}{casesother}% ANYTHING ELSE
%\DeclareIndexAssociation{gbprimleg}{legis}% LEGISLATION
%\DeclareIndexAssociation{gbsecleg}{legis}% LEGISLATION
\DeclareIndexAssociation{enprimleg}{legis}% LEGISLATION

\indexsetup{level=\section*,toclevel=section,noclearpage}

\DeclareBibliographyCategory{cited}
\AtEveryCitekey{\addtocategory{cited}{\thefield{entrykey}}}

\usepackage{calc}
\usepackage{lipsum}
\makeatletter
\newcommand{\tocfill}{\cleaders\hbox{$\m@th \mkern\@dotsep mu . \mkern\@dotsep mu$}\hfill}
\makeatother
\newcommand{\abbrlabel}[1]{\makebox[3cm][l]{\textbf{#1}\ \tocfill}}
\newenvironment{abbreviations}{\begin{list}{}{\renewcommand{\makelabel}{\abbrlabel}%
        \setlength{\labelwidth}{3cm}\setlength{\leftmargin}{\labelwidth+\labelsep}%
                                              \setlength{\itemsep}{0pt}}}{\end{list}}


\begin{document}
\renewcommand\baselinestretch{1.5}
\baselineskip=24pt

%\maketitle

\begin{titlepage}

\begin{center}



\vspace*{\fill}
\centering

{\Huge\textsc{On the legitimacy of economic development takings}}\\[3cm]


{\large Sjur Kristoffer Dyrkolbotn}\\

\large {Thesis submitted for the degree of Doctor of Philosophy, to the School of Law at Durham University} \\

%\emph{{Your College}}\\
\vspace*{\fill}

 

\vfill

{\Large \today}\\
{99 638 words excluding bibliography}

\end{center}

\end{titlepage}


%set the number of sectioning levels that get number and appear in the contents
\setcounter{secnumdepth}{4}
\setcounter{tocdepth}{2}

%\frontmatter

%\addcontentsline{toc}{chapter}{List of Abbreviations}

\tableofcontents

%
\begin{center}
\vspace{4cm}
I hereby certify that this thesis is the result of my own work except where otherwise indicated and due acknowledgement is given.
\vspace{1cm}

I also certify that this thesis is XXXXX words long excluding the bibliography.\\

\vspace{4cm}


\begin{tabular}{lr}
& DATE OF SUBMISSION \\
& \\
SIGNED & DATE \\
\end{tabular}


\end{center}


% ----------------------------------------------------------------------


%%% Local Variables: 
%%% mode: latex
%%% TeX-master: "../thesis"
%%% End: 

%\addcontentsline{toc}{chapter}{Abstract}

% Thesis Abstract -----------------------------------------------------

% NOTE: As with acknowledgements, I had to create a new format for this -- I couldn't get the original one to work. As with the acknowledgements, if you are able to fix the code so it's less messy, do pass the fix back to the Law Faculty.


%\begin{abstractslong}    %uncommenting this line, gives a different abstract heading
%\begin{abstracts}        %this creates the heading for the abstract page

\begin{quoting}
  \singlespace
    \begin{center}
  {\LARGE \bfseries  On the Legitimacy of Economic Development Takings }\\
  \vspace*{0.5cm}
      {\large Sjur Kristoffer Dyrkolbotn}\\
  \vspace*{0.1cm}  
      {\large \emph{Ustinov College}}\\
  \vspace*{0.2cm}  
    {\normalsize Thesis submitted to the School of Law at Durham University for the degree of Doctor of Philosophy}

  \vspace*{0.2cm}  
    {\normalsize \today}\\
  \vspace*{0.5cm}  
    {\normalsize \bfseries Abstract}      
  \end{center}
  {\parindent0pt
For most governments, facilitating economic growth is a top priority. Sometimes, in their pursuit of this objective, governments interfere with private property. Often, they do so by indirect means, for instance through their power to regulate permitted land uses or by adjusting the tax code. However, many governments are also prepared to use their power of eminent domain in the pursuit of economic development. That is, they sometimes compel private owners to give up their property to make way for a new owner that is expected to put the property to a more economically profitable use. %This new owner is sometimes the government itself, represented by one of its administrative bodies. But in many cases it will be a private company, operating for profit, possibly in cooperation with government entities through some form of public-private partnership.
}
\vspace{0.7mm}

This thesis asks how the law should respond to government actions of this kind, often referred to as {\it economic development takings}. The thesis makes two main contributions in this regard. First, in Part I, it proposes a theoretical foundation for reasoning about the legitimacy of economic development takings, including an assessment of standards for judicial review. The thesis goes on to link the legitimacy question to the work done by Elinor Ostrom and others on sustainable management of common pool resources. Specifically, it is argued that institutions for local self-governance that treat development potentials as common pool resources can often undercut arguments in favour of using eminent domain for economic development.

Then, in Part II, the thesis puts the theory to the test by considering takings of property for hydropower development in Norway. It is argued that current eminent domain practices appear illegitimate, according to the normative theory developed in Part I. At the same time, the Norwegian system of land consolidation offers an alternative to eminent domain that is already being used extensively to facilitate community-led hydropower projects. The thesis investigates this as an example of how to design self-governance arrangements to increase the democratic legitimacy of decision-making regarding property and economic development.

%This shows how local governance arrangements can work, suggesting that more attention should be devoted to studying the nexus between property, common pool resource management, and eminent domain.

%This theoretical basis is formulated independently of specific jurisdictions, but based on considering existing approaches to the legitimacy question from England and Wales, the United States, and at the European Court of Human Rights. In addition, the thesis draws a link between the legitimacy question and the work done by Elinor Ostrom and others on sustainable management of common pool resources. Specifically, it is argued that institutions for local self-governance that treat development potentials as common pool resources can often undercut arguments in favour of using eminent domain for economic development.

%Such rules are in place in most developed countries, and the fundamental status of property has been expressed explicitly in both the US constitution and the European Convention of Human Rights. The tension between these provisions and the practice of taking property for economic development, in many cases for commercial profit, is clear and worth considering further.

%\vspace{0.7mm}

%The second part of the thesis puts the theory developed in the first part to the test by considering takings for hydropower development in Norway. Under Norwegian law, the right to exploit the hydropower in most streams and rivers belong to the riparian owners. That is, the right to the hydropower belongs to the people who own the land over which the water flows, usually local community members. To acquire these rights, energy companies tend to rely on the government's power of eminent domain. Recently, however, local communities have begun to protest this practice, by arguing that they should be allowed to take a more active role in managing their own resources. This has resulted in tensions in Norway, shedding light on the legitimacy question as it arises in the context of Norwegian expropriation law. In addition, new light has been shed on the role of the so-called land consolidation courts, which are now increasingly asked to deliver alternatives to eminent domain in hydropower cases. The thesis investigates this in depth and argues that the unique system of land consolidation found in Norway demonstrates how to design self-governance arrangements that can increase the democratic legitimacy of decision-making regarding property and economic development.

\end{quoting}


%\end{abstracts}
%\end{abstractslong}


% ----------------------------------------------------------------------


%%% Local Variables: 
%%% mode: latex
%%% TeX-master: "../thesis"
%%% End: 


% Thesis Acknowledgements ------------------------------------------------


%\begin{acknowledgementslong} %uncommenting this line, gives a different acknowledgements heading
%\begin{acknowledgements}      %this creates the heading for the acknowlegments

\begin{quoting}
  \singlespace
    \begin{center}
  {\LARGE \bfseries  Acknowledgements}\\
  \vspace*{0.5cm}
  \end{center}
\noindent
I want to thank all these people.

\end{quoting}


%\end{acknowledgements}
%\end{acknowledgmentslong}

% ------------------------------------------------------------------------

%%% Local Variables: 
%%% mode: latex
%%% TeX-master: "../thesis"
%%% End: 


% A note on the format. I could not get the acknowledgements macro to work properly, despite some effort, and so I redesigned it rather messily, as you will see above. If you enter text, it will work -- but the more technologically competent among you will probably be able to fix it. If and when that is done, if you could send the resultant document back to the Law Faculty to update it, that would be great. 


\section*{List of Abbreviations}
\addcontentsline{toc}{chapter}{List of Abbreviations}

\begin{abbreviations}
\item[CPO]{Compulsory Purchase Order}
\item[CPR]{Common Pool Resource}
\item[ECHR]{European Convention of Human Rights}
\item[ECtHR]{European Court of Human Rights}
\item[EEA]{European Economic Area}
\item[EFTA Court]{Court of Justice of the European Free Trade Association States}
\item[ICCPR]{International Covenant on Civil and Political Rights}
\item[ICESCR]{International Covenant on Economic, Social and Cultural Rights}
\item[LAD]{Land Assembly District}
\item[NVE]{Norges Vassdrags- og Energidirektorat (Norwegian Water and Energy \linebreak Directorate)}
\item[P1(1)]{Article 1 of the First Protocol to the European Convention of Human Rights}
\item[UDHR]{Universal Declaration of Human Rights}
\end{abbreviations}

%% this section includes various indexes/tables of cases and legislation

\chapter*{Cases Cited}
\addcontentsline{toc}{chapter}{Cases Cited}
\markboth{CASES CITED}{CASES CITED}

\printindexearly[casesgb]% ENGLAND & WALES
\printindexearly[casessc]% SCOTLAND (GB too, of course, but ...)
\printindexearly[casesau]% AUSTRALIA
\printindexearly[casesnz]% NEW ZEALAND
\printindexearly[casesca]% CANADA
\printindexearly[casesus]% UNITED STATES
\printindexearly[casesother]% OTHERS

\chapter*{Legislation Cited}
\addcontentsline{toc}{chapter}{Legislation Cited}
\markboth{LEGISLATION CITED}{LEGISLATION CITED}

\printindexearly[legis]% ALL LEGISLATION

% NOTE:
% To generate the indexes properly you need to run the following commands (in a terminal shell -- you need to navigate to the Thesis file in terminal. There will be guidance online for how to navigate within Terminal. Otherwise, most scientists should be able to help you!):
% splitindex -- thesis -s oscola (THIS IS THE COMMAND WHICH WORKS FOR ME)
% splitindex thesis --s oscola thesis (TRY THIS IF THE FIRST ONE DOESN'T WORK)

%\addcontentsline{toc}{chapter}{Abstract}
%\listoffigures

%\include{LawTable/LawTable}
%\listoftables
%\addcontentsline{toc}{chapter}{List of Tables}
%\printglossary  %% Print the nomenclature
%\addcontentsline{toc}{chapter}{Nomenclature}

%\renewcommand{\nomname}{Abbreviations}
%\printglossary

%\mainmatter
%\include{Chapter1/introduction}
%\include{Chapter1/Chapter1_final}
\chapter{Introduction and Summary of Main Themes}\label{chap:1}

\begin{quote} \small
Thieves respect property. They merely wish the property to become their property that they may more perfectly respect it.\footnote{\cite[58]{chesterton08}.}
\end{quote}
\begin{quote} \small
[Granting] a takings power, then, may not be viewed as an act that wrenches away property rights and places an asset outside the world of property protection. Rather, it may be seen as an act within the larger super-structure of property.\footnote{\cite[583]{bell09}.}
\end{quote}
%
%A human being needs only a small plot of ground on which to be happy, and even less to lie beneath. %\footnote{Johan Wolfgang von Goethe, {\it The sorrows of young Werther and selected writings}.}
%\end{quote}
%“That's what makes it ours - being born on it, working on it, dying on it. That makes ownership, not a %paper with numbers on it.”
%― John Steinbeck, The Grapes of Wrath
%
%\cite{waring11} (``It is a testament to the elasticity of the concept of
%property that it is able to represent all things to all people, and to accommodate so many conflicting %calls''); 
%

Property can be an elusive concept, especially to property lawyers.\footnote{See, e.g., \cite[225-226]{waring09} (``It is a testament to the elasticity of the concept of 
property that it is able to represent all things to all people, and to accommodate so many conflicting calls.''); \cite[252]{gray91} (``But then, just as the desired object comes finally within reach, just as the notion of property seems reassuringly three-dimensional, the phantom figure dances away through our fingers and dissolves into a formless void.'').} Indeed, in legal language, the word itself often only functions as a metaphor -- an imprecise shorthand that refers to a complex and diverse web of doctrines, rules, and practices, each pertaining to different ``sticks'' in a bundle of rights.\footnote{See generally \cite{grey80,klein11}.} Should we conclude that property as a unifying term is lost to the law? It certainly seems hard to pin it down. In the words of Kevin Gray, when a close scrutiny of property law gets under way, property itself seems like it ``vanishes into thin air''.\footnote{See \cite[306-307]{gray91}. See also \cite[81]{grey80} (arguing that the eventual consequence of the bundle view is that property will cease to be an important category for legal and political reasoning).}

Arguably, however, property never truly disappears.\footnote{See \cite[159]{gray94} (``We are continually prompted by stringent, albeit intuitive, perceptions of 'belonging'.'').} Indeed, there is empirical evidence to suggest that humans come to the world with an innate concept of property, one which pre-exists any particular arrangements used to distribute it or mould it as a legal category.\footnote{See \cite{stake06}.} Specifically, humans and a seemingly select group of other animals appear to have an intuitive ability to recognise {\it thievery}, the taking of property (not necessarily one's own) by someone who is not entitled to do so.\footnote{See \cite[11-13]{brosnan11}; \cite[159]{gray94}.}

%\footnote{See \cite[...]{gray94} (``In this context we are still not far removed from the
%primitive, instinctive cries of identification which resound in the
%playgroup or playground:
%'That's not yours; it's mine.''').}

Taken in this light, Proudhon's famous dictum, ``property is theft'', might be more than a seemingly contradictory comment on the origins of inequality.\footnote{For Proudhon's theory of property generally, distinguishing between {\it de facto} possession and {\it de jure} property (regarded as theft), see \cite{strong14}.} It might point to a deeply rooted aspect of property itself, namely its role as an anchor for the distinction between legitimate and illegitimate acts of taking.

%Therefore, it also seems that the following becomes a key question in property theory: how should we think about takings, and when are they legitimate?

In this thesis, I will study takings of a special kind, namely those that are sanctioned by a government in the pursuit of some public use or interest. Specifically, the word {\it taking} will be used to refer to an exercise of the government's power of eminent domain.\footnote{See generally \cite{stoebuck72} (clarifying the status of eminent domain from a US perspective, tracing its roots back to early civil law writers such as Grotius and Bynkershoek). Takings will also be referred to as expropriations, especially in the context of Norwegian law. In England and Wales, the corresponding notion is that of compulsory purchase.} In legal language, especially in the US, takings by eminent domain are often referred to as takings {\it simpliciter}, while other kinds of ``takings'' require further qualification, e.g., in case of ``takings'' based on contract, taxation, or adverse possession. The US terminology is intuitive and helps bring the issue of legitimacy to the forefront, so I will adopt it throughout this thesis.\footnote{Sometimes it is convenient to draw up the notion of a taking more widely than I do in this thesis, e.g., to include adverse possession, see \cite[19-21]{waring09}. However, such a broader notion will not be used in this thesis. This choice appears natural in light of how the thesis focuses specifically on takings motivated by a government's desire to facilitate concrete economic development projects.}

When guided by eminent domain, the taking of private property without the owners' consent is not theft. But it is not necessarily that far removed from it either; the default assumption is that takings are legitimate, but if they are not, one may well be permitted to call them by a different name.\footnote{See \cite[8-10]{gray11} (discussing case law from the US, with state judges describing illegitimate takings as ``plunder'', ``rapine'', and ``robbery'').}

More generally, the idea that the government's power to take is not unlimited seems fundamental. Indeed, the expectation that an owner might find occasion to resist an act of taking, and may or may not have good grounds for doing so, appears deeply rooted in pre-legal intuitions.\footnote{See, e.g., \cite[159]{gray94}.} This raises the question of how to approach the legitimacy of takings in legal reasoning and what conceptual categories we can benefit from when doing so. This is the key question that is addressed in this thesis, for the special case of so-called {\it economic development takings}.\footnote{For a sample of scholarship based on this term, see \cite{cohen06,somin07,wilt09,yellin11}.}

Such takings occur when a government uses the power of eminent domain to stimulate economic growth, typically by providing property to for-profit companies. The canonical US example is {\it Kelo v City of New London}, which resulted in great controversy and a surge of academic work on the legitimacy of takings.\footcite{kelo05}

The {\it Kelo} case concerned several homes that were taken by the government in order to accommodate private enterprise, namely the construction of new research facilities for Pfizer, the multi-national pharmaceutical company. Several home-owners, among them Suzanne Kelo, protested the taking on the basis that it served no public use and was therefore illegitimate under the Fifth Amendment of the US Constitution. The Supreme Court eventually rejected their arguments, but this decision created a backlash that appears to be unique in the history of US jurisprudence.\footnote{See generally \cite{somin08}.}

In their mutual condemnation of the {\it Kelo} decision, commentators from very different ideological backgrounds came together in a shared scepticism towards the legitimacy of economic development takings.\footnote{See \cite[1413-1415]{bell06} (``Everyone hates {\it Kelo}'', commenting on how criticism was harsh from across the political spectrum).} Interestingly, their scepticism lacked a clear foundation in US law at the time, as the {\it Kelo} decision did not appear particularly controversial in light of established eminent domain doctrines.\footnote{See, e.g., \cite[1418]{bell06} (``The most astounding feature of {\it Kelo}, as even the case's harshest critics agree, is that from a legal standpoint, the ruling broke no new ground.'').} Hence, when the response was overwhelmingly negative, from both sides of the political spectrum, it seems that people were responding to a deeper notion of what counts as legitimate.

%Indeed, the critical response to {\it Kelo} appears to have been a reflection of widely shared sentiments. As such, it also arguably involved pre-legal notions pertaining to legitimacy. Simply stated, people from across the political spectrum simply found the outcome {\it unfair}.

If the law is meant to deliver justice, widely shared intuitions about legitimacy deserve attention from legal scholars. In the US, legitimacy intuitions pertaining to economic development takings have indeed received plenty of attention after {\it Kelo}. Despite the outcome of the case, it is now hard to deny that cases such as {\it Kelo} belong to a separate category of takings that raises special legal questions.\footnote{See, e.g., \cite{cohen06,somin07}.} As this change in the narrative was largely the result of a popular movement, there is reason to think that the category of economic development takings is a powerful addition to the discourse on legitimacy, potentially relevant also outside of the US.

%When cases like {\it Kelo} are portrayed as being primarily about bestowing a benefit on powerful commercial interests, it becomes natural to question their legitimacy, irrespective of details in the surrounding legal framework. But when is it appropriate to deride economic development takings in this way? Moreover, should the law provide a basis for the courts to intervene, to rescind illegitimate takings?%Both of these questions will be addressed in this thesis. To address them effectively, 
To explore this further, it should first be acknowledged that there is a {\it risk} that takings for economic development can be improperly influenced by commercial, rather than public, interests. This risk is clearly higher in economic development situations than in cases when takings take place to benefit a concretely identified public interest, such as the building of a new school or a public road. Hence, it is intuitively reasonable to single out economic development takings for special attention at the political and normative level. However, should the categorisation also be recognised as a basis for justiciable restrictions on the takings power?

This is not obvious, as it conflicts with the prevalent idea that governments enjoy a ``wide margin of appreciation'' when it comes to their use of eminent domain.\footnote{This expression has been used by the European Court of Human Rights, see \cite[54]{james86}. In the US, the same attitude was clearly a factor motivating the majority in \cite{kelo05}.} However, as the US debate shows, it might be hard to deny judicial review as soon as the special features of economic development takings are brought into focus. This points to the first main theme of this thesis: an analysis of economic development takings as a conceptual category for legal reasoning about property protection.

\section{Property Theory and Economic Development Takings}\label{sec:1:1}

In Part I, this thesis will argue that economic development takings should be recognised as a distinct category of takings at the theoretical level, with respect to fundamental rules that protect private property. This claim will be made on the basis of a theory of property that is broader than typical approaches found in the law and in legal scholarship. Specifically, the thesis rejects the view that private property should be understood as a form of entitlement protection.\footnote{For a famous entitlements-based view of private property, see \cite{calabresi72}.}

Instead, chapter \ref{chap:2} will argue for a social function understanding, with an emphasis on human flourishing as the normative foundation for private property.\footnote{For human flourishing theories of property generally, see \cite[chapter 5]{alexander10}.} In short, property should be protected because it can help people flourish, as members of a democratic society.\footnote{See also \cite[1089]{crawford11}.} Moreover, property is meant to serve this function not only for the owners themselves, but also for other members of their communities.\footnote{See generally \cite{gray94,alexander09d,alexander14}.}

Such an ambitious take on property must necessarily also give rise to a broader assessment of legitimacy when the state interferes with it.\footnote{See also \cite{underkuffler06}.} This is what inspires my initial discussion on economic development takings in chapter \ref{chap:2}. There I will present the basic definition of such takings and discuss the {\it Kelo} case in some more detail. Specifically, I will argue that Justice O'Connor's strongly worded dissent -- finding that the taking should be rescinded -- is consistent with, and conducive to, a social function perspective on property.\footnote{See \cite[494-505]{kelo05}.}

It should be emphasised that the focus will be on the question of when a taking is legitimate as such, not the question of how much compensation should be paid to the owners. Of course, the two questions are related; the amount of compensation  can influence the degree of legitimacy of the interference. Some scholars go further and argue that the legitimacy question is primarily about finding the appropriate mechanism for awarding compensation.\footnote{See generally \cite{fennell04,bell07,lehavi07}.} With the theoretical approach to property adopted in this thesis, this view must be rejected; the social functions of property are not reducible to financial entitlements. Moreover, the aim of this thesis is to discuss precisely those aspects of legitimacy that {\it cannot} be addressed through compensation. The link to compensation will be mentioned when it seems relevant, but the compensation issues that arise for economic development takings will not be analysed in any depth.\footnote{For such an analysis, see \cite{dyrkolbotn15a}.}

While I will advocate for a broad approach to the question of legitimacy of economic development takings, this thesis will focus on owners and their communities. Questions pertaining to the environment and social welfare will be considered as they arise in disputes about takings, not as issues in their own right. These aspects of economic development will therefore receive less attention here than they would in a thesis focusing specifically on environmental law or social and economic rights.  That said, one of the main arguments made in chapter 2 is that the social function perspective on property implies that we should take broader societal and environmental effects into account also when assessing the legitimacy of interfering with private property. The thesis argues for this coming from property theory, and in so doing will also touch on the conservation and social justice dimensions of economic development. In addition, chapter 2 will argue that if the social functions of private property support egalitarianism and equity at the local level, then private property can serve as an anchor for justice also with respect to the environment and the social and economic rights of non-owners. This point will also be made in Part II of the thesis, when discussing the issue of hydropower development in Norway. In future work, I hope to develop the idea further, to embark on research that will connect the social function account of property even more closely with environmental law and social and economic rights.

Chapter \ref{chap:3} builds on the property theory developed in chapter \ref{chap:2} by giving a more in-depth presentation of the legitimacy question, leading to a proposal for a justiciable legitimacy standard that makes judicial intervention possible. To make the discussion concrete, the chapter first offers a brief review and comparison of jurisprudential developments in the US, the UK, and at the European Court of Human Rights. These jurisdictions are chosen specifically for their many close connections with the Norwegian system, making them a natural reference point also for the case study in Part II of the thesis. Moreover, all the countries discussed are in comparable socio-economic situations, with property having a similar social function across the jurisdictions studied. This permits a comparative discussion that focuses specifically on the issue of takings, reducing the risk that the analysis will be distorted by significant differences in the social and economic context of takings law across different jurisdictions. The broader insights gained from the work done in this thesis might still be highly relevant to jurisdictions that have not been explicitly discussed. But a further treatment of this issue, for instance with respect to jurisdictions from the developing world, raises additional questions that must be left for future work.

Based on the choice of jurisdictions justified above, the thesis reviews several approaches to judicial review, culminating in a recommendation for a perspective based on institutional fairness that I trace to recent developments at the European Court of Human Rights. Specifically, the Court in Strasbourg has begun to look more actively at the systemic reasons why violations of human rights occur, in order to address structural weaknesses at the institutional level in the signatory states.\footnote{See generally \cite{leach10}.} This approach is arguably one that fits very well with the sort of analysis carried out by Justice O'Connor in {\it Kelo}, perhaps more so than the approach induced by the Fifth Amendment.

Importantly, the institutional perspective appears to be a sensible middle ground between procedural and substantive approaches to legitimacy, directing us to focus on decision-making processes without giving up on substantive fairness assessments. To ensure fairness, in particular, is not just about making good decisions, but also about how those decisions came about, and how likely it is that the system will have disproportionate effects at the societal level. This way of thinking about legitimacy brings me to the second main theme of this thesis, concerning the question of {\it democratic merit}.

\section{A Democratic Deficit in Takings Law?}\label{sec:1:2}

As discussed in chapter \ref{chap:2}, two of the key social functions of property is to promote social justice and to facilitate democratic decision-making.\footnote{See also, with further references, \cite{rose96,jackson10}.} In addition, property is meant to serve as a bridge between individual needs, community interests, and policy making at the national and the international stage. Through the law of property, societal priorities can be communicated to owners and communities without depriving them of their right to self-governance.\footnote{This is discussed in depth in chapter \ref{chap:2}, sections \ref{sec:2:4} and \ref{sec:2:5}.}

These functions of property are easily undermined if there is an excessive concentration of power and wealth among the elites of society. As indicated by Justice O'Connor's dissent in {\it Kelo}, this is one of the key reasons why economic development takings should be looked at with suspicion. In short, the concern is that economic development takings can both reflect and exacerbate a {\it democratic deficit}.

%hat can be applied to economic development takings. This test consists of a list of indicators that can suggest eminent domain abuse. The role of property in this regard is particularly clearly felt at the personal and local level, as a fair distribution of property is a highly effective safeguard for the basic rights of individuals and communities. In addition, property is meant to serve a key function as a bridge between the local level and policy making at the national and the international stage. Through the law of property, many societal priorities that might otherwise necessitate direct governmental control and interference can be effectively communicated to communities without depriving them of their right to self-governance.

There are many symptoms that can suggest a lack of democratic legitimacy, and to make the discussion more concrete, chapter \ref{chap:3} proposes a legitimacy test consisting of nine indicators of eminent domain abuse. The first six points are due to Kevin Gray, while the final three are additions I propose on the basis of the work done in this thesis.\footnote{For Gray's original points see \cite{gray11}.} I call the resulting list the extended Gray test, a heuristic for inquiring into the legitimacy of an economic development taking. 

Arguably, the most important indicator is the one pertaining to the overall democratic merit of the taking (one of my additions). When taken together with the other points, this indicator should induce an assessment of legitimacy against the decision-making process as a whole. Hence, it emphasises the institutional fairness perspective. If a taking fails the legitimacy test on this point, it might indicate an existing weakness of the system or a trend towards deterioration of the institutional framework surrounding eminent domain. This problem, moreover, might not be noticeable unless one considers an aggregated view of all the indicators of the extended Gray test, to shed light on what they tell us about the democratic legitimacy of existing practices.

Admittedly, asking courts to test for legitimacy is an incomplete response to the worry that economic development takings might result from, and give rise to, a democratic deficit at the societal level. This point has been argued by some US scholars, who claim that increased judicial scrutiny is neither a necessary nor a sufficient response to takings such as {\it Kelo}.\footnote{See generally \cite{lehavi07,heller08}.} Instead, these scholars try to come up with institutional innovations that can restore legitimacy in cases when the government wishes to ensure economic development on private property.

The most notable work in this direction so far is that of Heller and Hills, proposing what they call Land Assembly Districts (LADs) as possible alternatives to the use of eminent domain.\footnote{See \cite{heller08}.} The idea is that LADs will be set up to replace the traditional takings procedure in cases where property rights are fragmented and the potential takers have commercial incentives. The basic mechanism is one of self-governance; the owners themselves should be allowed to decide whether or not development takes place, using an appropriate collective choice mechanism (possibly as simple as a majority vote). In this way, the holdout problem can be solved (individual owners cannot threaten to block development to inflate the value of their properties). At the same time, the local community's right to manage its own property is recognised and respected.
 
The LAD proposal is closely linked to more general ideas about self-governance and sustainable resource management, particularly the theories developed by Elinor Ostrom and others.\footnote{See \cite{ostrom90}. For the connection with property theory generally, see \cite{ostrom10b,rose11,fennel11}.} On the basis of a large body of empirical work, these scholars have formulated and refined a range of design principles for institutions that can promote good self-governance at the local level.\footnote{For a more recent empirical assessment (and refinement), see \cite{cox10}.}

At the end of chapter \ref{chap:3}, I argue that this work can be used to address the legitimacy of takings in a principled way, to arrive at refinements or alternatives to the proposal made by Heller and Hills. Alternatives to expropriation based on self-governance can be a powerful way to address the worry that economic development takings might otherwise be associated with a democratic deficit. At the same time, the context-dependence of solutions along these lines make sweeping reform proposals unlikely to succeed. Rather, it is important that the institutions that are used are appropriately matched to local conditions.\footnote{See \cite[92]{ostrom90}.} This sets the stage for the second part of the thesis, consisting of a case study of takings for Norwegian hydropower development.

%For instance, a setting where property is evenly distributed among members of the local community might suggest a very different type of institution compared to a setting where the relevant property rights are all in the hands of a small number of absentee landlords. In short, the idea of using self-governance structures in place of eminent domain necessitates a more concrete approach, a move away from property theory towards propty practice. 

%The first key objective of this case study is to apply the theory developed in the first part to analyse the legitimacy of takings for hydropower. The second objective is to study a concrete institutional alternative to expropriation in more depth, namely the system of {\it land consolidation courts}. In Norway, these courts are empowered to set up self-governance organisations for local resource management and economic development, if necessary against the will of individual owners.

%In light of this, the case study will shed light on both of the two key conclusions drawn in the theoretical part of the thesis.

%Alternative test the theoretical assertions made about how to approach legitimacy. Specifically, the question to be addressed is to what extent the traditional narrative of takings is capable of doing justice to the property conflicts that have arisen regarding the development of hydropower in Norway.

\noo{ I arrive at several objections against the details of the particular institutional arrangements proposed, particularly with regards to their likely effectiveness. It seems, in particular, that both proposals fail to recognise the full extent to which prevailing regulatory frameworks concerning land use and planning would have to be reformed in order to make their proposals work.

At the same time, I argue that these novel institutional proposals are extremely useful in that they point towards a novel way to frame the issue of legitimacy in takings law. In particular, I explore the hypothesis that traditional procedural arrangements surrounding takings suffer from a democratic deficit, a particularly powerful source of discontent in economic development cases.

This idea is the second key focus point of my thesis. First, I approach it from a theoretical point of view, by exploring the notion of {\it participation} and its importance to the issue of legitimacy, particularly in the context of economic development. It seems, in particular, that {\it exclusion} could be a particular worrying consequence of certain kinds of economic development takings, namely those that lack democratic legitimacy in the local community where the direct effects of the taking are most clearly felt.

I believe this to be a promising hypothesis, and I back it up by considering the social function theory of property and the notion of human flourishing which has recently been proposed as a normative guide for reasoning about property interests. I pay particular attention to the importance of communities that has been highlighted in recent work, as a way to bridge the gap between individualistic and collectivist ideas about fairness in relation to property.

I take this a step further, by arguing that a focus on communities naturally should bring institutions of local democracy to the forefront of our attention. The role that property plays in facilitating democracy has been emphasised before by other scholars, and I think it has considerable merit. However, I also argue that it is important to resist the temptation of viewing its role in this regard through an individualistic prism. It is especially important to take into account additional structural dimensions that may supervene on both property and democracy, such as tensions between the periphery and the centre, the privileged and the marginalised, as well as between urban and rural communities.

It is especially important, I think, to appreciate the effect takings can have on local democracy. For one, excessive taking of property from certain communities might be a symptom of failures of democracy as well as structural imbalances between different groups and interest. But even more worrying are cases when the takings themselves, brought on by a commercially motivated rationale, appears to undermine the authority of local arrangements for collective decision-making and self-governance. This dimension of legitimacy, in particular, is one that I devote special attention to throughout this thesis.

I also believe, however, that it is hard to get very far with this sub-theme through theoretical arguments alone. Hence, to explore it in more depth, I go on to assess it from an empirical angle, by offering a detailed case study of takings of Norwegian waterfalls for the purpose of hydropower development. This case study, in turn, will allow me to cast light on two further key themes, that I now introduce. %This brings me to the second part of my thesis, which in turn consists of two main themes, where the latter aims to bring me back towards a more general setting, by delivering some recommendations for how best to deal with economic development takings.
}
%I go on to consider the hypothesis that economic development takings demonstrate that takings law suffer from a {\it democratic deficit}.

\section{Putting The Theory to the Test}\label{sec:1:3}

In Norwegian law, the takings question begins and ends with the issue of compensation.\footnote{See generally \cite{dyrkolbotn15,dyrkolbotn15a}.} If an owner has grievances about the act of taking as such, rather than the amount of money they receive, takings law has very little to offer. In fact, it does not appear to offer anything that does not already follow from general administrative law. The owner can argue that the taking decision was in breach of procedural rules, or grossly unreasonably, but the chance of succeeding is slim.\footnote{See \cite[384-386]{dyrkolbotn15b}.}

%This narrative of legitimacy is not unique to Norway. It seems that in Europe, unlike in the US, the issue of legitimacy is often seen as predominantly concerned with the issue of compensation. In particular, the jurisprudence at the ECtHR is typically focused on compensatory issues. Moreover, while many constitutions of Europe, including the Norwegian, include public interest clauses, the courts make little or no use of these when adjudicating takings complaints. In the words of the ECtHR, the member states enjoy a ``wide margin of appreciation'' when it comes to determining what counts as a public interest.

In cases involving hydropower development, the position of property owners is also strongly influenced by sector-specific legislation, as well as special administrative and market practices. Chapter \ref{chap:4} begins the case study by discussing this in more depth, setting the stage for the discussion on expropriation that follows in chapter \ref{chap:5}. A first important observation is that the hydropower sector in Norway was liberalised in the early 1990s.\footnote{The crucial legislative reform was the \cite{ea90}.} This means that the energy companies benefiting from eminent domain are now commercial enterprises, not public utilities.

A second important observation is that the right to harness the power of water is considered private property in Norway, typically owned in common by members of nearby rural communities.\footnote{See \indexonly{wra00}\dni\cite[13]{wra00}. This arrangement has long historical roots and makes intuitive sense in a mountainous country with a very vast number of small and medium sized rivers coming down from steep outfield mountains. For the historical development of the law on this point, see \cite[109-116]{nordtveit15}.} This does not mean that freely running water, as a substance, is subject to private property. What it means is that riparian owners have an additional stick in the bundle of rights that the law associates with being the owner of land over which water flows. A useful comparison can be made with fishing rights; the right to the hydropower in a river arises from landownership, but it is conceived of as a separate, transferable, right in property.\footnote{Apart from this explicit recognition of water power as a separate right in property, the Norwegian system of riparian rights appears to be historically quite similar to the riparian common law, see generally \cite{howarth15}.} It is referred to in Norwegian as a ``fallrett'', which can be translated as a {\it waterfall right}. This thesis will therefore often refer to waterfalls and owners of waterfalls when discussing the right to harness the power of water in a river.\footnote{In some cases, especially historically, a waterfall right would be formally registered as a separate unit of real property to facilitate transfer to someone other than the owner of the surrounding agricultural land. However, waterfall rights can also be formally registered as rights of use attaching to the real properties from which they arise. In relation to Norwegian expropriation law, and for the purposes of this thesis, the distinction between these two ways of registering waterfall rights will not play an important role and will not be discussed further.}

As discussed in Part II of this thesis, hydroelectric companies in Norway have traditionally had easy access to privately owned waterfalls, made possible through the government's power of eminent domain. However, since deregulation, local owners have begun to resist such takings. This has been motivated by the fact that owners can now undertake their own hydropower projects as a commercial pursuit; unlike the situation before liberalisation, owner-led development projects can demand access to the electricity grid as producers.\footnote{See, e.g., \cite{uleberg08}.} This has led to heightened tensions between takers and owners, tensions that the water authorities are now forced to grapple with on a regular basis.

%As a result, local owners now regularly protest expropriation of their rights on the grounds that they wish to {\it participate} in economic development, by carrying out alternative development projects, or by cooperating with the energy companies who wish to take their water rights. Hence, while liberalisation has rendered takings for hydropower as takings for profit, it has also empowered local owners and communities to propose alternatives. Unsurprisingly, this has led to tensions that the water authorities are now forced to grapple with on a regular basis.\footnote{See Chapter \ref{chap:4}, Section \ref{sec:4:4}.}

Chapter \ref{chap:4} argues that despite their improved position following liberalisation, local owners remain marginalised under the regulatory framework. Specifically, despite political support for locally organised small-scale development, the large energy companies have continued to enjoy a privileged position in their dealings with the water authorities. Building on this observation, chapter \ref{chap:5} goes on to discuss eminent domain in more depth. The chapter tracks the position of owners under the law and administrative practices that relate to takings of waterfalls. The key finding is that expropriation is usually an {\it automatic consequence} of a large-scale development license.\footnote{In some cases, this follows explicitly from the water resource legislation, while in other cases it follows from administrative practice. For further details, see below in chapter \ref{chap:5}, section \ref{sec:5:3}.} That is, commercial companies that succeed in obtaining large-scale development licenses will almost always be granted the right to expropriate. This right will be granted, moreover, with little or no prior assessment as to the appropriateness of depriving local communities of their resources. Indeed, the fact that expropriation tends to follow automatically from a license to develop has led the water authorities to focus their attention on the licensing question and associated procedures. No distinction appears to be made between cases involving expropriation and cases that do not. This has a significant effect on the level of procedural protection offered to local owners. For instance, according to written testimony during a recent Supreme Court case on legitimacy, the water authorities do not recognise any duty to give individual notice to local owners before processing applications that involve expropriation of their waterfalls.\footnote{The case in question was \cite{jorpeland11}.}

While the appropriateness of taking property from local people is rarely discussed, the issue of how hydropower affects the environment has received increased attention in recent decades. Sometimes, environmental impact assessments will uncover negative effects and the water authorities will reject development applications, also when the applicant is a large energy company. In general, the framework for management of hydropower in Norway has an important conservation dimension that is clearly recognised by the government.\footnote{See \cite{backer12} (presenting Norwegian environmental law).}\noo{However, conservation issues are often orthogonal to the property question; in some cases, local owners of waterfalls will oppose large-scale development projects that damage the environment, while in other cases, environmental interests will block small-scale projects that the owners themselves would like to carry out. This reflects the importance of environmental issues.} However, as argued in this thesis, the effect on local owners and communities still receives little or no attention.\footnote{See especially the discussion in chapter \ref{chap:5}, sections \ref{sec:5:6} and \ref{sec:5:7}.} In order to explore this phenomenon and investigate its consequences, the thesis focuses specifically on the taking of private property, while conservation issues remain in the background.\footnote{That said, chapters 4 and 5 will touch on two key debates regarding environmental law in Norway in recent years. The first pertains to the sector-based approach to natural resource regulation, which some argue is at odds with the holistic approach encouraged by international law instruments. The second debate, which is closely related to the first, concerns the extent of the government's duty to assess alternative resource uses and development schemes when considering license applications for concrete projects. See generally \cite{winge13,backer10}.}

%This means that conservation issues remains in the background.  discuss the special issues that arise when the property rights of local owners are restricted to conserve the local environment, a relatively common example of property interference in water law, but one that does not qualify as an economic development taking. 

\noo{Although conservation is not dealt with in any depth, environmental issues will be discussed when they have a bearing on the legitimacy questions that arise when energy companies expropriate waterfalls. As I will demonstrate in chapter \ref{chap:5}, environmental organisations and energy companies both enjoy a strong positions within the regulatory framework. Recently, there has been a tendency for these two power groups to reach compromises, such that large-scale development projects are allowed to go ahead while small-scale projects are stopped because of their environmental effects. Indeed, environmental groups and commercial companies now appear to be reaching a form of mutual understanding that large-scale development is {\it better} for the environment than small-scale projects. As discussed in chapter \ref{chap:5}, this conclusion rests on what appears to be a very narrow and arguably misguided understanding of what sustainability and conservation should entail in the context of hydropower development. %Moreover, when environmental groups and large energy companies unite in this way, it raises the worry that local owners and their communities will be marginalised further.chapter \ref{chap:5} will demonstrate that the owners' position during the licensing assessment stage is highly precarious, contrasting both with the strong position of the development companies and the similarly influential role played by conservation interests.
}
In relation to the compensation issues that arise following expropriation, the owners' legal position initially grew stronger after liberalisation. Specifically, the lower courts started to compensate local owners for the lost opportunity to profit from hydropower.\footnote{See \cite{uleberg08} (specifically, it was observed that waterfalls now had a market value, due to the increasing prevalence of owner-led hydropower).} This led to a dramatic increase in compensation payments compared to earlier practice.\footnote{See especially the discussion in chapter \ref{chap:5}, section \ref{sec:5:5:1}.} However, a recent decision from the Supreme Court appears to largely reverse this development, since a large-scale license may now be considered proof that alternative development by owners is unforeseeable and therefore not compensable.\footnote{See \cite{otra13}.} 

In light of this and other data discussed in chapter \ref{chap:5}, my conclusion is that recent takings for hydropower do not in fact pass the extended Gray test. The current practices appear illegitimate with respect to the theory of property developed in Part I of the thesis. At the same time, Norwegian law offers a promising institutional path towards the restoration of legitimacy in economic development contexts. Specifically, the unique framework for land consolidation found in Norway can serve such a function. This has already been demonstrated in the context of hydropower development, where land consolidation courts have been able to successfully organise development projects on behalf of owners who wish to undertake development but disagree about how it should be done. This brings me to the fourth key theme of this thesis.

\section{A Judicial Framework for Compulsory Participation}\label{sec:4}

\noo{ In Norway, the distribution of property rights across the rural population is traditionally highly egalitarian.\footnote{This is discussed in more depth in chapter \ref{chap:4}, Section \ref{sec:4:2}.} This meant that the farmers in Norway soon became an active political force, particularly as representative democracy started to gain ground as a form of government in the 19th century.\footnote{As early as in 1837, the Norwegian parliament was so dominated by farmers that it came to be described as the ``farmer's parliament''. See \cite{hommerstad14}.}

%The Norwegian farmers were often little more than small-holders, and had few privileges to protect. Hence, they became liberals of sorts (although also known for their fiscal conservatism). The farmers as a class were responsible for pushing through important early reforms, such as the abolition of noble titles and the establishment of democratically elected municipality governments.

%However, the municipality governments were not the first example of local decision-making institutions in Norway.
The highly fragmented ownership of land meant that institutions for collective decision making had to be introduced early on in Norwegian history; some even argue that the first realisation of a truly direct democracy can be traced to Norway in the Viking age.\footnote{See \cite[23]{titlestad14}.} One of the ancient institutions for collective action is the land consolidation court. 
}
The fourth and final key theme, presented in chapter \ref{chap:6}, consists of an assessment of the Norwegian land consolidation courts. These courts have the power to order owners to undertake or allow development projects (without depriving them of their property), as an alternative to expropriation. Moreover, they are presently used in this way in the context of hydropower development. The large energy companies almost never use consolidation, but local communities often do.\footnote{According to the Court Administration, as of 2009, land consolidation proceedings had facilitated a total of 164 small-scale hydropower projects with a total annual energy output of about 2 TWh per year (enough electricity to supply a city of about 250 000 people), see \cite{dom09}.} In these cases, the land consolidation courts have proved themselves effective in making self-governance work, also in cases when some of the owners do not with to undertake development.

\noo {The typical scenario is that the owners disagree about who owns what and cannot agree on how to organise development. In other cases, some of the owners, or even a majority of them, do not wish any development at all. In these cases, it is possible for the courts to {\it compel} them to participate. 

In these situations, it is less clear how well consolidation works in practice. Plainly, there has not been enough cases of this sort to draw a clear conclusion, especially not in situations when those who favour development are a minority among the owners. However, the consolidation alternative still appears highly preferable to the expropriation alternative, especially in terms of legitimacy. Specifically, the owners who are compelled to participate do not loose their property and are not excluded from the decision-making process.}

The land consolidation alternative can make a great difference, especially since it strives to ensure legitimacy through participation. The potential democratic deficit associated with economic development takings  is dealt with by mechanisms that seek to enable owners to take active part in the management of their property in the public interest. At the same time, the procedure can be quite effective, since participation is compulsory and the consolidation judge may intervene to settle conflicts and establish organisational order. Chapter \ref{chap:6} also addresses possible objections to the procedure, but concludes that the continued development of the land consolidation institution provides the best way forward for addressing problems associated with economic development takings in Norway.

%Finally, the institution of land consolidation is assessed against Land Assembly Districts, and -- more generally -- against the idea of self-governance frameworks for managing common pool resources. I argue that it compares favourably, both because it comes equipped with in-built judicial safeguards, but also because it has such a broad scope. I note, however, that its successful use is dependent on both political will and an ability to retain key feature even in the presence of new and powerful stakeholders in the consolidation process.

If the integrity and efficiency of the procedure can be preserved, it appears to have great potential as an alternative to eminent domain in general, also in cases involving large-scale development and cooperation with external commercial actors. Moreover, while the system is designed to work in a setting of egalitarian property rights, it is interesting to consider whether key features of the procedure could inspire solutions to the takings problem in other jurisdictions. %Specifically, the fact that the procedure focuses on benefiting properties rather than owners means that a broader understanding of property can itself suggest a broader range of possible applications. 

It might well be, for instance, that a land consolidation approach coupled with a human flourishing understanding of property can be a good way of including non-owners in the process, in jurisdictions where property rights are not distributed as widely among the population as in Norway. This might make the procedure more complex and give rise to new risks of abuse by local elites, but it seems like an interesting idea to explore in future work. 

In short, the consolidation alternative provides a starting point for an approach to legitimacy that takes a wider view of what property is, and what role it can and should play in a democratic society. In this way, the chapter on consolidation also returns to the conceptual premise discussed in the first chapter of this thesis, whereby the purpose of property is to promote human flourishing.

%the core features of land consolidation for economic development can be preserved and developed further, 
\noo{ In the second part of the thesis, I put the theoretical framework to the test by applying it to a concrete case study, namely that of Norwegian hydropower. Following liberalisation of the energy sector in the early 1990s, hydropower is now a commercial pursuit in Norway. Moreover, there is a long tradition for granting energy producers the power to acquire property compulsorily, including the necessary rights to exploit the energy of water, rights that are subject to private property under Norwegian law. This has resulted in tension and controversy, however, as the original owners of these rights, typically local farmers and small-holders, see the commercial potential of hydropower being transferred to other commercial interests, to the detriment of their own, and their communities', interest in self-governance and economic benefit.}

\section{Some Terminological Clarifications}

{\it Property} is a key notion in this thesis. As mentioned already, it is an elusive legal term, with different decomposable meanings depending on the context of use and the jurisdiction within which we find ourselves. In the first part of this thesis, the notion is explored conceptually, to develop a theory of property's role and purpose within law and society. The details of how the notion is defined in a given jurisdiction will not be our concern. In general, there is quite some variation in this regard, among the different jurisdictions considered in this thesis. With respect to the European Convention of Human Rights, for instance, we encounter a notion of property (or ``possession'') that is very wide, one that also covers social welfare entitlements and immaterial benefits, including future pension payments and goodwill acquired by holding a professional title.\footnote{See \cite[73-77]{allen05}.} This is a broader concept of property than that usually encountered in private law, also in jurisdictions that incorporate the Convention into their national legal order. The issues that can arise form this, when several distinct notions of property co-exist in the law, will not be considered here; the thesis will remain focused on ``classical'' instances of property, typically property in land and related resources.

That said, the theoretical argument made in chapter 2 might well be relevant for property lawyers working with disputed definitions of private property within a specific legal framework. Indeed, the theory presented in this thesis can be used to argue normatively that a given jurisdiction relies on a notion of property that is either too wide or too narrow to cater to important social functions. For instance, it would be interesting to consider the implications that a social function understanding can have in the context of intellectual property, specifically to shed light on the normative question of what kinds of immaterial property the law {\it should} recognise. However, this line of research must be left for future work.

There is one special type of property encountered in this thesis that deserves a special mention. This is the notion of {\it common property} in land and natural resources. This notion is notoriously ambiguous, used to refer to at least three different kinds of legal arrangements.\footnote{\cite[714-715]{bishop75}. See also \cite[12-13]{fennel11}.} First there are open-access resources, which are sometimes (erroneously) referred to as common property. These resources are characterised by the fact that everyone is in principle entitled to make use of them. Hence, it is more accurate to say that they are resources that have no owner. The use of such resources is typically managed by the government through regulation, sometimes under a public trust doctrine. The questions of sustainable resource management and governance that arise in this regard are interesting in their own right, but are not considered in any depth in this thesis.

The second type of legal arrangement often referred to as common property is the collection of rights and responsibilities attaching to common land. This is land over which a specific group of people enjoy use rights and where special rules are in place to regulate the exercise of these rights and the management of the underlying resources. A typical example is found in the law of the commons in England and Wales, as regulated today by the Commons Act 2006.\footnote{See generally \cite{rodgers10}.} Use rights on the commons can be thought of as property rights, but under individualistic accounts of what property is, it might not be appropriate to do so. The distinguishing feature of rights in common is that they provide an anchor for a special legal framework, a set of rules, institutions and customs that pertain specially to the communal character of such rights. This function of rights in common can be distorted if those rights are fitted into an entitlements-based framework for maintaining rights in property.\footnote{For a concrete example, see \cite[469--471]{rodgers10} (analysing the effect of the Commons Registration Act 1965 (England and Wales)).}

Under a social function theory of property, by contrast, it becomes much more appropriate (at the conceptual level, at least) to think of rights in common as private property rights. Moreover, as discussed further in chapter 2, the social function theory can support arguments to the effect that {\it all} instances of property, even traditional forms of private property, can be viewed as being part of a commons in an abstract sense of the word.\footnote{See also \cite[16-18]{fennel11} (``Property, as experienced on the ground, is never wholly individual nor wholly 
held in common, but instead always represents a mix of ownership types.'').} This point will also be made in chapter 6, when I discuss how land consolidation can be used to {\it set up} institutions for collective management of private property rights within a local community. This form of property intervention can be understood as an effort to bring key ideas behind the commons to bear on private property rights. The connection with the commons is made at the theoretical level; the thesis will not address existing legal frameworks used to regulate rights in common as such. Specifically, the special issues that arise when new forms of economic development interfere with such rights will be left for future work.

\noo{Moreover, according to wider, more functional, definitions of what private property is, rights in common may well be covered. There can be little doubt, for example, that the use rights of individuals having rights in common over some resource {\it do} constitute property rights (``possessions'') within the meaning of Article 1 of Protocol 1 of the European Convention of Human Rights.

At the same time, the distinguishing feature of rights in common is that they are surrounded by a special legal framework, a set of rules, institutions and customs that pertain specially to the communal character of such rights. This thesis will not investigate concrete examples of such legal frameworks, except briefly in chapter 4 when I present different property regimes found in Norway.}

%That said, commons also tend to come with many specific regulatory provisions and institutional arrangements, none of which 

%positive law regulations which might not bthis thesis will not explore in any depth those special rules and arrangements that are in place to regulate the commons in any specific It should be noted, however, that the special questions that might arise when economic development takings take place in the commons will not be addressed in this thesis. The theory developed herein should be applicable, but further exploration of this must be left for future work.

The third legal arrangement that can be referred to as common property is encountered when a property is owned, in a conventional private law sense, by a group of owners. Shared forms of private ownership are supported by most jurisdictions, including those considered in this thesis. Shared private ownership is particularly important in Part II of the thesis, since the takings discussed there will typically involve rights to hydropower that are owned by several private parties in common under a legal framework that resembles the concept of a tenancy in common, known from commonwealth jurisdictions. A brief presentation of this form of private ownership in Norway is provided in chapter 4, along with a discussion of the importance of egalitarianism in Norway.\footnote{See chapter 4, section 4.2.1.}

%affect local communities as a whole, not just individuals. However, the cases considered will all be cases where individuals have recognised rights in property, meaning that the cases fall uncontroversially within the ambit of takings law. The margins of takings law, encountered for instance if groups of non-owners make proprietary claims based on customary use rights or the like, will not be considered in any depth. However, the theory of property developed in this thesis might suggest making normative claims to the effect that legal standing in takings proceedings should be extended to cover a larger group of legal persons than those presently recognised. A closer examination of this is left for future work. 

%In the second part of the thesis, when dealing specifically with Norwegian law, we will encounter a few specific forms of private property that deserve a special mention. 
%In Norway, we find a uniquely egalitarian distribution of land ownership, where land and the resources found on it are typically owned by groups of local small-holders, not landlords or public bodies. This form of shared ownership is considered a conventional form of private ownership under Norwegian law. There are also some large commons in Norway, but they are of lesser practice importance due to the prevalence of shared private ownership over outfields. Further details on property arrangements found in Norway are provided in chapter \ref{chap:3}.

{\it Legitimacy} is a second key notion used in this thesis. The notion is consistently used in a normative sense, to describe that an interference in private property appears morally justified.\footnote{For a concise presentation of moral legitimacy, see \cite[438-441]{thomas14}. See also \cite{michelman04,priel11}. Moral legitimacy becomes particularly important under natural law theories, since such theories make the moral status of a rule directly relevant to the question of its legal validity. However, moral legitimacy is also relevant on a positivist understanding of law; it is a descriptive fact that moral considerations shape the law, not only through explicit law-making, but also because judges are unable to completely separate formal reasoning about legal content and validity from moral reasoning about legitimacy. For a longer argument to this effect, coming from a self-described positivist, see \cite[1801-1802]{fallon05}.} It is not used as a term with a specific descriptive meaning within a given jurisdiction. A key aim of this thesis is to address the question of when an interference in private property {\it should} be regarded as legitimate. All the jurisdictions I consider have their own specific rules in place that are meant to ensure legitimacy in takings law. The most common legal terms that are used in this context are the notions of {\it public use}, {\it public purpose}, and {\it public interest}. Specifically, a typical takings provision states that the public must benefit, directly or indirectly, in order for an interference in private property to count as legitimate. Such provisions or their near equivalents can be found in a range of different jurisdictions, including all those studied in this thesis.

The meanings of the terms used are similar across different contexts and legal systems. Still, since these are formal legal terms, it is worth keeping in mind that their meaning is relative to the jurisdiction under consideration. For instance, while most of the jurisdictions considered in this thesis do not recognise any substantial difference between public use, public interest and public purpose, some jurisdictions maintain such distinctions and attach important legal consequences to them. Most famously, the position that public use literally means use by the public, and is therefore quite distinct from public interest and public purpose, is forcefully advocated by several US legal scholars, including at least one member of the Supreme Court.\footnote{As demonstrated by \cite{kelo05}.}

When terms such as public use, public interest or public purpose are used in this thesis, their exact meaning correspond to the meaning given to them by the jurisdiction under consideration. If the terms occur in theoretical discussions, their meaning should be understood according to a natural language interpretation that points to the general idea behind using terms like these in the law of takings. The reason why notions such as public interest and public use are important is that they can help enforce the natural idea that interferences in private property should only occur for the good of the people. This much is common to all jurisdictions considered in this thesis. %However, how the general idea is implemented varies quite considerably. To account for this, the thesis will briefly clarify the meaning of the terms whenever they appear in the context of a concrete jurisdiction.

Adding to the generally accepted starting point, the thesis will argue that legitimacy also requires decision-making to take place in an equitable and inclusive manner, such that owners and others who depend closely on the properties in question have a say that is commensurate with what is at stake for them. This perspective, combining procedural and substantive ideals of fairness, will not rely on finer distinctions between notions such as public use, public purpose and public interest. In my opinion, this is a strength of the theory developed in this thesis, an escape from what Gregory Alexander calls the ``formalist trap'', characterised by an exaggerated focus on constitutional property clauses and how they are formulated.\footnote{See \cite[Chapter 1]{alexander06}.} As Alexander argues, excessive formalism can cloud the issue of legitimacy because it blocks from view those important institutional and political processes that determine the actual level of protection given to property and its owners within a given jurisdiction. Building on this, my thesis will develop an integrated approach to legitimacy based on the idea that private ownership is meant to be an anchor for democracy and a promoter of human flourishing.

% will be used throughout this thesis, and it will be used in a normative sense.


%These are presented in  section \ref{sec:x} of chapter \ref{ In general, we find a uniquely egalitarian distribution of land ownership in Norway, where undeveloped land and the resources found on it are typically owned by groups of local small-holders, not landlords or public bodies. In section \ref{sec:1:5}, we already mentioned the concept of a waterfall right, used to refer to the right to harness power from a river, an historically important stick in the property bundle associated with landownership in Norway. Moreover, we mentioned briefly that waterfall rights are usually held in common by members of the local population. %Enjoying private ownership in common is not unusual in Norway, particularly in rural areas, and the law of property in Norway reflects this in various ways.

%In some cases, this is because a river suitable for hydropower development will run across many distinct private properties. Hence, the relevant waterfall rights are held in common in the narrow sense that an assembly of private rights is required in order for development to take place. However, in most cases, waterfalls suitable for hydropower development will be owned in common in a somewhat stronger sense. Indeed, outfields in Norway are often held under a specific form of co-ownership, such that each small-holding in the local community owns a share in the land surrounding their local community. This form of co-ownership has no exact common law equivalent, but is most similar to the tenancy in common. However, there is no requirement that the co-ownership takes place behind a trust -- all individual shareholders are formally registered as owners of their share of the land and their is a presumption in favour of continued co-ownership accompanied by productive use of co-owned land, not a presumption in favour of sale and individuation as seen in UK law.

%In the Common Ownership Act 1965, further rules are given to regulate the use of land under co-ownership. The main principle is that each owner has a right to the ``normal'' enjoyment of the property, taken in light of the local conditions, customs, and the original purpose of the co-ownership arrangement (if it is known). Moreover, an individual owner's use must not exceed what corresponds to his share of the property and must not be unduly burdensome to the other owners. If damage occurs, moreover, compensation must be paid. To some extent, the majority shareholders can enforce a specific use of the property which would also be binding on the minority. This includes new forms of commercial activity on the property. If such activity is organised against the will of a minority, the minority will still be entitled to take part in the enterprise. 

%There are limits to what the majority can do. Importantly, they cannot do anything that will limit the ``normal'' use of the property by any owner. Also, they cannot do anything to dramatically change the character of the property, sell it, or use it as security for debt. Because of this, gridlock can often result if the owners disagree fundamentally about how to manage their land. For real property, particularly in rural areas, the standard way of resolving such situations is to bring a case before the Norwegian land consolidation courts. These courts are empowered to either dissolve the system of co-ownership or else to organise joint use of the land. Indeed, the prevalence of common ownership over outfields is one of the reasons why land consolidation courts are so important in Norway, and it also explains why they have been granted wide powers to help organise the use of privately owned land. I return to the details of this in chapter \ref{chap:6}, as part of a broader discussion on how the institute of land consolidation can be used as an alternative to eminent domain in economic development situations.

%In addition to the form of co-ownership regulated in the Common Ownership Act 1965, there are two other special forms of ownership of land found in Norway that should be briefly mentioned. Both pertain to land over which a large group of people enjoy extensive rights of use that have been recognised as so-called common rights under Norwegian land law. There is always an owner of the land in the normal private law sense of the word, but special rules are in place to protect the group of people who enjoy use rights. These rules presuppose that the land is owned either by the state or a council of the local community (which might not include everyone who enjoys use rights over the land). There is no concept of a commons in Norway that attaches to land owned by private individuals, which is quite natural given that private landlords and tenant farming is  absent from the structure of rural landownership in Norway.

%If the owner of common land is the state, the relevant legislation that protects the rights of the local people is the State Commons Act 19... If the land is owned by a local community, the relevant legislation is the Village Commons Act 19... The details differ, but the main principle of both acts is that they offer special protection to use rights holders, especially with regard to traditional land uses that local farmers depend on for their livelihoods.

%After the industrial revolution, there was some doubt as to whether common rights gave non-owners a claim to waterfall rights, or whether waterfall rights were held exclusively by the landowners. This was particularly important for land owned by the state, since common rights was the only potential route for local community members to claim a proprietary stake in local hydropower resources (in village commons, by contrast, the owners would typically themselves be local community members). The question was settled by the Supreme Court in the case of {\it Vinstra} in 196.. Here it was held that no rights to waterfalls on state-owned lands could be derived from rights in common over that land. For this reason, the takings issue does not arise with respect to hydropower development on such land, at least not with respect to the waterfall rights as such. 

%Of course, questions still arise regarding the fate of local communities when development takes place on state-owned land where local people enjoy rights in common. However, questions that arise specifically with respect to Norwegian commons law will not be dealt with in this thesis.\footnote{When we consider the case of {\it Alta} in Chapter \ref{chap:5}, we will encounter state-owned land where the aboriginal Sami population has claimed to enjoy rights in property similar to common rights. In recent years, this claim has met with some recognition within the Norwegian legal order, giving rise to yet another form of property in Norway. For further details, see the discussion in Chapter \ref{chap:5} section \ref{chap:5:x}.}
%However, when the {\it Alta} case was decided, members of the Sami population were considered as rights holders in the traditional private law sense of the word, no different from non-aboriginal holders of property and use rights elsewhere in Norway. See the

%In Chapter \ref{chap:5}, we will discuss the {\it Alta} case in some depth. This case was a takings case arising from hydropower development in Finnmark, a part of Norway where the state is traditionally regarded as the owner of all outfields. The state's ownership of land in this region tends to be at odds with the aboriginal interests of the Sami people. Traditionally, the state's ownership was consider to be entirely unencumbered by aboriginal entitlements except where Sami use rights had been explicitly recognised. Moreover, the rights of the Sami people did not have the protected status granted to rights in common over state land.\footnote{Some scholars disputed this, by arguing that Sami rights should be viewed as common rights by analogy with the legislation in place for state commons.} 

%Hence, in the {\it Alta} case, the formal standing of the Sami people was derived from expressly recognised use and property rights that would be lost or depreciate in value following the development. Specifically, the Supreme Court rejected claims based on aboriginal rights, and the case did not involve takings of waterfalls as such. Still, the case has been considered an important precedent for disputes surrounding expropriation of waterfalls, since it dealt with many aspects of administrative law pertaining to the licensing procedure surrounding hydropower development. In addition, as I discuss briefly in chapter \ref{chap:5}, the case marked a watershed moment in the legal history of the Sami people, whose rights over land in Finnmark have since received greater recognition within the Norwegian legal order. Today, in the special context of Sami land, the law appears to be moving towards a framework where the Norwegian state is increasingly seen as a custodian of Sami lands, rather than an owner in the standard private law sense.

%In other parts of Norway, a similar perspective has not developed. Natural resources owned by the state, or taken under eminent domain, has the same legal status as private property, except that the owner happens to be the state. There is no recognised legal sense in which the state is held to be a custodian of land, and there is no legal doctrine according to which state-owned lands are supposed to be held in trust on behalf of the people. However, in a recent revision of the Constitution, a new section was introduced that compels the government to preserve the environment and promote sustainability. The exact wording is as follows:


%This provision replaces a similarly broad sustainability provision that was first introduced in the Constitution in 1992. In practice, the sustainability requirement has left little impact in Norwegian law, with no consequences discernible at all within the law of property. No one, to my knowledge, has proposed to read section 112 as having any direct bearing on the state's rights and responsibilities as the owner of land. Rather, the section is typically understood to give rise to a general obligation to promote sustainability through regulation, meaning that regulatory failures could conceivably be challenged under the provision. So far, however, few challenges of this kind have appeared and none have been successful. . Indeed, it has been argued that the constitutional sustainability provision as such has been something of a failure.

%After the new formulation was introduced in 2014, there have been some indications that the legal status of the provision might be about to change, in the direction of becoming more easily justiciable. In fact, a group of Norwegian environmental lawyers are presently preparing a case where they will challenge the Norwegian state with not doing enough to fight climate change, a legal action that will be brought under section 112 of the Constitution. In the law of property, however, there is no indication that the provision will become important any time soon. Similarly, in the law of hydropower, there have been no indications to suggest that the provision will be considered relevant to disputes between developers and local people, especially not when such disputes arise with respect to the issue of expropriation. For this reason, the sustainability provision in the Norwegian constitution will not be examined further in this thesis. Of course, a normative argument could well be made that the provision {\it should} entail greater regard for the interests and property rights of local people. Such an argument might perhaps also be backed up by considerations based on sustainability research and international environmental law. Further exploration of economic development takings from this angle will be left for future work.

%First, we will encounter ownership of waterfalls, a concept that appears to be unique to Norwegian law. As mentioned in section \ref{sec:1:5}, above, the right to harness power from a waterfall in Norway has a recognised status as an incident of private landownership. Moreover, it can be transferred separately from the surrounding land, voluntarily or through expropriation. 

%This is the Norwegian property type referred to as a ``vannfall'', literally translated: a waterfall. This is a legal term with a specific (although disputed) meaning in Norwegian law. In the second part of the thesis the word waterfall will be used in the specific sense that the word ``vannfall'' is used under Norwegian law. The intuitive understanding of the term is suggestive but incomplete, so a clarification is in order: a waterfall is used in the law to refer to a power, namely that of water flowing along a given stream or river. That is, the waterfall is a term the law uses to refer to the energy that can be harnessed from a river from a point A to a point B, where A and B are typically given as the altitude where a given waterfall begins to where it ends. 

%What this means is that a waterfall owner, under Norwegian law, has a right to the hydropower of a river. This right usually emerges from ownership of land over which the water flows, but it is considered a distinct ``stick'' in the bundles of riparian owners. It is also (on some conditions) separately alienable. This is important because it means that in Norway, developing a hydropower plant requires ownership of the waterfall, nor merely access to suitable sites for building the dam and the station. In Norwegian law, one does not take the view that the building of a hydropower plant creates the hydropower. Rather, the power of the water already exists and already has owners, usually the local landowners. To some extent, waterfall rights under Norwegian law can be compared to fishing rights in English law.

\noo {\section{Structure of the Thesis}\label{sec:1:5}

My thesis is divided into two parts. Part I sets up a theoretical framework for reasoning about property and proceeds to study the legitimacy of economic development takings in more depth. Part II consists of a case study of takings for hydropower, focusing on how expropriation and alternatives to it work on the ground in Norway. In brief, the structure of the chapters are as follows.

Chapter 2 introduces the topic of this thesis and presents the social function theory of property. The chapter argues that the descriptive core of this theory should be accepted irrespective of one's normative inclinations; the social function approach is simply more accurate than other theories. From this descriptive assertion, the category of economic development takings arises naturally. To address it normatively, the chapter argues that the notion of human flourishing provides the appropriate starting point. On this basis, the chapter discusses economic development takings and {\it Kelo} in more depth, to introduce the key question of legitimacy.

Chapter 3 proceeds to address the legitimacy question in more depth. The chapter starts from considering the procedural approach to legitimacy, illustrated by the law of England and Wales. Following up on this, the substantive approach is considered, illustrated by the law of the US. Finally, the chapter argues for a middle ground between the two, an institutional fairness perspective that is also linked to recent developments at the ECtHR. Following up on this, the chapter presents the extended Gray test; a set of indicators of eminent domain abuse suitable for an institutional fairness approach. The chapter concludes by discussing the possibility of providing institutional alternatives to expropriation for economic development, taking inspiration from the theory of self-governance for common pool resources.

Chapter 4 introduces the case study of takings for hydropower in Norway. The chapter briefly presents hydropower in the law, focusing on the licensing legislation. Then the chapter investigates hydropower in practice, noting that the liberalisation of the electricity market in the early 1990s has had a dramatic effect. Specifically, the chapter emphasises how local owners of water resources are now in a better position to develop these themselves, since they can access the electricity grid as producers on equal terms as larger companies. The chapter goes on to study the tension that has resulted between large-scale development facilitated by expropriation and small-scale development facilitated by local property rights. Despite early signs that small-scale solutions enjoyed political support, the large energy companies now appear to be reasserting their control over the hydropower sector, to the detriment of owners and their local communities.

Chapter 5 discusses expropriation of hydropower in more depth. The chapter starts by giving a brief overview of Norwegian expropriation law, before noting that expropriation for hydropower often takes place on the basis of special rules that leave owners with less protection. The history of the law is discussed in quite some detail, to show how the law has gradually developed to undermine local property rights over water resources. Following up on this, the chapter discusses case law on the expropriation and licensing, focusing on the legitimacy question (which is addressed in Norway almost solely on the basis of procedural standards). The chapter studies the recent Supreme Court case of {\it Jørpeland} in depth, to shed light on how current administrative practices impact on owners and their communities. The conclusion is that current takings practices do not appear legitimate.

Chapter 6 discusses land consolidation as an alternative to expropriation. The chapter starts by clarifying the notion of land consolidation and how the Norwegian understanding of that terms is much wider than that found in other jurisdictions. Following up on this, the chapter discusses consolidation as an alternative to expropriation, by focusing on those tools that the consolidation courts have at their disposal in this regard. Then the chapter gives a more in-depth presentation of some cases when consolidation was used to organise small-scale hydropower development. Finally, a discussion is provided on the prospect of using consolidation to replace expropriation more generally, in Norway and possibly also in other jurisdictions.

Chapter 7 contains my conclusions, formulated as an attempt at connecting the concrete and abstract aspects of this work around two threads, tracking property's relationship with excluding and taking on the one hand and its relationship with giving and participation on the other. My final conclusion is that the latter two notions characterise true property, and that property as such is worth defending.

} 

\include{Chapter1/1_revised}
%\newcommand{\isr}[1]{{#1}}

\chapter{Possible Approaches to the Legitimacy Question}\label{chap:3}

\section{Introduction}\label{sec:3:1}

%In the previous chapter, I introduced the social function perspective of property and argued in favour of a normative approach to property based on the notion of human flourishing. Moreover, I argued that economic development takings make up a separate category of interference with private property, deserving of special attention. I also placed this category in the theoretical landscape, by relating it to the theory of property presented in the first part of the chapter. Specifically, I argued that economic development takings raise questions that require us to depart from the individualistic, entitlements-based narrative that has tended to dominate in property theory.

There are many ways of thinking about the legitimacy of takings. Moreover, how one chooses to approach this in the abstract is likely to depend not only on one's legal training, but also on more overarching visions of society. Specifically, it seems that one's approach to the legitimacy question will invariably depend also on one's vision of the relationship between the government, the law, and the institution of private property in a democratic system. To ask what imbues an act of taking with legitimacy, is to ask how this relationship should be.

This chapter considers the question of legitimacy of economic development takings on the basis of a social function understanding inspired by the norm of human flourishing. To move towards a justiciable legitimacy standard on this basis, the chapter first consider a rough outline of existing approaches to legitimacy, based on evidence from England and Wales, the United States, and the European Court of Human Rights.

Specifically, the chapter starts by considering the idea that the legislature itself imbues each instance of a taking with legitimacy, as the result of a decision made in a legitimate manner within a democracy. This narrative has long held sway in England and Wales, giving rise to a focus on procedural aspects, leaving little room for substantive judicial scrutiny of takings. The doctrine of deference, in particular, has developed as an overarching norm that guide the courts when faced with controversial takings. %In England and Wales, this perspective carries great weight, particularly historically, when parliament itself would authorise most takings directly through so-called private Acts. 

The sheer size and complexity of the modern state, with its ever growing presence in the private sphere, puts this perspective of legitimacy under strain. To some extent, it can be upheld by a well-organised executive, compelled to remain faithful to parliament and the ideas of democracy. However, a threat to the stability and success of such a procedural approach can arise from the lack of any clearly defined safeguards to protect against institutional failure and substantive abuse. If property as an institution begins to falter, for instance because takings for profit become too prevalent, the courts might find themselves unable to intervene on behalf of those democratic ideals that motivate the principle of deference in the first place. This chapter will argue that recent cases of economic development takings in England illustrate this worry, suggesting the need for substantive approaches to legitimacy in the law.

Following up on this, the chapter goes on to consider the US, where the public use restriction is considered to be an important substantive limitation on the government's power to take property. I argue that a contextual approach to the public use requirement, based on broad assessments of local conditions, was prevalent among the state courts in early public use cases. This arguably also reflects a social function understanding of property, connecting the public use test to the property theorising in chapter 1. Moreover, I note that there has been a resurgence of extensive public use scrutiny after {\it Kelo}, particularly at the state level. However, this change appears to have been largely ineffective at curbing dubious uses of eminent domain. Specifically, it has been argued that recent reforms, and broad substantive standards such as the public use requirement, are likely to become only symbolic nods to the danger of abuse: hidden within the complex arrangements of modern government, business goes on more or less as before.\footnote{See generally \cite{somin09}.}

%I track the history of public use scrutiny in some depth, showing that it was widespread and extensive at the state level, especially until a contrasting position of almost unconditional deference to the legislature was adopted by the Supreme Court in the case of {\it Berman}. After this, at the federal level at least, the public use restriction was effectively stripped of its content.\footnote{See \cite{berman54}.} The eventual backlash of this came with {\it Kelo}, which was decided in keeping with precedent, but which gave rise to severe doubts among the justices, particularly those who looked at the history of the public use doctrine and how it had worked prior to {\it Berman}.

This in turn raises the issue of how to combine the institutional and the procedural perspective on legitimacy, to ensure that substantive standards actually translate into effective protection. This brings me to the third approach to legitimacy, which I call the institutional fairness approach. I argue that this approach has been adopted recently by the ECtHR in Strasbourg, as they have developed a system of pilot judgements to deal with their vastly increasing case load. The idea of such judgements is that the Court will focus on systemic problems, to determine whether they should order the state to take general measures to improve their own institutions. By doing this, the Court will protect itself from having to deal with many similar cases. Instead, it can move on to novel issues of principle that need to be considered.

Quite apart from the practical motivation behind this development, I argue that the institutional perspective on fairness that it conditions is the way forward towards testing for legitimacy in takings cases. It should work well because it allows courts to adopt a middle ground between the procedural and the substantive approach. I consider the case of {\it Hutten-Czapska v Poland} in some depth to argue for the merits of this approach.\footnote{See \cite{hutten06}.}

Following up on this, I consider the question of how the courts should proceed to assess the legitimacy of economic development taking against such an institutional fairness perspective.\footnote{There is not yet any case law on this from the ECtHR.} Building on a list of conditions due to Kevin Gray, I propose a concrete heuristic for this purpose. In addition to the original points made by Gray, I add three of my own, inspired by the discussion in this and the previous chapter. 

A legitimacy test can never provide more than a partial solution to the legitimacy problem. Specifically, in cases when the desire for economic development is a genuine reflection of democratic decision-making, the follow-up question is how to better enable the collective to communicate this desire to private owners, without resorting to eminent domain. I address this question by looking to the theory of governance for common pool resources, developed by Elinor Ostrom and others. Specifically, I note how the connection between this theory and property law suggests the possibility that new institutions should be introduced to allow the collective to push for economic development involving privately owned property. In fact, such a proposal has already been made, by Heller and Hills, who proposed that so-called {\it Land Assembly Districts} could replace the use of eminent domain in many cases when holdouts make economic development on privately owned land hard to implement.  

I analyse the proposal in some depth, pointing out problems to suggest that Land Assembly Districts are not the final answer to the legitimacy question for economic development takings. Specifically, I note the underlying tension between the ideal of self-governance and the fear of tyranny by local elites. This goes to show that the cure for illegitimacy, much like its diagnosis, depends on the circumstances, and what one regards as the property's proper function. In light of this, I believe the critical examination of Land Assembly Districts marks a natural end to this chapter, as well as to the theoretical part of this thesis as a whole.

\section{England and Wales: Legitimacy through Parliament}\label{sec:3:2}

In England and Wales, the principle of parliamentary sovereignty and the lack of a written constitutional property clause has led to expropriation being discussed mostly as a matter of administrative law and property law, not as a constitutional issue.\footnote{See generally \cite{taggart98}.} Moreover, the use of compulsory purchase -- the term used to denote takings in the UK -- has not been restricted to particular purposes as a matter of principle.\footnote{See, e.g., \cite[48-49]{waring09}.} The uses that can justify taking property by compulsion are those uses that parliament regard as worthy of such consideration.\footnote{See \cite[48-49]{waring09}.} However, as private property has typically been held in high regard, the power of compulsory purchase has traditionally been exercised with caution.\footnote{See \cite[47-48]{waring09}.}

In his {\it Commentaries}, William Blackstone famously described property as the ``third absolute right'' that was ``inherent in every Englishman''.\footnote{See \cite[134-135]{blackstone79}. The first right, according to Blackstone, is security, while the second is liberty.} Moreover, Blackstone expressed a very restrictive view on the appropriateness of expropriation, pointing out that it was only the legislature that could legitimately interfere with property rights. He warned against the dangers of allowing private individuals, or even public tribunals, to be the judge of whether or not the common good could justify takings. Blackstone went as far as to say that the public good was ``in nothing more invested'' than the protection of private property.\footcite[134-135]{blackstone79}

In terms of historical accuracy, Blackstone's claims about property in England and Wales can be questioned. Specifically, it has been argued that his description of property might be shaped not so much by practical reality as by political values gaining ground among the bourgeoisie after the decline of the feudal system.\footnote{See \cite[34-35]{waring09} (describing Blackstone's account as a ``myth'').} However, the fact remains that compulsory purchase powers appear to have been granted relatively infrequently during his time, with no great increase in prevalence until the industrial revolution and the birth of the modern state.\footnote{See \cite[15]{allen00}. That said, recent scholarship has pointed out that expropriation appears to have taken place more frequently than previously thought, particularly following the glorious revolution, see \cite{hoppit11}.} Moreover, the conferral of such powers would typically require parliamentary involvement on a case-by-case basis, a practice reflecting that takings of private property, although far from unheard of, were indeed considered draconian.\footnote{See \cite[43-46]{nulty12}.}

Interestingly, the procedure followed by parliament in takings cases often resembled a judicial procedure; the interested parties were given an opportunity to present their case to parliament committees that would then effectively decide whether or not compulsion was warranted.\footnote{See \cite[13-16]{allen00}.} On the one hand, the direct involvement of parliament in the decision-making is suggestive of a fundamental respect for property rights. But at the same time, parliamentary sovereignty meant that the question of legitimacy was rendered mute as soon as compulsory purchase powers had been granted. The courts were not in a position to scrutinize takings at all, much less second-guess parliament as to whether or not a taking was for a legitimate purpose.\footnote{See, e.g., \cite[643]{nulty12}.}

During the 19th Century, as an industrial economy developed, so-called {\it private} acts, granting compulsory purchase powers to specific legal persons, grew massively in scope and importance.\footnote{See \cite[204]{allen00}.} Railway companies, in particular, regularly benefited from such acts.\footnote{\cite[204]{allen00}. See generally \cite{kostal97}.} During this time, the expanding scope of private-to-private transfers for economic development led to high-level political debate and controversy.\footnote{See \cite[204]{allen00}.} Usually, it would attract particular opposition from the House of Lords.\footcite[204]{allen00} Interestingly, this opposition was not only based on a desire to protect individual property owners. It also often reflected concerns about the cultural and social consequences of changed patterns of land use.\footcite[204]{allen00}

Hence, the early {\it political} debate on economic development takings in the UK shows some reflection of a social function approach to property protection. At the same time, as society changed following increasing industrialisation, a more expansive approach to compulsory purchase would eventually emerge as the norm.\footnote{Arguably, the social function perspective helps explain why this happened. Indeed, the expanded use of private takings in England during the 19th century, particularly in connection with the railways, might have served a more easily justifiable social function than that commonly associated with economic development takings today. Waring, in particular, notes how railway takings tended to affect aristocratic landowners rather than marginalised groups (``unlike private takings today, the railway legislation was most likely to affect those who could best defend their property rights from attack''), see \cite[111]{waring09}.} The idea that economic development could justify takings gradually became less controversial.

Today, the law on compulsory purchase in England is regulated in statute. Hence, parliament rarely gets involved on a case-by-case basis, and the role of the courts is largely limited to the application and interpretation of statutory rules.\footnote{See \cite[116-121]{waring09}. Some common law rules still play an important role, such as the {\it Pointe Gourde} rule, which stipulates that changes in value due to the compensation scheme itself should be disregarded when calculating compensation to the owner. The rule takes it name the case of \cite{gourde47}. The underlying principle, including also statutory regulations with a similar effect, is referred to as the ``no scheme'' principle, see \cite{lawcom01}. The principle is found in many jurisdictions, see \cite{sluysmans14}. It is often quite contentious, and notoriously hard to apply in practice. For a recent clarification of (some aspects of) the principle, see \cite{waters04}. I note that a strict interpretation of the no-scheme principle effectively precludes benefit sharing between takers and owners, a phenomenon that is also relevant in the context of economic development takings. See generally \cite{dyrkolbotn15}.} Moreover, with respect to the question of legitimacy of takings more broadly, the starting point for English courts is that this is a matter of ordinary administrative law.\footnote{See \cite{taggart98}.} More recently, the \cite{hra98} adds to this picture, since it incorporates the property clause in P1(1) into English law. Even so, the usual approach in England is to judge objections against compulsory purchase orders on the basis of the statutes that warrant them, rather than constitutional principles or human rights provisions that protect property.\footnote{See \cite[121-132]{waring09}. The important statutes are the \cite{ala81}, the \cite{lca61}, the \cite{cpa65}, the \cite{tcpa90} and the \cite{pcpa04}.} It is typical for statutory authorities to include standard reservations to the effect that some public benefit must be identified in order to justify a compulsory purchase order, but the scope of what constitutes a legitimate purpose can be very wide. For instance, to justify a taking under the \cite{tcpa90}, it will generally suffice to argue that it will ``facilitate the carrying out of development, redevelopment or improvement on or in relation to the land''.\footcite[226]{tcpa90}

While various governmental bodies are authorised to issue compulsory purchase orders (CPOs), a CPO typically has to be confirmed by a government minister.\footnote{See \cite[48]{waring09}.} The affected owners are given a chance to comment, and if there are objections, a public inquiry is typically held. The inspector responsible for the inquiry then reports to the relevant government minister, who makes the final decision about whether or not it should be granted, and on what terms. The CPO may later be challenged in court, but then on the basis of the statute authorising it, not on the basis of whether or not the purpose mentioned in that statute is legitimate as such.\footnote{See, e.g., \cite[48-49]{waring09}. The typical way to launch an attack on a taking would be to argue that it serves a purpose that falls outside the scope of the statute authorising it, or, more subtly, that the administrative decision-maker took irrelevant purposes into account when granting the power. See, e.g., \cite{sainsbury10}.} 

That said, the idea that property may only be compulsorily acquired when the public stands to benefit permeates the system. Indeed, this has also been regarded as a constitutional principle, for instance by Lord Denning in {\it Prest v Secretary of State for Wales}.\footnote{See \cite[198]{prest82} (``I regard it as a principle of our constitutional law that no citizen is to be deprived of his land by any public authority against his will, unless it is expressly authorised by Parliament and the public interest decisively so demands.'').} Moreover, in {\it R v Secretary of State for Transport, ex p de Rothschild}, Slade LJ spoke of ``a warning that, in cases where a compulsory purchase order is under challenge, the draconian nature of the order will itself render it more vulnerable to successful challenge''.\footcite[938]{rothschild89}

In keeping with the principle of parliamentary sovereignty, this warning targets judicial review of administrative decision-making, not legislation. Despite this limitation, the English approach to legitimacy has traditionally proved quite effective in preventing controversy from arising with respect to the use of eminent domain.\footnote{See generally \cite{allen10}.} An underlying respect for private property, as well as the idea that the authority to interfere with it rests on the authority of parliament, appears to have influenced the decision-making framework and the surrounding administrative practices. Hence, legitimacy has become an objective to be pursued through legislation, regulation, and administrative practice, not judicial scrutiny.\footnote{For a more detailed analysis of how this works, noting, among other things, that higher levels within the executive are also meant to act as safeguards of private property, filling -- to some extent -- the possible role of courts in this regard, see \cite[85-100]{allen08}.}

However, England and Wales have also seen controversial economic development takings being challenged in court. Indeed, such cases appear to have become more frequent.\footnote{See generally \cite{gray11}.} For instance, in the case of {\it Alliance}, many properties were taken in order to facilitate the construction of a new stadium for the football club Arsenal.\footcite{alliance06} Some owners who stood to lose their business premises protested on the basis that the purpose was dubious, pointing also to the fact that the inspector in charge of the public inquiry had recommended against the takings.\footcite[6-7]{alliance06} Their arguments also invoked P1(1) of the ECHR, to overcome the limitations of traditional judicial review in England and Wales. However, these argument were all quite summarily rejected by the Court.\footnote{See \footcite[6-7]{alliance06}. For a critical discussion, describing the Court's assessment against P1(1) as ``worryingly brief'', see \cite{gray11}.}

Arguably, the {\it Alliance} case reflects a weakness of the English approach to legitimacy. This weakness, moreover, appears to go beyond whatever doubts one might have about the principle of parliamentary sovereignty applied to property as a constitutional and/or human right. Specifically, if the framework laid down or condoned by parliament greatly empowers the administrative branch, while failing to appropriately regulate administrative practices, the deference due to parliament might effectively become undue deference to the executive branch. If the {\it practice} of using compulsory purchase continues to expand in relation to for-profit undertakings, there appears to be a significant risk of abuse associated with broad powers granted to the executive to take property for economic development. Plainly, if values such as those expressed by Blackstone are discredited further, there appears to be a lack of alternative sources for legitimacy in a system so reliant on a narrative of pure procedure.

To some extent, it would be possible for the Supreme Court to develop a more restrictive stance on compulsory purchase to address this, within the established constitutional order. In fact, there are some signs that this might be about to happen, specifically with respect to the broad powers granted under the the \cite[226]{tcpa90}. In the case of {\it R (Sainsbury's Supermarkets Ltd) v Wolverhampton City Council}, Lord Walker cited {\it Kelo} and went on to comment that ``economic regeneration brought about by urban redevelopment is no doubt a public good, but ``private to private'' acquisitions by compulsory purchase may also produce large profits for powerful business interests, and courts rightly regard them as particularly sensitive''.\footnote{See \cite[82]{sainsbury10}.}

However, the outcome of the {\it Sainsbury} case arguably also underscores the weaknesses of an indirect approach to legitimacy through administrative law. Instead of relying on Lord Walker's observations about the sensitivity of economic development takings, the majority of the Court quashed the compulsory purchase order on the basis that the local government had taken into account promises that the taker had made regarding a regeneration project in a different part of town. This was regarded as contravening section 226 of the \cite{tcpa90}, which only directs attention at the potential for improvements on or in relation to ``the land'', i.e., the land that is subject to compulsory purchase. The reasoning behind the decision, therefore, rests largely on a technicality, not any substantive assessment of legitimacy.

On a more purposive assessment, the taking in {\it Sainsbury} should arguably even have been upheld: the owner and the taker were both large commercial companies, they each owned a share of a plot of land suitable for joint development, they both wanted to develop at the expense of the other party, and the taker appeared to have the best overall plan for the community. Ironically, the English approach resulted in such a taking being struck down as illegitimate, while the taking in {\it Alliance}, involving the displacement of local people in favour of a football club, received little or no scrutiny at all. In light of this, it seems that alternatives to the traditional idea of legitimacy should be considered, at least if one agrees with Lord Walker's characterisation of economic development takings as ``particularly sensitive''.\footnote{See \cite[82]{sainsbury10}.}

\section{The US: Legitimacy through Public Use}\label{sec:3:3}

By contrast to the situation in England and Wales, the US Constitution is a basis for judicial review also with respect to the federal and state legislatures. Considering its status as a basis for potentially extensive review, the Constitution is remarkably terse. The takings clause, arriving as the final clause of the fifth amendment, reads simply ``nor shall private property be taken for public use, without just compensation''.\footnote{See \cite{us}.}

The compensation requirement is clearly stated, if embryonic, but the takings clause is also understood to include the requirement that property may only be taken for ``public use''. This is the aspect of the clause that will interest me in this thesis, since it provides an anchor for legitimacy that is particularly relevant -- and contentious -- in relation to economic development takings.\footnote{The compensation requirement is also important, of course, but the problems is gives rise to are rather more technical, pertaining also more to the entitlements-aspect of property protection, not the social function dimensions I focus on in this thesis. For a more in-depth assessment of the compensation issue in the context of economic development takings, see \cite{dyrkolbotn15}.}
Specifically, the question is to what extent such takings offend against the clause: is a taking for economic development by a commercial company really a taking for ``public use''?

Going back to the time when the fifth amendment was introduced, there is not much historical evidence explaining why the takings clause was included in the Bill of Rights.\footnote{See \cite{fifth}.} Moreover, there is little in the way of guidance as to how the takings clause was originally understood. James Madison, who drafted it, commented that his proposals for constitutional amendments were intended to be uncontroversial.\footnote{See letters from Madison to Edmund Randolph dated 15 June 1789 and from Madison to Thomas Jefferson dated 20 June 1789, both included in \cite{madison79}.} Hence, it is natural to regard the takings clause as a codification of an existing principle, not a novel proposal. Indeed, several state constitutions pre-dating the Bill of Rights also included takings clauses, seemingly based on codifying principles from English common law.\footcite[See][299]{johnson11} 

%As Meidinger notes, the Americans had never really charged the British with abuse of eminent domain, and private property had tended to be respected, also in the colonies.\footcite[17]{meidinger80} This undoubtedly influenced early US law.

Just like English scholars at the time, early American scholars emphasised the importance of private property. James Kent, for instance, described the sense of property as ``graciously implanted in the human breast'' and declared that the right of acquisition ``ought to be sacredly protected''.\footnote{See \cite[see][257]{kent27}.} Indeed, the Supreme Court itself expressed similar sentiments early on, when it spoke of the impossibility of passing a law that ``takes property from A and gives it to B''.\footnote{This was a {\it de dicta} in \cite[388]{calder98}. See also \cite[310]{vanhorne95}.}

However, just as in England, this early US attitude changed in response to industrial advances and a desire for economic development. As the 19th century progressed, eminent domain was used more frequently, now also to benefit (privately operated) railroad operations, hydroelectric projects, and the mining industry.\footcite[23-33]{meidinger80} During this time, it also became increasingly common for landowners to challenge the legitimacy of takings in court, undoubtedly a consequence of the fact that eminent domain was used more widely, for new kinds of projects.\footcite[24]{meidinger80} 

Controversy over the public use requirement arose particularly often with respect to the so-called mill acts.\footnote{\cite[24]{meidinger80}. See also \cite[306-313]{johnson11} and \cite[251-252]{horwitz73}.} Such acts were found throughout the US, many of them dating from pre-industrial times when mills were primarily used to serve the farming needs of agrarian communities.\footnote{A total of 29 states had passed mill acts, with 27 still in force, when a list of such acts was compiled in \cite[17]{head85}. According to Justice Gray, at pages 18-19 in the same, the ``principal objects'' for early mill acts had been grist mills typically serving local agrarian needs at tolls fixed by law, a purpose which was generally accepted to ensure that they were for public use.} Following economic and technological advances, provisions originally enacted to serve local farming purposes were now being used by developers wishing to harness hydropower for manufacturing and hydroelectric plants.\footnote{See, e.g., \cite[18-21]{head85} and \cite[449-452]{minn06}.}

It is important to note, however, that mill acts could not be used to authorise large-scale compulsory transfer of natural resources from owners to non-owners.\footnote{See the discussion in \cite{head85}.} Rather, mill acts provided management tools that could be used to ensure that owners of water resources could make better use of their rights. This would sometimes involve allowing riparian owners to interfere with, or take a necessary part of, the property of their neighbours, e.g., by constructing dams that would flood neighbouring land.\footnote{See, e.g., \cite[265]{staples03}.} However, the primary purpose of most mill acts was to facilitate rational coordination among owners, to the benefit of their community as a whole. This point was frequently made by the courts to justify upholding takings on the basis of mill acts, including takings that would benefit the manufacturing industry.\footnote{See \cite{fiske31}. See also the discussion (including references to other cases) in \cite{head85}.}

%As the industrial use of mill acts increased in scope, the original aim of these acts gradually became overshadowed by the strength of the commercial interests involved. 
%This, in turn, lead to public use controversies arising in relation to provisions that had not previously raised any doubts.\footnote{See \cite{head86}.} The case law on
More generally, case law on public use from the state courts at this time was characterised by a highly contextual understanding of property protection and the meaning of public use.\footnote{See, e.g, \cite{scudder32} (taking upheld, but said that ``the great principle remains that there must be a public use or benefit. That is indispensable. But what that shall consist of, or how extensive it shall be to authorize an appropriation of private property, is not easily reducible to a general rule.''); \footcite[409]{seawell76} (taking for a mineral company upheld on the basis that mining was the ``greatest of the industrial pursuits'' in the state of Nevada and that the benefits of the industry were ``distributed as much, and sometimes more, among the laboring classes than with the owners of the mines and mills''.); \footcite[337]{ryerson77} (taking struck down, by a Court that was ``not disposed to say that incidental benefit to the public could not under any circumstances justify an exercise of the right of eminent domain''.\footcite[337]{ryerson77}. See also \cite{gray11} (with many references to state courts striking down takings as impermissible).} Arguably, the case law on public use from the states even deserves to be categorised as an early example of a legitimacy approach based on a social function understanding of property. Moreover, it was held to be of high quality, as indicated by the early Supreme Court jurisprudence on economic development takings, as discussed in the next section.

\subsection{Legitimacy as Discussed by the Supreme Court}\label{sec:3:3:1}

Initially, the Supreme Court held that the takings clause in the US Constitution did not apply to state takings at all.\footcite{barron33} Federal takings, on the other hand, were of limited practical significance since the common practice was that the federal government would rely on the states to condemn property on its behalf.\footcite[30]{meidinger80}

This changed towards the end of the 19th century, particularly following the decision in {\it Trombley v Humphrey}, where the Supreme Court of Michigan struck down a taking that would benefit the federal government.\footcite{trombley71} Not long after, in 1875, the first Supreme Court adjudication of a federal taking occurred, marking the start of the development of the federal doctrine on public use and legitimacy.\footcite{kohl75} 

At the same time, the Supreme Court began to hear takings cases originating from the states, first on the basis of the due process clause of the fourteenth amendment, introduced after the civil war.\footnote{See, e.g, \cite{head85}.} Later, in 1897, the Supreme Court held that state takings could be scrutinized also against the takings clause of the fifth amendment.\footnote{See \cite{chicago97}.}

\noo{ The early 20th century was a period of great optimism about the ability of {\it laissez faire} capitalism to ensure progress and economic growth, a sentiment that was reflected in the federal case law on eminent domain. A particularly clear expression of this can be found in {\it Mt Vernon-Woodberry Cotton Duck Co v Alabama Interstate Power Co}.\footcite{vernon16}  This case dealt with the legitimacy of condemnation arising from the construction of a hydropower plant. The Supreme Court held that it was legitimate, with the presiding judge arguing briskly as follows:

\begin{quote}The principal argument presented that is open here, is that the purpose of the condemnation is not a public one. The purpose of the Power Company's incorporation, and that for which it seeks to condemn property of the plaintiff in error, is to manufacture, supply, and sell to the public, power produced by water as a motive force. In the organic relations of modern society it may sometimes be hard to draw the line that is supposed to limit the authority of the legislature to exercise or delegate the power of eminent domain. But to gather the streams from waste and to draw from them energy, labor without brains, and so to save mankind from toil that it can be spared, is to supply what, next to intellect, is the very foundation of all our achievements and all our welfare. If that purpose is not public, we should be at a loss to say what is. The inadequacy of use by the general public as a universal test is established. The respect due to the judgment of the state would have great weight if there were a doubt. But there is none.\footcite[32]{vernon16}
\end{quote}

On the one hand, the Court notes the importance of deference to the {\it state} judgement (not specifically the judgement of the state legislature). On the other hand, it prefers to conclude on the basis of its own assessment of the purpose of the taking. This assessment, however, is not grounded in the facts of the case or the circumstances in Alabama. Rather, it is based on sweeping assertions about ``all our welfare'' and the desire to ``save mankind from toil that it can be spared''. This marks a contrast with the approach of state courts, as discussed in the previous subsection.
}

In federal takings cases, the Supreme Court showed little willingness to enforce a strict public use requirement. In {\it United States v Gettysburg Electric Railway Co}, a case from 1896, deference to the legislature in federal takings cases was referred to as a principle that should be observed unless the judgement of the legislature was ``palpably without reasonable foundation''.\footcite[680]{gettysburg96} 

Importantly, however, such a deferential stance was not adopted in cases originating from the states. In {\it Cincinatti v Vester}, a case from 1930, the Supreme Court commented that ``it is well established that, in considering the application of the Fourteenth Amendment to cases of expropriation of private property, the question what is a public use is a judicial one''.\footcite[447]{vester30}

In the earlier case of {\it Hairston v Danville \& W R Co}, from 1908, the same was expressed by Justice Moody, who surveyed the state case law and declared that ``the one and only principle in which all courts seem to agree is that the nature of the uses, whether public or private, is ultimately a judicial question.''\footcite[606]{hairston08} Justice Moody continued by describing in more depth the typical approach of the state courts in determining public use cases:

\begin{quote}
The determination of this question by the courts has been influenced in the different states by considerations touching the resources, the capacity of the soil, the relative importance of industries to the general public welfare, and the long-established methods and habits of the people. In all these respects conditions vary so much in the states and territories of the Union that different results might well be expected.\footcite[606]{hairston08}
\end{quote}

Justice Moody goes on to give a long list of cases illustrating this aspect of state case law, showing how assessments of the public use issue had been inherently contextual.\footcite[607]{hairston08} Following up on this, he points out that ``no case is recalled'' in which the Supreme Court overturned ``a taking upheld by the state {\it court} as a taking for public uses in conformity with its laws'' (my emphasis). After making clear that situations might still arise where the Supreme Court would not follow state courts on the public use issue, Justice Moody goes on to conclude that the cases cited ``show how greatly we have deferred to the opinions of the state courts on this subject, which so closely concerns the welfare of their people''.\footcite[606]{hairston08}

{\it Hairston} is important for three reasons. First, it makes clear that initially, the deferential stance in cases dealing with state takings was primarily directed at state courts rather than legislatures and administrative bodies. Second, it demonstrates federal recognition of the fact that a consensus had emerged in the states, whereby scrutiny of the public use determination was consistently regarded as a judicial task.\footnote{Indeed, {\it Hariston} provides the authority for {\it Vester} on this point. See \cite[606]{vester30}.} Third, it provides a valuable summary of the contextual approach to the public use test that had developed at the state level. 

The {\it Hairston} Court clearly looked favourably on the case law from state courts. Importantly, when a deferential stance was adopted, this was clearly contingent on the assumption that state courts would continue to administer the public use test with the required vigour. Despite this, {\it Hairston} would later be cited as an early authority in favour of almost unconditional deference to legislators.\footnote{In fact, it was cited in this way also by the majority in {\it Kelo}, see \cite[482-483]{kelo05}.} 

This happened in {\it US ex rel Tenn Valley Authority v Welch}, concerning a federal taking.\footcite[552]{welch46} The Court first cited {\it US v Gettysburg Electric R Co} as an authority in favour of deference with regards to the public use limitation.\footcite{gettysburg96} The Court then paused to note that {\it Vester} later relied on the opposite view, namely that the public use test was a judicial responsibility.\footcite{vester30} The Court then attempts to undercut this by setting up a contrast between {\it Vester} and {\it Hairston}, by selectively quoting the observation made in the latter case that the Supreme Court had never overruled the state courts on the public use issue.\footnote{See \cite[552]{welch46}.} Hence, {\it Hairston} is effectively used to argue against judicial scrutiny, in a manner that is quite incommensurate with the full rationale behind the Court's decision in that case.

Later, {\it Welch} was used as an authority in the case of {\it Berman v Parker}.\footcite{berman54} This case concerned condemnation for redevelopment of a partly blighted residential area in the District of Colombia, which would also condemn a non-blighted department store. In a key passage, the Court states that the role of the judiciary in scrutinizing the public purpose of a taking is ``extremely narrow''.\footcite[32]{berman54} The Court provides only two references to previous cases to back up this claim, one of them being {\it Welch}.\footnote{The other case, {\it Old Dominion Land Co v US}, concerned a federal taking of land on which the military had already invested large sums in buildings. The Court commented on the public use test by saying that ``there is nothing shown in the intentions or transactions of subordinates that is sufficient to overcome the declaration by Congress of what it had in mind. Its decision is entitled to deference until it is shown to involve an impossibility. But the military purposes mentioned at least may have been entertained and they clearly were for a public use'', see \cite[66]{dominion25}. A partial quote, to the effect that deference to the legislature is in order except when it involves an ``impossibility'', was used to justify the decision in \cite[240]{midkiff84}.}

Moreover, both of the cases cited were concerned with federal takings, while in {\it Berman} the Court explicitly says that deference is due in equal measure to the state legislature.\footcite[32]{berman54} It is possible to regard this merely as a {\it dictum}, since the District of Columbia is governed directly by Congress. However, {\it Berman} was to have a great impact on future cases. In effect, it undermined a large body of case law on judicial review of takings without engaging with it at all.

In {\it Hawaii Housing Authority v Midkiff}, the Supreme Court further entrenched the principle expressed in {\it Berman}.\footcite{midkiff84} Here the state of Hawaii had made use of eminent domain  to break up an oligopoly in the housing sector. Given the circumstances of the case, it would have been natural to argue in favour of this taking on the basis that it served a proper public purpose.

However, the Court instead decided to rely on the doctrine of deference, shunning away from scrutinizing the takings purpose. Justice O'Connor, in particular, observed that ``judicial deference is required because, in our system of government, legislatures are better able to assess what public purposes should be advanced by an exercise of eminent domain''.\footcite[244]{midkiff84}

Effectively, what had been a doctrine of deference towards state courts had now transformed into a doctrine of deference towards state legislatures (and, in practice, the executive branch). In light of this, it had to be expected that {\it Kelo} would be decided in favour of the taker.\footnote{In fact, as pointed out by Somin, the {\it Kelo} case represents a slight tightening of the earlier line on public use. See \cite{somin07}.} However, the history of the public use requirement tells us that this outcome was by no means inevitable. Hence, the question arises whether legitimacy can be increased by reviving the public use test. The next section sheds some light on this, on the basis of legislative developments in the US after {\it Kelo}.

\subsection{Economic Development Takings after {\it Kelo}}\label{sec:3:3:2}

Following {\it Kelo}, much attention was directed at the danger of eminent domain abuse in the US.\footnote{See generally \cite{somin09}.} Moreover, the {\it Kelo} decision itself proved extremely unpopular. Surveys show that as many as 80-90 \% believe that it was wrongly decided, an opinion widely shared also among the political elite.\footcite[2109]{somin09}

Many states responded by introducing reforms aimed at limiting the use of eminent domain for economic development.\footnote{For an overview and critical examination of the myriad of state reforms that have followed {\it Kelo}, I point to \cite{eagle08}. See also \cite{somin09}.} Within two years, 44 states had passed post-{\it Kelo} legislation in an attempt to achieve this.\footnote{See \cite{castle}.} Various legislative techniques were adopted. Some states, including Alabama, Colorado and Michigan, enacted explicit bans on economic development takings and takings that would benefit private parties.\footcite[See][107-108]{eagle08} In South Dakota, the legislature went even further, banning the use of eminent domain: ``(1) For transfer to any private person, nongovernmental entity, or other public-private business entity; or (2) Primarily for enhancement of tax revenue''.\footnote{South Dakota Codified Laws § 11-7-22-1, amended by House Bill 1080, 2006 Leg, Reg Ses (2006).}

In other states, more indirect measures were taken, such as in Florida, where the legislature enacted a rule whereby property taken by the government could not be transferred to a private party until 10 years after the date it was condemned.\footcite[809]{eagle08} Many states also offered lengthy lists of uses that were to count as public, designed to restrict the room for administrative discretion while allowing condemnations for purposes that were regarded as particularly important.\footcite[804]{eagle08}

%As Somin has pointed out, state reforms enacted by the public through referendums tend to be more restrictive than reforms passed through the state legislature.\footcite[2143]{somin09} Many of the more radical reform proposals, moreover, were not endorsed by any of the branches of government, but were initiated by activist groups as ballot measures.\footnote{In some US states, initiative processes make it possible for activist groups to put measures on the ballot without prior approval by the state legislature. See \cite[2148]{somin09}.} As Somin observes, the reforms taking place via this route would be comparatively strict, testifying to the power of direct democracy.\footnote{See \cite[2143-2149]{somin09}.}

{\it Kelo} has clearly had a great effect on the discourse of eminent domain in the US. However, the effects of the many state reforms that have been enacted are less clear. According to Somin, most of these reforms have in fact been ineffective, despite the overwhelming popular and political opposition against economic development takings.\footcite[2170-2171]{somin09} At the same time, property lawyers report a greater feeling of unease regarding the correct way to approach the public use requirement, expressing hope that the Supreme Court will soon revisit the issue.\footnote{See \cite{murakami13} (``Until the Supreme Court revisits the issue, we predict that this question will continue to plague the lower courts, property owners, and condemning authorities'').} 

Why have legislative reforms proved inadequate? Part of the reason, according to Somin, is that people are ``rationally ignorant'' about the economic takings issue.\footnote{See \cite[2170]{somin09}.} For most people, it is unlikely that eminent domain will come to concern them personally or that they will be able to influence policy in this area. Hence, it makes little sense for them to devote much time to learn more about it. This, in turn, helps create a situation where experts can develop and sustain a system based on practices that a majority of citizens actually oppose.\footcite[2163-2171]{somin09} To back up this analysis, Somin points out that surveys seem to show that people generally overestimate the effectiveness of eminent domain reform, by mistaking symbolic legislative measures for materially significant changes in the law.\footcite[2155-2157]{somin09}

Arguably, this also shows that the legislative approach so far, which has focused on introducing more elaborate and detailed versions of the public use restriction, need to be supplemented by different kinds of proposals. Specifically, it seems important to also target the structural processes that result in the taking of private land for economic development. After all, it is when we direct attention at the decision-making involved in bringing about actual takings that we will locate those stakeholders who cannot afford to remain rationally ignorant about eminent domain. %These processes, it seems, need to be imbued with greater legitimacy. %In particular, it might be that owners themselves should be granted a better chance to participate in the management of their property, even when this involves deliberating on, and possibly taking part in, large-scale development projects. After all, it is the feeling that owners' and their communities' feeling that they are being treated unfairly that tend to lie at the root of controversies surrounding takings for economic development.\footnote{For a similar perspective, see \cite{underkuffler06}.}

This points towards another perspective on legitimacy, whereby focus shifts towards the institutional setting where the relevant decisions are made. Importantly, cases such as {\it Kelo} suggest that this needs to involve more than administrative law and ideas about procedural due process. Specifically, institutional legitimacy appears to have an important substantive component whereby a decision is legitimate only in so far as it results from democratically legitimate decision-making within an administrative framework that is generally conducive to fair and proportional outcomes. Arguably, recent developments at the ECtHR point towards a perspective on legitimacy that emphasises this interconnectedness between substantive and procedural aspects of fairness at the institutional level.

\section{Recent Developments at the ECtHR: Legitimacy as Institutional Fairness}\label{sec:3:4}

It is often said that the P1(1) of the ECHR consists of three rules. The first rule guarantees a right to `peaceful enjoyment of possessions', the second rule regulates the legitimacy of `deprivation' and the third rule regulates how the states can legitimately `control the use of property'.\footnote{See \cite[61]{sporrong82}.}

When dealing with complaints pertaining to P1(1), the Court in Strasbourg will typically first consider which of these three rules it should apply.\footnote{See \cite[102-104]{allen05}.} However, as noted by Allen, it is not clear that this choice has any great significance for the outcome.\footnote{See \cite[104-105]{allen05}.} In practice, the evaluation of legitimacy proceeds in much the same way regardless of which rule is used, with an emphasis on the {\it fair balance} that needs to be struck when states interfere with private property rights.\footnote{See \cite[[103]{allen05}. It is also typically assumed that an interference is only legitimate when it takes place for an appropriate purpose, but here the ECtHR has consistently maintained a deferential stance, pointing to the `wide margin of appreciation' that the member states enjoy in this regard. See, e.g., \cite[54]{james86}.}

In this regard, it is important to note that the Court has gradually adopted a more active role in assessing whether or not particular instances of interference are proportional and able to strike a fair balance between the interests of the public and the property owners. As argued by Allen, this has caused P1(1) to attain a wider scope than what was originally intended by the signatories.\footcite[1055]{allen10}

In the early case law behind this development, the focus was predominantly on the issue of compensation, with the Court gradually developing the principle that while P1(1) does not entitle owners to full compensation in all cases of interference, the fair balance will likely be upset unless at least some compensation is paid, based on the market value of the property in question.\footnote{See \cite[103]{scordino06}. The case also illustrates that the Court has adopted a fairly strict approach to the question of when it is legitimate to award less than full market value.}

However, the fair balance test encompasses more than the issue of compensation. In particular, the hunting cases discussed in Chapter 1 show that the Court in Strasbourg is willing to reflect broadly on the context and purpose of interference, to critically assess the social function of the taking.\footnote{See Section \ref{sec:hunt} of Chapter 1.} Moreover, institutional aspects of fairness have come to play an important role in the Court's reasoning in some other recent cases involving property.\footnote{See \cite{hutten06,lindheim12}.} This is particularly clearly demonstrated by the case of {\it Hutten-Czapska v Poland}.\footnote{See \cite{hutten06}.}

\subsection{Legitimacy {\it Erga Omnes}}\label{sec:3:4:1}

The striking conclusion in {\it Hutten-Czapska v Poland}, underscoring the institutional turn at the ECtHR, was that the case demonstrated `systemic violation of the right of property'.\footcite[239]{hutten06} The case concerned a house that had been confiscated during the Second World War. After the war, the property was transferred back to the owners, but in the meantime, the ground floor had been assigned to an employee of the local city council.\footcite[20-31]{hutten06} The state implemented strict housing regulations during this time, which eventually led to the applicant's house being placed under direct state management.\footcite[20-31]{hutten06} Following the end of communist rule in 1990, the owners were given back the right to manage their property, but it was still subject to strict regulation that protected the rights of the tenants.\footcite[31-53]{hutten06} In addition to rent control, rules were in place that made it hard to terminate the rental contracts.\footcite[20-53]{hutten06} 

After an in-depth assessment of the relevant parts of Polish law and administrative practice, the Grand Chamber of the ECtHR concluded that there had been a violation of P1(1). Importantly, they did not reach this conclusion by focusing on the owners and the interference that had taken place with respect to their individual entitlements. Rather, they focused on the overall character of the Polish system for rent control and housing regulation, as exemplified by the applicant's situation.

Specifically, the consequences for the owners were considered not in isolation, but in order to shed light on a broader question of sustainability.\footcite[60-61]{hutten06} The Court was particularly concerned with the fact that the total rent that could be charged for the house in question was not sufficient to cover the running maintenance costs.\footcite[224]{hutten06} In particular, it was noted that the consequence of this would be ``inevitable deterioration of the property for lack of adequate investment and modernisation''.\footnote{\cite[224]{hutten06}.}

In the end, the Court highlighted how three factors combined to bring both owners and their properties  to a precarious position. First, the rigid rent control system made it hard to sustainably manage rental property. Second, tenancy regulation made it hard for owners to terminate tenancy agreements. Third, the Court noted that the state itself had set up many tenancy agreements during the days of direct state management, shedding doubt on the fairness of the obligations that these contracts imposed on owners.\footcite[224-225]{hutten06} 

The Court's reasoning in {\it Hutten-Czapska} is also interesting because of how the `social rights' of the tenants is placed on an equal footing to the property rights of the owners.\footcite[225]{hutten06} Arguably, property rights and social rights are not considered merely as separate sets of entitlements, locked in opposition to one another. In the reasoning of the Court they also appear as mutually dependent social functions, both hampered by an unsustainable approach to property and housing during the communist era and beyond.\footnote{Specifically, the Court attached great significance to the finding that rents were too low to cover maintenance costs, see \cite[224]{hutten06}. A lack of incentives for maintenance is clearly a threat to tenants as much as to owners, illustrating the interdependence between the two groups. Despite this, when summing up their reasoning in broad strokes, the Court itself reverts back to a traditional narrative when it speaks about the `conflicting interests of landlords and tenants'. See \cite[225]{hutten06}.}

In this regard, the Court places considerable weight on the precarious situation of the owners and `the absence of any legal ways and means making it possible for them either to offset or mitigate the losses incurred in connection with the maintenance of property or to have the necessary repairs subsidised by the State in justified cases'.\footnote{See \cite[224]{hutten06}.} Moreover, the Court commented that the `burden cannot, as in the present case, be placed on one particular social group, however important the interests of the other group or the community as a whole'.\footnote{See \cite[225]{hutten06}.} Importantly, however, the Court did not set out to censor the political reasoning that motivated the rent control scheme, but rather focused on the fact that it had not been implemented properly.\footnote{See \cite[224]{hutten06}.}

On this basis, the Court concluded that there had been a systemic violation of P1(1), and ordered Poland to take measures to rectify the `malfunctioning of Polish housing regulation'.\footnote{See \cite[237]{hutten06}. The basis relied on for formulating such an order was Article 41 in the ECHR, first used in this way in the case of \cite{broniowski05}.} Hence, the lack of legitimacy was pronounced with a kind of {\it erga omnes} effect, establishing an obligation for Poland directed at all its citizens in equal measure, not merely the applicant.\footnote{There was some dissent as to whether or not this was an appropriate response, with Judge Zagrebelsky in particular arguing against it on the grounds that it would see the Court ``entering territory belonging specifically to the realm of politics''. See \cite{hutten06}.} Judgements of this kind, known as ``pilot judgements'', have now gained formal recognition as a distinct procedural form that the ECtHR can use to address systemic problems.\footnote{See generally \cite{leach10}.}

The institutional approach conditioned by the introduction of pilot judgements might point to the core function that the ECtHR is likely to serve in the future.\footnote{See, e.g., \cite{greer12} (arguing that a ``constitutional pluralism'' approach to adjudication -- better filtered, more principled, yet still context sensitive --  is the way ahead for the ECtHR).} Indeed, the ECtHR will hardly be able to protect human rights in Europe on a case-by-case basis. Nor would it seem appropriate for it to do so, given its remoteness to local conditions and its relative lack of democratic accountability. However, when the Court is able to identify systemic failures that look set to systematically give rise to imbalances and unfairness, it seems appropriate that it should take action.

This is particularly clear when, as in the case of {\it Hutten-Czapska}, the Court notes that the applicants have insufficient options available for achieving a fair balance by appealing to institutions within the domestic legal order. In such cases, it seems appropriate for the Court to demand a change at the level of the state's own institutions, giving rise to a broad duty for the state to improve those institutions. Moreover, by scrutinizing the procedures and principles that the states apply when fulfilling this duty, it is likely that the Court will still be able to steer and unify the development of the case law on human rights, at least to the extent that this is required to meet minimum standards.

Against the deferential implications of this shift of attention, it could be argued that the judicial or administrative bodies of the signatory states can easily circumvent their obligations by providing superficial reforms or biased assessments of the facts in human rights cases, to avoid embarrassment for the state's political or bureaucratic elites. However, this might then be raised as a more procedurally oriented complaint before the ECtHR, perhaps also against Articles 6 (fair trial) and 13 (effective remedy).\footnote{I note that this also fits with recent developments at the ECtHR, toward somewhat broader scrutiny under Article 6, see \cite{khamidov07}.}  

In this way, the Court can streamline its functions, by always aiming to direct attention at issues that arise at a higher level of abstraction.\footnote{A similar argument was given by Judge Zupan\u{c}i\u{c} in \cite{hutten06} (``Is it better for Poland to be condemned in this Court 80,000 times and to pay all the costs and expenses incurred in 80,000 cases, or is it better to say to the country concerned: “Look, you have a serious problem on your hands and we would prefer you to resolve it at home...! If it helps, these are what we think you should take into account as the minimum standards in resolving this problem...”? Which one of the two solutions is more respectful of national sovereignty?'').} This, in my view, seems highly desirable. The ECtHR should not aim to micromanage the signatory states, particularly not in relation to a norm such a P1(1), which the Court itself regards as highly dependent on context. By shifting attention towards institutional fairness, the Court can avoid getting stuck in deference to the states without overstepping its bounds with regards to the democratic process.

Indeed, the case of {\it Hutten-Czapska} is highly suggestive of the merits of such a perspective, not only because of the special measures ordered, but also because the Court reasoned on the basis of institutional information to identify systemic weaknesses of Polish housing regulation.\footnote{Specifically, it seems that the shift signalled by recent cases on property at the ECtHR does not end with a new take on remedies, but also signals some changes in the way the Court approaches the fair balance determination under P1(1). This seems natural; if the Court looks for systemic violations, not (only) individual transgressions, its substantive assessments of fairness will naturally be influenced.} Another example is the recent case of {\it Lindheim and others v Norway}.\footnote{See \cite{lindheim12}.} Here the applicants complained that their rights had been violated by a Norwegian act that gave lessees the right to demand indefinite extensions of ground leases on pre-existing conditions.\footcite[119]{lindheim12}

The Court agreed that this was a breach of P1(1). Moreover, it engaged in the same form of assessment as it had adopted in {\it Hutten-Czapska}. Specifically, it concluded that the Ground Lease Act 1996 as such was the underlying source of the violation. The problem was not merely that this act had been applied in a way that offended the rights of the applicants. In light of this, the Court did not only award compensation, it also ordered that general measures had to be taken by the Norwegian state to address the structural shortcomings that had been identified.\footnote{See \cite{lindheim12}.}

The Court also commented that its decision should be regarded in light of ``jurisprudential developments in the direction of a stronger protection under Article 1 of Protocol No. 1''.\footcite[135]{lindheim12} However, in light of the change in perspective that accompanies this development, it is interesting to ask in what sense exactly the protection is stronger. In particular, it is not {\it prima facie} clear that the Court's remark should be read as a statement expressing a change in its understanding of the content of individual rights under P1(1). 

Rather, it may be read as a statement to the effect that the Court has assumed greater authority to address structural problems under that provision. This might even allow the Court to conclude that a violation has occurred due to structural unfairness, even when it is not possible to trace this back to any abnormal decision that specifically targets the individual entitlements of applicants.

If this is true, it could make a big difference in cases involving takings for economic development. As illustrated by Justice O'Connor's dissent in {\it Kelo}, a main concern here is that such takings are likely to have ``perverse'' consequences at the structural level, because they lack democratic merit. \footnote{To quote Justice O'Connor's dissent in {\it Kelo}, see \cite{kelo05}.} In light of cases such as {\it Hutten-Czapska} and {\it Lindheim}, I think the ECtHR would have been likely to approach {\it Kelo} in a manner consistent with Justice O'Connor's approach.

Whether they would reach the same conclusion seems more uncertain, particularly since confidence in the states' ability and willingness to regulate private-public partnerships might be higher in Europe than in the US.\footnote{For a discussion from the point of view of English law, arguing that the prevailing regulatory regime limits the risk of eminent domain abuse largely through regulation of the takings power rather than strict property protection, see \cite{allen08}.} However, it seems unlikely that the ECtHR would follow the majority in {\it Kelo}, by simply deferring to the determinations made by the granting authority. Rather, Justice O'Connor's predictions about the fallout of the {\it Kelo} decision would likely have been of significant interest also to the justices at the Court in Strasbourg.

To conclude, I think notions of institutional fairness can help us locate a welcome middle ground between largely procedural notions of justiciable legitimacy, such as those found in England and Wales, and substantive notions, such as those found in the US. The question remains how Courts adopting such a middle ground should proceed when presented with a concrete case of alleged eminent domain abuse. In the next section, I present a possible heuristic.

\section{The Gray Test}\label{sec:3:5}

Pointing to early US case law on public use as a ``laboratory of elementary proprietary ideas'', Kevin Gray builds on the evidence found there to provide a set of conditions for recognising what he calls ``predatory takings''. \footnote{See \cite[28-30]{gray11}.} His conditions capture key aspects of eminent domain abuse that I believe should be recognised by a theory of economic development takings inspired by the notion of human flourishing and an institutional perspective on legitimacy. Below, I briefly present the criteria proposed by Gray, as well as three riders that I believe suggest themselves on the basis of the discussions presented earlier in this and the previous chapter. I will refer to the resulting set of conditions as the {\it Grey test}, to be understood as a proposed general heuristic for assessing the legitimacy of takings, especially in situations when there are strong commercial interests present on the taker side.

Several combinations of conditions might be sufficient to justify designating a taking as eminent domain abuse. The purpose of the Gray test is not to produce a definite set of such conditions that provide a final answer in any case. Rather, the aim is to provide a heuristic to facilitate concrete assessment against the social, economic and political circumstances surrounding the taking in question. If an economic development taking represents an abuse of power, one would expect it to run afoul with regard to some, and probably several, of the criteria set out in the following points.

\subsubsection*{Balance of Power among the Parties}

In a typical case of eminent domain abuse, the parties that stand to benefit will be more economically and politically powerful than those from whom property is taken.\footnote{See \cite[30-31]{gray11}. Gray himself omits any explicit mention of political power, but it is present in Justice O'Connor's dissent in {\it Kelo}, and in my view clearly belongs here.} This can be reflected in the takers' ability to solicit legal assistance and other services to defend the taking, as well as in the  owners' inability to launch a coordinated defence.\footnote{See \cite[30-31]{gray11}.} If there is an imbalance of power, this is particularly likely to be noticeable early on, during the planning stages, before the decision to condemn has actually been made. 

After the decision has been made, the procedural position of the owners might improve. However, this might not serve to restore any meaningful balance between the parties; when special procedural protections kick in, it will often be too late for the owners to launch an effective defence against the taking. For instance, strict rules concerning cost reimbursement for costs incurred {\it after} the decision to take has already been made, is not a sufficient response to an imbalance of power, especially not in legal systems that do not offer extensive judicial review of takings purposes.

More generally, a possible imbalance of power should be assessed against the decision-making process as a whole, going back to the first initiative made for taking the property in question. A critical  assessment of what role the owners have played in the decision-making process is a good way to uncover more information about imbalances of power, and whether or not such imbalances could have unduly influenced the outcome.

\subsubsection*{The Net Effect on the Parties}

As Gray notes, a hallmark of eminent domain abuse is that the net effect of the taking is a ``significant transfer of valued resource from one set of owners to another''\footnote{See \cite[31]{gray11}.} In itself, this is not a conclusive sign of abuse, but it directs us to ask two important questions. First, we should inquire critically into the main purpose of the taking. Is the transfer of resources between the parties an acknowledged motive or an unacknowledged side-effect of some ostensibly distinct public purpose?

In the latter case, it might be clear that the public purpose is only a pretext for benefiting the taker, in which case it counts as clear evidence of abuse. In less obvious cases, if the transfer of resources arising from the fulfilment of the public purpose was not properly discussed and critically examined by the decision-maker, this too can point towards predation.

The assessment will be different if redistribution of (control over) resources is openly acknowledged as part of the rationale justifying eminent domain. In such cases, it is pertinent to ask further  questions about the economic and social status of the parties, and the structure of the decision-making process, to shed light on whether the redistributive motive itself appears democratically legitimate. If there is eminent domain abuse, one would except the taking to fail to stand up to scrutiny in this regard.

In some cases, it might be debatable whether a taking passes the net effect test. However, the importance of scrutiny is still significant, since it helps bring the crucial questions into the open, thereby ensuring higher quality of the decision-making regarding the taking. Indeed, if the Gray test is applied at an early stage of the proceedings, this in itself might help increase acceptance of the decisions reached. Making room for more extensive legitimacy tests in takings law might well end up bolstering the government's power to take property, as long as the power is used faithfully.

\subsubsection*{Initiative}

In many suspicious economic development takings, the party benefiting commercially from the taking is the party that initially made the suggestion for using eminent domain.\footnote{See \cite[32]{gray11}.} In uncontroversial cases, on the other hand, the initiative tends to come from some government body that seeks to pursue a specific policy goal, e.g., to provide a public service or bestow a benefit on a particular group that is found to be in need of support. The contrast between this and cases when the initiative lies with the commercial beneficiaries themselves point to a disturbance of the decision-making underlying the decision to use eminent domain. As such, it is an important hallmark of abuse.

To investigate further under this point, one should take into account the wider social and political context of the taking, particularly the position of the parties involved. If the beneficiary is both more powerful and privileged than the owners {\it and} takes the initiative for the taking, this is clearly a sign pointing towards predation. On the other hand, if the beneficiaries are marginalised groups who could only expect any consideration if they were to take the initiate themselves, the situation might have to be viewed differently. %In these cases, the fact that the system leaves room for marginalised non-owners to acquire property interests might have to be considered a strength rather than a weakness. Still, as discussed in later points, the appropriateness of using eminent domain for redistributive purposes can be questioned, even if the redistributive goal itself appears democratically legitimate. In such cases, however, the question of legitimacy is unlikely to turn on the initiative test.

\subsubsection*{Location}

The location, in a broad sense of the word, of the property that is taken, can be a strong indicator that eminent domain is inappropriate.\footnote{See \cite[33-34]{gray11}.} For instance, cases involving the taking of dwellings are naturally more suspect than cases involving the taking of barren or unused plots of land. Similarly, the taking of property that is important to the subsistence of the current owner should raise the bar for when a taking may be considered legitimate. Moreover, if the taker's choice of location appears to be one of convenience rather than necessity, this points towards predation. It is particularly telling if alternative locations would be less intrusive, or obviate the need for using eminent domain altogether.

%The location of the property can also attain relevance independently of the current owner. For instance,

On the other hand, the location of the property can sometimes point towards {\it increased} legitimacy of a taking that would otherwise appear suspect. This might be the case, for instance, if the property that is taken has special value to the taker or the community specifically because of its strategic importance with respect to the taker's own property or the rights of non-owners.\footnote{For instance, if riparian owners cannot make rational use of the water flowing over their land without intruding on the land of their neighbours, using eminent domain to resolve this might be considerably less suspect than other kinds of economic development takings. This particular scenario was much discussed in the US during the 19th century, in relation to mill acts which authorised neighbour-to-neighbour takings of limited property rights needed for development. See the discussion in Chapter 2, Section \ref{sec:us}. For an example involving non-owners, consider the increased legitimacy of interference in cases when property rights frustrate efforts to secure rights of non-owners, such as rights to drinking water in cases when riparian owners prevent non-owners access to water for their basic needs.} The proper balance of burdens and benefits might still be upset, but a taking that fits smoothly into a `special value' narrative will be less suspect than one that does not.

\subsubsection*{Social Merit}

As Gray notes, a taking that is hard to justify on the basis of its social merits is more likely to be predatory.\footnote{See \cite[34]{gray11}. Gray writes of lap-dancing clubs and cigarette factories as examples of purposes that are suspect. Importantly, such purposes might well fulfil a public interest requirement via the economic development narrative, yet still fail a social merit test that focuses rather on the social dimensions of the use to which the property will be put.} This asks for closer scrutiny of the kinds of public interests that can be used to justify a taking. If the justification narrative surrounding a taking revolves solely around `trickle-down' effects and the successful business ventures that the taking will facilitate, there is reason to be suspicious. Specifically, if the taking cannot sustain a social merit narrative, whereby attention is shifted away from purely economic considerations, this is a strong independent indication that the taking might count as predation.

The point here is not that the language of social merit should replace the language of public use or public interest as some kind of conclusive test of legitimacy. Rather, the point is that one should always be encouraged to analyse takings specifically in terms of non-economic, social, effects. This is particularly important in difficult cases, because it can help us arrive at a better understanding of where exactly the taking sits on the gray scale between admissible governance and predatory exploitation. 

%If a taking appears to stand up to scrutiny only when embedded in a purely economic narrative, this in itself suggests a lack of legitimacy. Indeed, even if one concedes that incidental economic effects are relevant, it seems clear that however one circumscribes a notion such as public use, this notion certainly encompass {\it more} than merely those incidental economic effects that tend to occupy center stage in legitimacy disputes.\footnote{Indeed, it bears emphasising that those arguing against economic development takings might achieve more by emphasising non-economic aspects, compared to arguing that incidental economic benefits should not at all be considered relevant as a justification for eminent domain.}

\subsubsection*{Environmental Impact}

According to Gray, a typical feature of eminent domain abuse is that it has an adverse environmental impact. Moreover, a typical feature of eminent domain abusers is that they show disregard for such adverse affects.\footnote{See \cite[34]{gray11} (``predatory takers tend to be relatively unperturbed if they lay waste to the earth'').} This is an additional element that pertains specifically to the status of the taker, asking us to consider whether it is appropriate to grant their activities public interest status. It is not primarily a question of how the development stands with regard to environmental regulation. Rather, what is at stake is whether or not the characteristics of the taker and the development plans make it appropriate to use the power of eminent domain. 

It might be appropriate to use environmental law as a starting point, but the relevant environmental standard with regard to the legitimacy question should be drawn up more strictly than the standards generally applied to the type of development in question. Indeed, one should be entitled to expect {\it more} in terms of environmental awareness and concern from a developer and a development plan that benefit from the power of eminent domain. 

Arguably, the mere fact that takers engage in active lobbying for leniency in relation to environmental standards can be enough to shed doubt on the proposition that they act in the public interest. What might otherwise be considered natural and admissible behaviour for a common commercial company can be improper or inadmissible behaviour for one that benefits from eminent domain powers. I note that this particular observation has general import, pertaining to a potentially wider set of obligations that takers may be expected to take on, not only environmental ones. This brings me to the first rider that I propose to add to Gray's original evaluation points.

\subsubsection*{Rider 1: Regulatory Effects}

As discussed in Chapter 2, property has an important regulatory effect, also outside the realm of positive law. This effect typically changes following a taking, sometimes quite dramatically.  For instance, if locally owned property is taken by external commercial actors for high-intensity commercial use, the post-taking regulatory status of the property will most likely be completely different to its status prior to the interference. Moreover, the changed status might have as much to do with informal social functions as it has to do with positive regulation.

It might be, for instance, that the property in question is found in a jurisdiction that emphasises  the freedom of owners to do as they please without state interference. In this case, the fallout of allowing external commercial actors to take locally owned property can be particularly severe, as the new owner is likely to be unconstrained by locally grounded systems for sustainable resource management. In these cases, there is a risk that there will be a `tragedy of the taken', arising from how the taking undermines an important building block of sustainability. 

Indeed, a society based on egalitarianism and strict limits on state interference might find it especially difficult to appropriately restrain the actions of actors who use the eminent domain power to accumulate property for high-intensity use.\footnote{This problem can of course arise independently of the use of eminent domain, e.g., in the context of land grabs arising from voluntary or semi-voluntary transactions. However, the situation appears particularly problematic if the state itself is complicit in bringing about the problem, by undermining property's social function through the use of the takings power.} If this is resolved by increasing the state's power to interfere with private property through regulation, the effect can be a further undermining of local management frameworks, increased subsequent use of eminent domain, and a general spreading and amplification of the democratic deficit already inherent in the original act of taking.

A different regulatory concern is that the legal status of the property can change, for instance because the development in question brings it under the scope of different rules. If so, it should be examined whether the new rules offer weaker protection for the local community, the environment, or the general public interest, in which case it reflects badly on the initial decision to use eminent domain.%\footnote{The case study of Norwegian waterfall expropriation will offer an example of this mechanism, c.f., Chapter 5, Section \ref{sec:x}.}

\subsubsection*{Rider 2: Impact on Non-Owners}

Following up on the theoretical arguments made in Chapter 1, it is appropriate to direct special attention at the status of non-owners directly affected by economic development takings. It is of particular interest to ascertain whether or not the interests of such non-owners were given due consideration prior to the decision to use eminent domain. If their interests appear to have been neglected, or have not been considered at all, there is additional reason to be sceptical of the purported public interest of the taking. Indeed, just as disregard for the environment is a typical sign of predation, a general disregard for local non-owners is also an indicator of abuse.

To shed further light on this, one might first ask what role non-owners played in the decision-making process. If the non-owners directly affected by the taking were allowed to express their opinion, and enjoyed some measure of influence, this can enhance legitimacy. If, on the other hand, the most immediately affected members of the public were not consulted, or not given a proper voice in the proceedings, it indicates abuse.

There is also an important substantive aspect to consider: how is the taking going to affect property dependants without recognised ownership rights? If it is clear that they will suffer severe adverse effects, for instance by being displaced from their homes or by loosing their livelihoods, this must be counted as an indication of predation irrespective of any mitigating procedural arrangements used to create the impression of ``consultation'' or the like.

Importantly, it also follows from the social function perspective that awarding compensation can not by itself excuse shortcomings in this regard. If people are displaced, for instance, the fact that new dwellings are provided somewhere else does not detract from the fact that a community has been destroyed. It is possible that the needs of the public necessitate such a drastic interference with property's proper function, but this should then at once give rise to a more in-depth scrutiny of legitimacy. Moreover, the bar to pass the legitimacy test should be raised considerably in such cases.

\subsubsection*{Rider 3: Democratic Merit}

Perhaps the most important characteristic to consider when assessing the legitimacy of a taking is its democratic merit. In an important sense, putting a taking to the test against this measure serves to encapsulate all the other points raised above. Specifically, it asks us to consider the totality of these factors in order to judge whether good governance standards have been observed within a system based on democratic decision-making. The inquiry made in this regard should not be focused on second-guessing government policies, but should compel us to take seriously the idea that a commitment to democracy places real constraints on the exercise of government power. In this way, an overarching focus on democratic merit can hopefully render the principles of scrutiny expressed by the Gray test as a possible template for courts across different jurisdictions, with respect to both constitutional and human rights provisions.

The overarching question that arises with respect to democratic merit is whether the taking in question 
can be said to arise from a legitimate process of decision-making, in the pursuit of a fair and equitable outcome. It bears emphasising that in line with a more modern appreciation of the meaning of democracy and human rights, the relevant assessment under this point involves both procedural and substantive elements. Fairness in itself is a constraint on the democratic process, particularly when fundamental economic and social rights are involved. At the same time, the notion of democratic merit rightly brings procedural questions to the foreground. Indeed, it might be a weakness of Gray's original proposal that it does not single out procedural issues for special consideration.

On the one hand, it is inappropriate to reduce the takings question to a matter of administrative law. But on the other hand, the way in which the taking decision was made can often tell us much about its legitimacy, including how it stands with regard to broader notions of fairness. It is particularly important, in this regard, to inquire into the position of local owners and communities during the planning process leading up to the decision to use eminent domain. In the context of property as a human right, moreover, a stricter standard might be appropriate here, compared to that which would otherwise follow from administrative law.

Importantly, our commitment to property as a social institution requires us to take into account that the owners generally make up the group of people who will be most directly affected by any decision involving the future of their property. As such, they should normally be granted a decisive voice in decision-making processes leading up to economic development. At the same time, the social function account leaves room for recognising that this presumption in favour of emphasising the rights of owners can be defeated by the context. It is clear, for instance, that the substantive interests of absentee landlords might be limited compared to the substantive interests of local non-owners who depend more directly on the property for their livelihoods. In these cases, the social function approach allows us to recognise that a taking might have significant democratic merit, even if it is based on a form of decision-making that prioritises the interests and participation rights of non-owners.

Nevertheless, within a system based on private property rights it will always be appropriate to show caution in this regard. The presumption should always be, within such a system, that the owners are the primary stakeholders in decision-making processes involving their property. Moreover, if this presumption is defeated, it would usually point to a structural weakness of property's function within society, a weakness that should arguably be addressed by more general reforms of property, not by inflating the state's power to undermine it. If caution is not observed here, property can soon become a less secure basis on which to support local communities, including those marginalised groups that are most in need of protection from predators.

By itself, however, a legitimacy test cannot make property a more secure basis for promoting good outcomes in cases when the public desires economic development. What the Gray test provides is a list of possible symptoms to look our for when attempting to diagnose a suspected case of eminent domain abuse. Hopefully, this can help flag problems and limit damages, but it can not be regarded as a solution, especially not in cases when the public's apparent desire for economic development is a genuine reflection of a democratic commitment. 

In short, after diagnosing a lack of legitimacy, the question becomes how to find a cure, preferably without harming the patient, i.e., the democratic system. In the next section, I consider this challenge in more depth, premised on the idea that there is a need for alternatives to eminent domain in cases when the collective wishes to take decisive steps to promote economic development on privately owned land.

\section{Alternatives to Takings for Economic Development}\label{sec:3:6}

As mentioned briefly in Chapter 2, the work of Ostrom and others on common pool resources suggests that sustainable resource management can often be better achieved through local self-governance than through markets or states.\footnote{See generally \cite{ostrom90}. For a recent exposition of the main ideas, placing the work in a broader academic context, see the revised version of Ostrom's Nobel Lecture, \cite{ostrom10}.} The connection between this work and property theory is highly interesting, and has been explored in some recent work, particularly by US legal scholars.\footnote{See generally \cite{rose11,fennel11}.} As these scholars have observed, the connection can be made at a very high level of generality. Indeed, in a democracy, property as such has a kind of (partial) commons structure, since property as an institution depends on the collective choices we make regarding the legal order.\footnote{For similar observations, see \cite[51]{rose90}; \cite[577]{heller01}.} In cases when property is made subject to eminent domain, this perspective becomes particularly salient, since then the collective explicitly withdraws its backing for the rights of the owner, in favour of collective decision-making about the future of the property in question.\footnote{A common pool resource is typically identified by the fact that exclusion is difficult or costly, while use can cause depletion (and hence should be limited), see, e.g., \cite[57]{ostrom10b}. Hence, the mere fact that some property is apt to be regulated, or taken, by the collective, demonstrates that property has common pool characteristics (although these might be imposed by the polity, rather than arising from the nature of the underlying good).} Moreover, in case of an economic development taking, the property in question typically pertains to land or some other natural resource, which invariably form part of a larger resource system with some common pool characteristics.\footnote{See, e.g., \cite[16]{fennel11} (``we are {\it always} operating at least partially within a commons of some sort''). I also mention Smith's notion of a ``semicommons'', used to describe settings where common pool arrangements for resource management interact with individual property rights, see generally \cite{smith00,smith02}.}

Importantly, to designate something as a common pool resource does not in any way imply that the resource in question is open-access or that it is held as a form of common property, a public trust, or under some other legal construction moving away from the sphere of private property.\footnote{See, e.g., \cite[58]{ostrom10b}.} Perhaps more controversially, designating something as a common pool resource does not in any way imply that the resource {\it should} be removed from this sphere.\footnote{See \cite[58]{ostrom10b} (``there is no automatic association of common-pool resources with common-property regimes -- or, with any other particular type of property regime'').} According to Ostrom and Hess, the appropriate property regime for a given common pool resource is a pragmatic question that depends on the circumstances.\footnote{See \cite[58]{ostrom10b}.}

It should be noted, however, that this neutral position on the relationship between property and common pool resources is premised on a bundle of rights understanding.\footnote{See \cite[59]{ostrom10b}.} Potentially, a more ambitious theory of property could suggest a different perspective. Specifically, the question arises as to how theories of common pool resource management relates to the social function account. This is a particularly interesting avenue for future work, as it could shed light on the normative stance that private property can be a good basis for sustainable self-governance, at least when backed up by a human flourishing account of what private property should be.

In this thesis, I will limit myself to noting how the link between property and theories of commons governance provides a possible route towards an institutional perspective on how to solve legitimacy problems associated with economic development takings.

To make progress in this regard, it will be useful to first briefly consider one of the most important theoretical legacies of Ostrom's work, namely a list of eight design principles that she formulated on the basis of empirical studies.\footnote{See \cite[90]{ostrom90}.} These principles were formulated because they seemed to be particularly crucial in ensuring good governance at the local level, and have since been supported by a growing body of empirical evidence.\footnote{See \cite{cox10} (the authors also suggest splitting some of the original principles in two parts, resulting in a slightly more fine-grained list, not needed in this thesis). } In brief, the so-called CPR principles are the following:

\begin{enumerate}
\item {\bf Well-defined boundaries:} There should be a clearly defined boundary around the resource in question, and a clear distinction should exist between members of the user community, who are entitled to access the resource, and non-members, who may be excluded. This will internalise the costs of resource exploitation and other externalities, ensuring that proper incentives for sustainable management arise within the community of resource users.\footnote{Importantly, the possibility of excluding non-members marks a distinction between open-access resources and common pool resources, where the latter appears much less susceptible to a commons tragedy than the former, because externalities are internalised to a clearly defined community. See \cite[91-92]{ostrom90}.}
\item {\bf Congruence between appropriation and provision rules and local conditions:} Management principles should be flexible and responsive to changing local conditions. Moreover, management practices should be anchored in the economic, social, and cultural practices prevalent at the local level. In addition, the individual benefits should generally exceed the individual costs associated with membership in the community of users, and collectively managed benefits should be distributed fairly among community members.\footnote{See \cite[92]{ostrom90}.}
\item {\bf Collective-choice arrangements:} The individual members of the user community should have an opportunity to participate in decision-making processes regarding the rules that govern the user community and the resource management. In addition to securing fairness and legitimacy, this will enhance the quality of the decision-making, as the users themselves have first-hand knowledge and low-cost access to information about their situation and the state of the resource in question.\footnote{See \cite[93]{ostrom90}.}
\item {\bf Monitoring:} There should be mechanisms in place to ensure that the behaviour of users is monitored for violations of management rules. To increase efficiency, monitoring should be locally organised. Moreover, to ensure local responsiveness and legitimacy, individuals acting as monitors should themselves be members of the user community or in some way answerable to this community.\footnote{See \cite[94-100]{ostrom90}.}
\item {\bf Graduated sanctions:} There should be an effective system in place for penalising violations of user community rules. These penalties should be graduated so that more severe or repeated violations are sanctioned more severely than minor or one-time transgressions.\footnote{See \cite[94-100]{ostrom90}.}
\item {\bf Conflict-resolution mechanisms:} The user community should be endowed with low-cost procedures for conflict resolution. These procedures should be sensitive to local conditions, to ensure local legitimacy.\footnote{See \cite[100-101]{ostrom90}.}
\item {\bf Minimum recognition of rights:} The user community should be protected from interference by external actors, including government agencies. As a minimum, the existence of local institutions and the right to self-governance should be recognised and respected by external government authorities.\footnote{See \cite[101]{ostrom90}.}
\item {\bf Nested enterprises:} There should be vertical integration between local, small-scale, management institutions and larger institutions aimed at protecting and furthering non-local interests. This integration should be based on the minimum recognition of rights mentioned in the previous point. Furthermore, it should provide a template for integrated decision-making about larger scale issues, where local competences are employed incrementally in more general settings, involving also institutions working on behalf of municipalities, regions, states and the international community. Local institutions for resource management should not only be respected by such larger scale structures, they should also feed into larger scale decision-making and be called to respond to greater community needs.\footnote{See \cite[101-102]{ostrom90}.}
\end{enumerate}

There are at least two interesting connections between self-governance principles such as these and the issue of economic development takings, especially as that issue is approached in this thesis, on the basis of the social function theory of property. First, one may observe that when economic development takings appear to lack legitimacy with respect to social functions, this is typically also an indication that the surrounding framework for resource management is not well-designed. In particular, it appears that the Gray test closely tracks many of the design principles proposed by Ostrom.

For instance, consider the balance of power between the owners and beneficiaries of a taking, the first point to consider according to the Gray test. When a taking fails on this point, doubts naturally arise also with regard to the underlying framework for resource management, particularly aspects pertaining to the recognition of local rights, the adequacy of collective-choice arrangements, and the congruence between appropriation, provision and local conditions. If property is taken by powerful actors, chances are that these actors are not representative of local community interests. Moreover, takings characterised by an imbalance of power typically indicate that the government is in fact quite unwilling to recognise the rights of local people, even when these rights are formally recognised as property rights.

By contrast, the situation might be different if it involves a taking that is not suspect according to the Gray test. For instance, if property is taken from absentee landlords and given to local land users in order to facilitate development, this might be an honest attempt at setting up a management framework that complies with CPR principles. In such a case, one would also not expect the balance of power between owners and takers to point towards abuse.

The second link between CPR design and economic development takings is arguably even more interesting. This link becomes apparent as soon as we shift attention away from diagnosing a lack of legitimacy towards coming up with alternative management principles that can restore it. Specifically, work done on local governance of common pool resources point to an {\it alternative} way of approaching the goal of economic development in cases that might otherwise result in the use of eminent domain. 

This has not received much attention in the literature so far. One notable exception, discussed in depth in the following subsection, is the work of Heller, Dagan and Hills.\footnote{The work of Lehavi and Licht also deserves a brief mention, even though it focuses on compensation rather than alternatives to eminent domain. The reason is that this work relies on proposing a novel institution that also touches on issues related to self-governance. In particular, Lehavi and Licht propose that post-taking, collective, price bargaining should be carried out on behalf of owners by a Special Purpose Development Company, in an effort to give them a chance to get their share of the commercial benefit arising from development. See \cite{lehavi07}. For a more in-depth discussion of this proposal, and the compensatory approach to economic development takings more generally, see \cite{dyrkolbotn15}.} Looking at their work will serve to make the abstract discussion above more concrete, and will set the stage for a comparison between their proposal and solutions that can be facilitated by the system of land consolidation presented in Chapter 5.

\subsection{Land Assembly Districts}\label{sec:3:6:1}

In an article from 2001, Heller and Dagan considered the connection between CPR design and overarching (liberal) property values.\footnote{See \cite{heller01}.} From this, they arrived at a proposal for what they call a ``liberal commons'', which adds some design constraints rooted in a desire to protect individual autonomy and minority rights. In particular, they emphasise the value of exit, the opportunity for members of the governance structure to alienate their share in the commons resource (conceived of as a property right).\footnote{See \cite[567-572]{heller01}.} The right of owners to leave the collective is thought of as a safety mechanism, to prevent failing institutions from trapping its members in a state of oppression. This, it is argued, is an important overarching design constraint, described as a ``liberal'' idea, that should complement the other design principles for local management of common pool resources.\footnote{Despite their commitment to protect the right of exit, Heller and Dagan are also aware of the destabilising effect exit can have on an otherwise well-functioning institution. To address this, they discuss additional mechanisms, such as rights of first refusal, that can ensure that exit does not prove too disruptive to the local collective, as long as a sufficient number of members choose to remain. See \cite[596-702]{heller01}.}

In a later article, responding to the {\it Kelo} controversy, Heller and Hills build on the idea of the liberal commons by proposing a novel approach to the takings issue, consisting of a proposal for a new institutional framework that can facilitate land assembly for economic development. The key institutional innovation is the {\it Land Assembly Districts} (LADs), institutions that are meant to enable property owners in a specific area to make a collective decision about whether or not to sell their land to a developer or a municipality.\footcite[1469-1470]{heller08} The idea is that while anyone will be able to propose and promote the formation of a LAD, the official planning authorities and the owners themselves must consent before it is formed.\footcite[1488-1489]{heller08} Clearly, some kind of collective action mechanism is required to allow the owners to make such a decision. 

Heller and Hills suggest that voting under the majority rule will be adequate in this regard, at least in most cases.\footnote{See \cite[1496]{heller08}. However, when many of the owners are non-residents who only see their land as an investment, Heller and Hills note that it might be necessary to consider more complicated voting procedures, for instance by requiring separate majorities from different groups of owners. See \cite[1523-1524]{heller08}.} How to allocate voting rights in the LAD is given careful consideration, with Heller and Hills opting for the proposal that they should in principle be given to owners in proportion to their share in the land belonging to the LAD.\footnote{See \cite[1492]{heller08}. For a discussion of the constitutional one-person-one-vote principle and a more detailed argument in \isr{favour} of the property-based proposal, see \cite[1503-1507]{heller08}.} Owners can opt out of the LAD, but in this case, eminent domain can be used to transfer the land to the LAD using a conventional eminent domain procedure.\footcite[1496]{heller08}

Heller and Hills envision an important role for governmental planning agencies in approving, overseeing and facilitating the LAD process. Their role will be most important early on, in approving and spelling out the parameters within which the LAD is called to function.\footcite[1489-1491]{heller08} While it is not discussed at any length, the assumption appears to be that the planning authorities will define the scope of the LAD by specifying the nature of the development it can pursue in quite some depth. Hence, the powers of the planning authority appear likely to remain quite extensive.

If the owners do not agree to forming a LAD, or if they refuse to sell to any developer, Heller and Hills suggest that the government should be precluded from using eminent domain to assemble the land.\footcite[1491]{heller08} This is a crucial aspect of their proposal that sets the suggestion apart from other proposals for institutional reform that have appeared after {\it Kelo}. A LAD will not only ensure that the owners get to bargain with the developers over compensation, it will also give them an opportunity to refuse any development to go ahead. Hence, the proposal shifts the balance of power in economic development cases, giving owners a greater role also in preparing the decision whether or not to develop, and on what terms. Hence, the LAD proposal promises to address the democratic deficit of economic development takings, without failing to \isr{recognise} that the danger of holdouts is real and that institutions are needed to avoid it.

There are some problems with the model, however. First, it seems that planning authorities might have an incentive to refuse granting approval for LAD formation. After all, doing so entails that they give up the power of eminent domain for the land in question. For this reason, Heller and Hills propose that a procedure of judicial review should exist whereby a decision to deny approval for LAD formation can be scrutinized.\footcite[1490]{heller08} However, the question then arises as to how deferential  courts should be in this regard, echoing the conundrum that engulfs the safeguard intended by the public use restriction. Presumably, one would want the courts to strictly scrutinise LAD rejections, to instil that LADs should normally be promoted. However, would the courts be comfortable providing such scrutiny, also against a government body claiming that the ``public interest'' speaks against LAD formation? This would likely depend on the exact formulation and spirit of the LAD-enabling legislation. To work as intended, some sort of presumption in favour of LAD approval appears to be in order, but this in turn can have the effect of making it easier for powerful landowners to abuse the LAD system, e.g., by pushing through LADs that enable them to impose their will on other community members.

This worry is related to a second possible objection against the LAD proposal, concerning the practicalities of the process leading up to the LAD's decision on whether or not to accept a given offer. Is it possible to organise such a process in a manner that is at once efficient, inclusive and informative, without making it too costly and time consuming? Here Heller and Hills envision a system of public hearings, possibly \isr{organised} by the planning authorities, where potential developers meet with owners and other interested parties to discuss plans for development.\footnote{See \cite[1490-1491]{heller08}. It might also be necessary for the planning authorities or other government agencies to take on some responsibilities with respect to providing guidance and assistance to less resourceful members among the owners.} The process envisioned here would resemble existing planning procedures to such an extent that additional costs could hopefully be kept at a minimum. 

The significant difference would concern the relative influence of the different actors, with the   owners as a group receiving a considerable boost as a result of the LAD. Rather than being sidelined by a narrative that sees the use of eminent domain as the culmination of planning, the owners are now likely to occupy center stage throughout, as they now will have the final say on whether or not the development will go ahead.

This raises the question of how the interests of other locals, without property rights, will be protected. Heller and Hills assumes that local non-owners will also be represented during the stages leading up to the LAD's final decision, but their role in the process is not clarified in any detail.\footcite[1490-1491]{heller08} This raises the worry that LADs might undermine local democracy by giving property owners a privileged position with respect to policy questions that should be decided jointly by all members of the community. 

If property rights are distributed evenly among community members, the risk of abuse in this regard might be limited. Moreover, the local anchoring that LADs provide should also benefit non-owners, by bringing the decision-making process closer to the people most directly affected, including non-owners. If some members of the local community remain marginalised, this should arguably be regarded as a regulatory failure or a reflection of underlying inequality in society, not a shortcoming of the LAD proposal. In these cases, a reasonable approach might even be to {\it expand} the function of LADs, by granting voting rights to a larger class of local property dependants, not only formally titled owners.%\footnote{The important invariant to maintain, I believe, is that the locally anchored institution should be the active, invested, agent, while more centralised and/or expert-dominated government bodies should act as passive, impartial, regulators. In the processes leading to economic development takings, this equation is typically reversed, with government bodies and commercial companies being the active agents, while the owners and the local community are the passive agents whose property rights and dependencies place some nominal limits on the authority of other parties (limits which, due to the weakness of owners as a group, tend to be easily disregarded).}

%However, the LAD proposal raises some problematic issues pertaining to the proposed mechanism of collective decision-making. As Kelly points out in a commentary, the idea of majority voting might be inherently flawed for decision-making about land assembly.\footcite{kelly09} For instance, if different owners value their property differently, majority voting will tend to \isr{disfavour} those with the strongest views, either in \isr{favour} of, or against, assembly. If these viewpoints are assumed to be non-strategic and genuine reflections of the welfare associated with the land, this can result in inefficiencies. In short, the problem is that a majority can often be found that does not take due account of minority interests. This is worrying, since it can undermine property's function as a means for minorities to protect and assert themselves on the basis of merit rather than voting power.

%For instance, if a minority of owners are planning development on their own land, and this conflicts with some LAD proposal targeting a larger area, the minority might find it difficult to defend themselves against the force of the LAD. Indeed, such a minority might effectively loose the battle for their property as soon as a LAD is formed, if the development description underlying LAD formation is incompatible with the kind of development they wish to pursue. For such owners, a presumption in favour of LAD formation might prove highly disadvantageous.\footnote{Of course, one might imagine these landowners opting out of the LAD, or pursuing their own interests independently of it. However, they are then unlikely to be better off than they would be in a no-LAD regime. In fact, it is easy to imagine that they could come to be further \isr{marginalised}, since the existence of the LAD, acting `on behalf of the owners', might detract from any dissenting voices on the owner-side.}
 
%Indeed, developers might come to rely on LADs to push through {\it de facto} condemnations of property, through a procedure that leaves minorities less protected than the traditional takings process. Indeed, it would be theoretically possible for any landowner to use a LAD to condemn any neighbouring property smaller than their own. Eventually, a whole community might be taken over by one or a few powerful landowners, through a sequence of appropriately designed LADs and development projects. %The government should prevent this, of course, but experiences with eminent domain for economic development illustrate that they might well fail in this regard.

The ideal of the LAD proposal is clearly stated and highly attractive. LADs should help to establish self-governance for land assembly and economic development. In particular, Heller and Hills argue that LADs should have ``broad discretion to choose any proposal to redevelop the \isr{neighbourhood} -- or reject all such proposals''.\footcite[See][1496]{heller08} As they put it, two of the main goals of LAD formation is to ensure ``preservation of the sense of individual autonomy implicit in the right of private property and preservation of the larger community's right to self-government''.\footcite[See][1498]{heller08} The problem is that these ideals turn out to be at odds with some of the concrete rules that Heller and Hills propose, particularly those aiming to ensure good governance of the LAD itself.

In relation to the governance issue, Heller and Hills emphasise, in direct contrast to their comments about ``broad discretion'' and ``self-governance'', that ``LADs exist for a single narrow purpose -- to consider whether to sell a neighborhood''.\footcite[See][1500]{heller08} This is a good thing, according to Heller and Hills, since it provides a safeguard against mismanagement, serving to prevent LADs from becoming battle grounds where different groups attempt to co-opt the community voice to further their own interests. As Heller and Hills puts it, the narrow scope of LADs will ensure that ``all differences of interest based on the constituents' different activities and investments, therefore, merge into the single question: is the price offered by the assembler sufficient to induce the constituents to sell?''.\footcite[1500]{heller08}

This means that there is a significant internal tension in the LAD proposal, between the broad goal of self-governance on the one hand and the fear of \isr{neighbourhood} bickering and majority tyranny on the other. Indeed, it is hard to see how LADs can at once have both a ``narrow purpose'' as well as enjoy ``broad discretion'' to choose between competing proposals for development. If such discretion is granted to LADs, what prevents special interest groups among the landowners from promoting development projects that will be particularly \isr{favourable} to them, rather than to the landowners as a group? What is to prevent landowners from making behind-the-scene deals with \isr{favoured} developers at the expense of their \isr{neighbours}? It might be difficult to come up with rules that prevent mechanisms of this kind, without also making substantive self-governance an impossibility.

Moreover, as Kelly points out in a commentary, the idea of majority voting might be inherently vulnerable to abuse when applied to  decision-making about land assembly.\footcite{kelly09} The problem is that if different owners value their property differently, majority voting will tend to \isr{disfavour} those with the strongest views, either in \isr{favour} of, or against, assembly. If these viewpoints are assumed to be non-strategic and genuine reflections of the welfare associated with the land, this can result in inefficiencies and abuse. In short, the problem is that a majority can often be found that does not take due account of minority interests. This is worrying, since it can undermine property's function as a means for minorities to protect and assert themselves on the basis of right rather than voting power.

This weakening of property can be enough to allow powerful groups to pursue nefarious objectives. Indeed, it is conceivable that developers might come to rely on LADs to push through {\it de facto} condemnations of property, through a procedure that leaves minorities less strongly protected than the traditional takings process. For instance, it would be theoretically possible for any landowner to use a LAD to cheaply condemn any neighbouring property that is smaller or less valuable (giving rise to fewer votes) than their own. Eventually, a whole community might be taken over by one or a few powerful landowners, through a sequence of appropriately designed LADs and development projects. %The government should prevent this, of course, but experiences with eminent domain for economic development illustrate that they might well fail in this regard.

If a LAD is tightly regulated, required to offer the land on an open auction, and obliged to only look at the price, this might limit the risk of abuse. But it will not give owners broad discretion to consider the social functions of property when choosing among development \isr{proposals}. In my view, therefore, it is undesirable to restrict the operations of LADs in this way. It is easy to imagine cases where competing proposals, perhaps emerging from within the community of owners themselves, will be made in response to the formation of a LAD. Such proposals may involve novel solutions that are superior to the original development plans, in which case it is hard to see why they should be disregarded simply because they are less commercially attractive, or because the  developer interested in pursuing such a proposal cannot offer the highest payment to the owners. In the end, the decision that the LAD makes concerns the future of the community as a whole. This is not an exercise in profit-maximization, and there are good reasons to believe that LAD regulation should encourage a broad perspective, not enforce a narrow one.

By contrast, Heller and Hills are quite determined that the degree of self-governance needs to be limited in favour of strict regulation to reduce the risk of LAD abuse. In particular, they argue that ``LAD-enabling legislation should require especially stringent disclosure requirements and bar any landowner from voting in a LAD if that landowner has any affiliation with the assembler''.\footcite{heller08} Here, the notion of self-governance is made very thin indeed, as owners will effectively be barred from using LADs as a template for gaining the right to participate in development projects themselves.

Moreover, new questions arise. For one, what is meant by ``affiliation''? Say that a landowner happens to own shares in some of the companies proposing development. Should they then be barred from voting? If so, should they be barred from voting on all proposals, or just those involving companies in which they are a shareholder? If the answer is yes, how can this be justified? Would it not be easy to construe such a rule as discrimination against landowners who happen to own shares in development companies? On the other hand, if the landowner in question is allowed to vote on all other proposals, would it not be natural to suspect that their vote is biased against assembly that would benefit a competing company? Or what about the case when some of the landowners are employed by some of the development companies? Should such owners be barred from voting on proposals that could benefit their employers? This seems quite unfair as a general rule. But in some cases, employment relations could play a decisive factor in determining the outcome of a vote. This might happen, for instance, if an important local employer proposes development in a \isr{neighbourhood} where it has a large number of employees. Heller and Hills give no clear answer to the questions arising in this regard, and at this point, the circle has in some sense closed in on their proposal. Indeed, just as courts today struggle with the ``public use'' requirement, it seems that the proposed ``affiliation'' criterion for depriving someone of their voting rights would provide a very shaky basis for judicial review. 

More generally, it seems that how to best organise a LAD remains an open problem. The challenge is to ensure that LADs deliver a real possibility of self-determination, while also ensuring good governance and protection against abuse. That it remains unclear how to do this is acknowledged by Hiller and Hills themselves, who point out that further work is needed and that only a limited assessment of their proposal can be made in the absence of empirical data.\footnote{See \cite[]{heller08}.} Later in the thesis, I will shed light on this challenge when I consider the Norwegian framework for land consolidation. This framework can be looked at as a sophisticated institutional embedding of many of the central ideas of LADs. In particular, I will discuss how Norwegian land consolidation can be employed in cases of economic development, and how it is increasingly used as an alternative to expropriation in cases of hydropower development. This will allow me to shed further light on the issues that are left open by Heller and Hills' important article.

\section{Conclusion}\label{sec:3:7}

The legitimacy issue is at the heart of this thesis. There are many ways of approaching it, catering to different ideas about the appropriate role that the courts should play in safeguarding private property. This chapter has tried to distil an approach that is particularly suited in cases when property is taken for economic development. 

This led to a proposal for an institutional fairness approach that combines procedural and substantive standards, to arrive at a template for assessing the democratic quality of the decision-making as such, not merely the outcome. This is appropriate because it helps address a key worry associated with an economic development taking: that the decision to take represents an abuse of power, reflecting badly on the institutions that gave rise to it.
%I argued that such an institutional approach to fairness has started developing at the ECtHR, as a consequence of its development of the pilot judgement framework for assessing cases that might indicate systemic problems at the state institutional level.

On this basis, the chapter went on to provide a possible heuristic for assessing the legitimacy of economic development takings. This heuristic was based on six legitimacy indicators provided by Gray, with three new ones added, based on the work done in this and the previous chapter. The resulting heuristic, the Gray test, should be able to identify cases of eminent domain abuse, particularly those that offend against social functions of property at the institutional level, rather than merely the financial entitlements of owners.

Testing for failure is only the first step towards increased legitimacy, to be followed up by proposals for structural improvements. The question is how to respect property and its social functions without giving up on the idea that the collective has an overarching responsibility to regulate property and its uses, in keeping with the principle of democracy. This chapter proposed looking to the work done by Elinor Ostrom and others on common pool resources. Specifically, some key design principles for local self-governance was presented, along with an argument that these could be used as a starting point for coming up with institutions to replace eminent domain for economic development.

The proposal for Land Assembly Districts, due to Heller and Hills, was considered because it represents a first pass at such a solution to the legitimacy problem in the US. It was argued that the proposal is marked by a severe tension between the overarching goal of self-governance and the need to prevent abuses of power at the local level. In the end, the proposal did not appear to deliver on the initial promise of self-governance, because there were simply too many limits placed on the authority of the local decision-makers.

Arguably, this points to the need for adapting a less abstract perspective on legitimacy, to encourage flexible mechanisms that can be adapted to the circumstances. Limitations and safeguards against abuse that might be reasonable in an inner city neighbourhood with many poor tenants might be entirely misplaced in a village of equally positioned home-owners. This insight also echoes one of the key design principles formulated by Ostrom, concerning the need to maintain congruence with local conditions. According to the empirical evidence available regarding common pool resource management, a one-size-fits-all approach to legitimacy at the local level is not going to work.\footnote{See \cite{cox10}.}

This observation marks the end of the first part of the thesis. In the next part, I will adopt a more concrete perspective, by considering takings of water and land rights for Norwegian hydropower. This will lead to an analysis of legitimacy of takings for this purpose along the lines of the Gray test, as well as a case study of Norwegian land consolidation as an alternative to eminent domain. In this way, the second part will aim to shed light on key aspects of the theory developed in the first part, while exploring further the idea that social functions run as a common thread through individual property rights.
\part{A Case Study of Norwegian Waterfalls}

\chapter{Norwegian Waterfalls and Hydropower}\label{chap:4}

\section{Introduction}\label{sec:4:1}

Norway is country of many mountains, fjords and rivers, where around 95 \% of the annual domestic electricity supply comes from hydropower.\footnote{See \cite{statistikk13}.} The right to harness energy from rivers, streams and waterfalls generally belongs to local landowners under a riparian system whereby many water rights are derived from ownership of a riverbed.\footnote{This arrangement is rooted in the first known legal sources in Norway, the so-called ``Gulating'' laws, thought to have been in force well before AD 1000. See \cite[111-112,120]{robberstad81}.}  Historically, waterfalls were very important to local communities, particularly as a source of power for grist and saw mills.\footnote{See \cite[121]{tvedt13}.} %%Indeed, the fact that peasants in Norway controlled local water resources can help explain why they were relatively free, both economically and socially, compared to many other places in Europe.\footnote{See \cite[121]{tvedt13}.}

Following the industrial revolution, local ownership and management came under increasing pressure. At the beginning, this pressure was exerted by private commercial interests, often foreign investors, who saw the industrial potential in hydropower and started speculating in Norwegian water resources.\footnote{See \cite[30-31]{nou04}.} Later, the pressure was exerted mainly by the government, following the introduction of new legislation to regulate the development of hydroelectric power.\footnote{See \cite[41-57]{thue96} (describing the  regulatory system set up during this time).} This legislation set up a system that gave highly preferential treatment to public utilities over private actors, including local owners.\footnote{See \cite[46]{thue96} (describing legislation introduced to promote public utilities, including new expropriation authorities directed at local owners of waterfall).} At first, the motivation behind this reform was to facilitate a decentralised form of government control, led by public utilities controlled by the municipality governments.\footnote{See \cite[44-47]{thue96}.} However, the hydroelectric sector underwent gradual centralisation, a process that gained momentum after the Second World War when the state itself assumed a leading role.\footnote{See \cite[59-85]{thue96}. For the history of the state's involvement with hydropower generally, see \cite{thue06, skjold06,thue06b}.} After this, local communities and local riparian owners became increasingly marginalised. In particular, local communities were systematically pressured into shutting down their hydroelectric plants, often as a condition for being granted access to electricity through the national, monopolised, electricity grid.\footnote{See \cite[p.111]{hindrum94}.}

Then, in the early 1990s, the electricity sector was liberalised, largely inspired by the market-orientation and privatisation of the public sector in the UK under Thatcher.\footnote{See generally \cite{midttun98}.} The production sector was decoupled from the grid sector, while public utilities were reorganised as commercial companies.\footnote{See \cite[86]{efta07} (describing how Norwegian electricity companies, most of which are still (partly) publicly owned, now operate as for-profit, limited liability companies).} At the same time, the regulatory system was decoupled from political decision-making processes, to become more expert-based.\footnote{See \cite[26-27]{brekke12}.} Moreover, the sector underwent additional centralisation, a result of mergers and acquisitions among former public utilities.\footnote{See \cite[583]{bibow03}. I mention that despite significant continuous centralisation from the Second World War to this day, the Norwegian hydroelectric sector is still relatively decentralised compared to other countries, e.g., the UK, see \cite[181]{midttun98}. Arguably, this is a lasting influence of a tradition based on local, egalitarian, ownership of water resources.}

Following the reform, access rights to the national grid are meant to be granted equally to all potential actors on the energy market, including private companies.\footnote{See generally \cite{hammer96}. For an interesting presentation and analysis of grid-based markets in general, see \cite{falch04}.} After the passage of the \cite{ea90}, the energy companies that operate the national grid (the grid is divided into regions) are no longer authorised to shut out competitors.\footnote{See the \indexonly{ea90}\dni\cite[3-4]{ea90}.} A side-effect of this is that it has become possible for local landowners to undertake their own hydropower projects. Local owners can now access the grid to sell the electricity they produce on Nord Pool, the largest electrical energy market in Europe.\footnote{See generally \cite{larsen06,larsen08,larsen12}.} This has led to increased tension between local interests and established hydropower companies. The following fundamental question has arisen: who is entitled to benefit from rivers and waterfalls, and who is entitled to a say in decision-making processes concerning their use?

This chapter sets the stage for discussing this question in more depth, by detailing how the hydropower sector is organised. It looks both to the law and to commercial and administrative practices. Special attention is directed at those aspects that have changed following liberalisation, and which have resulted in conflicts involving property. The main goal is to show that the tension between large-scale development companies and local owners can only be understood on the basis of a social function perspective on riparian ownership. To set the stage for making this point, the chapter first provides a brief overview of the legal system, emphasising the role that private property has played in the development of Norwegian democracy.\footnote{The classic reference on Norwegian constitutional law is \cite{andenes06}.}

\section{Norway in a Nutshell}\label{sec:4:2}

\noo{ %\footnote{It should be noted that the executive branch also enjoys considerable legislative power under Norwegian law. Both informally, because it prepares new legislation, and also formally, because it has wide delegated powers to issue so-called {\it directives} (forskrifter). Indeed, it is typical for acts of parliament to include a general delegation rule which permits the executive to legislate further on the matters dealt with in the act, by clarifying and filling in the gaps left open by it.}

Norway is a constitutional monarchy, based on a representative system of government.\footnote{For Norwegian constitutional law generally, see \cite{andenes06}.} The executive branch is led by the King in Council, the Cabinet, headed by the Prime Minister. Legislative power is vested in the Storting, the Norwegian parliament, elected by popular vote in a multi-party setting. In 1884, the parliamentary system first triumphed in Norway, as the cabinet was forced to resign after it lost the confidence of parliament. The principle has since obtained the status of a constitutional custom. In particular, the cabinet can not continue to sit if parliament expresses mistrust against it. However, an express vote of confidence is not required. In practice, due to the multi-party nature of Norwegian politics, minority cabinets are quite common. These can sustain themselves by making long-term deals with supporting parties, or by looking for a majority on a case-by-case basis.

The judiciary is organised in three levels, with 70 district courts, 6 courts of appeal, and the Supreme Court. The district courts have general jurisdiction over most legal matters; there is no division between constitutional, administrative, civil, criminal courts. \footnote{However, there are distinct procedural rules for civil and criminal cases and a special court for land consolidation, see the \cite{lca79}. Moreover, both the district courts and the courts of appeal follow special procedural rules in appraisal disputes, for instance when compensation is awarded following expropriation, see the \cite{aa17}.} The courts of appeal have a similarly broad scope. Moreover, the right to appeal is ensured in most cases.\footnote{The right to an appeal is not absolute. In civil cases, it is generally required that the stakes are above a certain lower threshold, measured in terms of the appellants' financial interest in the outcome. See the \indexonly{cda05}\dni\cite[29-13]{cda05}.} The Supreme Court, on the other hand, operates a very strict restriction on the appeals it will allow.\footnote{See the \indexonly{cda05}\dni\cite[30-4]{cda05}.} It typically only hears cases if a matter of principle is at stake, or if the law is thought to be in need of clarification.\footnote{See, generally, \cite{skoghoy08}.}
}
The Norwegian legal system is often said to be based on a special ``Scandinavian'' variety of civil law, which includes strong common law elements: legislation is not as detailed as elsewhere in continental Europe, some legal areas lack a firm legislative basis, it is generally accepted that courts develop the law, and the opinions of the Supreme Court are often of crucial importance when the lower courts interpret and apply legislation.\footnote{See, generally, \cite{bernitz07}.} In this regard, it should be noted that the Supreme Court operates a very strict restriction on the leave to appeal.\footnote{See the \indexonly{cda05}\dni\cite[30-4]{cda05}.} It typically only hears cases if a matter of principle is at stake, or if the law is thought to be in need of clarification.\footnote{See, generally, \cite{skoghoy08}.} Moreover, legislation remains the primary source used to resolve most legal disputes. When applying it, the courts tend to place great weight on preparatory documents procured by the executive branch. These documents are widely regarded as expressions of legislative intent, even though Parliament is not usually involved in their preparation.

The Constitution of Norway dates back to 1814 and was heavily influenced by then recent political movements, particularly in the US and France.\footnote{See generally \cite{mestad14}.} Moreover, it was influenced by a desire for self-determination, as Norway was at that time a part of Denmark-Norway, largely controlled by the Danish elite. Following the Napoleonic wars, Norwegian politicians sought to take advantage of Denmark's weakened position to gain independence and they drafted the Constitution with this objective in mind. In the end, Norway was forced to enter into a union with Sweden (who was backed by the winning side of the Napoleonic wars), but the Constitution remained in place. Moreover, after the triumph of the parliamentary system in 1884, Norway would also eventually gain independence, in 1905, following a peaceful and democratic transition process.\footnote{See generally \cite{sejersted15}.}

During the 19th century, farmers and smallholders emerged as a powerful group in Norwegian politics. This was in large part due to the fact that they were also landowners, whose rights and contributions were not limited to traditional farming.\footnote{See generally \cite{hommerstad14}. The ``classic'' presentation of the political influence of farmers in Norway is \cite{koht26}.} This had not always been the case; during the middle ages, the Norwegian farmer had usually been a tenant.\footnote{See generally \cite{myking05}.} However, tenant farmers always enjoyed a significant degree of control over the management of the land and its natural resources.\footnote{See \cite[59-60]{pryser99}; \cite[226-238]{myking05}.} Moreover, between the 17th and the end of the 18th century, almost all Norwegian tenant farmers bought their land from their landlords.\footnote{See, e.g., \cite[108]{nordtveit15}.} As a result, the distribution of land ownership in Norway had  become highly egalitarian at the time of the Constitution.

In the years that followed, the landed nobility in Norway was further marginalised. The Constitution itself prohibited the establishment of new noble titles and estates.\indexonly{grunnloven14}\dni\footcite[23|118]{grunnloven14} Then, in 1821, all hereditary titles were abolished (although existing nobles kept their titles for their lifetimes).\footnote{See \cite{adel09}.} By the middle of the 19th century, farmers and smallholders had gained significant political influence. In fact, they emerged as the leading political class, alongside the city bureaucrats.\footnote{See generally \cite{hommerstad14}.} During this time, Norway also introduced a system of powerful local municipalities. These were organised as representative democracies, becoming miniature versions of the cherished, as of yet unfulfilled, nation state (Norway was still in a union with Sweden at this time). Even today, municipalities retain a great deal of power in Norway, particular in relation to land use planning.\footnote{They are the primary decision-makers for spatial planning, as pursuant to \cite{pb08}.} They represent a highly decentralised political structure, with a total of 428 municipalities as of 01 January 2013.\footnote{This is down from the all-time high of 747 in 1930. There have long been proposals to reduce the number of municipalities further, but so far the political resistance against this has prevented major reforms. See \cite{kommuner14} (report to the Ministry from an expert committee on municipality reform, 2014).} %Enjoying private ownership in common is not unusual in Norway, particularly in rural areas, and the law of property in Norway reflects this in various ways.

\subsection{Property regimes in Norway}\label{sec:4:2:1}

There can be no doubt that the egalitarian distribution of property rights found in Norway was crucial to the development 
of a democratic economic and political order, especially in rural parts of the country.\footnote{For a comparative discussion, focusing on how egalitarianism influenced the industrialisation process in Norway, setting it apart from the industrialisation process in the UK, see \cite{brox13}.} Moreover, landownership itself was never understood in purely individualistic terms, but rather as an important building block of local communities. A reflection of this is the fact that outfields in Norway tend to be held under a form of common ownership, whereby each smallholding in the local community owns a share in the surrounding land and its resources.

This type of co-ownership has no exact common law equivalent, but is most similar to the tenancy in common.\footnote{However, there is no requirement that the co-ownership takes place behind a trust -- all individual shareholders are formally registered as owners of their share of the land and there is a presumption in favour of continued co-ownership accompanied by collective productive use of the land, not alienation or individuation.} In the \cite{coa65}, further rules are given to regulate the relationship between the co-owners and their use of the property they share. The main principle is that each owner has a right to the ``normal'' enjoyment of the property, determined by looking at the natural conditions, the local customs, and the original purpose of the co-ownership arrangement (if it is known).\footnote{See the \indexonly{coa65}\dni\cite[4]{coa65}.} Moreover, an individual owner's use must not exceed what corresponds to their share of the property and must not be unduly burdensome to the other owners.\footnote{The share in commonly held real property was historically determined based on the amount of rent (``skyld'') that each farmer paid to the landowner (a figure that was also used to determine the level of taxation). While most tenant farmers in Norway had bought their land by the end of the 18th century, the notion of ``skyld'' was kept as a measure of the share each farm had in the commonly owned larger estates. For further details, see \cite{ravna09a}.}

To some extent, the majority shareholders can enforce a specific use of the property against the will of a minority.\footnote{See the \indexonly{coa65}\dni\cite[4]{coa65}.} This includes new forms of commercial activity, including activities requiring additional investments in the property. If such activities are organised against the will of a minority, the minority will still be entitled to take part in the enterprise.\footnote{See the \indexonly{coa65}\dni\cite[5]{coa65}.} There are limits to what the majority can do; they can only order uses for which the property is deemed ``suitable'', and they cannot do anything to dramatically change the character of the property, sell it, or use it as security for debt.\footnote{See the \indexonly{coa65}\dni\cite[4]{coa65} paras 1 and 3.} Moreover, if the majority does something to interfere with the use rights of a minority shareholder, compensation must be paid.\footnote{See the \indexonly{coa65}\dni\cite[4]{coa65} para 4.} Because of these restrictions, gridlock can often result if the owners disagree fundamentally about how to manage their land.

For real property, particularly in rural areas, the standard way of resolving conflicts among co-owners is to bring a case before the Norwegian land consolidation courts. These courts are empowered to dissolve systems of co-ownership (if certain conditions are met), but they can also be used to organise joint use of the land, to avoid dissolution. The prevalence of common ownership over outfields means that land consolidation courts are important in rural Norway, and it also explains why these courts have been granted such wide powers to help organise the use of privately owned land. I return to the details of this in chapter \ref{chap:6}, as part of a broader discussion on how the institute of land consolidation can be used as an alternative to eminent domain in economic development situations.

In addition to the form of co-ownership regulated in the \cite{coa65}, there are two other special forms of ownership of land found in Norway that should be briefly mentioned. Both pertain to land over which a large group of people enjoy extensive rights of use that have been recognised as so-called ``almenningsretter'' (common rights) under Norwegian land law. Land to which common rights attach will also have an {\it in rem} owner in the private law sense, but special rules are in place to protect the group of people who enjoy common rights. These rules presuppose that the land is owned either by the state or by a majority of the individuals who enjoy use rights.\footnote{Historically, the state was not considered the owner of the commons in the private law sense of the word, but rather as a custodian with special regulatory powers. However, perceptions of this changed over time, with the Supreme Court eventually concluding that the state was to be considered the owner of the state commons in the full private law sense of ownership, see \cite{vinstra63}.} If the owner of common land is the state, the primary legislation that protects the rights of the local people is the \cite{ma75}. If the land is owned by the rights holders themselves, the commons is known as a village commons, and the relevant legislation is the \cite{vca92}. The details differ, but the main purpose of both Acts is to offer special protection for rights in common, especially with regard to traditional land uses that local farmers depend on for their livelihoods.\footnote{For further details on the commons in Norwegian law, see \cite{stenseth05}.} There is no extant concept of a commons in Norway that attaches to land owned by a minority of the rights holders or other groups of private individuals.\footnote{Traditionally, there was also a concept of a ``private commons'', but this has all but disappeared from the law since the land in question has typically been transformed into co-owned private land or a village commons. The courts might still occasionally derive usage rights over private land from earlier rights in common over that land, but these use rights would be unlikely to receive legal recognition as specially protected commons rights.} This is a testament to Norwegian egalitarianism that can partly be explained by the fact that private landlords and tenant farming is absent from the structure of rural landownership and have been for quite some time.

\subsection{The importance of water resources}\label{sec:4:2:2}

Local control over water resources, ensured through property rights, has always been very important to farmers and rural communities in Norway. According to Terje Tvedt, 10 000-30 000 mills were in operation in Norway in the 1830s.\footnote{See \cite[121]{tvedt13}.} As Tvedt argues, the fact that these mills were under local control was particularly important because it helped ensure self-sufficiency. In addition, saw mills became an important source of extra income for Norwegian farming communities. As mentioned in chapter \ref{chap:1}, the right to harness power from a river is regarded as a separate unit of private property in Norway, referred to as a waterfall.\footnote{Historically, the law emphasised ownership of traditional agrarian water resources, such as fishing rights. However, new sticks were added to the waterfall bundle over the years, including the right to develop hydropower, see \cite[14-32]{vislie44}. See also \cite[108]{nordtveit15}. For a detailed presentation of the history of water law in pre-industrial times, I refer to \cite{motzfeld08}.} The system is riparian, so by default, a waterfall belongs to the owner of the land over which the water flows.\footnote{See the \indexonly{wra00}\dni\cite[13]{wra00}.} The landowners do not own the water as such -- freely running water is not subject to ownership -- and the riparian owners' right to withhold or divert water is limited.\footnote{See the \indexonly{wra00}\dni\cite[8|15]{wra00}.} However, the waterfall owners have the exclusive right to harness the potential energy in the water over the stretch of riverbed belonging to them. This right can be partitioned off from rights in the surrounding land, and large-scale hydropower schemes typically involve such a separation of water rights from land rights; the energy company acquires the right to harness the energy, while the local landowners retain ownership of the surrounding land.

%However, as mentioned in chapter \ref{chap:1}, the right to the hydropower in a river is considered to be a separate property right in the bundle of rights typically held by riparian owners, referred to as the right to the waterfall.%, a terminology I will also adopt-\footnote{The Norwegian term `fall' has a somewhat broader meaning than its English counterpart, `waterfall'. The word `fall' is used to describe a continuous section of any stream or river, typically identified by giving the total difference in altitude over the relevant stretch of riverbed. Furthermore, the Norwegian term `falleier' refers to a legal person who possesses the rights to the hydropower over such a section. In this thesis, I will typically refer to the owners of waterfalls, streams and rivers with the intended reading being the same as the Norwegian notion of a `falleier'. If special qualification is needed, for instance to distinguish between different classes of riparian owners, I will make a note of this explicitly.}

\noo{
In the introduction, I already highlighted the concept of a waterfall right, used to refer to the right to harness power from a river, an historically important stick in the property bundle associated with landownership in Norway. } 

I have already mentioned that undeveloped waterfall rights are usually held in common by members of the local population. In some cases, this is because a river suitable for hydropower development runs across many distinct private properties. Hence, the relevant waterfall rights are held in common in the narrow sense that an assembly of private rights is required in order for development to take place. However, in most cases, waterfall rights will be owned in common in a stronger sense since they attach to land that is already in shared ownership, regulated by the \cite{coa65}. % co-ownership of the surrounding land. %In these cases, it is also typically appropriate to identify the group of owners with the local community, which tend to be farming communities where each smallholding will tend to have a proprietary stake in the local river.

%Indeed, outfields in Norway are often held under a specific form of co-ownership,
After the industrial revolution, there was some doubt as to whether rights in common over land extended to waterfall rights, or whether waterfall rights were held exclusively by the landowners. This question was particularly important for land owned by the state, since common rights would be the only potential route for local community members to claim a proprietary stake in local hydropower resources.\footnote{In village commons, by contrast, the owners themselves are also local community members, although the group of owners is typically smaller than the group of people who have use rights in common.} The question was settled by the Supreme Court in the case of {\it Vinstra} in 1963.\footnote{See \cite{vinstra63}.} Here the Court held that no rights to waterfalls in state commons could be derived from communal use rights over that land. It follows that the takings issue does not arise with respect to hydropower development on commons land, at least not with respect to the waterfall rights as such. For this reason, the Norwegian framework for regulating common rights will not be considered further in this thesis.%Questions that arise specifically with respect to common rights are not practicallywill not be dealt with in at any length in this thesis.%\footnote{When we consider the case of {\it Alta} in Chapter \ref{chap:5}, we will encounter state-owned land where the aboriginal Sami population has claimed to enjoy rights in property similar to common rights. In recent years, this claim has met with some recognition within the Norwegian legal order, giving rise to yet another form of property in Norway. For further details, see the discussion in Chapter \ref{chap:5} section \ref{chap:5:x}.} In most cases, conflicts between local communities and energy companies take place in a setting where the local population have full property rights, as co-owners, over the natural resource in question. An important exception to this is the {\it Alta} case, pertaining to a hydropower project in the north of Norway, with a significant population of Sami people, an indigenous minor was decided, members of the Sami population were considered as rights holders in the traditional private law sense of the word, no different from non-aboriginal holders of property and use rights elsewhere in Norway.\footnote{See the...} 

In chapter \ref{chap:5}, I will discuss the {\it Alta} case, which involved the special property regime found in the north of Norway.\footnote{\cite{alta82}.} The case arose in the 1970s after the national government had decided to build a hydropower project in Finnmark, a part of Norway where the state is traditionally regarded as the owner of the outfields. The state's ownership of land in this region tends to be at odds with the interests of the Sami people, an indigenous group from northern Scandinavia. Traditionally, the state's ownership was consider to be standard private law ownership, more or less entirely unencumbered by indigenous interests, except when Sami rights had been explicitly recognised by the state. In particular, the use rights of the Sami people did not enjoy the protected status given to rights in common over state land elsewhere in Norway.%\footnote{Some scholars disputed this, by arguing that Sami rights should be viewed as common rights by analogy with the legislation in place for state commons.}

Hence, in the {\it Alta} case, the dispute revolved around expressly recognised use and property rights that would be negatively affected by the development. Communal claims based on indigenous rights were summarily rejected, and the case did not involve takings of waterfalls (since the state already held these rights). Still, {\it Alta} has later been considered an important precedent for disputes surrounding expropriation of waterfalls, since it dealt with many aspects of administrative law pertaining to the licensing procedure surrounding hydropower development. As discussed in chapter 5, the case also marked a watershed moment in the legal history of the Sami people, whose rights over land in Finnmark have since received greater recognition within the Norwegian legal order. Today, in the special context of Sami land, the law appears to be moving towards a framework where the Norwegian state is increasingly seen as a custodian of Sami lands, rather than an owner in the standard private law sense.\footnote{See generally \cite{ravna05,bull07,ravna12s}.}

In other parts of Norway, a similar perspective has yet to develop. Natural resources owned by the state, or taken by it under eminent domain, has the same legal status as private property. There is no legal basis for regarding the state as a custodian, and there is no legally enforcible sense in which land owned by the state is held in trust on behalf of the people. However, in a recent revision of the Constitution, a new section 112 was introduced that compels the government to preserve the environment and promote sustainability. The exact wording is as follows:

\begin{quote}
Every person has the right to an environment that is conducive to
health and to a natural environment whose productivity and diversity
are maintained. Natural resources shall be managed on the basis of
comprehensive long-term considerations which will safeguard this
right for future generations as well. \\
In order to safeguard their right in accordance with the foregoing
paragraph, citizens are entitled to information on the state of the
natural environment and on the effects of any encroachment on nature
that is planned or carried out. The authorities of the state shall take measures for the
implementation of these principles.\footnote{See the \indexonly{grunnloven14}\dni\cite[112]{grunnloven14}.}
\end{quote}

This provision replaces a sustainability clause that was first introduced in the Constitution in 1992, following the influential Rio summit at the United Nations. This clause left little or no impact on Norwegian law, with no consequences discernible at all within the law of property.\footnote{See generally \cite{fauchald07}.} After the new formulation was introduced in 2014, there have been some indications that the legal status of the sustainability provision might be about to change. Indeed, the legislator itself expressed a desire to make the provision more easily justiciable.\footnote{See \cite[246]{dok16}. As an example of where this might lead in the future, I mention that a group of Norwegian environmental lawyers are presently considering the possibility of initiating a class action against the Norwegian state for not doing enough to fight climate change, using section 112 as a legal basis. See \cite{gjengedal15}.} However, there have been no indication so far that section 112 will become relevant in the law of hydropower. Moreover, it seems highly unlikely that the section will attain relevance with respect to the issue of expropriation. For this reason, section 112 will not be examined further in this thesis.\footnote{Of course, a normative argument could well be made that the provision {\it should} entail greater regard for the interests and property rights of local people. Such an argument might perhaps also be backed up by considerations based on international environmental law. Further exploration of economic development takings from this angle will be left for future work.}

%For conservation issues pertaining specifically to water resources, the situation with respect to section 112 is slightly different. Here it seems that the sustainability clause could eventually become a justiciable standard that will restrict the authority of the water authorities to permit development. 
Although the sustainability clause in the Constitution is of marginal importance in the law of hydropower, environmental interests are quite well protected during the licensing procedure in hydropower cases. The rules and practices that apply in this regard are presented in more depth in section \ref{sec:4:3} below. Before looking at the details, it should be mentioned that Norway has implemented the Water Framework Directive of the European Union.\footnote{See \cite{water00}. It has been implemented in Norwegian law as the Directive Regarding Frameworks for Water Management, FOR-2006-12-15-1446.} This directive attempts to ensure that water resources are managed according to an ecosystem perspective where the regulator takes an holistic approach to planning in water systems. Such a perspective is at odds with the sector-based procedures that still dominate in Norwegian water law. Some have argued that the Norwegian implementation of the directive has not sufficiently recognized the need for structural reforms.\footnote{See \cite{hanssen14}.} However, the holistic perspective on water resources and sustainability now appears to be gaining ground, especially at the political level. As we will see towards the end of this chapter, the effect of this on local communities and owners has been mixed. In some cases, an increased emphasis on conservation and planning has had a negative effect on communities since it further inflates the power of those that have enough political and financial capital to exert their influence on the planning process. %This includes commercial development interests, who now often appear to be cooperating with environmental interests in an effort to gain control over Norwegian water resources.

To understand how water law works in Norway, it is important to keep in mind that the importance of water does not primarily arise from the fact that water is scarce, but mainly from the fact that it is so plentiful. Not only is water power the main source of domestic energy, it also occupies a special place in Norwegian culture. It is important to the identity of many communities, particularly in the western part of the country, where majestic waterfalls are considered symbols both of the hardship of the natural conditions and the sturdiness of local people. The implications for the tourism industry are also significant, as the natural environment attracts visitors and economic activity to regions of Norway that are otherwise threatened by stagnation and depopulation.

In light of this, it is not surprising that there is a tradition in Norway for local resistance against development projects that are considered damaging to the environment. In the 1960s and 70s, when the state embarked on their most ambitious projects, local environmental movements became nationally significant, as symbols of resistance against centralisation, exploitation of weaker groups, and environmental destruction.\footnote{See \cite{nilsen08}.}

This illustrates that water resources are embedded in the social fabric in such a way that a purely entitlements-based approach to property rights in these resources would be largely inappropriate. Rather, the case of Norwegian water seems to be well-suited for an investigation based on a social function view on property. As I show in this and the following two chapters, rivers and waterfalls serve to bring out tensions between rights and obligations in property, while also shedding light on the question of how to organise decision-making processes regarding economic development.

In the next section, I argue that the present law on hydropower in Norway tends to recognise only a small part of the relevant picture. On the one hand, it recognises the financial entitlements of individual owners, which it tries to balance against the regulatory needs of the state. But it largely fails to take into account that owners have broader interests, even obligations, relating to the sustainable management of their streams and their waterfalls. Moreover, the law appears largely unable to prevent commercial companies and special interest groups from exerting a strong pull on various state bodies, particularly those that are only weakly grounded in processes of democratic decision-making.

\section{Hydropower in the Law}\label{sec:4:3}
\noo{
Norwegian rivers, and especially rivers suitable for hydropower schemes, tend to run across outfields that are owned jointly by local farmers. Hence, rights to streams and waterfalls are typically held among several members of the rural community.\footnote{Rivers tend to run through land that has not to been enclosed. Moreover, in places where there has been a land enclosure, water rights are often explicitly left out, such that they are still considered jointly owned rights belonging to the community of local farmers. For more details on (forms of) joint ownership among Norwegian farmers, see, e.g., \cite[570]{stenseth07a}.} Local owners might not be willing to give up their ownership to facilitate development, especially not on terms proposed by external developers. Hence, the authority to expropriate has been an important legal instrument for state-backed hydropower companies. It has also been used extensively, particularly after the state made hydropower development a priority following the Second World War.
}
The law of hydropower in Norway is marked by a tension where, on the one hand, the right to harness power from rivers and waterfalls is considered private property, while on the other hand, it has become common to speak of hydropower as a resource belonging to the public. Since the \cite{ica17} was amended in 2008, this ambivalence in the discourse surrounding hydropower has also been part of the statutory provisions regulating hydropower development. I quote the two relevant sections side by side below:%\footnote{The first quote is taken from the general water law, with roots going back a thousand years to the so-called ``Gulating'' laws mentioned in the introduction. The second quote is taken from a law directed specifically at large-scale hydropower, introduced during the early days of the hydropower industry.}

{\begin{minipage}[t]{16em}
 \begin{aquote}{\tiny \indexonly{wra00}\dni\cite[13]{wra00}} \footnotesize A river system belongs to the owner of the land it covers, unless otherwise dictated by special legal status. [...]

The owners on each side of a river system have equal rights in exploiting its hydropower.
\end{aquote}  
\end{minipage}}
{\begin{minipage}[t]{22em}
\begin{aquote}{\tiny \indexonly{ica17}\dni\cite[1]{ica17} (after amendment in 2008)} \footnotesize Norwegian water resources belong to the general public and are to be managed in their interest. This is to be ensured by public ownership.
\end{aquote}
\end{minipage}} \\

The intended reading of section 1 of the \cite{ica17}, quoted on the right above, is that it provides a ``general starting point''.\footnote{See \cite[72]{otprp61}.} According to the Ministry, it expresses what has always been the purpose of the legislation used to regulate large-scale hydropower.\footnote{See \cite[72]{otprp61}.} 

This should not be understood as an explicit attack on the principle of private ownership expressed in section 13 of the \cite{wra00}, quoted on the left above.\footnote{There are no indications in the preparatory materials that the Ministry sought to confront the principles of ownership encoded in the \cite{wra00}.} However, the Ministry's comment underscores the extent to which the government regards it as natural to interfere with private rights to waterfalls, to pursue policies that it regards to be in the interest of the public. Taken in this light, section 1 of the \cite{ica17} reflects the prevailing opinion that there are few, if any, recognised limits on the state's power to manage privately owned water resources.\footnote{For a reflection of the same attitude, citing the state's broad regulatory competence as the main reason not to nationalise Norwegian water power rights, I refer to the preparatory documents underlying the \cite{wra00}. See \cite[152-153]{nou94}.}

This aspect of the Norwegian system has become particularly significant following the liberalisation of the electricity sector in the early 1990s.\footnote{See, e.g., \cite{larsen06}.} Since then, there have been an increasing number of cases where owners who are interested in undertaking their own development schemes attempt to fend off commercial energy companies wishing to expropriate.\footnote{See, e.g., \cite{sofienlund07}.} Importantly, the state has tended to side with the commercial companies in these cases, granting them the authority to expropriate for economic development. This has resulted in several Supreme Court decisions on hydropower and expropriation in the past few years, all of which have been in favour of the energy companies.\footnote{See  \cite{uleberg08,jorpeland11,klovtveit11,otra13}.} Before discussing these cases in more detail in the next chapter, I provide an in-depth analysis of hydropower in the law and in practice, to shed further light on the underlying conflict that has led to the recent surge in cases on expropriation. First, I briefly present the key legislation regulating the hydropower sector.

\subsection{The Water Resources Act}\label{sec:wra00}

The \cite{wra00} contains the basic rules regarding water management in Norway. This Act is not only concerned with hydropower, but regulates the use of river systems and groundwater generally.\footnote{See the \indexonly{wra00}\dni\cite[1]{wra00}. A river system is defined as ``all stagnant or flowing surface water with a perennial flow, with appurtenant bottom and banks up to the highest ordinary floodwater level'', see the \indexonly{wra00}\dni\cite[2]{wra00}. Artificial watercourses with a perennial flow are also covered (excluding pipelines and tunnels), along with artificial reservoirs, in so far as they are directly connected to groundwater or a river system, see the \indexonly{wra00}\dni\cite[2a, 2b]{wra00}.} In section 8, the Act sets out the basic license requirement for anyone wishing to undertake measures in a river system.\footnote{Measures in a river system are defined as interventions that ``by their nature are apt to affect the rate of flow, water level, the bed of a river or direction or speed of the current or the physical or chemical water quality in a manner other than by pollution'', see the \indexonly{wra00}\dni\cite[3a]{wra00}.} The main rule is that if such measures may be of ``appreciable harm or nuisance'', then a license is required.\footnote{See the \indexonly{wra00}\dni\cite[8]{wra00}. There are two exceptions, concerning measures to restore the course or depth of a river, and concerning the landowner's reasonable use of water for his permanent household or domestic animals, see the \indexonly{wra00}\dni\cite[12|15]{wra00}.} The water authorities themselves decide if this condition is met.\footnote{See the \indexonly{wra00}\dni\cite[18]{wra00}.} In relation to hydropower development, it is established practice that most hydropower projects over 1000 KW will be deemed to require a license.\footnote{See, e.g., \cite{nve09}. Exceptions are possible, for instance projects that upgrade existing plants, or which utilise water flowing between artificial reservoirs.}

The basic assessment criterion is that a license ``may be granted only if the benefits of the measure outweigh the harm and nuisances to public and private interests affected in the river system or catchment area''.\footnote{See the \indexonly{wra00}\dni\cite[25]{wra00}.} Hence, the water authorities are empowered to decide whether a licence {\it should} be granted, if they find that the benefits outweigh the harms. The courts are very reluctant to censor the discretion of the administrative decision-makers on this point.\footnote{This is an expression of the principle of ``freedom of discretion'' ({\it det frie forvaltningsskjønn}) for the administrative branch, a fundamental tenet of Norwegian administrative law. See generally \cite[71-74]{eckhoff14}.}

\noo{The Ministry of Petroleum and Energy maintains indirect control over the assessment process by issuing directives regarding the administrative procedure in licensing cases.\footnote{See section 65 of the \cite{wra00}.} In addition, the procedure is determined in large part by administrative practices developed by the water authorities themselves. A further discussion of the licensing procedure is given in section \ref{sec:4:3:1}. }

\noo{In principle, many of the rules in the \cite{paa67} apply at the administrative stage. However, these rules are of limited practical relevance compared to sector-specific practices. This has raised controversy in recent years, particularly in cases involving expropriation, as discussed in chapter \ref{chap:5}, section \ref{sec:5:6}. 

A few basic procedural rules are encoded directly in the \cite{wra00}. This includes rules to ensure that the application is sufficiently documented, so that the authorities have enough information to assess its merits.\footnote{See the \indexonly{wra00}\dni\cite[23]{wra00}.} Moreover, a basic publication requirement is expressed, stating that applications are public documents and that the applicant is responsible for giving public notice. The intention is that interested parties should be given an opportunity to comment on the plans.\footnote{See the \indexonly{wra00}\dni\cite[24]{wra00}. There are some exceptions to the requirement to give public notice, however. It may be dropped in case it appears superfluous, or if the application must be rejected or postponed, see the \indexonly{wra00}\dni\cite{wra00} ss 24a--24c.} More detailed rules for public notice of applications are given in section 27-1 of the \cite{pb08}, which also applies to licensing applications under section 8 of the \cite{wra00}.\footnote{In addition, I mention section 22, which regulates the relationship between licensing and planning in relation to water resources. In essence, the section stipulates that the water authorities may prioritise planning over assessment of individual licensing cases, e.g., by refusing to take applications under consideration if they interfere with ongoing planning procedures. However, the section leaves significant room for discretion in this regard. It also bears noting that watercourse planning is placed under centralised government control. This contrasts with land use planning in general, which is mainly the responsibility of the local municipality governments. See generally \cite{sp}.} }

\noo{Furthermore, an important rule of principle is given in section 22, regarding the relationship between applications for licenses and governmental ``master plans'' for the use or protection of river systems in a greater area. These plans have no clear legislative basis, but were introduced through parliamentary action in the 1980s, when the parliament decided to initiate such planning in an effort to introduce a more holistic basis for assessment of licensing applications.\footnote{Today, the planning authority is delegated to the Directorate of Natural Preservation and the NVE. See \cite{sp}.}
 
According to section 22 of the \cite{wra00}, if a river system falls within the scope of a master plan that is under preparation, an application to undertake measures in this river system may be delayed or rejected without further consideration.\footnote{See the \indexonly{wra00}\dni\cite[22]{wra00}, para 1.} Moreover, a license may only be granted if the measure is without appreciable importance to the plan.\footnote{See the \indexonly{wra00}\dni\cite[22]{wra00}, para 1.} In addition, once a plan has been completed, the processing of applications is to be based on it, meaning that an application which is at odds with some master plan may be rejected without further consideration.\footnote{See the \indexonly{wra00}\dni\cite[22]{wra00}, para 2.} It is still possible to obtain a license for such a project, but if it harnesses less hydropower than the project indicated by the plan, section 22 states that only the Ministry may grant it.\footnote{See the \indexonly{wra00}\dni\cite[22]{wra00}, para 2.}
}

The rules in the \cite{wra00} apply to any measures in river systems, not only hydropower projects. However, special rules 
that apply to hydropower cases are described in other statutes, the most important being the \cite{wra17}. \noo{ which is specifically aimed at a certain subgroup of hydropower schemes, namely those that involve regulation of the flow of water in a river system.\footnote{See section \ref{sec:wra17} below.} However, according to section 19 of the \cite{wra00}, many provisions from the \cite{wra17} also apply to unregulated, run-of-river, schemes, if they generate more than 40 GWh per annum.\footnote{See the \indexonly{wra00}\dni\cite[19]{wra00}.} }

\subsection{The Watercourse Regulation Act}\label{sec:wra17}

In order to maximise the output of a hydropower scheme, the flow of water may be regulated using dams or diversions. Regulation was particularly important in the early days of hydropower, before the national electricity grid was developed.\footnote{See \cite[83]{uleberg08}.} Indeed, in the early days, it was common for electricity producers to get paid based on the stable effect they were able to deliver, rather than the total amount of energy they harnessed.\footnote{See \cite{sofienlund07}.}

Today, this has changed, as producers get paid based on the total amount of electricity they deliver,  measured in kilowatt hours (KWh). The price fluctuates over the year, and the supply-side is still influenced by instability in the waterflow in Norwegian rivers. However, the smoothing effect of the national grid means that run-of-river schemes can be carried out profitably, even if most of the electricity from the plant is produced during peak periods.

Despite the growing importance of run-of-river schemes, many key rules regarding hydropower development are still found in the \cite{wra17}. This Act defines regulations as ``installations or other measures for regulating a watercourse's rate of flow''. It also explicitly states that this covers installations that ``increase the rate of flow by diverting water''.\footnote{See the \indexonly{wra17}\dni\cite[1]{wra17}.} The core rule of the Act is that watercourse regulations that affect the rate of flow of water above a certain threshold are subject to a special licensing requirement.\footnote{See the \indexonly{wra17}\dni\cite[2]{wra17}.}

The threshold is defined in terms of the notion of a ``natural horsepower'', such that a license is required if the regulation yields an increase of at least 400 natural horsepower in the river. Natural horsepower is a gross estimate of the power that can be harnessed from a river continuously for at least 350 days a year.\footnote{See the \indexonly{wra17}\dni\cite[2]{wra17}.} The definition is a simple mathematical expression, given below:

$$
nat.hp(Q,H) = 13.33 \times H \times Q
$$

This formula states that the natural horsepower of a regulation project ($nat.hp(Q,H)$) is a function of two variables, $H$ and $Q$. The constant factor $13.33$ is the force of gravity of Earth exerted on a mass of 1 kg (or, approximately, 1 litre of water). The variable $H$ is the difference in altitude (measured in metre) from the intake dam to the power generator. The variable $Q$ is the amount of water (measured in litre) continuously available per second of the day, for at least 350 days per year. The result is then a gross estimate (assuming no energy loss) of the stable horsepower output of a hydroelectric plant that harnesses the power of $Q$ litres of water per second over a difference in altitude of $H$ metres.

Section 2 of the \cite{wra17} asks for the {\it increase} of this figure after regulation. To arrive at this number, one first uses the formula with $Q$ taken to be $Q_1$, the stable water flow prior to regulation, before calculating it with $Q$ taken to be $Q_2$, the stable water flow after regulation. The difference between the second and the first figure ($nat.hp(Q_2,H) - nat.hp(Q_1,H)$) is the increase of natural horsepower resulting from regulation.

Effectively, at a time when electricity had to be produced at a stable effect, from a stable source of power, this increase in natural horsepower was a gross estimate of the value added to the river by regulation. In the present context, it suffices to say that if a hydropower project involves regulation at all (i.e., if it is not a run-of-river scheme), it will indeed yield 400 natural horsepower or more. Hence, a special license will be required pursuant to section 2 of the \cite{wra17}. 

The criteria for granting a regulation license are similar to those for granting a license pursuant to the \cite{wra00}. In particular, section 8 of the \cite{wra17} states that a license should ordinarily be issued only if the benefits of the regulation are deemed to outweigh the harm or inconvenience to public or private interests.\footnote{See the \indexonly{wra17}\dni\cite[8]{wra17}.} In addition, it is made clear that other deleterious or beneficial effects of importance to society should be taken into account.\footnote{See the \indexonly{wra17}\dni\cite[8]{wra17}.} Finally, if an application is rejected, the applicant can demand that the decision is submitted for review by Parliament.\footnote{See the \indexonly{wra17}\dni\cite[8]{wra17}.} %However, the \cite{wra17} contains more detailed rules regarding the procedure for dealing with license applications, c.f., section \ref{sec:4:3:1} below.

\noo{The most practically important is that the applicant is obliged to carry out an impact assessment pursuant to the \cite{pb08}. This means that the applicant must organise a hearing and submit a detailed report on positive and negative effects of the development, prior to submitting a formal application for a licence. Effectively, at least {\it two} detailed rounds of assessment are therefore required before a license is granted.

In addition to prescribing impact assessments, the \cite{wra17} contains more specific rules concerning the second public hearing that should take place, when the application as such is processed. First, the applicant should make sure that the application is submitted to the affected municipalities and other interested government bodies.\indexonly{wra17}\dni\footcite[6]{wra17} Second, the applicant should send the application to organisations, associations and the like whose interests are ``particularly affected''.\indexonly{wra17}\dni\footcite[6]{wra17} Along with the application, these interested parties should be given notification of the deadline for submitting comments, which should not be less than three months.\footnote{See the \indexonly{wra17}\dni\cite[6]{wra17}.} The applicant is also obliged to announce the plans, along with information about the deadline for comments, in at least one commonly read newspaper, as well as the Norwegian Official Journal.\footnote{See the \indexonly{wra17}\dni\cite[6]{wra17}. The Norwegian Official Journal is the state's own announcement periodical.}
}
%A license pursuant to the \cite{wra17} might be cumbersome to obtain, but a successful application also results in a significant benefit. Most importantly, the license holder then automatically has a right to expropriate the necessary rights needed to undertake the project, including the right to inconvenience other owners.\footnote{See \cite[16]{wra17}.} Hence, expropriation is a side-effect of a regulation license. Even so, the issue of expropriation rarely receives any special consideration in regulation cases. In particular, the assessment undertaken by the water authorities is focused on the licensing issue, which does not compel them to direct any special attention towards owners' interests.\footnote{I demonstrate this, and discuss it in much more depth, in Chapter \ref{chap:4}, Section \ref{sec:jorpeland}.}

In general, the issue of who owns and controls the water resources in question receives little attention in relation to licensing applications, both pursuant to the \cite{wra17} and the \cite{wra00}. Instead, the focus is on weighing environmental interests against the interest of increasing the electricity supply and facilitating economic development. The issue of resource ownership is more prominent in relation to a third important statute, namely the \cite{ica17}.

\subsection{The Industrial Licensing Act}\label{sec:ica17}

In the early 20th century, industrial advances meant that Norwegian waterfalls became increasingly interesting as objects of foreign investment. To maintain national control of water resources, Parliament passed an Act in 1909 that made it impossible to purchase valuable waterfalls without a special license.\footnote{See \cite[59]{falkanger87}.} The follow-up to this Act is the \cite{ica17}, which is still in force. It applies to potential purchasers and leaseholders of rivers that may be exploited so that they yield more than 4000 natural horsepower.\footnote{Unlike section 2 of the \cite{wra17}, this asks only for the number of horsepower in the river (after regulation), not the {\it increase} of this number.}

To reach this number requires a substantial regulation, so the Act does not apply to many run-of-river hydropower schemes, even large-scale projects. Originally, the main rule in the \cite{ica17} stated that all licenses granted to private parties were time-limited, and that the waterfalls would become state property without compensation when they expired, after at most 60 years.\footnote{See the previous \indexonly{ica17}\dni\cite[2]{ica17}, in force before the amendment on 26 September 2008.} This was known as the rule of {\it reversion} in Norwegian law.\footnote{This is a misnomer, however, in light of how most rivers and waterfalls were originally owned by local smallholders, not the state.}

In a famous Supreme Court case from 1918, the rule was upheld after having been challenged by owners on constitutional grounds.\footnote{See \cite{johansen18}.} This was based on the finding that reversion represented a form of regulation of property, not expropriation. Hence, it could not be challenged on the basis of section 105 of the Constitution, even though the owners were not awarded any compensation. 

While the rule of reversion withstood internal challenges, it was eventually struck down by the EFTA Court in 2007, as a breach of the EEA agreement.\footnote{See \cite{efta07}. The EEA (European Economic Area) agreement sets up a framework for the free movement of goods, persons, services and capital between Norway, Iceland, Lichtenstein and the European Union. The EFTA (European Free Trade Association) oversees the implementation of the EEA for those members of EFTA that are also members of the EEA (all except Switzerland). For further details, see generally \cite{bull94,magnussen02,fredriksen09}.} This conclusion was based on the fact that reversion only applied to privately owned companies, which the Court regarded as an illegitimate form of discrimination. After this ruling, the \cite{ica17} was amended. Today, only companies where the state controls more than 2/3 of the shares may purchase waterfalls or rivers to which the Act applies.\footnote{See the \indexonly{ica17}\dni\cite[2]{ica17}.}

This means that such rivers and waterfalls can only be bought, leased or expropriated by companies in which the state is a majority shareholder. In practice, however, landowners are still able to sell the land from which the right to a waterfall originates, even if this also means transferring the waterfall to a new owner. The rule is typically only enforced when riparian rights as such are transferred, specifically for the purpose of large-scale hydropower development. In particular, small-scale development and large run-of-river schemes can still usually be carried out by local owners. The policy justification for the (amended) \cite{ica17} is based on the idea that giving preference to state-owned actors will protect the public. However, this perspective clashes with the fact that the electricity sector itself has been liberalised. The state may be a majority shareholder in the most powerful companies, but these companies are now run according to commercial principles, with little or no direct political involvement.\footnote{See \cite[86]{efta07}.}

Hence, as the EFTA court highlights in its judgement on reversion, there appears to be a lack of convincing policy reasons why state-owned companies should be given preferential treatment.\footnote{See \cite[84-87]{efta07}.} In light of this, Norway's response to the Court's decision is a curious one: instead of creating a level playing field, the preference given to state-owned commercial companies is made even more marked, as privately owned companies are now excluded from one segment of the hydropower market altogether.

%Of course, the public benefits indirectly from the fact that public bodies, as shareholders, are entitled to dividends. But it is not clear why this benefit should be considered in a different light than other indirect financial benefits which might as well be extracted from private companies, e.g., through taxation. Moreover, public-private partnerships are still permitted, as private actors may own up to two-thirds of ``state-owned'' companies. What this means is that the preferential treatment given to state actors is in fact also extended to those private actors that the state happen to prefer. Interestingly, this style of regulation contrasts quite sharply with some of the key ideas behind the basic building block of the liberalised electricity market, namely the \cite{ea90}.

\subsection{The Energy Act}\label{sec:ea}

Before 1990, the Norwegian electricity sector was tightly regulated by the government.\footnote{See generally \cite{bye05,skjold07}.} The responsibility for the national grid was divided between various public utilities that would also typically engage in electricity production, wielding monopoly power within their districts. The most powerful utilities were controlled by the state, which also developed large-scale hydropower to supply the metallurgical industry with cheap electricity.\footnote{See \cite[67-71]{thue96}.} However, the county councils and the municipalities maintained a significant stake in the hydroelectric sector, as they often controlled the utilities responsible for the electricity supply in their own local area.\footnote{See \cite[85]{thue96}.} 
Prior to 1990, there was no real competition on the electricity market, and the local monopolists could deny other energy producers access to their segment of the distribution grid.\footnote{See \cite[83-84]{uleberg08}.}

This system was abandoned following the passage of the \cite{ea90}.\footnote{See generally \cite{bibow11}.} This Act set up a new regulatory framework, where management of the grid was decoupled from the hydropower production sector.\footnote{See generally \cite{bye05}.} In particular, the Act established a system whereby consumers could choose their electricity supplier freely. At the same time, the Act aimed to ensure that producers were granted non-discriminatory access to the electricity grid. This laid the groundwork for what has today become an international market for the sale of electricity, namely the Nord Pool.\footnote{See generally \cite{galtung07}.}

In response to this, monopoly companies were reorganised, becoming commercial companies that were meant to compete against each other, and against new actors that entered the market.\footnote{See \cite{claes11}.} In addition to commercialisation, the market-orientation of the sector has also lead to centralisation, as many of the locally grounded municipality companies have disappeared as a result of mergers and acquisitions.\footnote{Today, the 15 largest companies, largely controlled by the state and some prosperous city municipalities, own roughly 80\% of Norwegian hydropower, measured in terms of annual output. See \cite[28]{otprp61}. The process causing this concentration started long before the market-oriented reform of the sector. In particular, after the Second World War, there was a significant push by the state towards increased centralisation, see \cite{skjold06,thue06b}.} As a result, the local and political grounding of the electricity sector, which used to be ensured through decentralised municipal ownership, has been significantly weakened.

At the same time, the fact that any developer of hydropower is now entitled to connect to the national grid gives private actors a possibility of entering the Norwegian electricity market. They may do so not merely as (minority) shareholders in former utilities, but also as {\it competitors}, as long as they stick to run-of-river or small-scale hydropower.\footnote{See generally \cite{larsen06,larsen08,larsen12}.} In the next section, I give a step-by-step presentation of the licensing procedure for hydropower, which serves to summarise the legislative framework and provide information about the institutional framework within which it is called to function.

\subsection{The Licensing Procedure}\label{sec:4:3:1}

The water authorities in Norway are centrally organised. The most important body is the Norwegian Water Resources and Energy Directorate (NVE), based in Oslo. In many cases, the NVE have been delegated authority to grant development licenses themselves, but in case of large-scale development, they only prepare the case, then hand it over to the Ministry of Petroleum and Energy.\footnote{See Delegation of 19 December 2000, from the Ministry of Petroleum and Energy (FOR-2000-12-19-1705) and Directive of 15 December 2000, from the King in Council (FOR-2000-12-15-1270), pursuant to the \indexonly{wra00}\dni\cite[64]{wra00}.} The Ministry, in turn, gives its recommendation to the King in Council, who makes the final decision.\footnote{See Directive of 15 December 2000, from the King in Council (FOR-2000-12-15-1270).} Parliament must also be consulted for regulations that will yield more than 20 000 natural horsepower.\footnote{See the \indexonly{wra17}\dni\cite[2]{wra17}.}

As indicated by the survey of relevant legislation given in previous sections, there are many categories of hydropower projects. Moreover, different categories call for different licenses. Hence, the first step in the application process is for the developer to determine exactly what licenses they require. Generally speaking, the larger the project is, the more licenses it requires. 
\noo{
%This is further complicated by the fact that some categories overlap, since they are based on different measuring sticks for assessing the scale of an hydropower project. %One important parameter is the power of the hydropower generator, measured in MW (Megawatts). There are four categories of hydropower formulated on this basis: the micro plants (less than $0.1$ MW), the mini plants (less than $1$ MW), the small-scale plants (less than $10$ MW), and the large-scale plants (more than $10$ MW). In practice, one tends to use small-scale hydropower more loosely, to refer to all projects less than 10 MW. Still, a further qualification is sometimes required. For example, the authority to grant a license for a micro or mini plant has been delegated to the regional county councils since 2010, in an effort to reduce the queue of small-scale applications at the NVE.\footnote{See Delegation letter from the Ministry of Petroleum and Energy, dated 07 December 2009, available at \url{http://www.nve.no} (accessed 24 August 2014). The county council is an elected regional government institution situated between the municipalities and the central government. There are 19 county councils in Norway as of 01 January 2015. They are comparatively less important than both the municipalities and the central government, but have several  responsibilities, particularly in relation to infrastructure, education and resource management. See generally \cite{berg15}.} The council's decision is based on a (simplified) assessment made by the regional office of the NVE. In addition, licenses for micro and mini plants may be granted even in watercourses that have protected status pursuant to environmental law.\footnote{See Decision no 240, Stortinget (2004-2005), St.prp.nr.75 (2003-2004) and Innst.S.nr.116 (2004-2005).}

For small-scale plants proper, the authority to grant a license is delegated to the NVE, with the Ministry serving as the instance of appeal.\footnote{See Delegation of 19 December 2000, from the Ministry of Petroleum and Energy (FOR-2000-12-19-1705).} For large-scale plants, the granting authority is the King in Council, based on a recommendation from the Ministry.\footnote{See Directive of 15 December 2000, from the King in Council (FOR-2000-12-15-1270).} However, in practice, the decision is usually closely based on assessments and recommendations provided by the NVE.\footnote{For a detailed guide to the administrative process for large-scale applications, published by the NVE, see \cite{stokker10}.}

While the relevant licensing authority depends on the effect of the plant, the kind of license required depends on a different categorisation, relating to the level of planned water regulation, measured in natural horsepower. Here, there are three categories: run-of-river schemes  (less than $500$ natural horsepower), non-industrial regulations ($500 - 4000$ natural horsepower), and industrial regulations (more than $4000$ natural horsepower).\footnote{See the \indexonly{wra17}\dni\cite[2]{wra17} and the \indexonly{wra17}\dni\cite[1,2]{ica17}.}

Almost all hydropower schemes require a license pursuant to section 8 of the \cite{wra00}.\footnote{As mentioned in section \ref{sec:wra00}, the exceptions are very small schemes (usually mini or micro) that are deemed to be relatively uncontroversial. Such schemes only require a license pursuant to the \cite{pb08}.} For run-of-river schemes, no further licenses are required for the development itself, although an operating license pursuant to the \cite{ea90} is typically required for the electrical installations.\footnote{See the \indexonly{ea90}\dni\cite[3-1]{ea90}.} For schemes involving a non-industrial regulation, an additional license pursuant to section 8 of the \cite{wra17} is required. Industrial regulation schemes require yet another license, pursuant to section 2 of the \cite{ica17}.
}
As is to be expected, the complexity of the licensing procedure tends to increase with the number of different licenses required. However, the licensing applications tend to be dealt with in parallel, so that all licenses are granted at the same time, following a unified assessment. In practice, when the \cite{wra17} applies, it structures the procedure as a whole, also those aspects that pertain to other licenses.\footnote{Recall that the \cite{wra17} applies to most projects involving regulation, as well as all projects that will yield more than 40 GWh/year.} In these cases, there is a detailed examination of environmental effects. The procedure usually includes a designated impact assessment, with a separate public hearing, that the applicant must complete before the water authorities will consider their application.\footnote{See Directive of 19 December 2014 (FOR-2014-12-19-1758), pursuant to the \indexonly{pb08}\dni\cite[1-2,14-6]{pb08}.}
\noo{
In addition, yet another categorisation of hydropower schemes is used to determine the relevant application procedure. This categorisation is based on the annual production of the proposed plant, measured in GWh/year. There are three categories: simple schemes (less than $30$ GWh/year), intermediate schemes ($30 - 40$ GWh/year), and complicated schemes (more than $40$ GWh/year). As mentioned in section \ref{sec:wra17}, the most important rules in the \cite{wra17} apply to complicated schemes, regardless of whether or not the scheme involves a regulation.\footnote{See the \indexonly{wra00}\dni\cite[19]{wra00}.} In addition, applications for such schemes must be accompanied by an impact assessment pursuant to section 14-6 of the \cite{pb08}.

This means that the applicant is required to organise a public hearing prior to submitting their formal application, to collect opinions on the project and provide an overview of benefits and negative effects of the plans, particularly as they relate to environmental concerns.\footnote{See Directive of 19 December 2014 (FOR-2014-12-19-1758), pursuant to the \indexonly{pb08}\dni\cite[1-2,14-6]{pb08}.} In practice, if an impact assessment is required this significantly increases the scope and complexity of the application process.

For intermediate schemes that do not involve regulation, the rules in the \cite{wra17} do not apply. However, impact assessments {\it may} still be required.\footnote{See \cite[20]{stokker10}.} Here the threshold of 30 GWh/year has been set as an additional threshold by the NVE, who have been delegated authority to require impact assessments for hydropower projects even when these yield less than 40 GWh/year.\footnote{See Directive of 19 December 2014 (FOR-2014-12-19-1758).} For the intermediate schemes, NVE decides whether an impact assessment is required on a case-by-case basis. For simple schemes, on the other hand, impact assessments will not be required. Such schemes make up the core of what is described as small-scale hydropower in daily language.
}
The time from application to decision can vary widely, depending on the complexity of the case, the level of controversy it raises, and the priority it receives by the licensing authority. Usually, the assessment stage itself will last 1-3 years, sometimes longer.\footnote{See \cite[84-85]{nou129}.} While large-scale schemes involve more complicated procedures, they are also typically given higher priority than small-scale schemes. In recent years, following the surge of interest in small-scale development, a processing queue has formed at the NVE.\footnote{See \cite[84]{nou129}.} This means that small-scale applications typically have to wait a long time, sometimes several years, before the NVE begins processing them.\footnote{See \cite[84]{nou129}.}

%As I will discuss in more depth in the next chapter, the issue of expropriation is rarely given special attention during the application assessment. This is so even in cases when an application to expropriate waterfalls is submitted alongside the licensing applications. The issue of expropriation is rarely singled out for special treatment, at least not in cases of large-scale development. %Moreover, as mentioned in Section \ref{sec:hl}, an automatic right to expropriate follows from section 16 of the \cite{wra17}.

The applicant is expected to submit application notices for publication in local newspapers, and for larger projects there will typically also be an information meeting arranged in the local area, where the applicant and the authorities appear side by side, presenting the plans and the licensing procedure respectively.\footnote{See \cite[23]{stokker10}.} For large-scale projects, it is also common for the applicant to distribute brochures widely in the local area. These procedural arrangements arguably reflect some concern for the interests of local populations. However, the procedure is organised in a way that also creates the impression that the applicant enjoys significant state-backing from the start.

Indeed, applicants not only communicate with locals in place of the authorities, they are also given responsibility for many material aspects of the assessment process, including the often crucial assessment of possible alternatives.\footnote{See \cite[24]{stokker10}.} This would seem to raise competency questions, particularly in cases where the owners  present owner-led development as an alternative to expropriation.\footnote{It also raises questions about the impartiality of the assessment of alternatives motivated by environmental concerns (especially with respect to the question of what alternatives to evaluate in detail). Quite generally, how the government executes its duty to assess alternatives in licensing cases is a thorny issues in Norwegian environmental law, see generally \cite{backer10,winge13}.} However, even in these cases, the applicant hoping to expropriate will be tasked with evaluating the owners' plans on behalf of the government.\footnote{This remarkable form of administrative subcontracting has been given a stamp of approval by the Supreme Court, see \cite[51-55]{jorpeland11}.}

More generally, as discussed in greater depth in chapter \ref{chap:5}, the protection offered to waterfall owners is very limited. For instance, the government does not recognise a duty to notify these owners individually, to ensure that they are informed of what is at stake for them as owners of a very valuable resource. Rather, a generic letter is typically sent by the applicant to all affected private parties. The statement that private property ``will be expropriated'' unless a settlement is reached has also been observed.\footnote{In the case of \cite{sauda09}.}

After the hearing stage, the NVE will usually compile a final report along with a recommendation and send it to the interested parties for comments.\footnote{See the \indexonly{wra17}\dni\cite[6]{wra17}.} It is established practice that local owners do {\it not} count as interested parties in this regard.\footnote{See \cite[46]{jorpeland11}.} Hence, while the municipalities and various environmental interest groups are informed of how the case progresses and asked to comment prior to the final decision, the owners must inquire on their own accord if they wish to be kept up to date on the application process.\footnote{In a written statement to the Supreme Court in the case of \cite{jorpeland11}, the director of the hydropower division of the Ministry pointed out that the documents would be made available on the web page of the NVE and that local owners had to ``look after their own interests''.}

In summary, the procedural framework surrounding licensing of hydropower development leaves local owners in a precarious position, especially when the applicants wish to expropriate their waterfalls. At the same time, the liberalisation of the electricity sector means that owners are in a far better position than before when it comes to developing hydropower themselves. This is discussed in more depth in the next section.
%Given that expropriation is often an automatic side-effect of a development license, this already suggests that legitimacy issues are likely to likely to arise when waterfalls are taken for hydropower. I return to this issue specifically in the next chapter. First, I will discuss market practices in more depth, focusing on the changes that resulted from the liberalisation reform of the early 1990s. 

\section{Hydropower in Practice}\label{sec:4:4}

The history of hydropower in Norway can be roughly divided into four stages. The first stage was the development that took place prior to 1909. During this time, private actors dominated, with public ownership playing a minor role.\footnote{See \cite{otprp61}.} Moreover, there were many private interests speculating in acquiring Norwegian waterfalls, anticipating the value that these would have for industrial development.\footnote{See \cite[30-31]{nou04}.}

After 1909, the introduction of licensing obligations and the rule of reversion made it much harder for private companies to acquire waterfalls that were suitable large-scale industrial development. At the same time, local municipalities began to invest in hydropower to provide electricity to their citizens, a service they were increasingly being obliged to provide.\footnote{See \cite{otprp61}.} This marked the start of the second stage of hydropower development, which saw the development of a more strictly regulated sector. However, this sector was also highly decentralised, for a large part dominated by local actors.

In fact, throughout the first half of the 20th century, most hydroelectric plants were small-scale plants that supplied local communities with electricity.\footnote{See \cite[11]{utbygd46}. This is a report from the water directorate published in 1946, showing that as of 31 December 1943, $97.8 \%$ of all hydroelectric plants in Norway were small-scale plants. However, these plants contributed only $28 \%$ of the total hydroelectric power installed at that time.} Moreover, as late as in 1943, $89 \%$ of all hydroelectric power stations in Norway were still private, many of which were mini and micro plants that were owned and operated by the local community.\footnote{See \cite[6]{utbygd46}. See also \cite[111]{hindrum94}.} However, many bigger plants were also under private ownership, and $57 \%$ of the total hydroelectric power available at this time was supplied by the private sector. 
%This clearly illustrates the importance of smaller, local initiatives, in the process of providing Norway with electricity, particularly in rural areas. Interestingly, while the micro and mini plants accounted for $72.9 \%$ of the total number of plants, they only accounted for $1.6 \%$ of the total electricity supply.\footnote{See \cite[7]{utbygd46}.}

By the end of 1943, $80 \%$ of the Norwegian population had access to electricity at home.\footnote{In rural areas, the corresponding figure was $70 \%$, see \cite[7]{utbygd46}.} Hence, the decentralised approach to hydropower development, based on private ownership and local control, had not been an impediment to the supply of electricity to most of the country's population.

However, the regulatory regime was soon to undergo a significant change, designed to facilitate industrial development and increased state control. This change came quite rapidly after the Second World War, when the central government began to invest heavily in hydropower, often to ensure economic development by subsidising the metallurgical industry.\footnote{See \cite[59-65]{thue96}.} This period saw increased marginalisation of small private electricity companies, as well as local owners.\footnote{At the same time, powerful (private) metallurgical interests benefited greatly, sometimes also at the expense of the general supply of electricity. See \cite[65-71]{thue96}.} Indeed, it was often demanded, as a condition for allowing local communities access to the national electricity grid, that local hydroelectric plants had to be shut down.\footnote{See \cite[111]{hindrum94}.} During this time, the development of hydropower was seen as an important aspect of rebuilding the nation. However, the goal was not primarily to supply the public with electricity, but rather to facilitate a specific kind of economic development that the central government regarded as desirable.\footnote{See \cite[59]{thue96}.}

The state-dominated system set up on this basis remained in place until the 1970s, when increasingly vocal opposition from environmental groups and local populations led to some reforms.\footnote{See \cite[71-75]{thue96}.} As the scale of typical development projects had increased significantly compared to earlier times, new projects would tend to meet with broader and better organised forms of resistance. In many cases, municipal and regional government institutions would join in opposition against large-scale development.\footnote{See \cite[71-72]{thue96}.} The typical response from the state was to introduce measures that sought to pacify the regional and municipal government opposition, which was considered more serious than opposition from local people and environmental groups. The standard approach was to grant an increased share of the financial benefit to local and regional institutions of government, to instil support for state-led development plans.\footnote{See \cite[73-76]{nilsen08}.} The centralisation process in the hydroelectric sector slowed down somewhat during this time.\footnote{See \cite[85]{thue96}.} However, despite limiting the discontent among local power groups, high-profile controversies continued to arise, most notably the {\it Alta} case discussed in the next chapter.

The fourth stage of hydropower development began in 1990 after the passage of the \cite{ea90}. The liberalisation that followed saw the transformation of the hydropower sector into a commercial market, based on profit-maximising and competition. As a result, the structure of decentralised management withered away further, as many municipality companies were either bought up by more commercially aggressive actors or forced to merge and change their business practices in order to remain competitive.\footnote{See \cite[583]{bibow03} (commenting on the increased concentration of power on the electricity market, following acquisitions and mergers after 1990).} At the same time, a new decentralised force emerged in the sector, in the form of local owner-led projects.\footnote{See section \ref{sec:4:4} below.}

The core idea behind the \cite{ea90} was that the electricity sector should be restructured in such a way that production and sale of electricity, activities deemed suitable for market regulation, would be kept organisationally separate from electricity distribution over the national grid, a natural monopoly. Grid companies are now obliged to facilitate access for producers, and after an amendment in 2009 this applies also when access necessitates new investments.\footnote{See Act no 105 of 19 June 2009 regarding changes in the \cite{ea90}.} However, the energy producer seeking access is typically required to reimburse the grid company for the cost of new investments, as determined in the first instance by the grid company itself (the NVE serves a supervisory function).\footnote{See Directive of 7 December 1990 (FOR-1990-12-07-959), s 3-4.} In addition, grid companies may still deny access in cases when the needed investments are not ``socio-economically rational''.\footnote{See the \indexonly{ea90}\dni\cite[3-4]{ea90}. The authority to decide whether this requirement is fulfilled is vested with the Ministry.}

\noo{However, the Act itself does not explain in any depth how this is to be achieved. In practice, the divide has not been strictly implemented. Most of the large energy companies in Norway continue to maintain interests in both distribution, production and sale of electricity, a phenomenon known as ``vertical integration''.\footnote{See \cite[580-583]{bibow03}.} In fact, the degree of vertical integration in the electricity sector initially increased after the passage of the \cite{ea90}.\footnote{See \cite[583]{bibow03}.}

\noo{ To some extent, the water authorities have responded to this by making use of their authority to give organisational directives when they grant distribution licenses.\footnote{See the \indexonly{ea90}\dni\cite[4-1]{ea90}, para 2, no 1.} For instance, electricity companies are now required to keep separate accounts for production, distribution and sale of electricity.\footnote{See Directive of 11 March 1999 (FOR-1999-03-11-302), s 4-4 a and s 2-6, issued by the NVE pursuant to Directive of 7 December 1990 (FOR-1990-12-07-959), s 9-1, pursuant to the \indexonly{ea90}\dni\cite[10-6]{ea90}.} It is also required that transactions across these functional divides are clearly marked, and that they are based on market prices.\footnote{See Directive of 11 March 1999 (FOR-1999-03-11-302), s 2-8.}}

The water authorities responded to this by accepting increased concentration of ownership, while also ordering distribution activities to be kept organisationally separate from other activities, for instance through the establishment of a special subsidiary company.\footnote{See \cite[581-582]{bibow03}.} Typically, a conglomerate structure is used, with a single parent company that controls both the distribution company, the production company and the sales company. Indeed, this model has now been implemented by most of the large energy companies in Norway.\footcite[582]{bibow03}

\noo{It seems unclear whether this approach really achieves the stated objective. By adopting the conglomerate model of organisation, the major players on the market have successfully gained control over a larger share of both the production and distribution facilities for electricity. Hence, these actors effectively control the core infrastructure that makes up the backbone of the Norwegian electricity sector. The {\it intention} is that monopoly power should only be exercised with respect to the distribution grid on non-discriminatory terms. But is this realistic when the conglomerate controlling the grid operator has significant stakes also in production and the trade of electricity?

This question calls for a separate study, and }
The extent to which this is an adequate response to increased concentration of power in the electricity sector will not be addressed in any depth here. However, I will direct attention at one aspect that arises with particular urgency for small-scale development of hydropower, concerning access to the grid. It is quite common, in particular, that small-scale projects remain unrealised because the grid is regarded to lack sufficient capacity to accommodate new electricity.\footnote{See, e.g., \cite[84,161-162]{nou129}.}
}

Often, the relevant grid company will be a sister company of an energy producer operating in direct competition with the company seeking access. This can raise questions about the impartiality of the assessments carried out by the grid company. In expropriation cases, this becomes an issue particularly in relation to the assessment of the cost of undertaking an alternative development scheme.\footnote{This assessment is often crucial, because it provides information about the value of the development potential that the owners stand to loose.} Riparian owners are rarely pleased when they realise that the expropriating party is part of the same conglomerate as the grid company that estimates the grid connection costs associated with owner-led development.\footnote{See, e.g., \cite{smibelg15}.}

%It has been pointed out, in particular, that the practical consequence of liberalization has been that the local accountability of the electricity sector has been lost, both organizationally and politically.\footnote{See \cite{agnell11}.
Meanwhile, the market-orientation of the electricity sector has reduced the level of political control and accountability. According to Brekke and Sataøen, this serves to set the reform that took place in Norway apart from similar energy reforms in Sweden and the UK.\footnote{See \cite{brekke12}.} Moreover, Brekke and Sataøen argue that this has resulted in a lack of legitimacy that has been a significant contributory cause of recent national-scale controversies, particularly with regards to the development of the national grid.\footnote{The most serious case so far is that of {\it Sima - Samnanger}, concerning a new distribution line that will cut through the area known as {\it Hardanger}, a scenic part of south-western Norway. The plans met with significant resistance at both the national and the local level, but the government pushed ahead, leading to confrontations that also involved some acts of civil disobedience. See \cite[22-23]{brekke12}.}

At the same time, the growth of the small-scale hydropower sector gives local communities a new voice, as market participants, thereby acting as a counterweight to centralisation and expert-rule. Since the mid- to late 1990s, the small-scale sector has grown significantly. In a recent report, the potential for profitable small-scale hydropower projects was estimated to be around 20 TWh per year.\footnote{See \cite{aanesland09}. For comparison, suggesting the scale of this potential, I mention that the total consumption of electricity in Norway in 2013 amounted to about 120 TWh, see \cite{statistikk13}. According to the government, about one third of the remaining potential for hydropower in Norway, measured in annual energy output, will come from small-scale projects. See \cite[231]{nou129}.} On this basis, the authors of the report estimate that the total present-day value of all waterfalls suitable for small-scale hydropower is about 35 billion Norwegian kroner, i.e., about 3.5 billion pounds.\footnote{See \cite[1]{aanesland09}.} This calculation is based on a model where the waterfalls are exploited in cooperation with an external commercial company, inspired by existing agreements between owners and the limited company {\it Småkraft AS}. Hence, the calculation might be an underestimate of what small-scale hydropower could represent for local communities if they remain in charge of development themselves.

Small-scale hydropower has become socially and political significant in Norway. In the report mentioned above, it is estimated that the value of rivers and waterfalls amount to just under 50 \% of the total equity in Norwegian agriculture.\footcite[1]{aanesland09} Moreover, hydropower is increasingly seen as a possibility for declining regions to counter depopulation and poverty. In some communities, small-scale hydropower is the only growth industry. For these communities, pursuing hydropower development at the local level also provides a way to regain some autonomy with respect to how local natural resources should be managed. Hence, small-scale hydropower takes on great political and social importance, not just for the owners of waterfalls, but for the community as a whole.

For an example of a community where small-scale hydropower has played such a role, I point to Gloppen, a municipality in the county of Sogn og Fjordane, in the western part of Norway. Here, 19 hydropower plants have been built in recent years, all except one by local owners themselves, amounting to a total production of over 250 GWh per year. This prompted the mayor to comment that ``small scale hydro-power is in our blood''.\footnote{See \cite{starheim12}.} When interviewed, he also directed attention at the fact that hydropower had many positive ripple effects, since it significantly increased local investment in other industries, particularly agriculture, which had been severely on the decline.

To achieve such effects, it is important to organise development in an appropriate manner. Moreover, to explain how waterfalls came to be as valuable as they are today, it is crucial to direct attention to the way in which waterfall owners initially asserted themselves on the market. In the following, I do this by giving an in-depth presentation of an early model for local involvement in hydropower development, presented at a seminar in 1996.\footnote{See \cite{dyrkolbotn96}.} This model contains an early expression of several ideas that would prove influential to the development of the small-scale hydropower sector.

%However, certain other aspects of the model have not been widely adopted. These are aspects that pertain to the balance of power between owners and cooperating developers, as well as the relationship that should be established with larger communities of non-owners, including environmental groups and other water users. Hence, considering the model in some depth, and assessing its impact, will allow me to shed light on desirable social functions of waterfall ownership, and the extent to which such functions are fulfilled on the market today.

\section{{\it Nordhordlandsmodellen}}\label{sec:4:5}

In five brief points, the {\it Nordhordlandsmodellen} sets out a framework for cooperation between waterfall owners, professional hydroelectricity companies, local communities, and society as a whole.\footnote{See \cite{dyrkolbotn96}. The model was the result of a collaboration between Otto Dyrkolbotn, a farmer and a lawyer, and Arne Steen, the director of {\it Nordhordland Kraftlag}, a municipality-owned energy company.} 

The first point makes clear that the aim of cooperation should be to ensure local ownership and control: external interests should never be allowed to hold more than 50 \% of the shares in the development company. If the company is organised as a limited liability enterprise, then the model stipulates that local residents -- not necessarily owners -- are to be given a right of preemption in the event that shares come up for sale. %The possibility of organising the development company as a local cooperative is also mentioned.\footnote{References needed.}

The second point of the model sets out a method for valuing the riparian rights prior to development. It stipulates that the appraisal should reflect the real value of such rights, normally estimated on the basis of lease capitalisation. More concretely, the valuation should be based on the premise that the riparian owners will be entitled to rent based on the level of annual production in the planned hydropower project. Then, for the purpose of appraisal, the expected rent per annum is capitalised to find the present value of the riparian rights, relative to the development project in question.\footnote{This approach stands in stark contrast to the earlier valuation method, discussed in chapter \ref{chap:5}, section \ref{sec:5:4:1}.}

After such a value has been calculated, the model stipulates that owners are to be given a choice of either leasing out their water rights to receive rent, or to use the capitalised value of (part of) this rent as equity to acquire shares in the development company. The third point in the model then offers a clarification, by stating that the development company should not in any event acquire ownership of riparian rights, but only a time-limited right of use. After 25-35 years, this usufruct should fall away and the waterfall should revert back to the owners of the surrounding land, free of charge. This is the proposed rule even in cases when the landowners themselves initially control the majority of the shares in the development company. Hence, the rule places a limit on alienation; no separation of water rights from land rights is allowed to last for more than 35 years. 

The {\it Nordhordland} model demonstrated the commercial viability of this organisational model by pointing to a concrete municipality-owned energy company that had stated its willingness to cooperate with owners on such terms, to help with financing and share the risk.\footnote{The company in question is Nordhordland Kraftlag, where one of the authors of {\it Nordhordlandsmodellen}, Arne Steen, was a director.} 

Following up on this organisational blueprint, the fourth and fifth points of the model describe the intended role of the local development company in society, by stressing the relationship between hydropower and other interests and potential uses of the affected river. Importantly, the model stipulates that potential developers should be willing to take on formal obligations towards other user groups. Moreover, obligations should not only be negatively defined, as duties to minimise or avoid harms. Positive obligations should also be introduced, such as duties to improve other qualities of the river system, and to engage in active cooperation with other users.

It is made clear that the overall aim is to ensure sustainable management of the river system as a whole. Interestingly, the model predicts that active local ownership will achieve more in this regard than what can be achieved through governmental regulation alone. This claim is illustrated by a concrete example of a case in which the local owners decided to pursue a scheme that was less environmentally invasive than the project endorsed by the water authorities.\footnote{Today, this project has become Svartdalen Kraftverk, finalised in 2006. It produces 30 GWh annually, enough electricity for about 1500 households.}

The model goes on to emphasise the need for integrated processes of resource planning and decision-making, to ensure that hydropower development is not approached as an isolated economic and environmental concern, but looked at in a broader social and political context. To achieve this, it is argued that local communities need to play an important role in the management of water resources. \noo{Another concrete example follows, regarding {\it Romarheimsvassdraget}, a river system in the municipality of Lindås, in the county of Hordaland.

This river system was originally intended for large-scale development undertaken by BKK AS, without the participation of local owners.\footnote{BKK AS is one of the 15 biggest hydropower companies in Norway, and would later also purchase Nordhordland Kraftlag.} The project would involve a total of three river systems, such that the water from {\it Romarheimselva} and another river would be diverted to a neighbouring municipality for hydropower development there. The local owners argued against these plans by proposing a number of smaller development schemes. Eventually, they were successful, as the NVE agreed to endorse an alternative consisting of 7 distinct run-of-river projects to be undertaken by local owners.\footnote{See \cite{vann25}.} } 

It is important to note that when {\it Nordhordlandsmodellen} was formulated, owner-led development of hydropower was still a recent phenomenon, driven forward by individual owners and local groups that saw the potential and had enough know-how to get organised. Later, commercial companies have emerged that specialises in cooperating with local owners.\footnote{See, e.g., \cite{larsen06}.} Today, this has made it relatively easy for owners to initiate small-scale hydropower development. Moreover, owners are often approached by interested commercial actors who wish to cooperate with them. Most of them rely on cooperation on terms that reflect the main ideas expressed in the first three points of {\it Nordhordlandsmodellen}.

However, several adjustments have become standard, and these systematically benefit the external partner: the requirement that locals should at all times control a majority of the shares is dropped, the period of usufruct is typically longer than 35 years, the reversion to the landowners after this time is made conditional on payment for machines and installations, and no preemption rights are granted to local residents.\footnote{See generally \cite{hauge15}.} However, the core idea that riparian rights are to be valued based on a capitalisation of future rent is accepted. This means, in turn, that local owners rarely need to raise any additional capital to acquire shares in the development company. Moreover, the rent itself can become a significant source of income.

There are two main approaches to calculating this rent. The first approach, introduced already in {\it Nordhordlandsmodellen}, specifies the rent as a percentage of the gross revenue from sale of electricity, today often around 10-20 \%.\footnote{Source: contracts presented to the court in \cite{sauda09} (available from the author upon request). See also \cite[55-57]{hauge15}.} In this way, passive owners need not take on any risk related to the performance of the hydropower company. The second approach has been developed by the company Småkraft AS, which is now the leading market actor specialising in cooperation with local owners.\footnote{It is owned by several large-scale actors on the energy market, see \url{www.smaakraft.no}.} According to their model, riparian owners are paid a share of the annual {\it surplus} from hydropower generation.\footnote{See \cite[57-60]{hauge15} (also discussing variants of this contractual idea, based on how the surplus is actually defined in the contract).}

This share is usually higher than the rent payable based on the net revenue; often, the owners are entitled to $50 \%$ of the profit.\footnote{Source: contracts presented to the court in \cite{sauda09} (available from the author upon request). See also \cite[58]{hauge15}.} Hence, if the project is a success, the riparian owners might be better compensated. However, the owners have to accept some risks as though they were shareholders, and they do so even though they might not have much of a say in how the company is run.\footnote{To limit the risk for owners, companies such as Småkraft AS also operates a system of ``guaranteed'' rent, but this rent is usually quite a lot less than what the owners could expect from an agreement based solely on rent based on gross income. Source: contracts presented to the court in \cite{sauda09} (available from the author upon request).}

To illustrate the financial scale of the rent agreements that have now become standard, let us consider a typical small-scale hydropower plant that produces 10 GWh annually. With an electricity price of NOK 0.3 per KWh, this gives the hydropower plant an annual gross income of NOK 3 million. If the rent payable is 20 \%, the waterfall owners will receive NOK 600 000 annually, approximately GBP 60 000. This is many times more than what the owners could hope to receive according to the traditional method for calculating compensation following expropriation.\footnote{For an example based on comparing two concrete cases, see chapter \ref{chap:5}, section \ref{sec:5:4:1}. See also \cite[283-289]{hauge15}.}

%By contrast, if the rights were expropriated, the traditional method of calculating compensation would be unlikely to result in more than NOK 600 000 as a {\it one time payment} for a waterfall that yields 10 GWh per annum.\footnote{For further details on the compensation issue, see \cite{dyrkolbotn14,dyrkolbotn15,dyrkolbotn15a}. Sometimes, the difference in valuation would be even greater, since the natural horsepower of a development project is highly sensitive to the level of regulation of the waterfall, much more so than the value of the development. For an demonstration of how this affected compensation according to the natural horsepower method, one may consider the case \cite{hellandsfoss97}, which went to the Supreme Court. Here the owners were paid just over NOK 1 million for a waterfall that would yield 100 GWh per annum.}

All in all, the financial consequences of the ideas expressed in {\it Norhordlandsmodellen} have been dramatic. However, the latter two points of the model, addressing the importance of holistic and inclusive management of river systems, have not had the same degree of influence. In the next section, I address what appears to be a negative consequence of this for the small-scale industry, threatening to undermine its status as a sustainable alternative to large-scale exploitation.

\section{The Future of Hydropower}\label{sec:4:6}

In recent years, there has been a growing tension between the small-scale hydropower sector and environmental groups. There is talk of a brewing ``hydropower battle'', as environmentalists grow increasingly critical of what they regard as predatory practices.\footnote{See \cite{haltbrekken12}.}
Reports on small-scale producers who are alleged to have violated environmental regulations help fuel the negative impression of the industry.\footnote{In 2010, the NVE conducted randomised inspections and announced that 4 out of 5 mini and micro plants operated in violation of regulations pertaining to the amount of water they may use at any given time. See \cite{ulovlig10}. In the largest newspaper in Norway, this was reported under the heading that four out of five small-scale plants break the law, see \cite{ulovlig10b}. This is misleading, since mini and micro plants are distinct from small-scale plants proper. Most importantly, the former kinds of plants do not usually require a sector-specific development license. Because of this, it also seems plausible that the reported violations might in large part be due to a lack of knowledge and professionalism, not predation. I remark that questions later emerged regarding the accuracy of the report itself. Apparently, one of the plants that was reported to have violated regulations did not even exist, see \cite{tvilsom10}.}

On the regulatory side, the water authorities have now adopted much stricter procedures to assess licenses for small-scale hydropower.\footnote{See \cite{lie12}.} In addition, different planning routines have been adopted to ensure that small-scale schemes are no longer considered individually, but in so-called ``packages'', collecting together applications from the same area. As a consequence of these changes, the number of rejected small-scale applications have increased dramatically in recent years.\footnote{In 2013, the number of rejections tripled compared to previous years, while the number of accepted applications remained stable. See \cite{sunde14b}.}

At the same time, powerful market actors who favour a traditional mode of exploitation have seized the opportunity to lobby more aggressively against small-scale hydropower, in favour of large-scale projects.\footnote{See, e.g., \cite{alexandersen14}.} Such projects, they argue, are preferable also from an environmental point of view. In recent years, this argument has proven influential in many quarters, particularly among state agencies, such as the NVE and the Norwegian Environmental Agency.\footnote{See \cite{nilsen11}.} It has also been claimed that this perspective is backed up by research done on environmental effects of small-scale and large-scale projects.\footnote{See generally \cite{bakken12,bakken14}.}

The core environmental argument against small-scale solutions has a very simply structure: small-scale plants indirectly affect a greater total area of land per electricity unit produced, therefore they are considered more intrusive than large-scale schemes.\footcite[96-99]{bakken14} The stated premise of this reasoning is no doubt correct; several small-scale plants, at many different locations, are required to match the energy produced by a single larger plant, hence a greater land area will be affected. However, this quantitative observation has no bearing on the issue of how small-scale plants qualitatively affect the surrounding environment, compared to large-scale projects. In particular, the parameters used to compare small-scale and large-scale developments tend to be defined in terms of generic buffer zones that do not take into account differences in the severity of different kinds of environmental intrusions. For instance, as long as both installations are observable by passers by, a small cabin with a turbine inside is considered to have the same ``scenic impact'' as an imposing concrete dam that stretches out for 100 meters and significantly distorts the water level in a lake.\footnote{See \cite[95]{bakken14}.}

\noo{The only buffer zone that is not defined in this way is the {\it scenic} buffer, the area from which some installation can be seen. Here the model takes into account that a large installation should be assessed using a larger buffer zone than a small one, since the former is visible over a greater area. But even for this parameter, no distinction is made based on the actual visual impression; a large dam that dries up a river and makes it possible to regulate the water level in a lake by several meters counts the same as a small cabin with a generator inside, as long as both can be seen.\footnote{See \cite[95]{bakken14}.} For the other parameters, the data analysis is even more dubious, since the buffers are set uniformly according to general rules of thumb.\footnote{See \cite[95]{bakken14}.} For instance, a conflict with a threatened species is assumed to arise whenever a technical installation occurs within a certain distance from its natural habitat.\footnote{See \cite[95]{bakken14}.} Importantly, nothing is said about the severity of conflict, and no distinction is made between a minor installation and a massive disturbance.}

Despite the apparent lack of qualitative arguments, the idea that large-scale development is better for the environment now appears to be gaining ground in Norway. This represents a complete reversal compared to the political narrative that has dominated for the last 15-20 years. Indeed, the merits of small-scale development was strongly emphasised by political leaders around the turn of the century. In his New Year's speech 01 January 2001, the Prime Minister went as far as to declare that the age of large-scale development was over.\footnote{See, e.g., \cite[34]{haltbrekken12}.} The same phrase was then repeated in the policy platforms of two successive national governments, in 2005 and 2009 respectively.\footnote{See the ``Soria Moria'' declaration from 2005, p 57, and ``Soria Moria II'', from 2009, p 52 (available at \url{www.regjeringen.no}).}

However, as administrative practices and case law on hydropower shows, the end of large-scale exploitation has proved impossible to implement. Despite being official policy at the highest level of government for almost 15 years, large-scale development interests continue to dominate in the hydropower sector.\footnote{I believe the material presented in this thesis warrants making this claim. Moreover, it is underscored by the two recent Supreme Court decisions in \cite{jorpeland11} and \cite{otra13}.} Interestingly, the leading national politicians are now changing their position as well.\footnote{See \cite{liemin14} (reporting on recent public statements made by the Minister of Petroleum and Energy in support of large-scale development).} Arguably, this demonstrates how the politicians have responded to pressure from high-ranking bureaucrats and large energy companies.\footnote{In addition to their environmental arguments, these actors also rely on more familiar arguments in favour of large-scale development, especially the idea that large-scale development is more efficient. See \cite{lie12}. For a contrasting view on efficiency, emphasising the efficiency benefits associated with small-scale development and a decentralised approach to hydropower, see \cite[5]{inn101}.}

The political shift observed at present is likely to result in a further weakening of property and the rights of local communities. For example, it provides indirect political legitimacy to the NVE, which pursues an explicit policy of prioritising applications for large-scale projects when these come into conflict with small-scale schemes in the same rivers.\footnote{See \cite[3]{nve12}. See also \cite{lie12}.} Hence, the NVE is likely to refuse to consider applications from owners as long as there are applications pending that might result in the expropriation of their property.\footnote{For a concrete example of this, see \cite{smibelg15}.}

At the same time, the small-scale industry itself has occasionally sought to undermine property rights, 
possibly in an effort to mimic the successes of their large-scale competitors. The industry has argued, in particular, that expropriation should be made more easily available as a tool for small-scale developers and owners who wish to take property from reluctant neighbours.\footnote{See \cite{brekken07,brekken08}. The articles are written by a leading Norwegian energy lawyer, apparently in his capacity as legal representative of ``Småkraftforeningen'', an interest organisation for small-scale hydropower.} This argument rests on a peculiar form of anti-discrimination reasoning; as long as large-scale developers are allowed to take property by force, small-scale developers should be allowed to do the same. In a world where takings are endemic, this might make some sense. However, it is hardly an attitude that helps the small-scale industry preserve its image as the more sustainable hydropower option.

At the same time, the industry is beginning to struggle financially because the price of electricity has been much lower in recent years than what had previously been forecast.\footnote{See \cite{sunde14}.} Moreover, it has become clear that some of the investors on the market have engaged in speculative practices, by aggressively entering into agreements with local owners, without carrying out much hydropower development.\footnote{See \cite{endresen14}.}

These critical remarks should not detract from the fact that the growth in small-scale hydropower has led to dramatically increased benefit sharing with many local owners of rivers and waterfalls. However, recent events indicate that it is inappropriate to look at this development in isolation from other concerns. When assessing the future of small-scale hydropower and local property rights to waterfalls, it seems important to also take into account the broader societal consequences of new commercial practices. If one fails in this regard, the pernicious image of owners as socially passive ``profit-maximisers'' gains a firmer hold both on the political and the legal narrative. The negative consequences this can have for property as an institution are already apparent in Norway, as I will argue in the next chapter. 

More generally, recent developments in the hydropower industry illustrate that an entitlements-based perspective on waterfalls is inappropriate, since local ownership is meant to facilitate sustainable management first, and profit-seeking only second. This insight is also strongly implicit in {\it Nordhordlandsmodellen}. However, as the current debate is evolving, it seems to be at risk of disappearing from view.

\noo{To counter this, I believe the social function view of property must be developed further, so that concrete policy recommendations can be formulated on its basis. The aim, I believe, should be to arrive at frameworks for participatory decision-making regarding hydropower that allows local owners and communities to contribute constructively when society desires commercial development based on  their water rights.

I return to this issue in chapter \ref{chap:6}, where I argue that the Norwegian institution of land consolidation can be used to achieve this. First, I will zoom in on the issue of expropriation, where the mechanisms identified in this section often lead to concrete legal disputes. This will bring into focus important issues surrounding the status of economic development takings under Norwegian law.
}

\section{Conclusion}\label{sec:4:7}

Water resources have been, and still are, very important to Norway as a nation. Not only does the energy of streaming water provide electricity to people and industries, it also provides a source of profit, prestige and power to those who harness it. Historically, many rural communities in Norway benefited greatly from this, as it was they who managed local water resources.

Plainly, they did rather well. By the end of 1943, at a time when small-scale plants still outnumbered large-scale plants 45 to 1, around $80 \%$ of the population had access to electricity at home. The government, especially local governments, also felt responsible for the supply of electricity to the public, but they generally assumed this responsibility without encroaching on local populations who wished to manage their own resources.

As discussed in this chapter, the situation changed dramatically after the Second World War, when the government, especially the central government, assumed more direct control over the nation's water resources. This led to a situation where local owners became increasingly marginalised. In recent years, there has been a partial reversal of this trend, as the liberalisation of the electricity market has enabled local owners and communities to take part in hydropower development once again.

The result has been a growing tension between large-scale and small-scale development, which in turn corresponds to a tension between the owners of waterfalls and the energy companies that wish to expropriate. In the next chapter I will explore this tension in more depth, as I investigate the rules and practices relating to expropriation for hydropower development.
\chapter{Taking Waterfalls}\label{chap:5}

\section{Introduction}\label{sec:5:1}

The Norwegian water authorities have extensive powers to take waterfalls for hydropower development. However, they rarely need to reflect on this power, not even when they use it. The reason is that expropriation tends to be an {\it automatic} consequence of a development license; those who obtain a license to develop a large-scale hydropower plant almost always obtain also a license to expropriate all private property rights required for this purpose.

In some cases, this follows from section 16 of the \cite{wra17}, which gives license holders a right to expropriate all property rights needed for the development in question.\footnote{As mentioned briefly in the previous chapter.} However, even outside the scope of these provisions, the same approach to expropriation tends to be adopted. Specifically, the authorities adhere to the presumption that whenever a license to undertake large-scale development should be granted, then so should a license to expropriate.\footnote{The leader of the hydropower licensing division of the NVE expressed this presumption in  \cite{flatby08} (noting also that the same presumption is not applied for small-scale projects, e.g., when some owners wish to expropriate from neighbours who oppose development).}

The expropriation presumption has remained in place even though the regulatory and economic context of riparian expropriation has changed dramatically after the liberalisation of the electricity sector.
This is significant, especially due to how licensing cases are processed. As discussed in the previous chapter, the administrative licensing assessment tends to focus on the environmental consequences of development, with little attention devoted to how the loss of property rights affects the owners and their local communities. This is so even though a license to develop is in effect also a license to expropriate.

How did this system come about, and where does it leave local owners whose waterfalls are targeted by large-scale proposals? This chapter addresses these two questions in depth. 

First, the history of the law is presented. This will serve to demonstrate that the current state of affairs is in fact a complete reversal compared to the legal framework that was in place before the advent of state-initiated industrial hydropower development. Further to this, the chapter discusses more recent changes, specifically the changes in the expropriation regime that were implemented following liberalisation of the electricity sector. 

To make expropriation available as a tool for commercial companies, the earlier rules had to be modified. Specifically, public interest requirements had to be relaxed and limitations on private-to-private transfers had to be abrogated. The manner in which this was achieved, with only minimal parliamentary involvement, is in itself worth noting when addressing the legitimacy of current practices.

After the historical assessment, the chapter illustrates how the water authorities and the courts interpret and apply the rules currently in place. Specifically, I give a detailed presentation of the recent Supreme Court case of {\it Jørpeland}.\footnote{See \cite{jorpeland11} (I mention that I acted as legal counsel for the owners in this case).} Plainly, {\it Jørpeland} demonstrates that the standing of owners is very weak under administrative law. Arguably, this is a result of a tradition whereby the expropriation issue is overshadowed by the licensing question.

More generally, {\it Jørpeland} and other recent cases suggest that the Supreme Court adhere to a very narrow perspective on the meaning of property protection, taking it to be an issue that begins and ends with the question of compensation. In this regard, owners were initially able to make some progress towards a more equitably level of compensation, but as this chapter shows, the early progress made on this point is likely to be reversed following the Supreme Court decision in the case of {\it Otra II}.\footnote{See \cite{otra13}.}

I conclude this chapter with a more overarching assessment based on the theoretical framework presented in Part I of the thesis, to shed further light on the legitimacy of rules and practices surrounding takings of waterfalls in Norway. I argue that the current system is likely to systematically result in takings that fail the Gray test presented in Chapter \ref{chap:2}. This sets the stage for the final chapter, where I consider land consolidation as a legitimacy-enhancing alternative to expropriation for hydropower development.

\section{Norwegian Expropriation Law: A Brief Overview}\label{sec:5:2}

As mentioned in Chapter \ref{chap:2}, the right to property is entrenched in section 105 of the Norwegian Constitution. There it is made clear that when property is taken for public use, full compensation is to be paid to the owner. The formulation bears a striking resemblance to the formulation of the US takings clause in the fifth amendment. However, there is no active public use debate in Norway. The meaning of public use is hardly ever discussed by the courts, and according to legal scholars, the public use formulation places no limit at all on the state's authority to expropriate.\footnote{See \cite[249]{aall04}. For a comment to more or less the same effect, made by the court of appeal, see \cite{sauda09}.}

However, it is a rule of unwritten constitutional law that administrative decisions which affect the rights of individuals can only be carried out when they are positively authorised by law.\footnote{See generally \cite{hogberg11}.} Moreover, the Constitution is not understood as providing an authority for the state to expropriate. It merely expresses the presupposition that expropriation is possible.\footnote{See, e.g., \cite[6]{fleischer86}.} Hence, when applying eminent domain, the government needs to justify this on the basis of a more specific authorising provision. 

Historically, there was no general act relating to expropriation, and a range of different acts provided the necessary authority to expropriate for specific purposes such as roads, public buildings, and schools.\footnote{See \cite[11-12]{nut54}.} Today, many of these authorities have been broadened and included in the \cite{ea59}.\footnote{Act no 3 of 23 October 1959 Relating to Expropriation of Real Property.} After an amendment in 2001, this act includes an authority for the government to authorise expropriation of property and usufructs in order to facilitate ``hydropower production''.\footcite[2 no 51]{ea59} This is understood to include the authority to expropriate waterfalls.\footnote{See, e.g., 
\cite{sauda08}.}

According to the \cite{ea59}, expropriation can only be authorised if the benefits undoubtedly outweigh the harms, as determined by a discretionary assessment typically conducted by the water directorate.\footnote{See \cite[2]{ea59}.} Formally, the authorising authority is the King in Council. However, this authority can be delegated to ministries or other state bodies that the King in Council can instruct.\footnote{See \cite[5]{ea59}.} The compensation to the owner is determined following a judicial procedure administered by the so-called appraisal courts.\footnote{\cite[2]{ea59}.} This is the name given to the regular civil courts when they hear appraisal cases, observing the special procedure set out in the \cite{aa17}. The appraisal procedure emphasises the importance of factual assessment and lay discretion (the appraisal court typically sits with four lay judges).\footnote{See \cite[11-12]{aa17}.} In addition, there are special rules regarding costs, indicating that the expropriating party is usually required to pay for the procedure, include the owners' legal expenses.\footnote{See \cite[54]{aa17}.} In other regards, the appraisal procedure resembles a typical adversarial process before a civil court.\footnote{See generally \cite{dyrkolbotn15}.} 

The \cite{ea59} states that unless the Kind in Council decides otherwise, expropriation orders may only be granted to state or municipality bodies. This is formulated as a limiting principle, but in effect it serves as a general authorisation for the executive to decide, without parliamentary involvement, that a larger class of legal persons may be granted expropriation licenses. 

For many purposes, directives have been issued that extend the class of possible beneficiaries to any legal person, including companies operating for profit. In 2001, such a directive was issued for the authority to expropriate in favour of hydropower production.\footnote{See Directive no 391 of 06 April 2001.} 

In addition to providing a general authority for expropriation, the \cite{ea59} also contains several procedural rules. These are collected in Chapter 3 of the Act. Here the Act sets out minimal requirements for what an application for an expropriation license must include: it should make clear who will be affected, how the property is to be used, and what the purpose of acquisition is.\footnote{See \cite[11]{ea59}.} In addition, the Act requires the applicant to specify exactly what property they require, and to include information about the type of property in question and the current use that is made of it.

The owners must be notified, and the starting point is that every owner should be given individual notice, although this obligation is relaxed when it is ``unreasonable difficult'' to fulfil\footnote{See \cite[12]{ea59}, para 2.} In such cases, it is sufficient that the documents of the case are made available at a suitable place in the local area. In addition, a public announcement must then  be made in the official notification publication of the government, as well as in two widely read local newspapers.\footnote{See \cite[12]{ea59}.}

The licensing authority is required to ensure that the facts of the case are clarified to the ``greatest extent possible''.\footnote{The Norwegian expression is ``best råd er'', which literally means ``best possible way''. See \cite[12]{ea59}, para 2.} This formulation seems very strict, but is also highly non-specific. In practice, the level of scrutiny given to the expropriation question under Norwegian law varies greatly depending on the relevant sector-specific administrative practice.\footnote{See \cite[380-381]{dyrkolbotn15}.} Moreover, established practice from several fields, including the hydropower sector, suggests that when expropriation takes place to implement a public plan or a licensed development, little attention is devoted to expropriation as a special issue.\footnote{For zoning plans, see \cite{namsos98,bo99}. For hydropower, see \cite{jorpeland11}.}

The applicant must cover costs incurred by owners in relation to a pending application for expropriation.\footnote{See \cite[15]{ea59}.} The exact formulation is that the applicant is obliged to cover the costs that ``the rules in this chapter carry with them''. That is, the applicant is obliged to cover the costs that are related to the owners' rights pursuant to Chapter 3 of the \cite{ea59}. In practice, an owner will be denied costs if the competent authority takes the view that they are unreasonable or disproportionate to their interests in the case.\footnote{If the case progresses to an appraisement dispute, the competent authority to decide on costs is the appraisement court. Otherwise, the decision is left with the executive. See \cite[15]{ea59}.} Finally, the decision to grant an expropriation license must be justified, and the parties should be informed of the reasons for the decision.\footnote{See \cite[12]{ea59}, para 3.}

In addition to the procedural rules in the \cite{ea59}, the rules of the \cite{paa67} also apply in expropriation cases. These rules largely stipulate the same requirements as those discussed above, so I omit a detailed presentation. However, I mention that it has been controversial whether or not these rules provide any basis at all for scrutiny of established practices adopted by the water authorities. Specifically, it has been argued that the licensing procedures spelled out in the \cite{wra00} and the \cite{wra17} are exhaustive in hydropower cases.\footnote{See \cite{jorpeland11a}.} In the case of {\it Jørpeland}, the Supreme Court held that general rules of administrative law did apply in theory, but went quite far in suggesting that they would have limited significance in practice, as sector-specific rules and practices would take priority.\footnote{See \cite{jorpeland11}.}

\section{Taking Waterfalls by Obtaining a Development License}\label{sec:5:3}

As I mentioned in Chapter \ref{chap:3}, Section \ref{sec:wra17}, the \cite{wra17} establishes an automatic right to expropriate rights needed to implement a licensed watercourse regulation. This does not include a right to expropriate rivers and waterfalls needed for the hydropower development. However, it includes a right to transfer water away from a river for development somewhere else.

This has the {\it de facto} effect of a waterfall expropriation, since the water as such is taken by the expropriating party for use somewhere else. Moreover, it has always been treated as waterfall expropriation in relation to the compensation issue.\footnote{See \cite{jorpeland11}.} Formally, however, the interference is not considered an expropriation of real property, but rather seen as an expropriation of a right to deprive rivers of water, a sort of easement whereby the developer acquires the right to interfere with the rights of riparian owners in source rivers.

In theory, the rules in the \cite{ea59} and the \cite{paa67} still apply when the right to expropriate follows automatically from a development license. Indeed, the rules in the \cite{paa67} express general principles of administrative law, pertaining to all kinds of individual decisions, including both expropriation and licensing decisions. The \cite{ea59}, for its part, explicitly states that it applies to property interferences authorised under the \cite{wra17}.\footnote{See \cite[30]{ea59}.} However, it is also stated that the rules in the \cite{ea59} only apply in so far as they are ``suitable'' and do not ``contradict'' sector-specific rules.\footcite[30]{ea59} This points to the potential caveat that while a range of procedural rules apply in theory, there is a risk that they will be ignored in practice, if they are deemed unsuitable by the authorities.

This is practically significant in hydropower case. Specifically, the established practice among the water authorities is to regard the procedural rules in the \cite{wra17} as exhaustive.\footnote{This was made clear through the case of \cite{jorpeland11}, where this practice also got a stamp of approval from the Supreme Court.} In addition, the material assessment requirement in the \cite{ea59} is not considered to have any independent significance alongside the assessment criterion in the \cite{wra17}.\footnote{Again, see \cite{jorpeland11}.} This is so even though case law on the former assessment criterion emphasises the interests of affected property owners in a way that case law and administrative practice on the licensing issue does not.\footnote{In addition, the formulation in \cite[2]{ea59} contains the additional qualification that the benefit of interference must ``undoubtedly'' outweigh the harm, meaning that this clearly must be the case (pertaining to the evidence, not the weight of the benefit compared to the harm), see \cite{lovenskiold09}. No corresponding requirement is included in the \cite[8]{wra17}. Instead, the formulation there is that a license should ``normally'' not be given unless the benefits outweigh the harms. See also \cite[325-236]{haagensen02} (arguing that the ``normally'' qualification is without practical significance).}

As a consequence of how the law is understood on this point, it is very hard for owners to challenge the legality of a decision to allow expropriation of their riparian rights, especially when expropriation takes place pursuant to the \cite{wra17}.\footnote{It follows from the discussion in Chapter \ref{chap:4} that large-scale development projects almost always involve a license pursuant to the \cite{wra17} (or such that the rules from this act, including section 16 on expropriation apply pursuant to the \cite{wra00}).} Moreover, even if section 16 of the \cite{wra17} does not apply, the water authorities tend to approach the affected owners in a similar way. In particular, the practices observed with regard to the issue of property interference is largely the same in all cases when the administrative branch classifies the license application as pertaining to a large-scale project.\footnote{See \cite{flatby08}.}

For such projects, the water authorities rely on the presumption that an expropriation license should be granted whenever a development license is granted.\footnote{See \cite{flatby08}.} Hence, in order to defend themselves, owners must proceed in a roundabout manner by addressing the licensing question as such. In practice, there is little or no room for arguing on the basis of rules that protect private property. Moreover, in order to argue that the expropriation is unlawful on procedural grounds, the owners must effectively demonstrate that the water authorities dealt with the case in contravention of sector-specific rules and practices pertaining to the licensing question.

This is a daunting task, particularly in light of how the relevant case law developed during the period of monopoly regulation of the electricity sector. In practice, the courts will largely defer to the administrative branch, also when it comes to interpreting the relevant procedural rules.\footnote{This  deferential stance was articulated in the {\it Alta} case discussed in Section \ref{sec:twp} below.} As a consequence, procedural objections pertaining only or primarily to the expropriation decision are highly unlikely to succeed.\footnote{I am not aware of any case where such an argument has succeeded. In Section \ref{sec:jorpeland}, I will further demonstrate the present situation by tracking in detail the extent to which the Supreme Court is prepared to tolerate procedural shortcomings pertaining to the expropriation issue in hydropower cases.}

To shed further light on why this is so, I will now give a chronological presentation of how the law on expropriation of waterfalls has developed as part of the legal framework for management of hydropower. As will become clear, the current situation was by no means inevitable, but rather the result of a series of reforms that gradually undermined property as an anchor for active local community participation in hydropower development.

\section{Taking Waterfalls for Progress}\label{sec:4}

Historically, Norwegian law did not contain a general authority for expropriation of riparian rights.\footnote{See \cite[29]{amundsen28}.} In the \cite{wra88}, a range of provisions authorised appropriation of water rights and land for specific purposes, but the criteria were narrow.\footnote{See \cite[69-85]{dahl88}. In addition, the purpose of expropriation was largely understood to be binding also on future use, so that the taker would not gain unrestricted control over the rights they acquired. Rather, they were obliged to use these rights to pursue the specific public purpose for which expropriation was authorised. See, e.g., \cite[133-140]{rygh12}.} Rivers and waterfalls as such could never be made subject to expropriation, and expropriation of other water rights could only be permitted in so far as the affected owners were not thereby deprived of any water power that they could reasonably make use of themselves.\footnote{See \cite[58|60]{dahl88}.}

Specifically, expropriation for hydropower development was not permitted, except to the benefit of riparian owners who needed to acquire surrounding land in order to exploit their existing water rights.\footnote{See the \cite[15-16]{wra88}. See also the commentary in \cite[60-65]{dahl88}.} Moreover, riparian owners could apply for licenses to engage in various industrial exploits, in some cases also when this would prove damaging to other landowners, for instance through deprivation of water or flooding.\footnote{See \cite[14]{wra88}. See also the commentary in \cite[54-60]{dahl88}.} These rules are similar to many of the rules found in contemporaneous mill acts from the US, discussed in Chapter \ref{chap:2}. As in the US, these rules could be classified as giving rise to economic development takings. However, the source of the economic development potential as such was not supposed to be taken from the owners under these rules.\footnote{See \cite[168-170]{dahl88}.} Rather, takings were only warranted with respect to additional rights that existing owners needed to realise the full potential of their own resources.

In fact, an important principle of Norwegian expropriation law at this time was that no property could be taken if the taker's interest in that property was part of the current owner's bundle of rights.\footnote{See \cite[168-170]{dahl88}.} This applied regardless of whether or not the owners, subjectively speaking, were likely to pursue the interest in question in an optimal way. On the basis of this principle, expropriation of waterfalls for hydropower development was not permissible. The reason was simple: the right to develop hydropower was considered part of the owners' bundle of rights. 
Hence, it could not be taken from them, as a matter of principle. 

By contrast, if ancillary land was needed by someone wishing to make optimal use of {\it their} waterfall rights, expropriation was possible. In these cases, the takers did not seek to take the owners' rights as much as to negate them, in order to fully enjoy their own. More generally, expropriation at this time was considered a way to resolve conflicts between rights, not a way to redistribute them.\footnote{See \cite[168-170]{dahl88}.}

Following industrial advances, the interest in hydropower exploded in the late 19th century.\footnote{See \cite[58-59]{falkanger87}.} As a result, the state increasingly came to see it as a political priority to regulate the hydropower sector, especially to prevent foreign speculators and industrialists from acquiring ownership of Norwegian resources.\footnote{See \cite[58-59]{falkanger87}.} As discussed in Chapter \ref{chap:3}, the most important expressions of this came in the form of two new licensing acts, namely the \cite{wra17} (Section \ref{sec:wra17} and the \cite{ica17} (Section \ref{sec:ica17}).

Following up on this, parliament soon passed legislation that authorised expropriation of riparian rights for the benefit of public bodies, also when the purpose was hydropower development.\footnote{Legislation that made it possible to expropriate waterfalls to the benefit of the municipalities was introduced in 1911, and a similar authority that authorised expropriation in favour of the state appeared in 1917, see \cite[29]{amundsen28}.} In 1940, these authorities were consolidated and integrated in the general water resources legislation, through the \cite{wra40}.\footnote{This act has since largely been replaced by the \cite{wra00}.} According to this act, the authority to expropriate waterfalls could be granted only to the state and the municipalities. Moreover, the municipalities could only expropriate waterfalls when the purpose was to provide electricity to the local district.\footnote{See the \cite[148]{wra40}. See also the commentary in \cite[201-210]{sorensen41}.} Private parties could not expropriate except in exceptional circumstances, when they already owned more than 50 \% of the riparian rights they sought to exploit.\footnote{See the \cite[55]{wra40}. See also the commentary in \cite[70-74]{sorensen41}. I remark that this was a novel rule in the 1940 Act, which contradicted earlier theories about the legitimacy of allowing expropriation for private benefit.} 

In all cases of waterfall expropriation, it was felt that benefit sharing with local owners was required. Hence, special rules were introduced to ensure that takers would have to pay {\it more} than full compensation (typically a 25 \% premium, but in some cases the owner was also given a right to opt for compensation in the form of a proportion of the electricity output of the plant).\footnote{See \cite[70-91,184,210]{sorensen41}.}

As I showed in Chapter \ref{chap:4}, the electricity supply in Norway just after the passage of the \cite{wra40} was already well developed, with 80 \% of the population having access to electricity. Moreover, in the rural areas the supply often came from one among a vast number of small, local, power plants. In light of the progress already made and the highly decentralised structure of the hydroelectric sector at this time, one might have expected expropriation to remain a relatively rare occurrence.

However, the prevalence of expropriation to facilitate hydropower development increased greatly after the war, as the state itself became engaged much more actively with hydropower development, also for commercially oriented industrial purposes.\footnote{See \cite[59-71]{thue96}. See also \cite{skjold06}.} Hence, despite the spirit and wording of the \cite{wra40}, this was the time when expropriation of rivers and waterfalls became a measure to facilitate economic development. Moreover, this seems to have had little do with the supply of electricity to the people. Rather, it appears to have resulted from increased political demand for hydropower to support the metallurgical industry, combined with the fact that the hydropower sector was reorganised and brought under increasingly centralised political control.\footnote{See \cite[69-71]{thue96}.}

Following this, a growing share of the financial benefits from development would accrue to urban areas, as local development companies were replaced by state companies and companies dominated by prosperous city municipalities.\footnote{In 2007, as the result of a gradual centralisation process, the 15 largest hydropower companies in Norway, which are largely controlled by the state and some city municipalities, owned roughly 80\% of Norwegian hydropower, measured in terms of annual output. In 2006, the public owners of hydropower in Norway benefited from receiving more than NOK 9 billion in dividends. See \cite[28]{otprp61}.} In addition, a highly idiosyncratic compensation method was adopted in expropriation cases, ensuring that waterfalls could be purchased very cheaply from the original owners.

\subsection{The Natural Horsepower Method}\label{sec:5:4:1}

In Section \ref{sec:wra17}, I presented the notion of a natural horsepower, used to determine when a development project requires development licenses pursuant to the \cite{wra17} and the \cite{ica17}. As mentioned, the natural horsepower of a development scheme is a gross measure of the stable electric effect harnessed following the development. Specifically, it measures the electric output available for at least 350 days each year, a figure that is sensitive to fluctuations in the supply of water.\footnote{See \cite{sofienlund07}.} 

For this reason, the number of natural horsepower in a development project says little about the total amount of energy that the development harnesses in a year. As a result it also says very little about the value of the development project, as energy producers today get paid for all the energy they produce, not just that which they can guarantee in advance.\footnote{See \cite[83-84]{uleberg08}.} Prior to the establishment of a national grid, this was different. Without a grid, fluctuations in electric output would not be evened out by supply from other parts of the country, so the importance of maintaining a stable supply was much greater. Indeed, energy producers would often get paid based on the amount of electric effect they could deliver stably over the year, not the total amount of energy harnessed.\footnote{See \cite[83]{uleberg08}.}

Hence, early in the 20th century, the notion of natural horsepower gave a good indication of the value of a development project. Therefore, it was also a good measure of how much a developer would be willing to pay for access to riparian rights.\footnote{See \cite[83]{uleberg08}.} Indeed, it was used by the market for waterfalls that existed prior to state regulation. The price of a waterfall, specifically, was typically calculated on the basis of the price that the developer was willing to pay per natural horsepower that the planned development would yield.\footnote{See \cite[83]{uleberg08}.} The total payment offered to the owners, consequently, would be found by multiplying the natural horsepower of the development with the price offered per natural horsepower.

This method was also adopted by appraisal courts to fix the level of compensation following expropriation.\footnote{\cite[521]{vislie02}.} Moreover, when the notion of natural horsepower fell into disuse among energy producers, because it no longer reflected the actual value of development projects, the courts did not modify their compensation practices. They stuck with the natural horsepower method, which was now applied on a customary basis, not as a way of calculating realistic economic values.\footnote{See, e.g., \cite[1599]{hellandsfoss99} (the Supreme Court comments that the method is used customarily because the market provides ``little guidance'').}

Over time, the price level became more and more unrealistic as a measure of the value of waterfalls as a natural resource. After the liberalisation of the hydropower sector in the early 1990s, the discrepancy in this regard became particularly extreme. For instance, in 1999, the appraisal court of appeal awarded a one time payment of NOK 722 068 in compensation for a waterfall that yields 152 GWh per annum.\footnote{See \cite{hellandsfoss99}.} By comparison, in the case of {\it Sauda} from 2009, where a market-based valuation method was used, the owners of the {\it Maldal} river were awarded NOK 1 149 044 in compensation as a {\it yearly payment} for a waterfall that would yield 36.5 GWh per annum.\footnote{See \cite{sauda09}.} If we assume an interest rate of 4 \%, this corresponds to a one time payment of NOK 28 726 100. This, in turn, corresponds to NOK 787 017 per 1 GWh produced annually. That is, the owners of {\it Maldal} were paid in the excess of 150 times more for their waterfall than the owners of {\it Hellandsfoss}.

The mismatch between economic values and compensation payments had been noted long before the liberalisation of the electricity sector. In fact, the existence of a major discrepancy had been noted as early as in the 1950s, by the head of the water directorate himself. In an article published in an internal newsletter in 1956, the director commented that the natural horsepower method did not result in compensation payments that reflected the true economic value of waterfalls as a natural resource.\footnote{See \cite{rogstad56}.} Moreover, he speculated that the method could be sustained only through exploiting the lack of knowledge about hydropower development among owners.\footnote{See \cite{rogstad56}.}

One might think that the continued use of the natural horsepower method, in a situation when the water authorities themselves were aware of its shortcomings, would result in controversy. However, at this time, the local owners of waterfalls did not attack the method in court. Active resistance on this point would not be seen until much later, after the liberalisation of the electricity sector, as discussed in Sections \ref{sec:uleberg} and \ref{sec:recent} below. However, conflicts arose with respect to other aspects of the regulatory framework regarding hydropower, as discussed in the following section.

\subsection{Increased Scale of Development and Increased Tension}\label{sec:5:4:2}

As discussed in the previous chapter, the state pursued increasingly complex hydropower projects after the Second World War. At this time, technological and economic advances also made it more feasible to divert water over great distances (typically through tunnels), to collect several different rivers in a common reservoir for joint exploitation. Such projects became known as ``gutter'' projects, and they grew greatly in scope during the post-War years. Since the relevant licensing procedure was covered by the \cite{wra17}, the practical importance of the expropriation authority in section 16 of this act also increased dramatically.\footnote{See \cite[11]{innst59}. This was a proposition to parliament regarding an amendment of the \cite{wra17}. The amendment proposed to remove an earlier rule that applied only to diversion regulations, whereby a license to divert water from a river should {\it normally} only be granted when the riparian owners in the source river agreed to the measure. This rule made licenses harder to obtain in the diversion cases. However, following the department's recommendation, the rule was removed in 1959. The department argued that the rule had an ``unfortunate effect'' on the administrative procedure in large-scale diversion cases, noting also the vastly increasing complexity and scale of typical diversion regulations. The minority in the parliamentary committee recommended against the amendment, noting that it would ``greatly increase'' the authority to expropriate waterfalls, contrasting with the expropriation rules in the \cite{wra40}, see \cite[14]{innst59}. The majority countered this argument by maintaining that the regulatory power of the state would be used to prevent any abuse of power, and that the practical significance of the amendment would be limited to ensuring a ``more rational'' procedural approach to large-scale applications, see \cite[14]{innst59}.}

As mentioned in the previous chapter, the opposition to hydropower grew proportionally to the scale and complexity of typical development projects.\footnote{See generally \cite[64-65]{nilsen08}.} The critical focus was often on environmental effects, but the interests of local people also featured in these debates. Moreover, local interest were often aligned with the environmental interests.\footnote{See \cite[72-73]{nilsen08}.}

In the first cases that reached the Supreme Court from this era, the question of legitimacy was not raised in full breadth. Instead, the early cases concerned specific legal points, such as the issue of whether (informal) agreements and understandings between owners, municipalities and the central government were binding on future decision-making processes regarding development.\footnote{See \cite{aura61,mardola73}.} In addition, the question arose as to what extent additional compensation should be paid for `damages' and `inconveniences' caused by large-scale development, in addition to the compensation calculated using the natural horsepower method. Finally, questions arose over the status of non-waterfall owners who owned land that was still crucial to the development, for instance because it would be flooded or used to construct regulation installations. Should such owners receive compensation based on the value of their rights for development purposes, or should they only receive compensation based on current property uses, as before?

The Supreme Court consistently rejected the claims of owners and local communities. First, it was held that landowners who did not own waterfalls were now entitled to compensation based on the value of their rights as an asset for hydropower development.\footnote{See \cite[332-333]{tokke63}.} Second, it was held that when compensation was awarded to waterfall owners according to the natural horsepower method, then this would preclude additional compensation for harms and nuisances associated with large-scale watercourse regulation.\footnote{See  \cite{vikfalli71,driva82}.}

In the case of {\it Aura}, the owners argued that they had originally agreed to sell their water rights to a private developer, on the understanding that a specific development project would take place, not involving diversion of water.\footnote{See \cite[1284]{aura61} (the original transaction took place in 1906-1910, when there was still a market for sale of waterfalls to private speculators and developers).} Hence, the owners thought that the purchaser of their water rights had not acquired a right to divert the water away from the river. Still, when the government later acquired the water rights in question, they decided to embark on a more intrusive project that {\it would} involve water diversion. For this reason, the owners argued that they were entitled to additional compensation. 

The claim was rejected by the Supreme Court, which held that insufficient evidence had been provided to establish that the sale of the water rights was made conditional on a specific type of development.\footnote{See \cite[1285-1286]{aura61}.} Moreover, it was held -- on the basis of the facts -- that the sale of the water rights had {\it not} been restricted to only cover the waterfall (i.e., the right to harness hydropower from the river in question). According to the Supreme Court, the fact that the rights in question had been referred to as ``water rights'' meant that the right to divert away the water was also included.\footnote{See \cite[1284-1285]{aura61}.}

In the later case of {\it Mardøla}, the situation was similar, with the crucial difference being that these local owners had not sold ``water rights''; their contract explicitly stated that what had been sold was the waterfalls.\footnote{See \cite[112]{mardola73} (the voluntary sale dated back to the early 20th century, when the market for waterfall was still unregulated).} However, the government interpreted this to mean that they had a right to divert water away from the river, without paying any additional compensation.\footnote{See \cite[112]{mardola73}.} This contradicted the premise of {\it Aura}, where the decision to allow a diversion was premised on the fact that {\it not only} the waterfalls had been acquired by the developer. Still, in {\it Mardøla}, the Supreme Court cites {\it Aura} as the primary authority in favour of a {\it general rule} by which the sale of a ``waterfall'' also includes the right to divert water away from the river.\footnote{See \cite[112]{mardola73}.} No explanation is provided by the Court to reconcile this with what was actually said in {\it Aura}.\footnote{Arguably, the Court's finding on this point has since been overruled by \cite{jorpeland11}. Here a waterfall right was defined explicitly as a right to exploit the hydropower in a river along its present trajectory. This definition was provided in order to avoid the conclusion that a diversion of water by someone other than the waterfall owner (in the source river) amounts to a waterfall expropriation. It bears noting that if the Court had concluded in keeping with the precedent set by {\it Mardøla}, by holding that the diversion right is part of the waterfall bundle, it would have shed serious doubt on the legitimacy of the established practice of allowing diversions under section 16 of the \cite{wra17}, with no prior acquisition of the waterfall rights in the source river.}

In {\it Mardøla}, both the owners and the local municipality had explicitly agreed to support the central government on the understanding that a specific development plan would be adopted. Later, this plan was abandoned in favour of a project that was deemed by some local owners to be both less beneficial and more intrusive. Hence, both the owners and the municipality argued that the resulting development license was invalid. The Supreme Court conceded that prior statements made by the water authorities had been striking, serving to create a clear expectation among the locals for a specific development plan.\footnote{See \cite[111]{mardola73}.}

However, the Court chose to rely on what it described as a ``general presumption'' against the position that the central government is bound in its decision-making by prior statements.\footnote{See \cite[110]{mardola73}.} According to the Supreme Court, the statements made by the water directorate in {\it Mardøla} were not clearly endorsed by the Ministry and the Parliament, and could therefore not be regarded as binding on the final decision.\footnote{See \cite[111]{mardola73}.} 

%This finding seems reasonable enough. However, one rather crucial question was not addressed at all: is it legitimate procedure for the water authorities to make ``striking statements'' of the kind offered in {\it Mardøla}, when this serves to silence opposition and induce support for development during the assessment stages? The Supreme Court apparently did not wish to consider this question, raising the possibility that it felt the answer would not have been to its liking.

The case of {\it Mardøla} illustrates the increasing tension that arose regarding hydropower in the 1970s, as well as a tendency on part of the Supreme Court to side with large-scale development interests. Indeed, the development in {\it Mardøla} stirred up a high level of controversy that also resulted in civil disobedience and criminal prosecution of environmental activists.\footnote{See \cite{mar71}.} The case also illustrates how the central government attempted to minimise tensions by entering into dialogue with local authorities and owners. Crucially, this dialogue was not premised on a legal framework that ensured local participation, and, despite appearances, did not necessarily result in any new entitlements for local people.\footnote{However, the increased tension during this time did sometimes lead to additional benefits being bestowed on local power groups. These new measures were typically relatively minor, and they were directed primarily at municipalities and regional government bodies rather than local owners. See generally \cite[75-76]{nilsen08}.}

\noo{ In effect, the {\it Mardøla} Court sanctioned an approach whereby the water authorities could limit local opposition by expressing commitments early on, which could then simply be ignored at a later stage of the decision-making process. At this later stage, it would be too late for the local population to launch an effective opposition, e.g., it would be too late for them to aligning themselves with environmental activists. More generally,}

Moreover, despite occasional concessions being made to local and regional government institutions, controversies continued to arise. The culmination came with the case of {\it Alta}, where the question of procedural legitimacy was raised in full breadth. To this day, the {\it Alta} case remains the most important Supreme Court precedent in the area of hydropower law.

\subsection{The {\it Alta} Controversy}\label{sec:5:4:3}

The {\it Alta} case went before the Supreme Court in 1982 after a long period of high-intensity conflict going back to the mid-seventies.\footnote{See \cite{alta82}. For commentaries, see \cite{eckhoff82,boe83,hagvar88}.} In {\it Alta}, the affected local population largely lacked formal title to the property they sought to defend. This was because the development in question would take place in the northernmost part of Norway, in the native land of the Sami people.\footnote{For Sami law generally, see \cite{skogvang02}.}

Norway has a history of discrimination against the Sami, and as their culture is largely nomadic, their land rights were never formalised in private law.\footnote{See \cite[149-156]{ravna12s}} As a result, land and natural resources in the county of Finnmark are largely owned by the state, at least in the sense of the state appearing as the nominal {\it in rem} owner.\footnote{In the past 30 years, partly as a response to the controversy of the {\it Alta} case, there has been a gradual change in attitude, whereby the rights of the Sami people receives greater legal recognition. In 2007, formal title to most of the land in the county of Finnark was transferred to a special state agency which is regulated by a special statute that obliges it to manage the land with due regard to customary and prescriptive rights of aboriginal groups and local people. See generally \cite{bull07}.}

Due to the sensitive context of interference, the {\it Alta} plans met with particularly strong criticism from local people, as well as environmental groups and groups fighting for aboriginal rights. A broad political movement was mobilised in opposition to the plans, eventually resulting in several serious cases of civil disobedience.\footnote{This included hunger strikes and attempts at sabotage, see \cite[80-83]{nilsen08}. For the Alta controversy generally, see \cite{altawiki,hjorthol06}.} The case also came before the courts, as the local population and environmental groups claimed, primarily on the basis of administrative law, that the development licenses that had been granted were invalid.\footnote{See \cite{eckhoff82}.}

The {\it Alta} case did not involve expropriation of the right to harness hydropower. However, because of the priority given to the licensing procedure over specific expropriation procedures, the principles expressed in {\it Alta} also largely determine the legal position of waterfall owners whose rights to hydropower are expropriated.\footnote{See \cite{sauda09,jorpeland11}.} Moreover, the case involved expropriation of other property rights as well as special usufructuary rights held by the Sami people.

{\it Alta} was admitted to the Supreme Court in plenum, directly on appeal from the district court.\footnote{This is a special arrangement available in cases that raise important questions of principle, see \cite[30-2]{cda05} and \cite[5]{ca15}.} The presiding judge commented that as far as he knew, it was the longest and most extensive civil case that the Court had ever heard.\footcite[254]{alta82} In an opinion totalling 138 pages, the Court considers a long range of objections against the development licenses, all of which are either rejected or held to provide insufficient reasons to declare the licenses invalid.

\noo{The opponents of the {\it Alta} development also argued on the basis of human rights and international law.\footnote{First, on the basis of articles 1 and 27 of the \cite{fnp}. Second, on the basis of \cite{ilo107} (later replaced by \cite{ilo169}). Third, on the basis of P1(1) of the \cite{echr}.} As noted by Eckhoff, these arguments raised subtle legal questions about how to apply the relevant principles of international law to a concrete dispute over hydropower development.\footnote{See \cite[351-352]{eckhoff82}. One of the most important international instruments, namely ILO Convention No 107, was not ratified by Norway at the time of {\it Alta} (Norway later ratified its replacement, ILO Convention No 169). However, it was argued that it had the status of customary international law. See generally \cite{eide80}.} However, the Court refused to consider such  questions, finding that the negative effect of the hydroelectric plant was not so severe as to raise  human rights issues.\footnote{See \cite[299-300]{alta82}. See also \cite[351-352]{eckhoff82}.}}

First, the Court summarily rejects arguments based on indigenous and human rights law on the basis that the interference in question would not be sufficiently severe to raise any issues in this regard. After concluding in this way, the Supreme Court goes on to approach the case on the basis of administrative law instead.\footnote{See \cite[351-352]{eckhoff82}. It also bears noting that the most important international instrument protecting indigenous rights, ILO Convention No 107, was not ratified by Norway at the time of {\it Alta} (Norway later ratified its replacement, ILO Convention No 169). Still, it was argued that it had the status of customary international law, an argument not considered in any depth by the Supreme Court. See generally \cite{eide80}.} The focus was solely on the procedural rules of the \cite{wra17}. In this regard, the opponents of the {\it Alta} development had pointed to a large number of purported shortcomings of the decision-making process. 

First, it had been argued that the original licensing application did not meet the requirements stipulated in section 5 of the \cite{wra17}. Essentially, the original application contained little more than technical details about the planned development, with hardly any identification or assessment of deleterious effects.\footnote{See \cite[264-265]{alta82}.} This shortcoming had been openly acknowledge by the water authorities themselves, who had nevertheless initiated a public hearing.\footnote{See \cite[265]{alta82}.}

The Supreme Court concluded that this was ``clearly unfortunate''.\footcite[265]{alta82} However, several reports and assessments had subsequently been provided, to fill the gaps left open by the initial application. For this reason, the Supreme Court held that the initial mistakes were irrelevant, since it was the licensing process as a whole that should be assessed.\footnote{See \cite[265-266]{alta82}.} Shortcomings at specific stages in the assessment would not be given weight unless they could be seen to imbue the process with a dubious character overall.\footcite[265]{alta82}

The Court then moved on to assess whether the process as a whole fulfilled the procedural requirements of sections 5 and 6 in the \cite{wra17}. In addition, the Court considered whether the assessment of the licensing criteria in section 8 of the \cite{wra17} had been sufficiently detailed.\footnote{Compare also section 16 of the \cite{paa67}, requiring assessments to be as detailed as possible.}

In addition to assessing a large amount of information regarding the situation in {\it Alta} and how it had been assessed by the water authorities, the {\it Alta} Court also made some important statements of principle. In particular, the Court held that since a licensing decision itself is discretionary, it is appropriate to grant the executive some margin of appreciation also with regard to the question of how to interpret vague requirements of administrative law.\footnote{See \cite[262-264]{alta82}.}

The Court made a second decision of principle when it supported the state's contention that the administrative licensing assessment did not have to be as thorough as that required in a subsequent appraisal dispute.\footnote{See \cite[279|330]{alta82}.} This also serves to downplay the risk of factual error; if mistakes are made with regard to the owners' losses at the assessment stage, these mistakes can be corrected later by a correct compensation award.

In fact, the {\it Alta} Court agreed that the license had been based on erroneous information about some issues, particularly regarding alternative ways to meet the need for electricity in Finnmark.\footnote{See \cite[346-357]{alta82}.} However, the Supreme Court did not regard the factual errors in this regard as relevant to the licensing decision.\footnote{See \cite[346]{alta82}.} 

Here a third clarification of principle took place. The Court held, in particular, that the duty to consider alternatives -- different ways in which the public purpose could be satisfied -- is very limited in hydropower cases.\footnote{See \cite[346]{alta82}.} On this basis, the Court argues that factual errors and inadequate information regarding alternatives is less relevant.\footcite[346]{alta82} Apparently, since this information is not required in the first place, if the authorities get it wrong, it does not as easily count as a breach of procedure.

The Court's perspective on alternatives appears to have been at odds with how parliament had actually approached the licensing question, on three separate occasions.\footnote{See \cite[342]{alta82}.} Indeed, there was little doubt that the favourable political assessment of the {\it Alta} development depended strongly on the perceived electricity crisis in Finnmark and the supply situation in Norway generally, as well as the perceived inadequacies of alternative solutions.\footnote{See \cite[338-347]{alta82}.}

Hence, it is quite remarkable how little attention the Court directs towards the factual errors and the inadequate information that had been provided concerning alternatives.\footnote{See also the surprise expressed in \cite[349-351]{eckhoff82}.} By contrast, the Court goes into painstaking detail regarding issues that seem to have been far less important to the political decision-makers.

The dismissive attitude towards the duty to correctly assess alternatives is a controversial aspect of the {\it Alta}-decision.\footnote{See \cite[311]{haagensen02}. For criticism of the Supreme Court on this point, see \cite[580-584]{backer86}.} On this point especially, the decision has met with criticism from commentators arguing that the decision shows the extent to which the courts in Norway tend to identify themselves with other organs of state.\footnote{See, e.g., \cite[64]{graver88} (commenting also that ``government prestige'' was at stake).} Some have taken a more positive approach by arguing that {\it Alta} would be unlikely to become a leading precedent, especially with regard to the duty to assess alternatives.\footnote{See \cite[580-584]{backer86}.} But this has been proven wrong. Indeed, {\it Alta} continues to receive favourable citations by the Supreme Court, both in relation to hydropower as well as with regard to administrative law more generally.\footnote{See \cite{ambassade09,jorpeland11}.}

It should be mentioned, however, that after the {\it Alta} decision, the legal position of the Sami people has improved quite significantly.\footnote{See generally \cite{gauslaa07}. Gauslaa presents the emergence of {\it Sami law}, a collection of rules and principles serving to protect established land use patterns and the Sami way of life while also giving the Sami people a better opportunity to partake in decision-making processes that affect them as group.} Moreover, the controversy surrounding {\it Alta} has been regarded as a catalyst for change in this regard.\footnote{See \cite[156]{ravna12s}.} Hence, it is unlikely that the courts today would be as quick as the {\it Alta} court to dismiss arguments based on aboriginal rights.\footnote{See \cite[180]{gauslaa07}.}

However, with regard to local owners more generally, the {\it Alta} decision is considered to express key principles that still apply.\footnote{See \cite{jorpeland11}. See also \cite[312]{haagensen02}.} 
At the same time, the context surrounding takings for hydropower development have changed significantly since {\it Alta}. First, as discussed in the previous chapter, takings of waterfalls now occur in a very different economic context. Moreover, as discussed in the next section, the legal context has also changed, giving rise to a situation where expropriations of waterfalls have become pure takings for profit.

\section{Taking Waterfalls for Profit}\label{sec:5:5}

Following the introduction of the \cite{wra00}, the legislative authority to expropriate waterfalls  was extended and incorporated in the \cite{ea59}. Furthermore, for the first time in Norwegian history, it would become possible for private commercial interests to openly expropriate waterfalls.\footnote{In cases involving diversion of water, a {\it de facto} right to expropriate could be granted to private actors already under section 16 of the \cite{wra17}. See the discussion in Section \ref{sec:c}.} This change in the law was not singled out for political consideration. In fact, the increased scope of expropriation was not mentioned at all when the Ministry presented their proposal to parliament. Rather, the new expropriation authority was described merely as a ``simplification'' of existing law.\footcite[223-225]{otprp39}

The original proposal stemmed from the report handed to the Ministry by a commission appointed to prepare a new act relating to water resources. The report totals almost 500 pages, but devotes only three of those pages to discussing the new expropriation authority.\footnote{See \cite[235-237]{nou094}.} Here the committee notes that a range of different authorities for expropriation has long co-existed in the law, with many of them positing strict and specific public interest requirements as a precondition for granting a license. This, the commission argues, is not a very ``pedagogical'' way of providing expropriation authorities.\footcite[235]{nou94} Moreover, the commission notes that it runs the risk of omitting important purposes for which expropriation should be possible. Hence, the commission proposes to replace all older authorities by a sweeping authority that will make expropriation possible for any project that involves ``measures in watercourses''.\footcite[235-236]{nou94}

The commission comments that their formulation might seem wide, but remarks that this is not a problem since the executive can simply refuse to issue an expropriation order when they regard expropriation as undesirable.\footcite[235]{nou94} The commission does not reflect on the consequences of such a perspective, neither in relation to property rights nor in relation to the balance of power between the legislature, the executive and the courts. Instead, the commission offers a brief presentation of the rationale behind dropping the local supply restriction for municipal expropriation. They comment that these rules complicate the law and might make desirable expropriations impossible.\footcite[235]{nou94} Nothing is said to clarify what kind of desirable expropriations the committee think might be left out. 

Importantly, the committee do not relate their proposals to the recent liberalisation of the energy sector. Hence, the obvious practical consequence of their proposal, namely that expropriation of waterfalls would be made available as a profit-making tool for commercial companies, is not discussed or critically assessed. The issue of {\it who} should be permitted to benefit from an expropriation license is also dealt with only superficially. In this regard, the commission structure their presentation around the redemption rule of the \cite{wra40}. As mentioned briefly in Section \ref{sec:twp}, this rule made it possible for the majority owners of a waterfall to compulsorily acquire minority rights, if this was necessary to facilitate hydropower development. Hence, it was a rule that provided only a limited opportunity for private takings, restricted to owners themselves or external developers that had been able to reach a deal with a locally based majority.

The main justification given by the commission for introducing a general private takings authority is that the special redemption rule had not been much used.\footcite[236]{nou94} Why this is an argument in favour of opening up for private expropriation in general is not made clear. It seems just as natural to regard it as an argument {\it against} doing so. Why extend the possibility for private expropriation if the demand for such expropriation has been limited?

Presumably, the commission thought there would be a demand for private expropriation in the future, but this is not stated explicitly, nor is the appropriateness of it discussed. As to the requirement that private takers must already control a majority of the waterfall rights in the local area, the commission only remarks that it regards such a restriction as old-fashioned.\footcite[236]{nou94} No discussion is offered regarding the consequences for local communities, if it is dropped.

Since the passage of the \cite{wra00}, it has become clear that the new authority for expropriation is a particularly controversial aspect of the act. This is because cases of waterfall expropriation today tend to imply that local owners are deprived of a small-scale development potential in favour of a commercial company. This has resulted in a new body of case law developing on takings for hydropower, as discussed in the next few sections.

\section{{\it Sauda}}\label{sec:5:5:1}

In {\it Sauda}, a case before the court of appeal, the riparian owners formally protested a license that granted a private company the right to expropriate their rivers and waterfalls.\footnote{See \cite{sauda07} (the decision from the district appraisal court) and \cite{sauda09} (the decision from the appraisal court of appeal).} In the district court, the owners' principal argument was that the executive could not grant such a right to a private party, since the legislation authorising private expropriation of waterfalls had not been properly authorised by parliament.\footnote{See \cite{sauda07}.}

This argument appeared weak, since the \cite{ea59} had been amended to ensure that the executive would be authorised to decide what legal persons could expropriate for hydropower purposes. However, the owners argued that the executive had not appropriately informed Parliament that this would be the consequence of the amendment. In particular, the amendment itself had been passed as a mere formality following the adoption of the \cite{wra00}. 

The owners presented the written testimony of two members of the parliamentary committee that had prepared the Act.\footnote{Presented to the Court in \cite{sauda07} (available from the author on request).} Neither of them could recollect that they had been aware that the Act would make private expropriation possible. This had apparently not been communicated to them by the executive. Moreover, it was not explicitly stated anywhere in the Act itself. Rather, it followed implicitly from three different sections in two separate acts.\footnote{The \cite[51]{wra00} and \cite[2][3]{ea59} respectively.} In the entire collection of preparatory documents, the change was discussed only once, and then only very briefly, in the report from the committee to the Ministry.\footnote{See the discussion in Section \ref{sec:twpp} above.}

On this basis, the owners argued that the purported expropriation authority was not constitutionally valid, since parliament had not intended it. Unsurprisingly, this argument was rejected.\footnote{See \cite{sauda07}. It should also be noted that the appraisal court of appeal did not even consider this argument, since it was held that a separate expropriation decision was not needed at all. Instead, the necessary authority to expropriate was taken to be section 16 of the \cite{wra17}. See \cite{sauda09}. See also the discussion in Section \ref{sec:special} above.} According to the court, it had to be assumed that parliament understood the consequences of their own legislation.\footnote{See \cite{sauda07}.} 

In addition to the constitutional complaint, the owners in {\it Sauda} also raised procedural objections. They argued, in particular, that the expropriation question had been insufficiently assessed by the water authorities and that the administrative expropriation decision was therefore invalid.\footnote{See \cite{sauda09}.} The court did not agree, but the procedural arguments at stake here foreshadow the later case of {\it Jørpeland}, discussed in more depth in Section \ref{sec:jorpeland}.

While the owners in {\it Sauda} lost the validity dispute, the level of compensation they received was dramatically increased compared to earlier practice. Because of this, the development company appealed the decision to the Supreme Court, with the owners lodging a counter-appeal regarding the question of legitimacy. However, the Supreme Court decided not to hear the case, probably because it had recently considered the compensation in the case of {\it Uleberg}, discussed in the next section.

\subsection{{\it Uleberg}}\label{sec:5:5:2}

Just before the {\it Sauda} case was decided by the court of appeal, the Supreme Court had addressed the compensation question in the case of {\it Uleberg}.\footnote{See \cite{uleberg08}.} Here the Supreme Court agreed in principle that the natural horsepower method was not binding on the appraisal courts. Specifically, the Court held that market value compensation could be awarded just in case small-scale development by owners would have been ``foreseeable'' in the absence of expropriation.\footnote{See \cite[81]{uleberg08}.} In {\it Uleberg}, this was not the case. Specifically, the Supreme Court found that the relevant date of valuation was in 1968, when the waterfall rights in question had been transferred to the developer by a voluntary agreement.\footnote{See \cite[70]{uleberg08}.}

This agreement stated that the final payment to the owners should be fixed by the appraisal courts at the time when the development took place. Both the appraisal court and the appraisal court of appeal took this to mean that the valuation should be based on the value of the waterfall at the time when the compensation was awarded. However, the Supreme Court disagreed, holding instead that the intended reading was that the valuation should be based on the value of the waterfall at the date when the voluntary agreement was made (with interest paid for the delay).\footnote{See \cite[71]{uleberg08}.}

Furthermore, the Supreme Court then stated, without any substantive argument, that since this was the date of valuation, the natural horsepower method should be used.\footnote{See \cite[62]{uleberg08}.} Presumably, this was based on the opinion that it was obvious that owner-led development would have been `unforeseeable' at this time. The exact meaning of the foreseeability requirement has since become a much contested issue, resulting in several Supreme Court cases pertaining specifically to the compensation question.

\subsection{Recent Developments on Compensation}\label{sec:5:5:3}

Since {\it Uleberg}, there have been many controversial cases involving expropriation of waterfalls.\footnote{See generally \cite{larsen06,larsen08,larsen12}.} In most of these, the issue of compensation has occupied center stage. With respect to this issue, owners initially appeared to be gaining significant ground, as the appraisal courts started to apply a market-based method quite systematically, resulting in dramatically increased compensation payments.\footnote{See the discussion on the natural horsepower method above, in Section \ref{sec:nathp}.}

The large energy companies consistently resisted this development, typically by arguing that small-scale hydropower was unforeseeable and therefore not compensable according to the principle expressed in {\it Uleberg}.\footnote{See, e.g., \cite{klovtveit11,otra10,otra13}.} Moreover, the large energy companies would tend to argue that a license to undertake large-scale development was by itself conclusive evidence in support of the claim that small-scale development was unforeseeable.\footnote{See, e.g., \cite[17]{otra10}.} The large-scale development license showed, according to the large energy companies, that a license to undertake small-scale development could not be regarded as foreseeable.

This line of argument clearly conflicts with the so-called no-scheme principle, whereby compensation for expropriated property is to be based on the situation such as it would have been in the absence of the expropriation scheme.\footnote{This principle is also referred to as the ``Pointe Gourde'' principle in common law, and is sometimes known as the ``elimination rule''  in Europe. The basic idea is found in many jurisdictions, although details of the rule can vary. For a more detailed presentation of the version that applies in Norway, including a comparison with England and an assessment of the special issues that arise when the principle is applied to economic development takings, see \cite{dyrkolbotn15}.} In the absence of plans for large-scale development, it is often quite clear that the owners would have succeeded in obtaining a license to undertake small-scale hydropower. However, the large-scale energy companies maintained that small-scale hydropower should be considered unforeseeable even in these cases, since large-scale development was the preferred option for the licensing authorities.

In most early cases before the lower courts, this argument failed. Moreover, in the case of {\it Otra I}, it appeared as though it was rejected also by the Supreme Court.\footnote{See \cite[31-48]{otra10}.} However, the Court did not focus specifically on the no-scheme principle and how it should be applied in hydropower cases. Moreover, the taker in that case succeeded in having the appraisal court of appeal's decision overturned on the basis that inadequate reasons had been provided to justify the amount of compensation awarded to owners.\footnote{See \cite[52]{otra10}.} The court of appeal therefore had to hear the case again. This time, the taker was able to successfully argue that small-scale hydropower was unforeseeable. Hence, the court of appeal used the natural horsepower method to calculate compensation.\footnote{In fact, the court used a slightly modified version of the method, first developed in \cite{sauda09}, serving to make the discrepancy between market value and compensation slightly less pronounced. See \cite{otra12}.} 

The owners duly appealed the decision to the Supreme Court, which agreed to consider the case for a second time.\footnote{See \cite{otra13}.} But this time, the Supreme Court endorsed the understanding of the no-scheme principle of the large energy companies. Specifically, the Court refused to censor the appraisal court of appeal's assessment of foreseeability, even though it was based explicitly on the premise that the expropriation project was preferable from the point of view of the licensing authorities.\footnote{See \cite[53-54]{otra13}.}

If the precedent set by {\it Otra II} stands, market value compensation will generally not be awarded in future cases where waterfalls are expropriated in favour of large-scale schemes.\footnote{The precedent has already been used to deny small-scale compensation in the case of \cite{smibelg15} (appeal to the Supreme Court denied).} However, it should be noted that the Supreme Court has been very vague on how exactly it understands the no-scheme principle in these cases. Instead of tackling this issue directly, the Court has chosen to rely largely on deference to the foreseeability determinations carried out by the appraisal courts.

This is clearly illustrated by the earlier case of {\it Kløvtveit}.\footnote{See \cite{klovtveit11}.} Here the Supreme Court agreed with the appraisal court of appeal that it might in principle be foreseeable that the owners, in the absence of expropriation, could have cooperated with the taker to implement the expropriation project. This too contradicts the no-scheme principle, but unlike the reasoning of {\it Otra II}, it also provides an alternative route to market value compensation, on the basis of a valuation of the expropriation project itself. In effect, it points to an approach that promises to deliver a form of {\it benefit sharing} between owners and takers. 

For this reason, {\it Kløvtveit} is an interesting decision. However, it seems quite unlikely that it will become an important precedent for the future. Its importance was undermined already by {\it Otra II}, when the presiding judge explicitly denied that cooperation between the taker and the owners was a realistic scenario in that case.\footnote{See \cite[69-71]{otra13}.} Moreover, {\it Kløvtveit} itself was eventually sent back to the appraisal court of appeal, because the Supreme Court held that the date of valuation had been incorrectly determined.\footnote{See \cite[35-39]{klovtveit11}.} On the second hearing in the appraisal court of appeal, cooperation between owners and taker was regarded as unforeseeable, so market value compensation was denied.\footnote{See \cite{klovtveit13}.}

In fact, no case heard by the Supreme Court so far has concluded with compensation based on market values. In the end, the natural horsepower method has always been used. This, no doubt, sends a clear signal to the appraisal courts. In the future, the likelihood is that we will see a resurgence of the natural horsepower method and a return to compensation awards amounting to tiny fractions of the actual values that are taken from local owners.

In light of this development, the broader issue of legitimacy becomes increasingly important. The financial entitlements of owners and communities, which seemed to be more strongly protected after {\it Uleberg}, are again at great risk of being undermined. Moreover, as the social function theory of property indicates, the issue of legitimacy goes well beyond the individual financial entitlements of owners. It also pertains to the status of the local communities, the duties of owners in this regard, sustainable management, and the democratic legitimacy of decision-making regarding natural resources. These aspects have not received any attention from Norwegian courts so far. However, as I have already mentioned, the case of {\it Jørpeland} saw the procedural legitimacy of hydropower takings come to the forefront, for the first time since the case of {\it Alta}. In addition to clarifying legal points in this regard, the case also sheds light on the practices adopted by the water authorities in expropriation cases. Hence, it provides an excellent opportunity for a closer inquiry into the legitimacy question.

\section{A detailed case study: {\it Ola Måland v Jørpeland Kraft AS}}\label{sec:5:6}

The expropriating party was a public-private commercial partnership, Jørpeland Kraft AS. Originally, this limited liability company was jointly owned by Scana Steel Stavanger AS, with 1/3 of the shares, and Lyse Kraft AS, with the remainder.\footnote{See \cite[2]{jorpeland09}.} Lyse Kraft AS is a publicly owned energy company with the city municipality of Stavanger being its largest shareholder. Scana Steel Stavanger AS, on the other hand, was a subsidiary of the publicly traded Scana Steel Industrier ASA. The largest shareholder of this parent company is a leading business person and one of the richest people in south-west Norway, based in Stavanger.\footnote{See \cite{birkevold09} (the business man in question is John Arild Ertvaag).}

Scana Steel Stavanger had long operated a steel mill in the small town of Jørpeland, belonging to the municipality of Stranda, in Rogaland county, south-west Norway. The source of energy was a relatively small hydropower plant harnessing energy from the river that reaches the sea near Jørpeland.\footnote{See \cite{aadland09}.} At the height of activity, the mill had about 1200 employees and was an important local institution.\footnote{See \cite[11]{meland82}.} However, after going bankrupt and being reorganised in 1977, the importance of the steel mill declined significantly.\footnote{See \cite[8-15]{meland82}.} After a second bankruptcy in 2015, Scana Steel Stavanger seized to exist. The mill was then reorganised yet again, and the number of employees was reduced from around 100 to around 30.\footnote{See \cite{jossang15}.}

In parallel with the decline of the steel mill, the hydropower plant in Jørpeland was rebuilt and expanded, not to supply energy for local industry, but to sell electricity on the national grid.\footnote{See \cite{aadland09}.} Jørpeland Kraft AS was charged with undertaking this development, which was thereby decoupled from the steel mill operations. In 2011, the same year when {\it Måland} came before the Supreme Court, Scana Steel Stavanger AS sold their shares in Jørpeland Kraft AS to the German investment company Aquilla Capital.\footnote{See \cite{sandvik11}.} The story of Jørpeland, therefore, nicely illustrates broader trends in the history of the hydropower sector in Norway, as discussed in the previous chapter.

The river that gave rise to controversy in {\it Måland} was not located in the same municipality as Jørpeland, but in a different valley across a mountain range, in the municipality of Hjelmeland. The contested license in {\it Måland} gave Jørpeland Kraft AS the right to divert the water from this river for electricity production at Jørpeland. In the following, I present the facts of the case in more detail, before considering the legal questions that were addressed by the courts.

\subsection{The Facts of the Case}\label{sec:5:6:1}

One relatively small river from which Jørpeland Kraft AS suggested to extract water was not located in Jørpeland. Rather, it runs through the neighbouring municipality of Hjelmeland, on the other side of a mountain range, until it eventually reaches the sea at Tau, another neighbouring municipality. 

The plans to divert the river would deprive the riparian owners of water along some 15 km of riverbed, all the way from the mountains on the border between Hjelmeland and Jørpeland, to the sea at Tau. Not all the water would be removed, but the flow of water would be greatly reduced in the upper part of the river known as {\it Sagåna}, the rights to which is held jointly by Ola Måland and five other local farmers from Hjelmeland.

The water in question comes from a lake called \emph{Brokavatn}, located 646 meters above sea level, where altitude soon drops rapidly, making the river suitable for hydropower development. Plans were already in place for such a project, which would use the water from just below the altitude of Brokavatn, to the valley in which the original owners' farms are located, about 80 meters above sea level. 

A rough estimate of the potential of this project was made by the NVE itself, stating that the energy yield would be 7.49 GWh per annum.\footnote{See \cite[16]{jorpeland09}.} This is about five times more energy than the water from Brokavatn would contribute to the project proposed by Jørpeland Kraft AS.\footnote{See \cite[19]{jorpeland09}.}

Importantly, the estimate was not made in relation to the expropriation case, but as part of a national project to survey the remaining energy potential in Norwegian rivers.\footnote{The survey was carried out in 2004 and its results are summarised in \cite{jensen04}.} Ola Måland and the other owners of the river were not identified as significant stakeholders and were not notified of the assessment that had been made. Moreover, even after Jørpeland Kraft AS had submitted a formal application for permission to divert the water, the owners were not notified by the water authorities.\footnote{See \cite[16]{jorpeland09}. However, a generic orientation letter was apparently sent by Jørpeland Kraft AS, a letter that the owners themselves could not remember having received. See \cite[5|8]{jorpeland11a}.}

Moreover, the procedural approach to the case was the traditional one, with an assessment directed at evaluating the environmental impact. Many interest groups were called on to comment on environmental consequences, and public debate arose with respect to the balancing of commercial interests and the desire to preserve wildlife and nature.\footnote{See \cite[19]{jorpeland09}.}

One of the local owners, Arne Ritland, also commented on the proposed project. He did this in an informal letter sent directly to Scana Steel Stavanger AS.\footnote{See \cite[17]{jorpeland09}.} In this letter, he inquired for further information and protested the proposed diversion of water from Brokavatn. He also mentioned the possibility that an alternative hydropower project could be undertaken by original owners, but he did not go into any details, stating only that a locally owned hydropower plant had previously been in operation in the area. 

The plant he was referring to dates back to the time before there was a national grid. It ensured a local supply of electricity, but has since been shut down, in keeping with the general trend mentioned in Chapter \ref{chap:3}.

Arne Ritland received a reply from Scana Steel Stavanger AS, which stated that more information on the project and its consequences would soon be provided. Ritland did not pursue the matter further at this time. Meanwhile, Scana Steel Stavanger AS submitted his letter to the NVE, who in turn presented it as a comment directed at the application.\footnote{\cite[18]{jorpeland09}.}

This prompted the majority owner of Jørpeland Kraft AS, Lyse Kraft AS, to undertake their own survey of alternative hydropower in Sagåna.\footnote{See \cite[19]{jorpeland09}.} The conclusions were sent to the water authorities, but the owners were not informed that such an investigation was being conducted.\footnote{See \cite[23]{jorpeland09}.} Moreover, the water authorities did not take steps to investigate the commercial potential of local hydropower on their own accord. Instead, they referred to the conclusion presented by Jørpeland Kraft AS, stating that if the local owners decided to build two hydropower plants in Sagåna, then one of them, in the upper part of the river, would not be profitable, neither with nor without the contested water. The other project, in the lower part, could apparently still be carried out, even after the diversion.\footnote{See \cite[23]{jorpeland09}.}

No mention was made of what the original owners stood to loose, nor was there any argument given as to why it made sense to build two separate small-scale power plants in Sagåna. Nevertheless, the NVE handed the expropriating party's findings over to the Ministry, without conducting their own assessment and without informing the original owners.\footnote{See \cite[22-23]{jorpeland09}.}

In addition to the report made by Jørpeland Kraft AS, the municipality government of Hjelmeland also commented on the possibility of local hydropower. In their statement to the NVE, they directed attention to the data in the NVE's own national survey, which suggested that a single hydropower plant in Sagåna would be a highly beneficial undertaking.\footnote{See \cite[19]{jorpeland09}.} On this basis, they protested the diversion, arguing that original owners should be given the possibility of undertaking such a project.

This statement was not communicated to the original owners, and in their final report the NVE dismissed it by stating that the most efficient use of the water would be to transfer it and harness it at Jørpeland.\footnote{See \cite[19]{jorpeland09}.}

In addition to the statement made by Ritland, one other property owner, Ola Måland, commented on the plans.\footnote{See \cite[17]{jorpeland09}.} He did so without having any knowledge of the commercial potential of the waterfall and without having been informed of the statement made by the municipality of Hjelmeland. Therefore, Måland expressed his support for Jørpeland Kraft's plans, citing that the risk of flooding in Sagåna would be reduced.\footnote{He later joined the other owners in opposition to the expropriation.} He also phrased his letter in such a way that it could be interpreted as a statement on behalf of the owners as a group.\footnote{See \cite[17]{jorpeland09}.} However, Måland was the only person who signed.

In the final report to the Ministry, the NVE refer to Måland's letter and state that the original owners are in favour of the plans.\footnote{See \cite[19]{jorpeland09}.} For this reason, the NVE concludes that the opinion of the municipality of Hjelmeland should not be given any weight.\footnote{See \cite[19]{jorpeland09}.} The NVE neglects to mention that Arne Ritland's statement strongly opposed expropriation. Moreover, earlier in the report, where all incoming statements are reported, Ritland is referred to as a private individual, while Ola Måland is referred to as a property owner who speaks on behalf of the owners as a group.

The report made by the NVE was not communicated to the affected local owners at all, so the owners had no chance of correcting mistakes. However, the report was sent to many other stakeholders, including the municipality of Hjelmeland.\footnote{See \cite[24]{jorpeland09}.} In light of the report, the municipality changed their original position and informed the Ministry that they would not press for local hydropower, since this was not what the affected owners (i.e., Ola Måland) wanted.\footnote{See \cite[24]{jorpeland09}.}

This happened without the owners' knowledge. However, while the case was being prepared by the water authorities, the original owners had begun to seriously consider the potential for hydropower on their own accord. In late 2006, Jørpeland Kraft's application reached the Ministry and a decision was imminent. At the same time, the owners were under the impression that they would receive further information before the case progressed to the assessment stage.

As all the owners, including Ola Måland, had now come to realise the commercial value of the water from Brokavatn, they approached the NVE, inquiring about the status of the plans proposed by Jørpeland Kraft AS. They were subsequently informed that an opinion in support of the transfer had already been delivered to the Ministry. This communication took place in late November 2006, summarised in minutes from meetings between local owners, dated 21 and 29 November.\footnote{Presented to the courts, available upon request.} On 15 December 2006, the King in Council granted a concession for Jørpeland Kraft AS to transfer the water from Brokavatn to Jørpeland.\footnote{See \cite[3]{jorpeland09}.}

At this point, it had become clear to the original owners that the water from Brokavatn would be crucial to the commercial potential of their own project. They also retrieved expert opinions that strongly indicated that the NVE was wrong when they concluded that diverting the water would be the most efficient use of the water.\footnote{See \cite[23]{jorpeland09}.} In light of this, the owners decided to question the legality of the licence (with the corresponding permission to expropriate). They argued, in particular, that the administrative decision to grant the license was invalid.

In the following section, I present the main legal arguments relied on by the parties, as well as a summary of how the three national courts judged the case.

\subsection{Legal Arguments}\label{sec:5:6:2}

First, the owners argued that procedural mistakes had been made by the water authorities when preparing the case.\footnote{See \cite[12]{jorpeland09}.} This, in turn, had resulted in factual mistakes forming the basis of the decision to grant the development license. Since the outcome might have been different if these mistakes had not been made, the owners concluded that the development license could not be upheld.

Second, the owners argued that expropriation of their rights would result in a disproportionate loss of an economic development potential.\footnote{See \cite[5]{jorpeland11a}.} Moreover, they argued that the economic loss would clearly be greater than the gain also from the point of view of the public, since the owners were in a position to make more efficient use of the contested water. Therefore, allowing expropriation would only serve to benefit the commercial interests of Jørpeland Kraft AS, to the detriment of both local and public interests.

Third, the owners argued that the government had not fulfilled its duty to consider the case with due care.\footnote{See \cite[12]{jorpeland09}.} In particular, the assessment of local community interests and the interests of local owners had not been satisfactory. Particular attention was directed at the fact that local owners had not been informed about the progress of the case, and had not been told of assessments pertaining to their interests.

Fourth, the owners argued that irrespective of how the matter stood with respect to national law, the expropriation was unlawful because it would be in breach of the provisions in P1(1) of the ECHR regarding the protection of property.\footnote{See \cite[07-08]{jorpeland09}.}

Jørpeland Kraft AS protested, arguing first that the report from the NVE was not based on factually erroneous information.\footnote{See \cite[16]{jorpeland11}.} As a response to the apparent mistakes that had been made, Jørpeland Kraft AS argued that these did not in any event undermine the quality of the report as a whole.\footnote{See \cite[2]{jorpeland11a)}.} Moreover, it was argued that Måland had probably discussed the diversion of water with other affected owners, and that they had all agreed to support it.\footnote{See \cite[2]{jorpeland11a}.} Furthermore, according to Jørpeland Kraft AS, all the procedural rules of the \cite{wra17} had been observed. Other procedural rules might be relevant, but only in so far they were considered compatible with the rules in the \cite{wra17}.\footnote{See \cite[16]{jorpeland11}.} It was also argued that it was not for the courts to subject the assessment of public and private interests to any further scrutiny, since this was a matter for the administrative branch.\footnote{See \cite[2]{jorpeland11a}.} Finally, Jørpeland Kraft AS argued that diverting the water did not represent a breach of the owners' human rights.\footnote{See \cite[2]{jorpeland11a}.} They argued for this by pointing to the fact that the procedural rules had been followed and that the material decision was beyond reproach. Moreover, Jørpeland Kraft AS argued that since the owners would be compensated financially for whatever loss they incurred, it was clear that no human rights issues were at stake.\footnote{See \cite[2]{jorpeland11a}.}

\subsection{The Lower Courts}\label{sec:5:6:3}

The matter went before the district court in the city of Stavanger, which decided in favour of the owners on 20 May 2009.\footnote{See \cite{jorpeland09}.} The district court agreed with the local  owners that the decision to grant the license was based on an erroneous account of the relevant facts.\footnote{See \cite[25]{jorpeland11}.} Moreover, the court concluded that it was evident that allowing the applicants to use the water from Brokavatn in their own hydroelectric scheme would be the most efficient way of harnessing the hydropower potential.\footnote{See \cite[22-23]{jorpeland09}.} This, the court noted, directly contradicted what the NVE had stated in their report.\footnote{See \cite[23]{jorpeland09}.}

The court backed up its conclusion on the facts by giving several direct quotes from the report made by the NVE. On the legal side, they relied on a well-established principle of administrative law: while the exercise of discretionary powers is usually not subject to review by court, a decision based on factual mistakes is invalid if it can be shown that the mistakes in question were such that they could have affected the outcome.\footnote{See \cite[407-410]{eckhoff14}. For the requirement that the mistakes must have been such that they could have affected the outcome, see \cite[41]{paa67}} Since the small-scale alternative would in fact represent a more effective use of the water in question, the court was not in doubt that this principle applied here.\footnote{See \cite[25]{jorpeland09}.}

Since the district court held that the license to allow diversion was invalid because it was based on factual mistakes, there was no need to consider claims regarding the legitimacy of the diversion with respect to human rights law. However, the district court did comment that the traditional procedure used to deal with diversion cases was inadequate and had to be supplemented by looking to the procedural rules in the \cite{ea59} and the \cite{paa67}.\footnote{See \cite[21]{jorpeland09}.} 

Moreover, the court made a crucial statement about expropriation of riparian rights in general, regarding the duty of the water authorities to properly assess whether or not an expropriation license should be granted.\footnote{The duty is a general principle of administrative law, expressed both in \cite[12]{ea59} and \cite[16]{paa67}.} This duty, the court held, included a duty to properly consider negative effects on small-scale development potentials.\footnote{See \cite[22]{jorpeland09}.} According to the court, this was the natural consequence of the increasing interest in small-scale development. If this principle had become part of Norwegian hydropower law, it would have had significant implications for the water authorities, directly confronting their traditional lack of interest in the expropriation question. However, it was not to be, as the court's decision was overturned on appeal.

Indeed, the court of appeal approached the case very different than the district court. Specifically, its decision did not rely on any close assessment of the facts and the report made by the NVE. Instead, the court of appeal largely based its decision on the opinion that the rules in the \cite{wra17} exhaustively regulate the administrative procedure in watercourse regulation cases.\footnote{See \cite[7]{jorpeland11a}.} According to the court of appeal, the procedural rules in the \cite{ea59} and the \cite{paa67} do not apply at all to diversions of water authorised under section 16 of the \cite{wra17}.\footnote{See \cite[7]{jorpeland11a}.} 

This finding was based on the argument that the more specific rules of the \cite{wra17} have priority under the so-called {\it lex specialis} principle, which applies in case of conflict between different sets of rules, giving priority to those that are more specific.\footnote{See \cite[7]{jorpeland11a}.} Apparently, the court thought that there was a conflict between principles of administrative law and the rules that apply specifically in hydropower cases.\footnote{The decision is not entirely clear on this point, however, as the court also makes a sweeping remark to the effect that the rules in the \cite{wra17} conform to all ``basic and general'' procedural demands of administrative law. This, however, seems to be a reference to unwritten principles, not those specific provisions included in the \cite{paa67} and the \cite{ea59} which were found not to apply.} With regard to the procedural rules of the \cite{wra17}, the court does not go into much detail, but concludes that the assessment of the water authorities met all general requirements and was clearly adequate. Regarding the factual basis for the license, the court did not comment at all on much of the evidence presented to them. Moreover, the Court did not address those quoted segments of the report from the NVE that had formed the basis for the district court's decision.

Specifically, the court of appeal never mentions the objection to the transfer made by the muncipality of Hjelmeland, nor does it mention the fact the small-scale alternative suggested there would use the contested water more effectively.

Instead, the court of appeal points out that the NVE was well aware of the possibility of developing small-scale hydropower, was well-informed about such development, and had considered it during their assessment.\footnote{See \cite[9]{jorpeland11a}.} The court of appeal notes that the NVE's written assessment on this point was brief, but argues that this must be understood as a natural response to what the court of appeal describes as a lack of input from local owners.\footnote{See \cite[9]{jorpeland11a}.}

The owners appealed the court of appeal's decision to the Supreme Court, which decided to hear the  juridical aspects of the case.\footnote{See \cite[8]{jorpeland11}. Specifically, the Supreme Court would not engage in any independent factfinding, but only consider legal questions, including how the law should be applied to the facts.}

\subsection{The Supreme Court}\label{sec:5:6:4}

The Supreme Court approached the case in much the same way as the court of appeal. Regarding the facts, the Court emphasises that the majority owner of Jørpeland Kraft AS had considered the possibility that a hydroelectric scheme could be undertaken by local property owners.\footnote{See \cite[53]{jorpeland11}.} As mentioned, this resulted in a report based on the premise that the owners could have built two separate small-scale plants in the same river. The conclusion is that one of these would be unprofitable regardless of the diversion, while the other one could still be carried out.\footnote{See \cite[23]{jorpeland09}.} However, the report does not explain why anyone would want to build two consecutive small-scale plants in the same river, an approach that diverges from all other expert reports retrieved about small-scale potentials.\footnote{See \cite[16|23]{jorpeland09}.}

In any event, the most relevant question would be what the owners stood to loose when the water from Brokavatn was diverted away from Sagåna. Both the report and the Supreme Court remained silent on this point. Moreover, the Court does not mention that the report was never handed over to the applicants, nor that the details of the calculations were never independently considered by the NVE. Just like the court of appeal, the Supreme Court also neglects to mention that small-scale development would be a more efficient use of the water, according to the national survey of small-scale potentials carried out by the NVE itself.\footnote{See \cite[16]{jorpeland09}.} Furthermore, no mention is made of the fact that the NVE claims that the opposite is true in the report to the Ministry, contradicting also the statement made by the municipality of Hjelmeland.

Regarding the legal questions raised by the case, the Supreme Court rejects the view that the procedural rules in the \cite{ea59} and the \cite{paa67} do not apply to the case.\footnote{See \cite[32-34]{jorpeland11}.} However, the Court holds that these procedural rules do not imply a more extensive duty to assess the expropriation question, compared to established practices in hydropower cases.\footnote{See \cite[51-52]{jorpeland11} (citing also the {\it Alta} case, \cite{alta82}).} 

There is no rule in the \cite{wra17} which states that the authorities are required to consider specifically the question of how the regulation affects the interests of property owners. Moreover, administrative practice suggests that this is not done, except perhaps to some extent when the issue is explicitly raised during the hearing.\footnote{See \cite{stokker10}. This is the water authorities' own guideline for the assessment of large-scale applications. The previous version of this guideline (which also fails to mention the interests of owners) was presented to the Supreme Court. The Court also refers to it explicitly when it comments that existing practices are beyond reproach. See \cite[51]{jorpeland11}.} However, a rule explicitly demanding this is found in section 2 of the \cite{ea59}. This is not regarded as a procedural rule, however, as it pertains to the material considerations that the administrative branch is required to carry out in expropriation cases. 

Indeed, according to the Supreme Court, the rule does not apply at all when expropriation takes place on the basis of section 16 of the \cite{wra17}.\footnote{See \cite[30]{jorpeland11}.} This is the conclusion despite the fact that section 30 of the \cite{ea59} explicitly states that the provisions of that act apply to expropriations pursuant to the \cite{wra17}, in so far as they are compatible with the rules therein. It would appear to follow, by implication, that the Supreme Court does {\it not} think that directing more attention at owners' interests, as prescribed by section 2 of the \cite{ea59}, is compatible with the \cite{wra17}. 

This is a clear rejection of the principled position taken by the district court, whereby the water authorities should generally be obliged to consider small-scale alternatives before allowing expropriation. According to the Supreme Court, no special procedural obligations arise at all in such cases, which can still effectively be processed as thought the riparian rights already belong to the applicant. In short, expropriation is to remain a non-issue during the licensing process pursuant to the \cite{wra17}. 

Formally, this implication of {\it Jørpeland} only applies to expropriations carried out on the basis of section 16 of that act. However, in practice, there is reason to believe that the impact will be the same for all cases involving large-scale hydropower development. Indeed, the water authorities themselves do not appear to make any significant distinction between large-scale applications based on whether or not a separate license to expropriate waterfalls is formally required.\footnote{See \cite{flatby08}.}

\noo{ %It also bears noting that the facts in {\it Jørpeland} appear to suggest that the procedural shortcomings underlying that case were much more obvious than the shortcomings complained of in {\it Alta} (although the scale of the underlying conflict was much greater in {\it Alta}).
To further illustrate the extent to which {\it Jørpeland} signals a dismissive attitude towards owners and local communities, I will conclude by offering a quote from Harald Solli, director of the hydropower licensing section at the Ministry. Sollie submitted written evidence to the Supreme Court regarding the practices observed in cases involving expropriation of water power. Below, I quote two exchanges that demonstrate how current practices leave local owners in a precarious position.

\begin{quote}
Q: In cases pursuant to the \cite{wra17}, is it common for the water authorities to send prior written notices to the private owners that may be affected by a loss of a small-scale hydropower potential? \\
A: The procedural rules that apply to cases pursuant to the \cite{wra17} are found in section 6. To give such a written notice to private owners is not required. As far as I am aware, it is also not done, but I have no first-hand knowledge of this, since the NVE is responsible for the case at this stage. \\
Q: In cases such as this, should owners affected by the loss of a small-scale hydropower potential be kept informed about the factual basis on which the authorities plan to make their decision? I am thinking especially about cases when the authorities do in fact provide an assessment of the potential for small-scale hydropower on private properties. \\
A: Affected owners must look after their own interests. The assessments made by the NVE in their report is a public document, and it can be accessed through the homepage of the NVE.
\end{quote}

By their reasoning in \emph{Jørpeland}, it appears that the Supreme Court gave this dismissive attitude towards local owners a stamp of approval. In light of this, I believe the study of the law in a socio-legal setting becomes all the more relevant. For while the dismissive attitude might be a part of the national legal order, it seems pertinent to ask if it is a reasonable attitude to take towards local owners of valuable natural resources. Also, one may ask if a case can be made with respect to human rights, by arguing that the protection awarded is insufficient with regard to P1(1). This point was raised in \emph{Jørpeland}, but did not receive any attention from the Supreme Court.\footnote{The {\it Jørpeland} case resulted in a complaint to the ECtHR which has yet to be considered by the Court.}
}

\section{Predation?}\label{sec:5:7}

How should takings of waterfalls be assessed according to the normative theory developed in the first part of this theory? In Chapter \ref{chap:3}, I presented the Gray test, a set of key assessment points for determining whether a taking violates important property norms.\footnote{See Chapter \ref{chap:3}, Section \ref{sec:5}.} In the following, I briefly assess takings of waterfalls against the criteria of the Gray test, to shed further light on the normative status of current practices observed in Norway.

\subsubsection{The Balance of Power}\label{sec:5:7:1}

In light of the presentation so far, it is safe to conclude that typical large-scale waterfall expropriations in Norway are marked by a severe imbalance of power between the taker and the owners. The economic and political power of local communities is clearly very limited compared to that of the large energy companies. Moreover, it is interesting to observe that this imbalance is accentuated by procedural arrangements and practices presided over by the water authorities. As demonstrated by the case of {\it Jørpeland}, the formal position of owners and local communities under administrative law is very weak in hydropower cases. Hence, in addition to shedding doubt on the legitimacy of current practices in Norway, assessing waterfall takings against the balance of power criterion also underscores that this criterion is related to administrative law.

Ideally, procedural rules should function so as to maintain an appropriate balance of power between the different actors involved in an administrative dispute. At least, the rich and powerful should not be allowed to dominate decision-making processes within the polity, at the expense of those most intimately affected by the decisions reached. If the administrative branch fails in this regard, or acts in such a way that existing imbalances are worsened, this is surely a cause for additional criticism with respect to the balance of power criterion. I believe the case study so far shows that the framework for management of Norwegian hydropower is deserving of such criticism.

\subsubsection{The Net Effect on the Parties}\label{sec:5:7:2}

The immediate financial effect that a taking for hydropower has on the owners depends on how the compensation is calculated. As discussed in Section \ref{sec:5:4:3}, the law on this point has been in turmoil in recent years. In the late 2000s, there were signs that a commercially realistic valuation method might become dominant, leading in turn to a dramatic increase in compensation compared to earlier practice based on the natural horsepower method. But this trend now appears to have been reversed, as the energy companies have successfully argued that a license for large-scale development counts as proof that owner-led projects would not in any case have been `foreseeable' (because the necessary licenses would not have been granted). For this reason, the argument goes, owners suffer no actual loss when their resources are taken from them.\footnote{See \cite{otra13}.}

The local owners are in an even weaker position when it comes to indirect financial effects, as well as social and political effects, such as harms done to the cohesion and prosperity of the local community. In this regard, losses are not only under-compensated, they are typically not acknowledged at all, neither by the executive nor by the courts. The effects that go unnoticed range from the concrete, such as losses incurred because the expropriation proceedings drag out in time, to the abstract, such as the damage that is done to democracy when owners and local municipality governments are replaced by energy companies as the primary resource managers in the local district.\footnote{In \cite{smibelg15}, the owner submitted an application for small-scale hydropower in 2005 which the water authorities refused to process on account of a pending large-scale application. In 2015, compensation was awarded based on the natural horsepower method, with no compensation for, or even acknowledgement of, the owners' loss in the 10 year period where the water authorities refused to process their applications.}

\subsubsection{Initiative}\label{sec:5:7:3}

It follows from the regulatory framework that almost all cases involving expropriation for hydropower development originate from applications submitted by commercial companies.\footnote{See especially Chapter \ref{chap:4}, Section \ref{sec:4:4} and Chapter \ref{chap:5}, Section \ref{sec:5:3}.} The energy company draws up the plans and initiates the expropriation proceedings, by submitting a request for a development license to the water authorities. The main purpose, which is usually acknowledged by both the applicant and the water authorities, is to make money. Hence, it is usually hard or impossible to argue that takings of waterfalls for hydropower development in Norway are motivated by any direct public interests. %Indeed, applying the initiative test will suffice to conclude that the primary motive is profit-making. 

Exceptions to this are possible, in so far as the energy companies themselves embody public service functions. In some cases one might argue that they do, but such arguments are becoming increasingly unconvincing due to the fact that most energy companies have been reorganised as for-profit enterprises whose activities are largely unconstrained by institutions of local government.\footnote{See, e.g., the EFTA Court's description of the industry, \cite{efta06}.}

\subsubsection{Location}\label{sec:5:7:4}

Compared to notorious US cases such as {\it Kelo} and {\it Poletown}, the stakes for the owners appear lower in hydropower cases from Norway.\footnote{See \cite{poletown81,kelo05}.} However, as mentioned in Chapter 4, riparian rights are often of great importance to Norwegian farming communities and the subsistence of its members.\footnote{See especially Chapter \ref{chap:4}, Sections \ref{sec:4:2}, \ref{sec:4:4} and \ref{sec:4:5}.} Indeed, the taking of riparian rights from a local community might well contribute significantly to depopulation, although indirectly rather than by physical displacement.\footnote{Today, it is very unlikely that the Norwegian government would sanction physical displacement of people from their homes in order to facilitate hydropower development. However, this state of affairs cannot be taken for granted; the current political attitude on this point appears to have arisen in large part due to extensive and forceful anti-development activism during the 1970s, especially in relation to the {\it Alta} case (which initially involved plans to physically displace a local Sami community). See \cite{altawiki}.} Moreover, in many rural communities, small-scale hydropower appears to be the only growth industry, as farming is becoming increasingly unprofitable and communities are threatened by stagnation and decline.\footnote{For an example, I refer again to the case of the Gloppen municipality, discussed in Chapter \ref{chap:4}, Section \ref{sec:4:4}.}

Hence, the location criterion suggests that takings of waterfalls merit heightened critical scrutiny, especially due to the importance of the property that is taken to the subsistence of the local communities forced to give it up.

\subsubsection{Social Merit}\label{sec:5:7:5}

There is no shortage of electric energy in Norway, and electricity prices are very low compared to the rest of Europe.\footnote{See Chapter \ref{chap:4}, Section \ref{sec:4:1}.} Indeed, development projects such as {\it Jørpeland} are not motivated by any particular need to supply more energy to the Norwegian people or local industry, but openly pursued as commercial endeavours.\footnote{See Chapter \ref{chap:5}, Section \ref{sec:5:6}.} Hence, they do not appear to have any particular social merit.

On the contrary, waterfall takings can contribute to creating social ills. In south-western Norway, for instance, where {\it Jørpeland} is located, the average income for a sheep farmer corresponds to about half of the minimum wage that farmers are required to pay to full-time farm workers.\footnote{According to the Norwegian Bioresearch Institute, the average sheep farmer could expect to earn NOK 65 per hour from working at their farm in 2012. See \cite[50]{smesdal14}. The minimum wage for unskilled farm workers during the same time was NOK 123.15 per hour. The minimum wage for 16-17 year old vacation workers was NOK 83.75 per hour. See \cite{tariff12}.} During harvesting season, sheep farmers wishing to hire 16 year old vacation workers are required to pay the kids about 30 \% more per hour than they themselves can expect to earn from running their own farms. In short, sheep farming communities in western Norway, such as that affected by the taking in {\it Jørpeland}, are struggling.

In this context, it seems that the social harm created by expropriation, whereby disadvantaged rural communities are deprived of the opportunity to manage their own water resources, should be a pressing concern. Because of the traditional approach to hydropower, focusing solely on environmental harms, the social merit of maintaining local ownership and control over resources receives little or no attention from the water authorities in expropriation cases. This in itself suggests that typical cases of waterfall expropriation in Norway will tend to fail the social merit test.

\subsubsection{Environmental Impact}\label{sec:5:7:6}

It is clear that hydropower development can have negative environmental impacts. Hence, it is important that the value of development is appropriately balanced against environmental interests. To ensure this is a core aspect of the regulatory system. As discussed in Chapter 4, local initiatives for small-scale hydropower are now typically scrutinized quite intensely in this regard, particularly after reforms in recent years. By contrast, large-scale projects appear increasingly likely to receive preferential treatment. %Since 2000, only one such project has been denied a license by the water authorities.\footnote{....}
 
Moreover, the large companies are clearly in a better position to exert pressure on the regulator and to invest in lobbying in order to overcome regulatory hurdles. The fact that large-scale solutions continued to receive priority, despite it being official government policy for a decade that no more large-scale plants should be built, is an indication of the severity of this effect. Hence, while the debate continues regarding the comparative environmental merits of different kinds of hydropower, it appears safe to conclude that the dynamics of power on display in relation to environmental issues raise further doubts about the legitimacy of waterfall takings.

\subsubsection{Regulatory Impact}\label{sec:5:7:7}

When waterfall rights are expropriated, they also become a separate commodity, divorced from the surrounding land rights. They are also typically removed from the sphere of municipal control on land use, falling instead under the regulatory jurisdiction of the centralised water authorities. Hence, the regulatory context shifts from one emphasising holistic resource management and local community needs to one which focuses mainly on facilitating hydropower development.

Moreover, the fact that the takers of waterfalls are powerful actors might make it harder for regulators to do their job. After expropriation, the parties who stand to loose from increased regulation are the state-supported energy companies. They are therefore likely to oppose stricter standards, and to do so in a manner that is much more forceful than any lobbying one might expect from local community owners of waterfalls. Hence, takings in this sector appear likely to cause systemic imbalances and a push for less intrusive government control, or government regulation on terms dictated by the major market players. A sign of this effect can be found in recent controversial decisions made with regard to the national grid, where the interests of the electricity industry appears to have completely overshadowed broad public opposition against further environmental intrusions in valuable nature areas in the west of Norway.

\subsubsection{Impact on Non-Owners}\label{sec:5:7:8}

Non-owners can exercise some influence during the licensing procedure. However, this requires them to be organised or aligned with special interest groups. Organisations, rather than individuals, are entitled to the greatest protection under the \cite{wra17}. The non-owners most directly affected by hydropower development are usually local residents, from the same community as the waterfall owners. These owners have little chance of being heard in the process, except if they find that their interests are aligned with those of more powerful stakeholders, such as national or regional environmental groups. In general, the means available for local non-owners to partake in the decision-making do not appear commensurate with the local stakes in hydropower cases.

The transfer of property to a large-scale owner, moreover, changes the dynamic of interaction between owners and non-owners. Formally, the transfer of riparian rights away from the jurisdiction of municipal governments is particularly significant, since it significantly reduces the level of (local) democratic control over the use of the water resource. In addition, one should again consider the informal effect of transferring property away from local community members to large corporations. Unlike local owners, corporations that take waterfalls appear highly unlikely to interact with local non-owners on equal terms.


\subsubsection{Democratic Merit}\label{sec:5:7:9}

Following \emph{Jørpeland}, it seems that owners' right to participate in decision-making processes regarding the use of their rivers and waterfalls is extremely limited under Norwegian law. The regulatory system effectively negates private property rights by making expropriation an automatic consequence of any large-scale development license granted to any non-owner. The original justification for this might be found in the idea that the regulatory power of the state should take precedence over private proprietary entitlements. However, after the liberalisation of the energy sector, this idea has transformed completely into a practice of systematic prioritisation of powerful commercial interests at the expense of local communities. This has happened despite the political commitment to end large-scale development, which remained official government policy for over a decade.

In light of this, it is especially hard to see any democratic merit in the practice of taking waterfalls for profit, to the benefit of large-scale development companies. Overall, it seems clear that according to the Gray test, current rules and practices regarding takings for hydropower render such takings highly suspect with regard to the question of legitimacy.

\section{Conclusion}\label{sec:5:8}

This chapter has explored expropriation of waterfalls, focusing on the legitimacy issue. The presentation has focused on making the case that property rights have effectively been rendered subservient to the management framework set up by the sector-oriented water resource legislation. Specifically, the chapter tracks a transformation of the regulatory framework whereby the licensing authority is now used by the government to exercise {\it de facto} proprietary control over water resources, unconstrained by the fact that these resources remain in private ownership.

As such, this chapter has shed further light on the tension identified at the beginning of the previous chapter, between hydropower as private property and hydropower as a national asset. Importantly, however, the flavour of this particular ``national asset'' is strongly influenced by the liberalisation of the electricity sector. In particular, this chapter has made the case that takings of waterfalls today are pure takings for profit. Moreover, the government itself does not even feel the need to argue otherwise, since expropriation simply follows automatically from large-scale development licenses.

The chapter used the case of {\it Jørpeland} to shed light on the effect that this can have in practice, showing how owners desiring to carry out alternative projects can be completely marginalised in the decision-making process, regardless of the merits of their proposals. In light of this, based on a combination of concrete and general observations about the Norwegian system, I concluded that this system fails to deliver on legitimacy in the sense of the word explored in Part I of this thesis. 

This chapter has identified the problem, so the question now become how to resolve it. I address this in the next chapter, by considering the institution of land consolidation, which local owners have made very active use of in cases when {\it they} wish to impose economic development on recalcitrant neighbours. Arguably, the consolidation approach represents a solution in line with Ostrom's design principles for local self-governance and common pool resource management. However, the system does have some idiosyncratic features which, if anything, increases the degree of control that stakeholders other than the owners can exercise over the resource in question. This and more is explored in depth in the next chapter of this thesis.


%\chapter{Just compensation}\label{chap:5}

\section{Introduction}\label{sec:into5}

In this Chapter, I consider the question of compensation for waterfalls in more depth. The main issue that arises is whether or not owners should be compensated for the loss of a commercial hydropower potential. If so, the compensation payments can be very large, so large that expropriation will no longer be a feasible option. Traditionally, however, no such compensation was awarded and the amounts paid to owners were negligible. In fact, owners would often have been left with nothing at all, were it not for the fact that a theoretical compensation formula was developed which avoided this outcome, by ensuring some degree of benefit sharing.

The question of whether or not to base compensation on the loss of a commercial hydropower potential is closely related to the so-called ``no scheme'' principle, according to which compensation is to be based on the value of the property such as it would have been if the expropriation scheme had not been authorised. If one takes the view that hydropower development is the prerogative of the party that obtained such an authorisation, it follows from the principle that no compensation is payable to the owners of the waterfalls, at least not for the hydropower potential. The value of hydropower, in particular, is then regarded as being due to the scheme, not due to the natural resource that the waterfall represents. This perspective was implicitly adopted in Norwegian law from the early 20th to the early 21st century, when it began to loose ground due to the liberalization of the energy sector.

The structure of this chapter is as follows: In Section \ref{sec:nsp}, I provide a comparative and theoretical context for the case study that is to follow. I do so by discussing the origin and current status of the no-scheme principle in UK law. This facilitates a broader perspective on the data presented on Norwegian law in subsequent sections.  It also allows me to make a more general point, namely that the need to distinguish between commercial/private and public values inherent in a development project arises with great force when one attempts to apply the no-scheme principle in the context of an economic development taking. I identify the lack of a well-developed framework for making such a distinction as one of the main problems associated with such takings. The main worry is that the principle, when applied to commercial values associated with a development scheme, results in {\it discrimination}. Some categories of owners are entitled to commercial benefits that other categories of owners are effectively deprived of by an application of the no-scheme principle.

In Section \ref{sec:norcom}, I go on to present Norwegian compensation law, with a focus on various manifestations of the no-scheme principle. I also present the special judicial procedure used to award compensation following expropriation. I pay particular attention to the fact that it relies on the use of lay appraisers, who also have considerable influence over the application of the law in such cases. This system, I argue, is potentially very flexible, allowing compensation awards to be based on broad and contextual fairness considerations, rather than static application of special rules. It means, in particular, that the no-scheme principle is not applied without exception, even if it does have status as a general principle. I finish this section by noting that the traditional system has been somewhat undermined since WW2, following legislation specifically aimed at reducing compensation payments and narrowing the room for lay discretion in appraisal disputes.

In Section \ref{sec:nathp}, I move on to consider compensation for waterfalls. The first thing I note is that the no-scheme principle was not traditionally applied, since it would have led to little or no compensation for owners during the monopoly era. Instead, a theoretical method was used, based on the notion of natural horsepower. This method was meant to give owners a share of the benefit in hydropower development and was developed by the appraisal courts early in the 20th century. It was modelled on the market for waterfalls that had existed prior to monopolization, but as the years went by, the method became farther and farther removed from the physical and economic realities of the hydropower industry. Increasingly, it resulted in no more than a symbolic form of benefit sharing with owners.

Following liberalization the traditional method has been abandoned for several categories of cases, a development that I discuss in Section \ref{sec:fa}. The crucial condition for applying the new method, based on market-value assessment, is that an alternative development scheme would have been ``foreseeable'' in the absence of the expropriation project. How to interpret the meaning of ``foreseeable'', and how to determine the scope of the ``expropriation project'', are crucial issues currently being worked out in Norwegian case law. I link the reform in this area with the institutional framework surrounding appraisal disputes. I note, in particular, that the natural horsepower method was first abandoned by the appraisal courts, with the lay appraisers assuming a leading role. 

Unfortunately, the market-based method also raises problems, the most severe of which are related to the scope of the no-scheme principle. The lack of clarity about its scope and implication has created a situation where it appears that if owners are lucky, or employ skilled arguers, they can collect a very substantial sum of money with little or no effort and with no social responsibilities attached. On the other hand, if they are unlucky, they are forced to give up what is often the most valuable asset of their local community for nothing but a symbolic payment. I conclude by arguing that a much better approach would be to try and get owners involved in sustainable hydropower in a way that can remove the need for expropriation altogether. 

This sets the stage for the last chapter of this thesis, where I return to the question of how to replace expropriation by mechanisms of participatory democracy, referring back also to the discussion in Chapter \ref{chap:1}.

%As development is now organized as a commercial pursuit, this should in principle be possible, since the owners {\it do} have an incentive to get involved, also in cases when the public dictate the set of possible terms through strict regulation. In practice, however, what is needed is a mechanism for organizing such owner-involvement. This mechanism will undoubtedly also need to be endowed with powers of coercion if it is to be effective.
%
%The Supreme Court struck down their judgement on a technicality, but refused to reject the principle that lay people were free to adopt a new method in cases when the traditional method would not adequately reflect the value of ``foreseeable'' use. I argue that this shows the strength of a long tradition of respecting the discretion of lay people in appraisement disputes. Many legal scholars, in particular, had previously regarded the natural horsepower method as a {\it rule of law}, set by precedent.
%
%The method has not been abandoned as a matter of principle, however. As made clear recently by the Supreme Court, it is still to be applied in cases when a calculation based on ``foreseeable use'' does not lead to higher compensation payments. The crucial question becomes what exactly is meant by this notion. I address this in some depth, by pointing to how Norwegian law in general is  marked by a tendency to disregard any use that is not sanctioned by public plans, including in cases when these plans themselves provide the rationale for expropriation. This appears to be contrary to the no-scheme principle, demonstrating more generally that only ``one half'' of the principle tend to apply in a Norwegian setting. In so far as the principle precludes giving the owner a share of the expropriation surplus, it is applied, but in so far as it entitles him to compensation based on a future use that is rendered unforeseeable by the planning underlying the expropriation license, it is not.
%
%There are some exceptions to this, however, and the Supreme Court has indicated that one of them applies to hydropower cases. At the same time, however, it has been stressed that even if the expropriation plans themselves are not binding for the compensation assessment, the ``public rationale'' underlying these plans must be taken into consideration when awarding compensation. In effect, this means that compensation is not offered for alternative uses in so far as the project proposed by the expropriating party is superior and could not be undertaken by the current owner. In effect, it seems that a partly {\it subjective} standard is introduced into compensation law, whereby local owners are denied compensation for a commercial value that is deemed to be such that it is only realizable by the expropriating party.
%
%The Supreme Court has not been entirely consistent about the scope and exact content of the ``public rationale'' principle, however,   and the issue is still very much contested in Norwegian courts. In Section \ref{sec:ko}, I illustrate the current unclear state of the law by contrasting two recent Supreme Court cases. In the first, the court embraced an objective version of the ``public rationale'' principle by holding that as the expropriating party's project resulted in more public benefits, compensation could be based on the premise that the owners' foreseeable use of the waterfalls was to cooperate with the large energy company in realizing the plans, to take their share of its commercial potential. 
%
%In {\it Otra II} on the other hand, the Court held that this should not be the conclusion in so far as cooperation was deemed to be ``impractical'', following a concrete assessment of the facts. It seems quite clear that the notion of ``impracticality'', as it was used here, serves to introduce a subjective assessment standard, contrary to what otherwise dictated by Norwegian compensation law. 
%I go on to consider the merits of {\it Otra II} against human rights law, anticipating also the outcome of the appeal currently lodged with the ECtHR in Strasbourg.
%
%Finally, I conclude that the case law on compensation demonstrates the intrinsic inadequacy of a narrow perspective on takings for profit. It seems clear, in particular, that all of the approaches currently in use to calculate compensation for waterfalls leave great room for bickering, manipulation and long-winded court battles. Moreover, the factual premise for the calculation is typically extremely uncertain, meaning that the whole procedure appears as something of a gamble, for both owners and developers. Hence, the developers favour the use of the natural horsepower method, which is completely removed from the reality of hydropower, but deliver predictably low compensation payments that will not prove too damaging to the profit-margin of the development company. On the other hand, owners have an incentive to push for compensation mechanisms that will allow them to collect the entire financial potential of hydropower development without actually investing any effort in planning or administerting such development, and without subjecting themselves to any of the risks involved. 

\section{The ``no scheme'' principle}\label{sec:nsp}

In most jurisdictions, a fundamental principle relating to compensation following expropriation is that compensation should be calculated without taking into account changes in the property's value that are due to the expropriation, or the scheme underlying it. In short, compensation should be based on the owner's loss, not the taker's gain. In a recent Law Commission consultation paper, this principle is referred to as the \emph{no-scheme} rule, a terminology I will also adopt here, noting that while the exact details of the rule might differ between jurisdictions, the underlying principle appears to play a crucial role in both civil and common law traditions for regulating compensation following expropriation.\footnote{I am not aware of a single jurisdiction that does not include some rule corresponding to (aspects of) the no-scheme principle. I mention that in addition to the jurisdictions discussed in this section, no-scheme rules are also found in pure civil law jurisdictions like Germany and the Netherlands, see \cite[5,21]{sluysmans14}.}

While the no-scheme principle is easy enough to comprehend when it is stated in general terms, it raises many difficult questions when it is to be applied in concrete cases. What the rule asks of the valuers, in particular, is quite daunting; they are forced to consider a counterfactual ``no-scheme world'', and they must calculate the value of the property based on the workings of such an imaginary world. The crucial question that arises, of course, is the question of what exactly this world should be taken to look like.

In the first instance, it might be tempting to state simply that this is a ``question of fact for the arbitrator in each case'', as expressed by the Privy Council in \emph{Fraser}, a Canadian case from 1917.\footnote{\cite[194]{fraser17}.} However, as the history of the no-scheme rule has shown, this point of view is not tenable.\footnote{For a history of the rule in UK law, clearly illustrating the difficulty in interpreting it and applying it to concrete cases, I point to Appendix D of \cite{lawcom03}. See also \cite{lawcom01}.}  The problem is that the nature of the no-scheme world cannot be determined without making a vast range of assumptions, many of which appear to depend on how one understands the law. The challenges that arise were discussed in great detail by Lord Nicholls in the recent case of \emph{Waters}. He described the task as ``daunting'', noting also that some of the more recent statutory provisions ``defy ready comprehension''.\footnote{\cite[19]{waters04}.}

\noo{
\begin{quote}
The extreme complexity of the issues that I have had to consider, the
uncertainty in the law, the obscurity of the statutory provisions, and
the difficulties of looking back over a long period of time in order to
decide what would have happened in the no-scheme world
demonstrate, in my view, that legislation is badly needed in order to
produce a simpler and clearer compensation regime. I believe that
fairness, both to claimants and to acquiring authorities, requires
this
\end{quote}
}
The Lords clearly saw \emph{Waters} as an opportunity to offer a clarification on the no-scheme rule and how to interpret it. In particular, their judgement went into more detail than what seemed necessary for the case at hand. Even if it was not needed for the result, the Lords also addressed many of the issues raised by the Law Commission in their recent report, focusing particularly on resolving the tension which was identified there between the principle relied on in the \emph{Pointe Gourde} case and the reasoning adopted in the so-called \emph{Indian} case from 1939.\footnote{\cite{indian39,gourde47}.} In the \emph{Indian} case, the scheme was given a very narrow interpretation, with Lord Romer interpreting the scope as follows.\footcite[319]{indian39}

\begin{quote}
The only difference that the scheme has made is that the acquiring
authority, who before the scheme were possible purchasers only, have
become purchasers who are under a pressing need to acquire the
land; and that is a circumstance that is never allowed to enhance the
value.
\end{quote}

Importantly, this did not entail that the purchaser's demand for the property was to be disregarded, since, as Lord Romer puts it:\footcite[316-317]{indian39}

\begin{quote}
[...] The fact is that the only possible purchaser of a potentiality is
usually quite willing to pay for it […]
\end{quote}

In \emph{Pointe Gourde}, a different stance appears to have been adopted.\footcite{gourde47} The case concerned a quarry that was expropriated for the construction of a US naval base in Trinidad. The quarry had value to the owner as a business, and the valuer had found that if the quarry had not been forcibly acquired, it could also have supplied the US navel base on a voluntary basis, thereby increasing its profits. However, the value of this potential fell to be disregarded, with Lord MacDermott describing the no-scheme rule as follows:\footcite[572]{gourde47}

\begin{quote}
It is well settled that compensation for the compulsory acquisition of
land cannot include an increase in value, which is entirely due to the
scheme underlying the acquisition
\end{quote}

Seemingly, this is at odds with the position taken by Lord Romer in the {\it Indian} case. It seems clear that in the absence of a compulsory purchase order, the US would have been ``quite willing'' to pay for the quarry's services. Still, this potential had to be disregarded. 

In \emph{Waters}, both Lord Nicholls and Lord Scott addressed the tension between the two decisions in great detail. They then offered a reconciliatory interpretation, one which seems to narrow the no-scheme rule compared to how it has most commonly been understood following \emph{Pointe Gourde}. Moreover, the House of Lords also noted the need for reform and legislation, with Lord Scott describing the current state of the law as ``highly unsatisfactory''.\footcite[164]{waters04}

To explain how a seemingly simple principle could become so troubling in practice, I think it is important to start by noting that after the introduction of extensive planning legislation in the 20th century, development of property tends to be contingent on governmental licenses and plans. Moreover, the power to expropriate is often granted as a result of comprehensive regulation of the property-use in an area, often following public plans that encompass more than the particular project that will benefit from compulsory purchase. As a result, it has become increasingly difficult to ascertain what is meant by the ``scheme'' in compensation cases. Does it include the whole planning history leading to expropriation, does it only refer to the power to expropriate, or is it something in between?

A fine balancing act must be made when attempting to answer this question. Under a wide interpretation of ``the scheme'', forcing the valuer to entertain many counterfactual assumptions, the property owner might come to feel that he is not compensated for his true loss, but rather an imaginary one. Indeed, the no-scheme world that the valuer must consider can end up being far removed from the actual one, forcing him to go back many years, perhaps decades, to establish what would have been the status of the property in question if the sequence of planning steps eventually leading to expropriation had not taken place. 

This can leave the property owner in an unpredictable and very weak position. Taken to extremes, the no-scheme principle can then also come to run amiss with respect to human rights law and constitutional provisions protecting private property. On the other hand, if the scheme is interpreted too narrowly, one runs the risk of endangering important public schemes by compelling the public to pay extortionate amounts. In many cases, it is undoubtedly true that the value of property is increased by public investments and plans for the area in which the property is found. Moreover, one may ask if it is right to pay compensation based on increases in value that result from investments and plans that would not have materialised unless the power to expropriate had been anticipated. This, it may be argued, would be a form of double payment that should be avoided.

As noted by the Law Commission, it is important to keep in mind that the no-scheme rule serves at two distinct purposes.\footcite[69-70]{lawcom03} First, the rule has an important \emph{positive} dimension, enhancing compensation payments. Property owners are not only compensated for the direct loss of their property, but also for the possible depreciation of their property's value following the decision to carry out a scheme which requires expropriation. Seemingly, this is easy to justify: It seems intuitively unreasonable if the deleterious effects of a threat of compulsion is permitted to result in reduced compensation payments.

However, under the extensive planning regimes common today, it is not clear where to draw the line. When is the regulation leading up to the scheme to be regarded as reflecting general public control over property use, and when is it to be regarded as a measure specifically aimed at compelling private owners to give up their property? As we will see when we consider the role of the no-scheme rule in Norwegian law, this question can easily become highly controversial, especially when it is linked with the more general question of whether or not the state should be liable to pay compensation for regulation that adversely affects the potential for future development. In jurisdictions that do not recognize owners' right to such compensation, like Norway and England, it is easily argued that the positive aspect of the no-scheme rule must be limited correspondingly. Why should a depreciation of value following regulation imply compensation when the property is eventually expropriated, but not otherwise?

In addition to its positive dimension, the no-scheme rule also has an important \emph{negative} dimension, expressed in {\it Pointe Gourde} as the principle that an {\it increase} in value should be disregarded when it is ``entirely due to the scheme''. The negative dimension has attracted more interest and controversy than the positive dimension, especially in the UK. This is also the aspect of the rule that was at the center of attention in {\it Waters}.

It is not surprising that the negative aspect of the no-scheme principle more often results in complaints, as property owners stand to loose whenever it is applied. However, on a traditional understanding of the public purpose of expropriation, the negative aspect of the rule is also seemingly easy to justify. In \emph{Waters}, Lord Nicholls describes the most important policy reasons as follows:\footcite[18]{waters04}

\begin{quote}
When granting a power to acquire land compulsorily for a particular purpose Parliament cannot have intended thereby to increase the value of the subject land. Parliament cannot have intended that the acquiring authority should pay as compensation a larger amount than the owner could reasonably have obtained for his land in the absence of the power. For the same reason there should also be disregarded the ``special want'' of an acquiring authority for a particular site which arises from the authority having been authorised to acquire it.
\end{quote}

This appears like a reasonable justification. Notice, however, that Lord Nicholls avoids using the word ``scheme''. In particular, he does not identify the scheme's absence as the measuring stick for ascertaining on what basis parliament intends compensation to be based. Rather, Lord Nicholls speaks of what the owner could reasonably have obtained in the \emph{absence of the power} to acquire the land compulsory. In this way, he seems to prescribe a rather narrow interpretation of the negative dimension of the no-scheme rule.\footnote{See also the commentary offered in \cite{newuk}.} It is the power to expropriate that should not give rise to an increased value, nothing at all is said at this stage about the scheme that benefits from it.

It would appear, therefore, that there is nothing in principle that prevents the property from being compensated on the basis of its value in a scheme that differs from the scheme underlying expropriation only in that it does not have such powers. Indeed, this subtle caveat appears to be rather crucial for the remainder of Lord Nicholls' arguments, when he attempts to reconcile the principle adopted in the \emph{Indian} case with the \emph{Pointe Gourde} case.

It would lead me too far astray to go into all the subtle details about the interpretation of the no-scheme rule in UK law and the possible implications of \emph{Waters}. Rather, I would like to focus on one specific aspect, namely the application of the principle when the scheme in question is a commercial enterprise. The UK Supreme Court touched on this issue in the recent case of  \emph{Bocardo}.\footnote{\cite{bocardo10}.} The case was decided under dissent, suggesting that the clarifications offered in \emph{Waters} have not been as conclusive as one might have hoped.

\emph{Bocardo} concerned a reservoir of petroleum that extended beneath the appellant's estate. The petroleum could not be extracted without carrying out works beneath their land. The first question that arose was whether or not extraction of the petroleum amounted to an infringement of property rights. This was answered in the affirmative. The second question that arose was what principle of compensation should be adopted to compensate the owner. The Supreme Court, following some deliberation, found that the general rules applied, meaning that the case should be decided on the basis of an application of the no-scheme principle.

However, opinions differed as to the correct interpretation of this principle, as well as how the facts should be held against the law. The crucial point of disagreement arose with respect to whether or not the special suitability, or \emph{key value}, of the appellant's land, \emph{pre-existed} the petroleum scheme.

In \emph{Waters}, the House of Lords had cited and expressed support for the following passage, taken from Mann LJ's judgement in \emph{Batchelor}.\footnote{\cite[361]{batchelor89}. Cited by Lord Nicholls at \cite[65]{waters04}.}

\begin{quote}
If a premium value is ``entirely due to the scheme underlying the acquisition'' then it must be disregarded. If it was pre-existent to the acquisition it must in my judgement be regarded. To ignore the pre-existent value would be to expropriate it without compensation and would be to contravene the fundamental principle of equivalence.
\end{quote}

%(see \emph{Horn v Sunderland Corporation})
Relying on this distinction between the potentialities that are ``pre-existing'' and those that are due to the scheme, the minority in \emph{Bocardo}, led by Lord Clarke, made the following observation.\footcite[42]{bocardo10}

\begin{quote}
Anyone who had obtained a licence to search, bore for and get the petroleum under Bocardo’s
land would have had precisely the same need to obtain a wayleave to obtain access
to it if it was not to commit a trespass. So it was not the respondents' scheme that
gave the relevant strata beneath Bocardo’s land its peculiar and unusual value. It
was the geographical position that its land occupies above the apex of the
reservoir, coupled with the fact that it was only by drilling through Bocardo’s land
that any licence holder could obtain access to that part of the reservoir that gives it
its key value.
\end{quote}

This view was rejected by the majority, led by Lord Brown, who interpreted the no-scheme rule quite differently:\footcite[83]{bocardo10}

\begin{quote}To my mind it is impossible to characterise the key value in the ancillary
right being granted here as ``pre-existent'' to the scheme. There is, of course,
always the chance that a statutory body with compulsory purchase powers may
need to acquire land or rights over land to accomplish a statutory purpose for
which these powers have been accorded to them. But that does not mean that upon
the materialisation of such a scheme, the ``key'' value of the land or rights which
now are required is to be regarded as “pre-existent”.
\end{quote}

While the case was resolved in keeping with this view, the dissent suggests that the clarification in \emph{Waters} has not resolved all issues. Moreover, it suggests that special questions arise when the expropriation scheme itself involves the realisation of a commercial potential inherent in the land that is taken. Is it permissible for government to grant the value of this potential to the taker -- by granting him the necessary licenses -- without subsequently recognizing the potential as having been taken from the owner? 

This issue does not \emph{not} primarily depend on the scope of the scheme as such. In {\it Bocardo}, for instance, it was obvious that the scheme was the entire project aimed at extracting petroleum from the reserve, including the necessary works beneath the appellant's estate. But even so, it was still unclear whether the special value of the appellant's land could be said to have been {\it caused} by the scheme. The issue that arises in these kinds of situations is ontological: When should we attribute a given value to an act of government, and when should we attribute it to nature, as a fruit of the land? Or in more practical legal terms: When is a given property value that is unlocked by a development scheme part of the original owner's bundle of rights?

To answer this question, it is tempting to look for a causal link between scheme and value, to substantiate the claim that the value was not in fact pre-existent. But as \emph{Bocardo} illustrates, it is not always obvious what should be taken as good evidence for such a link. It seems that one's perspective on this will tend to depend also on one's point of view on the much more general question of what values one recognize as inherent in property rights.

When Lord Clarke remarked that the state, following nationalisation in 1934, could have given the right to extract the petroleum to \emph{someone else}, he was certainly correct. Hence, I also agree with him that ``the key value was not created by the 1934 Act or the grant of the petroleum licence to Star''.\footnote{See \cite[163]{bocardo10}.} But whose value was it, and was it a commercially realisable value? Here, Lord Clarke appears to assume that the value must belong to the property owner and that this owner would also have been able to make a profit from it in the absence of the expropriation scheme. This, I believe, is a leap that requires further justification. Just because some property has key value does not mean that the owner of the property is entitled to that value, or that it can ever be translated into a financial profit.

On the one hand, it is easy to agree with Lord Clarke that compulsory acquisition of a wayleave is no precondition for an extraction scheme. The project could well have been carried out by a developer who was willing to pay the owner for the special suitability of his land. But on the other hand, it does not seem obvious that the owner is meant to be able to demand such payment under the regulatory system currently in place. Hence, even in the absence of a causal link between scheme and value, one might be entitled to conclude that the special value falls to be disregarded because it has already effectively been removed from the owner's bundle.

In the case of {\it Bocardo}, I think this perspective would have been particularly helpful to Lord Brown, who argued that the value of the strata was not pre-existent. As it stands, his argument seems rather strained. After all, it was the physical conditions that gave the land its value, not the abstract fact that a development license had been granted. However, by looking at his argument in more depth, it is tempting to rephrase his conclusion by saying that he regarded the special suitability of the strata as having no commercial value under the prevailing regulatory regime.

In the end, I am agnostic about the correct way to judge {\it Bocardo}, but I think the crucial question that it raised was the following: did parliament intend to give petroleum developers a right to extract substrata resources without sharing the profits with affected surface owners? If no clear answer is available, conflicts can result, particularly if the question itself is obfuscated, as I think it was in {\it Bocardo}. It seems to me, in particular, that the focus on causality and the notion of ``pre-existence'' was not very helpful. Rather, I think the crucial keyword should have been benefit sharing.

The first question to ask in this regard is what parliament intended when it set up the current regulatory framework. If this is unclear or the evidence suggests that benefit sharing was not intended, the question becomes whether or not benefit sharing is nevertheless required on the basis of constitutional or human rights law. In a case like {\it Bocardo}, the latter question is unlikely to arise with any great force. It seems to me, in particular, that the question of how to deal with a property's ``key value'' in relation to other property is usually a question that can be resolved merely by pointing to the legitimate public interest in avoiding unwanted holdouts.

Even so, if the courts engage with the question of benefit sharing without being explicit about it, the lack of democratic accountability can become a worry. I think it is important to emphasize the political sensitivity of the range of complex rules found in compensation law. If not, a crisp political question risks becoming obfuscated to the extent that it can only be engaged with in a meaningful way by legal professionals. This, in turn, increases the chance of abuse and undue influence of special interest groups. While most people remain ignorant of the political work done by the courts in this regard, those who stand to gain the most are free to lobby and argue on technical points to gradually shape the law of benefit sharing according to their own interests. A conceptual shift might be needed to prevent this development from becoming precarious to the legitimacy of compensation law in general, and the no-scheme rule in particular. 

In addition, the question becomes much more pressing in cases when the development potential as such is subject to expropriation. An extreme case arises when natural resources are expropriated. For an illustration which also links up to my case study, I mention particularly the cases of \emph{Cedars} (1914) and \emph{Fraser} (1917), two Canadian compensation disputes regarding expropriation for hydropower. They were cited as important authorities by both the Law Commission and the House of Lords in \emph{Waters}.\footnote{\cite{cedars14,fraser17}.} 

In \emph{Fraser}, it was the waterfalls themselves that were subject to expropriation, yet the Privy Council still found that the value of the potential for hydropower exploitation of these falls should be disregarded when compensating them. The reasoning adopted seems to follow a standard ``value to the owner'' approach. However, reflecting back on {\it Bocardo}, it is hard to see how anyone could think that the value of the waterfalls were not ``pre-existent'' to the scheme to develop them. Surely, as a natural resource, a waterfall has significant value in itself, independently of any particular ``scheme''? 

Not so, according to the Privy Council, who found that the owners of waterfalls could not themselves have developed hydropower. Here, a subjective standard was in effect employed, whereby the bundle of rights associated with a property depended not only on the property itself but also on the nature of its owner. This unequal treatment of owners is such that is could, in my opinion, now be attacked from the point of view of human rights and constitutional law.\footnote{Although such an approach might not be required to overrule them, as the Canadian cases already appear to be at odds with both {\it Waters} and {\it Bocardo}.}

However, in order to make such an attack, it is necessary to use a working distinction between commercial and non-commercial aspects of a development scheme. The pre-existence test is inadequate. For instance, there can be no doubt that the energy inherent in water pre-exists any scheme seeking to harness it. Moreover, it seems clear that energy has great value, meaning that the value of a waterfall pre-exists any scheme for hydropower exploitation. However, we must also ask: what \emph{kind} of value is it?

To illustrate why this is a relevant consideration, consider a case where the property value is enhanced for the owner because of a personal attachment. In this case, it seems fair to differentiate, so that the owner's subjective attachment to the property is taken into account, potentially leading to a higher compensation payment then any other owner would receive. It is irrelevant, moreover, whether or not the particular aspect of the property to which the owner is attached is pre-existing. The relevant consideration is simply whether or not the value in question is such that one thinks it {\it should} be compensated. The value is {\it not} commercial, however, but personal (and, in so far as it receives recognition, also public). This is {\it why} differential treatment becomes justifiable. 

Similarly, in so far as a piece of land is particularly suited for building a school, it seems unproblematic to deny benefit sharing with the owner. In this case, the suitability is pre-existent, but it reflects a value to the public, not to commerce. Hence, a disregard rule can safely be applied, even though the public would been willing to pay large amounts in friendly negotiations. But what if the land was not particularly suited for a school, but for a shopping mall? Here I believe a different standard is needed. It seems, in particular, that benefit sharing is required in this case since one would otherwise illegitimately discriminate between owners. Why should the owners of shares in a shopping mall be allowed to profit, when the owners of the suitable land are not?

As a practical test, I propose the heuristic whereby one regards the commercial value of the development as evidence that disregard rules like the no-scheme principle should not be applied. The underlying rationale behind this heuristic is based on the public interest requirement. It seems to me, in particular, that disregard rules are also in need of justification based on the needs of the public.
In my opinion, the public interest/purpose requirement extends to compensation in such a way that a value needs to be identified as a public value in order for it to be legitimate to disregard this value when compensating the owner. 

More generally, I fail to see how it could ever be legitimate to apply a no-scheme principle unless it serves the public good. If the principle is applied in a way that results in a commercial benefit to the taker and a commercial loss to the owner, I would argue that it renders the expropriation as a whole unsafe in relation to the public interest requirement. One aspect of the interference, at least, then lacks proper motivation. From this I arrive at the general conclusion that values which are recognized as commercial should never be disregarded.

The distinction between commercial and public values is obviously not written in stone, but is down to a political decision. Moreover, it can hardly be regarded as permanent. In addition, it can often be difficult to assess where the line is to be drawn, especially in cases when public-private partnerships are relied on to provide public services. Nevertheless, it seems to me that the public interest requirement in constitutional and human rights law makes it necessary to be explicit about private and public values also in relation to compensation. Moreover, it seems like doing so could be very helpful in many cases, such as {\it Bocardo} and {\it Fraser}.

For instance, even if the public value of hydropower pre-exists an hydropower scheme, this does \emph{not} necessarily mean that there is any pre-existent commercial value in hydropower. What counts as {\it commercial} value, in particular, must first be answered. This, moreover, depends entirely on whether or not the public has settled on a regulatory regime that allows commercial exploitation.
Hence, I arrive at the following suggestion for a modified version of the ``pre-existence'' test: An owner should always be compensated for the value of any pre-existent \emph{commercial} value that his property has.\footnote{Certainly, a clarification along these line would not resolve all issues. It would not, for instance, offer any conclusive guidance with respect to the specific issues related to "key value" raised in \emph{Bocardo}.} 

To answer the question of what should be regarded as a pre-existing commercial value, one must take a broad look at the prevailing regulatory regime. Moreover, one must expect that the assessment will depend on the context of regulation, in particular the extent to which the state \emph{allows} the disputed value to be commercially realized. The law relating to compensation should be such that it can tolerate significant changes in these parameters. The theoretical question that arises concerns only the conceptual foundation for the assessment. The actual lines that must be drawn are all drawn in the sand, as usual.

In the next section, I will address Norwegian compensation law to shed light on some such lines that have been drawn in relation to waterfalls, which have recently been washed away and redrawn following liberalization of the hydropower sector. This will allow me to shed light both on the no-scheme rule and alternatives to it. 

%Moreover, I note how the Norwegian system was originally based on a rejection of the idea that all disputes had to be resolved uniformly on the basis of a battery of specific rules. Instead, great emphasis was placed on the discretion of lay people. In later years, however, Norwegian compensation law has developed along a similar trajectory to that of the UK. The no-scheme principle, in particular, has now been addressed in so many different ways and by some many different sources of authority that it appears just as much in defiance of ready comprehension as in the UK before {\it Waters}. I will now try to untangle the web somewhat, while moving towards the special points I would like to make based on the idiosyncratic case of waterfalls. 
%
%But the assessment itself was above all else discretionary. Legitimacy of the process was ensured in a bottom-up fashion, by the involvement of lay people sitting as appraisers, alongside a regular judge.
%
%I note, however, that this system has largely been modified so that, today, the appraisal courts are far more constrained from the top down, by legislation and the Supreme Court. I argue that this has been a source of difficulty for the system, particularly in relation to the no-scheme principle which, as I argued above, necessitates a concrete assessment, and can not -- should not -- be resolved by all-encompassing principles. I note, in particular, how the increasingly constrained room for discretion by lay people means that the distinction between commercial and public value -- which must now be determined centrally -- becomes muddled. In many cases, the idea acting as a premise for the general rules applied simply does not correspond to reality. This often leads to unacknowledged commercial windfalls for takers, arising when owners are denied compensation for commercially valuable rights that the law presupposes to be wholly public, even though they are not. 

\section{Appraisal courts and ``foreseeable alternatives''}

The owner's right to compensation following expropriation of property is enshrined in very simple terms in Section 105 of the Norwegian Constitution.\footnote{\cite[105]{grunnloven14}.} This section states simply that \emph{full compensation} is to be paid in all cases when the public interest warrants the compulsory acquisition of property. For more than 150 years, this was the sole legislative basis for compensation rules in Norway. The methods used to calculate full compensation in different scenarios developed entirely through case law.

According to a long legal tradition in Norway, the discretionary aspects of property valuation is regulated by a special procedure, with a significant reliance on so called \emph{unwilling appraisers}. These are members of the public who have no interests in the case at hand. They may be chosen, however,  specifically for their suitability in judging the value of the contested property, either because they are resident in the local area or because they have special expertise.

The appraisal procedure has a long history, going back to customary law that pre-dates the constitution. The rules regulating it today are found in the \cite{aa17}.\footnote{Act no 1 of 1 June 1917 relating to Appraisal Disputes and Expropriation Cases.} Appraisal cases are organised similarly to civil disputes, and the procedure is administered by the district courts.\footnote{\cite[5]{aa17}.} Appraisal courts are usually composed of a panel consisting of one professional judge and four appraisers, with no special juridical qualifications. 

The standard arrangement is that appraisers are chosen from the general public in the district where the property in question is located. But the Act opens up for the possibility that appraisers may also be chosen for their special technical expertise.\footnote{See \cite[11|12]{aa17}.} Their role in the procedure is on par with the judge, and the panel decides jointly both the legal and the technical questions, usually on the basis of technical reports put forth by the parties. These reports are presented during the main hearing, and may be challenged by the parties, in more or less the same way as the district court hears evidence in a regular civil dispute.\footnote{See particularly \cite[22|27]{aa17}, with further references to the \cite{da05} (Act No 90 of 17 June 2005 relating to the Mediation and Procedure in Civil Disputes).} 

There is a possibility for appeal to the appraisal Court of Appeal, which is the regular Court of Appeal sitting as an appraisal court in accordance with the rules of the \cite{aa17}. The right to having an appeal heard is not absolute; whether the appraisal Court of Appeal will hear the case depends on its importance, according to rules that correspond to those in place for regular civil disputes.\footnote{See \cite[32]{aa17}.} The procedure closely corresponds to the procedure followed in appraisal disputes at the district level.\footnote{See \cite[38]{aa17}.} However, the decision made by the appraisal Court of Appeal is final as far the appraisal assessment is concerned. An appeal to the Supreme Court can only be accepted on legal grounds.

As a consequence of this, the appraisal courts have been very important in interpreting and developing the law relating to compensation in Norway. Their importance was particularly great all the while the meaning of ``full compensation'' was not clarified further in statute. The presence of lay people sitting as judges is consistent with how many civil disputes are resolved. But what makes the appraisal courts unique is that in these cases the lay people where traditionally allowed to engage with the issue under very few restraints, beyond procedural rules and the words of the Constitution.

At the same time, the practical viewpoint enforced by the procedural form meant that legal questions would often remain in the background in such cases. Typically, the legal issues would only come to the forefront if the Supreme Court decided to hear the case as a matter of principle. 

The primary criticism voiced against this system, particularly following the Second World War, was that it gave the appraisal courts too much discretionary power. Hence, the argument went, legislation was needed to make the outcome of appraisal cases more predictable.\footnote{See, for instance, Part 2, Chapter 1 of \cite{nut69}, handed over to the Ministry of Justice by the so called Husaas committee, appointed by the King in Council 6 Aug 1965.} However, while the law regarding compensation was not formalized in written form, there were legal scholars who developed theories and aimed to explicate its content based on the body of case law that was available.

Also, the Supreme Court did regularly hear cases concerning legal arguments regarding compensation, and they developed a consistent position on at least some of the more critical and recurring legal issues. At this time, the central source of legal reasoning regarding appraisal was still to be found in the constitution itself. As a result, theories of compensation law tended to be \emph{absolutist}, in the sense that they looked directly to wording in Section 105, also when tackling specific problems of interpretation. 

\subsection{Constitutional absolutism}

Absolutism was widely endorsed by Norwegian legal scholars as late as in the 1940s. The well-known legal scholar Magne Schjødt summed it up as follows:\footcite[177]{schjodt47}

\begin{quote}
When an owner is entitled to compensation, he is entitled to have his full economic loss covered. He should receive full compensation, see p 42 ff. This is the great principle that remains absolute and any dispute must be resolved on its basis.
\end{quote}

A typical example of the style of legal reasoning that this view gave rise to can be found in thes writings of the prominent legal scholar Frede Castberg. He specifically addressed also the no-scheme principle, by asking about the extent to which increases in value due to the scheme underlying expropriation was to be taken into account when calculating compensation. His reasoning in this regard was based directly on a reading of the Constitution. Moreover, it was based on the principle of \emph{equality}, which was at that time considered particularly crucial in understanding constitutional law. The following quote serves to sum up Castberg's position on the no-scheme principle:\footcite[268]{castberg64b}

\begin{quote}
The owner is entitled to full compensation. The expropriation should not leave him worse off economically than other owners. Hence if the public has knowledge that an industrial undertaking is being planned, that a railway will be built etc, and this affects the value of property generally in a district, then the increased value of the property that will be expropriated must be taken into account. If not, the owners of such property will be worse off than other owners from the same district. On the other hand, if the expectation of the scheme underlying expropriation leads to a general depreciation of value, then it is this new value -- not the original value -- that is relevant for calculating compensation. The crucial question is what the actual value is, when expropriation takes place.
\end{quote}

% e mention that the problem analyzed by Castberg in this passage has been considered in many jurisdiction, and is dealt with in common law by the so called \emph{no-scheme} rule. This is more a principle than a single rule, and it is typically understood as a mechanism that is meant to ensure that changes in value due to the scheme underlying expropriation are disregarded.\footnote{For an history of the rule in common law (primarily the UK), which also illustrates the difficulty in interpreting it and applying it to concrete cases, we point to Appendix D of Law Commission Report No 286, 2003} In comparative terms, Castberg appears to favor a \emph{narrow} interpretation of the principle -- a restrictive view on when additional value due to the scheme should be disregarded -- quite close in spirit to the so called \emph{Indian} case from 1939\footnote{\emph{Vyricherla Narayana Gajapatiraju v Revenue Divisional Officer, Vizagapatam} [1939] AC 302.}, which was been much discussed in common law and was dealt with extensively by the House of Lords as late as in 2004.\footnote{In the case of \emph{Waters and other v Welsh National Assembly} [2004] UKHL 19. 
%The primary precedent for a broader interpretation of the non-statutory no-scheme rule, on the other hand, is \emph{Pointe Gourde}, \emph{Pointe Gourde Quarrying and Transport Co v Sub-Intendent of Crown Lands} [1947] AC 565, PC, 572, per Lord MacDermott. This case proved highly influential for the understanding of compensation rules in the post-war period, in many common law jurisdictions, but has recently been challenged by a renewed interest in more narrow viewpoints such as that expressed in the \emph{Indian} case, see  \cite{newuk} and also the case of \emph{Star Energy Weald Basin Limited and another (Respondents) v Bocardo SA (Appellant) [2010] UKSC 35}.}In the context of Norwegian law, it is of particular interest to note how Castberg's views in this regard is arrived at through considering the constitution itself, founded on the principle of equality.\footnote{In this way, he arrives at a narrow no-scheme rule quite abstractly, and through a different route than the one adopted in the \emph{Indian} case, where the outcome appears to have turned crucially on the particular facts in the case, a close reading of precedent, as well as the perceived fairness of the result.}

As Castberg bases his analysis on the exact wording of the Constitution, he does not engage in any reasoning based on the extent to which it can be regarded as socially fair for the public to pay compensation for value that arise due to the beneficial consequences of the public project itself. Crucially, he does not address the concern that this can be seen as a form of double payment. Such pragmatic and utilitarian points were not widely used to interpret the law in the legal tradition Castberg was part of. This, in particular, is why I think it is appropriate to use the label of constitutional absolutism to describe this kind of reasoning.

However, it is not correct to think that such reasoning is necessarily ``owner-friendly''. To see this, it is enough to note that Castberg, in the quote above, explicitly states that depreciation of value due to the scheme should not be disregarded. In addition, Castberg did not intend to reject the no-scheme principle altogether. In particular, he explicitly denied that owners of expropriated property should ever be able to claim compensation based on the special want of the acquiring party:\footcite[268]{castberg64b}

\begin{quote}
The situation is different if the property has increased value due to the expectation that it will be expropriated. The owner can not demand that this increase is compensated since that would be the same as giving him a special advantage compared to those from whom no property is expropriated.
\end{quote}

Hence, Castberg accepts a narrow version of the no-scheme principle, similar in spirit to that presented by Lord Romer in the {\it Indian} case. Castberg's view appears to have been shared by many academics of his day, and it was also largely reflected in case law from the Supreme Court.\footnote{See below.} At the same time, the very nature of the system for deciding appraisal disputes gave the local appraisers great freedom in adapting the principles in a way that suited the concrete circumstances.

To some extent, this would also involve making an assessment of what was regarded as a fair and just outcome. Hence, while the theory of the time was absolutist, case law was more multi-faceted. Importantly, fairness was seen as a concrete issue that had to be addressed on a case by case basis, an approach that would not necessarily lead to general rules. The Supreme Court largely sanctioned this approach, by passively respecting the discretion of the appraisal courts, as vested in them within an absolutist theoretical framework.

But as long as they did not cross the line with regards to the constitution, the appraisal courts were largely allowed to adopt broader viewpoints as well. The point was, however, that such viewpoints were \emph{not} extensively codified in terms of special principles used to deal with special case types or issues. Rather, they arose as a logical consequence of the way in which appraisal disputes were organized. Social justice and fairness perspectives were not excluded, but could in fact play an important role in practice.. However, such perspectives arose \emph{indirectly} through a \emph{decentralized} system which gave local courts great freedom when applying the law.

The way in which the no-scheme principle was applied serves as a nice illustration of this. On the one hand, the theoretical views of Castberg were widely accepted, but at the same time they were regarded as no more than guidelines that had to be adapted to the circumstances. Moreover, it was not unheard of for the appraisers to disagree with the judge about how this should be done, and to award compensation according to a different understanding of the law than that favoured by the judge. 

This happened, for instance, in the case of \emph{Tuddal}, where land was expropriated for construction of a power grid.\footcite{tuddal56} In relation to this, the expropriating party also acquired the right to use a private road. According to the juridical judge in the appraisal court of appeal, and consistent with the teaching of Castberg, compensation should be awarded solely on the basis of what the owners stood to lose. In this case, that would mean compensation based on the increased cost in maintaining the road resulting from increased use. However, the lay appraisers found this result unreasonable and awarded compensation also for the special value the use of the road would have for the acquiring party. The Supreme Court, although they found fault with the reasons given by the law appraisers, agreed that such compensation was possible in principle. The presiding judge offered the following perspective:\footcite[111]{tuddal56}

\begin{quote}
Since they were the private owners of the road, A/S Tuddal could, before the expropriation, refuse to let the Water Authorities make use of it. Hence it might be possible for A/S Tuddal, through negotiation and voluntary agreement with the Water Authorities or others with a similar interest, to demand a reasonable fee, and in this way achieve a greater total benefit than full compensation for damages and disadvantages. Following the expropriation, it is no longer possible for A/S Tuddal, in its dealings with the Water Authorities, to economically benefit from their ownership of the road in this way. If the company suffer an economic loss as a result of this, I believe they are entitled to compensation. Whether or not such an opportunity as I have mentioned -- all things considered -- was present at the time of the expropriation, falls to the appraisal court to decide, on the basis of whether or not an economic loss is suffered beyond that which follows from damages and disadvantages. On this basis, I assume that the appraisal court of appeal's decision to awarded compensation for the value of the right of way that is acquired can not -- in and of itself -- be regarded as an erroneous application of the law.
\end{quote}

The Supreme Court's reasoning illustrates two points. First, we see how the Supreme Court adopts absolutism in its interpretation of the law. Through careful use of wording, the compensation premium is not conceptualized as compensation based on the value of the road to the acquiring authority, but rather as compensation for the loss of potential profit following from a voluntary agreement. Hence, a seeming contradiction with the no-scheme principle is avoided.\footnote{This particular interpretation of full compensation led to arguments in the post-war period, regarding whether or not owners had a right to compensation based on the loss of profit from hypothetical voluntary agreements with the acquiring party. In the end, a consensus formed that this type of compensation should not in general be awarded. See \cite{nut69},Part 2, Chapter 4, Section 2.E.}

But {\it Tuddal} also illustrates a second important point, namely that the Supreme Court was prepared to defer greatly to the judgement of the appraisal court. It is stated explicitly that it falls for this court to decide whether or not the opportunity to profit from the road by negotiating with the expropriating party was present at the time of expropriation. This is particularly noteworthy in light of the dissent of the juridical judge in the appraisal court of appeal, and in light of the dominant legal theorizing of the day, which did  not seem to support the idea that a premium should ever be paid in a situation like this. Hence, the decision tells us that the Supreme Court went far in defending the discretion of the laypeople, as a \emph{systemic} feature. They tested with great caution whether it was truly outside the permissible legal boundary, but concluded that it should simply be regarded as an exercise of the lay judgement that the system presupposed.

This impression of the case is accentuated when considering other cases dealing with similar issues. Across the board, I note a strong  tendency to defend the role of the laypeople in the appraisal process. A particularly clear expression of this can be found in \emph{Marmor}, also from 1956, where the Supreme Court overturned a decision made by the appraisal court of appeal on the grounds that the court had been {\it too} reliant on general principles.\footnote{\cite{marmor56}.} This, the Supreme Court held, offend against both the principle of full compensation and the principle of discretionary evaluation by laymen.

The case involved expropriation of a private railway track, for the construction of a public railway. It was clear that the track which was being expropriated did not have market value in general, so the expropriating party argued that the value of these tracks to the public railway should not be taken into account when calculating compensation. The appraisal court of appeal agreed, pointing to the standard teaching of the day. The Supreme Court, on the other hand, struck down the decision because they felt that a standardized approach to the case was inappropriate given the circumstances. The presiding judge argued as follows:\footcite[498-499]{marmor56}

\begin{quote}
In my opinion one can not simply assume that a property does not have market value when it has no value for anyone other than the expropriating party. The question needs to be assessed concretely. I agree with the expropriating party -- as has also been confirmed on several occasions by the Supreme Court -- that in general one should not take into consideration the special value that the purpose of expropriation gives the property. This should not lead to a spike in compensation payments. On the other hand, I can not agree that it is automatically reasonable, or in keeping with Section 105 of the constitution, if the expropriating party in cases like the present one could acquire property at a price that is below what it would be natural and commercially appropriate to pay in a voluntary purchase.
\end{quote}

Again, I note the two main building blocks used in the argument: First, the standard reference to the constitution, and secondly, a reference to the need for \emph{concrete assessment}. This further reflects the strong confidence that the Supreme Court had in the integrity and autonomy the appraisal procedure. Moreover, I notice how, in this case, absolutism regarding the constitutional protection of property is \emph{not} used to argue for specific rules or principles, but rather to back up the argument that compensation should result from real assessment, and not be overly reliant on such rules. In the case of {\it Marmor}, this was the outcome even if the rules in question had the status of valid guidelines that had also been backed up by a series of Supreme Court decisions.

In addition to making the overreaching remarks quoted above, the Supreme Court also gave pointers as to the kinds of facts that should be considered. For instance, the presiding judge paid particular attention to the wider \emph{context} of expropriation, and the manner in which expropriation was used to benefit certain interests. He also noted how expropriation had come to replace voluntary agreement as the standard means of acquisition for this type of development. Therefore, the practice of using expropriation effectively prevented a market from developing, a market that might otherwise have appeared naturally:\footcite[499]{marmor56}

\begin{quote}
I also point to the fact that the case concerns an area of activity where the expropriating party has a {\it de facto} monopoly which makes it impossible for anyone else to make use of the property for the same purpose. This in itself makes it questionable to simply assume that the lack of financial value for other purchasers provides the appropriate basis for calculating compensation. When considering this question, it is also appropriate to take into account that we have lately seen a great increase in the use of expropriation to undertake projects such as this. Compulsion is becoming the primary mode for acquisition of property -- not voluntary sale following friendly negotiations.
\end{quote} 

In my opinion, the importance of this decision, which makes it highly relevant even today, is not that it seems to favour a narrow interpretation of the no-scheme principle. In fact, I think it is erroneous to read the judgement as expressing general support for any particular interpretation. In addition, I do not think the decision can be read as supporting a general principle by which compensation can always be based on the value of hypothetical agreements that could have been made with the expropriating party. Rather, I take the judgement to be an expression of scepticism towards blind obedience to \emph{any} set of detailed rules for calculating compensation that serve to limit the room for lay discretion.

At the very least, it seems clear upon closer inspection of the argument that the main objective of the Court was not to express any particular view regarding the content of the no-scheme principle, but rather to instil to the appraisal courts that they could not use this rule as an excuse not to engage in concrete assessment to ensure a reasonable outcome in keeping with the constitution.

I believe this point is important to stress. It illustrates how absolutism need not, and did not, result in a rigid system with little room for assessment based on justice and fairness, broadly conceived. Quite the contrary, the absolutism endorsed by the Supreme Court, and inherent in the Norwegian system of appraisal courts, was not characterized by blind obedience to specific rules, like those proposed by Castberg. Rather, the system was flexible, and it was explicitly intended to function such that fairness assessments based on concrete circumstances could be accommodated.\footnote{Going back to even older legal scholarship, we see that this view on the meaning of absolutism has a long history in Norway. It is present, for instance, in the work of the famous 19th century scholar Aschehough, who stressed the link between the constitution and the appraisal procedure when he considered the (then) hypothetical situation that legislation would be introduced with the specific aim of reducing the level of compensation payments following expropriation. See \cite[48]{aschehough93} 

%\begin{quote}
%If it becomes common practice to award compensation payments that are unreasonably high, this would make important public projects more expensive and difficult to carry out, greatly to the detriment of society. In many cases it might not be possible to rely on legislation to prevent such excessive compensation payments, since this would restrict the appraisers too much. To some extent this might be possible, however, and as far as it goes, parliament must be permitted to do so. However, if enacted rules clearly lead to less than full compensation in an individual case, they will be overruled by Section 105 of the constitution, and fall to be disregarded in that particular case.
%\end{quote}
%
%This quote is important because it does not rely on any particular interpretation of the constitutional demand for full compensation, but sees this inherently as an issue that needs to be resolved by concrete assessment of individual cases. Absolutism to Aschehough implies freedom and responsibility for the appraisers; freedom to judge individual cases by its merits, and a responsibility to award full compensation, irrespectively of any specific rules that might be in place to curtail excessive payments. The important point is that Aschehough here sees absolutism as a principle that should be applied to cases, not to principles. He does not argue that rules introduced to limit compensation payments would be inadmissible merely because they might sometimes suggest less than full compensation. Rather, he takes it for granted that it falls to the appraisal courts to apply the rules in a way that would prevent such outcomes. As long as the appraisal courts remain free to apply the rules in such a way that full compensation is awarded, specific rules intending to prevent excessive payments can happily coexist with absolutism.
%
%The subtle view taken by Aschehough was largely overlooked in debates following the introduction of the Compensation Act 1973, which served to introduce radical rules of exactly the kind he had predicted and considered 80 years earlier. The consequence was, as I will discuss in more depth in the next section, that the Supreme Court was forced to actively steer the interpretation of the Act to ensure that section 105 would not be violated in concrete cases. Hence, the introduction of legislation served to destabilize the system, by narrowing the room for lay judgements and increasing the reliance on legislation and special principles developed by the Supreme Court. This development is the subject of the next subsection.

%More generally, the 60s and 70s appears to be a period when the crucial role of the appraisal procedure was to some extent forgotten, and also undermined, following a heated political and ideological debate regarding the appropriateness and admissibility of introducing rules to ensure that compensation payments were brought down to a lower level. This had deep and lasting effects on Norwegian compensation law, and it is popularly described as a period when the social democrats won recognition for the principle that social fairness suggested the introduction of compensation rules and disregards that were more extensive than what had previously been considered appropriate. 
%
%This was conceived of as a fight for social justice against outdated and conservative ideas of constitutional absolutism. But it seems to us that this view of the history of Norwegian compensation law is erroneous, and largely unhelpful. The approach taken by Aschehough, in particular, placing emphasis on the important role played by the appraisers in achieving fairness and justice in concrete cases, does not appear to contradict social democratic goals at all. In fact, it seems that his approach might be better suited to serve such goals, and to accommodate a variety of different political opinions and ideas, than an approach which is based on attempting to flesh out in painstaking detail how the appraisal courts should go about achieving the balance between social fairness and owners' rights. We will return to this point later, but first we will take a closer look at the history of the radical Compensation Act 1973 and the censorship to which it was subjected by the Supreme Court, leading to the Compensation Act 1984, currently in place.

\subsection{Pragmatism}\label{sec:prag}

Following the Second World War, the social democratic \emph{Labour Party} gained a secure grip on political power in Norway. As a result, many reforms were carried out that would reshape Norwegian society. One of the most important reforms concerned the introduction of extensive planning law to ensure that land use was put under public control.\footnote{See generally \cite{thomassen97,kleven11}.} As a result of this, the period also saw expropriation being used more extensively to further public projects, such as the large scale construction of hydropower to ensure general supply of electricity.\footnote{See generally \cite{skjold06,thue06b}.} As a result of these changes, the opinion was soon voiced that there was a need for a more uniform approach to compensation, which collected some basic principles in a common body of written law. In addition, it was an explicitly stated political goal to bring compensation payments down.

In 1965, the so called \emph{Husaas committee} was appointed by the King and charged with the task of assessing the compensation rules currently in place.\footnote{Appointed by the King in Council on 6. Aug 1965.} The committee was also ordered to make a concrete suggestion regarding the need for additional principles of compensation, and to consider if these should be given in the form of a special compensation act. Initially there was some doubt as to the extent to which is was at all permissible to give rules regulating compensation, as the constitution itself addressed the matter. 

However, the committee noted that some rules had already been introduced for specific case types, for instance in relation to expropriation for hydropower development.\footnote{As discussed in Chapter \ref{chap:..}, Section \ref{sec:...} in relation to the \cite[16]{wra17}.} In addition, legal scholars of the day were generally of the opinion that compensation rules could be given, on the understanding that the courts would deviate from them in so far as they seemed to go against the Constitution. Hence, the Constitution was not understood to stand in the way of more specific rules.\footnote{\cite[136-137]{nut69}.} According to the minority of five, no such rules were actually needed, but the majority of ten disagreed.\footcite[137]{nut69} }

When considering the question of what kind of rules should be introduced, the Committee looked to case law as well as existing literature on compensation. They were faced with highly divergent opinions on the subject. Since WW2, in particular, a pragmatic view on property rights had taken hold, whereby an absolute right to property was increasingly felt to stand in the way of efforts to rebuild the country and ensure development following the great war. The Labour party had secured a firm grip on government at this point, so there was also an ideological shift taking place that emphasised the importance of building a welfare state over protecting the entitlements of individuals.

This was by no means a consensus view among legal scholars, however, and it was particularly contentious with regards to property. 

%As a result, some disagreed strongly with the very idea of legislation regarding compensation, and tensions arose that have led to much legal controversy and are still important in the law today.

%The majority pointed out that a vague general principle such as that provided by the constitution would by necessity have to be interpreted in order to be applied to concrete cases.\footnote[137]{nut69} Hence, it was not only permissible, but also desirable, for parliament to give more detailed instructions as to how is should be applied and understood by the courts and the appraisal courts. Leaving it to the judiciary to flesh out the exact meaning of full compensation through case law, it was felt, was not appropriate in a regulatory regime where expropriation had become increasingly important as a means to ensure modernization and development of critical infrastructure.

%In addition to this, the Supreme Court itself had recently expressed its support for a new view on regulation of property use, supported by contemporary legal scholars and politicians, whereby the State was regarded as having wide discretionary powers to determine how property should be used. This right to regulate, in particular, was increasingly coming to be seen as a right that did not infringe on property rights, so that the State would not have to compensate owners if they exercised it, except in special cases.\footnote{See, in particular, Rt. 1970 p. 67.}.

This problem area was mapped out in some detail by the Husaas committee, who traced the pragmatic view on compensation, identifying it using the following quote by the leading scholar Knoph from \cite[113]{knoph39}.

\begin{quote}
Since Section 105 is a rule prescribing practical justice, directed at parliament, and not an ethical postulate of absolute validity, it must be permitted to make technical legal considerations, so that one accepts compensation rules that lead to correct and just results on average, even if it does not grant the owner full individual justice in every case.
\end{quote}

Many disagreed vehemently with this perspective, based on absolutist principles.\footnote{See, e.g., \cite[20-22]{robberstad57};\cite[44]{schjodt47}.} The prominent legal scholar Schjødt, for instance, describes Knoph's reading of the law scathingly as follows:\footcite[44]{schjodt47}

\begin{quote}Luckily it has not had any effect on judicial practice whatsoever. No court of law would accept that compensation should be set according to a norm that may be practical and just in general, but does not grant the owner full compensation in all individual cases.
\end{quote}

By the late 1960s, however, Knoph's view was beginning to find favour among legal scholars.\footnote{See \cite[17]{fleischer68};\cite[41]{opshal68}.} When assessing the writings on the subject, the Husaas committee noted this tension. In response to it, they proposed a set of general principles for compensation which are still largely with us today. They were moving in a pragmatic direction, but rather cautiously. Hence, they refrained from encoding principles that would appear too offensive to the absolutists, even if the pervading political sentiment was that compensation rights had to be limited to ensure more effective state regulation of property use.

Importantly, the Husaas committee distilled from case law the principle that owners could only demand compensation based on the value of a specific use of the property when this use was ``foreseeable''.
The committee sought to codify this idea, which they saw as expressing an interpretation of Section 105 that was already largely entrenched in case law.\footcite[134]{nut69} This led to the following conclusion:\footnote{\cite[142]{nut69}.}

\begin{quote}
It is the view of the committee that it is correct to encode in the act the principle that the owner is entitled to compensation based on the value that results from taking into account the foreseeable and natural use of the property, given its location and the surrounding conditions. The exact meaning of ``natural and foreseeable'' use must be decided after a concrete assessment in individual cases. By encoding this general principle, however, it will become clear that compensation should not be based on private or public plans unless these plans coincide with the use of the property that is natural and foreseeable, independently of the scheme underlying expropriation.
\end{quote}

Importantly, I note how the committee actually does more than just encode a foreseeability constraint. They also state outright that this constraint is taken to imply the no-scheme principle, since they stipulate that the assessment of what counts as foreseeable and natural must be made independently of the scheme underlying expropriation. Since this statement is made quite generally, it also seems that the committee expresses a broader view on the no-scheme principle than that endorsed by Castberg.\footcite[268]{castberg64b} It is no longer only the special want of the expropriating party that should not be taken into account, the entire scheme ``underlying'' expropriation should be disregarded.

But in fact, this view was not in keeping with the political motivation for an act regarding compensation. It was too owner-friendly. Hence, the Ministry of Justice deviated from it in their final proposition to parliament. Instead of encoding existing principles, they sought a more aggressively pragmatic system whereby compensation would in general be based on the value of the \emph{current use} of the property.\footnote{\cite[19-20]{otprp59}.} In this way, the argument went, the public no longer had to pay a financial premium to owners based on possible future uses that would in any event, in most cases, be reliant on public development permissions.\footnote{\cite[17-20]{otprp59}.} 

%Such permissions, it was argued, could never be foreseeable in circumstances when it was in the public interest that the property should be expropriated, and hence all future development potential should in principle fall to be disregarded.

%The Ministry commented on this as follows:\footnote{\cite[19-20]{otprp59}.}
%
%\begin{quote}
%The Ministry is of the opinion that it is particularly important to arrive at a rule that can bring the assessment of property value down to a realistic level, and believes that the natural starting point for such an assessment must be the current use of the property, especially for expropriation of real property. As mentioned, it is the opinion of the Ministry that a practice has developed that gives too much weight to more or less uncertain future possibilities for the property, something that has led to a sharp rise in compensation payments.
%\end{quote}

After intense debate in parliament, where the minority center-right parties all opposed its introduction, the current use rule was eventually encoded in section 4, no 1 of the \cite{ca73}.\footnote{Act No 4 of 26 March 1973 Regarding Compensation following Expropriation of Real Property.} This was largely seen as a social democratic victory and a clear indication that the pragmatic approach to property protection was gaining ground. When clarifying their principled starting point regarding what should count as \emph{realistic}, the Ministry made the following assertion regarding the scope of the constitutional protection offered in Section 105, showing the ideological underpinnings of the new Act:\footcite[17]{otprp70}

\begin{quote}
However, a right to complete -- or almost complete -- equality can not be derived from the constitution. It must be taken into account that we are here discussing equality with regards to increases in property value that are, in themselves, undeserved. [...]  %  The starting point must be that it is not, in and of itself, contrary to the constitution that one property owner do not benefit from the same increase in value as another, when the increase in value, for both of them, is due to public investment and does not stem from their own efforts. \\ \\
Certainly, it would be best to avoid any kind of inequality, if it was possible. But the examples we have considered illustrate that, today, inequality between property owners is tolerated with regards to public investments and regulation, and that, moreover, practical and economic considerations dictate that we \emph{should} make use of differential treatment in this regard.
\end{quote}

This echoes Knoph, but also goes much further. In particular, the Ministry explicitly states that differential treatment is appropriate in the context of expropriation, and, by implication, that this should be done precisely to avoid compensation payments that include compensation for ``undeserved'' increases in value. Also, in proposing that compensation payments should be based on current use, the scope of ``undeserved value'' is made very wide. In principle it would seem to include \emph{any} value that could be attributed to an as of yet unrealized potential that the property in question might have. The question of whether or not this value was reflected in the market value of the property, in particular, was not regarded as relevant. This was in itself radical, since market value based on the likely use of an ``average buyer'' had previously been the dominant starting point for appraissal courts when awarding compensation.\footcite[112-113]{nut69}

The conceptual significance of this change in perspective should not be underestimated. Here the Ministry stood firmly behind a pragmatic view. Perceived social fairness was the overriding constraint, also with respect to constitutional property protection. However, on taking this view to its logical conclusion, it was recognized that any general compensation rules that might be introduced should themselves be subject to a fairness test, so that, for instance, the current use principle could not itself be absolute or without exception. 

Rather, it could only be applied in so far as it served the overreaching goal of social justice and fairness which was regarded as the fundamental component of property protection that made such a rule possible. This, in particular, seems like a crucial observation, and one that has in my opinion been overlooked, with unfortunate consequence for the subsequent debate and development of the law. Indeed, it echoes the sentiment behind the age-old procedural arrangement that placed high value on the free discretion of the appraisal courts. Hence, it points to the possibility of finding some \emph{common ground} between absolutist and pragmatist views on compensation.

Sensible voices from both camps seemed to arrive at the conclusion that in the end, there was no way around a concrete and contextual assessment, where social fairness values are (hopefully) used as a guide. In an attempt to translate aspects of such a perspective into legislation, the Ministry set out two exceptions to the current use rule. The first, which received by far the most attention, was based on the notion of equality between owners in same local area.\footcite[19]{otprp70} It stipulated that the appraisal courts should be free to deviate from the current use rule in so far as it felt that it was reasonable to do so in order to ensure a reasonable degree of equality between neighbouring owners.\footnote{This principle was eventually encoded in section 5, no 1-3 of the \cite{ca73}. It would prove highly controversial, since it was only formulated as a rules that ``could'' be used to increase the compensation. In \emph{Kløfta}, the Supreme Court eventually deviated from this and overruled the Act by making clear that additional compensation was \emph{obligatory} in a range of cases when the intention had clearly been that the rule should be used sparingly. In this way, and possibly inadvertently, the Supreme Court ended up defending owners' interest by \emph{limiting} the power of the appraisal courts.}

However, the Ministry also noted the need for a second exception, which is in my opinion far more important and interesting. This exception pointed to the need to ensure equality between the taker and the owner, in so far as the taker could not be regarded as the embodiment of purely public values.

%
%\begin{quote}
%One is aware that the principle of current use compensation cannot be without exception. Even though this rule will be fair in general it can, in some cases, disproportionately disadvantage property owners. One has therefore suggested rules that modify the principle to some extent. These are given for somewhat different reasons. \\ \\
%
%One case addresses the situation when current use compensation means that a property owner will be significantly worse off that other owners of similar property in the same district, according to how these properties are normally used. In these cases, the principle of equality suggest that the owner receives some -- but not necessarily full -- compensation for the inequality that would otherwise arise from the fact that his property was made subject to expropriation. %Etter departementets oppfatning har en ekspropriat etter grunnloven ikke noe krav på å bli satt helt i samme stilling som om ekspropriasjonen ikke var skjedd, en forskjellbehandling innen rimelige grenser må grunnloven tillate når dette tilsies av tungtveiende samfunnsmessige grunner. 
%\end{quote}

Importantly, the rule sought to address precisely the situation that arises when the taker benefits commercially from the expropriation. Moreover, it addressed the question of the \emph{power balance} between the expropriating party and the owner. In the words of the Ministry:\footcite[19]{otprp70}

\begin{quote}
The second modification we make has to do with the relationship between the property owner and the expropriating party. If the use of the property that the expropriation presupposes gives the property a value that is significantly higher than the value suggested by current use, this will entail a transfer of value from the property owner to the acquiring party. In some cases this might be unreasonable. As an example of when this can become an issue, we mention an agricultural property that is expropriation for the purposes of industrial production. In such a case it might be natural that the owner receives a certain share in the increased value that the new use of the property will lead to.[...] %This would be different than, say, a situation where an agricultural property is expropriated for constructing a road or for setting up recreational outdoor grounds. In such cases, the expropriation will not lead to any such economically advantageous use of the property that will give the expropriating party an economic advantage. 

To establish a flexible system, the Ministry has concluded that it is practical that the King gives rules concerning the cases where an enhanced compensation payment, based on these principles, might be appropriate. This should not be decided by individual assessment, but governed by rules for special case types. Hence, the proposed Act states that the King can pass regulation concerning this matter.
\end{quote}

This quote goes right to the heart of one of the main problems of economic development takings, and proposes a possible remedy. However, the Ministry took the view that this remedy should {\it not} be administered by the appraisal courts, but should be left in the hands of the executive. Already here I note a reason worry whether this could then ever become an ineffective way of achieving fairness in practice. Indeed, the fact that this aspect of the 1973 Act has been largely overlooked and forgotten seems to prove my point. No rules such as those proposed by the Ministry as a possibility have in fact been introduced, the entire profit still goes to the taker in cases when commercial schemes benefit from expropriation.

%More broadly, it is hard to disagree that the context of expropriation must by necessity come to play a crucial role for any approach based on compensating the ``deserved'' value. What this value should be taken to be, in particular, can hardly be determined once and for all and in general terms, but must rather be subject to continuous revision depending on how expropriation is \emph{actually used} in society. This includes looking to the purpose it is meant to serve, the parties who stand to benefit, and the groups who tend to loose their land.

%Indeed, stipulating that compensation should be ``deserved'' appears to provide a benchmark that is just as unclear as the stipulation that compensation should be ``full''. It seems, in particular, that the inherent ambiguity of these terms allows us to draw two conclusions: first, that they might very well have the same meaning, and second, that they cannot possibly be defined once 
%and for all by any act of parliament, or by any decision in the Supreme Court.
%
%But this suggest, against the Ministry and the overall spirit of the 1973 Act, that the system of appraisal courts has an important role to play in ensuring fairness in individual cases. It is hard to see how the objective of social fairness and justice for the individual can be reached without making heavy use of appraisers with discretionary competence. 

The procedural and contextual aspect of fairness seems to have been overlooked by those pushing for the 1973 Act. Since the appraisal courts were regarded as compensating owners too generously, their freedom of discretion was seen as a problem rather than as a path towards a solution. I think this regrettable. If the new Act had been slightly more temperate in its approach, by encouraging the appraisal courts to take a broader view on fairness, rather than to force them to adopt current use value as a baseline, it might have been a success. Instead, it caused an outcry, with attention shifting away from practical matters towards doctrinal issues. The primary such issue, and the most serious one, concerned the question of whether the Act as such was in breach of the constitution. This was eventually considered by the Supreme Court in the case of \emph{Kløfta} in 1976.\footnote{\cite{klofta76}.}. 

Following this decision, the 1973 Act would be significantly reinterpreted to make it appear less offensive to the constitutional standard of full compensation. However, it seems to me that the Supreme Court largely accepted that the intention behind the Act should be respected and that appraisal practice needed to be adjusted accordingly. In this, the Supreme Court signalled loyalty to the political system and the democratic process. However, in implementing this adjustment in practice, they also, possibly inadvertently, set up a system where the role of the local appraisal courts were undermined even further.

Not only were they constrained by an Act that seemed to run counter to the Constitution, they were now also ordered from above to openly deviate from its exact wording, but only for a select group of cases meeting certain pre-defined criteria. In essence, the Supreme Court itself assumed greater control over how compensation law was to be applied, no longer merely in broad strokes, but increasingly also by developing special rules for specific case types.\footnote{The clearest indication of this shift is found in recent case law wherein the Supreme Court has provided a myriad of detailed rules and directions regarding how appraisal courts should decide on the thorny issue of whether to consider public plans binding for the compensation award or to disregard them under a no-scheme rule. See generally \cite[7-9]{nou03}.} In the following section, I describe this in more detail.

\subsection{The top-down approach}\label{sec:regab}

Following the decision in {\it Kløfta}, the \cite{84} was introduced. It reverted back to the ``foreseeability'' test proposed by the Husaas committee. In section 4, it is stated that financial compensation (as opposed to compensation in kind) is to be based on either value of use or value of sale, whichever is highest.\footcite[4]{ca84} Sections 5 and 6 describes in more detail how the calculation should be carried out.

In both regards, the principal requirement is that the value is calculated based on a use of the property that is foreseeable and natural given the surrounding conditions. In relation to the value of sale, there is an additional requirement, namely that the use must be one that an ``average'' buyer would be likely to make of the property. Hence, the value of sale should be set as a general market value, not a value arising from selling the property to a specially interested party.

The extent to which the foreseeability requirement entails that the use in question has to be in accordance with public plans currently in place has been disputed. In general, compensation is only based either on uses permitted by public plans currently in place or uses that seem likely to be permitted in the future. In Norwegian law, whether a use is foreseeable is an ``either/or'' question. 

No compensation is given to reflect the so-called ``hope'' value, namely the part of a property's value that depends on the perceived likelihood of a change in planning status and future possibilities.\footnote{By contrast, compensation tends to include such an element in the UK.} If a permission for future use is deemed likely, it is subsequently regarded as a certainty for purposes of compensation, although the present-day value of a future possibility is usually calculated in a way that takes interest and inflation into account.\footnote{These calculations tend to be notoriously schematic, however, quite far removed from the realities of the financial system.} Similarly, if a future possibility is deemed unlikely, no compensation is paid for it whatsoever.

In effect, the best an owner can hope for in Norway is that the likelihood of having received more than full compensation is greater than the likelihood of having received less.

Tensions and disputes tend to arise either directly in relation to the foreseeability test, or else in relation to one of the disregard rules that encode aspects of the no-scheme principle. The disregard rules included in statute are all formulated in relation to the value of sale, although they are also regarded as applying to value of use assessments. To some extent, they may also be redundant, in so far as they already follow from the foreseeability test.\footnote{I recall that the Husaas committee itself thought that a rather wide no-scheme principle would follow already from the foreseeability test. See above.}

The main disregard rule included explicitly in the \cite{ca84} is formulated very similarly to the no-scheme rule in the UK. It states that one should not take into account changes in value that can be attributed to the ``expropriation measure''.\footcite[5]{ca84} Interestingly, the notion of an ``expropriation measure'' is defined in the Act. In section 2, the expropriation measure is said to be the ``activity, installation, or purpose'' benefiting from expropriation. Hence, while there is a definition, it is (as expected) very vague. 

If the definition had only included the second item -- that of an ``installation'' -- it would amount to a meaningful restriction. However, as it stands, an ``expropriation measure'' seems like it could include just about anything that stands in some kind of relationship to the expropriation order.
The purpose of the expropriation is included in the list, which, if taken literally, would lead to rather absurd results. 

For instance, if houses next to a small public road are expropriated for the construction of a motorway, the wording suggests that even the presence of the public road must be disregarded when assessing their value. This would seem to follow, in particular, in so far as the pubic road was built in pursuance of the same purpose as the motorway now being constructed. Luckily, the rule is not understood in this way in practice. 

However, a second rule expressed in section 5 of the \cite{ca84} states that changes in value due to other investments that the expropriating party has carried out, or plans to carry out, also falls to be disregarded. The condition is that they have {\it either} been carried out in relation to the expropriation measure {\it or} during the last 10 years.\footnote{See the third and fourth paragraph of \cite[5]{ca84}.} Hence, the disregard rule in section 5 is stronger than most no-scheme rules, in that previous or planned investments must sometimes be disregarded even if they stand in no relation between the expropriation scheme besides being carried out by the same party. In so far as the expropriating party is a public body, even this is relaxed, since all investments carried out by {\it any} public body is then to be disregarded, limited only by the 10 years rule.\footcite[5]{ca84} 

%When the constitutionality of the Compensation Act 1973 came before the Supreme Court in \emph{Kløfta}, they chose to sit as a grand chamber and they reached a decision under dissent, being divided into two fractions, consisting of 9 and 8 supreme judges respectively. However, both fractions approached the problem of constitutionality by endorsing an interpretation of Section 5 nr. 1 in the Compensation Act 1973 that gave the exception to the current use much wider scope than what had been intended by parliament. The majority went farthest, and unlike the minority they also regarded the compensation payment in the concrete case to be insufficient. The first voter for the majority commented as follows on the constitutional aspect of the case.\footcite[7-8]{klofta76}
%
%\begin{quote}
%[...] But the main question in this case, is whether or not it is in keeping with Section 105 to generally award compensation at a level below the market value that could legally be estimated, and that the owner could actually have achieved, if expropriation had not taken place. In my view, this involves allowing expropriation to transfer a right that the owner had, with a value to which he was entitled. If he is refused compensation for this value, he would, depending on the circumstances, be left significantly worse off than others in a similar position, who owns property that is not expropriated. Such a result I cannot accept. It would be a breach of established customary law and a practice that has been established throughout the years both by the appraisal courts and the Supreme Court. I refer particularly to Rt 1951 s. 87 (particularly p. 89, Opdahl). This practice is in itself a significant contribution to interpreting Section 105 on this point.
%\end{quote}
%
%I note the emphasis placed on \emph{market value} in the majority's reasoning. This may appear to be in keeping with an absolutist doctrine, but as I have mentioned, it can have unfortunate, possibly unintended, consequences for property owners, especially when combined with a restrictive view on what counts as foreseeable future development. I note, however, a technical point that might be of some significance for the interpretation of \emph{Klofta}: Instead of stating outright that a market value rule follows from the wording of the Constitution as such, the majority takes the view that this interpretation suggests itself based on the compensation practice that had currently been established. This might limit the scope of the majority's remarks in this regard, but it also serves to give further support to the claim that the role of the appraisal courts, and their assessments, still had a strong position in Norwegian compensation law at the time of \emph{Kløfta}. 
%
%I remark that the minority disagreed on the constitutional status of the market value rule. Indeed, it was in this regard that the difference of opinion between the minority and the majority was most clearly felt. The minority, in particular, explicitly rejected the view that this rule could be derived from the constitution itself, and they also disagreed with the understanding that it would have status as a constitutional rule simply because it had been adopted in practice. This bestowed merely the status of ordinary legal precedent. As expressed by the minority:\footcite[23-24]{klofta76}
%
%\begin{quote}
%Case law from this area cannot be understood as preventing parliament from changing the rules in accordance with what they regard as necessary. That would prevent a reasonable and natural development and would not be in keeping with the consensus view that Section 105 of the constitution is a rule that must be interpreted in light of, and adapted to, how society has developed and how the law is viewed. I believe the practice that have evolved cannot be decisive if a new situation and new needs require a different solution. Whether the Compensation Act is in breach of the right to full compensation enshrined in the constitution, must depend on an interpretation of the wording in the constitution itself.[...] \\ \\
%In my opinion, neither the intentions of parliament nor the way they are sought implemented through Sections 4 and 5 are in breach of the equality principle upon which Section 105 of the constitution is based. It does not follow from the constitution that an owner is in all circumstances -- and irrespectively of the economic forces from which the market value results -- entitled to compensation that is at least as great as the greatest legal value that the property could represent on a free market. A different matter is that Section 105 of the constitution could be important to the interpretation and application of the rules.
%\end{quote} 
%
%Hence, the market value rule was explicitly renounced as a constitutional principle by the minority, who nevertheless conceded that the constitution could be used to interpret Sections 4 and 5 of the Compensation Act 1973. Both the minority and the majority agreed, however, that  it would be wrong to go on to consider Section 4 of the Compensation Act 1973 in isolation. For the majority, this would clearly have led to the Compensation Act 1973 being held to be in breach of the constitution, something that was avoided since the Supreme Court chose to consider the law as a whole, with the majority using the reasoning detailed above to argue for a new interpretation of Section 5, rather than as a means to undermine Section 4. Still, their interpretation of Section 5 went well beyond what it seemed that parliament had intended, leading some scholars to claim that \emph{Kløfta} should be read as holding that the Compensation Act 1973 was unconstitutional.\footcite[477]{andenes86} In the words of the majority:\footcite[12-13]{klofta76}
%
%\begin{quote}
%The purpose of this rule is to award compensation beyond current use in cases where valuations according to section 4 could be in breach with section 105 of the Constitution. As it stands, section 5 no 1 is not sufficiently suited for this purpose. By its wording it gives the appraisal courts an opportunity to assess whether or not it is reasonable to award additional compensation, even when the conditions for this is otherwise met, and even then with the limitation that the compensation would otherwise be significantly unreasonable. Such a free position for the individual appraisal courts -- without possibility of legal appeal -- would not be in keeping with the purpose of the rule and the demand for full compensation set out in the Constitution.
%\end{quote}
%
%On this basis, the Supreme Court chose to interpret section 5 no 1 in such a way that whenever the conditions were fulfilled, the appraisal courts were \emph{obliged} to award additional compensation, On this basis they found that the property owners in \emph{Kløfta} was entitled to have their compensation looked at again, in a new round before the appraisal courts. The minority agreed in principle, yet did not go as far as the majority, concluding that based on the particular facts at hand section 5 had been adequately considered by the appraisal court in this particular case.\footcite[22]{klofta76} In addition, the majority went quite far in suggesting that ``full compensation'' entitled the owner to {\it market value} compensation, whenever this would result in a higher award than a ``value of use'' approach.\footcite[14]{klofta76} Moreover, they adopted a more narrow interpretation of the (negative) no-scheme rule, whereby public plans not closely related to the expropriation project should not be disregarded.\footcite[15-16]{klofta76} In these matters, the minority took a different view, arguing against market value as a general benchmark and in favour of a broader no-scheme rule.\footnote{\cite[22-23|30-31]{klofta16}.}
%
%The upshot of \emph{Kløfta} was that section 5 no 1 came to be seen as an obligatory rule, leading to compensation having to be enhanced whenever the current use rule led to payments that did not reflect the market value of comparable properties. However, the conditions stated in section 5 no 2 and no 3 were still regarded as relevant, and in interpreting these conditions, a body of law developed whereby the market value rule was applied in a way that would come to involve significant reduction in compensation compared to what would result from practice as it had been prior to the Compensation Act 1973. In this way, the pragmatic approach proved triumphant, not because current use value was introduced as the general starting point, on the contrary, but because a range of new disregards were introduced to reduce the level of compensation in a range of different circumstances. After \emph{Kløfta}, in particular, the following rules were all considered legitimate ways to decrease the level of compensation.

%In section 5 no 3 and no 4, the Expropriation Compensation Act 1973 encoded the following three disregard principles that are all, to varying degrees, still important in compensation law today. 
%
%\begin{enumerate}
%\item Changes in value that are due to the expropriation scheme should be disregarded, both when these are already carried out as well as when they are planned, c.f., section 5 no 2 of the \cite{ca73}.
%\item To the extent that it is regarded reasonable, \emph{increases} in value that are due to public plans or investments should be disregarded, irrespectively of whether or not they have already been carried out, c.f., section 5 no 2 of the \cite{ca73}.
%\item An increased value falls to be disregarded if it results from considering a use of the property which is not in accordance with public plans, c.f., section 5 no 3 of the \cite{ca73}.
%\end{enumerate}
%
%While the \cite{ca73} has now been replaced by the \cite{ca84}, the formulation given in the 

These rules severely limits the level of compensation payments, and in many cases it appears to make the principle of full compensation based on market value rather illusory, even if this was the principle endorsed by the Supreme Court in {\it Kløfta}. On the one hand, the foreseeability test can serve to rule of value arising from any use of the property that is not in keeping with the current public plan. At the same time, the no-scheme rule explicitly encoded in section 5 can be used to also disregard values that are due to this plan, particularly if they are regarded as standing in some relation to the expropriation measure. 

The outcome could easily become, logically speaking, that no compensation can be awarded whatsoever. However, the system tends to revert back to the current use compensation in such cases. For instance, if agricultural land is expropriated for the purpose of a motorway, and it would otherwise appear foreseeable that it might be used for housing in the future, the compensation will usually be based on agricultural use because the value for housing is disregarded under a foreseeability test while possible increases in value due to the motorway plan itself is disregarded under the no-scheme rule.

In practice, with virtually all novel economic activity making use of land is dependent on acquiring new planning permissions, the current use rule will typically be applied as intended by the \cite{ca73}. The main difference is that the rule is not thought of, or described, as an absolute. It rather tends to arise merely as a side effect of other rules.\footnote{A similar point was made in \cite{stordrange94}.} Outcomes that are in keeping with current use thinking will typically be designated as ``full compensation based on market value'' -- the standard phrase adopted in most appraisal judgements -- notwithstanding the fact that the accuracy of such a description depends on the disregards that have been applied.

%The \cite{ca84} was eventually introduced to reflect the principles laid down in \emph{Kløfta}, but it did not in any essentially way change or influence the course of the law that had already been set. Its main purpose was to bring the wording of the legislation more into keeping with how the law was interpreted by the Supreme Court. It explicitly returned to the starting point of the Husaas committee, namely that the compensation should be based on the value of the "foreseeable use" that the owner himself, or an average buyer, might make of the property. But it maintained and endorsed disregard rules no 1-3, except for restricting disregard no 2 to public investments, such that increased value due to public plans currently in place could not be disregarded.\footnote{In this way, the paradox mentioned above, that compensation could become impossible to award because there was no possible basis upon which to calculate it, was avoided.}

The statutory rules do not provide clear guidance as to how the disregard rules should be understood or applied, nor do they consider or resolve the question of when, if ever, they would need to be applied with caution in order not to go against the constitution. However, following {\it Kløfta}, there has been a growing expectation that cases where such issues arise should be resolved by crisp rules, not by the discretion of the appraisal courts. The age when the appraisal courts were considered free to assess cases directly against the Constitution is gone. Rather, an ethos had taken hold where the need to curb the freedom of appraisers, in the interest of ensuring predictability and centralized control, is emphasized.

As a result, difficult cases now routinely end up in the Supreme Court. Here, difficult circumstances are used as the basis for formulating more and more specific rules for special case types. As an example of this mechanism, it is enlightening to consider the case law surrounding the question of whether public plans currently in place are binding when calculating compensation. This rule cannot apply without exception, as recognized already by the \cite{ca73}. But when is it permissible to deviate from it?

The question has arisen in many Supreme Court cases following {\it Kløfta}. \emph{Østensjø} concerned land that was being expropriated for housing purposes, but such that one unlucky owner would only contribute land used for infrastructure that would serve the larger housing project.\footnote{\cite{ostensjo77}.} In this case, the Supreme Court agreed that he was entitled to compensation based on value of his land for housing purposes, irrespectively of the fact that \emph{his} land could not be used in this way according to the plan. However, in many other cases, the disregard rule is upheld even when it is hard to see it as either fair or just, simply on account of it having status as a general rule.\footnote{For instance in \cite{malvik93}. In this case, owners of property used for a motorway were only entitled to compensation based on current agricultural use because the planned motorway-use was assumed binding for the compensation assessment under the market value approach.}

One example is found in \emph{Sea Farm} which dealt with the issue of whether or not the owner of a commercial property should be awarded compensation for the value of investments carried out by the previous tenant.\footcite{seafarm08} There was no doubt that the owner was entitled to these investments, but since the acquiring authority was the only purchaser who was likely to benefit commercially from them, no compensation was awarded for the loss of these investments. This, in particular, followed from a strict reading of the requirement that compensation should be based on the foreseeable use that an "average" buyer could make of the property, encoded in Section 5 of the Compensation Act 1984. Adherence to the wording used in the act seems to have taken priority over an assessment based on the facts of the case. It seems difficult to argue that it would be either unjust or unreasonable, in particular, to compensate the owner for investments that would prove commercially valuable to the acquiring party.\footnote{The decision was sharply criticized by a former supreme judge. See \cite{skoghoy08}.}

In my opinion, this example illustrates how the development of compensation law towards greater reliance on specific rules rather than concrete assessment based on general principles can be harmful. I also threatens to undermine the idea behind the special procedure used to decide appraisal disputes, which has a long history in Norwegian law.\footnote{One might ask if it has status of constitutional customary law, especially since it concerns the mechanism by which a constitutional rule is meant to be upheld.} It also seems to severely underestimate the extent to which compensation rules, when applied to concrete cases, must and should be interpreted based on the context of the case. It seems difficult, if not completely impossible, to achieve social fairness and individual justice by a set of specific rules on the basis of which all legal issues can be resolved mechanically by blind application of such rules. %Moreover, it would be wrong to think that Section ... of the Appraisal Act 1917, encoding the principle that laymen should take part in the decision-making both with regards to legal and technical matters that arose in appraisal disputes.

In the following section, I will turn to waterfalls and hydropower. Interestingly, the compensation practices developed in this regard often deviate significantly from the general approach to compensation. The special approach developed, in particular, as a result of the perceived unfairness of denying benefit sharing altogether in such cases. Hence, looking to waterfalls serves to underscore my point about the importance of a flexible system. 

%The main benefit sharing principle that was developed was known as the {\it natural horsepower method} for calculating compensation for waterfalls following expropriation. It was initially developed by the appraisal courts as an ad hoc approach to ensuring some benefit sharing in hydropower cases. Later, however, many came to regard it as a binding principle of customary law. Gradually, it came to be applied by the courts with little or no regard for how well it suited the circumstances of the case and the changing realities of the hydropower sector. As a result, the method became hopelessly outdated, leading to compensation payments that had little or nothing to do with the actual value of waterfalls for hydropower. 

%Today, while the method has been abandoned for certain case types, it is still applied as the default rule for compensation waterfalls.
%
%
% address this issue in more detail, and we will argue for a different conceptual approach to compensation law, grounded both in the procedural tradition of appraisal courts and the more subtle parts of the absolutist and pragmatic theoretical traditions. It seems to me that the most striking lesson that should be drawn from considering the history of Norwegian compensation law is that a \emph{contextual} view of compensation has been a common denominator that both the absolutist and pragmatist camps have endorsed. Unfortunately, this common element was overshadowed by political conflict regarding the weighing of different values. However, there can be little doubt that social fairness and individual justice should \emph{both} to be regarded as important objectives for compensation rules. Moreover, while they may sometimes be opposing, they need not be, and their exact relationship depends largely on the circumstances. It seems to us that it is simply inappropriate to let particular political sentiments regarding their relationship and relative importance, sentiments that are usually dependent on the particulars of the prevailing political, social and economic conditions, dictate the development of the legal framework for resolving compensation disputes.
%
%Considering current trends and recent issues in expropriation law, particularly related to commercial expropriation, further suggests that a different perspective is needed on this matter. In particular, we believe it is time to recall the idea of the independent and impartial discretion of the appraisal court, relying on the good common sense of laymen as well as the legal expertise of judges. The appraisal courts should in our opinion be set with the task of more actively evaluating how fairness and justice is best served in individual cases, at least if the overall goal is truly to arrive at a socially fair and individually just compensation system. We discuss this idea in more detail in the final section below.

\section{``Natural horsepowers''}

Following the introduction of a general expropriation authority covering waterfalls in the early 20th century, the question of how to value waterfalls came before the appraisal courts. The regulatory regime that was established made private commercial development difficult or impossible, and this in turn meant that the commercial market for waterfalls all but disappeared. Hence, a strict application of the no-scheme rule could lead to no compensation being paid at all. Arguably, a waterfall had no value to anyone except the acquiring authority, since no alternative development scheme could be regarded as foreseeable.

The appraisal courts did not follow this point of view to its logical conclusion. Instead, they introduced a theoretical formula for calculating waterfall compensation. In effect, this method served to create an artificial market for waterfalls, controlled by the appraisal courts. Initially, this artificial market was modelled on the actual market that had existed prior to the regulatory reform. Over time, however, the waterfall ``market'' would slide further and further into the legal sphere, away from the physical and commercial reality of hydropower development.

The key notion used to determine the price of a waterfall on this market was that of a {\it natural horsepower}, a gross measure of electric effect.\footnote{A horsepower, of course, is an old-fashioned unit of effect which is still sometimes used, e.g., in relation to cars. In the context of electricity, it is replaced by {\it Watts}, such that 1 horespower (hp) = 745.69 Watts.} As I mentioned in Chapter \ref{chap:4}, the lack of a national grid at this time meant that the value of a hydropower plant was largely determined by the stable effect that the plant could deliver, not the total amount of electricity that could be produced. This, in turn, was a function of the degree of water regulation implemented by the hydropower developer. 

To simplify the calculation, the natural horsepower of a waterfall was introduced as a gross estimate of the stable effect that could be ensured given a choice regarding the level of regulation of the watercourse. The value of the waterfall itself was then determined by fixing a price per natural horsepower. This price was set on the basis of prices paid for other waterfalls, with some adjustments typically carried to take into account the level of cost and benefits associated with the hydropower project in question.

As I remarked in Chapter \ref{chap:4}, the notion of a natural horsepower is used in other contexts as well, for instance to determine what kind of licenses a development project requires. The use made of it to calculate compensation for waterfalls had no legislative basis, but arose as a result of the appraisal courts' efforts to calculate market prices. After the actual market based on the natural horsepower method disappeared, the method stuck and was applied as a matter of custom.\footnote{See generally the description of the history of the method given by the Supreme Court in \cite{uleberg08}.}
%
%
%prove shockingly unfair to owners of waterfalls. Presumably, since waterfalls could not be exploited for any significant commercial gain except through hydro-power exploitation, disregarding the hydro-power scheme when calculating compensation could lead to nil or close to nil being awarded to the owner. But this was not seen as an acceptable outcome, and instead the Norwegian courts introduced a special method to compensate waterfalls that gave the owner a \emph{share in the value of the hydro-power scheme} for which expropriation was taking place.
%
%Norway did not at this time have any legislation specifically aimed at regulating compensation following expropriation, and when formulating the special rules for compensation of waterfalls, the Norwegian courts seems to have relied on an analogical application of the gross valuation techniques introduced in the Industrial Concession Act 1917 and the Watercourse Regulation Act 1917.\footnote{Act No. 17 of 14 December 1917 relating to Regulations of Watercourses and Act No. 16 of 14 December 1917 relating to Acquisition of Waterfalls, Mines and other Real Property}. Neither of these acts were aimed at compensating owners, but they relied on methods for assessing the potential and significance of hydro-power projects with respect to the question of whether or not a special concession from the State was required.\footnote{To acquire the waterfall and the right to regulate the water-flow respectively.} In effect, by relying on the methods of valuation introduced there, the compensation mechanism that was introduced deviated completely from the "value to the owner" principle. On the other hand, it also closely mimicked the manner in which owners of waterfalls would be compensated on the market in the early days, prior to the introduction of our concession laws, when speculators would pay for waterfalls on the basis of what they assumed to get out of them in subsequent hydro-power projects.

In the Supreme Court case of \emph{Hellandsfoss}, some 80 years after it was first introduced, the natural horsepower method was described and put into context as follows:\footcite[1599]{hellandsfoss97}
\begin{quote}
The principle set out in the Compensation Act, Section 5, is that compensation should be determined on the basis of an estimation of what ordinary buyers would pay for the property in a voluntary sale, taking into account such use of the property as could reasonably be anticipated. For waterfalls, however, this often offers little guidance, and the value of waterfall rights have traditionally been determined based on the number of natural horsepowers in the fall, which are then multiplied by a price per unit. The unit price is determined after an overall assessment of the waterfall, including the cost of the scheme, its location, and levels of compensation paid for similar types of waterfalls in the past. The number of natural horsepowers is calculated by the formula ``natural horsepower = $13.33 \ \times \ Qreg \ \times \ Hbr$'', where $Qreg$ is the regulated water flow and $Hbr$ is the height of the waterfall.
\end{quote}

In this formula, $Qreg$ represents a quantity of water, measured in cubic meters per second (m3/sec), while $Hbr$ represents height measured in meters. The number $13.33$ is the force of the gravitational pull on earth measured in horsepower. 

In the standard account of the natural horsepower method, it is often said that the number of natural horsepower in a waterfall is a measure of gross effect, giving us the amount of ``raw'' power in the waterfall.\footnote{See \cite{vislie02}.} This is not accurate. Indeed, from the quote given above it is clear that the natural horsepower does {\it not} depend only on the nature of the waterfall. It also depends on the specific plans for development presented by the expropriating party. In particular, the quantity $Qreg$ is entirely a function of how the developer {\it chooses} to develop the waterfall, in that it measures the ``regulated water flow''.\footnote{In addition, the quantity $Hbr$ depends on the height over which the developer plans to make use of the water. The development potential that the owner is deprived of can amount to either more or less than this, depending on the nature of alternative schemes.}

In {\it Hellandsfoss}, the Supreme Court itself glosses over this point when it speaks of the ``natural horsepower in the fall''. It would be more accurate to speak of the natural horespower of the particular development scheme benefiting from expropriation.\footnote{Regulation of a watercourse can involve building a reservoir and/or installations that transfer water from one river to another. Then, if there is excess water, for instance due to flooding, water can be stored in the dam for later use. When there is no drought, the stored water can be released. In this way, it becomes possible to even out the water-flow over the year. Today, however, many hydropower plants, particularly smaller ones, involve little or not regulation. Instead, such run-of-river scheme operate by harnessing energy from whatever water is present in the river at any given time.}

But how exactly is the regulated water flow determined for the purposes of compensation estimation? In section 2 of the \cite{ica17}, it is said that the regulated water flow is to be determined ``on the basis of the increase of the low water flow of the watercourse, which the regulation is supposed to cause beyond the water flow which is considered foreseeable for 350 days a year.'' Hence, the idea is that only the {\it increase} in water flow is to be measured. This means that if the developer proposes a run-of-river project with no regulation, then the natural horsepower of the project will automatically be $0$.\footnote{In fact, things could become even worse for the owner, since the proposed project might lead to $Qreg$ becoming a {\it negative} number. This follows from section 10 of the \cite{wra00}. Here the NVE is given the power to compel the owner of a hydropower scheme to ensure that a certain quantity of water is always allowed to pass through the intake of the plant. This flow of water is typically referred to as the {\it minimum water flow}, but is sometimes used in place of the low water-flow before regulation when calculating the natural horsepower of a project. The idea behind imposing a minimum water-flow is to reduce the negative environmental impact. For many run-of-river schemes, the minimum water flow ordered by the NVE is higher than the low water flow after regulation. Hence, if the minimum water flow is subtracted from the low water flow after regulation, the result is a negative number. That is, one might end up with a {\it negative} $Qreg$.} But this outcome was averted in practice by an {\it ad hoc} adaptation of the traditional method. In relation to compensation, it became established practice to omit the deduction of the previous water flow, so that one would use the entire low water-flow after regulation as $Qreg$. That is, the quantity used for $Qreg$ when computing the natural horsepower of a waterfall for purposes of compensation is the estimated amount of water that is present in the river for at least 350 days a year after regulation.

This means that the natural horsepower of a development scheme has little bearing on the amount of energy that will actually be harnessed from it. Today, modern electricity generators can produce electricity at varying levels of effect, depending on the water-flow of the river. But the water-flow is not a constant as assumed by the natural horsepower formula. Rather, it varies considerably over the year.

As a result, the natural horsepower of a regulation does not have much to do with the value of neither waterfalls nor hydroelectric plants.\footnote{See generally \cite{sofienlund08}.} Indeed, the annual income of a hydroelectric plant has nothing to do with natural horsepower, it is solely a function of the price paid per kilowatthour and the total number of kilowatthours harnessed over the year (kWh/year).\footcite{sofienlund08} The amount of energy generated in a power plant could be measured in other units than kWh, e.g. in terms of the amount of horsepower-hours per year. But the important point to keep in mind is that an energy producer gets paid for the amount of energy he can deliver, \emph{not} the effect he can maintain in his station over a long duration of time. %Hence, even if if we uwould se kilowatt instead of horsepower and talk of the natural kilowatt of a hydropower plant, the quantity we are discussing is the same, and still has little or no bearing on the value of the waterfall.

Talking of natural horsepower therefore serves to give a skewed picture of the potential of a waterfall, especially for run-of-river projects. It is not unusual that the low water-flow in a river amounts to only about 3-5 \% of the average water supply. In modern hydropower projects, one would expect 70-80 \% of this water-flow to be harnessed for energy production even in the absence of any regulation. Hence, in these cases, the natural horsepower method, as it was traditionally applied, would only compensates the owners for about 5 \% of the energy that would actually be harnessed from their waterfalls.\footnote{sofienlund08}

This observation, which is trivial given a rudimentary understanding of the energy business, was not made in the context of expropriation until late in the 1990s. Moreover, the point was raised against the advice of legal experts who regarded the established method as a principle of customary law.\footnote{In the aforementioned case of {\it Hellandsfoss}, for instance, a local owner raised the issue with his legal council, who advised against raising it as an issue before the appraisal courts. The owner listened to his legal council, resulting in a compensation payment that is only a small fraction of what he would be entitled to under the method used in some more recent cases, e.g., in \cite{sauda08}. Source: Private correspondence.} At the same time, both engineers and government officials were well aware of the inadequacies of the method, as illustrated for instance by the following passage from a governmental report made in 1991:\footnote{\cite[19]{otprp50}, discussing the notion of natural horsepower in connection to the uses made of that term in other parts of the law.}

\begin{quote}
The Ministry of Petroleum and Energy has considered moving a proposition for changing the hydrological definitions in the Industrial Concession Act 1917 and the Watercourse Regulation Act 1917. Today the act uses a calculation method based on an increase in regulated water-flow, i.e. that of natural horsepower.[.......] The hydrological definitions of these acts, supposed to indicate how much electricity can be generated, were made on the basis of technical and operative conditions differing very much from contemporary circumstances. In implementing the definitions referred to above one has tried to adapt to the new technological realities of the present day. Therefore, in practice, a calculation based on current production is used instead. From several quarters, particularly the Association of Waterfall Regulators, there has been raised a strong wish to authorize this practice by altering the definitions of the relevant laws. The Department of Oil and Energy agree, but have not as yet made a sufficient elucidation of the issues to be able to move a proposition of alteration of these acts.
\end{quote}

The quote shows that in administrative practice, it had become common to deviate from the definition of a natural horsepower, since it no longer reflected a relevant figure. A similar move would not be made in the context of expropriation for another 20 years.\footnote{A ``natural horsepower'' calculation modified along the lines described by the Ministry in 1991 is now sometimes used also in compensation cases, following its adoption for some of the waterfalls that were expropriated in the case of \cite{sauda08}.}

Within the ranks of the specialized water authorities, the inadequacies of the natural horsepower method had been known even longer. Here it had also been noted that the method did not give rise to realistic estimates of the value of waterfalls. The first record I can find of such an admission dates back to 1957, from an article written by the director at the NVE which was published in their internal newsletter.\footnote{See \cite{....}. The director even went as far as to illustrate a different method, which would also be outdated given today's regulatory regime, but which would reflect contemporary \emph{actual} valuations, used by the NVE itself.}

Considering the physics behind the traditional method is enough to reveal that it fails to give rise to valuations that reflect the value of waterfalls, under any reasonable set of assumptions about the correct general compensation principles one should adopt. Important in this regard is the fact that  the method relies on data that depends entirely on the expropriating party's project. The compensation to the owner depends not on their loss, but on the technical details of the project that the expropriating party proposes. This clearly deviates from even a narrow interpretation of the no-scheme principle.

However, while the idea of compensating the owner of waterfalls by a price per natural horsepower is fundamentally flawed at the theoretical level, there are even more serious concerns that arise when one begins to consider the way in which the unit price has been determined {\it in practice}. The traditional approach to this question has had a particularly dramatic effect on the level of compensation payments. 

In case law based on the traditional method, it is often said that the price set per natural horsepower is set according to ``market price'' for waterfalls. But for the most part, what this means is that the court looks to prices awarded in earlier compensation cases. This practice gave rise to a price level that was entirely artificial. It reflected, more than anything else, the power balance between buyer and seller in the courtroom. It was certainly no genuine market value, even if it was described as such. This has become very clear after the adoption of new, genuinely market-based, methods in recent years.\footnote{See generally \cite{larsen08}.}

Indeed, while the unit price for a natural horsepower did increase somewhat during the first 80 years that the traditional method was used, this increase neither reflected the value of hydropower in particularly nor the level of inflation in general.\footnote{See \cite{sofienlund08}.} Moreover, while the price-level was determined by the courts, some voluntary agreements were also made on the basis of the same method. These could then in turn be used to back up the claim that this was a genuine market-based valuation principle. In this way, it became possible to legitimize an increasing imbalance of power between owners and purchasers. In the end, this imbalance became extreme.

For instance, in 2002 a waterfall belonging to local landowners in the rural community of Måren, located in south-western Norway, was sold for the sum of kr 45 000 (roughly £ 4500), based on traditional calculations.\footnote{Source: private correspondence.} The waterfall has now been exploited in a small-scale hydro-power plant belonging to the large energy company BKK, with annual energy output of 21 GWh.\footnote{$http://www.bkk.no/om_oss/anlegg-utbygging/Kraftverk_og_vassdrag/andre-vassdrag/article29899.ece$} For comparison, I mention that in the case of \emph{Sauda}, where a more realistic market-based method was used, the owners received a compensation which totalled about 1 kr/kWh annual production.\footnote{LG-2007-176723 (I acted as council for some of the owners in this case).} Applied to the Måren case, this would take the compensation from kr 45 000 to kr 21 000 000. That is, the price would have been almost 500 times higher.\footnote{In fact, the Måren waterfalls were cheaper to exploit, so in reality, one would expect that the new method applied to Måren would yield even greater compensation per kWh. I also remark that the value awarded in \emph{Sauda} was market-value, not value of use. It was assumed, in particular, that the owners would have to cooperate with a ``professional'' energy company to develop hydropower. This, in effect, halved the compensation awarded, since the Court's decision was based on the premise that the professional company was willing to pay about 50\% of the profit as rent to the owners.}

The case of Måren illustrates an important point, namely that when the traditional method was used, and described as the ``market value'' of waterfalls by the courts, this became a self-fulfilling prophecy. The prices paid in voluntary transactions were influenced by the practice adopted by the courts far more than the other way around. This, indeed, appears to be a general danger in cases when expropriation is widely used for some particular purpose. The prices paid can easily be kept artificially low by developers making use of expropriation as soon as prices begin to rise. In that way, by relying on what is ostensibly ``market value'' compensation, an artificial price level can be established and maintained. 

I mention that in a setting where the owners are politically powerful and can exert undue influence on the compensation process, the effect can be reversed, so that the ``market based'' approach leads to inflated compensation levels, including elements of holdout value. The general point is that the market approach can be turned to the advantage of the most resourceful and powerful groups, particularly in situations when expropriation is widely used for a particular kind of development. In such cases, a market-based approach is not as politically neutral and ``objective'' as its proponents tend to argue.


The potential severity of this mechanism is nicely illustrated by the case of Norwegian waterfalls. In my opinion, preventing such a mechanism from undermining the fairness of a compensation regime is a main challenge associated with regulatory systems that presuppose extensive use of expropriation. Moreover, in case expropriation is used to further economic development by commercial actors, it is likely that the effect will be detrimental to owners, while creating increased financial incentives for developers to favour expropriation. In this way, a vicious circle is established which can make it hard to break out of the ``expropriation loop'', even though alternatives exist that fulfil the same public interests while ensuring far more equitable forms of benefit sharing and participation.

\section{{\it Kløvtveit} and {\it Otra Kraft}}

Following the liberalization of the Norwegian energy sector in the 1990s, the traditional method came under increasing pressure. It was argued to be unjust by owners and it was held to be illogical by engineers working on developing small-scale hydropower.\footnote{See generally \cite{dyrkolbotn96}.} Eventually, legal professionals followed suit and came to the realization that established compensation  rules based on market value could be applied.\footnote{See generally \cite{larsen06}.} 

Indeed, a new market for waterfalls had begun to develop at this point, following the increased interest in small-scale hydropower and the formation of new companies specializing in cooperating with local owners. For transactions of rights to waterfalls taking place in this market, the traditional method of valuation was not used. In fact, waterfalls were rarely sold at all, but rather leased to the development company for an annual fee. Typically, this fee was calculated by fixing a percentage of the energy produced during the year, and compensating the owners of the waterfall by multiplying this with the market price for electricity obtained throughout the year, possibly deducting production specific taxes, but with no deduction of other cost. In effect, owners would get a fee corresponding to a set percentage of annual gross income in the hydro-power plant.\footnote{See \cite{larsen06}.}

Usually, such a fee entitles the owners to 10-20\% of the income from sale of electricity, depending on the cost of the project. Moreover, it is common that the owners are entitled to up to 50\% of the income derived from so-called \emph{green certificates}, a support mechanism for new renewable energy projects, corresponding to the Renewables Obligation in the UK.\footnote{See http://www.ofgem.gov.uk/Sustainability/Environment/RenewablObl/ for further details.} Essentially, and somewhat simplified, the scheme allows the energy producer to collect a premium on his sale of electricity, which, owning to its ``green'' status, is valued more highly by buyers (usually electricity suppliers), who are required to ensure that a certain proportion of the energy they offer to their customers is considered green. In Norway, such a scheme has been talked about for years, but was only put in force in 2012.\footnote{http://www.regjeringen.no/en/dep/oed/Subject/energy-in-norway/electricity-certificates.html?id=517462} Currently, energy producers can claim a premium of about 2 pp per KWh per year, meaning that about a third of the annual income for new renewable energy projects comes from the sale of green certificates.\footnote{While the premium must be expected to go down somewhat as the certificate market matures and more energy producers acquire "green" status, it will certainly remain an important source of extra income for renewable energy producers also in the future.}

Since these leasehold agreements tie compensation to the fate of the hydropower project, several questions arise when attempting to estimate a present-day value of a waterfall on this market. The valuers first have to determine what the most likely project looks like. Then they have to determine what the annual production will be. After this, they must assess the cost of constructing the plant, something that will in turn make it possible to estimate the level of rent likely to be paid to the waterfall owners. Then, since this rent is set as a percentage of the income from sale of electricity and energy certificates, the need arises to stipulate future prices, usually for as long as 40 years (the usual length of a leasehold). Finally, a present-day value can be calculated based on this cash flow.

The appraisal courts began to use just such a model around 2005. The first case of this kind to reach the Supreme Court was \emph{Uleberg}. In the appraisal Court of Appeal, the lay appraisers overruled the juridical judge and awarded compensation based on the new method. The Supreme Court ordered a retrial on a technicality, but it also commented that it supported the adoption of the new method in cases when \emph{alternative} small-scale development was deemed a \emph{foreseeable} use of the waterfall in the absence of the expropriation scheme.\footnote{\cite{uleberg08}.} Since \emph{Uleberg}, the new method has continued to be used in many cases before appraisal courts.\footnote{See generally \cite{larsen06,larsen08,larsen11}, a series of Norwegian papers discussing the new method.}

It is important to note that it was the lay appraisers that pushed for a new method initially, against the judgement of the legal professionals. This shows, in my opinion, that the old system of lay judgement in appraisal disputes still plays a role in Norway. Moreover, it demonstrates that it has positive qualities that should be preserved in the future. However, the new method is certainly not without its own problems. 

Unsurprisingly, it tends to lead to a rather protracted process of valuation, mostly dominated by experts. Moreover, given all the uncertain elements of the calculation, it is typical that the opposing parties produce expert witnesses that diverge significantly in their valuations. While this can be problematic, the fundamental \emph{legal} challenge arises with respect to the no-scheme rule. In particular, what hydropower scheme should the compensation be based on? Several questions arise, as listed below.

\begin{itemize}
\item (1) Is it foreseeable that the waterfall could be used in a hydropower project in the absence of a power to expropriate?
\item (2) If the answer to question (1) is yes, what would such a scheme look like?
\item (3) Is it foreseeable that such a scheme would obtain the necessary licenses?
\item (4) Does the no-scheme rule imply that the project benefiting from expropriation cannot be regarded as a foreseeable scheme for the purpose of compensation?
\item (5) Is the fact that the scheme underlying expropriation obtained a development license to be regarded as evidence that no other scheme would be likely to obtain such a license?
\item (6) How should compensation be calculated if it is determined that no hydropower scheme would have been foreseeable in the absence of the power to expropriate? 
\end{itemize}

In some cases, for instance when the project benefiting from expropriation is not commercially viable but is carried out for public purposes with the help of special state funding, the answer to question (1) might be no. However, in most cases, the question will be answered in the affirmative, since the scheme benefiting from expropriation already serves as an indication that the waterfall can be commercially harnessed for energy. However, here the no-scheme rule comes into play and creates severe difficulty once we reach question (2). For what kind of scheme can be assumed foreseeable all the while we are obliged to disregard the scheme underlying expropriation? 

In most cases so far, the owners have claimed that compensation should be based on the value of a small-scale hydropower scheme. Since such a scheme is likely to be clearly distinct from the expropriation scheme, one might think that the no-scheme rule will not come into play. This, however, is not necessarily the case. It appears, in particular, that the answer to question (3), asking about the likelihood of obtaining licenses, will still depend on how one views the no-scheme rule. It seems, in particular, that anyone who answers question (5) in the affirmative, will be inclined to say that the alternative project could not expect to get planning permission. This is so, such a person might argue, precisely \emph{because} licenses were granted to the expropriating party. This line of reasoning has been consistently advocated by the large energy companies, ever since the new method emerged.\footnote{See, e.g., \cite{klovtveit11,otra11,otra13}. The argument is often sugar-coated by pointing to the reasons underlying the decision to grant a license -- typically energy efficiency -- rather than by focusing on the formal license itself. In this way, one arrives at an interpretation of the no-scheme rule whereby the scheme can perhaps be said to have been disregarded even though one still takes into account reasons why it should be preferred over other schemes.}

Then the question arises: Is someone who reasons like this at odds with the no-scheme rule? It would seem so, but remember the earlier discussion on the no-scheme rule in Norwegian law, where I noted that the rule has tended to be applied much more narrowly along its positive dimension. Following up on this, it can be argued that while the expropriation scheme is to be disregarded for the purpose of compensation valuation, the regulation underlying the scheme -- or at least the rationale behind this regulation -- is nevertheless to be taken into account. If this point of view is adopted, then the conclusion can easily become that alternative development is to be regarded as unforeseeable. The reason, moreover, will be precisely the fact that the expropriation scheme received a development license. 

Indeed, this line of reasoning was given a stamp of approval in the recent Supreme Court case of \emph{Otra II}.\footcite{otra13} Here, the presiding judge made the following remarks, quoting Gulating Lagmannsrett (the appraisal Court of Appeal), expressing his support, and adding a few comments of his own.\footcite[]{otra13}

\begin{quote}
"[....] The Court of Appeal finds it difficult to distinguish this case from other cases when it has been established that alternative development is not foreseeable. It does not seem relevant whether this is the case because the alternative is not commercially viable or because the alternative must yield to a different exploitation of the waterfall" 
I agree with the Court of Appeal, and I would like to add the following: As the survey of the general principles have shown, it is assumed, both in the Expropriation Act, Sections 5 and 6, and in case-law, that only the value of a foreseeable alternative should be compensated. This starting point means that it would be in breach of the general arrangement if a waterfall that can not be used in foreseeable small-scale hydro-power was to be compensated as if it could be put to such use.
\end{quote}

Having used the development license granted to the expropriating party as evidence that alternative development was unforeseeable, the Court needed to answer question (6) by coming up with some alternative way of compensating the owners.  To do so, the Court was again faced with considering the implications of the expropriation scheme. One possibility would be to ensure that the negative and positive dimensions of the no-scheme rule came to be aligned with one another. That is, as the expropriation scheme was used to rule out alternatives, one might then proceed to use it also as the basis for valuation. Indeed, this is what the Supreme Court did. But at this point, the adherence to the no-scheme rule and a market-based approach spelled doom for the waterfall owners. As the presiding judge reasoned:\footcite[]{otra13}

\begin{quote}
Based on the arguments presented to the Supreme Court, I find it safe to assume that there does not today exist any market for the sale and leasing of waterfalls for which alternative development is not foreseeable, but where the waterfalls can be used in more complex hydro-power schemes. The appellants have not been able to produce documents or prices to document the existence of such a market
\end{quote}

The implicit assumption is that in order to value the waterfall according to its potential for hydropower production, a market needs to be identified. It is \emph{not} considered sufficient that the scheme for which expropriation takes place is itself a hydropower project, on the basis of which the  waterfall value could be assessed following exactly the same steps as in the new method. I also remark that it is very hard to imagine how a market of the kind asked for here could ever develop. After all, any alternative buyers are, by the Court's reasoning a few lines earlier, effectively excluded from being taken into account. In this case, if there was to be a market, it would presumably have to be one that emerged entirely out of the benevolence of the expropriating party.

In fact, the Supreme Court's reasoning in \emph{Otra II} serve as an excellent example of the type of reasoning that makes the no-scheme rule highly problematic for cases of expropriation that benefit commercial schemes. On the one hand, the rule can be used to argue that the inherent value of the scheme itself should not provide a basis for calculating the compensation. On the other hand, it can be used to argue that alternatives must be disregarded in so far as they represent the same kind of exploitation as the expropriation scheme, because they are inferior to it according to the state.

When taken to its logical conclusion, this line of reasoning leads to an offensive result; The commercial value of the property is not to be compensated because the optimal commercial use is the use that the expropriating party aims to make of it. Note that the conclusion is not just that this optimal value, inherent in the scheme, should not be compensated. No, the conclusion in \emph{Otra II} was that \emph{no} compensation could be estimated for any use of the same \emph{kind}, since such use was not foreseeable, owing to the absence of a market.

It is certainly possible to argue that this decision represent a misguided application of the no-scheme rule. In effect, the Supreme Court allowed the licences given to the expropriating party to act as evidence that alternative development was unforeseeable, while it used the no-scheme rule to argue that the hydropower scheme for which this planning permission was given could not itself form basis for compensation payment based on market value.  On the other hand, it seems that even if we disregard the scheme completely, it is unnatural to base the compensation payment on the value of a hydropower scheme that is less beneficial, both commercially and in terms of resource efficiency, than the scheme for which expropriation takes place. Such a scheme would not, one must presume, \emph{actually} have been carried out, regardless of the questions of whether or not it would have been given licenses in the absence of a preferable scheme. 

However, it is not seem particularly difficult to determine what would have been a foreseeable use in these cases, if one assumed only that the power to expropriate had not been granted. If so, it would seem all but certain that a scheme corresponding closely to that underlying expropriation would be implemented. This scheme, however, would be carried out on the basis of sharing the commercial benefit with the owners, not on the basis of expropriation. 

But in \emph{Otra II}, this line of thought was also rejected.\footnote{Although this was in part due to the point not having been argued before the Court of Appeal.} Instead, the Court states that a return to the traditional method is in order. However, they do not apply it in the traditional way. Rather, they sanction a modified version of it that moves away from compensation based on the level of stable effect towards compensation based on average effect.\footnote{That is, they replace the low water-flow by the average water-flow in the definition of Qref, c.f., Section \ref{sec:nathp}.}In addition, they also sanctioned the use of a significantly increased unit price compared to earlier times.

What to make of this? In fact, it seems hard indeed to make sense of since, effectively, by relying on the traditional method, the Supreme Court contradicts its own conclusion that compensations should be based on market value. Instead, they rely on a method that, in effect, is based on an attempt to quantify the value of the waterfall as it is being used by the expropriating party in his project. However, by relying on a technical method that has been completely outdated, it becomes difficult to assess the outcome properly, at least for a non-expert. This is so even after the modifications have been implemented, which make the method appear somewhat less irrational from a physical point of view.

But it is still noteworthy that the Supreme Court prefers the obscurity of the traditional method, as an established custom, over the explicit conclusion that it simply is not tenable to adopt the ``value to the owner'' principle in cases like this, as least not as that principle is construed in Norwegian law.

In any event, I think there is good reason to be critical of the Supreme Court for sanctioning the view that alternative development was unforeseeable in {\it Otra II}. Still, it is not possible to escape the fact that this reflects a general tendency in Norwegian law, whereby the positive dimension of the no-scheme rule is much weaker than the negative part. Even if it appears unreasonable, it might very well be a correct application of national law. Moreover, it could very well have been that alternative development was unforeseeable for \emph{some other reason}, for instance because the only commercially viable exploitation was the scheme planned by the expropriating party. In this case, the problem of how to compensate the owners in the absence of an alternative form of exploitation would still arise. It is this question, in particular, which seems entirely unsatisfactorily resolved under an application of a ``value to the owner'' principle.

This is witnessed by \emph{Otra II}, and, in fact, it appears that the Supreme Court, in their decision  \emph{not} to follow their own reasoning to its logical consequence, makes quite a powerful statement. For all intents and purposes, the Supreme Court \emph{rejects} the "value to the owner" principle, but they obscure this by wrapping it up in the traditional method, which is deeply flawed. However, the problem it attempts to solve appears significant, and it pertains directly to the question discussed more generally in Section \ref{sec:noscheme}, namely how to compensate owners that loose their land to commercial schemes. 

It seems that even the fiercest supporters of limiting owners' right to compensation tend to find it too offensive to apply this principle when it leaves the owners with no form of compensation in cases when they are forced to give up property to purely commercial undertakings. Indeed, such a practice would surely also be in breach of the human rights law. In these cases, the subjective aspect of the ``value to the owner'' principle is impossible to maintain. If the commercial value falls to be disregarded for no other reason than the fact that the state happens to have granted planning permission to the expropriating party rather than the owner, this is not only dubious with respect to human rights protecting property, but also appears to be a case of \emph{discrimination}, e.g., as prohibited by ECHR Article 14.

The problem does not arise when the buyer sees value in the property that is of a different \emph{kind} than that realizable by \emph{any} private owner. In this case, the rule simply states that the owner should not be able to demand that ``public value'' is transformed into commercial value just for him. This appears like a reasonable principle. But when there is commercial value already present on the "public" side of the transaction, it seems completely unwarranted that the public should be allowed to transfer this value from the owner to someone else without compensation. Thus, it seems that more accurately and acceptably, the ``value to the owner'' principle should be thought of as a ``commercial value'' principle. It seems, in particular, that the principle need to be stripped of any suggestion that a preferential financial position is to be awarded to whoever benefits from expropriation.\footnote{Exceptions might be possible to imagine, but, one would think, only when they can be construed as falling under the ``public value'' banner in some way.}

It seems unfortunate that this aspect has not been made explicit, and the difficulties that arise in the absence of this nuance are nicely illustrated by the case of Norwegian waterfalls. Still, as the case of \emph{Otra II} indicates, an interpretation of the ``value to the owner'' principle along less offensive lines is in reality already in place with regards to Norwegian hydro-power. Here it seems that ``value to the owner'' has in fact \emph{never} been applied in the traditional way. Hopefully, rather than obscuring this fact by relying on an unsatisfactory and artificial method for calculating the compensation, the future will see further developments that recognize the need for new principles. 

It should be recognized, in particular, that as the law has been applied for the last 80 years, despite its grave flaws and injustices, there has always been an implicit recognition in Norwegian law that the owners of waterfalls are \emph{entitled to their share} of the commercial benefits of hydropower. 
In fact, in the recent Supreme Court case of \emph{Kløvtveit}, a novel approach along the lines I am advocating was applied in circumstances similar to that of {\it Otra II}.\footcite{klovtveit11} The conclusion here too was that alternative development was not foreseeable. However, unlike in \emph{Otra II}, the lay appraisers in the Court of Appeal had compensated the owners based on the fact that they regarded it as foreseeable that in the absence of the scheme, the waterfalls would have been exploited in exactly the same way, except that it would have happened in the form of \emph{cooperation} between the owners and the expropriating party. By this line of reasoning, the Court effectively seems to have adopted a more modern ``commercial value'' principle, to replace the traditional method. 

For commercial projects, it seems that in the absence of a power to expropriate, any rational buyer would look to cooperate with the owners. This would not necessarily be a safe assumption to make for non-commercial projects. Such projects may fail to provide the necessary incentives for cooperation, even though they should nevertheless be carried out in the public interest.

I mention that \emph{Kløvtveit} was discussed in \emph{Otra II}. But the presiding judge chose to focus on what he regarded as the ``practical problems'' associated with the prospect of cooperation and a compensation award calculated on this premise. The cooperation model was not the center of attention in the case, however, so one can only hope that \emph{Kløvtveit}, rather than \emph{Otra II}, will become the influential precedent for future cases.

\section{Conclusion}

In this chapter I have presented the current compensation regime associated with waterfalls and I have related it to the broader question of how compensation should be calculated when commercial companies benefit from the property that is taken. I focused particularly on the no-scheme principle, which plays an important role in this regard in many jurisdictions, including in Norway. But I also emphasized another aspect particular to the Norwegian system, namely its reliance on the judgement of lay appraisers. 

I noted that the appraisal courts would typically operate largely unconstrained by specific evaluation rules, as they were directly guided by the Constitution and its requirement that ``full compensation'' had to be paid. I argued that this system was both flexible and capable of facilitating broad fairness considerations. The notion that constitutional absolutism was a rigid system, I argued, is largely unfounded. The system had a procedural flexibility that should not be underestimated and which served as a counterbalance to its seeming adherence to a strict dogma.

In fact, when moving on to consider the case of waterfalls in more depth, I noted how this flexibility was used to great effect. It allowed the appraisal courts to ensure that some benefit sharing was maintained in hydropower cases even after the regulatory system was transformed so that benefit sharing following expropriation would be hard or impossible to achieve in a system based on a legal formalization of the no-scheme principle. For over 80 years, the courts happily deviated from it entirely when awarding compensation for waterfalls. Remarkably, this practice continued even after legislation was passed that provided much more specific guidelines to the appraisal courts, and which seemingly enforced a strict no-scheme principle in Norwegian law.\footnote{More generally, however, I noted how this legislation, and the Constitutional battles that followed it, has lead to a development whereby the appraisers are somewhat marginalized and the Supreme Court itself has assumed greater power in directing them, by providing their own interpretation of a body of legislation that contains many specific rules that are hard to apply to concrete cases in a uniform fashion.}

However, as the appraisal courts were marginalized by increasing levels of top-down control, first by the legislator and later by the Supreme Court, the method that was developed to compensate waterfalls would itself develop into a fixed and rigid rule. It was not adapted, in particular, to reflect technological and economic progress. Since these were particularly rapid and ground-breaking in the energy sector, the result was a very severe mismatch between the real value of waterfalls and the compensation paid following expropriation. 

In the final part of the chapter I then considered recent cases where the traditional method has been abandoned in favour of a market-based approach which is based on the general rules governing compensation today. I found that while these cases tend to result in payments that more closely reflect actual commercial values, they raise severe problems of their own. Here, in particular, the no-scheme principle re-emerges on the scene with full force, becoming a very effective tool for those who seek to argue that hydropower development is not a fruit of property but belong to those who obtain expropriation and development licenses from the state. 

If such arguments are successful, the market-value approach can lead to worse outcomes for local owners of waterfalls than what they would be entitled to under the traditional method. The deeper question that arises, of course, is the following: what is equitable benefit sharing in these cases, and how can it be ensured? A second question is whether owners can in fact {\it demand} some level of benefit sharing on the basis of human rights law. This question is now coming into focus in Norway, as the Supreme Court's decision in {\it Otra II} has been brought before the ECtHR.
 
It is my opinion that the best way to ensure benefit sharing under a compensatory approach is to revive the old system of an independent appraisal procedure relying on the discretion of lay people from the local area. The most important aspect of this, I believe, is that it enhances the democratic legitimacy of the compensatory approach. It is clear that there is a great deal of uncertainty in the kinds of calculations one must engage in to assess the commercial value of a waterfall. Therefore, the temptation to rely blindly on experts and special rules that are not properly understood becomes great. This might reduce the uncertainty involved, but only to some extent. Moreover it also highly increased the risk of unfairness and opens up the possibility that powerful interests can sereptisiously usurp the procedure for their own interests. Compared to this, a system based on direct fairness assesments carried out by noraml people, on the basis of (hopefully) neutral information provided by experts, might well be the best option.

However, I think the inherent difficulty in devising appropriate compensation mechanisms for commercial potentials suggest that the compensatory approach might be misguided altogether. In addition, as soon as one begins to look at the social function of property, and its role in human flourishing, it seems that any kind of financial compensation is going to provide an inadequate reply to deprivation of commercially interesting property. Such a system speaks volumes about {\it who} society deems capable of carrying out commercial projects. The discrimination suffered by property owners of the {\it wrong kind}, should also not be underestimated. 

In light of this, I think it is appropriate to consider alternatives to expropriation in cases when economic rationales dictate economic development. Interestingly, the Norwegian system has an entire legal framework in place that can elegantly facilitate such a shift, should enough political be mustered to compel government and developers to make use of it. In the case of hydropower development, it already being put to the test in an increasing number of cases for substantial development projects, as local developers tend to shun away from outright expropriation of property belonging to unwilling neighbours. 

The land consolidation mechanisms that can be used to facilitate compulsory development in these situations form part of an ancient semi-juridical system of land management in Norway. In my opinion, this framework also points towards the future, as it provides a highly flexible approach for dealing with property and economic development under varying degrees of compulsion. In my next and final chapter I will present it in more depth and argue that it can often provide solutions that are both more effective and more equitable than solutions arrived at in a system that relies on expropriation. The compensation issue, in particular, is resolved simply by giving unwilling owners low-risk financial instruments tied to the development that is ordered to take place on their property.


\include{Chapter5/endelig_5}
\chapter{Conclusion}\label{chap:7}

\begin{quote} \small
Proudhon got it all wrong. Property is not theft -- it is fraud.\footnote{\cite[252]{gray91}.}
\end{quote}

\begin{quote} \small
That's what makes it ours -- being born on it, working on it, dying on it. That makes ownership, not a paper with numbers on it.\footnote{\cite[33]{steinbeck39}.}
\end{quote}
\noo{
\begin{quote} \small
Gode bønder nå til motstand reises. Enhver som må gi fra seg farsarvi til kongens grever mener det er det rene ran. \\
Good farmers prepare to resist. Those who must give up their father's inheritance to the lords of the king consider it plain robbery.\footnote{From the so-called {\it Bersogliviser}, an old Norse poem addressed to King Magnus (c. 1024 – 25 October 1047), who was accused of treating his subjects badly, including taking their properties, see \cite[259]{titlestad12}. The King listened to the farmers' protest and was thereafter referred to as Magnus ``the Good''.}
\end{quote}
}
%There is reason to think that fraudulent behaviour requires some, but not too much, in the way of intellectual refinement. Arguably, therefore, the case can be made that 

%The concept of property has received its share of criticism, but it remains omnipresent.

%
\noo{ t. Further to this, legitimacy-enhancing alternative to expropriation were considered. Specifically, the thesis argued that ideas taken from the research on common pool resources could inspire novel solutions for local self-governance that could obviate the need for using eminent domain to ensure economic development. 



%A concrete proposal along this line, due to Heller and Hills, was considered in some depth. 

%An important tension was identified within this proposal, between the ideal of self-governance and the danger of abuse at the local level. This, in turn, reinforced one of the key design principles of common pool resource management, namely that institutional arrangements need to be sensitive to the local context. This set the stage for the next part of the thesis, where the issue of economic development takings was looked at concretely, from the point of view of waterfall expropriation foor hydropower development in Norway. 

%Hence, to formulate a one-size-fits-all institution to replace eminent domain for economic development is unlikely to work. At least, such a solution would have to be far more flexible than the proposal made by Heller and Hills, which in the end offers owners very little in the way of participation in decision-making regarding economic development on their properties.


In such cases, it might well be that the balancing of different reasons for and against the taking has taken place prior to the decision to interfere with property. The plans for development themselves may well precede any specific property-oriented implementation steps, such as the use of eminent domain. It might even be that democratically accountable bodies responsible for land use planning have already concluded that some local community interests must give way to other interests.

In these cases, it might be tempting to argue that a narrow takings narrative is appropriate because it pertains only to the final implementation step, which is the only one that involves property rights. But this argument, I believe, rests on a flawed perception of what property is, and should be, in a democratic society. Invariably, property has to do with decision-making and power. If the decision-making process does not grant significant self-determination rights to affected property owners, a taking is already in progress. It might be justified, but it is still a taking. 

More worryingly, it is clear that this kind of taking carries with it a great potential for differential treatment, discrimination, and corruption. The traditional takings narrative does a good job of setting up a framework that makes it difficult to simply pay higher compensation to certain kinds of people, without offering any justification. But with respect to the aspects of taking not recognised, e.g., pertaining to what role the owner has during the planning stages, differences in treatment will not even be notices. But if property is owned by the right sorts of people, then invariably it {\it will} come with considerable decision-making power. 
}


\noo{
This has been a thesis in two parts that has addressed economic development takings from two distinct angles. In the first part, a theoretical discussion was provided, which started from the notion of property itself and gradually made its way towards the question of legitimacy of takings by exploring a broader meaning of property than that typically embraced by the law. The aim was to argue {\it why} property should be protected, while also providing a template for recognising and discussing special issues that arise for economic development takings. 

Following up on this, the thesis explored different approaches to legitimacy, culminating in a concrete proposal for a legitimacy test, inspired by the work of Kevin Gray. Furthermore, the thesis presented a template for formulating alternatives to outright expropriation, inspired by the work of Elinor Ostrom and others on local self-governance of common pool resources. The aim here was to suggest possible ways to restore legitimacy in cases when an either-or approach to legitimacy of property interference is not appropriate. Better than that it should be interference bottom-up than interference top-down, provided adequate safeguards are put in place to protect local minorities from abuse.

The discussion remained quite abstract in the first part of the thesis. By contrast, the second part of the thesis approached the question from the opposite angle, through an in-depth case study of takings for hydropower development in Norway. The thesis first presented and discussed the social, economic, and legal context of such takings, before proceeding to study practices and rules relating to expropriation of waterfalls in greater depth. This also provided an opportunity to apply the legitimacy test proposed in the first part of the thesis. The conclusion was that current expropriation practices in Norway fail in this regard, as the property rights of local people are very weakly protected in situations when large commercial companies wish to undertake hydropwoer development.

However, the thesis went on to observe that the land consolidation procedure represents a possible alternative to expropriation, one that is already being used extensively in Norway to facilitate owner-led hydropower development. Here the emphasis is on organising joint use of property, possible involving compulsion whereby owners must partake in development against their will. The procedure is judicial in nature, moreover, so also provides safeguards against abuse by local elites. Moreover, it is primarily a service to the owners and their properties, but is required to actively promote solutions that are in the public interest. After recent changes in the law, the scope of obligations that can be imposed on owners for the common good is also likely to increase, making the land consolidation alternative appear like a realistic option even in large-scale development situations that will necessarily also involve non-local actors. The thesis argued that land consolidation is a good example of the kind of institution that can function as an alternative to expropriation in hard cases. It has already proven itself in a setting of egalitarian property, in communities where property rights are held by local people. The possibility of employing it in situations where this is not the case remains more uncertain, but is an interesting prospect. Arguably, this would require including non-owners in the process as well, on the basis of connections to property that might not otherwise be recognised as property interests in the law. Specifically, it seems that one might want to rely on broader social function ideas of property in relation to consolidation, even if a more traditional account is maintained in other areas of the law. In relation to financial law or tax law, one might well wish to entertain an artificially narrow concept of property for efficiency reasons, even if a much broader concept is required in the context of compelled economic development. Moreover, the land consolidation procedure appears attractive because it fills a gap between planning and property, a gap that otherwise appears susceptible to infiltration by powerful commercial interests.

%The thesis made the case that these issues are important enough to suggest that economic development takings should be approached as a special category, also in the law.

%From this, the theoretical discussion continued by an exploration of different ways to approach the legitimacy question for such takings. 
}

%In this final section of my thesis, I would like to take a step back to briefly follow two broader threads that I believe run through my thesis. 
This has been a thesis in two parts, each of which have approached the issue of economic development takings. The first part took a theoretical approach, starting from the notion of property itself, to answer the question of {\it why} it should be protected. This, in turn, gave rise to a framework for assessing the legitimacy of economic development takings, and for formulating alternatives to it that could obviate the need for dispossessing current owners.

The second part of the thesis approached the issue of legitimacy concretely, by giving a case study of takings of waterfalls for hydropower development in Norway. The political, social and economic context was also analysed, leading to an application of the Gray test formulated in the first part of the thesis. Moreover, the case study considered the possibility of alternatives to expropriation, by assessing the Norwegian institution of land consolidation, which is now used extensively by local owners who wish to undertake hydropower development themselves.

% The current approach to takings for waterfalls in Norway were found wanting, with the current takings practices appearing to fail several, if not all, the points set out by the Gray test. However, the final chapter of the thesis studied a possible alternative to expropriation, which paradoxically is also actively used in the context of hydropower development in Norway. This framework, however, is so far used only when some of the local owners themselves wish to carry out development, and need to sort out their internal disagreements, possible even by compelling unwilling neighbours to join them in their endeavours. 

%The thesis argued that this alternative, although not necessarily applicable in other contexts, provides an interest

%More generally, and especially in its proposal for a possible solution, I hope the thesis has made a valuable contribution to the study of economic development takings. 

In the following, I offer a brief summary of the main points discussed in each chapter. While doing so, I hope to shed further light on a broader thread that runs through the work done in this thesis, pertaining to the importance of social justice considerations in takings law and the function of private property as an anchor for local self-governance and sustainable resource management in democratic societies.

%To conclude the thesis, I will take a step back to consider two broader threads that I believe run through my work, pertaining to the nature of property and how to ensure that it functions as a force for good in democratic societies.

%The first concerns the many senses of taking that have been brought into focus throughout the analysis, while the second concerns ways in which the law can help to give back some of the legitimacy that is typically lost when eminent domain is used to facilitate economic development.

%\section{Summary of main points}

\subsubsection*{Property theory}

%To motivate the theoretical work done in chapter 2, the thesis used the example of Donald Trump coming to Scotland to build a golf resort on the Aberdeen coast, on a site of special scientific interest. This looked like it was going to result in the compulsory acquisition of land, as the local authorities were asked to considered issuing compulsory purchase orders in support of Trump's project. However, in the end, the plans did {\it not} require takings, as Trump was able to make creative use of property rights he acquired voluntarily, against the complaints of his neighbours.

%Importantly, this turn of events did not make the example less relevant to the thesis. Rather, it served to highlight that the question of economic development takings cannot be looked at in isolation from the surrounding social and political context. Clearly, the grievances of the local people in Balmedie who were marginalised by Trump did not turn on whether or nor he formally condemned property, but rather on the manner in which the social functions of property in the local area conclusively changed in his favour, at the expense of others. Trump's actions, then, arguably still amounted to takings,  even if not in a formal legal sense of the word. Moreover, the example of Trump coming to Scotland allowed me to emphasise the importance of context when assessing both the nature of property, the many ways of taking, and the meaning of protecting owners against predation.

%The protection sought by those who opposed Trump's golf course did not target their entitlements as individuals. Rather, it targeted the community, as the owners felt it would be detrimental to the community, and to their lives, if Trump was allowed to redefine the social functions of local property. After Trump decided not to pursue compulsory purchase, protecting the property of these members of the community became a question of {\it restricting} the degree of dominion that Trump could exercise over his own property. Hence, under a conventional and overly simplistic way of looking at these matters, protecting property became tantamount to restricting its use, a seeming paradox.

To arrive at a theoretical framework for discussing economic development takings, chapter 2 considered various theories of property. The chapter noted that there are differences between civil law and common law theorising about ownership, but concluded that these differences are not particularly relevant to the questions studied in this thesis. In particular, the chapter observed that neither the bundle theory, dominant in the common law world, nor the dominion theory, taught to many civil law jurists, helps to clarify the distinguishing features of economic development takings. Specifically, traditional thinking both in civil and common law jurisdictions has a tendency to abstract away from the sensitive social and political context of such takings; there is a clear tendency, especially in takings cases, that private property is approached as a set of individual entitlements rather than an interconnected web of social functions. This, in turn, means that the broader societal effects of takings are often not considered when addressing the legitimacy of property interference against general protection principles found in constitutional and human rights law.

The thesis set out to arrive at a theoretical foundation for thinking about property that would support a more comprehensive approach to takings in the context of economic development. The chapter focused especially on property theories that emphasise that property plays a crucial role in many social and political relations within a society. Such social function theories, it was argued, provide us with crucial {\it descriptive} insights into the workings of property and its role in the legal order. In this regard, the thesis advanced a position different to that adopted by many previous scholars in the social function tradition, by arguing that we should actively try to decouple descriptive insights from normative claims about property. It was argued that it we succeed in doing this, the social function theory could serve as  common ground for further value-driven debates that cross ideological divisions.

Following up on this argument, the chapter further clarified the normative starting point adopted in this thesis, by expressing support for the human flourishing theory proposed by Alexander and Pe\~{n}alver. This theory is based on the premise that property rights {\it should} be integrated in the legal order in such a way that they enable -- possibly even compel -- individuals and their communities to participate in social and political processes. Specifically, the chapter argued that property and its associated social functions should be approached as a crucial anchor for democracy, since private property can potentially provide a powerful anchor for participatory decision-making on the basis of mutual obligation and respect, targeting owners and non-owners alike. Specifically, the human flourishing theory rightly emphasises the {\it duties} associated with private property, especially those duties that are directed at non-owners and which therefore necessitate inclusive institutions for collective decision-making in cases when conflicting interests are at stake.

Importantly, the human flourishing theory emphasises how such institutions can arise from the structure of property itself, rather than as a purely external framework imposed by the authority of the state. For this reason, the human flourishing theory contains a further important insight, concerning the scope of the state's obligation to protect property. In particular, the human flourishing theory asks us to acknowledge that protecting property also implies a commitment to protecting the right to some degree of self-governance for local communities, where the stake an individual has in decision-making processes are reflected in their opportunity to influence those processes. Being an owner, with all the rights and responsibilities this entails, then places an individual right at the center stage of decisions pertaining to the use of property, including in situations involving plans for large-scale economic development. This highlighted the relevance of property theory to the question of legitimacy in economic development takings.

Following up on the theoretical argument, chapter 2 sketched out what a social function perspective could imply in practice by considering {\it Kelo}, the paradigmatic case of a taking for economic development in the US. It was observed that the disagreement within the Supreme Court seemed to turn, in part, on the willingness of the justices to adopt a social function perspective on the case. Justice O'Connor, who led the dissenting minority, departed from a formalistic, entitlements-based approach, when she argued that takings for economic development should be prohibited because such takings would systematically bestow benefits on powerful commercial interests at the expense of owners from weaker social groups. In addition, both Justice O'Connor and Justice Kennedy, who voted with the majority to uphold the taking, emphasised the danger that the commercial incentives associated with using eminent domain to facilitate economic development could distort decision-making processes, creating a democratic deficit within government institutions endowed with eminent domain powers. These two themes, pertaining to the social fairness and procedural legitimacy of takings for economic development, has remained in focus throughout this thesis.

\subsubsection*{Testing for legitimacy and looking for alternatives to eminent domain}

In chapter 3, I gave a positive-law presentation of the legitimacy question that arises for economic development takings.
I considered existing approaches from the UK, the US, and at the ECtHR. As explained in the introduction to the thesis, this choice led me to consider a set of jurisdictions that adopt distinctly different approaches to the legitimacy question, yet remain easily comparable both to each other and to Norway, which was investigated in more depth in the subsequent case study. 

In the UK, the debate on legitimacy is traditionally structured around procedural rules, while in the US the traditional starting point, at least in before the higher courts, is a substantive assessment of the merits of takings. In the end, the chapter argued in favour of an approach that combines procedural and substantive standards, with the intention being that the latter should be used also to assess the fairness and democratic merit of the decision-making procedure, not merely the outcome. Furthermore, it was argued that such an institutional approach to fairness has started developing at the ECtHR in response to the newly introduced framework for pilot judgements applied in cases that might indicate systemic problems at the state level.

The chapter went on to propose a concrete heuristic for assessing the legitimacy of economic development takings against any standard that implies a commitment on the state's part to ensure a ``fair balance'' between the opposing interests involved. The chapter referred to this heuristic as the Gray test, since it is strongly influenced by previous work done by Kevin Gray, a leading UK scholar who specialises in land law. Indeed, the proposed heuristic includes a list of legitimacy indicators provided by Gray in his work on the legitimacy of takings. In addition to these original indicators, the thesis went on to add three additional points, based on the work done in this thesis. The resulting heuristic should be able to identify many, if not most, cases of eminent domain abuse, especially those that occur in the context of economic development. Furthermore, the Gray test incorporates several aspects of the social function theory of property, looking at the legitimacy issue from a broad vantage point of how the interference in property will affect communities and local democracies, not merely individual owners. This makes the Gray test stand out in the literature on takings so far, especially in the UK and the US, where individualistic perceptions of what property is have tended to dominate.

Following up on the discussion of legitimacy testing, chapter 3 considered the question of how to increase legitimacy without giving up on the idea that the collective should be entitled to push adamantly for economic development on private land. It was argued that the work done by Elinor Ostrom and others on common pool resources provides a suitable starting point for making institutional proposals to achieve a better framework for making decisions about economic development on privately owned land. Specifically, the chapter briefly presented Ostrom's design principles for local self-governance of natural resources, which can be used as a starting point also for designing procedures to replace eminent domain for economic development.

The chapter went on to consider a proposal that has already been made along these lines, namely the idea of Land Assembly Districts, due to Heller and Hills. This proposal was analysed in depth, and the chapter pointed out some potential problems with it, including a tension between the overarching goal of self-governance and the perceived danger of tyranny by local elites and abuse of local minorities. In the end, the chapter concluded that a single institutional framework is unlikely to work in all settings; institutions for self-governance need to be attuned to local conditions, so they cannot be made too general or justified in overly theoretical terms. Indeed, sensitivity to local conditions is one of the key design principles proposed by Ostrom, and should influence also the proposals we come up with for institutional reform in the law of takings.

This observation marked the end of the first part of the thesis. In the second part, the thesis considered the case of Norwegian hydropower. This led to an analysis of legitimacy of takings for this purpose along the lines of the Gray test, as well as a case study of the land consolidation courts and their power to set up local institutions for self-governance that can obviate the need for eminent domain. In this way, the second part went on to apply key aspects of the theory developed in the first part, while exploring further the idea that social functions run as a common thread through individual property rights.

\subsubsection*{Norwegian waterfalls and their social functions}

Chapter 4 introduced the case study and provided background information that placed it in a broader context with respect to Norwegian law. The legal and regulatory framework surrounding hydropower development was discussed, and its history was traced back to pre-industrial times. The chapter emphasised that local rights to hydropower have a long tradition in Norway, with communities of smallholders typically holding the rights to harness power from local rivers in common, as incidents of their shared ownership of the surrounding outfields. This arrangement dates from a time when grist mills and saw mills were important to many local communities, whose proprietary rights provided them with an opportunity to benefit from local resources.\footnote{As noted in chapter 4, Norwegian tenant farmers also enjoyed such rights before they started buying back their farms in the 17th and 18th century. In particular, tenants would typically have quite extensive rights to natural resources found in the outfields. This highlights that private ownership of land was always imbued with egalitarian social functions in Norway, not simply an encapsulation of individual entitlements for the privileged classes.} 

After the advent of the industrial age, and particularly following the Second World War, the state took the view that hydropower was a public good that should be exploited for industrial development in the interest of the general public. This resulted in a tension between local self-governance, rooted in private property, and central management, rooted in  the authority of the state, that now permeates the law of hydropower. This tension became particularly severe after the liberalisation of the electricity sector in the early 1990s. This reform reorganised hydropower development as a commercial pursuit, meaning that when the state uses its regulatory power, it often bestows benefits on commercial companies at the expense of local resource owners. At the same time, local owners themselves were empowered by the liberalisation reform, since they could now themselves engage in commercial hydropower development. This was made possible by the fact that a market for electricity was set up, founded on the idea that all actors should have access to the electricity grid on non-discriminatory terms.

Chapter 4 discussed the resulting system in some depth, addressing also the question of whether or not the liberal market functions as intended. It was argued that the energy reform has marginalised the municipality governments, who used to play an important role because they were in charge of public utilities for the supply of electricity in their own local district. The result, it was argued, has been a dramatic concentration of power in the hands of commercial companies partly owned by the state, a centralisation effect that threatens to undermine the market-stimulating intentions behind the reform. Specifically, the chapter provided a critical assessment of the extent to which the regulatory framework is able to accommodate new actors and facilitate competition on non-discriminatory terms. Special attention was directed at the position of waterfall owners and the companies that specialise in cooperating with them. It was argued that owners and small-scale development companies currently suffer under a regime that tolerates, and sometimes encourages, discrimination against less dominant market actors.

The chapter went on to present how smaller market actors now tend to organise themselves. First, the chapter presented an early organisational model that emphasised respect for property rights, local communities, and the environment. According to this model, owners and local communities would typically retain controlling stakes in development projects, with external partners providing capital and technical expertise either as a paid service or in return for a minority stake in the enterprise. Many owner-led hydropower plants have since been built according to this model, demonstrating its commercial viability. 

However, later developments have led to an increased concentration of power and ownership also in the small-scale segment of the electricity production sector. Specifically, external partners now typically demand controlling stakes in development projects and do not concede to organisational provisions meant to protect local communities and environmental interests. Effectively, the owners and their communities are increasingly asked to remain on the sideline. In many cases, local people have agreed to this in return for a promise of higher compensation in the future, calculated as a percentage of income from the generation of electricity. However, it has turned out that such promises have often been made with little realism. Indeed, as shown in chapter 4, there have been some high-profile cases of speculation in this market, where upstart energy companies have entered into a large number of agreements with local owners without carrying out much actual development. %Recently, large portfolios of agreements have been sold on to third parties, including multi-national corporations.  who have then been known to re-negotiate the terms with local owners under the threat that no development will take place unless the agreements are made more favourable to the external partner.

In general, chapter 4 argued that recent developments in the small-scale sector marks a sharp departure from initial visions of this sector, visions that emphasised values such as local self-governance and environmental sustainability. Instead, the values and practices of leading small-scale actors have become increasingly similar to those that dominate among more established market actors. The chapter went on to argue that this might be a contributing reason why small-scale development now appears to be falling out of political favour. Today, critical voices claim that large-scale development is better, not only because it is more commercially optimal, but also because it is more environmentally friendly. The specific argument provided for this claim is that small-scale projects affect a greater number of square meters per energy unit they produce. As noted in chapter 4, this is no doubt true, since many small-scale plants will typically be required to match the energy output of a single large-scale plant. Unfortunately, issues relating to more substantive notions of sustainability, as well as issues relating to property rights, benefit sharing, and local participation in decision-making, appear only at the fringes of the present debate. Hence, policies are now at risk of being formulated on the basis of an incomplete picture of what is at stake. As suggested in chapter 4, this makes the present thesis a timely scholarly contribution that could also inform policy making on hydropower in Norway.

%By offering this diagnosis, chapter 4 foreshadowed many of the issues that was brought into focus in the subsequent chapter. There, the thesis looked specifically at expropriation of waterfalls, by tracking the position of owners under the current regime. It was argued that the law as it stands is based on a perspective that blocks out both the significant commercial interests of the taker, as well as the significant social functions and obligations of the original owners. The main point made was that the issue of expropriation invariably raise questions that seem difficult to address without adopting a broader view, which also takes into account the owners' communities and their role within it.
\subsubsection*{Expropriation as an automatic consequence of a development license}

In chapter 5, the legal framework surrounding expropriation of waterfalls was studied in depth. In addition to presenting the law, the chapter also discussed administrative practices developed by the water authorities. The chapter argued that the present system is based on the presumption that private property embodies private values, while public values are to be pursued through regulation, if necessary also by redistributing or negating property rights. In effect, chapter 5 tracks how this perspective has shaped the law of expropriation of waterfalls. In particular, the chapter made clear that the notion of property presupposed by Norwegian expropriation law is not based on a social function understanding of what property is and why it should be protected against interference. 

At the same time, the traditional view on public values implied a sharp distinction between commercial and public uses of property. Hence, the law of expropriation initially tended to restrict the takings power to narrowly defined purposes that clearly served the common good. This was the case also in the context of water law, where the government did not initial have any authority to expropriate waterfalls for the purpose of developing hydropower. However, as noted in chapter 5, the increasing focus on electricity production as a public service resulted in the introduction of new authorities to expropriate waterfalls for public utilities. Still, expropriation could not take place for commercial purposes or in favour of private companies.

The restrictions placed on the power to expropriate waterfalls gave legitimacy to the legal framework. Still, the increasing centralisation of the energy sector and the increasing scale of projects seen after the Second World War led to increased tension surrounding new hydropower projects. Tensions came to a high-point in the case of {\it Alta}, when the indigenous Sami population from the north of Norway objected to a project that would have detrimental effects on their livelihoods and, as they argued, their very way of life. In collaboration with environmental groups, they launched a legal challenge directed at the development license, resulting in one of the most comprehensive cases ever dealt with by the Supreme Court. In the end, the development interests triumphed, and the regulatory framework surrounding hydropower and expropriation was given a stamp of approval. 

Even though the development license was upheld, the {\it Alta} conflict contributed to increased awareness of indigenous rights in Norway, starting a development that has since resulted in better protection of Sami rights within the legal order. No similar effect was observed with respect to the law of hydropower; the precedent set by {\it Alta} remains leading in disputes over the legitimacy of licensing decisions and expropriation orders. However, as noted in chapter 5, the context of waterfall expropriation has changed dramatically since the liberalisation of the electricity sector. Unlike before, expropriation is now regularly ordered for commercial purposes, to the benefit of limited liability companies.\footnote{For a discussion of how the law was changed to make this kind of expropriation possible, see chapter 5, section ....}

This change of context has failed to make any discernible impact on the administrative practices adopted by the waer authorities. Rather, the traditional approach, which evolved when electricity generation was still organised as a public service, remains in force. As discussed in chapter 5, the key feature of these practices is that they render expropriation a {\it de facto} automatic consequence of obtaining a development license; if an energy company manges to obtain a development license for a large-scale project, the right to expropriate waterfalls is granted to it by default. As shown in chapter 5, this leaves the owners and their local communities in a precarious position in such cases. Essentially, they enjoy very limited protection under administrative law, not substantially more than arbitrary members of the public, and in some cases far less than the members of organised interest groups.

%This did not mainly apply to the question of the authority to expropriate, which was hardly raised at all in the period between the reversion controversy of the early 20th century and the market-reform of the early 1990s. It applied mainly to general procedural rules. Here the Supreme Court adopted a stance whereby these rules were themselves considered largely ``discretionary'' in nature. Hence, it would fall under the authority of the executive to determine their scope and application in concrete cases.

Chapter 5 discussed the practical fallout from this in depth when considering the case of {\it Jørpeland}. This case served to illustrate that administrative practices currently in place serve to make expropriation available as an effective tool for powerful market players that wish to gain the upper hand in competition for resources owned by weaker groups. Hence, the current state of expropriation law in Norway adds weight to Justice O'Connor's prediction in {\it Kelo}, where she predicted permitting economic development takings would give powerful commercial interests the opportunity to take property from weaker members of society.

Chapter 5 also noted that in the case of {\it Jørpeland}, the Supreme Court explicitly denied that established practices were in need of revision. Moreover, the Court refused to reconsider the established interpretation of the scope of procedural rules in hydropower cases. Effectively, a range of general rules of administrative law do not apply in hydropower cases, since the special legislation used to regulate licensing of hydropower take precedence. In a dramatic departure from the situation as it had been before commercial expropriation was permitted, this hydropower-specific legislation is now also understood to cover the expropriation proceedings, which are not considered separate from the licensing proceedings at all.

\subsubsection*{Land consolidation as an institutional framework for local self-governance and sustainable resource management}

In chapter 6, the Norwegian system of land consolidation was presented. It was shown to be a unique institution that combines a property-based approach to land management with a broad authority for the courts to intervene in order to organise collective decision-making and promote sustainable resource management at the local level. The chapter went on to show that the system of use directive has become widely used in recent years to facilitate hydropower projects organised by local owners. In some cases, it is also used to deprive some owners of their holdout power by compelling them to participate in a development project that they oppose.

In consolidation cases, interference in property does not take place because property rights must give way to public interests. Rather, consolidation relies on proof that benefits will outweigh harms at the local level, with respect to each affected property. However, this requirement targets the property as a functional unit, irrespective (in principle) of the specific interests of its current owner. Hence, depending on what functions are regarded as more important, the stated desires of the owner might have to give way to other priorities. At the same time, the compensatory perspective is abandoned; the owners' financial entitlements are subordinated to the interests found to be inherent in their properties. If the property as a functional unit does not suffer a loss, no compensation is payable to the owner as an individual. 

As demonstrated in chapter 6, this perspective diverges from a perspective that sees private property as a way to encapsulate the financial entitlements of individuals. Instead, the perspective on property inherent in the land consolidation model is much closer to that postulated by the social function theorists discussed in chapter 2, especially those that focus on human flourishing as the underlying purpose of private property rights. Indeed, land consolidation relies on a highly functional perspective on property: beneficial resource uses, not individual entitlements, take center stage throughout the process. This might limit the power of the owners to do as they please, but it does not marginalise them. After all, it is hard to deny that one of the primary functions of private property is to bestow rights and obligations on its owners. Moreover, in normal circumstances, it would be safe to assume that when a property benefits, then so does its owner.

For this reason, it also seems that consolidation can be used to address the democratic deficit of economic development takings in an elegant way. Chapter 6 addressed this possibility in depth and argued that the land consolidation courts can be seen as providing an institution authorised to design and implement self-governance arrangements that approach private property in land as a common pool resource. This connected the case study of Norwegian hydropower with the discussion provided at the end of chapter \ref{chap:2}, regarding institutional alternatives to eminent domain for economic development. In particular, it connects the system of land consolidation courts with theories about self-governance and common pool resource management.

Because it combines great flexibility with a well-structured judicial procedure including many safeguards, the land consolidation option appears attractive as a vehicle for sustainable self-governance of local resources found in a local community. Moreover, it provides the public with a means to impose their will on owners and communities, without having to resort to the use of expropriation. While ensuring that legitimacy policy goals can be pursued more effectively, this also raises the worry that consolidation abuse will emerge to replace eminent domain abuse. Chapter 6 argued that in order to address this worry, the land consolidation alternative needs to function as a service to owners and local communities, as a means of helping them to manage their properties in accordance with public interests. Arguably, this requires a clear commitment on part of the state to prevent abuse of consolidation measures by commercial interests and public-private partnerships that seek access to property rights held by weaker parties.

Assuming that such a commitment is made, chapter 6 argued that use directives issued by a land consolidation court can be empowering to local owners, who are presented with a direct sense in which their property rights contributes to collective decision-making with a democracy guided by the rule of law. Specifically, chapter 6 argued that land consolidation can be used to deal with many of the challenges that arise at the intersection between private property, local community, and economic development in the public interest. The Norwegian model might also inspire similar solutions elsewhere, particularly in jurisdictions that are committed to an egalitarian ideal of property ownership.

\noo{ 

\section{On the legitimacy of interference in property, and the social structures it helps sustain}

\section{Substance and procedure -- protecting the social functions of property by reviewing decision-making processes leading to interference} 

\section{Norwegian waterfalls -- properties, assets, and common goods}

\section{Property taken by default}

\section{Property given following participation}

\section{Conclusion}

%\section{Property Lost -- Taking and Excluding}\label{sec:7:1}

%According to some, the law does not like it when things get too academical; after all, the law is not in the business of settling philosophical debates, but rather used to resolve disputes.
%Rather, its task is to deliver effective management of disputes, involving concrete legal persons or governments.
%In fact, many leading legal theorists adopt a closely related position, when they argue that the law 

According to many legal scholars, the law is an instrument of order, grounded in social facts, not an arbiter of justice, grounded in moral theory.\footnote{This is a terse formulation of a position typically associated with legal positivism; it can be elaborated and restated in a multitude of ways, giving rise to many theoretical variations. The merits of the positivist account of the relationship between morality and law is a contested issue in the so-called Hart-Dworkin debate in legal philosophy, see \cite{hart12,dworkin86,shapiro07}.} Building on this, a pragmatist might be tempted to think that the law should be sceptical of embracing the complexities of equity when goals such as efficiency,  certainty, and control appear to be better served by simplification.\footnote{For a theory that emphasises pragmatism and chides moral theory, see \cite[109-110]{posner99} (``Holmes warned long ago of the pitfalls of misunderstanding law by taking its moral vocabulary too seriously. A big part of legal education consists of showing students how to skirt those pitfalls.'' (citations omitted)).} This way of thinking appears to have played a significant role in shaping the traditional approach to the legitimacy issue in the law of takings.

% would be in the interest of efficiency.
%According to positivist thought, the law is rightly sceptical of allowing things to get too theoretical; after all, the law is not in the business of settling ethical issues. Indeed, many believe that its main responsibility is to deliver effective management of disputes, involving concrete legal persons or governments.

It is clear that by focusing on individual owners and their losses, the law makes things easier for those called to administer it. Moreover, by assuming that all losses can be quantified in financial terms, the law's standard can be at least partly automated. The potentially broad question of legitimacy has been reduced to the question of when and how to award compensation, for which the ostensibly neutral idea of ``fair market value'' appears to provide a practical starting point, well suited for maintaining order in a capitalist society.\footnote{See, e.g., \cite[510-511]{acres79} (``In giving content to the just compensation requirement of the Fifth Amendment, this Court has sought to put the owner of condemned property `in as good a position pecuniarily as if his property had not been taken.' However, this principle of indemnity has not been given its full and literal force. Because of serious practical difficulties in assessing the worth an individual places on particular property at a given time, we have recognized the need for a relatively objective working rule. The Court therefore has employed the concept of fair market value to determine the condemnee's loss. Under this standard, the owner is entitled to receive `what a willing buyer would pay in cash to a willing seller' at the time of the taking.'' (citations omitted).)} A large chunk of the remaining work, in turn, can be delegated to the appraisers, allowing the courts to get on with other business. In addition to being effective, this comes with the added bonus of allowing the courts to distance themselves very clearly from the political overtones of the legitimacy question.

This approach to legitimacy is prevalent, but I believe it needs to be rejected. The reason, as argued in the first part of this thesis, is that property itself cannot be drawn up as narrowly as the compensation approach presupposes.\footnote{See the discussion in chapter \ref{chap:2}, section \ref{sec:2:4}.} The legitimacy of property interference cannot be understood in mechanical terms, as a matter for the appraises, regardless of how much weight we want to place on the value of efficiency and the ideal of deference to political decision-making. Specifically, unless the issue of legitimacy is recognised as being far more complex than pertaining solely to the question of when and how to calculate market values, there will be a significant mismatch between what property is and what the law pretends it to be.\footnote{As contended in Part I of this thesis, this much appears to be a descriptive insight, which does not seem to depend on one's moral philosophy, see the discussion in chapter \ref{chap:2}, section \ref{sec:2:4:3}.}
\noo{In a setting where takings are rare and happen only in extraordinary situations, such a mismatch might be tolerable. Arguably, the strong commitment to the sanctity of property by early writers such as Blackstone, apparently conflicting with historical records, could be sustained because takings were exceptional, ordered only after careful deliberation by a legislative body with an authority over property to match or exceed even that of an owner.\footnote{See chapter \ref{chap:3}, section \ref{sec:3:2}.}

Such a narrative is no longer plausible in a world where the state has expanded its activities so much that interference in private property, rather than being exceptional, have become a normal occurrence. This also means that the law cannot pretend not to interfere also with the complexities of property as a social phenomenon and an anchor for participation and human flourishing. This broader narrative of property must then also be considered in the law, especially in the law of takings.}
This is harmful at a structural level, especially if the idea of property as a fundamental right is to have a future. If the law continues to insist that property is nothing more than a form of entitlement protection, there might even be a case to be made that the notion should just be done away with in its entirety.\footnote{See generally \cite{grey80}.}

In the first part of this thesis, I presented a theory of property which suggests that this would be a tragedy, particularly for marginalised groups who are in need of protection against economic and social elites. Importantly, while international law and human rights conventions offer important clarifications and protections at the level of principles, what property provides is an imperfect, yet very powerful, framework for implementing such principles at the local level. This is property's promise, which it can only keep if it is recognised as having social functions going well beyond the protection of individual entitlements.

Arguably, principles of human rights should even be recognised as inhering in property as such, not only as mediated by the power of states.\footnote{See chapter \ref{chap:2}, section \ref{sec:2:5:1}.} A worry often voiced as a counterargument for direct horizontal application of human rights is that it can serve as an excuse for the state to do nothing.\footnote{See \cite[110]{manisuli07}.} For this reason, it might well be worth emphasising that if the owners fail to deliver on basic rights, the state is still responsible. However, the converse is equally true: if states fail, then owners still have obligations.

If there is no distinction between owners and states, by contrast, failure in one is also failure in the other. This seems dangerous, particularly if proprietary power is exercised only by a small group of people, regardless of whether they are commercial leaders or powerful government officials. Property should therefore be widely distributed among the population, to be rendered as a provider of basic rights and an anchor of democracy.

The first chapter of this thesis explored this idea in depth and argued that a broader notion of property needs to be acknowledged also by the law, particularly in the law of takings. If property serves a broad social function by sustaining a community and aiding in the delivery of basic rights to all its members, transferring that property to a non-local commercial owner is not merely a highly dubious redistribution of entitlements. In the end, it will also be the destruction of property, as it undermines property's most significant functions in relation to the overarching goal of human flourishing. In such cases, therefore, property is not only taken, it is lost.

\noo{
t is worth pausing to recognise that the bundle of rights theory did us a favour in this regard, in that it directed our attention at the multifaceted nature of property. However, to make progress, it was necessary to further unpack the property bundle, to get at the substantive content of property in life: social functions as opposed to legal abstractions.

\noo{ Moreover, I argued that while private property might often be found wanting, it remains a potentially powerful force for good in the world. As discussed in chapter 1, its roles as a building block of democracy and a protector of communities is particularly important in this regard. In addition, I discussed the importance of social obligations arising from property, and how they can potentially function as a guarantee that the basis rights of non-owners will be delivered at the local level.

If these aspects are recognised and embraced as a crucial part of the concept of property in the law, it should hopefully go some way towards restoring confidence that property is neither theft nor fraud, but a promise to work hard for a better future for all. By contrast, the idea of property as a financial entitlement do not appear to offer any such relief. If anything, dismissive attitudes to property will take their fuel from the idea of property as entitlement; entitlements, after all, can often appear undeserved, especially when they are not checked by corresponding obligations. 
}
%In this way, a threat emerges to the stability of the concept of property itself, as a legitimate part of the social and political order.
%In reality, of course, owning property has nothing to do with what one deserves, but rather what task one has been allotted on this earth, to pursue in keeping with one's beliefs and convictions.

%In cases involving regulation of property use, it might still be possible to keep this aspect away from undermining the concept of property as such. At least, it might ensure that only those well versed in the technical details of the law are able to recognise property for the ``phantom'' that it appears to be.

%However, when property is taken outright, even this containment strategy is bound to falter. This is particularly clear if taking become increasingly prevalent not only in situations of pressing public need, but also as a means for companies to turn a profit. In such a setting, even the most naive observer would surely be tempted to think that property as a concept must be altogether rather vacuous.  


%The impression that private property, in the end, is nothing but a shorthand to describe a special class of liability rules, leaves property open to further attack. Indeed, if property and ownership has only such a thin content, why worry about interfering with it in the public interest? At this point, however, it seems prudent to take a step back, to reconsider the origin of the feeling that property is no more than theft, or no better than fraud. Specifically, it might be appropriate to note that unlike property as a concept, the act of taking it without its owner's consent is quite likely to involve both theft or fraud as a matter of fact, not merely a manner of speech.

%In the second part of this thesis, I have explored this in further depth, by analysing the law and practices relating to the expropriation of waterfalls in Norway.

%However, as I noted in the first chapter of this thesis, property itself is highly multifaceted, serving a range of social and individual functions. 

The first chapter set out to do this, in order to get at the multitude of different ways in which a taking can impact on society and its members. The economic consequences of a taking might be the most easily recognisable, particularly in the economic development cases. But as I have argued in this thesis, other consequences can be just as important, particularly those pertaining to property as a building block of communities. If jointly owned property is taken from a community, with full compensation paid to all individual owners, the community suffers a distinct uncompensated loss, namely the loss of future self-governance opportunities. %In light of work done by Elinor Ostrom and others on the strength of self-governance solutions for sustainable resource management, the law of takings should arguably offer protection directed specifically at the potential loss of community.

%In the traditional narrative on takings, social and political effects are typically only recognised on one side of the takings equation, namely the side of the taker, particularly the public interest. This has also influenced the debate on economic development takings. In order to make sense of the broader sense of unfairness often associated with such takings, critics tend to focus on the taker rather than the owner, by questioning the legitimacy of the motives behind the taking.

%However, this might be tantamount to shifting a variable to the wrong side of the takings equation. In particular, the feeling of unfairness associated with economic development takings clearly arise from a sense in which the owners are victims of an abuse of power. So why shift attention to the taker? 

%Perhaps it is tempting to do so simply because the sense of unfairness at work here pertains to a broader notion of justice than that normally associated with property interests. If so, the entire narrative points to a shortcoming of the liberal idea of property. If even property's staunchest defenders must turn to notions of ``public interest'' (and the lack thereof), then why do we need property as a concept at all? Why not simply say that a licence to undertake economic development should not be granted unless all affected parties agree, or the public interest is sufficiently strong to go ahead against some of their wishes? What makes property special in this picture, if all that is at stake is the strength of the public interest used to justify imposing the state's will on private individuals?

%Clearly, the gaping hole in the opposition to economic development takings in the US has been a {\it positive} account of 

%This part of the thesis focused on coming up with an answer as to why property is worthy of protection in the first place, in cases where economic rationality appears to dictate that it should be put to more profitable uses. If there is a reason to resist this, it must be because there is something valuable in property that the law should protect, irrespective of the current owner's financial entitlements.

Moreover, the thesis argued that the dynamics of power in takings cases need to receive more attention: the practice of economic development takings can result in local communities being deprived of highly valuable political capital in order for politically powerful commercial interests to make a profit. I noted that this was also the main concern raised by Justice O'Connor in her {\it Kelo} dissent.\footnote{\cite{kelo05}.} In this way, the social function view is arguably also implicit in one of the most forceful voices that have spoken out against economic development takings on the basis of constitutional property law.

However, the thesis also noted a weakness with the typical approach to legitimacy in the US, via the public use requirement. Specifically, this requirement does not appear to get us very far towards a justiciable restriction on the takings power along the lines of reasoning adopted by Justice O'Connor. Unlike the majority in {\it Kelo}, building on recent precedent, and Justice Thomas, building on the original meaning of the public use restriction, Justice O'Connor's more institutionally oriented reasons for rejecting the taking appeared to lack a firm basis in law.

To address this, I pointed to recent developments at the ECtHR, where an institutional perspective on fairness appears to be developing, which might be more likely to embrace the social function perspective on property as a basis that can support a justiciable restriction on the states' takings power. At the same time, the position of the Court in Strasbourg might be conducive to an approach that can adopt broad scrutiny in controversial cases without becoming too entangled with the politics of those cases at the state level. Specifically, the value of deference could be given a firmer expression as a norm that compels recognition of diversity and local democracy, not a norm that calls for passivity or loyalty to governments in politically sensitive situations.

Following up on this, the thesis went on to formulate a legitimacy test based on a set of conditions formulated by Kevin Gray. Three additional points were added to this list, emphasising the regulatory context, the position of non-owners, and the broader issue of democratic merit. 
}

Arguably, takings of this kind are currently being carried out in Norway, to the benefit of large energy companies. Waterfalls are still nominally considered private property, belonging to members of the rural communities in which the water resources are found. However, as shown in Part II of this thesis, the practice of taking waterfalls from local communities has become so insensitive to the plight of owners that the law itself is ambiguous about whether private ownership of water resources has much content at all.

%It is important to note that there is nothing inevitable about this state of the law. Indeed, the historical context shows that in Norway, where water is anything but scarce and most rivers are non-navigable, property rights in waterfalls (as opposed to water as a substance), was long recognised as on par with property rights in land. Expropriation to pursue hydropower development was not permitted under any circumstance until the early 20th century. 

But this does not mean that proprietary power is no longer exercised over the power of water. Quite the contrary. The commercial companies that acquire waterfalls are quick to turn them into commodities whose primary purpose is to turn a profit. At the same time, water resources have now been encapsulated in development licenses; today, licenses from the government, not waterfalls as such, are the key assets in the hydropower production sector. The nature of property has thus transformed; ownership of hydropower has become a legal fiction, arising from a bundle of papers with numbers on them, not from physical and social proximity to the underlying resource.

The result is that the notion of property backing up this regime is now so thin that it arguably cannot be distinguished at all from the political and economic power that backs it up. As such, it is also no wonder that narratives of egalitarian property give way to narratives based on thinking about water resources as though they belong to the ``public''.\footnote{See the \indexonly{ica17}\dni\cite[1]{ica17}.} This is a politically defensible way of talking, to mask the reality that water resources are managed according to the will of dominating market players and powerful interest groups, both striving to protect their dominion over the fruits of the land.\footnote{I mention that state-owned companies are currently extending their dominion elsewhere as well, with potentially deleterious consequences. In Nepal, for instance, Norwegian hydropower companies generate large profits from their aid-funded cooperation with the Nepali government, apparently at the expense of local populations and the rights of the poor. See \cite{gaarder15} (report from one of the largest environmental organisations in Norway, discussing how the Nepali government entered into an agreement where the price of electricity would be set at an extortionate level, denominated in US dollars, and adjusted upwards proportionally to US consumer prices; apparently, the project involves a transfer of money from Nepal to Norway that far exceeds the flow of money going the other way); \cite[644]{peris12} (mentioning the presence of Norwegian actors, discussing the marginalisation of local people, and arguing that conventions on indigenous rights are unable to deliver social justice due to the ``democratic deficit'' of decision-making regarding hydropower development in Nepal).}

Due to this dynamic, the case of Norwegian waterfalls appears to be an example of how illegitimate takings can do more harm than to deprive owners of valued resources. Specifically, the case study shows how a lack of legitimacy can effectively deprive property of its meaning, as the sticks of the 
bundle are bent to suit the interests of the commercial and political elites. Such a system might well give us the impression that property is little more than theft, maintained in the law only as a fraud.

\noo{ However, given the historical context Perhaps, then, the nature of property itself has changed, so that there is nothing left except those financial entitlements that Norwegian expropriation law recognised. If so, the change has not come about by any legislative move, nor has it been preceded by any kind of debate. It has simply emerged, gradually and unplanned, as a result of sector-based regulation and administrative practices. The process, therefore, meets neither the requirements of land reform or expropriation. It is an unacknowledged process about which the law in Norway has had nothing much to say at all, for which silence still persists. 

%Property can be an elusive concept, especially to property lawyers. Indeed, in the law of property, the word itself typically only functions as a metaphor -- an imprecise shorthand that refers to a complex and diverse web of doctrines, rules, and practices, each pertaining to different ``sticks'' in a ``bundle'' of rights. Indeed, this bundle perspective dominates legal scholarship, especially in the common law world. Some even go as far as to argue that words such as ``property'' and ``ownership'' should be removed from the legal vocabulary altogether. 

%So is property as a unifying concept lost to the law? It certainly seems hard to pin it down. In the words of Kevin Gray, when a close scrutiny of property law gets under way, property itself seems like it ``vanishes into thin air''.\footnote{See \cite[306-307]{gray91}.} %Indeed, one may argue that different ideas of property, practical and theoretical, are behind most, if not all, the major conflicts and confrontations that have shaped the society in which we live.
%Responding to this, some prominent philosophers have taken the view that property is not a concept suitable for philosophical study at all.
%According to some philosophers, property as a concept is a lost cause, not suitable for conceptual analysis at all. Instead, these scholars have suggested that property should be taken as a pragmatic and contingent derivative of other notions, such as the social order, or, on a normative account, by regarding it as an expedient of {\it justice}.
%Moreover, legal scholars are usually content with theories of property that remain largely descriptive, settling for the more modest aim of exploring how best to think of property given the prevailing legal order, rather than trying to come up with theories to explicate its nature as a pre-legal concept. Indeed, even legal philosophers are sometimes found doubting that there even is such a thing as property.
%Arguably, however, property never truly disappears. Indeed, there is empirical evidence to suggest that humans come equipped with a {\it primitive} concept of property, one which pre-exists any particular arrangements used to distribute it or mould it as a legal category.\footnote{See\cite{stake06}.} Perhaps most notably, humans, along with a seemingly select group of other animals, appear to have an innate ability to recognise {\it thievery}, the taking of property (not necessarily one's own) by someone who is not entitled to do so.\footnote{See \cite[11-13]{brosnan11}.}

%Taken in this light, Proudhon's famous dictum ``property is theft'', might be more than a seemingly contradictory comment on the origins of inequality. It might point to a deeply rooted aspect of property itself, namely its role as an anchor for the distinction between legitimate and illegitimate acts of taking.

%But what is a taking, and when is it legitimate? In this thesis, I will aim to make a contribution to this question. I will study takings of a special kind, namely those that are implemented, or at least formally sanctioned, by a government. In legal language, especially in the US, such acts of government takings are often referred to as takings {\it simpliciter}, while talk of other kinds of ``takings'' require further qualification, e.g., in case of ``takings'' based on contract, tax or occupation. 

%The US terminology brings the issue of legitimacy to the forefront in an illustrative manner. We are reminded, in particular, that under the rule of law, taking is not the same as theft. Rather, the default assumption is that the takings that take place under the rule of law are legitimate. If they are not, we may call them by a different name, but not before. At the same time, it falls to the legal order to spell out in further detail what restrictions may be placed on the power to take. 

%Indeed, restrictions appear implicit in the very notion of taking. The idea that someone might find occasion to resist an act of taking, and may or may not have good grounds for doing so, appears fundamental to our pre-legal intuitions. But how should we approach the question of legitimacy of takings from the point of view of legal reasoning, and what conceptual categories can we benefit from when doing so? This is the key question that is addressed in this thesis.
}

\section{Property Regained -- Giving and Participating}\label{sec:7:2}

The converse of a taking is a {\it giving}. In the US, this term is sometimes used to refer to situations when private property owners benefit from state actions involving property.\footnote{See generally \cite{bell01}.} For instance, it might be characterised as a giving if the state allows someone to purchase property cheaply, or if regulation makes some properties appreciate in value. Arguably, and analogously to the case of takings, there is a case to be made that the state should sometimes charge people for disproportionate givings.\footnote{See \cite[590-604]{bell01}.}

The issue of when this is appropriate, if at all, will not be discussed here. Instead, I wish to direct attention at the terminology itself, and its subtle conceptual commitment to a top-down way of thinking about both takings and givings. Indeed, consider what happens if we turn the terminology on its head. This is not an implausible conceptual shift. After all, if the state takes property, the current owners will have to give it up.

The owners' act of giving, however, is rarely given much recognition or attention when we approach takings. Plainly, the owners' active participation as a giver -- not merely an injured party -- is typically considered irrelevant. Why is that? The obvious answer is that since the giving takes place under compulsion, it does not express any intention to give. However, there are many situations in life where actions are compelled, but where the person taking that action still gets some credit for it.\footnote{Paying taxes, for instance, is typically associated with being a good citizen.} Moreover, the owners can clearly decide to be more or less cooperative when faced with the government's wishes for their property. 

By shifting attention towards the choices that the owners have in this regard, we can hope to find a path towards increased legitimacy by recognising the owners as active participants. Indeed, even a purely symbolic recognition of the owners' role and the importance of their choices in dealing with a takings request, can serve to enhance subjective legitimacy. However, quite apart from recognising the constructive role that owners can play in the existing system, thinking about takings as givings also suggests the possibility that owners should be granted more choices and asked to take part more actively in the proceedings.

%The owners make an active contribution to the public purpose, instead of being regarded as obstacles to it. This, in turn, can become a starting point for coming up with arrangements where the owners are permitted to take up a more lasting interest also in the new use of the property that the public desires. 

In cases involving economic development, this way of thinking seems particularly appropriate. As they contribute to the project by giving their property, it seems only fair that the owners should have a stake also in the planning and the continued use of their property for economic gain. For instance, it might be appropriate to offer owners shares in the development company, and to make sure that the property is taken under a leasehold rather than by a full transfer of title. Better yet, one could allow the owners themselves to deliberate on how they wish to honour their commitment to the public, to formulate their own plans and implement their own solutions, based on continued interaction both among themselves and with representatives from the relevant bodies of government.

Interestingly, as shown in chapter \ref{chap:5} of this thesis, such an abstract and highly idealistic idea is in fact (partly) implemented through the system of land consolidation found in Norway. As noted, this institution is even equipped with the power to compel owners to come together and participate in specific endeavours. If a proposed development is judged as being beneficial to the properties in question, the owners may be stripped of their holdout power, not only as individuals but even as a group, without any property having to be condemned. The owners will retain their ownership even after their properties have been put to new and more productive uses.

In light of this, land consolidation can indeed replace expropriation, provided our understanding of what property is, and should be, is broad enough to say that the purpose we wish to pursue is also in the interest of the properties involved. This limitation, expressly encoded in the law of consolidation in Norway, is interesting also on the theoretical level, because it makes the {\it purpose} of property directly relevant to determining the extent of the government's power to interfere with it. At the same time,  in the context of economic development, the restriction is rarely going to prevent consolidation from going ahead. If development is both economically beneficial and represents a sustainable use of resources, the land consolidation courts should have no difficulty justifying consolidation measures on the basis that development is also in the interest of the properties involved. At the same time, the context of land consolidation, with its emphasis on problem solving and owner participation, means that the process will usually be far more inclusive towards owners than traditional expropriation proceedings. 

If property is in the hands of the few, while many property dependants are without formal ownership rights, the consolidation model might be less appropriate. However, in these cases, it might be possible to adapt it by allowing a larger group of local people to partake in the proceedings. In complex cases, most of them might find themselves unable to participate effectively in the process. However, a consolidation approach will still serve an important function in that it gives marginalised groups new opportunities for engaging with the democratic process.

%Under consolidation, it is truly more appropriate to speak of givings than to speak of takings. Moreover, what is given in these cases is not normally the property as such, but the right to determine how it is to be used, with the public having to turn to the current owners for help in realising the public purpose. It would still be possible to expropriate, or to rely on a mix between expropriation and consolidation. In Norway, the relationship between the two remains to be clarified in the law. A general rule of consolidation first, expropriation second, 
%should arguably be introduced in order to better promote the consolidation alternative.

%This thesis went on to explore how this alternative works in practice for hydropower development, noting that the current system works best when the owners are in reasonable agreement with each other and society about how development should proceed. Hence, the broader applicability of the idea can not be taken for granted. However, it seems to have great potential for being fine-tuned, to make it more effective in situations involving deeper disagreements and conflicting interests. There is a danger here, however, namely that increasing the power of the institution will also undermine its role as a democracy-on-demand for owners and the community. However, if safeguards can be put in place, the land consolidation model might well be a highly attractive alternative to expropriation, also in cases involving deeper conflicts about property uses.

The system can also serve an empowering and educational role. Local elites might often dominate the process in practice, but the fact that the procedure is judicial in nature means that abuses can be curbed while the disadvantaged may be granted access to more fruitful forms of citizenship. At least, access to the decision-making process should become easier for those who wish to challenge the local leadership.

More generally, the flexible, issue-focused, and transient nature of a consolidation court might have significant advantages. Specifically, concentrations of power and participatory fatigue are unlikely to become entrenched, given the context of decision-making; the group of people involved, the area covered, and the agenda to be deliberated on, will all change from situation to situation. This keeps participants engaged while diffusing power, thereby minimising the risk of elite tyranny and expert rule, otherwise easily enabled by rigid institutions and apathetic majorities.\footnote{This also relates closely to the worry that ``rational ignorance'' can thwart attempts at meaningful eminent domain reform, see chapter \ref{chap:4}, section \ref{sec:3:3:2}. See also \cite{somin09}.}

Further exploration of the transferability of the consolidation framework to other contexts will have to be left for future work. What this thesis has hopefully shown is that fully fledged institutional alternatives to expropriation for economic development already exist and that they can work in practice. Hence, my research can hopefully inspire more work in this direction both at the theoretical and empirical level. In my opinion, the ideas underlying the land consolidation system found in Norway are worth exploring further, also in other jurisdictions that rely on expropriation as a tool to facilitate economic development. In this context, it would be a great victory for both property and equity if public and private interests could come together in a way that will give rise to givings in the future, clearly distinguished from the takings of the past.
}
%\include{Chapter6/Chapter6}		
%\include{Chapter7/Chapter7}
%\include{Chapter8/Chapter8}

%\bibliographystyle{Classes/CUEDbiblio}
%\bibliographystyle{oxford_en}
%\bibliographystyle{Classes/jmb} % bibliography style
%\renewcommand{\bibname}{References} % changes default name Bibliography to References
%\addcontentsline{toc}{chapter}{Bibliography} %adds References to contents page
%\bibliographystyle{Classes/jmb} % bibliography style

%\printindex[echrcases]
\nocite{*}

%If you want to input ship names, put them HERE (after \nocite, before %bibliography.tex

\chapter*{Bibliography}
\addcontentsline{toc}{chapter}{Bibliography}

% This filter is used to identify works which are either of the inbook or incollection type
\defbibfilter{inbookorincoll}{%
  \( \type{inbook} \or \type{incollection} \)}

% Define a bibheading that prints a subheading, with appropriate addition to table of contents, and sets right and left marks accordingly
\defbibheading{mysubbibintoc}{%
  \section*{#1}%
  \addcontentsline{toc}{section}{#1}%
  \markboth{BIBLIOGRAPHY -- \MakeUppercase{#1}}{BIBLIOGRAPHY -- \MakeUppercase{#1}}}

% BOOKS

\printbibliography[title={Books}, type=book, heading=mysubbibintoc]

% WORKS IN COLLECTIONS

\printbibliography[title={Contributions to Collections}, filter=inbookorincoll, heading=mysubbibintoc]

% ARTICLES IN JOURNALS

\printbibliography[title={Articles}, type=article, heading=mysubbibintoc]

% ALL OTHER WORKS INCLUDING UNPUBLISHED MATERIAL

\printbibliography[title={Other Works}, nottype=book, nottype=jurisdiction, nottype=legal, nottype=legislation, nottype=article, nottype=inbook, nottype=incollection, heading=mysubbibintoc]
. I have left one example, commented out, which should work (assuming you have the case, etc.).

%\index[casesgb]{Achilleas, The@\emph{Achilleas,} The|see{Transfield Shipping Inc v Mercator Shipping Inc}}


%bibliography.tex

\chapter*{Bibliography}
\addcontentsline{toc}{chapter}{Bibliography}

% This filter is used to identify works which are either of the inbook or incollection type
\defbibfilter{inbookorincoll}{%
  \( \type{inbook} \or \type{incollection} \)}

% Define a bibheading that prints a subheading, with appropriate addition to table of contents, and sets right and left marks accordingly
\defbibheading{mysubbibintoc}{%
  \section*{#1}%
  \addcontentsline{toc}{section}{#1}%
  \markboth{BIBLIOGRAPHY -- \MakeUppercase{#1}}{BIBLIOGRAPHY -- \MakeUppercase{#1}}}

% BOOKS

\printbibliography[title={Books}, type=book, heading=mysubbibintoc]

% WORKS IN COLLECTIONS

\printbibliography[title={Contributions to Collections}, filter=inbookorincoll, heading=mysubbibintoc]

% ARTICLES IN JOURNALS

\printbibliography[title={Articles}, type=article, heading=mysubbibintoc]

% ALL OTHER WORKS INCLUDING UNPUBLISHED MATERIAL

\printbibliography[title={Other Works}, nottype=book, nottype=jurisdiction, nottype=legal, nottype=legislation, nottype=article, nottype=inbook, nottype=incollection, heading=mysubbibintoc]


% this section includes various indexes/tables of cases and legislation

\chapter*{Cases Cited}
\addcontentsline{toc}{chapter}{Cases Cited}
\markboth{CASES CITED}{CASES CITED}

\printindexearly[casesgb]% ENGLAND & WALES
\printindexearly[casessc]% SCOTLAND (GB too, of course, but ...)
\printindexearly[casesau]% AUSTRALIA
\printindexearly[casesnz]% NEW ZEALAND
\printindexearly[casesca]% CANADA
\printindexearly[casesus]% UNITED STATES
\printindexearly[casesother]% OTHERS

\chapter*{Legislation Cited}
\addcontentsline{toc}{chapter}{Legislation Cited}
\markboth{LEGISLATION CITED}{LEGISLATION CITED}

\printindexearly[legis]% ALL LEGISLATION

\end{document}
