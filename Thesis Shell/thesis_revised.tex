%input macros (i.e. write your own macros file called MacroFile1.tex)
\newcommand{\PdfPsText}[2]{
  \ifpdf
     #1
  \else
     #2
  \fi
}

\newcommand{\IncludeGraphicsH}[3]{
  \PdfPsText{\includegraphics[height=#2]{#1}}{\includegraphics[bb = #3, height=#2]{#1}}
}

\newcommand{\IncludeGraphicsW}[3]{
  \PdfPsText{\includegraphics[width=#2]{#1}}{\includegraphics[bb = #3, width=#2]{#1}}
}

\newcommand{\InsertFig}[3]{
  \begin{figure}[!htbp]
    \begin{center}
      \leavevmode
      #1
      \caption{#2}
      \label{#3}
    \end{center}
  \end{figure}
}


%%% Local Variables: 
%%% mode: latex
%%% TeX-master: "~/Documents/LaTeX/CUEDThesisPSnPDF/thesis"
%%% End: 

%\includeonly{Chapter5/endelig_5}
%NOTE: if you want to work on just one Chapter, you can take out the `%' sign on the previous line and compile the thesis accordingly. The above command, for instance, will give you just the first Chapter. The bonus of doing it this way is that your cross references and page numbers will remain as they are in the full file.

\documentclass[a4paper,twoside,openright,10pt]{thesisPSnPDF}

\usepackage[utf8]{inputenc}
\usepackage{babel}
%\usepackage{nomencl}
%\DeclareUnicodeCharacter{00A0}{ }
%\makenomenclature

%\newcommand{\isr}[1]{#1}
\newcommand{\dni}{\DNI}
\newcommand{\noo}[1]{}
\newcommand{\sjur}[1]{SJUR: #1}
\newcommand{\nathp}[1]{NatHp(#1)}

\def\signed #1{{\leavevmode\unskip\nobreak\hfil\penalty50\hskip2em
  \hbox{}\nobreak\hfil(#1)%
  \parfillskip=0pt \finalhyphendemerits=0 \endgraf}}

\newsavebox\mybox
\newenvironment{aquote}[1]
  {\savebox\mybox{#1}\begin{quote}}
  {\signed{\usebox\mybox}\end{quote}}

\usepackage{titlesec}
\titleformat{\chapter}[hang]
  {\normalfont\huge\bfseries\centering}{\thechapter}{20pt}{\Huge}

\addbibresource{thesis.bib}

% turn of those nasty overfull and underfull hboxes
\hbadness=10000
\hfuzz=50pt

% Put all the style files you want in the directory StyleFiles and usepackage like this:
%\usepackage{StyleFiles/watermark}

%The following indexes are to ensure the table of cases functions properly. You can leave this to one side for now, though it is worth learning early on how to make the table of cases. It is pretty easy; but it'd be a shame if it got to near submission and you couldn't figure out how to do it. 
% NB: I haven't provided for Northern Irish cases here
\makeindex[name=casesgb, title={England and Wales}, columns=1,intoc]
\makeindex[name=casessc, title={Scotland}, columns=1,intoc]
\makeindex[name=casesus, title={The United States}, columns=1,intoc]
\makeindex[name=casesnz, title={New Zealand}, columns=1,intoc]
\makeindex[name=casesau, title={Australia}, columns=1,intoc]
\makeindex[name=casesca, title={Canada}, columns=1,intoc]
\makeindex[name=legis, title={United Kingdom}, columns=1,intoc]
\makeindex[name=casesother, title={Norway}, columns=1,intoc]
\makeindex[name=casesechr, title={European Court of Human Rights}, columns=1,intoc]
\makeindex[name=casesjunk, title = {Other Jurisdictions}, columns = 1, intoc]
\makeindex[name=noleg, title = {Norway}, columns = 1, intoc]
\makeindex[name=prepno, title = {Norway}, columns = 1, intoc]
\makeindex[name=intleg, title = {Other Law Instruments}, columns = 1, intoc]
\DeclareIndexAssociation{gbcases}{casesgb}% ENGLAND
\DeclareIndexAssociation{sccases}{casessc}% SCOTLAND
\DeclareIndexAssociation{aucases}{casesau}% AUSTRALIA
\DeclareIndexAssociation{cacases}{casesca}% CANADA
\DeclareIndexAssociation{nzcases}{caseszn}% NEW ZEALAND
\DeclareIndexAssociation{uscases}{casesus}% UNITED STATES
\DeclareIndexAssociation{eucases}{casesother}% EU
\DeclareIndexAssociation{echrcases}{casesechr}% ECHR
\DeclareIndexAssociation{pilcases}{casesother}%
\DeclareIndexAssociation{othercases}{casesother}% ANYTHING ELSE
%\DeclareIndexAssociation{gbprimleg}{legis}% LEGISLATION
%\DeclareIndexAssociation{gbsecleg}{legis}% LEGISLATION
\DeclareIndexAssociation{enprimleg}{legis}% LEGISLATION

\indexsetup{level=\section*,toclevel=section,noclearpage}

\DeclareBibliographyCategory{cited}
\AtEveryCitekey{\addtocategory{cited}{\thefield{entrykey}}}

\usepackage{calc}
\usepackage{lipsum}
\makeatletter
\newcommand{\tocfill}{\cleaders\hbox{$\m@th \mkern\@dotsep mu . \mkern\@dotsep mu$}\hfill}
\makeatother
\newcommand{\abbrlabel}[1]{\makebox[3cm][l]{\textbf{#1}\ \tocfill}}
\newenvironment{abbreviations}{\begin{list}{}{\renewcommand{\makelabel}{\abbrlabel}%
        \setlength{\labelwidth}{3cm}\setlength{\leftmargin}{\labelwidth+\labelsep}%
                                              \setlength{\itemsep}{0pt}}}{\end{list}}


\begin{document}
\renewcommand\baselinestretch{1.5}
\baselineskip=24pt

%\maketitle

\begin{titlepage}

\begin{center}



\vspace*{\fill}
\centering

{\Huge\textsc{On the legitimacy of economic development takings}}\\[3cm]

\large {Thesis submitted to the School of Law at Durham University for the degree of Doctor of Philosophy}\\

by

{Sjur K. Dyrkolbotn}\\

%\emph{{Your College}}\\
\vspace*{\fill}

 

\vfill

{\Large Autumn 2014}\\
{c. 90 000 Words}

\end{center}

\end{titlepage}


%set the number of sectioning levels that get number and appear in the contents
\setcounter{secnumdepth}{4}
\setcounter{tocdepth}{2}

%\frontmatter

%\addcontentsline{toc}{chapter}{List of Abbreviations}

\tableofcontents

%
\begin{center}
\vspace{4cm}
I hereby certify that this thesis is the result of my own work except where otherwise indicated and due acknowledgement is given.
\vspace{1cm}

I also certify that this thesis is XXXXX words long excluding the bibliography.\\

\vspace{4cm}


\begin{tabular}{lr}
& DATE OF SUBMISSION \\
& \\
SIGNED & DATE \\
\end{tabular}


\end{center}


% ----------------------------------------------------------------------


%%% Local Variables: 
%%% mode: latex
%%% TeX-master: "../thesis"
%%% End: 

%\addcontentsline{toc}{chapter}{Abstract}
% Thesis Abstract -----------------------------------------------------

% NOTE: As with acknowledgements, I had to create a new format for this -- I couldn't get the original one to work. As with the acknowledgements, if you are able to fix the code so it's less messy, do pass the fix back to the Law Faculty.
\cleardoublepage
\addcontentsline{toc}{chapter}{Abstract}
%\begin{abstractslong}    %uncommenting this line, gives a different abstract heading
%\begin{abstracts}        %this creates the heading for the abstract page

\begin{quoting}
  \singlespace
    \begin{center}
  {\LARGE \bfseries  On the Legitimacy of Economic Development Takings }\\
  \vspace*{0.5cm}
      {\large Sjur Kristoffer Dyrkolbotn}\\
  %\vspace*{0.1cm}  
   %   {\large \emph{Ustinov College}}\\
  \vspace*{0.2cm}  
    {\normalsize Thesis submitted to Durham Law School at Durham University for the degree of Doctor of Philosophy}

  \vspace*{0.2cm}  
    {\normalsize \today}\\
  \vspace*{0.5cm}  
    {\normalsize \bfseries Abstract}      
  \end{center}
  {\parindent0pt
For most governments, facilitating economic growth is a top priority. Sometimes, in their pursuit of this objective, governments interfere with private property. Often, they do so by indirect means, for instance through their power to regulate permitted land uses or by adjusting the tax code. However, many governments are also prepared to use their power of eminent domain in the pursuit of economic development. That is, they sometimes compel private owners to give up their property to make way for a new owner that is expected to put the property to a more economically profitable use. %This new owner is sometimes the government itself, represented by one of its administrative bodies. But in many cases it will be a private company, operating for profit, possibly in cooperation with government entities through some form of public-private partnership.
}
\vspace{0.7mm}

This thesis asks how the law should respond to government actions of this kind, often referred to as {\it economic development takings}. The thesis makes two main contributions in this regard. First, in Part I, it proposes a theoretical foundation for reasoning about the legitimacy of economic development takings, including an assessment of possible standards for judicial review. Moreover, the thesis links the legitimacy question to the work done by Elinor Ostrom and others on sustainable management of common pool resources. Specifically, it is argued that using institutions for local self-governance to manage development potentials as common pool resources can potentially undercut arguments in favour of using eminent domain for economic development.

Then, in Part II, the thesis puts the theory to the test by considering takings of property for hydropower development in Norway. It is argued that current eminent domain practices appear illegitimate, according to the normative theory developed in Part I. At the same time, the Norwegian system of land consolidation offers an alternative to eminent domain that is already being used extensively to facilitate community-led hydropower projects. The thesis investigates this as an example of how to design self-governance arrangements to increase the democratic legitimacy of decision-making regarding property and economic development.

%This shows how local governance arrangements can work, suggesting that more attention should be devoted to studying the nexus between property, common pool resource management, and eminent domain.

%This theoretical basis is formulated independently of specific jurisdictions, but based on considering existing approaches to the legitimacy question from England and Wales, the United States, and at the European Court of Human Rights. In addition, the thesis draws a link between the legitimacy question and the work done by Elinor Ostrom and others on sustainable management of common pool resources. Specifically, it is argued that institutions for local self-governance that treat development potentials as common pool resources can often undercut arguments in favour of using eminent domain for economic development.

%Such rules are in place in most developed countries, and the fundamental status of property has been expressed explicitly in both the US constitution and the European Convention of Human Rights. The tension between these provisions and the practice of taking property for economic development, in many cases for commercial profit, is clear and worth considering further.

%\vspace{0.7mm}

%The second part of the thesis puts the theory developed in the first part to the test by considering takings for hydropower development in Norway. Under Norwegian law, the right to exploit the hydropower in most streams and rivers belong to the riparian owners. That is, the right to the hydropower belongs to the people who own the land over which the water flows, usually local community members. To acquire these rights, energy companies tend to rely on the government's power of eminent domain. Recently, however, local communities have begun to protest this practice, by arguing that they should be allowed to take a more active role in managing their own resources. This has resulted in tensions in Norway, shedding light on the legitimacy question as it arises in the context of Norwegian expropriation law. In addition, new light has been shed on the role of the so-called land consolidation courts, which are now increasingly asked to deliver alternatives to eminent domain in hydropower cases. The thesis investigates this in depth and argues that the unique system of land consolidation found in Norway demonstrates how to design self-governance arrangements that can increase the democratic legitimacy of decision-making regarding property and economic development.

\end{quoting}


%\end{abstracts}
%\end{abstractslong}


% ----------------------------------------------------------------------


%%% Local Variables: 
%%% mode: latex
%%% TeX-master: "../thesis"
%%% End: 


% Thesis Acknowledgements ------------------------------------------------

\cleardoublepage
%\begin{acknowledgementslong} %uncommenting this line, gives a different acknowledgements heading
%\begin{acknowledgements}      %this creates the heading for the acknowlegments
\addcontentsline{toc}{chapter}{Acknowledgements}
\begin{quoting}
  \singlespace
    \begin{center}
  {\LARGE \bfseries  Acknowledgements}\\
  \vspace*{0.5cm}
  \end{center}
\noindent

My supervisor, Professor Tom Allen, has been a great support and inspiration ever since we first corresponded about the possibility of me doing a PhD in Durham, in the spring of 2012. His style as a supervisor has been superb: calm and unperturbed, yet always sharp and focused, readily available to offer insightful comments and valuable guidance. Thank you, Tom.

Second, I would like to thank Durham Law School for offering me a place at their department and for treating me well while I have been there. Thanks also to Professor Leigh and Professor Masterman for taking an interest and giving me valuable comments following my first year review.

Third, I would like to thank Professor Jacques Sluysmans, Professor Hanri Mostert, and Professor Leon Verstappen, for welcoming me to their regular colloquia on expropriation law. Attending and speaking at these meetings has been a very valuable experience for me, allowing me to learn from expropriation lawyers and scholars from many different jurisdictions. A special thanks to Professor Sluysmans, Dr Emma Waring and Dr Stijn Verbijst for inviting me to contribute a chapter on Norway in their book on expropriation in Europe. Another special thanks to Dr Waring for sending me a copy of her doctoral thesis on private takings; it has been a great help and inspiration for my own work. Also a special thanks to Dr Ernst Marais and Bj\"{o}rn Hoops for organising an excellent conference in Rome and being very helpful and welcoming to new members of the expropriation research community. Hopefully, this community will stay together and continue to prosper.

Fourth, I would like to thank Ustinov College for welcoming me as a student in Durham and providing a relaxed and friendly atmosphere during my first year as a PhD student. Thanks also to the friends I met there, including Julia, Alan, Noel, Meghan, Lloyd, Peter, and Alma. A special thanks to Isabel Richardson, for being both a highly valued friend and an excellent proofreader.

Fifth, I would like to thank family, friends and colleagues in Norway, especially Ragnhild, Truls and Piotr (who is just on holiday in Germany, I am sure). A special thanks to my father and my brother, for motivation, guidance, and support. A special thanks also to my mother and my sisters, for kindness and inspiration. Thanks to Einar Sofienlund for sharing his insight and providing invaluable information about small-scale hydropower. Thanks also to Johan Fr Remmen, for making clear why this thesis should be written. Hopefully, I have made a good start towards doing justice to the subject.

Lastly, I would like to thank Marijn Visscher. No doubt, coming to Durham was the best decision I ever made.

\end{quoting}


%\end{acknowledgements}
%\end{acknowledgmentslong}

% ------------------------------------------------------------------------

%%% Local Variables: 
%%% mode: latex
%%% TeX-master: "../thesis"
%%% End: 


% A note on the format. I could not get the acknowledgements macro to work properly, despite some effort, and so I redesigned it rather messily, as you will see above. If you enter text, it will work -- but the more technologically competent among you will probably be able to fix it. If and when that is done, if you could send the resultant document back to the Law Faculty to update it, that would be great. 


\section*{List of Abbreviations}
\addcontentsline{toc}{chapter}{List of Abbreviations}

\begin{abbreviations}
\item[CPO]{Compulsory Purchase Order}
\item[CPR]{Common Pool Resource}
\item[ECHR]{European Convention of Human Rights}
\item[ECtHR]{European Court of Human Rights}
\item[EEA]{European Economic Area}
\item[EFTA Court]{Court of Justice of the European Free Trade Association States}
\item[ICCPR]{International Covenant on Civil and Political Rights}
\item[ICESCR]{International Covenant on Economic, Social and Cultural Rights}
\item[LAD]{Land Assembly District}
\item[NVE]{Norges Vassdrags- og Energidirektorat (Norwegian Water and Energy \linebreak Directorate)}
\item[P1(1)]{Article 1 of the First Protocol to the European Convention of Human Rights}
\item[UDHR]{Universal Declaration of Human Rights}
\end{abbreviations}

%% this section includes various indexes/tables of cases and legislation

\chapter*{Cases Cited}
\addcontentsline{toc}{chapter}{Cases Cited}
\markboth{CASES CITED}{CASES CITED}

\printindexearly[casesgb]% ENGLAND & WALES
\printindexearly[casessc]% SCOTLAND (GB too, of course, but ...)
\printindexearly[casesau]% AUSTRALIA
\printindexearly[casesnz]% NEW ZEALAND
\printindexearly[casesca]% CANADA
\printindexearly[casesus]% UNITED STATES
\printindexearly[casesother]% OTHERS

\chapter*{Legislation Cited}
\addcontentsline{toc}{chapter}{Legislation Cited}
\markboth{LEGISLATION CITED}{LEGISLATION CITED}

\printindexearly[legis]% ALL LEGISLATION

% NOTE:
% To generate the indexes properly you need to run the following commands (in a terminal shell -- you need to navigate to the Thesis file in terminal. There will be guidance online for how to navigate within Terminal. Otherwise, most scientists should be able to help you!):
% splitindex -- thesis -s oscola (THIS IS THE COMMAND WHICH WORKS FOR ME)
% splitindex thesis --s oscola thesis (TRY THIS IF THE FIRST ONE DOESN'T WORK)

%\addcontentsline{toc}{chapter}{Abstract}
%\listoffigures

%\include{LawTable/LawTable}
%\listoftables
%\addcontentsline{toc}{chapter}{List of Tables}
%\printglossary  %% Print the nomenclature
%\addcontentsline{toc}{chapter}{Nomenclature}

%\renewcommand{\nomname}{Abbreviations}
%\printglossary

%\mainmatter
%\part{Towards a Theory of Takings for Commercial Gain}

\noo{The gulf between the conscious and the unconscious must be embraced by explanation models. It is the distinguishing human trait, after all, that we grant ourselves ... The unconscious is only looked {\it at} by the conscious. No particular theoretical unit of the sub-conscious change predictably when you think about yourself (or, as some would say, ``form an intention''). What changes is {\it you}. The predictability of the outcome (or otherwise, in people who are unwell, for instance) is largely attributable to the workings of the sub-conscious, i.e., must await explanation in behavioural (physical/chemical) terms. The difference between a person and a dog is that by looking at themselves and the world in a (mental) language, persons are capable -- through a purely behavioural/physical mechanism -- of modifying their instincts. Thought prevails over instinct (sometimes), but the power of thought itself is purely instinctual... The power of language is in the world, not in language. Thank God!}

\chapter{Property, protection and privilege}\label{chap:1}

\begin{quote}
It's nice to own land.\footnote{Donald Trump}
\end{quote}

\section{Introduction}

In this chapter, I provide a bird's eye view on my topic, by placing it in the theoretical landscape. My aim is to explain the key concepts that I will rely on to make sense of the empirical data considered in subsequent chapters. I will also present the main values that I will look to when I give normative assessments. In addition, I will relate my theoretical approach to current strands in property theory, focusing on those aspects of property theorizing that I regard as particularly relevant to the work done in this thesis.

I will strive to show that my approach to the empirical data is sound and informative, while focusing on principles of {\it legal} reasoning. I will not provide an extensive presentation of concepts or theoretical approaches developed in other fields, such as political science, sociology, economy, or psychology. However, I note that all these fields engage in interesting ways with the notion of property, and I think a multi-disciplinary approach can be illuminating.\footnote{For some examples of relevant work from economics, psychology and political science respectively, consider \cite{miceli11,nadler08,katz97,carruthers04}.} Hence, while I focus on legal and --  to some extent -- philosophical theories of property, I will try to make a note of specific questions I consider that are also analysed in related fields.

The crucial argument made in this chapter is that the category of {\it economic development takings} is relevant to legal reasoning about certain kinds of situations when private property is taken by the state. This is not {\it prima facie} clear. In fact, I am prepared to face critics who will argue that the category makes no legal sense at all. Fortunately, it makes perfect intuitive sense; it targets situations when property is, quite literally, taken for economic development. In most cases I will consider, this is even the explicitly stated aim used to justify eminent domain. Hence, the factual basis for our categorization can not be questioned.

The theoretical basis, on the other hand, can not be taken for granted. Indeed, a superficial look at dominant legal approaches to property would seem to indicate that in most property regimes, the nature of the project benefiting from a taking should not be in focus when assessing the legitimacy of interference. Rules aiming to protect property invariably focus on the rights of the affected owner, making clear that she enjoys some degree of protection against uncompensated state interference. But how can we, on this basis, justify having regard to the {\it purpose} of the taking? What bearing does this have for the question of legitimacy with respect to the owner's rights? At first sight, it might seem unwarranted to think that it should matter at all. Are not owners' rights  equally interfered with when property is taken for some uncontroversially public project, like a new public road, compared to the situation when it is taken for economic development? Is it not in fact a little small-minded, even short-sighted, to worry about the taker's gain, instead of concentrating on what the owner, if anything, stands to lose?

Much of the work in this chapter, albeit theoretical, is aimed at countering this very concrete objection. I believe it is important to do so thoroughly, since it is an objection that threatens to undermine the conceptual basis for the kind of study that I present in this thesis. Moreover, it is an objection that I think it is inappropriate to dismiss without further comment. In the context of US law, it might be possible to do so, since economic development takings have, as a matter of fact, gained recognition as an important category of legal reasoning.\footnote{See generally \cite{cohen06,somin07,malloy08}.}  In Europe, however, this has not yet happened, at least not to the same extent.

The reason for this difference is not that US law contains special rules that emphasize the importance of the distinguishing features of economic development takings.\footnote{In fact, many state laws now {\it do} contain such rules, following the backlash of the controversial decision in \cite{kelo05}. However, such rules were introduced only after the category of economic development takings first came to prominence in legal discourse. See generally \cite{eagle08,somin09,jacobs11}.} Rather, the difference must largely be attributed to the fact that economic development takings have resulted in great political controversy in the US, a controversy that has influenced both the law and legal scholars.\footnote{See, e.g., \cite[1190-1192]{somin08}.} Hence, in the absence of a similar political climate in Europe, a conceptual investigation into the very idea of an economic development taking is warranted, if not also required.

As I argue in this chapter, providing an adequate account of such takings forces us to broaden our theoretical outlook compared to most traditional strands of legal reasoning about property. However, I find considerable support for the necessary conceptual reconfiguration when I consider recent trends in property theory, particularly those that focus on the {\it social function} of property.\footnote{See generally \cite{alexander09a,foster11,singer00,underkuffler03,alexander06,alexander10,dagan11}.} Indeed, the crux of the main argument presented in this chapter is that this function allows us, even compels us, to pay attention to the special dynamics of power that tend to manifest in cases when private property is taken by the state for someone else's profit.

Some might argue that the most straightforward way of describing economic development takings is to say simply that they are {\it unfair}. Indeed, this has some merit also as a conceptual position, since it would seem to explain quite effectively why takings for profit have become so unpopular in the US, particularly when people's homes are taken.\footnote{See, e.g., \cite[742-748]{nadler08}.}  However, a more thoughtful assessment reveals that matters are not quite so simple. Indeed, it seems that economic development takings are an almost unavoidable consequence of any system that emphasize both state control over property and public-private partnerships in the economic sector. To condemn this political model of government is a possible response, but not one I will pursue in this thesis. Rather, I will focus on getting to the heart of what characterizes economic development takings, so that I may address also the question of how to deal with them, to ensure that their positive functions can be fulfilled in ways less offending.\footnote{Even those who support an outright ban on economic development takings should be interested in demarcating the category more closely, to arrive at a better understanding of what exactly will be banned.}

Therefore, the stark contrast between the intuitive response that a taking for profit is unfair, and the legal assessment that what matters is only the loss to the owner, needs to be considered further. A tentative first reconciliation can be achieved by arguing that the feeling of unfairness is in itself a loss that the owner incurs, so that the law had better take it into account. But, of course, not any subjective feeling should be given weight in a legal context. The question, therefore, is what exactly the feeling of unfairness can help us uncover about the nature of economic development takings. Does it uncover something legally significant?

In my view, the answer is yes, and in the following I will argue for this position in depth. To motivate this largely abstract analysis, I will begin by considering a concrete scenario which illustrates the need for contextual assessment and more fine-grained conceptual categories for reasoning about cases when demands for economic development come to pose a threat against established patterns of property.

\section{Donald Trump in Scotland}\label{sec:dts}

On the 10th of July 2010, the property magnate Donald Trump opened his first golf-course in Scotland, proudly announcing that it would be the ``best golf-course in the world''.\footnote{http://www.golf.com/courses-and-travel/donald-trump-scotland-golf-course-lives-hype (accessed 06 July 2014).} Impressed with the unspoilt and dramatic seaside landscape of Scotland's east coast, the New Yorker, who made his fortune as a real estate entrepreneur, had decided he wanted to develop a golf course in the village of Balmedie, close to Aberdeen.

To realize his plans, Trump purchased the Menie estate in 2006, with the intention of turning it into a large resort with a five-star hotel, 950 timeshare flats, and two 18-hole golf-courses. The local authorities were not particularly keen on the idea, and planning permission was initially denied by Aberdeenshire Council. Particularly worrying to the councilors was the fact that the proposed site for the development was declared to be of special scientific interest under EU conservation legislation. The frailty and richness of the sand dune ecosystem, in particular, suggested that the land should be left unspoilt for future generations.\footnote{See \url{http://en.wikipedia.org/wiki/Donald_Trump#Scottish_golf_course} (accessed 06 July 2014).} Trump was not deterred, however, and started lobbying Scottish politicians to gain support. In the end, he was able to convince Scottish ministers that he should be given the go-ahead on the prospect of boosting the economy by creating some 6000 new jobs.\footnote{See \url{http://www.theguardian.com/world/2008/nov/04/donald-trump-scottish-golf-course} (accessed 06 July 2014).}

Activists continued to fight the development, launching the ``Tripping up Trump'' campaign to back up local residents who refused to sell their properties.\footnote{See \url{http://www.trippinguptrump.co.uk/} (accessed 03 August 2014).} One of these, the farmer and quarry worker Michael Forbes, expressed his opposition in particularly clear terms, declaring at one point that Trump could ``shove his money up his arse''.\footnote{See \url{http://www.scotsman.com/news/donald-trump-s-plea-to-homeowners-on-the-menie-estate-1-1370270}. (accessed 03 August 2014)} Trump, on his part, had described Forbes as a ``village idiot'' that lived in a ``slum''.\footnote{See \url{http://www.bbc.co.uk/news/10205781} (accessed 08 July 2014).} Moreover, he had suggested that Forbes was keeping his property in a state of disrepair to pressure up the price of the property.\footnote{See \url{http://edition.cnn.com/2007/WORLD/europe/10/10/trump.golf/} (accessed 03 August 2014).} Forbes was offended. He proudly declared that he would never consider selling, as the issue had become personal.\footnote{See \url{http://www.scotsman.com/news/scotland/top-stories/farmer-who-took-on-trump-triumphs-in-spirit-awards-1-2668649} (accessed 03 August 2014).}

At the height of the tensions, Trump considered his legal options, by asking the local council to consider issuing compulsory purchase orders (CPOs) that would allow him to take property from Forbes and other recalcitrant locals against their will.\footnote{See \url{http://www.thesundaytimes.co.uk/sto/news/uk_news/article184090.ece} (accessed 03 August 2014).} If carried out, this would have been an iconic example of an economic development taking. Moreover, it would not be the first time that the power of eminent domain had been used to the benefit of Donal Trump's business empire. In the 1990s, Trump famously succeeded in convincing Atlantic City to allow him to take the home of one Vera Coking, to facilitate further development of his casino facilities.\footcite[297-301]{jones00} This taking was eventually struck down by the New Jersey Superior Court, however, a result that was hailed as a milestone in the fight against ``eminent domain abuse'' in the US.\footnote{See \url{https://www.ij.org/cases/privateproperty} (accessed 12 August 2014). The case also caused a surge of attention directed at the issue, see \url{http://reason.com/archives/2008/03/03/litigating-for-liberty/4} (accessed 12 August 2014). For the decision itself, consult \cite{banin98}.}

In Scotland, Trump's plans were met with widespread outrage. The media coverage was wide, mostly negative, and an award-winning documentary was made which painted Trump's activities in Balmedie in a highly negative light.\footnote{See \url{http://www.youvebeentrumped.com}.} The controversy also made its way into UK property scholarship. Kevin Gray, in particular, a leading expert in property law, expressed his opposition by making clear that he thought the proposed taking would be an act of ``predation''.\footcite{gray11}

In fact, the case prompted Gray to formulate a number of key features that could be used to identify situations where compulsory purchase would be more likely to represent an abuse of power. He noted, in particular, that Trump's proposed takings would fall in line with a general tendency in the UK towards using compulsory purchase to benefit private enterprise, even in the absence of a clear and direct benefit to the public. Indeed, it was not unrealistic to think that CPOs might come to be used in Balmedie; if he had put his weight behind it, Trump might well have been able to make a successful case that existing statutory authorities could be used to justify takings of this kind.\footnote{In particular, the \cite{tcpsa97} contains a wide authority in s. 189, stating that local authorities has a general power to acquire land compulsorily in order to ``secure the carrying out of development, redevelopment or improvement''.} It would not be hard to argue that the public would benefit indirectly in terms of job-creation and increased tax revenues. Moreover, Scottish ministers had already gone far in expressing their support for the plans.

But then, in a surprise move, Trump announced he would not seek CPOs after all.\footnote{See \url{http://news.stv.tv/north/224662-relief-for-residents-trump-lifts-threat-of-compulsory-purchase-orders/} (accessed 03 August 2014).} Quite possibly, he was discouraged by the negative press. But in addition, he had found another strategy, namely that of containment: He erected large fences, planted trees and created artificial sand dunes, all serving to prevent the properties he did not control from becoming a nuisance to his golfing guests. One local owner, Susan Monroe, was fenced in by a wall of sand some 8 meters high. ``I used to be able to see all the way to the other side of Aberdeen'', she said, ``but now I just look into that mound of sand''.\footnote{See \url{http://www.theguardian.com/world/2012/jul/10/donald-trump-100m-golf-course} (accessed 03 August 2014).} She also lamented the lack of support from the Scottish government, expressing surprise that nothing could be done to stop Trump.

But there was little left to do. As soon as Trump decided to build around them, the neighboring property owners found themselves completely marginalized. After all, Trump had the backing of the government, having been declared as a job-creator whose activities would boost the economy in the region. He had even received an honorary doctorate at the Robert Gordon University, a move that prompted the previous vice-chancellor, Dr David Kennedy, to hand his own honorific back in protest.\footnote{See \url{http://www.bbc.co.uk/news/uk-scotland-north-east-orkney-shetland-11421376}.}

But in the end, it was not by taking the land of others that Trump triumphed in Scotland. Rather, he succeeded by exercising ``despotic dominion' over his own.\footnote{To quote William Blackstone, \cite[2]{blackstone79b}.} This proved highly effective;  after he fenced them in, his neighbors were hard to see and hard to hear. The Balmedie controversy went quiet, the golfers came, Trump got his way. As he declared during the grand opening: ``Nothing will ever be built around this course because I own all the land around it. [...] It's nice to own land.''\footnote{See \url{http://www.theguardian.com/world/2012/jul/10/donald-trump-100m-golf-course} (accessed 06 July 2014).}

\subsubsection*{\ldots}

The tale of Trump coming to Scotland not only serves to illustrate the kind of scenario that I will be looking at in this thesis, it also puts the work into perspective. It shows, in particular, that what it means to protect property against undue interference can depend highly on the circumstances. For a while, it looked like Balmedie was about to become a canonical case of an economic development taking. But in the end, it became rather an illustration of something far more subtle, namely that the meaning of protecting property rights depends highly on context, our own perspective, and the values we aim to promote. 

Moreover, we are reminded that the function of property as such is deeply shaped by social, political and economic structures. It seems clear, in particular, that Donald Trump's ownership of the Menie estate has a vastly different meaning than does Michael Forbes' ownership of his small farm. To many observers, the former kind of ownership will represent some combination of power, privilege and profit, while the latter will be regarded as coming imbued with a mix of defiance, community and sustenance. Very different values are inherent in these two forms of ownership, and after Trump came to Balmedie, they clashed in a way that required the legal order to prioritize between them.

According to Trump and his supporters, protecting property rights against interference in Balmedie no doubt involves protecting the governmentally sanctioned golf resort plans from backwards locals who attempt to fight progress. In this narrative, ``protection'' can maybe even be used justify compulsory acquisition of property rights that are regarded as a hindrance to the full enjoyment of property for more resourceful members of the community. But for Michael Forbes and the other local owners, protecting property rights is likely to have a completely different meaning. To them, protecting property means above all else to protect a local community against what they see as a disruptive and damaging plan that will see both them and their properties turned into golfing props. Again, adequate protection might require an interference in property, to prevent Trump from using his land in a way that causes damage to his neighbors. Regardless of who we support, we are forced to recognize that protection implies interference and vice versa. 

This shows the conceptual inadequacy of a simplistic perspective whereby protecting property rights is seen as a black-and-white proposition, a call for limits on the state's power to do good, enforced to protect owners' right to do as they please. In reality, the situation is  more subtle, involving a number of additional dimensions. Importantly, how we assess concrete situations where property is under threat depends crucially on what we perceive as the ``normal'' state of property, the alignment of rights and responsibilities that we deem to be worthy of protection. Our stance in this regard clearly depends on our values. But values themselves are in turn influenced by the context of assessment within which they arise. More problematically still, they may be influenced by our \emph{perception} of this context, rather than by reality.

For example, property activists in the US tend to regard the value of autonomy as a fundamental aspect of property. But this must be understood in light of the idea that US society is founded on an egalitarian distribution of property, where ownership is meant to empower ordinary people by facilitating self-sufficiency and self-governance.\footnote{See, e.g., \cite[173]{ely07}.} Hence, the autonomy inherent in property ownership is not thought of as being bestowed on the few, but on the many. Protecting autonomy of owners against state interference is not about protecting the privileges of the rich and powerful, but is embraced as a way to protect {\it against} abuse by the privileged classes.\footnote{This narrative is enthusiastically embraced by US activists who fight economic development takings, see, e.g., \url{http://www.castlecoalition.org/}.} 

This, however, is only an {\it idea} of property protection. It might not correspond to the reality surrounding the rules that have been molded in its image. Indeed, it has been noted that despite the great pathos of the egalitarian property idea, egalitarianism has actually played a marginal role to the development of US property law.\footnote{\cite[361]{williams98} (``Why does the egalitarian strain of republicanism have such a substantial presence in American property rhetoric outside the law but so little influence within it?'')} More worryingly still, research indicates that land ownership in the US, while hard to track due to the idiosyncrasies of the land registration system, is not actually all that egalitarian.\footcite[246-247]{jacobs98} In this way, we are confronted with the danger of a disconnect between  values, reality and the law.

In Scotland, a similar story unfolds. Here, the traditional concern is that land rights are mainly held by the elites.\footnote{See generally \cite{wightman96,wightman13}.} As a result, Scottish property activists tend to focus on values such as equality and fairness, calling also on the state to regulate and implement measures to achieve more egalitarian control over the land. Indeed, reforms have been passed that sanction interference in established property rights on behalf of local communities.\footnote{See generally \cite{lovett11,hoffman13}.} At the same time, cases like Balmedie illustrate that the Scottish government, now with enhanced powers of land administration, may well choose to align themselves with the large landowners. Moreover, research indicates that recent reforms in Scottish planning law, which serve to enhance the power of the central government, has the effect of undermining local communities and their capacity for self-governance.\footnote{See generally \cite{pacione13,pacione14}.} Again, the danger of a disconnect between influential property narratives and reality is brought into focus.

On the other hand, it seems that grass root property activists in the US and Scotland may well be closer in spirit than they seem. Upon closer examination I cannot help thinking that they share many of the same concerns and aspirations, and that the differences arise mainly from the fact that they operate in different contexts and engage with different discourses of property. The challenge is to find categories of understanding that allow us to make sense of their shared spirit, as well as the spirit they oppose.

I think the example of Balmedie suggests a possible first step, by illustrating the need for a framework that will allow commentators to  deny that there is any inconsistency between opposing compulsory purchase orders while also supporting strict property regulation in the context of fighting a golf resort. Both of these positions, moreover, should be viewed as efforts to protect property. To the classical ``individual rights v state interference'' debate, such a dual position can be hard to make sense of. But in my opinion, this only points to the vacuity of such a conventional narrative.

In general, I think it is hard, close to impossible, to make sense of many contemporary disputes over property if we do not have the conceptual acumen to distinguish between egalitarian property held under a stewardship obligation by members of a local community, and feudal property held by businesses for investment. Moreover, there is no contradiction between promoting the value of autonomy for one of these, while emphasizing state control and redistribution for the other. The broader theme is the contextual nature of property, and its implications for protection of property rights. In the coming sections, I will locate a theoretical basis that will allow me to take advantage of this viewpoint in my legal analysis.

\section{Theories of property}\label{sec:top}

What is property? In common law jurisdictions, the standard answer is that property is a collection of individual rights, or more abstractly, {\it entitlements}.\footnote{The term ``entitlement'' was used to great effect in the seminal article \cite{calabresi72}.} Being an owner, it is often said, amounts to being entitled to one or more among a bundle of ``sticks'', streams of protected benefits associated with, or even serving to define, the property in question.\footcite[357-358]{merrill01} This point of view was first developed by legal realists in response to the natural law tradition, which conceptualized property in terms of the owner's dominion over the owned thing, particularly his right to exclude others from accessing it.\footcite[193-195]{klein11} In civil law jurisdictions, rooted in Roman law, a dominion perspective is still often taken as the theoretical foundation of property, although it is of course recognized that the owner's dominion is never absolute in practice.\footnote{For a comparison between civil and common law understanding of property, see generally \cite{chang12}.}

In modern society, the extent to which an owner may freely enjoy his property is highly sensitive to government's willingness to protect, as well as its desire to regulate. To civil law theorists, this sensitivity has been thought of as giving rise to various restrictions in property rights, but for common law theorists, overlooking a legal system with roots in a relatively stable feudal tradition, it has been thought of as {\it constitutive} of property itself.\footcite[7]{chang12} Indeed, the bundle of rights theory has long historical roots in common law. Arguably, it was distilled from the traditional estates system for real property, which was turned into a theoretical foundation for thinking about property in the abstract.\footnote{See \cite[7]{chang12}   
(``The ``bundle of rights'' is in a sense the theory implicit in the common law system taken to its extreme, with its inherently analytical tendency, in contrast to the dogged holism of the civil law.'').} 

However, during the 18th and 19th century, natural law thinking was also highly influential in common law. This is evidenced, for instance, by the works of William Blackstone and James Kent.\footnote{See generally \cite{blackstone79b,kent27}.} But towards the end of the 19th century, it became increasingly hard to reconcile such an approach to property with the reality of increasing state regulation. Hence, the bundle metaphor that gained prominence in the early 1900s can be seen to indicate a return to a more modest perspective.\footcite[195]{klein11}

Property rights under the bundle theory are thought to be directed primarily towards other people, not things.\footnote{See \cite[357-358]{merrill01} (``By and large, this view has become conventional wisdom among legal scholars: Property is a composite of legal relations that holds between persons and only secondarily or incidentally involves a ``thing''.'').} This underscores a second point about property in the real world, namely that the content of rights in property are necessarily relative to the totality of the legal order. For instance, relying on a bundle metaphor, it becomes perfectly natural that a farmer's property rights protects him against trespassing tourists, but not against the neighbor who has an established right of way. 

By contrast, the dominion theory suggests viewing such situations as exceptions to the general rule of ownership, which implies a right to exclusion at its core. In the case of property, exceptions no doubt make up the norm. But in civil law jurisdictions one lives happily with this. It takes the grandeur away from the dominion concept, but it retains a nice and simple structure to property law. In the civil law world, it is common to say that what the owner holds is the {\it remainder} after all positive rights, serving to restrict his dominion, have been deducted.\footcite[25]{chang12} Moreover, while there may be many limitations and additional benefits attached to property, they are all in principle carved out of one initial right, namely that of the owner. In this way, the system becomes more easily navigable.

An interested party may ask, ``who owns this land?'' Then, under the dominion theory, a clear answer is expected and will usually be adequate, even if it does not give a complete picture of all relevant property rights. Under the bundle theory, on the other hand, one might be inclined to respond, ``to which stick are you referring?'' Clearly, this narrative is more complex, perhaps unduly so. 

Some common law scholars have recently elaborated on this to develop a critique of the bundle theory, by suggesting that it should at least be complemented by a firm theory of {\it in rem} rights in property. This, they argue, would allow the law to operate more effectively, by relying on a simple and clear rule that, although defeasible, will generally suffice to inform people about their relevant rights and duties in relation to property.\footnote{\footcite[793]{merrill01b} (``The unique advantage of in rem rights -- the strategy of exclusion -- is that they conserve on information costs relative to in personam rights in situations where the number of potential claimants to resources is large, and the resource in question can be defined at relatively low cost.''); \footcite[389]{merrill01} (``The right to exclude allows the owner to control, plan, and invest, and permits this to happen with a minimum of information costs to others.''). See also \cite{ellickson11} (arguing that Merrill and Smith's analysis nicely complements and improves upon the bundle theory).} 

There are also other, less pragmatic, reasons why a dominion approach might be preferable, even if the bundle metaphor is arguably more accurate. In particular, some scholars point out that the bundle theory does not adequately reflect the sense in which property is a right to a {\it thing}, serving to create an attachment that is not easily reducible to a set of interpersonal legal relationships.\footnote{\cite[1862]{merrill07}. For a slightly different take on attachment, highlighting how the thingness of property marks its conditional nature and transferability, see \cite[799-818]{penner96}.} In the US, where the bundle theory has traditionally been dominant, critique like this seems to be gaining ground.\footnote{See generally \cite{foster10}.}

But in this thesis, the efficiency and clarity of different property concepts will not be a primary concern, nor will personal attachments to things in themselves play a particularly important role.\footnote{I mention, however, that the personhood-aspects of property that are sometimes highlighted in this regard will also be relevant to my analysis of economic development takings. However, this is not something that I think warrants extensive engagement with the bundle v dominion debate. I note, for instance, that in the work of Margaret Jane Radin, one of the main proponents of persoonhood accounts, the bundle theory is not challenged as much as it is readjusted, although in places it also seems to be the object of some implicit criticism, see, e.g., \cite[127-130]{radin93}.}
Hence, I will remain largely agnostic about this aspect of the debate between dominion and bundle theorists. In particular, the differences between civil and common law traditions in this regard do not cause special problems for my analysis of economic development takings. However, I am also more broadly interested in the values that are promoted by different ways of looking at property, particularly with regards to the question of when interference is legitimate under constitutional and human rights law. Hence, I  now turn to the question of whether or not there are any significant differences between dominion and bundle theories in this regard.

Intuitively, one might think that bundle theorists are likely to endorse greater room for state interference in property rights. Indeed, thinking about property as sticks in a bundle may lead one to think that property rights are intrinsically limited, so that subsequent changes to their content -- carried out by a competent body -- are mere reflections of their nature, not a cause for complaint. In particular, the theory conveys the impression that property is highly malleable. For the theorists that developed the bundle of sticks metaphor in the late 19th and early 20th century, this aspect was undoubtedly very important. By providing a highly flexible concept of property, they helped the state gain conceptual authority to control and regulate. Indeed, this was the clear intention of many early proponents of the bundle theory -- the ``progressives'' of their day.\footcite[195]{klein11} The early bundle theorists not only developed a theory to fit the law as they saw it, they also contributed to change.

In relation to takings law, the progressives succeeded in gaining acceptance for the use of eminent domain to benefit a wider range of public purposes than had so far been considered legitimate.\footnote{See generally \cite{yale49}.} In particular, they argued successfully that the so-called ``public use'' restriction, which had previously been enforced quite strictly, particularly by state courts, should be greatly relaxed. This change was important in creating the situation which led to economic development takings becoming a contentious issue in the US, and so provides important background to the main topic of my thesis.  I return to the public use debate in the US in much more depth later, in Chapter \ref{chap:2}, Section \ref{sec:hop}. Here I would like to stress that I think there can be little doubt that the increased scope given to eminent domain in the early 20th century was mutually conducive to the conceptual reorientation that took place during the same time.

In relation to the different, but related, issue of so-called regulatory takings, the bundle theory even  became directly implicated. A regulatory taking occurs when governmental control over the use of property becomes so severe that it must be classified as a taking in relation to the law of eminent domain. Particularly in the US, the question of when regulation amounts to a regulatory taking is highly controversial. The stakes are high because takings have to be compensated in accordance with the Fifth Amendment of the US constitution. At the same time, the law is unclear; a lack of statutory rules means that regulatory takings cases are often adjudicated directly against constitutional property clauses (often the relevant state constitution, in the first instance).

If property is thought of as a malleable bundle of entitlements that exists only because it is recognized by the law, it becomes more natural to argue that when government regulates the use of property, it does not deprive anyone of property rights, but merely restructures the bundle. In the case of {\it Andrus v Allard}, the Supreme Court adopted such an argument when it declared that ``where an owner possesses a full ``bundle'' of property rights, the destruction of one ``strand'' of the bundle is not a taking, because the aggregate must be viewed in its entirety''.\footcite[65--66]{andrus79}

Historically, therefore, it seems that bundle theorists have been largely aligned with those that favor a less restrictive approach to eminent domain. But I think it is wrong to conclude that the bundle theory {\it necessarily} implies such a stance on takings. Indeed, some prominent scholars have argued for an almost entirely opposite view. Professor Epstein, in particular, goes far in suggesting that as every stick in the property bundle represents a property right, government should not be permitted to remove any of them without paying compensation.\footcite[232-233]{epstein11} Moreover, Epstein does not believe that the bundle theory is responsible for the fact that his view of property has not been widely endorsed by US courts. Instead, he thinks the main (negative) impact of ``progressive'' thinkers stems from their tendency to adopt a ``top-down'' approach to property. That is, Epstein directs attention towards their tendency to view property rights as vested in, and arising from, the power of the state, not the possessions of individuals.\footnote{\cite[227-228]{epstein11} (``In my view, the nub of the difficulty with modern property law does not stem from the bundle-of-rights conception, but from the top-down view of property that treats all property as being granted by the state and therefore subject to whatever terms and conditions the state wishes to impose on its grantees'').} 

In my opinion, Epstein's argument shows that adoption of the bundle theory can hardly be considered a determinate factor for the kind of protection private property enjoys in a given legal system. Moreover, Epstein successfully demonstrates that as a rhetorical device, the theory may well be turned on its head. Unsurprisingly, the substance of the law, in the end, turns primarily on the values one adheres to, not the theoretical constructions one relies on when expressing those values.\footnote{To further underscore this point, it may be mentioned that while US courts do in fact recognize that a regulation can amount to a taking, this is practically unheard of in several other common law jurisdictions, including England and Australia, which nevertheless paint property in a similar conceptual light. Moreover, while the issue of regulatory takings is considered central to constitutional property law in the US, it is considered a fairly marginal issue in England, see \cite{purdue10}.}

In the civil law world, the relationship between property theorizing and property values is similarly hard to pin down at the conceptual level. To illustrate, I will again point to the question of regulatory takings. In some countries, like Germany and the Netherlands, the right to compensation is quite strong, but in other civil law countries, such as France and Greece, it is very weak.\footnote{See generally \cite{alterman10}.} In particular, the exclusive dominion understanding of property does not commit us to any particular kind of policy on this point. Indeed, the theory appears to cater comfortably to a range of different politically determined solutions to the problem of striking a balance between the interests of owners and the interests of the state. 

On the one hand, the undeniable fact of modern society is that property rights are enforced, and limited, by the power of government. Hanging on to the idea of dominion, then, necessarily forces us to embrace also the idea that dominion is not enjoyed absolutely and that government may interfere in property rights. In this way, the theory may serve as a conceptual basis upon which to argue for a more relaxed approach to protection of property rights. These rights are not absolute anyway, so why worry about interfering in them for the common good? But this story too may be turned on its head: A libertarian may well use the same image to tell a tale of property being ripped apart at its seams. Hence, he may argue, unless we want to completely lose our grasp of what property is, we had better enhance the level of protection offered to property owners.

To me, the upshot is that the differences between common law and civil law theorizing about property are not significant enough to 
make them crucial to the questions studied in this thesis. In particular, the differences between the bundle theory and the dominion idea do not appear to speak decisively in favor of any particular approach to economic development takings, nor does it provide any clear justification for regarding such takings as special in the first place. Property enjoys constitutional protection and is a recognized human right across the divide, but what this means in practice is hard to deduce from either account.

In terms of descriptive content, both theories are too bold and oversimplified. They provide a manner of speech, but they do little to enhance our understanding of the reality of property rights in modern society. In particular, they do not provide a functional account of what role property plays in relation to the social, economic and political structures within which it resides. In terms of normative content, on the other hand, they are both too bland and imprecise. They simply do not offer much clear guidance as to what norms and values the institution of property is meant to serve. They give neat explanations of what property is, but do not tell us {\it why} it should be protected. 

In the following, I will address both these shortcomings by considering property theories with a wider scope. There are many candidates that could be considered. In a recent book on property theory, Alexander and Pe\~{n}alver present five key theoretical branches: 
\begin{itemize}
\item {\it Utilitarian} theories, focusing on property's role in helping to maximize utility or welfare with respect to individual preferences and desires.\footnote{\cite[Chapter 1]{alexander10}.} 
\item {\it Libertarian} theories, focusing on property's role in furthering individual autonomy and liberty, as well as the importance of protecting property against state interference, particularly attempts at redistribution.\footnote{\cite[Chapter 2]{alexander10}.} 
\item {\it Hegelian} theories, focusing on the importance of property to the development of personhood and self-realization, particularly the expression and embodiment of free will through control and attachment to one's possessions.\footnote{\cite[Chapter 3]{alexander10}.}
\item {\it Kantian} theories, focusing on how property arises to protect freedom and autonomy in a coordinated fashion so that {\it everyone} may potentially enjoy it, through the development of the state.\footnote{\cite[Chapter 4]{alexander10}.}
\item {\it  Human flourishing} theories, focusing on property's role in facilitating participation in a community, particularly as a template allowing the individual to develop as a moral agent in a world of normative plurality.\footnote{\cite[Chapter 5]{alexander10}.}
\end{itemize}

It it beyond the scope of this thesis to give a detailed presentation and assessment of all these theoretical branches and the various ideas that have been discussed within each of them. However, in Section \ref{sec:hf} below, I will present the human flourishing theory in more detail. This is because I believe that if it is adopted, it suggests making a range of new policy recommendations regarding how the law {\it should} approach the question of economic development takings. 

First, however, I note that all the theories mentioned above are highly normative, used actively to promote the adoption of particular values associated with property. While I am not unwilling to take a stand in this debate, my main objective is to study economic development takings descriptively, by giving a case study of Norwegian waterfalls and discussing its significance in terms of comparative and human rights law. Hence, before I move on to consider other aspects, I first need a theoretical framework that allows me to meaningfully discuss those aspects that make economic development takings unique. I would like to do so, moreover, without thereby committing myself to any particular stance on how to normatively assess those aspects. 

To arrive at such a foundation, I will rely on the descriptive parts of the so-called {\it social function theory} of property.\footnote{See generally \cite{foster11,mirow10,alexander09a}. Be aware that some authors, particularly in the US, also speak of the {\it social obligation} theory, using it more or less as a synonym for the social function theory.} While this theory is often implicated in normative theories, including the human flourishing theory, I argue that it has a descriptive core which is also of great significance. Its importance to my work in this thesis is underscored by the fact that I will draw on the social function theory to answer the pressing problem of what makes economic development takings a legitimate and useful category of legal reasoning. 

Let me first reiterate that it is not {\it prima facie} clear that the category makes any legal sense at all, due to the fact that many jurisdictions lack rules that explicitly make the purpose of interference a relevant measuring stick for assessing legitimacy. To respond successfully to this potential objection, I believe it is necessary to look at property's social functions. In fact, property scholars are becoming increasingly aware of the need to do this in general, as they note that existing theories are overly focused on a narrative that revolves around individual entitlements. Many still reject that this necessitates conceptual reconfiguration, but the social function idea of property appears to be gaining ground, also as an important aid in making sense of how the law actually works. I believe this descriptive aspect of the theory provides the most appropriate way to argue that it is theoretically desirable to regard economic development takings as a special issue in property law, and I will argue for this in Section \ref{sec:edt}.

However, before making my specific point about takings, I will present the social function theory of property more generally. I will focus on showing that it captures aspects that are already highly relevant -- behind the scenes -- to how property rules are understood and applied in concrete situations. It seems, in particular, that socio-legal arguments play an important, yet often unacknowledged, role when courts interpret fundamental rules that are meant to protect private property. Bringing those aspects into the open is in itself a worthwhile project to pursue, irrespectively of any further normative stances that the social function theory might give the theorist occasion to adopt.

\section{The social function of property}

There is a growing feeling among property scholars that the notion of property has been drawn too narrowly by many of the traditionally dominant theories of property. Some have even gone as far as to challenge the idea that property is a meaningful and well-defined concept at all. These scholars point out that what counts as property in a given legal system, and what property entails in that system, depends largely on its social and political context, tradition, and even chance.\footnote{For a particularly inspiring exposition of property's elusive nature, see \cite{gray91}.} In the US, a utilitarian law-and-economics approach -- which largely takes the social and political underpinnings of property for granted -- has long been regard as standard, but even there the tide is turning. While most US scholars still regard property as a robust and meaningful category of legal thought, many are increasingly turning away from assessing property rules narrowly against their effectiveness in maximizing individual utility and social welfare. Instead, these scholars adopt a holistic approach, which allows property's social function to come into focus. One of the main proponents of this conceptual shift is Gregory S. Alexander, professor at Cornell University. In a recent article, he writes:

\begin{quote} Welfarism is no longer the only game in the town of property theory. In the last several years a number of property scholars have begun developing various versions of a general vision of property and ownership that, although consistent with welfarism in some respects, purports to provide an alternative to the still-dominant welfarist account.[...] These scholars emphasize the social obligations that are inherent in ownership, and they seek to develop a non-welfarist theory grounding those inherent social obligations.\footcite[1017]{alexander11}
\end{quote}

To scholars coming from political science, sociology or human geography, this trend will not raise many eyebrows, except perhaps for the fact that it is a recent one. After all, in fields such as these, property has never been understood merely as a set of individual entitlements that are meant to result in increased welfare. Rather, property is seen as a crucial part of the fabric of society, one that entrenches privileges and bestows power.\footnote{See generally \cite{carruthers04}.} Even scholars who believe that the institution of property is a force for good, recognize that being an owner carries with it political capital, social responsibility, and membership in a community. Those aspects, moreover, are often regarded as more important than entitlements explicitly recognized in positive legal terms. Crucially, they are important not only to the individual owners but also to society as a whole. How property rights are distributed among the population, for instance, has obvious political and economic implications, serving as a source of power and prosperity to some groups, while marginalizing others.\footnote{See, e.g., \cite[23]{carruthers04}. (``The right to control, govern, and exploit things entails the power to influence, govern, and exploit people'').}

But what is the relevance of this to property law? Usually, jurists approach property in isolation from such concerns, and often they do so because of practical necessity. The political question of what the law should be might require musings about the purpose and social context of property, but in the day-to-day workings of the law, the story goes, such considerations play a lesser role, with the importance of clear and simple rules outweighing the possible benefit that would result from contextual and holistic assessment. But at the same time, no functioning theory of  property would deny that social aspects such as expectations and obligations play a role in relation to property {\it in life}. The problem, rather, is that classical theories of property may be accused of taking the pragmatic view too far, by failing to recognize that many social functions are {\it intrinsic} to property, so that they may sometimes be impossible to disregard, also when the law is applied to concrete disputes.

This accusation can be raised against both bundle and dominion theorists. They both tend to leave little conceptual room for considering property as a social phenomena. It is recognized, of course, that rights in property -- bundled or otherwise -- serve to regulate social relations. But this effect is typically regarded as belonging to the periphery of property as a legal category, more relevant to sociologists than to property scholars. In addition, it is uncommon to observe that the causal relation between property rights and society is bidirectional, since the meaning and content of property itself is partly determined by the very same social structures that property helps establish and sustain. When this aspect of property is not recognized, the risk is that subtle dependencies between property and the political order are not brought into focus, even when they play an important role in practice.

This is particularly clear when it comes to socially defined obligations attached to property. Hardly anyone would protest that in practical life, what an owner will do with his property is as much constrained by the expectations of others as it is by law. But in addition to influencing the owner subjectively, expectations can take on an objective character by being embedded strongly in the social fabric. This, in turn, may give rise to a norm, or even a custom, which may be legally relevant, either because the law gives direct effect to it, or because it influences how we interpret rules relating to the use of property.\footnote{See generally \cite{penalver09,alexander09}.}

This seems hard to dispute as a descriptive assertion, but traditional property theorists have surprisingly little regard for it. According to Alexander, the classical theories of property convey the impression that ``property owners are rights-holders first and foremost; obligations are, with some few exceptions, assigned to non-owners''.\footcite[1023]{alexander11} The social function theorists attempt to redress this imbalance by developing theories that can naturally accommodate an account of social obligations that attaches greater weight to them as objects of property. As Alexander explains, ``social obligation theorists do not reverse this equation so much as they balance it. Of course property owners are rights-holders, but they are also duty-holders, and often more than minimally so.''\footcite[1023]{alexander11} 

It should be noted that while it lay dormant for some time, particularly in the US, this idea is by no means new. Its first modern expression is often attributed to Le{\'o}n Duguit, a French jurist active early in the 20th century. In a series of lectures he gave in Buenos Aires in 1911, Duguit challenged the classic liberal idea of property rights by pointing to their context-dependence, adopting a line of argument strikingly similar to how recent scholars have criticized the law-and-economics discourse of modern times.\footnote{See \cite[1004-1008]{foster11}. For more details about Duguit's work and the contemporaries that inspired him, see generally \cite{mirow10}.} In particular, Duguit also pointed to the notion of obligation, stressing the fact that individual autonomy only makes sense in a social context, wherein people are also dependent on each other and related through membership in communities. Hence, depending on the social circumstances of the owner, his property could entail as many obligations as it would entail entitlements and dominion. This, according to Duguit, was not only the reality of property ownership in life, it was also a normatively sound arrangement that should inspire the law, more so than the unrealistic visions of property evoked by the liberal tradition.\footnote{See \cite[1005]{foster11} (``The idea of the social function of property is based on a description of social reality that recognizes solidarity as one of its primary foundations'', discussing Duguit's work). It should also be noted that Duguit was particularly concerned with owners' obligations to make productive use of their property, to benefit society as a whole. This raises the question, however, to which we shall return in more depth later, who exactly should be granted the power to determine what counts as ``productive use''. In this way, Duguit's work also serves to underscore one of the main challenges of regulatory frameworks that seek to incorporate and draw on property's social dimension. How should decisions be made in such regimes?} 

Similar thoughts have been influential in Europe, particularly in the post-WW2 rebuilding period. For instance, as I discuss further in Chapter \ref{chap:2}, Section \ref{sec:germany}, the constitution of Germany -- her {\it Basic Law} -- contains a property clause that explicitly includes a provision stating that property entails obligations as well as rights. As argued by Alexander, this has had a significant effect on German property jurisprudence, creating a clear and interesting contrast with US law.\footnote{See \cite[338]{alexander03} (``The German Constitutional Court has adopted an approach that is both purposive and contextual, while the U.S. Supreme Court has not'').} 

A social perspective on property was also influential during the debate among the European states that first drafted the property clause in the First Protocol to the European Convention of Human Rights.\footnote{See \cite[1063-1065]{allen10}.} Later, however, the liberal conception of property gained ground also in Europe, causing jurisprudential developments that have been particularly clear in the case law from the European Court of Human Rights.\footnote{See generally \cite{allen10}.} Even so, property theorizing in Europe is still influenced by a social function view on property, more so than in the US. The European Court of Human Rights, for instance, stresses the importance of {\it proportionality} and {\it fairness} when adjudicating property cases, suggesting the importance of a contextual approach to the balancing of the many private and public interests involved.\footnote{See generally \cite[Chapter 5]{allen05}.}

I will return to possible normative implications of the social function theory later, but here I would like to stress that in the first instance it merely asks us to recognize an empirical truth. Property does not arise in a vacuum, but from within a society. As a philosophical proposition, this is obvious and hardly anyone denies it. But the social function theory asks us to consider something more, namely that property {\it law} continues to influence, and be influenced by, the social structures that surround it. Perhaps most importantly, property both reflects and shapes relations of power among members of a society.\footnote{This aspect of property's social function was stressed in a recent ``statement of purpose'' made by leading property scholars in support of the social function theory, see \cite{alexander09a}.} Moreover, it does not act uniformly in this way -- the actual effect of property on power depends on the circumstances.

An indebted farmer who is prevented by state regulation from making profitable use of his land might come to find that his property has become a burden rather than a privilege. As a consequence, someone who has already amassed power and wealth elsewhere might be able to purchase it from him cheaply. Indeed, this might provide an excellent opportunity for the outsider to consolidate his position. He can ensure that his privileges become further entrenched, both socially, politically and economically. By acquiring a farm and transforming it to recreational property, he symbolically and practically asserts his dominance and power, while also reaping a potential financial benefit resulting from his investments in a more ``modern'' pattern of use. In some cases, this dynamic can even become endemic in an area, resulting in a complete reshaping of the social fabric surrounding property.

The story might go like this: First, impoverished farmers and other locals sell homes to holiday dwellers. Then house prices soar. As a result, local people with agrarian-related incomes can't afford local homes, causing even more people to sell their land to the urban middle class. In this way, a causal cycle is established, the social consequences of which can be vicious, particularly to the low-income people who are displaced.\footnote{The general mechanism is well-documented and known as {\it gentrification} in human geography (often qualified as rural gentrification when it happens outside urban areas). See generally \cite{weesep94,phillips93,slater06}. For a case study demonstrating the role that state regulation can play (perhaps inadvertently) in causing rural gentrification, see \cite[1027-1030]{darling05}.} My theoretical contention is the following: Setting out to protect property in a situation like this -- when property rights pull in different directions -- requires taking some stance on whose property, and which of property's functions, one is aiming to protect. In particular, should the law protect the property rights of local people who face displacement, or should it protect the property rights of outsiders wishing to invest?

Some may shun away from this way of posing the question, by arguing that it would be better to rely on clear rules that can deliver justice to owners with a minimum level of dependence on the particular social processes that property is involved in at any given time. I am inclined to disagree with such a stance from the outset, since justice itself is a notion that largely seems to depend on social conditions. However, my main point here is that the prospect of such ``socially neutral'' rules is simply illusory when both sides of a conflict are in a position to adopt a property narrative to argue for their interests. For an excellent example of such a situation, it is enough to return to the story of Donald Trump coming to Scotland that we told in Section \ref{sec:dts}.

As long as Trump threatened to use compulsory purchase, the local people could adopt a traditional ``pro-property'' stance against Trump. But as soon as Trump decided to fence them in by relying on his own property rights, they had to adopt a seemingly contradictory view, {\it against} property, on the basis that Trump's rights should be limited out of concern for the community. So how do we classify the anti-Trump stance with regards to property? The answer is unclear under classical theories, but under the social function stance, it becomes easy to resolve. The locals sought to protect property, but not just any property. The property they wanted to protect was the property which served the social function of sustaining the existing community. The property they wanted to protect was the property that {\it meant} something to them.

Undoubtedly, this was also the sentiment of Trump and his supporters, who could also make a case based on property. Hence, in conflicts such as these the law will invariably have to take a stand regarding which social functions it wishes to promote. In all likelihood, such a stand must also sometimes be taken by whoever {\it interprets} the law, since it is exceedingly unlikely that the legislature will ever be able to provide clear rules for resolving all conflicts of this kind. Lastly, and most controversially, the courts may find occasion to curtail the power of government -- perhaps even the legislature -- if their power is usurped by powerful actors wishing to undermine property's proper functions to further their own interests. This, in particular, becomes the question of constitutional and human rights limits to interference in property.

Property shapes and reflects societies, but it also shapes and reflects social commitments and dependencies within those societies.\footnote{See generally \cite{alexander09}.} Again, this function of property is highly dependent on context. A small business owner, by virtue of being a member of the local community, is discouraged from becoming a nuisance to his neighbors. Everything from erecting bright neon signs to proposing condemnation of neighboring properties are actions that he will be socially deterred from taking. If the local shop owner does not conform to social expectations, he will pay a social price. Indeed, most likely even an economic price, especially if his customer-base is local. At the same time, the local connection would serve to make the business owner positively invested in the well-being of the community. This would encourage everything from sponsoring local events to hiring local youths as part-time helpers.

But at the same time, the local business owner might be discouraged from changing his business model to become more competitive, if this is perceived as a threat to other members of the community. Economic rationality might suggest that he should expand, say, by physically acquiring more space and targeting new groups of customers, but social rationality might make this an untenable proposal. This, however, might render the business economically unsustainable, particularly if it is facing fierce competition from businesses that are not similarly constrained by community ties. Moreover, even if the business is in fact viable as long as the community remains in place to support it, the perception that there is room for improvement might increase external pressures both on the business and the community. Importantly, in the age of regulation for commercial facilitation, the state itself might exert pressure of this kind.

Then, if our local shop owner goes out of business, for whatever reason, the new owner might fail to become integrated in the community in the same way, with obvious consequences for the property's function in that community. Indeed, if we imagine that the new owner is a large commercial actor who is hoping to raze the community in order to build a new shopping center, we are at once reminded of the stark contrasts that can arise between various social functions of property. The property rights of a shop owner can be the life nerve of a community, while the exact same rights in the hands of someone else can spell destruction. While this is an undeniable empirical fact of property ownership, it is far from clear what its legal ramifications are. Here, it is tempting to embrace a normative stance, and argue for particular social values that the law {\it should} promote. However, I would like to hold on to the descriptive mode of analysis a little further. For it is perfectly clear that regardless of whose interests win out in the end, assessments of the social function of property will have played an important role in brining about that outcome.

This is true not only when the law explicitly requires that this function is to be taken into account, such as in relation to the property clause of the basic law of Germany. It also commonly becomes true, as courts search for information to guide them in their interpretation and application of statutory rules that are seemingly not concerned with social aspects of property. The classical example from the US is the case of {\it State v Shack}.\footcite{shack71} The case concerned the right of a farmer to deny others access to his land, a basic exercise of the right to exclusion often regarded as fundamental to the very definition of property. The controversy arose after the two defendants, who worked for organizations that provided health-care and legal services to migrant farmworkers, entered the land of a farmer without permission. They were there to provide services to the farmers employees, and when the farmer asked them to leave, they refused.

In the first instance. they were convicted of trespassing in keeping with New Jersey state law, but on appeal the Supreme Court of New Jersey overturned the verdict. The court held that the dominion of the land owner did not extend to dominion over people who were rightfully on his land. Hence, as long as the defendants were there at the request of the workers, the owner had to tolerate this. Importantly, the court argued for this result -- which was not based on any natural reading of the New Jersey trespass statute -- by pointing also to the fact that the community of migrant workers was particularly fragile and in need of protection. Their right -- in property -- to receive visitors where they work and live, therefore, had to be recognized, in spite of this limiting the farmer's exclusion right.

The lesson to take from this is that the social function of property can play a role even when this does not explicitly follow from any property rules. This, in turn, may be used to argue that a shift towards a social function theory is desirable. In so far as the property rules we rely on explicitly directs us to take the social aspect of property into account when applying the law, it might be permissible for the practically minded jurist to conclude that there is little need for theorizing about property's social dimension. This dimension, in so far as it is relevant, is quantified inside the law itself, not by theories that encompass it. But as a matter of fact, cases like {\it State v Shack} show that the social dimension can be relevant even when it is not mentioned in any authority, even in relation to clear rules that would otherwise appear to leave little room for statutory interpretation. It arises as relevant, in such cases, because the social dimension is intrinsic to property itself. 

This might still be a radical claim, but it is primarily a descriptive one. Indeed, even if the case of {\it State v Shack} had gone the other way, I would be inclined to take from it the same lesson. If the owner's right to exclusion had received priority over the workers right to receive guests and the owner's obligation to respect this right, that too would be an outcome that would likely underscore the social function of property. To illustrate this, it is helpful to look to an article written by Eric Claeys, where he is critical both of the social function theory in general and {\it State v Shack} in particular.\footcite{claeys09} Given the basis on which that decision was made, he is led to argue, however, by also pointing to those aspects of the social context that speak in favor of the farmer.\footnote{\cite[941-942]{claeys09}.} Indeed, since he aims to engage with the social function theorists, he cannot simply declare  that the trespass rules are absolute and that the social circumstances are irrelevant. 

Instead, he argues that by considering the circumstances in {\it more} depth, a different outcome suggests itself.\footnote{\cite[941]{claeys09} (``there are good reasons for suspecting that there was more blame to go around in Shack than comes across in the case's statement of facts'').} But even if this is true, it is no argument against the descriptive content of the social function theory, merely an argument against those who think that particular values need to be endorsed by anyone willing to look to the social context of a property dispute. In this regard, it is not hard to agree with Claeys that normative fundamentalism is wrong. Indeed, he might even have a point in criticizing some social function theorists for normative naivety.\footnote{\cite[945]{claeys09} (``Judges might think they are doing what is equitable and prudent. In reality, however, maybe
they are appealing to a perfectionist theory of politics to restructure the law, to redistribute property, and ultimately to dispense justice in a manner encouraging all parties to become dependent on them.'')} 

I do not follow Claeys, however, when he takes this to be an argument {\it against} the form of legal reasoning that social function theories promote, and which he himself skillfully engages in.\footnote{In particular, I do not follow the leap Claeys makes when he suggests that it is beneficial to keep ``discretely submerged'' what he describes as ``culture war overtones'' in legal reasoning.\cite[947]{claeys09}.} In {\it State v Shack}, for instance, such reasoning was clearly in order. To engage in it was far {\it less} naive than to dismiss it on the basis that it would be irrelevant to the case. Indeed, if it the social function view had been dismissed, the entitlement-based idea of property would in effect do {\it unacknowledged} normative work, with no basis in anything more authoritative than a palpably oversimplified idea of the meaning of property. 

However, I agree with Claeys that prudence is in order. Moreover, I am not saying that the social function theory does not have normative consequences. It clearly does. Invariably so, simply because it provides a new way of talking about property and analysing conflicts, which will in turn influence our normative assessments. This is also illustrated by {\it State v Shack}. Despite Claeys skillful advocacy, many would no doubt fail to be convinced of the social merit of recognizing a right to exclusion in this case. But the crucial aspect of the social function narrative is that it makes such aspects clear, not that it commits us to, or promises to deliver, any morally superior stance on property that can deliver ``correct'' outcomes in cases such as this.

This challenges a common assumption, among both detractors and supporters of the social function theory, who argue that the theory commits us to a particular form of normative assessment, in pursuit of the ``good''. Some even argue that property law should be studied from the point of view of virtue ethics.\footnote{See generally \cite{penalver09}.} Unsurprisingly, critics such as Claeys use this to launch attacks on the social function theory and its supporters, by arguing that it represents a way of thinking that will invariably lead to lessened constitutional property protection and greater risk of abuse of state powers.\footnote{See \cite{claeys09} (``The more ``virtue'' is a dominant theme in property regulation, the less effective ``property'' is in politics, as a liberal metaphor steering religious, ethnic, or ideological extremism out of the public square'').} Indeed, increasing the room for state interference is often seen as the aim of conceptual reconfiguration; the social function view of property tends to be associated with social democratic and/or redistributive political projects, by which the notion of property is recast to justify greater interference in established rights.\footnote{Despite his commitment to ``value-pluralism'', this motivation is also clearly felt in the work of Gregory Alexander. He argues, for instance, that the social obligations inherent in property imply that the ``state should be empowered and may even be obligated to compel the wealthy to share their surplus with the poor'', see \cite[746]{alexander09}. For an assessment linking similar views on property in Europe to the dominance of social democratic thought in the post-WW2 period, see \cite{allen10}.}

It is important to note, however, that while social democratic policies may be easier to justify by emphasizing the social function of property, the mere recognition that property has an important social dimension does not in itself offer any justification for policies of this kind. For one, policy reasons must be tied to the prevailing social and economic circumstances, they will not automatically succeed merely by virtue of a conceptual shift. In addition, it seems to me that the most crucial premises used in arguments for greater state control and state-led redistribution projects concern the nature of the state, not the functions of property.

In particular, why should we believe that the state is the ultimate social institution to which property {\it should} answer? Is it not, for instance, equally possible to contend that property should continue to answer to less formal social structures that it is already embedded in by virtue of owners' membership in local communities? If so, one might as well want to limit the state's role to that of ensuring fair play among individuals and communities. A contentious question, then, might be to what extent the state should actively promote certain kinds of communities in accordance with political goals. Embracing more direct state control, on the other hand, would no longer seem very natural, at least not as a goal in itself. Indeed, on the social function view, the very idea of direct state control seems to depend on the claim that more low-level social structures fail to function properly and, crucially, that state control is {\it better}. In my opinion, this requires a separate argument. Hence, to move uncritically between talk of the ``community'' and talk of the state, as writers like Pe\~{n}alver and Alexander sometimes do, is in my opinion inappropriate.

In fact, I am inclined to believe that it is only appropriate to equate community with the state in highly special situations, for instance if it can be shown that owners insulate themselves from, and engage in exploitative practices towards, other people and communities. Importantly, to argue that such a situation obtains requires a case to be made that is compelling both empirically and politically. In this regard, I believe theory alone has little to offer. This is a reason to conclude that the social function view of property in fact tells us very little about how widely the state should intervene in property in a given society. It allows us to recognize the {\it possibility} that the state may have to intervene on behalf of certain property values, say those that aim to protect communities. But this is no argument in favour of the position that the state should intervene more or less often than it currently does. Importantly, the theory can still serve a crucial purpose in that it allows us to reason more clearly about {\it when} it is appropriate for the state to intervene. For instance, the social function theory will later be used by me to single out economic development takings for special consideration. But this will not commit me to a particular normative stance on such takings.

More generally, it does not follow from the recognition that property structures are social in nature that {\it any} institution should actively seek to neither change nor protect those structures. The Humean position, namely that the existing distribution of property rights represents a socially emergent equilibrium, remains plausible. Moreover, the normative stance that this equilibrium is a {\it good} one (or at least as good as it gets) remains as contentious -- and as arguable -- as ever. For this reason, I believe it is appropriate to approach the social function theory as a descriptive theory in the first instance.

It is worth emphasizing that in taking this view I depart from the stance taken by many of the contemporary scholars who advocate on behalf of social function theories, including some that reject social democratic ideals. Hanoch Dagan, for instance, is a self-confessed liberal, but still explicitly and strongly argues for a social function understanding on the basis that it is morally superior. ``A theory of property that excludes social responsibility is unjust'', he writes, and goes on to argue that ``erasing the social responsibility of ownership would undermine both the freedom-enhancing pluralism and the individuality-enhancing multiplicity that is crucial to the liberal ideal of justice''.\footcite[1259]{dagan07}

If this is true, then it is certainly a persuasive argument for those who believe in a ``liberal idea of justice''. But for those who do not, or believe that property law is -- or should be -- largely agnostic on this point, a normative justification for the social function theory along these lines can only discourage them from adopting it. Such a reader would be understandably suspicious that the {\it content} of the social function theory -- as Dagan understands it -- is biased towards a liberal world view. The reader might agree that property continuously interacts with social structures, but reject the theory on the basis that it seems to carry with it a normative commitment to promote liberalism.

Danach stands out somewhat in the literature by focusing on {\it liberal} values, but as I have already indicated, he is not alone in proposing highly normative social function theories. Indeed, most contemporary scholars endorsing a social function view on property base themselves on highly value-laden assessments of property institutions.\footnote{See, e.g.,  \cite{alexander09,crawford11,davidson11,singer09,penalver09}.} While they provide interesting insights into the nature of property, I am struck by a feeling that these writers all tend to overstate the desirable normative implications of adopting a social function view. In addition, they appear to believe that accepting this view on property requires us to embrace certain values and reject others. Moreover, one is left with the impression that the social function theory has little to offer beyond the values with which it is imbued, which can in turn push the law in the direction that these writers deem desirable. 

I disagree that this is the case, at least for the social function theory as I understand it. Dagan's theory of property might be conducive to ``liberal justice'', but this is because it involves far more than what follows analytically from the proper recognition that  social functions should be considered relevant when adjudicating on the rights and obligations attached to property. Indeed, it is Dagan's clearly stated aim to propose a theory that promotes specific liberal values. ``There is room to allow for the virtue of social responsibility and solidarity'', he writes, continuing by suggesting that ``those who endorse these values should seek to incorporate them -- alongside and in perpetual tension with the value of individual liberty -- into our conception of private property''.\footcite[802]{dagan99} This view is reflected further in the concrete policy recommendations he makes, for instance in relation to the question of when it is appropriate to award less than ``full'' (market value) compensation for property following a taking.\footnote{See generally \cite{dagan14b}.}

My objection is not that his proposals are necessarily wrong, but that they need not be accepted in order to conclude that the social function of property should be given a more prominent place in property theory. Importantly, I think the focus on normative reasons threatens to overshadow the most straightforward reason for awarding social structures a more prominent place in the analysis, namely that they are almost always crucially important behind the scenes, even if they go unacknowledged. The social function theory, rather than being ``good, period'', as Danach suggests, is nothing more or less than accurate, irrespectively of one's ethical or political inclinations. As such, it provides the foundation for a debate where different values and norms can be presented in a way that is conducive to meaningful debate, on the basis of a minimal number of hidden assumptions and implied commitments. Thus, the first reason to accept the social function theory, for me, is epistemic rather than deontic.

That is not to say that normative theories should not be formulated on the basis of the social function theory, it merely means that I believe it is useful to maintain at least a theoretical division between the descriptive and normative aspects of such theorizing. I return to normative aspects in the next section, arguing that the commitment to ``human flourishing'' endorsed by Professor Alexander is a particularly well-argued norm that arises from value-based assessment of the social function of property. This, I argue, is in large part also due to the value-pluralism inherent in this idea, suggesting as a positive normative claim that our notions of property {\it should} allow a divergence of opinions and values to influence the law and its application in this area.

Moreover, I believe the history of the social function theory lends support to my claim that it is useful to emphasize that the theory gives us important descriptive insights that carry few normative commitments. This is particularly important, I believe, in a time when property scholars are showing greater willingness to explore new theoretical ideas. Theories can hardly be entirely value-neutral, nor is this a goal in itself. But in my opinion, a good theory is one that can serve as a common ground for further discussion based on disagreement about values and priorities. According to Kevin Gray, ``the stuff of modern property theory involves a consonance of entitlement, obligation and mutual respect''.\footcite[37]{gray11} It is important, I think, that the same measured perspective is reflexively applied towards theory itself, to diminish the worry that a broader theoretical outlook is the first step towards unchecked state power and rule by ``judicial philosopher-kings''.\footcite[944]{claeys09}

In the next subsection, I will argue in some more detail why such a cautious perspective is warranted, by considering how the Italian fascists appropriated the social function theory in 1930s. Building on the highly inspiring work of di Robilant, I will also briefly track how non-fascist property scholars opposed this development by focusing on value-pluralism, local self-governance and freedom.\footcite{robilant13} Importantly, these scholars embraced the social function theory as a common ground from which to launch a meaningful attack on more radical ideas, without alienating those with divergent views. Instead of clinging to the old-style liberal discourse that the fascists had either flatly rejected or completely subverted, many Italian non-fascists were willing to engage in a discourse revolving around property's social function, by spelling out a more measured set of ideas based on this premise. Crucially, this set the stage for a form of intellectual resistance that did not reject those aspects of fascism that had great appeal to the public and which arguably also reflected true insights into the unfairness and lack of sustainability of the established legal order.

\subsection{Rooting out fascism: {T}he tree of property}

While the social function theory makes intuitive sense, it is also highly abstract. Therefore, its exact content has been notoriously hard to pin down. This is recognized by contemporary scholars endorsing a normative view, who attempt to address this by proposing lists of values that should be taken into account while giving examples of how they should be used to inform the law in concrete areas or cases.\footnote{See, e.g., \cite{alexander14,alexander11,dagan07}.} Unsurprisingly, however, views soon diverge regarding the concrete import of a social function view on legal reasoning. Even so, the contemporary debate appears to be based on a common ground that is quite stable, also with respect to the overall notion of what good the theory can do. But as history shows, this state of affairs is by no means guaranteed. 

In a recent article, Anna di Robilant illustrates this point exceptionally well by tracking the history of social function theorizing in Italy during the fascist era. The fascist property scholars, she notes, were happy to embrace the social function theory, since it provided them with a conceptual starting point from which to develop their idea that rights and obligations in property should be made to answer to one core value: the interests of the state.\footnote{See \cite[908-909]{robilant13} (``Fascist property scholars had also appropriated the social function formula. For the Fascists, the social function of property meant the superior interest of the Fascist state.'').} This stance was as effective as it was oversimplified. As di Robilant notes, ``earlier writers had been hopelessly evasive about the meaning and content of the social element of property''.\footcite[909]{robilant13} Hence, the fascist approach filled a need for clarity about the implications of the main idea, which was by now attracting increasing support both from the public and the academic community. Established property doctrine, it was widely felt, was both ineffective and unfair to ordinary people. Rather than securing productivity and a livelihood for all, property was used mainly as an instrument for maintaining the privileged position of the elites. By promising to change this state of affairs, the fascists attracted many to their cause.

As di Robilant notes, supporters of the fascist idea of property made clear that ``social function meant the productive needs of the Fascist nation''.\footcite[909]{robilant13} But at the same time, they cleverly denied that there was a ``contradiction between subordinating individual property rights to the larger interest of the Fascist state and the liberal language of autonomy, personhood, and labor''.\footcite[900]{robilant13} In this way, fascist scholars could claim that fascist liberalism was true liberalism, thereby subverting the conceptual basis for the traditional idea of liberal justice.\footcite[900]{robilant13} In this situation, there was reason to suspect that clinging to liberal dogma would be a largely ineffective response. Moreover, it seemed undeniable that fascism's appeal was rooted in real concerns about the fairness and effectiveness of the liberal legal order. 

Hence, many non-fascists shunned away from uncritical defense of traditional liberalism. Instead, they agreed that property's social function should come into focus, but emphasized the plurality of values that could potentially inform this function, not the interests of the state. In addition, they also noted that property rights were invariably associated with {\it control} over resources, and that the social functions of property depended on the resources in question. To own property, they argued, provides individuals with a source of privacy, power and freedom that is, as a matter of fact, highly valued. It is valued, moreover, for its implications in a social context. To capture these insights, Italian scholars adopted the metaphor of a ``tree'', by describing the core social function of property as the trunk, while referring to the various resource-specific values attached to property as branches.\footcite[894-916]{robilant13} As di Robilant notes regarding these theorists:

\begin{quote}
The rise of Fascism, they realized, was the
consequence of the crisis of liberalism. It was the consequence of liberals' insensibility to new ideas about the proper balance between individual rights and the interest of the collectivity.\footcite[907]{robilant13}
\end{quote}

In light of this, the tree-theorists concluded that continued insistence on the protection of the autonomy of owners was not a viable response. Instead, they adopted a theory that ``acknowledges and foregrounds the social dimension of property'', but without committing themselves to fascist ideas about the supreme moral authority of the state.\footcite[907]{robilant13} The value of autonomy was in turn recast in terms of property's social function. Arguably, this served to make the case far more compelling. Protecting autonomy could be seen as an aspect of protecting property's freedom-enhancing function, both at the individual level and as a way of ensuring a right to self-governance and sustenance for families and local communities. This, moreover, could not easily be derided as tantamount to protecting unfair privilege and entitlement. In fact, the suggestion was made in an effort to protect democracy itself.

I believe the story of fascist appropriation of the social function theory provides further weight to my claim that it is sensible to  maintain a descriptive perspective on its core features. Indeed, the readiness with which the fascists embraced social function theorizing serves as a reminder that we cannot easily predict what normative values may come to be promoted on its basis. Hence, it is also call for continuous vigilance when it comes to normative assessment and debate. At the same time, we are reminded of the danger of attaching too much normative prestige to a theory that is abstract and open to various interpretations.

In particular, it seems to me that failure to recognize the descriptive nature of the core idea can lead to unrealistic expectations of what the social function theory actually provides. In addition, it will make it harder for the theory to gain acceptance as a conceptual common ground from which to depart when engaging in debate. Indeed, if no division is recognized between normative and descriptive aspects, the historical record would allow detractors to make a {\it prima facie} plausible attack on the social function theory by arguing that it is fascism in disguise, or that fascism, rather than liberal justice, is where we end up in practice should we chose to adopt it.\footnote{This would echo the claim already made by Claeys, that the theory (when coupled with virtue ethics) might become a slippery slope towards the kind of extremism and revolt against oppression that gave rise to the Rwanda genocide in the early 1990s \cite[926-927]{claeys09}.}

In response, one might retort that this is cherry picking the historical facts, or that the fascists misunderstood or perverted the theory. That is certainly plausible, but the point I am trying to make here is that this kind of debate is in itself unhelpful. Unless the social function theory is rendered neutral enough to be acceptable as the conceptual premise of debate, it is likely going to fail -- in a purely epistemic sense -- as a template for negotiating conflicts about property. Those who oppose the norms associated with the theory will oppose also the core descriptive content, if they feel that the latter commits them to the former. I believe that this, in turn, suggests that those advocating on behalf of the social function theory should take care to avoid rhetorical hubris. The main point to convey, I believe, is that the theory is in fact more accurate, in a purely epistemic sense, than other conceptualizations of property.

The story of the fascist appropriation of the social function theory also points to the danger that often attach to abstract theories with normative implications: That they allow us to opportunistically recast whatever values we wish to promote, by providing qualifications for them in abstract terms that are hard to refute. The fascists did this, and the non-fascists responded. Hence, in the end one could do little more than hope that the fascists' vision of their state as an ``ethical state'' that ``every man holds in his heart'' would eventually prove less attractive then the promise of self-governing communities bustling with diversity in life and character.

\subsection{Towards a normative stance}

The social function theory can facilitate a new kind of normative reasoning, arising from how the theory allows us to recognize more subtle distinctions between different kinds of property and different kinds of circumstances. For instance, staunch entitlements-based approach to autonomy will leave us with little room to differentiate between the protection of investment property and the protection of a home, unless such a distinction is explicitly provided for in the law. But a social function approach compels us to notice the difference and to acknowledge that it might be legally, as well as ethically, relevant. Hence, if we seek to argue for protection of investment property, we must in principle be prepared to face counter-arguments that revolve around particulars of the investor's role in society and his relationship to the community of people that are affected by how he manages his property. Similarly, if someone argues against protecting home ownership, we can respond by drawing on additional arguments based on the importance of the home both to the owner, her family and friends, and her community. Under the social function theory, it becomes generally relevant to address how a home creates a sense of belonging and provides a basis for membership in social structures.

I believe normative assessments should aim to be as concrete as possible. That said, I still think it is worthwhile to provide more abstract forms of expression for core values, to clarify the ethical premises that provide the basis for concrete value-based conclusions. To me, therefore, normative theories should aim to be meta-ethical, not just ethical. They should provide a vocabulary and a conceptual framework tailored to advancing one's values. However, they should recognize that the ultimate expression of those values is given in relation to concrete facts. This, I believe, is prudent in light of how abstract ethical assertions are necessarily imprecise, and run the risk of being distorted or exaggerated, particularly as they gain influence.

Invariably, the most accurate information regarding the values I rely on when assessing cases will be conveyed by my assessment of the cases, not by my theorizing. On a deeper level, I am inclined to believe that value-systems are more or less unique to individuals, so that ethical theories are helpful primarily in that they provide an introduction to keywords and important lines of argument that will recur in different forms. As such, they enhance understanding, making it easier to communicate ideas and opinions in such a way that potential respondents are likely to enjoy a somewhat less inaccurate impression of what they are responding to. 

In short, I believe that ethics make moral judgments communicable, allowing new ideas to be created in the minds of individuals. It should come as no surprise to the reader, therefore, that I believe in ethical men and women, but not in ``ethical Man'' or -- God forbid --  the ``ethical State''. Luckily, I find some support for this world view in recent theories that have been proposed as normative extensions of the social function theory of property. These are the subject of my next section.

\section{Human flourishing}\label{sec:hf}

Taking the social function theory seriously forces us to recognize that a person's relation to property can be partly constitutive of that person's social and personal identity, including both its political and economic components. Hence, property influences people's preferences, as well as what paths lie open to them when they consider their life choices.\footnote{See generally \cite{alexander09}.} This effect is not limited to the owner, it comes into play for anyone who is socially or economically connected to property in some way. The life-significance of property might be clearly felt by a potentially large group of non-owners as well.\footcite[128-129]{alexander09d} The importance of property is obviously reduced if we move away from it in terms of social or economic distance. Hence, in many cases, property will be most important to its owner, simply because she is closest to it. This is not always the case, however, especially not if property rights are unevenly distributed, or in the possession of disinterested or negligent owners. Moreover, as mathematically oriented sociologists take pride in pointing out, social connections are ubiquitous  and the world is often smaller than it seems.\footnote{See generally \cite{schnettler09}.}

Hence, there is certainly potential for making wide-reaching socio-normative claims on the basis of this perspective on the meaning and content of property. But which such claims {\it should} we be making? According to some, we should adjust our moral compass by looking to the overriding norm of {\it human flourishing} as a guiding principle of property law. Colin Crawford, for example, explicitly argues that the social function theory of property should ``secure the goal of human flourishing for all citizens within any state''.\footcite[1089]{crawford11} In a recent article, Alexander goes even further, by declaring that human flourishing is the ``moral foundation of private property''.\footcite[1261]{alexander14} 

As I have already explained, I believe -- in contrast to both Crawford and Alexander -- that it is useful to decouple such normative claims from the descriptive core of the social function theory.\footnote{Crawford comments that the social function theory on its own  ``is not self-defining and invites many interpretations'', see \cite[1089]{crawford11}. The normative theory he proposes is clearly aimed at filing this perceived gap, by pinning down normative commitments that Crawford believes are intrinsic to the theory. However, as I have already argued, I reject this approach, since it unwisely downplays the fact that the social function theory can serve as a common ground among commentators with widely divergent normative views. Indeed, Crawford himself refers unfavorably to a writer who addresses the social function theory, but who, according to Crawford, proposes that ``property's social function is best served by focusing on overall economic production and efficiency in a given society, allowing the market's invisible hand to work its magic'', \cite[see][1089]{crawford11}. Against Crawford, I would argue that it is better to counter such a claim by arguing why it is normatively wrong than by suggesting that people with such values should be discouraged from attempting to argue for them on the basis of a social function understanding of property. Rather, by encouraging such an argument it should become easier to make the case why the values promoted are ultimately undesirable. This, at least, should follow if Crawford is otherwise largely correct (as I think he might be).} I therefore refer to the more distinctly normative aspects of their work as human flourishing theorizing. 

Human flourishing has a good ring to it, but what does it mean? According to Alexander, several values are implicated, both public and private.\footnote{See generally \cite{alexander14,alexander11}.} Importantly, Alexander stresses that human flourishing is {\it value pluralistic}.\footnote{\cite[750-751]{alexander09}.} There is not one core value that always guarantees a rewarding life. To flourish means to negotiate a range of different impulses, both internal and external. Importantly, these all act in a social context which influences their meaning and impact.\footcite[1035-1052]{alexander11}

For the family of a homeowner, the value of the ownership tends to be great; a home is a home for any non-owner living there, just as much as it is a home for the owner. This, in turn, creates both commitments and opportunities for the owner, which may or may not find recognition in the law and our legal reasoning. Regardless of this, it certainly carries significant importance both to her life and the lives of those that depend on her. If property is rented out as a home to someone else, the importance of ownership may be {\it greater} to a non-owner. Indeed, assuming a society where tenancy is a well-functioning social institution, the continuation of the established property pattern might well be of greater importance to the tenant than it is to the owner.

The effect on non-owners can also be restrictive in socially desirable ways. If an apartment has an owner, it discourages squatters, for instance. Moreover, this effect clearly depends also on {\it who} the owner is and the choices she makes in managing her property. If the owner lives in the apartment, squatting is hopefully not going to be an issue. But even the owner of an unoccupied apartment can discourage squatting by managing her property well. However, if owners mismanage their apartments, for instance because they seek to obtain demolition licenses, squatters can take opportunity of this. The risk, of course, increases if housing cannot be afforded by a large number of society's members. In this case, it is natural to argue that something is amiss with the prevailing property structures.

Now, the social function theory of property can also come into play, since it allows us to attach significance to this also when discussing the property rights of individual owners.  In particular, we are not compelled to pretend as though possible failures of property as a social institution are irrelevant when considering rights and responsibilities attached to it. As a matter of fact, they are not; actual squatters clearly affect the owner, influencing both the meaning and the value of her property, both to her, potential buyers, the local government, the state, and other interested parties. Even the mere {\it risk} of squatting can play such a role. But a property theory which does not recognize the social function of property might not allow us to take this into due regard. As long as the standard expectation of an owner is to be able to enjoy her apartment free of squatters, an entitlements-based view on property could easily force us into denial regarding actual (risk of) squatters.

In particular, we would be led to consider squatting as an interference with the owner's rights which the state can not, on pain of disrespecting property, recognize as a legitimate response to mismanagement and imbalances in the property structure. The normative significance of real life -- where squatting often happens due to badly managed property -- is discounted  because our conceptual glasses block it out. Then, the almost unavoidable consequence is that the state also recognizes a {\it positive} obligation to forbid squatting, and to forcibly remove squatters on behalf of owners. Under a narrow entitlements-based conception, this is the natural outcome, and must be classified as an act of protecting private property. Hence, under classical liberal values, it also becomes {\it good}. Here, however, the social function theory permits us to take a highly divergent view, to carry forward different value-judgments.

In particular, if squatting is recognized as creating new interests and obligations attaching to the property, it may now be argued that  it is the use of state power to evict that is the most severe act of interference. Not only interference in whatever housing rights the squatters may have, but in fact also as an interference in {\it property}. Hence, such state action might itself be morally suspect and held to be in need of further justification. In the Netherlands, the Supreme Court adopted a line of reasoning reflecting these insights, when it held that the right not to be disturbed in one's home life also applied to squatters. Hence, the property owner could not forcibly evict people who had taken up residence in her property.\footnote{See NJ 1971/38. The court held that the lower court had erred in taking it proven that the ``house in the original charge was ``in use'' by the owner of this house'', as required by the statute under which the squatters were tried. Instead, the Supreme Court held that ``art.138, in so far as it mentions houses, is specifically aimed at protecting home rights, in connection with which the words ``in use'' (differently than the court judged) can only be understood as ``actually in use as a house'' , as in accordance with ordinary use of language''. The upshot was that it was the squatters, not the owner, who enjoyed protection under the statute. In terms of the bundle theory, a right thought to be in the owner's bundle was deemed to actually belong to the bundle of the squatters, as this corresponded better to the circumstances of the case and the purposes meant to be served by the statute in question.}

In South Africa, a somewhat similar line of reasoning was adopted in the recent case of {\it Modderklip East Squatters v. Modderklip Boerdery (Pty) Ltd}, analysed in depth by Alexander and Pen\~{n}alver.\footcite[154-160]{alexander11} The case dealt with squatting on a massive scale: Some 400 people had taken up residence on land owned by Modderklip Farm, apparently under the belief that it belonged to the city of Johannesburg. The owner attempted to have them evicted and obtained an eviction order, but the local authorities refused to implement it. Eventually, the settlement grew to 40 000 people and Modderklip Farm complained that its constitutional property rights had not been respected.

The Supreme Court of Appeal concluded that Modderklip's property rights had indeed been violated, but noted that so had the rights of the squatters, since the state had failed to provide them with adequate housing.\footnote{See \cite{modderklip04}.} However, they upheld the eviction order and granted Modderklip Farm compensation for the state's failure to implement it. The Constitutional Court, on the other hand, while agreeing that the eviction order was valid, concluded that as long as the state failed in its obligations towards the squatters, the order should not be implemented.\footcite{modderklip05} The eviction of the squatters, in particular, was made contingent upon an adequate plan for relocation. In the meantime, Modderklip would receive monetary compensation, from the state rather than the squatters. In this way, the Court recognized the social function of property; they refused to give full effect to Modderklip's property rights as long as that meant putting other rights in jeopardy. The fact that the squatters had no place to go, in particular, was allowed to influence the content of Modderklip's right, making it impermissible to implement a standing eviction order.

It is possible to cast this outcome as an interference in property rights that was regarded as acceptable in the public interest. However, the reconceptualization in terms of property itself having a social function appears highly attractive. Moreover, it is also consistent with the South African constitution, which also focuses on property's social dimension.\footnote{See section 25 of the Constitution of the Republic of South Africa, Act 108 of 1996.} Thinking about cases like {\it Modderklip} in terms of property's social function allows us to remove the state as an intermediary between the owner and the other interested parties, in this case the squatters. As argued by Alexander and Pe\~{n}alver, it becomes possible to think of the Court as adjudging based on Modderklip's own responsibility, as an owner, towards other members of the community that have an interest in the property.\footnote{\cite[157]{alexander11} (``The courts' unwillingness to ratify Modderklip's desire to remove the squatters from its land illustrates the courts' willingness to take seriously the obligations of owners, not only as they concern owners' direct relationship with the state but also in relation to the needs of other citizens'').}

On this basis, it becomes easier to conclude that it is permissible for courts to take the social context into account even in the absence of any specific state action or legislation to indicate that this should be done, or that the public interest is at stake. Indeed, one of the problems in {\it Modderklip} was that the state had failed also in its responsibility towards the squatters. Moreover, while the local sheriff had refused to implement it, an eviction order had in fact been granted. Hence, thinking of the case as interference in the public interest becomes difficult.

More importantly, by taking into account the social function of property, it becomes possible to argue for the outcome in Modderklip positively on the basis of property values. In this way, property is no longer seen to stand in the way of justice in cases such as this. We need not ``interfere'' with rights to secure an appropriate outcome, we only need to apply property law. As Alexander puts it in another recent article: 

\begin{quote} The values that are
part of property's public dimension in many instances are necessary
to support, facilitate, and enable property's private ends.
Hence, any account of public and private values that depicts them as categorically
separate is grossly misleading. One important consequence of this
insight is that many legal disputes that appear to pose a conflict between
the private and public spheres or that seemingly
require the involvement of public law can and
should, in fact, be resolved on the basis of private law -- the law
of property alone.\footcite[1295-1296]{alexander14} \end{quote}

Protection of property, when property is understood in this way, becomes a potential source of justice, also for squatters. The basic values attached to property -- freedom, liberty, autonomy -- have not really changed, but are applied in a new way. In particular, they no longer only apply to the owner's interest in property, but also to that of other individuals closely connected to it. This normative turn, I argue, will potentially strengthen the institution of property itself, while also serving to loosen the compulsiveness of the idea that the ultimate expression of the public interest is found in the actions taken by the state. It suggests rather the view that the public interest manifests wherever the public may reside, including in property. This conclusion requires taking a normative stance, but a minimal one; we merely extend the scope of values traditionally attached to property.\footnote{Arguably, cases such as {\it Modderklip} might be taken to suggest that the social function theory, as soon as it is applied for the purposes of normative assessment, will systematically guide us to conclude that owners are not entitled to as many benefits as would otherwise follow from their property rights. It is fortunate, therefore, that the entire remainder of the thesis will focus on economic development takings, where it will typically appear more natural to conclude the opposite. In these cases, on a common- sense understanding of justice, applying the social function theory will allow us to recognize a sense in which owners should receive {\it increased} protection and more benefits, as a consequence of how such interferences can prove particularly damaging, both to the owner and to the social fabric of democracy.} 

That said, in the case of {\it Modderklip} the court was clearly faced with a value conflict that it is hard to resolve by looking to traditional liberal values. If these apply equally to the squatters, we are left with deadlock rather than resolution. Indeed, this was also reflected in the outcome of the case, which did not resolve the matter, but merely concluded that the state had failed in its obligations towards both of the parties. What should the solution be in the end? Should the squatters be allowed to stay, following condemnation of Modderklip's land, or should alternative housing be provided so that the eviction order can be carried out? The answer requires us to resolve a normative conflict, and how to do so might not be obvious. Moreover, value pluralism suggests that we must be prepared to engage with multiple ways of looking at the matter. In the interest of stability of property as an institution, allowing the squatters to succeed in establishing lasting title to the land might be considered unwise. Against this pragmatic and largely technical value, one would have to consider the values of community and belonging that now attach the squatters to their new homes. These two values are largely incommensurable, and it is not clear how to choose between them.

Still, Alexander maintains that human flourishing provides an ``objective'' standard on which to approach dilemmas such as these. Moreover, he ``rejects the view that what is good or valuable for a person is determined entirely by that person's own evaluation of the matter''.\footcite[1263]{alexander14} Some things are good for people, Alexander argues, irrespectively of whether or not people know so themselves. Hence, it may perhaps be argued that what is truly good for Modderklip is to come to an arrangement with the squatters and the state, to resolve the problem amicably. Moreover, failure to do so might entitle the state to take action that would otherwise seem to undermine the stability of property. This, then, would be partly due to this being conducive also to the flourishing of the people behind Modderklip, not only the squatters.

That, clearly, might be derided as an overly intrusive and moralistic way to approach property law. More generally, as Alexander notes, the exact content of goodness is ``necessarily contestable''. It consists of a list of different values which are all open to dispute, both as to their relevance and their precise meaning.\footcite[1263]{alexander14} Alexander goes on to list some key values that he believes are central, but the list is not meant to be exhaustive.\footcite[1764-1776]{alexander14}

Among the key values that Alexander discusses, we find many core private values that are commonly seen as important goals for the institution of property. This includes values such as autonomy and self-determination, both of which will feature heavily later in this thesis. However, Alexander also considers several public values, such as equality, inclusiveness and community. These too will be important later, as I will draw on them in my own normative analysis of economic development takings. I will be particularly concerned with the value of {\it participation}, understood, following Alexander, in terms of its broad social function.\footcite[1275-1276]{alexander14}

In my view, this value is closely related to the value of democracy. Participation in local decision-making processes is the root which enables democracy to come to fruition at the regional and national level. Moreover, participation is a value that will give me occasion to make particular policy suggestions regarding the correct way to approach economic development takings. Devoting some time to discussing this value in the abstract will therefore be helpful.

Alexander traces the value of participation back to Aristotle and the republican tradition. He notes, however, that this tradition involves a notion of participation that is somewhat narrowly drawn. For thinkers in the republican tradition, participation tends to mean public participation, meaning people's engagement with the formal affairs of the polity.\footcite[1275]{alexander14} For Alexander, participation has a broader meaning, involving also the value of being included in a community. He writes:

\begin{quote}
We can understand participation more broadly as an aspect of inclusion. In this sense participation means belonging or membership, in a robust respect. Whether or not one actively participates in the formal affairs of the polity, one nevertheless participates in the life of the community if one experiences a sense of belonging as a member of that community.\footcite[1275]{alexander14}
\end{quote}

Importantly, participation in a community can have a crucial influence also on people's preferences and desires. In this way, it is also invariably relevant -- behind the scenes -- to any assessment of property that focuses on welfare, utility or public participation in the classical sense. As Alexander and Pe\~{n}alver put it, drawing on the work of Amartya Sen and Martha Nussbaum:\footcite{sen84,sen85,sen99,nussbaum00,nussbaum02}
\begin{quote}
The communities in which we find ourselves play crucial roles in the formation of our preferences, the extent of our expectations and the scope of our aspirations.\footcite[140]{alexander09}
\end{quote}
Therefore, for anyone adhering to welfarism, rational choice theory, utilitarianism or the like, neglecting the importance of community is not only normatively undesirable, it is also unjustified in an epistemic sense. In particular, it should be recognized as a descriptive fact that community is highly relevant to {\it any} normative theory that attempts to take into account the preferences and desires of individuals.\footnote{Again, I think Alexander and other theorists attempting to incorporate such ideas in property law could benefit from making this descriptive point separately, so as to enable it to be considered in isolation from the more contentious normative arguments they construct on the basis of it.} But Alexander and Pe\~{n}alver go further, by arguing that participation in a community should also be seen as an independent, irreducibly social, value, not merely as a determinant of individual preferences and a precondition for rational choice. They write:

\begin{quote}
Beyond nurturing the individual capabilities necessary for flourishing, communities of all varieties serve another, equally important function. Community is necessary to create and foster a certain sort of society, one that is characterized above all by just social relations within it. By ``just social relations'', we mean a society in which individuals can interact with each other in a manner consistent with norms of equality, dignity, respect, and justice as well as freedom and autonomy. Communities foster just relations with societies by shaping social norms, not simply individual interests.\footcite[140]{alexander09}
\end{quote}

This, I think, is a crucial aspect of participation. Moreover, it is one that it is hard, if at all possible, to incorporate in theories that take  preferences and other attributes of individuals as the basis upon which to reason about property. For instance, if people in a community come under pressure to sell their homes to a large commercial company that wishes to raze them in order to construct a shopping mall, it may be appropriate to consider this as an unjustifiable attack on their property rights. Importantly, this may be so {\it irrespectively} of what the individual owners themselves think they should do. If they are offered generous financial compensations for their homes, or are threatened by eminent domain, economic incentives might trump the value of social inclusion and participation for all or a majority of these owners. As a consequence, the community might decide to sell.  

Even so, in light of the value of community, it would be in order for planning authorities, maybe even the judiciary, to view such an  agreement as an {\it attack on their property}. It is clear, in particular, that by the sale of the land, the ``just social relations'' inhering in the community will be destroyed. The members of the community -- including all the non-owners -- will lose their ability to participate in those relations. More concretely, the nature of the property rights that once contributed to sustaining ``just relations'' will now be transformed into property rights that serve different purposes. This includes aiding the concentration of power and wealth in the hands of commercially powerful actors. Such a change in the social function of property might have to be regarded -- objectively speaking -- as a threat to participation, community and democracy. Hence, on the human flourishing theory, it is also a threat to property. Our property institutions, therefore, should protect against it.

To demonstrate the general significance of such a line of normative reasoning, it is illustrative to mention a scenario -- not directly implicating property -- that is currently beginning to attract much attention in legal scholarship. This scenario arises in relation to the right to {\it privacy}. This right, of course, is increasingly perceived to be coming under threat in the information age. Crucially, it is beginning to become clear to legal theorists that viewing privacy merely as a private right is not going to provide a sustainable template for dealing with this challenge.\footnote{See generally \cite{schafer14}.} It seems, in particular, that people are simply too willing to give it up. This, in turn, contributes to the formation of potentially harmful social structures on the web. In particular, the lack of privacy becomes an impediment to dignity, freedom and respect in web societies. In this way, both individuals and society as a whole will eventually suffer, although this truth is not reflected in our individual preferences. Hence, it has been proposed that privacy should be considered also as a {\it common good}, so that protecting the privacy of individuals, in some cases, is an imperative irrespectively of what these individuals themselves desire and prefer. Privacy, in this way, becomes also an obligation, mirroring the similar phenomenon that we have observed with respect to the right to property.

There is a subtle issue that arises on the basis of this kind of normative reasoning about individual rights. Is it appropriate, in particular, to still think of such reasoning -- and the obligations it gives rise to -- as an aspect of protecting individuals? Is it not more accurate to say that this is an {\it interference} with individual rights, undertaken to further the public interest? Indeed, when the individual himself does not want his property or privacy to be ``protected'', is it not somewhat perverse to insists that this is what is happening? 

I am inclined to answer in the negative. In my opinion, we are still talking about protecting individual rights, even when this means imposing protections on people that they themselves do not want. Undoubtedly, this is {\it also} an interference in their rights, but just as different rights of different people can sometimes come into conflict, I am inclined to think that the same right, for the same person, can sometimes come into conflict with itself. This happens, in particular, when it is not possible to simultaneously protect all those functions that this right seeks to promote. 

For instance, if someone protests a taking on environmental grounds and also rejects financial compensation as immoral, the courts should still award just compensation for the land, if they find that the taking is valid. If the owner wishes, he can purge himself by making a donation to charity. Similarly, if someone attempts to commit suicide, the health services are still obliged to help, even against the patients wishes. This remains the case, moreover, even though suicide is no longer considered a criminal offense in the public interest. 

Protecting individuals against their will is condescending, no doubt, but it is still different, and often preferable, from subordinating their interests to that of the general public. If the justification for an act of interference is a vague proclamation of the ``public interest'', the individual is marginalized from the very start. A balancing act might be required, but this renders the individual relevant only to one side of the equation. On the other hand, if the act of interference is simultaneously rendered as protection, enforcement of an obligation, or a measure to enable participation, the individual occupies center stage. In so far as the public interests triumph, it is not because the individual loses, but because the public is deemed to know best how to secure the goal of human flourishing, both for the individual herself and other members of the social structures that surrounds her.

For instance, external interests of both a private and a public nature can dictate that owners should avoid becoming a nuisance to their neighbors. But under a human flourishing theory, we are also able to portray this as a case of protecting the individual's membership in the community. The public does not ``side with the neighbors'', but undertakes measures to protect the relationship between the owner and his fellows. In my opinion, a conceptual approach to property law that makes this portrayal plausible is highly desirable. 

For a second example, consider situations when environmental concerns suggest imposing restrictions on what an owner is permitted to do with his land. This too can be rendered as an act of protecting property. But doing so requires the regulatory body to relate the interference positively to the individual's interests and obligations, to ensure that they avoid adopting a narrative where the regulation is rendered as an act of enforcing the will of unnamed others against the will of specific owners. In this way, public values and the public interest can be given considerable weight, but will have to be rendered less abstract. In particular, these interests must be related concretely to the social functions of the rights protecting the individuals interfered with. The baseline for assessment remains actual persons and their well-being, not some abstract ideal of ``goodness''. Moreover, implementation of the collective will becomes a guide towards human flourishing for a society of individuals, not a goal in itself.

An individual might well be offended if the state adopts this narrative and implements behavioral restrictions by declaring ``it's for your own good''. But, I would argue, that is exactly as it should be. Any restriction of individual freedom is an offense, but one that is sometimes appropriate. If this is conveyed to people with a marginalizing ``your interests are not as important as ours'', the response might well be silence. But beneath the silence we may find disinterested apathy, or worse: contempt and despair. The interference is no longer an insult, but this is not because it is any more convincing to proclaim that interference is ``necessary for the greater good''. Rather, the interference is no longer an insult because it fails to properly engage the individual at all. The role of the person interfered with becomes passive -- she becomes an obstacle that needs to be removed. If such a dynamic of governance develops, the individual might take from this the lesson that she is unimportant in the greater scheme of things, that her interests are subordinate to those of ``the others'', and that her voice is not meant to be heard.

This is normatively undesirable. It represents a situation when the social effect of interference might become detrimental to society itself, particularly to the institution of democracy. It damages its roots, namely the ``just social structures'' that Alexander identifies as being at the core of the human flourishing theory. A better alternative, then, is to interfere in a way that constructively targets the individual, aiming to protect her by enabling her -- and compelling her -- to protect others and partake in social and political life. This can then become interference aimed at bringing the individual into the fold, making her play her part, by raising her to fruitful citizenship. Such a paternal (or maternal) state is one that cares, but one that may also be overprotective, unfair, or plain stupid. Hence, it becomes natural to resist and to revolt, but not without also carrying forward care and love for the social, political and legal structures within which this agency is (hopefully) permitted to take place.

The upshot, I believe, is that condescension in property law can be a good thing. To conceptualize an act of restriction as a means to empower the persons restricted is something they might well find offensive, but it also renders interference more meaningful to them. It provides both a reason to take a more active role in relation to the interfering power, and a possible cause for constructive resistance. Importantly, it does not force the conclusion that the public resides behind closed doors, disinterested in what the affected individual have to offer. Instead, it is an approach that encourages a response, by focusing always on the persons interfered with, whenever interference is deemed necessary. This is the vision of a bottom-up, rather than a top-down, approach to imposing the collective will on individuals. I believe it has merit. 

It will remain in the background as I now move on to apply the theories discussed in this and preceding sections. In the next section,  I return to the issue that will remain in focus for the remainder of this thesis. First, I will introduce economic development takings by considering the seminal case of {\it Kelo v City of New London}\footcite{kelo05}, which brought this category to prominence in the US discourse on property law. Then I will assess the unique aspects of such takings against the social function theory, to provide an argument that the category has significance for legal reasoning in takings law, as well as with respect to property as a constitutionally protected human right. Finally, I will provide an abstract presentation of the values that I believe should be considered important when normatively assessing the law in this area. In doing so, I will draw on the human flourishing theory, setting out the main values that will inform the concrete policy assessments I provide later. 

\section{Economic development takings}\label{sec:edt}

Constitutional property rules in many jurisdictions indicate, with varying degrees of clarity, that eminent domain should only be used to take property either for ``public use', in the ``public interest'', or for a ``public purpose''. Such a restriction can be regarded as an unwritten rule of constitutional law, as in the UK, or it can be explicitly stated, as in the basic law of Germany.\footnote{See Chapter \ref{chap:2}. Section \ref{sec:contrast} below.} In some jurisdictions, for instance in the US and in Norway, explicit property clauses exist, but are not formulated clearly.\footnote{See Chapter \ref{chap:2}, Section \ref{sec:us} and Chapter 3, Section \ref{sec:norexp} below.}

Both the Norwegian and the US property clauses appear to refer to public use only as a precondition for the duty to pay compensation. However, they are also universally read as expressing the {\it presupposition} that the power of eminent domain is only to be used in the public interest.\footnote{In the literature, it is rare to even note that a different interpretation is linguistically possible. But see \cite[205]{berger78}.} Indeed, in cases when one might say that private property is ``taken'' for a non-public use without compensation, for instance in a divorce settlement, it is not commonly regarded as an exercise of eminent domain. Rather, it is justified by making reference to a different category of rules, meant to ensure enforcement of obligations that arise between private parties independently of the state's power to single out and compulsorily acquire specific properties.

The exact boundary between eminent domain and other forms of state interference in property may not always be clear, but I will not worry too much about it in this thesis. I note, moreover, that most, if not all, legal scholars seem to agree that the power of eminent domain is meant to be exercised in the public interest. However, differences of opinion emerge when we turn to the question of whether the presupposed public use or public interest in the property taken serves also to restrict the power to take. In the US, most scholars agree that some restriction is intended, but there is great disagreement about its extent.\footcite[205]{berger78} In Norway, on the other hand, a consensus has developed that the public use limitation is so wide that it hardly amounts to a restriction at all.\footnote{See, e.g., \cite[368]{aall10}.} Moreover, the courts defer almost completely to the assessments made by the executive branch regarding the purposes that may be used to justify a taking.\footcite[368]{aall10}

Some US scholars adopt a similar stance, but others argue that the public use presupposition should be read as a strict requirement, forbidding the use of eminent domain unless the public will make actual use of the property that is taken.\footnote{Compare \cite{bell06,bell09,claeys04,sandefur06}.} Most scholars fall in between these two extremes. They regard the public use restriction as an important, practically relevant, limitation, but they also emphasize that courts should normally defer to the legislature's assessment of what counts as a public use.\footnote{See, e.g., \cite{merrill86,alexander05}. The fact that US jurists usually stress deference to the legislature, not the executive branch, should be noted as a further contrast with Norway.}

As I discuss in more depth in Chapter \ref{chap:2}, Section \ref{sec:hop}, the debate in the US has its roots in case law developed by state courts -- the federal property clause was for a long time not enforced against states. This has changed, however, and today the Supreme Court has a leading role also in this area of US law. It has developed a largely deferential doctrine, resembling the understanding of the public use limitation under Norwegian law.\footnote{See \cite{berman54,midkiff84,kelo05}.} The difference is that in the US, cases raising the issue  still regularly arise, and still prove controversial. The most important such case in recent times was {\it Kelo}, decided by the Supreme Court in 2005.\footnote{kelo05} This case saw the public use question reach new heights of controversy in the US.\footnote{See, e.g., \cite{somin09}.}

{\it Kelo} centered around the legitimacy of taking property to implement a redevelopment plan that involved construction of research facilities for the drug company Pfizer. The home of Suzanne Kelo stood in the way of this plan, and the city decided to use the power of eminent domain to condemn it. Kelo protested, arguing that making room for a private research facility was not a permissible  `public use''. She was represented by the libertarian legal firm {\it Institute for Justice}, which had previously succeeded in overturning similar instances of eminent domain at the state level.\footnote{See \url{https://www.ij.org/cases/privateproperty}.} Kelo lost the case before the state courts, but the Supreme Court decided to take it on, and they looked at it in great detail.

The precedent set by earlier federal cases was clear: As long as the decision to condemn was ``rationally related to a conceivable public purpose'', it was to be regarded as consistent with the public use restriction.\footcite[241]{midkiff84} Moreover, the role of the judiciary in determining whether a taking was for a public purpose was regarded as ``extremely narrow''.\footcite[32]{berman54} It had even been held that deference to the legislature's public use determination was required ``unless the use be palpably without reasonable foundation'' or involved an ``impossibility''.\footnote{See \cite[66]{dominion25}; \cite[680]{gettysburg96}.}

This understanding had also been reflected in the outcome of concrete cases resembling the situation in {\it Kelo}: In {\it Hawaii}, the Supreme Court had upheld a taking that would benefit private parties, with no direct benefit to the public.\footnote{\cite{midkiff84}. For a more detailed discussion, see Chapter \ref{chap:2}, Section \ref{sec:hop} below.} In {\it Berman}, it had upheld a taking for economic redevelopment of a blighted area, even though the property taken was not itself blighted.\footnote{\cite{berman54}. For a more detailed discussion, see Chapter \ref{chap:2}, Section \ref{sec:hop}.} But in the case of {\it Kelo}, the court hesitated.

Part of the reason was no doubt that takings similar to {\it Kelo} had been heavily criticized at state level, with an impression taking hold across the US that eminent domain ``abuse'' was becoming a real problem.\footnote{See, e.g., \cite[667-669]{sandefur05}.} A symbolic case that had contributed to this worry was the infamous \textcite{poletown81}. In this case, General Motors had been allowed to raze a town to build a car factory, a decision that provoked outrage across the political spectrum.\footnote{See generally \cite{sandefur05}.} The case was similar to {\it Kelo} in that the taker was a powerful commercial actor who wanted to take homes. This, in particular, served to set the case apart from  {\it Hawaii}, which involved a taking in favor of tenants, and to some extent also {\it Berman}, which involved a taking of businesses (and homes) in the interest of combating blight. Moreover, the Michigan Supreme Court had recently decided to overturn {\it Poletown} in the case of \textcite{wayne04}. Hence, it seemed that the time had come for the Supreme Court to reexamine the public use questions.\footnote{See, e.g., \cite{sandefur05,claeys04}.}

Eventually, in a 5-4 vote, the court decided to apply existing precedent and held against Suzanne Kelo. The majority also made clear that economic development takings were indeed permitted under the public use restriction, also when the public benefit was indirect and a private company would benefit commercially.\footcite[469-470]{kelo05} The backlash of this decision was severe. According to Ilya Somin, the case ranks among the most disliked decision that the Court has ever made.\footcite[2]{somin11} Some 80 - 90 \% of the US public expressed great disapproval, with critical voices coming from across the political spectrum\footcite[2108-2110]{somin09} Why did the case prove so controversial? No doubt, the discontent with the decision was fueled in large part by the fact that it was seen as a case of the government siding with the rich and powerful, against ordinary people.\footnote{\cite[630-634]{baron07}} Indeed, the party that appeared to benefit the most from the taking was Pfizer -- a multi-billion dollar company -- while Suzanne Kelo, who stood to lose, was a middle class homeowner. In this context, the taking of Kelo's home seemed morally suspect, an act of favoritism showing disregard for less influential members of society.\footnote{See, e.g., \cite{underkuffler06}.}

In addition, it is worth noting that many commentators conceptualized the {\it Kelo} case by thinking of it as belonging to a special category, by describing it as an economic development taking, a {\it taking for profit}, or, more bluntly, a case of {\it Robin Hood in reverse}.\footcite{somin05} Categories such as these had no clear basis in the property discourse before {\it Kelo}. Indeed, in terms of established legal doctrine, it would be more appropriate to say that the case revolved entirely around the notion of ``public use''. 

However, when we consider the most common reasons given for condemning the outcome in {\it Kelo}, we readily grasp why critics felt it was natural to classify the case along an additional dimension. A survey of the literature shows that many critical voices made use of a combination of substantive and procedural arguments to  paint a bleak picture of the {\it context} surrounding the decision to take Kelo's home. Important concrete factors that critics tend to stress include the imbalance of power between the commercial company and the owner, the incommensurable nature of the opposing interests, the lack of regard for the owner displayed by the decision makers, the close relationship between the company and the government, and the feeling that the public benefit -- while perhaps not insignificant -- was made conditional on, and rendered subservient to, the commercial benefit that would be bestowed on the commercial beneficiary.\footnote{See, for instance, \cite{underkuffler06,somin07,sandefur06,cohen06,hafetz09,hudson10}.}  This dynamic, in which public bodies no longer seem to be leading and pushing the process forward, but are also -- to quite some extent -- being led and being pushed, is regarded as particularly suspicious. This, in turn, is derided as a perversion of legitimate decision-making, used to argue more broadly that economic development takings such as {\it Kelo} suffer from what I will refer to here as a {\it democratic deficit}.

From a theoretical point of view, I take all of this to suggest that many critics of {\it Kelo} effectively adopted a social function view on property, by paying close attention to the wider social and political context of the taking.\footnote{For a particularly clear example of this, see \cite{underkuffler06}.} Importantly, if we now turn to the social function theory of property, we are placed in a position to engage more actively with this form of reasoning, as an integrated part of our assessment of the law. This may then in turn give us cues as to how we should reason -- within the law -- to justify a departure from the course laid down by previous cases on the ``public use'' requirement, where such a perspective was not adopted. Indeed, it seems to me that this is exactly what the minority of the Supreme Court did, particularly Justice O'Connor, who formulated a strongly worded dissent.\footnote{\cite[494-505]{kelo05}. Justice O'Connor was joined by the four other dissenters, but Justice Thomas also formulated his own dissent, taking a more narrow view and arguing for the revival of a strict reading of the public use requirement, see \cite[505-523]{kelo05}.} She writes as follows:

\begin{quote}
Any property may now be taken for the benefit of another private party, but the fallout from this decision will not be random. The beneficiaries are likely to be those citizens with disproportionate influence and power in the political process, including large corporations and development firms. As for the victims, the government now has license to transfer property from those with fewer resources to those with more. The Founders cannot have intended this perverse result.\footcite[505]{kelo05}
\end{quote}

It seems to me that the values Justice O'Connor rely on in her assessment are closely related to the idea of human flourishing presented by Alexander and others, particularly those pertaining to the political function of property as an anchor for community and democracy. Indeed, the danger of powerful groups gaining control of the power of eminent domain does not only affect the individual entitlements of owners. It also affects society, as the economic rationality used to justify interference comes to result in an implicit political statement to the effect that the property of the rich and powerful is better protected, and valued higher by the state, than property owned by regular citizens, who reside in ordinary communities.

The effect of a traditional economic development taking is that property rights are transferred from the many to the few, taken from ordinary people and given to the powerful. Hence, these cases represent a possibly pernicious redistribution of property, not necessarily in financial terms -- depending on the level of compensation -- but surely in terms of property's social function. The structural imbalances of the condemnation process itself find permanent expression in the new distribution of property. The social structures of a living community are dismantled in favor of a social structure that revolves around the commercial interest of a company. The political and social power of the community is diminished, perhaps lost in its entirety, while the political and social power of the company increases.

It seems clear that to Justice O'Connor, this too is a negative consequence of the taking. Again, we notice that recognizing this effect requires a social function approach to property. There is no clearly quantifiable individual loss -- no one particular ``stick'' in the property bundle that is not compensated. Rather, it is the community itself that is lost, a community that was not directly implicated in any ``entitlement'', but which played a crucial role in providing meaning to the totality of the bundle enjoyed by the owner. Even if we extend our perspective to account for indirect individual losses, we are not doing justice to the loss in this regard. The owner might relocate, acquire new property with a similar meaning in a new community somewhere else. But that does not make up for the fact that {\it this} community is lost forever, as {\it this} property takes on new meanings and functions. The loss to Suzanne Kelo, therefore, might  even be a significant loss to the City of New London.

Of course, the economic and social gains of development might outweigh such negative effects on community. But, arguably, the balancing of interests required in this regard can only be carried out by an institution that sufficiently recognizes the owners' and their community's right to participation and self-governance. The presence of a highly active commercial third party, in particular, means that public participation in the standard sense might be insufficient. In economic development takings, the commercial company typically appears alongside the government, as a more or less integrated part of the institutional structure making the decision to condemn. The owners, however, do not enjoy a corresponding level of participation.

In particular, their interests are only negatively defined. They are adversely effected and may object, but under standard administrative regimes they play no constructive role in the process. For instance, they are not called on to take part in the development itself, or to assess its merits more broadly than by being asked to respond based on their own individual entitlements. In fact, I think this is one of the main problems with economic development takings. I will argue for this in more depth later, but I remark here that an important reason to focus on this aspect is that it involves precisely those values that economic development takings are most likely to offend against. In particular, if the loss of community outweighs the positive effect of economic development, this is unlikely to be recognized by a process that relies mainly on the positive contribution of the developer and the expert planners.\footnote{A similar point is made in \cite{underkuffler06}.} 

The objections made by owners, moreover, may not only be given too little weight given the imbalance of power between owners and developers. As long as owners themselves focus only on the individual loss, they may not get to those issues that are the most important for property's social function. However, I do not think it is sufficient to theoretically proclaim that these aspects need to be considered. To address the democratic deficit of economic development takings, it seems likely that institutional changes will have to be made, to give those functions a voice in the decision-making process. This should ensure greater involvement by the local community (including, perhaps, even non-owners) in the decision-making process relating to development. Not only should they be asked if they have objections. They should be be included in a constructive way, perhaps even be compelled to assume an active role in relation to the proposed project.

This is a proposal that envisages owners engaging directly with both government and potential developers, consider alternative schemes, and make their own proposals. In short, this asks for a system where owners participate as a community. According to the human flourishing theory as I understand it, this is not only a right, but also an obligation. It gives a plausible basis on which to strike down economic development takings, and to do so without giving up the value of judicial deference. In addition, it is a call for institutional reform, to search for new governance frameworks that will empower owners and their communities.

It seems to me that Justice O'Connor's argument reflects some of these ideas. Indeed, she seems to believe strongly that the taking of Kelo's home would be a particularly harmful interference in the ``just social structures'' surrounding it. Importantly, a piece-by-piece entitlement-based approach to {\it Kelo} could hardly justify the degree of disapproval seen in Justice O'Connor's opinion. After all, Kelo had been offered generous compensation, there had been no clear breach of concrete procedural rules, and the claim that the taking was {\it only} a pretext to bestow a benefit on Pfizer did not seem supported by the facts.\footnote{See \cite{bell06}.} Rather, it was the overall character of the taking that could be used to argue that it was illegitimate. In this picture, moreover, the perceived lack of a clearly identifiable and direct public benefit becomes only one of several factors.

In addition, the institutional, social and political aspects of the case come into focus. The economic implications are less important to Justice O'Connor. Even the importance of homeownership to personhood does not receive the same attention as structural aspects. The problem which overshadows everything else is the concern that economic development takings represent a form of governmental interference in property that might come to systematically favor the rich and powerful to the detriment of the less resourceful. Hence, such takings may help establish and sustain patterns of inequality. Hardly anyone would openly regard this as desirable; it is not hard to agree that if Justice O'Connor's predictions about the fallout of {\it Kelo} are correct, then this is indeed be ``perverse''. 

The question, of course, is whether her predictions are warranted. This is a call for empirical and contextual assessment of economic development takings, to help us gain a better understanding of how they actual affect political, social and bureaucratic processes. In addition, it raises the question of how to {\it avoid} negative effects, that is, how to design rules and procedures that can reduce the democratic deficit of economic development takings. As I now move away from theory towards concrete assessment of economic development takings, both these questions will be in focus.

\section{Conclusion}

In this chapter, I have presented the core notion of my thesis, that of an economic development taking. I started by noting that while the notion is straightforward enough to define factually, it is far from obvious what implications it has for legal reasoning. I illustrated the subtleties involved by considering a concrete example of a commercial scheme that looked like it might well result in compulsory acquisition of land, namely Donald Trump's controversial plans to develop a golf course on a site of special scientific interest close to Aberdeen, Scotland. In the end, the plans did {\it not} require takings, as Trump was able to make creative use of property rights he acquired voluntarily, against the complaints of recalcitrant neighbors.

This turn of events made the example even more relevant to the points I have been trying to make in this chapter. It served to highlight, in particular, that the question studied in this thesis is not a black-and-white issue that sees privileged property rights enthusiasts on one side of the equation balanced against the good will of the regulatory state on the other. Rather, the example of Trump's golf course allowed me to emphasize the importance of context when assessing both the nature of property rights and the meaning of protecting them. In particular, to protect the property rights of those opposing Trump's golf course was not about protecting just any property, it was about protecting the property of members in a local community that felt it would be detrimental to this community, and to their lives, if Trump was allowed to redefine it. In particular, after Trump decided not to pursue compulsory purchase, protecting the property of these members of the community became a question of {\it restricting} the degree of dominion that Trump could exercise over his own property. Hence, under a conventional and overly simplistic way of looking at these matters, protecting property then became tantamount to restricting its use, a seeming paradox.

To resolve this paradox, and to arrive at a better conceptual understanding of economic development takings, I looked to various theories of property. I noted that there are differences between civil law and common law theorizing about property, but I concluded that these differences are not particularly relevant to the questions studied in this thesis. In particular, I observed that neither the bundle theory, dominant in the common law world, nor the dominion theory, used by civil law jurists, helped me clarify economic development takings as a category of legal thought.

I then went on to consider more sophisticated accounts of property, noting that a range of different {\it normative} theories have been proposed. These differ with respect to the values that they think the institution of property should promote, and as such they were also relevant to the question of assessing economic development takings. However, they do not allow us to zoom in on such takings in a more value-neutral way, to argue that regardless of one's normative persuasions, one should acknowledge that they deserve special attention.

I argued that in order to make this point successfully, the traditional entitlements-based perspective on property had to be abandoned. Instead, I looked to the social function theory of property, which encourages us to take a more contextual perspective on rights and obligations inherent in property. In particular, I noted that the social function theory compels us to recognize the importance of property in regulating social and political relations. Hence, economic development takings are special because they redefine the meaning of the property that is taken and cause a lasting disturbance to the established economic, social and political relationships that exist between owners, communities, state bodies, and commercial actors. The social function theory asks us to acknowledge that property rules are hardly ever neutral with regards to such effects. I identified this as the key observation that allowed me to make sense of economic development takings as category of legal reasoning.

After concluding that the social function theory allowed me to formulate a coherent conceptual basis for studying such takings, I went on to argue that in the first instance, the theory should be understood as giving us purely {\it descriptive} insights into the workings of property and its role in the legal order. In this, I advanced a different stance than many property scholars, by arguing that it would be better to decouple the more normative aspects of the theory, to allow the social function theory to serve as a common ground for further value-based debate.

I then went on to clarify my own starting point for engaging in such debate, by expressing support for the human flourishing theory proposed by Alexander and Pe\~{n}alver. I noted that this theory focuses on how property enables communities and individuals to  participate in social and political processes. I argued that protecting this function of property was good, and that this value should be considered fundamental in property law. Moreover, I noted that the human flourishing theory also contains a further important insight, concerning the scope of the state's power to protect. In particular, the theory asks us to recognize that protecting property against interference that is harmful to human flourishing is a responsibility that the state has even in cases when the individual owners themselves neglect to defend their property, for instance because of financial incentives to remain idle. In other words, some functions of property are such that owners have an obligation to preserve them, while the state has a duty to protect them, potentially even against the will of the owners.

After this, I went on to provide some introductory remarks on economic development takings, drawing on the theoretical insights collected from preceding sections. To make the discussion concrete, I considered the case of {\it Kelo}, which propelled the notion of an economic development taking to the front of the takings debate in the US. I focused particularly on the dissenting opinion of Justice O'Connor, and I argued that she approached the issue in a way that is consistent with the theoretical basis proposed in this chapter.

I will now go on to make my analysis of economic development takings more concrete, by considering how such takings are dealt with in Europe and the US respectively. I note that the category has yet to receive much attention in Europe, so the discussion focuses on the US. Here, the attention this issues has received after {\it Kelo} has been staggering. To get a broader basis upon which to asses all the various arguments that have been presented, I consider the historical background to the issue as it is discussed in the US. This involves giving a detailed presentation of the public use restriction, as it was developed in case law from the states during in the 19th and early 20th century. I then connect this discussion with recent proposals to deal with economic development takings, responding to the backlash of {\it Kelo}, by aiming to address the democratic deficit of such takings.

Later, when I begin to consider the law relating to Norwegian hydropower, I will look back at the theoretical basis provided in the present chapter to guide the analysis. In particular, I focus on certain decision-making mechanisms that have developed on the ground in Norway, as a practical response to the increased tendency for local owners to engage in hydropower development. I will argue that this shows the conceptual strength of the idea that property is irreducibly embedded in community, and that its meaning and function is not -- and should not -- be ordained from above, but should be allowed to arise from its grassroots through continuously evolving institutions of participatory democracy.

%If property rights, particularly rights to land, are distributed fairly in a local community, property is not a privilege. Even if most people do not hold land rights, as long as no one holds excessive amounts, there is no reason why owners and non-owners should not be on equal footing in the local community. They are mutually dependent on one another; non-owners need access to natural resources, while owners need access to services. Moreover, the bonds of community will tend to ensure that owners are deterred from engaging in exploitative practices towards non-owners in much the same way as non-owners are deterred from undermining property rights. 
%Tensions reached new heights following the Supreme Court case of {\it Kelo v City of New London}.\footcite{kelo05}  The company Pfizer was allowed to expropriate homes for the construction of new research facilities, and the questions arose as to whether or not this constituted "public use" in the sense of the property clause in the Fifth Amendment to the US Constitution.  The majority 5-4 found that the expropriation was constitutional, but the decision was controversial. Arguably, the attitude following {\it Kelo} has shifted towards a greater feeling of unease regarding economic takings, both among legal scholars and members of the general public.

The case has also had a great impact on academic writing on takings law in the US, where economic development cases are now usually viewed as a distinct sub-class of takings which merit particular attention. Moreover, a consensus seems to be emerging that there is a need for novel approaches to deal with such takings, possibly even new legal frameworks to resolve the tensions that typically arise. Some authors argue for a simple solution: an outright ban on economic development takings. However, the majority of scholars take a more measured approach, recognizing the need for legal frameworks that can be used to promote development projects without making illegitimate use of compulsion against land owners.


\chapter{Taking Property for Profit}\label{chap:2}

In the previous chapter, I argued that economic development takings should be considered a separate category of interference in private property. I also placed it in the theoretical landscape, by relating it to the social function theory of property. Economic development takings, I argued, raise questions about the overall effect of interference, questions that require contextual assessment. In particular, I argued that they require us to depart from the individualistic, entitlements-based approach that otherwise dominates in property law.

The significance of a new conceptual category should not be overstated. While I think the economic development label is a very helpful tool when thinking about certain takings cases, I am not suggesting that these cases should be approached uniformly and on the basis of mechanical legal assessment. Rather, the importance of context indicates that a concrete approach is in order. Moreover, while I have argued for a certain way of reasoning about economic development takings, I have so far said little about what the law has to say about them. In this chapter, I consider this question, by giving an overview of how economic development cases are dealt with in some more or less representative jurisdictions.

First, I will comment briefly on the importance of economic development takings on the global stage.

\section{Introduction}

Public-private partnerships are becoming increasingly important to the world economic order.\footnote{See generally \cite{saussier13}.} To some, they are the illegitimate children of privatization and deregulation, while others see them as efforts to make the public sector more efficient and accountable. Either way, their numbers are growing, and they appear to be here to stay.\footnote{Although their potentially pernicious effects on stability and accountability has also been noted. See, e.g., \cite{baker03} (arguing that ``the Enron scandal can be better understood as an American form of public private partnership rather than just another example of capitalism run amok'').} In this situation, it is inevitable that when eminent domain is used to acquire property for economic development, those who directly benefit will often be commercial companies rather than public bodies. In the previous chapter, I pointed out how indirect public benefits are typically used to justify such takings. Standard legitimizing reasons include the prospect of new jobs, increased tax revenues, and various other economic and social ripple effects. However, as I have indicated, economic development takings have a tendency to result in controversy.

In the US after {\it Kelo}, they have also been at the forefront of the constitutional property debate. In the rest of the world, a similar shift in academic outlook has yet to take place, but expropriation-for-profit situations are increasingly coming into focus here as well.\footnote{See, e.g., \cite{gray11,waring13,verstappen14}.} If we lift our perspective slightly, to consider commercially motivated interference more generally, it even seems appropriate to speak of a crisis of confidence in property law, particularly in relation to land rights. This is most clearly felt in the developing world, where egalitarian systems of property use and ownership are coming under increasing pressure. It has been noted, in particular, that large-scale commercial actors are assuming control over an increasing share of the world's land rights, a phenomenon known as {\it land grabbing}.\footnote{See generally \cite{borras11}.} 

So far, most research on land grabbing has looked at how commercial interests, often cooperating with nation states, exploit weaknesses of local property institutions, to acquire land voluntarily, or from those who lack formal title. However, the danger of {\it Kelo}-type reasoning has also been recognized. In particular, it has been noted how the purported public interest in economic development can be used to justify land grabs that would otherwise appear unjustifiable. In a recent article, Smita Narula cites {\it Kelo} directly and warns that procedural safeguards alone might not provide sufficient protection against abuse. She writes:
\begin{quote}
Procedural safeguards, however, can all too easily be co-opted by a state because its claims about what constitutes a public purpose may not be easy to contest. Particularly within the context of land investments, states could use the very general and under-scrutinized language of ``economic development'' to justify takings in the public interest.\footcite[157]{narula13}
\end{quote}

This underscores the broader relevance of the study of economic development takings. In addition, it reminds us that the question of what can be justified in the name of ``economic development'' is a general one, not confined to particular systems for organizing property rights. To address this, and to restore confidence in the institution of property more generally, many turn towards {\it human rights}. These scholars argue that a human right to land should be recognized on the international stage, a right that would apply even when those most affected by a land grab lack formal title.\footnote{See generally \cite{schutter10,schutter11,kunnerman13}.} If successful, this approach promises to deliver basic protection against interference in established patterns of property use independently of how particular jurisdictions approach property.

In Europe, a human rights perspective is already of great practical significance due to the European Convention of Human Rights (ECHR) and the court in Strasbourg (ECtHR). But, of course, in the context of land grabbing, protecting land rights is not primarily a question of protecting the civil law ideal of individual dominion. Rather, it is a question of providing protection against large-scale transactions that destabilize or destroy established patterns of land use, to the detriment of local communities. Nevertheless, the questions raised by the public interest  narrative -- and the notion of ``economic development'' in particular -- must be expected to arise in much the same way as in cases when formal title is acquired following a state-authorized taking.

Hence, it is somewhat surprising that the special category of for-profit takings has not received more attention from the point of view of human rights law. In human rights discourse, the focus tends to be rather on fairness and proportionality as broad benchmarks, in addition to specific values related to food security and protection of livelihoods that arise with particular urgency in the context of third-world land grabs. But how to achieve effective protection depends as much on the development of firm categories and enforcible legal principles as it does on broad benchmarks and good intentions. In this regard, I think Narula is right to stress that the lack of a convincing approach to the notion of ``economic development'' is a crucial challenge.

On the one hand, economic development is no doubt a sound overreaching goal, particularly for poor nations. But at the same time, the risk of abuse is obvious when such a vague term is used to justify dramatic interferences in property. After all, interferences in property can cause severe disturbances in people's life. This, moreover, is true for middle-class US homeowner in much the same way as it is true for members of self-sustaining agrarian communities in Africa, although the stakes might be much higher for the latter.

As illustrated by {\it Kelo}, deep conflicts can arise in this regard also in developed democracies with long established systems of formal title. In the following, I will attempt to shed further light on the issue as it arises in such legal systems, without considering the additional complications that arise when property itself is -- formally speaking -- a more fluid concept. I note, however, that according to the social function view of property, there is no need to view formally recognized property rights as completely distinct from rights arising from property use. The two are intertwined and the difference between them is at most a matter of degree.\footnote{Moreover, if the human flourishing account of property values is successfully developed, there should even be hope that a unified normative treatment can be given at some point.}

However, my case study will look to Norwegian law, a prosperous European country with a long tradition of formal title to land. Hence, it is prudent to narrow down the discussion here by focusing on similar jurisdictions.\footnote{The relation with third-world land grabbing is a highly interesting question for future work.} I will do so now, beginning with a brief look at English and German law, to illustrate that there are great differences in how different European jurisdictions think about property in general, and takings in particular. Then I turn my attention to the ECHR and I focus on presenting the proportionality test that is now at the core of property adjudication at the ECtHR.

Following this, I move on to consider the US in greater depth, both the historical debate that led to {\it Kelo} and the suggestions for reform that have emerged following its backlash. A closer look is necessary because of the sheer magnitude of writing on this issue in the US. Moreover, while much of it is repetitive and coloured by the tense political climate, I believe some historical points, as well as some recent suggestions for reform, are highly relevant also to the international setting. To single out and analyse those aspects is the main aim of this part of the chapter. Indeed, the current debating climate in the US might be an indication of what is to come also in Europe, if concerns about the legitimacy of economic development takings are not taken seriously.

%I also highlight what I believe to be a connection between the situation in the US leading up to {\it Kelo} and the present situation in Europe, illustrated by the fact that the European Court of Human Rights is now explicitly endorsing ``stronger protection'' of property rights.  I attempt to identify the reasons behind calls for a stricter approach, arguing that it is connected to the fact that interferences in property under modern regulatory regimes is sanctioned in wide a range of different circumstances, serving to undermine their status as a necessary burden imposed on owner's according to the will of the greater public. In some cases, rather, takings appear to both owners and the public as improperly motivated and socially and politically unfair. I note that this happens particularly often in economic development cases, when commercial actors benefit to the detriment of local communities. I go on to list some concrete issues that arise with respect to such takings and that have been flagged as problematic in the literature.
%
%Following up on this, I consider various proposals that have been made to resolve tensions and limit the possibility of abuse in economic development cases. The differences of opinion that have been expressed in this regard have been quite substantial, and proposals have ranged from suggesting an outright ban on economic development takings  (Somin 2007; Cohen 2006) to suggesting that the best way forward is to reassess principles for awarding compensation in such cases (Householder 2007; Lehavi and Licht 2007).

%Much of the current theory focus on assessing traditional judicial safeguards that courts can rely on to prevent abuses, pertaining primarily to the material assessment of proportionality, public purpose, and compensation. 

%In the last part of the chapter, I will focus on a very interesting strand of recent work in the US, which shifts attention towards procedural rules that can help address the worry that economic development takings tend to suffer from a democratic deficit. The core concern is that the manner in which eminent domain decisions are typically made, and the way in which owners are compensated, might be unsuitable for economic development cases. Importantly, the need for special procedures has been noted, to restore legitimacy.\footnote{See generally \cite{lehavi07,heller08}.} This ties the US debate even closer to the European context, where proportionality, not public use, has become the key notion in property protection. Several recent suggestions from the US can be conceptualized as suggestions that aim to secure fairness and proportionality, while paying less attention to the formalistic question of what constitutes a ``public use''.
%
%%Also, it allows us to be very clear about a special concern that arises for economic takings cases: under current regulatory regimes, the government and the developer together often dominate the decision-making process completely, leaving the property owners marginalized. Hence, there is often a {\it democratic deficit} in such cases, resulting in discontent and a feeling that the taking is not in the public interest at all. Importantly, some recent writers hypothesize that if the proper balance can be restored in the decision-making process, so will the decision reached appear more legitimate, also with respect to the public use clause. In my opinion, this idea is crucial, and together with the question of compensation, which raises a similar structural problem, it will guide the rest of the work done in this thesis. 
%

In response to that worry, this chapter aims to  bring into focus the following key question: What principles can be used to ensure meaningful participation and just compensation in economic takings cases, without hindering socially and economically desirable development projects? The tentative answers provided in Section \ref{sec:ir} will set the stage for the remainder of the thesis, where they will be assessed in depth against the case study of Norwegian hydropower.

%In particular, I will consider two special semi-judicial procedural systems used in such cases in Norway, one targeting compensation following expropriation, and another used as an alternative to expropriation, particularly in cases when development requires cooperation among many owners.

%I conclude by arguing that approaches along procedural lines represent the best way forward in relation to addressing issues associated with economic development takings. This raises the following problem, however: what procedural principles can be used to ensure meaningful participation, without hindering socially and economically desirable development projects? This question sets the stage for the remainder of my thesis, where I conduct a case study of expropriation for the development of hydro-power in Norway. In particular, I will consider two special semi-judicial procedural systems used in such cases in Norway, one targeting compensation following expropriation, and another used as an alternative to expropriation, particularly in cases when development requires cooperation among many owners.

\section{A European contrast}\label{sec:contrast}

Economic development takings have not become as controversial in Europe as they are in the US, but there have been cases where the issue has come up, in several different jurisdictions.\footnote{For instance, in the UK, Ireland and Germany, as well as in Norway and Sweden. See \cite[466-483]{walt11}; \cite{stenseth10}.} The European Convention of Human Rights (ECHR) contains a property clause in Article 1 of Protocol No 1 (P1(1)), but the legitimacy of economic development takings has not yet been discussed in case law from the European Court of Human Rights (ECtHR). However, it is interesting to analyse cases like {\it Kelo} against P1(1), particularly since the ECtHR has developed a doctrine that focuses on ``proportionality'' and ``fairness'' rather than the purpose of interference.\footnote{See generally, \cite[Chapter 5]{allen05}. This approach may become even more significant as a source of property protection in the future, as the ECtHR have indicated that there are ``jurisprudential developments in the direction of a stronger protection under Article 1 of Protocol No. 1'', see \cite[135]{lindheim12}.}

The fundamental question raised by economic development takings can be formulated independently of specific property clauses as follows: When, if ever, is it permissible for governments to order compulsory transfer of property rights from citizens to for-profit legal persons in order to facilitate economic development?

In this section, I address economic development takings from the point of view of European sources. I first contrast English and German law, to show that there are significant differences between European jurisdictions in this regard. I then go on to give a more detailed presentation of the unifying property clause in P1(1) of the ECHR. The case law from the ECtHR is presented and analysed in some depth, in an effort to assess how the ECtHR would be likely to approach an economic development case such as {\it Kelo}. In particular, I argue that the ``proportionality'' doctrine offers an interesting approach to economic development cases. This doctrine stipulates that a ``fair balance'' must be struck  between the interests of the property owner and the public.\footcite[Chapter 5]{allen05} I argue that such a perspective could make it easier to get to the heart of why economic development takings are often seen as problematic, without getting lost in theoretical discussions about the meaning of  terms like ``public use'' or ``public purpose''. However, I also raise the concern that the ECtHR is not the appropriate institution for applying the proportionality test concretely. Its remoteness suggests that we should also look for more locally grounded legitimacy-enhancing institutions. Such institutions will likely be better able to assess the fairness of interference in context.

I go on to discuss whether existing government institutions can serve this purpose, arguing that local courts may well be the best candidates. However, I argue that active application of the ``proportionality''-doctrine in property cases has not yet developed fully at the local level. I also discuss possible shortcomings of local courts; as judicial bodies they are not intrinsically well-suited to carry out the kind of assessment that is required. Hence, I suggest that entirely new institutional proposals might be in order. I conclude by arguing that once the need for local grounding is recognized and met, the ECtHR has the potential to play an important and constrictive role in providing oversight and developing basic principles.

\subsection{England}\label{sec:england}

In England, the principle of parliamentary supremacy and the lack of a written constitutional property clause has led to expropriation being discussed mostly as a matter of administration and property law, not as a constitutional issue.\footcite{taggart98} Moreover, the use of compulsory purchase -- the term most often used to denote takings in the UK -- has not been restricted to particular purposes as a matter of principle. The uses that can warrant compulsory alienation of property are those that parliament regard as worthy of such consideration. However, as private property itself has long been recognized as a fundamental right, the power of compulsory purchase has typically been exercised with great caution. 

In his {\it Commentaries on English Law}, William Blackstone famously described property as the ``third absolute right'' that was ``inherent in every Englishman''.\footcite[134-135]{blackstone79}  Moreover, Blackstone expressed a very restrictive view on the possibility of expropriation, arguing that it was only for the legislature to interfere with property rights. He warned against the dangers of allowing private individuals, or even public tribunals, to be the judge of whether or not the ``common good'' could justify it. Blackstone went as far as to say that the public good was ``in nothing more invested'' than the protection of private property.\footcite[134-135]{blackstone79}

Historically, Blackstone's description conveys a largely accurate impression of takings practice in England. Indeed, Parliament itself would usually be the granting authority in expropriation cases, through so-called {\it private Acts}. Hence, compulsory purchase would not take place unless it had been discussed at the highest level of government. Moreover, the procedure followed by parliament in such cases strongly resembled a judicial procedure; the interested parties were given an opportunity to present their case to parliament committees that would then decide whether or not compulsion was warranted.\footnote{See \cite[13-16]{allen00}. While this procedure clearly reflected a protective attitude towards private property, recent scholarship has pointed out that expropriation was actually used more actively in Britain following the glorious revolution, see \cite{hoppit11}.} 

On the one hand, the direct involvement of parliament in the decision-making process reflected the fundamental respect for property rights that permeated the system. But at the same time, parliamentary supremacy also meant that the question of legitimacy was rendered mute as soon as compulsory purchase powers had been granted. The courts were not in a position to scrutinize takings at all, much less second-guess parliament as to whether or not the taking was for a legitimate purpose.

Eventually, an overworked parliament developed procedures for dealing more expeditiously with takings cases, and during the 19th Century, as an industrial economy developed, private Acts granting commercial companies the power to take land grew massively in scope and importance.\footnote{See \cite[204]{allen00}.} Private railway companies, in particular, regularly benefited from such Acts.\footnote{\cite[204]{allen00}. See generally \cite{kostal97}.} During this time, the expanding scope of private-to-private transfers for economic development lead to quite a bit of political debate and controversy. Usually, it would attract particular opposition from the House of Lords. Interestingly, this opposition was not only based on a desire to protect individual property owners. It also often reflected concerns about the cultural and social consequences of changed patterns of land use.\footcite[204]{allen00} 

Hence, the early debate on economic development takings in the UK shows some reflection of a contextual approach to property protection. However, as society itself changed dramatically following increasing industrialization, an expansive approach to compulsory purchase eventually emerged triumphant. At the same time, the idea that economic development could justify takings gradually became less controversial. 

Today, the law on compulsory purchase in England is regulated in statute and the role of courts is to a large extent limited to the application and interpretation of statutory rules. Some common law rules still play an important role, such as the {\it Pointe Gourde} rule discussed in more depth in Chapter \ref{chap:5}. With respect to the question of legitimacy, however, the starting point for English courts is that this is a matter of ordinary administrative law. 

More recently, the \cite{hra98} adds to this picture, since it incorporates the property clause in P1(1) into English law. But even so, the usual approach for English courts is to judge objections against compulsory purchase orders on the basis of the statutes that warrant them, rather than constitutional or human rights principles that protect property.\footnote{The important statutes are the \cite{ala81}, the \cite{lca61}, the \cite{tcpa90} and the \cite{pcpa04}. Acquisition of Land Act 1981, the Land Compensation Act 1961, the Town and Country Planning Act 1990 and the Planning and Compulsory Purchase Act 2004.} It is typical for statutory authorities to include standard reservations to the effect that some public benefit must be identified in order to justify a CPO, but the scope of what constitutes a legitimate purpose can be very wide. For instance, to warrant a taking under the \cite{tcpa90}, it is enough that it ``facilitates the carrying out of development, redevelopment and improvement on or in relation to the land''.\footcite[226]{tcpa90} 

While various governmental bodies are authorised to issue compulsory purchase orders (CPOs), a CPO typically has to be confirmed by a government minister. The affected owners are given a chance to comment and if there are objections, a public inquiry is typically held. The inspector responsible for the inquiry then reports to the relevant government minister, who makes the final decision about whether or not it should be granted, and on what terms. The CPO may then be challenged in court, but will usually only be scrutinized on the basis of whether or not it lies within the scope of the statute authorizing it. Hence, the discussion and evaluation at court is firmly grounded in statutory rules rather than constitutional principles.

That said, the idea that property may only be compulsorily acquired when the public stands to benefit permeates the system. Indeed, this has also been regarded as a constitutional principle, for instance by Lord Denning in {\it Prest v Secretary of State for Wales}.\footcite{prest82} He said:

\begin{quote}
It is clear that no minister or public authority can acquire any land compulsorily except the power to do so be given by Parliament: and Parliament only grants it, or should only grant it, when it is necessary in the public interest. In any case, therefore, where the scales are evenly balanced – for or against compulsory acquisition – the decision – by whomsoever it is made – should come down against compulsory acquisition. I regard it as a principle of our constitutional law that no citizen is to be deprived of his land by any public authority against his will, unless it is expressly authorised by Parliament and the public interest decisively so demands. If there is any reasonable doubt on the matter, the balance must be resolved in favour of the citizen.\footcite[198]{prest82}
\end{quote}

Lord Denning also supported the doctrine of necessity, as expressed by Forbes J in {\it Brown v Secretary for the Environment}:\footcite{brown78}

\begin{quote}It seems to me that there is a very long and respectable tradition for the view that an authority that seeks to dispossess a citizen of his land must do so by showing that it is necessary, in order to exercise the powers for the purposes of the Act under which the compulsory purchase order is made, that the acquiring authority should have authorisation to acquire the land in question.\footcite[291]{brown78}
\end{quote}

In practice, these principles are mostly implicit in legal reasoning, as a factor that influences the courts when they interpret statutory rules and carry out judicial review of administrative decisions. As Watkins LJ stated in {\it Prest}:

\begin{quote}
The taking of a person's land against his will is a serious invasion of his proprietary rights. The use of statutory authority for the destruction of those rights requires to be most carefully scrutinised. The courts must be vigilant to see to it that that authority is not abused. It must not be used unless it is clear that the Secretary of State has allowed those rights to be violated by a decision based upon the right legal principles, adequate evidence and proper consideration of the factor which sways his mind into confirmation of the order sought.\footcite[211-212]{prest82}
\end{quote}

In {\it R v Secretary of State for Transport, ex p de Rothschild}, Slade LJ referred to the judgment and made clear that he did not regard it as expressing a rule concerning the burden of proof in compulsory purchase cases, but rather as more general expressions about the severity of the interference and the importance of vigilance in such cases.\footnote{rothschild89} He said that they provided ``a warning that, in cases where a compulsory purchase order is under challenge, the draconian nature of the order will itself render it more vulnerable to successful challenge''.\footcite[938]{rothschild89}

A nice example of how these sentiments influence the assessment of legitimacy of takings, showing how it is applied in economic development cases, can be found in the recent case of {\it Regina (Sainsbury’s Supermarkets Ltd) v Wolverhampton City Council}.\footcite{sainsbury10} Here a CPO was granted to allow the company Tesco to acquire land from its competitor Sainsbury, in a situation when they were both competing for licenses to undertake commercial development on this land. The decisive factor that had led the local authorities to grant the CPO was that Tesco had offered to develop a different property in the same local area, which was currently in need of regeneration. 

Sainsbury protested, arguing that the local council could not strike such a deal on the use of its compulsory purchase power. It was argued, moreover, that taking the land for incidental benefits resulting from development in a different part of town was not legitimate under the Town and Country Planning Act 1990. The UK Supreme Court agreed 4-3, with Lord Walker in particular emphasizing the need for heightened judicial scrutiny in cases of private-to-private cases for economic development.\footcite[80-84]{sainsbury10} Lord Walker even cited {\it Kelo}, to further substantiate the need for a stricter standard in such cases.\footcite[81]{sainsbury10} 

However, the main line of reasoning adopted by the majority was based on an interpretation of the Town and Country Planning Act itself. In particular, the majority held that it was improper for the local council to take into consideration the development that Tesco had committed itself to carry out on a different site.\footcite[73-79]{sainsbury10} This, in particular, was not ``improvement on or in relation to the land'', as required by the Act.\footcite[336]{tcpa90} In addition, Lord Collins, who led the majority, said that ``the question of what is a material (or relevant) consideration is a question of law, but the weight to be given to it is a matter for the decision maker''.\footcite[70]{sainsbury10} Hence, the general importance of the decision for economic development cases is unclear.

Still, it is interesting to see how the purpose of the interference featured in the Supreme Court's interpretation and application of the statutory rules. The opinion of Lord Walker is particularly interesting, since he stresses that ``The land is to end up, not in public ownership and used for public purposes, but in private ownership and used for a variety of purposes, mainly retail and residential.''\footcite[81]{sainsbury10} He goes on to state that ``economic regeneration brought about by urban redevelopment is no doubt a public good, but ``private to private'' acquisitions by compulsory purchase may also produce large profits for powerful business interests, and courts rightly regard them as particularly sensitive.``\footcite[81]{sainsbury10}

Lord Walker then makes clear that he does not think it is impermissible, as such, for the local council to take into consideration positive effects on the local area, even when these do not directly result from the planned use of the land that is being acquired. Instead, he relies explicitly on the for-profit character of the taking, by arguing that ``the exercise of powers of compulsory acquisition, especially in a ``private to private'' acquisition, amounts to a serious invasion of the current owner's proprietary rights. The local authority has a direct financial interest in the matter, and not merely a general interest (as local planning authority) in the betterment and well-being of its area. A stricter approach is therefore called for.''\footcite[84]{sainsbury10} 

Lord Walker's opinion might indicate that the narrative of economic development takings is about to find its way into English case law. Moreover, a more critical approach might be adopted in the future, when compulsory purchase powers are made available to commercial companies wishing to undertake for-profit schemes. However, for schemes where the commercial aspect appears less dominant, English courts still appear very reluctant to quash CPOs, also when the purpose is economic development. This is so even in situations when the owners have requested a stricter standard of review on the basis of human rights law. 

For instance, in the case of {\it Smith \& Others v Secretary of State for Trade and Industry}, a caravan site was compulsorily acquired for development in connection with the London Olympic Games.\footcite{smith08} Some of the owners protested, including Romani Gypsies who used the caravans as their primary residence. A public inquiry was held, after which the inspector recommended that the CPO should not be confirmed until adequate relocation sites had been identified. However, due to the ``urgency, timing and importance '' of the project, the Secretary of State decided to go ahead before a relocation scheme was put in place (although he expressed commitment to ensuring satisfactory relocation).\footcite[10]{smith08} The owners argued that without satisfactory relocation plans, the interference in the property rights was not proportional and had to be struck down on the basis of human rights law, in particular Article 8 in the ECHR regarding respect for the home and private life.\footcite[27-51]{smith08}

The Court of Appeal considered the matter in great depth, applying the doctrine of proportionality developed at the ECtHR, which goes beyond the standard room for judicial review of administrative decisions under English law. However, the Court still concluded that the taking was proportional. This was largely based on the finding that ``the issue of proportionality has to be judged against the background that everyone accepts that an overwhelming case has been made out for compulsory acquisition of the sites for the stated objectives and that compulsory purchase is justified.''\footcite[42]{smith08} 

Justice Williams arrived at this conclusion after noting that the owners' {\it only} substantial objection against the CPO was that it was confirmed before adequate relocation measures had been agree on.\footcite[42]{smith08} Hence, the question as he saw did not concern the validity of using compulsory purchase powers, but merely the timing with which it had been ordered. On this basis, he framed the question of legitimacy as one relating to the ``necessity'' standard, according to which an infringement on Convention rights is only permissible when the public interest cannot be served in some other way.\footcite[43]{smith08} A strict reading of this standard holds that an interference must be the {\it least intrusive means} of achieving the stated aim.\footnote{Such a standard has been adopted in some Convention cases, for instance in \cite{samaroo01}.}

Justice Williams argued against such a strict reading, subscribing instead to a view expressed as an {\it obiter} in the case of {\it Pascoe v The First Secretary of State}. According to this view, an interference need not be the least intrusive means of achieving the public purpose, it is sufficient that the measure is ``reasonably necessary'' to achieve that aim.\footnote{See \cite[74-75]{pascoe06} (quoting \cite[25]{clay04}).} However, while noting his agreement with this approach, Justice Williams went on to apply the stronger necessity test, and found that even if this was applied the CPO in question would still be a proportional interference.\footcite[41-50]{smith08}

It seems clear that while the taking in question was for economic and recreational development purposes, the case was marked by the finding that the legitimacy of the aim of interference -- to facilitate the London Olympics -- was beyond reproach. Hence, there was no need for, or even room for, more detailed purposive reasoning of the kind that would later be applied by Lord Walker in {\it Sainsbury}. The fact that the taking was for economic development and recreation, not for a pressing public need, was not considered relevant, and was not held against the effects on the owners. This, in particular, was not how the issue of proportionality was conceptualized. Indeed, since the case was construed to be solely about the extent to which the CPO was ``necessary'' to further its stated aim, the proportionality test that was carried out, despite being detailed, was very narrow in scope. It concerned only proportionality of the means, not of the aim itself. The question of how to weigh the public interest in a multi-billion dollar sporting event against the security of someone's home was not considered.

In later cases, a dismissive attitude towards substantive review has been adopted also in situations when the owners have argued against takings by explicitly questioning the proportionality of the inference against the importance of the aim. In the case of {\it Alliance Spring Co Ltd v The First Secretary of State}, a large number of properties were expropriated to build a new football stadium for the football club Arsenal.\footcite{alliance06} Some owners who stood to lose their business premises as a result of the scheme protested the legitimacy of the order, pointing to the fact that the inspector in charge of the public inquiry had recommended against the takings.\footcite[6-7]{alliance06} As noted by Justice Collins, the main line of legal argument presented against the taking was that it did not serve a ``proper purpose''.\footcite[19]{alliance06} It is of note that in his evaluation of this argument, Justice Collins largely focuses on presenting the assessments carried out by the inspector and the Secretary of State, who went against the recommendation and confirmed the CPO. Finding that these assessments took all relevant matters into account and where not clearly unreasonable, Justice Collins goes on to conclude as follows: 

\begin{quote}
There is nothing in the material put before and accepted by the Inspector which persuades me that that decision was ill founded or was one which the Secretary of State was not entitled to reach. Developments which result in regeneration of an area are often led by private enterprise. Mr Horton perforce accepts that that is so, but submits that this is not the sort of situation where, for example, a private development is the anchor for a particular scheme. I disagree.\footcite[19]{alliance06}
\end{quote}

Hence, unlike the case of {\it Smith}, where the Court did in fact carry out its own assessment of proportionality, albeit only in relation to the question of necessity, the {\it Alliance} Court was content with deferring to the assessment carried out by the executive branch.\footnote{This has been criticized, e.g., by Kevin Grey who describes the reference to Convention Rights in Alliance as ``worryingly brief''. See \cite{gray11}.} As such, the case largely follows the set pattern of judicial review of CPOs from before the passing of the Human Rights Act 1998. This mean that it also stands in contrast to how English courts have approach the Convention in other kinds of cases, involving other rights, such as Article 8 in {\it Smith}. 

Whether the approach taken in {\it Alliance} is good law after {\it Sainsbury} is unclear; from Lord Walker's opinion, it seems that a more substantive assessment can be demanded for similar cases in the future. While this might not imply a different outcome for a case like {\it Alliance}, it would mean that courts would have to engage in independent review of the purpose and merits of contested CPOs that benefit commercial actors. In particular, English courts would have to change the way they approach such cases, by being more prepared to assess for themselves whether a fair balance is struck between the interests of the developer and the property owners. Hence, it is not unlikely that the category of economic development takings will become an important point of reference in the future, both for the law and those who study it.

\subsection{Germany}\label{sec:germany}

In German law we find an explicit constitutional property clause. In particular, Article 14 of the Basic Law ({\it Grundgesetz}) reads as follows:

\begin{quote}
(1) Property and the right of inheritance shall be guaranteed. Their content and limits shall be defined by the laws. \\
(2) Property entails obligations. Its use shall also serve the public good. \\
(3) Expropriation shall only be permissible for the public good. It may only be ordered by or pursuant to a law that determines the nature and extent of compensation. Such compensation shall be determined by establishing an equitable balance between the public interest and the interests of those affected. In case of dispute concerning the amount of compensation, recourse may be had to the ordinary courts.\footcite[14]{basic49}
\end{quote}

Apart from the fact that the property clause is explicit, I note two further characteristic features of the protection of property in Germany. First, the constitution explicitly stresses that property comes with social obligations as well as rights. The use of property should ``serve the public good''. On the other hand, it is also made clear that expropriation is only permissible when it is ``for the public good''. Hence, it follows immediately that the purpose of expropriation is a relevant factor when determining the legitimacy of a taking, irrespectively of the specific statute used to authorise it. Importantly, it is clear already from the outset that the question of legitimacy is a \emph{judicial} question, one which the courts can only answer if they form an opinion about that constitutes the ``public good''. 

This means that it is quite natural to approach the question of economic development takings from the point of view of constitutional law. Unlike in England, disputes over the legitimacy of such takings can be comfortably adjudicated directly against a ``public good'' restriction. While this sets Germany apart on the theoretical level, it is unclear how much of an effect it has had in practice. To shed some light on this question, we can look to the two major authorities on the legitimacy of economic development takings, the cases of {\it D\"{u}rkheimer Gondelbahn} and {\it Boxberg}.\footcite{durkheimer81,boxberg86} 

In both cases, the German Constitutional court found that expropriation to the benefit of commercial interests was illegitimate. However, the Court argued for this result on the basis that there was insufficient statutory authority for such takings in the concrete circumstances complained of. That is, the Court did not directly address the question of whether the relevant statutes were compliant with Article 14 of the basic law. Instead, they interpreted statutory authorities on the assumption that they had to be, following a pattern of reasoning that appears to be rather close to the approach followed by English courts in similar cases.\footnote{Although in {\it Dürkheimer Gondelbahn}, Böhmer J gave a separate concurring judgment where he argued for this result on the basis of the public good requirement of the basic law.} It seems, in particular, that even in Germany, the public purpose restriction is primarily relevant as a factor guiding the interpretation of statutory authorities.

That said, the cases of {\it D{\"u}rkheimer Gondelbahn} and {\it Boxberg} show that in situations when the public purpose of a taking is unclear, German courts seem inclined to favor a narrow interpretation of the relevant statute. In {\it Bloxberg}, several properties were expropriated in favor in favor of the car company Daimler Benz AG, for commercial purposes. The affected local communities suffered from high unemployment rates and a slow economy, so a {\it prima facie} reasonable cases could be made that allowing Daimler to acquire the land was in the public interest, as it would facilitate economic growth. However, the Federal Constitutional Court agreed with the owners that the expropriation was invalid. This, it held, was because the taking was outside the scope of the relevant statute, which authorised expropriations for ``planning purposes''. The owners had argued extensively using Article 14 of the Basic Law and the constitutional ``public good'' restriction clearly did play a role in the Court's reasoning. But at the same time, the Court stressed that private-to-private transfers that bestow financial benefit on the acquiring party may well satisfy the ``public good'' requirement. The important issue was whether a sufficiently strong public interest could be identified, irrespectively of any windfall benefits that might fall on private parties.

In light of this, I think it is wrong to exaggerate the importance of the explicit formulation of the public use test offered in the German constitution. Its importance seems to rest mainly in the fact that it provides a particularly authoritative expression guiding the national courts' application of statutory provisions regarding expropriation of property. But developments in common law, where the public use requirement is stressed as a guiding constitutional principle, might well point in the same direction. In principle, both German and English Courts are in a good position to respond to increased tension regarding economic development takings by developing a stricter standard of judicial review in such cases.

A different aspect of German law deserves special attention, however, since it does not appear to have any clear counterpart in the common law tradition. This is the  ``social-obligation'' norm in Article 14 (2), which points to a different conceptualization of property rights as such. As argued by Alexander, the distinguishing feature of the property clause in the German Constitution is that the value of property is thought to relate more strongly to its importance for human dignity and flourishing in a social context, rather than the protection of individual financial entitlements. As Alexander notes regarding the Germans' own conceptualization of their property clause:

\begin{quote}
This theory holds that the core purpose of property is not wealth maximization or the satisfaction of individual preferences, as the American economic theory of property holds, but self-realization, or self-development, in an objective, distinctly moral and civic sense. That is, property is fundamental insofar as it is necessary for individuals to develop fully both
as moral agents and participating members of the broader community.\footcite[745]{alexander03}
\end{quote}

With such a starting point, it is not surprising that in cases such as {\it Boxberg}, resembling {\it Kelo}, German Courts will tend to adopt a strict view on legitimacy. These are cases when the property rights infringed on serve a fundamentally different function for the two opposing private parties. To the owner, the property is a home, an important source of self-identity, autonomy, security and membership in a community. To the taker, it represents an obstacle to commercial development which needs to be removed. In such a situation, it is in keeping with the spirit of the social-obligation norm of property to offer enhanced protection to the homeowner. To this owner, the property serves a purpose which is fundamentally different, and arguably more worthy of protection, then the property's purpose for the developer. A taking in this situation might therefore, because of Article 14, require a particularly clear and strong public interest.

But unless there is an asymmetry between owner and taker, heightened scrutiny does not necessarily follow. Hence, it is interesting to speculate what German courts would have made of a case such as {\it Regina (Sainsbury’s Supermarkets Ltd) v Wolverhampton City Council}. Here, the interests of owner and taker were strictly commercial nature. Both owned part of the contested land and neither one could develop the land according to their plans without buying out the other. The enhanced protection of property offered under German law would probably not have much significance in such a case. 

In fact, it might well be that German courts would be {\it more} likely to accept such a taking. First, their conceptualization of property rights appears to allow greater flexibility to adapt the level of protection to the circumstances and the purposes of the property in question. So even if is correct that private-to-private transfers for commercial projects require a ``stricter approach'' in general, as argued by Lord Walker in \textcite{sainsbury10}, the fact that the interests of the owner were also purely commercial  might make this less relevant. Second, German courts might be more inclined to have regard to socially beneficial additional commitments entered into by the applicant, even if they do not concern the property that is taken. As a tie-breaker, looking to such commitments might be as good an approach as any other.\footnote{This was the view taken by the dissenting minority in \textcite{sainsbury10}.}

Of course, objections could still be raised on the basis of general administrative law. Indeed, some might see the case as an example of government ``auctioning'' off licenses to the highest bidder. This might well be regarded as an affront to good governance. I will not delve into German law to assess the case from this perspective. My point is simply that because of the purposive and contextual nature of Article 14, it seems unlikely that a case like \textcite{sainsbury10} would turn on constitutional property law.

To sum up, German constitutional law serves to create an interesting contrast with English law regarding the question of economic development takings. On the one hand, property appears to be better protected against such takings in Germany, but on the other hand, the extent to which increased protection is offered depends more closely on the social values involved. The German system appears to look more actively at the social function of property for guidance when resolving property disputes, thereby echoing some of the ideas discussed in Chapter \ref{chap:1}. 

In the next section, I will discuss the property clause in the ECHR, which explicitly serves to set up a minimum level of property protection that provides a common standard for all member states, including Germany and the UK.

\section{The Property Clause in the European Convention of Human Rights}

The starting point for property adjudication at the ECtHR is that States have a ``wide margin of appreciation'' with regards to the question of whether or not an interference in property rights is to be considered legitimate in pursuance of the public interest.\footcite[See][54]{james86} This question is thought to depend on democratically determined policies to such an extent that it is rarely appropriate for the Court to censor the assessments made by member states. At the same time, however, the Court has gradually come to take a more active role in assessing whether or not particular instances of interference are ``proportional'' and able to strike a ``fair balance'' between the interests of the public and the interests of the individual property owner.\footnote{See \cite[69]{sporrong82} and \cite[120]{james86}. The standard account of the protection against interference inherent in P1(1) describes it as consisting of three rules. First, there is the rule of {\it legality}, asserting that an interference needs to be authorized by statute. Second, there is the rule of {\it legitimacy}, making clear that interference should only take place in pursuance of a legitimate public purpose. Both of these rules are of little practical significance, however, as the margin of appreciation has been regarded as very wide in regards to both. The third rule is the ``fair balance'' principle, which is applied by the ECtHR in almost all cases when it finds that there has been a violation of P1(1). In the following, I focus only on this rule and on those aspects of it that I think are most relevant to the question of economic development takings. For a more detailed description of P1(1) generally, I refer to \cite{allen05}.} As argued by Tom Allen, this has caused P1(1) to attain a wider scope than what was originally intended by the signatories.\footcite[1055]{allen10}.

In the case law behind this development, the focus has predominantly been on the issue of compensation, with the Court gradually developing the principle that while P1(1) does not entitle owners to full compensation in all cases of interference, the fair balance will likely be upset unless at least some compensation is paid, based on the market value of the property in question.\footnote{See \cite[103]{scordino06}. The case also illustrates that the Court has come to adopt a fairly strict approach to the question of when it is legitimate to award less than full market value.} This focus on compensation has also been reflected in academic work on P1(1), which tends to address proportionality from an economic perspective, by investigating to what extent owners are entitled to compensation based on the market value of their property. Indeed, when considering the best known case law and literature on the subject, one is left with the impression that ``fair balance'' with regards to P1(1) is crucially linked to financial entitlements, primarily used as a standard that can justify a right to compensation that goes beyond what the wording of P1(1) might initially suggest.

In recent case law, however, it has become clear that the fair balance test encompass more than this, since it also gives the Court in Strasbourg occasion to reflect on the social context and purpose of interference, in a manner largely consistent with the social function approach to property. In {\it Chassagnou and others v France} the situation was that property owners were compelled to permit hunting on their land, following compulsory membership in a hunting association which was set up to manage hunting in the local area.\footcite{chassagnou99} They protested this on the grounds that they were ethically opposed to hunting, and the Court agreed that there had been a breach of P1(1). 

In the later case of {\it Hermann v Germany} the circumstances were similar, and the Court followed the precedent set in {\it Chassagnou}, commenting also that they had ``misgivings of principle'' about the argument that financial compensation could provide adequate protection in such a case.\footcite[See][91]{hermann12}  In this way, the hunting cases illustrate that to the ECtHR, the right to property is not seen as a mere financial entitlement. Moreover, the fair balance that must be struck could well pertain to other aspects, such as the owner's right to make use of his property in accordance with his convictions and to take part in decision-making processes regarding how it should be managed.\footnote{The assessment of proportionality should be concrete and contextual, and it is not based on a narrow or formalistic concept of property as dominion. This is demonstrated, for instance, by \cite{chabauty12}. Here the Court found no violation of P1(1) although the facts seemed close to those of {\it Chassagnou}. The case differed, however, in that the owner himself was not opposed to hunting, but wanted to withdraw his land from the hunters' association to enjoy exclusive hunting rights.}

In a different, but related, development, the Court has also adopted a contextual approach in recent cases involving rent control 
schemes and housing regulation. While there are obvious financial interests at stake in such cases, for both landlords and tenants, the Court has looked to the fairness of the underlying regulation more generally, by taking into account the local social, economic and political conditions. Moreover, the Court has not shunned away from using concrete cases as a starting point for providing an assessment of the sustainability of national law as such. In {\it Hutten-Czapska v Poland}, for instance, the Court concluded that the case demonstrated ``systemic violation of the right of property''.\footcite[239]{hutten06}

The case concerned a house that had been confiscated during WW2. After the war, the property was transferred back to the owners, but in the meantime, the ground floor had been assigned to an employee of the local city council. Moreover, the state implemented strict housing regulations during this time, which eventually meant that the applicant's house was placed under direct state management.\footcite[20-31]{hutten06} Following the end of communist rule in 1990, the owners were given back the right to manage their property, but it was still subject to strict regulation that protected the rights of the tenants.\footcite[31-53]{hutten06} In addition to rent control, rules were in place that made it hard to terminate the rental contracts. Hence, it became impossible for the owners to make use of the house themselves, as they wished to do.\footcite[20-53]{hutten06} 

After an in-depth assessment of the relevant parts of Polish law and administrative practice, the Grand Chamber of the ECtHR concluded that there had been a violation of P1(1). Importantly, they did not reach this conclusion by focusing on the house as a source of financial entitlements for the owners. Rather, they focused on the overall character of the Polish system for rent control and housing regulation, as it manifested in the concrete circumstances of the applicant's case. The financial consequences for the owners were considered, as was the financial situation of the tenants.\footcite[60-61]{hutten06} The Court was particularly concerned with the fact that the total rent that could be charged for the house was not sufficient to cover the running maintenance costs.\footcite[224]{hutten06} In particular, it was noted that the consequence of this would be ``inevitable deterioration of the property for lack of adequate investment and modernisation''.\footnote{\cite[224]{hutten06}.}

In the end, the Court concluded that the combination of a rigid rent control system, rules that made it hard for owners to terminate tenancy agreements, and the fact that the State itself had set up these agreements during the days of direct state management, meant that a fair balance had not been struck.\footcite[224-225]{hutten06} The contextual nature of the Court's reasoning is evidenced not only by the extent to which the concrete circumstances are assessed against the goal of fairness, but also by how the Court explicitly places the ``social rights'' of the tenants on equal footing with the property rights of the owners.\footcite[225]{hutten06} 

It is also of interest to note how the Court reasons towards the conclusion that the Polish legal order as such is at fault. In this regard, great weight is placed on the observation that the system suffers from a lack of adequate safeguards to protect owners against imbalances such as those identified in the present case. In particular, the Court comments on ``the absence of any legal ways and means making it possible for them either to offset or mitigate the losses incurred in connection with the maintenance of property or to have the necessary repairs subsidised by the State in justified cases''. Hence, the rent control scheme alone was not the whole problem, the Court also criticized what it saw as a defective way of implementing it.\footcite[224]{hutten06} Moreover, the Court did not censor the political reasoning that motivated Polish housing legislation, but concluded instead that the ``burden cannot, as in the present case, be placed on one particular social group, however important the interests of the other group or the community as a whole''. 

I think this is the most important aspect of the case, pointing to the core function that the ECtHR should embrace more generally. It seems to me, in particular, that objections can be raised against the appropriateness of having the Court in Strasbourg assess concretely what is fair regarding the relationship between owners and tenants in a specific house in Gdynia. Its remoteness to the local conditions, as well as its lack of sensitivity and accountability to locally grounded political processes, suggest that the Court is not ideally placed to carry out the kind of contextual assessment that it prescribes for such cases. In addition, the amount of resources and time needed to independently scrutinize these aspects convincingly risks undermining its ability to deal expediently with its case load. The ECtHR will hardly be able to protect human rights in Europe on a case-by-case basis.

Instead, the aim should always be to get at the systemic features that cause perceived imbalances. As in \textcite{hutten06}, the Court serves its function best when it is able to identify a sense in which the domestic legal order needs to be improved to better comply with human rights standards. This is particularly true when, as in that case, the Court notes that the applicants have insufficient options available for achieving a fair balance by appealing to institutions within the domestic legal order. By demanding {\it institutional} changes, in particular, the Court effectively delegates responsibility for ensuring the kind of fair balance that is required under the ECHR. Moreover, by scrutinizing the procedures and principles that the states apply when fulfilling this duty, it is likely that the Court will still be able to steer and unify the development of the case law. Importantly, they would then be able to do so without having to engage extensively in concrete assessments of fairness. 

Against this, one may argue that the judicial or administrative bodies of the signatory states can easily circumvent their obligations by giving a superficial or biased assessment of the facts in human rights cases, to avoid embarrassment for the state's political or bureaucratic elite. However, this might then be raised as a procedural complaint before the ECtHR, resulting in cases revolving around Articles 6 (fair trial) and 13 (effective remedy).\footnote{I note that this also fits with recent developments at the ECtHR, toward somewhat broader scrutiny under Article 6, see \cite{khamidov07}.}  In this way, the Court can streamline its functions, by always aiming to direct attention at issues that arise at a higher level of abstraction. This, in my view, is desirable. The ECtHR should not aim to micromanage the signatory states, particularly not in relation to a norm such a P1(1), which the Court itself regards as highly dependent on context.

However, the question arises as to what kind of institutions the Court should focus on in its effort to ensure fairness in relation to Convention rights such as property. It is not given, in particular, that directing attention towards domestic judicial bodies is the most appropriate approach. Rather, it seems logical to assume that those institutions most in need of reform will be those that are actually responsible for violations. A possible lack of an effective complaints procedure would be worrying, but hardly as problematic as possible systemic weaknesses that give rise to complaints in the first place. With this change in perspective, the Court can avoid getting stuck in deference to domestic judicial bodies, but still shift attention away from concrete assessment of alleged violations. They can do so, in particular, by concretely and critically assessing those rules and procedures that are identified as causally significant to individual complaints. \footnote{In the future, one might even encounter cases when the Court prefers to remain agnostic about whether a substantive violation occurred, focusing instead on the possible violation inherent in excessive systemic risks and a shortage of adequate safeguards.}

Indeed, I think the case of \textcite{hutten06} is suggestive of a move towards such a perspective. While the Court went into great detail about the facts of the case, it {\it also} looked at the case from an alternative perspective, more in line with the suggestion sketched above. In fact, I think it is likely that the Court will eventually veer even more towards such an approach, while deferring to national judicial bodies when it comes to concrete factual assessments. If not as a result of policy, I imagine this will happen from necessity, at least in relation to rights such as property, which now seem to flood the Court.

One might ask where this would leave the proportionality doctrine. In fact, I think this doctrine still makes good sense when framed in more abstract terms as the question of what kinds of rules, and what kinds of institutions, member states need to put in place to ensure fairness. In \textcite{hutten06}, the Court moved in this direction, when it explained the basic principle as follows:

\begin{quote}
In assessing compliance with Article 1 of Protocol No. 1, the Court must make an overall examination of the various interests in issue, bearing in mind that the Convention is intended to safeguard rights that are “practical and effective”. It must look behind appearances and investigate the realities of the situation complained of. In cases concerning the operation of wide-ranging housing legislation, that assessment may involve not only the conditions for reducing the rent received by individual landlords and the extent of the State’s interference with freedom of contract and contractual relations in the lease market, but also the existence of procedural and other safeguards ensuring that the operation of the system and its impact on a landlord’s property rights are neither arbitrary nor unforeseeable. Uncertainty – be it legislative, administrative or arising from practices applied by the authorities – is a factor to be taken into account in assessing the State’s conduct. Indeed, where an issue in the general interest is at stake, it is incumbent on the public authorities to act in good time, in an appropriate and consistent manner.\footcite[151]{hutten06} 
\end{quote}

I note how the Court builds on the earlier precedent set by cases such as \textcite{sporrong82} and \textcite{james86}. The first half of the quote, therefore, stresses that the Court itself must ``look to the realities of the situation''. However, in clarifying what is meant by this, the Court goes on to emphasize procedural aspects. In particular, it is made clear that the Court regards such aspects as an integral part of those ``realities'' that need to be assessed. Indeed, the Court even makes specific reference to the importance of several values that arise in the context of administrative law, such as predictability and effectiveness.

The passage above was subsequently quoted in {\it Lindheim and others v Norway}. In this case, the applicants complained that their rights had been violated by a recent Norwegian act that gave lessees the right to demand indefinite extensions of ground leases on pre-existing conditions.\footcite[119]{lindheim12}  In the end, the Court concluded that there had indeed been a breach of P1(1). Interestingly, they engaged in the same form of assessment that they had adopted in \textcite{hutten06}. They held, in particular, that it was the Act itself which was the underlying source of the violation, not merely its concrete application against the applicants. Hence, the Court did not only award compensation, it also ordered that general measures had to be taken by the Norwegian State to address the structural shortcomings that had been identified.

In this case, the Court also commented that its decision should be regarded in light of ``jurisprudential developments in the direction of a stronger protection under Article 1 of Protocol No. 1''.\footcite[135]{lindheim12} However, in light of the change in perspective that accompanies this development, it is interesting to ask in what sense exactly the protection is stronger. It is not {\it prima facie} clear, in particular, that the Court's remark should be read as a statement expressing a change in its understanding of the content of individual rights under P1(1). Rather, I am inclined to read it as a statement to the effect that the Court now assumes it has greater authority to address structural problems under that provision. This authority, in particular, is now seen to extend also to the fair balance requirement, not only the (much more narrowly drawn) legality and legitimacy rules. In effect, this also leads to stronger protection for individuals, since it allows the Court to conclude that a violation has occurred due to ``structural unfairness'', even when it is not possible to trace this back to any ``flawed'' decision that directly targets the applicants.

What is the relevance of all this to the issue of economic development takings? A great deal, I think. Indeed, I am struck by how the reasoning of the ECtHR in recent cases on hunting and rent control mirrors the kind of reasoning that Justice O'Connor engaged in when she considered {\it Kelo}. The emphasis is on structural aspects and fairness, but the considerations made in this regard are grounded on what the facts of the concrete case reveal about the rules and procedures involved. In this way, the contextual approach to property gains focus without losing its bite. Individual entitlements play a relatively minor role, the assessment is highly relational, focusing on where the system involved places different legal persons in relation to one another. The crux of arguments used to conclude violation is the observation that the system offends against the role that owners {\it should} occupy in order to be able to meet those obligations and exercise those freedoms that society normally regards as inherent to the form of property that they possess.

In this case, interference is not only unfair, it is also a failure of governance, and a structural inconsistency. In the case of \textcite{hutten06}, this boiled down to the observation that the system which had led to the complaint sought to resolve problems in the Polish housing sector in a manner that placed the burden ``on one particular social group'', namely the owners.\footcite[225]{hutten06} This conclusion was backed up by the concrete observation that the rules and procedures in place meant that owners who were expected to maintain their properties in good condition for their tenants were in fact prevented from doing so because they were not permitted to charge rents that would cover the costs.

In the case of {\it Kelo}, Justice O'Connor argued in  a similar fashion when she concluded that the system which had led to the decision to condemn Suzanne Kelo's house was likely to function so as to systematically ``transfer property from those with fewer resources to those with more''. This conclusion was backed up by the observation that the beneficiary in {\it Kelo} was a multi-billion dollar commercial company that had been allowed to take Kelo's home because this would lead to ``economic development''. To Justice O'Connor, there was little doubt that this could become a general pattern, if safeguards were not in place. Indeed, it must be presumed that a multi-million dollar company is always in a better position than a homeowner when it comes to arguing that  ``economic development'' will result from their ownership. More subtly, her opinion also hinted at the inconsistency involved in asserting abstractly that economic development would benefit the community indirectly, all the while the development would if fact require razing it.

To conclude, I think the ECtHR is more likely to approach a case like {\it Kelo} in the manner Justice O'Connor did. Whether they would reach the same result seems more uncertain, particularly since confidence in states' ability and willingness to regulate private-public partnerships might be higher in Europe. However, it seems unlikely that the ECtHR would follow the majority in {\it Kelo}, by simply deferring to the determinations made by the granting authority. Moreover, with the recent change in perspective towards a more structural assessment of property institutions at the ECtHR, it seems that Justice O'Connor's predictions about the ``fallout'' of the {\it Kelo} decision would likely strike a cord with the justices at Strasbourg. 

\section{The US perspective on economic takings}\label{sec:us}

I now consider US law in more depth. First, I track the development of the case law on the public use restriction in the Fifth Amendment and in various state constitutions, from the early 19th Century up to the present day.\footnote{The public use clause in the US constitution was not held to apply to state takings until the late 19th Century, see \cite{chicago97}.} Many writers assert that the 19th and early 20th Century was characterized by a ``narrow'' approach to public use which eventually gave way to a broader conception.\footnote{See, e.g., \cite[483]{walt11}; \cite[203-204]{allen00}. For a more in-depth argument asserting the same, see \cite{nichols40}.} Against this, I argue that it is more appropriate to think of this period as one when courts adopted a {\it broad} approach to judicial scrutiny of the takings purpose at state level. Importantly, I also argue that while different state courts expressed different theoretical views on the meaning of ``public use'', there was a growing consensus that the approach to judicial scrutiny should be contextual, focused on weighing the rationale of the taking against the concrete social, political and economic circumstances of the local area.\footnote{A summary of state case law that supports this view is given in the little discussed Supreme Court case of \cite{hairston08}.}  In particular, I argue that early state courts did not focus as much on the exact wording of the constitutional property clause as many later commentators have suggested.

I go on to show that the doctrine of deference that was developed by the Supreme Court early in the 20th Century was directed primarily at state courts, not state legislatures and administrative bodies.\footnote{See \cite{vester30} (echoing and citing \cite{hairston08}).} I then present the case of {\it Berman}, arguing that it was a significant departure from previous case law.\footcite{berman54} After {\it Berman}, deference was suddenly taken to mean deference to the (state) legislature, so there would be little or no room for judicial review of the takings purpose. I go on to present the subsequent developments at state level, characterized by increasing worry that the eminent domain power could be abused by powerful commercial actors. I discuss the case of {\it Poletown}, where a neighborhood of about 1000 homes was razed to provide General Motors with land to assemble a car factory.\footcite{poletown81} I link this to the subsequent controversy that arose over {\it Kelo}, suggesting that it should be seen as the eventual backlash of {\it Berman}, a consequence of abandoning the contextual approach to public use in favor of an almost absolute rule of deference.

After the historical overview, I go on to briefly present the vast amount of research that has targeted economic takings in the US after {\it Kelo}. I give special attention to writers that propose new legitimacy-enhancing institutions for facilitating economic development of jointly owned land. I focus on two proposals in particular, targeting compensation and participation respectively.\footcite{lehavi07,heller08} These proposals will serve as important reference points later on, when I consider the Norwegian appraisal  and land consolidation courts in Chapters 4 and 5.

\section{The history of the public use restriction}\label{sec:hop}

Going back to the time when the Fifth Amendment was introduced, there is not much historical evidence explaining why the takings clause was included in the bill of rights, and little in the way of guidance as to how it was originally understood. James Madison, who drafted it, commented that his proposals for constitutional amendments were intended to be uncontroversial to Congress.\footnote{See letters from Madison to Edmund Randolph dated 15 June 1789 and from Madison to Thomas Jefferson dated 20 June 1789, both included in \cite{madison79}.}  Hence, it is natural to regard it as a codification of an existing principle, rather than a novel proposal. Indeed, several State constitutions pre-dating the Bill of Rights also included takings clauses, and they all seem to be largely based on a codification of principles from English Common law.\footcite[See][299]{johnson11}

As we discussed in subsection \ref{sec:england} above, the typical English attitude from this time, which was also reflected in the law, held private property in very high regard. On this background it is not surprising that Madison regarded the property clause as an uncontroversial amendment.\footnote{Indeed, early American scholars also emphasized the importance of private property. For instance, in his famous {\it Commentaries}, James Kent described the sense of property as ``graciously implanted in the human breast'' and declared that the right of acquisition ``ought to be sacredly protected'', \cite[see][257]{kent27}.} Its importance may in fact have been greater as a legitimizing force, increasing confidence in the regulatory power of the newly established state by setting up clear parameters for the exercise of that power.  However, while the principle itself was regarded as self-evident, it was never clear what it would mean in practice, particularly in cases when takings where challenged on the basis that they were not for a ``public use'''.\footcite[See][317]{johnson11} 

There are two points that I would like to record about the early common law in the US  in this regard. First, the distinction between public use and public purpose does not appear to have been considered sharp. In his {\it Commentaries}, James Kent first makes clear that the power of eminent domain is for ``public use, and public use only", but then goes on to qualify this by stating that a taking which served a ``purpose not of a public nature'' would be unconstitutional.\footcite[See][275-276]{kent27}  He does not address this limitation in any detail, however, suggesting that it was not the subject of much debate at this time. To the founders, it seems that the right to compensation was considered the more important principle, something that is also reflected in the {\it Commentaries}.\footnote{James Kent held it to be  ``founded in natural equity'' and described it as an ``acknowledged principle of universal law'', \cite[see][276]{kent27}.} The public use limitation was probably taken for granted as a matter of principle, while it had not yet proved problematic as a matter of practical adjudication. Moreover, it appears to have been accepted that takings which clearly benefited the public would be legitimate regardless of whether or not the property was physically put to use by the public.\footcite{johnson11}

An interesting early illustration of how courts approached takings controversies at this time can be found in {\it Stowell v Flagg}, a Massachusetts case from 1814. In this case, a landowner complained that his land had been flooded by a mill and sought a remedy in common law. The mill owner protested, however, since he was entitled to flood the land according to a special mill act, which allowed him to exercise the power of eminent domain to gain the right to flood his neighbor (provided statutory compensation was paid). The focus in the case was on whether a common law claim for damages could still be made, irrespectively of the act's clear intention to deprive the affected neighbors of this opportunity. Hence, the court implicitly dealt with the legitimacy of the mill act itself, and they actively engaged with the public use requirement in the state constitution when making their assessment.\footcite{stowell14} In the end, they found that the act was legitimate, and they highlighted the purpose of the interference, commenting that ``these mills, early in the settlement of this country, were of great public necessity and utility''.\footcite[366]{stowell14} 

At the same time, however, the court had misgivings about how the act had come to be applied and expressed concern that ``the legislature, as well as the courts of law in this state, seem to have been disposed rather to enlarge, than to curtail, the power of mill owners''.\footcite[366]{stowell14} Still, after noting that affected land owners were entitled to compensation under the act, the court concluded that the act had to be observed and that it precluded any claims for damages under common law. Hence, the case is an early example of judicial deference to the legislature in takings cases, while also illustrating that the public use requirement was beginning to emerge as a potentially problematic issue in its own right. The presiding judge stated that he could not help thinking that the statute was ``incautiously copied from the ancient colonial and provincial acts'', but still held in favor of the mill owner,  concluding that ``as the law is, so must we declare it''.\footcite[368]{stowell14}

While judicial deference was recognized as a guiding principle early on in US takings law, it is important to note in this regard that eminent domain was seldom used in a way that would raise serious controversy. English common law, while lacking clearly defined constitutional safeguards, was, as we have already mentioned, based on a fundamentally cautious attitude, ensuring that the power would typically only be used as a last resort. As Professor Meidinger notes, the British were never really charged with abuse of eminent domain, and private property tended to be respected, also in the colonies.\footcite[17]{meidinger80} This undoubtedly influenced early US law. Indeed, the importance of constitutional limits on the taking power was made clear by the Supreme Court early on, as a matter of principle.\footnote{As reflected in {\it de dicta} comments from {\it Calder v Bull} and {\it Vanhorne’s Lessee v Dorrance}, see \cite[388]{calder98}; \cite[310]{vanhorne95}.} Hence, the relative lack of judicial interest in the question of legitimacy does not appear to have been due to a broad view on the scope of eminent domain, but an established practice of narrow use of that power, inherited from the English.

%The Legislature declare and enact, that such are the public exigencies, or necessities of the State, as to authorise them to take the land of A. and give it to B.; the dictates of reason and the eternal principles of justice, as well as the sacred principles of the social contract, and the Constitution, direct, and they accordingly declare and ordain, that A. shall receive compensation for the land. But here the Legislature must stop; they have run the full length of their authority, and can go no further: they cannot constitutionally determine upon the amount of the compensation, or value of the land. Public exigencies do not require, necessity does not demand, that the Legislature should, of themselves, without the participation of the proprietor, or intervention of a jury, assess the value of the thing, or ascertain the amount of the compensation to be paid for it. This can constitutionally be effected only in three ways.
%1. By the parties that is, by stipulation between the Legislature and proprietor of the land.
%2. By commissioners mutually elected by the parties.
%3. By the intervention of a Jury.

The traditional attitude to eminent domain would eventually give way to a more expansive approach, however. This development became particularly marked during the period of great economic expansion and industrialization in the mid to late 19th century, when eminent domain was increasingly used to benefit (privately operated) railroads, hydroelectric projects, and the mining industry.\footcite[23-33]{meidinger80} During this time, it also became increasingly common for landowners to challenge the legitimacy of takings in court, undoubtedly a consequence of the fact that eminent domain was now used more widely, for new kinds of projects.\footcite[24]{meidinger80} Controversy arose particularly often with respect to mill acts.\footnote{\cite[24]{meidinger80}. See also \cite[306-313]{johnson11} and \cite[251-252]{horwitz73}.} Such acts were found throughout the US, and many of them dated from pre-industrial times when mills were primarily used to serve the needs of self-sufficient agrarian communities.\footnote{A total of 29 states had passed mill acts, with 27 still in force, when a list of such acts was compiled in \cite[17]{head85}. According to Justice Gray, at pages 18-19 in the same, the ``principal objects'' for early mill acts had been grist mills typically serving local agrarian needs at tolls fixed by law, a purpose which was generally accepted to ensure that they were for public use.}  However, following economic and technological advances, acts that were once used to facilitate the construction of grist mills would increasingly also be relied on by developers wishing to harness hydropower for manufacturing, and eventually, for hydroelectric projects.\footnote{See, e.g., \cite[18-21]{head85} and \cite[449-452]{minn06}.}

The mill acts typically contained provisions that enabled the mill developer to condemn both property needed for the construction itself as well as the right to damage surrounding land by flooding or deprivation of water. Such takings became increasingly controversial, however, and many legitimacy cases came before state courts in the late 19th and early 20th century. In the next subsection I present some of these cases, to shed light on how states courts developed their own approach to the question of legitimacy of takings.

\subsection{Legitimacy in state courts}\label{subsec:state}

In the mil cases, we find the first clear evidence of how the public use requirement was put to use to enable state courts to scrutinize the legitimacy of takings. Generally speaking, when a court upheld an interference in private property, it would place decisive weight on the broader purpose of interference, typically by arguing that economic ripple effects ensured that the mill was in the public interest even if the public would not literally make use of it.\footnote{See, e.g., \cite{hazen53,scudder32,boston32}. A more comprehensive list of cases adopting a broad view can be found in \cite[617]{nichols40}.} By contrast, when a court decided that an interference was unconstitutional (with respect to state constitutions), it would often focus on the use made of the mill, arguing that it did not directly benefit the public in the sense required by the public use restriction.\footnote{See, e.g., \cite{sadler59,ryerson77,gaylord03,minn06}. A more comprehensive list can be found in {\it Public benefit or convenience as distinguished from use by the public as ground for the exercise of the power of eminent domain} 54 ALR 7 (American Law Reports, 1928).} For a time, a doctrine which sought to distinguish between takings for public use and takings for a public purpose, played quite a significant role in many states. Under this doctrine, only those takings that were deemed to qualify as public use takings under a narrow view of that term would be upheld.\footnote{Professor Nichols goes as far as to conclude that this emerged as the ``majority'' opinion on public use, see \footcite[617-618]{nichols40}. But contrast this with \cite{berger78} and \cite[24]{meidinger80}, who argue that the narrow view was only dominant in a handful of states, led by New York.}

%For instance, in the case of {\it Gaylord v. Sanitary Dist. of Chicago}, the Supreme Court of Illinois held the state Mill Act to be unconstitutional, as it was not limited to traditional flour mills. In doing so, the court observed that public use was ``something more than a mere benefit to the public''.\footcite[524]{gaylord03} Similar sentiments were expressed in other decisions striking down uses of eminent domain for mill construction, for instance in Vermont, Michigan and New York.\footnote{References.}

It is tempting to associate the narrow view on public use with a more restrictive attitude towards the use of eminent domain. Similarly, it is natural to assume that a broad view on public use suggests a more relaxed attitude. To some extent, the primary sources warrant this; unsurprisingly, those who endorsed a broad view on the public use question also often spoke in favor of judicial deference in legitimacy cases, while those endorsing a narrow view tended to emphasize the importance of constitutional safeguards against abuse of eminent domain. However, it seems that both groups were quite heterogeneous and that differences of opinion about the public use requirement did not necessarily reflect any deep ideological divisions.

It is clear, for instance, that many of the courts which favored a broad interpretation of public use still viewed the constitutional limitation on the takings power as an important safeguard, not only as a guarantee for compensation but also as a restriction on the purpose of takings. Indeed, it seems that most late 19th Century Courts, including those that upheld economic takings, were influenced by the growing body of case law across the US that actively scrutinized takings, sometimes striking them down. In particular, it seems that the strict deferential view was largely abandoned in economic takings cases during this period. Deference to the legislature still played an important role and was typically called on as an important argument in takings cases. However, it became much more common to discuss legitimacy also in terms of substantive arguments, by directly addressing the context and circumstances of the taking complained of. I believe this is an important insight to record about the case law from this period; despite differences of opinion about the meaning of public use, a consensus appears to have emerged that judicial review of legitimacy was appropriate and important in economic takings cases.

A good example is the case of {\it Dayton Gold \& Silver Mining Co. v. Seawell}, concerning a Nevada Act which stipulated that mining was a public use for which the power of eminent domain could be exercised to acquire additional rights needed to facilitate extraction.\footcite{seawell76} The Supreme Court of Nevada decided that the Act was constitutional and adopted a broad understanding of the property clause in the Nevada constitution.\footnote{Nev Const Art 8 § 1.} Interestingly, it argued for this interpretation partly on the basis that it would provide {\it better} protection for landowners:

\begin{quote}
If public occupation and enjoyment of the object for which land is to be condemned furnishes the only and true test for the right of eminent domain, then the legislature would certainly have the constitutional authority to condemn the lands of any private citizen for the purpose of building hotels and theaters. [...] Stage coaches and city hacks would also be proper objects for the legislature to make provision for, for these vehicles can, at any time, be used by the public upon paying a stipulated compensation. It is certain that this view, if literally carried out to the utmost extent, would lead to very absurd results, if it did not entirely destroy the security of the private rights of individuals. Now while it may be admitted that hotels, theaters, stage coaches, and city hacks, are a benefit to the public, it does not, by any means, necessarily follow that the right of eminent domain can be exercised in their favor.\footcite[410-411]{seawell76}
\end{quote}

The quote shows that a broad understanding of ``public use'' need not be synonymous with a less cautious attitude to abuse of the takings power. Indeed, while the Court decided to uphold the Act, it did so only after a very careful assessment of both legal arguments and factual circumstances. In particular, the Court considered the importance of mining, concluding that it was the ``greatest of the industrial pursuits'' in the state, and that all other interests were ``subservient'' to it.\footcite[409]{seawell76} Moreover, the Court commented that the benefits of the mining industry was ``distributed as much, and sometimes more, among the laboring classes than with the owners of the mines and mills''.\footcite[409]{seawell76}

This shows that the Court actively engaged with the purpose of the Act, thoughtfully assessing it against the constitution. Importantly, it did not do so in isolation, as a linguistic exercise or by attempting to recreate its ``original intent''. Rather, the court approached the constitutional safeguard by making detailed references to the prevailing social and economic conditions in the state of Nevada. The Court noted the importance of deference to the legislature on matters of policy, but it did so only after it had satisfied itself that the Act could be ``enforced by the courts so as to prevent its being used as an instrument of oppression to any one''.\footcite[412]{seawell76} More generally, the court commented as follows on the public purpose test that had to be performed in takings cases, elucidating on the principles on which it should be founded:

\begin{quote}
 Each case when presented must stand or fall upon its own merits, or want of merits. But the danger of an improper invasion of private rights is not, in my judgment, as great by following the construction we have given to the constitution as by a strict adherence to the principles contended for by respondent.\footcite[398]{seawell76}
\end{quote}

In light of this, {\it Dayton Gold \& Silver Mining Co. v. Seawell} must be regarded as an early example of a {\it contextual} approach to legitimacy, characterized by the willingness of the Court to engage in a fairly detailed analysis of the concrete circumstances and consequences of takings. A formalistic approach based on the phrase ``public use'' was abandoned, but not in favor of general deference. Rather, a more nuanced view was adopted, to respect the idea that the legislature should have the final say on policy while also recognizing that courts should play a crucial role in protecting citizens from abuse of the takings power. 

The case is not unique, but rather exemplifies the type of reasoning that was used in economic takings cases at this time. Interestingly, many common elements exist between courts that upheld and struck down such takings, irrespectively of whether or not they subscribed to a narrow or broad view on the public use test. One example is {\it Ryerson v. Brown}, a case often cited as an authority in favor of a narrow view.\footcite{ryerson77} Here the Supreme Court of Michigan explicitly qualifies its decision by stating that it is ``not disposed to say that incidental benefit to the public could not under any circumstances justify an exercise of the right of eminent domain'', hardly a clear endorsement of the narrow rule. The case concerned the constitutionality of a mill act, and while the court argues that public use should be taken to mean ``use in fact'', it is clear that ``use'' is understood rather loosely, not literally as physical use of the property that is taken.\footnote{The court explains its stance on the public use restriction by stating (emphasis added) ``it would be essential that the statute should require the use to be public in fact; in other words, that it should contain provisions entitling the public to {\it accommodations}.'' The court continues with an illustrative example: ``A flouring mill in this state may grind exclusively the wheat of Wisconsin, and sell the product exclusively in Europe; and it is manifest that in such a case the proprietor can have no valid claim to the interposition of the law to compel his neighbor to sell a business site to him, any more than could the manufacturer of shoes or the retailer of groceries. Indeed the two last named would have far higher claims, for they would subserve actual needs, while the former would at most only incidentally benefit the locality by furnishing employment and adding to the local trade''. See \cite[336]{ryerson77}.} Moreover, when clarifying its starting point for judicial scrutiny of mill acts, the court explains that ``in considering whether any public policy is to be subserved by such statutes, it is important to consider the subject from the standpoint of each of the parties''. Following up on this with regards to the act in question, the court finds that `` the power to make compulsory appropriation, if admitted, might be exercised under circumstances when the general voice of the people immediately concerned would condemn it''. After considering this and other possible consequences of mill development under the act, the court eventually declares it to be unconstitutional, summing up its assessment as follows: ``What seems conclusive to our minds is the fact that the questions involved are questions not of necessity, but of profit and relative convenience''.\footcite[336]{ryerson77}

Hence, far from nitpicking on the basis of the public use phrase, the court adopts a contextual approach to takings that is in fact rather similar to the approach of {\it Dayton Gold \& Silver Mining Co. v. Seawell}. The outcome it different, but it is also based on a different assessment of the context and the consequences of the takings complained of. Importantly, the case does not rest on any {\it a priori} assumption that economic takings of the kind in question could not meet a public use test -- no general rule is relied on at all. Hence, it is somewhat strange that later commentators have focused on the case for its comments on public use rather than its broad, albeit perhaps somewhat conservative, assessment of legitimacy. 

Many of the important cases from the late 19th Century, on both sides of the public use debate, shares many crucial features with the two cases discussed above.\footnote{See, e.g., \cite{scudder32} (Eminent domain power upheld, but said: ``The great principle remains that there must be a public use or benefit. That is indispensable. But what that shall consist of, or how extensive it shall be to authorize an appropriation of private property, is not easily reducible to a general rule. What may be considered a public use may depend somewhat on the situation and wants of the community for the time being.''), \cite{fallsburg03} (Eminent domain struck down, on holding that ``the private benefit too clearly dominates the public interest to find constitutional authority for the exercise of the power of eminent domain''), \cite[538]{board91} (Eminent domain struck down, qualified by ``not only must the purpose be one in which the public has an interest, but the state must have a voice in the manner in which the public may avail itself of that use'').} In my opinion, this points to an interesting alternative perspective on legitimacy adjudication from this time. Some commentators describe the case law as chaotic, with competing conceptions of constitutional limits competing for dominance.\footcite{berger78,meidinger80}. I think this is more accurate than saying that a narrow interpretation of public use developed as a general rule. However, I also find evidence that there was in fact a broad consensus in this period regarding the need for special judicial scrutiny of economic development cases. State courts widely engaged in contextual assessment of legitimacy, and they were conscious of the special challenges that arose in a time when eminent domain was being used to facilitate economic expansion that would benefit specific commercial actors. Differences of opinion about public use terminology was an important aspect of this, but it was rarely considered in isolation from other aspects. On a deeper lever, the fact that the public use debate was regarded as important in the first place clearly suggests that deference to the legislature was not held to be an exhaustive answer to the question of legitimacy. This, in my opinion, is an important observation which appears to have been somewhat overlooked in the literature. 

It is an observation that I think is relevant not only in relation to state law, but also when considering the takings doctrine that was later developed by the Supreme Court. While the narrow view of public use was indeed losing ground at the beginning of the 20th Century, the doctrine of extreme deference that was about to be adopted at the federal level represents a largely new development. The new deference was not originally directed at the legislature, in particular, but primarily towards the judiciary at the state level. Hence, it represent a development that is in some sense incomparable to the earlier case law from the states. The balance of power between states and the federal government also played an important role, which should not be overlooked.

\subsection{Legitimacy as discussed in the Supreme Court}\label{subsec:US}

Initially, the Supreme Court held that the takings clause in the US Constitution did not apply to state takings at all.\footcite{barron33} Federal takings, on the other hand, were of limited practical significance since the common practice was that the federal government would rely on the states to condemn property on their behalf.\footcite[30]{meidinger80}. This changed towards the end of the 19th Century, particularly following the decision in {\it Trombley v. Humphrey}, where the Supreme Court of Michigan struck down a taking that would benefit the federal government.\cite{trombley71} Not long after, in 1875, the first Supreme Court adjudication of a federal taking case occurred, marking the start of the development of the Supreme Court's own doctrine on public use and legitimacy.\footcite{kohl75} Eventually, in 1897, the Court would also hold that state takings could be scrutinized under the takings clause of the constitution.\footcite{chicago97} This was a development that can be traced to the passage of the Fourteenth Amendment to the Constitution after the civil war, concerning due process.\footcite{johnson11}. Indeed, some early Supreme Court cases dealing with state takings were adjudicated against the due process clause directly.\footnote{See, e.g., \cite{head85}.}

After the Supreme Court started developing its own case law on the legitimacy issue, the deferential stance soon became entrenched. As argued by Professor Horwitz, the mid to late 19th Century was the period in US history when control over property was transferred on a massive scale from agrarian communities to various agents of industrial expansion.\footcite{horwitz73} Moreover, it was a period of great optimism about the ability of {\it laissez faire} capitalism to ensure progress and economic growth. This was also reflected in the case law on eminent domain, particularly as developed by the Supreme Court. A particularly clear expression of this can be found in {\it Mt. Vernon-Woodberry Cotton Duck Co v Alabama Interstate Power Co}.\footcite{vernon16}  This case dealt with the legitimacy of a condemnation arising from the construction of a hydropower plant, which the Alabama Supreme Court had upheld against claims that it was unconstitutional under the constitution of Alabama. The presiding judge held that it was valid using quite brisk language:

\begin{quote}The principal argument presented that is open here, is that the purpose of the condemnation is not a public one. The purpose of the Power Company's incorporation, and that for which it seeks to condemn property of the plaintiff in error, is to manufacture, supply, and sell to the public, power produced by water as a motive force. In the organic relations of modern society it may sometimes be hard to draw the line that is supposed to limit the authority of the legislature to exercise or delegate the power of eminent domain. But to gather the streams from waste and to draw from them energy, labor without brains, and so to save mankind from toil that it can be spared, is to supply what, next to intellect, is the very foundation of all our achievements and all our welfare. If that purpose is not public, we should be at a loss to say what is. The inadequacy of use by the general public as a universal test is established. The respect due to the judgment of the state would have great weight if there were a doubt. But there is none.\footcite[]{vernon16}
\end{quote}

The quote serves as an indication of how deference was fast gaining ground, without yet being established doctrine. On the one hand, the Court stresses that deference to the {\it state} judgment (rather than the judgment of the legislature) should be given great weight in legitimacy cases. On the other hand, it prefers to conclude on the basis of its own assessment of the purpose of the taking. This assessment, however, is not particularly grounded in the circumstances on the ground in Alabama, being based rather on sweeping assertions about the ``organic relations of modern society'' and the desire to ``save mankind from toil that it can be spared''. 

This judgment, from 1916, was given during the so-called {\it Lochner} era of jurisprudence in the US, when the Supreme Court  would famously engage in active censorship of regulation that was meant to promote greater social and economic equality.\footcite{cohen08} In particular, much case law from this period witnesses to a general lack of deference. Hence, it is not unexpected to find that public use cases decided on the basis of substantive arguments. However, it is rather more surprising to find that deference actually played an increasingly important role in takings cases.\footnote{The {\it Lochner} era in general was characterized by courts engaging in censorship of state regulation, but this general tendency is not well reflected in how eminent domain law developed over the same period. This is interesting, as it points to the shortcoming of another commonly held view on property protection, namely that it largely serves the interests of property-owning elites, to the detriment of regulatory efforts to promote social equality. The cases through which {\it Lochner} era courts developed the deferential stance suggest a different interpretation; those who benefited most directly from takings in these cases were commercial interests, not vulnerable groups of society. Moreover, they benefited from acquiring land rights from members of agrarian communities, not from the elites. Hence allowing such takings to go ahead was no affront to the ideology of progress through {\it laissez faire} capitalism, quite the contrary. In particular, if it is true as many have argued, that the {\it Lochner} courts were ideologically committed to the promotion of unrestrained capitalism, there was little reason for them to oppose expansion of eminent domain into the commercial arena: those who would be likely to benefit were market actors who were proposing large scale commercial development projects. Indeed, the case law from this period makes it natural to argue that the deferential stance developed primarily to cater to the needs of the capitalists, under the perceived view that they represented the class which would bring progress and prosperity to the nation as a whole.} As early as { \it United States v. Gettysburg Electric Railway Co.}, a case from 1896, deference was described as a fundamental guiding principle, which should be adhered to except in very special circumstances.\footcite{gettysburg96} In particular, Justice Peckham lended his support to the following deferential stance on the public use test:

\begin{quote}
It is stated in the second volume of Judge Dillon's work on Municipal Corporations (4th Ed. § 600) that, when the legislature has declared the use or purpose to be a public one, its judgment will be respected by the courts, unless the use be palpably without reasonable foundation. Many authorities are cited in the note, and, indeed, the rule commends itself as a rational and proper one.\footcite[680]{gettysburg96}
\end{quote}

The case did not turn on the public use issue, however, as the condemned land would be used for battlefield memorials at Gettysburg, Pennsylvania, clearly a public use. In addition, the case concerned a federal takings, authorized by Congress. In later cases, the deferential stance was not adopted in cases originating from the states. As late as in 1930, in {\it Cincinatti v Vester}, the Supreme Court commented that the ``‘It is well established that, in considering the application of the Fourteenth Amendment to cases of expropriation of private property, the question what is a public use is a judicial one".\footcite[447]{vester30} In this judgment, Chief Justice Hughes also describes in more depth how the judicial assessment of the public use question should be carried out, echoing the contextual approach that had been developed in case law from the states.

\begin{quote}
In deciding such a question, the Court has appropriate regard to the diversity of local conditions and considers with great respect legislative declarations and in particular the judgments of state courts as to the uses considered to be public in the light of local exigencies. But the question remains a judicial one which this Court must decide in performing its duty of enforcing the provisions of the Federal Constitution.\footcite[447]{vester30}
\end{quote}

In {\it Hairston v. Danville \& W. R. Co.}, the same idea was expressed even more clearly by Justice Moody, who surveyed the state case law and declared that ``The one and only principle in which all courts seem to agree is that the nature of the uses, whether public or private, is ultimately a judicial question.''\footcite[606]{hairston08} He continued by describing in more depth the typical approach of the state courts in determining public use cases:

\begin{quote}
The determination of this question by the courts has been influenced in the different states by considerations touching the resources, the capacity of the soil, the relative importance of industries to the general public welfare, and the long-established methods and habits of the people. In all these respects conditions vary so much in the states and territories of the Union that different results might well be expected.\footcite[606]{hairston08}
\end{quote}

Justice Moody goes on to give a long list of cases illustrating this aspect of state case law, showing how assessments of the public use issue is inherently contextual and varies from state to state.\footcite[607]{hairston08} He then cites three further Supreme Court cases, pointing out that all of them express similar sentiments of support for state case law on this issue.\footnote{{\it Falbrook, Clark} and {\it Strickley}} Following up on this, he points out that ``no case is recalled'' in which the Supreme Court overturned ``a taking upheld by the state {\it court} as a taking for public uses in conformity with its laws'' (my emphasis). After making clear that situations might still arise where the Supreme Court would not follow state courts on the public use issue, Justice Moody goes on to conclude that the cases cited `` show how greatly we have deferred to the opinions of the state courts on this subject, which so closely concerns the welfare of their people''.\footcite[606]{hairston08}

I believe {\it Hairston} is an important case for two reasons. First, it makes clear that initially, the deferential stance in cases dealing with state takings was largely directed at the state courts rather than the state legislature. Second, it demonstrates federal recognition of the fact that a consensus had emerged in the states, whereby scrutiny of the public use determination was consistently regarded as a judicial task.\footnote{Indeed, {\it Hariston} provides the authority for {\it Vester} on this point. See \cite[606]{vester30}.} Moreover, the Court clearly looked favorably on the contextual approach adopted in such cases, whereby state courts would look to the concrete circumstances of the individual takings and acts complained of. The Court's approval of this tradition, in particular, is explicitly given as the reason for adopting a deferential stance. Put simply, the judicial test provided at state level was held to be of such high quality that there was little use for further scrutiny; a deferential stance was assumed, but made contingent on the fact that state courts would provide the required judicial scrutiny.

Despite this, {\it Hairston} would later be cited as an early authority in favor of almost unconditional deference in {\it US ex rel Tenn Valley Authority v Welch}.\footcite[552]{welch46} This case concerned a federal taking and it cited {\it US v Gettysburg Electric R Co} as an authority in favor of strong deference with regards to the public use limitation.\footcite{gettysburg96} However, the Court also paused to note that the later case of {\it City of Cincinnati v Vester} expressed the opposite view, that the public use test was a judicial responsibility.\footcite{vester30} In a very selective citation, the Court then purports to resolve this tension by quoting {\it Hairston} and the observation made there that the Supreme Court had never overruled the state courts in takings cases. Effectively, the importance of judicial scrutiny is thereby downplayed, although as we saw, the rationale behind {\it Hairston} was that state courts already offered high-quality judicial scrutiny of the public purpose.

{\it Welch} is particularly important because it is used as an authority in the later case of {\it Berman v Parker}, which endorses almost complete deference to the legislature regarding the public use issue.\footcite[32]{berman54} This case concerned condemnation for redevelopment of a partly blighted residential area in the District of Colombia, which would also condemn a non-blighted department store. In a key passage, the Court states that the role of the judiciary in scrutinizing the public purpose of a taking is ``extremely narrow''.\footcite[32]{berman54} The Court provides only two citations for this claim, one of them being {\it Welch}. The other case, {\it Old Dominion Land Co v US}, concerned a federal taking of land on which the military had already invested large sums in buildings.\footnote{The Court commented on the public use test by saying that ``there is nothing shown in the intentions or transactions of subordinates that is sufficient to overcome the declaration by Congress of what it had in mind. Its decision is entitled to deference until it is shown to involve an impossibility. But the military purposes mentioned at least may have been entertained and they clearly were for a public use''. See \cite[66]{dominion25} Hence, the Court took the view that courts should be cautious in second-guessing the intentions of Congress on the basis of what its subordinates had subsequently done and said. This is far from a general deferential stance on public use, and no cases are cited at all, suggesting further that the Court did not think its remarks would be of general significance. Still, a partial quote, used to substantiate  broad deference to the legislature (not only Congress, but also the states) except when it involves an ``impossibility'', has become commonplace. In particular, such a quote was used in the much discussed \cite[240]{midkiff84}.}
In my view, both cases are weak authorities for prescribing general deference regarding public use. Moreover, both cases are concerned with federal takings only, while in {\it Berman} the Court explicitly says that deference is due in equal measure to the state legislature.\footcite[32]{berman54} It is possible to see this as a {\it dictum}, since the District of Columbia is governed directly by Congress, but it is a passage that has had a great impact on future cases. In effect, {\it Berman} caused departure from a significant and consistent body of case law which recognized the important role of the judiciary, at state level, in assessing the purported public purpose of takings. It did so, moreover, without engaging with any of these cases at all.

In {\it Hawaii Housing Authority v Midkiff}, the Supreme Court further entrenched the principles of {\it Berman}, in a case where the state of Hawaii had made used of the takings power to break up an oligopoly in the housing sector.\footcite{midkiff84}  However, the fact that the case made it to the Supreme Court is perhaps suggestive of an increase in the level of worry and tension associated with eminent domain in the 1980s. Indeed, Justice Sandra Day O'Connor, joined by a unanimous Supreme Court, expressed general disapproval of private takings and she appears to have felt the need to provide further qualification for the deferential view, which she did in part by observing that ``judicial deference is required because, in our system of government, legislatures are better able to assess what public purposes should be advanced by an exercise of eminent domain''. Hence, judicial deference was not regarded as an absolute and systemic imperative, as in Berman, but made contingent on the fact that legislatures are ``better able'' than courts at conducting public purpose tests. Hence, some of the contextual ideas from earlier case law is echoed in the decision, but now with respect to the legislature. It should be noted that {\it Midkiff} follows {\it Berman} also in the authorities consulted, and does not consider the cases which had focused on the importance of judicial scrutiny at state level.

The purpose of interference in {\it Midkiff} was to break up an oligopoly to the benefit of tenants, not to further economic development by allowing commercial interests to take land. Hence, the rationale behind the interference is likely to have struck the Supreme Court as sound and just. Moreover, it seems that such an interference would be easy to uphold also under the doctrine of contextual judicial scrutiny of the public use determination. Indeed, Justice O'Connor partly relies on an assessment of the merits of the taking, pointing out that  ``regulating oligopoly and the evils associated with it is a classic exercise of a State's police powers''. In conclusion, the ``extremely narrow'' room for judicial review set up by {\it Berman} seems to have been replaced by a slightly more nuanced formulation, which nevertheless made clear that a legal precedent of deference had now become entrenched. Fine readings aside, {\it Midkiff} reaffirms the main principle:  the meaning of public use can be broad, and the room for judicial review of governmental assessments in this regard is narrow.

So far we have only commented on how the Supreme Court developed its own doctrine on the public use restriction in the early 20th Century. Given that its role in takings jurisprudence was limited up to this point, it is important to consider also the effect on state case law. In particular, what was the fallout of {\it Berman}, which failed to recognize the importance of the tradition for judicial scrutiny that had developed at the state level? A detailed assessment of this against primary sources will have to be left for future work. However, it seems clear that {\it Berman} had a significant effect, both conceptually and in practice. A clear indication of this can be found in the secondary literature. Indeed, most academics following WW2 seemed to converge towards the view that the public use requirement was of little or no judicial importance. Professor Merrill, in an influential paper from 1986, goes as far as to describe it as a ``dead letter''.\footcite{merrill86}. At the same time, eminent domain became more controversial in this period, as it was also put to use more aggressively by some states.

 Some concrete cases proved particularly controversial, and they were taken to illustrate the dangers of eminent domain, particularly in relation to economic development projects. While the takings power had traditionally been used mostly to condemn agrarian land rights, it was now regularly used to condemn middle class homes. The controversy surrounding the case of Poletown Neighborhood Council v. City of Detroit  illustrates this, and the case marks a watershed moment in the history of  economic development takings in the US.\footcite[See][380-381]{sandefur05} In {\it Poletown}, the Michigan Supreme Court held that it was not in violation of the public use requirement to allow General Motors to displace some 3500 people for the construction of a car assembly factory. The majority 5-2 cites {\it Berman}, commenting that its own room for review of the public use requirement is limited.\footcite[632-633]{poletown81}

The {\it Poletown} decision was controversial, and the minority, especially Justice Ryan, was highly critical of it. He objects both to the deferential stance in general and to the majority reading of {\it Berman} in particular, pointing out that the Supreme Court's doctrine of deference was in large part directed at the state courts.\footcite[668]{poletown81} Hence, he concludes, the majority's reliance on {\it Berman} is ``particularly disingenuous''.\footcite[668]{poletown81} 

Justice Ryan was not alone in his disapproval of {\it Poletown} and the case is widely regarded as the prelude to an era of increased tensions over economic development takings in the US. This would culminate with {\it Kelo} which, despite upholding an economic development taking, also signaled a move towards more active judicial review of the public use requirement. This effect of {\it Kelo} has become more clear over time, primarily due to state responses caused by widespread disapproval with the outcome. However, it has also been remarked that both the majority and minority opinions in {\it Kelo} indicate that the Supreme Court itself may not be entirely at ease with the doctrine of strict deference that developed after {\it Berman}. In the next subsection, I will give an overview of recent developments, particularly from the secondary literature.

\section{Economic development takings after Kelo}

The fact that {\it Kelo} was decided against the homeowner met with wide disapproval by the US public. In addition, many scholars expressed concern at what they saw as an ill advised ``abdication'' of the judiciary in takings cases. The minority opinions given in {\it Kelo}, particularly the opinion of Justice O'Connor, also proved influential, causing further attention to be directed at the perceived dangers of eminent domain abuse. A massive amount of literature has since appeared devoted to studying the ``problem'' of economic takings. Moreover,  many states have seen reforms aimed to curb the use of eminent domain for economic development.\footnote{For an overview and critical examination of the myriad of state reforms that have followed {\it Kelo}, I point to \cite{eagle08}. See also \cite{somin09}.} 

As of 2014, 44 states have passed post-{\it Kelo} legislation to curb the use of eminent domain for economic development.\footnote{According to the Castle Coalition, a property activist project associated with the Institute of Justice. See \url{http://www.castlecoalition.org/} for an up-to-date survey of state legislation on eminent domain.} Various legislative techniques have been adopted by the states to achieve this. Some states, including Alabama, Colorado, Michigan, enacted explicit bans on economic development takings and takings that would benefit private parties.\footcite[See][107-108]{eagle08} In South Dakota, the legislature went even further, banning the use of eminent domain  ``(1) For transfer to any private person, nongovernmental entity, or other public-private business entity; or (2) Primarily for enhancement of tax revenue''.\footnote{South Dakota Codified Laws § 11-7-22-1, amended by House Bill 1080, 2006 Leg, Reg Ses (2006).}

In other states, more indirect measures were also taken, such as in Florida, where the legislature enacted a rule whereby property taken by the government could not be transferred to a private party until 10 years after the date it was condemned.\footcite[809]{eagle08} Many states also offer inclusive, often lengthy, lists of uses that should count as public, allowing the states to restrict the eminent domain power while also allowing condemnations that are regarded as particularly important to the state.\footcite[804]{eagle08}
However, as argued by Somin, many of these legislative reforms are largely ineffective in preventing economic development takings.\footcite[2120]{somin09} Somin also points to another interesting trend, namely that state reforms enacted by the public through referendums tend to be far more restrictive and effective in preventing economic and private-to-private takings than reforms passed through the state legislature.\footcite[2143]{somin09} 

This is a further reflection of the extent to which the US public opposed the decision in {\it Kelo}. Surveys show that as many as 80-90 \% believe that it was wrongly decided, an opinion widely shared also among the political elite.\footcite[2109]{somin09} Indeed, {\it Kelo} has had a great effect on the discourse of eminent domain in the US, and this effect is perhaps of greater importance than the various state reforms that have been enacted. According to Somin, most of the reforms have in fact been ineffective, despite the overwhelming popular and political opposition against economic development takings.\footcite[2170-2171]{somin09} 

Somin is not alone in feeling that eminent domain reform has offered more than it could deliver, this is a sentiment that is expressed both by supporters and critics of {\it Kelo}. On the other hand, while practitioners have noted that it is largely business-as-usual in eminent domain law, they also report a greater feeling of unease regarding the public use requirement, expressing hope that the Supreme Court will soon revisit the issue.\footnote{See \cite{murakami13} (``Until the Supreme Court revisits the issue, we predict that this question will continue to plague the lower courts, property owners, and condemning authorities'').} In this way, the public backlash against {\it Kelo} has served as an influential reminder that the rationale behind eminent domain for economic development is largely out of sync with the sense of fairness and justice endorsed by most non-experts. 

The underlying cause of this, according to Somin, can be traced to the fact that people are ``rationally ignorant'' about the economic takings issue. For most people, it is unlikely that eminent domain will come to concern them personally or that they will be able to influence policy in this area. Hence, it makes little sense for them to devote much time to learn more about it. This, in turn, helps create a situation where experts can develop and sustain a system based on principles that, in fact, are opposed by a large majority of citizens.\footcite[2163-2171]{somin09} Indeed, Somin argues that surveys show how people tend to overestimate the effectiveness of eminent domain reform, possibly due to the fact that symbolic legislative measures are mistaken for materially significant changes in the law.\footcite{somin09}

I think Somin's analysis is on an interesting track, although it seems wrong to assume {\it a priori} that people's critical stance on economic development takings would necessarily remain in place if they educated themselves more on the issue. Rational ignorance, in particular, should be seen as a double-edged sword in disputes of this kind. But this does nothing to detract from the main message, which is that the {\it Kelo} backlash seems to have caused greater insecurity about what the law is, without being able to significantly curb those uses of eminent domain that have been deemed problematic. In my opinion, this shows that the static legislative approach to eminent domain reform, which has dominated the scene in the US so far, needs to be supplemented by more dynamic proposals. In particular, it seems important to target the decision-making processes surrounding planning and eminent domain, to look for principles by which this process can be imbued with legitimacy. 

In a country where the population expresses antagonism towards eminent domain for economic development, a more inclusive process will likely cause such takings to become more uncommon. On the other hand, if principles of good governance are put in place, it might also restore confidence in eminent domain as a procedure by which to implement democratically legitimate decisions about how to weigh the interests of landowners against the interests of the public. In the next subsection, I will consider two proposals for principles of this kind. The first targets specifically the question of how compensation is determined in economic development cases, a crucial aspect of legitimacy. The second proposal targets the decision-making process more broadly, by proposing a framework for land assembly that is meant to replace the use of eminent domain in certain circumstances.

%\noo{But it is not the general public that are the major stakeholders in such disputes, but rather the communities that are directly affected, including both the private property owners who will be burdened and those community members who stand to benefit. A good framework for balancing their interests relies on finding appropriate principles of good governance, so that governments can play an empowering role when such decisions are made. This is crucial for legitimacy of land use planning generally, but especially for eminent domain, where the gravity of the interference means that legitimacy is unlikely to arise unless the decision to condemn is firmly rooted in the interests of the main stakeholders. To the greatest possible extent, it also seems crucial to emphasize local conditions and ensure that the decision enjoys broad local support. 
%
%Shortly after {\it Poletown} was overturned, the case of Kelo saw the legitimacy of economic takings brought before the Supreme Court once again. This time there was real doubt and disagreement among the justices regarding the scope of the public use limitation. The case revolved around the legitimacy of condemning a home in favour of a research facility for the drug company Pfizer, which was part of a development plan for the City of New London.  The owner, Suzanne Kelo, argued that the condemnation of her home was in breach of the constitution, since it was a private-to-private taking ostensibly to the benefit of Pfizer rather than any clearly defined public use or interest.
%
%In Kelo, Justice Thomas adopted the strictest view on the public use test. He entirely disregarded  the precedent set by Berman and Midkiff in favour of constitutional originalism, the doctrine which asserts that direct assessment of the wording in the Constitution, and the intentions of the founding fathers, is the approach that should be used to decide constitutional cases. Following up on this he held that actual right of use for the public was the test that had to be applied in takings cases. The hundred years of precedent preceding Kelo was described as “wholly divorced from the text, history, and structure of our founding document", and thus Justice Thomas concluded that it had to be abandoned. 
%
%Justice O'Connor, in an expression of dissent joined by Chief Justice Rehnquist and Justices Scalia
%and Thomas, argued against legitimacy on less theoretical grounds, based on the facts of the case and the precedent that would be set for similar cases in the future. Her main legal argument was that while public use should be interpreted broadly, the possibility of positive ripple effects was not enough to justify private-to-private takings. In particular, Justice O'Connor took a very bleak view on the practical consequences that would arise from allowing economic takings that could be justified only by pointing only to indirect positive consequences for the public. She commented on the majority decision to uphold the taking as follows: 
%
%Any property may now be taken for the benefit of another private party, but the fallout from this decision will not be random. The beneficiaries are likely to be those citizens with disproportionate influence and power in the political process, including large corporations and development firms. As for the victims, the government now has license to transfer property from those with fewer resources to those with more. The Founders cannot have intended this perverse result.
%
%It seems that a major point of contention among the judges in the Supreme Court was whether or not these grim predictions was a realistic assessment of what the consequences of the decision would be. Surely, anyone who agrees with Justice O'Connor in her prediction of the fallout would also agree with here conclusion that it is perverse. But the majority in Kelo, in an opinion written by Justice Stevens, disagreed with her assessment, observing instead that a more restrictive view on economic takings would make it more difficult to cater to the "diverse and always evolving needs of society". 
%
%But the majority opinion also stressed that purely private takings where not permissible, and they attached great significance to the substantive assessment that the actual taking of Suzanne Kelo's home formed part of a comprehensive development plan that would not bestow special benefit on any particular group of individuals. Moreover, Justice Kennedy, in his concurring opinion, emphasised that states should not use public purpose as a pretext for interfering in property rights to the benefit of commercial actors.
%Hence the overall impression one is left with when considering Kelo in its historical and legal context is that it reflects an increasingly cautious attitude to economic takings. The precedent of virtually unlimited deference that was set in case law from the mid-to-late 19th Century was eschewed in favour of a more contextual approach where the merits and deeper purpose of the plans underlying a taking is not axiomatically beyond the scrutiny of the courts.
%
%From considering the reception of the case by the general public, we see even more clearly how Kelo in effect marks a change in the US towards greater scrutiny. 
%
%Indeed, the voices that have dominated in the aftermath of Kelo were critical of the decision and criticized the court for not offering better protection to property owners. The case also led to an a surge of academic interest in the pubic use restriction, with many arguing for further restrictions on the scope of the takings power. 
%Hence it seems that Justice O'Connor's opinion largely reflects contemporary worries about takings in the US, worries that are now also becoming increasingly relevant to how the law develops and is understood. Many states have changed their own eminent domain codes  following Kelo, to make it harder to undertake economic takings. Moreover, the federal government also banned such takings from taking place on the basis of federal takings powers.
%It will lead us astray to delve deeply into the question of what caused this change in perspective on economic takings in the US, but we can offer a few hypothesis. First, it seems that cases such as Poletown illustrates the potential danger inherent in making the power of eminent domain available to market players. In particular, the main worry that has been raised is that the pretext of public purpose may be in the process of becoming a powerful instrument for influential market actors to gain access to regulatory powers of government. As these powers has massively expanded in the post-WW2 period, so has the potential for abuse. In addition, it seems that while those who were adversely affected by eminent domain tended to be less privileged and resourceful groups of society, the takings power is now increasingly brought to bear also against members of the middle class, who are in a better position to fight it, both legally and on the political scene.
%
%While opinions differ greatly both regarding the extent of the problem and the causes of recent controversy, there is something near consensus in the US after Kelo that economic development takings raise special problems under the current system of eminent domain, and that these need to be addressed with a view to reducing tensions and restoring faith in the system. Indeed, even the majority in Kelo hint strongly at this when they say that  
%Some have argued forcefully that a strict reading of the public use requirement is the way forward, if not by strict interpretation then by an explicit ban on economic development takings.  However, it is tempting here to echo the worries expressed in Seawell, that a strict formalistic approach to legitimacy runs the risk not only of being inflexible, but also, eventually, of offering less  protection to property owners. How, then, should we reduce the risk of abuses?
%While many have focused on the question of banning economic taking, or reconsidering the public use clause, some have addressed this question from such a broader angle. In my opinion, this is the way forward. It seems, in particular, that a complete ban on economic development takings will leave a vacuum in the current economic system, which presupposes a great deal of cooperation between commercial and public interest. Particularly when it comes to economic development, the private-public partnership model has gained influence to the point that a ban on economic development takings would likely prove impossible to implement in a satisfactory manner. 
%More generally, it seems hard to address the problem of economic takings without considering the role they play in the larger economic context within which current rules and practices have developed. Based on such considerations, I believe the procedural approach to economic takings is the appropriate one. This perspective asks us to take a closer look at judicial safeguards for protecting the role of property owners in the decision-making processes that lead up to the use of eminent domain. To some extent one might approach this on the basis of existing legal principles, asking for better scrutiny of procedural aspects, or by making it easier to bring pretext claims before the courts. However, it might also require new ideas, and, in particular, the introduction of new institutions for decision-making and administration of the eminent domain process.
%
%In the next section, I will look at two concrete proposals in more detail, one concerning the decision-making step and the other concerning the calculation of compensation. 
%They will be important because they serve as starting points for the case study that is to follow, addressing mechanisms that we will return to in Chapters x and y when we look more closely at two Norwegian legal institutions that share many features with the theoretical roposals discussed in the next section.
%}

\section{Institutional proposals for increased legitimacy}\label{sec:ir}

In this subsection, I first present the Special Purpose Development Companies proposed by Lehavi and Licht.\footcite{lehavi07} I relate this proposals to theoretical approaches to the issue of compensation, before I go on to note some shortcomings and open questions that I will later address in my case study. I then go on to consider the Land Assembly Districts proposed by Heller and Hills.\footcite{heller08} I consider this proposal in light of the stated motivation, which is to design an effective mechanism of self-governance that can replace eminent domain in economic development cases. I present some unresolved questions and argue that there is a tension in the proposal between its narrow scope, imposed to prevent majority tyranny and other forms of abuse, and its broad goal of empowering local communities. 

\subsection{Special Purpose Development Companies}

The primary distinguishing feature of economic development takings is that they give the taker an opportunity to profit commercially from the development. This may even be the primary aim of the project, with the public benefiting only indirectly through potential economic and social ripple effects. Property owners facing condemnation in such circumstances might expect to take a share in the profit resulting from the use of their land. However, in many jurisdictions, including the US, the rules used to calculate compensation prevents owners from getting any share in the commercial surplus resulting from development.\footnote{See, e.g., \cite[965-966]{fennell04}.} In particular, various {\it elimination rules} are typically in place to ensure that compensation is based entirely on the pre-project value of the land that is being taken.\footcite[See][81]{ackerman06} The policy reasons for such rules is that they ensure that the public does not have to pay extra due to its own special want of the property. After all, this is one of the main purposes of using eminent domain in the first place; to ensure that the public does not have to pay extortionate prices for land needed for important projects. However, when the purpose of the project is itself commercial in nature, there appears to be a shortage of good policy reasons for excluding this value from consideration when compensation is calculated. This is especially true when, as in the US, compensation tends to be based on the market value of the land taken. Why should a commercial condemner's prospect of carrying out economic development with a profit be disregarded from the assessment of market value? In any fair and friendly transaction among rational agents, one would expect benefit sharing in a case like this. Yet for economic development backed up by eminent domain, the application of elimination rules ensures that all the profit goes to the developer. 

Some authors have argued that failures of compensation is at the heart of the economic takings issue and that worry over the public use restriction is in large part only a response to deeper concerns about the ``uncompensated increment'' of such takings.\footcite[See][962]{fennell04} In addition to the lack of benefit sharing, previous work has identified two further problems of compensation that also tend to become exasperated in economic development cases. First, the problem of ``subjective premium'' has been raised, pointing to the fact that property owners often value their own land higher than the market value, for personal reasons.\footcite[963]{fennell04} For instance, if a home is condemned, the homeowner will typically suffer costs not covered by market value, such as the cost of moving, including both the immediate ``objective'' logistic costs as well as more subtle costs, such as having to familiarize oneself with a new local community. Second, the problem of ``autonomy'' has been discussed, arising from the fact that an exercise of eminent domain deprives the landowner of her right to decide how to manage her property.\footnote{Discussed in \cite[966-967]{fennell04}. For a general personhood building theory of property law, see \cite{radin93}. For a general economic theory of the subjective value of independence, see \cite{benz08}.}

In \footcite{lehavi07}, the authors propose a novel approach for addressing the ``uncompensated increment'' in economic takings cases. Their proposal is based on a new kind of structure that they dub a {\it Special Purpose Development Corporation} (SPDC). The idea is that owners affected by eminent domain will be given a choice between standard pre-project market value and shares in a special company. This company will exist only to implement a specific step in the implementation of the development project: the transaction of the land-rights. The SPDC may choose either to offer their rights on an auction or else negotiate a deal with a designated developer.\footcite[1735]{lehavi07} Hence, the idea is to ensure that the owners are paid a value that reflects the post-project value of the land, but in such a way that the holdout problem is avoided. In particular, the SPDC will have a single task: to sell the land for the highest possible price within a given time frame.\footcite[1741]{lehavi07} After the sale is completed, the SPDC will divide the proceeds as dividends and be wound up.\footcite[1741]{lehavi07}

Other suggestions have taken a more static approach to compensation reform, such as proposing to give owners a fixed premium in cases of economic development, or developing mechanisms of self-assessment to ensure that compensation is based on the true value the owner attributes to his own land.\footnote{A range of static proposals have been proposed in the literature: Merrill proposes 150 \% of market value for takings that are deemed to be ``suspect'', including takings for which the nature of the public use is unclear, see \cite[90-93]{merrill86}. Krier and Serkin propose a system that provide compensation for a property's special suitability to its owner, or a system where compensation is based on the court's assessment of post-project value, see \cite[865-873]{krier04}. Fennell proposes a system of self-evaluation of property for takings purposes with tax-breaks given to those who value their property close to market value (to avoid overestimation), see \cite[995-996]{fennell04}. Bell and Parchomovsky also propose self-evaluation, but rely on a different mechanism to prevent overestimation; tax liability is based on the self-reported value and no property can be sold by its owner for less than his reported value, see \cite[890-900]{bell07}.} Compared to such proposals, the idea of SPDCs is more sophisticated and should be looked at in more depth. 

The conceptual premise for the proposal is that takings for economic development can be seen as compulsory incorporation, a pooling of resources useful in overcoming market failures.\footcite[1732-1733]{lehavi07} Just as the corporation is formed to consolidate assets in order to facilitate effective management, so is eminent domain used to assemble property rights in order to facilitate efficient organization of development. According to Lehavi and Licht, this also provides a viable approach to problems of ``opportunistic behavior''; hierarchical governance after assembly ensures that order and unity can be regained even if interests in the land are distributed among a large and heterogeneous group of potentially mischievous shareholders.\footcite[1733]{lehavi07} In the words of Lehavi and Licht:

\begin{quote}
The exercise of eminent domain powers thus resembles an incorporation by the government of all landowners with a view to brining all the critical assets under hierarchical governance. Establishing a corporation for this purpose and transferring land parcels to it thus would be merely a procedural manifestation of the substantive economic reality that already takes place in eminent domain cases.
\end{quote}

As soon as we look at the rationale behind economic development takings in this way, any remnant of good policy reasons for ensuring that the developer gets all the profit seems to disappear. Rather, we are led to consider compensation as an issue entirely separate from the exercise of the takings power. After the land has been reorganized by eminent domain and an SPDC has been formed, the land rights might as well be sold {\it freely} to a developer. In this way, the land will be sold for a price that is closer to an actual market value, on the market where the land is destined for development.\footcite[1735-1736]{lehavi07} More generally, the SPDC becomes an aid that the government can use to create more favorable market conditions for transferring land that has commercial potential in its public use. Due to the compulsory pooling of resources, no owner can exercise monopoly power by holding out, but due to decoupling of compensation from assembly, the owners can now negotiate with potential developers for a share of the resulting profit. Moreover, the fact that the SPDC offers its rights on an actual market can also help ensure that more information become available regarding the true economic value of the development, something that may in turn help ensure that only the good projects will be successful in acquiring land. Hence, according to Lehavi and Licht, an additional positive effect of SPDCs is that developers and governments will shun away from using the eminent domain power to benefit projects that are not truly welfare-enhancing.\footcite[1735-1736]{lehavi07}

In addition to these substantive consequences, the SPDC-proposal also stands out because it has a significant institutional component, pointing to its potential for restoring procedural legitimacy as well as substantive fairness. Lehavi and Licht discuss corporate governance issues at some length, but without committing themselves to definite answers about how the operations of the SPDC should be organized.\footcite[1040-1048]{lehavi07} Indeed, while their proposal is perhaps most interesting because of its procedural aspects, it also appears to be rather preliminary in this regard. The main idea is to let the SPDC structure piggyback on existing corporative structures, particularly those developed for securitization of assets.\footnote{See generally \cite{schwarcz94}. For an up-to-date overview, targeting special challenges that became apparent during the 2008 financial crisis, see \cite{schwarcz13}.} The basic idea is that the corporate structure should be insulated from the original landowners to the greatest possible extent; it should have a narrow scope, it should be managed by neutral administrators, and it should entrust a third party with its voting rights.\footcite[1742]{lehavi07} This is meant to prevent failures of governance within the SPDC itself, making it harder for majority shareholders and self-interested managers to co-opt the process. For instance, if a possible developer already holds a majority of the shares in an SPDC, this structure would prevent him from using this position to acquire the remaining land on favorable terms. 

Lehavi and Licht observe that under US law, the government would often be required to make shares in an SPDC available to the landowners as a public offering.\footcite[1745]{lehavi07} Lehavi and Licht deem this to be desirable, arguing that full disclosure will provide owners with a better basis on which to decide whether or not to accept SPDC shares in place of pre-project market value. It will also facilitate trading in such shares, so that they will become more liquid and therefore, presumably, more valuable.\footcite[1746]{lehavi07} 

Lehavi and Licht's proposal is interesting, but I think a fundamental objection can be raised against it. In particular, it seems that their governance model more or less completely alienate property owners from the decision-making process after SPDC formation. Limiting the participation of owners is to a large extent an explicit aim, since governance by experts is held to increase the chances of ensuring good governance. But is expert rule really the answer?

It seems that from the owners' point of view, Lehavi and Licht's proposals for governance reduces the SPDC to a mechanism whereby they can acquire certain financial entitlements. These may exceed those that would follow from standard compensation rules, but they do not directly empower owners vis-{\'a}-vis developers and the government. Instead, a largely independent structure will be introduced. It is this new organizational structure, rather than the owners, that will now become an important actor in the eminent domain process. In principle, it is meant to represent owners, but to what extent can it do so effectively? After all, it is specifically intended to operate as neutral player, charged with maximizing the price, nothing more. Hence, it appears that the SPDC will not be able to give owners an arena to negotiate on the basis of the personal and social importance they attribute to their land rights. How the problem of ``autonomy'' is addressed by the proposal is therefore hard to see and the ``subjective premium'' also appears to be in danger, unless it can be objectively quantified and covered by the surplus from a voluntary sale. But if such quantification is possible, then why not simply tell the appraiser to award some premium under standard compensation rules?

More generally, it seems to me that while all three categories of ``uncompensated increments'' are interesting to study from a financial viewpoint, severe doubts can be raised regarding the feasibility of addressing the subjective aspects of this as a question of compensation. It may be that issues related to ``subjective premium'' and ``autonomy'' are seen as public use issues for good reason; they are hard to quantify otherwise. Moreover, attempting to do so might do more harm than good. On the one hand, it might skew the political process, since owners that have been ``bought off'' don't object to ill-advised development projects, as long as they generate financial revenue. But what about projects that are undesirable for other reasons, for instance because they completely change the character of a neighborhood, or because they are harmful to the environment? On the other hand, the very idea that money can compensate for the subjective importance of property and autonomy can itself prove offensive. At least it seems likely that it would often come to be seen as inadequate and inefficient.\footnote{For more detailed criticism of the compensation approach to the public use issue, see \cite{garnett06}.} Moreover, an owner that is compelled to give up his home after an inclusive process where the public interest has been debated and clearly communicated is likely to feel like he incurs less costs related both to his subjective premium and his autonomy. Hence, the lack of participation in the decision-making process can in itself increase the uncompensated loss. Clearly, no externally managed ``bargain-oriented'' SPDC will be able to resolve this problem. Of course, some ``objective'' elements of, such as relocation costs or cost for juridical assistance, can still be addressed under the banner of compensation. But in most jurisdictions, they already are.\footnote{See, e.g., \cite[121-126]{garnett06}.} For more subtle aspects, the aftermath of {\it Kelo} itself can serve as an illustration of how a compensatory approach is unsatisfactory:

After the case, Suzanne Kelo remained defiant, until she eventually decided to settle in 2006, for an offer of \$ 442 155, more than \$ 319 000 above the appraised value.\footcite[1709]{lehavi07} Apparently, the other owners affected by the same taking were not particularly pleased, arguing that recalcitrant owners were actually rewarded for holding out.\footcite[1709]{lehavi07} On the other hand, there is no indication that Suzanne Kelo was not genuine in her opposition to the taking. Indeed, after the long struggle she had taken part in, it is easy to imagine that financial compensation, if it was to be an effective remedy at all, would have to be very high. Even after she had settled, Kelo apparently toured the country speaking out against economic takings. This, too, is a statement to the inadequacy of a purely financial approach to legitimacy. 

I conclude that SPDCs have serious shortcoming with regards to the subjective aspects of undercompensation, aspects that can only be addressed if the focus turns towards participation. However, SPDCs do seem promising when it comes to profit-sharing. This, after all, is what the structure is specifically aiming to achieve. In addition, I agree that SPDCs will likely have a positive effect on the other actors in the eminent domain process. In particular, I agree with Lehavi and Licht that greater openness is likely to result, revealing the true merits of development projects, at least in so far as these are translatable into financial terms. The fact that developers must negotiate with an SPDC who can threaten to make the land available an an open auction will likely deter developers and government from pursuing fiscally inefficient projects. Hence, the risk that governments will subsidized such projects by giving them cheap access to land will also be reduced. In addition, the presence of a third voice, speaking on behalf of owners, is likely to help achieve a better balance of power in development takings. 

Even if the individual landowners do not have a voice in this process, the fact that the landowners are better represented as a group is then still likely to have a positive effect on legitimacy. On the other hand, as long as the power of the SPDC is limited to choosing the best offer and negotiating over price, it seems that SPDCs will easily end up being dominated by developers and government. This is a particular concern in cases when competition fails to arise after SPDC formation. To ensure that there are other interested parties, in particular, sems like an important precondition for the proposal to work in practice. In this regard, it is important to realize that a lack of interest from other developers may not be due to the superiority of the original developer's plans. It might rather be due to the fact that the scope of the assembly giving rise to the SPDC is so defined as to make alternatives unfeasible. The danger of abuse in this regard seems significant, particularly when developers themselves participate in coming up with the plans that give rise to SPDC formation. 

Moreover, as long as owners remain marginalized in the planning phase, it is easy to imagine situations where the plan itself will be formulated in such a way that only one developer is in a position to successfully implement it. A simple example would be if a prospective developer already owns some of the land that is critical to the plan, and is able to ensure that this land is kept out of the scope of the SPDC. Clearly, if SPDCs are to operate effectively, such instances of manipulation need to be avoided, suggesting that the proposal as it stands needs to be fleshed out in greater detail.

The problems addressed here both seem to point to the fact that the SPDCs, while more flexible than other suggestions, are still too static to achieve many of their objectives. In particular, to arrive at genuine market conditions for assessing post-project value, there is still a need for changes in the dynamics of the planning process underlying the taking. Moreover, to ensure legitimacy, there is a need for a mechanism that goes beyond expert bargaining and provides owners with better access to the decisionmaking process. In the next subsection, I will consider a proposal that aims to address this, by proposing a framework for self-governance. 

\subsection{Land assembly districts}

In a recent article, Heller and Hills propose a new institutional framework for carrying out land assembly for economic development. Interestingly, it is meant to replace eminent domain altogether. The goal is to ensure democratic legitimacy while also creating a template for collective decision-making that will prevent inefficient gridlock and holdouts. The core idea is to introduce {\it Land Assembly Districts} (LADs), institutions that will enable property owners in a specific area to make a collective decision about whether or not to sell the land to a developer or a municipality.\footcite[1469-1470]{heller08} Anyone can propose and promote the formation of a LAD, but both the official planning authorities and the owners themselves must consent before it is formed.\footcite[1488-1489]{heller08} Clearly, some form of collective action mechanism is required to allow the owners to make such a decision. Hiller and Hill suggest that voting under the majority rule will be adequate in this regard, at least in most cases.\footnote{See \cite[1496]{heller08}. However, when many of the owners are non-residents who only see their land as an investment, Heller and Hills note that it might be necessary to consider more complicated voting procedure, for instance by requiring separate majorities from different groups of owners. See \cite[1523-1524]{heller08}.} How to allocate voting rights in the LAD is another issue that require careful consideration, but Heller and Hills land on the proposal that they should in principle be given to owners in proportion to their share in the land belonging to the LAD.\footnote{See \cite[1492]{heller08}. For a discussion of the constitutional one-person-one-vote principle and a more detailed argument in favor of the property-based proposal, see \cite[1503-1507]{heller08}.} Owners can opt out of the LAD, but in this case eminent domain can be used to transfer the land to the LAD using a conventional eminent domain procedure.\footcite[1496]{heller08}

Heller and Hills envision an important role for governmental planning agencies in approving, overseeing and facilitating the LAD process. Their role will be most important early on, in approving and spelling out the parameters within which the LAD is called to function.\footcite[1489-1491]{heller08} Hence, it appears to be assumed that the planning authorities will define the scope of the LAD by specifying the nature of the development it can pursue. A possible challenge that arises, and which Heller and Hills do not address at any length, is that the scope of the LAD needs to be broad enough to allow for meaningful competition and negotiation after LAD formation. At the same time, however, there will probably be a push, both by governments and initiating developers, to ensure that the scope is defined narrowly enough to give confidence that rezoning permissions will not be denied at a later stage. Another potential challenge is that the planning authorities might have an incentive to refuse granting approval for LAD formation, since it effectively entails that they give up the power of eminent domain for the land in question. For this reason, Heller and Hills propose that a procedure of judicial review should exist whereby a decision to deny approval for LAD formation can be scrutinized.\footcite[1490]{heller08} 

After the formation of the LAD, the government will have a less important position, but the planning authorities will still occupy an important facilitating role. Heller and Hills envision a system of public hearings, possibly organized by the planning authorities, where potential developers meet with owners and other interested parties to discuss  plans for development.\footnote{See \cite[1490-1491]{heller08}.} In this process, it is assumed that also other voices will be represented, such as owners of adjoining land, who can use this opportunity to express objections against the project. Their role in the process is not clarified, but presumably the planning authorities would be able to offer this group some protection, if not in relation to the LADs own operations, then later in relation to the decision whether or not to grant the licenses needed for the development project.

Importantly, if the owners do not agree to forming a LAD, or if they refuse to sell to any developer, the government will be precluded from using eminent domain against them to assemble the land.\footcite[1491]{heller08} This is the crucial novel idea that sets the suggestion apart from other proposals for institutional reform that have appeared after {\it Kelo}. LADs will not only ensure that the owners get to bargain with the developers over compensation, it will also give them an opportunity to refuse any development to go ahead, if they should so decide. Hence, the proposal shifts the balance of power in economic development cases, giving owners a greater role also in preparing the decision whether or not to develop, and on what terms. In my opinion, this makes the proposal stand out as particularly interesting in the recent literature on economic takings. It is the first concrete suggestion that addresses the democratic deficit in a dynamic, procedural manner, without failing to recognize that the danger of holdouts is real and that institutions are needed to avoid it, also in economic development cases.

There are some problems with the model, however. Kelly points out that the basic mechanism of majority voting is itself imperfect, and can lead both to overassembly and underassembly, depending on the circumstances.\footcite{kelly09} He points out, in particular, that if different owners value their property differently, majority voting will tend to disfavor those with the most extreme viewpoints, either in favor of, or against, assembly. If these viewpoints are assumed to be non-strategic and genuine reflections of the welfare associated with the land, the result can be inefficiency. In a nutshell, the problem is that a majority can often be found that does not take due account of minority interests. For instance, if some owners are planning alternative development, leading them to attribute a high {\it hope}-value to their land, they can safely be ignored as long as the majority have no such plans. This could become particularly bad in cases when the alternative development itself is more socially desirable than the development that will benefit from assembly. The role of the LAD in such cases will not improve the quality of the decision to develop, since it pushes the decision-making process into a track where those interests that {\it should} prevail are voiced only by a marginalized minority inside the new institution.\footnote{Of course, one might imaging these landowners opting out of the LAD, or pursuing their own interests independently of it. However, they are then unlikely to be better off than they would be in a no-LAD regime. In fact, it is easy to imagine that they could come to be further marginalized, since the existence of the LAD, acting ``on behalf of the owners'', might detract from any dissenting voices on the owner-side.}

More generally, the lack of clarity regarding the role of LADs in the planning process is a problem. As it stands, the proposal leaves it uncertain how LADs will affect the decision-making process regarding development. But the ideal is clearly stated: LADs should help to establish self-governance in land assembly cases. In particular, Heller and Hills argue that LADs should have ``broad discretion to choose any proposal to redevelop the neighborhood -- or reject all such proposals''.\footcite[See][1496]{heller08} As they put it, two of the main goals of LAD formation is to ensure `` preservation of the sense of individual autonomy implicit in the right of private property and preservation of the larger community's right to self-government''.\footcite[See][1498]{heller08} Unfortunately, these ideals are somewhat at odds with the concrete rules that Heller and Hills propose, particularly those aiming to ensure good governance of the LAD itself. 

In relation to the governance issue, Heller and Hills echo many of the ``corporate governance''-ideas that also feature heavily in Lehavi and Licht's proposal. Indeed, in direct contrast to their comments about ``broad discretion'' and ``self-governance'', Hiller and Hills also state that ``LADs exist for a single narrow purpose -- to consider whether to sell a neighborhood''.\footcite[See][1500]{heller08} This is a good thing, according to Heller and Hills, since it provides a safe-guard against mismanagement, serving to prevent LADs from becoming battle grounds where different groups attempt to co-opt the community voice to further their own interests. As Heller and Hills puts it, the narrow scope of LADs will ensure that ``all differences of interest based on the constituents' different activities and investments, therefore, merge into the single question: is the price offered by the assembler sufficient to induce the constituents to sell?''.\footcite[1500]{heller08}

But this means that there is an internal tension in the LAD proposal, between the broad goal of self-governance on the one hand and the fear of neighborhood bickering and majority tyranny on the other. It is hard to see, in particular, how the idea of LADs with a ``narrow purpose'' is compatible with a scenario where the LAD has ``broad discretion'' to choose between competing proposals for development. If such discretion may indeed be exercised, what is to prevent special interest groups among the landowners from promoting development projects that will be particularly favorable to them, rather than to the landowners as a group? And what is to prevent landowners from making behind-the-scene deals with favored developers at the expense of their neighbors? It seems like a great challenge to come up with rules that prevent mechanisms of this kind, without also constraining the landowners so much that meaningful ``self-governance' becomes an impossibility. If a LAD is obliged to only look at the price, this might prevent abuse. But it will not give owners broad discretion to choose among development proposal. Effectively, it will render LADs as little more than a variant of SPDCs, where the owners are awarded an extra bargaining-chip: the option to refuse all offers. 

In my view, such a restriction on the operations of LADs is not desirable. It is easy to imagine cases where competing proposals, perhaps emerging from within the community of owners themselves, will emerge in response to the formation of a LAD. Such proposals may involve novel solutions that are superior to the original development plans, in which case it is hard to see any good reason why they should not be taken into account, even if they are less commercially attractive. In particular, the formation of a LAD and the competition for development that ensues creates an opportunity for tapping into a greater pool of ideas for redevelopment, ideas which may then also be rooted more firmly in the local community. Surely, getting such proposals to the table would be desirable and it would take us to the heart of self-governance. At the same time, it is easy to acknowledge that problematic situations may arise, for instance if a majority forms in favor of a scheme that involves razing only the homes of the minority, maybe on the rationale that these are ``more blighted''. That would likely give rise to accusations of unfair play, which may or may not be warranted. But irrespectively of this, an alternative project of this kind might well be a better use of the land in question, also from the point of view of the public. Hence, it would seem that the planning authorities would be obliged to give it some serious consideration. Then, however, the LAD has truly become an arena for a new kind of power play among different interest, and a potential vehicle of force for whomever secures support from a majority of owners within the district.

In their proposal, Heller and Hills are aware of this potential problem, which they propose to resolve by strict regulation. In particular, they argue that ``LAD-enabling legislation should require especially stringent disclosure requirements and bar any landowner from voting in a LAD if that landowner has any affiliation with the assembler''.\footcite{heller08} But this raises further questions. For one, what is meant by ``affiliation'' here? Say that a land owner happens to own shares in some of the companies proposing development. Should he then be barred from voting? If so, should he be barred from voting on all proposals, or just those involving companies in which he is a shareholder? If the answer is yes, how would this be justified? Would it not be easy to construe such a rule as discrimination against landowners who happen to own shares in development companies? On the other hand, if the landowner in question is allowed to vote on all other proposals, would it not be natural to suspect that his vote is biased against assembly that would benefit a competing company? Or what about the case when some of the land owners are employed by some of the development companies? Should such owners be barred from voting on proposals that could benefit their employers? This seems quite unfair as a general rule, especially if a low-level employment relationship has such a dramatic effect. But in some cases even low-level ties could play a decisive factor. This might happen, for instance, if an important local employer proposes development in a neighborhood where it has a large number of employees.

Of course, the most pressing issue that arises is the following: who exactly should be empowered to make the determination of when an affiliation is such that an owner should be deprived of his voting rights? Heller and Hills give no answer, but it is easy to imagine that whoever is given this task in the first instance, the courts would soon enough be asked to consider the question. At this point, the circle has in some sense closed in on the proposal. In particular, one might ask: why is it easier to determine if someone can be deprived of his voting rights due to an ``affiliation'', than it is to determine if someone can be deprived of his land due to some planned ``public use''?

In any event, to come up with a set of rules ensuring that LADs can deliver both self-governance and good governance largely remains an open problem. This is acknowledged by Hiller and Hills themselves, who point out that further work is needed and that only a limited assessment of their proposal can be made in the absence of empirical data. Later in the thesis, I will shed light on this challenge when I consider the Norwegian rules relating to land consolidation, showing how these can be looked at as a highly developed institutional embedding of many of the central ideas of LADs. The assessment of how they function in cases of economic development, and how they are increasingly used as an alternative to expropriation in cases of hydro-power development, will allow me to shed further light on the issues that are left open by Heller and Hills' important article.

\section{Conclusion}

In this chapter, I have given a more in-depth presentation of economic development takings in law. I began by noting that the issue is particularly pressing for land users that are not regarded as bringing about economic growth. Hence, I argued that the issue is closely related to that of land grabbing, which is currently receiving much attention, both academic and political. Under the social function understanding of property there is in principle no difference between protecting property rights arising from formal title and property rights arising from use. That said, special issues arise in the latter case, not least because it is unclear how the law should deal with rights resulting from cultural practices that western property regimes are not designed to handle. In addition, I noted that special issues related to food security and poverty arise with particular urgency in relation to land grabbing.

Moreover, the nature of my case study makes it more natural for me to focus on traditional western systems of property law. Hence, I went on to shed more light on discuss how economic development takings are dealt with in such legal systems, focusing on Europe and the US respectively. For the case of Europe, this assessment was made more difficult by the fact that the category is not an established part of legal discourse. However, by looking to England and Germany as concrete examples, I noted that such cases do arise and that they are increasingly seen as controversial. I also noted that there is a contrast between how England and Germany approach such cases, as well as how they approach property more generally. Germany, in particular, goes further in explicitly recognizing the social functions of property, by actively looking to social and political values when assessing whether interferences are legitimate. In England, similar reasoning is at most applied indirectly, as takings are approached almost entirely as an issue of administrative law. 

I then went on to consider the property protection offered by P1(1) of the ECHR, and how it is applied by the Court in Strasbourg. I zoomed in on those aspect that I believe to be the most relevant for economic development takings. While I noted that this category has yet to be discussed by the ECtHR, I argued that a recent shift in the Court's property adjudication is suggestive of the fact that it would likely approach such cases similarly to how Justice O'Connor approached {\it Kelo}. In particular, I noted how the Court has recently adopted a stricter standard of assessment. This standard, I argued, is characterized primarily by increased sensitivity to systemic imbalances causing alleged P1(1) violations. Hence, to regard economic development takings as a special category appears to fit well with recent jurisprudential developments at the Court in Strasbourg.

I went on to consider US sources on economic development takings, noting that the issue has receive an extraordinary amount of attention in recent years. I adopted an historical approach to the material, by tracing the case law surrounding the public use restriction in the fifth amendment to the US constitution, which was much debated even before the specific issue of economic development takings rose to prominence. I focused particularly on case law developed by state courts, and I argued that it shows great sensitivity to the need for contextual assessment. Indeed, I argued that originally, many state courts implicitly adopted a social function view on property when they assessed such cases.

I then looked at the history of Supreme Court adjudication of public use cases. I noted that the doctrine of deference was developed early on, but that it was initially directed mainly at state courts. In fact, I showed that the Supreme Court itself noted with approval the contextual and in-depth approach these courts would rely on when dealing with this issue.

The shift, I argued, came with \textcite{berman54}, in which the Supreme Court adopted a deferential doctrine that was directed specifically at the {\it state legislature}. This, I argued, was quite a dramatic departure from the Court's previous attitude towards {\it state} takings. In fact, it was almost entirely backed up by precedent set in cases when {\it federal} takings had been ordered by Congress. I went on to consider the fallout of \textcite{berman84} at state level, which culminated with the infamous {\it Poletown} case. This case prompted wide-spread accusations of eminent domain abuse and thus it set the stage for {\it Kelo}.

After completing the historical overview, I went on to consider the literature after {\it Kelo}. I expressed particular support for those responses that focus on the need for {\it institutional} reform, to address precisely those dangers that Justice O'Connor pointed to in her minority opinion. As a shorthand, I proposed referring to the mechanisms she identified as the {\it democratic deficit} of economic development takings. 
% I zoomed in on two of those in particular, the Special Purpose Development Companies proposed by Lehavi and Licht, and the Land Assembly Districts suggested by Heller and Hills. I gave an in-depth presentation of these two proposals, pointing out strengths and weaknesses. 
%%In coming chapters, I will refer back to this as I consider similar institutions and mechanisms that are currently operating in Norwegian law relating to hydropower development.

I then gave a thorough presentation of two recent reform suggestions that might help address this deficit. Both are institutional in nature, based on setting up formally recognized coalitions of land owners that can act as a counterweight to the disproportional power of commercial beneficiaries. The first suggestion, by Lehavi and Licht, limited its attention to compensation, recognizing the need for a system whereby the land owners are compensated based on post-project value.  This idea in itself represents a fairly dramatic break with the currently dominant doctrine in takings law, where compensation is almost always, and in almost all jurisdictions, based on the pre-project value of the land, or the {\it value to the owner}.

In Chapter \ref{chap:5}, I will return to this principle in more depth, looking at how it developed internationally, and in Norway. I will also look at how it has now been abandoned in Norwegian law, for some case types involving hydro-power development. I relate this to the special role played by the appraisal courts in Norway. The local grounding of these courts, involving lay people sitting as court appointed appraisers, allows the law to be applied in a way that adapts to the concrete circumstance in a way that may enhance the perceived fairness and legitimacy of the taking. At the same time, however, the judicial procedure, with a (limited) possibility for appeal, puts in place safeguards against abuse.

The second suggestion I looked at in depth, proposed by Heller and Hills, focused not on compensation, but on the decision-making process leading to an economic development project being implemented. This proposal is based on the idea that local communities should be entitled to greater self-governance in such cases. At the same time, it recognizes the need for a mechanism to avoid inefficient and socially harmful gridlock due to holdouts among unwilling owners. Instead of eminent domain, however, a different mechanism is proposed: a land assembly district (LAD). 

This is also a new class of institutions, and I pointed out some problems and seeming inconsistencies in the proposal, regarding the exact role they will have in the planning process. I argued that while the risk of abuse and failure increases with the level of participation, so does the positive effect on legitimacy. I concluded that to reduce the democratic deficit in economic development cases, a wide power of participation must be granted to the land owners and their (immediate) communities. This is needed, in particular, to restore balance in the relationship between owners and others directly connected with the land, the planning authorities, and the commercial actors interested in development for profit. The question that is as of yet unresolved is how to organize such participation in a way that avoids obvious pitfalls, such as administrative inefficiency and tyranny by majorities or elites that gain control of the local agenda.

In Chapter \ref{chap:6}, I will shed light on this question by considering the Norwegian institution of land consolidation, which has very long traditions. It is a flexible frameworks which includes, among other things, a a template for establishing institutions that can function as a LAD. I will focus on how land consolidation functions in cases of economic development that would otherwise likely be pursued by eminent domain. Norwegian hydro-power development will again be in focus, but I will also discuss planning law and development more generally, as the Norwegian government is now considering making consolidation, traditionally a rural institution, a primary mechanism for land development also in urban areas.

Before I delve into this, I will use the next chapter to present an overview of Norwegian hydropower and the role of waterfalls as private property.
%\chapter{Introduction and Summary of Main Themes}\label{chap:1}

\begin{quote} \small
Thieves respect property. They merely wish the property to become their property that they may more perfectly respect it.\footnote{\cite[58]{chesterton08}.}
\end{quote}
\begin{quote} \small
[Granting] a takings power, then, may not be viewed as an act that wrenches away property rights and places an asset outside the world of property protection. Rather, it may be seen as an act within the larger super-structure of property.\footnote{\cite[583]{bell09}.}
\end{quote}
%
%A human being needs only a small plot of ground on which to be happy, and even less to lie beneath. %\footnote{Johan Wolfgang von Goethe, {\it The sorrows of young Werther and selected writings}.}
%\end{quote}
%“That's what makes it ours - being born on it, working on it, dying on it. That makes ownership, not a %paper with numbers on it.”
%― John Steinbeck, The Grapes of Wrath
%
%\cite{waring11} (``It is a testament to the elasticity of the concept of
%property that it is able to represent all things to all people, and to accommodate so many conflicting %calls''); 
%

Property can be an elusive concept, especially to property lawyers.\footnote{See, e.g., \cite[225-226]{waring09} (``It is a testament to the elasticity of the concept of 
property that it is able to represent all things to all people, and to accommodate so many conflicting calls.''); \cite[252]{gray91} (``But then, just as the desired object comes finally within reach, just as the notion of property seems reassuringly three-dimensional, the phantom figure dances away through our fingers and dissolves into a formless void.'').} Indeed, in legal language, the word itself often only functions as a metaphor -- an imprecise shorthand that refers to a complex and diverse web of doctrines, rules, and practices, each pertaining to different ``sticks'' in a bundle of rights.\footnote{See generally \cite{grey80,klein11}.} Should we conclude that property as a unifying term is lost to the law? It certainly seems hard to pin it down. In the words of Kevin Gray, when a close scrutiny of property law gets under way, property itself seems like it ``vanishes into thin air''.\footnote{See \cite[306-307]{gray91}. See also \cite[81]{grey80} (arguing that the eventual consequence of the bundle view is that property will cease to be an important category for legal and political reasoning).}

Arguably, however, property never truly disappears.\footnote{See \cite[159]{gray94} (``We are continually prompted by stringent, albeit intuitive, perceptions of 'belonging'.'').} Indeed, there is empirical evidence to suggest that humans come to the world with an innate concept of property, one which pre-exists any particular arrangements used to distribute it or mould it as a legal category.\footnote{See \cite{stake06}.} Specifically, humans and a seemingly select group of other animals appear to have an intuitive ability to recognise {\it thievery}, the taking of property (not necessarily one's own) by someone who is not entitled to do so.\footnote{See \cite[11-13]{brosnan11}; \cite[159]{gray94}.}

%\footnote{See \cite[...]{gray94} (``In this context we are still not far removed from the
%primitive, instinctive cries of identification which resound in the
%playgroup or playground:
%'That's not yours; it's mine.''').}

Taken in this light, Proudhon's famous dictum, ``property is theft'', might be more than a seemingly contradictory comment on the origins of inequality.\footnote{For Proudhon's theory of property generally, distinguishing between {\it de facto} possession and {\it de jure} property (regarded as theft), see \cite{strong14}.} It might point to a deeply rooted aspect of property itself, namely its role as an anchor for the distinction between legitimate and illegitimate acts of taking.

%Therefore, it also seems that the following becomes a key question in property theory: how should we think about takings, and when are they legitimate?

In this thesis, I will study takings of a special kind, namely those that are sanctioned by a government in the pursuit of some public use or interest. Specifically, the word {\it taking} will be used to refer to an exercise of the government's power of eminent domain.\footnote{See generally \cite{stoebuck72} (clarifying the status of eminent domain from a US perspective, tracing its roots back to early civil law writers such as Grotius and Bynkershoek). Takings will also be referred to as expropriations, especially in the context of Norwegian law. In England and Wales, the corresponding notion is that of compulsory purchase.} In legal language, especially in the US, takings by eminent domain are often referred to as takings {\it simpliciter}, while talk of other kinds of ``takings'' require further qualification, e.g., in case of ``takings'' based on contract, taxation, or adverse possession. The US terminology is intuitive and helps bring the issue of legitimacy to the forefront, so I will adopt it throughout this thesis.\footnote{Sometimes it is convenient to draw up the notion of a taking more widely than I do in this thesis, e.g., to include adverse possession, see \cite[19-21]{waring09}. However, such a broader notion will not be used in this thesis. This choice appears natural in light of how the thesis focuses specifically on takings motivated by a government's desire to facilitate concrete economic development projects.}

When guided by eminent domain, the taking of private property without the owners' consent is not theft. But it is not necessarily that far removed from it either; the default assumption is that takings are legitimate, but if they are not, one may well be permitted to call them by a different name.\footnote{See \cite[8-10]{gray11} (discussing case law from the US, with state judges describing illegitimate takings as ``plunder'', ``rapine'', and ``robbery'').}

More generally, the idea that the government's power to take is not unlimited seems fundamental. Indeed, the expectation that an owner might find occasion to resist an act of taking, and may or may not have good grounds for doing so, appears deeply rooted in pre-legal intuitions.\footnote{See, e.g., \cite[159]{gray94}.} This raises the question of how to approach the legitimacy of takings in legal reasoning and what conceptual categories we can benefit from when doing so. This is the key question that is addressed in this thesis, for the special case of so-called {\it economic development takings}.\footnote{For a sample of scholarship based on this term, see \cite{cohen06,somin07,wilt09,yellin11}.}

Such takings occur when a government uses the power of eminent domain to stimulate economic growth, typically by providing property to for-profit companies for use in a concrete development project. The canonical US example is {\it Kelo v City of New London}, which resulted in great controversy and a surge of academic work on the legitimacy of takings.\footcite{kelo05}

The {\it Kelo} case concerned several homes that were taken by the government in order to accommodate private enterprise, namely the construction of new research facilities for Pfizer, the multi-national pharmaceutical company. Several home-owners, among them Suzanne Kelo, protested the taking on the basis that it served no public use and was therefore illegitimate under the Fifth Amendment of the US Constitution. The Supreme Court eventually rejected their arguments, but this decision created a backlash that appears to be unique in the history of US jurisprudence.\footnote{See generally \cite{somin08}.}

In their mutual condemnation of the {\it Kelo} decision, commentators from very different ideological backgrounds came together in a shared scepticism towards the legitimacy of economic development takings.\footnote{See \cite[1413-1415]{bell06} (``Everyone hates {\it Kelo}'', commenting on how criticism was harsh from across the political spectrum).} Interestingly, their scepticism lacked a clear foundation in US law at the time, as the {\it Kelo} decision did not appear particularly controversial in light of established eminent domain doctrines.\footnote{See, e.g., \cite[1418]{bell06} (``The most astounding feature of {\it Kelo}, as even the case's harshest critics agree, is that from a legal standpoint, the ruling broke no new ground.'').} Hence, when the response was overwhelmingly negative, from both sides of the political spectrum, it seems that people were responding to a deeper notion of what counts as legitimate.

%Indeed, the critical response to {\it Kelo} appears to have been a reflection of widely shared sentiments. As such, it also arguably involved pre-legal notions pertaining to legitimacy. Simply stated, people from across the political spectrum simply found the outcome {\it unfair}.

If the law is meant to deliver justice, widely shared intuitions about legitimacy deserve attention from legal scholars. In the US, legitimacy intuitions pertaining to economic development takings have indeed received plenty of attention after {\it Kelo}. Despite the outcome of the case, it is now hard to deny that cases such as {\it Kelo} belong to a separate category of takings that raises special legal questions.\footnote{See, e.g., \cite{cohen06,somin07}.} As this change in the narrative was largely the result of a popular movement, there is reason to think that the category of economic development takings is in itself a powerful addition to the discourse on legitimacy, potentially relevant also outside of the US.

%When cases like {\it Kelo} are portrayed as being primarily about bestowing a benefit on powerful commercial interests, it becomes natural to question their legitimacy, irrespective of details in the surrounding legal framework. But when is it appropriate to deride economic development takings in this way? Moreover, should the law provide a basis for the courts to intervene, to rescind illegitimate takings?%Both of these questions will be addressed in this thesis. To address them effectively, 
To explore this further, it should first be acknowledged that there is a {\it risk} that takings for economic development can be improperly influenced by commercial, rather than public, interests. This risk is clearly higher in economic development situations than in cases when takings take place to benefit a concretely identified public interest, such as the building of a new school or a public road. Hence, it is already intuitively reasonable to single out economic development takings for special attention at the political and normative level. However, should the categorisation also be recognised as a basis for justiciable restrictions on the takings power?

This is not obvious, as it conflicts with the prevalent idea that governments enjoy a ``wide margin of appreciation'' when it comes to their use of eminent domain.\footnote{This expression has been used by the European Court of Human Rights, see \cite[54]{james86}. In the US, the same attitude was clearly a factor motivating the majority in \cite{kelo05}.} However, as the US debate shows, it might be hard to deny judicial review as soon as the special features of economic development takings are brought into focus. This points to the first main theme of this thesis: an analysis of economic development takings as a conceptual category for legal reasoning about property protection.

\section{Property Theory and Economic Development Takings}\label{sec:1:1}

In Part I, this thesis will argue that economic development takings should be recognised as a distinct category of takings at the theoretical level, with respect to fundamental rules that protect private property. This claim will be made on the basis of a theory of property that is broader than typical approaches found in the law and in legal scholarship. Specifically, the thesis rejects the view that private property should be understood as a form of entitlement protection.\footnote{For a famous entitlements-based view of private property, see \cite{calabresi72}.}

Instead, chapter \ref{chap:2} will argue for a social function understanding, with an emphasis on human flourishing as the normative foundation for private property.\footnote{For human flourishing theories of property generally, see \cite[chapter 5]{alexander10}.} In short, property should be protected because it can help people flourish, as members of a democratic society.\footnote{See also \cite[1089]{crawford11}.} Moreover, property is meant to serve this function not only for the owners themselves, but also for the other members of their communities.\footnote{See generally \cite{gray94,alexander09d,alexander14}.}

Such an ambitious take on property must necessarily also give rise to a broader assessment of legitimacy when the state interferes with it.\footnote{See also \cite{underkuffler06}.} This is what inspires my initial discussion on economic development takings in chapter \ref{chap:2}. There I will present the basic definition of the notion and discuss the {\it Kelo} case in some more detail. Specifically, I will argue that Justice O'Connor's strongly worded dissent -- finding that the taking should be rescinded -- is consistent with, and subtly conducive to, a social function perspective on property.\footnote{See \cite[494-505]{kelo05}.}

It should be emphasised that the focus will be on the question of when a taking is legitimate as such, not the question of how much compensation should be paid to the owners. Of course, the two questions are related; the amount of compensation offered can influence the degree of legitimacy of the interference. Some scholars go further and argue that the legitimacy question is primarily about finding the appropriate mechanism for awarding compensation.\footnote{See generally \cite{fennell04,bell07,lehavi07}.} With the theoretical approach to property adopted in this thesis, this view must be rejected; the social functions of property are not reducible to financial entitlements. Moreover, the aim of this thesis is to discuss precisely those aspects of legitimacy that {\it cannot} be addressed through compensation. The link to compensation will be mentioned when it seems relevant, but the compensation issues that arise for economic development takings will not be analysed in any depth.\footnote{For such an analysis, see \cite{dyrkolbotn15a}.}

While I will advocate for a broad approach to the question of legitimacy of economic development takings, this thesis will remain focused on private property. Other potential consequences of economic development, such as effects on the environment and social welfare impacts, will be considered as they arise in disputes about takings, not as issues in their own right. These aspects of economic development will therefore receive less attention here than they would in a thesis focusing specifically on environmental law or social and economic rights.  That said, one of the main arguments made in chapter 2 is that the social function perspective on property implies that we should take broader societal and environmental effects into account also when assessing the legitimacy of interfering with private property. The thesis argues for this coming from property theory, and in so doing will also touch on the conservation and social justice dimensions of economic development. In addition, chapter 2 will argue that if the social functions of private property support egalitarianism and equity at the local level, then private property can serve as an anchor for justice also with respect to the environment and the social and economic rights of non-owners. This point will also be made in Part II of the thesis, when discussing the issue of hydropower development in Norway. In future work, I hope to develop the idea further, to embark on research that will connect the social function account of property even more closely with environmental law and social and economic rights.

Chapter \ref{chap:3} builds on the property theory developed in chapter \ref{chap:2} by giving a more in-depth presentation of the legitimacy question, leading to a proposal for a justiciable legitimacy standard that makes judicial intervention possible. To make the discussion concrete, the chapter first offers a brief review and comparison of jurisprudential developments in the US, the UK, and at the European Court of Human Rights. These jurisdictions are chosen specifically for their many close connections with the Norwegian system, making them a natural reference point also for the case study in Part II of the thesis. Moreover, all the countries discussed are in comparable socio-economic situations, with property having a similar social function across the jurisdictions studied. This permits a comparative discussion that focuses specifically on the issue of takings, reducing the risk that the analysis will be distorted by significant differences in the social and economic context of takings law across different jurisdictions. The broader insights gained from the work done in this thesis might still be highly relevant to jurisdictions that have not been explicitly discussed. But a further treatment of this issue, for instance with respect to jurisdictions from the developing world, raises additional questions that must be left for future work.

Based on the choice of jurisdictions justified above, the thesis reviews several approaches to judicial review, culminating in a recommendation for a perspective based on institutional fairness that I trace to recent developments at the European Court of Human Rights. Specifically, the Court in Strasbourg has begun to look more actively at the systemic reasons why violations of human rights occur, in order to address structural weaknesses at the institutional level in the signatory states.\footnote{See generally \cite{leach10}.} This approach is arguably one that fits very well with the sort of analysis carried out by Justice O'Connor in {\it Kelo}, perhaps more so than the approach induced by the Fifth Amendment.

Importantly, the institutional perspective appears to be a sensible middle ground between procedural and substantive approaches to legitimacy, directing us to focus on decision-making processes without giving up on substantive fairness assessments. To ensure fairness, in particular, is not just about making good decisions, but also about how those decisions came about, and how likely it is that the system will have disproportionate effects at the societal level. This way of thinking about legitimacy brings me to the second main theme of this thesis, concerning the question of {\it democratic merit}.

\section{A Democratic Deficit in Takings Law?}\label{sec:1:2}

As discussed in chapter \ref{chap:2}, two of the key social functions of property is to promote social justice and to facilitate democratic decision-making.\footnote{See also, with further references, \cite{rose96,jackson10}.} In addition, property is meant to serve as a bridge between individual needs, community interests, and policy making at the national and the international stage. Through the law of property, societal priorities can be communicated to owners and communities without depriving them of their right to self-governance.\footnote{This is discussed in depth in chapter \ref{chap:2}, sections \ref{sec:2:4} and \ref{sec:2:5}.}

These functions of property are easily undermined if there is an excessive concentration of power and wealth among the elites of society. As indicated by Justice O'Connor's dissent in {\it Kelo}, this is one of the key reasons why economic development takings should be looked at with suspicion. In short, the concern is that economic development takings can both reflect and exacerbate a {\it democratic deficit}.

%hat can be applied to economic development takings. This test consists of a list of indicators that can suggest eminent domain abuse. The role of property in this regard is particularly clearly felt at the personal and local level, as a fair distribution of property is a highly effective safeguard for the basic rights of individuals and communities. In addition, property is meant to serve a key function as a bridge between the local level and policy making at the national and the international stage. Through the law of property, many societal priorities that might otherwise necessitate direct governmental control and interference can be effectively communicated to communities without depriving them of their right to self-governance.

There are many symptoms that can suggest a lack of democratic legitimacy, and to make the discussion more concrete, chapter \ref{chap:3} proposes a legitimacy test consisting of nine indicators of eminent domain abuse. The first six points are due to Kevin Gray, while the final three are additions I propose on the basis of the work done in this thesis.\footnote{For Gray's original points see \cite{gray11}.} I call the resulting list the Gray test, a heuristic for inquiring into the legitimacy of an economic development taking. 

Arguably, the most important indicator is the one pertaining to the overall democratic merit of the taking (one of my additions). When taken together with the other points, this indicator should induce an assessment of legitimacy against the decision-making process as a whole. Hence, it emphasises the institutional fairness perspective. If a taking fails the legitimacy test on this point, it might indicate an existing weakness of the system or a trend towards deterioration of the institutional framework surrounding eminent domain. This problem, moreover, might not be noticeable unless one considers an aggregated view of all the indicators of the Gray test, to shed light on what they tell us about the democratic legitimacy of existing practices.

Admittedly, asking courts to test for legitimacy is an incomplete response to the worry that economic development takings might result from, and give rise to, a democratic deficit at the societal level. This point has been argued by some US scholars, who claim that increased judicial scrutiny is neither a necessary nor a sufficient response to concerns about the institutional legitimacy of takings such as {\it Kelo}.\footnote{See generally \cite{lehavi07,heller08}.} Instead, these scholars try to come up with institutional innovations that can restore legitimacy in cases when the government wishes to ensure economic development on private property.

The most notable work in this direction so far is that of Heller and Hills, proposing what they call Land Assembly Districts (LADs) as possible alternatives to the use of eminent domain.\footnote{See \cite{heller08}.} The idea is that LADs will be set up to replace the traditional takings procedure in cases where property rights are fragmented and the potential takers have commercial incentives. The basic mechanism is one of self-governance; the owners themselves should be allowed to decide whether or not development takes place, by some sort of collective choice mechanism (possibly as simple as a majority vote). In this way, the holdout problem can be solved (individual owners cannot threaten to block development to inflate the value of their properties). At the same time, the local community's right to manage its own property is recognised and respected.
 
The LAD proposal is closely linked to more general ideas about self-governance and sustainable resource management, particularly the theories developed by Elinor Ostrom and others.\footnote{See \cite{ostrom90}. For the connection with property theory generally, see \cite{ostrom10b,rose11,fennel11}.} On the basis of a large body of empirical work, these scholars have formulated and refined a range of design principles for institutions that can promote good self-governance at the local level.\footnote{For a more recent empirical assessment (and refinement), see \cite{cox10}.}

At the end of chapter \ref{chap:3}, I argue that this work can be used to address the legitimacy of takings in a principled way, to arrive at refinements or alternatives to the proposal made by Heller and Hills. Specifically, it seems that alternatives to expropriation based on self-governance can be a powerful way to address the worry that economic development takings might otherwise be associated with a democratic deficit. At the same time, the context-dependence of solutions along these lines make sweeping reform proposals unlikely to succeed. Rather, it is important that the institutions that are used are appropriately matched to local conditions.\footnote{See \cite[92]{ostrom90}.} This sets the stage for the second part of the thesis, consisting of a case study of takings for Norwegian hydropower development.

%For instance, a setting where property is evenly distributed among members of the local community might suggest a very different type of institution compared to a setting where the relevant property rights are all in the hands of a small number of absentee landlords. In short, the idea of using self-governance structures in place of eminent domain necessitates a more concrete approach, a move away from property theory towards propty practice. 

%The first key objective of this case study is to apply the theory developed in the first part to analyse the legitimacy of takings for hydropower. The second objective is to study a concrete institutional alternative to expropriation in more depth, namely the system of {\it land consolidation courts}. In Norway, these courts are empowered to set up self-governance organisations for local resource management and economic development, if necessary against the will of individual owners.

%In light of this, the case study will shed light on both of the two key conclusions drawn in the theoretical part of the thesis.

%Alternative test the theoretical assertions made about how to approach legitimacy. Specifically, the question to be addressed is to what extent the traditional narrative of takings is capable of doing justice to the property conflicts that have arisen regarding the development of hydropower in Norway.

\noo{ I arrive at several objections against the details of the particular institutional arrangements proposed, particularly with regards to their likely effectiveness. It seems, in particular, that both proposals fail to recognise the full extent to which prevailing regulatory frameworks concerning land use and planning would have to be reformed in order to make their proposals work.

At the same time, I argue that these novel institutional proposals are extremely useful in that they point towards a novel way to frame the issue of legitimacy in takings law. In particular, I explore the hypothesis that traditional procedural arrangements surrounding takings suffer from a democratic deficit, a particularly powerful source of discontent in economic development cases.

This idea is the second key focus point of my thesis. First, I approach it from a theoretical point of view, by exploring the notion of {\it participation} and its importance to the issue of legitimacy, particularly in the context of economic development. It seems, in particular, that {\it exclusion} could be a particular worrying consequence of certain kinds of economic development takings, namely those that lack democratic legitimacy in the local community where the direct effects of the taking are most clearly felt.

I believe this to be a promising hypothesis, and I back it up by considering the social function theory of property and the notion of human flourishing which has recently been proposed as a normative guide for reasoning about property interests. I pay particular attention to the importance of communities that has been highlighted in recent work, as a way to bridge the gap between individualistic and collectivist ideas about fairness in relation to property.

I take this a step further, by arguing that a focus on communities naturally should bring institutions of local democracy to the forefront of our attention. The role that property plays in facilitating democracy has been emphasised before by other scholars, and I think it has considerable merit. However, I also argue that it is important to resist the temptation of viewing its role in this regard through an individualistic prism. It is especially important to take into account additional structural dimensions that may supervene on both property and democracy, such as tensions between the periphery and the centre, the privileged and the marginalised, as well as between urban and rural communities.

It is especially important, I think, to appreciate the effect takings can have on local democracy. For one, excessive taking of property from certain communities might be a symptom of failures of democracy as well as structural imbalances between different groups and interest. But even more worrying are cases when the takings themselves, brought on by a commercially motivated rationale, appears to undermine the authority of local arrangements for collective decision-making and self-governance. This dimension of legitimacy, in particular, is one that I devote special attention to throughout this thesis.

I also believe, however, that it is hard to get very far with this sub-theme through theoretical arguments alone. Hence, to explore it in more depth, I go on to assess it from an empirical angle, by offering a detailed case study of takings of Norwegian waterfalls for the purpose of hydropower development. This case study, in turn, will allow me to cast light on two further key themes, that I now introduce. %This brings me to the second part of my thesis, which in turn consists of two main themes, where the latter aims to bring me back towards a more general setting, by delivering some recommendations for how best to deal with economic development takings.
}
%I go on to consider the hypothesis that economic development takings demonstrate that takings law suffer from a {\it democratic deficit}.

\section{Putting The Theory to the Test}\label{sec:1:3}

In Norwegian law, the legitimacy questions that arise with respect to takings usually begin and end with the issue of compensation.\footnote{See generally \cite{dyrkolbotn15,dyrkolbotn15a}.} If an owner has grievances about the act of taking as such, rather than the amount of money they receive, takings law has very little to offer. In fact, it does not appear to offer anything that does not already follow from general administrative law. The owner can argue that the taking decision was in breach of procedural rules, or grossly unreasonably, but the chance of succeeding is slim.\footnote{See \cite[384-386]{dyrkolbotn15b}.}

%This narrative of legitimacy is not unique to Norway. It seems that in Europe, unlike in the US, the issue of legitimacy is often seen as predominantly concerned with the issue of compensation. In particular, the jurisprudence at the ECtHR is typically focused on compensatory issues. Moreover, while many constitutions of Europe, including the Norwegian, include public interest clauses, the courts make little or no use of these when adjudicating takings complaints. In the words of the ECtHR, the member states enjoy a ``wide margin of appreciation'' when it comes to determining what counts as a public interest.

In cases involving hydropower development, the position of property owners is also strongly influenced by sector-specific legislation, as well as special administrative and market practices. Chapter \ref{chap:4} begins the case study by discussing this in more depth, setting the stage for the discussion on expropriation that follows in chapter \ref{chap:5}. A first important observation is that the hydropower sector in Norway was liberalised in the early 1990s.\footnote{The crucial legislative reform was the \cite{ea90}.} This means that the energy companies benefiting from eminent domain are now commercial enterprises, not public utilities.

A second important observation is that the right to harness the power of water is considered private property in Norway, typically owned in common by members of nearby rural communities.\footnote{See \indexonly{wra00}\dni\cite[13]{wra00}. This arrangement has long historical roots and makes intuitive sense in a mountainous country with a very vast number of small and medium sized rivers coming down from steep outfield mountains. For the historical development of the law on this point, see \cite[109-116]{nordtveit15}.} This does not mean that freely running water, as a substance, is subject to private property. What it means is that riparian owners have an additional stick in the bundle of rights that the law associates with being the owner of land over which water flows. A useful comparison can be made with fishing rights; the right to the hydropower in a river arises from landownership, but it is conceived of as a separate, transferable, right in property.\footnote{Apart from this explicit recognition of water power as a separate right in property, the Norwegian system of riparian rights appears to be historically quite similar to the riparian common law, see generally \cite{howarth15}.} It is referred to in Norwegian as a ``fallrett'', which can be translated as a {\it waterfall right}. This thesis will therefore often refer to waterfalls and owners of waterfalls when discussing the right to harness the power of water in a river.\footnote{In some cases, especially historically, a waterfall right would be formally registered as a separate unit of real property to facilitate transfer to someone other than the owner of the surrounding agricultural land. However, waterfall rights can also be formally registered as rights of use attaching to the real properties from which they arise. In relation to Norwegian expropriation law, and for the purposes of this thesis, the distinction between these two ways of registering waterfall rights will not play an important role and will not be discussed further.}

As discussed in Part II of this thesis, hydroelectric companies in Norway have traditionally had easy access to privately owned waterfalls, made possible through the government's power of eminent domain. However, since deregulation, local owners have begun to resist such takings. This has been motivated by the fact that owners can now undertake their own hydropower projects as a commercial pursuit; unlike the situation before liberalisation, owner-led development projects can demand access to the electricity grid as producers.\footnote{See, e.g., \cite{uleberg08}.} This has led to heightened tensions between takers and owners, tensions that the water authorities are now forced to grapple with on a regular basis.

%As a result, local owners now regularly protest expropriation of their rights on the grounds that they wish to {\it participate} in economic development, by carrying out alternative development projects, or by cooperating with the energy companies who wish to take their water rights. Hence, while liberalisation has rendered takings for hydropower as takings for profit, it has also empowered local owners and communities to propose alternatives. Unsurprisingly, this has led to tensions that the water authorities are now forced to grapple with on a regular basis.\footnote{See Chapter \ref{chap:4}, Section \ref{sec:4:4}.}

Chapter \ref{chap:4} argues that despite their improved position following liberalisation, local owners remain marginalised under the regulatory framework. Specifically, despite political support for locally organised small-scale development, the large energy companies have continued to enjoy a privileged position in their dealings with the water authorities. Building on this observation, chapter \ref{chap:5} goes on to discuss eminent domain in more depth. The chapter tracks the position of owners under the law and administrative practices that relate to takings of waterfalls. The key finding is that expropriation is usually an {\it automatic consequence} of a large-scale development license.\footnote{In some cases, this follows explicitly from the water resource legislation, while in other cases it follows from administrative practice. For further details, see below in chapter \ref{chap:5}, section \ref{sec:5:3}.} That is, commercial companies that succeed in obtaining large-scale development licenses will almost always be granted the right to expropriate. This right will be granted, moreover, with little or no prior assessment as to the appropriateness of depriving local communities of their resources.

While the appropriateness of taking property from local people is rarely discussed, the issue of how hydropower affects the environment has received increased attention in recent decades. Sometimes, environmental impact assessments will uncover negative effects and the water authorities will reject development applications, also when the applicant is a large energy company. In general, the framework for management of hydropower in Norway has an important conservation dimension that is clearly recognised by the government. It should be noted that conservation issues are often orthogonal to the property question. In some cases, local owners of waterfalls will oppose large-scale development projects that damage the environment, while in other cases, environmental interests will block small-scale projects that the owners themselves would like to carry out. The important point made in this thesis is that while the Norwegian system has an elaborate framework in place to evaluate the environmental consequences of hydropower, the effect on local owners and communities receives little or no attention. This is also reflected in previous scholarship of the law of hydropower. For this reason, the present thesis focuses specifically on property, without giving a separate treatment of conservation issues. This means, for example, that the thesis will not discuss the special issues that arise when the property rights of local owners are restricted to conserve the local environment, a relatively common example of property interference in water law, but one that does not qualify as an economic development taking. 

Although conservation issues in water law are not dealt with in any depth in this thesis, such issues will be discussed when they have a bearing on the legitimacy questions that arise when energy companies expropriate waterfalls. As I will demonstrate in chapter \ref{chap:5}, environmental organisations and energy companies both enjoy a significantly stronger position within the regulatory framework than local owners and communities. Recently, there has been a tendency for these two power groups to reach compromises, such that large-scale development projects are allowed to go ahead while small-scale projects are stopped because of their environmental effects. Indeed, environmental groups and commercial companies now appear to be reaching a form of mutual understanding that large-scale development is {\it better} for the environment than small-scale projects. As discussed in chapter \ref{chap:5}, this conclusion rests on what appears to be a very narrow and arguably misguided understanding of what sustainability and conservation should entail in the context of hydropower development. Moreover, when environmental groups and large energy companies unite in this way, it raises the worry that local owners and their communities will be marginalised further.

Indeed, as chapter \ref{chap:5} will demonstrate, the owners' position during the licensing assessment stage is highly precarious, contrasting both with the strong position of the development companies and the similarly influential role played by conservation interests.\footnote{See especially the discussion in chapter \ref{chap:5}, sections \ref{sec:5:6} and \ref{sec:5:7}.} The fact that expropriation tends to follow automatically from a license to develop has led the water authorities to focus their attention on the licensing question and associated procedures. No distinction appears to be made between cases involving expropriation and cases that do not. This has a significant effect on the level of procedural protection offered to local owners. For instance, according to written testimony during a recent Supreme Court case on legitimacy, the water authorities do not recognise any duty to give individual notice to local owners before processing applications that involve expropriation of their waterfalls.\footnote{The case in question was \cite{jorpeland11}.}

In relation to the compensation issue, the owners' legal position initially grew stronger after liberalisation. Specifically, the lower courts started to compensate local owners for the lost opportunity to profit from hydropower.\footnote{See \cite{uleberg08} (specifically, it was observed that waterfalls now had a market value, due to the increasing prevalence of owner-led hydropower).} This led to a dramatic increase in compensation payments compared to earlier practice.\footnote{See especially the discussion in chapter \ref{chap:5}, section \ref{sec:5:5:1}.} However, a recent decision from the Supreme Court appears to largely reverse this development, since a large-scale license may now be considered proof that alternative development by owners is unforeseeable and therefore not compensable.\footnote{See \cite{otra13}.} 

In light of this and other data discussed in chapter \ref{chap:5}, my conclusion is that recent takings for hydropower do not in fact pass the Gray test. The current practices appear illegitimate with respect to the theory of property developed in Part I of the thesis.

At the same time, Norwegian law offers a promising institutional path towards the restoration of legitimacy in economic development contexts. Specifically, the unique framework for land consolidation found in Norway can serve such a function. This has already been demonstrated in the context of hydropower development, where land consolidation courts have been able to successfully organise development projects on behalf of owners who wish to undertake development but disagree about how it should be done. This brings me to the fourth key theme of this thesis.

\section{A Judicial Framework for Compulsory Participation}\label{sec:4}

\noo{ In Norway, the distribution of property rights across the rural population is traditionally highly egalitarian.\footnote{This is discussed in more depth in chapter \ref{chap:4}, Section \ref{sec:4:2}.} This meant that the farmers in Norway soon became an active political force, particularly as representative democracy started to gain ground as a form of government in the 19th century.\footnote{As early as in 1837, the Norwegian parliament was so dominated by farmers that it came to be described as the ``farmer's parliament''. See \cite{hommerstad14}.}

%The Norwegian farmers were often little more than small-holders, and had few privileges to protect. Hence, they became liberals of sorts (although also known for their fiscal conservatism). The farmers as a class were responsible for pushing through important early reforms, such as the abolition of noble titles and the establishment of democratically elected municipality governments.

%However, the municipality governments were not the first example of local decision-making institutions in Norway.
The highly fragmented ownership of land meant that institutions for collective decision making had to be introduced early on in Norwegian history; some even argue that the first realisation of a truly direct democracy can be traced to Norway in the Viking age.\footnote{See \cite[23]{titlestad14}.} One of the ancient institutions for collective action is the land consolidation court. 
}
The fourth and final key theme, presented in chapter \ref{chap:6}, consists of an assessment of the Norwegian land consolidation courts. These courts have the power to order owners to undertake or allow development projects (without depriving them of their property), as an alternative to expropriation. Moreover, they are presently used in this way in the context of hydropower development. The large energy companies almost never use consolidation, but local communities often do.\footnote{According to the Court Administration, as of 2009, land consolidation proceedings had facilitated a total of 164 small-scale hydropower projects with a total annual energy output of about 2 TWh per year (enough electricity to supply a city of about 250 000 people), see \cite{dom09}.} In these cases, the land consolidation courts have proved themselves effective in making self-governance work, also in cases when some of the owners do not with to undertake development.

\noo {The typical scenario is that the owners disagree about who owns what and cannot agree on how to organise development. In other cases, some of the owners, or even a majority of them, do not wish any development at all. In these cases, it is possible for the courts to {\it compel} them to participate. 

In these situations, it is less clear how well consolidation works in practice. Plainly, there has not been enough cases of this sort to draw a clear conclusion, especially not in situations when those who favour development are a minority among the owners. However, the consolidation alternative still appears highly preferable to the expropriation alternative, especially in terms of legitimacy. Specifically, the owners who are compelled to participate do not loose their property and are not excluded from the decision-making process.}

The land consolidation alternative can make a great difference, especially since it strives to ensure legitimacy through participation. The potential democratic deficit associated with economic development takings  is dealt with by mechanisms that seek to enable owners to take active part in the management of their property in the public interest. At the same time, the procedure can be quite effective, since participation is compulsory and the consolidation judge may intervene to settle conflicts and establish organisational order. Chapter \ref{chap:6} also addresses possible objections to the procedure, but concludes that the continued development of the land consolidation institution provides the best way forward for addressing problems associated with economic development takings in Norway.

%Finally, the institution of land consolidation is assessed against Land Assembly Districts, and -- more generally -- against the idea of self-governance frameworks for managing common pool resources. I argue that it compares favourably, both because it comes equipped with in-built judicial safeguards, but also because it has such a broad scope. I note, however, that its successful use is dependent on both political will and an ability to retain key feature even in the presence of new and powerful stakeholders in the consolidation process.

If the integrity and efficiency of the procedure can be preserved, it appears to have great potential as an alternative to eminent domain in general, also in cases involving large-scale development and cooperation with external commercial actors. Moreover, while the system is designed to work in a setting of egalitarian property rights, it is interesting to consider whether key features of the procedure could inspire solutions to the takings problem in other jurisdictions. %Specifically, the fact that the procedure focuses on benefiting properties rather than owners means that a broader understanding of property can itself suggest a broader range of possible applications. 

It might well be, for instance, that a land consolidation approach coupled with a human flourishing understanding of property can be a good way of including non-owners in the process, in jurisdictions where property rights are not distributed as widely among the population as in Norway. This might make the procedure more complex and give rise to new risks of abuse by local elites, but it seems like an interesting idea to explore in future work. 

In short, the consolidation alternative provides a starting point for an approach to legitimacy that takes a wider view of what property is, and what role it can and should play in a democratic society. In this way, the chapter on consolidation also returns to the conceptual premise discussed in the first chapter of this thesis, whereby the purpose of property is to promote human flourishing.

%the core features of land consolidation for economic development can be preserved and developed further, 
\noo{ In the second part of the thesis, I put the theoretical framework to the test by applying it to a concrete case study, namely that of Norwegian hydropower. Following liberalisation of the energy sector in the early 1990s, hydropower is now a commercial pursuit in Norway. Moreover, there is a long tradition for granting energy producers the power to acquire property compulsorily, including the necessary rights to exploit the energy of water, rights that are subject to private property under Norwegian law. This has resulted in tension and controversy, however, as the original owners of these rights, typically local farmers and small-holders, see the commercial potential of hydropower being transferred to other commercial interests, to the detriment of their own, and their communities', interest in self-governance and economic benefit.}

\noo{
\section{Some terminological and conceptual clarifications}

{\it Property} is a key notion in this thesis. As mentioned already, it is an elusive legal term, with different decomposable meanings depending on the context of use and the jurisdiction within which we find ourselves. In the first part of this thesis, the notion is explored conceptually, to develop a theory of property's role and purpose within law and society. The details of how the notion is defined in a given jurisdiction will not be our concern. In general, there is quite some variation in this regard, among the different jurisdictions considered in this thesis. With respect to the European Convention of Human Rights, for instance, we encounter a notion of property (or ``possession'') that is very wide. The property clause in the Convention relies on a concept of property that covers a range of social welfare entitlements and immaterial benefits, including future pension payments and goodwill acquired by holding a professional title.\footnote{See \cite[73-77]{allen05}.} This is a broader concept of property than that usually encountered in private law, also in jurisdictions that incorporate the Convention into their national law. The issues that can arise form this, when several distinct notions of property co-exist in the law, will not be considered in this thesis; rather, the thesis will remain focused on ``classical'' instances of property, typically property in land and related resources.

That said, the theoretical argument made in chapter 2 might well be relevant for property lawyers working with disputed definitions of private property within a specific legal framework. Indeed, the theory presented in this thesis can be used to argue normatively that a given jurisdiction relies on a notion of property that is either too wide or too narrow to cater to important social functions. For instance, it would be interesting to consider the implications that a social function understanding can have in the context of intellectual property, specifically to shed light on the normative question of what kinds of incorporeal property the law {\it should} recognise. However, this line of research must be left for future work.

There is one special type of property encountered in this thesis that deserves a special mention. This is the notion of {\it common property} in land and natural resources. This notion is notoriously ambiguous, used to refer to at least three different kinds of legal arrangements.\footnote{See generally \cite{fennel11}.} First there are open-access resources, which are sometimes (erroneously) referred to as common property. These resources are characterised by the fact that everyone is in principle entitled to make use of them. Hence, it is more accurate to say that they are resources that have no owner. The use of such resources is typically managed by the government through regulation, sometimes under a public trust doctrine. The questions of sustainable resource management and governance that arise in this regard are interesting in their own right, but are not considered in any depth in this thesis.

The second type of legal arrangement often referred to as common property is the collection of rights and responsibilities attaching to common land. This is land over which a specific group of people enjoy use rights and where special rules are in place to regulate the exercise of these rights and the management of the underlying resources. A typical example is found in the law of the commons in the UK.\footnote{See, e.g., \cite{rodgers11}.} Use rights on the commons can be thought of as property rights, but under individualistic accounts of what property is, it might not be appropriate to do so. The distinguishing feature of rights in common is that they provide an anchor for a special legal framework, a set of rules, institutions and customs that pertain specially to the communal character of such rights. This function of rights in commons can be distorted if those rights are fitted into an entitlements-based framework for maintaining rights in property.\footnote{See \cite{rodgers11} (criticising the Commons Registration Act 1965 (England and Wales) on such a basis).} 

Under a social function theory of property, by contrast, it becomes much more appropriate (at the conceptual level, at least) to think of rights in common as private property rights. Moreover, as discussed further in chapter 2, the social function theory can support arguments to the effect that {\it all} instances of property, even traditional forms of private property, can be viewed as a commons in an abstract sense.\footnote{See also \cite{fennel11}.} This point will also be made in chapter 6, when I discuss how land consolidation can be used to {\it set up} institutions for collective management of private property rights within a local community. This form of property intervention can be understood as an effort to bring key ideas behind the commons to bear on private property rights. That said, this thesis will not discuss any concrete legal frameworks that specifically target the traditional concept of a commons, except briefly in chapter 4 when presenting different property regimes found in Norway.

\noo{Moreover, according to wider, more functional, definitions of what private property is, rights in common may well be covered. There can be little doubt, for example, that the use rights of individuals having rights in common over some resource {\it do} constitute property rights (``possessions'') within the meaning of Article 1 of Protocol 1 of the European Convention of Human Rights.

At the same time, the distinguishing feature of rights in common is that they are surrounded by a special legal framework, a set of rules, institutions and customs that pertain specially to the communal character of such rights. This thesis will not investigate concrete examples of such legal frameworks, except briefly in chapter 4 when I present different property regimes found in Norway.}

%That said, commons also tend to come with many specific regulatory provisions and institutional arrangements, none of which 

%positive law regulations which might not bthis thesis will not explore in any depth those special rules and arrangements that are in place to regulate the commons in any specific It should be noted, however, that the special questions that might arise when economic development takings take place in the commons will not be addressed in this thesis. The theory developed herein should be applicable, but further exploration of this must be left for future work.

The third legal arrangement that can be referred to as common property is encountered when a property is owned, in the standard private law sense of the word, by a group of owners. Shared forms of private ownership are supported by most jurisdictions, including those considered in this thesis. Shared private ownership is particularly important in Part II of the thesis, since the takings discussed there will typically involve rights to water that are owned by several private parties in common under a legal framework that most closely resembles the common law concept of a tenancy in common. A brief presentation of this form of private ownership in Norway is provided in chapter 5, along with a discussion of the importance of egalitarianism in Norway.

%affect local communities as a whole, not just individuals. However, the cases considered will all be cases where individuals have recognised rights in property, meaning that the cases fall uncontroversially within the ambit of takings law. The margins of takings law, encountered for instance if groups of non-owners make proprietary claims based on customary use rights or the like, will not be considered in any depth. However, the theory of property developed in this thesis might suggest making normative claims to the effect that legal standing in takings proceedings should be extended to cover a larger group of legal persons than those presently recognised. A closer examination of this is left for future work. 

I%n the second part of the thesis, when dealing specifically with Norwegian law, we will encounter a few specific forms of private property that deserve a special mention. 
%In Norway, we find a uniquely egalitarian distribution of land ownership, where land and the resources found on it are typically owned by groups of local small-holders, not landlords or public bodies. This form of shared ownership is considered a conventional form of private ownership under Norwegian law. There are also some large commons in Norway, but they are of lesser practice importance due to the prevalence of shared private ownership over outfields. Further details on property arrangements found in Norway are provided in chapter \ref{chap:3}.

{\it Legitimacy} is a second key notion used in this thesis. The notion is consistently used in a normative sense, to describe that an interference in private property appears justified. It is not a term with a specific descriptive meaning within a given jurisdiction. A key aim of this thesis is to address the question of when an interference in private property {\it should} be regarded as legitimate. All the jurisdictions I consider have their own specific rules in place that are meant to ensure legitimacy in takings law. The most common legal terms that are used in this context are the notions of {\it public use}, {\it public purpose}, and {\it public interest}. Specifically, a typical takings provision states that the public must benefit, directly or indirectly, in order for an interference in private property to count as legitimate. Such provisions or their near equivalents can be found in a range of different jurisdictions, including all those studied in this thesis.

The meanings of the terms used are similar across different contexts and legal systems. Still, since these are formal legal terms, it is worth keeping in mind that their meaning is relative to the jurisdiction under consideration. For instance, while most of the jurisdictions considered in this thesis do not recognise any substantial difference between public use, public interest and public purpose, some jurisdictions maintain such distinctions and attach important legal consequences to them. In some cases, the correct way to make these distinctions is a highly controversial question. Most famously, the position that public use literally means use by the public, and is therefore quite distinct from public interest and public purpose, is forcefully advocated by several US legal scholars, including at least one member of the Supreme Court.\footnote{See \cite{kelo05}.}

When terms such as public use, public interest or public purpose are used in this thesis, their exact meaning corresponds to the meaning given to them by the jurisdiction under consideration. If the terms occur in theoretical discussions, their meaning should be understood according to a natural language interpretation that points to the general idea behind using terms like these in the law of takings. The reason why notions such as public interest and public use are important is that they can help enforce the natural idea that interferences in private property should only occur for the good of the people. This much is common to all jurisdictions considered in this thesis. %However, how the general idea is implemented varies quite considerably. To account for this, the thesis will briefly clarify the meaning of the terms whenever they appear in the context of a concrete jurisdiction.

At a more general level, the work done in this thesis suggests that it might be inappropriate to rely on formal terms such as public use or public interest when attempting to ensure legitimacy in takings law. Specifically, this thesis will argue that legitimacy requires decision-making to take place in an equitable and inclusive manner, such that the owners and those who depend most on the properties in question have a say that is commensurate with what is at stake for them. This perspective, combining procedural and substantive ideals of fairness, will not rely on finer distinctions between notions such as public use, public purpose and public interest. In my opinion, this is a strength of the theory developed in this thesis, an escape from what Gregory Alexander calls the ``formalist trap'', characterised by an exaggerated focus on constitutional property clauses and how they are formulated.\footnote{See \cite[Chapter 1]{alexander06}.} As Alexander argues, excessive formalism can cloud the issue of legitimacy because it blocks from view those important institutional and political processes that determine the actual level of protection given to property and its owners within a given jurisdiction. Building on this, my thesis will develop an integrated approach that looks at the institutional context and the substantive fairness of economic development takings, to arrive at a theory of legitimacy that focuses on the social functions of private ownership.

% will be used throughout this thesis, and it will be used in a normative sense.


%These are presented in  section \ref{sec:x} of chapter \ref{ In general, we find a uniquely egalitarian distribution of land ownership in Norway, where undeveloped land and the resources found on it are typically owned by groups of local small-holders, not landlords or public bodies. In section \ref{sec:1:5}, we already mentioned the concept of a waterfall right, used to refer to the right to harness power from a river, an historically important stick in the property bundle associated with landownership in Norway. Moreover, we mentioned briefly that waterfall rights are usually held in common by members of the local population. %Enjoying private ownership in common is not unusual in Norway, particularly in rural areas, and the law of property in Norway reflects this in various ways.

%In some cases, this is because a river suitable for hydropower development will run across many distinct private properties. Hence, the relevant waterfall rights are held in common in the narrow sense that an assembly of private rights is required in order for development to take place. However, in most cases, waterfalls suitable for hydropower development will be owned in common in a somewhat stronger sense. Indeed, outfields in Norway are often held under a specific form of co-ownership, such that each small-holding in the local community owns a share in the land surrounding their local community. This form of co-ownership has no exact common law equivalent, but is most similar to the tenancy in common. However, there is no requirement that the co-ownership takes place behind a trust -- all individual shareholders are formally registered as owners of their share of the land and their is a presumption in favour of continued co-ownership accompanied by productive use of co-owned land, not a presumption in favour of sale and individuation as seen in UK law.

%In the Common Ownership Act 1965, further rules are given to regulate the use of land under co-ownership. The main principle is that each owner has a right to the ``normal'' enjoyment of the property, taken in light of the local conditions, customs, and the original purpose of the co-ownership arrangement (if it is known). Moreover, an individual owner's use must not exceed what corresponds to his share of the property and must not be unduly burdensome to the other owners. If damage occurs, moreover, compensation must be paid. To some extent, the majority shareholders can enforce a specific use of the property which would also be binding on the minority. This includes new forms of commercial activity on the property. If such activity is organised against the will of a minority, the minority will still be entitled to take part in the enterprise. 

%There are limits to what the majority can do. Importantly, they cannot do anything that will limit the ``normal'' use of the property by any owner. Also, they cannot do anything to dramatically change the character of the property, sell it, or use it as security for debt. Because of this, gridlock can often result if the owners disagree fundamentally about how to manage their land. For real property, particularly in rural areas, the standard way of resolving such situations is to bring a case before the Norwegian land consolidation courts. These courts are empowered to either dissolve the system of co-ownership or else to organise joint use of the land. Indeed, the prevalence of common ownership over outfields is one of the reasons why land consolidation courts are so important in Norway, and it also explains why they have been granted wide powers to help organise the use of privately owned land. I return to the details of this in chapter \ref{chap:6}, as part of a broader discussion on how the institute of land consolidation can be used as an alternative to eminent domain in economic development situations.

%In addition to the form of co-ownership regulated in the Common Ownership Act 1965, there are two other special forms of ownership of land found in Norway that should be briefly mentioned. Both pertain to land over which a large group of people enjoy extensive rights of use that have been recognised as so-called common rights under Norwegian land law. There is always an owner of the land in the normal private law sense of the word, but special rules are in place to protect the group of people who enjoy use rights. These rules presuppose that the land is owned either by the state or a council of the local community (which might not include everyone who enjoys use rights over the land). There is no concept of a commons in Norway that attaches to land owned by private individuals, which is quite natural given that private landlords and tenant farming is  absent from the structure of rural landownership in Norway.

%If the owner of common land is the state, the relevant legislation that protects the rights of the local people is the State Commons Act 19... If the land is owned by a local community, the relevant legislation is the Village Commons Act 19... The details differ, but the main principle of both acts is that they offer special protection to use rights holders, especially with regard to traditional land uses that local farmers depend on for their livelihoods.

%After the industrial revolution, there was some doubt as to whether common rights gave non-owners a claim to waterfall rights, or whether waterfall rights were held exclusively by the landowners. This was particularly important for land owned by the state, since common rights was the only potential route for local community members to claim a proprietary stake in local hydropower resources (in village commons, by contrast, the owners would typically themselves be local community members). The question was settled by the Supreme Court in the case of {\it Vinstra} in 196.. Here it was held that no rights to waterfalls on state-owned lands could be derived from rights in common over that land. For this reason, the takings issue does not arise with respect to hydropower development on such land, at least not with respect to the waterfall rights as such. 

%Of course, questions still arise regarding the fate of local communities when development takes place on state-owned land where local people enjoy rights in common. However, questions that arise specifically with respect to Norwegian commons law will not be dealt with in this thesis.\footnote{When we consider the case of {\it Alta} in Chapter \ref{chap:5}, we will encounter state-owned land where the aboriginal Sami population has claimed to enjoy rights in property similar to common rights. In recent years, this claim has met with some recognition within the Norwegian legal order, giving rise to yet another form of property in Norway. For further details, see the discussion in Chapter \ref{chap:5} section \ref{chap:5:x}.}
%However, when the {\it Alta} case was decided, members of the Sami population were considered as rights holders in the traditional private law sense of the word, no different from non-aboriginal holders of property and use rights elsewhere in Norway. See the

%In Chapter \ref{chap:5}, we will discuss the {\it Alta} case in some depth. This case was a takings case arising from hydropower development in Finnmark, a part of Norway where the state is traditionally regarded as the owner of all outfields. The state's ownership of land in this region tends to be at odds with the aboriginal interests of the Sami people. Traditionally, the state's ownership was consider to be entirely unencumbered by aboriginal entitlements except where Sami use rights had been explicitly recognised. Moreover, the rights of the Sami people did not have the protected status granted to rights in common over state land.\footnote{Some scholars disputed this, by arguing that Sami rights should be viewed as common rights by analogy with the legislation in place for state commons.} 

%Hence, in the {\it Alta} case, the formal standing of the Sami people was derived from expressly recognised use and property rights that would be lost or depreciate in value following the development. Specifically, the Supreme Court rejected claims based on aboriginal rights, and the case did not involve takings of waterfalls as such. Still, the case has been considered an important precedent for disputes surrounding expropriation of waterfalls, since it dealt with many aspects of administrative law pertaining to the licensing procedure surrounding hydropower development. In addition, as I discuss briefly in chapter \ref{chap:5}, the case marked a watershed moment in the legal history of the Sami people, whose rights over land in Finnmark have since received greater recognition within the Norwegian legal order. Today, in the special context of Sami land, the law appears to be moving towards a framework where the Norwegian state is increasingly seen as a custodian of Sami lands, rather than an owner in the standard private law sense.

%In other parts of Norway, a similar perspective has not developed. Natural resources owned by the state, or taken under eminent domain, has the same legal status as private property, except that the owner happens to be the state. There is no recognised legal sense in which the state is held to be a custodian of land, and there is no legal doctrine according to which state-owned lands are supposed to be held in trust on behalf of the people. However, in a recent revision of the Constitution, a new section was introduced that compels the government to preserve the environment and promote sustainability. The exact wording is as follows:


%This provision replaces a similarly broad sustainability provision that was first introduced in the Constitution in 1992. In practice, the sustainability requirement has left little impact in Norwegian law, with no consequences discernible at all within the law of property. No one, to my knowledge, has proposed to read section 112 as having any direct bearing on the state's rights and responsibilities as the owner of land. Rather, the section is typically understood to give rise to a general obligation to promote sustainability through regulation, meaning that regulatory failures could conceivably be challenged under the provision. So far, however, few challenges of this kind have appeared and none have been successful. . Indeed, it has been argued that the constitutional sustainability provision as such has been something of a failure.

%After the new formulation was introduced in 2014, there have been some indications that the legal status of the provision might be about to change, in the direction of becoming more easily justiciable. In fact, a group of Norwegian environmental lawyers are presently preparing a case where they will challenge the Norwegian state with not doing enough to fight climate change, a legal action that will be brought under section 112 of the Constitution. In the law of property, however, there is no indication that the provision will become important any time soon. Similarly, in the law of hydropower, there have been no indications to suggest that the provision will be considered relevant to disputes between developers and local people, especially not when such disputes arise with respect to the issue of expropriation. For this reason, the sustainability provision in the Norwegian constitution will not be examined further in this thesis. Of course, a normative argument could well be made that the provision {\it should} entail greater regard for the interests and property rights of local people. Such an argument might perhaps also be backed up by considerations based on sustainability research and international environmental law. Further exploration of economic development takings from this angle will be left for future work.

%First, we will encounter ownership of waterfalls, a concept that appears to be unique to Norwegian law. As mentioned in section \ref{sec:1:5}, above, the right to harness power from a waterfall in Norway has a recognised status as an incident of private landownership. Moreover, it can be transferred separately from the surrounding land, voluntarily or through expropriation. 

%This is the Norwegian property type referred to as a ``vannfall'', literally translated: a waterfall. This is a legal term with a specific (although disputed) meaning in Norwegian law. In the second part of the thesis the word waterfall will be used in the specific sense that the word ``vannfall'' is used under Norwegian law. The intuitive understanding of the term is suggestive but incomplete, so a clarification is in order: a waterfall is used in the law to refer to a power, namely that of water flowing along a given stream or river. That is, the waterfall is a term the law uses to refer to the energy that can be harnessed from a river from a point A to a point B, where A and B are typically given as the altitude where a given waterfall begins to where it ends. 

%What this means is that a waterfall owner, under Norwegian law, has a right to the hydropower of a river. This right usually emerges from ownership of land over which the water flows, but it is considered a distinct ``stick'' in the bundles of riparian owners. It is also (on some conditions) separately alienable. This is important because it means that in Norway, developing a hydropower plant requires ownership of the waterfall, nor merely access to suitable sites for building the dam and the station. In Norwegian law, one does not take the view that the building of a hydropower plant creates the hydropower. Rather, the power of the water already exists and already has owners, usually the local landowners. To some extent, waterfall rights under Norwegian law can be compared to fishing rights in English law.

\noo {\section{Structure of the Thesis}\label{sec:1:5}

My thesis is divided into two parts. Part I sets up a theoretical framework for reasoning about property and proceeds to study the legitimacy of economic development takings in more depth. Part II consists of a case study of takings for hydropower, focusing on how expropriation and alternatives to it work on the ground in Norway. In brief, the structure of the chapters are as follows.

Chapter 2 introduces the topic of this thesis and presents the social function theory of property. The chapter argues that the descriptive core of this theory should be accepted irrespective of one's normative inclinations; the social function approach is simply more accurate than other theories. From this descriptive assertion, the category of economic development takings arises naturally. To address it normatively, the chapter argues that the notion of human flourishing provides the appropriate starting point. On this basis, the chapter discusses economic development takings and {\it Kelo} in more depth, to introduce the key question of legitimacy.

Chapter 3 proceeds to address the legitimacy question in more depth. The chapter starts from considering the procedural approach to legitimacy, illustrated by the law of England and Wales. Following up on this, the substantive approach is considered, illustrated by the law of the US. Finally, the chapter argues for a middle ground between the two, an institutional fairness perspective that is also linked to recent developments at the ECtHR. Following up on this, the chapter presents the Gray test; a set of indicators of eminent domain abuse suitable for an institutional fairness approach. The chapter concludes by discussing the possibility of providing institutional alternatives to expropriation for economic development, taking inspiration from the theory of self-governance for common pool resources.

Chapter 4 introduces the case study of takings for hydropower in Norway. The chapter briefly presents hydropower in the law, focusing on the licensing legislation. Then the chapter investigates hydropower in practice, noting that the liberalisation of the electricity market in the early 1990s has had a dramatic effect. Specifically, the chapter emphasises how local owners of water resources are now in a better position to develop these themselves, since they can access the electricity grid as producers on equal terms as larger companies. The chapter goes on to study the tension that has resulted between large-scale development facilitated by expropriation and small-scale development facilitated by local property rights. Despite early signs that small-scale solutions enjoyed political support, the large energy companies now appear to be reasserting their control over the hydropower sector, to the detriment of owners and their local communities.

Chapter 5 discusses expropriation of hydropower in more depth. The chapter starts by giving a brief overview of Norwegian expropriation law, before noting that expropriation for hydropower often takes place on the basis of special rules that leave owners with less protection. The history of the law is discussed in quite some detail, to show how the law has gradually developed to undermine local property rights over water resources. Following up on this, the chapter discusses case law on the expropriation and licensing, focusing on the legitimacy question (which is addressed in Norway almost solely on the basis of procedural standards). The chapter studies the recent Supreme Court case of {\it Jørpeland} in depth, to shed light on how current administrative practices impact on owners and their communities. The conclusion is that current takings practices do not appear legitimate.

Chapter 6 discusses land consolidation as an alternative to expropriation. The chapter starts by clarifying the notion of land consolidation and how the Norwegian understanding of that terms is much wider than that found in other jurisdictions. Following up on this, the chapter discusses consolidation as an alternative to expropriation, by focusing on those tools that the consolidation courts have at their disposal in this regard. Then the chapter gives a more in-depth presentation of some cases when consolidation was used to organise small-scale hydropower development. Finally, a discussion is provided on the prospect of using consolidation to replace expropriation more generally, in Norway and possibly also in other jurisdictions.

Chapter 7 contains my conclusions, formulated as an attempt at connecting the concrete and abstract aspects of this work around two threads, tracking property's relationship with excluding and taking on the one hand and its relationship with giving and participation on the other. My final conclusion is that the latter two notions characterise true property, and that property as such is worth defending.

} 
}
%\newcommand{\isr}[1]{{#1}}

\part{Towards a Theory of Economic Development Takings}

\chapter{Property, Protection and Privilege}\label{chap:2}

\noo{ \begin{quote}
It's nice to own land.\footnote{Donald Trump, as quoted in \cite{booth12}.}
\end{quote}

\begin{quote}
A human being needs only a small plot of ground on which to be happy, and even less to lie beneath.\footnote{Johan Wolfgang von Goethe, {\it The sorrows of young Werther and selected writings}.}
\end{quote}
}
\section{Introduction}\label{sec:2:1}

\noo{ This chapter presents a template for analysing economic development takings, based on legal theory.\footnote{I will not provide an extensive presentation of concepts or theoretical approaches developed in other fields, such as political science, sociology, economy, or psychology. However, all these fields engage in interesting ways with the notion of takings and property, and I will quote some sources from these fields as appropriate for my argument based on legal theory. See generally \cite{miceli11,nadler08,katz97,carruthers04}.} It will be argued that the category of economic development takings is relevant to legal reasoning about certain situations when private property is taken by the state. }

The category of economic development takings makes intuitive sense; it targets situations when property is literally taken for economic development. In most cases considered in this thesis, economic development is even the explicitly stated aim used to justify the exercise of eminent domain. However, as mentioned in the introduction, the legal relevance of the category cannot be taken for granted. Indeed, a superficial look at typical approaches to takings in the law would seem to indicate that the nature of the project benefiting from a taking is not usually a major issue when assessing the legitimacy of interference.\footnote{For instance, in Europe, the property jurisprudence at the European Court of Human Rights (ECtHR) deals almost exclusively with other aspects of legitimacy. The Court typically stresses that interference must be in the public interest, but then leaves this aspect of legitimacy behind after making clear that the member states enjoy a wide margin of appreciation in relation to the public interest requirement. See, e.g., \cite{james86,lindheim12} (however, the form and strength of the public interest is potentially relevant to the Court's fair balance assessment, as discussed in chapter \ref{chap:3}). Similarly, in the US in the 1980s, Merrill claimed that most observers thought of the public use clause in the Fifth Amendment of the US Constitution as nothing more than a ``dead letter'', see \cite[61]{merrill86}.} 

This chapter challenges that idea, by offering an argument as to why the purpose and context of a taking matters, not only as a question of public policy but also with respect to property as a fundamental right. From the point of view of US law, providing such an argument might not be strictly required, since economic development takings have already gained recognition as an important category of legal reasoning.\footnote{See generally \cite{cohen06,somin07,malloy08}.} However, a conceptual discussion of what exactly the category represents appears to be missing in the literature so far. In Europe, moreover, the category has so far not received much recognition as a legally relevant way to address the legitimacy of expropriation. Hence, in a comparative study, a conceptual investigation into the very idea of an economic development taking is a necessary first step.

This chapter argues that in order to make this step, we should broaden our theoretical outlook compared to traditional forms of legal reasoning about property. Moreover, it will be argued that a suitable conceptual reconfiguration is already implicit in recent strands of property theory, particularly those that focus on the {\it social function} of property.\footnote{See generally \cite{alexander09a,foster11,singer00,underkuffler03,alexander06,alexander10,dagan11}.} Indeed, the crux of the main argument presented in this chapter is that the social function view compels us to pay attention to the special dynamics of power that tend to manifest in cases when private property is taken by the state for economic development, especially in the context of commercial exploitation.

To make clear why such takings are special, this chapter abandons the traditional entitlements-based perspective on property in favour of a perspective that emphasises the ideal function of property as a guarantor of social justice and a building block of democracy and participatory decision-making, particularly at the local level. This allows us to shift attention away from the effect that a taking has on individuals one-by-one, towards the question of whether the purpose of the taking, and its broader societal effect, merits interfering with private property. When this question is recognised as falling within the sphere of property law, it also provides a potential basis for judicial review rooted in property protection.

This chapter will also argue that private property is important because it gives owners a right to take part in decision-making processes concerning economic development, a right that also typically gives owners a duty to participate, not only on their own behalf, but also on behalf of broader community interests. In my view, this highlights how property rights can empower local communities in their interactions with powerful commercial and central government interests. Clearly, the use of eminent domain can undermine this function of property, thereby threatening the democratic legitimacy of the decision-making process, by depriving local communities of a potentially robust source of participatory competence. Moreover, when property interests are transferred away from the local community on a permanent basis, this threatens to leave a lasting democratic deficit in the wake of economic development. Arguably, this is the key reason why we should recognise economic development takings as a separate conceptual category.

To motivate the theoretical work, I start by considering the Balmedie controversy, pertaining to Donald Trump's plans for a golf resort in Scotland. I use this concrete example to highlight tensions between property's different functions in the context of economic development, to motivate the theoretical arguments that follow.
\noo{
Then, in section \ref{sec:top}, I go on to discuss theories of property, to locate a suitable starting point for further analysis. I argue that neither of the two dominant property theories of the last century, the bundle theory and the dominion theory respectively, provide such a starting point. In section \ref{sec:socfunc}, I move on to consider the social function theory in more depth, to arrive at a more useful theoretical template. Moreover, I argue that the descriptive part of this theory can provide a valuable conceptual tool even if one does not agree with the normative assertions that are typically associated with it. In particular, I argue that normative considerations should be addressed separately from conceptual foundations.

I do so in section \ref{sec:hf}, by building on the human flourishing account of the purpose of property. I argue that the human flourishing theory provides us with a possible path towards answers to the normative questions that arise from the social function perspective. In section \ref{sec:edt}, I apply the theoretical framework developed in preceding sections to a preliminary investigation of economic development takings, to bring out the overarching question of legitimacy, which will occupy a central place in this thesis.}

\section{Donald Trump in Scotland}\label{sec:2:2}

On the 10th of July 2010, the property magnate Donald Trump opened his first golf-course in Scotland, proudly announcing that it would be the ``best golf-course in the world''.\footnote{See \cite{passow12}.} Impressed with the unspoilt and dramatic seaside landscape of Scotland's east coast, the New Yorker, who made his fortune as a real estate entrepreneur, had decided he wanted to develop a golf course in the village of Balmedie, close to Aberdeen.

To realise his plans, Trump purchased the Menie estate in 2006, with the intention of turning it into a large resort with a five-star hotel, 950 timeshare flats, and two 18-hole golf-courses.\footnote{See \cite{siddique08}.} The local authorities were divided on the issue of whether to grant planning permission, which was first denied by Aberdeenshire Council.\footnote{See, e.g., \cite{bbc07}.} One of the reasons for rejecting the plans was that the proposed site for the development had previously been declared to be of special scientific interest under conservation legislation.\footnote{See \cite{bbc07b}.} The frailty and richness of the sand dune ecosystem, it was argued, suggested that the land should be left unspoilt for future generations. Several members of the local population actively campaigned against the plans, with some also refusing to sell property that Trump wanted to include in his development project.\footnote{See \cite{scotsman10}.}

Trump was not deterred, and in the end he was able to convince Scottish ministers that he should be given the go-ahead on the prospect of boosting the economy by creating some 6000 new jobs.\footnote{See \cite{carrell08}. Trump's plans attracted significant public attention, and his interaction with Scottish decision-makers came under critical scrutiny by commentators, see, e.g., \cite{jenkins08}. For a more general assessment from the point of view of conservation interests in the UK, see \cite{koen13}.} Activists continued to fight the development, launching the ``Tripping up Trump'' campaign to back up local residents who refused to sell their properties.\footnote{See \cite{tripping15}.} One of these, the farmer and quarry worker Michael Forbes, expressed his opposition in particularly clear terms, declaring at one point that Trump could ``shove his money up his arse''.\footnote{See \cite{scotsman10}.} Trump, on his part, had described Forbes as a ``village idiot'' that lived in a ``slum''.\footnote{See \cite{bbc10}.} Moreover, he had suggested that Forbes was keeping his property in a state of disrepair on purpose, to coerce Trump to pay more for the land, to remove the blight.\footnote{See \cite{cnn07}.} Forbes was offended and he proudly declared that he would never consider selling, as the issue had become personal.\footnote{See \cite{ferguson12}.}

At the height of the tensions, Trump asked the local council to consider issuing compulsory purchase orders (CPOs) that would allow him to take property from Forbes and other recalcitrant locals against their will.\footnote{See \cite{macaskill09}. It would not have been the first time Donald Trump benefited from eminent domain. In the 1990s, he famously succeeded in convincing Atlantic City to allow him to take the home of Vera Coking, to facilitate further development of his casino facilities. But in this instance, Trump did not get his way. Indeed, the taking of Vera's home was eventually struck down by the New Jersey Superior Court, an influential result that was hailed as a milestone in the fight against ``eminent domain abuse'' in the US. See \cite[297-301]{jones00}. See also \cite{gillespie08}. For the decision itself, consult \cite{banin98}.} These plans met with widespread outrage. The media coverage was wide, mostly negative, and an award-winning documentary was made which painted Trump's activities in Balmedie in a highly negative light.\footnote{See \cite{baxter11}.} The controversy also found its way into UK property scholarship. Kevin Gray, in particular, a leading expert in property law, expressed his opposition by making clear that he thought the proposed taking would be an act of ``predation''.\footcite{gray11}

In fact, the case prompted Gray to formulate a number of key features that could be used to identify situations where compulsory purchase would be likely to represent an abuse of power. Gray noted, moreover, that Trump's proposed takings would fall in line with a general tendency in the UK towards using compulsory purchase to benefit private enterprise, even in the absence of a clear and direct benefit to the public. In light of this, it seemed realistic that CPOs might be used in Balmedie.\footnote{Moreover, a statutory authority is found in section 189 of the \cite{tcpsa97}, stating that local authorities have a general power to acquire land compulsorily in order to ``secure the carrying out of development, redevelopment or improvement''.} It would not be hard to argue that the public would benefit indirectly in terms of job-creation and increased tax revenues. Moreover, Scottish ministers had already gone far in expressing their support for the plans.

But then, in a surprise move, Trump announced he would not seek CPOs.\footnote{See \cite{scotsman11}.} Instead, he decided to pursue a different strategy, namely that of containment. He erected large fences, planted trees and created artificial sand dunes, all serving to prevent the properties he did not control from becoming a nuisance to his golfing guests. One local owner, Susan Monroe, was fenced in by a wall of sand some 8 meters high. ``I used to be able to see all the way to the other side of Aberdeen'', she said, ``but now I just look into that mound of sand''.\footnote{See \cite{booth12}.} She also lamented the lack of support from the Scottish government, expressing surprise that nothing could be done to stop Trump.

There was little left to do. As soon as the decision was made to build around them, the neighbouring property owners found themselves marginalised. Trump, on his part, was declared a valuable job-creator whose activities would boost the economy in the region. He even received an honorary doctorate at Robert Gordon University, a move that prompted the previous vice-chancellor, Dr David Kennedy, to hand his own honorific back in protest.\footnote{See \cite{bbc10b}.}

In the end, then, it was not by taking the land of others that Trump triumphed in Scotland. Rather, he succeeded by exercising ``despotic dominion'' over his own.\footnote{To quote Blackstone, see \cite[2]{blackstone79b}.} This proved highly effective. After he fenced them in, his neighbours were hard to see and hard to hear. The Balmedie controversy went quiet, the golfers came, Trump got his way. As he declared during the grand opening: ``Nothing will ever be built around this course because I own all the land around it. [...] It's nice to own land.''\footnote{See \cite{booth12}.}

\subsubsection*{\ldots}

The tale of Trump coming to Scotland serves to illustrate the kind of scenario that I will be looking at in this thesis. In addition, it puts my work into perspective. For a while, it looked like Balmedie was about to become a canonical case of an economic development taking. But in the end, it became an illustration of something more subtle, namely that what it means to protect property depends on value judgements regarding opposing property interests. In particular, while Trump achieved his ends in Scotland by relying on his own property rights, he did so by undermining the property rights of others, even if he did not formally condemn those rights.

This was made possible by an exercise of regulatory and financial power. Hence, we are reminded that the function of property as such is deeply shaped by social, political and economic structures. For the powerful owner, property can be used offensively to oppress weaker parties. For the marginalised, it might well be the last line of defence against oppression. Indeed, Donald Trump's ownership of the Menie estate has a vastly different meaning than does Michael Forbes' ownership of his small farm. To many observers, the former kind of ownership will represent some combination of power, privilege and profit, while the latter will be regarded as imbued with a mix of defiance, community and sustenance. Different values are inherent in these two forms of ownership, and when Trump came to Balmedie, they clashed in a way that required the legal order to prioritise between them.

In Trump's narrative, upholding the sanctity of property in Balmedie entails allowing him to protect his golf resort plans from what he regards as backwards locals who attempt to fight progress. If this is one's starting point, property protection might even come to involve the use of compulsory purchase of rights that are seen as a hindrance to the full enjoyment of property by a more resourceful owner. 

For Michael Forbes and the other local owners, protecting property has a completely different meaning. To them, it was paramount to protect the local community against what they saw as a disruptive and damaging plan, one that threatened to turn them and their properties into mere golfing props. Again, adequate protection might require an interference in property, to prevent Trump from using his land according to his own wishes, because this causes damage to his neighbours. 

In the case of Balmedie, we are forced to recognise that protection implies interference and vice versa. Moreover, we see how both sides of that equation can involve the interests, ambitions, fears and aspirations of private individuals. This shows the conceptual inadequacy of the idea that property protection is all about weighing private against public interests, to strike a balance between the state's power to do good and owners' right to do as they please. In reality, matters are often more subtle, involving a number of additional dimensions. Importantly, how we assess concrete situations where property is under threat depends crucially on what we perceive as the ``normal'' state of property, the alignment of rights and responsibilities that we deem  worthy of protection. Our stance in this regard clearly depends on our values. But values themselves are in turn influenced by the context of assessment within which they arise. An additional challenge is that our assessments are often influenced by our \emph{perception} of the relevant context, rather than by facts.

For example, property activists in the US tend to regard the value of autonomy as a fundamental aspect of property. But this must be understood in light of the idea that US society is founded on an egalitarian distribution of property, where ownership is meant to empower ordinary people by facilitating self-sufficiency and self-governance.\footnote{See, e.g., \cite[173]{ely07}; \cite{rose96}.} Hence, the autonomy inherent in property ownership is not thought of as being bestowed on the few, but on the many. Protecting autonomy of owners against state interference is not about protecting the privileges of the rich and powerful, but is embraced as a way to protect {\it against} abuse by the privileged classes.\footnote{This narrative is enthusiastically embraced by US activists who fight economic development takings, see, e.g., \cite{castle15}.} 

This, however, is only an {\it idea} of property protection. It might not correspond to the reality surrounding the rules that have been \isr{moulded} in its image. Indeed, it has been noted that despite the great pathos of the egalitarian property idea, egalitarianism has actually played a marginal role to the development of US property law.\footnote{\cite[361]{williams98} (``Why does the egalitarian strain of republicanism have such a substantial presence in American property rhetoric outside the law but so little influence within it?'')} More worryingly still, research indicates that land ownership in the US, which is hard to track due to the idiosyncrasies of the land registration system, is not actually all that egalitarian.\footcite[246-247]{jacobs98} In this way, we are confronted with the danger of a dissociation of values, reality and the law.

In Scotland, a similar story unfolds. Here, the traditional concern is that land rights are mainly held by the elites.\footnote{See generally \cite{wightman96,wightman13}.} As a result, Scottish property activists tend to focus on values such as equality and fairness, calling also on the state to regulate and implement measures to achieve more egalitarian control over the land. Indeed, reforms have been passed that sanction interference in established property rights on behalf of local communities.\footnote{See generally \cite{lovett11,hoffman13}.} At the same time, cases like Balmedie illustrate that the Scottish government, now with enhanced powers of land administration, may well choose to align themselves with the large landowners. Moreover, research indicates that recent reforms in Scottish planning law, which serve to enhance the power of the central government, have the effect of undermining local communities and their capacity for self-governance.\footnote{See generally \cite{pacione13,pacione14}.} Again, the danger of a disconnect between influential property narratives and reality is brought into focus.

On the other hand, it seems that \isr{grass roots} property activists in the US and Scotland may well be closer in spirit than they seem. Although their perception of the role of the state is very different, they appear to share many of the same concerns and aspirations. Arguably, differences arise mainly from the fact that they operate in different contexts and engage with different discourses of property. The challenge is to find categories of understanding that allow us to make sense of both their commonalities and their differences.

I think the example of Balmedie suggests a possible first step. It illustrates, in particular, the need for a framework that will allow us to recognise that opposing the use of compulsory purchase for economic development is perfectly consistent with supporting strict property regulation to prevent the establishment of golf resorts in fragile coastal communities. Both of these positions, moreover, should be viewed as efforts to protect property. To the classical debate about the limits of the state's authority over property, such a dual position can be hard to make sense of. But in my opinion, this only points to the vacuity of the conventional narrative.

In general, I think it is hard to make sense of many contemporary disputes over property if we do not have the conceptual tools to distinguish between (1) egalitarian property held under a stewardship obligation by members of a local community, and (2) neo-liberal property held by large enterprises for investment. Moreover, there is no contradiction between promoting the value of autonomy for one of these, while \isr{emphasising} the need for state control and redistribution when it comes to the other. The broader theme is the contextual nature of property and its implications for protection of property rights. In the coming sections, I will propose a theoretical basis that integrates this viewpoint into legal reasoning about interference in property rights.

\section{Theories of Property}\label{sec:2:3}

What is property? In common law jurisdictions, the standard answer is that property is a collection of individual rights, or more abstractly, a means of protecting {\it entitlements}.\footnote{The idea that property rules are a form of entitlement protection was developed to great effect in the seminal article \cite{calabresi72}.} Being an owner, it is often said, amounts to being entitled to one or more among a bundle of sticks, floating on streams of protected benefits associated with, and thereby serving to legally define, the property in question.\footnote{See \cite[357-358]{merrill01}. The ``classical'' references on the bundle of rights theory in the US and the UK respectively are \cite{hohfeld17,honore61}.} This point of view was first developed by legal realists in response to the natural law tradition, which \isr{conceptualised} property in terms of the owner's dominion over the owned thing, particularly his right to exclude others from accessing it.\footcite[193-195]{klein11} In civil law jurisdictions, rooted in Roman law, a dominion perspective is still often taken as the theoretical foundation of property, although it is of course \isr{recognised} that the owner's dominion is never absolute in practice.\footnote{For a comparison between civil and common law understanding of property, see generally \cite{chang12}.}

In modern society, the extent to which an owner may freely enjoy his property is highly sensitive to government's willingness to protect, as well as its desire to regulate. To dominion theorists, this sensitivity is typically thought of as giving rise to various restrictions on property, but for bundle theorists it is often thought of as {\it constitutive} of property itself.\footcite[7]{chang12} 

The bundle of rights theory has long historical roots in common law. Arguably, it was distilled from the traditional estates system for real property, which was turned into a theoretical foundation for thinking about property in the abstract.\footnote{See \cite[7]{chang12} (``The ``bundle of rights'' is in a sense the theory implicit in the common law system taken to its extreme, with its inherently analytical tendency, in contrast to the dogged holism of the civil law.'').} However, during the 18th and 19th century, natural law and dominion theorising was also influential in common law. This is evidenced, for instance, by the works of William Blackstone and James Kent.\footnote{See generally \cite{blackstone79b,kent27}.} Towards the end of the 19th century, it became increasingly hard to reconcile such an approach to property with the reality of increasing state regulation. Hence, the bundle metaphor that gained prominence in the early 1900s can be seen as a return to a more modest perspective.\footnote{See \cite[195]{klein11}.}

On the bundle account, property rights are thought to be directed primarily towards other people, not things.\footnote{See \cite[357-358]{merrill01} (``By and large, this view has become conventional wisdom among legal scholars: Property is a composite of legal relations that holds between persons and only secondarily or incidentally involves a ``thing''.'').} This underscores an important point about property in the real world, namely that the content of rights in property are necessarily relative to a social context as well as the totality of the legal order. For instance, when relying on a bundle metaphor it becomes easy to explain that a farmer's property rights protects him against trespassing tourists, but not against the \isr{neighbour} who has an established right of way.\footnote{It has been argued that this way of thinking about property, as a web of (legal and social) normative relations between persons, does not entail the bundle of sticks idea, see \cite[23-25]{dorfman10}. I agree, and I also believe that endorsing the property-as-relations perspective is largely appropriate, even if one does not otherwise agree with the bundle perspective. Historically, however, the two ideas have in fact been closely associated with one another, so presenting them together seems appropriate. Moreover, I will not actively enter into the theoretical debate on this point, since I believe that the {\it social function} account of property, discussed in more detail in section \ref{sec:2:4}, takes us further than both bundle and dominion perspectives. However, as will hopefully become clear, the social function theory itself may be seen as a continuation of the property-as-relations idea, catering also to a more holistic perspective on social structures (although it otherwise manages to remain largely neutral on the bundle v dominion issue).}

By contrast, the dominion theory suggests viewing such situations as exceptions to the general rule of ownership, which implies a right to exclusion at its core. In the case of property, exceptions no doubt make up the norm. But in civil law jurisdictions one lives happily with this. It takes the grandeur away from the dominion concept, but it retains a nice and simple structure to property law. In the civil law world, it is common to say that what the owner holds is the {\it remainder}, namely what is left after deducting all positive rights that restrict his dominion.\footcite[25]{chang12} Moreover, while there may be many limitations and additional benefits attached to property, they are all in principle carved out of one initial right, namely that of the owner. In this way, the civil law system can be more easy to navigate.

Some common law scholars have recently elaborated on this to develop a critique of the bundle theory, by suggesting that it should at least be complemented by a firm theory of {\it in rem} rights in property. This, they argue, would allow the law to operate more effectively, by relying on a simple and clear rule that, although defeasible, would generally suffice to inform people about their relevant rights and duties in relation to property.\footnote{\cite[793]{merrill01b} (``The unique advantage of in rem rights -- the strategy of exclusion -- is that they conserve on information costs relative to in personam rights in situations where the number of potential claimants to resources is large, and the resource in question can be defined at relatively low cost.''); \cite[389]{merrill01} (``The right to exclude allows the owner to control, plan, and invest, and permits this to happen with a minimum of information costs to others.''). See also \cite{ellickson11} (arguing that Merrill and Smith's analysis nicely complements and improves upon the bundle theory).} 

In addition, some scholars point out that the bundle theory does not adequately reflect the sense in which property is a right to a {\it thing}, serving to create an attachment that is not easily reducible to a set of interpersonal legal relationships.\footnote{\cite[1862]{merrill07}. For a slightly different take on attachment, highlighting how the `thingness' of property marks its conditional nature and transferability, see \cite[799-818]{penner96}.} In the US, where the bundle theory has traditionally been dominant, this critique seems to be gaining ground.\footnote{See generally \cite{klein11}.}

In this thesis, the efficiency and clarity of different property concepts will not be a primary concern, nor will personal attachments to things in themselves play a particularly important role.\footnote{I mention, however, that the \isr{personhood aspects} of property that are sometimes highlighted in this regard might be relevant to an analysis of economic development takings. The personhood account does not seem to depend on a rejection of the bundle theory, but might carry with it some implicit criticism of it, see, e.g., \cite[127-130]{radin93}.} Hence, I will remain largely agnostic about this aspect of the debate between dominion and bundle theorists. In particular, the differences between civil and common law traditions in this regard do not cause special problems for my analysis of economic development takings. For the purposes of this thesis, it is more important how different ways of looking at property can influence how we assess when interference is legitimate under constitutional and human rights law. Hence, I now turn to the question of whether or not there are any significant differences between dominion and bundle theories when it comes to the question of legitimacy of takings.

\subsection{Takings under Bundle and Dominion Accounts of Property}\label{sec:2:3:1}

Bundle theorists might be expected to have a comparatively relaxed attitude towards state interference in property rights. Indeed, thinking about property as sticks in a bundle may lead one to think that property rights are intrinsically limited, so that subsequent changes to their content, made by a competent body, are reflections of their nature, not a cause for complaint. In particular, the theory conveys the impression that property is highly malleable. 

For the theorists that developed the bundle of sticks metaphor in the late 19th and early 20th century, this aspect was undoubtedly very important. By providing a highly flexible concept of property, they helped the state gain conceptual authority to control and regulate.\footcite[195]{klein11} The early bundle theorists not only developed a theory to fit the law as they saw it, they also contributed to change.

In takings law, the bundle narrative has been particularly important in relation to the contentious issue of so-called regulatory takings.\footnote{See generally \cite{alterman10} (a comparative study of   regulatory takings in thirteen countries).} Such takings occur when government regulates the use of property so severely that it may be classified as a taking in relation to the law of eminent domain.\footnote{See \cite[1]{fischel95}. Regulatory takings proper arise when the (contested) right to compensation is inferred from a takings clause, such as in the US. However, the overarching question is how to deal with changes in property values caused by regulation, a question of universal importance across jurisdictions that may also be addressed by the legislator as a separate issue, see \cite[3-10]{alterman10}.} In the US, the question of when regulation amounts to a regulatory taking is highly controversial.\footnote{See generally \cite{fischel95}.} The stakes are high because takings have to be compensated in accordance with the Fifth Amendment of the US Constitution.\footnote{See \cite{fifth}.} At the same time, the law is unclear, and the limited amount of statutory law on the issue means that cases tend to be adjudicated against the Constitution, often the relevant state Constitution in the first instance.\footnote{See \cite[31-32]{alterman10}. Indeed, the federal doctrine on regulatory takings appears to be marked by deference to state courts. See \cite[66]{fischel95}.}

If property is thought of as a malleable bundle of entitlements that exists only because it is recognised by the law, it becomes natural to argue that when government regulates the use of property, it does not deprive anyone of property rights. It merely restructures the bundle. In the case of {\it Andrus v Allard}, the Supreme Court adopted such an argument when it declared that ``where an owner possesses a full ``bundle'' of property rights, the destruction of one ``strand'' of the bundle is not a taking, because the aggregate must be viewed in its entirety''.\footcite[65--66]{andrus79}

Hence, with regards to the issue of regulatory takings, the bundle theory was actively used by those who favour a less restrictive approach to interference with private property rights. However, it is wrong to conclude that the bundle theory {\it necessarily} implies a less restrictive stance on takings. Epstein, for instance, argues that as every stick in the property bundle represents a property right, government should not be permitted to remove any of them without paying compensation.\footcite[232-233]{epstein11} 

More generally, Epstein does not believe that the bundle theory is responsible for what he regards as a weakening of property rights in the US during the 20th century. Instead, he thinks this weakening resulted from a tendency among modern property scholars to adopt a ``top-down'' approach to property. According to Epstein, too many scholars view property rights as vested in, and arising from, the power of the state, not the possessions of individuals.\footnote{\cite[227-228]{epstein11} (``In my view, the nub of the difficulty with modern property law does not stem from the bundle-of-rights conception, but from the top-down view of property that treats all property as being granted by the state and therefore subject to whatever terms and conditions the state wishes to impose on its grantees.'').}

Epstein successfully shows that as a rhetorical device, the bundle of rights theory may be turned on its head compared to how it was used in {\it Andrus v Allard}. Moreover, his arguments illustrate that the bundle theory itself does not appear to dictate any particular position on the degree of protection that private property should enjoy against state interference.\footnote{To further underscore this point, it may be mentioned that while US courts \isr{recognise} that a regulation can amount to a taking, this is practically unheard of in several other common law jurisdictions, including England and Australia. This is despite the fact that these countries all paint property in a similar conceptual light. Moreover, while the issue of regulatory takings is considered central to constitutional property law in the US, it is considered a fairly marginal issue in England, see \cite{purdue10}.}

In the civil law world, the relationship between property theorising and property protection is similarly hard to pin down at the conceptual level. Again, the issue of regulatory takings illustrates this. In some civil law countries, like Germany and the Netherlands, the owner's right to compensation for burdensome land use regulation is strong, while in other civil law countries, such as France and Greece, it is very weak.\footnote{See generally \cite{alterman10}.} In particular, the exclusive dominion understanding of property does not appear to commit one to any particular kind of policy on this point. 

On the one hand, it cannot be denied that property rights are enforced, and limited, by the power of government. Hanging on to the idea of dominion, then, necessarily forces us to embrace also the idea that dominion is never absolute. In this way, the theory may serve as a conceptual basis for arguing in favour of a relaxed approach to state interference. If property rights are not absolute to start with, why worry about interfering in them for the common good? But, of course, this story too may be turned on its head. Indeed, a libertarian can use the image of limited dominion to argue that property is being ripped apart at its seams. If we want to maintain our grasp of what property is, such a person might argue, we better enhance the level of protection offered to property owners, to restore true dominion.

The upshot is that the differences between common law and civil law \isr{theorising} about property do not appear to be very relevant to the question of legitimacy in the context of state interference. In particular, the differences between the bundle theory and the dominion idea do not appear to speak decisively in \isr{favour} of any particular approach to economic development takings.

In terms of descriptive content, both theories appear oversimplified. They provide a manner of speech, but they do not really get us very far towards uncovering the reality of property rights in modern society. In particular, they do not provide a functional account of what role property plays in relation to the social, economic and political structures within which it resides.\footnote{A similar point is made in \cite[2-6]{alexander12}.}

In terms of normative content, on the other hand, both the bundle theory and the dominion theory appear rather bland. They simply do not offer much clear guidance as to what norms and values the institution of property is meant to promote. They give neat ways of presenting what property can look like, but do not tell us {\it why} it should be protected.\footnote{For a similar criticism, see \cite[535-536]{bell05} (proposing an instrumental theory of property, with both descriptive and normative implications, based on the idea that property exists to protect the value of stable ownership).}

\subsection{Broader Theories}\label{sec:2:3:2}

Based on the discussion so far, it seems that in order to make progress towards a theory of economic development takings we need to start from a property theory with a wider scope than both the bundle account and the dominion theory. There are many candidates that could be considered. In a recent monograph on property, Alexander and Pe\~{n}alver present five key theoretical branches:\footnote{See chapters 1 to 5 of \cite{alexander12}.}
\begin{itemize}
\item {\it Utilitarian} theories, focusing on property's role in helping to maximize utility or welfare with respect to individual preferences and desires. 
\item {\it Libertarian} theories, focusing on property's role in furthering individual autonomy and liberty, as well as the importance of protecting property against state interference, particularly attempts at redistribution. 
\item {\it Hegelian} theories, focusing on the importance of property to the development of personhood and \isr{self-realisation}, particularly the expression and embodiment of free will through control and attachment to one's possessions.
\item {\it Kantian} theories, focusing on how property arises to protect freedom and autonomy in a coordinated fashion so that {\it everyone} may potentially enjoy it, through the development of the state.
\item {\it  Human flourishing} theories, focusing on property's role in facilitating participation in a community, particularly as a template allowing the individual to develop as a moral agent in a world of normative plurality.
\end{itemize}

It it beyond the scope of this thesis to give a detailed presentation and assessment of all these theoretical branches. Suffice it to say that the utilitarian approach has been by far the most influential.\footnote{See \cite[11]{alexander12} (noting that there are many varieties of utilitarianism, including some law-and-economics theories for which the appropriateness of that label is contentious).} The basic tenet of this paradigm is that means-end analysis on the basis of exogenous preferences and utility measures provide a sound foundation on which to reason about law and policy.

In this thesis, I will depart from this form of analysis, by regarding property instead as an integral part of social structures. On this view, property can no longer be seen neither as an end in itself nor as a means to maximise some utility measure. Instead, property is understood in light of how it functionally relates to other building blocks of life, such as sustenance, economic activity, social interaction, interpersonal responsibility, preference change, deliberation, and democratic decision-making.

With such a starting point, I believe the human flourishing theory has more to offer than any of the other theoretical branches mentioned above. In section \ref{sec:2:5} below, I will emphasise how this theory suggests a range of new policy recommendations regarding how the law {\it should} approach the question of economic development takings.

Before I get to this, I will explore descriptive aspects of property theory in some more depth. Indeed, a potential objection against all the theories summarised above is that they are overly normative; they are largely used to argue for particular values associated with property, not to clarify the descriptive core of the notion. This is a challenge, since one of my main aims in this thesis is to argue for a descriptive proposition, namely that economic development takings make sense as a conceptual category for legal reasoning. Hence, before I move on to consider normative aspects, I first need a theoretical framework that allows me to pinpoint what makes economic development takings unique. I would like to do so, moreover, without thereby committing myself to any particular stance on how to normatively assess such takings.

To arrive at a suitable foundation in this regard, I will rely on the so-called {\it social function theory} of property.\footnote{See generally \cite{foster11,mirow10,alexander09a}. Be aware that some authors, particularly in the US, also speak of the {\it social obligation} theory, using it more or less as a synonym for the social function theory.} This theory is often thought of as a normative theory as well, in some sense a precursor to more overtly normative theories such as the human flourishing theory. However, I will argue that the social function theory has a descriptive core that can serve as a common ground for debate among scholars that do not necessarily share the same normative outlook. Crucially, the descriptive core of the social function theory also points towards a descriptive argument in favour of studying economic development takings.

\section{The Social Function of Property}\label{sec:2:4}

As an empirical observation, the fact that property has social functions is beyond doubt. For instance, it is clear that ownership of property gives rise to social obligations, not just rights. Hardly anyone would protest that in practical life, what an owner will do with their property is as much constrained by the expectations of others as it is by law. Moreover, the law of nuisance and rules relating to adverse possession both serve as simple examples that such expectations can also have a bearing on the legal status of property and its owners.\footnote{See \cite[314]{waldron85} (invoking social obligations as well as the notion of nuisance to explain what ownership is, at the conceptual level); \cite[197-198]{gerhard13} (proposing an abstract distinction between trespass and nuisance in US law based on the concept of duty, which, it is argued, is always symmetric in nuisance cases but not in case of trespass); \cite[1169-1172]{penalver07} (analysing adverse possession on the basis of a social functions understanding of property). See also \cite{pye07} (the ECtHR deciding to regard a UK case of adverse possession in bad faith as legitimate under the ECHR).}

Still, many property scholars have surprisingly little regard for social functions when they theorise about ownership. According to Alexander, the classical theories of property convey the impression that ``property owners are rights-holders first and foremost; obligations are, with some few exceptions, assigned to non-owners''.\footcite[1023]{alexander11} Theorists who emphasise property's social function attempt to redress this conceptual imbalance. As Alexander explains, ``social obligation theorists do not reverse this equation so much as they balance it. Of course property owners are rights-holders, but they are also duty-holders, and often more than minimally so''.\footnote{\cite[1023]{alexander11}.}

I remark that what Alexander and others sometimes refer to as the social obligation theory of property is covered by the social function theory as I understand it. However, the social function theory is broader in that it asks us to consider the legal relevance of social dependencies rooted in property more generally, not just obligations. Specifically, the social function theory as it is used in this thesis should be understood as a cross-jurisdictional reference point for a set of abstract ideas about property that includes the ideas of theorists such as Alexander, who also argue specifically for a social obligation norm in US property law.\footnote{For a similar understanding of the social function theory, see \cite{foster11}.} As I discuss in the next subsection, the idea that property serves important social functions is not new. Moreover, it often plays an important implicit role in shaping how property is understood in the law, also in Europe.

\subsection{Historical Roots and European Influence}\label{sec:2:4:1}

The first expression of the social function theory has been attributed to Le{\'o}n Duguit, a French jurist active early in the 20th century.\footnote{See generally \cite{foster11}.} In a series of lectures he gave in Buenos Aires in 1911, Duguit challenged the classic liberal idea of property rights by pointing to their context dependence, adopting a line of argument strikingly similar to how recent scholars have criticized utilitarian discourses about property.\footnote{See \cite[1004-1008]{foster11}. For more details about Duguit's work and the contemporaries that inspired him, see generally \cite{mirow10}.} In particular, Duguit also pointed to the notion of obligation, stressing the fact that individual autonomy only makes sense in a social context where people are dependent on each other as members of  communities. Hence, depending on the social circumstances of the owner, their property could entail as many obligations as entitlements. This, according to Duguit, was not only the inescapable reality of property ownership, it was also a normatively sound arrangement that should inspire the law, more so than individualistic, `liberal', visions of property as entitlement protection.\footnote{See \cite[1005]{foster11} (``The idea of the social function of property is based on a description of social reality that recognizes solidarity as one of its primary foundations'', discussing Duguit's work). It should also be noted that Duguit was particularly concerned with owners' obligations to make productive use of their property, to benefit society as a whole. This raises the question of who exactly should be granted the power to determine what counts as ``productive use''. In this way, Duguit's work also serves to underscore one of the main challenges of regulatory frameworks that seek to incorporate and draw on property's social dimension: how should decisions be made in such regimes?} The social function perspective has since become widespread in Latin America, as reflected for instance in the property clause in Article 21 of the American Convention of Human Rights.\footnote{\cite{achr}. See \cite[61-62]{banning02}. See also \cite{cunha11,mirow11,bonilla11}.}

Related ideas have also been influential in Europe, particularly during the rebuilding period after the Second World War. For instance, the constitution of Germany -- her {\it Basic Law} -- contains a property clause stating explicitly that property entails obligations as well as rights.\footnote{See \cite[14]{basic49}.} As argued by Alexander, this has had a significant effect on German property jurisprudence, creating a clear and interesting contrast with US law.\footnote{See \cite[738]{alexander03} (``The German Constitutional Court has adopted an approach that is both purposive and contextual, while the U.S. Supreme Court has not.''). See also \cite[476-483]{walt11} (noting that the German property clause also imposes a strict public purpose requirement which has been used by the Constitutional Court to strike down illegitimate takings that benefit commercial enterprises).}

A social perspective on property was also influential during the debate among the European states that first drafted the property clause in Article 1 of the First Protocol to the European Convention of Human Rights (P1(1) of the ECHR).\footnote{See \cite[1063-1065]{allen10}. Allen argues that the liberal conception of property has since gained ground in Europe, as reflected in jurisprudential developments at the ECtHR.} The article was eventually formulated as follows:\indexonly{echr}

\begin{quote} Every natural or legal person is entitled to the peaceful enjoyment of his possessions. No one shall be deprived of his possessions except in the public interest and subject to the conditions provided for by law and by the general principles of international law.
\end{quote}
\begin{quote}
The preceding provisions shall not, however, in any way impair the right of a state to enforce such laws as it deems necessary to control the use of property in accordance with the general interest or to secure the payment of taxes or other contributions or penalties.
\end{quote}

I will return to this clause in more depth in section \ref{sec:3:4} of chapter \ref{chap:3}. Here I note how it emphasises both the private right to peaceful enjoyment of possessions and the state's right to interfere with property in the general/public interest.\footnote{As argued by Allen, there is no real difference between ``general interest'' and ``public interest'' as used in P1(1) of the ECHR; both terms are understood very broadly, with the Court in Strasbourg emphasising the considerable ``margin of appreciation'' awarded to the states in this regard, see \cite{29-30]{allen05}.} Moreover, it does not explicitly introduce an absolute compensation requirement in case of expropriation by the state, setting it apart from many other property clauses, including that contained in the Fifth Amendment of the US Constitution. Arguably, this reflects a recognition of the social aspects of property.\footnote{See generally \cite{allen10}. For completeness, I mention that a human right to property is also included in Article 17 of the \cite{udhr}. However, its inclusion was controversial and no corresponding right was included in either of the two subsequent covenants containing binding provisions, see \cite{fne,fnp}. In light of this, the property clause in the UDHR appears to have limited practical significance. Similarly, a property clause is included in Article 14 of the \cite{banjul}. However, this property clause has apparently not been given much attention from the African Court, see \cite[83]{walt11}. However, it has been proposed that an internationally binding right to property should be introduced through a new international Covenant, a proposal that appears to be firmly based on a social function understanding of property, see generally \cite{hassmann13} (including a draft formulation of a property clause that states, among other things, that the human right to property does not apply to corporations).}

However, it also fits within a traditional narrative of private property, where social responsibilities attaching to property are regarded as arising from state objectives and policies, not ownership as such. Indeed, the chosen formulation in P1(1) appears to suggest that social aspects are external to private property, vested in the regulatory power of the state.\indexonly{echr}

This marks a possible tension with the social function theory, which asks us to recognise that social obligations are inherent in private property, attaching to owners directly. The importance of this in the present context is that a social function perspective can occasionally suggest stricter limits on state interference, not out of greater concern for individual entitlements, but out of concern for property's proper role as a building block of social and political life.

Despite the conventional formulation used in P1(1), such a perspective does in fact appear to play a role at the ECtHR.\noo{At least, the case law from the Court shows that social and political considerations are not only invoked in the context of adjudicating tensions between private entitlements and the perceived necessity of state interference in the public interest.

The ECtHR emphasises {\it proportionality} and {\it fairness} when adjudicating cases involving interference in property.\footnote{See generally \cite[chapter 5]{allen05}.} Importantly, these broad notions are assessed concretely against the context of interference, also to give appropriate weight to the social and political function of the property interfered with. As a result, specific social functions of property can justify greater protection against interference.} A series of cases involving hunting rights provide an example of this.\footnote{See \cite{chassagnou99,hermann12,chabauty12}.} In these cases, the Court in Strasbourg has explicitly granted stronger property protection to owners who oppose hunting on ethical grounds, compared to owners who want to retain exclusive hunting rights for themselves.

For the former group of owners, it has been held that the state may not compulsorily transfer hunting rights to hunting associations for collective management.\footnote{See \cite{chassagnou99, hermann12}.} For the latter group of owners, by contrast, the Court held in {\it Chabauty v France} that such transfers must be tolerated.\footnote{See \cite{chabauty12}.}\indexonly{echr}

For owners opposing hunting on ethical grounds, an interference with their hunting right is an interference with their moral duty to act in accordance with their beliefs. The belief that hunting is unethical gives the owners a personal obligation to prevent their hunting rights from being used. If owners are deprived of their opportunity to fulfil this obligation, it changes the social function of their property because it severs the link between the owners' value system and the use that is made of their property.

In {\it Chassagnou and others v France}, the Court regarded this as a particularly severe interference in property, which could not be upheld despite the fact that it had been carried out in the public interest to secure sustainable management of hunting rights. The Court concluded that ``compelling small landowners to transfer hunting rights over their land so that others can make use of them in a way which is totally incompatible with their beliefs imposes a disproportionate burden which is not justified under the second paragraph of Article 1 of Protocol No. 1''.\footnote{See \cite[85]{chassagnou99}.}\indexonly{echr}

Clearly, the Court is not expressing an opinion on the ethical status of hunting. However, it is recognised that owners are entitled to have unconventional personal convictions in this regard. Moreover, managing one's property in accordance with one's convictions is recognised as part of what it means to be an owner. Protecting this aspect of ownership appears to be more important to the Court in Strasbourg than protecting the right of exclusion for owners who wish to keep the fruits of the land to themselves, as demonstrated by the ruling in {\it Chabauty}. 

The hunting cases also demonstrate that even when the legal system does not explicitly recognise the value of a social function inherent in property, such a function can still come to play a role when the Court assesses the legitimacy of interference against P1(1). This is reassuring since, as I argue in the next section, the law of property invariably involves prioritising between different social functions, also in situations when this is not openly acknowledged by policy makers and judges.\indexonly{echr}

\subsection{The Impossibility of a Socially Neutral Property Regime}\label{sec:2:4:2}

Property both reflects and shapes relations of power among members of a society.\footnote{This aspect of property's social function was stressed in a recent ``statement of purpose'' made by leading property scholars in support of the social function theory, see \cite{alexander09a}. For a sociological perspective on this, see \cite[23]{carruthers04} (``The right to control, govern, and exploit things entails the power to influence, govern, and exploit people.'').} Moreover, it does not act uniformly in this way -- the effect depends on the circumstances. \noo{ An indebted farmer who is prevented by state regulation from making profitable use of their land might come to find that their property has become a burden rather than a privilege. As a consequence, someone who has already amassed power and wealth elsewhere might be able to purchase the land from the farmer cheaply. By acquiring a farm and transforming it to recreational property, the outsider will symbolically and practically assert their dominance and power, while also reaping a potential financial benefit from investing in a more modern function of property.

In some cases, this dynamic can become endemic in an area, resulting in a complete reshaping of the social fabric surrounding property.} Consider, for instance, a fairly typical scenario leading to depopulation of rural areas: first, impoverished farmers and other locals sell homes to holiday dwellers, causing house prices to soar. As a result, local people with agrarian-related incomes \isr{cannot} afford local homes, causing even more people to sell their land to the urban middle class. In this way, a causal cycle is established, the social consequences of which can be vicious, particularly to the low-income people who are displaced.\footnote{The general mechanism described here is well-documented and known as {\it gentrification} in human geography (often qualified as rural gentrification when it happens outside urban areas). See generally \cite{weesep94,phillips93,slater06}. For a case study demonstrating the role that state regulation can play (perhaps inadvertently) in causing rural gentrification, see \cite[1027-1030]{darling05}.} This gives rise to the following  theoretical contention: setting out to regulate property in a situation like this -- when property rights pull in different directions depending on your vantage point -- requires a principled stance on whose property, and which of property's functions, one is aiming to prioritise. Should the law emphasise the property rights of local people who face displacement, or should it protect the property rights of outsiders wishing to invest in holiday homes?

Some may \isr{shy} away from this way of posing the question,  arguing instead that it would be better to rely on neutral rules that treat all owners the same way. In a gentrification scenario, for instance, such an appeal to neutrality could be the first step in an argument against regulating the property market to prevent the displacement of local people. But would that truly be a socially neutral approach to residential property? Presumably, it would threaten the property interests of local owners, particularly those not wishing to sell their properties.\footnote{The threat might be more or less direct. For instance, a weakened community could reduce the values that make the properties attractive to their current owners, while rising property prices could give rise to rising property taxes that render this ownership unaffordable as well.} Hence, if {\it their} property rights are to be protected, regulation should be put in place.

Importantly, both sides of a conflict like this are in a position to adopt a property narrative to argue for their interests. Hence, it is also inappropriate to think that the law of property can remain neutral. Moreover, a traditional narrative of property might fail to make sense of the ensuing tension. Consider, for instance, the conflict between Donald Trump and the Balmedie locals, as discussed in section \ref{sec:2:2}.

As long as Trump threatened to use compulsory purchase, the local people could adopt a traditional ``pro-property'' stance against Trump. But as soon as Trump decided to fence them in by relying on his own property rights, they had to adopt a seemingly contradictory view on property, whereby Trump's property rights should be limited out of concern for the community. A traditionally minded observer might use this as an opportunity to accuse the locals of having an unprincipled attitude towards private property. By contrast, the social function approach suggests a very different picture.

The locals sought to protect property, but not just any property. The property they wanted to protect was the property which served the social function of sustaining the existing community. The property they wanted to protect was the property that {\it meant} something to them.\footnote{This is more than merely observing that they wanted to protect {\it their} property. In their desire to regulate the use of Trump's property, the locals also wanted to protect certain social functions inherent in that property, against Trump's own actions.} 

Trump and his supporters might well have entertained similar feelings about their property rights, and the development they wanted to carry out. Hence, in conflicts such as these the law will invariably have to take a stand regarding which property interests it wishes to promote. The social function theory asks us to be upfront about this, so that policy making and adjudication in hard cases can proceed on the basis of substantive arguments about social functions rather than unconvincing appeals to neutrality and deference.

\noo{
In all likelihood, such a stand must also sometimes be taken by whoever {\it interprets} the law, since it is exceedingly unlikely that the legislature will ever be able to provide deterministic rules for resolving all conflicts of this kind. Lastly and most controversially, the courts may find occasion to curtail the power of government -- perhaps even the legislature -- if such power is usurped by powerful actors wishing to undermine property's proper functions to further their own interests. 
This, in particular, raises the question of constitutional and human rights limits to interference in property, relative to those functions that are to be protected.}
 
While the law is forced to prioritise in case of conflict, social functions can also work together in a way that promotes certain property uses and decision-making structures for property management. This can even alleviate the pressure for top-down government regulation, with desirable consequences for both owners and the public interest.
%, as discussed in the next subsection.

%\noo{ \subsection{The Regulating Effect of Property}

%Property shapes and reflects societies, but it also shapes and reflects social commitments and dependencies within those societies.\footnote{See generally \cite{alexander09}.} 

Again, this function of property is highly dependent on context. Small business owners, by virtue of being members of the local community, might be socially discouraged from becoming a nuisance to their \isr{neighbours}, with no need for state interventions through detailed legislation or planning.\footnote{See, e.g., \cite[282-283]{ellickson91}.} However, if local owners go out of business and a non-local commercial owner replaces them, the regulatory effect of property can change dramatically. 

Indeed, if we imagine that the new owner hopes to raze the local community in order to build a new shopping center, we are at once reminded of the stark contrasts that can arise between various social functions of property. The property rights of small shop owners can be the lifeblood of a community, while the exact same rights in the hands of a large enterprise can give rise to its destruction.

Mechanisms like these can have important ramifications, not only for property, but also for the regulatory regime surrounding its use. For instance, it seems clear that if a new and more commercially aggressive owner is to be deterred from becoming a nuisance to neighbours, stricter forms of regulation might have to be put in place.\footnote{It has been argued that such a mechanism explains the advent of zoning and land use planning in the US, particularly the Supreme Court's willingness to accept it even though it limited the freedom of property owners. See \cite[99-100]{shoked11}.} The social responsibility that was previously anchored in the community must now be protected more forcefully by the state. In turn, this can cause the institution of property to weaken further, as the government assumes greater power to interfere.\footnote{For an early criticism of zoning in the US, pointing to the merits of a more flexible and highly decentralised approach based in large part on the law of nuisance, see \cite{ellickson73}.} A feedback effect might result, as increased regulation in turn threatens to make property ownership too burdensome or expensive for low-income, or even average-income, community members.\footnote{In the US, the term ``exclusionary zoning'' is used to describe such a mechanism when zoning contributes to pushing low-income people out of suburban, and increasingly also urban, areas. Since the zoning framework in the US is quite decentralised, the process is often pushed forward by locally based affluent home-owners who capture the zoning power of local governments to enhance their property values, e.g., by preventing intrusive development projects that could otherwise attract low-income people by increasing the demand for cheap labour and the supply of cheap housing. Due to this dynamic, the feedback mechanism is amplified: the standard proposals for reform to deal with exclusionary zoning involve further inflating the power of the government or the markets to impose large-scale development projects against the will of local communities. See, e.g., \cite[117-120]{mangin14} (referring favourably to the standard reform suggestion, but noting that it seems politically unrealistic to implement in the US). For a more subtle analysis, resulting in the proposal that home-value insurance should be introduced to make home-owners less likely to pursue exclusionary zoning, see \cite{fischel04}.} Hence, the most resourceful actors, those who are able to meet or influence the government's demands and/or protect themselves against interference, gain more  property, while the government gains more regulatory power.

The social function theory tells us that mechanisms of this kind need to be taken seriously by legal theorists and practitioners. The broader point at stake here can also be brought out in relation to the famous ``tragedy of the commons''.\footnote{See \cite{hardin68}.} In his seminal article, Hardin describes how individually rational users of a commons can eventually cause the depletion of that resource. The problem arises, according to Hardin, because individuals have no proper incentive to refrain from over-exploitation; the damage will be distributed among all resource users, so it will not outweigh the benefit of individual over-use in the short term.

%\footnote{See \cite{demsetz67} (according to Demsetz's theory, subdividing a commons and introducing individual exclusion rights characterises the introduction of a private property regime, the purpose of which is to internalise many externalities while making negotiations about any remaining externalities, e.g., non-localised pollution, more effective).}

In response, it has been typical to regard either state management or stronger individual exclusion rights as the answer.\footnote{See \cite[8-13]{ostrom90}.} State management is supposed to prevent over-exploitation through regulation, while individual exclusion rights are supposed to make it more difficult for resource users to shift the cost of over-exploitation onto other individuals.

Both of these strategies can potentially result in the sort of feedback effect discussed above, where the inadequacies of states and markets combine to make it necessary for both of them to inflate their power at the expense of local communities. However, a cycle of dispossession is not the inevitable outcome of the tragedy of the commons. Indeed, as Elinor Ostrom and others have shown, the traditional narrative overlooks the fact that commons tend to come with community structures that provide appropriate checks and balances through locally grounded institutions or social arrangements.\footnote{See generally \cite{ostrom90}.} As long as external forces do not threaten them, such arrangements can be more robust than either 
individual exclusion regimes or state control. Moreover, they can be anchored in the law of property.\footnote{For the connection between property law and governance theories for common pool resources generally, see \cite{rose11,fennel11}. It is worth emphasising that this link is not only relevant in relation to {\it natural} resources. Indeed, common pool theories have been much discussed also in the context of intellectual property, see \cite[38-43]{rose11}. That being said, in this thesis the focus will be on property forms that are moulded out of land and related resources. The special questions that might arise when the theoretical framework is applied to other forms of property will not be addressed.} I will return to this point in the next chapter, when I discuss possible alternatives to eminent domain in economic development situations.\footnote{See chapter \ref{chap:3}, section \ref{sec:3:6}.}

\noo{ The ideas of Ostrom on common pool management focus on local institutions for collective decision-making, not property rights. As a complementary viewpoint, the idea that local institutions for resource management can be anchored in private property represents a potentially attractive way of thinking.\footnote{For a survey of how US property scholars have been influenced by Ostrom's ideas, including a discussion on the relationship between (private) property and local institutions for self-governance, see \cite{rose11}.} The social function theory of property already suggests pursuing this idea. Based on the social function approach, the descriptive fact that property structures shape decision-making processes at the local level is enough to conclude that local institutions for resource management should not be looked at in isolation from the law of property.\footnote{It is also worth emphasising that there is nothing in this way of thinking that limits us to considering property in {\it natural} resources. Indeed, common pool theories have been much discussed also in the context of intellectual property, see \cite[38-43]{rose11}. That being said, in this thesis the focus will be on property forms that are moulded out of land and related resources. The special questions that might arise when the theoretical framework is applied to other forms of property will not be addressed.}

Moreover, by recognising property as an anchor for equity and decision-making at the local level, social obligations that inhere in private property may be recognised as existing independently of specific institutional arrangements. Hence, if local institutions are marred by corruption and malpractice, a social function theorist can take the normative stance that property ownership still carries with it duties to care for other property dependants in the community. This duty, moreover, would exist independently of the extent to which it is presently fulfilled through local practices and institutional arrangements. 
}
%section \ref{sec:2:5}, when I discuss the human flourishing theory of property and its promise of internalising economic and social rights for non-owners into the structure of property itself. First, I will argue that it is useful to distil a descriptive core from the social function theory, so that it may serve as a common ground for debate, allowing the interchange of ideas across normative divisions.

\subsection{The Descriptive Core of the Social Function Theory}\label{sec:2:4:3}

Social function theorists have been criticised for making too far-reaching normative claims. Eric Claeys, in particular, argues forcefully against normative fundamentalism and what he regards as normative naivety among social function theorists.\footnote{\cite[945]{claeys09}. (``Judges might think they are doing what is equitable and prudent. In reality, however, maybe they are appealing to a perfectionist theory of politics to restructure the law, to redistribute property, and ultimately to dispense justice in a manner encouraging all parties to become dependent on them.'')} Indeed, some social function theorists have gone very far in presenting the social function account of property as a normative theory, attaching specific political commitments to it along the way.

Hanoch Dagan, for instance, is a self-confessed liberal who argues for a social function understanding on the basis that it is morally superior. ``A theory of property that excludes social responsibility is unjust'', he writes, and goes on to argue that ``erasing the social responsibility of ownership would undermine both the freedom-enhancing pluralism and the individuality-enhancing multiplicity that is crucial to the liberal ideal of justice''.\footcite[1259]{dagan07}

If this is true, then it is certainly a persuasive argument for those who believe in a ``liberal idea of justice''. But for those who do not, or believe that property law is -- or should be -- as neutral as possible on this point, a normative argument along these lines can only discourage them from adopting a social function approach. Such a reader would be understandably suspicious that the {\it content} of the social function theory -- as Dagan understands it -- is biased towards a liberal world view. Such a reader might agree that property continuously interacts with social structures, but reject the theory on the basis that it seems to carry with it a normative commitment to promote liberalism.

Dagan is not alone in proposing highly normative social function theories. Indeed, most contemporary scholars endorsing a social function view on property base themselves on highly value-laden assessments of property institutions.\footnote{See, e.g.,  \cite{alexander09,crawford11,davidson11,singer09,penalver09}.} By contrast, the discussion in this chapter so far has aimed to demonstrate that the theory has significant merit already as a {\it descriptive} theory. In my opinion, this is also demonstrated by much of the normatively oriented work that has been done in the social function tradition. When this work focuses on making abstract normative assertions it threatens to overshadow what is arguably the most important insight, namely that considerations related to social functions {\it are} already important in many areas of property law, in many different jurisdictions.\footnote{See, e.g., \cite{gray94,mirow11,cunha11,bonilla11}.} Moreover, the social functions of property, and normative assertions about them, often play a role behind the scenes, where they do unacknowledged work among policy makers and judges alike. As Laura Underkuffler puts it:

\begin{quote}
Property rules, as they now exist, are contingent rules, complex rules, and normatively charged rules. They are crafted and applied in response to the politics of power, security, stability, greed, and a myriad of other aspects of human life.\footnote{\cite[376]{underkuffler10}.}
\end{quote}

Because it embraces this crucial insight, the core of the social function theory, rather than being ``good, period'' as Dagan suggests, is simply more accurate than other proposals, irrespective of one's ethical or political inclinations.\footcite[1259]{dagan07} The theory provides the foundation for a discussion where different values and norms can be presented in a way that is conducive to meaningful debate, on the basis of a minimal number of hidden assumptions and implied commitments. Thus, the first reason to accept the social function theory is epistemic, not deontic.

That is not to say that theories can ever be entirely value-neutral, nor that this should be a goal in itself. However, a good theory is one that can at least serve as a common ground for further discussion based on disagreement about values and priorities. \noo{According to Kevin Gray, ``the stuff of modern property theory involves a consonance of entitlement, obligation and mutual respect''.\footcite[37]{gray11} This is a rather loose way of putting it, but I believe it also points to a measured perspective that is ultimately highly appropriate.} Making room for normative divergences, moreover, can hopefully diminish the worry that a broader theoretical outlook is the first step towards unchecked state power and rule by ``judicial philosopher-kings'', as Claeys puts it.\footnote{See \cite[944]{claeys09}. There is further evidence to suggest that this is a real worry. Specifically, in a recent article, Anna di Robilant discusses how the Italian fascists were happy to embrace a social function perspective on property, because it helped them make the case that property should be made to answer to one core collective value: the interests of the state. As a form of resistance, many Italian property scholars agreed that the social function view was in order, but emphasised the plurality of values associated with this vision, values that have little or nothing to do with the interests of a Fascist state. See \cite{robilant13}.} At the same time, the new descriptive dimensions uncovered by the social function view can also inspire novel normative perspectives, as explored in the next section.

\section{Human Flourishing}\label{sec:2:5}

Taking the social function theory seriously forces us to \isr{recognise} that a person's relation to property can be partly constitutive of that person's social and personal capabilities, both in a political and an economic context.\footnote{I will explore some specific capabilities in more depth later on, when discussing economic and social rights, and the value of participation in democracy. For the notion of a capability more generally, proposed as a foundational concept for economic theory, see \cite{sen85}. For a discussion on the import of this work to property theory, see \cite[105]{alexander09}.} Moreover, property influences people's preferences, as well as what paths lie open to them when they consider their life choices.\footnote{See generally \cite{alexander09}.} This effect is not limited to the owner, it comes into play for anyone who is socially or economically connected to property in some way, including a potentially large group of non-owners.\footcite[128-129]{alexander09d}

Hence, there is great potential for making wide-reaching socio-normative claims on the basis of the social function perspective on the meaning and content of property. But which such claims {\it should} we be making? According to some, we should adjust our moral compass by looking to the overriding norm of {\it human flourishing} as a guiding principle of property law. In a recent article, Alexander goes as far as to declare that human flourishing is the ``moral foundation of private property''.\footcite[1261]{alexander14} 

Human flourishing has a good ring to it, but what does it mean? According to Alexander, several values are implicated, both public and private.\footnote{See generally \cite{alexander14,alexander11}.} Importantly, Alexander stresses that human flourishing is {\it value pluralistic}.\footnote{\cite[750-751]{alexander09}.} There is not one core value that always guarantees a rewarding life. To flourish means to negotiate a range of different impulses, both internal and external. Importantly, these act together in a social context that influences their meaning and impact.\footcite[1035-1052]{alexander11}

In the following, I consider some values that I regard as particularly important for the study of economic development takings. I start by the values enshrined in economic and social rights, which should arguably also inform our understanding and application of property law.

\subsection{Property as an Anchor for Economic and Social Rights}\label{sec:2:5:1}

The so-called ``second generation'' of human rights consists of basic economic, social and cultural  rights that complement the better known civil and political rights.\footnote{See generally \cite[1-14]{baderin07} (arguing that the ``second-generation'' terminology is unfortunate since it can give rise to the misconception that ESC rights are second-class).} This includes rights such as the right to housing, the right to food, and the right to work.\footnote{See \cite[6|7|11]{fnp}.} Economic and social rights of this kind often involve property. Specifically, they often involve interests in property that are not recognised as ownership, e.g., housing rights for squatters or rights to food and work for landless rural people.

If the notion of property is conceptualised in the traditional way, as an arrangement to protect individual entitlements, the relationship between private property and economic and social rights appears to be one filled with tension.\footnote{See, e.g., \cite[138]{garcia14}.} In particular, if economic and social rights require owners to give up some property entitlements, it becomes natural to portray property protection as standing in the way of social justice.

However, the human flourishing theory can be used to tell a very different story, namely one where economic and social rights are anchored in the notion of property itself. Importantly, the human flourishing theory compels us to take into account the interests and needs of property dependants other than owners. As Colin Crawford puts it, the purpose of property should be to ``secure the goal of human flourishing for all citizens within any state''.\footcite[1089]{crawford11} Consider, for instance, the right to housing. If the interests of a property owner come into conflict with the housing rights of a property dependant, the human flourishing theory encourages us to approach this as a tension {\it within} property, between different property functions.

With such a starting point, we should also acknowledge that the appropriate way to approach the rights of non-owners in relation to property might well depend on who the owner is and the choices they make in managing their property.\footnote{See, e.g., \cite[43]{walt11} (commenting on the principle of ``scaling'' of social obligations in German property law, whereby what can be demanded of owners depends on the context).} For instance, if owners live on their land and do not own much more than they need themselves, it becomes hard to maintain the criticism that their private property is somehow an affront to the housing rights of the landless. Moreover, from a practical point of view, squatting is unlikely to occur in these settings unless accompanied by severe trespass or dispossession. Similarly,  the owner of a commercial building can discourage squatting by managing the property well. Arguably, this too can undercut potential criticism on the basis of housing rights, especially if the owner uses the building to engage in a commercial activity that contributes to sustaining the local community.

On the other hand, if owners mismanage their properties, for instance because they seek to obtain demolition licenses or simply wish to await an expected rise in land values, squatters might take opportunity of this and feel encouraged to occupy the property. If private property is thought of merely as entitlement-protection, a property-protecting state might feel obliged to respond in a way that offends against the social and economic rights of the squatters. If this is considered an undesirable outcome, the government or its critics might in turn come to regard strong property protection as an affront to housing rights, even though the real problem is that property does not function as it should within society. Hence, the result can be that property structures are damaged further, as the state pursues policies of interference and centralised management, without addressing how private property as such can promote human flourishing.

By contrast, the human flourishing narrative suggests that both owners and non-owners might appropriately be viewed as victims if the state fails to protect property's proper functions. This perspective might even suggest itself when owners and non-owners would otherwise appear to be adversaries. For a concrete example, I mention the South African case of {\it Modderklip East Squatters v Modderklip Boerdery (Pty) Ltd}.\footnote{See \cite{modderklip05}. For two commentaries focusing specifically on its implications for property law and theory, see \cite{alexander09d,walt05}.}

The case dealt with squatting on a massive scale: some 400 people had initially taken up residence on land owned by Modderklip Farm, apparently under the belief that it belonged to the city of Johannesburg.\footnote{See \cite[4]{modderklip05}.} The owner attempted to have them evicted and obtained an eviction order, but the local authorities refused to implement it. Eventually, the settlement grew to 40 000 people and Modderklip Farm complained that its constitutional property rights had not been respected.\footnote{See \cite[8]{modderklip05}.}

The Supreme Court of Appeal concluded that Modderklip's property rights had indeed been violated, but noted that the housing rights of the squatters were also (in danger of) being violated; the squatters needed a place to go before they could be evicted.\footnote{See \cite[21-26]{modderklip04}.} Hence, the appropriate response was not to evict the squatters, but to pay compensation to Modderklip, for an ongoing violation of its property rights, caused by the state's failure to protect housing rights.

This outcome is consistent with a social function understanding of property, but the rationale behind it builds on a narrative that takes the perceived tension between property and housing rights as its point of departure.\footnote{See \cite[152-156]{walt05}. However, as van der Walt notes, the Supreme Court of Appeal took the traditional narrative in an interesting direction when it held that the failure of the state to protect the housing rights of the squatters was the cause of its failure to protect property. As van der Walt notes, the conflict between rights then became less important than the observation that the state had failed in its duty to protect {\it both} rights. This is an improvement on a narrative focused on the question of which right to prioritise, but still arguably over-emphasises the role of the state.} The Constitutional Court, by contrast, chose to remain agnostic on the issue of the state's duties with respect to both property and housing rights, as well as the relation between them.\footnote{See \cite[25]{modderklip05}.} Rather, the Constitutional Court focused on the state's failure to ``assist Modderklip'' in dealing with the ``burden imposed on it to provide accommodation to such a large number of occupiers''.\footnote{See \cite[49]{modderklip05}.} This was a failure of governance, and the state was ordered to pay compensation, not for a violation of property, but as an appropriate form of assistance to Modderklip.\footnote{For a more detailed presentation of the Court's decision, including references to the relevant governance provision of the South African constitution (guaranteeing access to court with suitable and efficient enforcement procedures), see \cite[156-158]{walt05}.} 

This shift of perspective can arguably be understood as a reflection of the Court's willingness to regard the needs of the squatters as giving rise to a social obligation for Modderklip {\it qua} owner. Effectively, the circumstances of the case meant that Modderklip's ownership entailed an obligation to respect the housing needs of a community of 40 000 people. Moreover, the primary duty-bearer had become the owner, not the state.

With such an approach, private property can be a potential source of justice for anyone, including squatters. The role of the state, meanwhile, becomes that of assisting those who are directly responsible for delivering justice on the ground, including owners such as Modderklip. In a detailed analysis of the case, Alexander and Pe\~{n}alver also argue in this direction. They suggest, in particular, that {\it Modderklip} serves as an illustration of how property owners themselves can have responsibilities towards property dependants, obligations that endure as long as private property remains in place.\footnote{\cite[157]{alexander09d} (``The courts' unwillingness to ratify Modderklip's desire to remove the squatters from its land illustrates the courts' willingness to take seriously the obligations of owners, not only as they concern owners' direct relationship with the state but also in relation to the needs of other citizens.''). It should be noted, moreover, that Modderklip was eager to sell the land to the government. See \cite[61]{modderklip05}.} 

This normative turn makes property owners addressees of obligations arising from the economic, social and cultural rights of non-owners, not by direct horizontal application of these rights, but through the law of property.\footnote{In this way, it arguably strengthens such rights, while potentially circumventing problems and objections associated with the idea of making such rights directly justiciable in disputes among private parties. For the question of horizontal application more generally, see \cite{manisuli07}. As the case of {\it Modderklip} demonstrates, embedding the duty of non-state actors in property will serve to ensure that the state is responsible for providing assistance in cases when the burdens of ownership become disproportionate. The duty of the state in these circumstances will be formally directed towards the owners, even if the beneficiaries of state action will be those property dependants with respect to whom the owners have obligations. This might prove conducive to more effective action also at the state level; presumably, the owners as a group would make a valuably ally for the landless squatters, in cases of failed governance.} In this way, the human flourishing theory points towards a novel way to address the rights of marginalised group. %when neither neo-liberal property nor state management is capable of delivering basic justice.%\footnote{For instance, it has been noted that in India, the human right to water has at times been simultaneously frustrated both by a non-egalitarian distribution of riparian rights as well as a regulatory framework that grants the state almost limitless proprietary power over water resources. See \cite[186]{cullet09}. To address shortcomings of the current system for water management in India, Cullet recommends a conceptual approach that ``leaves aside property rights altogether'' (both private and public) and instead emphasises that water is a common heritage of humankind. A possible alternative, highlighted by the social function theory, is to embrace human flourishing as the core value of property, possibly in a manner that is mutually conducive to recognising water as a common heritage. On such an account, property might regulate some uses of water, such as economic development, without thereby resulting in anyone being granted ownership of water as a substance or a right to charge people for access to drinking water. As discussed in Part II of this thesis, the traditional property regime for water resources in Norway arguably reflects a framework along these lines, applied in a context marked by egalitarian property with no scarcity of water for basic human needs. The appropriateness of proposing similarly spirited (but adjusted) property-based frameworks for different contexts is a matter for future work, but in my view, the human flourishing theory at least suggests that this might be worth exploring.}

The theory also strengthens the institution of property, highlighting why it might be appropriate to 
grant it extensive protection against interference. In particular, a human flourishing approach might serve as a bulwark against the idea that the ultimate expression of public interests can be found in the actions taken by the state. Instead, the theory directs attention at how public interests are expressed at their point of origin; values often associated with the public sphere, such as those pertaining to the economic and social rights of marginalised groups, are in fact legally relevant already at the level of private law.\footnote{See also \cite[1295-1296]{alexander14}.} As a consequence, the human flourishing account bolsters the view that public interests and obligations can acquire some justiciable relevance even in the absence of explicit international treaties, legislation or equitable decision-making within (inter)national institutions. 

Perhaps the most important structural aspect of this insight concerns the mechanisms used to resolve tensions between different property values. Importantly, it might not be necessary to introduce intermediaries between owners and other rights holders and property dependants. To introduce such intermediaries, whether they are state bodies, international institutions, NGOs, or commercial enterprises, carries with it the risk that the decision-making process can be captured by forces that either have ulterior motives or are simply too far removed from local conditions to deliver results on basic rights.\footnote{See, e.g., \cite{cullet13} (describing how non-state actors and developed countries increasingly seem to capture the agenda in international water policy, to the detriment of people and states in the developing part of the world); \cite{levien13} (analysing state-led processes of rural dispossession in India, arguing that states now often act as land brokers for private enterprises); \cite{mehta14} (two case studies, from India and Bolivia respectively, demonstrating that elite bias and other democratic deficits at the state level have frustrated efforts to deliver on water rights for marginalised groups in peri-urban areas).} It might be better, therefore, if basic rights are (also) anchored and implemented using private law solutions that target the local level, for instance by relying on the law of property.

\noo{ At least, it should be possible to pursue key economic and social values without massively increasing the power of non-local actors and weakening the institution of property. Moreover, it should be possible to more effectively enforce social obligations on private property owners, particularly when they are locally based. Achieving this in practice requires mechanisms that enable negotiations between competing private property interests, to facilitate a balancing of those interests through participatory decision-making rather than top-down state management. This highlights the importance of another property value that the human flourishing theory emphasises, namely that of participation, discussed in the following section.}

\subsection{Property as an Anchor for Democracy}\label{sec:2:5:2}

It is often argued that property is a crucial building block of democracy, as it both empowers and encourages owners to participate in the political process.\footnote{See generally \cite{rose96} (critically examining common arguments to support the claim that property is the most fundamental right, including the argument that it gives rise to, facilitates, and protects democracy). For an exposition of the converse link, explaining how property law is constrained and determined by the values and principles associated with democracy, see \cite{singer14}.} However, the notion of participation at work here often seems to be drawn up rather narrowly, as pertaining primarily to individual owners, and only to their engagement with the formal affairs of the polity.\footcite[1275]{alexander14} By contrast, the human flourishing theory gives participation a broader meaning, involving also the value of being included in a community. Alexander writes:

\begin{quote}
We can understand participation more broadly as an aspect of inclusion. In this sense participation means belonging or membership, in a robust respect. Whether or not one actively participates in the formal affairs of the polity, one nevertheless participates in the life of the community if one experiences a sense of belonging as a member of that community.\footcite[1275]{alexander14}
\end{quote}

Participating in a community can have a crucial influence also on people's preferences and desires.\footnote{For a more in depth discussion of this, see \cite[140]{alexander09}. Here, Alexander and Pe\~{n}alver draw on the work of Amartya Sen and Martha Nussbaum, see generally \cite{sen84,sen85,sen99,nussbaum00,nussbaum02}.} Therefore, for anyone adhering to welfarism, rational choice theory, or some other utilitarian dogma, neglecting the importance of communities is not only normatively undesirable, it is also unjustified in an epistemic sense. In particular, it should be \isr{recognised} as a descriptive fact that the idea of community is highly relevant to {\it any} normative theory that attempts to take into account the preferences and desires of individuals. But Alexander and Pe\~{n}alver go further, by arguing that participation in a community should also be seen as an irreducibly social value, not merely as a determinant of individual preferences and a precondition for rational choice. They write:

\begin{quote}
Beyond nurturing the individual capabilities necessary for flourishing, communities of all varieties serve another, equally important function. Community is necessary to create and foster a certain sort of society, one that is characterized above all by just social relations within it. By ``just social relations'', we mean a society in which individuals can interact with each other in a manner consistent with norms of equality, dignity, respect, and justice as well as freedom and autonomy. Communities foster just relations with societies by shaping social norms, not simply individual interests.\footcite[140]{alexander09}
\end{quote}

This, I believe, is a crucial aspect of participation. Moreover, it is a notion that invariably leads us to recognise that other property dependants should also have a voice, as they form part of the ``just social relations'' within the community to which the owners belong. In addition, this is a notion of participation that it is hard, if at all possible, to incorporate in theories that take preferences and other attributes of individuals as the basis upon which to reason about their legal status. Instead, the human flourishing perspective asks us to consider how property serves as an anchor for participation that shapes and influences community norms and preferences.

Protecting the function that property plays in this regard can at times require protecting it also against the actions of owners and their communities. This can happen, for example, in a community where people have come under pressure to sell their homes and their land to make way for large enterprises. If owners are offered generous financial compensations, or if they are threatened by eminent domain, economic incentives might trump the value of social inclusion and participation. As a consequence, the community might decide to sell.

Even so, in light of the value of community, it would be in order for planning authorities, maybe even the judiciary, to view such an agreement as an {\it attack on their property}. It is clear that by the sale of the land, the ``just social relations'' inhering in the community will come under pressure. Property rights that once contributed to sustaining these relations will be transformed into property rights that serve a very different purpose, namely that of aiding the concentration of power and wealth in the hands of the commercially powerful. Such a change in the social function of property might have to be regarded -- objectively speaking -- as a threat to participation, community and democracy. Therefore, it is arguable that our property institutions should protect against it, even if this implies limiting the freedom of owners and communities to do as they please.

In Norway, a range of such rules are in place to protect agricultural property, by limiting the owners' right to sell parcels of their land without local government consent, as well as by compelling them to reside on their property and to make use of it for agricultural production.\footnote{See \indexonly{la95}\dni\cite[8|12]{la95} and \indexonly{lca03}\dni\cite[4|5]{lca03}.} In addition, there are rules in place that guarantees certain principles of non-exclusion for outfield land; the owner of such land cannot prevent anyone from travelling over it, camping on it, and must even tolerate that visitors pick berries and roots for their own consumption.\footnote{See generally \cite{backer07}.} This is the so-called ``allemannsretten'' -- the right to roam -- which has been recognised in Norway since ancient times.

When the law actively promotes egalitarian and equitable property using arrangements such as these, the natural counterpart is to limit direct state interference. The danger otherwise is that the limited power of each individual property owner -- appropriate in a community of property owners -- is exploited by the state or other powerful stakeholders who might wish to usurp control over local resources and impose their will on local populations.

The broader issue at stake is highlighted by recent developments in South Africa, where rules resembling many of those in place for agricultural property in Norway have been proposed in a recent Act on land reform.\footnote{See \cite{steyn15}.} In South Africa, however, these rules have been proposed alongside a new framework of state ``custodianship'' of agricultural land, corresponding to a formulation recently introduced in the mineral and petroleum legislation.\footnote{See \cite{agri13} (holding that introducing custodianship over mineral and petroleum is not expropriation, meaning that no compensation is payable to owners, not even when the rights taken may subsequently be transferred --  in a different wrapping, under a new regulatory regime -- to third parties). The decision was made under dissent on the basis of the conclusion that {\it acquisition} by the state -- apparently not implied by ``custodianship'' --- is required for a deprivation of property to count as expropriation. On an uncharitable reading, the decision therefore appears to open the floodgates for economic development takings, which can now even go uncompensated as long as they are embedded in a regulatory narrative based on custodianship. The academic community in South Africa appears to be divided on the issues raised by the case, see \cite[428-451]{walt11}; \cite{marais15a,marais15b}.} If the proposal passes, the proper functioning of agricultural property in South Africa would seem to depend quite strongly on the benevolence and capability of the state, which will significantly increase its own power to interfere with private property.\footnote{Hence, the proposed land reform arguably demonstrates the appropriateness of Justice Froneman's dissent in {\it Agri}, see \cite[79-110]{agri13} (warning against the precedent set by the majority, holding that state custodianship of the type in question amounted to expropriation, but finding that compensation was not required in the circumstances of the case, looking also to the value of social justice and the history of apartheid in South Africa).} Importantly, the human flourishing perspective suggests that even when provisions to promote egalitarian ownership and community commitment are appropriate, provisions that inflate the state's authority might not be. The case study from Norway will illustrate that strict property rules to protect and promote self-governing agrarian communities can work well, but only as long as they are applied consistently and coupled with strong institutions of local democracy and strict limits on state power.\footnote{I discuss the role of agrarian property to the development of Norwegian democracy in more depth in chapter 4.}

This raises the question of what kind of institutions we need to enable local communities and owners to  flourish and make democratically legitimate decisions about how to use their properties. Arguably, there is no appropriate theoretical answer to this question, since institutions for participatory decision-making are successful only when they match local conditions.\footnote{See also the discussion in chapter \ref{chap:2}, section \ref{sec:2:6} (discussing the design principles for self-governance first presented by \cite{ostrom90}).} In the final chapter of the thesis, I return to discuss this concretely by looking to the Norwegian institution of land consolidation.

In the next section, I apply the theory developed so far to economic development takings. Specifically, I introduce this category of takings in more depth and present {\it Kelo} in further detail, drawing on the social function theory to carry out an assessment of the controversy that resulted.\noo{ v City of New London\footcite{kelo05}, which brought this category to prominence in the US discourse on property law. Then I will assess the unique aspects of such takings against the social function theory, to provide an argument that the category has significance for legal reasoning in takings law, as well as with respect to property as a constitutionally protected human right. Finally, I will provide an abstract presentation of the values that I believe are important when normatively assessing the law in this area. In doing so, I will draw on the human flourishing theory, setting out the main values that will inform the concrete policy assessments I provide later.}

\section{Economic Development Takings}\label{sec:2:6}

The notion of an economic development taking is in some sense self-explanatory: it targets situations when property is taken for economic development. However, the obvious follow-up question is a difficult one: what is meant by ``economic development''? In the literature on economic development takings, no clear answer has been provided.\footnote{See, e.g., \cite[558-567]{cohen06} (Cohen proposes a ban on economic development takings, comments on the difficulty of defining the notion precisely, before proceeding to pursue his stated aim indirectly, through a ban on takings benefiting private parties, with exceptions for certain non-profit undertakings).} Rather, one tends to rely on an intuitive understanding to classify takings as being for economic development, where typical cases are those where the decision-makers themselves emphasise the value of economic progress as a reason for authorising eminent domain.\footnote{See, e.g, \cite{somin07,ely09} (both authors also discuss how economic development, or even commercial profit, can be an unacknowledged motive, for instance when property is taken on the pretext of combating ``blight'').} The lack of clarity about what the category covers might seem like a theoretical challenge, possibly even a weakness. However, it can also be argued that the ambiguity of the notion of economic development forms part of the reason why economic development takings merit special attention in the first place.\footnote{See \cite{somin07} (arguing for a complete ban on the ``economic development rationale'', citing its vagueness as a reason why it should {\it never} be used to justify a taking).} 

Some scholars still prefer not to use the notion, choosing instead to speak of ``private takings'' when they discuss the legitimacy issues that arise in cases such as {\it Kelo}.\footnote{See, e.g., \cite{bell09}.} The notion of a private taking is very easy to define: such a taking occurs when the legal person taking title to the property is a non-governmental actor.\footnote{See \cite[519]{bell09}.} Arguably, however, this categorisation is quite unhelpful when the aim is to get at the legitimacy issues that arise specifically in economic development situations. For instance, it might well be that a private organisation, say a tightly regulated charity, functionally mimics a quintessential ``public'' taker. A public body, on the other hand, can well be functionally equivalent to a private enterprise, particularly if there is a lack of political oversight and democratic accountability. Imagine, for instance, a case involving a publicly owned limited liability company. According to the simple definition of a private taking, a taking by such a company would not meet the definition. This would be the conclusion even if the company's interests are completely or predominantly commercial, directed at maximising profit for the shareholders, not at providing a public service.\footnote{Some might argue that the distinction between private and public ownership is still significant. However, such an argument seems difficult to make independently of the social context. If a public company operates for profit and is insulated from political decision-making and principles of administrative law, it is hard to see why takings to benefit such a company should be regarded as {\it a priori} different from other kinds of economic development takings. In particular, it is hard to see why it should matter in such cases whether the associated public benefit is ensured through the payment of dividends, taxes, or some other mechanism. In any event, the public benefit will be indirect in these cases, arising from ordinary commercial activity.}

By contrast, the notion of an economic development taking points to the purpose of the taking, not the outward legal appearance of the taker. As such, it provides a less sharp distinction between different kinds of takings, but also seems more relevant to the question of legitimacy. Specifically, the category performs an important function in that it directs attention at the fact that there might be  inappropriate motives influencing the decision to take private property. Moreover, the main reason for paying particular attention to economic development takings is clear enough: the presence of strong economic incentives, often of a commercial nature, appears to increase the risk of eminent domain abuse.

The benefit of using a comparatively neutral and open-ended designation seems especially clear in mixed economies, where the influence of public-private partnerships can cause a general blurring of lines between private and public sectors.\footnote{For the growing importance of public-private partnerships to the world economic order, see generally \cite{saussier13}.} In such contexts, it seems particularly appropriate to devote special attention to cases where commercial interests stand to benefit -- directly or indirectly -- from a taking of private property. The presence of commercial incentives among the beneficiaries or their partners might contrast with the public spirited rationale provided to legitimise the taking. An important advantage of a categorisation based on the notion of economic development is that it can be used to flag cases where this contrast is present, suggesting that we should further scrutinize the legitimacy of the undertaking as a whole.

If we broaden our perspective even further and consider commercially motivated changes in property structures on the global stage, this perspective suggests itself with even greater force. In fact, it seems appropriate to speak of a crisis of confidence in property, particularly in relation to land rights, arising from how powerful commercial interests usurp proprietary power over an increasingly large share of the world's resources. This is the phenomenon known as {\it land grabbing}, which has received much critical attention in recent years.\footnote{See generally \cite{borras11}.}

So far, most research on land grabbing has looked at how commercial interests, often cooperating with nation states, exploit weaknesses of local property institutions, to acquire land voluntarily, or from those who lack formal title. However, the similarity between economic development takings and state-aided land grabbings in favour of large commercial companies is striking. Specifically, it has been noted that the purported public interest in economic development can be used to justify massive land grabs that would otherwise appear unjustifiable. In a recent article, Smita Narula cites {\it Kelo} directly and warns that procedural safeguards alone might not provide sufficient protection against abuse. She writes:

\begin{quote}
Procedural safeguards, however, can all too easily be co-opted by a state because its claims about what constitutes a public purpose may not be easy to contest. Particularly within the context of land investments, states could use the very general and under-scrutinized language of ``economic development'' to justify takings in the public interest.\footcite[157]{narula13}
\end{quote}

This underscores the broader relevance of the study of economic development takings. In addition, it asks us to keep in mind that the question of what can be justified in the name of ``economic development'' is a general one, not confined to particular systems for organizing property rights.\footnote{To address this, and to restore confidence in the institution of property more generally, some academics and policy makers have proposed a novel concept of property as a human right. It has been argued, in particular, that a human right to land should be \isr{recognised} on the international stage, a right that would apply even when those affected by a land grab lack formal title. If successful, this approach promises to deliver basic protection against interference in established patterns of property use independently of how particular jurisdictions approach property. Specifically, it would establish an important link needed to make the kinds of property protections discussed in this thesis justiciable in the context of land grabbing when those adversely affected lack formal title. See generally \cite{schutter10,schutter11,kunnerman13}.}

In India, for example, people have been displaced and dispossessed on a massive scale in the name of economic development.\footnote{See generally \cite{levien13}.} Furthermore, the state has actively used eminent domain to enable such processes. Indeed, it has been argued that the scope of eminent domain in India has become so wide that it allows for a ``complete assertion of power'' by the state.\footnote{See \cite[43]{cullet09};\cite{usha09}. For a concrete example of a case involving displacement on a very large scale, see \cite{cullet01}.} Interestingly, the scope of eminent domain has expanded alongside a judicial re-creation of property as a fundamental right (the right to property was removed from the Constitution in 1978).\footnote{See generally \cite{allen15}.} However, as argued by Allen, ``the Supreme Court’s emphasis on liberal entitlement, rather than solidarity or social obligation, is likely to deny the new right to property of relevance in cases where social justice is paramount.''.\footnote{See \cite[30]{allen15}.}

%In fact, the language of eminent domain has been used to justify controversial policy decisions also with respect to land that is not privately owned, but already under forms of state ownership/custodianship.\footnote{See \cite[141]{usha09}.} The notion of eminent domain is apparently so powerful that it can be used to silence opposition of all kinds, including that arising with respect to social and economic rights for the poor and the landless when they face displacement or loss of livelihoods.\footnote{See \cite[143-144]{usha09} (``the power of eminent domain has been interpreted as being close to absolute power of the State over all land and interests in land within its territory. The effect of this has been that those without access to land and rights over land (including the landless, artisans, women as a composite group), those who may have use rights but no titles, communities holding common rights and others with inchoate interests, have had to bear the burden heaved on to them by eminent domain.'').} 

This underscores the broader relevance of the theoretical framework developed in the first part of this thesis, especially the importance of emphasising the social functions of property. Moreover, while this thesis focuses on cases when those adversely affected by the use of eminent domain have recognised property interests, the social function perspective makes it natural to emphasise the wider societal effects of takings, including effects on non-owners. I return to this point in chapter \ref{chap:3} when I present a concrete proposal for a heuristic to test the legitimacy of economic development takings.

In the next section, I consider {\it Kelo} in more depth, to argue that strict judicial deference to legislative and executive decision-makers is inappropriate in this regard. I focus especially on Justice O'Connor's dissent, which I believe suggests a stricter standard of judicial review for economic development takings, without unduly undermining the value of deference to political decision-makers and the executive branch.

\subsection{{\it Kelo}: Casting Doubts on the Narrow Approach to Judicial Review}\label{sec:2:6:1}

In many jurisdictions, constitutional property rules indicate, with varying degrees of clarity, that eminent domain should only be used to take property either for ``public use'', in the ``public interest'', or for a ``public purpose''.\footnote{For instance, a rather unclear ``public use'' formulation is used in the takings clause of the US constitution, as well as in the Norwegian constitution, while both the ``public interest'' and the ``public purpose'' formulations (but not ``public use'') are used in the South African Constitution. See \cite[462]{walt11} (arguing that while ``public purpose'' would traditionally have been understood more narrowly, there is no generally observed difference between the two notions as they are now understood in South African law).} Such a restriction can be regarded as an unwritten rule of constitutional law, as in the UK, or it can be explicitly stated, as in the basic law of Germany.\footnote{See \cite[3-4]{sluysmans15}.} In some jurisdictions, for instance in the US and in Norway, explicit takings clauses exist, but do not provide much information about the intended scope of protection.\footnote{See chapter \ref{chap:3}, section \ref{sec:3:3} and chapter \ref{chap:5}, section \ref{sec:5:2}.}

The question arises to what extent these clauses give the judiciary a duty and a right to restrict the state's power to take property. In the US, most scholars agree that some judicial review based on the public use requirement is warranted, but there is great disagreement about its extent.\footcite[205]{berger78} In Norway, on the other hand, a consensus has developed whereby the notion of public use is interpreted so widely that it hardly amounts to a justiciable restriction at all.\footnote{See, e.g., \cite[368]{aall10}.} Indeed, the courts defer almost completely to the assessments made by the executive branch regarding the purposes that may be used to justify a taking.\footcite[368]{aall10}

Some US scholars adopt a similar stance, but others argue that the public use restriction should be read as a stricter requirement, forbidding the use of eminent domain unless the public will make actual use of the property that is taken.\footnote{Compare \cite{bell06,bell09,claeys04,sandefur06}.} Most scholars fall in between these two extremes. They regard the public use restriction as an important limitation, but they also \isr{emphasise} that the courts should normally defer to the legislature's assessment of what counts as a public use.\footnote{See, e.g., \cite{merrill86,alexander05}.}

As I discuss in more depth in chapter \ref{chap:3}, section \ref{sec:3:3}, the debate in the US has its roots in case law developed by state courts -- the federal property clause was for a long time not applied to state takings. This has changed, and today the Supreme Court has a leading role in this area of US law. It has developed a largely deferential doctrine, resembling the understanding of the public use limitation under Norwegian law.\footnote{See \cite{berman54,midkiff84,kelo05}.} The difference is that in the US, cases raising the issue still regularly arise and prove controversial. As mentioned in the introduction to this thesis, the most important such case in recent times was {\it Kelo}, decided by the Supreme Court in 2005.\footcite{kelo05} This case saw the public use question reach new heights of controversy in the US.\footnote{See, e.g., \cite{somin09}.}

As mentioned in the introduction, {\it Kelo} centred on the legitimacy of taking property to implement a redevelopment plan that involved the construction of research facilities for the drug company Pfizer. The homes of Suzanne Kelo and eight other home-owners stood in the way of this plan and the city decided to use the power of eminent domain to condemn them. Kelo and the other owners protested, arguing that making room for a private research facility was not a permissible ``public use''. The owners were represented by the libertarian legal firm {\it Institute for Justice}, which had previously succeeded in overturning similar instances of eminent domain at the state level.\footnote{See \cite{justice15}.} Kelo and the other owners lost the case before the state courts, but the Supreme Court decided to hear it and assessed its merits in great detail.

The precedent set by earlier federal cases such as {\it Berman} and {\it Midkiff} was clear: as long as the decision to condemn was ``rationally related to a conceivable public purpose'', it was to be regarded as consistent with the public use restriction.\footnote{See \cite[241]{midkiff84}; \cite{berman54}.} Moreover, the role of the judiciary in determining whether a taking was for a public purpose was regarded as ``extremely narrow''.\footcite[32]{berman54} It had even been held that deference to the legislature's public use determination was required ``unless the use be palpably without reasonable foundation'' or involved an ``impossibility''.\footnote{See \cite[66]{dominion25}; \cite[680]{gettysburg96}.}

Despite this, in the case of {\it Kelo}, the court hesitated. Part of the reason was no doubt that takings similar to {\it Kelo} had been heavily criticised at state level, with an impression taking hold across the US that eminent domain abuse was becoming a real problem.\footnote{See, e.g., \cite[667-669]{sandefur05}.} A symbolic case that had contributed to this worry was the infamous case of {\it Poletown}.\footcite{poletown81} In this case, General Motors had been allowed to raze a town to build a car factory, a decision that provoked outrage across the political spectrum.\footnote{See generally \cite{sandefur05}.} The case was similar to {\it Kelo} in that the taker was a powerful commercial actor who wanted to take homes. This, in particular, served to set the case apart from {\it Midkiff}, which involved a taking in \isr{favour} of tenants, and to some extent also {\it Berman}, which involved a taking of businesses (and homes) in the interest of removing blight.\footnote{\cite{berman54,midkiff84}.} Moreover, the Michigan Supreme Court had recently decided to overturn {\it Poletown} in the case of {\it Hatchcock}.\footcite{wayne04} Hence, it seemed that the time had come for the Supreme Court to re-examine the public use question.\footnote{See, e.g., \cite{sandefur05,claeys04}.}

Eventually, in a 5-4 vote, the court decided to apply existing precedent, leading it to uphold the taking of Kelo's home. The majority also made clear that economic development takings were indeed permitted under the public use restriction, also when the public benefit was indirect and a private company would benefit commercially.\footcite[469-470]{kelo05} This resulted in great political controversy in the US. According to Ilya Somin, the {\it Kelo} case ranks among the most disliked decisions in the history of the Supreme Court.\footcite[2]{somin11} 

Importantly, many commentators emphasised that {\it Kelo} was an economic development taking.\footnote{See, e.g., \cite{somin07,cohen06}.} This category had no clear basis in the property discourse before {\it Kelo}. Indeed, in terms of established legal doctrine, it would be more appropriate to say that the case revolved entirely around the notion of ``public use''. However, when considering the most common reasons given for condemning the outcome in {\it Kelo}, it becomes clear why many felt it was natural to classify the case along additional dimensions. A survey of the literature shows that many made use of a combination of substantive and procedural arguments to paint a bleak picture of the {\it context} surrounding the decision to take Kelo's home. Important aspects of this include the imbalance of power between the commercial company and the owners, the incommensurable nature of the opposing interests, the close relationship between the company and the government, and the feeling that the public benefit -- while perhaps not insignificant -- was made conditional on, and rendered subservient to, the commercial benefit that would be bestowed on a commercial beneficiary.\footnote{See, for instance, \cite{underkuffler06,somin07,sandefur06,cohen06,hafetz09,hudson10}.} Plainly, the decision to condemn in {\it Kelo} appeared to suffer from what I will refer to here as a {\it democratic deficit}.

The social function theory of property makes it natural to emphasise the worry that economic development takings can lack democratic merit. Moreover, the theory inspires reasoning that can justify a departure from the established doctrine of extreme deference, in favour of more substantial judicial review. It seems to me that such a perspective was indeed adopted by the minority of the Supreme Court in {\it Kelo}, particularly Justice O'Connor.\footnote{\cite[494-505]{kelo05}.} She wrote a strongly worded dissent, characterising the majority's decision as follows:

\begin{quote}
Any property may now be taken for the benefit of another private party, but the fallout from this decision will not be random. The beneficiaries are likely to be those citizens with disproportionate influence and power in the political process, including large corporations and development firms. As for the victims, the government now has license to transfer property from those with fewer resources to those with more. The Founders cannot have intended this perverse result.\footcite[505]{kelo05}
\end{quote}

As demonstrated by this quote, the overarching concern raised by Justice O'Connor was that allowing takings such as {\it Kelo} could legitimise a form of governmental interference in property that would systematically \isr{favour} the rich and powerful to the detriment of the less resourceful. In this way, the power of eminent domain could become a tool for establishing and sustaining patterns of inequality, under the pretence of providing an economic benefit. Hardly anyone would openly regard this as desirable. Indeed, one of the justices who voted with the majority, Justice Kennedy, formulated a separate concurring opinion to emphasise that detailed, albeit deferential, judicial scrutiny is appropriate in cases like {\it Kelo}. According to Justice Kennedy, this is needed to rule out that the public interest in economic development is used merely as a pretext to bestow benefits on private companies.\footnote{See \cite[490-493]{kelo05}. The form of judicial review Justice Kennedy proposes is rather weak, admitting only that the courts have a role to play in cases when the government decision does not appear to be rationally related to a legitimate purpose (known as rational basis review in US constitutional law).}

The main difference between his opinion and that of Justice O'Connor does not appear to hinge on how they interpret the meaning of public use. Rather, the crucial difference seems to arise from the fact that Justice O'Connor is more willing to address injustices associated with economic development takings at the systemic level. Her perspective is clearly a powerful one, at least partially responsible also for the wide disapproval of {\it Kelo} among the public. Indeed, if Justice O'Connor's predictions about the systemic fallout of {\it Kelo} are correct, most would probably agree that the result would be ``perverse''.

The crucial question therefore becomes whether her predictions are warranted. In fact, the main importance of her dissent might be that it flags this issue as a crucial one in relation to very typical uses of eminent domain in the modern world. In light of its high level of generality, Justice O'Connor's dissent becomes a call for empirical work, to shed light on how economic development takings actually come about, and how they affect political, social and bureaucratic processes. In addition, her dissent raises the question of how to {\it avoid} negative effects, that is, how to design rules and procedures that can help bring about desirable economic development without creating a democratic deficit. These will be the main themes that I discuss in the remainder of this thesis.

\section{Conclusion}\label{sec:2:7}

In this chapter, I have proposed a theoretical foundation for approaching the question of economic development takings. Specifically, I suggested that a social function perspective on property is an appropriate starting point for an analysis of such takings. Furthermore, I argued that the notion of human flourishing provides a good template for carrying out a normative analysis of when economic development takings are legitimate. This approach, in turn, led to an argument in favour of a broader style of judicial review of such takings, namely one that embraces considerations based on social justice and the ideal of democracy.

\noo{ To illustrate the subtleties that this approach helps shed light on, I considered a concrete example of a commercial scheme that looked like it might well result in compulsory acquisition of land, namely Donald Trump's controversial plans to develop a golf course on a site of special scientific interest close to Aberdeen, Scotland. In the end, the plans did {\it not} require takings, as Trump was able to make creative use of property rights he acquired voluntarily, against the complaints of his \isr{neighbours}.

This turn of events did not make the example less relevant to this thesis. Rather, it served to highlight that the question of economic development takings is not a black-and-white balancing act between property privileges on one side and the good intentions of the regulatory state on the other. Specifically, the example of Trump coming to Scotland allowed me to \isr{emphasise} the importance of context when assessing both the nature of property, the many ways of taking, and the meaning of protecting owners against predation.

The protection sought by those who opposed Trump's golf course did not target their entitlements as individuals. Rather, it targeted the community, as the owners felt it would be detrimental to the community, and to their lives, if Trump was allowed to redefine the social functions of local property. After Trump decided not to pursue compulsory purchase, protecting the property of these members of the community became a question of {\it restricting} the degree of dominion that Trump could exercise over his own property. Hence, under a conventional and overly simplistic way of looking at these matters, protecting property became tantamount to restricting its use, a seeming paradox.

To resolve this paradox, and to arrive at a better conceptual understanding of economic development takings, I looked to various theories of property. I noted that there are differences between civil law and common law theorising about property, but I concluded that these differences are not particularly relevant to the questions studied in this thesis. In particular, I observed that neither the bundle theory, dominant in the common law world, nor the dominion theory, taught to many civil law jurists, helped me clarify economic development takings as a category of legal thought.

I then went on to consider more sophisticated accounts of property, focusing on the social function theory, which emphasises how property structures, and is structured by, social and political relations within a society. 

I went on to argue that in the first instance, the social function theory should be understood as giving us {\it descriptive} insights into the workings of property and its role in the legal order. In this regard, I advanced a different stance than many property scholars, by arguing that we should aim to decouple descriptive insights from normative aspects of the theory, to allow the social function theory to serve as a common ground for further value-driven debate.

I then went on to clarify my own starting point for engaging in such debate, by expressing support for the human flourishing theory proposed by Alexander and Pe\~{n}alver. This theory is based on the premise that property {\it should} enable -- and even compel -- individuals and their communities to  participate in social and political processes. I argued that property's purpose in this regard is  fundamental to its proper role in a democratic society, as an anchor for participatory decision-making.  

}

\noo{Moreover, I noted that the human flourishing theory contains a further important insight, concerning the scope of the state's power to protect. In particular, the theory asks us to recognise that protecting property against interference that is harmful to human flourishing is a responsibility that the state has even in cases when the individual owners themselves neglect to defend their property, for instance as a result of financial incentives to remain idle. In other words, some functions of property are such that owners have an obligation to preserve them, while the state has a duty to protect them, potentially even against the will of the owners.}

On this basis, I went on to give a preliminary analysis of economic development takings. To make the discussion concrete, I considered the case of {\it Kelo}, which propelled the notion of an economic development taking to the front of the takings debate in the US. I focused particularly on the dissenting opinion of Justice O'Connor, and I argued that she approached the issue in a way that is consistent with the theoretical basis proposed in this chapter.

In the next chapter, I will continue my analysis of economic development takings, by considering the legitimacy question in more depth. Specifically, I ask what role the law can and should play in ensuring that the state's power to take property is not used improperly in the context of economic development. This will lead to two sets of broad policy recommendations for dealing with the ills of economic development takings, targeting the diagnosis and the cure respectively. %This, in turn, will set the stage for the second part of this thesis, where the recommendations provided in the next chapter will be tested and made more concrete through a case study of takings for Norwegian hydropower development.

\noo{
how such takings are dealt with in Europe and the US respectively. I note that the category has yet to receive much attention in Europe, so the discussion focuses on the US. Here this issue has received a staggering level of attention after {\it Kelo}. To get a broader basis upon which to \isr{assess} all the various arguments that have been presented, I consider the historical background to the issue as it is discussed in the US. This involves giving a detailed presentation of the public use restriction, as it was developed in case law from the states during in the 19th and early 20th century. I then connect this discussion with recent proposals to deal with economic development takings, responding to the backlash of {\it Kelo} by aiming to address the democratic deficit of such takings.

Later, when I begin to consider the law relating to Norwegian hydropower, I will look back at the theoretical basis provided in the present chapter to guide the analysis. In particular, I focus on certain decision-making mechanisms that have developed on the ground in Norway, as a practical response to the increased tendency for local owners to engage in hydropower development. I will argue that this shows the conceptual strength of the idea that property is irreducibly embedded in community, continuously evolving alongside institutions of participatory decision-making. }
%%\newcommand{\isr}[1]{{#1}}

\chapter{Possible Approaches to the Legitimacy Question}\label{chap:3}

\section{Introduction}\label{sec:3:1}

%In the previous chapter, I introduced the social function perspective of property and argued in favour of a normative approach to property based on the notion of human flourishing. Moreover, I argued that economic development takings make up a separate category of interference with private property, deserving of special attention. I also placed this category in the theoretical landscape, by relating it to the theory of property presented in the first part of the chapter. Specifically, I argued that economic development takings raise questions that require us to depart from the individualistic, entitlements-based narrative that has tended to dominate in property theory.

%There are many ways of thinking about the legitimacy of takings. Moreover, how one chooses to approach this in the abstract is likely to depend not only on one's legal training, but also on more overarching visions of society. Specifically, it seems that one's approach to the legitimacy question will invariably depend also on one's vision of the relationship between the government, the law, and the institution of private property in a democratic system. To ask what imbues an act of taking with legitimacy, is to ask how this relationship should be.

This chapter considers the question of legitimacy of economic development takings in more depth. First, it presents there existing approaches, based on evidence from England and Wales, the United States, and the European Court of Human Rights. The chapter starts by considering the idea that parliament itself imbues each instance of a taking with legitimacy, as the result of a decision made in a legitimate manner within a democracy. Such a narrative can easily become quite focused on procedural aspects, leaving little room for substantive judicial scrutiny of takings. The doctrine of deference, in particular, can come to develop as the main norm that guides the courts when faced with controversial takings. In England and Wales, this perspective carries great weight, particularly historically, when parliament itself would authorise most takings directly through so-called private Acts. 

I argue that the expanding state and the increasingly expanding use of eminent domain puts this perspective on legitimacy under strain. To some extent, it can be upheld by a well-organised executive, compelled to remain faithful to parliament and the ideas of democracy. However, a threat to the stability and success of such a procedural approach can arise from the lack of any clearly defined safeguards to protect against institutional failure and substantive abuse. If property as an institution begins to falter, for instance because takings for profit become too prevalent, the courts might find themselves unable to intervene on behalf of those democratic ideals that motivated the idea of deference in the first place. I will argue that recent cases of economic development takings in England illustrate this danger, suggesting that we should also consider substantive approaches to legitimacy.

Following up on this, I go on to consider the US, where the public use restriction is considered to be an important substantive limitation on the government's power to take property. I track the history of public use scrutiny in some depth, showing that it was widespread and extensive at the state level, at least until a contrasting position of almost unconditional deference to the legislature was adopted by the Supreme Court in the case of {\it Berman}. After this, at the federal level at least, the public use restriction was effectively stripped of its content.\footnote{See \cite{berman54}.} The eventual backlash of this came with {\it Kelo}, which was decided in keeping with precedent, but with severe doubts arising among the justices, particularly those who looked at the history of the public use doctrine and how it had worked prior to {\it Berman}.

I argue that a contextual approach to the public use requirement, based on broad assessments of local conditions, was prevalent among the state courts in early public use cases. This arguably also reflects a social function understanding of property, connecting the public use test to the property theorising in chapter 1. Moreover, I note that there has been a resurgence of extensive public use scrutiny after {\it Kelo}, particularly at the state level. I note, however, that this change appears to have been largely ineffective at curbing dubious uses of eminent domain. Specifically, it has been argued that recent reforms, and broad substantive standards such as the public use requirement, are likely to become only symbolic nods to the danger of abuse: hidden within the complex arrangements of modern government, there is business as usual regardless.

This in turn raises the issue of how to combine the institutional and the procedural perspective on legitimacy, to ensure that substantive standards actually translate into effective protection. This brings me to the third approach to legitimacy, which I call the institutional fairness approach. I argue that this approach has been adopted recently by the ECtHR in Strasbourg, as they have developed a system of pilot judgements to deal with their vastly increasing case load. The idea of such judgements is that the Court will focus on systemic problems, to determine whether they should order the state to take general measures to improve their own institutions. By doing this, the Court will protect itself from having to deal with many similar cases. Instead, it can move on to novel issues of principle that need to be considered.

Quite apart from the practical motivation behind this development, I argue that the institutional perspective on fairness that it conditions is the way forward towards testing for legitimacy in takings cases. It should work well because it allows courts to adopt a middle ground between the procedural and the substantive approach. I consider the case of {\it Hutten-Czapska v Poland} in some depth to argue for the merits of this approach.\footnote{See \cite{hutten06}.}

Following up on this, I consider the question of how the courts should proceed to assess the legitimacy of economic development taking against such an institutional fairness perspective.\footnote{There is not yet any case law on this from the ECtHR.} Building on a list of conditions due to Kevin Gray, I propose a concrete heuristic for this purpose. In addition to the original points made by Gray, I add three of my own, inspired by the discussion in this and the previous chapter. 

A legitimacy test can never provide more than a partial solution to the legitimacy problem. Specifically, in cases when the desire for economic development is a genuine reflection of democratic decision-making, the follow-up question is how to better enable the collective to communicate this desire to private owners, without resorting to eminent domain. I address this question by looking to the theory of governance for common pool resources, developed by Elinor Ostrom and others. Specifically, I note how the connection between this theory and property law suggests the possibility that new institutions should be introduced to allow the collective to push for economic development involving privately owned property. In fact, such a proposal has already been made, by Heller and Hills, who proposed that so-called {\it Land Assembly Districts} could replace the use of eminent domain in many cases when holdouts make economic development on privately owned land hard to implement.  

I analyse the proposal in some depth, pointing out problems to suggest that Land Assembly Districts are not the final answer to the legitimacy question for economic development takings. Specifically, I note the context dependence of their proposal, and the underlying tension between the ideal of self-governance and the fear of tyranny by local elites. This goes to show that the cure for illegitimacy, much like its diagnosis, depends on the circumstances, and what one regards as the property's proper function. In light of this, I believe the critical examination of Land Assembly Districts marks a natural end to this chapter, as well as to the theoretical part of this thesis as a whole.

\section{England and Wales: Legitimacy through Parliament}\label{sec:3:2}

In England and Wales, the principle of parliamentary sovereignty and the lack of a written constitutional property clause has led to expropriation being discussed mostly as a matter of administrative law and property law, not as a constitutional issue.\footnote{See generally \cite{taggart98}.} Moreover, the use of compulsory purchase -- the term used to denote takings in the UK -- has not been restricted to particular purposes as a matter of principle.\footnote{See, e.g., \cite[48-49]{waring09}.} The uses that can justify taking property by compulsion are those uses that parliament regard as worthy of such consideration.\footnote{See \cite[48-49]{waring09}.} However, as private property has typically been held in high regard, the power of compulsory purchase has traditionally been exercised with caution.\footnote{See \cite[47-48]{waring09}.}

In his {\it Commentaries}, William Blackstone famously described property as the ``third absolute right'' that was ``inherent in every Englishman''.\footnote{See \cite[134-135]{blackstone79}. The first right, according to Blackstone, is security, while the second is liberty.} Moreover, Blackstone expressed a very restrictive view on the appropriateness of expropriation, pointing out that it was only the legislature that could legitimately interfere with property rights. He warned against the dangers of allowing private individuals, or even public tribunals, to be the judge of whether or not the common good could justify takings. Blackstone went as far as to say that the public good was ``in nothing more invested'' than the protection of private property.\footcite[134-135]{blackstone79}

In terms of historical accuracy, Blackstone's claims about property in England and Wales can be questioned. Specifically, it has been argued that his description of property might be shaped not so much by practical reality as by political values gaining ground among the bourgeoisie after the decline of the feudal system.\footnote{See \cite[34-35]{waring09} (describing Blackstone's account as a ``myth'').} However, the fact remains that compulsory purchase powers appear to have been granted relatively infrequently during his time, with no great increase in prevalence until the industrial revolution and the birth of the modern state.\footnote{See \cite[15]{allen00}. That said, recent scholarship has pointed out that expropriation appears to have taken place more frequently than previously thought, particularly following the glorious revolution, see \cite{hoppit11}.} Moreover, the conferral of such powers would typically require parliamentary involvement on a case-by-case basis, a practice reflecting that takings of private property, although far from unheard of, were indeed considered draconian.\footnote{See \cite[43-46]{nulty12}.}

Interestingly, the procedure followed by parliament in takings cases often resembled a judicial procedure; the interested parties were given an opportunity to present their case to parliament committees that would then effectively decide whether or not compulsion was warranted.\footnote{See \cite[13-16]{allen00}.} On the one hand, the direct involvement of parliament in the decision-making is suggestive of a fundamental respect for property rights. But at the same time, parliamentary sovereignty meant that the question of legitimacy was rendered mute as soon as compulsory purchase powers had been granted. The courts were not in a position to scrutinize takings at all, much less second-guess parliament as to whether or not a taking was for a legitimate purpose.\footnote{See, e.g., \cite[643]{nulty12}.}

During the 19th Century, as an industrial economy developed, so-called {\it private} acts, granting compulsory purchase powers to specific legal persons, grew massively in scope and importance.\footnote{See \cite[204]{allen00}.} Railway companies, in particular, regularly benefited from such acts.\footnote{\cite[204]{allen00}. See generally \cite{kostal97}.} During this time, the expanding scope of private-to-private transfers for economic development led to high-level political debate and controversy.\footnote{See \cite[204]{allen00}.} Usually, it would attract particular opposition from the House of Lords.\footcite[204]{allen00} Interestingly, this opposition was not only based on a desire to protect individual property owners. It also often reflected concerns about the cultural and social consequences of changed patterns of land use.\footcite[204]{allen00}

Hence, the early {\it political} debate on economic development takings in the UK shows some reflection of a social function approach to property protection. At the same time, as society changed following increasing industrialisation, a more expansive approach to compulsory purchase would eventually emerge as the norm.\footnote{Arguably, the social function perspective helps explain why this happened. Indeed, the expanded use of private takings in England during the 19th century, particularly in connection with the railways, might have served a more easily justifiable social function than that commonly associated with economic development takings today. Waring, in particular, notes how railway takings tended to affect aristocratic landowners rather than marginalised groups (``unlike private takings today, the railway legislation was most likely to affect those who could best defend their property rights from attack''), see \cite[111]{waring09}.} The idea that economic development could justify takings gradually became less controversial.

Today, the law on compulsory purchase in England is regulated in statute. Hence, parliament rarely gets involved on a case-by-case basis, and the role of the courts is largely limited to the application and interpretation of statutory rules.\footnote{See \cite[116-121]{waring09}. Some common law rules still play an important role, such as the {\it Pointe Gourde} rule, which stipulates that changes in value due to the compensation scheme itself should be disregarded when calculating compensation to the owner. The rule takes it name the case of \cite{gourde47}. The underlying principle, including also statutory regulations with a similar effect, is referred to as the ``no scheme'' principle, see \cite{lawcom01}. The principle is found in many jurisdictions, see \cite{sluysmans14}. It is often quite contentious, and notoriously hard to apply in practice. For a recent clarification of (some aspects of) the principle, see \cite{waters04}. I note that a strict interpretation of the no-scheme principle effectively precludes benefit sharing between takers and owners, a phenomenon that is also relevant in the context of economic development takings. See generally \cite{dyrkolbotn15}.} Moreover, with respect to the question of legitimacy of takings more broadly, the starting point for English courts is that this is a matter of ordinary administrative law.\footnote{See \cite{taggart98}.} More recently, the \cite{hra98} adds to this picture, since it incorporates the property clause in P1(1) into English law. Even so, the usual approach in England is to judge objections against compulsory purchase orders on the basis of the statutes that warrant them, rather than constitutional principles or human rights provisions that protect property.\footnote{See \cite[121-132]{waring09}. The important statutes are the \cite{ala81}, the \cite{lca61}, the \cite{cpa65}, the \cite{tcpa90} and the \cite{pcpa04}.} It is typical for statutory authorities to include standard reservations to the effect that some public benefit must be identified in order to justify a compulsory purchase order, but the scope of what constitutes a legitimate purpose can be very wide. For instance, to justify a taking under the \cite{tcpa90}, it will generally suffice to argue that it will ``facilitate the carrying out of development, redevelopment or improvement on or in relation to the land''.\footcite[226]{tcpa90}

While various governmental bodies are authorised to issue compulsory purchase orders (CPOs), a CPO typically has to be confirmed by a government minister.\footnote{See \cite[48]{waring09}.} The affected owners are given a chance to comment, and if there are objections, a public inquiry is typically held. The inspector responsible for the inquiry then reports to the relevant government minister, who makes the final decision about whether or not it should be granted, and on what terms. The CPO may later be challenged in court, but then on the basis of the statute authorising it, not on the basis of whether or not the purpose mentioned in that statute is legitimate as such.\footnote{See, e.g., \cite[48-49]{waring09}. The typical way to launch an attack on a taking would be to argue that it serves a purpose that falls outside the scope of the statute authorising it, or, more subtly, that the administrative decision-maker took irrelevant purposes into account when granting the power. See, e.g., \cite{sainsbury10}.} 

That said, the idea that property may only be compulsorily acquired when the public stands to benefit permeates the system. Indeed, this has also been regarded as a constitutional principle, for instance by Lord Denning in {\it Prest v Secretary of State for Wales}.\footnote{See \cite[198]{prest82} (``I regard it as a principle of our constitutional law that no citizen is to be deprived of his land by any public authority against his will, unless it is expressly authorised by Parliament and the public interest decisively so demands.'').} Moreover, in {\it R v Secretary of State for Transport, ex p de Rothschild}, Slade LJ spoke of ``a warning that, in cases where a compulsory purchase order is under challenge, the draconian nature of the order will itself render it more vulnerable to successful challenge''.\footcite[938]{rothschild89}

In keeping with the principle of parliamentary sovereignty, this warning targets judicial review of administrative decision-making, not legislation. Despite this limitation, the English approach to legitimacy has traditionally proved quite effective in preventing controversy from arising with respect to the use of eminent domain.\footnote{See generally \cite{allen10}.} An underlying respect for private property, as well as the idea that the authority to interfere with it rests on the authority of parliament, appears to have influenced the decision-making framework and the surrounding administrative practices. Hence, legitimacy has become an objective to be pursued through legislation, regulation, and administrative practice, not judicial scrutiny.\footnote{For a more detailed analysis of how this works, noting, among other things, that higher levels within the executive are also meant to act as safeguards of private property, filling -- to some extent -- the possible role of courts in this regard, see \cite[85-100]{allen08}.}

However, England and Wales have also seen controversial economic development takings being challenged in court. Indeed, such cases appear to have become more frequent.\footnote{See generally \cite{gray11}.} For instance, in the case of {\it Alliance}, many properties were taken in order to facilitate the construction of a new stadium for the football club Arsenal.\footcite{alliance06} Some owners who stood to lose their business premises protested on the basis that the purpose was dubious, pointing also to the fact that the inspector in charge of the public inquiry had recommended against the takings.\footcite[6-7]{alliance06} Their arguments also invoked P1(1) of the ECHR, to overcome the limitations of traditional judicial review in England and Wales. However, these argument were all quite summarily rejected by the Court.\footnote{See \footcite[6-7]{alliance06}. For a critical discussion, describing the Court's assessment against P1(1) as ``worryingly brief'', see \cite{gray11}.}

Arguably, the {\it Alliance} case reflects a weakness of the English approach to legitimacy. This weakness, moreover, appears to go beyond whatever doubts one might have about the principle of parliamentary sovereignty applied to property as a constitutional and/or human right. Specifically, if the framework laid down or condoned by parliament greatly empowers the administrative branch, while failing to appropriately regulate administrative practices, the deference due to parliament might effectively become undue deference to the executive branch. If the {\it practice} of using compulsory purchase continues to expand in relation to for-profit undertakings, there appears to be a significant risk of abuse associated with broad powers granted to the executive to take property for economic development. Plainly, if values such as those expressed by Blackstone are discredited further, there appears to be a lack of alternative sources for legitimacy in a system so reliant on a narrative of pure procedure.

To some extent, it would be possible for the Supreme Court to develop a more restrictive stance on compulsory purchase to address this, within the established constitutional order. In fact, there are some signs that this might be about to happen, specifically with respect to the broad powers granted under the the \cite[226]{tcpa90}. In the case of {\it R (Sainsbury's Supermarkets Ltd) v Wolverhampton City Council}, Lord Walker cited {\it Kelo} and went on to comment that ``economic regeneration brought about by urban redevelopment is no doubt a public good, but ``private to private'' acquisitions by compulsory purchase may also produce large profits for powerful business interests, and courts rightly regard them as particularly sensitive''.\footnote{See \cite[82]{sainsbury10}.}

However, the outcome of the {\it Sainsbury} case arguably also underscores the weaknesses of an indirect approach to legitimacy through administrative law. Instead of relying on Lord Walker's observations about the sensitivity of economic development takings, the majority of the Court quashed the compulsory purchase order on the basis that the local government had taken into account promises that the taker had made regarding a regeneration project in a different part of town. This was regarded as contravening section 226 of the \cite{tcpa90}, which only directs attention at the potential for improvements on or in relation to ``the land'', i.e., the land that is subject to compulsory purchase. The reasoning behind the decision, therefore, rests largely on a technicality, not any substantive assessment of legitimacy.

On a more purposive assessment, the taking in {\it Sainsbury} should arguably even have been upheld: the owner and the taker were both large commercial companies, they each owned a share of a plot of land suitable for joint development, they both wanted to develop at the expense of the other party, and the taker appeared to have the best overall plan for the community. Ironically, the English approach resulted in such a taking being struck down as illegitimate, while the taking in {\it Alliance}, involving the displacement of local people in favour of a football club, received little or no scrutiny at all. In light of this, it seems that alternatives to the traditional idea of legitimacy should be considered, at least if one agrees with Lord Walker's characterisation of economic development takings as ``particularly sensitive''.\footnote{See \cite[82]{sainsbury10}.}

\section{The US: Legitimacy through Public Use}\label{sec:3:3}

By contrast to the situation in England and Wales, the US Constitution is a basis for judicial review also with respect to the federal and state legislatures. Considering its status as a basis for potentially extensive review, the Constitution is remarkably terse. The takings clause, arriving as the final clause of the fifth amendment, reads simply ``nor shall private property be taken for public use, without just compensation''.\footnote{See \cite{us}.}

The compensation requirement is clearly stated, if embryonic, but the takings clause is also understood to include the requirement that property may only be taken for ``public use''. This is the aspect of the clause that will interest me in this thesis, since it provides an anchor for legitimacy that is particularly relevant -- and contentious -- in relation to economic development takings.\footnote{The compensation requirement is also important, of course, but the problems is gives rise to are rather more technical, pertaining also more to the entitlements-aspect of property protection, not the social function dimensions I focus on in this thesis. For a more in-depth assessment of the compensation issue in the context of economic development takings, see \cite{dyrkolbotn15}.}
Specifically, the question is to what extent such takings offend against the clause: is a taking for economic development by a commercial company really a taking for ``public use''?

Going back to the time when the fifth amendment was introduced, there is not much historical evidence explaining why the takings clause was included in the Bill of Rights.\footnote{See \cite{fifth}.} Moreover, there is little in the way of guidance as to how the takings clause was originally understood. James Madison, who drafted it, commented that his proposals for constitutional amendments were intended to be uncontroversial.\footnote{See letters from Madison to Edmund Randolph dated 15 June 1789 and from Madison to Thomas Jefferson dated 20 June 1789, both included in \cite{madison79}.} Hence, it is natural to regard the takings clause as a codification of an existing principle, not a novel proposal. Indeed, several state constitutions pre-dating the Bill of Rights also included takings clauses, seemingly based on codifying principles from English common law.\footcite[See][299]{johnson11} 

%As Meidinger notes, the Americans had never really charged the British with abuse of eminent domain, and private property had tended to be respected, also in the colonies.\footcite[17]{meidinger80} This undoubtedly influenced early US law.

Just like English scholars at the time, early American scholars emphasised the importance of private property. James Kent, for instance, described the sense of property as ``graciously implanted in the human breast'' and declared that the right of acquisition ``ought to be sacredly protected''.\footnote{See \cite[see][257]{kent27}.} Indeed, the Supreme Court itself expressed similar sentiments early on, when it spoke of the impossibility of passing a law that ``takes property from A and gives it to B''.\footnote{This was a {\it de dicta} in \cite[388]{calder98}. See also \cite[310]{vanhorne95}.}

However, just as in England, this early US attitude changed in response to industrial advances and a desire for economic development. As the 19th century progressed, eminent domain was used more frequently, now also to benefit (privately operated) railroad operations, hydroelectric projects, and the mining industry.\footcite[23-33]{meidinger80} During this time, it also became increasingly common for landowners to challenge the legitimacy of takings in court, undoubtedly a consequence of the fact that eminent domain was used more widely, for new kinds of projects.\footcite[24]{meidinger80} 

Controversy over the public use requirement arose particularly often with respect to the so-called mill acts.\footnote{\cite[24]{meidinger80}. See also \cite[306-313]{johnson11} and \cite[251-252]{horwitz73}.} Such acts were found throughout the US, many of them dating from pre-industrial times when mills were primarily used to serve the farming needs of agrarian communities.\footnote{A total of 29 states had passed mill acts, with 27 still in force, when a list of such acts was compiled in \cite[17]{head85}. According to Justice Gray, at pages 18-19 in the same, the ``principal objects'' for early mill acts had been grist mills typically serving local agrarian needs at tolls fixed by law, a purpose which was generally accepted to ensure that they were for public use.} Following economic and technological advances, provisions originally enacted to serve local farming purposes were now being used by developers wishing to harness hydropower for manufacturing and hydroelectric plants.\footnote{See, e.g., \cite[18-21]{head85} and \cite[449-452]{minn06}.}

It is important to note, however, that mill acts could not be used to authorise large-scale compulsory transfer of natural resources from owners to non-owners.\footnote{See the discussion in \cite{head85}.} Rather, mill acts provided management tools that could be used to ensure that owners of water resources could make better use of their rights. This would sometimes involve allowing riparian owners to interfere with, or take a necessary part of, the property of their neighbours, e.g., by constructing dams that would flood neighbouring land.\footnote{See, e.g., \cite[265]{staples03}.} However, the primary purpose of most mill acts was to facilitate rational coordination among owners, to the benefit of their community as a whole. This point was frequently made by the courts to justify upholding takings on the basis of mill acts, including takings that would benefit the manufacturing industry.\footnote{See \cite{fiske31}. See also the discussion (including references to other cases) in \cite{head85}.}

%As the industrial use of mill acts increased in scope, the original aim of these acts gradually became overshadowed by the strength of the commercial interests involved. 
%This, in turn, lead to public use controversies arising in relation to provisions that had not previously raised any doubts.\footnote{See \cite{head86}.} The case law on
More generally, case law on public use from the state courts at this time was characterised by a highly contextual understanding of property protection and the meaning of public use.\footnote{See, e.g, \cite{scudder32} (taking upheld, but said that ``the great principle remains that there must be a public use or benefit. That is indispensable. But what that shall consist of, or how extensive it shall be to authorize an appropriation of private property, is not easily reducible to a general rule.''); \footcite[409]{seawell76} (taking for a mineral company upheld on the basis that mining was the ``greatest of the industrial pursuits'' in the state of Nevada and that the benefits of the industry were ``distributed as much, and sometimes more, among the laboring classes than with the owners of the mines and mills''.); \footcite[337]{ryerson77} (taking struck down, by a Court that was ``not disposed to say that incidental benefit to the public could not under any circumstances justify an exercise of the right of eminent domain''.\footcite[337]{ryerson77}. See also \cite{gray11} (with many references to state courts striking down takings as impermissible).} Arguably, the case law on public use from the states even deserves to be categorised as an early example of a legitimacy approach based on a social function understanding of property. Moreover, it was held to be of high quality, as indicated by the early Supreme Court jurisprudence on economic development takings, as discussed in the next section.

\subsection{Legitimacy as Discussed by the Supreme Court}\label{sec:3:3:1}

Initially, the Supreme Court held that the takings clause in the US Constitution did not apply to state takings at all.\footcite{barron33} Federal takings, on the other hand, were of limited practical significance since the common practice was that the federal government would rely on the states to condemn property on its behalf.\footcite[30]{meidinger80}

This changed towards the end of the 19th century, particularly following the decision in {\it Trombley v Humphrey}, where the Supreme Court of Michigan struck down a taking that would benefit the federal government.\footcite{trombley71} Not long after, in 1875, the first Supreme Court adjudication of a federal taking occurred, marking the start of the development of the federal doctrine on public use and legitimacy.\footcite{kohl75} 

At the same time, the Supreme Court began to hear takings cases originating from the states, first on the basis of the due process clause of the fourteenth amendment, introduced after the civil war.\footnote{See, e.g, \cite{head85}.} Later, in 1897, the Supreme Court held that state takings could be scrutinized also against the takings clause of the fifth amendment.\footnote{See \cite{chicago97}.}

\noo{ The early 20th century was a period of great optimism about the ability of {\it laissez faire} capitalism to ensure progress and economic growth, a sentiment that was reflected in the federal case law on eminent domain. A particularly clear expression of this can be found in {\it Mt Vernon-Woodberry Cotton Duck Co v Alabama Interstate Power Co}.\footcite{vernon16}  This case dealt with the legitimacy of condemnation arising from the construction of a hydropower plant. The Supreme Court held that it was legitimate, with the presiding judge arguing briskly as follows:

\begin{quote}The principal argument presented that is open here, is that the purpose of the condemnation is not a public one. The purpose of the Power Company's incorporation, and that for which it seeks to condemn property of the plaintiff in error, is to manufacture, supply, and sell to the public, power produced by water as a motive force. In the organic relations of modern society it may sometimes be hard to draw the line that is supposed to limit the authority of the legislature to exercise or delegate the power of eminent domain. But to gather the streams from waste and to draw from them energy, labor without brains, and so to save mankind from toil that it can be spared, is to supply what, next to intellect, is the very foundation of all our achievements and all our welfare. If that purpose is not public, we should be at a loss to say what is. The inadequacy of use by the general public as a universal test is established. The respect due to the judgment of the state would have great weight if there were a doubt. But there is none.\footcite[32]{vernon16}
\end{quote}

On the one hand, the Court notes the importance of deference to the {\it state} judgement (not specifically the judgement of the state legislature). On the other hand, it prefers to conclude on the basis of its own assessment of the purpose of the taking. This assessment, however, is not grounded in the facts of the case or the circumstances in Alabama. Rather, it is based on sweeping assertions about ``all our welfare'' and the desire to ``save mankind from toil that it can be spared''. This marks a contrast with the approach of state courts, as discussed in the previous subsection.
}

In federal takings cases, the Supreme Court showed little willingness to enforce a strict public use requirement. In {\it United States v Gettysburg Electric Railway Co}, a case from 1896, deference to the legislature in federal takings cases was referred to as a principle that should be observed unless the judgement of the legislature was ``palpably without reasonable foundation''.\footcite[680]{gettysburg96} 

Importantly, however, such a deferential stance was not adopted in cases originating from the states. In {\it Cincinatti v Vester}, a case from 1930, the Supreme Court commented that ``it is well established that, in considering the application of the Fourteenth Amendment to cases of expropriation of private property, the question what is a public use is a judicial one''.\footcite[447]{vester30}

In the earlier case of {\it Hairston v Danville \& W R Co}, from 1908, the same was expressed by Justice Moody, who surveyed the state case law and declared that ``the one and only principle in which all courts seem to agree is that the nature of the uses, whether public or private, is ultimately a judicial question.''\footcite[606]{hairston08} Justice Moody continued by describing in more depth the typical approach of the state courts in determining public use cases:

\begin{quote}
The determination of this question by the courts has been influenced in the different states by considerations touching the resources, the capacity of the soil, the relative importance of industries to the general public welfare, and the long-established methods and habits of the people. In all these respects conditions vary so much in the states and territories of the Union that different results might well be expected.\footcite[606]{hairston08}
\end{quote}

Justice Moody goes on to give a long list of cases illustrating this aspect of state case law, showing how assessments of the public use issue had been inherently contextual.\footcite[607]{hairston08} Following up on this, he points out that ``no case is recalled'' in which the Supreme Court overturned ``a taking upheld by the state {\it court} as a taking for public uses in conformity with its laws'' (my emphasis). After making clear that situations might still arise where the Supreme Court would not follow state courts on the public use issue, Justice Moody goes on to conclude that the cases cited ``show how greatly we have deferred to the opinions of the state courts on this subject, which so closely concerns the welfare of their people''.\footcite[606]{hairston08}

{\it Hairston} is important for three reasons. First, it makes clear that initially, the deferential stance in cases dealing with state takings was primarily directed at state courts rather than legislatures and administrative bodies. Second, it demonstrates federal recognition of the fact that a consensus had emerged in the states, whereby scrutiny of the public use determination was consistently regarded as a judicial task.\footnote{Indeed, {\it Hariston} provides the authority for {\it Vester} on this point. See \cite[606]{vester30}.} Third, it provides a valuable summary of the contextual approach to the public use test that had developed at the state level. 

The {\it Hairston} Court clearly looked favourably on the case law from state courts. Importantly, when a deferential stance was adopted, this was clearly contingent on the assumption that state courts would continue to administer the public use test with the required vigour. Despite this, {\it Hairston} would later be cited as an early authority in favour of almost unconditional deference to legislators.\footnote{In fact, it was cited in this way also by the majority in {\it Kelo}, see \cite[482-483]{kelo05}.} 

This happened in {\it US ex rel Tenn Valley Authority v Welch}, concerning a federal taking.\footcite[552]{welch46} The Court first cited {\it US v Gettysburg Electric R Co} as an authority in favour of deference with regards to the public use limitation.\footcite{gettysburg96} The Court then paused to note that {\it Vester} later relied on the opposite view, namely that the public use test was a judicial responsibility.\footcite{vester30} The Court then attempts to undercut this by setting up a contrast between {\it Vester} and {\it Hairston}, by selectively quoting the observation made in the latter case that the Supreme Court had never overruled the state courts on the public use issue.\footnote{See \cite[552]{welch46}.} Hence, {\it Hairston} is effectively used to argue against judicial scrutiny, in a manner that is quite incommensurate with the full rationale behind the Court's decision in that case.

Later, {\it Welch} was used as an authority in the case of {\it Berman v Parker}.\footcite{berman54} This case concerned condemnation for redevelopment of a partly blighted residential area in the District of Colombia, which would also condemn a non-blighted department store. In a key passage, the Court states that the role of the judiciary in scrutinizing the public purpose of a taking is ``extremely narrow''.\footcite[32]{berman54} The Court provides only two references to previous cases to back up this claim, one of them being {\it Welch}.\footnote{The other case, {\it Old Dominion Land Co v US}, concerned a federal taking of land on which the military had already invested large sums in buildings. The Court commented on the public use test by saying that ``there is nothing shown in the intentions or transactions of subordinates that is sufficient to overcome the declaration by Congress of what it had in mind. Its decision is entitled to deference until it is shown to involve an impossibility. But the military purposes mentioned at least may have been entertained and they clearly were for a public use'', see \cite[66]{dominion25}. A partial quote, to the effect that deference to the legislature is in order except when it involves an ``impossibility'', was used to justify the decision in \cite[240]{midkiff84}.}

Moreover, both of the cases cited were concerned with federal takings, while in {\it Berman} the Court explicitly says that deference is due in equal measure to the state legislature.\footcite[32]{berman54} It is possible to regard this merely as a {\it dictum}, since the District of Columbia is governed directly by Congress. However, {\it Berman} was to have a great impact on future cases. In effect, it undermined a large body of case law on judicial review of takings without engaging with it at all.

In {\it Hawaii Housing Authority v Midkiff}, the Supreme Court further entrenched the principle expressed in {\it Berman}.\footcite{midkiff84} Here the state of Hawaii had made use of eminent domain  to break up an oligopoly in the housing sector. Given the circumstances of the case, it would have been natural to argue in favour of this taking on the basis that it served a proper public purpose.

However, the Court instead decided to rely on the doctrine of deference, shunning away from scrutinizing the takings purpose. Justice O'Connor, in particular, observed that ``judicial deference is required because, in our system of government, legislatures are better able to assess what public purposes should be advanced by an exercise of eminent domain''.\footcite[244]{midkiff84}

Effectively, what had been a doctrine of deference towards state courts had now transformed into a doctrine of deference towards state legislatures (and, in practice, the executive branch). In light of this, it had to be expected that {\it Kelo} would be decided in favour of the taker.\footnote{In fact, as pointed out by Somin, the {\it Kelo} case represents a slight tightening of the earlier line on public use. See \cite{somin07}.} However, the history of the public use requirement tells us that this outcome was by no means inevitable. Hence, the question arises whether legitimacy can be increased by reviving the public use test. The next section sheds some light on this, on the basis of legislative developments in the US after {\it Kelo}.

\subsection{Economic Development Takings after {\it Kelo}}\label{sec:3:3:2}

Following {\it Kelo}, much attention was directed at the danger of eminent domain abuse in the US.\footnote{See generally \cite{somin09}.} Moreover, the {\it Kelo} decision itself proved extremely unpopular. Surveys show that as many as 80-90 \% believe that it was wrongly decided, an opinion widely shared also among the political elite.\footcite[2109]{somin09}

Many states responded by introducing reforms aimed at limiting the use of eminent domain for economic development.\footnote{For an overview and critical examination of the myriad of state reforms that have followed {\it Kelo}, I point to \cite{eagle08}. See also \cite{somin09}.} Within two years, 44 states had passed post-{\it Kelo} legislation in an attempt to achieve this.\footnote{See \cite{castle}.} Various legislative techniques were adopted. Some states, including Alabama, Colorado and Michigan, enacted explicit bans on economic development takings and takings that would benefit private parties.\footcite[See][107-108]{eagle08} In South Dakota, the legislature went even further, banning the use of eminent domain: ``(1) For transfer to any private person, nongovernmental entity, or other public-private business entity; or (2) Primarily for enhancement of tax revenue''.\footnote{South Dakota Codified Laws § 11-7-22-1, amended by House Bill 1080, 2006 Leg, Reg Ses (2006).}

In other states, more indirect measures were taken, such as in Florida, where the legislature enacted a rule whereby property taken by the government could not be transferred to a private party until 10 years after the date it was condemned.\footcite[809]{eagle08} Many states also offered lengthy lists of uses that were to count as public, designed to restrict the room for administrative discretion while allowing condemnations for purposes that were regarded as particularly important.\footcite[804]{eagle08}

%As Somin has pointed out, state reforms enacted by the public through referendums tend to be more restrictive than reforms passed through the state legislature.\footcite[2143]{somin09} Many of the more radical reform proposals, moreover, were not endorsed by any of the branches of government, but were initiated by activist groups as ballot measures.\footnote{In some US states, initiative processes make it possible for activist groups to put measures on the ballot without prior approval by the state legislature. See \cite[2148]{somin09}.} As Somin observes, the reforms taking place via this route would be comparatively strict, testifying to the power of direct democracy.\footnote{See \cite[2143-2149]{somin09}.}

{\it Kelo} has clearly had a great effect on the discourse of eminent domain in the US. However, the effects of the many state reforms that have been enacted are less clear. According to Somin, most of these reforms have in fact been ineffective, despite the overwhelming popular and political opposition against economic development takings.\footcite[2170-2171]{somin09} At the same time, property lawyers report a greater feeling of unease regarding the correct way to approach the public use requirement, expressing hope that the Supreme Court will soon revisit the issue.\footnote{See \cite{murakami13} (``Until the Supreme Court revisits the issue, we predict that this question will continue to plague the lower courts, property owners, and condemning authorities'').} 

Why have legislative reforms proved inadequate? Part of the reason, according to Somin, is that people are ``rationally ignorant'' about the economic takings issue.\footnote{See \cite[2170]{somin09}.} For most people, it is unlikely that eminent domain will come to concern them personally or that they will be able to influence policy in this area. Hence, it makes little sense for them to devote much time to learn more about it. This, in turn, helps create a situation where experts can develop and sustain a system based on practices that a majority of citizens actually oppose.\footcite[2163-2171]{somin09} To back up this analysis, Somin points out that surveys seem to show that people generally overestimate the effectiveness of eminent domain reform, by mistaking symbolic legislative measures for materially significant changes in the law.\footcite[2155-2157]{somin09}

Arguably, this also shows that the legislative approach so far, which has focused on introducing more elaborate and detailed versions of the public use restriction, need to be supplemented by different kinds of proposals. Specifically, it seems important to also target the structural processes that result in the taking of private land for economic development. After all, it is when we direct attention at the decision-making involved in bringing about actual takings that we will locate those stakeholders who cannot afford to remain rationally ignorant about eminent domain. %These processes, it seems, need to be imbued with greater legitimacy. %In particular, it might be that owners themselves should be granted a better chance to participate in the management of their property, even when this involves deliberating on, and possibly taking part in, large-scale development projects. After all, it is the feeling that owners' and their communities' feeling that they are being treated unfairly that tend to lie at the root of controversies surrounding takings for economic development.\footnote{For a similar perspective, see \cite{underkuffler06}.}

This points towards another perspective on legitimacy, whereby focus shifts towards the institutional setting where the relevant decisions are made. Importantly, cases such as {\it Kelo} suggest that this needs to involve more than administrative law and ideas about procedural due process. Specifically, institutional legitimacy appears to have an important substantive component whereby a decision is legitimate only in so far as it results from democratically legitimate decision-making within an administrative framework that is generally conducive to fair and proportional outcomes. Arguably, recent developments at the ECtHR point towards a perspective on legitimacy that emphasises this interconnectedness between substantive and procedural aspects of fairness at the institutional level.

\section{Recent Developments at the ECtHR: Legitimacy as Institutional Fairness}\label{sec:3:4}

It is often said that the P1(1) of the ECHR consists of three rules. The first rule guarantees a right to `peaceful enjoyment of possessions', the second rule regulates the legitimacy of `deprivation' and the third rule regulates how the states can legitimately `control the use of property'.\footnote{See \cite[61]{sporrong82}.}

When dealing with complaints pertaining to P1(1), the Court in Strasbourg will typically first consider which of these three rules it should apply.\footnote{See \cite[102-104]{allen05}.} However, as noted by Allen, it is not clear that this choice has any great significance for the outcome.\footnote{See \cite[104-105]{allen05}.} In practice, the evaluation of legitimacy proceeds in much the same way regardless of which rule is used, with an emphasis on the {\it fair balance} that needs to be struck when states interfere with private property rights.\footnote{See \cite[[103]{allen05}. It is also typically assumed that an interference is only legitimate when it takes place for an appropriate purpose, but here the ECtHR has consistently maintained a deferential stance, pointing to the `wide margin of appreciation' that the member states enjoy in this regard. See, e.g., \cite[54]{james86}.}

In this regard, it is important to note that the Court has gradually adopted a more active role in assessing whether or not particular instances of interference are proportional and able to strike a fair balance between the interests of the public and the property owners. As argued by Allen, this has caused P1(1) to attain a wider scope than what was originally intended by the signatories.\footcite[1055]{allen10}

In the early case law behind this development, the focus was predominantly on the issue of compensation, with the Court gradually developing the principle that while P1(1) does not entitle owners to full compensation in all cases of interference, the fair balance will likely be upset unless at least some compensation is paid, based on the market value of the property in question.\footnote{See \cite[103]{scordino06}. The case also illustrates that the Court has adopted a fairly strict approach to the question of when it is legitimate to award less than full market value.}

However, the fair balance test encompasses more than the issue of compensation. In particular, the hunting cases discussed in Chapter 1 show that the Court in Strasbourg is willing to reflect broadly on the context and purpose of interference, to critically assess the social function of the taking.\footnote{See Section \ref{sec:hunt} of Chapter 1.} Moreover, institutional aspects of fairness have come to play an important role in the Court's reasoning in some other recent cases involving property.\footnote{See \cite{hutten06,lindheim12}.} This is particularly clearly demonstrated by the case of {\it Hutten-Czapska v Poland}.\footnote{See \cite{hutten06}.}

\subsection{Legitimacy {\it Erga Omnes}}\label{sec:3:4:1}

The striking conclusion in {\it Hutten-Czapska v Poland}, underscoring the institutional turn at the ECtHR, was that the case demonstrated `systemic violation of the right of property'.\footcite[239]{hutten06} The case concerned a house that had been confiscated during the Second World War. After the war, the property was transferred back to the owners, but in the meantime, the ground floor had been assigned to an employee of the local city council.\footcite[20-31]{hutten06} The state implemented strict housing regulations during this time, which eventually led to the applicant's house being placed under direct state management.\footcite[20-31]{hutten06} Following the end of communist rule in 1990, the owners were given back the right to manage their property, but it was still subject to strict regulation that protected the rights of the tenants.\footcite[31-53]{hutten06} In addition to rent control, rules were in place that made it hard to terminate the rental contracts.\footcite[20-53]{hutten06} 

After an in-depth assessment of the relevant parts of Polish law and administrative practice, the Grand Chamber of the ECtHR concluded that there had been a violation of P1(1). Importantly, they did not reach this conclusion by focusing on the owners and the interference that had taken place with respect to their individual entitlements. Rather, they focused on the overall character of the Polish system for rent control and housing regulation, as exemplified by the applicant's situation.

Specifically, the consequences for the owners were considered not in isolation, but in order to shed light on a broader question of sustainability.\footcite[60-61]{hutten06} The Court was particularly concerned with the fact that the total rent that could be charged for the house in question was not sufficient to cover the running maintenance costs.\footcite[224]{hutten06} In particular, it was noted that the consequence of this would be ``inevitable deterioration of the property for lack of adequate investment and modernisation''.\footnote{\cite[224]{hutten06}.}

In the end, the Court highlighted how three factors combined to bring both owners and their properties  to a precarious position. First, the rigid rent control system made it hard to sustainably manage rental property. Second, tenancy regulation made it hard for owners to terminate tenancy agreements. Third, the Court noted that the state itself had set up many tenancy agreements during the days of direct state management, shedding doubt on the fairness of the obligations that these contracts imposed on owners.\footcite[224-225]{hutten06} 

The Court's reasoning in {\it Hutten-Czapska} is also interesting because of how the `social rights' of the tenants is placed on an equal footing to the property rights of the owners.\footcite[225]{hutten06} Arguably, property rights and social rights are not considered merely as separate sets of entitlements, locked in opposition to one another. In the reasoning of the Court they also appear as mutually dependent social functions, both hampered by an unsustainable approach to property and housing during the communist era and beyond.\footnote{Specifically, the Court attached great significance to the finding that rents were too low to cover maintenance costs, see \cite[224]{hutten06}. A lack of incentives for maintenance is clearly a threat to tenants as much as to owners, illustrating the interdependence between the two groups. Despite this, when summing up their reasoning in broad strokes, the Court itself reverts back to a traditional narrative when it speaks about the `conflicting interests of landlords and tenants'. See \cite[225]{hutten06}.}

In this regard, the Court places considerable weight on the precarious situation of the owners and `the absence of any legal ways and means making it possible for them either to offset or mitigate the losses incurred in connection with the maintenance of property or to have the necessary repairs subsidised by the State in justified cases'.\footnote{See \cite[224]{hutten06}.} Moreover, the Court commented that the `burden cannot, as in the present case, be placed on one particular social group, however important the interests of the other group or the community as a whole'.\footnote{See \cite[225]{hutten06}.} Importantly, however, the Court did not set out to censor the political reasoning that motivated the rent control scheme, but rather focused on the fact that it had not been implemented properly.\footnote{See \cite[224]{hutten06}.}

On this basis, the Court concluded that there had been a systemic violation of P1(1), and ordered Poland to take measures to rectify the `malfunctioning of Polish housing regulation'.\footnote{See \cite[237]{hutten06}. The basis relied on for formulating such an order was Article 41 in the ECHR, first used in this way in the case of \cite{broniowski05}.} Hence, the lack of legitimacy was pronounced with a kind of {\it erga omnes} effect, establishing an obligation for Poland directed at all its citizens in equal measure, not merely the applicant.\footnote{There was some dissent as to whether or not this was an appropriate response, with Judge Zagrebelsky in particular arguing against it on the grounds that it would see the Court ``entering territory belonging specifically to the realm of politics''. See \cite{hutten06}.} Judgements of this kind, known as ``pilot judgements'', have now gained formal recognition as a distinct procedural form that the ECtHR can use to address systemic problems.\footnote{See generally \cite{leach10}.}

The institutional approach conditioned by the introduction of pilot judgements might point to the core function that the ECtHR is likely to serve in the future.\footnote{See, e.g., \cite{greer12} (arguing that a ``constitutional pluralism'' approach to adjudication -- better filtered, more principled, yet still context sensitive --  is the way ahead for the ECtHR).} Indeed, the ECtHR will hardly be able to protect human rights in Europe on a case-by-case basis. Nor would it seem appropriate for it to do so, given its remoteness to local conditions and its relative lack of democratic accountability. However, when the Court is able to identify systemic failures that look set to systematically give rise to imbalances and unfairness, it seems appropriate that it should take action.

This is particularly clear when, as in the case of {\it Hutten-Czapska}, the Court notes that the applicants have insufficient options available for achieving a fair balance by appealing to institutions within the domestic legal order. In such cases, it seems appropriate for the Court to demand a change at the level of the state's own institutions, giving rise to a broad duty for the state to improve those institutions. Moreover, by scrutinizing the procedures and principles that the states apply when fulfilling this duty, it is likely that the Court will still be able to steer and unify the development of the case law on human rights, at least to the extent that this is required to meet minimum standards.

Against the deferential implications of this shift of attention, it could be argued that the judicial or administrative bodies of the signatory states can easily circumvent their obligations by providing superficial reforms or biased assessments of the facts in human rights cases, to avoid embarrassment for the state's political or bureaucratic elites. However, this might then be raised as a more procedurally oriented complaint before the ECtHR, perhaps also against Articles 6 (fair trial) and 13 (effective remedy).\footnote{I note that this also fits with recent developments at the ECtHR, toward somewhat broader scrutiny under Article 6, see \cite{khamidov07}.}  

In this way, the Court can streamline its functions, by always aiming to direct attention at issues that arise at a higher level of abstraction.\footnote{A similar argument was given by Judge Zupan\u{c}i\u{c} in \cite{hutten06} (``Is it better for Poland to be condemned in this Court 80,000 times and to pay all the costs and expenses incurred in 80,000 cases, or is it better to say to the country concerned: “Look, you have a serious problem on your hands and we would prefer you to resolve it at home...! If it helps, these are what we think you should take into account as the minimum standards in resolving this problem...”? Which one of the two solutions is more respectful of national sovereignty?'').} This, in my view, seems highly desirable. The ECtHR should not aim to micromanage the signatory states, particularly not in relation to a norm such a P1(1), which the Court itself regards as highly dependent on context. By shifting attention towards institutional fairness, the Court can avoid getting stuck in deference to the states without overstepping its bounds with regards to the democratic process.

Indeed, the case of {\it Hutten-Czapska} is highly suggestive of the merits of such a perspective, not only because of the special measures ordered, but also because the Court reasoned on the basis of institutional information to identify systemic weaknesses of Polish housing regulation.\footnote{Specifically, it seems that the shift signalled by recent cases on property at the ECtHR does not end with a new take on remedies, but also signals some changes in the way the Court approaches the fair balance determination under P1(1). This seems natural; if the Court looks for systemic violations, not (only) individual transgressions, its substantive assessments of fairness will naturally be influenced.} Another example is the recent case of {\it Lindheim and others v Norway}.\footnote{See \cite{lindheim12}.} Here the applicants complained that their rights had been violated by a Norwegian act that gave lessees the right to demand indefinite extensions of ground leases on pre-existing conditions.\footcite[119]{lindheim12}

The Court agreed that this was a breach of P1(1). Moreover, it engaged in the same form of assessment as it had adopted in {\it Hutten-Czapska}. Specifically, it concluded that the Ground Lease Act 1996 as such was the underlying source of the violation. The problem was not merely that this act had been applied in a way that offended the rights of the applicants. In light of this, the Court did not only award compensation, it also ordered that general measures had to be taken by the Norwegian state to address the structural shortcomings that had been identified.\footnote{See \cite{lindheim12}.}

The Court also commented that its decision should be regarded in light of ``jurisprudential developments in the direction of a stronger protection under Article 1 of Protocol No. 1''.\footcite[135]{lindheim12} However, in light of the change in perspective that accompanies this development, it is interesting to ask in what sense exactly the protection is stronger. In particular, it is not {\it prima facie} clear that the Court's remark should be read as a statement expressing a change in its understanding of the content of individual rights under P1(1). 

Rather, it may be read as a statement to the effect that the Court has assumed greater authority to address structural problems under that provision. This might even allow the Court to conclude that a violation has occurred due to structural unfairness, even when it is not possible to trace this back to any abnormal decision that specifically targets the individual entitlements of applicants.

If this is true, it could make a big difference in cases involving takings for economic development. As illustrated by Justice O'Connor's dissent in {\it Kelo}, a main concern here is that such takings are likely to have ``perverse'' consequences at the structural level, because they lack democratic merit. \footnote{To quote Justice O'Connor's dissent in {\it Kelo}, see \cite{kelo05}.} In light of cases such as {\it Hutten-Czapska} and {\it Lindheim}, I think the ECtHR would have been likely to approach {\it Kelo} in a manner consistent with Justice O'Connor's approach.

Whether they would reach the same conclusion seems more uncertain, particularly since confidence in the states' ability and willingness to regulate private-public partnerships might be higher in Europe than in the US.\footnote{For a discussion from the point of view of English law, arguing that the prevailing regulatory regime limits the risk of eminent domain abuse largely through regulation of the takings power rather than strict property protection, see \cite{allen08}.} However, it seems unlikely that the ECtHR would follow the majority in {\it Kelo}, by simply deferring to the determinations made by the granting authority. Rather, Justice O'Connor's predictions about the fallout of the {\it Kelo} decision would likely have been of significant interest also to the justices at the Court in Strasbourg.

To conclude, I think notions of institutional fairness can help us locate a welcome middle ground between largely procedural notions of justiciable legitimacy, such as those found in England and Wales, and substantive notions, such as those found in the US. The question remains how Courts adopting such a middle ground should proceed when presented with a concrete case of alleged eminent domain abuse. In the next section, I present a possible heuristic.

\section{The Gray Test}\label{sec:3:5}

Pointing to early US case law on public use as a ``laboratory of elementary proprietary ideas'', Kevin Gray builds on the evidence found there to provide a set of conditions for recognising what he calls ``predatory takings''. \footnote{See \cite[28-30]{gray11}.} His conditions capture key aspects of eminent domain abuse that I believe should be recognised by a theory of economic development takings inspired by the notion of human flourishing and an institutional perspective on legitimacy. Below, I briefly present the criteria proposed by Gray, as well as three riders that I believe suggest themselves on the basis of the discussions presented earlier in this and the previous chapter. I will refer to the resulting set of conditions as the {\it Grey test}, to be understood as a proposed general heuristic for assessing the legitimacy of takings, especially in situations when there are strong commercial interests present on the taker side.

Several combinations of conditions might be sufficient to justify designating a taking as eminent domain abuse. The purpose of the Gray test is not to produce a definite set of such conditions that provide a final answer in any case. Rather, the aim is to provide a heuristic to facilitate concrete assessment against the social, economic and political circumstances surrounding the taking in question. If an economic development taking represents an abuse of power, one would expect it to run afoul with regard to some, and probably several, of the criteria set out in the following points.

\subsubsection*{Balance of Power among the Parties}

In a typical case of eminent domain abuse, the parties that stand to benefit will be more economically and politically powerful than those from whom property is taken.\footnote{See \cite[30-31]{gray11}. Gray himself omits any explicit mention of political power, but it is present in Justice O'Connor's dissent in {\it Kelo}, and in my view clearly belongs here.} This can be reflected in the takers' ability to solicit legal assistance and other services to defend the taking, as well as in the  owners' inability to launch a coordinated defence.\footnote{See \cite[30-31]{gray11}.} If there is an imbalance of power, this is particularly likely to be noticeable early on, during the planning stages, before the decision to condemn has actually been made. 

After the decision has been made, the procedural position of the owners might improve. However, this might not serve to restore any meaningful balance between the parties; when special procedural protections kick in, it will often be too late for the owners to launch an effective defence against the taking. For instance, strict rules concerning cost reimbursement for costs incurred {\it after} the decision to take has already been made, is not a sufficient response to an imbalance of power, especially not in legal systems that do not offer extensive judicial review of takings purposes.

More generally, a possible imbalance of power should be assessed against the decision-making process as a whole, going back to the first initiative made for taking the property in question. A critical  assessment of what role the owners have played in the decision-making process is a good way to uncover more information about imbalances of power, and whether or not such imbalances could have unduly influenced the outcome.

\subsubsection*{The Net Effect on the Parties}

As Gray notes, a hallmark of eminent domain abuse is that the net effect of the taking is a ``significant transfer of valued resource from one set of owners to another''\footnote{See \cite[31]{gray11}.} In itself, this is not a conclusive sign of abuse, but it directs us to ask two important questions. First, we should inquire critically into the main purpose of the taking. Is the transfer of resources between the parties an acknowledged motive or an unacknowledged side-effect of some ostensibly distinct public purpose?

In the latter case, it might be clear that the public purpose is only a pretext for benefiting the taker, in which case it counts as clear evidence of abuse. In less obvious cases, if the transfer of resources arising from the fulfilment of the public purpose was not properly discussed and critically examined by the decision-maker, this too can point towards predation.

The assessment will be different if redistribution of (control over) resources is openly acknowledged as part of the rationale justifying eminent domain. In such cases, it is pertinent to ask further  questions about the economic and social status of the parties, and the structure of the decision-making process, to shed light on whether the redistributive motive itself appears democratically legitimate. If there is eminent domain abuse, one would except the taking to fail to stand up to scrutiny in this regard.

In some cases, it might be debatable whether a taking passes the net effect test. However, the importance of scrutiny is still significant, since it helps bring the crucial questions into the open, thereby ensuring higher quality of the decision-making regarding the taking. Indeed, if the Gray test is applied at an early stage of the proceedings, this in itself might help increase acceptance of the decisions reached. Making room for more extensive legitimacy tests in takings law might well end up bolstering the government's power to take property, as long as the power is used faithfully.

\subsubsection*{Initiative}

In many suspicious economic development takings, the party benefiting commercially from the taking is the party that initially made the suggestion for using eminent domain.\footnote{See \cite[32]{gray11}.} In uncontroversial cases, on the other hand, the initiative tends to come from some government body that seeks to pursue a specific policy goal, e.g., to provide a public service or bestow a benefit on a particular group that is found to be in need of support. The contrast between this and cases when the initiative lies with the commercial beneficiaries themselves point to a disturbance of the decision-making underlying the decision to use eminent domain. As such, it is an important hallmark of abuse.

To investigate further under this point, one should take into account the wider social and political context of the taking, particularly the position of the parties involved. If the beneficiary is both more powerful and privileged than the owners {\it and} takes the initiative for the taking, this is clearly a sign pointing towards predation. On the other hand, if the beneficiaries are marginalised groups who could only expect any consideration if they were to take the initiate themselves, the situation might have to be viewed differently. %In these cases, the fact that the system leaves room for marginalised non-owners to acquire property interests might have to be considered a strength rather than a weakness. Still, as discussed in later points, the appropriateness of using eminent domain for redistributive purposes can be questioned, even if the redistributive goal itself appears democratically legitimate. In such cases, however, the question of legitimacy is unlikely to turn on the initiative test.

\subsubsection*{Location}

The location, in a broad sense of the word, of the property that is taken, can be a strong indicator that eminent domain is inappropriate.\footnote{See \cite[33-34]{gray11}.} For instance, cases involving the taking of dwellings are naturally more suspect than cases involving the taking of barren or unused plots of land. Similarly, the taking of property that is important to the subsistence of the current owner should raise the bar for when a taking may be considered legitimate. Moreover, if the taker's choice of location appears to be one of convenience rather than necessity, this points towards predation. It is particularly telling if alternative locations would be less intrusive, or obviate the need for using eminent domain altogether.

%The location of the property can also attain relevance independently of the current owner. For instance,

On the other hand, the location of the property can sometimes point towards {\it increased} legitimacy of a taking that would otherwise appear suspect. This might be the case, for instance, if the property that is taken has special value to the taker or the community specifically because of its strategic importance with respect to the taker's own property or the rights of non-owners.\footnote{For instance, if riparian owners cannot make rational use of the water flowing over their land without intruding on the land of their neighbours, using eminent domain to resolve this might be considerably less suspect than other kinds of economic development takings. This particular scenario was much discussed in the US during the 19th century, in relation to mill acts which authorised neighbour-to-neighbour takings of limited property rights needed for development. See the discussion in Chapter 2, Section \ref{sec:us}. For an example involving non-owners, consider the increased legitimacy of interference in cases when property rights frustrate efforts to secure rights of non-owners, such as rights to drinking water in cases when riparian owners prevent non-owners access to water for their basic needs.} The proper balance of burdens and benefits might still be upset, but a taking that fits smoothly into a `special value' narrative will be less suspect than one that does not.

\subsubsection*{Social Merit}

As Gray notes, a taking that is hard to justify on the basis of its social merits is more likely to be predatory.\footnote{See \cite[34]{gray11}. Gray writes of lap-dancing clubs and cigarette factories as examples of purposes that are suspect. Importantly, such purposes might well fulfil a public interest requirement via the economic development narrative, yet still fail a social merit test that focuses rather on the social dimensions of the use to which the property will be put.} This asks for closer scrutiny of the kinds of public interests that can be used to justify a taking. If the justification narrative surrounding a taking revolves solely around `trickle-down' effects and the successful business ventures that the taking will facilitate, there is reason to be suspicious. Specifically, if the taking cannot sustain a social merit narrative, whereby attention is shifted away from purely economic considerations, this is a strong independent indication that the taking might count as predation.

The point here is not that the language of social merit should replace the language of public use or public interest as some kind of conclusive test of legitimacy. Rather, the point is that one should always be encouraged to analyse takings specifically in terms of non-economic, social, effects. This is particularly important in difficult cases, because it can help us arrive at a better understanding of where exactly the taking sits on the gray scale between admissible governance and predatory exploitation. 

%If a taking appears to stand up to scrutiny only when embedded in a purely economic narrative, this in itself suggests a lack of legitimacy. Indeed, even if one concedes that incidental economic effects are relevant, it seems clear that however one circumscribes a notion such as public use, this notion certainly encompass {\it more} than merely those incidental economic effects that tend to occupy center stage in legitimacy disputes.\footnote{Indeed, it bears emphasising that those arguing against economic development takings might achieve more by emphasising non-economic aspects, compared to arguing that incidental economic benefits should not at all be considered relevant as a justification for eminent domain.}

\subsubsection*{Environmental Impact}

According to Gray, a typical feature of eminent domain abuse is that it has an adverse environmental impact. Moreover, a typical feature of eminent domain abusers is that they show disregard for such adverse affects.\footnote{See \cite[34]{gray11} (``predatory takers tend to be relatively unperturbed if they lay waste to the earth'').} This is an additional element that pertains specifically to the status of the taker, asking us to consider whether it is appropriate to grant their activities public interest status. It is not primarily a question of how the development stands with regard to environmental regulation. Rather, what is at stake is whether or not the characteristics of the taker and the development plans make it appropriate to use the power of eminent domain. 

It might be appropriate to use environmental law as a starting point, but the relevant environmental standard with regard to the legitimacy question should be drawn up more strictly than the standards generally applied to the type of development in question. Indeed, one should be entitled to expect {\it more} in terms of environmental awareness and concern from a developer and a development plan that benefit from the power of eminent domain. 

Arguably, the mere fact that takers engage in active lobbying for leniency in relation to environmental standards can be enough to shed doubt on the proposition that they act in the public interest. What might otherwise be considered natural and admissible behaviour for a common commercial company can be improper or inadmissible behaviour for one that benefits from eminent domain powers. I note that this particular observation has general import, pertaining to a potentially wider set of obligations that takers may be expected to take on, not only environmental ones. This brings me to the first rider that I propose to add to Gray's original evaluation points.

\subsubsection*{Rider 1: Regulatory Effects}

As discussed in Chapter 2, property has an important regulatory effect, also outside the realm of positive law. This effect typically changes following a taking, sometimes quite dramatically.  For instance, if locally owned property is taken by external commercial actors for high-intensity commercial use, the post-taking regulatory status of the property will most likely be completely different to its status prior to the interference. Moreover, the changed status might have as much to do with informal social functions as it has to do with positive regulation.

It might be, for instance, that the property in question is found in a jurisdiction that emphasises  the freedom of owners to do as they please without state interference. In this case, the fallout of allowing external commercial actors to take locally owned property can be particularly severe, as the new owner is likely to be unconstrained by locally grounded systems for sustainable resource management. In these cases, there is a risk that there will be a `tragedy of the taken', arising from how the taking undermines an important building block of sustainability. 

Indeed, a society based on egalitarianism and strict limits on state interference might find it especially difficult to appropriately restrain the actions of actors who use the eminent domain power to accumulate property for high-intensity use.\footnote{This problem can of course arise independently of the use of eminent domain, e.g., in the context of land grabs arising from voluntary or semi-voluntary transactions. However, the situation appears particularly problematic if the state itself is complicit in bringing about the problem, by undermining property's social function through the use of the takings power.} If this is resolved by increasing the state's power to interfere with private property through regulation, the effect can be a further undermining of local management frameworks, increased subsequent use of eminent domain, and a general spreading and amplification of the democratic deficit already inherent in the original act of taking.

A different regulatory concern is that the legal status of the property can change, for instance because the development in question brings it under the scope of different rules. If so, it should be examined whether the new rules offer weaker protection for the local community, the environment, or the general public interest, in which case it reflects badly on the initial decision to use eminent domain.%\footnote{The case study of Norwegian waterfall expropriation will offer an example of this mechanism, c.f., Chapter 5, Section \ref{sec:x}.}

\subsubsection*{Rider 2: Impact on Non-Owners}

Following up on the theoretical arguments made in Chapter 1, it is appropriate to direct special attention at the status of non-owners directly affected by economic development takings. It is of particular interest to ascertain whether or not the interests of such non-owners were given due consideration prior to the decision to use eminent domain. If their interests appear to have been neglected, or have not been considered at all, there is additional reason to be sceptical of the purported public interest of the taking. Indeed, just as disregard for the environment is a typical sign of predation, a general disregard for local non-owners is also an indicator of abuse.

To shed further light on this, one might first ask what role non-owners played in the decision-making process. If the non-owners directly affected by the taking were allowed to express their opinion, and enjoyed some measure of influence, this can enhance legitimacy. If, on the other hand, the most immediately affected members of the public were not consulted, or not given a proper voice in the proceedings, it indicates abuse.

There is also an important substantive aspect to consider: how is the taking going to affect property dependants without recognised ownership rights? If it is clear that they will suffer severe adverse effects, for instance by being displaced from their homes or by loosing their livelihoods, this must be counted as an indication of predation irrespective of any mitigating procedural arrangements used to create the impression of ``consultation'' or the like.

Importantly, it also follows from the social function perspective that awarding compensation can not by itself excuse shortcomings in this regard. If people are displaced, for instance, the fact that new dwellings are provided somewhere else does not detract from the fact that a community has been destroyed. It is possible that the needs of the public necessitate such a drastic interference with property's proper function, but this should then at once give rise to a more in-depth scrutiny of legitimacy. Moreover, the bar to pass the legitimacy test should be raised considerably in such cases.

\subsubsection*{Rider 3: Democratic Merit}

Perhaps the most important characteristic to consider when assessing the legitimacy of a taking is its democratic merit. In an important sense, putting a taking to the test against this measure serves to encapsulate all the other points raised above. Specifically, it asks us to consider the totality of these factors in order to judge whether good governance standards have been observed within a system based on democratic decision-making. The inquiry made in this regard should not be focused on second-guessing government policies, but should compel us to take seriously the idea that a commitment to democracy places real constraints on the exercise of government power. In this way, an overarching focus on democratic merit can hopefully render the principles of scrutiny expressed by the Gray test as a possible template for courts across different jurisdictions, with respect to both constitutional and human rights provisions.

The overarching question that arises with respect to democratic merit is whether the taking in question 
can be said to arise from a legitimate process of decision-making, in the pursuit of a fair and equitable outcome. It bears emphasising that in line with a more modern appreciation of the meaning of democracy and human rights, the relevant assessment under this point involves both procedural and substantive elements. Fairness in itself is a constraint on the democratic process, particularly when fundamental economic and social rights are involved. At the same time, the notion of democratic merit rightly brings procedural questions to the foreground. Indeed, it might be a weakness of Gray's original proposal that it does not single out procedural issues for special consideration.

On the one hand, it is inappropriate to reduce the takings question to a matter of administrative law. But on the other hand, the way in which the taking decision was made can often tell us much about its legitimacy, including how it stands with regard to broader notions of fairness. It is particularly important, in this regard, to inquire into the position of local owners and communities during the planning process leading up to the decision to use eminent domain. In the context of property as a human right, moreover, a stricter standard might be appropriate here, compared to that which would otherwise follow from administrative law.

Importantly, our commitment to property as a social institution requires us to take into account that the owners generally make up the group of people who will be most directly affected by any decision involving the future of their property. As such, they should normally be granted a decisive voice in decision-making processes leading up to economic development. At the same time, the social function account leaves room for recognising that this presumption in favour of emphasising the rights of owners can be defeated by the context. It is clear, for instance, that the substantive interests of absentee landlords might be limited compared to the substantive interests of local non-owners who depend more directly on the property for their livelihoods. In these cases, the social function approach allows us to recognise that a taking might have significant democratic merit, even if it is based on a form of decision-making that prioritises the interests and participation rights of non-owners.

Nevertheless, within a system based on private property rights it will always be appropriate to show caution in this regard. The presumption should always be, within such a system, that the owners are the primary stakeholders in decision-making processes involving their property. Moreover, if this presumption is defeated, it would usually point to a structural weakness of property's function within society, a weakness that should arguably be addressed by more general reforms of property, not by inflating the state's power to undermine it. If caution is not observed here, property can soon become a less secure basis on which to support local communities, including those marginalised groups that are most in need of protection from predators.

By itself, however, a legitimacy test cannot make property a more secure basis for promoting good outcomes in cases when the public desires economic development. What the Gray test provides is a list of possible symptoms to look our for when attempting to diagnose a suspected case of eminent domain abuse. Hopefully, this can help flag problems and limit damages, but it can not be regarded as a solution, especially not in cases when the public's apparent desire for economic development is a genuine reflection of a democratic commitment. 

In short, after diagnosing a lack of legitimacy, the question becomes how to find a cure, preferably without harming the patient, i.e., the democratic system. In the next section, I consider this challenge in more depth, premised on the idea that there is a need for alternatives to eminent domain in cases when the collective wishes to take decisive steps to promote economic development on privately owned land.

\section{Alternatives to Takings for Economic Development}\label{sec:3:6}

As mentioned briefly in Chapter 2, the work of Ostrom and others on common pool resources suggests that sustainable resource management can often be better achieved through local self-governance than through markets or states.\footnote{See generally \cite{ostrom90}. For a recent exposition of the main ideas, placing the work in a broader academic context, see the revised version of Ostrom's Nobel Lecture, \cite{ostrom10}.} The connection between this work and property theory is highly interesting, and has been explored in some recent work, particularly by US legal scholars.\footnote{See generally \cite{rose11,fennel11}.} As these scholars have observed, the connection can be made at a very high level of generality. Indeed, in a democracy, property as such has a kind of (partial) commons structure, since property as an institution depends on the collective choices we make regarding the legal order.\footnote{For similar observations, see \cite[51]{rose90}; \cite[577]{heller01}.} In cases when property is made subject to eminent domain, this perspective becomes particularly salient, since then the collective explicitly withdraws its backing for the rights of the owner, in favour of collective decision-making about the future of the property in question.\footnote{A common pool resource is typically identified by the fact that exclusion is difficult or costly, while use can cause depletion (and hence should be limited), see, e.g., \cite[57]{ostrom10b}. Hence, the mere fact that some property is apt to be regulated, or taken, by the collective, demonstrates that property has common pool characteristics (although these might be imposed by the polity, rather than arising from the nature of the underlying good).} Moreover, in case of an economic development taking, the property in question typically pertains to land or some other natural resource, which invariably form part of a larger resource system with some common pool characteristics.\footnote{See, e.g., \cite[16]{fennel11} (``we are {\it always} operating at least partially within a commons of some sort''). I also mention Smith's notion of a ``semicommons'', used to describe settings where common pool arrangements for resource management interact with individual property rights, see generally \cite{smith00,smith02}.}

Importantly, to designate something as a common pool resource does not in any way imply that the resource in question is open-access or that it is held as a form of common property, a public trust, or under some other legal construction moving away from the sphere of private property.\footnote{See, e.g., \cite[58]{ostrom10b}.} Perhaps more controversially, designating something as a common pool resource does not in any way imply that the resource {\it should} be removed from this sphere.\footnote{See \cite[58]{ostrom10b} (``there is no automatic association of common-pool resources with common-property regimes -- or, with any other particular type of property regime'').} According to Ostrom and Hess, the appropriate property regime for a given common pool resource is a pragmatic question that depends on the circumstances.\footnote{See \cite[58]{ostrom10b}.}

It should be noted, however, that this neutral position on the relationship between property and common pool resources is premised on a bundle of rights understanding.\footnote{See \cite[59]{ostrom10b}.} Potentially, a more ambitious theory of property could suggest a different perspective. Specifically, the question arises as to how theories of common pool resource management relates to the social function account. This is a particularly interesting avenue for future work, as it could shed light on the normative stance that private property can be a good basis for sustainable self-governance, at least when backed up by a human flourishing account of what private property should be.

In this thesis, I will limit myself to noting how the link between property and theories of commons governance provides a possible route towards an institutional perspective on how to solve legitimacy problems associated with economic development takings.

To make progress in this regard, it will be useful to first briefly consider one of the most important theoretical legacies of Ostrom's work, namely a list of eight design principles that she formulated on the basis of empirical studies.\footnote{See \cite[90]{ostrom90}.} These principles were formulated because they seemed to be particularly crucial in ensuring good governance at the local level, and have since been supported by a growing body of empirical evidence.\footnote{See \cite{cox10} (the authors also suggest splitting some of the original principles in two parts, resulting in a slightly more fine-grained list, not needed in this thesis). } In brief, the so-called CPR principles are the following:

\begin{enumerate}
\item {\bf Well-defined boundaries:} There should be a clearly defined boundary around the resource in question, and a clear distinction should exist between members of the user community, who are entitled to access the resource, and non-members, who may be excluded. This will internalise the costs of resource exploitation and other externalities, ensuring that proper incentives for sustainable management arise within the community of resource users.\footnote{Importantly, the possibility of excluding non-members marks a distinction between open-access resources and common pool resources, where the latter appears much less susceptible to a commons tragedy than the former, because externalities are internalised to a clearly defined community. See \cite[91-92]{ostrom90}.}
\item {\bf Congruence between appropriation and provision rules and local conditions:} Management principles should be flexible and responsive to changing local conditions. Moreover, management practices should be anchored in the economic, social, and cultural practices prevalent at the local level. In addition, the individual benefits should generally exceed the individual costs associated with membership in the community of users, and collectively managed benefits should be distributed fairly among community members.\footnote{See \cite[92]{ostrom90}.}
\item {\bf Collective-choice arrangements:} The individual members of the user community should have an opportunity to participate in decision-making processes regarding the rules that govern the user community and the resource management. In addition to securing fairness and legitimacy, this will enhance the quality of the decision-making, as the users themselves have first-hand knowledge and low-cost access to information about their situation and the state of the resource in question.\footnote{See \cite[93]{ostrom90}.}
\item {\bf Monitoring:} There should be mechanisms in place to ensure that the behaviour of users is monitored for violations of management rules. To increase efficiency, monitoring should be locally organised. Moreover, to ensure local responsiveness and legitimacy, individuals acting as monitors should themselves be members of the user community or in some way answerable to this community.\footnote{See \cite[94-100]{ostrom90}.}
\item {\bf Graduated sanctions:} There should be an effective system in place for penalising violations of user community rules. These penalties should be graduated so that more severe or repeated violations are sanctioned more severely than minor or one-time transgressions.\footnote{See \cite[94-100]{ostrom90}.}
\item {\bf Conflict-resolution mechanisms:} The user community should be endowed with low-cost procedures for conflict resolution. These procedures should be sensitive to local conditions, to ensure local legitimacy.\footnote{See \cite[100-101]{ostrom90}.}
\item {\bf Minimum recognition of rights:} The user community should be protected from interference by external actors, including government agencies. As a minimum, the existence of local institutions and the right to self-governance should be recognised and respected by external government authorities.\footnote{See \cite[101]{ostrom90}.}
\item {\bf Nested enterprises:} There should be vertical integration between local, small-scale, management institutions and larger institutions aimed at protecting and furthering non-local interests. This integration should be based on the minimum recognition of rights mentioned in the previous point. Furthermore, it should provide a template for integrated decision-making about larger scale issues, where local competences are employed incrementally in more general settings, involving also institutions working on behalf of municipalities, regions, states and the international community. Local institutions for resource management should not only be respected by such larger scale structures, they should also feed into larger scale decision-making and be called to respond to greater community needs.\footnote{See \cite[101-102]{ostrom90}.}
\end{enumerate}

There are at least two interesting connections between self-governance principles such as these and the issue of economic development takings, especially as that issue is approached in this thesis, on the basis of the social function theory of property. First, one may observe that when economic development takings appear to lack legitimacy with respect to social functions, this is typically also an indication that the surrounding framework for resource management is not well-designed. In particular, it appears that the Gray test closely tracks many of the design principles proposed by Ostrom.

For instance, consider the balance of power between the owners and beneficiaries of a taking, the first point to consider according to the Gray test. When a taking fails on this point, doubts naturally arise also with regard to the underlying framework for resource management, particularly aspects pertaining to the recognition of local rights, the adequacy of collective-choice arrangements, and the congruence between appropriation, provision and local conditions. If property is taken by powerful actors, chances are that these actors are not representative of local community interests. Moreover, takings characterised by an imbalance of power typically indicate that the government is in fact quite unwilling to recognise the rights of local people, even when these rights are formally recognised as property rights.

By contrast, the situation might be different if it involves a taking that is not suspect according to the Gray test. For instance, if property is taken from absentee landlords and given to local land users in order to facilitate development, this might be an honest attempt at setting up a management framework that complies with CPR principles. In such a case, one would also not expect the balance of power between owners and takers to point towards abuse.

The second link between CPR design and economic development takings is arguably even more interesting. This link becomes apparent as soon as we shift attention away from diagnosing a lack of legitimacy towards coming up with alternative management principles that can restore it. Specifically, work done on local governance of common pool resources point to an {\it alternative} way of approaching the goal of economic development in cases that might otherwise result in the use of eminent domain. 

This has not received much attention in the literature so far. One notable exception, discussed in depth in the following subsection, is the work of Heller, Dagan and Hills.\footnote{The work of Lehavi and Licht also deserves a brief mention, even though it focuses on compensation rather than alternatives to eminent domain. The reason is that this work relies on proposing a novel institution that also touches on issues related to self-governance. In particular, Lehavi and Licht propose that post-taking, collective, price bargaining should be carried out on behalf of owners by a Special Purpose Development Company, in an effort to give them a chance to get their share of the commercial benefit arising from development. See \cite{lehavi07}. For a more in-depth discussion of this proposal, and the compensatory approach to economic development takings more generally, see \cite{dyrkolbotn15}.} Looking at their work will serve to make the abstract discussion above more concrete, and will set the stage for a comparison between their proposal and solutions that can be facilitated by the system of land consolidation presented in Chapter 5.

\subsection{Land Assembly Districts}\label{sec:3:6:1}

In an article from 2001, Heller and Dagan considered the connection between CPR design and overarching (liberal) property values.\footnote{See \cite{heller01}.} From this, they arrived at a proposal for what they call a ``liberal commons'', which adds some design constraints rooted in a desire to protect individual autonomy and minority rights. In particular, they emphasise the value of exit, the opportunity for members of the governance structure to alienate their share in the commons resource (conceived of as a property right).\footnote{See \cite[567-572]{heller01}.} The right of owners to leave the collective is thought of as a safety mechanism, to prevent failing institutions from trapping its members in a state of oppression. This, it is argued, is an important overarching design constraint, described as a ``liberal'' idea, that should complement the other design principles for local management of common pool resources.\footnote{Despite their commitment to protect the right of exit, Heller and Dagan are also aware of the destabilising effect exit can have on an otherwise well-functioning institution. To address this, they discuss additional mechanisms, such as rights of first refusal, that can ensure that exit does not prove too disruptive to the local collective, as long as a sufficient number of members chose to remain. See \cite[596-702]{heller01}.}

In a later article, responding to the {\it Kelo} controversy, Heller and Hills build on the idea of the liberal commons by proposing a novel approach to the takings issue, consisting of a proposal for a new institutional framework that can facilitate land assembly for economic development. The core idea is to introduce {\it Land Assembly Districts} (LADs), institutions that will enable property owners in a specific area to make a collective decision about whether or not to sell the land to a developer or a municipality.\footcite[1469-1470]{heller08} The idea is that while anyone will be able to propose and promote the formation of a LAD, the official planning authorities and the owners themselves must consent before it is formed.\footcite[1488-1489]{heller08} Clearly, some kind of collective action mechanism is required to allow the owners to make such a decision. 

Heller and Hill suggest that voting under the majority rule will be adequate in this regard, at least in most cases.\footnote{See \cite[1496]{heller08}. However, when many of the owners are non-residents who only see their land as an investment, Heller and Hills note that it might be necessary to consider more complicated voting procedures, for instance by requiring separate majorities from different groups of owners. See \cite[1523-1524]{heller08}.} How to allocate voting rights in the LAD is given careful consideration, with Heller and Hills opting for the proposal that they should in principle be given to owners in proportion to their share in the land belonging to the LAD.\footnote{See \cite[1492]{heller08}. For a discussion of the constitutional one-person-one-vote principle and a more detailed argument in \isr{favour} of the property-based proposal, see \cite[1503-1507]{heller08}.} Owners can opt out of the LAD, but in this case, eminent domain can be used to transfer the land to the LAD using a conventional eminent domain procedure.\footcite[1496]{heller08}

Heller and Hills envision an important role for governmental planning agencies in approving, overseeing and facilitating the LAD process. Their role will be most important early on, in approving and spelling out the parameters within which the LAD is called to function.\footcite[1489-1491]{heller08} While it is not discussed at any length, the assumption appears to be that the planning authorities will define the scope of the LAD by specifying the nature of the development it can pursue in quite some depth. Hence, despite the overarching goal of self-governance, the power of the planning authority still appears significant under the LAD proposal as it currently stands.

If the owners do not agree to forming a LAD, or if they refuse to sell to any developer, Heller and Hills suggest that the government should be precluded from using eminent domain to assemble the land.\footcite[1491]{heller08} This is a crucial aspect of their proposal that sets the suggestion apart from other proposals for institutional reform that have appeared after {\it Kelo}. A LAD will not only ensure that the owners get to bargain with the developers over compensation, it will also give them an opportunity to refuse any development to go ahead. Hence, the proposal shifts the balance of power in economic development cases, giving owners a greater role also in preparing the decision whether or not to develop, and on what terms. Hence, the LAD proposal promises to address the democratic deficit of economic development takings, without failing to \isr{recognise} that the danger of holdouts is real and that institutions are needed to avoid it.

There are some problems with the model, however. First, it seems that planning authorities might have an incentive to refuse granting approval for LAD formation. After all, doing so entails that they give up the power of eminent domain for the land in question. For this reason, Heller and Hills propose that a procedure of judicial review should exist whereby a decision to deny approval for LAD formation can be scrutinized.\footcite[1490]{heller08} However, the question then arises as to how deferential  courts should be in this regard, echoing the conundrum that engulfs the safeguard intended by the public use restriction. Presumably, one would want the courts to strictly scrutinise LAD rejections, to instil that LADs should normally be promoted. However, would the courts be comfortable providing such scrutiny, also against a government body claiming that the ``public interest'' speaks against LAD formation? This would likely depend on the exact formulation and spirit of the LAD-enabling legislation. To work as intended, some sort of presumption in favour of LAD approval appears to be in order, but this in turn can have the effect of making it easier for powerful landowners to abuse the LAD system, e.g., by pushing through LADs that enable them to impose their will on other community members.

This worry is related to a second possible objection against the LAD proposal, concerning the practicalities of the process leading up to the LAD's decision on whether or not to accept a given offer. Is it possible to organise such a process in a manner that is at once efficient, inclusive and informative, without making it too costly and time consuming? Here Heller and Hills envision a system of public hearings, possibly \isr{organised} by the planning authorities, where potential developers meet with owners and other interested parties to discuss plans for development.\footnote{See \cite[1490-1491]{heller08}. It might also be necessary for the planning authorities or other government agencies to take on some responsibilities with respect to providing guidance and assistance to less resourceful members among the owners.} The process envisioned here would resemble existing planning procedures to such an extent that additional costs could hopefully be kept at a minimum. 

The significant difference would concern the relative influence of the different actors, with the   owners as a group receiving a considerable boost as a result of the LAD. Rather than being sidelined by a narrative that sees the use of eminent domain as the culmination of planning, the owners are now likely to occupy center stage throughout, as they now will have the final say on whether or not the development will go ahead.

From this, however, arises the question of how the interests of other locals, without property rights, will be protected. Heller and Hills assumes that local non-owners will also be represented during the stages leading up to the LAD's final decision, but their role in the process is not clarified in any detail.\footcite[1490-1491]{heller08} This raises the worry that LADs might undermine local democracy by giving property owners a privileged position with respect to policy questions that should be decided jointly by all members of the community. The severity of this risk depends heavily on the circumstances. In a context of egalitarian property ownership and sensible government regulation of land uses and LAD operations, the risk should be minimal. In principle, the local anchoring that LADs provide should also benefit non-owners, by bringing the decision-making process closer and making it more easily accessible for the people most directly affected, including non-owners. Moreover, if some members of the local community remain marginalised, this should possibly be regarded as a regulatory failure or a reflection of underlying inequality in society, not a shortcoming of the LAD proposal as such. In these cases, a reasonable approach might even be to {\it expand} the function of LADs, by granting voting rights to a larger class of local property dependants, not only formally titled owners.\footnote{The important invariant to maintain, I believe, is that the locally anchored institution should be the active, invested, agent, while more centralised and/or expert-dominated government bodies should act as passive, impartial, regulators. In the processes leading to economic development takings, this equation is typically reversed, with government bodies and commercial companies being the active agents, while the owners and the local community are the passive agents whose property rights and dependencies place some nominal limits on the authority of other parties (limits which, due to the weakness of owners as a group, tend to be easily disregarded).}

However, the LAD proposal raises some highly problematic issues pertaining to the proposed mechanism of collective decision-making. As Kelly points out in a commentary, the basic mechanism of majority voting is deeply flawed.\footcite{kelly09} For instance, if different owners value their property differently, majority voting will tend to \isr{disfavour} those with the most extreme viewpoints, either in \isr{favour} of, or against, assembly. If these viewpoints are assumed to be non-strategic and genuine reflections of the welfare associated with the land, the result can be inefficiency. In short, the problem is that a majority can often be found that does not take due account of minority interests. 

For instance, if a minority of owners are planning development on their own land, and this conflicts with some LAD proposal targeting a larger area, the minority might find it difficult to defend themselves against the force of the LAD. Indeed, such a minority might effectively loose the battle for their property as soon as a LAD is formed, if the development description underlying LAD formation is incompatible with the kind of development they wish to pursue. For such owners, a presumption in favour of LAD formation might prove highly disadvantageous.\footnote{Of course, one might imagine these landowners opting out of the LAD, or pursuing their own interests independently of it. However, they are then unlikely to be better off than they would be in a no-LAD regime. In fact, it is easy to imagine that they could come to be further \isr{marginalised}, since the existence of the LAD, acting `on behalf of the owners', might detract from any dissenting voices on the owner-side.}
 
Indeed, developers might come to rely on LADs to push through {\it de facto} condemnations of property, through a procedure that leaves minorities less protected than the traditional takings process. Indeed, it would be theoretically possible for any landowner to use a LAD to condemn any neighbouring property smaller than their own. Eventually, a whole community might be taken over by one or a few powerful landowners, through a sequence of appropriately designed LADs and development projects.

Despite these worries, the ideal of the LAD proposal is clearly stated and highly attractive. LADs should help to establish self-governance for land assembly and economic development. In particular, Heller and Hills argue that LADs should have ``broad discretion to choose any proposal to redevelop the \isr{neighbourhood} -- or reject all such proposals''.\footcite[See][1496]{heller08} As they put it, two of the main goals of LAD formation is to ensure ``preservation of the sense of individual autonomy implicit in the right of private property and preservation of the larger community's right to self-government''.\footcite[See][1498]{heller08} The problem is that these ideals turn out to be at odds with some of the concrete rules that Heller and Hills propose, particularly those aiming to ensure good governance of the LAD itself.

In relation to the governance issue, Heller and Hills emphasise, in direct contrast to their comments about ``broad discretion'' and ``self-governance'', that ``LADs exist for a single narrow purpose -- to consider whether to sell a neighborhood''.\footcite[See][1500]{heller08} This is a good thing, according to Heller and Hills, since it provides a safeguard against mismanagement, serving to prevent LADs from becoming battle grounds where different groups attempt to co-opt the community voice to further their own interests. As Heller and Hills puts it, the narrow scope of LADs will ensure that ``all differences of interest based on the constituents' different activities and investments, therefore, merge into the single question: is the price offered by the assembler sufficient to induce the constituents to sell?''.\footcite[1500]{heller08}

This means that there is a significant internal tension in the LAD proposal, between the broad goal of self-governance on the one hand and the fear of \isr{neighbourhood} bickering and majority tyranny on the other. Indeed, it is hard to see how LADs can at once have both a ``narrow purpose'' as well as enjoy ``broad discretion'' to choose between competing proposals for development. If such discretion is granted to LADs, what prevents special interest groups among the landowners from promoting development projects that will be particularly \isr{favourable} to them, rather than to the landowners as a group? What is to prevent landowners from making behind-the-scene deals with \isr{favoured} developers at the expense of their \isr{neighbours}? It might be difficult to come up with rules that prevent mechanisms of this kind, without also making substantive self-governance an impossibility.

If a LAD is obliged to only look at the price, this might prevent abuse. But it will not give owners broad discretion to consider the social functions of property when choosing among development \isr{proposals}. In my view, it is undesirable to restrict the operations of LADs in this way. It is easy to imagine cases where competing proposals, perhaps emerging from within the community of owners themselves, will be made in response to the formation of a LAD. Such proposals may involve novel solutions that are superior to the original development plans, in which case it is hard to see any good reason why they should not be taken into account, even if they are proposed by a minority. Moreover, it is hard to see why they should be disregarded simply because they are less commercially attractive, or because the  developer interested in pursuing such a proposal cannot offer the highest payment to the owners. In the end, the decision that the LAD makes concerns the future of the community as a whole. This is not an exercise in profit-maximization, and there are good reasons to believe that LAD regulation should encourage a broad perspective, not enforce a narrow one.

However, when it comes to the details, Heller and Hills seem quire adamant that the degree of self-governance needs to be limited in favour of strict regulation to reduce the risk of LAD abuse. In particular, they argue that ``LAD-enabling legislation should require especially stringent disclosure requirements and bar any landowner from voting in a LAD if that landowner has any affiliation with the assembler''.\footcite{heller08} Here, the notion of self-governance is made even thinner, as owners will effectively be barred from using LADs as a template for gaining the right to participate in development projects themselves.

Moreover, new questions arise. For one, what is meant by ``affiliation''? Say that a landowner happens to own shares in some of the companies proposing development. Should they then be barred from voting? If so, should they be barred from voting on all proposals, or just those involving companies in which they are a shareholder? If the answer is yes, how can this be justified? Would it not be easy to construe such a rule as discrimination against landowners who happen to own shares in development companies? On the other hand, if the landowner in question is allowed to vote on all other proposals, would it not be natural to suspect that their vote is biased against assembly that would benefit a competing company? Or what about the case when some of the landowners are employed by some of the development companies? Should such owners be barred from voting on proposals that could benefit their employers? This seems quite unfair as a general rule. But in some cases, employment relations could play a decisive factor in determining the outcome of a vote. This might happen, for instance, if an important local employer proposes development in a \isr{neighbourhood} where it has a large number of employees. Heller and Hills give no clear answer to the questions arising in this regard, and at this point, the circle has in some sense closed in on their proposal. Indeed, just as courts today struggle with the ``public use'' requirement, it seems that the proposed ``affiliation'' criterion for depriving someone of their voting rights would provide a very shaky basis for judicial review. 

More generally, it seems that how to best organise a LAD remains an open problem. The challenge is to ensure that LADs deliver a real possibility of self-determination, while also ensuring good governance and protection against abuse. That it remains unclear how to do this is acknowledged by Hiller and Hills themselves, who point out that further work is needed and that only a limited assessment of their proposal can be made in the absence of empirical data.\footnote{See \cite[]{heller08}.} Later in the thesis, I will shed light on this challenge when I consider the Norwegian framework for land consolidation. This framework can be looked at as a sophisticated institutional embedding of many of the central ideas of LADs. In particular, I will discuss how Norwegian land consolidation can be employed in cases of economic development, and how it is increasingly used as an alternative to expropriation in cases of hydropower development. This will allow me to shed further light on the issues that are left open by Heller and Hills' important article.

\section{Conclusion}\label{sec:3:7}

The legitimacy issue is at the heart of this thesis. There are many ways of approaching it, catering to different ideas about the appropriate role that the courts should play in safeguarding private property. This chapter has tried to distil an approach that is particularly suited in cases when property is taken for economic development. 

This led to a proposal for an approach that combines procedural and substantive standards, to arrive at a template for assessing the fairness and democratic merit of the decision-making as such, not merely the outcome. This is appropriate because it helps address a key worry associated with an economic development taking: that the decision to take represents an abuse of power, reflecting badly on the institutions that gave rise to it.
%I argued that such an institutional approach to fairness has started developing at the ECtHR, as a consequence of its development of the pilot judgement framework for assessing cases that might indicate systemic problems at the state institutional level.

On this basis, the chapter went on to provide a possible heuristic for assessing the legitimacy of economic development takings, in a manner that is compatible with an institutional fairness perspective. This heuristic was based on six legitimacy indicators provided by Gray, with three new ones added, based on the work done in this and the previous chapter. The resulting heuristic, the Gray test, should be able to identify cases of eminent domain abuse, particularly those that offend against social functions of property at the institutional level, rather than merely the financial entitlements of owners.

Following up on this, the chapter considered the question of how to increase legitimacy without giving up on the idea that the collective should be entitled to push adamantly for economic development on private land. I argued that the work done by Elinor Ostrom and others on common pool resources provide a suitable starting point for making institutional proposals in this regard. Specifically, I briefly presented her design principles for local self-governance institutions, which I believe can be used as a starting point also for designing procedures to replace eminent domain for economic development.

Finally, I considered a proposal that has already been made along these line, namely the idea of Land Assembly Districts due to Heller and Hills. I analysed this proposals in some detail, pointing out some shortcomings, including a deep tension between the overarching goal of self-governance and the danger of majority and/or elite tyranny. Moreover, I noted that proposals such as these are hard to make at the theoretical level; the appropriate institution for self-governance needs to be attuned to local conditions. Indeed, this is one of the key design principles proposed by Ostrom. 

This observation marks the end of the first part of the thesis. In the next part, I will consider the case of Norwegian hydropower specifically. This will lead to an analysis of legitimacy of takings for this purpose along the lines of the Gray test, as well as a case study of an already operational alternative for self-governance in place of eminent domain. In this way, the second part will aim to shed light on key aspects of the theory developed in the first part, while exploring further the idea that social functions run as a common thread through individual property rights.
%\part{A Case Study of Norwegian Waterfalls}

\chapter{Norwegian Waterfalls and Hydropower}\label{chap:4}

\section{Introduction}\label{sec:4:1}

Norway is country of many mountains, fjords and rivers, where around 95 \% of the annual domestic electricity supply comes from hydropower.\footnote{See \cite{statistikk13}.} The right to harness energy from rivers, streams and waterfalls generally belongs to local landowners under a riparian system.\footnote{This arrangement is rooted in the first known legal sources in Norway, the so-called ``Gulating'' laws, thought to have been in force well before AD 1000. See \cite[111-112,120]{robberstad81}.}  Historically, waterfalls were very important to local communities, particularly as a source of power for grist and saw mills.\footnote{See \cite[121]{tvedt13}.} %%Indeed, the fact that peasants in Norway controlled local water resources can help explain why they were relatively free, both economically and socially, compared to many other places in Europe.\footnote{See \cite[121]{tvedt13}.}

Following the industrial revolution, local ownership and management came under increasing pressure. At the beginning, this pressure was exerted by private commercial interests, often foreign investors, who saw the industrial potential in hydropower and started speculating in Norwegian water resources.\footnote{See \cite[30-31]{nou04}.} Later, the pressure on local self-governance was exerted mainly by the government, following the introduction of new legislation to regulate the development of hydroelectric power.\footnote{See \cite[41-57]{thue96} (describing the  regulatory system set up during this time).} This legislation set up a system that gave highly preferential treatment to public utilities over private actors, including local owners.\footnote{See \cite[46]{thue96} (describing legislation introduced to promote public utilities, including new expropriation authorities directed at local owners of waterfall).} At first, the motivation behind this reform was to facilitate a decentralised form of government control, led by public utilities controlled by the municipality governments.\footnote{See \cite[44-47]{thue96}.} However, the hydroelectric sector underwent gradual centralisation, a process that gained momentum after the Second World War when the state itself assumed a leading role.\footnote{See \cite[59-85]{thue96}. For the history of the state's involvement with hydropower generally, see \cite{thue06, skjold06,thue06b}.} After this, local communities and local riparian owners became increasingly marginalised. In particular, local hydropower producers were systematically pressured into shutting down their hydroelectric plants, often as a condition for being granted access to electricity through the national, monopolised, electricity grid.\footnote{See \cite[p.111]{hindrum94}.}

Then, in the early 1990s, the electricity sector was liberalised, largely inspired by the market-orientation and privatisation of the public sector in the UK under Thatcher.\footnote{See generally \cite{midttun98}.} The production sector was decoupled from the grid sector, while public utilities where reorganised as commercial companies.\footnote{See \cite[86]{efta07} (describing how Norwegian electricity companies, most of which are still (partly) publicly owned, now operate as for-profit, limited liability companies).} At the same time, the regulatory system was decoupled from both political and commercial decision-making processes, to become more expert-based.\footnote{See \cite[26-27]{brekke12}.} Moreover, the sector underwent additional centralisation, as a result of mergers and acquisitions among former public utilities.\footnote{See \cite[583]{bibow03}. I mention that despite significant continuous centralisation from the Second World War to this day, the Norwegian hydroelectric sector is still relatively decentralised compared to other countries, e.g., the UK, see \cite[181]{midttun98}. Arguably, this is a lasting influence of a tradition based on local, egalitarian, ownership of water resources.}

Following the reform, access rights to the national grid are meant to be granted equally to all potential actors on the energy market, including private companies.\footnote{See generally \cite{hammer96}. For an interesting presentation and analysis of grid-based markets in general, see \cite{falch04}.} After the passage of the \cite{ea90}, the energy companies that operate the national grid (the grid is divided into regions) are no longer authorised to shut out competitors.\footnote{See the \indexonly{ea90}\dni\cite[3-4]{ea90}.} A side-effect of this is that it has become possible for local landowners to undertake their own hydropower projects. Local owners can now access the grid to sell the electricity they produce on Nord Pool, the largest electrical energy market in Europe.\footnote{See generally \cite{larsen06,larsen08,larsen12}.} This has led to increased tension between local interests and established hydropower companies. The following fundamental question has arisen: who is entitled to benefit from rivers and waterfalls, and who is entitled to a say in decision-making processes concerning their use?

This chapter sets the stage for discussing this question in more depth, by detailing how the hydropower sector is organised. It looks both to the law and to the commercial and administrative practices surrounding it. Moreover, special attention is directed at those aspects that have changed following liberalisation, resulting in conflicts involving property. Specifically, it is argued that the tension between large-scale development companies and local owners can only be understood on the basis of a social function perspective on riparian ownership. I start by giving a brief overview of the legal system more generally, emphasising also the role that private property has played in the development of Norwegian democracy.\footnote{The classic reference on Norwegian constitutional law is \cite{andenes06}.}

\section{Norway in a Nutshell}\label{sec:4:2}

\noo{ %\footnote{It should be noted that the executive branch also enjoys considerable legislative power under Norwegian law. Both informally, because it prepares new legislation, and also formally, because it has wide delegated powers to issue so-called {\it directives} (forskrifter). Indeed, it is typical for acts of parliament to include a general delegation rule which permits the executive to legislate further on the matters dealt with in the act, by clarifying and filling in the gaps left open by it.}

Norway is a constitutional monarchy, based on a representative system of government.\footnote{For Norwegian constitutional law generally, see \cite{andenes06}.} The executive branch is led by the King in Council, the Cabinet, headed by the Prime Minister. Legislative power is vested in the Storting, the Norwegian parliament, elected by popular vote in a multi-party setting. In 1884, the parliamentary system first triumphed in Norway, as the cabinet was forced to resign after it lost the confidence of parliament. The principle has since obtained the status of a constitutional custom. In particular, the cabinet can not continue to sit if parliament expresses mistrust against it. However, an express vote of confidence is not required. In practice, due to the multi-party nature of Norwegian politics, minority cabinets are quite common. These can sustain themselves by making long-term deals with supporting parties, or by looking for a majority on a case-by-case basis.

The judiciary is organised in three levels, with 70 district courts, 6 courts of appeal, and the Supreme Court. The district courts have general jurisdiction over most legal matters; there is no division between constitutional, administrative, civil, criminal courts. \footnote{However, there are distinct procedural rules for civil and criminal cases and a special court for land consolidation, see the \cite{lca79}. Moreover, both the district courts and the courts of appeal follow special procedural rules in appraisal disputes, for instance when compensation is awarded following expropriation, see the \cite{aa17}.} The courts of appeal have a similarly broad scope. Moreover, the right to appeal is ensured in most cases.\footnote{The right to an appeal is not absolute. In civil cases, it is generally required that the stakes are above a certain lower threshold, measured in terms of the appellants' financial interest in the outcome. See \indexonly{cda05}\dni\cite[29-13]{cda05}.} The Supreme Court, on the other hand, operates a very strict restriction on the appeals it will allow.\footnote{See the \indexonly{cda05}\dni\cite[30-4]{cda05}.} It typically only hears cases if a matter of principle is at stake, or if the law is thought to be in need of clarification.\footnote{See, generally, \cite{skoghoy08}.}
}
The Norwegian legal system is often said to be based on a special ``Scandinavian'' variety of civil law, which includes strong common law elements: legislation is not as detailed as elsewhere in continental Europe, some legal areas lack a firm legislative basis, it is generally accepted that courts develop the law, and the opinions of the Supreme Court are considered crucial to the legislative interpretation at the lower courts.\footnote{See, generally, \cite{bernitz07}.} The Supreme Court operates a very strict restriction on the leave to appeal.\footnote{See the \indexonly{cda05}\dni\cite[30-4]{cda05}.} It typically only hears cases if a matter of principle is at stake, or if the law is thought to be in need of clarification.\footnote{See, generally, \cite{skoghoy08}.} Moreover, legislation remains the primary source used to resolve most legal disputes. When applying it, the courts tend to place great weight on preparatory documents procured by the executive branch. These documents are widely regarded as expressions of legislative intent, even though parliament is not usually actively involved in the process during the preparatory stages.

The Constitution of Norway dates back to 1814 and was heavily influenced by then recent political movements, particularly in the US and France.\footnote{See generally \cite{mestad14}.} Moreover, it was influenced by a desire for self-determination, as Noway was at that time a part of Denmark-Norway, largely controlled by the Danish elite. Following the Napoleonic wars, Norwegian politicians sought to take advantage of Denmark's weakened position to gain independence and drafted a Constitution. In the end, Norway was forced to enter into a union with Sweden (who was backed by the winning side of the Napoleonic wars), but the Constitution remained in place. Moreover, after the triumph of the parliamentary system in 1884, Norway would also eventually gain independence, in 1905, following a peaceful and democratic transition process.\footnote{See generally \cite{sejersted15}.}

During the 19th century, farmers and smallholders emerged as a powerful group in Norwegian politics. This was in large part due to the fact that they were also landowners, whose rights and contributions were not limited to traditional farming.\footnote{See generally \cite{hommerstad14}. The ``classic'' presentation of the political influence of farmers in Norway is \cite{koht26}.} This had not always been the case; during the middle ages, the Norwegian farmer had usually been a tenant.\footnote{See generally \cite{myking05}.} However, tenant farmers always enjoyed a significant degree of control over the management of the land and its natural resources.\footnote{See \cite[59-60]{pryser99}; \cite[226-238]{myking05}.} Moreover, between the 17th and the end of the 18th century, almost all Norwegian tenant farmers bought their land from their landlords.\footnote{See \cite[108]{nordtveit15}.} 
As a result, the distribution of land ownership in Norway had already become highly egalitarian at the time of the Constitution.

Many resources attached to land were owned by the local community as such, as vast areas of non-arable outfields were either considered commons or else jointly owned by the local smallholders.\footnote{For a presentation of different forms of joint ownership in Norwegian outfields, see \cite{stenseth07a}.} As a result, Norway became a society were land ownership was not a privilege for the few, but held by the many, particularly compared to feudal Europe.\footnote{For a comparative discussion of this, focusing on how it influenced the industrialisation process in Norway, setting it apart from the industrialisation process in the UK, see \cite{brox13}.}

In 1814, the landed nobility in Norway was further marginalised. Indeed, the Constitution itself prohibited the establishment of new noble titles and estates.\indexonly{grunnloven14}\dni\footcite[23|118]{grunnloven14} Then, in 1821, all hereditary titles were abolished (although existing nobles kept their titles for their lifetimes).\footnote{See \cite{adel09}.} By the middle of the 19th century, ordinary farmers had gained even greater political influence. In fact, they emerged as the leading political class, alongside the city bureaucrats.\footnote{See generally \cite{hommerstad14}.} During this time, Norway also introduced a system of powerful local municipalities. These were organised as representative democracies, becoming miniature versions of the cherished, as of yet unfulfilled, nation state (Norway was still in a union with Sweden at this time). Even today, municipalities retain a great deal of power in Norway, particular in relation to land use planning.\footnote{They are the primary decision-makers for spatial planning, as pursuant to \cite{pb08}.} They represent a highly decentralised political structure, with a total of 428 municipalities in force as of 01 January 2013. \footnote{This is down from the all-time high of 747 in 1930. There have long been proposals to reduce the number of municipalities further, but so far the political resistance against this has prevented major reforms. See \cite{kommuner14} (report to the Ministry from an expert committee on municipality reform, 2014).}

Local control of water resources, ensured through property rights, was very important to farmers and rural communities in pre-industrial Norway. According to Terje Tvedt, 10 000-30 000 mills were in operation in Norway in the 1830s.\footnote{See \cite[121]{tvedt13}.} As Tvedt argues, the fact that these mills were under local control was particularly important because it helped ensure self-sufficiency. In addition, saw mills became an important source of extra income for Norwegian farming communities.

Today, the importance of water is clearly felt throughout Norwegian society. This is not because water is scarce, but rather because it is so plentiful. Not only is water power the main source of domestic energy. It also occupies a special place in Norwegian culture. It is important to the identity of many communities, particularly in the western part of the country, where majestic waterfalls are considered symbols both of the hardship of the natural conditions and the sturdiness of local people. The implications for the tourism industry are also significant, as the natural environment attracts visitors and economic activity to regions of Norway that are otherwise threatened by stagnation and depopulation.

In light of this, it is not surprising that there is a tradition in Norway for local resistance against large-scale development that is considered damaging to the environment. In the 1960s and 70s, when the state embarked on their most ambitious projects, local environmental movements became nationally significant, as symbols of resistance against centralisation, exploitation of weaker groups, and environmental destruction.\footnote{See \cite{nilsen08}.}

This illustrates that water resources are embedded in the social fabric in such a way that a purely entitlements-based approach to property rights in these resources would be largely inappropriate. Rather, the case of Norwegian water seems to be well-suited for an investigation based on a social function view on property. As I show in this and the following two chapters, rivers and waterfalls serve to bring out tensions between rights and obligations in property, while also shedding light on the question of how to organise decision-making processes regarding economic development.

In the next section, I argue that the present law on hydropower in Norway tends to recognise only a small part of the relevant picture. On the one hand, it recognises the financial entitlements of individual owners, which it tries to balance against the regulatory needs of the state. But it largely fails to take into account that owners have broader interests, even obligations, relating to the sustainable management of their streams and their waterfalls. Moreover, the law appears largely unable to prevent commercial interests from exerting a strong pull on various state bodies, particularly those that are only weakly grounded in processes of democratic decision-making.

\section{Hydropower in the Law}\label{sec:4:3}

As mentioned in chapter \ref{chap:1}, the right to harness power from a river is regarded as a separate unit of private property in Norway, referred to as a waterfall.\footnote{Historically, the law emphasised ownership of traditional agrarian water resources, such as fishing rights. However, new sticks were added to the waterfall bundle over the years, including the right to develop hydropower, see \cite[14-32]{vislie44}. See also \cite[108]{nordtveit15}. For a detailed presentation of the history of water law in pre-industrial times, I refer to \cite{motzfeld08}.} The system is riparian, so by default, a waterfall belongs to the owner of the land over which the water flows.\footnote{See the \indexonly{wra00}\dni\cite[13]{wra00}.} The landowners do not own the water as such -- freely running water is not subject to ownership -- and the riparian owners' right to withhold or divert water is limited.\footnote{See \indexonly{wra00}\dni\cite[8|15]{wra00}.} 

%However, as mentioned in chapter \ref{chap:1}, the right to the hydropower in a river is considered to be a separate property right in the bundle of rights typically held by riparian owners, referred to as the right to the waterfall.%, a terminology I will also adopt-\footnote{The Norwegian term `fall' has a somewhat broader meaning than its English counterpart, `waterfall'. The word `fall' is used to describe a continuous section of any stream or river, typically identified by giving the total difference in altitude over the relevant stretch of riverbed. Furthermore, the Norwegian term `falleier' refers to a legal person who possesses the rights to the hydropower over such a section. In this thesis, I will typically refer to the owners of waterfalls, streams and rivers with the intended reading being the same as the Norwegian notion of a `falleier'. If special qualification is needed, for instance to distinguish between different classes of riparian owners, I will make a note of this explicitly.}

However, the waterfall owners have the exclusive right to harness the potential energy in the water over the stretch of riverbed belonging to them. This right can be partitioned off from rights in the surrounding land, and large-scale hydropower schemes typically involve such a separation of water rights from land rights; the energy company acquires the right to harness the energy, while the local landowners retain ownership of the surrounding land.

Norwegian rivers, and especially rivers suitable for hydropower schemes, tend to run across outfields that are owned jointly by local farmers. Hence, rights to streams and waterfalls are typically held among several members of the rural community.\footnote{Rivers tend to run through land that has not to been enclosed. Moreover, in places where there has been a land enclosure, water rights are often explicitly left out, such that they are still considered jointly owned rights belonging to the community of local farmers. For more details on (forms of) joint ownership among Norwegian farmers, see, e.g., \cite[570]{stenseth07a}.} Local owners might not be willing to give up their ownership to facilitate development, especially not on terms proposed by external developers. Hence, the authority to expropriate has been an important legal instrument for state-backed hydropower companies. It has also been used extensively, particularly after the state made hydropower development a priority after the Second World War.

This has resulted in a tension where, on the one hand, rights to harness hydropower from streams and waterfalls are considered private property, while on the other hand, it has become common to speak of hydropower as a resource belonging to the public. Since the \cite{ica17} was amended in 2008, this ambivalence in the discourse surrounding hydropower has also been part of the statutory provisions regulating hydropower development. I quote the two relevant sections side by side below:%\footnote{The first quote is taken from the general water law, with roots going back a thousand years to the so-called ``Gulating'' laws mentioned in the introduction. The second quote is taken from a law directed specifically at large-scale hydropower, introduced during the early days of the hydropower industry.}

{\begin{minipage}[t]{16em}
 \begin{aquote}{\tiny \indexonly{wra00}\dni\cite[13]{wra00}} \footnotesize A river system belongs to the owner of the land it covers, unless otherwise dictated by special legal status. [...]

The owners on each side of a river system have equal rights in exploiting its hydropower.
\end{aquote}  
\end{minipage}}
{\begin{minipage}[t]{22em}
\begin{aquote}{\tiny \indexonly{ica17}\dni\cite[1]{ica17} (after amendment in 2008)} \footnotesize Norwegian water resources belong to the general public and are to be managed in their interest. This is to be ensured by public ownership.
\end{aquote}
\end{minipage}} \\

The intended reading of section 1 of the \cite{ica17}, quoted on the right above, is that it provides a ``general starting point''.\footnote{See \cite[72]{otprp61}.} According to the Ministry, it expresses what has always been the purpose of the legislation used to regulate large-scale hydropower.\footnote{See \cite[72]{otprp61}.} 

This should not be understood as an explicit attack on the principle of private ownership expressed in section 13 of the \cite{wra00}, quoted on the left above.\footnote{There are no indications in the preparatory materials that the Ministry sought to confront the principles of ownership encoded in the \cite{wra00}.} However, the Ministry's comment underscores the extent to which the government regards it as natural to interfere with private rights to waterfalls, to pursue policies that it regards to be in the public interest. Taken in this light, section 1 of the \cite{ica17} reflects the prevailing opinion that there are few, if any, recognised limits on the state's power to manage privately owned water resources.\footnote{For a reflection of the same attitude, citing the state's broad regulatory competence as the main reason not to nationalise Norwegian water power rights, I refer to the preparatory documents underlying the \cite{wra00}. See \cite[152-153]{nou94}.}

This aspect of the Norwegian system has become particularly significant following the liberalisation of the electricity sector in the early 1990s.\footnote{See, e.g., \cite{larsen06}.} Since then, there have been an increasing number of cases where owners who are interested in undertaking their own development schemes attempt to fend off commercial energy companies wishing to expropriate.\footnote{See, e.g., \cite{sofienlund07}.} Importantly, the state has tended to side with the commercial companies in these cases, granting them the authority to expropriate for economic development. This has resulted in several Supreme Court decisions on hydropower and expropriation in the past few years, all of which have been in favour of the energy companies.\footnote{See  \cite{uleberg08,jorpeland11,klovtveit11,otra13}.} Before discussing these cases in more detail in the next chapter, I provide an in-depth analysis of hydropower in the law and in practice, to shed further light on the underlying conflict that has led to the recent surge in cases on expropriation. First, I briefly present the key legislation regulating the hydropower sector.

\subsection{The Water Resources Act}\label{sec:wra00}

The \cite{wra00} contains the basic rules regarding water management in Norway.\footnote{I also mention the Water Framework Directive of the European Union, \cite{water00}. It has been implemented in Norwegian law as the Directive Regarding Frameworks for Water Management, FOR-2006-12-15-1446. In Norway, it only indirectly impacts on the hydropower licensing procedure, and does not appear to have had much influence at all on issues studied in this thesis. I mention that there is some concern that the Norwegian implementation of the directive has not sufficiently recognized the need for structural reforms to ensure holistic resource management; the established approach to water management, which is centralised and sector-based, remains largely in place. See \cite{hanssen14}.} This Act is not only concerned with hydropower, but regulates the use of river systems and groundwater generally.\footnote{See the \indexonly{wra00}\dni\cite[1]{wra00}. A river system is defined as ``all stagnant or flowing surface water with a perennial flow, with appurtenant bottom and banks up to the highest ordinary floodwater level'', see \indexonly{wra00}\dni\cite[2]{wra00}. Artificial watercourses with a perennial flow are also covered (excluding pipelines and tunnels), along with artificial reservoirs, in so far as they are directly connected to groundwater or a river system, see the \indexonly{wra00}\dni\cite[2a, 2b]{wra00}.} In section 8, the Act sets out the basic license requirement for anyone wishing to undertake measures in a river system.\footnote{Measures in a river system are defined as interventions that ``by their nature are apt to affect the rate of flow, water level, the bed of a river or direction or speed of the current or the physical or chemical water quality in a manner other than by pollution'', see the \indexonly{wra00}\dni\cite[3a]{wra00}.} The main rule is that if such measures may be of ``appreciable harm or nuisance''  to public interests, then a license is required.\footnote{See the \indexonly{wra00}\dni\cite[8]{wra00}. There are two exceptions, concerning measures to restore the course or depth of a river, and concerning the landowner's reasonable use of water for his permanent household or domestic animals, see the \indexonly{wra00}\dni\cite[12|15]{wra00}.} The water authorities themselves decide if this condition is met.\footnote{See \indexonly{wra00}\dni\cite[18]{wra00}.} In relation to hydropower development, it is established practice that most hydropower projects over 1000 KW will be deemed to require a license.\footnote{See, e.g., \cite{nve09}. Exceptions are possible, for instance projects that upgrade existing plants, or which utilise water flowing between artificial reservoirs.}

The basic assessment criterion is that a license ``may be granted only if the benefits of the measure outweigh the harm and nuisances to public and private interests affected in the river system or catchment area''.\footnote{See \indexonly{wra00}\dni\cite[25]{wra00}.} Hence, the water authorities are empowered to decide whether a licence {\it should} be granted, if they find that the benefits outweigh the harms. The courts are very reluctant to censor the discretion of the administrative decision-makers on this point.\footnote{This is an expression of the principle of ``freedom of discretion'' ({\it det frie forvaltningsskjønn}) for the administrative branch, a fundamental tenet of Norwegian administrative law. See generally \cite[71-74]{eckhoff14}.}

The Ministry of Petroleum and Energy maintains indirect control over the assessment process by issuing directives regarding the administrative procedure in licensing cases.\footnote{See section 65 of the \cite{wra00}.} In addition, the procedure is determined in large part by administrative practices developed by the water authorities themselves.\footnote{I return to a presentation of administrative practice in section \ref{sec:4:3:1}. In principle, many of the rules in the \cite{paa67} also apply. However, in practice, these rules are of limited practical relevance compared to sector-specific practices. This has raised controversy in recent years, particularly in cases involving expropriation, as discussed in chapter \ref{chap:5}, section \ref{sec:5:6}.} 

A few basic procedural rules are encoded directly in the \cite{wra00}. This includes rules to ensure that the application is sufficiently documented, so that the authorities have enough information to assess its merits.\footnote{See \indexonly{wra00}\dni\cite[23]{wra00}.} Moreover, a basic publication requirement is expressed, stating that applications are public documents and that the applicant is responsible for giving public notice. The intention is that interested parties should be given an opportunity to comment on the plans.\footnote{See \indexonly{wra00}\dni\cite[24]{wra00}. There are some exceptions to the requirement to give public notice, however. It may be dropped in case it appears superfluous, or if the application must be rejected or postponed, see \indexonly{wra00}\dni\cite{wra00} ss 24a--24c.} More detailed rules for public notice of applications are given in section 27-1 of the \cite{pb08}, which also applies to licensing applications under section 8 of the \cite{wra00}.\footnote{In addition, I mention section 22, which regulates the relationship between licensing and planning in relation to water resources. In essence, the section stipulates that the water authorities may prioritise planning over assessment of individual licensing cases, e.g., by refusing to take applications under consideration if they interfere with ongoing planning procedures. However, the section leaves significant room for discretion in this regard. It also bears noting that watercourse planning is placed under centralised government control. This contrasts with land use planning in general, which is mainly the responsibility of the local municipality governments. See generally \cite{sp}.}

\noo{Furthermore, an important rule of principle is given in section 22, regarding the relationship between applications for licenses and governmental ``master plans'' for the use or protection of river systems in a greater area. These plans have no clear legislative basis, but were introduced through parliamentary action in the 1980s, when the parliament decided to initiate such planning in an effort to introduce a more holistic basis for assessment of licensing applications.\footnote{Today, the planning authority is delegated to the Directorate of Natural Preservation and the NVE. See \cite{sp}.}
 
According to section 22 of the \cite{wra00}, if a river system falls within the scope of a master plan that is under preparation, an application to undertake measures in this river system may be delayed or rejected without further consideration.\footnote{See \indexonly{wra00}\dni\cite[22]{wra00}, para 1.} Moreover, a license may only be granted if the measure is without appreciable importance to the plan.\footnote{See \indexonly{wra00}\dni\cite[22]{wra00}, para 1.} In addition, once a plan has been completed, the processing of applications is to be based on it, meaning that an application which is at odds with some master plan may be rejected without further consideration.\footnote{See \indexonly{wra00}\dni\cite[22]{wra00}, para 2.} It is still possible to obtain a license for such a project, but if it harnesses less hydropower than the project indicated by the plan, section 22 states that only the Ministry may grant it.\footnote{See \indexonly{wra00}\dni\cite[22]{wra00}, para 2.}
}

The rules considered so far apply to any measures in river systems, not only hydropower projects. However, special procedures that apply to hydropower cases are described in other statutory provisions. The most important is the \cite{wra17}, which is specifically aimed at a certain subgroup of hydropower schemes, namely those that involve regulation of the flow of water in a river system.\footnote{See section \ref{sec:wra17} below.} However, according to section 19 of the \cite{wra00}, many provisions from the \cite{wra17} also apply to unregulated, run-of-river, schemes, if they generate more than 40 GWh per annum.\footnote{See \indexonly{wra00}\dni\cite[19]{wra00}.}

\subsection{The Watercourse Regulation Act}\label{sec:wra17}

In order to maximise the output of a hydropower scheme, the flow of water may be regulated using dams or diversions. Regulation was particularly important in the early days of hydropower, before the national electricity grid was developed.\footnote{See \cite[83]{uleberg08}.} Indeed, in the early days, it was common for electricity producers to get paid based on the stable effect they were able to deliver, rather than the total amount of energy they harnessed.\footnote{See \cite{sofienlund07}.}

Today, this has changed, as producers get paid based on the total amount of electricity they deliver,  measured in kilowatt hours (KWh). The price fluctuates over the year, and the supply-side is still influenced by instability in the waterflow in Norwegian rivers. However, the smoothing effect of the national grid means that run-of-river schemes can be carried out profitably, even if most of the electricity from the plant is produced during peak periods.

Despite the growing importance of run-of-river schemes, many key rules regarding hydropower development are still found in the \cite{wra17}. This Act defines regulations as ``installations or other measures for regulating a watercourse's rate of flow''. It also explicitly states that this covers installations that ``increase the rate of flow by diverting water''.\footnote{See \indexonly{wra17}\dni\cite[1]{wra17}.} The core rule of the Act is that watercourse regulations that affect the rate of flow of water above a certain threshold are subject to a special licensing requirement.\footnote{See \indexonly{wra17}\dni\cite[2]{wra17}.}

The threshold is defined in terms of the notion of a ``natural horsepower'', such that a license is required if the regulation yields an increase of at least 400 natural horsepower in the river. Natural horsepower is a measure of the gross estimate of the power that can be harnessed from a river stably for at least 350 days a year.\footnote{See \indexonly{wra17}\dni\cite[2]{wra17}.} The definition is a simple mathematical expression, given below:

$$
nat.hp(Q,H) = 13.33 \times H \times Q
$$

This formula states that the natural horsepower of a regulation project ($nat.hp(Q,H)$) is a function of two variables, $H$ and $Q$. The constant factor $13.33$ is the force of gravity of Earth exerted on a mass of 1 kg (or, approximately, 1 litre of water). The variable $H$ is the difference in altitude (measured in metre) from the intake dam to the power generator. The variable $Q$ is the amount of water (measured in litre) stably available every second for at least 350 days per year. The result is then a gross estimate (assuming no energy loss) of the stable horsepower output of the hydroelectric plant that harnesses the power of $Q$ litres of water per second over a difference in altitude of $H$ metres. 

Section 2 of the \cite{wra17} asks us for the {\it increase} of this figure after regulation. To arrive at this number, one first uses the formula with $Q$ taken to be $Q_1$, the stable water flow prior to regulation, before calculating it with $Q$ taken to be $Q_2$, the stable water flow after regulation. The difference between the second and the first figure ($nat.hp(Q_2,H) - nat.hp(Q_1,H)$) is the increase of natural horsepower resulting from regulation.

Effectively, at a time when electricity had to be produced at a stable effect, from a stable source of power, this increase in natural horsepower was a gross estimate of the value added to the river by regulation. In the present context, it suffices to say that if a hydropower project involves regulation at all (i.e., if it is not a run-of-river scheme), it will indeed yield 400 natural horsepower or more. Hence, a special license will be required pursuant to section 2 of the \cite{wra17}. 

The criteria for granting a regulation license are similar to those for granting a license pursuant to the \cite{wra00}. In particular, section 8 of the \cite{wra17} states that a license should ordinarily be issued only if the benefits of the regulation are deemed to outweigh the harm or inconvenience to public or private interests.\footnote{See \indexonly{wra17}\dni\cite[8]{wra17}.} In addition, it is made clear that other deleterious or beneficial effects of importance to society should be taken into account.\footnote{See \indexonly{wra17}\dni\cite[8]{wra17}.} Finally, if an application is rejected, the applicant can demand that the decision is submitted for review by Parliament.\footnote{See \indexonly{wra17}\dni\cite[8]{wra17}.}

The \cite{wra17} contains more detailed rules regarding the procedure for dealing with license applications. The most practically important is that the applicant is obliged to carry out an impact assessment pursuant to the \cite{pb08}. This means that the applicant must organise a hearing and submit a detailed report on positive and negative effects of the development, prior to submitting a formal application for a licence. Effectively, at least {\it two} detailed rounds of assessment are therefore required before a license is granted.

In addition to prescribing impact assessments, the \cite{wra17} contains more specific rules concerning the second public hearing that should take place, when the application as such is processed. First, the applicant should make sure that the application is submitted to the affected municipalities and other interested government bodies.\indexonly{wra17}\dni\footcite[6]{wra17} Second, the applicant should send the application to organisations, associations and the like whose interests are ``particularly affected''.\indexonly{wra17}\dni\footcite[6]{wra17} Along with the application, these interested parties should be given notification of the deadline for submitting comments, which should not be less than three months.\footnote{See \indexonly{wra17}\dni\cite[6]{wra17}.} The applicant is also obliged to announce the plans, along with information about the deadline for comments, in at least one commonly read newspaper, as well as the Norwegian Official Journal.\footnote{\indexonly{wra17}\dni\cite[6]{wra17}. The Norwegian Official Journal is the state's own announcement periodical.}

%A license pursuant to the \cite{wra17} might be cumbersome to obtain, but a successful application also results in a significant benefit. Most importantly, the license holder then automatically has a right to expropriate the necessary rights needed to undertake the project, including the right to inconvenience other owners.\footnote{See \cite[16]{wra17}.} Hence, expropriation is a side-effect of a regulation license. Even so, the issue of expropriation rarely receives any special consideration in regulation cases. In particular, the assessment undertaken by the water authorities is focused on the licensing issue, which does not compel them to direct any special attention towards owners' interests.\footnote{I demonstrate this, and discuss it in much more depth, in Chapter \ref{chap:4}, Section \ref{sec:jorpeland}.}

In general, the issue of who owns and controls the water resources in question receives little attention in relation to licensing applications, both pursuant to the \cite{wra17} and the \cite{wra00}. Instead, the focus is on weighing environmental interests against the interest of increasing the electricity supply and facilitating economic development. The issue of resource ownership is more prominent in relation to a third important statute, namely the \cite{ica17}.

\subsection{The Industrial Licensing Act}\label{sec:ica17}

In the early 20th century, industrial advances meant that Norwegian waterfalls became increasingly interesting as objects of foreign investment. To maintain national control of water resources, parliament passed an Act in 1909 that made it impossible to purchase valuable waterfalls without a special license.\footnote{See \cite[59]{falkanger87}.} The follow-up to this Act is the \cite{ica17}, which is still in force. It applies to potential purchasers and leaseholders of rivers that may be exploited so that they yield more than 4000 natural horsepower.\footnote{Unlike section 2 of the \cite{wra17}, this asks only for the number of horsepower in the river (after regulation), not the {\it increase} of this number.}

To reach this number requires a substantial regulation, so the Act does not apply to many run-of-river hydropower schemes, even large-scale projects. Originally, the main rule in the \cite{ica17} stated that all licenses granted to private parties were time-limited, and that the waterfalls would become state property without compensation when they expired, after at most 60 years.\footnote{See the previous \indexonly{ica17}\dni\cite[2]{ica17}, in force before the amendment on 26 September 2008.} This was known as the rule of {\it reversion} in Norwegian law.\footnote{This is a misnomer, however, in light of how most rivers and waterfalls were originally owned by local smallholders, not the state.}

In a famous Supreme Court case from 1918, the rule was upheld after having been challenged by owners on constitutional grounds.\footnote{See \cite{johansen18}.} This was based on the finding that reversion represented a form of regulation of property, not expropriation. Hence, it could not be challenged on the basis of section 105 of the Constitution, even though the owners were not awarded any compensation. 

While the rule of reversion withstood internal challenges, it was eventually struck down by the EFTA Court in 2007, as a breach of the EEA agreement.\footnote{See \cite{efta07}. The EEA (European Economic Area) agreement sets up a framework for the free movement of goods, persons, services and capital between Norway, Iceland, Lichtenstein and the European Union. The EFTA (European Free Trade Association) oversees the implementation of the EEA for those members of EFTA that are also members of the EEA (all except Switzerland). For further details, see generally \cite{bull94,magnussen02,fredriksen09}.} This conclusion was based on the fact that reversion only applied to privately owned companies, which the Court regarded as an illegitimate form of discrimination. After this ruling, the \cite{ica17} was amended. Today, only companies where the state controls more than 2/3 of the shares may purchase waterfalls or rivers to which the Act applies.\footnote{See the \indexonly{ica17}\dni\cite[2]{ica17}.}

This means that such rivers and waterfalls can only be bought, leased or expropriated by companies in which the state is a majority shareholder. In practice, however, landowners are still able to sell the land from which the right to a waterfall originates, even if this also means transferring the waterfall to a new owner. The rule is typically only enforced when riparian rights as such are transferred, specifically for the purpose of large-scale hydropower development. In particular, small-scale development and large run-of-river schemes can still usually be carried out by local owners. The policy justification for the (amended) \cite{ica17} is based on the idea that giving preference to state-owned actors will protect the public interest in Norwegian hydropower. However, this perspective clashes with the fact that the electricity sector itself has been liberalised. The state may be a majority shareholder in the most powerful companies, but these companies are now run according to 
commercial principles, with little or no direct political involvement.\footnote{See \cite[86]{efta07}.}

Hence, as the EFTA court highlights in its judgement on reversion, there appears to be a lack of convincing policy reasons why state-owned companies should be given preferential treatment.\footnote{See \cite[84-87]{efta07}.} In light of this, Norway's response to the Court's decision is a curious one: instead of creating a level playing field, the preference given to state-owned commercial companies is made even more marked, as privately owned companies are now excluded from one segment of the hydropower market altogether.

%Of course, the public benefits indirectly from the fact that public bodies, as shareholders, are entitled to dividends. But it is not clear why this benefit should be considered in a different light than other indirect financial benefits which might as well be extracted from private companies, e.g., through taxation. Moreover, public-private partnerships are still permitted, as private actors may own up to two-thirds of ``state-owned'' companies. What this means is that the preferential treatment given to state actors is in fact also extended to those private actors that the state happen to prefer. Interestingly, this style of regulation contrasts quite sharply with some of the key ideas behind the basic building block of the liberalised electricity market, namely the \cite{ea90}.

\subsection{The Energy Act}\label{sec:ea}

Before 1990, the Norwegian electricity sector was tightly regulated by the government.\footnote{See generally \cite{bye05,skjold07}.} The responsibility for the national grid was divided between various public utilities that would also typically engage in electricity production, wielding monopoly power within their districts. The most powerful utilities were controlled by the state, who also developed large-scale hydropower to supply the metallurgical industry with cheap electricity.\footnote{See \cite[67-71]{thue96}.} However, the county councils and the municipalities maintained a significant stake in the hydroelectric sector, as they often controlled the utilities responsible for the electricity supply in their own local area.\footnote{See \cite[85]{thue96}.} 
Prior to 1990, there was no real competition on the electricity market, and the local monopolists could deny other energy producers access to their segment of the distribution grid.\footnote{See \cite[83-84]{uleberg08}.}

This system was abandoned following the passage of the \cite{ea90}.\footnote{See generally \cite{bibow11}.} This Act set up a new regulatory framework, where management of the grid was decoupled from the hydropower production sector.\footnote{See generally \cite{bye05}.} In particular, the Act established a system whereby consumers could choose their electricity supplier freely. At the same time, the Act aimed to ensure that producers were granted non-discriminatory access to the electricity grid. This laid the groundwork for what has today become an international market for the sale of electricity, namely the Nord Pool.\footnote{See generally \cite{galtung07}.}

In response to this, monopoly companies were reorganised, becoming commercial companies that were meant to compete against each other, and against new actors that entered the market.\footnote{See \cite{claes11}.} In addition to commercialisation, the market-orientation of the sector has also lead to centralisation, as many of the locally grounded municipality companies have disappeared as a result of mergers and acquisitions.\footnote{Today, the 15 largest companies, largely controlled by the state and some prosperous city municipalities, own roughly 80\% of Norwegian hydropower, measured in terms of annual output. See \cite[28]{otprp61}. The process causing this concentration started long before the market-oriented reform of the sector. In particular, after the Second World War, there was a significant push by the state towards increased centralisation, see \cite{skjold06,thue06b}.} As a result, the local and political grounding of the electricity sector, which used to be ensured through decentralised municipal ownership, has been significantly weakened.

At the same time, the fact that any developer of hydropower is now entitled to connect to the national grid gives private actors a possibility of entering the Norwegian electricity market. They may do so not merely as (minority) shareholders in former utilities, but also as {\it competitors}, as long as they stick to run-of-river or small-scale hydropower.\footnote{See generally \cite{larsen06,larsen08,larsen12}.} In the next section, I give a step-by-step presentation of the licensing procedure for hydropower, which serves to summarise the legislative framework and provide information about the institutional framework within which it is called to function.

\subsection{The Licensing Procedure}\label{sec:4:3:1}

The water authorities in Norway are centrally organised. The most important body is the Norwegian Water Resources and Energy Directorate (NVE), based in Oslo. In many cases, the NVE have been delegated authority to grant development licenses themselves, but in case of large-scale development, they only prepare the case, then hand it over to the Ministry of Petroleum and Energy.\footnote{See delegation of 19 December 2000, from the Ministry of Petroleum and Energy (FOR-2000-12-19-1705) and Directive of 15 December 2000, from the King in Council (FOR-2000-12-15-1270), pursuant to \indexonly{wra00}\dni\cite[64]{wra00}.} The Ministry, in turn, gives its recommendation to the King in Council, who makes the final decision.\footnote{See Directive of 15 December 2000, from the King in Council (FOR-2000-12-15-1270).} Parliament must also be consulted for regulations that will yield more than 20 000 natural horsepower.\footnote{See \indexonly{wra17}\dni\cite[2]{wra17}.}

As indicated by the survey of relevant legislation given in previous sections, there are many categories of hydropower projects. Moreover, different categories call for different licenses. Hence, the first step in the application process is for the developer to determine exactly what kind of license they require. This is further complicated by the fact that some categories overlap, since they are based on different measuring sticks for assessing the scale of an hydropower project. 

One important parameter is the power of the hydropower generator, measured in MW (Megawatts). There are four categories of hydropower formulated on this basis: the micro plants (less than $0.1$ MW), the mini plants (less than $1$ MW), the small-scale plants (less than $10$ MW), and the large-scale plants (more than $10$ MW). In practice, one tends to use small-scale hydropower more loosely, to refer to all projects less than 10 MW. Still, a further qualification is sometimes required. For example, the authority to grant a license for a micro or mini plant has been delegated to the regional county councils since 2010, in an effort to reduce the queue of small-scale applications at the NVE.\footnote{See delegation letter from the Ministry of Petroleum and Energy, dated 07 December 2009, available at \url{http://www.nve.no} (accessed 24 August 2014). The county council is an elected regional government institution situated between the municipalities and the central government. There are 19 county councils in Norway as of 01 January 2015. They are comparatively less important than both the municipalities and the central government, but have several  responsibilities, particularly in relation to infrastructure, education and resource management. See generally \cite{berg15}.} The council's decision is based on a (simplified) assessment made by the regional office of the NVE. In addition, licenses for micro and mini plants may be granted even in watercourses that have protected status pursuant to environmental law.\footnote{See Decision no 240, Stortinget (2004-2005), St.prp.nr.75 (2003-2004) and Innst.S.nr.116 (2004-2005).}

For small-scale plants proper, the authority to grant a license is delegated to the NVE, with the Ministry serving as the instance of appeal.\footnote{See delegation of 19 December 2000, from the Ministry of Petroleum and Energy (FOR-2000-12-19-1705).} For large-scale plants, the granting authority is the King in Council, based on a recommendation from the Ministry.\footnote{See Directive of 15 December 2000, from the King in Council (FOR-2000-12-15-1270).} However, in practice, the decision is usually closely based on assessments and recommendations provided by the NVE.\footnote{For a detailed guide to the administrative process for large-scale applications, published by the NVE, see \cite{stokker10}.}

While the relevant licensing authority depends on the effect of the plant, the kind of license required depends on a different categorisation, relating to the level of planned water regulation, measured in natural horsepower. Here, there are three categories: run-of-river schemes  (less than $500$ natural horsepower), non-industrial regulations ($500 - 4000$ natural horsepower), and industrial regulations (more than $4000$ natural horsepower).\footnote{See \indexonly{wra17}\dni\cite[2]{wra17} and \indexonly{wra17}\dni\cite[1,2]{ica17}.}

Almost all hydropower schemes require a license pursuant to section 8 of the \cite{wra00}.\footnote{As mentioned in section \ref{sec:wra00}, the exceptions are very small schemes (usually mini or micro) that are deemed to be relatively uncontroversial. Such schemes only require a license pursuant to the \cite{pb08}.} For run-of-river schemes, no further licenses are required for the development itself, although an operating license pursuant to the \cite{ea90} is typically required for the electrical installations.\footnote{See \indexonly{ea90}\dni\cite[3-1]{ea90}.} For schemes involving a non-industrial regulation, an additional license pursuant to section 8 of the \cite{wra17} is required. Industrial regulation schemes require yet another license, pursuant to section 2 of the \cite{ica17}.

As is to be expected, the complexity of the licensing procedure tends to increase with the number of different licenses required. However, the licensing applications tend to be dealt with in parallel, so that all licenses are granted at the same time, following a unified assessment. In practice, when the \cite{wra17} applies, it structures the procedure as a whole, also those aspects that pertain to other licenses. 

In addition, yet another categorisation of hydropower schemes is used to determine the relevant application procedure. This categorisation is based on the annual production of the proposed plant, measured in GWh/year. There are three categories: simple schemes (less than $30$ GWh/year), intermediate schemes ($30 - 40$ GWh/year), and complicated schemes (more than $40$ GWh/year). As mentioned in section \ref{sec:wra17}, the most important rules in the \cite{wra17} applies to complicated schemes, regardless of whether or not the scheme involves a regulation.\footnote{See \indexonly{wra00}\dni\cite[19]{wra00}.} In addition, applications for such schemes must be accompanied by an impact assessment pursuant to section 14-6 of the \cite{pb08}.

This means that the applicant is required to organise a public hearing prior to submitting their formal application, to collect opinions on the project and provide an overview of benefits and negative effects of the plans, particularly as they relate to environmental concerns.\footnote{See Directive of 19 December 2014 (FOR-2014-12-19-1758), pursuant to the \indexonly{pb08}\dni\cite[1-2,14-6]{pb08}.} In practice, if an impact assessment is required this significantly increased the scope and complexity of the application processing.

For intermediate schemes that do not involve regulation, the rules in the \cite{wra17} do not apply. However, impact assessments {\it may} still be required.\footnote{See \cite[20]{stokker10}.} Here the threshold of 30 GWh/year has been set as an additional threshold by the NVE, who have been delegated authority to require impact assessments for hydropower projects even when these yield less than 40 GWh/year.\footnote{See Directive of 19 December 2014 (FOR-2014-12-19-1758).} For the intermediate schemes, NVE decides whether an impact assessment is required on a case-by-case basis. For simple schemes, on the other hand, impact assessments will not be required. Such schemes make up the core of what is described as small-scale hydropower in daily language.

The time from application to decision can vary widely, depending on the complexity of the case, the level of controversy it raises, and the priority it receives by the licensing authority. Usually, the assessment stage itself will last 1-3 years, sometimes longer.\footnote{See \cite[84-85]{nou129}.} While large-scale schemes involve more complicated procedures, they are also typically given higher priority than small-scale schemes. In recent years, following the surge of interest of small-scale development, a processing queue has formed at the NVE.\footnote{See \cite[84]{nou129}.} This means that small-scale applications typically have to wait a long time, sometimes several years, before the NVE begins processing them.\footnote{See \cite[84]{nou129}.}

%As I will discuss in more depth in the next chapter, the issue of expropriation is rarely given special attention during the application assessment. This is so even in cases when an application to expropriate waterfalls is submitted alongside the licensing applications. The issue of expropriation is rarely singled out for special treatment, at least not in cases of large-scale development. %Moreover, as mentioned in Section \ref{sec:hl}, an automatic right to expropriate follows from section 16 of the \cite{wra17}.

The applicant is expected to submit application notices for publication in local newspapers, and for larger projects there will typically be an information meeting arranged in the local area, where the applicant and the authorities appear side by side, presenting the plans and the licensing procedure respectively.\footnote{See \cite[23]{stokker10}.} For large-scale projects, it is also common for the applicant to distribute brochures widely in the local area. These procedural arrangements arguably reflect some concern for the interests of local populations. However, the procedure is organised in a way that can also create the impression that the applicant enjoys significant state-backing from the start.

Indeed, applicants not only communicate with locals in place of the authorities, they are also given responsibility for many material aspects of the assessment process, including the often crucial assessment of possible alternatives.\footnote{See \cite[24]{stokker10}.} This would seem to raise competency questions, particularly in cases where the owners themselves propose alternatives that the applicant hoping to expropriate will then assess on behalf of the government. However, the Supreme Court has not found any fault with this remarkable form of administrative subcontracting.\footnote{See \cite[51-55]{jorpeland11}.}

More generally, as shown in chapter \ref{chap:5}, the protection offered to waterfall owners is very limited. For instance, the government does not recognise a duty to notify these owners individually, to ensure that they are informed of what is at stake for them as owners of a very valuable resource. Rather, a generic letter is typically sent by the applicant to all affected private parties. The statement that private property ``will be expropriated'' unless a settlement is reached has also been observed.\footnote{In the case of \cite{sauda09}.}

After the hearing stage, the NVE will usually compile a final report along with a recommendation and send it to the interested parties for comments.\footnote{See \indexonly{wra17}\dni\cite[6]{wra17}.} It is established practice that local owners do {\it not} count as interested parties in this regard.\footnote{See \cite[46]{jorpeland11}.} Hence, while the municipalities and various environmental interest groups are informed of how the case progresses and asked to comment prior to the final decision, the owners must inquire on their own accord if they wish to be kept up to date on the application process.\footnote{In a written statement to the Supreme Court in the case of \cite{jorpeland11}, the director of the hydropower division of the Ministry pointed out that the documents would be made available on the web page of the NVE and that local owners had to ``look after their own interests''.}

In summary, the procedural framework surrounding licensing of hydropower development leaves local owners in a precarious position, especially when the applicants wish to expropriate their waterfalls. At the same time, the liberalisation of the electricity sector means that owners are in a far better position than before when it comes to developing hydropower themselves. This is discussed in more depth in the next section.
%Given that expropriation is often an automatic side-effect of a development license, this already suggests that legitimacy issues are likely to likely to arise when waterfalls are taken for hydropower. I return to this issue specifically in the next chapter. First, I will discuss market practices in more depth, focusing on the changes that resulted from the liberalisation reform of the early 1990s. 

\section{Hydropower in Practice}\label{sec:4:4}

The history of hydropower in Norway can be roughly divided into four stages. The first stage was the development that took place prior to 1909. During this time, private actors dominated, with public ownership playing a minor role.\footnote{See \cite{otprp61}.} Moreover, there were many private interests speculating in acquiring Norwegian waterfalls, anticipating the value that these would have for industrial development.\footnote{See \cite[30-31]{nou04}.}

After 1909, the introduction of licensing obligations and the rule of reversion made it much harder for private companied to acquire waterfalls that were suitable large-scale industrial development. At the same time, local municipalities began to invest in hydropower to provide electricity to its citizens, a service they were increasingly being obliged to provide.\footnote{See \cite{otprp61}.} This marked the start of the second stage of hydropower development, which saw the development of a more strictly regulated sector. However, this sector was also highly decentralised, for a large part dominated by local actors.

In fact, throughout the first half of the 20th century, most hydroelectric plants were small-scale plants that supplied local communities with electricity.\footnote{See \cite[11]{utbygd46}. This is a report from the water directorate published in 1946, showing that as of 31 December 1943, $97.8 \%$ of all hydroelectric plants in Norway were small-scale plants. However, these plants contributed only $28 \%$ of the total hydroelectric power installed at that time.} Moreover, as late as in 1943, $89 \%$ of all hydroelectric power stations in Norway were still private, many of which were mini and micro plants that were owned and operated by the local community.\footnote{See \cite[6]{utbygd46}. See also \cite[111]{hindrum94}.} However, many bigger plants were also under private ownership, and $57 \%$ of the total hydroelectric power available at this time was supplied by the private sector. 
%This clearly illustrates the importance of smaller, local initiatives, in the process of providing Norway with electricity, particularly in rural areas. Interestingly, while the micro and mini plants accounted for $72.9 \%$ of the total number of plants, they only accounted for $1.6 \%$ of the total electricity supply.\footnote{See \cite[7]{utbygd46}.}

By the end of 1943, $80 \%$ of the Norwegian population had access to electricity at home.\footnote{In rural areas, the corresponding figure was $70 \%$, see \cite[7]{utbygd46}.} Hence, the decentralised approach to hydropower development, based on private ownership and local control, had not been an impediment to the supply of electricity to most of the country's population.

However, the regulatory regime was soon to undergo a significant change, designed to facilitate industrial development and increased state control. This change came quite rapidly after the Second World War, when the central government began to invest heavily in hydropower, often to ensure economic development by subsidising the metallurgical industry.\footnote{See \cite[59-65]{thue96}.} This period saw increased marginalisation of small private electricity companies, as well as local owners.\footnote{At the same time, powerful (private) metallurgical interests benefited greatly, sometimes also at the expense of the general supply of electricity. See \cite[65-71]{thue96}.} Indeed, it was often demanded, as a condition for allowing local communities access to the national electricity grid, that local hydroelectric  plants had to be shut down.\footnote{See \cite[p.111]{hindrum94}.} During this time, the development of hydropower was seen as an important aspect of rebuilding the nation, a task carried out in the public interest, not primarily to supply the public with electricity, but rather to facilitate a specific kind of economic development that the central government regarded as desirable.\footnote{See \cite[59]{thue96}.}

The state-dominated system set up on this basis remained in place until the 1970s, when environmental concerns and discontent among local populations led to some reforms.\footnote{See \cite[71-75]{thue96}.} As the scale of development had grown significantly, new projects would tend to meet with significant opposition from various stakeholders, including environmental interest groups, local communities, as well as municipal and regional government institutions.\footnote{See \cite[71-72]{thue96}.} The typical response from the state was to introduce measures that sought to pacify the regional and municipal government opposition, which was considered more serious than opposition from local people and environmental groups. The standard approach was to grant an increased share of the financial benefit to local and regional institutions of government, to instil support for state-led development plans.\footnote{See \cite[73-76]{nilsen08}.} To some extent, this slowed down the centralisation process in the hydroelectric sector.\footnote{See \cite[85]{thue96}.} However, 
despite limiting the discontent among local power groups, high-profile controversies continued to arise, most notably the {\it Alta} case discussed in the next chapter.

The fourth stage of hydropower development began in 1990 after the passage of the \cite{ea90}. The liberalisation that followed saw the transformation of the hydropower sector into a commercial market, based on profit-maximising and competition. As a result, the structure of decentralised management withered away further, as many municipality companies were either bought up by more commercially aggressive actors or forced to merge and change their business practices in order to remain competitive.\footnote{See \cite[583]{bibow03} (commenting on the increased concentration of power on the electricity market, following acquisitions and mergers after 1990).} At the same time, a new decentralised force emerged in the sector, in the form of local owner-led projects.\footnote{See section \ref{sec:4:4} below.}

The core idea behind the \cite{ea90} was that the electricity sector should be restructured in such a way that production and sale of electricity, activities deemed suitable for market regulation, would be kept organisationally separate from electricity distribution over the national grid, a natural monopoly. However, the Act itself does not explain in any depth how this is to be achieved. In practice, the divide has not been strictly implemented. Most of the large energy companies in Norway continue to maintain interests in both distribution, production and sale of electricity, a phenomenon known as ``vertical integration''.\footnote{See \cite[580-583]{bibow03}.} In fact, the degree of vertical integration in the electricity sector initially increased after the passage of the \cite{ea90}.\footnote{See \cite[583]{bibow03}.}

\noo{ To some extent, the water authorities have responded to this by making use of their authority to give organisational directives when they grant distribution licenses.\footnote{See \indexonly{ea90}\dni\cite[4-1]{ea90}, para 2, no 1.} For instance, electricity companies are now required to keep separate accounts for production, distribution and sale of electricity.\footnote{See Directive of 11 March 1999 (FOR-1999-03-11-302), s 4-4 a and s 2-6, issued by the NVE pursuant to Directive of 7 December 1990 (FOR-1990-12-07-959), s 9-1, pursuant to \indexonly{ea90}\dni\cite[10-6]{ea90}.} It is also required that transactions across these functional divides are clearly marked, and that they are based on market prices.\footnote{See Directive of 11 March 1999 (FOR-1999-03-11-302), s 2-8.}}

The water authorities responded to this by accepting increased concentration of ownership, while also demanding that the distribution activities were kept organisationally separate from other activities, for instance through the establishment of a special subsidiary company.\footnote{See \cite[581-582]{bibow03}.} Typically, a conglomerate structure is used, with a single parent company that controls both the distribution company, the production company and the sales company. Indeed, this model has now been implemented by most of the large energy companies in Norway.\footcite[582]{bibow03}

\noo{It seems unclear whether this approach really achieves the stated objective. By adopting the conglomerate model of organisation, the major players on the market have successfully gained control over a larger share of both the production and distribution facilities for electricity. Hence, these actors effectively control the core infrastructure that makes up the backbone of the Norwegian electricity sector. The {\it intention} is that monopoly power should only be exercised with respect to the distribution grid on non-discriminatory terms. But is this realistic when the conglomerate controlling the grid operator has significant stakes also in production and the trade of electricity?

This question calls for a separate study, and }
The extent to which this is an adequate response to increased concentration of power in the electricity sector will not be addressed in any depth here. However, I will direct attention at one aspect that arises with particular urgency for small-scale development of hydropower, concerning access to the grid. It is quite common, in particular, that small-scale projects remain unrealised because the grid is regarded to lack sufficient capacity to accommodate new electricity.\footnote{See, e.g., \cite[84,161-162]{nou129}.}

Following an amendment of the Energy Act in 2009, grid companies are now obliged to facilitate access for producers, even when this necessitates new investments.\footnote{See Act no 105 of 19 June 2009 regarding changes in the \cite{ea90}.} However, the energy producer seeking access is typically required to reimburse the grid company for the cost of new investments, as determined in the first instance by the grid company itself (the NVE serves a supervisory function).\footnote{See Directive of 7 December 1990 (FOR-1990-12-07-959), s 3-4.} In addition, grid companies may still deny access in cases when the needed investments are not ``socio-economically rational''.\footnote{See \indexonly{ea90}\dni\cite[3-4]{ea90}. The authority to decide whether this requirement is fulfilled is vested with the Ministry.}

Often, the relevant grid company will be a sister company of an energy producer operating in direct competition with the company seeking access. This can raise questions about the impartiality of the assessments carried out by the grid company. In expropriation cases, this becomes an issue particularly in relation to the assessment of the cost of undertaking an alternative development scheme.\footnote{This assessment is often crucial, because it provides information about the value of the development potential that the owners stand to loose.} Riparian owners are rarely pleased when they realise that the expropriating party is part of the same conglomerate as the grid company that estimates the grid connection costs associated with owner-led development.\footnote{See, e.g., \cite{smibelg15}.}

%It has been pointed out, in particular, that the practical consequence of liberalization has been that the local accountability of the electricity sector has been lost, both organizationally and politically.\footnote{See \cite{agnell11}. 
Meanwhile, the market-orientation of the electricity sector has reduced the level of political control and accountability. Today, a management model based on economic rationality and expert-rule has become dominant. According to Brekke and Sataøen, this serves to set the reform that took place in Norway apart from similar energy reforms in Sweden and the UK.\footnote{See \cite{brekke12}.} Moreover, Brekke and Sataøen argue that this has resulted in a lack of legitimacy that has been a significant contributory cause of recent national-scale controversies, particularly with regards to the development of the national grid.\footnote{The most serious case so far is that of {\it Sima - Samnanger}, concerning a new distribution line that will cut through the area known as {\it Hardanger}, a scenic part of south-western Norway. The plans met with significant resistance at both the national and the local level, but the government pushed ahead, leading to confrontations that also involved some acts of civil disobedience. See \cite[22-23]{brekke12}.}

At the same time, the growth of the small-scale hydropower sector gives local communities a new voice, as market participants, thereby acting as a counterweight to centralisation and expert-rule. Since the mid- to late 1990s, the small-scale sector has grown significantly. In a recent report, the potential for profitable small-scale hydropower projects was estimated to be around 20 TWh per year.\footnote{See \cite{aanesland09}. For comparison, suggesting the scale of this potential, I mention that the total consumption of electricity in Norway in 2013 amounted to about 120 TWh, see \cite{statistikk13}. According to the government, about one third of the remaining potential for hydropower in Norway, measured in annual energy output, will come from small-scale projects. See \cite[231]{nou129}.} On this basis, the authors of the report estimate that the total present-day value of all waterfalls suitable for small-scale hydropower is about 35 billion Norwegian kroner, i.e., about 3.5 billion pounds.\footnote{See \cite[1]{aanesland09}.} This calculation is based on a model where the waterfalls are exploited in cooperation with an external commercial company, inspired by existing agreements between owners and the limited company {\it Småkraft AS}. Hence, the calculation might be an underestimate of what small-scale hydropower could represent for local communities if they remain in charge of development themselves.

Small-scale hydropower has become socially and political significant in Norway. In the report mentioned above, it is estimated that the value of rivers and waterfalls amount to just under 50 \% of the total equity in Norwegian agriculture.\footcite[1]{aanesland09} Moreover, hydropower is increasingly seen as a possibility for declining regions to counter depopulation and poverty. In some communities, small-scale hydropower is the only growth industry. For these communities, pursuing hydropower development at the local level also provides a way to regain some autonomy with respect to how local natural resources should be managed. Hence, small-scale hydropower takes on great political and social importance, not just for the owners of waterfalls, but for the community as a whole.

For an example of a community where small-scale hydropower has played such a role, I point to Gloppen, a municipality in the county of Sogn og Fjordane, in the western part of Norway. 19 hydropower schemes have already been carried out, all except one by local owners, amounting to a total production of over 250 GWh per year. This prompted the mayor to comment that ``small scale hydro-power is in our blood''.\footnote{See \cite{starheim12}.} When interviewed, he also directed attention at the fact that hydropower had many positive ripple effects, since it significantly increased local investment in other industries, particularly agriculture, which had been severely on the decline.

To achieve such effects, it is important to organise development in an appropriate manner. Moreover, to explain how waterfalls came to be as valuable as they are today, it is crucial to direct attention to the way in which waterfall owners initially asserted themselves on the market. In the following, I do this by giving an in-depth presentation of an early model for local involvement in hydropower development, presented at a seminar in 1996.\footnote{See \cite{dyrkolbotn96}.} This model contains an early expression of several ideas that would prove influential to the development of the small-scale hydropower sector.

%However, certain other aspects of the model have not been widely adopted. These are aspects that pertain to the balance of power between owners and cooperating developers, as well as the relationship that should be established with larger communities of non-owners, including environmental groups and other water users. Hence, considering the model in some depth, and assessing its impact, will allow me to shed light on desirable social functions of waterfall ownership, and the extent to which such functions are fulfilled on the market today.

\section{{\it Nordhordlandsmodellen}}\label{sec:4:5}

In five brief points, the {\it Nordhordlandsmodellen} sets out a framework for cooperation between waterfall owners, professional hydroelectricity companies, local communities, and society as a whole.\footnote{See \cite{dyrkolbotn96}. The model was the result of a collaboration between Otto Dyrkolbotn, a farmer and a lawyer, and Arne Steen, the director of {\it Nordhordland Kraftlag}, a municipality-owned energy company.} 

The first point makes clear that the aim of cooperation should be to ensure local ownership and control: external interests should never be allowed to hold more than 50 \% of the shares in the development company. If the company is organised as a limited liability enterprise, then the plan stipulates that local residents -- not necessarily owners -- are to be given a right of preemption in the event that shares come up for sale. %The possibility of organising the development company as a local cooperative is also mentioned.\footnote{References needed.}

The second point of the model sets out a method for valuing the riparian rights prior to development. It stipulates that the appraisal should reflect the real value of such rights, normally estimated on the basis of lease capitalisation. More concretely, the valuation should be based on the premise that the riparian owners will be entitled to rent based on the level of annual production in the planned hydropower project. Then, for the purpose of appraisal, the expected rent per annum is capitalised to find the present value of the riparian rights, relative to the development project in question.\footnote{This approach stands in stark contrast to the earlier valuation method used in the electricity sector, which relied on a purely theoretical assessment based on the aforementioned notion of a natural horsepower. See \cite{dyrkolbotn15,hellandsfoss97}.}

After such a value has been calculated, the model stipulates that owners are to be given a choice of either leasing out their water rights to receive rent, or to use the capitalised value of (part of) this rent as equity to acquire shares in the development company. The third point in the model then offers a clarification, by stating that the development company should not in any event acquire ownership of riparian rights, but only a time-limited right of use. After 25-35 years, this usufruct should fall away and the waterfall should revert back to the owners of the surrounding land, free of charge. This is the proposed rule even in cases when the landowners themselves initially control the majority of the shares in the development company. Hence, the rule places a limit on alienation; no separation of water rights from land rights is allowed to last for more than 35 years. The model demonstrates the commercial viability of this organisational model by pointing to a concrete municipality-owned energy company that has stated its willingness to cooperate with owners on such terms, to help with financing and share the risk.\footnote{The company in question is Nordhordland Kraftlag, where one of the authors of the model, Arne Steen, was a director.} 

Following up on this organisational blueprint, the fourth and fifth points of the model describe the intended role of the local development company in society, by stressing the relationship between hydropower and other interests and potential uses of the affected river. Importantly, the model stipulates that potential developers should be willing to take on formal obligations towards other user groups. Moreover, obligations should not only be negatively defined, as duties to minimise or avoid harms. Positive obligations should also be introduced, such as duties to improve other qualities of the river system, and to engage in active cooperation with other users.

The overall aim, it is made clear, is to ensure sustainable management of the river system as a whole. Interestingly, the model predicts that active local ownership will achieve more in this regard than what can be achieved through governmental regulation alone. This claim is illustrated by a concrete example of a case in which the local owners decided to pursue a scheme that was less environmentally invasive than the project endorsed by the water authorities.\footnote{Today, this project has become Svartdalen Kraftverk, finalised in 2006. It produces 30 GWh annually, enough electricity for about 1500 households, see \cite{wikisvart}.}

The model goes on to emphasise the need for integrated processes of resource planning and decision-making, to ensure that hydropower development is not approached as an isolated economic and environmental concern, but looked at in a broader social and political context. To achieve this, it is argued that local communities need to play an important role in the management of water resources. Another concrete example follows, regarding {\it Romarheimsvassdraget}, a river system in the municipality of Lindås, in the county of Hordaland.

This river system was originally intended for large-scale development undertaken by BKK AS, without the participation of local owners.\footnote{BKK AS is one of the 15 biggest hydropower companies in Norway, and would later also purchase Nordhordland Kraftlag.} The project would involve a total of three river systems, such that the water from {\it Romarheimselva} and another river would be diverted to a neighbouring municipality for hydropower development there. The local owners argued against these plans by proposing a number of smaller development schemes. Eventually, they were successful, as the NVE agreed to endorse an alternative consisting of 7 distinct run-of-river projects to be undertaken by local owners.\footnote{See \cite{vann25}.}

It is important to note that when {\it Nordhordlandsmodellen} was formulated, owner-led development of hydropower was still a recent phenomenon, driven forward by individual owners and local groups that saw the potential and had enough know-how to get organised. Later, commercial companies emerged that specialised in cooperating with local owners.\footnote{See, e.g., \cite{larsen06}.} This has made it relatively easy to initiate a process of owner-led development. Moreover, owners are often approached by interested commercial actors who wish to cooperate with them. Most of them rely on cooperation on terms that reflect the main ideas expressed in the first three points of {\it Nordhordlandsmodellen}.

However, several adjustments have become standard, and these systematically benefit the external partner: the requirement that locals should at all times control a majority of the shares is dropped, the period of usufruct is typically longer than 35 years, the reversion to the landowners after this time is made conditional on payment for machines and installations, and no preemption rights are granted to local residents.\footnote{See generally \cite{hauge15}.} However, the core idea that riparian rights are to be valued based on a capitalisation of future rent is accepted. This means, in turn, that local owners rarely need to raise any additional capital to acquire shares in the development company. Moreover, the rent itself can become a significant source of income.

There are two main approaches to calculating this rent. The first approach, introduced already in {\it Nordhordlandsmodellen}, specifies the rent as a percentage of the gross income from sale of electricity, today often around 10-20 \%.\footnote{Source: contracts presented to the court in \cite{sauda09} (available from the author upon request). See also \cite[55-57]{hauge15}.} In this way, passive owners need not take on any risk related to the performance of the hydropower company. The second approach has been developed by the company Småkraft AS, which is now the leading market actor specialising in cooperation with local owners.\footnote{It is owned by several large-scale actors on the energy market, see \url{www.smaakraft.no}.} According to their model, riparian owners are paid a share of the annual {\it surplus} from hydropower generation.\footnote{See \cite[57-60]{hauge15} (also discussing variants of this contractual idea, based on how the surplus is actually defined in the contract).}

This share is usually higher than the rent payable based on the net income; often, the owners are entitled to $50 \%$ of the profit.\footnote{Source: contracts presented to the court in \cite{sauda09} (available from the author upon request). See also \cite[58]{hauge15}.} Hence, if the project is a success, the riparian owners might be better compensated. However, the owners have to accept some risks as though they were shareholders, and they do so even though they might not have much of a say in how the company is run.\footnote{To limit the risk for owners, companies such as Småkraft AS also operates a system of ``guaranteed'' rent, but this rent is usually quite a lot less than what the owners could expect from an agreement based solely on rent based on gross income. Source: contracts presented to the court in \cite{sauda09} (available from the author upon request).}

To illustrate the financial scale of the rent agreements that have now become standard, let us consider a typical small-scale hydropower plant that produces 10 GWh annually. With an electricity price of NOK 0.3 per KWh, this gives the hydropower plant an annual gross income of NOK 3 million. If the rent payable is 20 \%, the waterfall owners will receive NOK 600 000 annually, approximately GBP 60 000. This is much more than what the owners could hope to receive according to the traditional method for calculating compensation following expropriation.\footnote{For an example based on comparing two concrete cases, see the discussion in chapter \ref{chap:5}, section \ref{sec:5:4:1}. For the compensation issue generally, see \cite{hauge15,dyrkolbotn15a}.}

%By contrast, if the rights were expropriated, the traditional method of calculating compensation would be unlikely to result in more than NOK 600 000 as a {\it one time payment} for a waterfall that yields 10 GWh per annum.\footnote{For further details on the compensation issue, see \cite{dyrkolbotn14,dyrkolbotn15,dyrkolbotn15a}. Sometimes, the difference in valuation would be even greater, since the natural horsepower of a development project is highly sensitive to the level of regulation of the waterfall, much more so than the value of the development. For an demonstration of how this affected compensation according to the natural horsepower method, one may consider the case \cite{hellandsfoss97}, which went to the Supreme Court. Here the owners were paid just over NOK 1 million for a waterfall that would yield 100 GWh per annum.}

All in all, the financial consequences of the ideas expressed in {\it Norhordlandsmodellen} have been dramatic. However, the latter two points of the model, addressing the importance of holistic and inclusive management of river systems, have not had the same degree of influence. In the next section, I address what appears to be a negative consequence of this for the small-scale industry, threatening to undermine its status as a sustainable alternative to large-scale exploitation.

\section{The Future of Hydropower}\label{sec:4:6}

In recent years, there has been a growing tension between the small-scale hydropower sector and environmental groups. There is talk of a brewing ``hydropower battle'', as environmentalists grow increasingly critical of what they regard as predatory practices.\footnote{See \cite{haltbrekken12}.}
Reports on small-scale producers who are alleged to have violated environmental regulations help fuel the negative impression of the industry.\footnote{In 2010, the NVE conducted randomised inspections and announced that 4 out of 5 mini and micro plants operated in violation of regulations pertaining to the amount of water they may use at any given time. See \cite{ulovlig10}. In the largest newspaper in Norway, this was reported under the heading that four out of five small-scale plants break the law, see \cite{ulovlig10b}. This is misleading, since mini and micro plants are distinct from small-scale plants proper. Most importantly, the former kinds of plants do not usually require a sector-specific development license. Because of this, it also seems plausible that the reported violations might in large part be due to a lack of knowledge and professionalism, not predation. I remark that questions later emerged regarding the accuracy of the report itself. Apparently, one of the plants that was reported to have violated regulations did not even exist, see \cite{tvilsom10}.}

On the regulatory side, the water authorities have now adopted much stricter procedures to assess licenses for small-scale hydropower.\footnote{See \cite{lie12}.} In addition, different planning routines have been adopted to ensure that small-scale schemes are no longer considered individually, but in so-called ``packages'', collecting together applications from the same area. As a consequence of these changes, the number of rejected small-scale applications have increased dramatically in recent years.\footnote{In 2013, the number of rejections tripled compared to previous years, while the number of accepted applications remained stable. See \cite{sunde14b}.}

At the same time, powerful market actors who favour a traditional mode of exploitation have seized the opportunity to lobby more aggressively against small-scale hydropower, in favour of large-scale projects.\footnote{See, e.g., \cite{alexandersen14}.} Such projects, they argue, are preferable also from an environmental point of view. In recent years, this argument has proven influential in many quarters, particularly among state agencies, such as the NVE and the Norwegian Environmental Agency.\footnote{See \cite{nilsen11}.} It has also been claimed that this perspective is backed up by research done on environmental effects of small-scale and large-scale projects.\footnote{See generally \cite{bakken12,bakken14}.}

The core environmental argument against small-scale solutions has a very simply structure: small-scale plants indirectly affect a greater total area of land per energy unit produced, therefore they are considered more intrusive than large-scale schemes.\footcite[96-99]{bakken14} The stated premise of this reasoning is no doubt correct, since several small-scale plants, at many different locations, are required to match the energy produced by a single larger plant. However, this quantitative observation has no bearing on the issue of how small-scale plants qualitatively effect the surrounding environment, compared to large-scale projects. In particular, the parameters used to compare small-scale and large-scale developments tend to be defined in terms of generic buffer zones that do not take into account differences in the severity of different kinds of environmental intrusions. For instance, as long as both installations are observable by passers by, a small cabin with a turbine inside is considered to have the same ``scenic impact'' as an imposing concrete dam that stretches out for 100 meters and significantly distorts the water level in a lake.\footnote{See \cite[95]{bakken14}.}

\noo{The only buffer zone that is not defined in this way is the {\it scenic} buffer, the area from which some installation can be seen. Here the model takes into account that a large installation should be assessed using a larger buffer zone than a small one, since the former is visible over a greater area. But even for this parameter, no distinction is made based on the actual visual impression; a large dam that dries up a river and makes it possible to regulate the water level in a lake by several meters counts the same as a small cabin with a generator inside, as long as both can be seen.\footnote{See \cite[95]{bakken14}.} For the other parameters, the data analysis is even more dubious, since the buffers are set uniformly according to general rules of thumb.\footnote{See \cite[95]{bakken14}.} For instance, a conflict with a threatened species is assumed to arise whenever a technical installation occurs within a certain distance from its natural habitat.\footnote{See \cite[95]{bakken14}.} Importantly, nothing is said about the severity of conflict, and no distinction is made between a minor installation and a massive disturbance.}

Despite the apparent lack of qualitative arguments, the idea that large-scale development is better for the environment now appears to be gaining ground in Norway. This represents a complete reversal compared to the political narrative that has dominated for the last 15-20 years. Indeed, the merits of small-scale development was strongly emphasised by political leaders around the turn of the century. In his New Year's speech 01 January 2001, the Prime Minister went as far as to declare that the age of large-scale development was over.\footnote{See, e.g., \cite[34]{haltbrekken12}.} The same phrase was then repeated in the policy platforms of two successive national governments, in 2005 and 2009 respectively.\footnote{See the ``Soria Moria'' declaration from 2005, p 57, and ``Soria Moria II'', from 2009, p 52 (available at \url{www.regjeringen.no}).}

However, as administrative practices and case law on hydropower shows, the end of large-scale exploitation has proved impossible to implement. Despite being official policy at the highest level of government for almost 15 years, large-scale development interests continue to dominate in the hydropower sector.\footnote{I believe the material presented in this thesis warrants making this claim. Moreover, it is underscored by the two recent Supreme Court decisions in \cite{jorpeland11} and \cite{otra13}.} Interestingly, the leading national politicians are now changing their position as well.\footnote{See \cite{liemin14} (reporting on recent public statements made by the Minister of Petroleum and Energy in support of large-scale development).} Arguably, this demonstrates how the politicians have responded to pressure from high-ranking bureaucrats and large energy companies.\footnote{In addition to their environmental arguments, these actors also rely on more familiar arguments in favour of large-scale development, especially the idea that large-scale development is more efficient. See \cite{lie12}. For a contrasting view on efficiency, emphasising the efficiency benefits associated with small-scale development and a decentralised approach to hydropower, see \cite[5]{inn101}.}

The political shift observed at present is likely to result in a further weakening of property and the rights of local communities. For example, it provides indirect political legitimacy to the NVE, who  pursue an explicit policy of prioritising applications for large-scale projects when these come into conflict with small-scale schemes in the same rivers.\footnote{See \cite[3]{nve12}. See also \cite{lie12}.} Hence, the NVE is likely to refuse to consider applications from owners as long as there are applications pending that might result in the expropriation of their property.\footnote{For a concrete example of this, see \cite{smibelg15}.}

At the same time, the small-scale industry itself has occasionally sought to undermine property rights, possibly in an effort to mimic the successes of their large-scale competitors. The industry has argued, in particular, that expropriation should be made more easily available as a tool for small-scale developers and owners who wish to take property from reluctant neighbours.\footnote{See \cite{brekken07,brekken08}. The articles are written by a leading Norwegian energy lawyer, apparently in his capacity as legal representative of ``Småkraftforeningen'', an interest organisation for small-scale hydropower.} This argument rests on a peculiar form of anti-discrimination reasoning; as long as large-scale developers are allowed to take property by force, small-scale developers should be allowed to do the same. In a world where takings are endemic, this might make some sense. However, it is hardly an attitude that helps the small-scale industry preserve its image as the more sustainable hydropower option.

At the same time, the industry is beginning to struggle financially because the price of electricity has been much lower in recent years than what had previously been forecast.\footnote{See \cite{sunde14}.} Moreover, it has become clear that some of the investors on the market have engaged in speculative practices, by aggressively entering into agreements with local owners, without carrying out much hydropower development.\footnote{See \cite{endresen14}.}

These critical remarks should not detract from the fact that the growth in small-scale hydropower has led to dramatically increased benefit sharing with many local owners of rivers and waterfalls. However, recent events indicate that it is inappropriate to look at this development in isolation from other concerns. When assessing the future of small-scale hydropower and local property rights to waterfalls, it seems important to also take into account the broader societal consequences of new commercial practices. If one fails in this regard, the pernicious image of owners as socially passive ``profit-maximisers'' gains a firmer hold both on the political and the legal narrative. The negative consequences this can have for property as an institution are already apparent in Norway, as I will argue in the next chapter. 

More generally, recent developments in the hydropower industry illustrate that an entitlements-based perspective on waterfalls is inappropriate, since local ownership is meant to facilitate sustainable management first, and profit-seeking only second. This insight is also strongly implicit in {\it Nordhordlandsmodellen}. However, as the current debate is evolving, it seems to be at risk of disappearing from view.

\noo{To counter this, I believe the social function view of property must be developed further, so that concrete policy recommendations can be formulated on its basis. The aim, I believe, should be to arrive at frameworks for participatory decision-making regarding hydropower that allows local owners and communities to contribute constructively when society desires commercial development based on  their water rights.

I return to this issue in chapter \ref{chap:6}, where I argue that the Norwegian institution of land consolidation can be used to achieve this. First, I will zoom in on the issue of expropriation, where the mechanisms identified in this section often lead to concrete legal disputes. This will bring into focus important issues surrounding the status of economic development takings under Norwegian law.
}

\section{Conclusion}\label{sec:4:7}

Water resources have been, and still are, very important to Norway as a nation. Not only does the energy of streaming water provide electricity to people and industries, it also provides a source of profit, prestige and power to those who harness it. Historically, many rural communities in Norway benefited greatly from this, as it was they who managed local water resources.

Plainly, they did rather well. By the end of 1943, at a time when small-scale plants still outnumbered large-scale plants 45 to 1, $80 \%$ of the population had domestic access to electricity. The government, especially local governments, also felt responsible for the supply of electricity to the public, but they generally assumed this responsibility without encroaching on local populations who wished to manage their own resources.

As discussed in this chapter, the situation changed dramatically after the Second World War, when the government, especially the central government, assumed more direct control over the nation's water resources. This led to a situation where local owners became increasingly marginalised. In recent years, there has been a partial reversal of this trend, as the liberalisation of the electricity market has enabled local owners and communities to take part in hydropower development once again.

The result has been a growing tension between large-scale and small-scale development, which in turn corresponds to a tension between the owners of waterfalls and the energy companies that wish to expropriate. In the next chapter I will explore this corresponding tension in more depth, as I investigate the rules and practices relating to expropriation for hydropower development.
%\chapter{Taking Waterfalls}\label{chap:5}

\section{Introduction}\label{sec:5:1}

The Norwegian water authorities have extensive powers to take waterfalls for hydropower development. However, they rarely need to reflect on this power, not even when they use it. The reason is that expropriation tends to be an {\it automatic} consequence of a development license; those who obtain a license to develop a large-scale hydropower plant almost always obtain also a license to expropriate the private property rights they require for this purpose.

In some cases, this follows from section 16 of the \cite{wra17}, which gives license holders a right to expropriate all property rights needed for the development in question. However, even outside the scope of these provisions, the same approach to expropriation tends to be adopted. Specifically, the authorities adhere to the presumption that whenever a license to undertake large-scale development should be granted, then so should a license to expropriate.\footnote{The leader of the hydropower licensing division of the NVE expressed this presumption in \cite{flatby08} (noting also that the same presumption is not applied for small-scale projects, e.g., when some owners wish to expropriate from neighbours who oppose development).}

The expropriation presumption has remained in place even though the regulatory and economic context of riparian expropriation has changed dramatically after the liberalisation of the electricity sector.
This is significant, especially due to how licensing cases are processed. As discussed in the previous chapter, the administrative licensing assessment tends to focus on the environmental consequences of development, with little attention devoted to how the loss of property rights affects the owners and their local communities. This is so even though a license to develop is in effect also a license to expropriate. How did this system come about, and where does it leave local owners whose waterfalls are targeted by large-scale proposals? This chapter addresses these two questions in depth. 

First, the history of the law is presented. This will serve to demonstrate that the current state of affairs developed gradually from a pre-industrial starting point where expropriation of waterfalls was generally not permitted. Further to this, the chapter discusses more recent changes, specifically the changes in the expropriation regime that were implemented following liberalisation of the electricity sector.

To make expropriation available as a tool for commercial companies, the earlier rules had to be modified. Specifically, public interest requirements had to be relaxed and limitations on private-to-private transfers had to be abrogated. The manner in which this was achieved, with only minimal parliamentary involvement, is in itself worth noting when addressing the legitimacy of current practices.

After the historical assessment, the chapter illustrates how the water authorities and the courts interpret and apply the rules currently in place. Specifically, I give a detailed presentation of the recent Supreme Court case of {\it Jørpeland}.\footnote{See \cite{jorpeland11} (I mention that I acted as legal counsel for the owners in this case).} This case demonstrates that the standing of owners is very weak under administrative law, a result of how the expropriation issue is overshadowed by the licensing question.

%More generally, {\it Jørpeland} and other recent cases suggest that the Supreme Court adhere to a very narrow perspective on the meaning of property protection, taking it to be an issue that begins and ends with the question of compensation. In this regard, owners were initially able to make some progress towards a more equitably level of compensation, but as this chapter shows, the early progress made on this point is likely to be reversed following the Supreme Court decision in the case of {\it Otra II}.\footnote{See \cite{otra13}.}

I conclude this chapter with a more overarching assessment based on the theoretical framework presented in Part I of the thesis, to shed further light on the legitimacy of rules and practices surrounding takings of waterfalls in Norway. This culminates in an argument suggesting that the current system systematically produces takings that fail the Gray test presented in chapter \ref{chap:2}. %This sets the stage for the final chapter, where I consider land consolidation as a legitimacy-enhancing alternative to expropriation for hydropower development.

\section{Norwegian Expropriation Law: A Brief Overview}\label{sec:5:2}

As mentioned in chapter \ref{chap:2}, the right to property is entrenched in section 105 of the Norwegian Constitution.\indexonly{grunnloven14} There it is made clear that when property is taken for public use, full compensation is to be paid to the owner. The formulation bears a close resemblance to the formulation of the US takings clause in the Fifth Amendment.\footnote{The Norwegian clause can be translated as follows: ``If the needs of the state require that any person shall surrender their movable or immovable property for public use, they shall receive full compensation from the Treasury.''. I remark that the translation provided by the parliamentary administration is plainly  incorrect. There the ``needs of the state'' have been replaced by ``the welfare of the state'', without any basis in the Norwegian text (the Norwegian expression used is ``statens tarv'', literally the state's need or, it might be argued, the state's interest). See \cite{grunnlovsjuks}.} However, there is no active public use debate in Norway. The meaning of public use is hardly ever discussed by the courts, and according to legal scholars, the public use formulation places no limit at all on the state's authority to expropriate.\footnote{See \cite[249]{aall04}; \cite{sauda09}.}

However, it is a rule of unwritten constitutional law that administrative decisions which affect the rights of individuals can only be carried out when they are positively authorised by law.\footnote{See generally \cite{hogberg11}.} Moreover, the Constitution is not understood as providing an authority for the state to expropriate. It merely expresses the presupposition that expropriation is possible.\footnote{See, e.g., \cite[6]{fleischer86}.} Hence, when applying eminent domain, the government needs to justify this on the basis of a more specific authorising provision. 

Historically, there was no general act relating to expropriation and a range of different acts authorised the executive to expropriate for specific purposes such as roads, public buildings, and schools.\footnote{See \cite[11-12]{nut54}.} Today, many of these authorities have been broadened and included in the \cite{ea59}. After an amendment in 2001, this Act includes an authority for the government to authorise expropriation of property and use rights in order to facilitate hydropower production.\indexonly{ea59}\dni\footcite[2 no 51]{ea59} This is understood to include the authority to expropriate waterfalls.\footnote{See, e.g., 
\cite{sauda07}.}

According to the \cite{ea59}, expropriation can only be authorised if the benefits undoubtedly outweigh the harms, as determined following a discretionary assessment.\footnote{\indexonly{ea59}\dni\cite[2]{ea59}.} Formally, the authorising authority lies with the King in Council. However, this authority can be delegated to ministries or other state bodies that the King in Council can instruct.\footnote{\indexonly{ea59}\dni\cite[5]{ea59}.} The compensation to the owner is determined following a judicial procedure administered by the so-called appraisal courts.\footnote{See the \indexonly{ea59}\dni\cite[2]{ea59}.} This is the name given to the regular civil courts when they hear appraisal cases, observing the special procedure set out in the \cite{aa17}. 

The appraisal procedure emphasises the importance of factual assessment and lay discretion (the appraisal court typically sits with four lay judges).\footnote{See \indexonly{aa17}\dni\cite[11-12]{aa17}.} In addition, there are special rules regarding costs, indicating that the expropriating party is usually required to pay for the procedure, including the owners' legal expenses.\footnote{\indexonly{aa17}\dni\cite[54]{aa17}.} In other regards, the appraisal procedure resembles a typical adversarial process before a civil court.\footnote{For a more in-depth discussion, see \cite[382-384]{dyrkolbotn15b}.}

The \cite{ea59} states that unless the King in Council decides otherwise, expropriation orders may only be granted to state or municipality bodies. This is formulated as a limiting principle, but in effect it serves as a general authorisation for the executive to decide, without parliamentary involvement, that a larger class of legal persons may be granted expropriation licenses. For many purposes, directives have been issued that extend the class of possible beneficiaries to any legal person, including companies operating for profit. In 2001, such a directive was issued for the authority to expropriate in favour of hydropower production.\footnote{See Directive no 391 of 06 April 2001.} 

In addition to providing a general authority for expropriation, the \cite{ea59} also contains several procedural rules. These are collected in chapter 3 of the Act. Here the Act sets out minimal requirements as to what an application for an expropriation license must include: it should make clear who will be affected, how the property is to be used, and what the purpose of acquisition is.\footnote{\indexonly{ea59}\dni\cite[11]{ea59}.} In addition, the Act requires the applicant to specify exactly what property they require, and to include information about the type of property in question and the current use that is made of it.

The Act also stipulates that owners must be notified, and that every owner should be given individual notice, although this obligation is relaxed when it is ``unreasonable difficult'' to fulfil.\footnote{\indexonly{ea59}\dni\cite[12]{ea59}, para 2.} In such cases, it is sufficient that the documents of the case are made available at a suitable place in the local area. In addition, a public announcement must then be made in the Norwegian Official Journal as well as in two widely read local newspapers.\footnote{\indexonly{ea59}\dni\cite[12]{ea59}.}

The licensing authority is required to ensure that the facts of the case are clarified to the ``greatest extent possible''.\footnote{The Norwegian expression is ``best råd er'', which literally means ``best possible way'', see the \indexonly{ea59}\dni\cite[12]{ea59}, para 2.} This formulation seems very strict, but is highly non-specific. In practice, the level of scrutiny given to the expropriation question under Norwegian law varies greatly depending on sector-specific administrative practices.\footnote{See \cite[380-381]{dyrkolbotn15}.} Moreover, established practice from several fields, including the hydropower sector, suggests that when expropriation takes place to implement a public plan or a licensed development, little attention is devoted to expropriation as a special issue.\footnote{For zoning plans, see \cite{namsos98,bo99}. For hydropower, see \cite{jorpeland11}.}

The applicant must cover costs incurred by owners in relation to a pending application for expropriation.\footnote{\indexonly{ea59}\dni\cite[15]{ea59}.} The exact formulation is that the applicant is obliged to cover the costs that ``the rules in this chapter carry with them''. That is, the applicant is obliged to cover the costs that are related to the owners' rights pursuant to chapter 3 of the \cite{ea59}. In practice, an owner will be denied costs if the competent authority takes the view that they are unreasonable or disproportionate to their interests in the case.\footnote{If the case progresses to an appraisal dispute, the competent authority to decide on costs is the appraisal court. Otherwise, the decision is left with the executive. See the \indexonly{ea59}\dni\cite[15]{ea59}.} Finally, the decision to grant an expropriation license must be justified, and the parties should be informed of the reasons for the decision.\footnote{\indexonly{ea59}\dni\cite[12]{ea59}, para 3.}

In addition to the procedural rules in the \cite{ea59}, the rules of the \cite{paa67} also apply in expropriation cases. These rules largely stipulate the same requirements as those discussed above, so I omit a detailed presentation. Instead, I go on to present sector-specific rules pertaining to takings for hydropower. %However, I mention that it has been controversial whether or not these rules provide any basis at all for scrutiny of established practices adopted by the water authorities. Specifically, it has been argued that the licensing procedures spelled out in the \cite{wra00} and the \cite{wra17} are exhaustive in hydropower cases.\footnote{See \cite{jorpeland11a}.} In the case of {\it Jørpeland}, the Supreme Court held that general rules of administrative law did apply in theory, but went quite far in suggesting that they would have limited significance in practice, as sector-specific rules and practices would take priority.\footnote{See \cite{jorpeland11}.}

\section{Taking Waterfalls by Obtaining a Development License}\label{sec:5:3}

As mentioned in the introduction, section 16 of the \cite{wra17} establishes an automatic right to expropriate rights needed to implement a licensed watercourse regulation. This does not include a right to expropriate rivers and waterfalls needed for the hydropower development as such. However, it includes a right to transfer water away from a river for development somewhere else. This has the {\it de facto} effect of a waterfall expropriation, since the water is transferred away from the source river.

This kind of expropriation has always been treated as waterfall expropriation in relation to the compensation issue.\footnote{See \cite{jorpeland11}.} Formally, however, the interference is not considered an expropriation of real property, but rather an expropriation of a right to remove the water, a sort of easement whereby the developer acquires the right to interfere with the rights of riparian owners in source rivers.

In theory, the rules in the \cite{ea59} and the \cite{paa67} still apply when the right to expropriate follows automatically from a development license. Indeed, the rules in the \cite{paa67} express general principles of administrative law, pertaining to all kinds of individual decisions, including both expropriation and licensing decisions. The \cite{ea59}, for its part, explicitly states that it applies to property interferences authorised under the \cite{wra17}.\footnote{See \indexonly{ea59}\dni\cite[30]{ea59}.} However, it is also stated that the rules in the \cite{ea59} only apply in so far as they are ``suitable'' and do not ``contradict'' sector-specific rules.\indexonly{ea59}\dni\footcite[30]{ea59} This points to the potential caveat that while a range of procedural rules apply in theory, there is a risk that they will be ignored in practice, if they are deemed unsuitable by the licensing authorities.

This is practically significant in hydropower cases. Specifically, the water authorities regard the \cite{wra17} as providing an exhaustive legislative basis for the licensing procedure.\footnote{This was made clear through the case of \cite{jorpeland11}, where this practice also got a stamp of approval from the Supreme Court.} This also means that the material assessment requirement in the \cite{ea59} is not considered to have any independent significance alongside the assessment criterion in the \cite{wra17}.\footnote{See \cite[30]{jorpeland11}.}

This is so, even though case law on the former assessment criterion emphasises the interests of affected property owners in a way that case law and administrative practice on the licensing issue does not.\footnote{In addition, as mentioned above, the formulation in \indexonly{ea59}\dni\cite[2]{ea59} contains the additional qualification that the benefit of interference must ``undoubtedly'' outweigh the harm. This is interpreted to mean that the benefit must {\it clearly} outweigh the harm (pertaining to the evidence, not the weight of the benefit compared to the harm), see \cite{lovenskiold09}. No corresponding requirement is included in the \indexonly{wra17}\dni\cite[8]{wra17}. Instead, the formulation there is that a license should ``normally'' not be given unless the benefits outweigh the harms. See also \cite[325-236]{haagensen02} (arguing that the ``normally'' qualification is without practical significance).} As a consequence of how the law is understood on this point, it is very hard for owners to challenge a decision to allow expropriation of their riparian rights, especially when expropriation takes place pursuant to the \cite{wra17}.\footnote{It follows from the discussion in chapter \ref{chap:4} that large-scale development projects almost always involve a license pursuant to the \cite{wra17} (or such that many of the rules from this Act, including section 16 on expropriation, apply pursuant to the \cite{wra00}).} Moreover, even if section 16 of the \cite{wra17} does not apply, the water authorities rely on the presumption that an expropriation license should be granted whenever a large-scale development license is granted.\footnote{See \cite{flatby08}.}

Hence, in order to defend themselves, owners must proceed in a roundabout manner by addressing the licensing question, for instance by arguing that large-scale development will be environmentally unsound or by presenting a detailed development plan of their own in the hope of convincing the water authorities of its merits. This is a daunting task, particularly in light of the continued influence of administrative practices developed during the period of monopoly regulation, which systematically favour the large energy companies.\footnote{See sections \ref{sec:5:4:2} and \ref{sec:5:4:3} below.}

To shed further light on how the current situation came about, I will now present the history of the law in this area. As will become clear, the current state of affairs was not inevitable, but rather the result of a series of reforms that gradually undermined property as an anchor for active community participation in hydropower development.

\section{Taking Waterfalls for Progress}\label{sec:5:4}

%Historically, Norwegian law did not permit expropriation of waterfalls for hydropower development.\footnote{See \cite[29]{amundsen28}.} 
In the now repealed \cite{wra88}, several provisions authorised appropriation of water-rights and land for various water-related purposes, but the criteria were very narrow.\footnote{See \cite[69-85]{dahl88}. In addition, the purpose of expropriation was largely understood to be binding also on future use, so that the taker would not gain unrestricted control over the rights they acquired. Rather, they were obliged to use these rights to pursue the specific public purpose for which expropriation was authorised. See, e.g., \cite[133-140]{rygh12}.} Waterfall rights as such could never be expropriated, and expropriation of other rights pertaining to the use of water could only be permitted in so far as the affected owners were not thereby deprived of any water-power that they could reasonably make use of themselves.\footnote{See \cite[58|60]{dahl88}.}

Specifically, expropriation for hydropower development was only permitted when it benefited waterfall owners who needed to acquire surrounding land in order to make better use of their own property rights.\footnote{See the \indexonly{wra88}\dni\cite[15, 16]{wra88}. See also the commentary in \cite[60-65]{dahl88}.} Moreover, riparian owners could apply for licenses to engage in various industrial exploits, in some cases also when this would prove damaging to other landowners, for instance through deprivation of water or flooding.\footnote{See \indexonly{wra88}\dni\cite[14]{wra88}. See also the commentary in \cite[54-60]{dahl88}.} These rules are similar to many of the rules found in contemporaneous mill Acts from the US, discussed in section \ref{sec:3:3} of chapter \ref{chap:3}. As in the US, these rules could be classified as giving rise to economic development takings. However, these rules were not intended to permit the source of the economic development potential as such to be taken from the owners.\footnote{See \cite[168-170]{dahl88}.} Rather, takings were only warranted with respect to additional rights that existing owners needed to realise the full potential of their own resources.

In fact, an important principle of Norwegian expropriation law at this time was that no property could be taken if the taker's interest in that property was part of the current owner's bundle of interests associated with the property.\footnote{See \cite[168-170]{dahl88}.} This applied regardless of whether or not the owners, subjectively speaking, were likely to pursue the interest in question in an optimal way. On the basis of this principle, expropriation of waterfalls for hydropower development was not permissible. The reason was simple: the right to develop hydropower was considered part of the owners' bundle of interests. Hence, it could not be taken from them, as a matter of principle.

By contrast, if ancillary land was needed by someone wishing to make optimal use of {\it their} waterfall rights, expropriation was possible. In these cases, the takers did not seek to take the owners' rights as much as to negate them, in order to fully enjoy their own. More generally, expropriation at this time was considered a way to resolve conflicts between rights, not a way to redistribute them.\footnote{See \cite[168-170]{dahl88}.}

Following industrial advances, the interest in hydropower exploded in the late 19th century.\footnote{See \cite[58-59]{falkanger87}.} As a result, the state increasingly came to see it as a political priority to regulate the hydropower sector, especially to prevent foreign speculators and industrialists from acquiring ownership of Norwegian resources.\footnote{See \cite[58-59]{falkanger87}.} As discussed in chapter \ref{chap:3}, the most important expressions of this came in the form of two new licensing Acts, namely the \cite{wra17} (section \ref{sec:wra17} and the \cite{ica17} (section \ref{sec:ica17}).

Following up on this, Parliament soon passed legislation that authorised expropriation of riparian rights -- including waterfalls -- for the benefit of public bodies, also when the purpose was hydropower development.\footnote{Legislation that made it possible to expropriate waterfalls to the benefit of the municipalities was introduced in 1911, and a similar authority that authorised expropriation in favour of the state appeared in 1917, see \cite[29]{amundsen28}.} In 1940, these authorities were consolidated and integrated in the general water resource legislation, through the \cite{wra40}.\footnote{This Act has since largely been replaced by the \cite{wra00}.} According to this Act, the authority to expropriate waterfalls could be granted only to the state and the municipalities. Moreover, the municipalities could only expropriate waterfalls when the purpose was to provide electricity to the local district.\footnote{See the \indexonly{wra40}\dni\cite[148]{wra40}. See also the commentary in \cite[201-210]{sorensen41}.} Private parties could not expropriate except in exceptional circumstances, when they already owned more than 50 \% of the riparian rights they sought to exploit.\footnote{See the \indexonly{wra40}\dni\cite[55]{wra40}. See also the commentary in \cite[70-74]{sorensen41}. I remark that this was a novel rule in the 1940 Act, which contradicted earlier theories about the legitimacy of allowing expropriation for private benefit.} 

In all cases of waterfall expropriation, it was felt that benefit sharing with local owners was required. Hence, special rules were introduced to ensure that takers would have to pay {\it more} than full compensation (typically a 25 \% premium, but in some cases the owner was also given a right to opt for compensation in the form of a proportion of the electricity output of the plant).\footnote{See \cite[70-91,184,210]{sorensen41}. For more on compensation, see below in section \ref{sec:5:4:1}.}

As I showed in chapter \ref{chap:4}, the electricity supply in Norway just after the passage of the \cite{wra40} was already well developed, with 80 \% of the population having access to electricity. Moreover, in the rural areas the supply often came from one among a vast number of small power plants. In light of the progress already made and the highly decentralised structure of the hydroelectric sector at this time, one might have expected expropriation to remain a relatively rare occurrence.

However, the prevalence of expropriation to facilitate hydropower development increased greatly after the war, as the state itself became engaged much more actively with hydropower development, also for commercially oriented industrial purposes.\footnote{See \cite[59-71]{thue96}. See also \cite{skjold06}.} Hence, despite the spirit and wording of the \cite{wra40}, this was the time when expropriation of rivers and waterfalls became a measure to facilitate large-scale transfers of resources.

This development had little do with supplying electricity to the people. Rather, it arose from increased political demand for hydropower to support the metallurgical and electrochemical industries, combined with the fact that the hydropower sector was reorganised and brought under increasingly centralised political control.\footnote{See \cite[69-71]{thue96}.} Following this, a growing share of the financial benefits from development would accrue to urban areas, as local development companies were replaced by state companies and companies dominated by prosperous city municipalities.\footnote{In 2007, as the result of a gradual centralisation process, the 15 largest hydropower companies in Norway, which are largely controlled by the state and some city municipalities, owned roughly 80\% of Norwegian hydropower, measured in terms of annual output. In 2006, the public owners of hydropower in Norway benefited from receiving more than NOK 9 billion in dividends. See \cite[28]{otprp61}.} In addition, a highly idiosyncratic compensation method was adopted in expropriation cases, resulting in a situation where waterfalls could be acquired very cheaply from local owners.

\subsection{The Natural Horsepower Method}\label{sec:5:4:1}

In section \ref{sec:wra17} of the previous chapter, I presented the notion of a natural horsepower, used to determine when a development project requires development licenses pursuant to the \cite{wra17} and the \cite{ica17}. As mentioned, the natural horsepower of a development scheme is a gross measure of the stable electric effect harnessed following the development. Specifically, it measures the electric output that can be maintained for at least 350 days each year, a figure that is sensitive to fluctuating water levels. In practice, a power plant can often generate electricity at a much higher level, by using more water whenever it is available.\footnote{See \cite{sofienlund07}.}

For this reason, the number of natural horsepower in a development project is an unreliable measure of the total amount of energy harnessed in a year. As a result, it is also an unreliable starting point for assessing the value of a development project. Today, energy producers get paid for all the energy they produce, not just that which they can guarantee in advance.\footnote{See \cite[83-84]{uleberg08}.} Prior to the establishment of a national grid, this was different. Without a grid, fluctuations in electric output would not be evened out by supply from other parts of the country. Hence, the importance of maintaining a stable supply was much greater. Indeed, energy producers would often get paid based on the amount of electric effect they could deliver stably over the year, not the total amount of energy harnessed.\footnote{See \cite[83]{uleberg08}.}

Hence, early in the 20th century, the notion of natural horsepower could be used to provide a sensible measure of how much a developer would be willing to pay for access to riparian rights.\footnote{See \cite[83]{uleberg08}.} Indeed, the notion was used in this way on the market for waterfalls that existed prior to state regulation. The price of a waterfall was typically calculated on the basis of the price that the developer was willing to pay per natural horsepower that the planned development would yield.\footnote{See \cite[83]{uleberg08}.} The total payment offered to the owners, consequently, would be found by multiplying the natural horsepower of the development with the price offered per natural horsepower.

This method was also adopted by appraisal courts to fix the level of compensation following expropriation.\footnote{\cite[521]{vislie02}.} Moreover, when the notion of natural horsepower fell into disuse among energy producers, because it no longer reflected the actual value of development projects, the courts did not modify their compensation practices. They stuck with the natural horsepower method, which was now applied on a customary basis, not as a way of calculating realistic economic values.\footnote{See, e.g., \cite[1599]{hellandsfoss99} (the Supreme Court comments that the method is used customarily because the market provides ``little guidance'').}

Over time, the price level became more and more unrealistic as a measure of the value of waterfalls as a natural resource. By the time of liberalisation in the early 1990s, the discrepancy had become extreme. To give an example: in 1999, the appraisal court of appeal awarded a one time payment of NOK 722 068 in compensation for a waterfall that yields 152 GWh per annum.\footnote{See \cite{hellandsfoss99}.} By comparison, in the case of {\it Sauda} from 2009, where a market-based valuation method was used, the owners of the {\it Maldal} river were awarded NOK 1 149 044 in compensation as a {\it yearly payment} for a waterfall that would yield 36.5 GWh per annum.\footnote{See \cite{sauda09}.} If we assume an interest rate of 4 \%, this corresponds to a one time payment of NOK 28 726 100. This, in turn, corresponds to NOK 787 017 per 1 GWh produced annually. That is, the owners of {\it Maldal} were paid in the excess of 150 times more for their waterfall than the owners of {\it Hellandsfoss}.

The clearest data available is from the time after liberalisation, when a more realistic method of valuation came to be used by some appraisal courts, based on the leasehold model discussed in section \ref{sec:3:5} of the previous chapter. However, the mismatch between economic values and compensation payments had in fact been noted much earlier. A major discrepancy had been identified as early as in the 1950s, by the head of the water directorate himself. In an article published in an internal newsletter in 1956, the director commented that the natural horsepower method did not result in compensation payments that reflected the true economic value of waterfalls as a natural resource.\footnote{See \cite{rogstad56}.} Moreover, he speculated that the method could be sustained only by exploiting the lack of knowledge about hydropower development among rural populations.\footnote{See \cite{rogstad56}.}

One might think that the continued use of the natural horsepower method, in a situation when the water authorities themselves were aware of its shortcomings, would result in controversy. However, at this time, the local owners of waterfalls did not attack the method in court. Active resistance on this point would not be seen until much later, after the liberalisation of the electricity sector, as discussed in sections \ref{sec:5:5:2} and \ref{sec:5:5:3} below. However, conflicts arose with respect to other aspects of the regulatory framework regarding hydropower, as discussed in the following section.

\subsection{Increased Scale of Development and Increased Tension}\label{sec:5:4:2}

As discussed in the previous chapter, the state pursued increasingly complex hydropower projects after the Second World War. At this time, technological and economic advances also made it more feasible to divert water over great distances, to collect several different rivers in a common reservoir for joint exploitation. Such projects became known as ``gutter'' projects, and they grew greatly in scope during the post-War years. Since the relevant licensing procedure was covered by the \cite{wra17}, the practical importance of the expropriation authority in section 16 of this Act also increased dramatically.

Initially, the law stood in the way of this development, since section 16 itself stated that a license to divert water should ``normally'' not be given unless the owners in the source rivers agreed to it.\footnote{See generally \cite{innst59}.} However, this rule was removed after an amendment in 1959.
The department argued that the rule had an unfortunate effect on the administrative procedure in large-scale diversion cases, noting also the vastly increasing complexity and scale of typical diversion regulations.\footnote{\cite[11]{innst59}.} The minority in the parliamentary committee recommended against the amendment, noting that it would ``greatly increase'' the authority to expropriate waterfalls, conflicting with the principles of the \cite{wra40}.\footnote{See \cite[14]{innst59}.} However, the majority rejected this argument by maintaining that the state would prevent any abuse of power, and that the practical significance of the amendment would be limited to ensuring a ``more rational''  approach to large-scale applications.\footnote{See \cite[14]{innst59}.}

As mentioned in the previous chapter, the opposition to hydropower grew proportionally to the scale and complexity of typical development projects.\footnote{See generally \cite[64-65]{nilsen08}.} The critical focus was often on environmental effects, but the interests of local people also featured in these debates. Moreover, local interest were often aligned with the environmental interests.\footnote{See \cite[72-73]{nilsen08}.}

%In the first cases that reached the Supreme Court from this era, the question of legitimacy was not raised in full breadth. Instead, the early cases concerned specific legal points, such as the issue of whether (informal) agreements and understandings between owners, municipalities and the central government were binding on future decision-making processes regarding development.\footnote{See \cite{aura61,mardola73}.} In addition, the question arose as to what extent additional compensation should be paid for `damages' and `inconveniences' caused by large-scale development, in addition to the compensation calculated using the natural horsepower method. Finally, questions arose over the status of non-waterfall owners who owned land that was still crucial to the development, for instance because it would be flooded or used to construct regulation installations. Should such owners receive compensation based on the value of their rights for development purposes, or should they only receive compensation based on the value of their current property uses, as had been the norm before?

Some controversies led to legal conflicts that came before the Supreme Court. Here the claims of owners and local communities were consistently rejected. First, the Court held that owners of ancillary rights (e.g., land needed for dams or buildings) were not entitled to compensation based on the value of their rights as an asset for hydropower development.\footnote{See \cite[332-333]{tokke63}.} Instead, they would only receive damages based on the value of their current use of the property. Second, it was held that when compensation was awarded to waterfall owners according to the natural horsepower method, then this would preclude additional compensation for harms and nuisances associated with large-scale watercourse regulation.\footnote{See \cite{vikfalli71,driva82}.}

In the case of {\it Aura}, the owners argued that they had originally agreed to sell their water rights to a private developer, on the understanding that a specific development project would take place, not involving diversion of water.\footnote{See \cite[1284]{aura61} (the original transaction took place in the period 1906-1910, when there was still a market for sale of waterfalls to private speculators and developers).} Hence, the owners thought that the purchaser of their water rights had not acquired a right to divert the water away from the river. Still, when the government later acquired the water rights in question, they decided to embark on a more intrusive project that {\it would} involve water diversion. For this reason, the owners argued that they were entitled to additional compensation. 

The claim was rejected by the Supreme Court, which held that insufficient evidence had been provided to establish that the sale of the water rights was made conditional on a specific type of development.\footnote{See \cite[1285-1286]{aura61}.} Moreover, it was held -- on the basis of the facts -- that the sale of the water rights had {\it not} been restricted to only cover the waterfall (i.e., the right to harness hydropower from the river in question). According to the Supreme Court, the fact that the rights in question had been referred to as ``water rights'' meant that the right to divert away the water was also included.\footnote{See \cite[1284-1285]{aura61}.}

In the later case of {\it Mardøla}, the situation was similar, with the crucial difference being that these local owners had not sold ``water rights''; their contract explicitly stated that what had been sold was the waterfalls.\footnote{See \cite[112]{mardola73} (the voluntary sale dated back to the early 20th century, when the market for waterfall was still unregulated).} However, the government interpreted this to mean that they had a right to divert water away from the river, without paying any additional compensation.\footnote{See \cite[112]{mardola73}.} This contradicted the premise of {\it Aura}, where the decision to allow a diversion was premised on the fact that {\it not only} the waterfalls had been acquired by the developer. Still, in {\it Mardøla}, the Supreme Court cites {\it Aura} as the primary authority in favour of a {\it general rule} by which the sale of a ``waterfall'' also includes the right to divert water away from the river.\footnote{See \cite[112]{mardola73}.} No explanation is provided by the Court to reconcile this with what was actually said in {\it Aura}.\footnote{Arguably, the Court's finding on this point has since been overruled by \cite{jorpeland11}. Here a waterfall right was defined explicitly as a right to exploit the hydropower in a river along its present trajectory. This definition was provided in order to avoid the conclusion that a diversion of water by someone other than the waterfall owner (in the source river) amounts to a waterfall expropriation. If the Court had instead concluded in keeping with the precedent set by {\it Mardøla}, by holding that the diversion right is part of the waterfall bundle, it would have shed serious doubt on the legitimacy of the established practice of allowing diversions under section 16 of the \cite{wra17}, with no prior acquisition of the waterfall rights in source rivers.}

%In {\it Mardøla}, both the owners and the local municipality had explicitly agreed to support the central government on the understanding that a specific development plan would be adopted. Later, this plan was abandoned in favour of a project that was deemed by some local owners to be both less beneficial and more intrusive. Hence, both the owners and the municipality argued that the resulting development license was invalid. The Supreme Court conceded that prior statements made by the water authorities had been striking, serving to create a clear expectation among the locals for a specific development plan.\footnote{See \cite[111]{mardola73}.}

%However, the Court chose to rely on what it described as a ``general presumption'' against the position that the central government is bound in its decision-making by prior statements.\footnote{See \cite[110]{mardola73}.} According to the Supreme Court, the statements made by the water directorate in {\it Mardøla} were not clearly endorsed by the Ministry and the Parliament, and could therefore not be regarded as binding on the final decision.\footnote{See \cite[111]{mardola73}.} %However, the Court did not address the question at all from a procedural angle, by inquiring into the legitimacy of creating what appeared to be a legitimate expectation on part of the owners and the municipality.

%This finding seems reasonable enough. However, one rather crucial question was not addressed at all: is it legitimate procedure for the water authorities to make ``striking statements'' of the kind offered in {\it Mardøla}, when this serves to silence opposition and induce support for development during the assessment stages? The Supreme Court apparently did not wish to consider this question, raising the possibility that it felt the answer would not have been to its liking.

The case of {\it Mardøla} illustrates the increasing tension that arose regarding hydropower in the 1970s, and arguably also a tendency on part of the Supreme Court to side with large-scale development interests. Indeed, the development in {\it Mardøla} stirred up a high level of controversy that also resulted in civil disobedience and criminal prosecutions.\footnote{See \cite{mar71}.} The culmination of the increasing tension surrounding hydropower at this time came with the case of {\it Alta}, where the question of procedural legitimacy was raised in full breadth. To this day, the {\it Alta} case remains the most important Supreme Court precedent in the area of hydropower law.

%The case also illustrates how the central government attempted to minimise tensions by entering into dialogue with local authorities and owners. Crucially, this dialogue was not premised on a legal framework that ensured local participation, and, despite appearances, did not necessarily result in any new entitlements for local people.\footnote{However, the increased tension during this time did sometimes lead to additional benefits being bestowed on local power groups. These new measures were typically relatively minor, and they were directed primarily at municipalities and regional government bodies rather than local owners. See generally \cite[75-76]{nilsen08}.}

\noo{ In effect, the {\it Mardøla} Court sanctioned an approach whereby the water authorities could limit local opposition by expressing commitments early on, which could then simply be ignored at a later stage of the decision-making process. At this later stage, it would be too late for the local population to launch an effective opposition, e.g., it would be too late for them to aligning themselves with environmental activists. More generally,}

%Moreover, despite occasional concessions being made to local and regional government institutions, controversies continued to arise.

\subsection{The {\it Alta} Controversy}\label{sec:5:4:3}

The {\it Alta} case went before the Supreme Court in 1982 after a long period of high-intensity conflict going back to the mid-seventies.\footnote{See \cite{alta82}. For commentaries, see \cite{eckhoff82,boe83,hagvar88}.} In {\it Alta}, the affected local population largely lacked formal title to the property they sought to defend. This was because the development in question would take place in the northernmost part of Norway, in the native land of the Sami people.\footnote{For Sami law generally, see \cite{skogvang02}.}

Norway has a history of discrimination against the Sami, and as their culture is largely nomadic, their land rights were never formalised in private law.\footnote{See \cite[149-156]{ravna12s}} As a result, land and natural resources in the county of Finnmark are largely owned by the state, at least in the sense of the state appearing as the nominal {\it in rem} owner.\footnote{In the past 30 years, partly as a response to the controversy of the {\it Alta} case, there has been a gradual change in attitude, whereby the rights of the Sami people receives greater legal recognition. In 2007, formal title to most of the land in the county of Finnark was transferred to a special state agency which is regulated by a special statute that obliges it to manage the land with due regard to customary and prescriptive rights of aboriginal groups and local people. See generally \cite{bull07}.}

Due to the sensitive context of interference, the {\it Alta} plans met with particularly strong criticism from local people, as well as environmental groups and groups fighting for aboriginal rights. A broad political movement was mobilised in opposition to the plans, eventually resulting in several serious cases of civil disobedience.\footnote{This included hunger strikes and attempts at sabotage, see \cite[80-83]{nilsen08}. For the {\it Alta} controversy generally, see \cite{hjorthol06,altawiki}.} 
The plans for development had initially involved a dam that would flood a village and displace its inhabitants, but this aspect of the project was abandoned. However, the conflict continued, with the local population arguing that the proposed hydropower plans would threaten their livelihoods. The case eventually came before the courts, when the locals joined forces with environmental groups to request judicial review of the development licenses that had been granted.\footnote{See \cite{eckhoff82}.}

The {\it Alta} case did not involve expropriation of waterfalls. However, because of the priority given to the licensing procedure over specific expropriation procedures, the principles expressed in {\it Alta} also largely determine the legal position of waterfall owners whose rights to hydropower are expropriated.\footnote{See \cite{sauda09,jorpeland11}.} Moreover, the case involved expropriation of other property rights as well as special usufructuary rights held by the Sami people.

{\it Alta} was admitted to the Supreme Court in plenum, directly on appeal from the district court.\footnote{This is a special arrangement available in cases that raise important questions of principle, see \indexonly{cda05}\dni\cite[30-2]{cda05} and \indexonly{ca15}\dni\cite[5]{ca15}.} The presiding judge commented that as far as he knew, it was the longest and most extensive civil case that the Court had ever heard.\footcite[254]{alta82} In an opinion totalling 138 pages, the Court considers a long range of objections against the development licenses, all of which are either rejected or held to provide insufficient reasons to declare the licenses invalid.

\noo{The opponents of the {\it Alta} development also argued on the basis of human rights and international law.\footnote{First, on the basis of articles 1 and 27 of the \cite{fnp}. Second, on the basis of \cite{ilo107} (later replaced by \cite{ilo169}). Third, on the basis of P1(1) of the \cite{echr}.} As noted by Eckhoff, these arguments raised subtle legal questions about how to apply the relevant principles of international law to a concrete dispute over hydropower development.\footnote{See \cite[351-352]{eckhoff82}. One of the most important international instruments, namely ILO Convention No 107, was not ratified by Norway at the time of {\it Alta} (Norway later ratified its replacement, ILO Convention No 169). However, it was argued that it had the status of customary international law. See generally \cite{eide80}.} However, the Court refused to consider such  questions, finding that the negative effect of the hydroelectric plant was not so severe as to raise  human rights issues.\footnote{See \cite[299-300]{alta82}. See also \cite[351-352]{eckhoff82}.}}

First, the Court summarily rejects arguments based on indigenous and human rights law on the basis that the impact of the planned development would not be sufficiently severe to raise any issues in this regard.\footnote{See \cite[351-352]{eckhoff82}. I note that the most important international instrument protecting indigenous rights, ILO Convention No 107, was not ratified by Norway at the time of {\it Alta} (Norway later ratified its replacement, ILO Convention No 169). Still, it was argued that it had the status of customary international law, an argument not considered in any depth by the Supreme Court. See generally \cite{eide80}.} After concluding in this way, the Supreme Court goes on to approach the case on the basis of administrative law. The focus was solely on the procedural rules of the \cite{wra17}. In this regard, the opponents of the {\it Alta} development had pointed to a large number of purported shortcomings of the decision-making process. 

First, it had been argued that the original licensing application did not meet the requirements stipulated in section 5 of the \cite{wra17}. Essentially, the original application contained little more than technical details about the planned development, with hardly any identification or assessment of deleterious effects.\footnote{See \cite[264-265]{alta82}.} This shortcoming had been openly acknowledged by the water authorities themselves, but a public hearing had nevertheless been initiated.\footnote{See \cite[265]{alta82}.}

The Supreme Court concluded that this was ``clearly unfortunate''.\footcite[265]{alta82} However, several reports and assessments had subsequently been provided, to fill the gaps left open by the initial application. For this reason, the Supreme Court held that the initial mistakes were irrelevant, since it was the licensing process as a whole that should be assessed.\footnote{See \cite[265-266]{alta82}.} Shortcomings at specific stages in the assessment would not be given weight unless they could be seen to imbue the process with a dubious character overall.\footcite[265]{alta82}

The Court then moved on to assess whether the process as a whole fulfilled the procedural requirements of sections 5 and 6 in the \cite{wra17}. In addition, the Court considered whether the assessment of the licensing criteria in section 8 of the \cite{wra17} had been sufficiently detailed. In addition to assessing a large amount of information regarding the negative effects of development in {\it Alta} and how these had been assessed by the water authorities, the {\it Alta} Court also made some important statements of principle. In particular, the Court held that since a licensing decision itself is discretionary, it is appropriate to grant the executive some margin of appreciation also with regard to the question of how to interpret vague requirements of administrative law.\footnote{See \cite[262-264]{alta82}.}

The Court made a second decision of principle when it supported the state's contention that the administrative licensing assessment did not have to be as thorough as that required in a subsequent appraisal dispute.\footnote{See \cite[279|330]{alta82}.} This observation was used to downplay the risk of factual error; if mistakes are made with regard to the owners' losses at the assessment stage, these mistakes can be corrected later by a correct compensation award.

In fact, the {\it Alta} Court agreed that the license had been based on erroneous information about some issues, particularly regarding alternative ways to meet the need for electricity in Finnmark.\footnote{See \cite[346-357]{alta82}.} However, the Supreme Court did not regard the factual errors in this regard as relevant to the licensing decision.\footnote{See \cite[346]{alta82}.} 

Here a third clarification of principle took place. The Court held, in particular, that the duty to consider alternatives -- different ways in which the public purpose could be satisfied -- is very limited in hydropower cases.\footnote{See \cite[346]{alta82}.} This was how the Court dealt with  factual errors and inadequate information concerning alternatives; since the information in question was not required in the first place, the mistakes had no bearing on the validity of the decision.\footcite[346]{alta82}

The Court's perspective on alternatives appears to have been at odds with how Parliament had approached the licensing question, on three separate occasions.\footnote{See \cite[342]{alta82}.} Indeed, there was little doubt that the favourable political assessment of the {\it Alta} development depended strongly on the perceived electricity crisis in Finnmark and the supply situation in Norway generally, as well as the perceived inadequacies of alternative solutions.\footnote{See \cite[338-347]{alta82}.}

Hence, it is quite remarkable how little attention the Court directs towards the factual errors and the inadequate information that had been provided in this regard.\footnote{See also the surprise expressed in \cite[349-351]{eckhoff82}.} By contrast, the Court goes into painstaking detail regarding issues that seem to have been far less important to the political decision-makers, and with respect to which the state's arguments appeared much stronger.

The dismissive attitude towards the duty to correctly assess alternatives is a controversial aspect of the {\it Alta}-decision.\footnote{See \cite[311]{haagensen02}. For criticism of the Supreme Court on this point, see \cite[580-584]{backer86}.} On this point especially, the decision has met with criticism from commentators arguing that the decision shows the extent to which the courts in Norway tend to identify themselves with other organs of state.\footnote{See, e.g., \cite[64]{graver88} (commenting also that ``government prestige'' was at stake).} Some have also argued that {\it Alta} would be unlikely to become a leading precedent, especially with regard to the duty to assess alternatives.\footnote{See \cite[580-584]{backer86}.} But this has been proven wrong. Indeed, {\it Alta} continues to receive favourable citations by the Supreme Court, both in relation to hydropower and with respect to administrative law generally.\footnote{See \cite{ambassade09,jorpeland11}.}

It should be mentioned, however, that after the {\it Alta} decision, the legal position of the Sami people has improved quite significantly.\footnote{See generally \cite{gauslaa07}. Gauslaa presents the emergence of {\it Sami law}, a collection of rules and principles serving to protect established land use patterns and the Sami way of life while also giving the Sami people a better opportunity to partake in decision-making processes that affect them as group.} Moreover, the controversy surrounding {\it Alta} has been regarded as a catalyst for change in this regard.\footnote{See \cite[156]{ravna12s}.} Hence, it is unlikely that the courts today would be as quick as the {\it Alta} court to dismiss arguments based on aboriginal rights.\footnote{See \cite[180]{gauslaa07}.}

However, with regard to local owners more generally, the {\it Alta} decision is considered to express key principles that still apply.\footnote{See \cite{jorpeland11}. See also \cite[312]{haagensen02}.} 
At the same time, the context surrounding takings for hydropower development have changed significantly. First, as discussed in the previous chapter, takings of waterfalls now occur in a very different economic context. Moreover, as discussed in the next section, the legal context has also changed, giving rise to a situation where expropriations of waterfalls have become clear takings for profit.

\section{Taking Waterfalls for Profit}\label{sec:5:5}

Following the introduction of the \cite{wra00}, the legislative authority to expropriate waterfalls  was expanded and incorporated in the \cite{ea59}. This change in the law was not singled out for political consideration. In fact, the increased scope of expropriation was not mentioned at all when the Ministry presented their proposals to Parliament. Rather, the new expropriation authority was described merely as a ``simplification'' of existing law.\footcite[223-225]{otprp39}

The original proposal stemmed from the report handed to the Ministry by a committee appointed to prepare a new Act relating to water resources. The report totals almost 500 pages, but devotes only three of those pages to discussing the new expropriation authority.\footnote{See \cite[235-237]{nou94}.} Here it is noted that a range of different authorities for expropriation has long co-existed in the law, with many of them positing strict and specific public interest requirements as a precondition for granting a license. This, it is argued, is not a very ``pedagogical'' way of providing expropriation authorities.\footcite[235]{nou94} Moreover, it is suggested that such a system runs the risk of omitting important purposes for which expropriation should be possible. Hence, the committee proposes to replace all older authorities by a sweeping authority that will make expropriation possible for the purpose of facilitating ``measures in watercourses''.\footcite[235-236]{nou94}

The committee notes that such a formulation might seem wide, but remarks that this is not a problem since the executive can simply refuse to issue an expropriation order when they regard expropriation as undesirable.\footcite[235]{nou94} There is no discussion of the consequences of such a perspective, neither in relation to property rights nor in relation to the balance of power between the legislature, the executive and the courts.

\noo{Instead, the commission offers a brief presentation of the rationale behind dropping the local supply restriction for municipal expropriation. They comment that these rules complicate the law and might make desirable expropriations impossible.\footcite[235]{nou94} Nothing is said to clarify what kind of desirable expropriations the committee think might be left out. 

The committee do not relate their proposals to the recent liberalisation of the energy sector. Hence, the obvious practical consequence of their proposal, namely that expropriation of waterfalls would be made available as a profit-making tool for commercial companies, is not discussed or critically assessed. The issue of {\it who} should be permitted to benefit from an expropriation license is also dealt with only superficially. In this regard, the committee structures its presentation around the redemption rule of the \cite{wra40}. As mentioned briefly in section \ref{sec:twp}, this rule made it possible for the majority owners of a waterfall to compulsorily acquire minority rights, if this was necessary to facilitate hydropower development. Hence, it was a rule that provided only a limited opportunity for private takings, restricted to owners themselves or external developers that had been able to reach a deal with a locally based majority.

The main justification given by the commission for introducing a general private takings authority is that the special redemption rule had not been much used.\footcite[236]{nou94} Why this is an argument in favour of opening up for private expropriation in general is not made clear. It seems just as natural to regard it as an argument {\it against} doing so. Why extend the possibility for private expropriation if the demand for such expropriation has been limited?

Presumably, the commission thought there would be a demand for private expropriation in the future, but this is not stated explicitly, nor is the appropriateness of it discussed. As to the requirement that private takers must already control a majority of the waterfall rights in the local area, the commission only remarks that it regards such a restriction as old-fashioned.\footcite[236]{nou94} No discussion is offered regarding the consequences for local communities, if it is dropped.}

However, the later case of {\it Sauda} shed light on this question, as it emerged that the new authority would, for the first time in Norwegian history, make it possible for private commercial interests to openly expropriate waterfalls.\footnote{In cases involving diversion of water, a {\it de facto} right to expropriate could be granted to private actors already under section 16 of the \cite{wra17}. See the discussion in section \ref{sec:5:4:2} above.}

%when it emerged that the members of parliament themselves had not been aware that the new legislation would result in private takings of waterfalls.

%t emerged that the members of parliamentary committee preparing the act had 

%Since the passage of the \cite{wra00}, it has become clear that the new authority for expropriation is a particularly controversial aspect of the act. This is because cases of waterfall expropriation today tend to imply that local owners are deprived of a small-scale development potential in favour of a commercial company. This has resulted in a new body of case law developing on takings for hydropower, as discussed in the next few sections.

\subsection{{\it Sauda}}\label{sec:5:5:1}

In {\it Sauda}, a case before the court of appeal, the riparian owners formally protested a license that granted a private company the right to expropriate their rivers and waterfalls.\footnote{See \cite{sauda07} (the decision from the district appraisal court) and \cite{sauda09} (the decision from the appraisal court of appeal).} In the district court, the owners' principal argument was that the executive could not grant such a right to a private party, since the legislation authorising private expropriation of waterfalls had not been properly authorised by Parliament.\footnote{See \cite{sauda07}.}

This argument appeared weak, since the \cite{ea59} contains a provision which implies that the executive is authorised to decide what legal persons can benefit from expropriation licenses under that act.\footnote{See section 3 of the \cite{ea59}. As mentioned in section \ref{sec:5:2} above, the provision gets to the point in a rather roundabout way, by stating that no one except state bodies may be granted a permission to expropriate, {\it unless} the King in Council has decided otherwise.} However, the owners argued that the executive had not appropriately informed Parliament that private takings of waterfalls would result from including the waterfall expropriation authority in the \cite{ea59}. It was pointed out, in particular, that the crucial amendment to the \cite{ea59} had been passed as a mere formality following the adoption of the \cite{wra00}, on the basis of the Ministry's description of the new expropriation authority as a ``simplification'' of existing law.

To back up their constitutional argument, the owners presented the written testimony of two members of the parliamentary committee that had prepared and voted for the \cite{wra00} and the associated amendments of the \cite{ea59}.\footnote{Presented to the Court in \cite{sauda07} (available from the author on request).} Neither of them could recollect that they had been aware that their actions would make private expropriation possible. Plainly, they had not been told, and had not realised on their own, that this would be the result. Their ignorance was not in fact very surprising, since the crucial change in the law was only apparent as an implicit consequence of the combined effect of three different sections in two separate acts.\footnote{See the \indexonly{wra00}\dni\cite[51]{wra00} and the \indexonly{ea59}\dni\cite[2, 3]{ea59}.} In the entire collection of preparatory documents, the change was discussed only once, and then only very briefly, in the report from the committee to the Ministry.

On this basis, the owners argued that the purported expropriation authority was not constitutionally valid, since Parliament had not intended it. Unsurprisingly, this argument was rejected by the district court.\footnote{See \cite{sauda07}.} It had to be assumed that members of Parliament understood the consequences of their own legislative actions. 

Interestingly, however, when the case went before the court of appeal, the issue was not discussed at all, since the court decided to rely on a {\it different} authority as the legislative basis for allowing the expropriation to go ahead. Specifically, since the expropriation in question involved a diversion of water, the court of appeal held that the taker did not in fact require the expropriation license it had been granted pursuant to the \cite{ea59}. Section 16 of the \cite{wra17} would suffice.\footnote{See \cite{sauda09}. See also the discussion in section \ref{sec:5:2} above.} Hence, the constitutional question could be laid to rest.

In addition to the constitutional complaint, the owners in {\it Sauda} also raised procedural objections. They argued, in particular, that the expropriation question had been insufficiently assessed by the water authorities.\footnote{See \cite{sauda09}.} The court did not agree, and the procedural arguments at stake here foreshadow the later Supreme Court case of {\it Jørpeland}, discussed in more depth in section \ref{sec:5:6}.

While the owners in {\it Sauda} lost the validity dispute, the level of compensation they received was dramatically increased compared to earlier practice. Because of this, the development company appealed the decision to the Supreme Court, with the owners lodging a counter-appeal regarding the question of legitimacy. The Supreme Court decided not to hear the case, possibly because it had recently considered the compensation issue in the case of {\it Uleberg}, discussed in the next section.

\subsection{{\it Uleberg}}\label{sec:5:5:2}

Just before the {\it Sauda} case was decided by the court of appeal, the Supreme Court had addressed the compensation question in the case of {\it Uleberg}.\footnote{See \cite{uleberg08}.} Here the Supreme Court agreed in principle that the natural horsepower method was not binding on the appraisal courts. Specifically, the Court held that market value compensation could be awarded just in case small-scale development by owners would have been ``foreseeable'' in the absence of expropriation.\footnote{See \cite[81]{uleberg08}.} In {\it Uleberg}, this was not the case. Specifically, the Supreme Court found that the relevant date of valuation was in 1968, when the waterfall rights in question had been transferred to the developer by a voluntary agreement.\footnote{See \cite[70]{uleberg08}.}

This agreement stated that the final payment to the owners should be fixed by the appraisal courts at the time when the development took place. Both the appraisal court and the appraisal court of appeal took this to mean that the valuation should be based on the value of the waterfall at the time when the compensation was awarded. However, the Supreme Court disagreed, holding instead that the intended reading was that the valuation should be based on the value of the waterfall at the date when the voluntary agreement was made (with interest paid for the delay).\footnote{See \cite[71]{uleberg08}.}

Furthermore, the Supreme Court then stated, without any substantive argument, that the natural horsepower method should then be used.\footnote{See \cite[62]{uleberg08}.} Presumably, this was based on the opinion that it was obvious that owner-led development would have been `unforeseeable' at this time. The exact meaning of the foreseeability requirement has since become a much contested issue, resulting in several Supreme Court cases pertaining specifically to the compensation question.

\subsection{Recent Developments on Compensation}\label{sec:5:5:3}

Since {\it Uleberg}, there have been many controversial cases involving expropriation of waterfalls.\footnote{See generally \cite{larsen06,larsen08,larsen12}.} In most of these, the issue of compensation has occupied center stage. With respect to this issue, owners initially appeared to be gaining significant ground, as the appraisal courts started to apply a market-based method quite systematically, resulting in dramatically increased compensation payments.\footnote{See the discussion on the natural horsepower method above, in section \ref{sec:5:4:1}. See also \cite[278-290]{hauge15}.}

The large energy companies consistently resisted this development, typically by arguing that small-scale hydropower was unforeseeable and therefore not compensable according to the principle expressed in {\it Uleberg}.\footnote{See, e.g., \cite{klovtveit11,otra10,otra13}.} Moreover, the takers would tend to argue that a license to undertake large-scale development was by itself conclusive evidence  that small-scale development was unforeseeable.\footnote{See, e.g., \cite[17]{otra10}.} The large-scale development license showed, according to the takers, that a license to undertake small-scale development would not have been granted.

This line of argument conflicts with the so-called no-scheme principle, whereby compensation for expropriated property is to be based on the situation such as it would have been in the absence of the expropriation scheme.\footnote{This principle is found in many jurisdictions. See \cite[20-21]{sluysmans15}; \cite[1722]{lehavi07}. The common law formulation is also often referred to as the {\it Pointe Gourde} rule, named after \cite{gourde47}.} This principle is often contentious, and notoriously hard to apply in practice.\footnote{For a recent clarification of (some aspects of) the principle as applied in England and Wales, see \cite{waters04,spirerose09}. There have been many calls for legislative reform as well, which have so far not been taken up by the government, see \cite[152]{waring15}. In Norway, the principle has proven itself equally problematic (if not more so), see for instance \cite{nou03} (including an aggressive dissent from a professor of law opposing a widening of the no-scheme principle, especially in cases when the `scheme' involves protecting the environment). I also mention that in economic development situations, a wide application of the no-scheme principle can have the effect of precluding benefit sharing between takers and owners, giving rise to a legitimacy issue that is discussed in depth in \cite{dyrkolbotn15a}.} However, in hydropower cases, its meaning is typically clear enough: compensation should be based on the situation as it would have been in the absence of plans for large-scale development. In such a situation, it is often clear that waterfall owners would have succeeded in obtaining a license to undertake small-scale hydropower. Hence, the energy companies must contradict the no-scheme principle when they argue that small-scale hydropower should be considered unforeseeable {\it because} the licensing authorities prefer large-scale development.

In most early cases before the lower courts, this argument failed. Moreover, in the case of {\it Otra I}, it appeared as though it was rejected also by the Supreme Court.\footnote{See \cite[31-48]{otra10}.} However, the Court did not focus specifically on the no-scheme principle and how it should be applied in hydropower cases. Moreover, the taker in that case succeeded in having the appraisal court of appeal's decision overturned on the basis that inadequate reasons had been provided to justify the amount of compensation awarded to owners.\footnote{See \cite[52]{otra10}.} The court of appeal therefore had to hear the case again. This time, the taker was able to successfully argue that small-scale hydropower was unforeseeable. Hence, the court of appeal used the natural horsepower method to calculate compensation.\footnote{In fact, the court used a slightly modified version of the method, first developed in \cite{sauda09}, serving to make the discrepancy between market value and compensation slightly less pronounced. See \cite{otra12}.} 

The owners duly appealed the decision to the Supreme Court, which agreed to consider the case for a second time.\footnote{See \cite{otra13}.} But this time, the Supreme Court endorsed the understanding of the no-scheme principle argued for by the energy company. Specifically, the Court refused to censor the appraisal court of appeal's assessment of foreseeability, even though it was based explicitly on the premise that the expropriation project was preferable from the point of view of the licensing authorities.\footnote{See \cite[53-54]{otra13}.}

If the precedent set by {\it Otra II} stands, compensation for the loss of alternative development opportunities will generally not be awarded in future cases where waterfalls are expropriated in favour of large-scale schemes.\footnote{The precedent has already been used to deny such compensation in the case of \cite{smibelg15} (appeal to the Supreme Court denied).} However, it should be noted that the Supreme Court has been very vague on how exactly it understands the no-scheme principle in these cases. Instead of tackling this issue directly, the Court has chosen to rely largely on deference to the foreseeability determinations carried out by the appraisal courts.

This is clearly illustrated by the earlier case of {\it Kløvtveit}.\footnote{See \cite{klovtveit11}.} Here the Supreme Court agreed with the appraisal court of appeal that it might in principle be foreseeable that the owners, in the absence of expropriation, could have cooperated with the taker to implement the expropriation project. This too contradicts the no-scheme principle, but unlike the reasoning of {\it Otra II}, it also provides an alternative route to substantial compensation, on the basis of a valuation of the expropriation project itself. In effect, it points to an approach that promises to deliver a form of {\it benefit sharing} between owners and takers. 

For this reason, {\it Kløvtveit} is an interesting decision. However, it seems quite unlikely that it will become an important precedent for the future. Its importance was undermined already by {\it Otra II}, when the presiding judge explicitly denied that cooperation between the taker and the owners was a realistic scenario in that case.\footnote{See \cite[69-71]{otra13}.} Moreover, {\it Kløvtveit} itself was eventually sent back to the appraisal court of appeal, because the Supreme Court held that the date of valuation had been incorrectly determined.\footnote{See \cite[35-39]{klovtveit11}.} On the second hearing in the appraisal court of appeal, cooperation between owners and taker was regarded as unforeseeable, so market value compensation was denied.\footnote{See \cite{klovtveit13}.}

In fact, no case heard by the Supreme Court so far has concluded with compensation based on commercial  values. In the end, the natural horsepower method has always been used. This, no doubt, sends a clear signal to the appraisal courts. In the future, it seems likely that we will see a resurgence of the natural horsepower method and a return to compensation awards amounting to tiny fractions of the values that are taken from local owners.

In light of this development, the broader issue of legitimacy becomes increasingly important. The financial entitlements of owners and communities, which seemed to be better protected after {\it Uleberg}, are again at risk of being undermined. Moreover, as the social function theory of property indicates, the issue of legitimacy goes well beyond the individual financial entitlements of owners. It also pertains to the future of the local communities, the duties of owners in this regard, sustainable management, and the democratic legitimacy of decision-making regarding natural resources. These aspects have not received any attention from Norwegian courts so far. However, as I have already mentioned, the case of {\it Jørpeland} saw the procedural legitimacy of hydropower takings come to the forefront, for the first time since the case of {\it Alta}. In addition to clarifying legal points in this regard, the case also sheds light on the practices adopted by the water authorities in expropriation cases. Hence, it provides an excellent opportunity for a closer inquiry into the legitimacy question.

\section{A detailed case study: {\it Ola Måland v Jørpeland Kraft AS}}\label{sec:5:6}

The expropriating party was a public-private commercial partnership, Jørpeland Kraft AS. Originally, this limited liability company was jointly owned by Scana Steel Stavanger AS, with 1/3 of the shares, and Lyse Kraft AS, with the remainder.\footnote{See \cite[2]{jorpeland09}.} Lyse Kraft AS is a publicly owned energy company with the city municipality of Stavanger being the dominating shareholder. Scana Steel Stavanger AS, on the other hand, was a subsidiary of the publicly traded Scana Steel Industrier ASA. The largest shareholder of the parent company is a private individual, a leading business person and one of the richest people in the city of Stavanger.\footnote{See \cite{birkevold09} (the business man in question is John Arild Ertvaag).}

Scana Steel had long operated a steel mill in the small town of Jørpeland, belonging to the municipality of Stranda, in Rogaland county, south-west Norway. The source of energy was a relatively small hydropower plant harnessing energy from the river that reaches the sea near Jørpeland.\footnote{See \cite{aadland09}.} At the height of activity, the mill had about 1200 employees and was an important local institution.\footnote{See \cite[11]{meland82}.} However, after going bankrupt and being reorganised in 1977, the importance of the steel mill declined significantly.\footnote{See \cite[8-15]{meland82}.} After a second bankruptcy in 2015, Scana Steel Stavanger AS was wound up. The mill has since been reorganised yet again, with the number of employees reduced from around 100 to around 30.\footnote{See \cite{jossang15}.}

In parallel with the decline of the steel mill, the hydropower plant in Jørpeland was rebuilt and expanded, not to supply energy for local industry, but to sell electricity on the national grid.\footnote{See \cite{aadland09}.} Jørpeland Kraft AS was put in charge of this development, which was thereby decoupled from the steel mill operations. In 2011, the same year when {\it Måland} came before the Supreme Court, Scana Steel Stavanger AS sold their shares in Jørpeland Kraft AS to the German investment company Aquilla Capital.\footnote{See \cite{sandvik11}.} In light of this, the story of Jørpeland nicely illustrates broader trends in the history of hydropower, reflecting the centralisation and commercialisation pressures discussed in chapter \ref{chap:4}.

The river that gave rise to controversy in {\it Måland} was not located in the same municipality as Jørpeland, but in a different valley across a mountain range, in the rural municipality of Hjelmeland. The contested license in {\it Måland} gave Jørpeland Kraft AS the right to divert the water from this river for electricity production at Jørpeland. In the following, I present the facts of the case in more detail, before considering the legal questions that were addressed by the courts.

\subsection{The Facts of the Case}\label{sec:5:6:1}

The plans to divert water from Hjelmeland would deprive riparian owners of potential hydropower along some 15 kilometres of riverbed, all the way from the mountains on the border between Hjelmeland and Jørpeland, to the sea at Tau. Not all the water would be removed, but the flow of water would be greatly reduced in the upper part of the river known as Sagåna, the rights to which is held jointly by Ola Måland and five other local farmers from Hjelmeland.

The water in question comes from a lake called Brokavatn, located 646 meters above sea level, where altitude soon drops rapidly, making the river suitable for hydropower development. Plans were already in place for such a project, which would use the water from just below the altitude of Brokavatn, to the valley in which the original owners' farms are located, about 80 meters above sea level. 

A rough estimate of the potential of this project was made by the NVE itself, stating that the energy yield would be 7.49 GWh per annum.\footnote{See \cite[16]{jorpeland09}.} This is about five times more energy than the water from Brokavatn would contribute to the project proposed by Jørpeland Kraft AS.\footnote{See \cite[19]{jorpeland09}.}

This estimate was not made in relation to the expropriation case, but as part of a national project to survey the remaining energy potential in Norwegian rivers.\footnote{The survey was carried out in 2004 and its results are summarised in \cite{jensen04}.} The owners were not informed of these assessments carried out regarding their resources. Moreover, even after Jørpeland Kraft AS had submitted a formal application for permission to divert the water, the owners were not notified that this would deprive them of a valuable natural resource.\footnote{See \cite[16]{jorpeland09}. The owners claimed not to have been notified of the diversion plans at all, but Jørpeland Kraft AS was able to convince the court of appeal that {\it they} had sent a generic orientation letter to substantially affected private individuals, including the owners of Sagåna. See \cite[5|8]{jorpeland11a}.}

Moreover, the procedural approach to the case was the traditional one, with an assessment directed at evaluating the environmental impact. Many interest groups were called on to comment on environmental consequences, and public debate arose with respect to the balancing of commercial interests and the desire to preserve wildlife and nature.\footnote{See \cite[19]{jorpeland09}.}

One of the local owners, Arne Ritland, also commented on the proposed project. He did this in an informal letter sent directly to Scana Steel Stavanger AS.\footnote{See \cite[17]{jorpeland09}.} In this letter, he inquired for further information and protested the proposed diversion of water from Brokavatn. He also mentioned the possibility that an alternative hydropower project could be undertaken by original owners, but he did not go into any details, stating only that a locally owned hydropower plant had previously been in operation in the area.\footnote{The plant he was referring to dates back to the time before there was a national grid. It ensured a local supply of electricity, but has since been shut down, in keeping with the general trend mentioned in chapter \ref{chap:4}, section \ref{sec:4:4}.}

Arne Ritland received a reply from Scana Steel Stavanger AS, which stated that more information on the project and its consequences would soon be provided. Ritland did not pursue the matter further at this time. Meanwhile, Scana Steel Stavanger AS submitted his letter to the NVE, who in turn presented it as a comment directed at the application.\footnote{\cite[18]{jorpeland09}.}

This prompted the majority owner of Jørpeland Kraft AS, Lyse Kraft AS, to undertake their own survey of alternative hydropower in Sagåna.\footnote{See \cite[19]{jorpeland09}.} The conclusions were sent to the water authorities, but the owners were not informed that such an investigation was being conducted.\footnote{See \cite[23]{jorpeland09}.} Moreover, the water authorities did not take steps to investigate the commercial potential of local hydropower on their own accord. Instead, they referred to the conclusion presented by Jørpeland Kraft AS, stating that if the local owners decided to build two hydropower plants in Sagåna, then one of them, in the upper part of the river, would not be profitable, not even with the contested water. The other project, in the lower part, could apparently still be carried out, even after the diversion.\footnote{See \cite[23]{jorpeland09}.}

No mention was made of what the original owners stood to loose, nor was there any argument given as to why it made sense to build two separate small-scale power plants in Sagåna. Nevertheless, the NVE handed the expropriating party's findings over to the Ministry, without conducting their own assessment and without informing the original owners.\footnote{See \cite[22-23]{jorpeland09}.}

In addition to the report made by Jørpeland Kraft AS, the municipality government of Hjelmeland also commented on the possibility of local hydropower. In their statement to the NVE, they directed attention to the data in the NVE's own national survey, which suggested that a single hydropower plant in Sagåna would be a highly beneficial undertaking.\footnote{See \cite[19]{jorpeland09}.} On this basis, they protested the diversion, arguing that original owners should be given the possibility of undertaking such a project. This statement was not communicated to the original owners, and in their final report the NVE stated that the most efficient use of the water would be to transfer it and harness it at Jørpeland.\footnote{See \cite[19]{jorpeland09}.}

In addition to the statement made by Ritland, one other property owner, Ola Måland, commented on the plans.\footnote{See \cite[17]{jorpeland09}.} He did so without having any knowledge of the commercial potential of the waterfall and without having been informed of the statement made by the municipality of Hjelmeland. On this basis, Måland expressed his support for Jørpeland Kraft's plans, citing that the risk of flooding in Sagåna would be reduced. He also phrased his letter in such a way that it could be interpreted as a statement on behalf of the owners as a group.\footnote{See \cite[17]{jorpeland09}.} However, Måland was the only person who signed.

In the final report to the Ministry, the NVE refers to Måland's letter and state that the owners are in favour of the plans.\footnote{See \cite[19]{jorpeland09}.} For this reason, the NVE concludes that the opinion of the municipality of Hjelmeland should not be given any weight.\footnote{See \cite[19]{jorpeland09}.} The NVE neglects to mention that Arne Ritland's statement strongly opposed the development. %Moreover, earlier in the report, where all incoming statements are reported, Ritland is referred to as a private individual, while Ola Måland is referred to as a property owner who speaks on behalf of the owners as a group.

The report made by the NVE was not communicated to the affected local owners at all, so the owners had no chance of correcting the mistakes that had been made. However, the report was sent to many other stakeholders, including the municipality of Hjelmeland.\footnote{See \cite[24]{jorpeland09}.} In light of the report, the municipality changed their original position and informed the Ministry that they would not press for local hydropower, since this was not what the affected owners (i.e., Ola Måland) wanted.\footnote{See \cite[24]{jorpeland09}.}

Again, this happened without the owners' knowledge. However, while the case was being prepared by the water authorities, the original owners had begun to seriously consider the potential for hydropower on their own accord. In late 2006, Jørpeland Kraft's application reached the Ministry and a decision was imminent. At the same time, the owners were under the impression that they would receive further information before the case progressed to the assessment stage.

All the owners, including Ola Måland, had now come to realise the value of the water from Brokavatn. Hence, they approached the NVE, inquiring about the status of the plans proposed by Jørpeland Kraft AS. They were subsequently informed that an opinion in support of the transfer had already been delivered to the Ministry. This communication took place in late November 2006, summarised in minutes from meetings between local owners, dated 21 and 29 November. On 15 December 2006, the King in Council granted a concession for Jørpeland Kraft AS to transfer the water from Brokavatn to Jørpeland.\footnote{See \cite[3]{jorpeland09}.}

At this point, it had become clear to the original owners that the water from Brokavatn would be crucial to the commercial potential of their own project. They also retrieved expert opinions that further substantiated that the NVE was wrong to conclude that diversion to Jørpeland would be the most efficient use of the water.\footnote{See \cite[23]{jorpeland09}.} In light of this, the owners decided to question the legality of the licence (with the corresponding permission to expropriate).

In the following section, I present the main legal arguments relied on by the parties, as well as a summary of how the three national courts judged the case.

\subsection{Legal Arguments}\label{sec:5:6:2}

First, the owners argued that procedural mistakes had been made by the water authorities when preparing the case.\footnote{See \cite[12]{jorpeland09}.} This, in turn, had resulted in factual mistakes forming the basis of the decision to grant the development license. Since the outcome might have been different if these mistakes had not been made, the owners concluded that the development license could not be upheld.

Second, the owners argued that expropriation of their rights would result in a disproportionate loss of an economic development potential.\footnote{See \cite[5]{jorpeland11a}.} Moreover, they argued that the economic loss would clearly be greater than the gain also from the point of view of the public, since the owners were in a position to make more efficient use of the contested water. Therefore, allowing expropriation would only serve to benefit the commercial interests of Jørpeland Kraft AS, to the detriment of both local and public interests.

Third, the owners argued that the government had not fulfilled its duty to consider the case with due care.\footnote{See \cite[12]{jorpeland09}.} In particular, the assessment of local community interests and the interests of local owners had not been satisfactory. The local owners should have been informed about the progress of the case, especially when assessments were made pertaining to their hydropower interests.

Fourth, the owners argued that irrespective of how the matter stood with respect to national law, the expropriation was unlawful because it would be in breach of the provisions in P1(1) of the ECHR regarding the protection of property.\footnote{See \cite[07-08]{jorpeland09}.}\indexonly{echr}

Jørpeland Kraft AS protested, arguing first that the report from the NVE was not based on factually erroneous information.\footnote{See \cite[16]{jorpeland11}.} With respect to the apparent mistakes that had been made, Jørpeland Kraft AS argued that these did not in any event undermine the quality of the report as a whole.\footnote{See \cite[2]{jorpeland11a}.} Moreover, it was argued that Måland had probably discussed the diversion of water with other affected owners, and that they had all probably agreed to support it.\footnote{See \cite[2]{jorpeland11a}.} 

Furthermore, according to Jørpeland Kraft AS, all the procedural rules of the \cite{wra17} had been observed. Other procedural rules might be relevant, but only if they were compatible with sector-specific practices.\footnote{See \cite[16]{jorpeland11}.} It was also argued that it was not for the courts to subject the assessment of public and private interests to any further scrutiny, since this was a matter for the administrative branch.\footnote{See \cite[2]{jorpeland11a}.} Finally, Jørpeland Kraft AS argued that diverting the water did not represent a breach of the owners' human rights.\footnote{See \cite[2]{jorpeland11a}.} They argued for this by pointing to the fact that the procedural rules had been followed and that the material decision was discretionary. Moreover, Jørpeland Kraft AS argued that since the owners would be compensated financially for whatever loss they incurred, it was clear that no human rights issues were at stake in the case.\footnote{See \cite[2]{jorpeland11a}.}

\subsection{The Lower Courts}\label{sec:5:6:3}

The matter went before the district court in the city of Stavanger, which decided in favour of the owners on 20 May 2009.\footnote{See \cite{jorpeland09}.} The district court agreed with the local  owners that the decision to grant the license was based on an erroneous account of the relevant facts.\footnote{See \cite[25]{jorpeland11}.} Moreover, the court concluded that it was evident that allowing the applicants to use the water from Brokavatn in their own hydroelectric scheme would be the most efficient way of harnessing the hydropower potential.\footnote{See \cite[22-23]{jorpeland09}.} This, the court noted, directly contradicted what the NVE had stated in their report.\footnote{See \cite[23]{jorpeland09}.}

The court backed up its conclusion by giving several direct quotes from the report made by the NVE. On the legal side, they relied on a well-established principle of administrative law: while the exercise of discretionary powers is usually not subject to review by court, a decision based on factual mistakes is invalid if it can be shown that the mistakes in question were such that they could have affected the outcome.\footnote{See \indexonly{paa67}\dni\cite[41]{paa67}; \cite[407-410]{eckhoff14}.} Since the small-scale alternative would in fact represent a more effective use of the water in question, the court was not in doubt that this principle applied here.\footnote{See \cite[25]{jorpeland09}.}

After this, there was no need to consider claims regarding the legitimacy of the diversion with respect to human rights law. However, the district court commented that the traditional procedure used to deal with diversion cases was inadequate and had to be supplemented by looking to the procedural rules in the \cite{ea59} and the \cite{paa67}.\footnote{See \cite[21]{jorpeland09}.}

Moreover, the court made a crucial statement about expropriation of riparian rights in general, regarding the duty of the water authorities to properly assess whether or not an expropriation license should be granted.\footnote{Compare \indexonly{ea59}\dni\cite[12]{ea59} and \indexonly{ea59}\dni\cite[16]{paa67}.} This duty, the court held, included a duty to properly consider negative effects on small-scale development potentials.\footnote{See \cite[22]{jorpeland09}.} According to the court, this was the natural consequence of the increasing interest in small-scale development. 

If the principle expressed by the district court on this point had been allowed to stand, it would have had significant implications for the water authorities. Specifically, it would directly confront their traditional lack of interest in the expropriation question. However, it was not to be, as the district court's decision was overturned on appeal.

The court of appeal approached the case very different than the district court. Specifically, its decision did not rely on any close assessment of the facts against the report made by the NVE. Instead, the court of appeal largely based its decision on the opinion that the rules in the \cite{wra17} exhaustively regulate the administrative procedure in watercourse regulation cases.\footnote{See \cite[7]{jorpeland11a}.} According to the court of appeal, the procedural rules in the \cite{ea59} and the \cite{paa67} do not apply at all to diversions of water authorised under section 16 of the \cite{wra17}.\footnote{See \cite[7]{jorpeland11a}.} 

This finding was based on the argument that the more specific rules of the \cite{wra17} have priority under the so-called {\it lex specialis} principle, which applies in case of conflict between different sets of rules, giving priority to those that are more specific.\footnote{See \cite[7]{jorpeland11a}.} Apparently, the court thought that the procedural rules of the \cite{ea59} and the \cite{paa67} conflict with the rules that apply specifically in hydropower cases.\footnote{The court also made a sweeping remark to the effect that the rules in the \cite{wra17} conform to all ``basic and general'' procedural demands of administrative law. This seems to be a reference to core unwritten principles, not specific provisions included in other Acts.} With regard to the procedural rules of the \cite{wra17}, the conclusion is that the NVE's assessment met all general requirements and was clearly adequate. Regarding the factual basis for the license, the court did not comment on most of the evidence presented to it. Moreover, the Court did not address those quoted segments of the report from the NVE that had formed the basis for the district court's decision.

Specifically, the court of appeal never mentions the objection to the transfer made by the municipality of Hjelmeland, nor does it mention the fact the small-scale alternative proposed by the municipality would use the contested water more effectively. Instead, the court of appeal points out that the NVE was well aware of the possibility of developing small-scale hydropower, was well-informed about such development, and had considered it during their assessment.\footnote{See \cite[9]{jorpeland11a}.} The court notes that the NVE's written assessment on this point was brief, but argues that this must be understood as a natural response to what the court of appeal describes as a lack of input from local owners.\footnote{See \cite[9]{jorpeland11a}.}

The owners appealed the court of appeal's decision to the Supreme Court, which decided to hear the  juridical aspects of the case.\footnote{See \cite[8]{jorpeland11}. Specifically, the Supreme Court would not engage in any independent fact-finding, but only consider legal questions, including how the law should be applied to the facts.}

\subsection{The Supreme Court}\label{sec:5:6:4}

The Supreme Court approached the case in much the same way as the court of appeal. Regarding the facts, the Court emphasised that the majority owner of Jørpeland Kraft AS had considered the possibility that a hydroelectric scheme could be undertaken by local property owners.\footnote{See \cite[53]{jorpeland11}.} As mentioned, Lyse Kraft AS had indeed made a report on this, concluding that one small-scale plant would be unprofitable regardless of the diversion, while another one, further down in the river, could still be carried out.\footnote{See \cite[23]{jorpeland09}.} As mentioned earlier, the report did not explain why anyone would want to build two consecutive small-scale plants so close to each other in the same river.\footnote{See \cite[16|23]{jorpeland09}.}

In any case, the follow-up question was what the owners stood to loose when the water from Brokavatn was diverted away from Sagåna. Both the report and the Supreme Court remained silent on this point. Moreover, the Court does not mention that the report was never handed over to the applicants, nor that the details of the calculations were never independently considered by the NVE. Just like the court of appeal, the Supreme Court also neglects to mention that small-scale development would be a more efficient use of the water, according to the national survey of small-scale potentials carried out by the NVE itself.\footnote{See \cite[16]{jorpeland09}.} Furthermore, no mention is made of the fact that the NVE claims that the opposite is true in the report to the Ministry, contradicting also the statement made by the municipality of Hjelmeland.

Regarding the legal questions raised by the case, the Supreme Court rejects the view that the procedural rules in the \cite{ea59} and the \cite{paa67} do not apply to the case.\footnote{See \cite[32-34]{jorpeland11}.} However, the Court holds that these procedural rules do not imply a more extensive duty to assess the expropriation question, compared to established practices in hydropower cases.\footnote{See \cite[51-52]{jorpeland11} (citing also the {\it Alta} case, \cite{alta82}).}

Importantly, there is no rule in the \cite{wra17} which states that the authorities are required to consider specifically the question of how the regulation affects the interests of property owners. This is also rarely done in practice.\footnote{See \cite{stokker10}. This is the water authorities' own guideline for the assessment of large-scale applications. The previous version of this guideline (which also fails to mention the interests of owners) was presented to the Supreme Court. The Court also refers to it explicitly when it comments that existing practices are beyond reproach. See \cite[51]{jorpeland11}.} However, a rule explicitly demanding this is found in section 2 of the \cite{ea59}. 

This is not regarded as a procedural rule, as it pertains to the {\it content} of material considerations. Hence, according to the Supreme Court, the rule does not apply at all when expropriation takes place on the basis of section 16 of the \cite{wra17}.\footnote{See \cite[30]{jorpeland11}.} This is the conclusion despite the fact that section 30 of the \cite{ea59} explicitly states that the provisions of that Act apply to expropriations pursuant to the \cite{wra17}, in so far as they are compatible with the rules therein. It would appear to follow, by implication, that the Supreme Court does {\it not} think that directing more attention at owners' interests is compatible with the \cite{wra17}.

This is a clear rejection of the principled position taken by the district court, whereby the water authorities should generally be obliged to consider small-scale alternatives before allowing expropriation. According to the Supreme Court, no special procedural obligations arise on account of the owners' interests in small-scale development. Essentially, cases that involve expropriation can be processed as though the waterfalls already belonged to the applicant; expropriation is permitted to remain a non-issue during the licensing process pursuant to the \cite{wra17}.\footnote{Moreover, the procedural standards that the water authorities were held to in {\it Jørpeland} appear to be essentially the same as those formulated by the Court in \cite{alta82}. In particular, the {\it Jørpeland} Court maintained a controversial position on the duty to assess alternatives, see \cite[157]{winge13}.}

Formally, this implication of {\it Jørpeland} only applies to expropriations carried out on the basis of section 16 of that act. However, in practice, there is reason to believe that the impact will be the same for all cases involving large-scale hydropower development. Indeed, the water authorities themselves do not appear to make any significant distinction between cases where expropriation is automatic and cases where a separate license is formally required.\footnote{See \cite{flatby08}.}

\noo{ %It also bears noting that the facts in {\it Jørpeland} appear to suggest that the procedural shortcomings underlying that case were much more obvious than the shortcomings complained of in {\it Alta} (although the scale of the underlying conflict was much greater in {\it Alta}).
To further illustrate the extent to which {\it Jørpeland} signals a dismissive attitude towards owners and local communities, I will conclude by offering a quote from Harald Solli, director of the hydropower licensing section at the Ministry. Sollie submitted written evidence to the Supreme Court regarding the practices observed in cases involving expropriation of water power. Below, I quote two exchanges that demonstrate how current practices leave local owners in a precarious position.

\begin{quote}
Q: In cases pursuant to the \cite{wra17}, is it common for the water authorities to send prior written notices to the private owners that may be affected by a loss of a small-scale hydropower potential? \\
A: The procedural rules that apply to cases pursuant to the \cite{wra17} are found in section 6. To give such a written notice to private owners is not required. As far as I am aware, it is also not done, but I have no first-hand knowledge of this, since the NVE is responsible for the case at this stage. \\
Q: In cases such as this, should owners affected by the loss of a small-scale hydropower potential be kept informed about the factual basis on which the authorities plan to make their decision? I am thinking especially about cases when the authorities do in fact provide an assessment of the potential for small-scale hydropower on private properties. \\
A: Affected owners must look after their own interests. The assessments made by the NVE in their report is a public document, and it can be accessed through the homepage of the NVE.
\end{quote}

By their reasoning in \emph{Jørpeland}, it appears that the Supreme Court gave this dismissive attitude towards local owners a stamp of approval. In light of this, I believe the study of the law in a socio-legal setting becomes all the more relevant. For while the dismissive attitude might be a part of the national legal order, it seems pertinent to ask if it is a reasonable attitude to take towards local owners of valuable natural resources. Also, one may ask if a case can be made with respect to human rights, by arguing that the protection awarded is insufficient with regard to P1(1). This point was raised in \emph{Jørpeland}, but did not receive any attention from the Supreme Court.\footnote{The {\it Jørpeland} case resulted in a complaint to the ECtHR which has yet to be considered by the Court.}
}

\section{Predation?}\label{sec:5:7}

How should takings of waterfalls be assessed according to the normative theory developed in the first part of this theory? In chapter \ref{chap:3}, I presented the Gray test, a set of key assessment points for determining whether a taking violates important property norms.\footnote{See chapter \ref{chap:3}, section \ref{sec:3:5}.} In the following, I briefly assess takings of waterfalls against the criteria of the Gray test, to shed further light on the normative status of current practices observed in Norway.

\subsubsection{The Balance of Power}\label{sec:5:7:1}

In light of the presentation so far, it is safe to conclude that typical large-scale waterfall expropriations in Norway are marked by a severe imbalance of power between the taker and the owners. The economic and political power of local communities is clearly very limited compared to that of the large energy companies. Moreover, it is interesting to observe that this imbalance is accentuated by procedural arrangements and practices presided over by the water authorities. As demonstrated by the case of {\it Jørpeland}, the formal position of owners and local communities under administrative law is very weak in hydropower cases. Hence, in addition to shedding doubt on the legitimacy of current practices in Norway, assessing waterfall takings against the balance of power criterion also underscores that this criterion is related to administrative law.

Ideally, procedural rules should function so as to maintain an appropriate balance of power between the different actors involved in an administrative dispute. At least, the rich and powerful should not be allowed to dominate decision-making processes within the polity, at the expense of those most intimately affected by the decisions reached. If the administrative branch fails in this regard, or acts in such a way that existing imbalances are worsened, this is surely a cause for additional criticism with respect to the balance of power criterion. I believe the case study so far shows that the framework for management of Norwegian hydropower is deserving of such criticism.

\subsubsection{The Net Effect on the Parties}\label{sec:5:7:2}

The immediate financial effect that a taking for hydropower has on the owners depends on how the compensation is calculated. As discussed in section \ref{sec:5:4:3}, the law on this point has been in turmoil in recent years. In the late 2000s, there were signs that a commercially realistic valuation method might become dominant, leading in turn to a dramatic increase in compensation compared to earlier practice based on the natural horsepower method. But this trend now appears to have been reversed, as the energy companies have successfully argued that a license for large-scale development counts as proof that owner-led projects would not in any case have been `foreseeable' (because the necessary licenses would not have been granted). For this reason, the argument goes, owners suffer no actual loss when their resources are taken from them.\footnote{See \cite{otra13}.}

The local owners are in an even weaker position when it comes to indirect financial effects, as well as social and political effects, such as harms done to the cohesion and prosperity of the local community. In this regard, losses are not only under-compensated, they are typically not acknowledged at all, neither by the executive nor by the courts. The effects that go unnoticed range from the concrete, such as losses incurred because the expropriation proceedings drag out in time, to the abstract, such as the damage that is done to democracy when owners and local municipality governments are replaced by energy companies as the primary resource managers in the local district.\footnote{In \cite{smibelg15}, the owners submitted applications for small-scale hydropower in several affected rivers in 2005, but the water authorities refused to process these on account of the pending large-scale application. In 2015, compensation was awarded based on the natural horsepower method, with no compensation for, or discussion of, the owner's loss in the 10 year period where the water authorities refused to process the owners' applications.}

\subsubsection{Initiative}\label{sec:5:7:3}

It follows from the regulatory framework that almost all cases involving expropriation for hydropower development originate from applications submitted by commercial companies.\footnote{See especially chapter \ref{chap:4}, section \ref{sec:4:4} and chapter \ref{chap:5}, section \ref{sec:5:3}.} The energy company draws up the plans and initiates the expropriation proceedings, by submitting a request for a development license to the water authorities. The main purpose, which is usually acknowledged by both the applicant and the water authorities, is to make money. Hence, it is usually hard or impossible to argue that takings of waterfalls for hydropower development in Norway are motivated by any direct public interests. %Indeed, applying the initiative test will suffice to conclude that the primary motive is profit-making. 

Exceptions to this are possible, in so far as the energy companies themselves embody public service functions. In some cases one might argue that they do, but such arguments are becoming increasingly unconvincing due to the fact that most energy companies have been reorganised as for-profit enterprises whose activities are largely unconstrained by institutions of local government.\footnote{See, e.g., the EFTA Court's description of the industry, \cite{efta07}.}

\subsubsection{Location}\label{sec:5:7:4}

\noo{Compared to notorious US cases such as {\it Kelo} and {\it Poletown}, the stakes for the owners appear lower in hydropower cases from Norway.\footnote{See \cite{poletown81,kelo05}.} However,} As mentioned in chapter 4, riparian rights are often of great importance to Norwegian farming communities and the subsistence of its members.\footnote{See especially chapter \ref{chap:4}, sections \ref{sec:4:2}, \ref{sec:4:4} and \ref{sec:4:5}.} Indeed, the taking of riparian rights from a local community might well contribute significantly to depopulation, although indirectly rather than by physical displacement.\footnote{Today, it is very unlikely that the Norwegian government would sanction physical displacement of people from their homes in order to facilitate hydropower development. However, this state of affairs cannot be taken for granted; the current political attitude on this point appears to have arisen in large part due to extensive and forceful anti-development activism during the 1970s, especially in relation to the {\it Alta} case (which initially involved plans to physically displace a local Sami community). See \cite{altawiki}.} Moreover, in many rural communities, small-scale hydropower appears to be the only growth industry, as farming is becoming increasingly unprofitable and communities are threatened by stagnation and decline.\footnote{For an example, I refer again to the case of the Gloppen municipality, discussed in chapter \ref{chap:4}, section \ref{sec:4:4}.}

Hence, the location criterion suggests that takings of waterfalls merit heightened critical scrutiny, especially due to the importance of the property that is taken to the subsistence of the local communities forced to give it up.

\subsubsection{Social Merit}\label{sec:5:7:5}

There is no shortage of electric energy in Norway, and electricity prices are very low compared to the rest of Europe.\footnote{See chapter \ref{chap:4}, section \ref{sec:4:1}.} Indeed, development projects such as {\it Jørpeland} are not motivated by any particular need to supply more energy to the Norwegian people or local industry, but openly pursued as commercial endeavours.\footnote{See chapter \ref{chap:5}, section \ref{sec:5:6}.} Hence, they do not appear to have any particular social merit.

On the contrary, waterfall takings can contribute to creating social ills, by exacerbating rural decline and depopulation. In south-western Norway, for instance, the average income for a sheep farmer corresponds to about half of the minimum wage that farmers are required to pay to full-time farm workers.\footnote{According to the Norwegian Bioresearch Institute, the average sheep farmer in the western part of Norway could expect to earn NOK 65 per hour from working at their farm in 2012. See \cite[50]{smesdal14}. The minimum wage for unskilled farm workers during the same time was NOK 123.15 per hour. The minimum wage for 16-17 year old vacation workers was NOK 83.75 per hour. See \cite{tariff12}.} During harvesting season, sheep farmers wishing to hire 16 year old vacation workers are required to pay the kids about 30 \% more per hour than they themselves can expect to earn from running their own farms. In short, sheep farming communities in western Norway, such as that affected by the taking in {\it Jørpeland}, are struggling.

In this context, it seems that the social harm created by expropriation, whereby disadvantaged rural communities are deprived of the opportunity to manage their own water resources, should be a pressing concern. Because of the traditional approach to hydropower, focusing solely on environmental harms, the social merit of maintaining local ownership and control over resources receives little or no attention from the water authorities in expropriation cases. This in itself suggests that typical cases of waterfall expropriation in Norway will tend to fail the social merit test.

\subsubsection{Environmental Impact}\label{sec:5:7:6}

It is clear that hydropower development can have negative environmental impacts. Hence, it is important that the value of development is appropriately balanced against environmental interests. To ensure this is a core aspect of the regulatory system. As discussed in chapter 4, local initiatives for small-scale hydropower are now typically scrutinized quite intensely in this regard, particularly after reforms in recent years. By contrast, large-scale projects appear increasingly likely to receive preferential treatment. %Since 2000, only one such project has been denied a license by the water authorities.\footnote{....}
 
Moreover, the large companies are clearly in a better position to exert pressure on the regulator in order to overcome regulatory hurdles. The fact that large-scale solutions continued to receive priority, despite it being official government policy for a decade that no more large-scale plants should be built, is an indication of the severity of this effect. Hence, while the debate continues regarding the comparative environmental merits of different kinds of hydropower, it appears safe to conclude that the dynamics of power on display in relation to environmental issues raise further doubts about the legitimacy of waterfall takings.

\subsubsection{Regulatory Impact}\label{sec:5:7:7}

When waterfall rights are expropriated, they also become a separate commodity, divorced from the surrounding land rights. They are also typically removed from the sphere of municipal control on land use, falling instead under the regulatory jurisdiction of the centralised water authorities.\footnote{Specially, the water resource legislation will take priority over the otherwise rather municipality-empowering \cite{pb08}.} Hence, the regulatory context shifts from one emphasising holistic resource management and local community needs to one which focuses mainly on facilitating hydropower development.

Moreover, the fact that the takers of waterfalls are powerful actors might make it harder for regulators to do their job. After expropriation, the parties who stand to loose from increased regulation are the state-supported energy companies. They are therefore likely to oppose stricter standards, and to do so in a manner that is much more forceful than any lobbying one might expect from local community owners of waterfalls. Hence, takings in this sector appear likely to cause systemic imbalances and a push for less intrusive government control, or government regulation on terms dictated by the major market players. A sign of this effect can be found in recent controversial decisions made with regard to the national grid, where the interests of the electricity industry appears to have completely overshadowed broad public opposition against further environmental intrusions in valuable nature areas in the west of Norway.

\subsubsection{Impact on Non-Owners}\label{sec:5:7:8}

Non-owners can exercise some influence during the licensing procedure. However, this typically requires them to be organised or aligned with special interest groups, to effectively stand up to the authority of the prospective takers, who are expected by the government to take the lead during the administrative process.\footnote{See the discussion in chapter \ref{chap:4}, section \ref{sec:4:3:1}.} If there are objections, organisations, rather than individuals, are prioritised under the \cite{wra17}. The non-owners most directly affected by hydropower development are usually local residents, from the same community as the waterfall owners. They have little chance of being heard in the process, except if they find that their interests are aligned with those of more powerful stakeholders, such as national or regional environmental groups. In general, the means available for local non-owners to influence the decision-making do not appear commensurate with the local stakes in hydropower cases.

The transfer of property to a large-scale owner, moreover, changes the dynamic of interaction between owners and non-owners. Formally, the transfer of riparian rights away from the jurisdiction of municipal governments is particularly significant, since it significantly reduces the level of (local) democratic control over the use of the water resource. In addition, one should again consider the informal effect of transferring property away from local community members to large corporations. Unlike local owners, corporations that take waterfalls appear highly unlikely to interact with local non-owners on equal terms.

\subsubsection{Democratic Merit}\label{sec:5:7:9}

Following \emph{Jørpeland}, it seems that owners' right to participate in decision-making processes regarding the use of their rivers and waterfalls is extremely limited under Norwegian law. The regulatory system effectively negates private property rights by making expropriation an automatic consequence of any large-scale development license granted to any non-owner. The original justification for this might be found in the idea that the regulatory power of the state should take precedence over private proprietary entitlements. However, after the liberalisation of the energy sector, this idea has transformed into a practice of systematic prioritisation of powerful commercial interests at the expense of local communities. This has happened despite the political commitment to end large-scale development, which remained official government policy for over a decade.

In light of this, it is difficult to see any democratic merit in the practice of taking waterfalls for profit, to the benefit of large-scale development companies. Overall, it seems clear that according to the Gray test, current rules and practices regarding takings for hydropower render such takings highly suspect with regard to the question of legitimacy.

\subsection{A Brief Note on Contagion}

The lack of legitimacy identified in this chapter presents as a systemic problem in Norway, where the  regulatory framework appears rigged in favour of large state-backed companies. There are signs that the structural and ideological premise of this model is currently being exported abroad. According to a recent report on Norway's involvement in hydropower in Nepal, the same basic pattern is repeated there: the Norwegian energy companies generate large profits from their cooperation with the Nepali government, at the expense of local populations and the rights of the poor.\footnote{See \cite{gaarder15} (discussing how the Nepali government entered into an agreement where the price of electricity would be set at a high level, denominated in US dollars, and adjusted upwards proportionally to US consumer prices; apparently, the project involves a transfer of money from Nepal to Norway that far exceeds the flow of money going the other way). The report was commissioned from Fremtiden i Våre Hender, one of the largest environmental organisations in Norway.} In addition to the 15 per cent of Nepali electricity that Norwegian companies already control, there are plans for a massive new project that will make Norwegian interests dominant while doubling the electricity output of Nepal. This project was initially put on hold, but now seems to be back on the table in anticipation of a deal that will allow the Norwegians to sell Nepali electricity to more affluent consumers in India.

Hence, it appears that what was intended as a form of development assistance, subsidised over the aid budget of the Norwegian government, has become a vehicle of oppression, arguably tracking the trajectory of the hydropower sector within Norway's own borders.\footnote{Moreover, the Nepali state appears particularly susceptible to this kind of abuse due to national elitism and the lack of adequate institutions for local democracy and self-governance with respect to natural resources. See \cite[644]{peris12} (commenting on hydropower development specifically, arguing that international conventions on indigenous rights are unable to deliver social justice on the ground due to the ``democratic deficit'' of decision-making regarding development in Nepal; arguing also that small-scale hydropower offers a potential legitimacy-enhancing alternative).} There is reason to think that this mechanism is on display elsewhere as well. The largest hydropower company in Norway, the state-owned, Statkraft SF has considerable ownership stakes in (controversial) hydropower projects in many countries, including Turkey, Brazil, India and Laos.

This illustrates the importance of maintaining an institutional fairness perspective on legitimacy, to realise the danger of excessive deference in situations when organs of state appear to be loosing their sense of direction, embarking instead on a path towards tyranny, dispossession, and unbridled greed. In these situations, there is reason to think that an egalitarian distribution of property will be one of the first democratic institution to come under attack, targeted for its often unspoken opposition to elitism and usury.

\section{Conclusion}\label{sec:5:8}

This chapter has explored expropriation of waterfalls, focusing on the legitimacy issue. The presentation has demonstrated how waterfall rights have effectively been rendered subservient to the management framework set up by the sector-oriented water resource legislation. Specifically, the chapter tracked a transformation of the regulatory framework whereby the licensing authority is now used by the government to exercise {\it de facto} proprietary control over water resources, largely unconstrained by the fact that these resources nominally remain in private ownership.

As such, this chapter has shed further light on the tension identified at the beginning of the previous chapter, between hydropower as private property and hydropower as a `national asset'. Importantly, the flavour of this national asset is strongly influenced by the liberalisation of the electricity sector. In particular, this chapter has made the case that takings of waterfalls today are pure takings for profit. In most cases, the government itself does not even feel the need to argue otherwise, since expropriation simply follows automatically from large-scale development licenses granted to commercial energy companies.

The chapter used the case of {\it Jørpeland} to shed light on the effect that this can have in practice, showing how owners desiring to carry out alternative projects can be completely marginalised in the decision-making process, regardless of the merits of their proposals. In light of this, based on a combination of concrete and general observations about the Norwegian system, I concluded that this system fails to deliver on legitimacy in the sense of the word explored in Part I of this thesis. 

This chapter has thus identified a legitimacy problem, raising the question of how to resolve it. This is addressed in the next chapter, where I consider the institution of land consolidation and how it is used as an alternative to expropriation in Norway.% which local owners have made very active use of in cases when {\it they} wish to impose economic development on recalcitrant neighbours. Arguably, the consolidation approach represents a solution in line with Ostrom's design principles for local self-governance and common pool resource management. However, the system does have some idiosyncratic features which, if anything, increases the degree of control that stakeholders other than the owners can exercise over the resource in question. This and more is explored in depth in the next chapter of this thesis.
%\chapter{Just compensation}\label{chap:5}

\section{Introduction}\label{sec:into5}

In this Chapter, I consider the question of compensation for waterfalls in more depth. The main issue that arises is whether or not owners should be compensated for the loss of a commercial hydropower potential. If so, the compensation payments can be very large, so large that expropriation will no longer be a feasible option. Traditionally, however, no such compensation was awarded and the amounts paid to owners were negligible. In fact, owners would often have been left with nothing at all, were it not for the fact that a theoretical compensation formula was developed which avoided this outcome, by ensuring some degree of benefit sharing.

The question of whether or not to base compensation on the loss of a commercial hydropower potential is closely related to the so-called ``no scheme'' principle, according to which compensation is to be based on the value of the property such as it would have been if the expropriation scheme had not been authorised. If one takes the view that hydropower development is the prerogative of the party that obtained such an authorisation, it follows from the principle that no compensation is payable to the owners of the waterfalls, at least not for the hydropower potential. The value of hydropower, in particular, is then regarded as being due to the scheme, not due to the natural resource that the waterfall represents. This perspective was implicitly adopted in Norwegian law from the early 20th to the early 21st century, when it began to loose ground due to the liberalization of the energy sector.

The structure of this chapter is as follows: In Section \ref{sec:nsp}, I provide a comparative and theoretical context for the case study that is to follow. I do so by discussing the origin and current status of the no-scheme principle in UK law. This facilitates a broader perspective on the data presented on Norwegian law in subsequent sections.  It also allows me to make a more general point, namely that the need to distinguish between commercial/private and public values inherent in a development project arises with great force when one attempts to apply the no-scheme principle in the context of an economic development taking. I identify the lack of a well-developed framework for making such a distinction as one of the main problems associated with such takings. The main worry is that the principle, when applied to commercial values associated with a development scheme, results in {\it discrimination}. Some categories of owners are entitled to commercial benefits that other categories of owners are effectively deprived of by an application of the no-scheme principle.

In Section \ref{sec:norcom}, I go on to present Norwegian compensation law, with a focus on various manifestations of the no-scheme principle. I also present the special judicial procedure used to award compensation following expropriation. I pay particular attention to the fact that it relies on the use of lay appraisers, who also have considerable influence over the application of the law in such cases. This system, I argue, is potentially very flexible, allowing compensation awards to be based on broad and contextual fairness considerations, rather than static application of special rules. It means, in particular, that the no-scheme principle is not applied without exception, even if it does have status as a general principle. I finish this section by noting that the traditional system has been somewhat undermined since WW2, following legislation specifically aimed at reducing compensation payments and narrowing the room for lay discretion in appraisal disputes.

In Section \ref{sec:nathp}, I move on to consider compensation for waterfalls. The first thing I note is that the no-scheme principle was not traditionally applied, since it would have led to little or no compensation for owners during the monopoly era. Instead, a theoretical method was used, based on the notion of natural horsepower. This method was meant to give owners a share of the benefit in hydropower development and was developed by the appraisal courts early in the 20th century. It was modelled on the market for waterfalls that had existed prior to monopolization, but as the years went by, the method became farther and farther removed from the physical and economic realities of the hydropower industry. Increasingly, it resulted in no more than a symbolic form of benefit sharing with owners.

Following liberalization the traditional method has been abandoned for several categories of cases, a development that I discuss in Section \ref{sec:fa}. The crucial condition for applying the new method, based on market-value assessment, is that an alternative development scheme would have been ``foreseeable'' in the absence of the expropriation project. How to interpret the meaning of ``foreseeable'', and how to determine the scope of the ``expropriation project'', are crucial issues currently being worked out in Norwegian case law. I link the reform in this area with the institutional framework surrounding appraisal disputes. I note, in particular, that the natural horsepower method was first abandoned by the appraisal courts, with the lay appraisers assuming a leading role. 

Unfortunately, the market-based method also raises problems, the most severe of which are related to the scope of the no-scheme principle. The lack of clarity about its scope and implication has created a situation where it appears that if owners are lucky, or employ skilled arguers, they can collect a very substantial sum of money with little or no effort and with no social responsibilities attached. On the other hand, if they are unlucky, they are forced to give up what is often the most valuable asset of their local community for nothing but a symbolic payment. I conclude by arguing that a much better approach would be to try and get owners involved in sustainable hydropower in a way that can remove the need for expropriation altogether. 

This sets the stage for the last chapter of this thesis, where I return to the question of how to replace expropriation by mechanisms of participatory democracy, referring back also to the discussion in Chapter \ref{chap:1}.

%As development is now organized as a commercial pursuit, this should in principle be possible, since the owners {\it do} have an incentive to get involved, also in cases when the public dictate the set of possible terms through strict regulation. In practice, however, what is needed is a mechanism for organizing such owner-involvement. This mechanism will undoubtedly also need to be endowed with powers of coercion if it is to be effective.
%
%The Supreme Court struck down their judgement on a technicality, but refused to reject the principle that lay people were free to adopt a new method in cases when the traditional method would not adequately reflect the value of ``foreseeable'' use. I argue that this shows the strength of a long tradition of respecting the discretion of lay people in appraisement disputes. Many legal scholars, in particular, had previously regarded the natural horsepower method as a {\it rule of law}, set by precedent.
%
%The method has not been abandoned as a matter of principle, however. As made clear recently by the Supreme Court, it is still to be applied in cases when a calculation based on ``foreseeable use'' does not lead to higher compensation payments. The crucial question becomes what exactly is meant by this notion. I address this in some depth, by pointing to how Norwegian law in general is  marked by a tendency to disregard any use that is not sanctioned by public plans, including in cases when these plans themselves provide the rationale for expropriation. This appears to be contrary to the no-scheme principle, demonstrating more generally that only ``one half'' of the principle tend to apply in a Norwegian setting. In so far as the principle precludes giving the owner a share of the expropriation surplus, it is applied, but in so far as it entitles him to compensation based on a future use that is rendered unforeseeable by the planning underlying the expropriation license, it is not.
%
%There are some exceptions to this, however, and the Supreme Court has indicated that one of them applies to hydropower cases. At the same time, however, it has been stressed that even if the expropriation plans themselves are not binding for the compensation assessment, the ``public rationale'' underlying these plans must be taken into consideration when awarding compensation. In effect, this means that compensation is not offered for alternative uses in so far as the project proposed by the expropriating party is superior and could not be undertaken by the current owner. In effect, it seems that a partly {\it subjective} standard is introduced into compensation law, whereby local owners are denied compensation for a commercial value that is deemed to be such that it is only realizable by the expropriating party.
%
%The Supreme Court has not been entirely consistent about the scope and exact content of the ``public rationale'' principle, however,   and the issue is still very much contested in Norwegian courts. In Section \ref{sec:ko}, I illustrate the current unclear state of the law by contrasting two recent Supreme Court cases. In the first, the court embraced an objective version of the ``public rationale'' principle by holding that as the expropriating party's project resulted in more public benefits, compensation could be based on the premise that the owners' foreseeable use of the waterfalls was to cooperate with the large energy company in realizing the plans, to take their share of its commercial potential. 
%
%In {\it Otra II} on the other hand, the Court held that this should not be the conclusion in so far as cooperation was deemed to be ``impractical'', following a concrete assessment of the facts. It seems quite clear that the notion of ``impracticality'', as it was used here, serves to introduce a subjective assessment standard, contrary to what otherwise dictated by Norwegian compensation law. 
%I go on to consider the merits of {\it Otra II} against human rights law, anticipating also the outcome of the appeal currently lodged with the ECtHR in Strasbourg.
%
%Finally, I conclude that the case law on compensation demonstrates the intrinsic inadequacy of a narrow perspective on takings for profit. It seems clear, in particular, that all of the approaches currently in use to calculate compensation for waterfalls leave great room for bickering, manipulation and long-winded court battles. Moreover, the factual premise for the calculation is typically extremely uncertain, meaning that the whole procedure appears as something of a gamble, for both owners and developers. Hence, the developers favour the use of the natural horsepower method, which is completely removed from the reality of hydropower, but deliver predictably low compensation payments that will not prove too damaging to the profit-margin of the development company. On the other hand, owners have an incentive to push for compensation mechanisms that will allow them to collect the entire financial potential of hydropower development without actually investing any effort in planning or administerting such development, and without subjecting themselves to any of the risks involved. 

\section{The ``no scheme'' principle}\label{sec:nsp}

In most jurisdictions, a fundamental principle relating to compensation following expropriation is that compensation should be calculated without taking into account changes in the property's value that are due to the expropriation, or the scheme underlying it. In short, compensation should be based on the owner's loss, not the taker's gain. In a recent Law Commission consultation paper, this principle is referred to as the \emph{no-scheme} rule, a terminology I will also adopt here, noting that while the exact details of the rule might differ between jurisdictions, the underlying principle appears to play a crucial role in both civil and common law traditions for regulating compensation following expropriation.\footnote{I am not aware of a single jurisdiction that does not include some rule corresponding to (aspects of) the no-scheme principle. I mention that in addition to the jurisdictions discussed in this section, no-scheme rules are also found in pure civil law jurisdictions like Germany and the Netherlands, see \cite[5,21]{sluysmans14}.}

While the no-scheme principle is easy enough to comprehend when it is stated in general terms, it raises many difficult questions when it is to be applied in concrete cases. What the rule asks of the valuers, in particular, is quite daunting; they are forced to consider a counterfactual ``no-scheme world'', and they must calculate the value of the property based on the workings of such an imaginary world. The crucial question that arises, of course, is the question of what exactly this world should be taken to look like.

In the first instance, it might be tempting to state simply that this is a ``question of fact for the arbitrator in each case'', as expressed by the Privy Council in \emph{Fraser}, a Canadian case from 1917.\footnote{\cite[194]{fraser17}.} However, as the history of the no-scheme rule has shown, this point of view is not tenable.\footnote{For a history of the rule in UK law, clearly illustrating the difficulty in interpreting it and applying it to concrete cases, I point to Appendix D of \cite{lawcom03}. See also \cite{lawcom01}.}  The problem is that the nature of the no-scheme world cannot be determined without making a vast range of assumptions, many of which appear to depend on how one understands the law. The challenges that arise were discussed in great detail by Lord Nicholls in the recent case of \emph{Waters}. He described the task as ``daunting'', noting also that some of the more recent statutory provisions ``defy ready comprehension''.\footnote{\cite[19]{waters04}.}

\noo{
\begin{quote}
The extreme complexity of the issues that I have had to consider, the
uncertainty in the law, the obscurity of the statutory provisions, and
the difficulties of looking back over a long period of time in order to
decide what would have happened in the no-scheme world
demonstrate, in my view, that legislation is badly needed in order to
produce a simpler and clearer compensation regime. I believe that
fairness, both to claimants and to acquiring authorities, requires
this
\end{quote}
}
The Lords clearly saw \emph{Waters} as an opportunity to offer a clarification on the no-scheme rule and how to interpret it. In particular, their judgement went into more detail than what seemed necessary for the case at hand. Even if it was not needed for the result, the Lords also addressed many of the issues raised by the Law Commission in their recent report, focusing particularly on resolving the tension which was identified there between the principle relied on in the \emph{Pointe Gourde} case and the reasoning adopted in the so-called \emph{Indian} case from 1939.\footnote{\cite{indian39,gourde47}.} In the \emph{Indian} case, the scheme was given a very narrow interpretation, with Lord Romer interpreting the scope as follows.\footcite[319]{indian39}

\begin{quote}
The only difference that the scheme has made is that the acquiring
authority, who before the scheme were possible purchasers only, have
become purchasers who are under a pressing need to acquire the
land; and that is a circumstance that is never allowed to enhance the
value.
\end{quote}

Importantly, this did not entail that the purchaser's demand for the property was to be disregarded, since, as Lord Romer puts it:\footcite[316-317]{indian39}

\begin{quote}
[...] The fact is that the only possible purchaser of a potentiality is
usually quite willing to pay for it […]
\end{quote}

In \emph{Pointe Gourde}, a different stance appears to have been adopted.\footcite{gourde47} The case concerned a quarry that was expropriated for the construction of a US naval base in Trinidad. The quarry had value to the owner as a business, and the valuer had found that if the quarry had not been forcibly acquired, it could also have supplied the US navel base on a voluntary basis, thereby increasing its profits. However, the value of this potential fell to be disregarded, with Lord MacDermott describing the no-scheme rule as follows:\footcite[572]{gourde47}

\begin{quote}
It is well settled that compensation for the compulsory acquisition of
land cannot include an increase in value, which is entirely due to the
scheme underlying the acquisition
\end{quote}

Seemingly, this is at odds with the position taken by Lord Romer in the {\it Indian} case. It seems clear that in the absence of a compulsory purchase order, the US would have been ``quite willing'' to pay for the quarry's services. Still, this potential had to be disregarded. 

In \emph{Waters}, both Lord Nicholls and Lord Scott addressed the tension between the two decisions in great detail. They then offered a reconciliatory interpretation, one which seems to narrow the no-scheme rule compared to how it has most commonly been understood following \emph{Pointe Gourde}. Moreover, the House of Lords also noted the need for reform and legislation, with Lord Scott describing the current state of the law as ``highly unsatisfactory''.\footcite[164]{waters04}

To explain how a seemingly simple principle could become so troubling in practice, I think it is important to start by noting that after the introduction of extensive planning legislation in the 20th century, development of property tends to be contingent on governmental licenses and plans. Moreover, the power to expropriate is often granted as a result of comprehensive regulation of the property-use in an area, often following public plans that encompass more than the particular project that will benefit from compulsory purchase. As a result, it has become increasingly difficult to ascertain what is meant by the ``scheme'' in compensation cases. Does it include the whole planning history leading to expropriation, does it only refer to the power to expropriate, or is it something in between?

A fine balancing act must be made when attempting to answer this question. Under a wide interpretation of ``the scheme'', forcing the valuer to entertain many counterfactual assumptions, the property owner might come to feel that he is not compensated for his true loss, but rather an imaginary one. Indeed, the no-scheme world that the valuer must consider can end up being far removed from the actual one, forcing him to go back many years, perhaps decades, to establish what would have been the status of the property in question if the sequence of planning steps eventually leading to expropriation had not taken place. 

This can leave the property owner in an unpredictable and very weak position. Taken to extremes, the no-scheme principle can then also come to run amiss with respect to human rights law and constitutional provisions protecting private property. On the other hand, if the scheme is interpreted too narrowly, one runs the risk of endangering important public schemes by compelling the public to pay extortionate amounts. In many cases, it is undoubtedly true that the value of property is increased by public investments and plans for the area in which the property is found. Moreover, one may ask if it is right to pay compensation based on increases in value that result from investments and plans that would not have materialised unless the power to expropriate had been anticipated. This, it may be argued, would be a form of double payment that should be avoided.

As noted by the Law Commission, it is important to keep in mind that the no-scheme rule serves at two distinct purposes.\footcite[69-70]{lawcom03} First, the rule has an important \emph{positive} dimension, enhancing compensation payments. Property owners are not only compensated for the direct loss of their property, but also for the possible depreciation of their property's value following the decision to carry out a scheme which requires expropriation. Seemingly, this is easy to justify: It seems intuitively unreasonable if the deleterious effects of a threat of compulsion is permitted to result in reduced compensation payments.

However, under the extensive planning regimes common today, it is not clear where to draw the line. When is the regulation leading up to the scheme to be regarded as reflecting general public control over property use, and when is it to be regarded as a measure specifically aimed at compelling private owners to give up their property? As we will see when we consider the role of the no-scheme rule in Norwegian law, this question can easily become highly controversial, especially when it is linked with the more general question of whether or not the state should be liable to pay compensation for regulation that adversely affects the potential for future development. In jurisdictions that do not recognize owners' right to such compensation, like Norway and England, it is easily argued that the positive aspect of the no-scheme rule must be limited correspondingly. Why should a depreciation of value following regulation imply compensation when the property is eventually expropriated, but not otherwise?

In addition to its positive dimension, the no-scheme rule also has an important \emph{negative} dimension, expressed in {\it Pointe Gourde} as the principle that an {\it increase} in value should be disregarded when it is ``entirely due to the scheme''. The negative dimension has attracted more interest and controversy than the positive dimension, especially in the UK. This is also the aspect of the rule that was at the center of attention in {\it Waters}.

It is not surprising that the negative aspect of the no-scheme principle more often results in complaints, as property owners stand to loose whenever it is applied. However, on a traditional understanding of the public purpose of expropriation, the negative aspect of the rule is also seemingly easy to justify. In \emph{Waters}, Lord Nicholls describes the most important policy reasons as follows:\footcite[18]{waters04}

\begin{quote}
When granting a power to acquire land compulsorily for a particular purpose Parliament cannot have intended thereby to increase the value of the subject land. Parliament cannot have intended that the acquiring authority should pay as compensation a larger amount than the owner could reasonably have obtained for his land in the absence of the power. For the same reason there should also be disregarded the ``special want'' of an acquiring authority for a particular site which arises from the authority having been authorised to acquire it.
\end{quote}

This appears like a reasonable justification. Notice, however, that Lord Nicholls avoids using the word ``scheme''. In particular, he does not identify the scheme's absence as the measuring stick for ascertaining on what basis parliament intends compensation to be based. Rather, Lord Nicholls speaks of what the owner could reasonably have obtained in the \emph{absence of the power} to acquire the land compulsory. In this way, he seems to prescribe a rather narrow interpretation of the negative dimension of the no-scheme rule.\footnote{See also the commentary offered in \cite{newuk}.} It is the power to expropriate that should not give rise to an increased value, nothing at all is said at this stage about the scheme that benefits from it.

It would appear, therefore, that there is nothing in principle that prevents the property from being compensated on the basis of its value in a scheme that differs from the scheme underlying expropriation only in that it does not have such powers. Indeed, this subtle caveat appears to be rather crucial for the remainder of Lord Nicholls' arguments, when he attempts to reconcile the principle adopted in the \emph{Indian} case with the \emph{Pointe Gourde} case.

It would lead me too far astray to go into all the subtle details about the interpretation of the no-scheme rule in UK law and the possible implications of \emph{Waters}. Rather, I would like to focus on one specific aspect, namely the application of the principle when the scheme in question is a commercial enterprise. The UK Supreme Court touched on this issue in the recent case of  \emph{Bocardo}.\footnote{\cite{bocardo10}.} The case was decided under dissent, suggesting that the clarifications offered in \emph{Waters} have not been as conclusive as one might have hoped.

\emph{Bocardo} concerned a reservoir of petroleum that extended beneath the appellant's estate. The petroleum could not be extracted without carrying out works beneath their land. The first question that arose was whether or not extraction of the petroleum amounted to an infringement of property rights. This was answered in the affirmative. The second question that arose was what principle of compensation should be adopted to compensate the owner. The Supreme Court, following some deliberation, found that the general rules applied, meaning that the case should be decided on the basis of an application of the no-scheme principle.

However, opinions differed as to the correct interpretation of this principle, as well as how the facts should be held against the law. The crucial point of disagreement arose with respect to whether or not the special suitability, or \emph{key value}, of the appellant's land, \emph{pre-existed} the petroleum scheme.

In \emph{Waters}, the House of Lords had cited and expressed support for the following passage, taken from Mann LJ's judgement in \emph{Batchelor}.\footnote{\cite[361]{batchelor89}. Cited by Lord Nicholls at \cite[65]{waters04}.}

\begin{quote}
If a premium value is ``entirely due to the scheme underlying the acquisition'' then it must be disregarded. If it was pre-existent to the acquisition it must in my judgement be regarded. To ignore the pre-existent value would be to expropriate it without compensation and would be to contravene the fundamental principle of equivalence.
\end{quote}

%(see \emph{Horn v Sunderland Corporation})
Relying on this distinction between the potentialities that are ``pre-existing'' and those that are due to the scheme, the minority in \emph{Bocardo}, led by Lord Clarke, made the following observation.\footcite[42]{bocardo10}

\begin{quote}
Anyone who had obtained a licence to search, bore for and get the petroleum under Bocardo’s
land would have had precisely the same need to obtain a wayleave to obtain access
to it if it was not to commit a trespass. So it was not the respondents' scheme that
gave the relevant strata beneath Bocardo’s land its peculiar and unusual value. It
was the geographical position that its land occupies above the apex of the
reservoir, coupled with the fact that it was only by drilling through Bocardo’s land
that any licence holder could obtain access to that part of the reservoir that gives it
its key value.
\end{quote}

This view was rejected by the majority, led by Lord Brown, who interpreted the no-scheme rule quite differently:\footcite[83]{bocardo10}

\begin{quote}To my mind it is impossible to characterise the key value in the ancillary
right being granted here as ``pre-existent'' to the scheme. There is, of course,
always the chance that a statutory body with compulsory purchase powers may
need to acquire land or rights over land to accomplish a statutory purpose for
which these powers have been accorded to them. But that does not mean that upon
the materialisation of such a scheme, the ``key'' value of the land or rights which
now are required is to be regarded as “pre-existent”.
\end{quote}

While the case was resolved in keeping with this view, the dissent suggests that the clarification in \emph{Waters} has not resolved all issues. Moreover, it suggests that special questions arise when the expropriation scheme itself involves the realisation of a commercial potential inherent in the land that is taken. Is it permissible for government to grant the value of this potential to the taker -- by granting him the necessary licenses -- without subsequently recognizing the potential as having been taken from the owner? 

This issue does not \emph{not} primarily depend on the scope of the scheme as such. In {\it Bocardo}, for instance, it was obvious that the scheme was the entire project aimed at extracting petroleum from the reserve, including the necessary works beneath the appellant's estate. But even so, it was still unclear whether the special value of the appellant's land could be said to have been {\it caused} by the scheme. The issue that arises in these kinds of situations is ontological: When should we attribute a given value to an act of government, and when should we attribute it to nature, as a fruit of the land? Or in more practical legal terms: When is a given property value that is unlocked by a development scheme part of the original owner's bundle of rights?

To answer this question, it is tempting to look for a causal link between scheme and value, to substantiate the claim that the value was not in fact pre-existent. But as \emph{Bocardo} illustrates, it is not always obvious what should be taken as good evidence for such a link. It seems that one's perspective on this will tend to depend also on one's point of view on the much more general question of what values one recognize as inherent in property rights.

When Lord Clarke remarked that the state, following nationalisation in 1934, could have given the right to extract the petroleum to \emph{someone else}, he was certainly correct. Hence, I also agree with him that ``the key value was not created by the 1934 Act or the grant of the petroleum licence to Star''.\footnote{See \cite[163]{bocardo10}.} But whose value was it, and was it a commercially realisable value? Here, Lord Clarke appears to assume that the value must belong to the property owner and that this owner would also have been able to make a profit from it in the absence of the expropriation scheme. This, I believe, is a leap that requires further justification. Just because some property has key value does not mean that the owner of the property is entitled to that value, or that it can ever be translated into a financial profit.

On the one hand, it is easy to agree with Lord Clarke that compulsory acquisition of a wayleave is no precondition for an extraction scheme. The project could well have been carried out by a developer who was willing to pay the owner for the special suitability of his land. But on the other hand, it does not seem obvious that the owner is meant to be able to demand such payment under the regulatory system currently in place. Hence, even in the absence of a causal link between scheme and value, one might be entitled to conclude that the special value falls to be disregarded because it has already effectively been removed from the owner's bundle.

In the case of {\it Bocardo}, I think this perspective would have been particularly helpful to Lord Brown, who argued that the value of the strata was not pre-existent. As it stands, his argument seems rather strained. After all, it was the physical conditions that gave the land its value, not the abstract fact that a development license had been granted. However, by looking at his argument in more depth, it is tempting to rephrase his conclusion by saying that he regarded the special suitability of the strata as having no commercial value under the prevailing regulatory regime.

In the end, I am agnostic about the correct way to judge {\it Bocardo}, but I think the crucial question that it raised was the following: did parliament intend to give petroleum developers a right to extract substrata resources without sharing the profits with affected surface owners? If no clear answer is available, conflicts can result, particularly if the question itself is obfuscated, as I think it was in {\it Bocardo}. It seems to me, in particular, that the focus on causality and the notion of ``pre-existence'' was not very helpful. Rather, I think the crucial keyword should have been benefit sharing.

The first question to ask in this regard is what parliament intended when it set up the current regulatory framework. If this is unclear or the evidence suggests that benefit sharing was not intended, the question becomes whether or not benefit sharing is nevertheless required on the basis of constitutional or human rights law. In a case like {\it Bocardo}, the latter question is unlikely to arise with any great force. It seems to me, in particular, that the question of how to deal with a property's ``key value'' in relation to other property is usually a question that can be resolved merely by pointing to the legitimate public interest in avoiding unwanted holdouts.

Even so, if the courts engage with the question of benefit sharing without being explicit about it, the lack of democratic accountability can become a worry. I think it is important to emphasize the political sensitivity of the range of complex rules found in compensation law. If not, a crisp political question risks becoming obfuscated to the extent that it can only be engaged with in a meaningful way by legal professionals. This, in turn, increases the chance of abuse and undue influence of special interest groups. While most people remain ignorant of the political work done by the courts in this regard, those who stand to gain the most are free to lobby and argue on technical points to gradually shape the law of benefit sharing according to their own interests. A conceptual shift might be needed to prevent this development from becoming precarious to the legitimacy of compensation law in general, and the no-scheme rule in particular. 

In addition, the question becomes much more pressing in cases when the development potential as such is subject to expropriation. An extreme case arises when natural resources are expropriated. For an illustration which also links up to my case study, I mention particularly the cases of \emph{Cedars} (1914) and \emph{Fraser} (1917), two Canadian compensation disputes regarding expropriation for hydropower. They were cited as important authorities by both the Law Commission and the House of Lords in \emph{Waters}.\footnote{\cite{cedars14,fraser17}.} 

In \emph{Fraser}, it was the waterfalls themselves that were subject to expropriation, yet the Privy Council still found that the value of the potential for hydropower exploitation of these falls should be disregarded when compensating them. The reasoning adopted seems to follow a standard ``value to the owner'' approach. However, reflecting back on {\it Bocardo}, it is hard to see how anyone could think that the value of the waterfalls were not ``pre-existent'' to the scheme to develop them. Surely, as a natural resource, a waterfall has significant value in itself, independently of any particular ``scheme''? 

Not so, according to the Privy Council, who found that the owners of waterfalls could not themselves have developed hydropower. Here, a subjective standard was in effect employed, whereby the bundle of rights associated with a property depended not only on the property itself but also on the nature of its owner. This unequal treatment of owners is such that is could, in my opinion, now be attacked from the point of view of human rights and constitutional law.\footnote{Although such an approach might not be required to overrule them, as the Canadian cases already appear to be at odds with both {\it Waters} and {\it Bocardo}.}

However, in order to make such an attack, it is necessary to use a working distinction between commercial and non-commercial aspects of a development scheme. The pre-existence test is inadequate. For instance, there can be no doubt that the energy inherent in water pre-exists any scheme seeking to harness it. Moreover, it seems clear that energy has great value, meaning that the value of a waterfall pre-exists any scheme for hydropower exploitation. However, we must also ask: what \emph{kind} of value is it?

To illustrate why this is a relevant consideration, consider a case where the property value is enhanced for the owner because of a personal attachment. In this case, it seems fair to differentiate, so that the owner's subjective attachment to the property is taken into account, potentially leading to a higher compensation payment then any other owner would receive. It is irrelevant, moreover, whether or not the particular aspect of the property to which the owner is attached is pre-existing. The relevant consideration is simply whether or not the value in question is such that one thinks it {\it should} be compensated. The value is {\it not} commercial, however, but personal (and, in so far as it receives recognition, also public). This is {\it why} differential treatment becomes justifiable. 

Similarly, in so far as a piece of land is particularly suited for building a school, it seems unproblematic to deny benefit sharing with the owner. In this case, the suitability is pre-existent, but it reflects a value to the public, not to commerce. Hence, a disregard rule can safely be applied, even though the public would been willing to pay large amounts in friendly negotiations. But what if the land was not particularly suited for a school, but for a shopping mall? Here I believe a different standard is needed. It seems, in particular, that benefit sharing is required in this case since one would otherwise illegitimately discriminate between owners. Why should the owners of shares in a shopping mall be allowed to profit, when the owners of the suitable land are not?

As a practical test, I propose the heuristic whereby one regards the commercial value of the development as evidence that disregard rules like the no-scheme principle should not be applied. The underlying rationale behind this heuristic is based on the public interest requirement. It seems to me, in particular, that disregard rules are also in need of justification based on the needs of the public.
In my opinion, the public interest/purpose requirement extends to compensation in such a way that a value needs to be identified as a public value in order for it to be legitimate to disregard this value when compensating the owner. 

More generally, I fail to see how it could ever be legitimate to apply a no-scheme principle unless it serves the public good. If the principle is applied in a way that results in a commercial benefit to the taker and a commercial loss to the owner, I would argue that it renders the expropriation as a whole unsafe in relation to the public interest requirement. One aspect of the interference, at least, then lacks proper motivation. From this I arrive at the general conclusion that values which are recognized as commercial should never be disregarded.

The distinction between commercial and public values is obviously not written in stone, but is down to a political decision. Moreover, it can hardly be regarded as permanent. In addition, it can often be difficult to assess where the line is to be drawn, especially in cases when public-private partnerships are relied on to provide public services. Nevertheless, it seems to me that the public interest requirement in constitutional and human rights law makes it necessary to be explicit about private and public values also in relation to compensation. Moreover, it seems like doing so could be very helpful in many cases, such as {\it Bocardo} and {\it Fraser}.

For instance, even if the public value of hydropower pre-exists an hydropower scheme, this does \emph{not} necessarily mean that there is any pre-existent commercial value in hydropower. What counts as {\it commercial} value, in particular, must first be answered. This, moreover, depends entirely on whether or not the public has settled on a regulatory regime that allows commercial exploitation.
Hence, I arrive at the following suggestion for a modified version of the ``pre-existence'' test: An owner should always be compensated for the value of any pre-existent \emph{commercial} value that his property has.\footnote{Certainly, a clarification along these line would not resolve all issues. It would not, for instance, offer any conclusive guidance with respect to the specific issues related to "key value" raised in \emph{Bocardo}.} 

To answer the question of what should be regarded as a pre-existing commercial value, one must take a broad look at the prevailing regulatory regime. Moreover, one must expect that the assessment will depend on the context of regulation, in particular the extent to which the state \emph{allows} the disputed value to be commercially realized. The law relating to compensation should be such that it can tolerate significant changes in these parameters. The theoretical question that arises concerns only the conceptual foundation for the assessment. The actual lines that must be drawn are all drawn in the sand, as usual.

In the next section, I will address Norwegian compensation law to shed light on some such lines that have been drawn in relation to waterfalls, which have recently been washed away and redrawn following liberalization of the hydropower sector. This will allow me to shed light both on the no-scheme rule and alternatives to it. 

%Moreover, I note how the Norwegian system was originally based on a rejection of the idea that all disputes had to be resolved uniformly on the basis of a battery of specific rules. Instead, great emphasis was placed on the discretion of lay people. In later years, however, Norwegian compensation law has developed along a similar trajectory to that of the UK. The no-scheme principle, in particular, has now been addressed in so many different ways and by some many different sources of authority that it appears just as much in defiance of ready comprehension as in the UK before {\it Waters}. I will now try to untangle the web somewhat, while moving towards the special points I would like to make based on the idiosyncratic case of waterfalls. 
%
%But the assessment itself was above all else discretionary. Legitimacy of the process was ensured in a bottom-up fashion, by the involvement of lay people sitting as appraisers, alongside a regular judge.
%
%I note, however, that this system has largely been modified so that, today, the appraisal courts are far more constrained from the top down, by legislation and the Supreme Court. I argue that this has been a source of difficulty for the system, particularly in relation to the no-scheme principle which, as I argued above, necessitates a concrete assessment, and can not -- should not -- be resolved by all-encompassing principles. I note, in particular, how the increasingly constrained room for discretion by lay people means that the distinction between commercial and public value -- which must now be determined centrally -- becomes muddled. In many cases, the idea acting as a premise for the general rules applied simply does not correspond to reality. This often leads to unacknowledged commercial windfalls for takers, arising when owners are denied compensation for commercially valuable rights that the law presupposes to be wholly public, even though they are not. 

\section{Appraisal courts and ``foreseeable alternatives''}

The owner's right to compensation following expropriation of property is enshrined in very simple terms in Section 105 of the Norwegian Constitution.\footnote{\cite[105]{grunnloven14}.} This section states simply that \emph{full compensation} is to be paid in all cases when the public interest warrants the compulsory acquisition of property. For more than 150 years, this was the sole legislative basis for compensation rules in Norway. The methods used to calculate full compensation in different scenarios developed entirely through case law.

According to a long legal tradition in Norway, the discretionary aspects of property valuation is regulated by a special procedure, with a significant reliance on so called \emph{unwilling appraisers}. These are members of the public who have no interests in the case at hand. They may be chosen, however,  specifically for their suitability in judging the value of the contested property, either because they are resident in the local area or because they have special expertise.

The appraisal procedure has a long history, going back to customary law that pre-dates the constitution. The rules regulating it today are found in the \cite{aa17}.\footnote{Act no 1 of 1 June 1917 relating to Appraisal Disputes and Expropriation Cases.} Appraisal cases are organised similarly to civil disputes, and the procedure is administered by the district courts.\footnote{\cite[5]{aa17}.} Appraisal courts are usually composed of a panel consisting of one professional judge and four appraisers, with no special juridical qualifications. 

The standard arrangement is that appraisers are chosen from the general public in the district where the property in question is located. But the Act opens up for the possibility that appraisers may also be chosen for their special technical expertise.\footnote{See \cite[11|12]{aa17}.} Their role in the procedure is on par with the judge, and the panel decides jointly both the legal and the technical questions, usually on the basis of technical reports put forth by the parties. These reports are presented during the main hearing, and may be challenged by the parties, in more or less the same way as the district court hears evidence in a regular civil dispute.\footnote{See particularly \cite[22|27]{aa17}, with further references to the \cite{da05} (Act No 90 of 17 June 2005 relating to the Mediation and Procedure in Civil Disputes).} 

There is a possibility for appeal to the appraisal Court of Appeal, which is the regular Court of Appeal sitting as an appraisal court in accordance with the rules of the \cite{aa17}. The right to having an appeal heard is not absolute; whether the appraisal Court of Appeal will hear the case depends on its importance, according to rules that correspond to those in place for regular civil disputes.\footnote{See \cite[32]{aa17}.} The procedure closely corresponds to the procedure followed in appraisal disputes at the district level.\footnote{See \cite[38]{aa17}.} However, the decision made by the appraisal Court of Appeal is final as far the appraisal assessment is concerned. An appeal to the Supreme Court can only be accepted on legal grounds.

As a consequence of this, the appraisal courts have been very important in interpreting and developing the law relating to compensation in Norway. Their importance was particularly great all the while the meaning of ``full compensation'' was not clarified further in statute. The presence of lay people sitting as judges is consistent with how many civil disputes are resolved. But what makes the appraisal courts unique is that in these cases the lay people where traditionally allowed to engage with the issue under very few restraints, beyond procedural rules and the words of the Constitution.

At the same time, the practical viewpoint enforced by the procedural form meant that legal questions would often remain in the background in such cases. Typically, the legal issues would only come to the forefront if the Supreme Court decided to hear the case as a matter of principle. 

The primary criticism voiced against this system, particularly following the Second World War, was that it gave the appraisal courts too much discretionary power. Hence, the argument went, legislation was needed to make the outcome of appraisal cases more predictable.\footnote{See, for instance, Part 2, Chapter 1 of \cite{nut69}, handed over to the Ministry of Justice by the so called Husaas committee, appointed by the King in Council 6 Aug 1965.} However, while the law regarding compensation was not formalized in written form, there were legal scholars who developed theories and aimed to explicate its content based on the body of case law that was available.

Also, the Supreme Court did regularly hear cases concerning legal arguments regarding compensation, and they developed a consistent position on at least some of the more critical and recurring legal issues. At this time, the central source of legal reasoning regarding appraisal was still to be found in the constitution itself. As a result, theories of compensation law tended to be \emph{absolutist}, in the sense that they looked directly to wording in Section 105, also when tackling specific problems of interpretation. 

\subsection{Constitutional absolutism}

Absolutism was widely endorsed by Norwegian legal scholars as late as in the 1940s. The well-known legal scholar Magne Schjødt summed it up as follows:\footcite[177]{schjodt47}

\begin{quote}
When an owner is entitled to compensation, he is entitled to have his full economic loss covered. He should receive full compensation, see p 42 ff. This is the great principle that remains absolute and any dispute must be resolved on its basis.
\end{quote}

A typical example of the style of legal reasoning that this view gave rise to can be found in thes writings of the prominent legal scholar Frede Castberg. He specifically addressed also the no-scheme principle, by asking about the extent to which increases in value due to the scheme underlying expropriation was to be taken into account when calculating compensation. His reasoning in this regard was based directly on a reading of the Constitution. Moreover, it was based on the principle of \emph{equality}, which was at that time considered particularly crucial in understanding constitutional law. The following quote serves to sum up Castberg's position on the no-scheme principle:\footcite[268]{castberg64b}

\begin{quote}
The owner is entitled to full compensation. The expropriation should not leave him worse off economically than other owners. Hence if the public has knowledge that an industrial undertaking is being planned, that a railway will be built etc, and this affects the value of property generally in a district, then the increased value of the property that will be expropriated must be taken into account. If not, the owners of such property will be worse off than other owners from the same district. On the other hand, if the expectation of the scheme underlying expropriation leads to a general depreciation of value, then it is this new value -- not the original value -- that is relevant for calculating compensation. The crucial question is what the actual value is, when expropriation takes place.
\end{quote}

% e mention that the problem analyzed by Castberg in this passage has been considered in many jurisdiction, and is dealt with in common law by the so called \emph{no-scheme} rule. This is more a principle than a single rule, and it is typically understood as a mechanism that is meant to ensure that changes in value due to the scheme underlying expropriation are disregarded.\footnote{For an history of the rule in common law (primarily the UK), which also illustrates the difficulty in interpreting it and applying it to concrete cases, we point to Appendix D of Law Commission Report No 286, 2003} In comparative terms, Castberg appears to favor a \emph{narrow} interpretation of the principle -- a restrictive view on when additional value due to the scheme should be disregarded -- quite close in spirit to the so called \emph{Indian} case from 1939\footnote{\emph{Vyricherla Narayana Gajapatiraju v Revenue Divisional Officer, Vizagapatam} [1939] AC 302.}, which was been much discussed in common law and was dealt with extensively by the House of Lords as late as in 2004.\footnote{In the case of \emph{Waters and other v Welsh National Assembly} [2004] UKHL 19. 
%The primary precedent for a broader interpretation of the non-statutory no-scheme rule, on the other hand, is \emph{Pointe Gourde}, \emph{Pointe Gourde Quarrying and Transport Co v Sub-Intendent of Crown Lands} [1947] AC 565, PC, 572, per Lord MacDermott. This case proved highly influential for the understanding of compensation rules in the post-war period, in many common law jurisdictions, but has recently been challenged by a renewed interest in more narrow viewpoints such as that expressed in the \emph{Indian} case, see  \cite{newuk} and also the case of \emph{Star Energy Weald Basin Limited and another (Respondents) v Bocardo SA (Appellant) [2010] UKSC 35}.}In the context of Norwegian law, it is of particular interest to note how Castberg's views in this regard is arrived at through considering the constitution itself, founded on the principle of equality.\footnote{In this way, he arrives at a narrow no-scheme rule quite abstractly, and through a different route than the one adopted in the \emph{Indian} case, where the outcome appears to have turned crucially on the particular facts in the case, a close reading of precedent, as well as the perceived fairness of the result.}

As Castberg bases his analysis on the exact wording of the Constitution, he does not engage in any reasoning based on the extent to which it can be regarded as socially fair for the public to pay compensation for value that arise due to the beneficial consequences of the public project itself. Crucially, he does not address the concern that this can be seen as a form of double payment. Such pragmatic and utilitarian points were not widely used to interpret the law in the legal tradition Castberg was part of. This, in particular, is why I think it is appropriate to use the label of constitutional absolutism to describe this kind of reasoning.

However, it is not correct to think that such reasoning is necessarily ``owner-friendly''. To see this, it is enough to note that Castberg, in the quote above, explicitly states that depreciation of value due to the scheme should not be disregarded. In addition, Castberg did not intend to reject the no-scheme principle altogether. In particular, he explicitly denied that owners of expropriated property should ever be able to claim compensation based on the special want of the acquiring party:\footcite[268]{castberg64b}

\begin{quote}
The situation is different if the property has increased value due to the expectation that it will be expropriated. The owner can not demand that this increase is compensated since that would be the same as giving him a special advantage compared to those from whom no property is expropriated.
\end{quote}

Hence, Castberg accepts a narrow version of the no-scheme principle, similar in spirit to that presented by Lord Romer in the {\it Indian} case. Castberg's view appears to have been shared by many academics of his day, and it was also largely reflected in case law from the Supreme Court.\footnote{See below.} At the same time, the very nature of the system for deciding appraisal disputes gave the local appraisers great freedom in adapting the principles in a way that suited the concrete circumstances.

To some extent, this would also involve making an assessment of what was regarded as a fair and just outcome. Hence, while the theory of the time was absolutist, case law was more multi-faceted. Importantly, fairness was seen as a concrete issue that had to be addressed on a case by case basis, an approach that would not necessarily lead to general rules. The Supreme Court largely sanctioned this approach, by passively respecting the discretion of the appraisal courts, as vested in them within an absolutist theoretical framework.

But as long as they did not cross the line with regards to the constitution, the appraisal courts were largely allowed to adopt broader viewpoints as well. The point was, however, that such viewpoints were \emph{not} extensively codified in terms of special principles used to deal with special case types or issues. Rather, they arose as a logical consequence of the way in which appraisal disputes were organized. Social justice and fairness perspectives were not excluded, but could in fact play an important role in practice.. However, such perspectives arose \emph{indirectly} through a \emph{decentralized} system which gave local courts great freedom when applying the law.

The way in which the no-scheme principle was applied serves as a nice illustration of this. On the one hand, the theoretical views of Castberg were widely accepted, but at the same time they were regarded as no more than guidelines that had to be adapted to the circumstances. Moreover, it was not unheard of for the appraisers to disagree with the judge about how this should be done, and to award compensation according to a different understanding of the law than that favoured by the judge. 

This happened, for instance, in the case of \emph{Tuddal}, where land was expropriated for construction of a power grid.\footcite{tuddal56} In relation to this, the expropriating party also acquired the right to use a private road. According to the juridical judge in the appraisal court of appeal, and consistent with the teaching of Castberg, compensation should be awarded solely on the basis of what the owners stood to lose. In this case, that would mean compensation based on the increased cost in maintaining the road resulting from increased use. However, the lay appraisers found this result unreasonable and awarded compensation also for the special value the use of the road would have for the acquiring party. The Supreme Court, although they found fault with the reasons given by the law appraisers, agreed that such compensation was possible in principle. The presiding judge offered the following perspective:\footcite[111]{tuddal56}

\begin{quote}
Since they were the private owners of the road, A/S Tuddal could, before the expropriation, refuse to let the Water Authorities make use of it. Hence it might be possible for A/S Tuddal, through negotiation and voluntary agreement with the Water Authorities or others with a similar interest, to demand a reasonable fee, and in this way achieve a greater total benefit than full compensation for damages and disadvantages. Following the expropriation, it is no longer possible for A/S Tuddal, in its dealings with the Water Authorities, to economically benefit from their ownership of the road in this way. If the company suffer an economic loss as a result of this, I believe they are entitled to compensation. Whether or not such an opportunity as I have mentioned -- all things considered -- was present at the time of the expropriation, falls to the appraisal court to decide, on the basis of whether or not an economic loss is suffered beyond that which follows from damages and disadvantages. On this basis, I assume that the appraisal court of appeal's decision to awarded compensation for the value of the right of way that is acquired can not -- in and of itself -- be regarded as an erroneous application of the law.
\end{quote}

The Supreme Court's reasoning illustrates two points. First, we see how the Supreme Court adopts absolutism in its interpretation of the law. Through careful use of wording, the compensation premium is not conceptualized as compensation based on the value of the road to the acquiring authority, but rather as compensation for the loss of potential profit following from a voluntary agreement. Hence, a seeming contradiction with the no-scheme principle is avoided.\footnote{This particular interpretation of full compensation led to arguments in the post-war period, regarding whether or not owners had a right to compensation based on the loss of profit from hypothetical voluntary agreements with the acquiring party. In the end, a consensus formed that this type of compensation should not in general be awarded. See \cite{nut69},Part 2, Chapter 4, Section 2.E.}

But {\it Tuddal} also illustrates a second important point, namely that the Supreme Court was prepared to defer greatly to the judgement of the appraisal court. It is stated explicitly that it falls for this court to decide whether or not the opportunity to profit from the road by negotiating with the expropriating party was present at the time of expropriation. This is particularly noteworthy in light of the dissent of the juridical judge in the appraisal court of appeal, and in light of the dominant legal theorizing of the day, which did  not seem to support the idea that a premium should ever be paid in a situation like this. Hence, the decision tells us that the Supreme Court went far in defending the discretion of the laypeople, as a \emph{systemic} feature. They tested with great caution whether it was truly outside the permissible legal boundary, but concluded that it should simply be regarded as an exercise of the lay judgement that the system presupposed.

This impression of the case is accentuated when considering other cases dealing with similar issues. Across the board, I note a strong  tendency to defend the role of the laypeople in the appraisal process. A particularly clear expression of this can be found in \emph{Marmor}, also from 1956, where the Supreme Court overturned a decision made by the appraisal court of appeal on the grounds that the court had been {\it too} reliant on general principles.\footnote{\cite{marmor56}.} This, the Supreme Court held, offend against both the principle of full compensation and the principle of discretionary evaluation by laymen.

The case involved expropriation of a private railway track, for the construction of a public railway. It was clear that the track which was being expropriated did not have market value in general, so the expropriating party argued that the value of these tracks to the public railway should not be taken into account when calculating compensation. The appraisal court of appeal agreed, pointing to the standard teaching of the day. The Supreme Court, on the other hand, struck down the decision because they felt that a standardized approach to the case was inappropriate given the circumstances. The presiding judge argued as follows:\footcite[498-499]{marmor56}

\begin{quote}
In my opinion one can not simply assume that a property does not have market value when it has no value for anyone other than the expropriating party. The question needs to be assessed concretely. I agree with the expropriating party -- as has also been confirmed on several occasions by the Supreme Court -- that in general one should not take into consideration the special value that the purpose of expropriation gives the property. This should not lead to a spike in compensation payments. On the other hand, I can not agree that it is automatically reasonable, or in keeping with Section 105 of the constitution, if the expropriating party in cases like the present one could acquire property at a price that is below what it would be natural and commercially appropriate to pay in a voluntary purchase.
\end{quote}

Again, I note the two main building blocks used in the argument: First, the standard reference to the constitution, and secondly, a reference to the need for \emph{concrete assessment}. This further reflects the strong confidence that the Supreme Court had in the integrity and autonomy the appraisal procedure. Moreover, I notice how, in this case, absolutism regarding the constitutional protection of property is \emph{not} used to argue for specific rules or principles, but rather to back up the argument that compensation should result from real assessment, and not be overly reliant on such rules. In the case of {\it Marmor}, this was the outcome even if the rules in question had the status of valid guidelines that had also been backed up by a series of Supreme Court decisions.

In addition to making the overreaching remarks quoted above, the Supreme Court also gave pointers as to the kinds of facts that should be considered. For instance, the presiding judge paid particular attention to the wider \emph{context} of expropriation, and the manner in which expropriation was used to benefit certain interests. He also noted how expropriation had come to replace voluntary agreement as the standard means of acquisition for this type of development. Therefore, the practice of using expropriation effectively prevented a market from developing, a market that might otherwise have appeared naturally:\footcite[499]{marmor56}

\begin{quote}
I also point to the fact that the case concerns an area of activity where the expropriating party has a {\it de facto} monopoly which makes it impossible for anyone else to make use of the property for the same purpose. This in itself makes it questionable to simply assume that the lack of financial value for other purchasers provides the appropriate basis for calculating compensation. When considering this question, it is also appropriate to take into account that we have lately seen a great increase in the use of expropriation to undertake projects such as this. Compulsion is becoming the primary mode for acquisition of property -- not voluntary sale following friendly negotiations.
\end{quote} 

In my opinion, the importance of this decision, which makes it highly relevant even today, is not that it seems to favour a narrow interpretation of the no-scheme principle. In fact, I think it is erroneous to read the judgement as expressing general support for any particular interpretation. In addition, I do not think the decision can be read as supporting a general principle by which compensation can always be based on the value of hypothetical agreements that could have been made with the expropriating party. Rather, I take the judgement to be an expression of scepticism towards blind obedience to \emph{any} set of detailed rules for calculating compensation that serve to limit the room for lay discretion.

At the very least, it seems clear upon closer inspection of the argument that the main objective of the Court was not to express any particular view regarding the content of the no-scheme principle, but rather to instil to the appraisal courts that they could not use this rule as an excuse not to engage in concrete assessment to ensure a reasonable outcome in keeping with the constitution.

I believe this point is important to stress. It illustrates how absolutism need not, and did not, result in a rigid system with little room for assessment based on justice and fairness, broadly conceived. Quite the contrary, the absolutism endorsed by the Supreme Court, and inherent in the Norwegian system of appraisal courts, was not characterized by blind obedience to specific rules, like those proposed by Castberg. Rather, the system was flexible, and it was explicitly intended to function such that fairness assessments based on concrete circumstances could be accommodated.\footnote{Going back to even older legal scholarship, we see that this view on the meaning of absolutism has a long history in Norway. It is present, for instance, in the work of the famous 19th century scholar Aschehough, who stressed the link between the constitution and the appraisal procedure when he considered the (then) hypothetical situation that legislation would be introduced with the specific aim of reducing the level of compensation payments following expropriation. See \cite[48]{aschehough93} 

%\begin{quote}
%If it becomes common practice to award compensation payments that are unreasonably high, this would make important public projects more expensive and difficult to carry out, greatly to the detriment of society. In many cases it might not be possible to rely on legislation to prevent such excessive compensation payments, since this would restrict the appraisers too much. To some extent this might be possible, however, and as far as it goes, parliament must be permitted to do so. However, if enacted rules clearly lead to less than full compensation in an individual case, they will be overruled by Section 105 of the constitution, and fall to be disregarded in that particular case.
%\end{quote}
%
%This quote is important because it does not rely on any particular interpretation of the constitutional demand for full compensation, but sees this inherently as an issue that needs to be resolved by concrete assessment of individual cases. Absolutism to Aschehough implies freedom and responsibility for the appraisers; freedom to judge individual cases by its merits, and a responsibility to award full compensation, irrespectively of any specific rules that might be in place to curtail excessive payments. The important point is that Aschehough here sees absolutism as a principle that should be applied to cases, not to principles. He does not argue that rules introduced to limit compensation payments would be inadmissible merely because they might sometimes suggest less than full compensation. Rather, he takes it for granted that it falls to the appraisal courts to apply the rules in a way that would prevent such outcomes. As long as the appraisal courts remain free to apply the rules in such a way that full compensation is awarded, specific rules intending to prevent excessive payments can happily coexist with absolutism.
%
%The subtle view taken by Aschehough was largely overlooked in debates following the introduction of the Compensation Act 1973, which served to introduce radical rules of exactly the kind he had predicted and considered 80 years earlier. The consequence was, as I will discuss in more depth in the next section, that the Supreme Court was forced to actively steer the interpretation of the Act to ensure that section 105 would not be violated in concrete cases. Hence, the introduction of legislation served to destabilize the system, by narrowing the room for lay judgements and increasing the reliance on legislation and special principles developed by the Supreme Court. This development is the subject of the next subsection.

%More generally, the 60s and 70s appears to be a period when the crucial role of the appraisal procedure was to some extent forgotten, and also undermined, following a heated political and ideological debate regarding the appropriateness and admissibility of introducing rules to ensure that compensation payments were brought down to a lower level. This had deep and lasting effects on Norwegian compensation law, and it is popularly described as a period when the social democrats won recognition for the principle that social fairness suggested the introduction of compensation rules and disregards that were more extensive than what had previously been considered appropriate. 
%
%This was conceived of as a fight for social justice against outdated and conservative ideas of constitutional absolutism. But it seems to us that this view of the history of Norwegian compensation law is erroneous, and largely unhelpful. The approach taken by Aschehough, in particular, placing emphasis on the important role played by the appraisers in achieving fairness and justice in concrete cases, does not appear to contradict social democratic goals at all. In fact, it seems that his approach might be better suited to serve such goals, and to accommodate a variety of different political opinions and ideas, than an approach which is based on attempting to flesh out in painstaking detail how the appraisal courts should go about achieving the balance between social fairness and owners' rights. We will return to this point later, but first we will take a closer look at the history of the radical Compensation Act 1973 and the censorship to which it was subjected by the Supreme Court, leading to the Compensation Act 1984, currently in place.

\subsection{Pragmatism}\label{sec:prag}

Following the Second World War, the social democratic \emph{Labour Party} gained a secure grip on political power in Norway. As a result, many reforms were carried out that would reshape Norwegian society. One of the most important reforms concerned the introduction of extensive planning law to ensure that land use was put under public control.\footnote{See generally \cite{thomassen97,kleven11}.} As a result of this, the period also saw expropriation being used more extensively to further public projects, such as the large scale construction of hydropower to ensure general supply of electricity.\footnote{See generally \cite{skjold06,thue06b}.} As a result of these changes, the opinion was soon voiced that there was a need for a more uniform approach to compensation, which collected some basic principles in a common body of written law. In addition, it was an explicitly stated political goal to bring compensation payments down.

In 1965, the so called \emph{Husaas committee} was appointed by the King and charged with the task of assessing the compensation rules currently in place.\footnote{Appointed by the King in Council on 6. Aug 1965.} The committee was also ordered to make a concrete suggestion regarding the need for additional principles of compensation, and to consider if these should be given in the form of a special compensation act. Initially there was some doubt as to the extent to which is was at all permissible to give rules regulating compensation, as the constitution itself addressed the matter. 

However, the committee noted that some rules had already been introduced for specific case types, for instance in relation to expropriation for hydropower development.\footnote{As discussed in Chapter \ref{chap:..}, Section \ref{sec:...} in relation to the \cite[16]{wra17}.} In addition, legal scholars of the day were generally of the opinion that compensation rules could be given, on the understanding that the courts would deviate from them in so far as they seemed to go against the Constitution. Hence, the Constitution was not understood to stand in the way of more specific rules.\footnote{\cite[136-137]{nut69}.} According to the minority of five, no such rules were actually needed, but the majority of ten disagreed.\footcite[137]{nut69} }

When considering the question of what kind of rules should be introduced, the Committee looked to case law as well as existing literature on compensation. They were faced with highly divergent opinions on the subject. Since WW2, in particular, a pragmatic view on property rights had taken hold, whereby an absolute right to property was increasingly felt to stand in the way of efforts to rebuild the country and ensure development following the great war. The Labour party had secured a firm grip on government at this point, so there was also an ideological shift taking place that emphasised the importance of building a welfare state over protecting the entitlements of individuals.

This was by no means a consensus view among legal scholars, however, and it was particularly contentious with regards to property. 

%As a result, some disagreed strongly with the very idea of legislation regarding compensation, and tensions arose that have led to much legal controversy and are still important in the law today.

%The majority pointed out that a vague general principle such as that provided by the constitution would by necessity have to be interpreted in order to be applied to concrete cases.\footnote[137]{nut69} Hence, it was not only permissible, but also desirable, for parliament to give more detailed instructions as to how is should be applied and understood by the courts and the appraisal courts. Leaving it to the judiciary to flesh out the exact meaning of full compensation through case law, it was felt, was not appropriate in a regulatory regime where expropriation had become increasingly important as a means to ensure modernization and development of critical infrastructure.

%In addition to this, the Supreme Court itself had recently expressed its support for a new view on regulation of property use, supported by contemporary legal scholars and politicians, whereby the State was regarded as having wide discretionary powers to determine how property should be used. This right to regulate, in particular, was increasingly coming to be seen as a right that did not infringe on property rights, so that the State would not have to compensate owners if they exercised it, except in special cases.\footnote{See, in particular, Rt. 1970 p. 67.}.

This problem area was mapped out in some detail by the Husaas committee, who traced the pragmatic view on compensation, identifying it using the following quote by the leading scholar Knoph from \cite[113]{knoph39}.

\begin{quote}
Since Section 105 is a rule prescribing practical justice, directed at parliament, and not an ethical postulate of absolute validity, it must be permitted to make technical legal considerations, so that one accepts compensation rules that lead to correct and just results on average, even if it does not grant the owner full individual justice in every case.
\end{quote}

Many disagreed vehemently with this perspective, based on absolutist principles.\footnote{See, e.g., \cite[20-22]{robberstad57};\cite[44]{schjodt47}.} The prominent legal scholar Schjødt, for instance, describes Knoph's reading of the law scathingly as follows:\footcite[44]{schjodt47}

\begin{quote}Luckily it has not had any effect on judicial practice whatsoever. No court of law would accept that compensation should be set according to a norm that may be practical and just in general, but does not grant the owner full compensation in all individual cases.
\end{quote}

By the late 1960s, however, Knoph's view was beginning to find favour among legal scholars.\footnote{See \cite[17]{fleischer68};\cite[41]{opshal68}.} When assessing the writings on the subject, the Husaas committee noted this tension. In response to it, they proposed a set of general principles for compensation which are still largely with us today. They were moving in a pragmatic direction, but rather cautiously. Hence, they refrained from encoding principles that would appear too offensive to the absolutists, even if the pervading political sentiment was that compensation rights had to be limited to ensure more effective state regulation of property use.

Importantly, the Husaas committee distilled from case law the principle that owners could only demand compensation based on the value of a specific use of the property when this use was ``foreseeable''.
The committee sought to codify this idea, which they saw as expressing an interpretation of Section 105 that was already largely entrenched in case law.\footcite[134]{nut69} This led to the following conclusion:\footnote{\cite[142]{nut69}.}

\begin{quote}
It is the view of the committee that it is correct to encode in the act the principle that the owner is entitled to compensation based on the value that results from taking into account the foreseeable and natural use of the property, given its location and the surrounding conditions. The exact meaning of ``natural and foreseeable'' use must be decided after a concrete assessment in individual cases. By encoding this general principle, however, it will become clear that compensation should not be based on private or public plans unless these plans coincide with the use of the property that is natural and foreseeable, independently of the scheme underlying expropriation.
\end{quote}

Importantly, I note how the committee actually does more than just encode a foreseeability constraint. They also state outright that this constraint is taken to imply the no-scheme principle, since they stipulate that the assessment of what counts as foreseeable and natural must be made independently of the scheme underlying expropriation. Since this statement is made quite generally, it also seems that the committee expresses a broader view on the no-scheme principle than that endorsed by Castberg.\footcite[268]{castberg64b} It is no longer only the special want of the expropriating party that should not be taken into account, the entire scheme ``underlying'' expropriation should be disregarded.

But in fact, this view was not in keeping with the political motivation for an act regarding compensation. It was too owner-friendly. Hence, the Ministry of Justice deviated from it in their final proposition to parliament. Instead of encoding existing principles, they sought a more aggressively pragmatic system whereby compensation would in general be based on the value of the \emph{current use} of the property.\footnote{\cite[19-20]{otprp59}.} In this way, the argument went, the public no longer had to pay a financial premium to owners based on possible future uses that would in any event, in most cases, be reliant on public development permissions.\footnote{\cite[17-20]{otprp59}.} 

%Such permissions, it was argued, could never be foreseeable in circumstances when it was in the public interest that the property should be expropriated, and hence all future development potential should in principle fall to be disregarded.

%The Ministry commented on this as follows:\footnote{\cite[19-20]{otprp59}.}
%
%\begin{quote}
%The Ministry is of the opinion that it is particularly important to arrive at a rule that can bring the assessment of property value down to a realistic level, and believes that the natural starting point for such an assessment must be the current use of the property, especially for expropriation of real property. As mentioned, it is the opinion of the Ministry that a practice has developed that gives too much weight to more or less uncertain future possibilities for the property, something that has led to a sharp rise in compensation payments.
%\end{quote}

After intense debate in parliament, where the minority center-right parties all opposed its introduction, the current use rule was eventually encoded in section 4, no 1 of the \cite{ca73}.\footnote{Act No 4 of 26 March 1973 Regarding Compensation following Expropriation of Real Property.} This was largely seen as a social democratic victory and a clear indication that the pragmatic approach to property protection was gaining ground. When clarifying their principled starting point regarding what should count as \emph{realistic}, the Ministry made the following assertion regarding the scope of the constitutional protection offered in Section 105, showing the ideological underpinnings of the new Act:\footcite[17]{otprp70}

\begin{quote}
However, a right to complete -- or almost complete -- equality can not be derived from the constitution. It must be taken into account that we are here discussing equality with regards to increases in property value that are, in themselves, undeserved. [...]  %  The starting point must be that it is not, in and of itself, contrary to the constitution that one property owner do not benefit from the same increase in value as another, when the increase in value, for both of them, is due to public investment and does not stem from their own efforts. \\ \\
Certainly, it would be best to avoid any kind of inequality, if it was possible. But the examples we have considered illustrate that, today, inequality between property owners is tolerated with regards to public investments and regulation, and that, moreover, practical and economic considerations dictate that we \emph{should} make use of differential treatment in this regard.
\end{quote}

This echoes Knoph, but also goes much further. In particular, the Ministry explicitly states that differential treatment is appropriate in the context of expropriation, and, by implication, that this should be done precisely to avoid compensation payments that include compensation for ``undeserved'' increases in value. Also, in proposing that compensation payments should be based on current use, the scope of ``undeserved value'' is made very wide. In principle it would seem to include \emph{any} value that could be attributed to an as of yet unrealized potential that the property in question might have. The question of whether or not this value was reflected in the market value of the property, in particular, was not regarded as relevant. This was in itself radical, since market value based on the likely use of an ``average buyer'' had previously been the dominant starting point for appraissal courts when awarding compensation.\footcite[112-113]{nut69}

The conceptual significance of this change in perspective should not be underestimated. Here the Ministry stood firmly behind a pragmatic view. Perceived social fairness was the overriding constraint, also with respect to constitutional property protection. However, on taking this view to its logical conclusion, it was recognized that any general compensation rules that might be introduced should themselves be subject to a fairness test, so that, for instance, the current use principle could not itself be absolute or without exception. 

Rather, it could only be applied in so far as it served the overreaching goal of social justice and fairness which was regarded as the fundamental component of property protection that made such a rule possible. This, in particular, seems like a crucial observation, and one that has in my opinion been overlooked, with unfortunate consequence for the subsequent debate and development of the law. Indeed, it echoes the sentiment behind the age-old procedural arrangement that placed high value on the free discretion of the appraisal courts. Hence, it points to the possibility of finding some \emph{common ground} between absolutist and pragmatist views on compensation.

Sensible voices from both camps seemed to arrive at the conclusion that in the end, there was no way around a concrete and contextual assessment, where social fairness values are (hopefully) used as a guide. In an attempt to translate aspects of such a perspective into legislation, the Ministry set out two exceptions to the current use rule. The first, which received by far the most attention, was based on the notion of equality between owners in same local area.\footcite[19]{otprp70} It stipulated that the appraisal courts should be free to deviate from the current use rule in so far as it felt that it was reasonable to do so in order to ensure a reasonable degree of equality between neighbouring owners.\footnote{This principle was eventually encoded in section 5, no 1-3 of the \cite{ca73}. It would prove highly controversial, since it was only formulated as a rules that ``could'' be used to increase the compensation. In \emph{Kløfta}, the Supreme Court eventually deviated from this and overruled the Act by making clear that additional compensation was \emph{obligatory} in a range of cases when the intention had clearly been that the rule should be used sparingly. In this way, and possibly inadvertently, the Supreme Court ended up defending owners' interest by \emph{limiting} the power of the appraisal courts.}

However, the Ministry also noted the need for a second exception, which is in my opinion far more important and interesting. This exception pointed to the need to ensure equality between the taker and the owner, in so far as the taker could not be regarded as the embodiment of purely public values.

%
%\begin{quote}
%One is aware that the principle of current use compensation cannot be without exception. Even though this rule will be fair in general it can, in some cases, disproportionately disadvantage property owners. One has therefore suggested rules that modify the principle to some extent. These are given for somewhat different reasons. \\ \\
%
%One case addresses the situation when current use compensation means that a property owner will be significantly worse off that other owners of similar property in the same district, according to how these properties are normally used. In these cases, the principle of equality suggest that the owner receives some -- but not necessarily full -- compensation for the inequality that would otherwise arise from the fact that his property was made subject to expropriation. %Etter departementets oppfatning har en ekspropriat etter grunnloven ikke noe krav på å bli satt helt i samme stilling som om ekspropriasjonen ikke var skjedd, en forskjellbehandling innen rimelige grenser må grunnloven tillate når dette tilsies av tungtveiende samfunnsmessige grunner. 
%\end{quote}

Importantly, the rule sought to address precisely the situation that arises when the taker benefits commercially from the expropriation. Moreover, it addressed the question of the \emph{power balance} between the expropriating party and the owner. In the words of the Ministry:\footcite[19]{otprp70}

\begin{quote}
The second modification we make has to do with the relationship between the property owner and the expropriating party. If the use of the property that the expropriation presupposes gives the property a value that is significantly higher than the value suggested by current use, this will entail a transfer of value from the property owner to the acquiring party. In some cases this might be unreasonable. As an example of when this can become an issue, we mention an agricultural property that is expropriation for the purposes of industrial production. In such a case it might be natural that the owner receives a certain share in the increased value that the new use of the property will lead to.[...] %This would be different than, say, a situation where an agricultural property is expropriated for constructing a road or for setting up recreational outdoor grounds. In such cases, the expropriation will not lead to any such economically advantageous use of the property that will give the expropriating party an economic advantage. 

To establish a flexible system, the Ministry has concluded that it is practical that the King gives rules concerning the cases where an enhanced compensation payment, based on these principles, might be appropriate. This should not be decided by individual assessment, but governed by rules for special case types. Hence, the proposed Act states that the King can pass regulation concerning this matter.
\end{quote}

This quote goes right to the heart of one of the main problems of economic development takings, and proposes a possible remedy. However, the Ministry took the view that this remedy should {\it not} be administered by the appraisal courts, but should be left in the hands of the executive. Already here I note a reason worry whether this could then ever become an ineffective way of achieving fairness in practice. Indeed, the fact that this aspect of the 1973 Act has been largely overlooked and forgotten seems to prove my point. No rules such as those proposed by the Ministry as a possibility have in fact been introduced, the entire profit still goes to the taker in cases when commercial schemes benefit from expropriation.

%More broadly, it is hard to disagree that the context of expropriation must by necessity come to play a crucial role for any approach based on compensating the ``deserved'' value. What this value should be taken to be, in particular, can hardly be determined once and for all and in general terms, but must rather be subject to continuous revision depending on how expropriation is \emph{actually used} in society. This includes looking to the purpose it is meant to serve, the parties who stand to benefit, and the groups who tend to loose their land.

%Indeed, stipulating that compensation should be ``deserved'' appears to provide a benchmark that is just as unclear as the stipulation that compensation should be ``full''. It seems, in particular, that the inherent ambiguity of these terms allows us to draw two conclusions: first, that they might very well have the same meaning, and second, that they cannot possibly be defined once 
%and for all by any act of parliament, or by any decision in the Supreme Court.
%
%But this suggest, against the Ministry and the overall spirit of the 1973 Act, that the system of appraisal courts has an important role to play in ensuring fairness in individual cases. It is hard to see how the objective of social fairness and justice for the individual can be reached without making heavy use of appraisers with discretionary competence. 

The procedural and contextual aspect of fairness seems to have been overlooked by those pushing for the 1973 Act. Since the appraisal courts were regarded as compensating owners too generously, their freedom of discretion was seen as a problem rather than as a path towards a solution. I think this regrettable. If the new Act had been slightly more temperate in its approach, by encouraging the appraisal courts to take a broader view on fairness, rather than to force them to adopt current use value as a baseline, it might have been a success. Instead, it caused an outcry, with attention shifting away from practical matters towards doctrinal issues. The primary such issue, and the most serious one, concerned the question of whether the Act as such was in breach of the constitution. This was eventually considered by the Supreme Court in the case of \emph{Kløfta} in 1976.\footnote{\cite{klofta76}.}. 

Following this decision, the 1973 Act would be significantly reinterpreted to make it appear less offensive to the constitutional standard of full compensation. However, it seems to me that the Supreme Court largely accepted that the intention behind the Act should be respected and that appraisal practice needed to be adjusted accordingly. In this, the Supreme Court signalled loyalty to the political system and the democratic process. However, in implementing this adjustment in practice, they also, possibly inadvertently, set up a system where the role of the local appraisal courts were undermined even further.

Not only were they constrained by an Act that seemed to run counter to the Constitution, they were now also ordered from above to openly deviate from its exact wording, but only for a select group of cases meeting certain pre-defined criteria. In essence, the Supreme Court itself assumed greater control over how compensation law was to be applied, no longer merely in broad strokes, but increasingly also by developing special rules for specific case types.\footnote{The clearest indication of this shift is found in recent case law wherein the Supreme Court has provided a myriad of detailed rules and directions regarding how appraisal courts should decide on the thorny issue of whether to consider public plans binding for the compensation award or to disregard them under a no-scheme rule. See generally \cite[7-9]{nou03}.} In the following section, I describe this in more detail.

\subsection{The top-down approach}\label{sec:regab}

Following the decision in {\it Kløfta}, the \cite{84} was introduced. It reverted back to the ``foreseeability'' test proposed by the Husaas committee. In section 4, it is stated that financial compensation (as opposed to compensation in kind) is to be based on either value of use or value of sale, whichever is highest.\footcite[4]{ca84} Sections 5 and 6 describes in more detail how the calculation should be carried out.

In both regards, the principal requirement is that the value is calculated based on a use of the property that is foreseeable and natural given the surrounding conditions. In relation to the value of sale, there is an additional requirement, namely that the use must be one that an ``average'' buyer would be likely to make of the property. Hence, the value of sale should be set as a general market value, not a value arising from selling the property to a specially interested party.

The extent to which the foreseeability requirement entails that the use in question has to be in accordance with public plans currently in place has been disputed. In general, compensation is only based either on uses permitted by public plans currently in place or uses that seem likely to be permitted in the future. In Norwegian law, whether a use is foreseeable is an ``either/or'' question. 

No compensation is given to reflect the so-called ``hope'' value, namely the part of a property's value that depends on the perceived likelihood of a change in planning status and future possibilities.\footnote{By contrast, compensation tends to include such an element in the UK.} If a permission for future use is deemed likely, it is subsequently regarded as a certainty for purposes of compensation, although the present-day value of a future possibility is usually calculated in a way that takes interest and inflation into account.\footnote{These calculations tend to be notoriously schematic, however, quite far removed from the realities of the financial system.} Similarly, if a future possibility is deemed unlikely, no compensation is paid for it whatsoever.

In effect, the best an owner can hope for in Norway is that the likelihood of having received more than full compensation is greater than the likelihood of having received less.

Tensions and disputes tend to arise either directly in relation to the foreseeability test, or else in relation to one of the disregard rules that encode aspects of the no-scheme principle. The disregard rules included in statute are all formulated in relation to the value of sale, although they are also regarded as applying to value of use assessments. To some extent, they may also be redundant, in so far as they already follow from the foreseeability test.\footnote{I recall that the Husaas committee itself thought that a rather wide no-scheme principle would follow already from the foreseeability test. See above.}

The main disregard rule included explicitly in the \cite{ca84} is formulated very similarly to the no-scheme rule in the UK. It states that one should not take into account changes in value that can be attributed to the ``expropriation measure''.\footcite[5]{ca84} Interestingly, the notion of an ``expropriation measure'' is defined in the Act. In section 2, the expropriation measure is said to be the ``activity, installation, or purpose'' benefiting from expropriation. Hence, while there is a definition, it is (as expected) very vague. 

If the definition had only included the second item -- that of an ``installation'' -- it would amount to a meaningful restriction. However, as it stands, an ``expropriation measure'' seems like it could include just about anything that stands in some kind of relationship to the expropriation order.
The purpose of the expropriation is included in the list, which, if taken literally, would lead to rather absurd results. 

For instance, if houses next to a small public road are expropriated for the construction of a motorway, the wording suggests that even the presence of the public road must be disregarded when assessing their value. This would seem to follow, in particular, in so far as the pubic road was built in pursuance of the same purpose as the motorway now being constructed. Luckily, the rule is not understood in this way in practice. 

However, a second rule expressed in section 5 of the \cite{ca84} states that changes in value due to other investments that the expropriating party has carried out, or plans to carry out, also falls to be disregarded. The condition is that they have {\it either} been carried out in relation to the expropriation measure {\it or} during the last 10 years.\footnote{See the third and fourth paragraph of \cite[5]{ca84}.} Hence, the disregard rule in section 5 is stronger than most no-scheme rules, in that previous or planned investments must sometimes be disregarded even if they stand in no relation between the expropriation scheme besides being carried out by the same party. In so far as the expropriating party is a public body, even this is relaxed, since all investments carried out by {\it any} public body is then to be disregarded, limited only by the 10 years rule.\footcite[5]{ca84} 

%When the constitutionality of the Compensation Act 1973 came before the Supreme Court in \emph{Kløfta}, they chose to sit as a grand chamber and they reached a decision under dissent, being divided into two fractions, consisting of 9 and 8 supreme judges respectively. However, both fractions approached the problem of constitutionality by endorsing an interpretation of Section 5 nr. 1 in the Compensation Act 1973 that gave the exception to the current use much wider scope than what had been intended by parliament. The majority went farthest, and unlike the minority they also regarded the compensation payment in the concrete case to be insufficient. The first voter for the majority commented as follows on the constitutional aspect of the case.\footcite[7-8]{klofta76}
%
%\begin{quote}
%[...] But the main question in this case, is whether or not it is in keeping with Section 105 to generally award compensation at a level below the market value that could legally be estimated, and that the owner could actually have achieved, if expropriation had not taken place. In my view, this involves allowing expropriation to transfer a right that the owner had, with a value to which he was entitled. If he is refused compensation for this value, he would, depending on the circumstances, be left significantly worse off than others in a similar position, who owns property that is not expropriated. Such a result I cannot accept. It would be a breach of established customary law and a practice that has been established throughout the years both by the appraisal courts and the Supreme Court. I refer particularly to Rt 1951 s. 87 (particularly p. 89, Opdahl). This practice is in itself a significant contribution to interpreting Section 105 on this point.
%\end{quote}
%
%I note the emphasis placed on \emph{market value} in the majority's reasoning. This may appear to be in keeping with an absolutist doctrine, but as I have mentioned, it can have unfortunate, possibly unintended, consequences for property owners, especially when combined with a restrictive view on what counts as foreseeable future development. I note, however, a technical point that might be of some significance for the interpretation of \emph{Klofta}: Instead of stating outright that a market value rule follows from the wording of the Constitution as such, the majority takes the view that this interpretation suggests itself based on the compensation practice that had currently been established. This might limit the scope of the majority's remarks in this regard, but it also serves to give further support to the claim that the role of the appraisal courts, and their assessments, still had a strong position in Norwegian compensation law at the time of \emph{Kløfta}. 
%
%I remark that the minority disagreed on the constitutional status of the market value rule. Indeed, it was in this regard that the difference of opinion between the minority and the majority was most clearly felt. The minority, in particular, explicitly rejected the view that this rule could be derived from the constitution itself, and they also disagreed with the understanding that it would have status as a constitutional rule simply because it had been adopted in practice. This bestowed merely the status of ordinary legal precedent. As expressed by the minority:\footcite[23-24]{klofta76}
%
%\begin{quote}
%Case law from this area cannot be understood as preventing parliament from changing the rules in accordance with what they regard as necessary. That would prevent a reasonable and natural development and would not be in keeping with the consensus view that Section 105 of the constitution is a rule that must be interpreted in light of, and adapted to, how society has developed and how the law is viewed. I believe the practice that have evolved cannot be decisive if a new situation and new needs require a different solution. Whether the Compensation Act is in breach of the right to full compensation enshrined in the constitution, must depend on an interpretation of the wording in the constitution itself.[...] \\ \\
%In my opinion, neither the intentions of parliament nor the way they are sought implemented through Sections 4 and 5 are in breach of the equality principle upon which Section 105 of the constitution is based. It does not follow from the constitution that an owner is in all circumstances -- and irrespectively of the economic forces from which the market value results -- entitled to compensation that is at least as great as the greatest legal value that the property could represent on a free market. A different matter is that Section 105 of the constitution could be important to the interpretation and application of the rules.
%\end{quote} 
%
%Hence, the market value rule was explicitly renounced as a constitutional principle by the minority, who nevertheless conceded that the constitution could be used to interpret Sections 4 and 5 of the Compensation Act 1973. Both the minority and the majority agreed, however, that  it would be wrong to go on to consider Section 4 of the Compensation Act 1973 in isolation. For the majority, this would clearly have led to the Compensation Act 1973 being held to be in breach of the constitution, something that was avoided since the Supreme Court chose to consider the law as a whole, with the majority using the reasoning detailed above to argue for a new interpretation of Section 5, rather than as a means to undermine Section 4. Still, their interpretation of Section 5 went well beyond what it seemed that parliament had intended, leading some scholars to claim that \emph{Kløfta} should be read as holding that the Compensation Act 1973 was unconstitutional.\footcite[477]{andenes86} In the words of the majority:\footcite[12-13]{klofta76}
%
%\begin{quote}
%The purpose of this rule is to award compensation beyond current use in cases where valuations according to section 4 could be in breach with section 105 of the Constitution. As it stands, section 5 no 1 is not sufficiently suited for this purpose. By its wording it gives the appraisal courts an opportunity to assess whether or not it is reasonable to award additional compensation, even when the conditions for this is otherwise met, and even then with the limitation that the compensation would otherwise be significantly unreasonable. Such a free position for the individual appraisal courts -- without possibility of legal appeal -- would not be in keeping with the purpose of the rule and the demand for full compensation set out in the Constitution.
%\end{quote}
%
%On this basis, the Supreme Court chose to interpret section 5 no 1 in such a way that whenever the conditions were fulfilled, the appraisal courts were \emph{obliged} to award additional compensation, On this basis they found that the property owners in \emph{Kløfta} was entitled to have their compensation looked at again, in a new round before the appraisal courts. The minority agreed in principle, yet did not go as far as the majority, concluding that based on the particular facts at hand section 5 had been adequately considered by the appraisal court in this particular case.\footcite[22]{klofta76} In addition, the majority went quite far in suggesting that ``full compensation'' entitled the owner to {\it market value} compensation, whenever this would result in a higher award than a ``value of use'' approach.\footcite[14]{klofta76} Moreover, they adopted a more narrow interpretation of the (negative) no-scheme rule, whereby public plans not closely related to the expropriation project should not be disregarded.\footcite[15-16]{klofta76} In these matters, the minority took a different view, arguing against market value as a general benchmark and in favour of a broader no-scheme rule.\footnote{\cite[22-23|30-31]{klofta16}.}
%
%The upshot of \emph{Kløfta} was that section 5 no 1 came to be seen as an obligatory rule, leading to compensation having to be enhanced whenever the current use rule led to payments that did not reflect the market value of comparable properties. However, the conditions stated in section 5 no 2 and no 3 were still regarded as relevant, and in interpreting these conditions, a body of law developed whereby the market value rule was applied in a way that would come to involve significant reduction in compensation compared to what would result from practice as it had been prior to the Compensation Act 1973. In this way, the pragmatic approach proved triumphant, not because current use value was introduced as the general starting point, on the contrary, but because a range of new disregards were introduced to reduce the level of compensation in a range of different circumstances. After \emph{Kløfta}, in particular, the following rules were all considered legitimate ways to decrease the level of compensation.

%In section 5 no 3 and no 4, the Expropriation Compensation Act 1973 encoded the following three disregard principles that are all, to varying degrees, still important in compensation law today. 
%
%\begin{enumerate}
%\item Changes in value that are due to the expropriation scheme should be disregarded, both when these are already carried out as well as when they are planned, c.f., section 5 no 2 of the \cite{ca73}.
%\item To the extent that it is regarded reasonable, \emph{increases} in value that are due to public plans or investments should be disregarded, irrespectively of whether or not they have already been carried out, c.f., section 5 no 2 of the \cite{ca73}.
%\item An increased value falls to be disregarded if it results from considering a use of the property which is not in accordance with public plans, c.f., section 5 no 3 of the \cite{ca73}.
%\end{enumerate}
%
%While the \cite{ca73} has now been replaced by the \cite{ca84}, the formulation given in the 

These rules severely limits the level of compensation payments, and in many cases it appears to make the principle of full compensation based on market value rather illusory, even if this was the principle endorsed by the Supreme Court in {\it Kløfta}. On the one hand, the foreseeability test can serve to rule of value arising from any use of the property that is not in keeping with the current public plan. At the same time, the no-scheme rule explicitly encoded in section 5 can be used to also disregard values that are due to this plan, particularly if they are regarded as standing in some relation to the expropriation measure. 

The outcome could easily become, logically speaking, that no compensation can be awarded whatsoever. However, the system tends to revert back to the current use compensation in such cases. For instance, if agricultural land is expropriated for the purpose of a motorway, and it would otherwise appear foreseeable that it might be used for housing in the future, the compensation will usually be based on agricultural use because the value for housing is disregarded under a foreseeability test while possible increases in value due to the motorway plan itself is disregarded under the no-scheme rule.

In practice, with virtually all novel economic activity making use of land is dependent on acquiring new planning permissions, the current use rule will typically be applied as intended by the \cite{ca73}. The main difference is that the rule is not thought of, or described, as an absolute. It rather tends to arise merely as a side effect of other rules.\footnote{A similar point was made in \cite{stordrange94}.} Outcomes that are in keeping with current use thinking will typically be designated as ``full compensation based on market value'' -- the standard phrase adopted in most appraisal judgements -- notwithstanding the fact that the accuracy of such a description depends on the disregards that have been applied.

%The \cite{ca84} was eventually introduced to reflect the principles laid down in \emph{Kløfta}, but it did not in any essentially way change or influence the course of the law that had already been set. Its main purpose was to bring the wording of the legislation more into keeping with how the law was interpreted by the Supreme Court. It explicitly returned to the starting point of the Husaas committee, namely that the compensation should be based on the value of the "foreseeable use" that the owner himself, or an average buyer, might make of the property. But it maintained and endorsed disregard rules no 1-3, except for restricting disregard no 2 to public investments, such that increased value due to public plans currently in place could not be disregarded.\footnote{In this way, the paradox mentioned above, that compensation could become impossible to award because there was no possible basis upon which to calculate it, was avoided.}

The statutory rules do not provide clear guidance as to how the disregard rules should be understood or applied, nor do they consider or resolve the question of when, if ever, they would need to be applied with caution in order not to go against the constitution. However, following {\it Kløfta}, there has been a growing expectation that cases where such issues arise should be resolved by crisp rules, not by the discretion of the appraisal courts. The age when the appraisal courts were considered free to assess cases directly against the Constitution is gone. Rather, an ethos had taken hold where the need to curb the freedom of appraisers, in the interest of ensuring predictability and centralized control, is emphasized.

As a result, difficult cases now routinely end up in the Supreme Court. Here, difficult circumstances are used as the basis for formulating more and more specific rules for special case types. As an example of this mechanism, it is enlightening to consider the case law surrounding the question of whether public plans currently in place are binding when calculating compensation. This rule cannot apply without exception, as recognized already by the \cite{ca73}. But when is it permissible to deviate from it?

The question has arisen in many Supreme Court cases following {\it Kløfta}. \emph{Østensjø} concerned land that was being expropriated for housing purposes, but such that one unlucky owner would only contribute land used for infrastructure that would serve the larger housing project.\footnote{\cite{ostensjo77}.} In this case, the Supreme Court agreed that he was entitled to compensation based on value of his land for housing purposes, irrespectively of the fact that \emph{his} land could not be used in this way according to the plan. However, in many other cases, the disregard rule is upheld even when it is hard to see it as either fair or just, simply on account of it having status as a general rule.\footnote{For instance in \cite{malvik93}. In this case, owners of property used for a motorway were only entitled to compensation based on current agricultural use because the planned motorway-use was assumed binding for the compensation assessment under the market value approach.}

One example is found in \emph{Sea Farm} which dealt with the issue of whether or not the owner of a commercial property should be awarded compensation for the value of investments carried out by the previous tenant.\footcite{seafarm08} There was no doubt that the owner was entitled to these investments, but since the acquiring authority was the only purchaser who was likely to benefit commercially from them, no compensation was awarded for the loss of these investments. This, in particular, followed from a strict reading of the requirement that compensation should be based on the foreseeable use that an "average" buyer could make of the property, encoded in Section 5 of the Compensation Act 1984. Adherence to the wording used in the act seems to have taken priority over an assessment based on the facts of the case. It seems difficult to argue that it would be either unjust or unreasonable, in particular, to compensate the owner for investments that would prove commercially valuable to the acquiring party.\footnote{The decision was sharply criticized by a former supreme judge. See \cite{skoghoy08}.}

In my opinion, this example illustrates how the development of compensation law towards greater reliance on specific rules rather than concrete assessment based on general principles can be harmful. I also threatens to undermine the idea behind the special procedure used to decide appraisal disputes, which has a long history in Norwegian law.\footnote{One might ask if it has status of constitutional customary law, especially since it concerns the mechanism by which a constitutional rule is meant to be upheld.} It also seems to severely underestimate the extent to which compensation rules, when applied to concrete cases, must and should be interpreted based on the context of the case. It seems difficult, if not completely impossible, to achieve social fairness and individual justice by a set of specific rules on the basis of which all legal issues can be resolved mechanically by blind application of such rules. %Moreover, it would be wrong to think that Section ... of the Appraisal Act 1917, encoding the principle that laymen should take part in the decision-making both with regards to legal and technical matters that arose in appraisal disputes.

In the following section, I will turn to waterfalls and hydropower. Interestingly, the compensation practices developed in this regard often deviate significantly from the general approach to compensation. The special approach developed, in particular, as a result of the perceived unfairness of denying benefit sharing altogether in such cases. Hence, looking to waterfalls serves to underscore my point about the importance of a flexible system. 

%The main benefit sharing principle that was developed was known as the {\it natural horsepower method} for calculating compensation for waterfalls following expropriation. It was initially developed by the appraisal courts as an ad hoc approach to ensuring some benefit sharing in hydropower cases. Later, however, many came to regard it as a binding principle of customary law. Gradually, it came to be applied by the courts with little or no regard for how well it suited the circumstances of the case and the changing realities of the hydropower sector. As a result, the method became hopelessly outdated, leading to compensation payments that had little or nothing to do with the actual value of waterfalls for hydropower. 

%Today, while the method has been abandoned for certain case types, it is still applied as the default rule for compensation waterfalls.
%
%
% address this issue in more detail, and we will argue for a different conceptual approach to compensation law, grounded both in the procedural tradition of appraisal courts and the more subtle parts of the absolutist and pragmatic theoretical traditions. It seems to me that the most striking lesson that should be drawn from considering the history of Norwegian compensation law is that a \emph{contextual} view of compensation has been a common denominator that both the absolutist and pragmatist camps have endorsed. Unfortunately, this common element was overshadowed by political conflict regarding the weighing of different values. However, there can be little doubt that social fairness and individual justice should \emph{both} to be regarded as important objectives for compensation rules. Moreover, while they may sometimes be opposing, they need not be, and their exact relationship depends largely on the circumstances. It seems to us that it is simply inappropriate to let particular political sentiments regarding their relationship and relative importance, sentiments that are usually dependent on the particulars of the prevailing political, social and economic conditions, dictate the development of the legal framework for resolving compensation disputes.
%
%Considering current trends and recent issues in expropriation law, particularly related to commercial expropriation, further suggests that a different perspective is needed on this matter. In particular, we believe it is time to recall the idea of the independent and impartial discretion of the appraisal court, relying on the good common sense of laymen as well as the legal expertise of judges. The appraisal courts should in our opinion be set with the task of more actively evaluating how fairness and justice is best served in individual cases, at least if the overall goal is truly to arrive at a socially fair and individually just compensation system. We discuss this idea in more detail in the final section below.

\section{``Natural horsepowers''}

Following the introduction of a general expropriation authority covering waterfalls in the early 20th century, the question of how to value waterfalls came before the appraisal courts. The regulatory regime that was established made private commercial development difficult or impossible, and this in turn meant that the commercial market for waterfalls all but disappeared. Hence, a strict application of the no-scheme rule could lead to no compensation being paid at all. Arguably, a waterfall had no value to anyone except the acquiring authority, since no alternative development scheme could be regarded as foreseeable.

The appraisal courts did not follow this point of view to its logical conclusion. Instead, they introduced a theoretical formula for calculating waterfall compensation. In effect, this method served to create an artificial market for waterfalls, controlled by the appraisal courts. Initially, this artificial market was modelled on the actual market that had existed prior to the regulatory reform. Over time, however, the waterfall ``market'' would slide further and further into the legal sphere, away from the physical and commercial reality of hydropower development.

The key notion used to determine the price of a waterfall on this market was that of a {\it natural horsepower}, a gross measure of electric effect.\footnote{A horsepower, of course, is an old-fashioned unit of effect which is still sometimes used, e.g., in relation to cars. In the context of electricity, it is replaced by {\it Watts}, such that 1 horespower (hp) = 745.69 Watts.} As I mentioned in Chapter \ref{chap:4}, the lack of a national grid at this time meant that the value of a hydropower plant was largely determined by the stable effect that the plant could deliver, not the total amount of electricity that could be produced. This, in turn, was a function of the degree of water regulation implemented by the hydropower developer. 

To simplify the calculation, the natural horsepower of a waterfall was introduced as a gross estimate of the stable effect that could be ensured given a choice regarding the level of regulation of the watercourse. The value of the waterfall itself was then determined by fixing a price per natural horsepower. This price was set on the basis of prices paid for other waterfalls, with some adjustments typically carried to take into account the level of cost and benefits associated with the hydropower project in question.

As I remarked in Chapter \ref{chap:4}, the notion of a natural horsepower is used in other contexts as well, for instance to determine what kind of licenses a development project requires. The use made of it to calculate compensation for waterfalls had no legislative basis, but arose as a result of the appraisal courts' efforts to calculate market prices. After the actual market based on the natural horsepower method disappeared, the method stuck and was applied as a matter of custom.\footnote{See generally the description of the history of the method given by the Supreme Court in \cite{uleberg08}.}
%
%
%prove shockingly unfair to owners of waterfalls. Presumably, since waterfalls could not be exploited for any significant commercial gain except through hydro-power exploitation, disregarding the hydro-power scheme when calculating compensation could lead to nil or close to nil being awarded to the owner. But this was not seen as an acceptable outcome, and instead the Norwegian courts introduced a special method to compensate waterfalls that gave the owner a \emph{share in the value of the hydro-power scheme} for which expropriation was taking place.
%
%Norway did not at this time have any legislation specifically aimed at regulating compensation following expropriation, and when formulating the special rules for compensation of waterfalls, the Norwegian courts seems to have relied on an analogical application of the gross valuation techniques introduced in the Industrial Concession Act 1917 and the Watercourse Regulation Act 1917.\footnote{Act No. 17 of 14 December 1917 relating to Regulations of Watercourses and Act No. 16 of 14 December 1917 relating to Acquisition of Waterfalls, Mines and other Real Property}. Neither of these acts were aimed at compensating owners, but they relied on methods for assessing the potential and significance of hydro-power projects with respect to the question of whether or not a special concession from the State was required.\footnote{To acquire the waterfall and the right to regulate the water-flow respectively.} In effect, by relying on the methods of valuation introduced there, the compensation mechanism that was introduced deviated completely from the "value to the owner" principle. On the other hand, it also closely mimicked the manner in which owners of waterfalls would be compensated on the market in the early days, prior to the introduction of our concession laws, when speculators would pay for waterfalls on the basis of what they assumed to get out of them in subsequent hydro-power projects.

In the Supreme Court case of \emph{Hellandsfoss}, some 80 years after it was first introduced, the natural horsepower method was described and put into context as follows:\footcite[1599]{hellandsfoss97}
\begin{quote}
The principle set out in the Compensation Act, Section 5, is that compensation should be determined on the basis of an estimation of what ordinary buyers would pay for the property in a voluntary sale, taking into account such use of the property as could reasonably be anticipated. For waterfalls, however, this often offers little guidance, and the value of waterfall rights have traditionally been determined based on the number of natural horsepowers in the fall, which are then multiplied by a price per unit. The unit price is determined after an overall assessment of the waterfall, including the cost of the scheme, its location, and levels of compensation paid for similar types of waterfalls in the past. The number of natural horsepowers is calculated by the formula ``natural horsepower = $13.33 \ \times \ Qreg \ \times \ Hbr$'', where $Qreg$ is the regulated water flow and $Hbr$ is the height of the waterfall.
\end{quote}

In this formula, $Qreg$ represents a quantity of water, measured in cubic meters per second (m3/sec), while $Hbr$ represents height measured in meters. The number $13.33$ is the force of the gravitational pull on earth measured in horsepower. 

In the standard account of the natural horsepower method, it is often said that the number of natural horsepower in a waterfall is a measure of gross effect, giving us the amount of ``raw'' power in the waterfall.\footnote{See \cite{vislie02}.} This is not accurate. Indeed, from the quote given above it is clear that the natural horsepower does {\it not} depend only on the nature of the waterfall. It also depends on the specific plans for development presented by the expropriating party. In particular, the quantity $Qreg$ is entirely a function of how the developer {\it chooses} to develop the waterfall, in that it measures the ``regulated water flow''.\footnote{In addition, the quantity $Hbr$ depends on the height over which the developer plans to make use of the water. The development potential that the owner is deprived of can amount to either more or less than this, depending on the nature of alternative schemes.}

In {\it Hellandsfoss}, the Supreme Court itself glosses over this point when it speaks of the ``natural horsepower in the fall''. It would be more accurate to speak of the natural horespower of the particular development scheme benefiting from expropriation.\footnote{Regulation of a watercourse can involve building a reservoir and/or installations that transfer water from one river to another. Then, if there is excess water, for instance due to flooding, water can be stored in the dam for later use. When there is no drought, the stored water can be released. In this way, it becomes possible to even out the water-flow over the year. Today, however, many hydropower plants, particularly smaller ones, involve little or not regulation. Instead, such run-of-river scheme operate by harnessing energy from whatever water is present in the river at any given time.}

But how exactly is the regulated water flow determined for the purposes of compensation estimation? In section 2 of the \cite{ica17}, it is said that the regulated water flow is to be determined ``on the basis of the increase of the low water flow of the watercourse, which the regulation is supposed to cause beyond the water flow which is considered foreseeable for 350 days a year.'' Hence, the idea is that only the {\it increase} in water flow is to be measured. This means that if the developer proposes a run-of-river project with no regulation, then the natural horsepower of the project will automatically be $0$.\footnote{In fact, things could become even worse for the owner, since the proposed project might lead to $Qreg$ becoming a {\it negative} number. This follows from section 10 of the \cite{wra00}. Here the NVE is given the power to compel the owner of a hydropower scheme to ensure that a certain quantity of water is always allowed to pass through the intake of the plant. This flow of water is typically referred to as the {\it minimum water flow}, but is sometimes used in place of the low water-flow before regulation when calculating the natural horsepower of a project. The idea behind imposing a minimum water-flow is to reduce the negative environmental impact. For many run-of-river schemes, the minimum water flow ordered by the NVE is higher than the low water flow after regulation. Hence, if the minimum water flow is subtracted from the low water flow after regulation, the result is a negative number. That is, one might end up with a {\it negative} $Qreg$.} But this outcome was averted in practice by an {\it ad hoc} adaptation of the traditional method. In relation to compensation, it became established practice to omit the deduction of the previous water flow, so that one would use the entire low water-flow after regulation as $Qreg$. That is, the quantity used for $Qreg$ when computing the natural horsepower of a waterfall for purposes of compensation is the estimated amount of water that is present in the river for at least 350 days a year after regulation.

This means that the natural horsepower of a development scheme has little bearing on the amount of energy that will actually be harnessed from it. Today, modern electricity generators can produce electricity at varying levels of effect, depending on the water-flow of the river. But the water-flow is not a constant as assumed by the natural horsepower formula. Rather, it varies considerably over the year.

As a result, the natural horsepower of a regulation does not have much to do with the value of neither waterfalls nor hydroelectric plants.\footnote{See generally \cite{sofienlund08}.} Indeed, the annual income of a hydroelectric plant has nothing to do with natural horsepower, it is solely a function of the price paid per kilowatthour and the total number of kilowatthours harnessed over the year (kWh/year).\footcite{sofienlund08} The amount of energy generated in a power plant could be measured in other units than kWh, e.g. in terms of the amount of horsepower-hours per year. But the important point to keep in mind is that an energy producer gets paid for the amount of energy he can deliver, \emph{not} the effect he can maintain in his station over a long duration of time. %Hence, even if if we uwould se kilowatt instead of horsepower and talk of the natural kilowatt of a hydropower plant, the quantity we are discussing is the same, and still has little or no bearing on the value of the waterfall.

Talking of natural horsepower therefore serves to give a skewed picture of the potential of a waterfall, especially for run-of-river projects. It is not unusual that the low water-flow in a river amounts to only about 3-5 \% of the average water supply. In modern hydropower projects, one would expect 70-80 \% of this water-flow to be harnessed for energy production even in the absence of any regulation. Hence, in these cases, the natural horsepower method, as it was traditionally applied, would only compensates the owners for about 5 \% of the energy that would actually be harnessed from their waterfalls.\footnote{sofienlund08}

This observation, which is trivial given a rudimentary understanding of the energy business, was not made in the context of expropriation until late in the 1990s. Moreover, the point was raised against the advice of legal experts who regarded the established method as a principle of customary law.\footnote{In the aforementioned case of {\it Hellandsfoss}, for instance, a local owner raised the issue with his legal council, who advised against raising it as an issue before the appraisal courts. The owner listened to his legal council, resulting in a compensation payment that is only a small fraction of what he would be entitled to under the method used in some more recent cases, e.g., in \cite{sauda08}. Source: Private correspondence.} At the same time, both engineers and government officials were well aware of the inadequacies of the method, as illustrated for instance by the following passage from a governmental report made in 1991:\footnote{\cite[19]{otprp50}, discussing the notion of natural horsepower in connection to the uses made of that term in other parts of the law.}

\begin{quote}
The Ministry of Petroleum and Energy has considered moving a proposition for changing the hydrological definitions in the Industrial Concession Act 1917 and the Watercourse Regulation Act 1917. Today the act uses a calculation method based on an increase in regulated water-flow, i.e. that of natural horsepower.[.......] The hydrological definitions of these acts, supposed to indicate how much electricity can be generated, were made on the basis of technical and operative conditions differing very much from contemporary circumstances. In implementing the definitions referred to above one has tried to adapt to the new technological realities of the present day. Therefore, in practice, a calculation based on current production is used instead. From several quarters, particularly the Association of Waterfall Regulators, there has been raised a strong wish to authorize this practice by altering the definitions of the relevant laws. The Department of Oil and Energy agree, but have not as yet made a sufficient elucidation of the issues to be able to move a proposition of alteration of these acts.
\end{quote}

The quote shows that in administrative practice, it had become common to deviate from the definition of a natural horsepower, since it no longer reflected a relevant figure. A similar move would not be made in the context of expropriation for another 20 years.\footnote{A ``natural horsepower'' calculation modified along the lines described by the Ministry in 1991 is now sometimes used also in compensation cases, following its adoption for some of the waterfalls that were expropriated in the case of \cite{sauda08}.}

Within the ranks of the specialized water authorities, the inadequacies of the natural horsepower method had been known even longer. Here it had also been noted that the method did not give rise to realistic estimates of the value of waterfalls. The first record I can find of such an admission dates back to 1957, from an article written by the director at the NVE which was published in their internal newsletter.\footnote{See \cite{....}. The director even went as far as to illustrate a different method, which would also be outdated given today's regulatory regime, but which would reflect contemporary \emph{actual} valuations, used by the NVE itself.}

Considering the physics behind the traditional method is enough to reveal that it fails to give rise to valuations that reflect the value of waterfalls, under any reasonable set of assumptions about the correct general compensation principles one should adopt. Important in this regard is the fact that  the method relies on data that depends entirely on the expropriating party's project. The compensation to the owner depends not on their loss, but on the technical details of the project that the expropriating party proposes. This clearly deviates from even a narrow interpretation of the no-scheme principle.

However, while the idea of compensating the owner of waterfalls by a price per natural horsepower is fundamentally flawed at the theoretical level, there are even more serious concerns that arise when one begins to consider the way in which the unit price has been determined {\it in practice}. The traditional approach to this question has had a particularly dramatic effect on the level of compensation payments. 

In case law based on the traditional method, it is often said that the price set per natural horsepower is set according to ``market price'' for waterfalls. But for the most part, what this means is that the court looks to prices awarded in earlier compensation cases. This practice gave rise to a price level that was entirely artificial. It reflected, more than anything else, the power balance between buyer and seller in the courtroom. It was certainly no genuine market value, even if it was described as such. This has become very clear after the adoption of new, genuinely market-based, methods in recent years.\footnote{See generally \cite{larsen08}.}

Indeed, while the unit price for a natural horsepower did increase somewhat during the first 80 years that the traditional method was used, this increase neither reflected the value of hydropower in particularly nor the level of inflation in general.\footnote{See \cite{sofienlund08}.} Moreover, while the price-level was determined by the courts, some voluntary agreements were also made on the basis of the same method. These could then in turn be used to back up the claim that this was a genuine market-based valuation principle. In this way, it became possible to legitimize an increasing imbalance of power between owners and purchasers. In the end, this imbalance became extreme.

For instance, in 2002 a waterfall belonging to local landowners in the rural community of Måren, located in south-western Norway, was sold for the sum of kr 45 000 (roughly £ 4500), based on traditional calculations.\footnote{Source: private correspondence.} The waterfall has now been exploited in a small-scale hydro-power plant belonging to the large energy company BKK, with annual energy output of 21 GWh.\footnote{$http://www.bkk.no/om_oss/anlegg-utbygging/Kraftverk_og_vassdrag/andre-vassdrag/article29899.ece$} For comparison, I mention that in the case of \emph{Sauda}, where a more realistic market-based method was used, the owners received a compensation which totalled about 1 kr/kWh annual production.\footnote{LG-2007-176723 (I acted as council for some of the owners in this case).} Applied to the Måren case, this would take the compensation from kr 45 000 to kr 21 000 000. That is, the price would have been almost 500 times higher.\footnote{In fact, the Måren waterfalls were cheaper to exploit, so in reality, one would expect that the new method applied to Måren would yield even greater compensation per kWh. I also remark that the value awarded in \emph{Sauda} was market-value, not value of use. It was assumed, in particular, that the owners would have to cooperate with a ``professional'' energy company to develop hydropower. This, in effect, halved the compensation awarded, since the Court's decision was based on the premise that the professional company was willing to pay about 50\% of the profit as rent to the owners.}

The case of Måren illustrates an important point, namely that when the traditional method was used, and described as the ``market value'' of waterfalls by the courts, this became a self-fulfilling prophecy. The prices paid in voluntary transactions were influenced by the practice adopted by the courts far more than the other way around. This, indeed, appears to be a general danger in cases when expropriation is widely used for some particular purpose. The prices paid can easily be kept artificially low by developers making use of expropriation as soon as prices begin to rise. In that way, by relying on what is ostensibly ``market value'' compensation, an artificial price level can be established and maintained. 

I mention that in a setting where the owners are politically powerful and can exert undue influence on the compensation process, the effect can be reversed, so that the ``market based'' approach leads to inflated compensation levels, including elements of holdout value. The general point is that the market approach can be turned to the advantage of the most resourceful and powerful groups, particularly in situations when expropriation is widely used for a particular kind of development. In such cases, a market-based approach is not as politically neutral and ``objective'' as its proponents tend to argue.


The potential severity of this mechanism is nicely illustrated by the case of Norwegian waterfalls. In my opinion, preventing such a mechanism from undermining the fairness of a compensation regime is a main challenge associated with regulatory systems that presuppose extensive use of expropriation. Moreover, in case expropriation is used to further economic development by commercial actors, it is likely that the effect will be detrimental to owners, while creating increased financial incentives for developers to favour expropriation. In this way, a vicious circle is established which can make it hard to break out of the ``expropriation loop'', even though alternatives exist that fulfil the same public interests while ensuring far more equitable forms of benefit sharing and participation.

\section{{\it Kløvtveit} and {\it Otra Kraft}}

Following the liberalization of the Norwegian energy sector in the 1990s, the traditional method came under increasing pressure. It was argued to be unjust by owners and it was held to be illogical by engineers working on developing small-scale hydropower.\footnote{See generally \cite{dyrkolbotn96}.} Eventually, legal professionals followed suit and came to the realization that established compensation  rules based on market value could be applied.\footnote{See generally \cite{larsen06}.} 

Indeed, a new market for waterfalls had begun to develop at this point, following the increased interest in small-scale hydropower and the formation of new companies specializing in cooperating with local owners. For transactions of rights to waterfalls taking place in this market, the traditional method of valuation was not used. In fact, waterfalls were rarely sold at all, but rather leased to the development company for an annual fee. Typically, this fee was calculated by fixing a percentage of the energy produced during the year, and compensating the owners of the waterfall by multiplying this with the market price for electricity obtained throughout the year, possibly deducting production specific taxes, but with no deduction of other cost. In effect, owners would get a fee corresponding to a set percentage of annual gross income in the hydro-power plant.\footnote{See \cite{larsen06}.}

Usually, such a fee entitles the owners to 10-20\% of the income from sale of electricity, depending on the cost of the project. Moreover, it is common that the owners are entitled to up to 50\% of the income derived from so-called \emph{green certificates}, a support mechanism for new renewable energy projects, corresponding to the Renewables Obligation in the UK.\footnote{See http://www.ofgem.gov.uk/Sustainability/Environment/RenewablObl/ for further details.} Essentially, and somewhat simplified, the scheme allows the energy producer to collect a premium on his sale of electricity, which, owning to its ``green'' status, is valued more highly by buyers (usually electricity suppliers), who are required to ensure that a certain proportion of the energy they offer to their customers is considered green. In Norway, such a scheme has been talked about for years, but was only put in force in 2012.\footnote{http://www.regjeringen.no/en/dep/oed/Subject/energy-in-norway/electricity-certificates.html?id=517462} Currently, energy producers can claim a premium of about 2 pp per KWh per year, meaning that about a third of the annual income for new renewable energy projects comes from the sale of green certificates.\footnote{While the premium must be expected to go down somewhat as the certificate market matures and more energy producers acquire "green" status, it will certainly remain an important source of extra income for renewable energy producers also in the future.}

Since these leasehold agreements tie compensation to the fate of the hydropower project, several questions arise when attempting to estimate a present-day value of a waterfall on this market. The valuers first have to determine what the most likely project looks like. Then they have to determine what the annual production will be. After this, they must assess the cost of constructing the plant, something that will in turn make it possible to estimate the level of rent likely to be paid to the waterfall owners. Then, since this rent is set as a percentage of the income from sale of electricity and energy certificates, the need arises to stipulate future prices, usually for as long as 40 years (the usual length of a leasehold). Finally, a present-day value can be calculated based on this cash flow.

The appraisal courts began to use just such a model around 2005. The first case of this kind to reach the Supreme Court was \emph{Uleberg}. In the appraisal Court of Appeal, the lay appraisers overruled the juridical judge and awarded compensation based on the new method. The Supreme Court ordered a retrial on a technicality, but it also commented that it supported the adoption of the new method in cases when \emph{alternative} small-scale development was deemed a \emph{foreseeable} use of the waterfall in the absence of the expropriation scheme.\footnote{\cite{uleberg08}.} Since \emph{Uleberg}, the new method has continued to be used in many cases before appraisal courts.\footnote{See generally \cite{larsen06,larsen08,larsen11}, a series of Norwegian papers discussing the new method.}

It is important to note that it was the lay appraisers that pushed for a new method initially, against the judgement of the legal professionals. This shows, in my opinion, that the old system of lay judgement in appraisal disputes still plays a role in Norway. Moreover, it demonstrates that it has positive qualities that should be preserved in the future. However, the new method is certainly not without its own problems. 

Unsurprisingly, it tends to lead to a rather protracted process of valuation, mostly dominated by experts. Moreover, given all the uncertain elements of the calculation, it is typical that the opposing parties produce expert witnesses that diverge significantly in their valuations. While this can be problematic, the fundamental \emph{legal} challenge arises with respect to the no-scheme rule. In particular, what hydropower scheme should the compensation be based on? Several questions arise, as listed below.

\begin{itemize}
\item (1) Is it foreseeable that the waterfall could be used in a hydropower project in the absence of a power to expropriate?
\item (2) If the answer to question (1) is yes, what would such a scheme look like?
\item (3) Is it foreseeable that such a scheme would obtain the necessary licenses?
\item (4) Does the no-scheme rule imply that the project benefiting from expropriation cannot be regarded as a foreseeable scheme for the purpose of compensation?
\item (5) Is the fact that the scheme underlying expropriation obtained a development license to be regarded as evidence that no other scheme would be likely to obtain such a license?
\item (6) How should compensation be calculated if it is determined that no hydropower scheme would have been foreseeable in the absence of the power to expropriate? 
\end{itemize}

In some cases, for instance when the project benefiting from expropriation is not commercially viable but is carried out for public purposes with the help of special state funding, the answer to question (1) might be no. However, in most cases, the question will be answered in the affirmative, since the scheme benefiting from expropriation already serves as an indication that the waterfall can be commercially harnessed for energy. However, here the no-scheme rule comes into play and creates severe difficulty once we reach question (2). For what kind of scheme can be assumed foreseeable all the while we are obliged to disregard the scheme underlying expropriation? 

In most cases so far, the owners have claimed that compensation should be based on the value of a small-scale hydropower scheme. Since such a scheme is likely to be clearly distinct from the expropriation scheme, one might think that the no-scheme rule will not come into play. This, however, is not necessarily the case. It appears, in particular, that the answer to question (3), asking about the likelihood of obtaining licenses, will still depend on how one views the no-scheme rule. It seems, in particular, that anyone who answers question (5) in the affirmative, will be inclined to say that the alternative project could not expect to get planning permission. This is so, such a person might argue, precisely \emph{because} licenses were granted to the expropriating party. This line of reasoning has been consistently advocated by the large energy companies, ever since the new method emerged.\footnote{See, e.g., \cite{klovtveit11,otra11,otra13}. The argument is often sugar-coated by pointing to the reasons underlying the decision to grant a license -- typically energy efficiency -- rather than by focusing on the formal license itself. In this way, one arrives at an interpretation of the no-scheme rule whereby the scheme can perhaps be said to have been disregarded even though one still takes into account reasons why it should be preferred over other schemes.}

Then the question arises: Is someone who reasons like this at odds with the no-scheme rule? It would seem so, but remember the earlier discussion on the no-scheme rule in Norwegian law, where I noted that the rule has tended to be applied much more narrowly along its positive dimension. Following up on this, it can be argued that while the expropriation scheme is to be disregarded for the purpose of compensation valuation, the regulation underlying the scheme -- or at least the rationale behind this regulation -- is nevertheless to be taken into account. If this point of view is adopted, then the conclusion can easily become that alternative development is to be regarded as unforeseeable. The reason, moreover, will be precisely the fact that the expropriation scheme received a development license. 

Indeed, this line of reasoning was given a stamp of approval in the recent Supreme Court case of \emph{Otra II}.\footcite{otra13} Here, the presiding judge made the following remarks, quoting Gulating Lagmannsrett (the appraisal Court of Appeal), expressing his support, and adding a few comments of his own.\footcite[]{otra13}

\begin{quote}
"[....] The Court of Appeal finds it difficult to distinguish this case from other cases when it has been established that alternative development is not foreseeable. It does not seem relevant whether this is the case because the alternative is not commercially viable or because the alternative must yield to a different exploitation of the waterfall" 
I agree with the Court of Appeal, and I would like to add the following: As the survey of the general principles have shown, it is assumed, both in the Expropriation Act, Sections 5 and 6, and in case-law, that only the value of a foreseeable alternative should be compensated. This starting point means that it would be in breach of the general arrangement if a waterfall that can not be used in foreseeable small-scale hydro-power was to be compensated as if it could be put to such use.
\end{quote}

Having used the development license granted to the expropriating party as evidence that alternative development was unforeseeable, the Court needed to answer question (6) by coming up with some alternative way of compensating the owners.  To do so, the Court was again faced with considering the implications of the expropriation scheme. One possibility would be to ensure that the negative and positive dimensions of the no-scheme rule came to be aligned with one another. That is, as the expropriation scheme was used to rule out alternatives, one might then proceed to use it also as the basis for valuation. Indeed, this is what the Supreme Court did. But at this point, the adherence to the no-scheme rule and a market-based approach spelled doom for the waterfall owners. As the presiding judge reasoned:\footcite[]{otra13}

\begin{quote}
Based on the arguments presented to the Supreme Court, I find it safe to assume that there does not today exist any market for the sale and leasing of waterfalls for which alternative development is not foreseeable, but where the waterfalls can be used in more complex hydro-power schemes. The appellants have not been able to produce documents or prices to document the existence of such a market
\end{quote}

The implicit assumption is that in order to value the waterfall according to its potential for hydropower production, a market needs to be identified. It is \emph{not} considered sufficient that the scheme for which expropriation takes place is itself a hydropower project, on the basis of which the  waterfall value could be assessed following exactly the same steps as in the new method. I also remark that it is very hard to imagine how a market of the kind asked for here could ever develop. After all, any alternative buyers are, by the Court's reasoning a few lines earlier, effectively excluded from being taken into account. In this case, if there was to be a market, it would presumably have to be one that emerged entirely out of the benevolence of the expropriating party.

In fact, the Supreme Court's reasoning in \emph{Otra II} serve as an excellent example of the type of reasoning that makes the no-scheme rule highly problematic for cases of expropriation that benefit commercial schemes. On the one hand, the rule can be used to argue that the inherent value of the scheme itself should not provide a basis for calculating the compensation. On the other hand, it can be used to argue that alternatives must be disregarded in so far as they represent the same kind of exploitation as the expropriation scheme, because they are inferior to it according to the state.

When taken to its logical conclusion, this line of reasoning leads to an offensive result; The commercial value of the property is not to be compensated because the optimal commercial use is the use that the expropriating party aims to make of it. Note that the conclusion is not just that this optimal value, inherent in the scheme, should not be compensated. No, the conclusion in \emph{Otra II} was that \emph{no} compensation could be estimated for any use of the same \emph{kind}, since such use was not foreseeable, owing to the absence of a market.

It is certainly possible to argue that this decision represent a misguided application of the no-scheme rule. In effect, the Supreme Court allowed the licences given to the expropriating party to act as evidence that alternative development was unforeseeable, while it used the no-scheme rule to argue that the hydropower scheme for which this planning permission was given could not itself form basis for compensation payment based on market value.  On the other hand, it seems that even if we disregard the scheme completely, it is unnatural to base the compensation payment on the value of a hydropower scheme that is less beneficial, both commercially and in terms of resource efficiency, than the scheme for which expropriation takes place. Such a scheme would not, one must presume, \emph{actually} have been carried out, regardless of the questions of whether or not it would have been given licenses in the absence of a preferable scheme. 

However, it is not seem particularly difficult to determine what would have been a foreseeable use in these cases, if one assumed only that the power to expropriate had not been granted. If so, it would seem all but certain that a scheme corresponding closely to that underlying expropriation would be implemented. This scheme, however, would be carried out on the basis of sharing the commercial benefit with the owners, not on the basis of expropriation. 

But in \emph{Otra II}, this line of thought was also rejected.\footnote{Although this was in part due to the point not having been argued before the Court of Appeal.} Instead, the Court states that a return to the traditional method is in order. However, they do not apply it in the traditional way. Rather, they sanction a modified version of it that moves away from compensation based on the level of stable effect towards compensation based on average effect.\footnote{That is, they replace the low water-flow by the average water-flow in the definition of Qref, c.f., Section \ref{sec:nathp}.}In addition, they also sanctioned the use of a significantly increased unit price compared to earlier times.

What to make of this? In fact, it seems hard indeed to make sense of since, effectively, by relying on the traditional method, the Supreme Court contradicts its own conclusion that compensations should be based on market value. Instead, they rely on a method that, in effect, is based on an attempt to quantify the value of the waterfall as it is being used by the expropriating party in his project. However, by relying on a technical method that has been completely outdated, it becomes difficult to assess the outcome properly, at least for a non-expert. This is so even after the modifications have been implemented, which make the method appear somewhat less irrational from a physical point of view.

But it is still noteworthy that the Supreme Court prefers the obscurity of the traditional method, as an established custom, over the explicit conclusion that it simply is not tenable to adopt the ``value to the owner'' principle in cases like this, as least not as that principle is construed in Norwegian law.

In any event, I think there is good reason to be critical of the Supreme Court for sanctioning the view that alternative development was unforeseeable in {\it Otra II}. Still, it is not possible to escape the fact that this reflects a general tendency in Norwegian law, whereby the positive dimension of the no-scheme rule is much weaker than the negative part. Even if it appears unreasonable, it might very well be a correct application of national law. Moreover, it could very well have been that alternative development was unforeseeable for \emph{some other reason}, for instance because the only commercially viable exploitation was the scheme planned by the expropriating party. In this case, the problem of how to compensate the owners in the absence of an alternative form of exploitation would still arise. It is this question, in particular, which seems entirely unsatisfactorily resolved under an application of a ``value to the owner'' principle.

This is witnessed by \emph{Otra II}, and, in fact, it appears that the Supreme Court, in their decision  \emph{not} to follow their own reasoning to its logical consequence, makes quite a powerful statement. For all intents and purposes, the Supreme Court \emph{rejects} the "value to the owner" principle, but they obscure this by wrapping it up in the traditional method, which is deeply flawed. However, the problem it attempts to solve appears significant, and it pertains directly to the question discussed more generally in Section \ref{sec:noscheme}, namely how to compensate owners that loose their land to commercial schemes. 

It seems that even the fiercest supporters of limiting owners' right to compensation tend to find it too offensive to apply this principle when it leaves the owners with no form of compensation in cases when they are forced to give up property to purely commercial undertakings. Indeed, such a practice would surely also be in breach of the human rights law. In these cases, the subjective aspect of the ``value to the owner'' principle is impossible to maintain. If the commercial value falls to be disregarded for no other reason than the fact that the state happens to have granted planning permission to the expropriating party rather than the owner, this is not only dubious with respect to human rights protecting property, but also appears to be a case of \emph{discrimination}, e.g., as prohibited by ECHR Article 14.

The problem does not arise when the buyer sees value in the property that is of a different \emph{kind} than that realizable by \emph{any} private owner. In this case, the rule simply states that the owner should not be able to demand that ``public value'' is transformed into commercial value just for him. This appears like a reasonable principle. But when there is commercial value already present on the "public" side of the transaction, it seems completely unwarranted that the public should be allowed to transfer this value from the owner to someone else without compensation. Thus, it seems that more accurately and acceptably, the ``value to the owner'' principle should be thought of as a ``commercial value'' principle. It seems, in particular, that the principle need to be stripped of any suggestion that a preferential financial position is to be awarded to whoever benefits from expropriation.\footnote{Exceptions might be possible to imagine, but, one would think, only when they can be construed as falling under the ``public value'' banner in some way.}

It seems unfortunate that this aspect has not been made explicit, and the difficulties that arise in the absence of this nuance are nicely illustrated by the case of Norwegian waterfalls. Still, as the case of \emph{Otra II} indicates, an interpretation of the ``value to the owner'' principle along less offensive lines is in reality already in place with regards to Norwegian hydro-power. Here it seems that ``value to the owner'' has in fact \emph{never} been applied in the traditional way. Hopefully, rather than obscuring this fact by relying on an unsatisfactory and artificial method for calculating the compensation, the future will see further developments that recognize the need for new principles. 

It should be recognized, in particular, that as the law has been applied for the last 80 years, despite its grave flaws and injustices, there has always been an implicit recognition in Norwegian law that the owners of waterfalls are \emph{entitled to their share} of the commercial benefits of hydropower. 
In fact, in the recent Supreme Court case of \emph{Kløvtveit}, a novel approach along the lines I am advocating was applied in circumstances similar to that of {\it Otra II}.\footcite{klovtveit11} The conclusion here too was that alternative development was not foreseeable. However, unlike in \emph{Otra II}, the lay appraisers in the Court of Appeal had compensated the owners based on the fact that they regarded it as foreseeable that in the absence of the scheme, the waterfalls would have been exploited in exactly the same way, except that it would have happened in the form of \emph{cooperation} between the owners and the expropriating party. By this line of reasoning, the Court effectively seems to have adopted a more modern ``commercial value'' principle, to replace the traditional method. 

For commercial projects, it seems that in the absence of a power to expropriate, any rational buyer would look to cooperate with the owners. This would not necessarily be a safe assumption to make for non-commercial projects. Such projects may fail to provide the necessary incentives for cooperation, even though they should nevertheless be carried out in the public interest.

I mention that \emph{Kløvtveit} was discussed in \emph{Otra II}. But the presiding judge chose to focus on what he regarded as the ``practical problems'' associated with the prospect of cooperation and a compensation award calculated on this premise. The cooperation model was not the center of attention in the case, however, so one can only hope that \emph{Kløvtveit}, rather than \emph{Otra II}, will become the influential precedent for future cases.

\section{Conclusion}

In this chapter I have presented the current compensation regime associated with waterfalls and I have related it to the broader question of how compensation should be calculated when commercial companies benefit from the property that is taken. I focused particularly on the no-scheme principle, which plays an important role in this regard in many jurisdictions, including in Norway. But I also emphasized another aspect particular to the Norwegian system, namely its reliance on the judgement of lay appraisers. 

I noted that the appraisal courts would typically operate largely unconstrained by specific evaluation rules, as they were directly guided by the Constitution and its requirement that ``full compensation'' had to be paid. I argued that this system was both flexible and capable of facilitating broad fairness considerations. The notion that constitutional absolutism was a rigid system, I argued, is largely unfounded. The system had a procedural flexibility that should not be underestimated and which served as a counterbalance to its seeming adherence to a strict dogma.

In fact, when moving on to consider the case of waterfalls in more depth, I noted how this flexibility was used to great effect. It allowed the appraisal courts to ensure that some benefit sharing was maintained in hydropower cases even after the regulatory system was transformed so that benefit sharing following expropriation would be hard or impossible to achieve in a system based on a legal formalization of the no-scheme principle. For over 80 years, the courts happily deviated from it entirely when awarding compensation for waterfalls. Remarkably, this practice continued even after legislation was passed that provided much more specific guidelines to the appraisal courts, and which seemingly enforced a strict no-scheme principle in Norwegian law.\footnote{More generally, however, I noted how this legislation, and the Constitutional battles that followed it, has lead to a development whereby the appraisers are somewhat marginalized and the Supreme Court itself has assumed greater power in directing them, by providing their own interpretation of a body of legislation that contains many specific rules that are hard to apply to concrete cases in a uniform fashion.}

However, as the appraisal courts were marginalized by increasing levels of top-down control, first by the legislator and later by the Supreme Court, the method that was developed to compensate waterfalls would itself develop into a fixed and rigid rule. It was not adapted, in particular, to reflect technological and economic progress. Since these were particularly rapid and ground-breaking in the energy sector, the result was a very severe mismatch between the real value of waterfalls and the compensation paid following expropriation. 

In the final part of the chapter I then considered recent cases where the traditional method has been abandoned in favour of a market-based approach which is based on the general rules governing compensation today. I found that while these cases tend to result in payments that more closely reflect actual commercial values, they raise severe problems of their own. Here, in particular, the no-scheme principle re-emerges on the scene with full force, becoming a very effective tool for those who seek to argue that hydropower development is not a fruit of property but belong to those who obtain expropriation and development licenses from the state. 

If such arguments are successful, the market-value approach can lead to worse outcomes for local owners of waterfalls than what they would be entitled to under the traditional method. The deeper question that arises, of course, is the following: what is equitable benefit sharing in these cases, and how can it be ensured? A second question is whether owners can in fact {\it demand} some level of benefit sharing on the basis of human rights law. This question is now coming into focus in Norway, as the Supreme Court's decision in {\it Otra II} has been brought before the ECtHR.
 
It is my opinion that the best way to ensure benefit sharing under a compensatory approach is to revive the old system of an independent appraisal procedure relying on the discretion of lay people from the local area. The most important aspect of this, I believe, is that it enhances the democratic legitimacy of the compensatory approach. It is clear that there is a great deal of uncertainty in the kinds of calculations one must engage in to assess the commercial value of a waterfall. Therefore, the temptation to rely blindly on experts and special rules that are not properly understood becomes great. This might reduce the uncertainty involved, but only to some extent. Moreover it also highly increased the risk of unfairness and opens up the possibility that powerful interests can sereptisiously usurp the procedure for their own interests. Compared to this, a system based on direct fairness assesments carried out by noraml people, on the basis of (hopefully) neutral information provided by experts, might well be the best option.

However, I think the inherent difficulty in devising appropriate compensation mechanisms for commercial potentials suggest that the compensatory approach might be misguided altogether. In addition, as soon as one begins to look at the social function of property, and its role in human flourishing, it seems that any kind of financial compensation is going to provide an inadequate reply to deprivation of commercially interesting property. Such a system speaks volumes about {\it who} society deems capable of carrying out commercial projects. The discrimination suffered by property owners of the {\it wrong kind}, should also not be underestimated. 

In light of this, I think it is appropriate to consider alternatives to expropriation in cases when economic rationales dictate economic development. Interestingly, the Norwegian system has an entire legal framework in place that can elegantly facilitate such a shift, should enough political be mustered to compel government and developers to make use of it. In the case of hydropower development, it already being put to the test in an increasing number of cases for substantial development projects, as local developers tend to shun away from outright expropriation of property belonging to unwilling neighbours. 

The land consolidation mechanisms that can be used to facilitate compulsory development in these situations form part of an ancient semi-juridical system of land management in Norway. In my opinion, this framework also points towards the future, as it provides a highly flexible approach for dealing with property and economic development under varying degrees of compulsion. In my next and final chapter I will present it in more depth and argue that it can often provide solutions that are both more effective and more equitable than solutions arrived at in a system that relies on expropriation. The compensation issue, in particular, is resolved simply by giving unwilling owners low-risk financial instruments tied to the development that is ordered to take place on their property.


%\chapter{Enabling Participation in Hydropower Development}\label{chap:6}

\section{Introduction}\label{sec:6:1}

%Traditional narratives of property tend to direct attention at owners and their choices, while traditional narratives of sustainable development tend to direct attention at resources and their uses. In light of the discussion in this thesis so far, it seems that it might be a good idea to allow these two perspectives to meet up, before expropriation becomes necessary or -- as in the case of large-scale hydropower in Norway -- automatic.

In recent years, land consolidation has been used to facilitate hydropower projects in Norway. So far, however, it has been used almost exclusively for small-scale development projects organised by local owners. In these situations, expropriation orders are rarely sought and rarely authorised, even if some owners object to the development plans.\footnote{See \cite{brekken08}.} Instead, various consolidation measures are used, including the practically important ``use directives'', serving to set up organisational frameworks for compulsory implementation of a development plan, possibly against some of the the owners' wishes.

Essentially, a use directive can be used to take some of the holdout power away from the owners, without depriving them of their property. Instead, owners are encouraged to cooperate and participate in a decision-making process that has economic development as an overarching aim. Some argue that because of the compulsion involved land consolidation can leave owners in a precarious position by weakening their private property rights.\footnote{See, e.g., \cite{stenseth07}.} By contrast, this chapter sets out to make the opposite case, namely that the use of consolidation for economic development can be used to strengthen property as an institution, particularly when use directives replace traditional expropriation proceedings.
\noo{
In consolidation cases, interference in property does not take place because property rights must give way to public interests. Rather, consolidation relies on proof that benefits will outweigh harms at the local level, with respect to each affected property. However, this requirement targets the property as a functional unit, irrespective (in principle) of the specific interests of its current owner. Hence, depending on what functions are regarded as more important, the stated desires of the owner might have to give way to other priorities.

Land consolidation therefore relies on what appears to be a highly functional perspective on property: beneficial resource uses, not individual entitlements, take center stage throughout the process. This might limit the power of the owners to do as they please, but it does not marginalise them. After all, it is hard to deny that one of the primary functions of private property is to bestow rights and obligations on its owners. Moreover, in normal circumstances, it would be safe to assume that when a property benefits, then so does whoever owns it.

For this reason, it also seems that consolidation can be used to address the democratic deficit of economic development takings in an elegant way. This chapter addresses this possibility in more depth. Moreover, it sets out to show that the land consolidation court can be seen as a framework for self-governance that is conducive to sustainable management of property as a common pool resource. 

Specifically, this chapter argues that land consolidation can be used to deal with many of the challenges that arise at the intersection between private property, local community, and economic development in the public interest. The Norwegian model might also inspire similar solutions elsewhere, particularly in jurisdictions that are committed to an egalitarian ideal of property ownership. 
}
The chapter starts by introducing this idea in some more depth, before clarifying the consolidation alternative and its building blocks. It then goes on to demonstrate how it works in practice, in the context of hydropower development. It concludes with a discussion of possible objections to the legitimacy of consolidation and a brief assessment of the possibility of exporting the Norwegian consolidation model to other jurisdictions.

\section{Land Consolidation as an Alternative to Expropriation}\label{sec:6:2}

The notion of land consolidation is widely used on the international stage, but it is somewhat ambiguous. Often, it refers to mechanisms whereby boundaries in real property are redrawn to reduce fragmentation, without affecting the relative value of the different owners' holdings.\footnote{See, e.g., the entry on {\it land consolidation} in \cite{mayhew09}.} However, it is also common to use consolidation to refer to mechanisms for pooling together small parcels of land to create larger units.\footnote{See, e.g., \cite{lerman06}.} There is a tension between these two notions of consolidation, with some claiming that consolidation in the latter sense is sometimes used to surreptitiously bestow benefits on powerful property owners, at the expense of weaker groups.\footcite[237-239]{lipton09}

In light of this, I should stress that I will use the term land consolidation in a very broad sense in this chapter, much wider than {\it both} of the interpretations mentioned above. Land consolidation, as I use the term, refers to any mechanism by which the state intervenes, at the request of some interested party, to (re)organise property rights and uses in a local area. Hence, a consolidation measure might as well involve {\it increased} fragmentation of property, if this is deemed a rational form of consolidation of the property {\it values} involved. Importantly, I also use land consolidation to refer to efforts directed at {\it managing} property, not just redrawing boundaries.

%Moreover, in Heller and Hills' work on land assembly districts, a comparison with land consolidation is presented, based on a broad understanding of that term (although not as broad as that found in Norway).\footcite{heller08} One of my main aims in this chapter is to pick up on this, by offering a more detailed comparison with the Norwegian system and its application in the context of hydropower development. The Norwegian system deserves special attention in this regard because it is particularly broad, especially in its authority to issue use directives.

Some might argue that this terminology is strained, but I adopt it for a reason. It is motivated by the fact that in Norway, the institution known as ``jordskifte'', which is officially translated as land consolidation, has exactly such a broad meaning.\footnote{See, e.g., \cite{reiten09,rognes03}. The notion of consolidation at work in Norway appears to be quite unique, but I note that land consolidation also has a relatively broad scope in many other jurisdictions of continental Europe, as well as in Japan and in parts of the developing world. See generally \cite{sky07,vitikainen04}.} Since land consolidation measures in Norway can be used to interfere with property rights quite extensively, one may ask about the legitimacy of consolidation, held against rules that protect private property owners.\footnote{For an analysis of the Norwegian land consolidation process held against the provisions of the ECHR, I refer to \cite{utgard09}.} 

There is a shortage on case law on this regarding the Norwegian system, but legitimacy issues have been raised before the ECtHR regarding the Austrian system of land consolidation, which is also equipped with broad powers to interfere with private rights. Specifically, the Austrian system has been found to offend against the property norm in P1(1) of the ECHR in cases when the consolidation procedure has dragged out in time, while severely restricting the owners' use of their property.\footnote{See \cite{erkner87,poiss87}.} 

In some situations, it may be argued that a land consolidation measure {\it is} a form of expropriation, even if it is not recognised as such by the legislature or the executive. In the US, for instance, a land consolidation provision ordering escheat (to Native American tribes) of fractional property interests in Native American reservations was struck down as an uncompensated taking by the Supreme Court.\footnote{See \cite{hodel87}.}

In relation to the legitimacy issue, the Norwegian system stands out in two important respects. First, the consolidation procedure is managed by judicial bodies, namely the land consolidation courts.\footnote{See generally \cite{langbach09}. The fact that the land consolidation process is administered by a judicial body appears to be unique to Norway, see \cite[45]{sky01}.} Second, land consolidation is largely seen as a service to owners, not a tool for increased state control and top-down management.\footnote{See generally \cite{sky09}.} In particular, a case before the land consolidation courts is almost always initiated by (some of) the affected owners themselves and the court often acts as a ``problem-solver'', aiming to facilitate dialogue and cooperation among owners.\footnote{See generally \cite{rognes98,rognes03,rognes07}.} Finally, the so-called no-loss requirement is a core principle of consolidation law, stating that no consolidation measure can take place unless the benefits make up for the harms, for all the properties involved.\footnote{See the \indexonly{lca79}\dni\cite[3 a)]{lca79}. In terms of economic theory, this amounts to requiring that all measures should lead to Pareto improvements, see \cite[59-61]{miceli11}. Moreover, what is required is actual improvement, not merely {\it potential} improvement (known as Kaldor-Hicks improvement, see \cite[61-63]{miceli11}). Importantly, the no-loss guarantee requires Pareto improvements in or in relation to the affected {\it properties}; no mention is made of their respective individual owners. As a result, the no-loss criterion is averse to the monetization of benefits and harms -- it is normally not possible to fulfil the no-loss requirement by paying compensation to the owners of adversely affected properties, see \cite[394]{sky09}. Improvements must typically be rendered in kind, in a manner that offsets potential losses to the property as such, independently of the owner's own (hidden or revealed) valuations, see \cite[371-372]{sky09}. Hence, the starting point here is completely different from that normally assumed in economic analyses of takings law, where complete monetization and individuation is standard (usually starting from the notion of the owners' {\it reservation price} -- the price at which they would be willing to sell if they behaved non-strategically).} Indeed, this remains one of the key principles of land consolidation in Norway.\footnote{See generally \cite{rygg98}.}

The combination of a judicial procedure that emphasises owner-participation and a no-loss criterion that ensures local benefits means that, arguably, land consolidation in Norway {\it strengthens} property as an institution. Moreover, land consolidation can serve as an effective countermeasure against two of the most widely discussed challenges to any property regime. 

First, consolidation can serve to protect an egalitarian distribution of property against the threat of inefficiency and underdevelopment that is otherwise associated with fragmentation.\footnote{See generally \cite{heller98}.} Importantly, it can do so without disturbing the underlying property structure and without bestowing disproportionate benefits or harms on certain owners or other select groups. In particular, land consolidation can ensure commercial development without pooling together property rights and without handing property over to powerful market actors. 

Second, land consolidation can be used to manage common pool resources (if understood as being jointly owned), to tackle problems of over-exploitation and under-investment arising from how harms and benefits may be insufficiently targeted among members of a potentially large group of resource users.\footnote{See generally \cite{hardin68,demsetz67}.} Specifically, land consolidation can ensure sustainable management of jointly owned resources without necessarily forcing an enclosure process (enclosure {\it can} be the result of land consolidation, but it is only one of many measures in the consolidation toolbox).

In short, land consolidation can be used to address both anti-commons and commons problems, in a way that protects, and possibly enhances, desirable social functions of property, through a judicial system that combines participatory and adversarial decision-making. Furthermore, land consolidation is based on a conceptual premise that -- potentially -- offers increased protection to owners and their properties, by recognising them as members of a community that are mutually dependent on each other. In this way, the form of property protection offered in the context of land consolidation is distinct from the protection offered in the context of expropriation. But it is not necessarily weaker.

%The vision of land consolidation at work here is one that sees it as a means for setting up a mini-democracy on demand, to organise decision-making processes in a way that grants those most intimately affected -- the owners and (possibly) other property dependants -- a say that is proportional to their stake in the matter at hand. Importantly, since land consolidation can be used to impose specific uses of property, it can also be an {\it effective} alternative to expropriation, a compulsory measure that can obviate the need for depriving owners of their property rights. 

%Depending on the compensation regime, the costs associated with the use of eminent domain can be much higher than those associated with land consolidation. Hence, consolidation might be a better approach also from a purely financial perspective. 

In the context of economic development, the no-loss criterion will generally be possible to fulfil  through benefit sharing. Indeed, it becomes the responsibility of the land consolidation court to {\it ensure} that a sufficient degree of benefit sharing results, so that consolidation measures may be applied in accordance with the law.\footnote{For a detailed discussion of the extent of the court's duties in this regard, also discussing recent changes in the law that might indicate a weakening of the no-loss guarantee, see \cite{hauge15}.} Moreover, ensuring a fair distribution of benefits is usually regarded as one of the key goals of consolidation, independently of the no-loss criterion. Often, this is taken to mean that the benefits should be distributed among the affected properties in accordance with their relative value prior to the consolidation measure.\footnote{This principle is not as strictly encoded as the no-loss criterion, but is formulated as an ``ought''-rule. See the \indexonly{lca79}\dni\cite[31|41]{lca79}. In my opinion, this is a weakness of the current framework. I mention that for the special case of consolidation to implement a zoning plan, the rule is absolute, see \indexonly{lca79}\dni\cite[3 b)]{lca79}.}

This way of thinking can clearly be applied to address the compensation issue that arises following an  economic development taking.\footnote{For the compensation issue generally, see \cite{fennell04,bell07}. The land consolidation approach to benefit sharing also parallels key insights contained in the proposal for compensation reform made in \cite{lehavi07} (proposing that a special institution should be set up to allow owners to bargain for higher compensation in for-profit situations).} However, the principle of benefit sharing at work in consolidation is usually not compensatory, but rather one that sees the owners as active participants in the development project, also when it takes place against their will. This is a highly interesting shift of attention, particularly from the point of view of human flourishing conceptions of property. On such accounts, it can make good sense to impose obligations on owners to participate in the fulfilment of public interests, particularly when they and their properties also stand to benefit from doing so.\footnote{See the discussion on the social function theory and human flourishing in chapter \ref{chap:2}, sections \ref{sec:2:4} and \ref{sec:2:5}.}

%The emphasis on benefit sharing in land consolidation also reveals a concrete advantage of this institution compared to traditional expropriation. In particular, while benefit sharing is typically required under consolidation law, it is hardly ever achieved through compensation in the context of expropriation for economic development.\footnote{This is largely due to the so-called {\it no scheme} principle, which states that compensation to the owner following expropriation should not reflect changes in value that are due to the expropriation scheme. I am not aware of a single jurisdiction that does not include a variant of this principle. For a detailed investigation into the question of whether or not it stands in the way of benefit sharing in economic development cases, I point to \cite{dyrkolbotn15}.} This means that the use of land consolidation in place of expropriation has considerable potential also in relation to the worry that owners are undercompensated following economic development takings.\footnote{Many scholars adhering to an entitlements-based perspective on property argue that the tendency for undercompensation is in fact the core problem associated with economic development takings.\cite{fennel04,lehavi07,bell07}.}

The fact that consolidation implies benefit sharing and owner participation means that commercially motivated developers may have an {\it incentive} to favour eminent domain over consolidation. Hence, the question becomes whether or not owners should be able to use land consolidation as a {\it defence} against expropriation. If owners are granted such a right, it would become a very powerful version of what is known in some jurisdictions as the ``self-realisation'' mechanism, a rule whereby owners can sometimes preclude a proposed taking by proposing to implement the required development themselves.\footnote{Rules to this effect are found in several jurisdictions in continental Europe (including a very limited rule to this effect in Norway, pertaining to housing projects), see \cite[13-14]{sluysmans15}.} Even in the absence of any legislation explicitly granting owners the right to rely on consolidation as a self-realisation argument, one might ask whether owners can already achieve the desired effect in practice. Can owners preclude expropriation by asking the court to organise the desired development as a consolidation measure?

As long as the expropriation application is still pending a final decision, the owners could theoretically hope to achieve this. Moreover, consolidation measures to implement economic development would generally fulfil the no-loss criterion. Hence, the land consolidation courts should in fact be {\it obliged} to take on such a case, even if there were also plans for expropriation. However, I am not aware of any case where the consolidation courts have actually intervened in this way. They might hesitate to get involved, particularly if the proposed development is large-scale. Moreover, even if they did decide to get involved, it is not clear how the expropriation authorities would react. In principle, an ongoing consolidation case, or even a formally valid use directive, would not in itself prevent expropriation from taking place. However, it might then become harder to justify an economic development taking as being in the public interest.

After an expropriation order has been granted, things are very different. The law as it stands leaves no room for a consolidation defence in these cases. Quite the contrary, the land consolidation courts would have to respect a valid expropriation order and might even be called on to implement it, by awarding replacement land or financial damages to affected owners.\footnote{See \indexonly{lca79}\dni\cite[6]{lca79}.}

In the future, if the consolidation alternative to expropriation is to develop successfully, it seems that the owners' right to request consolidation in place of expropriation must be strengthened. So far, there are no signs of this happening in Norway. However, the use of consolidation as an alternative to expropriation has received attention from a different angle, as a potentially valuable service to developers who seek a more efficient way to deal with property owners.\footnote{See \cite[84]{prop12}.}

Following a change in the law that takes effect in 2016, private developers without established property interests will be granted the right to bring a case before the land consolidation courts, to seek help in implementing projects that would otherwise necessitate expropriation.\footnote{See \indexonly{lca13}\dni\cite[1-5(3)]{lca13}.} Developers might well be motivated to do so, since this could result in reduced administrative costs and (cheaper) compensation arrangements, e.g., compensation in kind through land readjustment.

In light of this, one must ask the following: will land consolidation remain a service to owners, or will it become a service to developers who seek cheap access to property owned by others? This question is about to become pressing in Norway, as the scope of land consolidation continually broadens, making it interact with expropriation law to a greater extent than before.\footnote{In addition to the new rules granting developers a formal standing in certain consolidation disputes, this development is also strongly felt in the move to apply land consolidation in the context of urban development, outside the traditional scope of agricultural pursuits. See generally \cite{stenseth07}.}

\noo{However, the issue of benefit sharing is bound to come up, particularly in the context of commercial development. In this regard, the risk for developers is that they will be compelled to share the benefits with the owners. However, in order for this to happen, the Land Consolidation Court must actively take steps to make it happen, by recognising the owners' right to benefit sharing. Moreover, while benefit sharing is a fundamental principle for land consolidation among owners, it remains to be seen if this way of thinking will be preserved when new and powerful external actors enter the scene.}

The idea that consolidation can serve as an alternative to expropriation also raises practical questions regarding how it would work in practice. Here there is already much interesting empirical data available, arising mainly from situations when some owners wish to undertake economic development projects on jointly owned land against the will of other owners, possibly also in cooperation with external developers.

In the context of hydropower development, using land consolidation in this way has become very important in recent years. In 2009, the Court Administration reported that land consolidation had helped realise 164 small-scale hydropower projects with a total annual energy output of about 2 TWh per year.\footnote{See \cite{gevinst09}. For the scale, I mention that 2 TWh per year is roughly what it takes to supply Bergen with electricity, the second largest city in Norway with around 250 000 inhabitants.} Moreover, in the Supreme Court case of {\it Kløvtveit}, discussed briefly in the previous chapter, the importance of land consolidation was recognised also in the context of expropriation.\footnote{See chapter \ref{chap:5}, section \ref{sec:5:5:3}.} Specifically, the presiding judge pointed to the prevalence of consolidation in the context of hydropower as a justification for requiring a commercial taker to pay additional compensation to the owners. According to the Court, it would have been possible for the taker to cooperate with the owners rather than expropriate from them. Increased compensation was then required because the taker should not be allowed to benefit financially from choosing not to cooperate.\footnote{See \cite{klovtveit11}.}

To set the stage for a more in-depth presentation of consolidation for hydropower development, I will now give some further details about the Norwegian system, focusing on the system of use directives. %Then, in Section \ref{sec:lch}, I consider land consolidation to facilitate small-scale hydropower specifically. I approach this as a test case for the proposition that land consolidation can be a legitimacy-enhancing alternative to expropriation for economic development more generally.

%%%%%%%%%%%%%%%%%%%%%%%%%%%%%%%%%%%%%

\section{Land Consolidation in Norway}\label{sec:6:3}

Rules regarding land consolidation have a long history in Norwegian law. The first known consolidation rules were included already in King Magnus Lagabøte's \emph{landslov} (law of the land) from 1274, the first piece of written legislation known to have been introduced at the national level in Norway.\footnote{See chapter 4, section 2 of \cite{nou02}.} The earliest rules targeted jointly held rights in farming land, giving owners and rights holders on that land an opportunity to demand apportionment that would give them exclusive rights on a single parcel.\footnote{The share in joint rights belonging to each individual farm was historically determined based on the amount of rent (``skyld'') that each farmer paid to the landowner (a figure that was also used to determine the level of taxation). As mentioned in chapter \ref{chap:4}, most tenant farmers in Norway had bought back their land by the end of the 18th century. However, the notion of ``skyld'' was kept as a measure of the share each farm had in the now jointly owned larger estates. The notion is still important, for instance in apportionment proceedings, as discussed in \cite{ravna09a}.} The land consolidation courts still provide this function, but additional rules were introduced during the 19th century. At this time, the main use of land consolidation was to pool together fragments and divide up jointly owned land, to create larger single-owner parcels that could facilitate higher-intensity farming.\footnote{The fragmented system of land ownership that was consolidated at this time served an interesting function in the earlier agrarian economy, to promote governance through a combination of scattered individual rights and property held in common. See generally \cite{smith00,smith02}.} However, it was noted that complete individuation of property rights was not necessarily required or desirable. Rather, collective-action mechanisms was introduced, to facilitate economic development without disturbing the established governance structures associated with agrarian property rights.\footnote{This idea was behind a range of provisions introduced during the 19th century, not all pertaining to land consolidation. For instance, a special management structure was set up to govern forestry on common land, to avoid overexploitation and ensure rational management without necessitating enclosure. See generally \cite{stenseth10a}.}

\noo{ The rules regarding use directives emerged from this context. The initial objective was to enable rural communities to adapt to changing economic conditions without fundamentally altering them or leading to displacement or depopulation. Moreover, the scope of use directives was typically limited to the regulation and reorganisation of already established forms of joint use.\footnote{See the discussion in \cite[35-37]{nou76} and \cite[47-48]{otprp56}.} It was relatively uncommon to employ use directives to facilitate completely new kinds of development. Over the last few decades, this has changed. Today, use directives are increasingly applied also to organise development projects that are not agricultural in the traditional sense, even for properties that have no prior connection with one another. 

Before discussing this in more detail, it will be helpful to recognise three main categories of consolidation tools, as summarised in the following table:}

Today, consolidation measures can be roughly grouped into the following three categories:\footnote{I consciously omit the compensatory function that a consolidation court can serve by acting as an appraisal court, e.g., in expropriation cases, see \indexonly{lca13}\dni\cite[1-4(d)]{lca13}. This is arguably not a consolidation power at all, but rather an additional function that distracts from the uniqueness of consolidation.}

\begin{itemize}
\item \emph{Apportionment of land}: Rules that empower the court to dissolve systems of joint ownership by apportioning to each estate a parcel corresponding to its share, or by reallocating property through exchange of land.\footnote{This is the traditional form of land consolidation in Norway and the main legislative basis for it is provided in the \indexonly{lca79}\dni\cite[2]{lca79}.}
\item \emph{Delimitation of boundaries:} Rules that empower the court to determine, mark and describe boundaries between properties and the content and extent of different rights of use attached to the land.\footnote{The main legislative basis for this form of consolidation is found in the \indexonly{lca79}\dni\cite[88]{lca79}.}
\item \emph{Directives for use}: Rules that empower the court to prescribe rules for the use of land that can benefit from joint management, including setting up organisational units for carrying out specific development projects.\footnote{These rules are found in the \indexonly{lca79}\dni\cite[2 c), 34, 35]{lca79}.}
\end{itemize}

In all cases, the consolidation court can only employ these tools when they are called on to do so by someone with legal standing.\footnote{See \indexonly{lca79}\dni\cite[5]{lca79}.} This was traditionally limited to the owners and those holding perpetual rights of use.\footnote{See \indexonly{lca79}\dni\cite[5]{lca79}.} Today, the government also has legal standing in many kinds of consolidation cases, but most cases (about 90 \%) are still initiated by owners.\footnote{See, e.g., \cite[135]{bjerva12}.} From 2016, when the \cite{lca13} comes into force, legal standing will be granted to a larger class of actors, including development companies that could otherwise obtain an expropriation licence.\footnote{See \indexonly{lca13}\dni\cite[1-5(3)]{lca13}.} Moreover, legal standing will be granted to all rights- and ground leaseholders.\footnote{See \indexonly{lca13}\dni\cite[1-5(1)]{lca13}.}

After a case has been brought before the court, the consolidation court can implement consolidation measures in so far as they are needed to alleviate problems and difficulties preventing rational use of the affected land.\footnote{See the \indexonly{lca79}\dni\cite[1]{lca79}.} To determine whether or not this requirement has been met, the court will look to the prevailing economic and social situation, as well as predictions for the future.\footnote{See generally \cite{reiten09}.} In this regard, the court is also influenced by what it regards as the prevailing public interests in property use. The role of the perceived public interest is gaining importance; recent reforms have underscored that considerations based on the common good should inform the decisions made by the consolidation courts.\footnote{See generally \cite{prop12} (proposal from the Ministry of Agriculture to the parliament regarding the Consolidation Act 2013).}

%The contextual nature of land consolidation has always been clear. Indeed, the basic building blocks of the current system can be traced back to the influence of technological advances in farming and the modernisation processes that Norwegian society underwent in the 19th century. The law responded to these changes by making consolidation an increasingly powerful instrument for change and development. It was also at this time that it was decided to establish a tribunal system for administering the process, first in the Land Consolidation Act from 1857, which was revised and developed further in 1882 and 1950.\footnote{An overview of the history of consolidation law is given in Chapter 3 of \cite{prop12}.} 

The procedural rules of consolidation closely mimics those that pertain to regular civil courts. This ensures that consolidation measures are only applied by the court following a public hearing where all involved parties are given an opportunity to present their case, give supporting evidence, and contradict each others' testimony. In the following section, I briefly elaborate on the consolidation process step by step.

\noo{

As mentioned in the previous section, the exact relationship with expropriation looks set to become a more pressing issue in the future. In 2005, the Ministry of Agriculture made some comments in this regard, in connection with a revision of the \cite{lca79} that gave consolidation greater applicability in urban areas and with respect to implementing public plans.\footnote{See, in particular, \indexonly{lca79}\dni\cite[2 h), 2 i)]{lca79}.} Some members of the preparatory committee had raised the concern that giving consolidation extended scope in this way would be problematic since it would encroach on expropriation law. Also, the concern was raised that it would effectively render consolidation as a form of expropriation. The Ministry disagreed, commenting as follows.

\begin{quote}
The Ministry would like to point out that one of the main preconditions for consolidation is that a net profit is created for the land in question. This profit is then divided among the parties in an orderly fashion. Individually, the law also guarantees that no one suffers a loss, see s 3 a). [...] In the Ministry's opinion, expropriation takes place on a different factual and legal basis. In cases of expropriation the public makes decisions that deprives the parties of economic values. The purpose then becomes to compensate them in accordance with s 105 of the Constitution, not to increase the value of their land or the annual income they may derive from it.\footnote{See chapter 3.3 of \cite{otprp78} (report to parliament from the Ministry regarding changes in the \indexonly{lca79}\dni\cite{lca79}.}
\end{quote}

When preparing the new Act, the Ministry of Agriculture reiterated this position, but did not reflect further on the question of the exact relationship between consolidation and expropriation. The Ministry observed, however, that changing the law so that expropriating parties could appear in consolidation cases was \emph{reasonable}, since it would then be left up to the developer whether to make use of their permission to expropriate or to rely on consolidation.\footnote{See \cite[84]{prop12}.}

The choice made by the expropriating party in this regard will be of great importance to the affected owners and rights holders. In particular, as the Ministry makes clear, it is an absolute precondition for the implementation of a rights-altering consolidation measure that it serves to make the structure of ownership and use more favourable. This requirement, moreover, refers explicitly to the \emph{area within which consolidation takes place}.\footnote{See the \indexonly{lca13}\dni\cite[3-3]{lca13}.} No similar rule is in place to protect the affected local community following expropriation. Moreover, the practices that have developed for dealing with consolidation cases are centred on the interests of the local owners and their communities to a far greater extent than prevailing expropriation procedures.

For instance, the rule regarding expropriation that corresponds most closely to the no-loss rule requires merely that the benefit to private and public interest exceeds the disadvantages \emph{overall}, not locally and certainly not for each individual plot of land.\footnote{See the \indexonly{ea59}\dni\cite[2]{ea59}.} At the same time, consolidation rules do not place any restrictions on the kinds of development that can be carried out. The consolidation rules pertain instead to \emph{how} it should be organised. 

Moreover, the consolidation courts must always base their decisions on existing public regulations of the property use.\footnote{In the \indexonly{lca13}\dni\cite[3-17]{lca13} it is explicitly stated that the consolidation court cannot prescribe solutions that are not in keeping with such regulation. However, it is also made clear that the consolidation court itself can apply for necessary planning permissions on behalf of the owners and the land in question.} Hence, if the public interest suggests a particular form of land use, the fact that a planning decision detailing development of such use is implemented through consolidation does not entitle the court to review the plans themselves, going against the public interest. But it does introduce an obligation, emerging at the time of implementation, to turn specifically to the interests of original owners and rights holders. Importantly, the court must look for solutions that minimise the burden and maximises the benefit for all the properties involved.
}

\subsection{The Consolidation Process}\label{sec:6:3:1}

A consolidation case is usually initiated by an owner or someone holding use rights.\footnote{See section 5, paragraph 1 of the \indexonly{lca79}\dni\cite{lca79}.} The request for consolidation measures is to be directed at the relevant district consolidation court, one of the 34 district courts for land consolidation that have been set up by the King in accordance with section 7 of the \cite{lca79}. The request is meant to include further details about the affected properties, the owners and rights holder involved, as well as the specific issues that consolidation should address.

However, this requirement is not usually interpreted very strictly, meaning that the consolidation court will often be inclined to take steps to clarify further what the case should encompass, more so than in regular civil disputes.\footcite[39]{langbach09} However, the court may reject the consolidation request if it finds that it suffers from formal shortcomings, pursuant to the same rules as those that apply to civil disputes.\footnote{See section 12, paragraph 2 of the \cite{lca79}, which refers to section 16-5 of the \cite{cda05}.}

If the court decides that the request is well-formed and that it includes sufficient detail to permit material consideration, it goes on to prepare public hearings, following the rules set out in chapter 3 of the \cite{lca79}. These rules mirror those that are in place for civil hearings, including the duty to inform affected parties, the parties' right to present their claims, as well as their duty and right to give testimony and provide evidence supporting it.\footnote{See the \indexonly{lca79}\dni\cite[13|15|17 a)|18]{lca79}.} As in civil cases, a decision is usually made only after at least one oral hearing where the parties may present and comment on the evidence and the issues raised by the case.

Unlike in civil cases, the main hearing typically takes place on the disputed land itself and often revolves around practical rather than legal issues. Moreover, a consolidation case will usually not take the form of a two-party adversarial process, but rather present as a multi-party discussion where the court interacts with a large number of persons who may have both common and conflicting interests in the outcome. The typical case involves 5-10 people, but in some cases there can be hundreds of parties involved.\footcite[39]{langbach09} In addition, it is quite common that the parties are not represented by legal counsel.\footcite[109-111]{rognes00} And even if they are, the owners themselves are typically expected to take an active part in the proceedings.\footnote{See generally \cite{rognes00}.}

The request for consolidation will be the court's point of departure when assessing the case. However, the court is not bound by the claims put forth by the parties. This again marks a differences with most civil disputes. With a few exceptions explicitly listed in statute, the consolidation court may decide to use any measure that it deems suitable to ensure a favourable structure of rights and ownership for the future. However, there is some restriction placed on the court in that the measures taken must be regarded as \emph{necessary} in light of considerations based on the original request.\footnote{See sections 26 and 29 of the \cite{lca79}.} In short, the court should remain focused on the issues raised by the parties, but is free to address these issues using the tools they deem most suited for the job. The consolidation court, in particular, is meant to be a `problem solver', more so than an ordinary civil court.\footnote{See generally \cite{rognes07}.}

When a decision is reached, the parties are notified and the decision is presented and argued for in keeping with the rules of the \cite{cda05}.\footnote{See the \indexonly{lca79}\dni\cite[7|22]{lca79}.} The appropriate format for the decision depends on its content. A regular civil ruling is the form used for decisions that only involve ascertaining the boundaries between properties, while a special ``consolidation decision'' is used to implement apportionment and directives of use. The difference in form affects the appeals procedure; while civil rulings are dealt with by the regular courts of appeal, the consolidation decisions can only be appealed to one of 4 designated consolidation courts of appeal.\footnote{See the \indexonly{lca79}\dni\cite[61]{lca79}.}

The procedural rules remain largely the same at the consolidation court of appeal, which provides an entirely new consolidation assessment.\footnote{See section 69 of the \cite{lca79}.} The decision of the consolidation court of appeal can only be appealed on the grounds that it is based on an incorrect understanding of the law, or that procedural mistakes were made. The ordinary appeal courts hear the case in the first instance, while the Supreme Court is the last instance of possible appeal.\footnote{See section 71 of the \cite{lca79}.}

In general, consolidation cases are different from other civil cases mainly in that they have a fundamentally different scope. A consolidation case is not primarily concerned with deciding the merits of individual claims, but focuses on introducing structures of ownership and rights that will prove favourable to the community of owners. In this respect, the process has an administrative character. However, the fact that it is organised similarly to a civil dispute means that the affected parties can arguably expect to contribute more to the decision-making process than they do when decisions are made by administrative bodies.

Given the special context of arbitration, it is not surprising that the judges appointed to the consolidation courts are required to have a special skill set, different from that of regular civil law judges. In fact, consolidation judges are required to have successfully completed a special master level degree in consolidation. This is not a law degree, but a distinct form of professional education.\footnote{See section 7, paragraph 5 of the \cite{lca79}. The degree in question is currently offered only at the Norwegian College of Life Sciences and Agriculture.}

The consolidation court also relies on the participation of lay people who sit alongside the specialist judge.\footnote{See section 8 of the \cite{lca79}.} These lay judges are appointed by the specialist judge from a committee of lay persons that are elected by the local municipalities.\footnote{See section 8 of the \cite{lca79} (the appointment itself is regulated in the \indexonly{ca15}\dni\cite[64]{ca15}).} Ideally, the appointed laymen should have special knowledge of the issues raised by the case. However, they are drawn from the general population.\footnote{See section 9, paragraph 5 of the \cite{lca79}.}

Summing up, the consolidation process has both administrative, adversarial and participatory characteristics. While the content and scope of the court's decision will often have an administrative flavour and is not primarily directed at settling any specific dispute, the process is judicial. Hence everyone is entitled, and to some extent even \emph{obliged}, to have their voice heard and to partake in the process. Moreover, while the process is guided and overseen by the court, the decisions made will be based on considerations arising from the interests of the properties involved, usually as expressed by the parties in their own words.\footnote{See generally \cite{rognes07}.}

More generally, the court is tasked with determining what is best for the land as a productive unit in the local community, in light of all relevant economic, social and political facts, including the fact that the current owners will remain in charge after the consolidation procedure ends.\footnote{See generally \cite{reiten09,sky09}.} To flag the dual nature of the consolidation process, it is tempting to designate it as a process of judicially structured \emph{deliberation}. The final decision-making authority rests with the court, but the court is required to act on behalf of the rights holders, on the basis of their wishes, but always also in the best interest of their properties and their community.

%This form of decision-making based on multi-party deliberation is interesting in its own right, as it provides a template for management of land that caters to the idea of public oversight and control as well as to the idea of local participation and self-governance. It is a form of land management that seems especially suitable as a means to implement concrete projects undertaken in the public interest, particularly when these would otherwise appear to adversely affect individual land owners and local communities.

For this reason, land consolidation is perfectly situated for providing an additional institutional layer in situations when the public wishes to facilitate or even compel economic development involving privately owned property. In the next section, I present the rules pertaining to use directives in more detail, to elaborate on how consolidation can be used to replace expropriation.

\subsection{Organising the Use of Property}\label{sec:6:3:2}

Traditionally, use directives targeted property rights that were owned jointly or for which some form of shared use had already been established.\footnote{In accordance with s 2 c) of the \cite{lca79}.} However, in the 1979 Act, the power of the courts to issue use directives was extended, so that directives could also be issued when there was no prior connection between the rights and properties in question. The requirement was that \emph{special reasons} made this desirable.\footnote{See s 2 c), para 2 of the \cite{lca79}.} Traditional examples include directives for the shared use of a private road which crosses several different properties, or regulation of hunting that takes place across property boundaries.

The rules pertaining to use directives emerged as an alternative to apportionment of jointly owned property, a more subtle and less invasive measure that could often give rise to the same positive effect as a full division of ownership, without leading to unwanted fragmentation or excessive pooling of resources. Hence, in the now repealed Land Consolidation Act 1950 it was stated that use directives should be the \emph{primary} mechanism of consolidation, such that apportionment could only take place if such directives were deemed insufficient to reach the goal of creating more favourable conditions for the use of the land.\footnote{See section 3 no 3 and 4 of the Land Consolidation Act 1950 and the discussion in \cite[30-37]{nou76}.} In the \cite{lca79}, the two mechanisms were formally put side by side, but the intention behind this was to ensure greater flexibility of the system, not to reduce the scope of use directives. Quite the contrary, the 1979 Act explicitly intended to promote the increased use of such directives, also in conjunction with other measures.\footnote{See the discussion in \cite[35-37]{nou76} and \cite[47-48]{otprp56}.}

Since the Act was introduced, there has been a gradual increase in the willingness of the courts to rely on use directives to facilitate \emph{new development} on the land, not just as a means to regulate an existing activity.\footcite[103]{otprp57} The \cite{lca79} lists a range of different circumstances in which such directives can be applied.\footnote{See section 35 of the \cite{lca79}.} But the list is not understood to be exhaustive. Hence, as the notion of agriculture has broadened to include activities such as small-scale hydropower development, the scope of use directives has followed suit.

In the \cite{lca13}, the list has been replaced altogether by a general rule which makes it clear that the consolidation courts have the authority to give directives whenever they regard this to be favourable to the properties involved.\footnote{See section 3-8 of the \cite{lca13}.} \noo{In addition to this, the new Act also introduces a general rule which gives the court authority to set up joint ownership when a joint use directive is deemed insufficient to achieve the purpose.\footnote{See section 3-5 of the \cite{lca13}.} Hence, apportionment and pooling of property is now on equal footing, although a priority rule is introduced for the latter; pooling will only be considered if directives of joint use are regarded as an insufficient means to ensure more favourable conditions.} The new Act maintains the principle that directives regarding joint use of properties with no prior connection can only be given if there are special reasons for it. However, this requirement is not intended to be very strict and the Ministry of Agriculture was initially inclined to remove it.\footnote{For a discussion on this see \cite[140-141]{prop12}.} However, it was eventually decided that it should be kept in order to flag that two distinct questions arise in such cases. First, the court must consider whether or not joint use is in fact desirable, before moving on to the question of how it should be organised.

In addition to giving directives prescribing how joint use is to be organised, the consolidation court can give rules compelling owners to take joint action to realise potentials inherent in their land. Rules to this effect were novel to the \cite{lca79}. According to this Act, joint action can only be prescribed in circumstances covered by one of the points in a concrete list of conditions.\footnote{The rules are given in the \cite{lca79} ss 2 e), 42-44.} Moreover, joint action directives can only be directed at {\it in rem} property owners, not other parties.\footnote{See the \indexonly{lca79}\dni\cite[34 a)]{lca79}.} Following the new \cite{lca13}, however, the consolidation courts will be authorised to prescribe joint action also to groups of use right holders. In addition, the existing list of circumstances that warrant joint action will be replaced by a general joint action rule, potentially increasing the scope of such directives.\footnote{See section 3-9 of the \cite{lca13}.}

When commenting on this change in the law, the Ministry noted that the joint action rules currently in place have been widely used. Indeed, applying them is now one of the core responsibilities of the consolidation courts.\footnote{See \cite[146]{prop12}.} Joint action directives can even include prescriptions for joint investments.\footnote{See section 3-9 of the \cite{lca13}.} On the one hand, this means that such directives can be used to facilitate capital-intensive new development, making consolidation a more effective tool to implement economic development. On the other hand, questions arise regarding the extent to which it is legitimate to rely on compulsion in this regard, when the current owners are required to contribute financially or put themselves at financial risk.

The magnitude of investments required to undertake complex projects can soon become quite burdensome for individual owners. The \cite{lca79} attempts to resolve this by a rule stating that if a development project will involve ``great risk'', the court must set up two \emph{distinct} organisational units to undertake it.\footnote{See the \indexonly{lca79}\dni\cite[34 b)|42]{lca79}.} First, the rights needed to undertake the scheme will be pooled together and managed by an owners' association. Then, to undertake the scheme itself, a separate development company will be set up on behalf of the owners.

In this way, the risk is diverted away from the individual owners onto a company controlled by them. This company will be entitled to the profit from the scheme, but it will also be required to pay rent to the owners' association on terms agreed on by the parties with the help of the court.\footnote{See s 34 b) of the \cite{lca79}.} The owners are entitled to shares in the development company proportional to their share of the relevant rights in the land, as determined by the consolidation court. However, they are not obliged to acquire any such shares if they do not wish to do so. If they do not, they will still benefit from membership in the owners' association.

This two-tier system provides a mechanism that can also empower owners to undertake large-scale projects, possibly by setting up partnerships with external commercial actors. Moreover, the owners' association is not always obliged to lease out the development rights to a specific owner-controlled development company. The exact rules depend on the statutes of the association, as determined by the consolidation court, but typically it will be possible for a majority of owners to lease out the development rights to an external developer, should they choose to do so. In this regard, conflicts may arise, if some of the owners wish to undertake development themselves, while others wish to strike a deal with an external company. The challenge for the consolidation court, illustrated concretely in the next section, is to organise the owners' association in such a way that the chance of later conflicts is minimised.

After the new consolidation Act takes effect in 2016, both planning authorities and commercial developers may be granted legal standing in the consolidation process. This might prove particularly useful in connection with large-scale industrial development, as it might otherwise be hard to implement such projects successfully. In these cases, the consolidation courts can now function as an arena for interaction and deliberation between the three main groups of stakeholders: the public, the local owners, and the commercially motivated developers.

\noo{Despite the potential for disagreement among owners, I believe it is a strength of the system that  owners retain decision-making power and a right to benefit, even when complex development schemes are to be implemented. Indeed, the rules currently found in Norwegian consolidation law adds weight to the claim that and consolidation might point to an alternative and possibly fruitful way of implementing development projects in a system which presupposes that development takes place through commercial initiatives on the basis of public  planning and control.}

%For now, I conclude that the system currently in place already provides tools that allow consolidation courts to organise large-scale development on behalf of owners, even when this requires considerable property (re)organisation and diversification of risk. Importantly, I note that the consolidation rules also point to a form of implementation that is likely to allow the public to exercise more extensive oversight and control. This follows from the fact that the system clearly \emph{curbs} the power and influence of purely commercial forces by emphasising both the owners' interests and the social, economic and political aims which motivate the underlying planning decisions. Effectively, commercial development through consolidation gives the public a greater say during the implementation stage. After all, the organisational structure and the implementation plans are formulated by courts which are explicitly obliged to consider public and societal interests.

%Such an arena is so far missing at the implementation stage of big development projects. At this stage, owners and their communities in particular tend to become completely marginalised, particularly when expropriation is used.

%It remains unclear to what extent the Norwegian consolidation rules will actually be used to give property owners a leading voice in development projects involving their properties. The tension between expropriation and consolidation has yet to arise in case law from this angle. However, consolidation is beginning to receive much attention as a practical alternative to expropriation. Hence, I believe it is only a matter of time before deeper questions of participation rights and benefit sharing will also arise.

To sum up, use directives are highly versatile tools that may be used to organise extensive projects of land development on behalf of local owners. This form of development organisation makes it possible for original owners to maintain their interest in the land, obviating the need for expropriation, while giving the public a greater opportunity to influence and control  how their planning decisions are implemented in practice.

In the next section, I consider in depth the particular case of hydropower, where the consolidation courts have recently started to make use of a wide arsenal of its tools to facilitate owner-led development.

\section{Compulsory Participation in Hydropower Development}\label{sec:6:4}

In this section, I look at four recent cases in detail, all of which involved directives of use for hydropower development by local owners. The waterfalls and rivers dealt with in these cases are all located in the county of \emph{Hordaland}, in south-western Norway. Three of the cases involved small-scale hydro-power which some of the owners wanted to develop themselves, while the fourth was a case when the owners were also considering a development plan which would involve cooperation with an external energy company. The cases are particularly useful because we have access to data on how the process of consolidation was perceived by the owners themselves.\footnote{This material is due to Sæmund Stokstad, who conducted interviews for his master thesis on land consolidation, devoted to the study of how consolidation measures can be used to facilitate hydropower development. See \cite{stokstad11}.}

In the following, I present each case separately, focusing on the organisational issues, the solutions prescribed by the court, and the reception among the parties.

\subsection{\emph{Vika}}\label{sec:6:4:1}

The case was brought before the consolidation court in 2005, by riparian owners who had all agreed to pursue hydropower development.\footcite{vika05} The owners disagreed on how to organise the owners' association, and on how the shares in this association should be divided among the properties involved, 15 in total.\footnote{See \cite[25-28]{stokstad11}.} However, a consensus had formed regarding the main organisational principle, namely that the owners would rent out their waterfall to a separate development company which every owner would have a right (but not a duty) to take part in. 

The parties in \emph{Vika} were closely involved in the consolidation process and the statutes for the owners' association were based on suggestions made by the owners themselves. The main point of disagreement concerned how the shares in this association should be allotted, a question that was made more difficult by the fact that some owners benefited from old water-mill rights in the river.\footnote{See \cite[26]{stokstad11}. In the end, the consolidation court held that these rights were tied to the form of use relevant at the time they were established. Hence, the rights were not regarded as having any financial value and could therefore be extinguished without compensation, as provided for in the \indexonly{lca79}\dni\cite[2|36|38]{lca79}.}

There was also some disagreement about whether the voting rights in the owners' association should be tied to the number of shares belonging to each owner, or if the owners should simply be allotted one vote each, irrespectively of their share of the relevant riparian rights. The consolidation court went for the first option.\footnote{See \cite[26]{stokstad11}.} However, the way shares where allotted deserves special mention. In particular, the court decided to take into account that some additional water entered the main river from smaller rivers where only a sub-group of the owners held riparian rights.\footnote{See \cite[26]{stokstad11}.} These owners' share in the association was increased accordingly. This is surprising in light of Norwegian water law, as ownership of riparian rights usually arises from ownership of land along the relevant riverbed, regardless of where the water itself comes from.\footnote{See the \indexonly{wra00}\dni\cite[13]{wra00}.} Hence, this is an illustration of how the land consolidation court can opt for organisational solutions that seem rational given the concrete circumstances, even if they do not follow from any generally recognised principles of law. 

The statutes of the owners' association in {\it Vika} also contains a second interesting provision, based on a suggestion made by the owners.\footnote{See \cite[26]{stokstad11}.} This provision states that all rights in the association are to be tied to the underlying agricultural properties so that they can not be sold separately. In Norway, a division of agricultural property requires permission from the local municipality.\footnote{See section 12 of the \cite{la95}.} In recent years, however, this protection of farming communities has grown weaker in practice. It is interesting, therefore, that the owners in \emph{Vika} decided that a dissociation of water rights from the underlying agricultural properties should be expressly forbidden.

According to Stokstad, a general consensus had developed among the parties whereby the land consolidation procedure was seen as a great success.\footnote{See \cite[39-41]{stokstad11}.} It allowed for an orderly and fair decision-making process regarding the conflicts that had arisen. The resolution of the case followed continuous interaction between the owners and the court, where everyone felt they had been given an opportunity to have their voice heard. 

Initially, the situation had been tense, but the consolidation process had resolved all conflicts. Some owners also pointed to the fact that the main hearing had been physically conducted in the local community, in a meeting hall that was neutral yet familiar to the owners. This also gave them a feeling that they were meant to actively partake in the decision-making process. 

When the interviews were conducted, 5 years after the case was concluded, the owners also appeared to agree that the association was working as intended and that the climate of cooperation among the owners was good. The hydropower scheme itself had been completed in 2008, yielding an annual production of around 15 GWh per year, providing enough energy for around 700 households.\footnote{See \cite[41]{stokstad11}.}

Moreover, following the experience of land consolidation, a culture of deliberation towards consensus had developed among the owners. The owners now emphasised the search for a common ground, aiming to reach agreement on important issues. This was reflected, for instance, in the fact that the owner who contributed the land for the power station was given a generous annual fee, in addition to his compensation as a riparian owner.

According to Stokstad, this fee exceeds what he might have gotten if this decision had been left to the discretion of the consolidation court.\footnote{See \cite[40]{stokstad11}.} Hence, it reflects a premium that the owners were now willing to pay to ensure agreement and a continued good climate for cooperation.

In light of this, the case of \emph{Vika} serves as an excellent example of how land consolidation can empower local communities and enable them to embark on substantial development projects.

\subsection{\emph{Oma}}\label{sec:6:4:2}

The case of \emph{Oma} was brought before the courts in 2006.\footcite{oma06} The case involved four properties. The owners of three of them, $A,B$ and $C$, wanted to develop hydropower, while the fourth owner, $D$, was opposed to the plans.\footnote{See \cite[36-39]{stokstad11}.} Rather than attempting to expropriate the necessary rights from owner $D$, owners $A,B$ and $C$ took the case to the consolidation court. They argued that development would benefit all the properties involved. Moreover, they pointed out that an alternative project which would not make use of owner $D$'s rights would be less profitable. Hence, in their view, the consolidation court should compel $D$ to cooperate in a joint scheme. Owner $D$ protested, arguing that the project would not economically benefit him, and that it would also be to the detriment of his plans to build holiday cottages in the same area. 

The case of \emph{Oma} differs from that of \emph{Vika} since the question of whether it was appropriate to use compulsion was more prominent. In the end, the court agreed with the majority that an owners' association with compulsory membership should be set up.\footnote{In doing so, the court relied on s 2 c) of the \cite{lca79}.} To justify the use of compulsion against $D$, the court commented specifically on owner $D$'s plans for building holiday homes, noting first that he was unlikely to be given planning permission, and secondly that a hydropower plant would not adversely affect such plans in any significant way.\footnote{See \cite[36-37]{stokstad11}.} Moreover, the court noted that while owner $D$'s rights were relatively minor, they were quite crucial for the profitability of the project, particularly because owner $D$ controlled the best location for the construction of a dam to collect the water used in the scheme. Overall, the court's conclusion was that a joint hydropower scheme would be a better option for everyone than a project that did not include owner $D$'s property.

The question then arose as to how the shares in the owners' association should be divided among the owners and their land. In regard to this question, the court departed significantly from one of the basic principles of Norwegian hydropower law. This is the principle stating that no right to hydropower can be derived from being in possession of land suitable for the construction of dams or other facilities necessary to exploit riparian rights.\footnote{The principle is well-established in expropriation law, going back to the Supreme Court decision in \cite{herlandsfossen22}. The principle was challenged unsuccessfully following the increased scale of development after the Second World War, as discussed in chapter \ref{chap:5}, section \ref{sec:5:4:2}.} The land consolidation court broke with this principle in the case of \emph{Oma}, deciding instead to set the value of the land designated for construction of a dam and a power station at $6 \%$ of the total value of the rights that went into the owners' association.\footnote{See \cite[36]{stokstad11}.}

The proportion of financial benefit and decision-making power awarded to the unwilling owner $D$ thus increased accordingly, since these rights were all held by him. In fact, his share went from $1.75 \%$ to $7.75 \%$, so the consolidation process itself led to a situation where he would have a far greater incentive for supporting the development. Hence, the decision in \emph{Oma} was more to the benefit of owner $D$ than any other among the involved parties. If the rights in question had been expropriated, $D$ would have been given next to nothing in compensation and would lose his rights forever. Instead, the solution prescribed by the consolidation court gave him a lasting and substantial interest in local hydro-power.

According to Stokstad, interviews conducted with the parties show how the process and outcome of consolidation served to create a much better climate for further cooperation.\footnote{See \cite[44-45]{stokstad11}.} Indeed, when the interviews where conducted, 4 years after the courts' decision, owner $D$ had changed his mind and was now in favour of the development. Moreover, he had also decided that he wanted to take part in the development company. He was not obliged to do so, but his right to take part was ensured by the agreement with the development company, regulated by the statutes of the owners' association.\footnote{The owners' right to take part in the development company is obligatory in some situations, pursuant to \dni\cite[34 b) no 3]{lca79}\indexonly{lca79}.}

The owners all reported that the consolidation process had been very successful and that the court had listened to them, allowing everyone to have their voices heard. Moreover, some owners reported that the court had cleverly maintained a bird's eye view on the best way to develop the land in question, ensuring both long term benefits to all involved properties as well as creating an improved climate for cooperation and mutual understanding. The consensus was that making concessions to owner $D$ was appropriate and had been in the interest of everyone involved. In 2011, the hydropower project was completed and today its output is roughly 5 GWh per year.\footnote{See \cite[45]{stokstad11}.}

\emph{Oma} serves as a good illustration of how consolidation can be an effective instrument for facilitating locally controlled development, also in cases when this requires the use of compulsion against some owners. Interestingly, in this case the successful outcome appears to be partly due to the fact that the consolidation court actively used its discretionary powers when deciding how to organise joint use. This power allowed them to deviate from established rights-based legal doctrine and adopt a more context-dependent approach, pursuing solutions that better suited the situation. Interesting legal questions arise in this regard, particularly regarding the extent to which the consolidation court can deviate from sector-based doctrines when organising development.

For instance, one may ask what would have happened if the majority owners in \emph{Oma} had appealed the decision to the regular courts on the basis that $D$ was awarded too many shares in the owners' association. Would this be regarded as a question of the court's interpretation of the law regarding the owners' \emph{rights}, or would it be regarded as a discretionary decision regarding the best way to organise development? If a rights-based perspective was adopted, the decision would almost certainly be overturned. If not, it would seem beyond reproach, as an exercise of the consolidation courts' discretionary power.

A second interesting question that arises is whether or not consolidation can work as well as it did in \emph{Oma} in cases where conflicts run more deeply, or where the parties favouring development are a minority among the owners. The next two cases I consider shed some light on this issue.

\subsection{\emph{Djønno}}\label{sec:6:4:3}

This case was brought before the courts in 2006, by a local owner $A$ who wanted to develop hydropower in a small river crossing his land, the so called \emph{Kvernhusbekken}.\footcite{djonno06} $A$ wanted the court to help him implement a hydropower project, by compelling the other owners, $B, C$ and $D$, to rent out their share of the waterfall on terms dictated by the court.\footnote{See \cite[28-31]{stokstad11}.} The starting point for the other owners was that they did not want hydropower development. Hence, they were not willing to rent out their rights to owner $A$ or any other developer. There was also a dispute regarding the ownership of the waterfall rights, with $A$ believing initially that he controlled a large majority. It soon became clear that this was not the case. As it turned out, owner $A$'s share of the riparian rights was only $5 \%$, so his financial interest in hydropower was in fact very limited compared to the owners who did not want any development.

On the other hand, the land rights needed for the necessary physical constructions were predominantly held by owner $A$ alone. For this reason, $A$ maintained that the court should compel the other owners to allow him to go ahead with his development plans. The court agreed that hydropower would be a rational use of the waterfall, and initially assessed the case against the rules relating to compulsory joint action.\footnote{See the \indexonly{lca79}\dni\cite[2 e)]{lca79}.} This could have resulted in concrete directives regarding how the hydropower development should be carried out, including at the level of specific investments and building steps.

However, the court eventually held that this approach would place too much of a burden on the owners opposing hydropower. Hence, it chose to resolve the case using directives for joint use. By doing so, the court also restricted the scope of their decision to the establishment of an owners' association that would be responsible for renting out the rights. 

The model used for the owners' association was similar to the one the court adopted in \emph{Oma}. This included allocating shares in the owners' association in a way that took into account the special importance of land needed for physical constructions. In total, this land was held to correspond to $6 \%$ of the shares in the association. Since these rights were held by owner $A$ alone, his share in the association doubled. In addition to this, owner $A$ purchased the shares from owner $B$, so that his total share ended up amounting to $22 \%$. Still, for the majority, membership in the association was imposed on them against their will.

The wording of the statutes for the association took into account that it would be run by a majority of unwilling shareholders. In particular, it was stated clearly that the association was going to rent out the rights in the waterfall such that hydropower could be developed. In \emph{Oma} and \emph{Vika}, by contrast, the statutes only stated that this was the \emph{purpose} of the association, leaving the shareholders with the freedom to determine whether or not to go through with development.

In interviews, those who were compelled to take part in the association against their will expressed dissatisfaction and surprise at the result. Moreover, while the association had apparently tried to be loyal to the wording of the statutes, by looking for interested developers, there had been no willingness among the majority to engage actively with this work. No deals had been made, no separate development company had been set up, and the conflict among the owners was ongoing. Hence, while the case of \emph{Djønno} is an example that consolidation can be used even when it involves compulsion against the majority of owners, it also serves to illustrate that the chance of a successful outcome may be more limited. 

The question arises as to how such cases should be dealt with by courts in the future. According to owner $A$, the problem was that the directives of use were not specific enough. In his opinion, the directives should not have been restricted to merely setting up an owners' association for renting out the rights. In addition, the court should have actively addressed the question of how the development company should be organised. Among the majority owners, on the other hand, the prevailing feeling was that the development in question was more or less doomed to fail from the start, since it was unwanted.

Hence, the case of {\it Djønno} illustrates that when the courts are not prepared to actively organise the development company, compulsory participation might fail in practice unless a majority agrees that development should take place.

\subsection{\emph{Tokheim}}\label{sec:6:4:4}

This case was brought before the consolidation court in 2008, by the owners of \emph{Tokheimselva}.\footcite{tokheim08} The five owners all agreed that development should take place, but they disagreed about how it should be done and about the proportion of each owners' share of the riparian rights.\footnote{See \cite[34-36]{stokstad11}.} Some owners argued that development should be organised by the owner community, while other owners thought it would be best to rent out the rights to an external developer. The case was further complicated by the fact that the proposed development was so substantial that it could require a waterfall transferral license pursuant to the \cite{ica17}. As discussed in chapter \ref{chap:4}, such a license can only be given to a company in which the state controls at least $\frac{2}{3}$ of the shares.\footnote{See the discussion in chapter \ref{chap:4}, section \ref{sec:4:3}.}

The consolidation court eventually decided to set up an owners' association. However, unlike in the previous cases I have considered, there was no adjustment made for land that would be needed for physical constructions. Instead, the statutes state that owners will be entitled to a lump sum estimated on the basis of the damages and disadvantages that a concrete hydropower project will bring. This marks a different kind of departure from established practice in expropriation law; specifically, it rejects the established principle that owners can be compensated on the basis of \emph{either} the value of their waterfalls \emph{or} the damages and disadvantages caused by the project, not both.\footnote{See for instance the case of \cite{vikfalli71}. See also the discussion in chapter \ref{chap:5}, section \ref{sec:5:4:2}.}

In other respects, the statutes for the owners' association are similar to those adopted in the previously considered cases. Specifically, the statutes do not resolve the controversial question of how to carry out development. Moreover, nothing is said about the extent to which interested owners should be given a right of first refusal with respect to the development rights held by the owners' association. This was an important issue raised by the case, but the consolidation court explicitly decided not to address it. %In particular, the statutes of the owners' association explicitly provides separate rules depending on how the development is to be carried out. 

In interviews, the owners expressed that they were happy with how the case was dealt with by the court.\footnote{See \cite[43-44]{stokstad11}.} Everyone was heard and the owners' association was set up in consultation with the parties. However, the main issue, concerning {\it who} should develop the waterfall, was still unresolved after the case concluded. Some of the owners expressed criticism against the court on this basis.

The case of \emph{Tokheim} serves to illustrate that established practices of consolidation, while being well received and understood by local owners, face some new challenges in relation to hydropower, challenges that consolidation courts might be reluctant to take on. It seems that the court in \emph{Tokheim} felt that it was not in a position to assess the question of what kind of development would be best. The court was particularly cautious about expressing an opinion about the legal status of the project with respect to the relevant licensing legislation. %The court did not, in particular, form an opinion about how local owners should proceed to carry out their own large-scale development in a waterfall subject to the \cite{ica17}.

It remains to be seen whether such an agnostic attitude can be maintained by the consolidation courts, as local owners increasingly turn to them for help in resolving disputes regarding hydropower. Moreover, it will be interesting to see how the new \cite{lca13} will influence case law in this area. It seems that a case like \emph{Tokheim} could benefit from the court taking a broader view, possible even by including government bodies as parties in the case, to clarify the licensing status of the proposed development.

\section{Assessment and Future Challenges}\label{sec:6:5}

The cases discussed in the previous section show that the system of land consolidation can work as a practical alternative to expropriation in the context of hydropower development. At the same time, the cases suggest that the land consolidation courts may find it hard to deliver effective directives if owners disagree fundamentally about how their water resources should be managed. In addition, land consolidation courts are clearly less effective in situations when they are forced to consider rules and regulations from other areas of the law, outside their traditional area of expertise. Specifically, the land consolidation courts might be overly cautious about implementing solutions that they fear will contradict sector-specific provisions. Furthermore, the land consolidation courts might be unwilling to intervene in a potential conflict between owners and powerful commercial interests, especially if the sector-specific rules seem to speak in favour of expropriation.

Paradoxically, the potential weaknesses of the land consolidation courts in this regard may be enhanced by the worry that the courts are not authorised to make use of sufficiently strong forms of compulsion against owners. This worry, specifically, can give rise to the argument that the public interest in development is unlikely to be realised through the use of consolidation measures alone. Hence, one may fall back on expropriation, to the detriment of all owners, including those that also oppose development by consolidation.

The consolidation alternative seems to be quite vulnerable to this mechanism. This is illustrated by the Supreme Court case of {\it Holen v Holen}, concerning a conflict between a small quarry and a neighbouring farmer.\footcite{holen95} In order to continue extracting his minerals, the owner of the quarry would have to interfere with the property of a neighbouring owner who was using his land for more traditional forms of agriculture. The farmer was unwilling to reach an agreement with the quarry owner, so the latter brought a case before the land consolidation court. The court noted that it would be possible to reach an accommodation that would benefit both parties and issued use directives that would allow the quarry to continue its operations.

The directives gave the quarry owner access to the farmer's land, who was in turn granted replacement property from the quarry owner. The consolidation court also noted that the quarry would, in the future, be likely to extract minerals that belonged to the farmer. Hence, a directive was issued that gave the quarry owner a right to extract these minerals, provided he paid market value for them. 

Hence, not only was the farmer awarded replacement property for agricultural purposes, he was also granted a share of the benefits that would result from the continued operation of his neighbour's quarry. This was clearly beneficial to his property, economically speaking. The owner himself, however, objected to the arrangement. The Supreme Court found in his favour. This was not because they sanctioned his right to block the continued operations of the quarry, or because they thought the replacement property or the payment model was inappropriate. Instead, the Court held that the farmer's  right to extract minerals could not be transferred to the quarry by a consolidation measure, even if the farmer was ensured payment for his share of the total mineral rights.\footnote{See \cite[1481]{holen95}.} Specifically, such a transfer was not held to fall within the meaning of organising joint use of their properties. 

The perspective underlying this decision is interesting, because it underscores a reluctance to use land consolidation in what would otherwise be a fairly typical economic development scenario. However, {\it Holen v Holen} was decided in 1995, and as I have already mentioned, the law has developed in recent years in the direction of increased use of land consolidation as an alternative to expropriation. Moreover, it might be possible to counter the reasoning of the Supreme Court in {\it Holen} by an organisational arrangement that allows unwilling owners to take active part in the development company later on, if they change their mind. In that case, arguably, the arrangement as a whole falls back inside the meaning of joint use. Still, {\it Holen v Holen} reminds us that critics might be able to raise convincing formal objections against compulsion in land consolidation, on the basis of earlier case law. 

As mentioned earlier in this chapter, some scholars argue that land consolidation sometimes offers less protection to owners than administrative expropriation.\footnote{See \cite[318-319]{stenseth07}.} In expropriation cases, it is true that a range of procedural rules tend to apply, pertaining to notification to the owners, impact assessments, a duty to provide guidance and reasons for the decision, and a possibility (sometimes several) for administrative appeal.\footnote{See \cite[377-382]{dyrkolbotn15b}.} Moreover, after an expropriation order has been granted, the owner can still challenge it before the appraisal courts, in principle at the expropriating party's expense.\footnote{See \cite[382-384]{dyrkolbotn15b}.}

In practice, however, the administrative expropriation procedure often leaves the owners completely marginalised, as they are overshadowed by more powerful stakeholders. This is particularly clear in situations when expropriation arises as a result of more comprehensive planning or licensing procedures, such as in the context of hydropower development.\footnote{See the discussion in chapter \ref{chap:5}. For the same point with respect to planning more generally, see \cite[376]{dyrkolbotn15b}.} In addition to this, the possibility of raising validity objections before the courts in expropriation cases is mostly a theoretical one in Norway.\footnote{For a discussion on this with further references, see \cite[384-386]{dyrkolbotn15}.} The courts almost always defer to the discretion of the administrative decision-maker in such cases.

More generally, the narrative of expropriation is one where the owners have to endure a loss in the public interest, for which they must be compensated as individuals. By contrast, the narrative of consolidation is one where the owners themselves are tasked with making a {\it contribution} to the development project, in the best interests of both the local community and greater society. In particular, the owner's role is no longer that of a passive obstacle to development. Rather, the owner is placed in the position of active {\it participant}, one who might yet have to be nudged to fulfil their potential. In addition, the properties as such receive recognition as important resource units, independently of the interests of their current owners. Moreover, the owners as a {\it group} come into focus, as the process is meant to facilitate rational {\it collective} action.

This is achieved by placing owners in a partly deliberative, partly adversarial, context, which not only tolerates, but also presupposes, their active input to the decision-making process. In addition, the {\it grounds} for imposing compulsory measures that interfere with property rights need to be anchored explicitly in the social functions of the affected properties, not individual interests. A measure is warranted only when it enhances property values, also in the sense of improving conditions for the communities that take their livelihoods from the affected properties. Clearly, this broader sense in which consolidation serves to protect property is not matched by any administrative safeguards in expropriation law.

Hence, I conclude that land consolidation is highly attractive, even when it involves compulsion directed against owners. In a system based on private property rights, it seems only reasonable that owners and their communities retain their position as primary stakeholders, even if the public is adamant that large-scale development needs to take place on their property.

\noo{ principled objections against land consolidation in expropriation contexts appear largely misplaced when for the sub-group of takings that realise commercial potentials. However, a second question arises, of a more practical nature. Will the land consolidation process work in practice, if it is applied to organise commercial development. Increased powers of compulsion might be required, and in keeping with my argument above, I believe such powers may well be granted, as long as land consolidation remains directed at improving the situation for existing properties and their owners, rather than bestowing benefits on someone else.}
 
It should be noted that this positive assessment of consolidation as an alternative to expropriation is  premised on the fact that property in Norway is distributed in an egalitarian manner among the members of local populations, especially in rural areas. If land consolidation is used outside of this context, even in urban Norway, one might ask whether the processes truly empowers the community, not only the landowners. 

Clearly, the current system of land consolidation is an incomplete solution to the legitimacy issues that arise in such cases. Moreover, land consolidation might even have indirect effects that make those issues harder to resolve. For instance, it might be that consolidation will undermine local democracy by allowing powerful owners to remove certain property issues from the broader political agenda. The consolidation courts could become arenas used by powerful owners to prevent marginalised group from accessing decision-making processes of great societal significance.

On the other hand, it is wrong to assume that increased protection of owners cannot possibly benefit non-owners. As long as the owners are themselves members of the local community, the fact that they are offered increased protection through land consolidation might positively affect the community as a whole. As I discussed at length in chapter \ref{chap:2}, the social function theory of property asks us to recognise this effect. Indeed, a local community represented by a handful of its own property owners might be in a much better position to participate in decision-making than a local community represented by career politicians, expert planners, or judges.

In addition to this comes the fact that the land consolidation process has a judicial form, meaning that a judge is already present, as an administrator and a representative of public interests. Moreover, the judge is required to consider what is best for the properties, not the owners. Hence, on a human flourishing account of which functions property should fulfil, the interests of non-owners need to be taken into account. On such an account, the owners participating in consolidation are indeed meant to be representatives of their communities, with obligations as well as rights.

That said, if the scope of consolidation is very broad, or the community representation is too narrow, consolidation itself might come to lack legitimacy. Hence, in some contexts, it would perhaps be advisable to further extend the class of persons that can be recognised as having legal standing in consolidation disputes, to include non-owners without formally recognised property rights (e.g., neighbours, tenants or employees). However, there are reasons for caution in this regard. 

First, there are obvious pragmatic concerns related to the increased cost and complexity of the procedure. From the Norwegian experience, it seems that a few hundred parties would be manageable, but more than that takes us to uncharted territory. Second, moving away from property as a basis for legal standing in consolidation might make it easier for powerful external actors to unduly influence the process. This can be an indirect effect, arising from good intentions. For instance, if a large-scale development involves razing an impoverished part of town, it will not be a good idea to give full legal standing in consolidation to the employees of the development company that stands to benefit. This would be so even if the employees could be classified as `locals' under some imprecise standard for determining who to include in the consolidations process.

%In the standard narrative about the dangers of majoritarian democracy, this problem needs to be dealt with by the central government or by the law, not by the locals themselves. This is where the social function notion of property, as exemplified by the land consolidation system, offers an alternative. By placing functions of property center stage when making decisions, the majority rule can be replaced by a property-based form of {\it unanimity}. This, in effect, is what the no-loss principle does, on a high level of abstraction, when it declares that no property, no matter how small or insignificant, must be worse off after consolidation.\footnote{This, indeed, is how Pareto improvements among a group of individuals are interpreted theoretically: they are improvements that all owners will consent to.} 
%Of course, the meaning of this must still be debated and defined by persons, but by emphasising that the improvement of property, not personal gain, is the purpose, the decision-making process now takes on some of those characteristics that make some argue that disinterested experts should be granted the power to decide about land uses.

The overarching question is how to ensure that land consolidation courts remain respectful towards owners and their communities, while providing a good basis for equitable and participatory decision-making about how to manage property. In Norway, any legal person with a right to expropriate may now act as a party to a consolidation dispute, so this challenge is becoming increasingly pressing. What will the role of the new parties be? Will they become potential partners that owners can rely on to implement projects in the public interest, or will they be regarded as the main stakeholders, whom the land consolidation courts should assist so that they may successfully impose their will on local owners? It will be very interesting to follow this development further.

\noo{ How this story will unfold in Norway might also shed interesting light on the question of whether the Norwegian system of land consolidation for economic development could work well in other jurisdictions. In small-scale rural settings where property is distributed evenly among local people, the procedure might be expected to work well. However, in other contexts, the challenges that now face the Norwegian system might be even more acute. Even so, the underlying idea and premise of consolidation as a means to organise compulsory development seems to have great potential.
}

Clearly, the consolidation proposal is closely related to the theoretical argument made in chapter \ref{chap:3}, in favour of alternatives to expropriation based on local institutions for self-governance. As noted by Ostrom and others, congruence with local conditions is crucial to the success of such institutions.\footnote{See \cite[92]{ostrom90}.} Hence, no single institutional framework is likely to work in all cases. This was also the lesson drawn from the critical assessment of the Land Assembly Districts proposed by Heller and Hills.\footnote{See the discussion in chapter \ref{chap:2}, section \ref{sec:2:6}.} This proposal, aiming for self-governance while striving to limit the risk of abuse in all situations, ends up catering only to a very thin notion of participation, arguably without effectively reassuring those who worry about new ways in which governance might fail.

In light of this, the idea of having a designated institution to interface between governments and communities, while overseeing and gently directing decision-making at the community level, might have wide applicability. At a high level of abstraction, this appears to be a key idea inherent in the Norwegian system of land consolidation.

The consolidation procedure is temporary, but consolidation leaves a lasting effect on how decision-making takes place within the affected area. A consolidation court can even set out to {\it design} an institution for local self-governance. This institution can then subsequently be tasked with resource management after the judicial proceedings come to an end. When the consolidation courts in Norway set up owners' associations to manage local water resources, this represents an instance of such an idea.

Applied more generally, this can become a way for the law to approach the theory of common pool resource management, to establish a mechanism for formal integration of its insights into the legal order. At this level of abstraction, the idea of judicially administered consolidation of local property could potentially have a very wide range of applications, across a range of different jurisdictions and property contexts.

These observations are preliminary at this stage, but I think they point to a promising direction for future work. The intuitive appeal of the consolidation idea appears significant, especially because of its potential as a means to enlist the help of a judicial body to build and improve democracies from the ground up - starting with people in their communities, and their link to the properties they rely on for their subsistence and well-being.

\section{Conclusion}\label{sec:6:6}

The Norwegian institution of land consolidation is a very broad one. Specifically, it also includes  measures seeking to compel owners to use their property in the public interest. This procedure is conceptualised as a service to owners, with a no-loss guarantee in place to ensure that consolidation measures are only implemented when the benefits make up for the harms for all the involved properties individually. As such, consolidation can be a powerful alternative to a taking for economic development, in the public interest and carried out with the active support of the owners themselves.

In practice, however, land consolidation works best when there is a basic agreement among the owners that development is desirable, or at least tolerable. The procedure as it currently exists in Norway   appears to be less effective when there is deep disagreement about whether or not development should proceed at all. Hence, to make the proposal even more suited to replace expropriation for economic development, it might be necessary to enhance the power of the land consolidation court, also in the direction of extending its authority to compel owners to engage with development projects that they fundamentally disagree with.

If this is done, it is important to simultaneously ensure that land consolidation remains a service to the owners. This cannot be taken for granted. Indeed, recent changes in the law grant developers the right to act as parties in consolidation cases and to bring cases before the courts themselves, if they favour it over expropriation. On the one hand, this will enhance the power of the land consolidation court, making it more effective in dealing with cases that involve external parties. On the other hand, there is a possibility that the presence of new and powerful stakeholders will change the nature of the land consolidation process itself, so that it transforms from a property-enhancing institution for self-governance into a planning and implementation instrument for developers.

Despite this, I believe this chapter has demonstrated that the land consolidation regime in Norway functions in a way that sheds interesting light on collective-action alternatives to expropriation. Moreover, land consolidation has been shown to be a highly versatile framework, much more so than other suggestions, such as the land assembly districts proposed by Heller and Hills.\footcite{heller08} Specifically, it seems that the institution of land consolidation can provide a useful kind of democracy-on-demand for decision-making about economic development, in a way that promises to maintain a reasonable balance of power between owners, local communities, and society as a whole. In this vision, external commercial actors are at worst going to be partners in crime, at best partners in prosperity. They will not, however, be allowed to dictate the terms of development.
\chapter{Conclusion}\label{chap:7}

\begin{quote} \small
Proudhon got it all wrong. Property is not theft -- it is fraud.\footnote{\cite[252]{gray91}.}
\end{quote}

\begin{quote} \small
That's what makes it ours -- being born on it, working on it, dying on it. That makes ownership, not a paper with numbers on it.\footnote{\cite[33]{steinbeck39}.}
\end{quote}
\noo{
\begin{quote} \small
Gode bønder nå til motstand reises. Enhver som må gi fra seg farsarvi til kongens grever mener det er det rene ran. \\
Good farmers prepare to resist. Those who must give up their father's inheritance to the lords of the king consider it plain robbery.\footnote{From the so-called {\it Bersogliviser}, an old Norse poem addressed to King Magnus (c. 1024 – 25 October 1047), who was accused of treating his subjects badly, including taking their properties, see \cite[259]{titlestad12}. The King listened to the farmers' protest and was thereafter referred to as Magnus ``the Good''.}
\end{quote}
}
%There is reason to think that fraudulent behaviour requires some, but not too much, in the way of intellectual refinement. Arguably, therefore, the case can be made that 

%The concept of property has received its share of criticism, but it remains omnipresent.

%
\noo{ t. Further to this, legitimacy-enhancing alternative to expropriation were considered. Specifically, the thesis argued that ideas taken from the research on common pool resources could inspire novel solutions for local self-governance that could obviate the need for using eminent domain to ensure economic development. 



%A concrete proposal along this line, due to Heller and Hills, was considered in some depth. 

%An important tension was identified within this proposal, between the ideal of self-governance and the danger of abuse at the local level. This, in turn, reinforced one of the key design principles of common pool resource management, namely that institutional arrangements need to be sensitive to the local context. This set the stage for the next part of the thesis, where the issue of economic development takings was looked at concretely, from the point of view of waterfall expropriation foor hydropower development in Norway. 

%Hence, to formulate a one-size-fits-all institution to replace eminent domain for economic development is unlikely to work. At least, such a solution would have to be far more flexible than the proposal made by Heller and Hills, which in the end offers owners very little in the way of participation in decision-making regarding economic development on their properties.


In such cases, it might well be that the balancing of different reasons for and against the taking has taken place prior to the decision to interfere with property. The plans for development themselves may well precede any specific property-oriented implementation steps, such as the use of eminent domain. It might even be that democratically accountable bodies responsible for land use planning have already concluded that some local community interests must give way to other interests.

In these cases, it might be tempting to argue that a narrow takings narrative is appropriate because it pertains only to the final implementation step, which is the only one that involves property rights. But this argument, I believe, rests on a flawed perception of what property is, and should be, in a democratic society. Invariably, property has to do with decision-making and power. If the decision-making process does not grant significant self-determination rights to affected property owners, a taking is already in progress. It might be justified, but it is still a taking. 

More worryingly, it is clear that this kind of taking carries with it a great potential for differential treatment, discrimination, and corruption. The traditional takings narrative does a good job of setting up a framework that makes it difficult to simply pay higher compensation to certain kinds of people, without offering any justification. But with respect to the aspects of taking not recognised, e.g., pertaining to what role the owner has during the planning stages, differences in treatment will not even be notices. But if property is owned by the right sorts of people, then invariably it {\it will} come with considerable decision-making power. 
}


\noo{
This has been a thesis in two parts that has addressed economic development takings from two distinct angles. In the first part, a theoretical discussion was provided, which started from the notion of property itself and gradually made its way towards the question of legitimacy of takings by exploring a broader meaning of property than that typically embraced by the law. The aim was to argue {\it why} property should be protected, while also providing a template for recognising and discussing special issues that arise for economic development takings. 

Following up on this, the thesis explored different approaches to legitimacy, culminating in a concrete proposal for a legitimacy test, inspired by the work of Kevin Gray. Furthermore, the thesis presented a template for formulating alternatives to outright expropriation, inspired by the work of Elinor Ostrom and others on local self-governance of common pool resources. The aim here was to suggest possible ways to restore legitimacy in cases when an either-or approach to legitimacy of property interference is not appropriate. Better than that it should be interference bottom-up than interference top-down, provided adequate safeguards are put in place to protect local minorities from abuse.

The discussion remained quite abstract in the first part of the thesis. By contrast, the second part of the thesis approached the question from the opposite angle, through an in-depth case study of takings for hydropower development in Norway. The thesis first presented and discussed the social, economic, and legal context of such takings, before proceeding to study practices and rules relating to expropriation of waterfalls in greater depth. This also provided an opportunity to apply the legitimacy test proposed in the first part of the thesis. The conclusion was that current expropriation practices in Norway fail in this regard, as the property rights of local people are very weakly protected in situations when large commercial companies wish to undertake hydropwoer development.

However, the thesis went on to observe that the land consolidation procedure represents a possible alternative to expropriation, one that is already being used extensively in Norway to facilitate owner-led hydropower development. Here the emphasis is on organising joint use of property, possible involving compulsion whereby owners must partake in development against their will. The procedure is judicial in nature, moreover, so also provides safeguards against abuse by local elites. Moreover, it is primarily a service to the owners and their properties, but is required to actively promote solutions that are in the public interest. After recent changes in the law, the scope of obligations that can be imposed on owners for the common good is also likely to increase, making the land consolidation alternative appear like a realistic option even in large-scale development situations that will necessarily also involve non-local actors. The thesis argued that land consolidation is a good example of the kind of institution that can function as an alternative to expropriation in hard cases. It has already proven itself in a setting of egalitarian property, in communities where property rights are held by local people. The possibility of employing it in situations where this is not the case remains more uncertain, but is an interesting prospect. Arguably, this would require including non-owners in the process as well, on the basis of connections to property that might not otherwise be recognised as property interests in the law. Specifically, it seems that one might want to rely on broader social function ideas of property in relation to consolidation, even if a more traditional account is maintained in other areas of the law. In relation to financial law or tax law, one might well wish to entertain an artificially narrow concept of property for efficiency reasons, even if a much broader concept is required in the context of compelled economic development. Moreover, the land consolidation procedure appears attractive because it fills a gap between planning and property, a gap that otherwise appears susceptible to infiltration by powerful commercial interests.

%The thesis made the case that these issues are important enough to suggest that economic development takings should be approached as a special category, also in the law.

%From this, the theoretical discussion continued by an exploration of different ways to approach the legitimacy question for such takings. 
}

%In this final section of my thesis, I would like to take a step back to briefly follow two broader threads that I believe run through my thesis. 
This has been a thesis in two parts, each of which have approached the issue of economic development takings. The first part took a theoretical approach, starting from the notion of property itself, to answer the question of {\it why} it should be protected. This, in turn, gave rise to a framework for assessing the legitimacy of economic development takings, and for formulating alternatives to it that could obviate the need for dispossessing current owners.

The second part of the thesis approached the issue of legitimacy concretely, by giving a case study of takings of waterfalls for hydropower development in Norway. The political, social and economic context was also analysed, leading to an application of the Gray test formulated in the first part of the thesis. Moreover, the case study considered the possibility of alternatives to expropriation, by assessing the Norwegian institution of land consolidation, which is now used extensively by local owners who wish to undertake hydropower development themselves.

% The current approach to takings for waterfalls in Norway were found wanting, with the current takings practices appearing to fail several, if not all, the points set out by the Gray test. However, the final chapter of the thesis studied a possible alternative to expropriation, which paradoxically is also actively used in the context of hydropower development in Norway. This framework, however, is so far used only when some of the local owners themselves wish to carry out development, and need to sort out their internal disagreements, possible even by compelling unwilling neighbours to join them in their endeavours. 

%The thesis argued that this alternative, although not necessarily applicable in other contexts, provides an interest

%More generally, and especially in its proposal for a possible solution, I hope the thesis has made a valuable contribution to the study of economic development takings. 

In the following, I offer a brief summary of the main points discussed in each chapter. While doing so, I also hope to shed further light on two broader threads that run through the work done in this thesis, pertaining to the nature of property and how to ensure that it functions as a force for good in democratic societies.

%To conclude the thesis, I will take a step back to consider two broader threads that I believe run through my work, pertaining to the nature of property and how to ensure that it functions as a force for good in democratic societies.

%The first concerns the many senses of taking that have been brought into focus throughout the analysis, while the second concerns ways in which the law can help to give back some of the legitimacy that is typically lost when eminent domain is used to facilitate economic development.

\section{Property theory and the social obligations of ownership}

In the first part of the thesis, I presented a number of ways of looking at private property. 

\subsubsection{Chapter 1}

To illustrate the subtleties that this approach helps shed light on, I considered a concrete example of a commercial scheme that looked like it might well result in compulsory acquisition of land, namely Donald Trump's controversial plans to develop a golf course on a site of special scientific interest close to Aberdeen, Scotland. In the end, the plans did {\it not} require takings, as Trump was able to make creative use of property rights he acquired voluntarily, against the complaints of his neighbours.

This turn of events did not make the example less relevant to this thesis. Rather, it served to highlight that the question of economic development takings is not a black-and-white balancing act between property privileges on one side and the good intentions of the regulatory state on the other. Specifically, the example of Trump coming to Scotland allowed me to emphasise the importance of context when assessing both the nature of property, the many ways of taking, and the meaning of protecting owners against predation.

The protection sought by those who opposed Trump's golf course did not target their entitlements as individuals. Rather, it targeted the community, as the owners felt it would be detrimental to the community, and to their lives, if Trump was allowed to redefine the social functions of local property. After Trump decided not to pursue compulsory purchase, protecting the property of these members of the community became a question of {\it restricting} the degree of dominion that Trump could exercise over his own property. Hence, under a conventional and overly simplistic way of looking at these matters, protecting property became tantamount to restricting its use, a seeming paradox.

To resolve this paradox, and to arrive at a better conceptual understanding of economic development takings, I looked to various theories of property. I noted that there are differences between civil law and common law theorising about property, but I concluded that these differences are not particularly relevant to the questions studied in this thesis. In particular, I observed that neither the bundle theory, dominant in the common law world, nor the dominion theory, taught to many civil law jurists, helped me clarify economic development takings as a category of legal thought.

I then went on to consider more sophisticated accounts of property, focusing on the social function theory, which emphasises how property structures, and is structured by, social and political relations within a society. 

I went on to argue that in the first instance, the social function theory should be understood as giving us {\it descriptive} insights into the workings of property and its role in the legal order. In this regard, I advanced a different stance than many property scholars, by arguing that we should aim to decouple descriptive insights from normative aspects of the theory, to allow the social function theory to serve as a common ground for further value-driven debate.

I then went on to clarify my own starting point for engaging in such debate, by expressing support for the human flourishing theory proposed by Alexander and Pe\~{n}alver. This theory is based on the premise that property {\it should} enable -- and even compel -- individuals and their communities to  participate in social and political processes. I argued that property's purpose in this regard is  fundamental to its proper role in a democratic society, as an anchor for participatory decision-making.  

Moreover, I noted that the human flourishing theory contains a further important insight, concerning the scope of the state's power to protect. In particular, the theory asks us to recognise that protecting property against interference that is harmful to human flourishing is a responsibility that the state has even in cases when the individual owners themselves neglect to defend their property, for instance as a result of financial incentives to remain idle. In other words, some functions of property are such that owners have an obligation to preserve them, while the state has a duty to protect them, potentially even against the will of the owners.

I then applied the theoretical framework developed in preceding sections to a preliminary investigation of economic development takings, to bring out the overarching question of legitimacy, which will occupy a central place in this thesis.

\subsubsection{Chapter 2}



\section{On the legitimacy of interference in property, and the social structures it helps sustain}

\section{Substance and procedure -- protecting the social functions of property by reviewing decision-making processes leading to interference} 

\section{Norwegian waterfalls -- properties, assets, and common goods}

\section{Property taken by default}

\section{Property given following participation}

\section{Conclusion}

%\section{Property Lost -- Taking and Excluding}\label{sec:7:1}

%According to some, the law does not like it when things get too academical; after all, the law is not in the business of settling philosophical debates, but rather used to resolve disputes.
%Rather, its task is to deliver effective management of disputes, involving concrete legal persons or governments.
%In fact, many leading legal theorists adopt a closely related position, when they argue that the law 

According to many legal scholars, the law is an instrument of order, grounded in social facts, not an arbiter of justice, grounded in moral theory.\footnote{This is a terse formulation of a position typically associated with legal positivism; it can be elaborated and restated in a multitude of ways, giving rise to many theoretical variations. The merits of the positivist account of the relationship between morality and law is a contested issue in the so-called Hart-Dworkin debate in legal philosophy, see \cite{hart12,dworkin86,shapiro07}.} Building on this, a pragmatist might be tempted to think that the law should be sceptical of embracing the complexities of equity when goals such as efficiency,  certainty, and control appear to be better served by simplification.\footnote{For a theory that emphasises pragmatism and chides moral theory, see \cite[109-110]{posner99} (``Holmes warned long ago of the pitfalls of misunderstanding law by taking its moral vocabulary too seriously. A big part of legal education consists of showing students how to skirt those pitfalls.'' (citations omitted)).} This way of thinking appears to have played a significant role in shaping the traditional approach to the legitimacy issue in the law of takings.

% would be in the interest of efficiency.
%According to positivist thought, the law is rightly sceptical of allowing things to get too theoretical; after all, the law is not in the business of settling ethical issues. Indeed, many believe that its main responsibility is to deliver effective management of disputes, involving concrete legal persons or governments.

It is clear that by focusing on individual owners and their losses, the law makes things easier for those called to administer it. Moreover, by assuming that all losses can be quantified in financial terms, the law's standard can be at least partly automated. The potentially broad question of legitimacy has been reduced to the question of when and how to award compensation, for which the ostensibly neutral idea of ``fair market value'' appears to provide a practical starting point, well suited for maintaining order in a capitalist society.\footnote{See, e.g., \cite[510-511]{acres79} (``In giving content to the just compensation requirement of the Fifth Amendment, this Court has sought to put the owner of condemned property `in as good a position pecuniarily as if his property had not been taken.' However, this principle of indemnity has not been given its full and literal force. Because of serious practical difficulties in assessing the worth an individual places on particular property at a given time, we have recognized the need for a relatively objective working rule. The Court therefore has employed the concept of fair market value to determine the condemnee's loss. Under this standard, the owner is entitled to receive `what a willing buyer would pay in cash to a willing seller' at the time of the taking.'' (citations omitted).)} A large chunk of the remaining work, in turn, can be delegated to the appraisers, allowing the courts to get on with other business. In addition to being effective, this comes with the added bonus of allowing the courts to distance themselves very clearly from the political overtones of the legitimacy question.

This approach to legitimacy is prevalent, but I believe it needs to be rejected. The reason, as argued in the first part of this thesis, is that property itself cannot be drawn up as narrowly as the compensation approach presupposes.\footnote{See the discussion in chapter \ref{chap:2}, section \ref{sec:2:4}.} The legitimacy of property interference cannot be understood in mechanical terms, as a matter for the appraises, regardless of how much weight we want to place on the value of efficiency and the ideal of deference to political decision-making. Specifically, unless the issue of legitimacy is recognised as being far more complex than pertaining solely to the question of when and how to calculate market values, there will be a significant mismatch between what property is and what the law pretends it to be.\footnote{As contended in Part I of this thesis, this much appears to be a descriptive insight, which does not seem to depend on one's moral philosophy, see the discussion in chapter \ref{chap:2}, section \ref{sec:2:4:3}.}
\noo{In a setting where takings are rare and happen only in extraordinary situations, such a mismatch might be tolerable. Arguably, the strong commitment to the sanctity of property by early writers such as Blackstone, apparently conflicting with historical records, could be sustained because takings were exceptional, ordered only after careful deliberation by a legislative body with an authority over property to match or exceed even that of an owner.\footnote{See chapter \ref{chap:3}, section \ref{sec:3:2}.}

Such a narrative is no longer plausible in a world where the state has expanded its activities so much that interference in private property, rather than being exceptional, have become a normal occurrence. This also means that the law cannot pretend not to interfere also with the complexities of property as a social phenomenon and an anchor for participation and human flourishing. This broader narrative of property must then also be considered in the law, especially in the law of takings.}
This is harmful at a structural level, especially if the idea of property as a fundamental right is to have a future. If the law continues to insist that property is nothing more than a form of entitlement protection, there might even be a case to be made that the notion should just be done away with in its entirety.\footnote{See generally \cite{grey80}.}

In the first part of this thesis, I presented a theory of property which suggests that this would be a tragedy, particularly for marginalised groups who are in need of protection against economic and social elites. Importantly, while international law and human rights conventions offer important clarifications and protections at the level of principles, what property provides is an imperfect, yet very powerful, framework for implementing such principles at the local level. This is property's promise, which it can only keep if it is recognised as having social functions going well beyond the protection of individual entitlements.

Arguably, principles of human rights should even be recognised as inhering in property as such, not only as mediated by the power of states.\footnote{See chapter \ref{chap:2}, section \ref{sec:2:5:1}.} A worry often voiced as a counterargument for direct horizontal application of human rights is that it can serve as an excuse for the state to do nothing.\footnote{See \cite[110]{manisuli07}.} For this reason, it might well be worth emphasising that if the owners fail to deliver on basic rights, the state is still responsible. However, the converse is equally true: if states fail, then owners still have obligations.

If there is no distinction between owners and states, by contrast, failure in one is also failure in the other. This seems dangerous, particularly if proprietary power is exercised only by a small group of people, regardless of whether they are commercial leaders or powerful government officials. Property should therefore be widely distributed among the population, to be rendered as a provider of basic rights and an anchor of democracy.

The first chapter of this thesis explored this idea in depth and argued that a broader notion of property needs to be acknowledged also by the law, particularly in the law of takings. If property serves a broad social function by sustaining a community and aiding in the delivery of basic rights to all its members, transferring that property to a non-local commercial owner is not merely a highly dubious redistribution of entitlements. In the end, it will also be the destruction of property, as it undermines property's most significant functions in relation to the overarching goal of human flourishing. In such cases, therefore, property is not only taken, it is lost.

\noo{
t is worth pausing to recognise that the bundle of rights theory did us a favour in this regard, in that it directed our attention at the multifaceted nature of property. However, to make progress, it was necessary to further unpack the property bundle, to get at the substantive content of property in life: social functions as opposed to legal abstractions.

\noo{ Moreover, I argued that while private property might often be found wanting, it remains a potentially powerful force for good in the world. As discussed in chapter 1, its roles as a building block of democracy and a protector of communities is particularly important in this regard. In addition, I discussed the importance of social obligations arising from property, and how they can potentially function as a guarantee that the basis rights of non-owners will be delivered at the local level.

If these aspects are recognised and embraced as a crucial part of the concept of property in the law, it should hopefully go some way towards restoring confidence that property is neither theft nor fraud, but a promise to work hard for a better future for all. By contrast, the idea of property as a financial entitlement do not appear to offer any such relief. If anything, dismissive attitudes to property will take their fuel from the idea of property as entitlement; entitlements, after all, can often appear undeserved, especially when they are not checked by corresponding obligations. 
}
%In this way, a threat emerges to the stability of the concept of property itself, as a legitimate part of the social and political order.
%In reality, of course, owning property has nothing to do with what one deserves, but rather what task one has been allotted on this earth, to pursue in keeping with one's beliefs and convictions.

%In cases involving regulation of property use, it might still be possible to keep this aspect away from undermining the concept of property as such. At least, it might ensure that only those well versed in the technical details of the law are able to recognise property for the ``phantom'' that it appears to be.

%However, when property is taken outright, even this containment strategy is bound to falter. This is particularly clear if taking become increasingly prevalent not only in situations of pressing public need, but also as a means for companies to turn a profit. In such a setting, even the most naive observer would surely be tempted to think that property as a concept must be altogether rather vacuous.  


%The impression that private property, in the end, is nothing but a shorthand to describe a special class of liability rules, leaves property open to further attack. Indeed, if property and ownership has only such a thin content, why worry about interfering with it in the public interest? At this point, however, it seems prudent to take a step back, to reconsider the origin of the feeling that property is no more than theft, or no better than fraud. Specifically, it might be appropriate to note that unlike property as a concept, the act of taking it without its owner's consent is quite likely to involve both theft or fraud as a matter of fact, not merely a manner of speech.

%In the second part of this thesis, I have explored this in further depth, by analysing the law and practices relating to the expropriation of waterfalls in Norway.

%However, as I noted in the first chapter of this thesis, property itself is highly multifaceted, serving a range of social and individual functions. 

The first chapter set out to do this, in order to get at the multitude of different ways in which a taking can impact on society and its members. The economic consequences of a taking might be the most easily recognisable, particularly in the economic development cases. But as I have argued in this thesis, other consequences can be just as important, particularly those pertaining to property as a building block of communities. If jointly owned property is taken from a community, with full compensation paid to all individual owners, the community suffers a distinct uncompensated loss, namely the loss of future self-governance opportunities. %In light of work done by Elinor Ostrom and others on the strength of self-governance solutions for sustainable resource management, the law of takings should arguably offer protection directed specifically at the potential loss of community.

%In the traditional narrative on takings, social and political effects are typically only recognised on one side of the takings equation, namely the side of the taker, particularly the public interest. This has also influenced the debate on economic development takings. In order to make sense of the broader sense of unfairness often associated with such takings, critics tend to focus on the taker rather than the owner, by questioning the legitimacy of the motives behind the taking.

%However, this might be tantamount to shifting a variable to the wrong side of the takings equation. In particular, the feeling of unfairness associated with economic development takings clearly arise from a sense in which the owners are victims of an abuse of power. So why shift attention to the taker? 

%Perhaps it is tempting to do so simply because the sense of unfairness at work here pertains to a broader notion of justice than that normally associated with property interests. If so, the entire narrative points to a shortcoming of the liberal idea of property. If even property's staunchest defenders must turn to notions of ``public interest'' (and the lack thereof), then why do we need property as a concept at all? Why not simply say that a licence to undertake economic development should not be granted unless all affected parties agree, or the public interest is sufficiently strong to go ahead against some of their wishes? What makes property special in this picture, if all that is at stake is the strength of the public interest used to justify imposing the state's will on private individuals?

%Clearly, the gaping hole in the opposition to economic development takings in the US has been a {\it positive} account of 

%This part of the thesis focused on coming up with an answer as to why property is worthy of protection in the first place, in cases where economic rationality appears to dictate that it should be put to more profitable uses. If there is a reason to resist this, it must be because there is something valuable in property that the law should protect, irrespective of the current owner's financial entitlements.

Moreover, the thesis argued that the dynamics of power in takings cases need to receive more attention: the practice of economic development takings can result in local communities being deprived of highly valuable political capital in order for politically powerful commercial interests to make a profit. I noted that this was also the main concern raised by Justice O'Connor in her {\it Kelo} dissent.\footnote{\cite{kelo05}.} In this way, the social function view is arguably also implicit in one of the most forceful voices that have spoken out against economic development takings on the basis of constitutional property law.

However, the thesis also noted a weakness with the typical approach to legitimacy in the US, via the public use requirement. Specifically, this requirement does not appear to get us very far towards a justiciable restriction on the takings power along the lines of reasoning adopted by Justice O'Connor. Unlike the majority in {\it Kelo}, building on recent precedent, and Justice Thomas, building on the original meaning of the public use restriction, Justice O'Connor's more institutionally oriented reasons for rejecting the taking appeared to lack a firm basis in law.

To address this, I pointed to recent developments at the ECtHR, where an institutional perspective on fairness appears to be developing, which might be more likely to embrace the social function perspective on property as a basis that can support a justiciable restriction on the states' takings power. At the same time, the position of the Court in Strasbourg might be conducive to an approach that can adopt broad scrutiny in controversial cases without becoming too entangled with the politics of those cases at the state level. Specifically, the value of deference could be given a firmer expression as a norm that compels recognition of diversity and local democracy, not a norm that calls for passivity or loyalty to governments in politically sensitive situations.

Following up on this, the thesis went on to formulate a legitimacy test based on a set of conditions formulated by Kevin Gray. Three additional points were added to this list, emphasising the regulatory context, the position of non-owners, and the broader issue of democratic merit. 
}

Arguably, takings of this kind are currently being carried out in Norway, to the benefit of large energy companies. Waterfalls are still nominally considered private property, belonging to members of the rural communities in which the water resources are found. However, as shown in Part II of this thesis, the practice of taking waterfalls from local communities has become so insensitive to the plight of owners that the law itself is ambiguous about whether private ownership of water resources has much content at all.

%It is important to note that there is nothing inevitable about this state of the law. Indeed, the historical context shows that in Norway, where water is anything but scarce and most rivers are non-navigable, property rights in waterfalls (as opposed to water as a substance), was long recognised as on par with property rights in land. Expropriation to pursue hydropower development was not permitted under any circumstance until the early 20th century. 

But this does not mean that proprietary power is no longer exercised over the power of water. Quite the contrary. The commercial companies that acquire waterfalls are quick to turn them into commodities whose primary purpose is to turn a profit. At the same time, water resources have now been encapsulated in development licenses; today, licenses from the government, not waterfalls as such, are the key assets in the hydropower production sector. The nature of property has thus transformed; ownership of hydropower has become a legal fiction, arising from a bundle of papers with numbers on them, not from physical and social proximity to the underlying resource.

The result is that the notion of property backing up this regime is now so thin that it arguably cannot be distinguished at all from the political and economic power that backs it up. As such, it is also no wonder that narratives of egalitarian property give way to narratives based on thinking about water resources as though they belong to the ``public''.\footnote{See the \indexonly{ica17}\dni\cite[1]{ica17}.} This is a politically defensible way of talking, to mask the reality that water resources are managed according to the will of dominating market players and powerful interest groups, both striving to protect their dominion over the fruits of the land.\footnote{I mention that state-owned companies are currently extending their dominion elsewhere as well, with potentially deleterious consequences. In Nepal, for instance, Norwegian hydropower companies generate large profits from their aid-funded cooperation with the Nepali government, apparently at the expense of local populations and the rights of the poor. See \cite{gaarder15} (report from one of the largest environmental organisations in Norway, discussing how the Nepali government entered into an agreement where the price of electricity would be set at an extortionate level, denominated in US dollars, and adjusted upwards proportionally to US consumer prices; apparently, the project involves a transfer of money from Nepal to Norway that far exceeds the flow of money going the other way); \cite[644]{peris12} (mentioning the presence of Norwegian actors, discussing the marginalisation of local people, and arguing that conventions on indigenous rights are unable to deliver social justice due to the ``democratic deficit'' of decision-making regarding hydropower development in Nepal).}

Due to this dynamic, the case of Norwegian waterfalls appears to be an example of how illegitimate takings can do more harm than to deprive owners of valued resources. Specifically, the case study shows how a lack of legitimacy can effectively deprive property of its meaning, as the sticks of the 
bundle are bent to suit the interests of the commercial and political elites. Such a system might well give us the impression that property is little more than theft, maintained in the law only as a fraud.

\noo{ However, given the historical context Perhaps, then, the nature of property itself has changed, so that there is nothing left except those financial entitlements that Norwegian expropriation law recognised. If so, the change has not come about by any legislative move, nor has it been preceded by any kind of debate. It has simply emerged, gradually and unplanned, as a result of sector-based regulation and administrative practices. The process, therefore, meets neither the requirements of land reform or expropriation. It is an unacknowledged process about which the law in Norway has had nothing much to say at all, for which silence still persists. 

%Property can be an elusive concept, especially to property lawyers. Indeed, in the law of property, the word itself typically only functions as a metaphor -- an imprecise shorthand that refers to a complex and diverse web of doctrines, rules, and practices, each pertaining to different ``sticks'' in a ``bundle'' of rights. Indeed, this bundle perspective dominates legal scholarship, especially in the common law world. Some even go as far as to argue that words such as ``property'' and ``ownership'' should be removed from the legal vocabulary altogether. 

%So is property as a unifying concept lost to the law? It certainly seems hard to pin it down. In the words of Kevin Gray, when a close scrutiny of property law gets under way, property itself seems like it ``vanishes into thin air''.\footnote{See \cite[306-307]{gray91}.} %Indeed, one may argue that different ideas of property, practical and theoretical, are behind most, if not all, the major conflicts and confrontations that have shaped the society in which we live.
%Responding to this, some prominent philosophers have taken the view that property is not a concept suitable for philosophical study at all.
%According to some philosophers, property as a concept is a lost cause, not suitable for conceptual analysis at all. Instead, these scholars have suggested that property should be taken as a pragmatic and contingent derivative of other notions, such as the social order, or, on a normative account, by regarding it as an expedient of {\it justice}.
%Moreover, legal scholars are usually content with theories of property that remain largely descriptive, settling for the more modest aim of exploring how best to think of property given the prevailing legal order, rather than trying to come up with theories to explicate its nature as a pre-legal concept. Indeed, even legal philosophers are sometimes found doubting that there even is such a thing as property.
%Arguably, however, property never truly disappears. Indeed, there is empirical evidence to suggest that humans come equipped with a {\it primitive} concept of property, one which pre-exists any particular arrangements used to distribute it or mould it as a legal category.\footnote{See\cite{stake06}.} Perhaps most notably, humans, along with a seemingly select group of other animals, appear to have an innate ability to recognise {\it thievery}, the taking of property (not necessarily one's own) by someone who is not entitled to do so.\footnote{See \cite[11-13]{brosnan11}.}

%Taken in this light, Proudhon's famous dictum ``property is theft'', might be more than a seemingly contradictory comment on the origins of inequality. It might point to a deeply rooted aspect of property itself, namely its role as an anchor for the distinction between legitimate and illegitimate acts of taking.

%But what is a taking, and when is it legitimate? In this thesis, I will aim to make a contribution to this question. I will study takings of a special kind, namely those that are implemented, or at least formally sanctioned, by a government. In legal language, especially in the US, such acts of government takings are often referred to as takings {\it simpliciter}, while talk of other kinds of ``takings'' require further qualification, e.g., in case of ``takings'' based on contract, tax or occupation. 

%The US terminology brings the issue of legitimacy to the forefront in an illustrative manner. We are reminded, in particular, that under the rule of law, taking is not the same as theft. Rather, the default assumption is that the takings that take place under the rule of law are legitimate. If they are not, we may call them by a different name, but not before. At the same time, it falls to the legal order to spell out in further detail what restrictions may be placed on the power to take. 

%Indeed, restrictions appear implicit in the very notion of taking. The idea that someone might find occasion to resist an act of taking, and may or may not have good grounds for doing so, appears fundamental to our pre-legal intuitions. But how should we approach the question of legitimacy of takings from the point of view of legal reasoning, and what conceptual categories can we benefit from when doing so? This is the key question that is addressed in this thesis.
}

\section{Property Regained -- Giving and Participating}\label{sec:7:2}

The converse of a taking is a {\it giving}. In the US, this term is sometimes used to refer to situations when private property owners benefit from state actions involving property.\footnote{See generally \cite{bell01}.} For instance, it might be characterised as a giving if the state allows someone to purchase property cheaply, or if regulation makes some properties appreciate in value. Arguably, and analogously to the case of takings, there is a case to be made that the state should sometimes charge people for disproportionate givings.\footnote{See \cite[590-604]{bell01}.}

The issue of when this is appropriate, if at all, will not be discussed here. Instead, I wish to direct attention at the terminology itself, and its subtle conceptual commitment to a top-down way of thinking about both takings and givings. Indeed, consider what happens if we turn the terminology on its head. This is not an implausible conceptual shift. After all, if the state takes property, the current owners will have to give it up.

The owners' act of giving, however, is rarely given much recognition or attention when we approach takings. Plainly, the owners' active participation as a giver -- not merely an injured party -- is typically considered irrelevant. Why is that? The obvious answer is that since the giving takes place under compulsion, it does not express any intention to give. However, there are many situations in life where actions are compelled, but where the person taking that action still gets some credit for it.\footnote{Paying taxes, for instance, is typically associated with being a good citizen.} Moreover, the owners can clearly decide to be more or less cooperative when faced with the government's wishes for their property. 

By shifting attention towards the choices that the owners have in this regard, we can hope to find a path towards increased legitimacy by recognising the owners as active participants. Indeed, even a purely symbolic recognition of the owners' role and the importance of their choices in dealing with a takings request, can serve to enhance subjective legitimacy. However, quite apart from recognising the constructive role that owners can play in the existing system, thinking about takings as givings also suggests the possibility that owners should be granted more choices and asked to take part more actively in the proceedings.

%The owners make an active contribution to the public purpose, instead of being regarded as obstacles to it. This, in turn, can become a starting point for coming up with arrangements where the owners are permitted to take up a more lasting interest also in the new use of the property that the public desires. 

In cases involving economic development, this way of thinking seems particularly appropriate. As they contribute to the project by giving their property, it seems only fair that the owners should have a stake also in the planning and the continued use of their property for economic gain. For instance, it might be appropriate to offer owners shares in the development company, and to make sure that the property is taken under a leasehold rather than by a full transfer of title. Better yet, one could allow the owners themselves to deliberate on how they wish to honour their commitment to the public, to formulate their own plans and implement their own solutions, based on continued interaction both among themselves and with representatives from the relevant bodies of government.

Interestingly, as shown in chapter \ref{chap:5} of this thesis, such an abstract and highly idealistic idea is in fact (partly) implemented through the system of land consolidation found in Norway. As noted, this institution is even equipped with the power to compel owners to come together and participate in specific endeavours. If a proposed development is judged as being beneficial to the properties in question, the owners may be stripped of their holdout power, not only as individuals but even as a group, without any property having to be condemned. The owners will retain their ownership even after their properties have been put to new and more productive uses.

In light of this, land consolidation can indeed replace expropriation, provided our understanding of what property is, and should be, is broad enough to say that the purpose we wish to pursue is also in the interest of the properties involved. This limitation, expressly encoded in the law of consolidation in Norway, is interesting also on the theoretical level, because it makes the {\it purpose} of property directly relevant to determining the extent of the government's power to interfere with it. At the same time,  in the context of economic development, the restriction is rarely going to prevent consolidation from going ahead. If development is both economically beneficial and represents a sustainable use of resources, the land consolidation courts should have no difficulty justifying consolidation measures on the basis that development is also in the interest of the properties involved. At the same time, the context of land consolidation, with its emphasis on problem solving and owner participation, means that the process will usually be far more inclusive towards owners than traditional expropriation proceedings. 

If property is in the hands of the few, while many property dependants are without formal ownership rights, the consolidation model might be less appropriate. However, in these cases, it might be possible to adapt it by allowing a larger group of local people to partake in the proceedings. In complex cases, most of them might find themselves unable to participate effectively in the process. However, a consolidation approach will still serve an important function in that it gives marginalised groups new opportunities for engaging with the democratic process.

%Under consolidation, it is truly more appropriate to speak of givings than to speak of takings. Moreover, what is given in these cases is not normally the property as such, but the right to determine how it is to be used, with the public having to turn to the current owners for help in realising the public purpose. It would still be possible to expropriate, or to rely on a mix between expropriation and consolidation. In Norway, the relationship between the two remains to be clarified in the law. A general rule of consolidation first, expropriation second, 
%should arguably be introduced in order to better promote the consolidation alternative.

%This thesis went on to explore how this alternative works in practice for hydropower development, noting that the current system works best when the owners are in reasonable agreement with each other and society about how development should proceed. Hence, the broader applicability of the idea can not be taken for granted. However, it seems to have great potential for being fine-tuned, to make it more effective in situations involving deeper disagreements and conflicting interests. There is a danger here, however, namely that increasing the power of the institution will also undermine its role as a democracy-on-demand for owners and the community. However, if safeguards can be put in place, the land consolidation model might well be a highly attractive alternative to expropriation, also in cases involving deeper conflicts about property uses.

The system can also serve an empowering and educational role. Local elites might often dominate the process in practice, but the fact that the procedure is judicial in nature means that abuses can be curbed while the disadvantaged may be granted access to more fruitful forms of citizenship. At least, access to the decision-making process should become easier for those who wish to challenge the local leadership.

More generally, the flexible, issue-focused, and transient nature of a consolidation court might have significant advantages. Specifically, concentrations of power and participatory fatigue are unlikely to become entrenched, given the context of decision-making; the group of people involved, the area covered, and the agenda to be deliberated on, will all change from situation to situation. This keeps participants engaged while diffusing power, thereby minimising the risk of elite tyranny and expert rule, otherwise easily enabled by rigid institutions and apathetic majorities.\footnote{This also relates closely to the worry that ``rational ignorance'' can thwart attempts at meaningful eminent domain reform, see chapter \ref{chap:4}, section \ref{sec:3:3:2}. See also \cite{somin09}.}

Further exploration of the transferability of the consolidation framework to other contexts will have to be left for future work. What this thesis has hopefully shown is that fully fledged institutional alternatives to expropriation for economic development already exist and that they can work in practice. Hence, my research can hopefully inspire more work in this direction both at the theoretical and empirical level. In my opinion, the ideas underlying the land consolidation system found in Norway are worth exploring further, also in other jurisdictions that rely on expropriation as a tool to facilitate economic development. In this context, it would be a great victory for both property and equity if public and private interests could come together in a way that will give rise to givings in the future, clearly distinguished from the takings of the past.
% \chapter{Sixth Chapter Title}


\section{First Section}
\subsection{First Subsection}
Here is some text. 

\subsection{Second Subsection}

\section{Conclusion}
Here is some more text. 		
% \chapter{Seventh Chapter Title}


\section{First Section}
\subsection{First Subsection}
Here is some text. 

\subsection{Second Subsection}

\section{Conclusion}
Here is some more text. 
% \chapter{Eighth Chapter Title}


\section{First Section}
\subsection{First Subsection}
Here is some text. 

\subsection{Second Subsection}

\section{Conclusion}
Here is some more text. 

%\bibliographystyle{Classes/CUEDbiblio}
%\bibliographystyle{oxford_en}
%\bibliographystyle{Classes/jmb} % bibliography style
%\renewcommand{\bibname}{References} % changes default name Bibliography to References
%\addcontentsline{toc}{chapter}{Bibliography} %adds References to contents page
%\bibliographystyle{Classes/jmb} % bibliography style

%\printindex[echrcases]
\nocite{*}

%If you want to input ship names, put them HERE (after \nocite, before %bibliography.tex

\chapter*{Bibliography}
\addcontentsline{toc}{chapter}{Bibliography}

% This filter is used to identify works which are either of the inbook or incollection type
\defbibfilter{inbookorincoll}{%
  \( \type{inbook} \or \type{incollection} \)}

% Define a bibheading that prints a subheading, with appropriate addition to table of contents, and sets right and left marks accordingly
\defbibheading{mysubbibintoc}{%
  \section*{#1}%
  \addcontentsline{toc}{section}{#1}%
  \markboth{BIBLIOGRAPHY -- \MakeUppercase{#1}}{BIBLIOGRAPHY -- \MakeUppercase{#1}}}

% BOOKS

\printbibliography[title={Books}, type=book, heading=mysubbibintoc, category = cited]

% WORKS IN COLLECTIONS

\printbibliography[title={Contributions to Collections}, filter=inbookorincoll, heading=mysubbibintoc, category = cited]

% ARTICLES IN JOURNALS

\printbibliography[title={Articles}, type=article, heading=mysubbibintoc, category = cited]

% ALL OTHER WORKS INCLUDING UNPUBLISHED MATERIAL

\printbibliography[title={Other Works}, nottype=book, nottype=jurisdiction, nottype=legal, nottype=legislation, nottype=article, nottype=inbook, nottype=incollection, heading=mysubbibintoc, category = cited]
. I have left one example, commented out, which should work (assuming you have the case, etc.).

%\index[casesgb]{Achilleas, The@\emph{Achilleas,} The|see{Transfield Shipping Inc v Mercator Shipping Inc}}


%bibliography.tex

\chapter*{Bibliography}
\addcontentsline{toc}{chapter}{Bibliography}

% This filter is used to identify works which are either of the inbook or incollection type
\defbibfilter{inbookorincoll}{%
  \( \type{inbook} \or \type{incollection} \)}

% Define a bibheading that prints a subheading, with appropriate addition to table of contents, and sets right and left marks accordingly
\defbibheading{mysubbibintoc}{%
  \section*{#1}%
  \addcontentsline{toc}{section}{#1}%
  \markboth{BIBLIOGRAPHY -- \MakeUppercase{#1}}{BIBLIOGRAPHY -- \MakeUppercase{#1}}}

% BOOKS

\printbibliography[title={Books}, type=book, heading=mysubbibintoc, category = cited]

% WORKS IN COLLECTIONS

\printbibliography[title={Contributions to Collections}, filter=inbookorincoll, heading=mysubbibintoc, category = cited]

% ARTICLES IN JOURNALS

\printbibliography[title={Articles}, type=article, heading=mysubbibintoc, category = cited]

% ALL OTHER WORKS INCLUDING UNPUBLISHED MATERIAL

\printbibliography[title={Other Works}, nottype=book, nottype=jurisdiction, nottype=legal, nottype=legislation, nottype=article, nottype=inbook, nottype=incollection, heading=mysubbibintoc, category = cited]


% this section includes various indexes/tables of cases and legislation

\chapter*{Cases Cited}
\addcontentsline{toc}{chapter}{Cases Cited}
\markboth{CASES CITED}{CASES CITED}

\printindexearly[casesgb]% ENGLAND & WALES
\printindexearly[casessc]% SCOTLAND (GB too, of course, but ...)
\printindexearly[casesau]% AUSTRALIA
\printindexearly[casesnz]% NEW ZEALAND
\printindexearly[casesca]% CANADA
\printindexearly[casesus]% UNITED STATES
\printindexearly[casesother]% OTHERS

\chapter*{Legislation Cited}
\addcontentsline{toc}{chapter}{Legislation Cited}
\markboth{LEGISLATION CITED}{LEGISLATION CITED}

\printindexearly[legis]% ALL LEGISLATION

\end{document}
