%\documentclass[12pt,a4paper]{memoir} % for a long document
\documentclass[12pt,a4paper]{article} % for a short document

\usepackage[utf8]{inputenc} % set input encoding to utf8
\usepackage[style = oscola]{biblatex}

% Don't forget to read the Memoir manual: memman.pdf

\title{{\huge Summary and progress report:} \\ ``On the Legitimacy of Economic Takings: A case study of Norwegian waterfalls taken for hydropower development''}
\author{Sjur K Dyrkolbotn}
\date{} % Delete this line to display the current date

\addbibresource{thesis.bib}

%%% BEGIN DOCUMENT
\begin{document}

\maketitle

\section*{Summary of Contribution}

My thesis will contribute to research on the legitimacy of economic development takings, an area at the crossroads between property law, planning law, constitutional and human rights law.\footnote{See generally, \cite[Chapter 12]{epstein85};  \cite{merrill86,malloy08}.} It is an area of increasing interest to scholars, particularly in the US after the controversial decision in {\it Kelo v City of New London}.\footcite{kelo05} In this case, the Supreme Court held, in a 5-4 decision, that a taking for economic development should be upheld, despite objections that this would be suspiciously favorable to the drug company Pfizer. In particular, Suzanne Kelo, who stood to lose her home, argued forcefully that the taking was not for public use, as required by the Fifth Amendment of the US constitution. While the majority of the Supreme Court disagreed, the case was met with outrage in the US public and provoked intense political and academic debate.\footnote{According to surveys, the decision in {\it Kelo} was opposed by 80-90 \% of the US population, and more than 40 states have passed post-{\it Kelo} legislation to restrict the use of takings for economic development, see \cite{somin09}.}

So far, economic development takings have not become as controversial in Europe, but there have been cases where the issue has come up, in several different jurisdictions.\footnote{For instance, in the UK, Ireland and Germany, as well as in Norway and Sweden. See \cite[466-483]{walt11}; \cite{stenseth10}.} The European Convention of Human Rights (ECHR) contains a property clause in Article 1 of Protocol No. 1 (P1(1)), but the legitimacy of economic development takings has not yet been discussed in case law from the European Court of Human Rights (ECtHR). However, it is interesting to analyse cases like {\it Kelo} against P1(1), particularly since the ECtHR has developed a doctrine that focuses on ``proportionality'' and ``fairness'' rather than the purpose of interference.\footnote{See generally, \cite[Chapter 5]{allen05}. Also note that there are ``jurisprudential developments in the direction of a stronger protection under Article 1 of Protocol No. 1'', see \cite[135]{lindheim12}.}

The fundamental question raised by economic development takings can be formulated independently of specific property clauses as follows: when, if ever, is it permissible for governments to order compulsory transfer of property rights from citizens to for-profit legal persons in order to facilitate economic development? In my thesis, I shed new light on this question by offering a comparative assessment, as well as by analysing it against a case study I have conducted on waterfalls taken for hydropower development in Norway. The three main contributions of my research are summarized in the following.

First, I offer a systematic overview of recent developments in Norwegian law and administrative practices relating to the expropriation of waterfalls for hydropower development. Since the liberalization of the energy sector in the early nineties, private owners of waterfalls, usually farmers and other members of the rural community, have engaged in small-scale hydropower development on a scale not previously seen in Norway. At the same time, established energy companies continue to benefit from expropriation, now often in competition with younger companies that want to develop hydropower in cooperation with private owners. This has led to many legal disputes and a dramatic change in the way compensation is calculated in cases when waterfalls are expropriated. At the same time, the principle that energy companies may take waterfalls compulsorily for commercial gain is upheld, now often to the direct detriment of owner-led projects. My thesis will provide the first detailed presentation of this development and its legal and administrative ramifications.

Second, I offer a comparative perspective on economic development takings, by describing their status under Norwegian expropriation law and comparing this to the status of such takings in Europe and in the US. This includes an assessment of such takings against the ECHR. As part of this work, I also provide a comparative analysis of so-called {\it elimination} principles, which can be found in both common law and civil law jurisdictions.\footnote{The most important such principle developed in common law is typically referred to as the {\it Pointe Gourde} rule, after \cite{gourde47}.} Such principles raise special questions in economic development cases, since they can give the taker an additional financial incentive to rely on compulsion rather than a voluntary transaction.\footnote{In effect, they allow the developer to refuse the owner a share in the profit arising from development.} To further analyse important points addressed in the comparative study, I will draw on the data I have assembled concerning waterfalls. I will show, in particular, how many of the challenges that have been raised in the literature are currently addressed concretely in Norwegian case law on hydropower development. My work is the first to make this connection and to study Norwegian waterfalls in this light.

Third, I discuss a branch of recent scholarship that focuses on the design of legitimacy-enhancing institutions that can supplement or replace standard eminent domain procedures in economic development cases.\footnote{I focus particularly on the proposals from \cite{lehavi07,heller08}.} After describing and assessing this research, I go on to consider two institutions that currently operate in Norway. Both are judicial in nature, and both have played an important role in empowering waterfall owners. The first is the system of {\it appraisal courts}, used to award compensation in takings cases.\footnote{Act No 1 of 1 June 1917 relating to Appraisal and Expropriation Disputes.}  These courts are characterized by the special procedural rules they observe, involving the participation of lay people on the bench, alongside expert appraisers and legally educated judges. The recent changes to how compensation is awarded in waterfalls cases originate from new practices adopted by the appraisal courts, practices that the Supreme Court has later (partly) upheld. Hence, I argue that the case of waterfalls witnesses to the potential of appraisal courts as legitimacy-enhancing institutions. I discuss them at some length, arguing that they establish an interesting balance of power over compensation determination between legal professionals, expert appraisers and lay people. At the same time, I note that their effect on legitimacy is limited by their narrow focus on compensation; they do not, in particular, directly empower property owners in decisionmaking processes related to the taking decision.

However, a second class of judicial bodies that can be used to this effect are the {\it land consolidation courts}.\footnote{Act No 77 of 21 December 1979 relating to Land Consolidation etc.} These courts are also governed by special procedural rules, and they sit with specially educated judges, not judges with an ordinary legal education. Moreover, they have the power to organize development on behalf of property owners, if necessary by force. However, they are constrained by the fundamental principle that a development project can not be ordered unless it is regarded as beneficial to all the properties involved.\footnote{Section 3(a) of the Land Consolidation Act 1979.} To achieve this, the court may reorganize and restructure rights and boundaries in real property, and it may establish corporate structures to limit the financial risks involved. However, it will rarely engage in such measures unless it enjoys wide support among the property owners. In particular, the court is largely meant to be providing a service to the property owners, providing them with an arena where they can deliberate about how to manage their land and resolve conflicts about land use and ownership.

However, since the land consolidation courts can compel owners to partake in economic development, it can also often function as an alternative to eminent domain. Unsurprisingly, the legitimacy of development facilitated by land consolidation is often much greater among owners and their local community than development facilitated by the use of expropriation. This is clearly felt in waterfall cases, where land consolidation is increasingly being used instead of expropriation, particularly for owner-led projects (especially when only some of the owners support such development). Land consolidation was originally an instrument for rational owner-led management of jointly owned agricultural land, but the Norwegian government is currently broadening the scope of the procedure, to include urban development projects.\footcite[788-790]{stenseth10} In my thesis, I argue that land consolidation is a particularly interesting institution to look at as an alternative to eminent domain in economic development cases. Doing so ties land consolidation directly to an interesting recent proposal in the US literature, and I offer a comparative analysis.\footcite{heller08} My thesis is the first to make this link and to study land consolidation in such a light.

Overall, I think my thesis will make a valuable contribution to the study of economic development takings. I contribute to the theoretical foundations for such research, by approaching the topic comparatively and from the point of view of European law. Following up on this, I offer a concrete assessment of legitimacy based on as-of-yet unexplored data from Norway.

\section*{Structure of thesis and plan for completion}

\section*{Chapter 1: Setting the stage}\label{chap:1}

In this chapter, I introduce the subject, placing it in the theoretical landscape as a branch of constitutional and human rights law. 
I argue for a broad conception of property, drawing on the social-obligation norm explored in recent scholarship.\footnote{See generally, \cite{alexander06,dagan07,underkuffer07}.} Under this conception, property entails both rights and obligations, and is inexorably linked to the owner's  membership in a community. I contend that protection of property so conceived is crucial to the well-being of both individuals, local communities, and society as a whole. Then I comment briefly on some global threats to property, such as neocolonialism and ``land-grabbing'', both arguably made worse by the lack of legal protection for, and promotion of, sustainable property structures.\footcite{narula13} I note that while property reform can be a crucial component in facilitating sustainable development, the need for principles protecting vulnerable groups can often render such reforms ineffective or counterproductive.\footcite{smith03}

I then go on to argue that with a broad conception of property, it becomes natural to consider property as part of a broader discourse on justice. Upholding the rights and obligations associated with property can then be seen as an element in ensuring fairness, not as a conservative dogma standing in the way of reform. Following up on this conceptual starting point, I introduce the debate on  economic development takings in the US and Europe. This debate has flourished, especially in the US, and I suggest that this might be related to a shift in perceptions of property, along the theoretical lines laid out in the first part of the chapter. To substantiate this I present the groundbreaking case of {\it Kelo}, where the US Supreme Court dealt with the issue of economic development takings in unprecedented depth.\footcite{kelo05}

The majority eventually decided to uphold the takings complained of, but both the majority and the minority expressed some reservations regarding the established tradition of deference to the legislature in determining what counts as a legitimate purpose in takings cases.\footcite[563-565]{thomas09} The dissenting opinion of Justice O'Connor was particularly clear, and I highlight how her argument actively engages with the possible political and social ramifications of economic development takings. She expresses concern, in particular, that the power of eminent domain can be co-opted by powerful commercial interests and used against vulnerable groups.\footcite[505]{kelo05}

The backlash of {\it Kelo} was immense, with commentators describing it as possibly the worst received decision by the Supreme Court in modern history.\footcite{somin07} According to surveys, 80-90 \% of the US public disapproved of the decision, including a large majority among the political elites.\footcite{somin09} The case also prompted an unprecedented level of state legislation attempting to curb the use of eminent domain for economic development.\footnote{So far, some 44 states have enacted eminent domain reforms of this kind, see \url{http://www.castlecoalition.org/} (a property activist project that maintains an eminent domain ``report card'' system for states).}

Hence, I conclude that {\it Kelo} marks the topic of economic development takings as one that deserves special attention. I go on to argue that in conjunction with a contextual shift at the theoretical level, we should approach this class of takings concretely, looking at their effect on people and their communities. This will add substance to a comparative constitutional and human rights analysis, allowing us to ground the discussion in the socio-legal reality of how economic development takings play out under specific legal and regulatory regimes. Following up on this assertion, I move on to the main part of my thesis, where I will study the workings of economic development takings in more depth, relying on a comparative analysis of European and US law and data from a case study on ownership of waterfalls and hydropower development in Norway. \\

\noindent {\bf Progress so far:} This chapter is based on elements of work done on Chapter \ref{chap:2} that I felt would be more suitable for a somewhat extended introductory chapter. A rough draft of about 5000 words is in place and the crucial sources have been identified. Some original writing remains. \\

\noindent {\bf Estimated time to completion:} 1-2 months.

\section*{Chapter 2: A Comparative Overview of Economic Development Takings}\label{chap:2}

In this Chapter, I give an overview of the theory and practice surrounding economic development takings in the US and Europe. I first contrast English and German law, before I go on to give a more detailed presentation of the unifying property clause in P1(1) of the ECHR. The case law from the ECtHR is presented and analysed in some depth, in an effort to assess how the ECtHR would be likely to approach an economic development case such as {\it Kelo}. In particular, I argue that the ``proportionality'' doctrine offers an interesting approach to economic development cases. This doctrine stipulates that a ``fair balance'' must be struck  between the interests of the property owner and the public.\footcite[Chapter 5]{allen05} I argue that such a perspective could make it easier to get to the heart of why economic development takings are often seen as problematic, without getting lost in theoretical discussions about the meaning of  terms like ``public use'' or ``public purpose''. However, I also raise the concern that the ECtHR is not the appropriate institution for applying the proportionality test concretely. Its remoteness suggests that we should also look for more locally grounded legitimacy-enhancing institutions. Such institutions will likely be better able to assess the fairness of interference in context.

I go on to discuss whether existing government institutions can serve this purpose, arguing that local courts may well be the best candidates. However, I note that active application of the ``proportionality''-doctrine in property cases has not yet developed at the local level, at least not in Norway and the UK. I also discuss possible shortcomings of local courts; as judicial bodies they are not intrinsically well-suited to carry out the kind of assessment that is required. Hence, I suggest that entirely new institutional proposals might be in order. I conclude by arguing that once the need for local grounding is recognized and met, the ECtHR has the potential to play an important and constrictive role in providing oversight and developing basic principles.

In the last part of the Chapter, I consider US law, where economic development takings are already considered as an important special category. I track the development of the case law on the public use restriction in the Fifth Amendment and in various state constitutions, from the early 19th Century up to the present day.\footnote{The public use clause in the US constitution was not held to apply to state takings until the late 19th Century, see \cite{chicago97}.} Many writers assert that the 19th and early 20th Century was characterized by a ``narrow'' approach to public use which eventually gave way to a broader conception.\footnote{See, e.g., \cite[483]{walt11}; \cite[203-204]{allen00}. For a more in-depth argument asserting the same, see \cite{nichols40}.} Against this, I argue that it is more appropriate to think of this period as one when courts adopted a {\it broad} approach to judicial scrutiny of the takings purpose at state level. Importantly, I also argue that while different state courts expressed different theoretical views on the meaning of ``public use'', there was a growing consensus that the approach to judicial scrutiny should be contextual, focused on weighing the rationale of the taking against the concrete social, political and economic circumstances of the local area.\footnote{A summary of state case law that supports this view is given in the little discussed Supreme Court case of \cite{hairston08}.}  In particular, I argue that early state courts did not focus as much on the exact wording of the constitutional property clause as many later commentators have suggested.

I go on to show that the doctrine of deference that was developed by the Supreme Court early in the 20th Century was directed primarily at state courts, not state legislatures and administrative bodies.\footnote{See \cite{vester30} (echoing and citing \cite{hairston08}).} I then present the case of {\it Berman}, arguing that it was a significant departure from previous case law.\footcite{berman54} After {\it Berman}, deference was suddenly taken to mean deference to the (state) legislature, so there would be little or no room for judicial review of the takings purpose. I go on to present the subsequent developments at state level, characterized by increasing worry that the eminent domain power could be abused by powerful commercial actors. I discuss in particular the case of {\it Poletown}, where a neighborhood of about 1000 homes was razed to provide General Motors with land to assemble a car factory.\footcite{poletown81} I link this to the subsequent controversy that arose over {\it Kelo}, suggesting that it should be seen as the eventual backlash of {\it Berman}, a consequence of abandoning the contextual approach to public use in favor of an almost absolute rule of deference.

After the historical overview, I go on to briefly present the vast amount of research that has targeted economic takings in the US after {\it Kelo}. I give special attention to writers that propose new legitimacy-enhancing institutions for facilitating  economic development that requires compulsion and coordination. I focus on two proposals in particular, targeting compensation and participation respectively.\footcite{lehavi07,heller08} These proposals will serve as important reference points later on, when I consider the Norwegian appraisal  and land consolidation courts in Chapters 4 and 5. \\

\noindent {\bf Progress so far:} An almost complete draft of just under 20 000 words. \\

\noindent {\bf Estimated time to completion:} Less than a month.

\section*{Chapter 3: Norwegian Waterfalls and Hydropower}\label{chap:3}

In this Chapter, I present the case study of Norwegian waterfalls in detail. I begin by describing the historical and present-day importance of water to Norwegian culture and economy, before I present the legal framework surrounding the development of hydropower. I also give a terse presentation of Norwegian expropriation law. Most of the chapter is devoted to presenting recent changes in the Norwegian hydropower sector, whereby local owners of waterfalls now take more active part in hydropower development.

I pay particular attention to tensions that have developed between established energy companies, who would traditionally rely on eminent domain to acquire waterfalls, and waterfall owners, who oppose this tradition vigorously due to their own interest in owner-led development of hydropower. As a result, the law relating to takings for development of hydropower is under stress, and many cases have come before the courts in recent years. In general, while owners have declared victory in relation to some aspects of the way in which compensation is awarded, established energy companies have continued to enjoy relatively easy access to waterfalls through eminent domain.

This observation leads me to the next two chapters, where I give a more in depth analysis of the two main issues that have arisen in Norway, corresponding nicely to issues that have also been discussed in relation to economic development takings elsewhere. \\

\noindent {\bf Progress so far:} A draft of about 16 000 words. \\

\noindent {\bf Estimated time to completion:} Less than a month.

\section*{Chapter 4: Compensation}\label{chap:4}

In this Chapter, I first present Norwegian rules for awarding compensation, including an historical overview of the main tensions that have shaped current rules.\footnote{Including the case of {\it Kløfta} Rt. 1976 s. 1, where the Supreme Court effectively held that provisions included in the Compensation Act 1974 went against the property clause in section 105 of the Norwegian Constitution. (Although, according to some, the Court only ``reinterpreted'' the offending provisions in light of constitutional law).} Then I offer a comparative assessment focused on the use of so-called {\it elimination} rules, know in common law as the ``no-scheme'' rule or the ``Pointe Gourde'' principle.\footcite{gourde47} Noting that such rules are hard to justify for economic development takings, I present the Norwegian system of appraisal courts. I show how these courts played a crucial role in formulating new principles for awarding compensation in waterfall cases, by implementing a form of benefit-sharing that is precluded under standard elimination rules.

I go on to discuss more broadly the extent to which property owners should be given a share of the surplus generated by projects benefiting from eminent domain. I argue that this is indeed reasonable in economic development cases, since the rationale for elimination is missing. I note that this has been observed by many commentators in several different jurisdictions, but that suggestions differ as to how the appropriate premium should be calculated.\footcite{lehavi07,crow07,stenseth10} For Norwegian waterfalls compensations now include such a premium, provided certain conditions are fulfilled. Such awards are calculated based on the hypothetical consideration of what would happen if owners {\it participated} in development. That is, compensation is estimated on the basis of the assumption that owners would willingly commit their land to a hydropower project, in return for shares or securities in the development company. I relate this manner of calculating compensation with other suggestions that have emerged in the literature.

I go on to note that while better compensation is legitimacy-enhancing, it is not easy to get compensation awards right, particularly not when they involve complex patterns of hypothetical assessment. Essentially, such assessment requires the appraiser to consider the future income that would have been generated from a hypothetical hydropower project. I show that in practice, it is not clear if this model for waterfall compensation has actually been a success. The outcomes vary widely with the opinions of the experts, and the complexity of the simulation models used can mean that a relatively slight change to some of the premises will lead to massive differences in the outcome. I also note that the models fail to capture the more ``subjective'' value that owners might attribute to their waterfalls, for instance the scenic and environmental value that some owners feel far exceeds any potential revenue from hydropower.

Following up on the observation that compensation reform has a limited legitimacy-enhancing effect, I go on to consider the natural next step: instead of just simulating participation by owners, why not replace eminent domain by a procedure which makes participation a reality? I observe that an institution for achieving this already exists in Norway, and this leads me naturally to the next chapter. \\

\noindent {\bf Progress so far:} A draft of about 17 000 words. \\

\noindent {\bf Estimated time to completion:} Less than one month.

\section*{Chapter 5: Participation}\label{chap5}

In this Chapter, I start by discussing the extent to which property owners and local communities have a voice in decisionmaking processes concerning land use under Norwegian law. I note that this depends greatly on both the type of owner, the type of property and the type of land use in question. I then zoom in on waterfalls and hydropower, where the decisionmaking procedure is organized in such a way that waterfall owners and their local communities are largely placed on the sideline. This is particularly clear in relation to the use of eminent domain, since Norweigan law operates a system where the takings power is often {\it automatically} conferred on any developer that succeeds in obtaining a development license from the central water authorities.

I then go on to present alternatives to eminent domain under Norwegian law, focusing on the land consolidation courts. These courts come equipped with wide powers to reorganize and manage jointly owned property. Traditionally, land consolidation was used only for agricultural land, but its scope is widening. In particular, land consolidation has recently been introduced as a mechanism to replace eminent domain also in urban development projects.\footcite[788-790]{stenseth10}

Land consolidation has also become widely used for hydropower development, particularly for owner-led projects, including cases when some of the owners object to development. I present this use of land consolidation in depth and I use it to analyse the land consolidation court as a potentially legitimacy-enhancing institution for compulsory economic development of jointly owned land. I compare it with other proposals, particularly the land assembly districts (LADs) introduced by Heller and Hills.\footcite{heller08} I show that while land consolidation courts can have much wider scope, they can also be used to simulate the behavior of LADs. However, as they are judicial in nature they include many extra safeguards, including some which are explicitly in place to protect owners. In particular, a fundamental rule in land consolidation law is that no development can be ordered unless it will benefit all the properties involved, possibly after some reorganization of boundaries and rights.\footnote{Section 3(a) of the Land Consolidation Act 1979.}

I conclude that the land consolidation procedure is a viable and interesting alternative to (traditional) eminent domain procedures, particularly in cases of economic development. I show that they embody both sensitivity to the needs of society as well as  contextual safeguards that empower owners and their local communities. Since they are not yet fully developed for economic development projects outside the traditional agricultural context, it remains to be seen if they can deliver on what they promise. I conclude by suggesting that in order for land consolidation to become a success, it is important that reforms needed to expand their scope do not simultaneously destroy their defining characteristics. Importantly, land consolidation courts should continue to be service-providers to communities of owners. If they retain this character, I argue that they can serve as excellent frameworks for empowering owners in their interactions with each other, their local community, and society as a whole. \\

\noindent {\bf Progress so far:} A draft of about 16 000 words. \\

\noindent {\bf Estimated time to completion:} Less than a month.

\section*{Chapter 6: Conclusion}\label{chap:6}

In this Chapter, I will summarize the results of my research and offer concluding comments. \\

\noindent {\bf Profress so far:} Remains to be written. \\

\noindent {\bf Estimated time to completion:} Less than a month.

\end{document}


