%%%%%%%%%%%%%%%%%%%%%%%%%%%%%%%%%%%%%%%%%
% Beamer Presentation
% LaTeX Template
% Version 1.0 (10/11/12)
%
% This template has been downloaded from:
% http://www.LaTeXTemplates.com
%
% License:
% CC BY-NC-SA 3.0 (http://creativecommons.org/licenses/by-nc-sa/3.0/)
%
%%%%%%%%%%%%%%%%%%%%%%%%%%%%%%%%%%%%%%%%%

%----------------------------------------------------------------------------------------
%	PACKAGES AND THEMES
%----------------------------------------------------------------------------------------

\documentclass{beamer}

\mode<presentation> {

% The Beamer class comes with a number of default slide themes
% which change the colors and layouts of slides. Below this is a list
% of all the themes, uncomment each in turn to see what they look like.

%\usetheme{default}
%\usetheme{AnnArbor}
%\usetheme{Antibes}
%\usetheme{Bergen}
%\usetheme{Berkeley}
%\usetheme{Berlin}
%\usetheme{Boadilla}
%\usetheme{CambridgeUS}
%\usetheme{Copenhagen}
%\usetheme{Darmstadt}
%\usetheme{Dresden}
%\usetheme{Frankfurt}
%\usetheme{Goettingen}
%\usetheme{Hannover}
%\usetheme{Ilmenau}
%\usetheme{JuanLesPins}
%\usetheme{Luebeck}
%\usetheme{Madrid}
%\usetheme{Malmoe}
%\usetheme{Marburg}
%\usetheme{Montpellier}
%\usetheme{PaloAlto}
%\usetheme{Pittsburgh}
%\usetheme{Rochester}
%\usetheme{Singapore}
%\usetheme{Szeged}
%\usetheme{Warsaw}

% As well as themes, the Beamer class has a number of color themes
% for any slide theme. Uncomment each of these in turn to see how it
% changes the colors of your current slide theme.

%\usecolortheme{albatross}
%\usecolortheme{beaver}
%\usecolortheme{beetle}
%\usecolortheme{crane}
%\usecolortheme{dolphin}
%\usecolortheme{dove}
%\usecolortheme{fly}
%\usecolortheme{lily}
%\usecolortheme{orchid}
%\usecolortheme{rose}
%\usecolortheme{seagull}
\usecolortheme{seahorse}
%\usecolortheme{whale}
%\usecolortheme{wolverine}

%\setbeamertemplate{footline} % To remove the footer line in all slides uncomment this line
%\setbeamertemplate{footline}[page number] % To replace the footer line in all slides with a simple slide count uncomment this line

%\setbeamertemplate{navigation symbols}{} % To remove the navigation symbols from the bottom of all slides uncomment this line
}

\usepackage{graphicx} % Allows including images
\usepackage{booktabs} % Allows the use of \toprule, \midrule and \bottomrule in tables
\usepackage[utf8]{inputenc} % set input encoding to utf8
\usepackage[style = oscola,backend=biber]{biblatex}



%----------------------------------------------------------------------------------------
%	TITLE PAGE
%----------------------------------------------------------------------------------------
\usepackage[utf8]{inputenc}

\addbibresource{thesis.bib}

\title[]{On the Legitimacy of Economic Development Takings} % The short title appears at the bottom of every slide, the full title is only on the title page

\author{Sjur K Dyrkolbotn} % Your name
\institute[Durham University, Utrecht University] % Your institution as it will appear on the bottom of every slide, may be shorthand to save space
{
Durham University, Utrecht University \\ % Your institution for the title page
\medskip
\textit{s.k.dyrkolbotn@durham.ac.uk} % Your email address
}
\date{10 December 2015} % Date, can be changed to a custom date

\begin{document}

\begin{frame}
\titlepage % Print the title page as the first slide
\end{frame}

\begin{frame}
\frametitle{Overview} % Table of contents slide, comment this block out to remove it
\tableofcontents % Throughout your presentation, if you choose to use \section{} and \subsection{} commands, these will automatically be printed on this slide as an overview of your presentation
\end{frame}

%----------------------------------------------------------------------------------------
%	PRESENTATION SLIDES
%----------------------------------------------------------------------------------------

%------------------------------------------------
\begin{frame}
\frametitle{My personal motivation} % Sections can be created in order to organize your presentation into discrete blocks, all sections and subsections are automatically printed in the table of contents as an overview of the talk
%------------------------------------------------

\begin{itemize}
\item Grew up on a farm, that held hydropower rights developed in an owner-led small-scale plant.
\item Worked as a lawyer, specialising in hydropower cases, especially cases involving expropriation of hydropower from local community owners.
\item A lack of interest in legitimacy issues in these cases, which suggested a comparative look, fuelled initially by comparison with the US debate following Kelo.
\end{itemize}
\end{frame}
\begin{frame}
\frametitle{Main challenges}
\begin{itemize}
\item The case law on expropriation for hydropower in Norway necessitated a view of property that is distinct from the typical liberal or neo-liberal accounts.
\item To connect the theoretical work on property and its social functions with the case study on hydropower; needed to cover quite a lot of ground in relatively short amount of space -- a challenge of balancing the need for maintaining a clear direction in the argument, without oversimplification of various issues encountered along the way. 
\item A new theoretical foundation needed to break away from the narrow entitlements-based narrative that often reduces legitimacy issues to compensation questions; needed also to highlight the link between property and social and economic rights in the context of Norwegian hydropower.
\end{itemize}
\end{frame}
\begin{frame}
\frametitle{Main contributions}

\begin{itemize}
\item Contributing to the development of a new approach to legitimacy of economic development takings, inspired by the social function perspective and illustrated by a case study.
\item Further developing the Gray test, placing it in the comparative landscape of different approaches to judicial review of takings, and testing it by applying it to the case of waterfalls.
\item Further developing the idea that institutions for self-governance and common pool resource management can be used to restore legitimacy. A novel assessment of the Norwegian land consolidation courts as meta-level institutions that can be used to set up self-governance frameworks in a democratically legitimate way, while ensuring congruence with local conditions and appropriate nesting of decision-making.
\item Hopefully also showing that a social function view of property can be used to anchor discussions on issues that are also closely intertwined with, and sensitive to, the issue of environmental and social justice, including rights of non-owners.
\end{itemize}
\end{frame}
\begin{frame}

\frametitle{Why property? What is property?}

\begin{itemize}
\item First, because I felt it was the most relevant way to approach the Norwegian cases. In Norway, hydropower is controversial for two reasons: effect on local property-owning communities and effect on environment. Today, environment quite well protected (increasingly so), while local property rights are being completely undermined (Example: the ``deal'' between DNT and BKK).
\item More generally, when applicable, property provides a very firm and concrete anchor for social and economic rights of marginalised groups -- clearly an underlying political issue, embedded in a property narrative. Arguably, under some understanding of what property is, some ESC rights {\it are} property rights and obligations, e.g., the right of local residents with respect to development that result in dispossession can be understood as proprietary rights under a bundle perspective -- a system granting (property) rights to local people in the development potential of local resources. (Example: property rights to unemployment benefits, liabilities as ownership of toxic assets).
\end{itemize}
\end{frame}

\begin{frame}
\begin{itemize}
\item Culminates in proposal that private property might {\it not} be what the neo-liberals claim it to be. A need for broader notions, potentially beneficial also to broader narratives of environmental and social justice.
\item Why should the neo-liberals have a monopoly on using property narratives to further their commercial and political interests? 
\begin{itemize}
\item The narrative impact: Property intuitions deeply rooted in human nature, a powerful image that is presently used mainly by one side of the debate on environmental and social justice.
\item The substantive impact: Property arrangements are well-known, can be moulded largely within the sphere of private and can be highly precise and efficient at delivering new solution (e.g., if knowledge of the DNA sequence of a particular seed can count as private property, why should not ESC scholars also be able to refer to some ESC rights as arising from property, either through rights or responsibilities? Why not protect ESC rights and the environment through the law of property, e.g., propose reforms in the law of adverse possession or attach positive environmental obligations to private property (e.g., as found in the riparian law of the UK).)
\end{itemize}
\end{itemize}
\end{frame}

\begin{frame}

\frametitle{Is it not harder to make property egalitarian than to introduce ESC rights?}
\begin{itemize}
\item If by ESC rights one means rights that are secure, and that apply also in cases when national interests (or big business) want to negate those rights, then probably not.
\item Also, egalitarianism might become the consequence, rather than a necessary condition, for a successful implementation of social function ideas; responsibility of large and powerful owners might 
help fulfilment of basic rights to non-owners, while also providing incentives (or nudges) towards greater egalitarianism (e.g., many responsibilities towards non-owners plus restrictions on sale to non-locals or rights of pre-emption/first refusal, will eventually result in locals taking over the ownership in an egalitarian manner -- as happened in Norway during the 18th century.)
\end{itemize}
\end{frame}

\begin{frame}

\frametitle{What about the non-owners and the environment?}

\begin{itemize}
\item The social function theory and corresponding emphasis on property should not replace state regulation or independent state responsibility for ESC rights, but complement it and prevent the regulatory power of the state from taking forms that can lead to oppression or abuse by elites.
\end{itemize}
\end{frame}

\end{document}
