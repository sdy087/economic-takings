% Thesis Abstract -----------------------------------------------------

% NOTE: As with acknowledgements, I had to create a new format for this -- I couldn't get the original one to work. As with the acknowledgements, if you are able to fix the code so it's less messy, do pass the fix back to the Law Faculty.
\cleardoublepage
\addcontentsline{toc}{chapter}{Abstract}
%\begin{abstractslong}    %uncommenting this line, gives a different abstract heading
%\begin{abstracts}        %this creates the heading for the abstract page

\begin{quoting}
  \singlespace
    \begin{center}
  {\LARGE \bfseries On the Legitimacy of Economic Development Takings }\\
  \vspace*{0.5cm}
      {\large Sjur Kristoffer Dyrkolbotn}\\
  %\vspace*{0.1cm}  
   %   {\large \emph{Ustinov College}}\\
  \vspace*{0.2cm}  
    {\normalsize Thesis submitted to Durham Law School at Durham University for the degree of Doctor of Philosophy}

  \vspace*{0.2cm}  
    {\normalsize 24th March 2016}\\
  \vspace*{0.5cm}  
    {\normalsize \bfseries Abstract}      
  \end{center}
  {\parindent0pt
For most governments, facilitating economic growth is a top priority. Sometimes, in their pursuit of this objective, governments interfere with private property. Often, they do so by indirect means, for instance through their power to regulate permitted land uses or by adjusting the tax code. However, many governments are also prepared to use their power of eminent domain in the pursuit of economic development. That is, they sometimes compel private owners to give up their property to make way for a new owner that is expected to put the property to a more economically profitable use. %This new owner is sometimes the government itself, represented by one of its administrative bodies. But in many cases it will be a private company, operating for profit, possibly in cooperation with government entities through some form of public-private partnership.
}
\vspace{0.7mm}

This thesis asks how the law should respond to government actions of this kind, often referred to as {\it economic development takings}. The thesis makes two main contributions in this regard. First, in Part I, it proposes a theoretical foundation for reasoning about the legitimacy of economic development takings, including an assessment of possible standards for judicial review. Moreover, the thesis links the legitimacy question to the work done by Elinor Ostrom and others on sustainable management of common pool resources. Specifically, it is argued that using institutions for local self-governance to manage development potentials as common pool resources can potentially undercut arguments in favour of using eminent domain for economic development.

Then, in Part II, the thesis puts the theory to the test by considering takings of property for hydropower development in Norway. It is argued that current eminent domain practices appear illegitimate, according to the normative theory developed in Part I. At the same time, the Norwegian system of land consolidation offers an alternative to eminent domain that is already being used extensively to facilitate community-led hydropower projects. The thesis investigates this as an example of how to design self-governance arrangements to increase the democratic legitimacy of decision-making regarding property and economic development.

%This shows how local governance arrangements can work, suggesting that more attention should be devoted to studying the nexus between property, common pool resource management, and eminent domain.

%This theoretical basis is formulated independently of specific jurisdictions, but based on considering existing approaches to the legitimacy question from England and Wales, the United States, and at the European Court of Human Rights. In addition, the thesis draws a link between the legitimacy question and the work done by Elinor Ostrom and others on sustainable management of common pool resources. Specifically, it is argued that institutions for local self-governance that treat development potentials as common pool resources can often undercut arguments in favour of using eminent domain for economic development.

%Such rules are in place in most developed countries, and the fundamental status of property has been expressed explicitly in both the US constitution and the European Convention of Human Rights. The tension between these provisions and the practice of taking property for economic development, in many cases for commercial profit, is clear and worth considering further.

%\vspace{0.7mm}

%The second part of the thesis puts the theory developed in the first part to the test by considering takings for hydropower development in Norway. Under Norwegian law, the right to exploit the hydropower in most streams and rivers belong to the riparian owners. That is, the right to the hydropower belongs to the people who own the land over which the water flows, usually local community members. To acquire these rights, energy companies tend to rely on the government's power of eminent domain. Recently, however, local communities have begun to protest this practice, by arguing that they should be allowed to take a more active role in managing their own resources. This has resulted in tensions in Norway, shedding light on the legitimacy question as it arises in the context of Norwegian expropriation law. In addition, new light has been shed on the role of the so-called land consolidation courts, which are now increasingly asked to deliver alternatives to eminent domain in hydropower cases. The thesis investigates this in depth and argues that the unique system of land consolidation found in Norway demonstrates how to design self-governance arrangements that can increase the democratic legitimacy of decision-making regarding property and economic development.

\end{quoting}


%\end{abstracts}
%\end{abstractslong}


% ----------------------------------------------------------------------


%%% Local Variables: 
%%% mode: latex
%%% TeX-master: "../thesis"
%%% End: 
