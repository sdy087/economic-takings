%\documentclass[12pt,a4paper]{memoir} % for a long document
\documentclass[12pt,a4paper]{article} % for a short document

\usepackage[utf8]{inputenc} % set input encoding to utf8
\usepackage[style = oscola]{biblatex}

% Don't forget to read the Memoir manual: memman.pdf

\title{Progress report ``On the Legitimacy of Economic Takings: A case study of Norwegian waterfalls taken for hydropower development''}
\author{Sjur K Dyrkolbotn}
\date{} % Delete this line to display the current date

\addbibresource{thesis.bib}

%%% BEGIN DOCUMENT
\begin{document}

\maketitle

\section*{Summary of Contribution}

My thesis will contribute to research on the legitimacy of economic development takings, an area at the crossroads between property law, planning law, constitutional and human rights law.\footnote{See generally, \cite[Chapter 12]{epstein85};  \cite{merrill86,malloy08}.} It is an area of increasing interest to scholars, particularly in the US after the controversial decision in {\it Kelo v City of New London}.\footcite{kelo05} In this case, a majority of the Supreme Court upheld a taking for economic development, despite objections that this would be suspiciously favorable to the drug company Pfizer. In particular, Suzanne Kelo, who stood to lose her home, argued forcefully that the taking was not for public use, as required by the Fifth Amendment of the US constitution. While the majority of the Supreme Court disagreed, the case was met with outrage in the US public and provoked intense political and academic debate.\footnote{According to surveys, the decision in {\it Kelo} was opposed by 80-90 \% of the US population, and more than 40 states have passed post-{\it Kelo} legislation to restrict the use of takings for economic development, see \cite{somin09}.}

So far, economic development takings have not become as controversial in Europe, but there have been cases where the issue has come up, in various European jurisdictions.\footnote{For instance, in the UK, Ireland and Germany, as well as in Norway and Sweden. See \cite[466-483]{walt11}; \cite{stenseth10}.} The European Convention of Human Rights (ECHR) also contains a property clause in Article 1 of Protocol No. 1 (P1(1)), but the legitimacy of economic development takings has not yet been discussed in the case law at the European Court of Human Rights (ECtHR). However, it is interesting to analyse cases like {\it Kelo} against P1(1), particularly since the ECtHR has developed a doctrine that focuses on ``proportionality'' and ``fairness'' rather than the purpose of interference.\footnote{See generally, \cite[Chapter 5]{allen05}. Also note that there are ``jurisprudential developments in the direction of a stronger protection under Article 1 of Protocol No. 1'', see \cite[135]{lindheim12}.}

The fundamental question raised by economic development takings can be formulated independently of specific property clauses as follows: under what conditions, if any, is it permissible for governments to order compulsory transfer of property rights from citizens to for-profit companies to facilitate economic development? In my thesis, I shed new light on this question by analysing it against a case study I have conducted on waterfalls taken for hydropower development in Norway. The three main contributions I make to the field are summarized in the following.

First, I offer a systematic overview of recent developments in Norwegian law and administrative practices relating to the expropriation of waterfalls for hydropower development. Since the liberalization of the energy sector in the early nineties, private owners of waterfalls, usually farmers and other members of the rural community, have engaged in small-scale hydropower development on a scale not previously seen in Norway. At the same time, established energy companies continue to benefit from expropriation, now often in competition with new companies who want to develop hydropower in cooperation with private owners. This has led to many legal disputes and a dramatic change in the way compensation is calculated in cases when waterfalls are expropriated. At the same time, the principle that energy companies may take waterfalls compulsorily for commercial gain is upheld, now often to the direct detriment of owner-led hydropower projects. My thesis will provide the first detailed presentation of this development and its legal and administrative ramifications.

Second, I offer a comparative perspective on economic development takings, by describing their status under Norwegian expropriation law and comparing this to the status of such takings in Europe and in the US, including an assessment of such takings against the ECHR. As part of this work, I also provide a comparative analysis of so-called {\it elimination} principles, which can be found in both common law and civil law jurisdictions.\footnote{The most important such principle developed in common law is typically referred to as the {\it Pointe Gourde} rule, after \cite{gourdexx}.} Such principles raise special questions in economic development cases, since they can give the taker an additional financial incentive to rely on compulsion rather than a voluntary transaction. To elucidate the important points addressed in the comparative study, I will draw on my case study of waterfalls. I will show, in particular, how many of the challenges that have been raised are currently addressed concretely in the Norwegian case law on waterfalls and hydropower development. My work is the first to make this connection and to study Norwegian waterfalls in this light.

Third, I discuss a branch of recent scholarship that focus on the design of legitimacy-enhancing institutions that can supplement or replace standard eminent domain procedures in economic development cases. After describing and assessing this research, I go on to consider two institutions that currently operate in Norway. Both are judicial in nature, and both have played an important role in empowering waterfall owners in Norway. The first is the system of {\it appraisal courts}, used to award compensation in takings cases. Appraisal courts are characterized by the special procedural rules they observe, involving the participation of lay people on the bench, alongside expert appraisers and legally educated judges. The recent changes to how compensation is awarded in waterfalls cases originate from new practices adopted by the appraisal courts, practices that the Supreme Court have later upheld. Hence, the case of waterfalls witnesses to the potential of appraisal courts as legitimacy-enhancing institutions. I discuss them at some length, arguing that they establish an interesting balance of power over compensation determination between legal professionals, expert appraisers and lay people. At the same time, I note that their effect on legitimacy is limited by their narrow focus on compensation; they do not, in particular, directly empower property owners in the decisionmaking process prior to the taking decision.  

However, a second class of judicial bodies that achieve just this are the {\it land consolidation courts}. These courts are also governed by special procedural rules, and they sit with specially educated judges, not judges with an ordinary legal education. Moreover, they have the power to organize development on behalf of property owners, if necessary by force. However, they are constrained by the fundamental principle that a development project can not be ordered unless it is regarded as beneficial to all the properties involved. To achieve this, the court may reorganize and restructure rights and boundaries in real property, and it may establish corporate structures to diversify the risk involved. However, it will rarely engage in such measures unless it enjoys wide support among the property owners. In particular, the court is largely meant to be a providing a service to the property owners, providing them with an arena where they can deliberate about how to manage their land and resolve conflicts about land use and ownership.

However, since the land consolidation courts can compel owners to partake in economic development, it can also often function as an alternative to eminent domain. Unsurprisingly, the legitimacy of development facilitated by land consolidation is often much greater among owners and their local community, than development facilitated by the use of expropriation. This is clearly felt in waterfall cases, where land consolidation is increasingly being used instead of expropriation, particularly for owner-led projects (especially when only some of the owners support such development). Land consolidation was originally an instrument for rational owner-led management of jointly owned agricultural land, but the Norwegian government is currently considering broadening the scope of the procedure, to include urban development projects. In my thesis, I argue that land consolidation is a particularly interesting institution to look at as an alternative to eminent domain in economic development cases. Doing so ties land consolidation directly to an interesting recent proposal in the US literature, and I offer a comparative analysis. My thesis is the first to make this link and to study land consolidation in this light.

Overall, I think my thesis will make a valuable contribution to the study of economic development takings. I contribute to the theoretical foundation of such research, by setting out how to approach this topic comparatively and from the point of view of European law. Following up on this, I offer a concrete assessment of legitimacy based on as-of-yet unexplored empirical data from Norway.

\section*{Structure of thesis and plan for completion}


\section*{Chapter 1: Setting the stage}

In this chapter I introduce the subject, placing it in the theoretical landscape as a branch of constitutional and human rights law. The right to property is widely recognized as a human right and many jurisdictions also have constitutional property clauses protecting this right. At the same time, the practical import of property rights and provisions that aim to protect it is largely dependent on local property law and planning law regimes. Hence, the very idea of a fundamental rule protecting property owners -- across different jurisdictions - has been regarded as conceptually problematic. However, there is growing consensus that appropriate systems of property protection are crucial to the well-being of both individuals, local communities, and society as a whole. This has become particularly clear in the developing world, where commentators lament the lack of safeguards against neocolonialsim and ``landgrabbing'', both propelled forward by lack of legal protection for sustainable property structures. In addition, while property reform is often a crucial component in facilitating sustainable development, it has been observed that the lack of strong guiding principles often render such reforms ineffective or counterproductive. 

In this landscape, a narrow view of property rights as individual entitlements is inadequate. Rather, an holistic concept of situated property is in order, where ownership entails both rights and responsibilities within a community. This has been a major tenet in recent scholarship on human rights and constitutional property clauses. To make such clauses relevant and effective for a range of different political, economic, and social conditions, it is necessary to reconceptualize them in light of property's social dimension. It is crucial that property rights and obligations become integrated into a broader discourse of justice, such that upholding those rights and obligations can be seen as an element in ensuring fairness, not as a conservative dogma standing in the way of reform.

In the first chapter, I present this theoretical approach to property in some detail, before I go on to relate it to the debate on economic development takings in the US and Europe. This debate has flourished, especially in the US, and I suggest that this might be related to a shift in perceptions of property, along the theoretical lines laid out in the first part of the chapter. To substantiate this I present the groundbreaking case of {\it Kelo}, where the US Supreme Court dealt with the issue of economic takings in unprecedented depth. Eventually, the majority decided to uphold an economic development taking against constitutional objections, despite claims that the taking would be suspiciously favorable to the drug company Pfizer. However, both the majority and the minority expressed doubts about the principle of judicial deference to the legislature in such cases, a principle which had been firmly enthrenched in case law since the 1950s. The minority also spoke out against what they perceived to be an injustice, with Justice O'Connor in particular arguing that allowing commercial agents to co-opt the takings process set a dangerous precedent more generally. In her words: 

\begin{quote}


\end{quote}

I believe this quote illustrates the power of a contextual approach to property protection, and how it has come to play a major role in the debate on economic takings in the US. In particular, I note that Justice O'Connor looks actively at the political, social and economic ramifications of the case, arguing in favor of protection of the owner's rights on this basis rather than on the basis of her individual entitlements.

The backlash of {\it Kelo} was immense, with commentators describing it as the worst received decision by the Supreme Court in modern history. According to surveys, 80 -- 90 \% of the US public disapproved of the decision, including a large majority among the political elites.\footnote{....} The case also prompted an unprecedented level of state legislation attempting to curb the use of eminent domain for economic development.\footnote{As of 2010, some 44 states had enacted eminent domain reforms of this kind.}

I conclude by suggesting that the {\it Kelo} case marks the topic of economic development takings as one that deserves special attention. More generally, I suggest that in conjunction with a contextual theoretical shift at the theoretical level, a practical approach to property rights and obligations requires that we identify suitable categories of case types that can be studied in light of the special circumstances that render them distinct. I argue that the one-size-fits-all approach must be abandoned and that we should  shatter the illustion that property protection is a self-contained issue that can be studied in isolation from socio-legal aspects.

Following up on this assertion, I move on to the main part of my thesis, where I will study economic development takings in depth, using data from Norway to shed light on the special issues that arise for this category.

{\bf Progress so far:} This chapter is based on elements of work done on Chapter 2 that I now feel would be more suitable for an introductory chapter. A rough draft of about 5000 words is in place and the crucial sources have been identitfied. Some original writing remains. \\

{\bf Estimated time to completion:} 1 -- 2 months.

\section*{Chapter 2: A Comparative Overview of Economic Development Takings}

In this Chapter I give an overview of the theory and practice surrounding economic development takings in the US and Europe. I start by considering Europe, where such takings have not yet proved as controversial as in the US. I argue that this is related to the fact that takings tend to be better justified as expressions of the public will. I argue that this is the case in part due to the prevalence of less romantic perceptions of property, but also due to better safeguards against abuse. I note, however, that many different constitutional approaches to property co-exist in Europe and that different jurisdictions differ in how they approach cases of economic development takings. I illustrate this by contrasting English and German law, before I go on to give a more detailed presentation of the unifying property clause in the European Convention of Human Rights. 

I present the case law developed at the European Court of Human Rights in Strasbourg, tracking it in some depth. In particular, I note the development from the ``wide margin of appreciation''-doctrine established in early cases such as {\it James}, to the ``increased protection for property owners'' signalled in recent cases such as {\it Lindheim}. Importantly, I discuss the fact that in relation to the ECHR, the ``public purpose'' requirement for interference is of little practical relevance, due to the wide margin of discretion awarded to states in this regard. Instead, the court adjudicates cases based on their doctrine of ``fairness'' and ``proportionality'', allowing them to check whether an acceptable balance has been struck between the interests of the owner and the interests of the public. No case has so far been adjudicated that deals specifically with economic takings, but the fact that such cases appear to arise increasingly often in the member states, suggests that they will soon arise also at the ECtHR. 

I conclude by noting that the proportionality doctrine developed at the ECtHR could give the Court a good way to approach economic development cases, as it would allow them to single out for special consideration the probleamtic aspects of such cases. In particular, instead of getting stuck in theoretical an abstract discussions about the meaning of the term ``public purpose'', the Court may consider instead the context and consequences of such takings, judging whether or not they can be upheld against a standard of fairness. On the other hand, I also note the potential danger in the fact that this assessment is then carried out by a judicial body that is remote to the reality of the situation complained of. In general, doubts can be raised whether decentralized and local institutions would not be better suited to implement a contextual approach to property protection. This, of course, might be the aim of the current developments at the ECtHR: to ensure that member states themselves take steps to institutionalize adeqatue mechanisms to protect property in the context. But how can this effect be achieved, without also making ECtHR an institution that interfers with policy and gradually becomes a political player in Europe? I return to this question later, in Chapter \ref{sec:4} and \ref{chap:5}, when describing how two institutions peculiar to Norway have played an important role in changing how Norwegian law 
facilitates hydropower development, towards more local owner-participation and benefit sharing.

In the last part of the Chaper I consider US law, where economic development takings are already considered as an important special category. I track the development of the case law on the public use restriction in the Fifth Amendment of the US Constitution, from the early19th Century up to the present day. Against some commentators, I argue that the early period was not marked by a ``narrow'' approach to public use, but rather by a {\it broad} approach to judicial scrutiny of the takings purpose at state level. I argue that while different courts expressed different theoretical views on the meaning of ``public use'', the period was marked by a growing consensus that the approach to judicial scrutiny should be contextual, focused on weighing the rationale of takings against the concrete social, political and economic circumstances of the local area. In partiular, early state courts did not focus as much on the exact wording of the constiutional property clause as some later commentators seem to suggest.

I go on to argue that the doctrine of deference that was developed by the Supreme Court early in the 20th Century was directed primarily at state courts, not state legislatures and administrative bodies. I then present the case of {\it Berman}, suggesting that it was a significant departure from previous case law and that it dramatically extended the doctrine of deference. Now, deference was suddenly taken to mean general deference to the legislature, signalling that there should be little or no room for judicial review of the takings purpose.  I then present the subsequent developments at state level, characterized by increasing worry that the eminent domain power could be abused by powerful commercial actors. I discuss in particular the case of {\it Poletown}, where a neighborhood of some 3000 low-income people was razed to give General Motors land to assemble a car factory. I link this to the subsequent controversy that arose over {\it Kelo}, suggesting that it this case and its aftermath should be seen as the eventual backlash of {\it Berman}, which effectively abandoned contextualism in favor of an almost absolute rule of deference.

After the historical overview, I go on to briefly present the vast amount of academic work that has targeted economic takings in the US after {\it Kelo}. I note that many suggestions are static in nature, and that they have already been followed up on at the state level, where acts have been passed to curb eminent domain for economic development. According to many, these reforms have been largely ineffective, however, which I take to suggest the shortcoming of static approaches more generally. I therefore go on to consider more dynamic suggestions that are based on proposing the introduction of legitimacy-enhancing institutions for facilitating  economic development that requires compulsion and coordination. I focuse on two proposals in particular, targeting compensation and participation respectively. These proposals will serve as concrete reference points later on, when I consider the Norwegina appraisal  and land consolidation courts in Chapters \ref{chap:4} and \ref{chap:5}.

I conclude my chapter my noting how the institutional approach seems to suggest itsef on basis both of the European and US experience. On the one hand, deference to the legislature (or a ``wide margin of appreciation'') is established doctrine, supported by the sense that politically accountable institutions are better able to judge what counts as a pulbic purpose. On the other hand, the need for concrete and contextual safeguards against abuse is also clearly felt, as commercial interest otherwise threaten to co-opt the political proces sin their favor. With this tension as a starting point, I go on to consider the case of Norwegian waterfalls, where many of the issues raised in this chapter crystallize.

{\bf Progress so far:}
An almost complete draft of just under 20 000 words.

{\bf Estimated time to completion:}

Less than a month.

\section*{Chapter 3: Norwegian Waterfalls and Hydropower}

In this Chapter, I present the case of Norwegian waterfalls in detail. I begin by describing the importance of water in Norway, before I present the legal framework surrounding the development of hydropower. I include a bried presentaiton of Norwegian expropriation law in general.

{\bf Progress so far:}

A draft of about 16 000 words.

{\bf Estimated time to completion:}

Less than a month.


\section*{Chapter 4: Compensation}

In this Chapter, I present Norwegian rules for awarding compensation, offering a comparative analysis focused on the applicaiton of the ``no-scheme'' rule, also known from common law as the ``Pointe Gourde'' principle. 

{\bf Progress so far:}

A draft of about 17 000 words.

{\bf Estimated time to completion:}

Less than one month.

\section*{Chapter 5: Participation}

In this Chapter, I present owners' access to decisionmaking processes regarding development under Norwegian law. I focus on cases when the developer wishes to use eminent domain, detailing the procedural rules protecting the interests of owners and their communitiyes. Again, I use waterfalls as a running example, based on the recent case of {\it M{\aa}land} which broadly addressed the owners' right to participation prior to a decision to use eminent domain. I argue that the current system fails to ensure legitimacy, particualrly in cases when eminent domain is used for economic development, and particualrly when this development is subject to special centralized management (as is the case for hydropower).

In then go on to present alternatives to eminent domain under Norwegian law, focusing on the land consolidation courts, a special judicial body with wide powers to reorganize and manage property owned jointly by many people. I note how these courts are increasingly coming to replace the use of eminent domain in cases of small scale locally initiated hydropower projects, including in cases when some of the owners object to development. Moreover, I note how the Norwegian authorities are currently considering making use of land consolidation as an alternative to exproopriation more genrally, also in urban areas and with resepct to general planning law. 

I analyse the land consolidation court as a potentially legitiamxcy enhancing institution for economid development, comparing it in particular with the proposed land assembly districts introduce dby Professor Heller and Hills. sI argue that while land consolidation court s have wider scope, they can also be used in effectively the same way as Heller and Hills propose that their LADs are internded to function. \cite{heller08}

I concluse that ladn consolidation lprovideure  is a viabl and interesting alternative to (traditional) emeinet domain lprocedrue aw in cases of economid deleopemnt, arguing that it embodies both snesitivity to societies needs and the need for contextual sfaeguaards that empowe owners and local communities.

{\bf Progress so far:}

A draft of about 16 000 words.

{\bf Estimated time to completion:} Less than a month.

\section*{Chapter 6: Conclusion}

In this Chapter I will sum up the result of my research.

{\bf Profress so far:} Remains to be written.


{\bf Estimated time to completion:} Less than a month.

\printbibliography

\end{document}









is explicitly to ensure that all affected owners 


ies for 


The first proposal, set out in some detail by Professors hhh and hhhh, argues for the introduction of special purpose development companies that will enable property owners to bargain with developers about compensation. This, in particular, is meant to overcome problems of legitimacy arising from the fact that no-scheme rules and other legal and practical mechanisms that make just compensation difficult to achieve for economic development cases. I observe that a closely related way of thinking has recently made its way into Norwegian law via the waterfall cases. In particular, in certain circumstances the courts have now begun to award compensation for waterfalls based on the assumption that the developer and the property owner would have had an incentive to cooperate and share the benefits of development were it not for the expropriation license. 

the Norwegian system of appraisal courts, with laymen sitting as judges in appraisal disputes, I shed light on a related construction that 

take a share of the profit from a commercial project benefiting from eminent domain. It 


relate my case study to two special proposals that have been  


 , in particular, is critically addressed against the observation that in economic takings cases, it can lead to takings that effectively deprive the 

special category of cases 


begun to make use of this resource in small-scale hydropower projects. 

will be a novel contribution in three important ways. 



economic develpresents a case study of expropriation used to benefit the Norwegian hydropower sector
