%\documentclass[12pt,a4paper]{memoir} % for a long document
\documentclass[11pt,a4paper]{article} % for a short document

\usepackage[utf8]{inputenc} % set input encoding to utf8
\usepackage[style = oscola]{biblatex}

% Don't forget to read the Memoir manual: memman.pdf

\date{} % Delete this line to display the current date

\newcommand{\noo}[1]{}

\addbibresource{thesis.bib}

%%% BEGIN DOCUMENT
\begin{document}
\noo{
In many jurisdictions, the law encodes a principle to the effect that compensation should be calculated  without taking into account changes in the property's value that are due to the expropriation, or the scheme underlying it. In short, compensation should be based on the owner's loss, not the taker's gain. In this article, I am referring to this as the no-scheme principle, following the terminology of a recent UK Law Commission consultation paper. The exact details of various no-scheme rules differ between jurisdictions, but the underlying principle appears to play a crucial role in both civil and common law traditions.\footnote{I am not aware of any jurisdiction that does not include some rule corresponding to (aspects of) the no-scheme principle. I mention that in addition to the jurisdictions discussed in this section, no-scheme rules are also found in pure civil law jurisdictions like Germany and the Netherlands, see \cite[5,21]{sluysmans14}.}
}

\title{On Benefit Sharing and the Compensatory Approach to Economic Development Takings}

\maketitle

\section{Introduction}\label{sec:into5}

In this article, I address the compensatory approach to the legitimacy of so-called economic development takings, which involve a commercial entity, often privately owned, benefiting from the State's power to compulsorily acquire property. The first question I discuss has to do with the status of the development project when calculating compensation. Should the award to the owner reflect the commercial value of the development potential, or should this value be disregarded under a so-called ``no-scheme'' rule?\footnote{A no-scheme rule is a rule that is meant to ensure that compensation is calculated without taking into account changes in the property's value that are due to the expropriation, or the scheme underlying it. This terminology is established in the UK, see, e.g., \cite{lawcom01}.}

I start in Section \ref{sec:back} by considering the US debate on the legitimacy of economic development takings. I note, in particular, how some authors have argued that the legitimacy issue can be dealt with using a compensation mechanism that ensures benefit sharing.\footnote{See generally \cite{merrill86,fennell04,krier04,lehavi07,bell07,householder07}.} The broad aim of my article is to shed some light on this suggestion, through a comparative case study.

In Section \ref{sec:noscheme}, I present the no-scheme principle, and I note that it is the main obstacle to benefit sharing in practice. The principle is found in many jurisdictions, including the US. However, here I focus on recent case law from the UK, which I use to formulate a normative stance regarding the relationship between the idea of benefit sharing and the no-scheme principle.\footnote{I believe my comparative approach is justified, as the core idea of the no-scheme principle seems to be largely the same across different jurisdictions. In fact, I am not aware of a single jurisdiction that does not include some rule corresponding to (aspects of) the no-scheme principle. In addition to the jurisdictions discussed in this article, I mention that no-scheme rules are also found in civil law jurisdictions like Germany and the Netherlands, see \cite[5,21]{sluysmans14}.} In particular, I argue that recent developments in the UK illustrate that the principle can be interpreted and implemented in a way that does not necessarily block benefit sharing, at least not in economic development cases.

Section \ref{sec:norway} starts off the second part of my article, which is devoted to a case study of compensation procedures in Norway, particularly in the context of expropriation of waterfalls for hydropower development. This case study serves two main purposes. First, it is meant to shed light on some concrete issues that may arise when the tension between the no-scheme principle and the need for benefit sharing manifests in concrete cases. Second, and more broadly, the case study sheds some light on the feasibility of achieving fairness {\it in practice}, according to the normative stance I endorse. My conclusion is somewhat pessimistic. In particular, I believe the case study illustrates inherent weaknesses of the compensatory approach to legitimacy, clearly felt as long as an overarching objective is to achieve meaningful benefit sharing with owners.

Finally, in Section \ref{sec:conc}, I offer a conclusion. Here I summarize my findings, and argue that we should devote more attention to proposals that move beyond compensatory viewpoints and attempt to formulate institutional {\it alternatives} to expropriation. \noo{ for instances by proposing novel frameworks for collective action and compulsory participation in economic development projects.\footnote{I mention that similar ideas have emerged also in the context of the US debate on economic development takings, particularly in Heller and Hills' proposal that so-called {\it land assembly districts} can be set up to obviate the need for many kinds of economic development takings. See \cite{heller08}.}}

\section{The Compensatory Approach to Economic Development Takings}\label{sec:back}

Arguably, the primary distinguishing feature of economic development takings is that they give (private) takers the opportunity to profit commercially at the expense of owners and their communities.\footnote{For a small sample of US scholarship on this, see \cite{alexander05,underkuffer06,somin07,somin09}.} This can even be the primary aim of such projects, with the public benefiting only indirectly through potential economic and social ripple effects. Property owners facing condemnation in such circumstances might expect to take a share in the profit resulting from the use of their land. In practice, however, this is rarely achieved through compensation.\footnote{See, e.g., \cite[965-966]{fennell04}.} Instead, a no-scheme rule typically kicks in, resulting in compensation awards that are based on the pre-project value of the property that is being taken.\footnote{See, e.g., \cite[81]{freilich06}.}

The policy reasons usually given for no-scheme rules are based on the idea that the public should not have to pay extra due to its own special want of property. After all, one of the main purposes of using the power of eminent domain is to ensure that the public does not have to pay extortionate prices for land when important projects need to be carried out. However, when the expropriation project itself has a commercial flavour, there appears to be a shortage of good policy reasons for excluding its commercial value from consideration when compensation is calculated.

This is especially clear when, as in the US, compensation tends to be based on the market value of the land taken. Why should the taker's prospect of carrying out economic development with a profit be disregarded from the assessment of market value? In fair and friendly transactions among rational agents, one would expect benefit sharing in cases like this. Yet for economic development backed up by eminent domain, no-scheme rules ensure that all the profit goes to the developer. In light of this, some authors have argued that failures of compensation is at the heart of the legitimacy issue, and that worries over the public use restriction in the US Constitution is largely a response to concerns about the ``uncompensated increment'' of such takings.\footcite[See][962]{fennell04} 

I mention that two further problems of compensation have also been identified, that are not related to the notion of benefit sharing. First, the problem of ``subjective premium'' has been raised, pointing to the fact that property owners often value their own land higher than the market value, for personal reasons.\footcite[963]{fennell04} For instance, if a home is condemned, the homeowner will typically suffer costs not covered by market value, such as the cost of moving, including both the immediate ``objective'' logistic costs as well as more subtle costs, such as the cost of having to familiarize oneself with a new local community. Second, the problem of ``autonomy'' is also natural to discuss, since an exercise of eminent domain deprives the landowner of her right to decide how to manage her property.\footnote{See \cite[966-967]{fennell04}. For a general personhood building theory of property law, see \cite{radin93}. For a general economic theory of the subjective value of independence, see \cite{benz08}. For an even broader, non-individualistic and non-utilitarian, theory of property, based on notions such as {\it human flourishing} and {\it social obligation}, see the work of Alexander and others, e.g., \cite{alexander09a}.} However, I think it is doubtful whether compensation can ever be an effective remedy in this regard.

Here I will only consider the challenge of achieving fair benefit sharing. However, I note that achieving fairness in relation to other aspects of value often seem even harder. Hence, setting up a framework for benefit sharing is merely a first step towards a successful compensatory approach to economic development takings. But it is clearly a very significant step, and it has received quite some attention in recent scholarship.

In my opinion, the most interesting suggestion so far is due to Lehavi and Licht.\footcite{lehavi07} They propose that a new kind of institution should be introduced, called a {\it Special Purpose Development Corporation} (SPDC). The idea is that owners affected by eminent domain will be given a choice between standard pre-project market value and shares in the SPDC. This company will exist only to implement a specific step in the implementation of the development project, the transaction of the land-rights. The SPDC may choose either to offer their rights on an auction or else negotiate a deal with a designated developer.\footcite[1735]{lehavi07} Hence, the idea is to ensure that the owners are paid a value that reflects the post-project value of the land, but in such a way that the holdout problem is avoided. In particular, the SPDC will have a single task, namely to sell the land for the highest possible price within a given time frame.\footcite[1741]{lehavi07} After the sale is completed, the SPDC will divide the proceeds as dividends and be wound up.\footcite[1741]{lehavi07}

Other suggestions have taken a more static approach to compensation reform, such as proposing to give owners a fixed premium in cases of economic development, or developing mechanisms of self-assessment to ensure that compensation is based on the true value the owner attributes to his own land. A range of such proposals have been made: Merrill proposes 150 \% of market value for takings that are deemed to be ``suspect'', including takings for which the public use requirement raises doubts.\footcite[90-93]{merrill86} Krier and Serkin propose a system that provide compensation for a property's special suitability to its owner, or a system where compensation is based on the court's assessment of post-project value.\footcite[865-873]{krier04} Fennell proposes a system of self-evaluation of property for takings purposes with tax-breaks given to those who value their property close to market value (to avoid overestimation).\footcite[995-996]{fennell04} Bell and Parchomovsky also propose self-evaluation, but rely on a different mechanism to prevent overestimation; tax liability is based on the self-reported value and no property can be sold by its owner for less than his reported value. \footcite[890-900]{bell07}

Compared to such proposals, I think Lehavi and Licht's suggestion is more subtle. However, I think it is appropriate to distinguish between the particularities of their institutional proposal and the conceptual premise that underlies it. In this way, one may address the latter without being sidetracked by objections based on purely practical challenges.

To this end, I assume that the core idea of the SPDC proposal is to regard takings for economic development as a form of compulsory incorporation, a pooling of resources useful in overcoming market failures.\footcite[1732-1733]{lehavi07} Indeed, just as regular corporations are formed to package assets for effective management, so is eminent domain used to assemble property rights in order to facilitate efficient organization of development. In the words of Lehavi and Licht:

\begin{quote}
The exercise of eminent domain powers thus resembles an incorporation by the government of all landowners with a view to brining all the critical assets under hierarchical governance. Establishing a corporation for this purpose and transferring land parcels to it thus would be merely a procedural manifestation of the substantive economic reality that already takes place in eminent domain cases.
\end{quote}

As soon as we look at the rationale behind economic development takings in this way, any remnants of good policy reasons for ensuring that the developer gets all the profit disappear. The justification for eminent domain in economic development cases only extends to the necessary pooling of resources, it provides no reason to deprive owners of the commercial potential inherent in their land.\footcite[1735-1736]{lehavi07} 

This perspective is clear, and in my view very attractive. On the practical side, however, it seems very difficult to come up with a reliable pricing mechanism that truly does justice to the idea. Merely setting up an SPDC seems insufficient, since it provides no guarantee that there will be an {\it actual}, well-functioning, market for its rights. 

In particular, the fact that some development proposal has commercial potential does not by itself ensure that a market will form. Rather, it seems that most cases of eminent domain for economic development arise from land use planning that set up {\it de facto} development monopolies, by specifying a form of development that is desired by a specific developer. Indeed, the designated developer often takes active part in the planning process, in some cases to the point of being the primary author of the plans.

It seems largely unrealistic to think that other potential developers will be interested in competing for rights that are packaged to facilitate a specific development project undertaken by a specific party.\footnote{For an example, consider the infamous case of \cite{poletown81}, where around 1300 homes were condemned at the request of General Motors, who wished to set up a new car assembly plant where the homes had been. Although this was a taking for profit it seems hard to imagine that any other developer would be interested in bargaining for the right to purchase the entire site in question, taking into account its scale and how it had been laid out especially so that it would suit plans formulated by General Motors.} Hence, to work in practice, it seems that the SPDC proposal needs to be accompanied by significant reforms in land use planning procedures. Alternatively, one has to fall back on some discretionary mechanism where appraisers are asked to determine what the price {\it should be}, or {\it would have been} if there was a market, taking into account the commercial value of the project in question.

I believe this latter approach is basically unavoidable, at least until one is able to formulate a much more comprehensive account of how to restructure planning to provide better benefit sharing and increased participation rights for affected owners and competing developers. I briefly return to this particular challenge towards the end of my article, but for now I focus on the conceptual premise, namely that the owners should be paid based on their entitlement to a share of the incorporated value of their property rights for the purpose of development. This, in particular, is the crucial idea. Moreover, it seems to directly confront the no-scheme idea.

The next section is devoted to discussing the resulting tension in more depth. I will argue that, contrary to appearances, it is in fact possible to justify benefit sharing in economic development cases in a way that is consistent with (one interpretation of) the no-scheme principle. To do this, I look to the UK, where recent case law on the no-scheme rule seems to suggest an interpretation that remains quite open -- in principle, at least -- to benefit sharing in cases involving commercial development.

\section{The No-Scheme Principle}\label{sec:noscheme}

The no-scheme principle is easy enough to comprehend when it is presented in general terms, as the idea that the effect of the expropriation scheme should not be allowed to influence the compensation award. But difficult questions arise as soon as the idea is to be applied in concrete cases. What the principle asks of the valuers, in particular, is quite demanding. They are forced to consider a counterfactual world where the expropriation scheme is not present, and they must calculate the value of the property based on the workings of such an imaginary world. But what exactly should this world be taken to look like?

It might be tempting to consider this as a ``question of fact for the arbitrator in each case'', as expressed by the Privy Council in \emph{Fraser}, a Canadian case from 1917.\footnote{\cite[194]{fraser17}.} However, as the history of the no-scheme rule has shown, this point of view is not tenable.\footnote{For a history of the rule in UK law, clearly illustrating the difficulty in interpreting it and applying it to concrete cases, I point to Appendix D of \cite{lawcom03}. See also \cite{lawcom01}.}  The problem is that the nature of the no-scheme world cannot be determined without making a vast range of assumptions, many of which appear to depend on how one understands the law. The challenges that arise were discussed in great detail by Lord Nicholls in the recent case of \emph{Waters}.\footcite{waters04} He described the task as ``daunting'', noting also that some of the more recent statutory provisions ``defy ready comprehension''.\footnote{\cite[19]{waters04}.}

In {\it Waters}, the Lords made a particular point out of resolving a tension that was identified between the principle relied on in the \emph{Pointe Gourde} case from 1947 and the reasoning adopted in the so-called \emph{Indian} case from 1939.\footnote{\cite{indian39,gourde47}.} 

In the \emph{Indian} case, the scheme was given a very narrow interpretation, with Lord Romer interpreting the scope as follows.\footcite[319]{indian39}

\begin{quote}
The only difference that the scheme has made is that the acquiring
authority, who before the scheme were possible purchasers only, have
become purchasers who are under a pressing need to acquire the
land; and that is a circumstance that is never allowed to enhance the
value.
\end{quote}

Importantly, this did not entail that the purchaser's demand for the property was to be disregarded, since, as Lord Romer puts it:\footcite[316-317]{indian39}

\begin{quote}
[...] The fact is that the only possible purchaser of a potentiality is
usually quite willing to pay for it […]
\end{quote}

In \emph{Pointe Gourde}, a different stance appeared to have been adopted.\footcite{gourde47} The case concerned a quarry that was expropriated for the construction of a US naval base in Trinidad. The quarry had value to the owner as a business, and the valuer had found that if the quarry had not been forcibly acquired, it could also have supplied the US navel base on a voluntary basis, thereby increasing its profits. However, the value of this potential fell to be disregarded, with Lord MacDermott describing the no-scheme rule as follows:\footcite[572]{gourde47}

\begin{quote}
It is well settled that compensation for the compulsory acquisition of
land cannot include an increase in value, which is entirely due to the
scheme underlying the acquisition.
\end{quote}

Seemingly, this is at odds with the position taken by Lord Romer in the {\it Indian} case. In particular, it seems clear that in the absence of a compulsory purchase order, the US military would in fact have been ``quite willing'' to pay for the quarry's services.

In \emph{Waters}, both Lord Nicholls and Lord Scott addressed the tension between the two decisions in great detail. They then offered a reconciliatory interpretation, which narrows the no-scheme rule compared to how it has sometimes been understood following \emph{Pointe Gourde}.\footcite[242-244]{baum14} Moreover, the Lords noted the need for reform and legislation, with Lord Scott describing the current state of the law as ``highly unsatisfactory''.\footcite[164]{waters04}

To explain how a seemingly simple principle could become so troubling in practice, I think it is important to start by noting that after the introduction of extensive planning legislation in the 20th century, development of property tends to be contingent on governmental licenses and plans. Moreover, the power to expropriate is often granted as a result of comprehensive regulation of the property-use in an area, often following public plans that encompass more than the particular project that will benefit from compulsory purchase. As a result, it has become increasingly difficult to ascertain what is meant by the ``scheme'' in compensation cases. Does it include the whole planning history leading to expropriation, does it only refer to the power to expropriate, or is it something in between?

A fine balancing act is made when attempting to answer this question. Under a wide interpretation of the scheme, forcing the valuer to entertain many counterfactual assumptions, the property owner might come to feel that he is not compensated for his true loss, but rather an imaginary one. Indeed, the no-scheme world that the valuer must consider can end up being far removed from the actual one, forcing him to go back many years, perhaps decades, to establish what would have been the status of the property in question if the sequence of planning steps eventually leading to expropriation had not taken place. 

This can leave the property owner in an unpredictable and very weak position. Taken to extremes, the no-scheme principle can then also come to run amiss with respect to human rights law and constitutional provisions protecting private property. On the other hand, if the scheme is interpreted too narrowly, one runs the risk of endangering important public schemes by compelling the public to pay extortionate amounts. In many cases, it is undoubtedly true that the value of property is increased by public investments and plans for the area in which the property is found. Moreover, one may ask if it is right to pay compensation based on increases in value that result from investments and plans that would not have materialised unless the power to expropriate had been anticipated. This, it may be argued, would be a form of double payment that should be avoided.

It is also important to keep in mind that the no-scheme principle embodies two distinct purposes that can branch out and give rise to quite distinct rules.\footnote{See \cite[69-70]{lawcom03}.} First, the principle has an important \emph{positive} dimension, which serves to enhance compensation payments. Property owners are not only to be compensated for the direct loss of their property, but also for the possible depreciation of their property's value following the decision to permit a scheme which requires expropriation. Seemingly, this is easy to justify: It seems intuitively unreasonable if the deleterious effects of a threat of compulsion is permitted to result in reduced compensation payments.

However, under the extensive planning regimes common today, it is not clear where to draw the line. When is the regulation leading up to the scheme a reflection of public control over property use, and when should it be regarded as a measure specifically aimed at compelling private owners to give up their property? As we will see when we consider the role of the no-scheme principle in Norwegian law, this question can easily become highly controversial. Moreover, it is a question that may be linked with the more general question of whether or not the State should be liable to pay compensation for regulation that adversely affects the potential for future development.\footnote{In some jurisdictions, one will sometimes regard this as a taking in its own right, known as a {\it regulatory taking} in the US.} In jurisdictions that do not usually recognize any right to such compensation, such as Norway and the UK, it is easily argued that the positive aspect of the no-scheme principle must be limited correspondingly. Why should a depreciation of value following regulation imply compensation when the property is eventually expropriated, but not otherwise?

In addition to its positive dimension, the no-scheme principle also has an important \emph{negative} dimension, expressed in {\it Pointe Gourde} as the principle that an {\it increase} in value should be disregarded when it is ``entirely due to the scheme''.\footcite[]{gourde47} The negative dimension has attracted more interest and controversy than the positive dimension, especially in the UK. The rules pertaining to this aspect of the principle were also at the center of attention in {\it Waters}.

It is not surprising that the negative aspect of the no-scheme principle more often results in complaints, as property owners stand to loose whenever it is applied. However, on a traditional understanding of the public purpose of expropriation, the negative aspect of the rule is also seemingly easy to justify. In \emph{Waters}, Lord Nicholls describes the most important policy reasons as follows:\footcite[18]{waters04}

\begin{quote}
When granting a power to acquire land compulsorily for a particular purpose Parliament cannot have intended thereby to increase the value of the subject land. Parliament cannot have intended that the acquiring authority should pay as compensation a larger amount than the owner could reasonably have obtained for his land in the absence of the power. For the same reason there should also be disregarded the ``special want'' of an acquiring authority for a particular site which arises from the authority having been authorised to acquire it.
\end{quote}

This appears like a reasonable justification. Notice, however, that Lord Nicholls avoids using the word ``scheme''. Rather, he speaks of what the owner could reasonably have obtained in the \emph{absence of the power} to acquire the land compulsory. In this way, he seems to prescribe a rather narrow interpretation of the negative dimension of the no-scheme rule.\footnote{See also the commentary offered in \cite{crow07}.} It is the power to expropriate that should not give rise to an increased value, nothing at all is said at this stage about the scheme that benefits from it.

It would appear, therefore, that there is nothing in principle that prevents the property from being compensated on the basis of its value in a scheme that differs from the scheme underlying expropriation only in that it has not been granted a power to expropriate. Indeed, this seems rather crucial for the remainder of Lord Nicholls' arguments, wherein he reconciles the principle adopted in the \emph{Indian} case with that of \emph{Pointe Gourde}.

It would lead me too far astray to go into all the subtle details about the interpretation of the no-scheme rule in UK law and the possible implications of \emph{Waters}. Rather, I would like to focus on the application of the principle when the scheme in question is a commercial enterprise. The UK Supreme Court touched on this issue in the recent case of \emph{Bocardo}.\footnote{\cite{bocardo10}.} The case was decided under dissent, suggesting that the clarifications offered in \emph{Waters} have not been as conclusive as hoped.

In \emph{Bocardo}, a reservoir of petroleum extended beneath the appellant's estate. Moreover, the petroleum could not be extracted without carrying out works beneath the appellant's land. The first question that arose was whether or not extraction of the petroleum amounted to an infringement of property rights. This was answered in the affirmative. The second question that arose was how to compensate the owner. The Supreme Court, following some deliberation, found that the general rules applied, meaning that the case should be decided on the basis of an application of the no-scheme principle.

However, opinions differed as to the correct interpretation of this principle, and as to how the facts should be held against the law. The crucial point of disagreement arose with respect to whether or not the special suitability, or \emph{key value}, of the appellant's land, \emph{pre-existed} the petroleum scheme.

In \emph{Waters}, the House of Lords had cited and expressed support for the following passage, taken from Mann LJ's judgement in \emph{Batchelor}.\footnote{\cite[361]{batchelor89}. Cited by Lord Nicholls at \cite[65]{waters04}.}

\begin{quote}
If a premium value is ``entirely due to the scheme underlying the acquisition'' then it must be disregarded. If it was pre-existent to the acquisition it must in my judgement be regarded. To ignore the pre-existent value would be to expropriate it without compensation and would be to contravene the fundamental principle of equivalence.
\end{quote}

%(see \emph{Horn v Sunderland Corporation})
Relying on this distinction between the potentialities that are ``pre-existing'' and those that are due to the scheme, the minority in \emph{Bocardo}, led by Lord Clarke, made the following observation.\footcite[42]{bocardo10}

\begin{quote}
Anyone who had obtained a licence to search, bore for and get the petroleum under Bocardo’s
land would have had precisely the same need to obtain a wayleave to obtain access
to it if it was not to commit a trespass. So it was not the respondents' scheme that
gave the relevant strata beneath Bocardo’s land its peculiar and unusual value. It
was the geographical position that its land occupies above the apex of the
reservoir, coupled with the fact that it was only by drilling through Bocardo’s land
that any licence holder could obtain access to that part of the reservoir that gives it
its key value.
\end{quote}

This view was rejected by the majority, led by Lord Brown, who interpreted the no-scheme rule quite differently:\footcite[83]{bocardo10}

\begin{quote}To my mind it is impossible to characterise the key value in the ancillary
right being granted here as ``pre-existent'' to the scheme. There is, of course,
always the chance that a statutory body with compulsory purchase powers may
need to acquire land or rights over land to accomplish a statutory purpose for
which these powers have been accorded to them. But that does not mean that upon
the materialisation of such a scheme, the ``key'' value of the land or rights which
now are required is to be regarded as “pre-existent”.
\end{quote}

The case was resolved in keeping with this view, but the dissent shows that difficult, unresolved, questions can easily arise when the no-scheme principle is applied to an expropriation scheme that realises a commercial potential inherent in the land that is taken. Is it permissible for government to grant the value of this potential to the taker -- by granting him the necessary licenses -- without subsequently recognizing the potential as having been taken from the owner? 

This issue does not \emph{not} primarily depend on the scope of the scheme as such. In {\it Bocardo}, it was obvious that the scheme was the entire project aimed at extracting petroleum from the reserve, including the necessary works beneath the appellant's estate. But even so, it was still unclear whether the special value of the appellant's land could be said to have been {\it caused} by the scheme. The deeper question that arises in these kinds of situations seems to be almost ontological: When should we attribute a given value to an act of government, and when should we attribute it to nature, by regarding it as a fruit of the land? Or in more practical legal terms: When is a given property value that is unlocked by a development scheme part of the original owner's bundle of rights?

To answer this question, it is tempting to look for a causal link between scheme and value, to answer whether or not the value was pre-existent. But as \emph{Bocardo} illustrates, it is not obvious what should be taken as good evidence for such a link. It seems that one's perspective on this will tend to depend also on one's point of view on the much more general question of what values one recognize as inherent in property rights.

Lord Clarke remarked that the State, following nationalisation in 1934, could have given the right to extract the petroleum to \emph{someone else}.\footcite[163]{bocardo10} He was certainly correct. Hence, I also agree with him that ``the key value was not created by the 1934 Act or the grant of the petroleum licence to Star''.\footnote{See \cite[163]{bocardo10}.} But whose value was it, and was it a commercially realisable value? Here Lord Clarke appears to assume that the value must belong to the property owner and that this owner would also have been able to make a profit from it in the absence of the expropriation scheme. This, I believe, is a leap that requires further justification. Just because some property appears to have pre-existent value does not mean that the owner of the property is entitled to that value, or that it can ever be translated into a financial profit.

On the one hand, it is easy to agree with Lord Clarke that compulsory acquisition of a wayleave is no precondition for an extraction scheme. The project could well have been carried out by a developer who was willing to pay surface owners for the special suitability of their land. But on the other hand, it does not seem obvious that the owner is meant to be able to demand such payment under the regulatory system currently in place. Hence, even in the absence of a causal link between scheme and value, one might be entitled to conclude that the special value falls to be disregarded because it has already effectively been removed from the owner's bundle (or was never part of it in the first place).

In the case of {\it Bocardo}, I think this perspective would have been particularly helpful to Lord Brown, who argued that the value of the strata was not pre-existent. As it stands, his argument seems rather strained. After all, it was the physical conditions that gave the land its value, not the abstract fact that a development license had been granted. However, by looking at his argument in more depth, it is tempting to rephrase his conclusion by saying that he regarded the special suitability of the strata as having no commercial value under the prevailing regulatory regime.

In the end, I am agnostic about the correct way to judge {\it Bocardo}. Basically, I think the main question in that case was whether parliament had intended to give petroleum developers a right to extract substrata resources without sharing the profits with surface owners whose property had key value in relation to the extraction process. As no clear answer was available, conflict resulted. Moreover, the question itself became obfuscated. It seems to me, in particular, that the focus on causality and the notion of ``pre-existence'' was not very helpful. Rather, I think attention should have been explicitly directed at the issue of benefit sharing.

In general, the first question to ask when attempting to answer whether the value of some development potential should be compensated is whether this potential is supposed to be realisable for the property owner given the regulatory framework in place for the relevant kind of development. If this is unclear or the evidence suggests that owner realisation is not intended, the question becomes whether or not a claim can still be made on the basis of constitutional or human rights law. This question should then be openly addressed as a question of whether or not owners can demand benefit sharing.

If, on the other hand, courts engage with the question of benefit sharing surreptitiously, without being explicit about it, the lack of democratic accountability can become a worry. Indeed, I think it is important to emphasize the political sensitivity of the range of complex rules found in compensation law. Moreover, the underlying political question should not be obfuscated to the extent that it can only be engaged with in a meaningful way by legal professionals.

If this happens, most people will likely remain ignorant of the political significance of work done by the courts. Worryingly, those who stand to gain the most from this will be those who are in a good position to lobby and argue on technical points to gradually shape the law of benefit sharing according to their own interests. I think a conceptual shift might be needed to prevent this mechanism from becoming precarious to the legitimacy of compensation law in general, and the no-scheme rule in particular.

In addition, I think the questions touched on in {\it Bocardo} become much more pressing in cases when the main development potential as such is subject to expropriation, such as when natural resources are expropriated. In such cases, I believe one is right to ask if it is at all permissible to deny benefit sharing with the owner. In particular, it seems hard to imagine a defensible public interest in bestowing the entire commercial benefit on the expropriating party, particularly when this party is a powerful commercial actor. It is telling that this is never the explicitly stated aim of such takings, only a (sometimes unacknowledged) side-effect. Hence,  constitutional and human rights limitations on the takings power can become relevant here. 

However, I note that benefit sharing in these cases might not be prevented by the no-scheme principle, as that principle is now understood in the UK. In light of {\it Waters} and {\it Bocardo}, it seems possible to get around no-scheme rules by arguing that commercial development potentials pre-exist particular development schemes. If so, such values should be allowed to influence compensation payments. If this stance is accepted, the compensatory approach to the legitimacy of economic development takings is off to a good start, as benefit sharing is shown to be possible even within the no-scheme framework. 

The question remains, however, how far this gets us in practice. I address this question in subsequent sections, by looking to Norwegian law. Here the main issues have crystallized further, particularly in relation to expropriation of waterfalls for hydropower development.

\section{Norwegian Compensation Law}\label{sec:norway}

The owner's right to compensation following expropriation is enshrined in simple terms in Section 105 of the Norwegian Constitution.\footnote{\cite[105]{grunnloven14}.} The rule is simply that \emph{full compensation} is to be paid in all cases when the public interest necessitates compulsory acquisition of property. For more than 150 years, this was the sole legislative basis for compensation rules in Norway. The methods used to calculate full compensation in different scenarios developed entirely through case law. However, in 1973, legislation was introduced to make the system more predictable and -- to some extent -- in order to ensure that compensation payments were reduced.

In the following, I start by presenting the traditional system for calculating compensation, which is based on a judicial process that relies heavily on the discretion of lay people. Then I present the legislation currently in place, which is based on the notion that compensation should only be paid based on ``foreseeable'' uses of property. This sets the stage for Section \ref{sec:waterfalls}, where I discuss expropriation of waterfalls, for which compensation practices deviate from what is otherwise the norm, particularly in relation to the no-scheme principle.

\subsection{Appraisal Courts and ``Full Compensation''}\label{sec:appraisal}

According to a long legal tradition in Norway, the discretionary aspects of property valuation is regulated by a special procedure, with a significant reliance on so called \emph{unwilling appraisers}. These are members of the public who have no special connection to the case at hand or the parties involved in it. They may be chosen specifically for their suitability in judging the value of the contested property, either because they are resident in the local area or because they have special expertise.\footnote{See \cite[11|12]{aa17}.}

The appraisal procedure has a long history, and the rules regulating it today are found in the Appraisal Act 1917.\footnote{Act no 1 of 1 June 1917 relating to Appraisal Disputes and Expropriation Cases.} Appraisal cases are organised similarly to civil disputes, and the procedure is administered by the regular district courts.\footnote{\cite[5]{aa17}.} In appraisal disputes, these courts usually sit as a panel consisting of one professional judge and four appraisers, who have no special juridical qualifications.

Their role in the procedure is on par with the judge, and the panel decides jointly both the legal and the technical questions, usually on the basis of technical reports commissioned by the parties. These reports are presented during the main hearing and may be challenged by the parties, in more or less the same way as the district court hears evidence in a regular civil dispute.\footnote{See particularly \cite[22|27]{aa17}, with further references to the \cite{da05} (Act No 90 of 17 June 2005 relating to the Mediation and Procedure in Civil Disputes).} 

There is a possibility for appeal to the appraisal court of appeal, which is the regional court of appeal sitting as an appraisal court in accordance with the rules of the \cite{aa17}. The right to an appeal hearing is not absolute, it depends on the importance of the case, according to rules that correspond to those in place for regular civil disputes.\footnote{See \cite[32]{aa17}.} The procedure closely corresponds to the procedure followed in appraisal disputes at the district level.\footnote{See \cite[38]{aa17}.} However, the decision made by the appraisal Court of Appeal is final as far the appraisal assessment is concerned. An appeal to the Supreme Court can only be accepted on legal grounds (including procedural complaints).

As a consequence of this, the appraisal courts have been very important in interpreting and developing the law relating to compensation in Norway. Their importance was particularly great when the meaning of ``full compensation'' was not further clarified in statute. At the same time, the practical viewpoint enforced by the procedural form meant that legal questions often remained in the background in appraisal cases. Even today, legal issues only tend to come to the forefront if the Supreme Court decides to hear a case as a matter of principle. 

A typical example of the traditional form of legal reasoning in compensation cases, from the time before legislation was introduced, can be found in the writings of the prominent legal scholar Frede Castberg. He specifically addressed the no-scheme principle and his reasoning in this regard was based directly on a reading of the Constitution. Moreover, he relied on the principle of \emph{equality}, which was at that time considered particularly crucial in constitutional law. The following quote serves to sum up Castberg's position:\footcite[268]{castberg64b}

\begin{quote}
The owner is entitled to full compensation. The expropriation should not leave him worse off economically than other owners. Hence if the public has knowledge that an industrial undertaking is being planned, that a railway will be built etc, and this affects the value of property generally in a district, then the increased value of the property that will be expropriated must be taken into account. If not, the owners of such property will be worse off than other owners from the same district. On the other hand, if the expectation of the scheme underlying expropriation leads to a general depreciation of value, then it is this new value -- not the original value -- that is relevant for calculating compensation. The crucial question is what the actual value is, when expropriation takes place.
\end{quote}

As Castberg bases his analysis on the exact wording of the Constitution, he does not engage in reasoning based on (social) fairness considerations. However, it is not correct to think that his reasoning is particularly ``owner-friendly''. Indeed, Castberg explicitly states that depreciation of value due to an expropriation scheme should not be disregarded. Moreover, and despite appearances, the intention is not to reject the no-scheme principle altogether. In particular, Castberg denies that owners of expropriated property should ever be able to claim compensation based on the special want of the acquiring party:\footcite[268]{castberg64b}

\begin{quote}
The situation is different if the property has increased value due to the expectation that it will be expropriated. The owner can not demand that this increase is compensated since that would be the same as giving him a special advantage compared to those from whom no property is expropriated.
\end{quote}

Hence, Castberg accepts a narrow version of the no-scheme principle, similar in spirit to that presented by Lord Romer in the {\it Indian} case. Castberg's view appears to have been shared by many academics of his day, and it was also largely reflected in case law from the Supreme Court. 

At the same time, the very nature of the system for deciding appraisal disputes gave the local appraisers great freedom in adapting the principles in a way that suited the circumstances of concrete cases. To some extent, this would also involve making an assessment of what was regarded as a fair and just outcome. Hence, while the theory of the time seemed to leave little room for such arguments, case law was much more multi-faceted. 

Importantly, fairness was seen as a concrete issue that had to be addressed on a case-by-case basis, not on the basis of general rules. The Supreme Court largely sanctioned this approach, by respecting the discretion of the appraisal courts. The way in which the no-scheme principle was applied serves as a nice illustration of this. On the one hand, the theoretical views of Castberg were widely accepted, but at the same time they were regarded as no more than guidelines that had to be adapted to the circumstances. 

In fact, it was not unheard of for the lay appraisers to disagree with the judge about how this should be done. This happened, for instance, in the case of \emph{Tuddal}, where land was expropriated for the construction of a power grid.\footcite{tuddal56} Crucially, the expropriating party also acquired the right to use a private road. According to the juridical judge in the appraisal court of appeal, compensation should be awarded solely on the basis of what the owners stood to lose. In his view, this meant compensation based on the increased cost of maintaining the road due to increased use.

However, the lay appraisers found this result unreasonable and awarded compensation also for the special value the use of the road would have for the acquiring party. The Supreme Court found fault with the reasons given by the law appraisers, but agreed that such compensation was possible in principle. In reaching this conclusion, they reasoned that the owners were in effect compensated for the loss of a good bargaining position, an approach held to be appropriate under the ``full compensation'' principle.\footcite[111]{tuddal56}

The bargaining rule used to justify the decision in {\it Tuddal} is no longer considered good law. But the decision illustrates a broader point, namely that the Supreme Court was prepared to defer greatly to the judgement of the appraisal court. It was for this court to decide whether or not the lost opportunity to profit from negotiating with the expropriating party should be compensated.

This aspect of the decision is particularly noteworthy in light of the dissenting opinion of the juridical judge. Also, it seems to contradict the dominant legal theorizing of the day, which did not seem to support the idea that a premium should be paid. Hence, the {\it Tuddal} decision tells us that the Supreme Court went far in defending the discretion of the lay people, as a \emph{systemic} feature. The Court tested with great caution whether it was truly outside the permissible legal boundary, but concluded that it should simply be regarded as an exercise of the lay judgement that the system presupposed.

This impression of the case is accentuated when considering other cases dealing with similar issues. Across the board, I note a strong tendency to defend the role of the lay people in the appraisal process. A particularly clear expression of this can be found in \emph{Marmor}, also from 1956, where the Supreme Court overturned a decision made by the appraisal court of appeal on the grounds that the court had been too reliant on general principles.\footnote{\cite{marmor56}.}

The case involved expropriation of a private railway track, for the construction of a public railway. It was clear that the track which was being expropriated did not have market value in general. Moreover, the expropriating party argued -- on the basis of a no-scheme principle -- that the value of these tracks to the public railway should not be taken into account when calculating compensation. The appraisal court of appeal agreed, pointing to the standard teaching of the day. 

The Supreme Court, on the other hand, struck down the decision because it felt that a standardized approach to the case was inappropriate given the circumstances.\footcite[498-499]{marmor56} To justify this conclusion, the presiding judge paid particular attention to the wider \emph{context} of expropriation, and the manner in which expropriation was used to benefit certain interests. 

Importantly, he also noted how expropriation had come to replace voluntary agreement as the standard means of acquisition for this type of development. Therefore, the practice of using expropriation effectively prevented a market from developing, a market that might otherwise have appeared naturally. He said:\footcite[499]{marmor56}

\begin{quote}
I also point to the fact that the case concerns an area of activity where the expropriating party has a {\it de facto} monopoly which makes it impossible for anyone else to make use of the property for the same purpose. This in itself makes it questionable to simply assume that the lack of financial value for other purchasers provides the appropriate basis for calculating compensation. When considering this question, it is also appropriate to take into account that we have lately seen a great increase in the use of expropriation to undertake projects such as this. Compulsion is becoming the primary mode for acquisition of property -- replacing voluntary sale following friendly negotiations.
\end{quote}

In my opinion, the importance of this decision is not primarily that it seems to endorse a narrow interpretation of the no-scheme principle. In fact, I think it is erroneous to read the judgement as expressing support for any particular interpretation. Rather, I take the judgement to be an expression of scepticism towards uncritical obedience to \emph{any} set of general rules for calculating compensation, particularly if these limit the room for lay discretion. Moreover, I note that the contextual factors that the Supreme Court highlights are particularly relevant in the context of economic development takings.

The days of {\it Tuddal} and {\it Marmor} are gone. In particular, there has been a shift of attention towards ensuring predictability in compensation disputes. This development has led to legislation that limits the freedom of the appraisal courts. The key notion that has been introduced is that of ``forseeability'', whereby owners are only entitled to compensation for uses of their property that are regarded as likely and natural, according to an increasingly long list of rules that pertain to various case types.

\subsection{The Foreseeability Test}\label{sec:fore}

Following World War II, the social democratic \emph{Labour Party} gained a secure grip on political power in Norway. As a result, many reforms were carried out that would reshape Norwegian society. One of the most important reforms introduced extensive planning legislation to ensure that land use was placed more firmly under public control.\footnote{See generally \cite{thomassen97,kleven11}.} As a result, this period also saw expropriation being used more extensively to further public projects, such as hydropower development for the supply of electricity.\footnote{See generally \cite{skjold06,thue06b}.} As a result of these changes, many felt that a more uniform approach to compensation was needed. In addition, it became an explicitly stated political goal to bring compensation payments down.\footnote{\cite[19-20]{otprp70}.}

In 1965, the so called \emph{Husaas committee} was appointed by the King and charged with the task of assessing the compensation rules currently in place.\footnote{Appointed by the King in Council on 6 August 1965.} The committee was also ordered to make a concrete suggestion regarding the need for additional principles of compensation, and to consider if these should be given in the form of a special compensation act. Initially there was some doubt as to the extent to which is was at all permissible to give rules regulating compensation, as the constitution itself addressed the matter. 

However, the committee noted that some rules had already been introduced for specific case types, for instance in relation to expropriation for hydropower development.\footnote{See, e.g., \cite[16]{wra17}.} In addition, legal scholars of the day were generally of the opinion that compensation rules could be given, on the understanding that the courts would simply deviate from them in so far as they seemed to go against the Constitution.\footnote{\cite[136-137]{nut69}.}

Following this initial clarification, the Husaas committee formulated an overarching principle that has since become crucial, namely that owners are only entitled to compensation based on a ``foreseeable'' use of their property. The committee argued that this was an interpretation of ``full compensation'' that was already largely entrenched in case law, which should now be explicitly encoded in statute.\footcite[134]{nut69}

The forseeability test was taken to also imply the no-scheme principle. In particular, it was assumed that the assessment of foreseeability would be made independently of the scheme underlying expropriation.\footnote{\cite[142]{nut69}. Since this assumption was made quite generally, it also corresponds to a broader view on the no-scheme principle than that endorsed by Castberg, see \cite[268]{castberg64b}.} Hence, it was no longer only the special want of the expropriating party that should not be taken into account -- the entire scheme should be disregarded.\footnote{\cite[142]{nut69}.}

This view was rejected by the government. Interestingly, it was considered too owner-friendly, as it was felt that it did not sufficiently limit the possibility for compensation based on non-actual property uses. As a result, the Ministry of Justice proposed an Act to parliament that deviated quite significantly from the proposals of the Husaas committee.

Instead of encoding what was perceived to be mostly existing principles, the Ministry pushed through a more aggressive reform whereby compensation would normally be based on the value of the \emph{current use} of the property.\footnote{\cite[19-20]{otprp70}.} The main argument in favour of this was that the public should not have to pay a financial premium to owners based on possible future uses that would not be permitted unless the public was willing to grant their permission.\footnote{\cite[17-20]{otprp70}.} 

The Ministry set up two exceptions to the current use rule, based on fairness and constitutional considerations. The first, which received by far the most attention, was based on a desire to ensure some degree of equality between owners.\footcite[19]{otprp70} This exception stipulated that the appraisal courts should be free to deviate from the current use rule in so far as it felt that it was reasonable to do so in order to prevent affected owners from being overly disadvantaged compared to other owners, not affected by expropriation.\footnote{This principle was eventually encoded in section 5, no 1-3 of the \cite{ca73}. It would prove highly controversial, since it was only formulated as rule that ``could'' be used to increase the compensation. In \emph{Kløfta}, the Supreme Court eventually deviated from this and overruled the Act by making clear that additional compensation was \emph{obligatory} in a range of cases when the intention had clearly been that the rule should be used sparingly. In this way, and possibly inadvertently, the Supreme Court ended up defending owners' interest by \emph{limiting} the power of the appraisal courts.}

The second exception received far less attention, but in my opinion it is the more interesting of the two. This exception was motivated by a desire to ensure equality between the taker and the owner, particularly in so far as the taker could not be regarded merely as an embodiment of public values. Importantly, the rule sought to address precisely the situation that arises when the taker benefits commercially from the expropriation. In the words of the Ministry:\footcite[19]{otprp70}

\begin{quote}
The second modification we make has to do with the relationship between the property owner and the expropriating party. If the use of the property that the expropriation presupposes gives the property a value that is significantly higher than the value suggested by current use, this will entail a transfer of value from the property owner to the acquiring party. In some cases this might be unreasonable. As an example of when this can become an issue, we mention an agricultural property that is expropriation for the purposes of industrial production. In such a case it might be natural that the owner receives a certain share in the increased value that the new use of the property will lead to.[...]

To establish a flexible system, the Ministry has concluded that it is practical that the King gives rules concerning the cases where an enhanced compensation payment, based on these principles, might be appropriate. This should not be decided by individual assessment, but governed by rules for special case types. Hence, the proposed Act states that the King can pass regulation concerning this matter.
\end{quote}

This quote went right to the heart of the benefit sharing problem in economic development takings, and it proposed a possible remedy. However, the Ministry took the view that this remedy should {\it not} be administered by the appraisal courts, but should be left in the hands of the executive. Hence, there was reason to worry that the suggestion would not work well in practice. 

Indeed, this particular aspect of the 1973 Act was largely overlooked and forgotten and no rules such as those proposed by the Ministry were ever introduced. In general, procedural and contextual aspects seem to have been overlooked by those pushing for the 1973 Act. Since the appraisal courts were regarded as compensating owners too generously, their freedom of discretion was seen as a weakness rather than a strength.

I think this is regrettable. If the new Act had been more temperate in its approach, by encouraging the appraisal courts to take a broader view on fairness, it might have been a success. Instead, it caused an outcry, with attention shifting away from practical matters towards doctrinal issues. The primary such issue, and the most serious one, concerned the question of whether the Act as such was in breach of the Constitution. This question was considered by the Supreme Court in the case of \emph{Kløfta} in 1976.\footnote{\cite{klofta76} (Kløfta).}

Here the 1973 Act would be significantly reinterpreted to make it appear less offensive to the constitutional standard of full compensation. Essentially, it was held that the appraisal courts sometimes {\it had to} deviate from the current use rule, provided certain conditions were met. At the same time, however, the Supreme Court largely accepted the rationale behind the Act and agreed that appraisal practice needed to be adjusted accordingly. Moreover, in implementing the adjustment in practice, the Court arguably also contributed to further undermining the appraisal courts.

Not only were these courts now constrained by an Act that seemed to go against the Constitution, they were also ordered from above to openly deviate from its exact wording. However, they were told only to do so in a select group of cases meeting certain pre-defined criteria. In essence, the Supreme Court itself assumed greater control over how compensation law was to be applied, no longer merely in broad strokes, but increasingly also by developing special rules for specific case types.\footnote{The clearest indication of this shift is found in recent case law wherein the Supreme Court has provided a myriad of detailed rules and directions regarding how appraisal courts should decide on the thorny issue of whether to consider public plans binding for the compensation award or to disregard them under a no-scheme rule. See generally \cite[7-9]{nou03}.}

Following the decision in {\it Kløfta}, the Expropriation Compensation Act 1984 was introduced, which is still in force today. This Act reverted back to the ``foreseeability'' test proposed by the Husaas committee. In general, compensation is to based either on the value of use or the value of sale of the property, whichever is highest.\footcite[4]{ca84} The Act regulates in further detail how the assessment is to be carried out, also by explicitly introducing disregard rules that encode various aspects of the no-scheme principle.\footcite[5-6]{ca84}

The guiding idea is that the value of the property is calculated based on a use of the property that is foreseeable and natural given the surrounding conditions. In relation to the value of sale, there is an additional requirement, namely that the use must be one that an ``average'' buyer would be likely to make of the property. Hence, the value of sale should be set as a general market value, not a value arising from selling the property to a specially interested party.

The extent to which the foreseeability requirement entails that the use in question has to be in accordance with current land use plans is particularly thorny. In general, the Norwegian system stands out because plans are usually {\it not} disregarded, even in situations when they are closely related to, or directly authorise, the expropriation scheme.\footcite[7-9]{nou03} Importantly, this approach is usually to the {\it disadvantage} of the owner, since the no-scheme principle still applies in a way that prevents benefit sharing through compensation. In practice, when land use plans are not disregarded, the compensation award tends to be based on current use assessments, just like the 1973 Act established as the general rule. Hence, in Norwegian law there is a clear asymmetry between the negative and positive aspect of the no-scheme principle. In general, the positive aspect -- leading to increased compensation -- is applied narrowly, while the negative aspect -- leading to reduced compensation -- is applied broadly.

\noo{
has been disputed. In general, compensation is based on uses permitted by public plans, but there are some exceptions applied in cases when the plan itself forms the basis of expropriation.\footnote{In Norwegian law, whether a use is foreseeable is an ``either/or'' question. No compensation is given to reflect what UK lawyers refer to as ``hope'' value, namely the part of a property's value that depends on the perceived likelihood of a change in planning status and future possibilities. If a permission for future use is deemed likely, it is subsequently regarded as a certainty for purposes of compensation, although the present-day value of a future possibility is usually calculated in a way that takes interest and inflation into account. Similarly, if a future possibility is deemed unlikely, no compensation is paid for it whatsoever.} In this way, the law often dictates outcomes that are quite close to those that would follow under a current use rule, but without having to state outright that such a rule exists. In particular,
}

It is not surprising, then, that tensions and disputes often arise in relation to the foreseeability test, or else in relation to one of the disregard rules that encode aspects of the no-scheme principle.
\noo{
These rules, in particular, tend to be insufficient to resolve cases when the nature of the ``scheme'' is not clear, or when it is not clear what should be taken to fall under the disregard because it is due to the scheme. In this way, the legal landscape resembles that of the UK, although there is a marked tendency (to the owners' disadvantage) of regarding the use prescribed by public plans to be binding on the compensation estimation.
}
Moreover, following {\it Kløfta}, there has been a growing expectation that hard cases should be resolved by crisp rules. Appraisal courts are no longer considered free to assess cases directly against the Constitution. Rather, an ethos has taken hold where the need to curb the freedom of appraisers, in the interest of ensuring predictability and centralized control, is usually emphasized.

As a result, difficult cases now routinely end up in the Supreme Court who often censor the appraisal courts. More and more specific rules for special case types are produced. This is particularly marked in relation to the question of how to deal with land use plans that give rise to expropriation. 
Currently, it is thought that an exception to the main rule applies to certain kinds of roads and public buildings, so that a plan designating a property to one of these uses should be disregarded for compensation purposes.\footnote{In practice, this means that the appraisal court is free to compensate the owner on the basis that some other use would have been foreseeable in the absence of the expropriation, e.g., housing uses.} This is consistent with the no-scheme principle, since the uses in question tend to presuppose expropriation, so that the plans themselves must be said to be part of the expropriation scheme.

But where should the line be drawn? The Supreme Court now largely takes it upon itself to shape the law in this area, down to a questionable level of detail. The two most recent decisions stipulate that a plan for a sports hall is to be disregarded, while a plan for a public footpath is not. The distinction between the two seems rather arbitrary, however, particularly as this now becomes a precedent, so that it will tend to apply to footpaths and sports halls in general.

In my opinion, this illustrates how the development of compensation law towards greater reliance on static rules in place of concrete assessment. It also threatens to undermine the idea behind the special procedure used to decide appraisal disputes, which has a long history in Norwegian law.\footnote{One might ask if it has status of constitutional customary law, especially since it concerns the mechanism by which a constitutional rule is meant to be upheld.} Moreover, I think it underestimates the extent to which compensation rules, when applied to concrete cases, must and should be interpreted based on the context of the case.

In the following section, I will turn to waterfalls and hydropower. Interestingly, the compensation practices here often deviate significantly from those observed in relation to other kinds of development. In particular, a more traditional approach continues to dominate these cases, since the rules enacted through legislation do not readily apply. However, I show that while the traditional method of compensating waterfalls is based on benefit sharing, it has become highly standardized and artificial over time. Today, it is perceived as deeply unjust by owners who may be deprived of extremely valuable natural resources and paid only a fraction of their true value.\footnote{See generally \cite{sofienlund07}.}

This points to the inherent difficulty of attempting to ensure reasonable benefit sharing through compensation. Looking at compensation rules as they are applied to waterfalls today further underscores this, as the courts now grapple with the disintegration of the traditional method, and the question of what should replace it.

\section{Compensation for Waterfalls}\label{sec:waterfalls}

Following the introduction of a general expropriation authority covering waterfalls in the early 20th century, the question of how to value waterfalls came before the appraisal courts. The regulatory regime that was established made private commercial development difficult or impossible, and this in turn meant that the commercial market for waterfalls all but disappeared. Hence, a strict application of the no-scheme rule could lead to no compensation being paid at all. In practice, a waterfall would usually have little or no value to anyone except the acquiring authority, since no alternative development scheme could be regarded as foreseeable.

The appraisal courts did not follow this point of view to its logical conclusion. Instead, they introduced a theoretical formula for calculating waterfall compensation. In effect, this method served to create an artificial market for waterfalls, controlled by the appraisal courts. Moreover, it was decided that additional benefit sharing should be ensured by a rule stipulating that the expropriating party should always pay a 20 \% premium in hydropower cases.\footnote{See \cite[16]{wra17}.} Hence, the traditional method for compensating waterfalls in Norway is based on ideas that are very similar to those that are now considered in the US debate on economic development takings, cf, section \ref{sec:back}. Moreover, these rules have been in place for almost 100 years in Norway, giving us a good basis for assessing how well they work in practice.

In the next subsection I present the traditional method in more detail, before I present more recent developments, whereby the liberalization of the energy sector has caused a revision of established compensation practices.

\subsection{A Case Study of Compensatory Benefit Sharing: The Natural Horsepower Method}\label{sec:nathp}

Initially, the artificial market created to compensate waterfall owners was modelled on the actual market that had existed prior to the regulatory reforms of the early 1900s.\footnote{It should be noted that these reforms did not dismantle the actual market for waterfalls overnight, but they gave State and municipality companies so many advantages over other actors, as well as owners, that the market was no longer sustained except through the compensation practices of the appraisal courts. The waterfall ``market'' would slide further and further into the legal sphere, away from the physical and commercial reality of hydropower development. The ``market price'' for waterfalls increasingly came to mean simply the prices paid in recent appraisal disputes.} The key notion used to determine the price of a waterfall on this market was that of a {\it natural horsepower}, a gross measure of electric effect.\footnote{A horsepower is an old-fashioned unit of effect which is still sometimes used, e.g., in relation to cars. In the context of electricity, it has been largely replaced by {\it Watts}.} The lack of a national grid at this time meant that the value of a hydropower plant was largely determined by the stable effect that the plant could deliver, not the total amount of electricity that could be produced.

The notion of natural horsepower was originally introduced to simplify calculations, as a gross estimate of the stable effect that the plant could deliver. The value of the waterfall itself was then determined by fixing a price per natural horsepower. This price was set on the basis of prices paid for other waterfalls, with some adjustments typically carried out to take into account the level of cost and benefits associated with the hydropower project in question.

The use made of the natural horsepower method to calculate compensation for waterfalls had no legislative basis, but arose as a result of the appraisal courts' efforts to calculate market prices. After the actual market based on the natural horsepower method disappeared, the method stuck and was applied customarily.\footnote{See generally the description of the history of the method given by the Supreme Court in \cite{uleberg08}.} 

In the standard account of the natural horsepower method, it is often said that the number of natural horsepower in a waterfall is a measure of gross effect, giving us the amount of ``raw'' power in the waterfall.\footnote{See \cite{vislie02}.} This is not accurate, as the natural horsepower ``of the waterfall'' actually depends crucially on what kind of development project the expropriating party proposes.

Hence, it is far more accurate to speak of the natural horespower of the development scheme benefiting from expropriation. Moreover, the natural horsepower of a development project is a measure of gross stable effect, so it also depends crucially on the nature of the proposed watercourse regulation.\footnote{Regulation of a watercourse can involve building a reservoir and/or installations that transfer water from one river to another. Then, if there is excess water, for instance due to flooding, water can be stored in the dam for later use. When there is no drought, the stored water can be released. In this way, it becomes possible to even out the water-flow over the year.} Today, however, many hydropower plants, particularly smaller ones, involve little or no regulation. Instead, such run-of-river scheme operate by harnessing energy from whatever water is present in the river at any given time. For these projects, the natural horsepower can be zero or close to zero, depending on what formula is used when performing the necessary calculations.\footnote{See generally \cite{sofienlund07}.}

This means, moreover, that the natural horsepower of a development scheme often has little or no bearing on the amount of energy that will actually be harnessed from the hydropower plant. Moreover, the annual income of a hydroelectric plant no longer has anything whatsoever to do with its natural horsepower.\footnote{See \cite{sofienlund07}.} The income is solely a function of the price paid per kilowatthour and the total number of kilowatthours harnessed over the year (kWh/year). Today,  energy producers gets paid for the amount of energy they can deliver, \emph{not} the effect they can maintain in their station over a long duration of time.

Talking of natural horsepower therefore serves to give a skewed picture of the potential of a waterfall, especially for run-of-river projects.
Within the ranks of the specialized water authorities, the inadequacies of the notion has long been common knowledge. The first statement I can find to this effect -- made by the director of the water authorities -- dates all the way back to 1956, from an internal newsletter published by the water directorate.\footnote{See \cite{rogstad56}.} Here it is made clear that compensation practices generally fail to reflect actual values of waterfalls and it is more than suggested that the system operates by exploiting owners' lack of knowledge regarding the true value of the natural resources that they own.

While the idea of compensating the owner of waterfalls by a price per natural horsepower is flawed at the theoretical level, there are even more serious concerns that arise when one begins to consider the way in which the unit price has been determined {\it in practice}. The traditional approach in this regard has had a particularly dramatic effect on the level of compensation payments. 

In case law based on the traditional method, it is often said that the price set per natural horsepower is set according to ``market price'' for waterfalls. But for the most part, what this means is that the court looks to prices awarded in earlier compensation cases. This gives rise to a price level that is entirely artificial. More than anything else, the prices paid reflect the power balance between buyer and seller in the courtroom. This has become very clear after the adoption of new, genuinely market-based, methods in recent years.\footnote{See generally \cite{larsen06,larsen08,larsen12}.}

\noo{ Indeed, while the unit price for a natural horsepower did increase somewhat during the first 80 years that the traditional method was used, this increase neither reflected the value of hydropower in particularly nor the level of inflation in general.\footnote{See \cite{sofienlund07}.}} Moreover, while the price-level was determined by the courts, some voluntary agreements were also made on the basis of the same method. These could then in turn be used to back up the claim that this was a genuine market-based valuation principle. In this way, it became possible to legitimize an increasing imbalance of power between owners and purchasers. In the end, this imbalance became extreme.

For instance, in 2002 a waterfall belonging to local landowners in the rural community of Måren, located in south-western Norway, was sold for the sum of kr 45 000 (roughly £ 4500), based on traditional calculations.\footnote{Source: Private correspondence.} The waterfall has now been exploited in a small-scale hydro-power plant belonging to the large energy company BKK, with annual energy output of 21 GWh.\footnote{\url{http://www.bkk.no/om_oss/anlegg-utbygging/Kraftverk_og_vassdrag/andre-vassdrag/article29899.ece}} For comparison, I mention that in the case of \emph{Sauda}, where a more realistic market-based method was used, the owners received a compensation which totalled about 1 kr/kWh annual production.\footnote{LG-2007-176723.}

Applied to the Måren case, this would take the compensation from kr 45 000 to kr 21 000 000. That is, the price would have been almost 500 times higher.\footnote{In fact, the Måren waterfalls were cheaper to exploit, so in reality, one would expect that the new method applied to Måren would yield even greater compensation per kWh. I also remark that the value awarded in \emph{Sauda} was market-value, not value of use. It was assumed, in particular, that the owners would have to cooperate with a ``professional'' energy company to develop hydropower. This, in effect, halved the compensation awarded, since the Court's decision was based on the premise that the professional company was willing to pay about 50\% of the profit as rent to the owners.}

The case of Måren illustrates an important point, namely that when the traditional method was used, and described as the ``market value'' of waterfalls by the courts, this became a self-fulfilling prophecy. The prices paid in voluntary transactions were influenced by the practice adopted by the courts far more than the other way around. 

This points to a more general mechanism. In particular, when expropriation is widely used for some particular purpose, prices for property that can be used for this purpose can be kept artificially low by developers choosing to make use of expropriation rather than entering into friendly negotiations. 

Instead of competing for a deal with owners, developers can compete to be the first to acquire an expropriation license from the State. In this way, even if the system prescribed ``market-value'' compensation, an artificial price level can be established and maintained through the use of expropriation.\footnote{I mention that in a setting where the owners are politically powerful and can exert undue influence on the compensation process, the effect can be reversed, so that the ``market based'' approach leads to inflated compensation levels, including elements of holdout value. The general point is that the market approach can be turned to the advantage of the most resourceful and powerful groups, particularly in situations when expropriation is widely used for a particular kind of development. In such cases, a market-based approach is not as politically neutral and ``objective'' as its proponents tend to argue.}

In my opinion, preventing such a mechanism from undermining the fairness of the compensation regime is a main challenge associated with regulatory systems that presuppose extensive use of expropriation for economic development. Failure to address this appropriately can create financial incentives for developers to favour expropriation over friendly negotiations or cooperation with owners. In this way, a vicious circle can form, making it hard to break out of the ``expropriation loop''.

\subsection{New Methods for Compensating Waterfalls}\label{sec:new}

In the 1990s, the Norwegian energy sector was liberalized, and the traditional method for compensating owners came under increasing pressure. It was argued to be unjust and illogical by engineers and smaller companies working on developing small-scale hydropower. Eventually, legal professionals followed suit and came to the realization that established compensation rules based on market value should be applied.\footnote{\cite{larsen06}.}

Indeed, a new market for waterfalls had begun to develop at this point, following increased interest in small-scale hydropower, also from new companies specializing in cooperation with local owners. For transactions of rights to waterfalls taking place on this market, the traditional method of valuation was not used. In fact, waterfalls were rarely sold at all, but rather leased to the development company for an annual fee. Typically, this fee was calculated by fixing a percentage of the energy produced during the year, and compensating the owners of the waterfall by multiplying this with the market price for electricity obtained throughout the year, possibly after deducting production specific taxes, but with no deduction of other costs. In effect, owners would get a fee corresponding to a set percentage of annual gross income in the hydro-power plant.\footnote{See \cite{larsen06,sofienlund07}.} Today, such a fee usually entitles the owners to 10-20\% of the gross income, depending on the cost of the project. \noo{Moreover, it is common that the owners are entitled to up to 50\% of the income derived from so-called \emph{green certificates}, a support mechanism for new renewable energy projects, corresponding to the Renewables Obligation in the UK.\footnote{See http://www.ofgem.gov.uk/Sustainability/Environment/RenewablObl/ for further details.} Such a scheme has been talked about for years, but was only put in force in 2012.\footnote{http://www.regjeringen.no/en/dep/oed/Subject/energy-in-norway/electricity-certificates.html?id=517462} Currently, energy producers can claim a premium of about 2 pp per KWh per year, meaning that about a third of the annual income for new renewable energy projects comes from the sale of green certificates.\footnote{While the premium must be expected to go down somewhat as the certificate market matures and more energy producers acquire ``green'' status, it will certainly remain an important source of extra income for renewable energy producers also in the future.}}

Since leasehold agreements tie compensation to the fate of the hydropower project, several questions arise when attempting to estimate a present-day value of a waterfall on this market. The valuers first have to determine what the most likely project looks like. Then they have to determine what the annual production will be. After this, they must assess the cost of constructing the plant, something that will in turn make it possible to estimate the level of rent likely to be paid to the waterfall owners. Then, since this rent is set as a percentage of the income from sale of electricity and energy certificates, the need arises to stipulate future prices, usually for as long as 40 years (the usual length of a leasehold). Finally, a present-day value can be calculated based on future cash flows.

The appraisal courts began to use such a model around 2005. The first case of this kind to reach the Supreme Court was \emph{Uleberg}. In the appraisal court of appeal, the lay appraisers overruled the juridical judge and awarded compensation based on the new method. The Supreme Court ordered a retrial on a technicality, but it also commented that it supported the adoption of the new method in cases when \emph{alternative} small-scale development was deemed a \emph{foreseeable} use of the waterfall in the absence of the expropriation scheme.\footnote{\cite{uleberg08}.} Since \emph{Uleberg}, the new method has been used in several cases before the appraisal courts.\footnote{See generally \cite{larsen06,larsen08,larsen12}, a series of Norwegian papers discussing the new method.}

It is interesting to note that it was the lay appraisers that pushed for a new method initially, sometimes in opposition to the juridical judge. I believe this shows that the old system of lay judgement in appraisal disputes still plays a role in Norway. Moreover, I also think it demonstrates that the system has positive qualities that should be preserved in the future. However, that is not to say that the new method is without problems.

Unsurprisingly, it tends to lead to a rather protracted process of valuation, mostly dominated by experts. Moreover, given all the uncertain elements of the calculation, it is typical that the opposing parties produce expert witnesses that diverge significantly in their valuations. While this can be problematic, the fundamental \emph{legal} challenge arises with respect to the no-scheme rule. In particular, what hydropower scheme should the compensation be based on? Several questions arise, as listed below.

\begin{itemize}
\item (1) Is it foreseeable that the waterfall could be used in a hydropower project in the absence of a power to expropriate?
\item (2) If the answer to question (1) is yes, what would such a scheme have looked like?
\item (3) Is it foreseeable that the scheme from (2) would obtain the necessary licenses?
\item (4) Does the no-scheme rule imply that no project similar or identical to that benefiting from expropriation can be regarded as foreseeable for the purpose of compensation?
\item (5) Is the fact that the scheme underlying expropriation obtained a development license to be regarded as evidence that alternative (inferior) schemes would not have received such a license?
\item (6) How should compensation be calculated if it is determined that no (alternative) hydropower scheme would have been foreseeable?
\end{itemize}

In some cases, for instance when the project benefiting from expropriation is not commercially viable but is carried out for public purposes with the help of special state funding, the answer to question (1) might be no. However, in most cases, the question will be answered in the affirmative, since the scheme benefiting from expropriation already serves as an indication that the waterfall can be commercially harnessed for energy. However, here the no-scheme rule comes into play and creates severe difficulty once we reach question (2). For what kind of scheme can be assumed foreseeable all the while we are obliged to disregard the scheme underlying expropriation? 

In most cases so far, the owners have claimed that compensation should be based on the value of a small-scale hydropower scheme. Since such a scheme is likely to be clearly distinct from the expropriation scheme (which tends to be large-scale), one might think that the no-scheme rule will not come into play. This, however, is not necessarily the case. It appears, in particular, that the answer to question (3), asking about the likelihood of obtaining licenses, will still depend on how one views the no-scheme rule. It seems, in particular, that anyone who answers question (5) in the affirmative, will be inclined to say that the alternative project could not expect to obtain a development licence.

This is so, such a person might argue, precisely \emph{because} licenses were granted to the expropriating party. This line of reasoning has been consistently advocated by the large energy companies, ever since the new method emerged.\footnote{See, e.g., \cite{klovtveit11,otra10,otra13}. The argument is often sugar-coated by pointing to the reasons underlying the decision to grant a license -- typically energy efficiency -- rather than by focusing on the formal license itself. In this way, one arrives at an interpretation of the no-scheme rule whereby the scheme can perhaps be said to have been disregarded even though one still takes into account reasons why it should be preferred over other schemes.} Is their argument at odds with the no-scheme rule? It would seem so, but remember the earlier discussion on the no-scheme rule in Norwegian law, where I noted that the rule has tended to be applied much more narrowly along its positive dimension. 

Following up on this, it can be argued that while the expropriation scheme is to be disregarded for the purpose of compensation valuation, the land use regulation underlying the scheme -- or at least the rationale behind this regulation -- is nevertheless to be taken into account. If this point of view is adopted, then the conclusion can easily become that alternative development is to be regarded as unforeseeable. This points to the danger that simply abandoning the no-scheme rule might not achieve fairness for owners at all, but create new ways for powerful takers to limit compensation payments.

Indeed, the line of reasoning described above was given a stamp of approval in the recent Supreme Court case of \emph{Otra II}.\footcite{otra13} Here the Court concluded that the development plans of the expropriating party were so superior that alternative development was unforeseeable. Hence, the no-scheme principle was not applied very widely along its positive dimension. In particular, the principle did not prevent the conclusion that the superiority of the expropriation scheme meant that alternatives would have been unforeseeable.

After concluding in this way, the Court needed to answer question (6) by coming up with some alternative way of compensating the owners.  To do so, the Court was again faced with considering the implications of the expropriation scheme. One possibility would be to simply ignore the no-scheme principle completely. Indeed, as the superiority of the expropriation scheme was used to rule out alternatives, it seemed natural to use it also as the basis for valuation. This is what the appraisal court of appeal had done, and the Supreme Court sanctioned the approach.

But at this point, the adherence to a ``market-value'' approach spelled doom for the waterfall owners. The presiding judge, in particular, reasoned as follows:\footcite[]{otra13}

\begin{quote}
Based on the arguments presented to the Supreme Court, I find it safe to assume that there does not today exist any market for the sale and leasing of waterfalls for which alternative development is not foreseeable, but where the waterfalls can be used in more complex hydropower schemes. The appellants have not been able to produce documents or prices to document the existence of such a market.
\end{quote}

The implicit assumption is that in order to value the waterfall according to its potential for hydropower production, a market needs to be identified. It was \emph{not} considered sufficient that the scheme for which expropriation took place was itself a commercial project. It is very hard to imagine how a market of the kind asked for here could ever develop. After all, alternative buyers (the existence of whom had been documented) were all excluded from consideration because their development schemes were held to be unforeseeable. Hence, if there was to be a market, it would have to be one that emerged entirely due to the benevolence of the expropriating party.

Taken to its logical conclusion, the line of reasoning in {\it Otra II} leads to an offensive result: The commercial value of the property taken should not be compensated {\it because} the optimal commercial use is the use that the expropriating party aims to make of it. Unsurprisingly, the Supreme Court shunned away from explicitly endorsing this conclusion. Instead, it sanctioned compensation based on benefit sharing, using the natural horsepower method.

Leaving aside questions about the Court's application of the foreseeability test and the no-scheme principle, was it appropriate to prescribe the traditional method? In light of the evidence that has emerged regarding its shortcomings, this seems highly questionable. But was there a better alternative?

Towards an answer, let us follow Lord Nicholls in avoiding talk of schemes and reason instead on the basis of what would have happened in the absence of a power to expropriate. In cases such as {\it Otra II}, it seems likely that a scheme corresponding closely to that underlying expropriation would still be implemented. This scheme, however, would then have to be carried out on the basis of {\it actual} cooperation and benefit sharing with owners.

If we think like this, focus shifts from theoretical benefit sharing imposed through compensation, towards trying to the determine -- under the foreseeability test -- what would have been the actual benefit sharing in a joint enterprise. This subtle change of perspective could make a dramatic difference, as it would place us in a position to reject the natural horsepower method, legal tradition notwithstanding. Instead, we could now anchor the compensation assessment in a concrete, albeit counterfactual, scenario.

In \emph{Otra II}, this line of thought was not considered in any depth, but mentioned briefly and rejected.\footnote{Although this may in part have been due to the fact that the point was not raised  before the court of appeal.} However, in the Supreme Court case of \emph{Kløvtveit} -- in circumstances similar to that of {\it Otra II} -- such an argument succeeded.\footcite{klovtveit11} Here too, alternative development was not foreseeable, but unlike in \emph{Otra II}, the lay appraisers in the court of appeal decided to compensate the owners based on the premise that it was foreseeable that they would have cooperated with the expropriating party in the absence of the expropriation scheme. 

That is, the appraisal court of appeal held that the waterfalls would have been exploited in exactly the same way, except through cooperation, not expropriation.\footnote{Hence, the court effectively adopted a compensation approach based on the same conceptual premise as that of Lehavi and Licht's SPDC proposal, as discussed in Section \ref{sec:back}.} Crucially, this also meant that the court was free to replace the natural horsepower method by what they thought the parties would actually have done to set up a cooperation project. Here the appraisal court found that a leasehold model would have been likely, since this was the model generally used on the waterfall market.

I think the approach of {\it Kløvtveit} is far superior to that of {\it Otra II}. For commercial projects, it seems that in the absence of a power to expropriate, any rational buyer would look to cooperate with the owners.\footnote{This would not necessarily be a safe assumption to make for non-commercial projects. Such projects may fail to provide the necessary incentives for cooperation, even though they should nevertheless be carried out in the public interest.} Hence, the foreseeability test itself may be applied, an approach that will naturally lead to the rejection of the natural horsepower method in favour of more realistic forms of benefit sharing. 

I mention that \emph{Kløvtveit} was discussed in \emph{Otra II}. But the presiding judge chose to focus on what he regarded as the ``practical problems'' associated with the prospect of cooperation and a compensation award calculated on this premise. The cooperation model was not the center of attention in the case, however, so one can only hope that \emph{Kløvtveit}, rather than \emph{Otra II}, will become the influential precedent for future cases.

However, the eventual outcome of {\it Kløvtveit} serves as a warning that the solution offered here is very frail. In particular, the expropriating party in {\it Kløvtveit} successfully argued that the following sinister question had to be put to the appraisal court: If the owners had decided to cooperate with the expropriating party, when exactly would this joint endeavour have applied for a development licence? It was held that the no-scheme rule applied here, so that the actual time-line of the expropriation project fell to be disregarded. But could it then still be regarded as foreseeable that a joint application for development would have been successful, at a later point in time? As it turned out, the answer to this question, based on a concrete assessment, was that a licence would probably not have been granted. Hence, the cooperation project fell to be disregarded as unforeseeable after all. The case therefore conclude just as the {\it Otra II}, with compensation awarded according to the natural horsepower method.

Indeed, every single case where the Supreme Court has commented that the traditional method may be abandoned when alternative developments are foreseeable, have been decided on the basis that alternatives are {\it not} foreseeable. In my opinion, this serves to illustrate the frailty and instability of a system that attempts to ensure benefit sharing through complicated counterfactual assessments of what {\it would have happened} if an expropriation scheme had not existed. 

The pitfalls are many, and the room for creative arguments by resourceful parties is very great. So great, in fact, that the goal of achieving meaningful benefit sharing in hydropower cases seems rather remote in many, if not most, concrete cases. In the end, the new methods that have been introduced might amount to little more than slight variations of an inherently flawed idea of attempting to ensure fairness through compensation.

\section{Conclusion and Future Work}\label{sec:conc}

In this article, I have explored the possibility of enhancing the legitimacy of economic development takings by establishing compensation practices that ensure better benefit sharing with owners. I started with a brief presentation of the US debate, where this suggestion has received theoretical attention. I noted that from a practical point of view, one of the major obstacle to benefit sharing through compensation appears to be the no-scheme principle, whereby changes in value due to the expropriation scheme are not supposed to influence compensation awards.

Following up on this, I took a closer look at the no-scheme principle, based on recent case law from the UK. I argued that the principle -- as it is understood there -- does not necessarily stand in the way of benefit sharing, at least not as long as the commercial potential of the land that is taken can be said to pre-exist the development scheme that unlocks it. I argued that despite appearances, this question is not primarily a question of fact, but a much deeper question of what meaning property is taken to have within society and within the regulatory framework set up to regulate (commercial) land use. 

I argued that while this is a politically sensitive question that should normally be left to political decision-makers, some situations might call for a more principled approach in light of constitutional and human rights perspectives. I argued that such a situation arises when the regulatory system presupposes commercial exploitation of a development potential, but acts in such a way that the potential is taken from the land owners and handed over to some external commercial entity. Here, the lack of legitimate reasons for denying benefit sharing is acute, suggesting a critical look at established compensation practices.

Following up on this, I went on to consider Norwegian expropriation law. I first gave a general overview, focusing on the no-scheme principle, before I turned to the case of waterfall expropriation, for which the principle has never been understood as a hindrance to benefit sharing. I noted, however, how the traditional approach to waterfall expropriations has gradually become more and more unsatisfactory, causing great tension and a recent revision of established compensation practices. I analysed these developments and concluded that although it is now recognized -- in principle -- that the loss of a hydropower potential should be compensated, this is hard to implement in practice. This, I argued, points towards the inherent inadequacy of a system that attempts to ensure benefit sharing through compensation rather than actual participation. 

I think my conclusion points towards interesting avenues for future work. First, I note that my analysis can serve as an argument in favour of looking at alternatives to expropriation in cases involving economic development. This is a perspective that has already emerged in the US debate, particularly through the work of Heller and Hills, who have proposed a novel institution for collective action -- the {\it land assembly district} -- that they think can obviate the need for expropriation in many cases when economic development is desired by the public.\footcite{heller08}

I think this kind of work points to the future in the debate on economic development takings. Second, I would like to briefly mention that Norway has an entire legal framework in place that can shed further light also on this proposal. I am referring here to the system of so-called {\it land consolidation courts}, special tribunals that are empowered to organize development projects involving fragmented property rights.\footnote{\cite{lca79}.} Importantly, the legal framework leaves room for {\it compelling} owners to cooperate and participate, in some cases also with external commercial actors. At the same time, it is a fundamental principle of land consolidation that measures can only be ordered if they are beneficial to all the owners and properties involved.\footnote{See generally \cite{ravno08}.}

In relation to hydropower development, the framework is already being put to use in an increasing number of cases.\footnote{\cite{stokstad11}.} Moreover, there are forces in the Norwegian government that are pushing for land consolidation as an alternative to expropriation more generally.\footnote{A new Act on land consolidation will take effect from 2016, and in this Act any party who is authorised to expropriate property is also authorised -- as an alternative or complementary measure -- to initialise and act as a party to a land consolidation dispute that seeks to organize the development project.} In my opinion, this points towards the future, as it promises to provide a highly flexible approach for dealing with property and economic development using varying degrees of compulsion. Moreover, it provides an interesting case study against which to judge other proposals, such as that put forth by Heller and Hills. A further study of this, however, must be left for a another article.

\end{document}


 This is then also an argument in support of the approach taken by Heller and Hills, in their article on so-called land assembly districts as an alternative to expropriation in economic development cases. 



to work 

approach taken by Hiller and Hills in the US takings debate, in as much as they propose exactly the kind of collec




due to many ways in which the compensatory approach opens up for arguments based on counterfactual scenarios that are hard to judge accurately.


.





 becomes particularly acute, and the need for ensuring 


it seems unwarranted to say that the value is inherent in the scheme, not the property. This, in effect, would amount to 


. The act of expropriation, in these cases, serves to transfer commercial potentials from the owner to another commercial act

 and the act of expropriating it thereby serves to bestow this potential on someone other than the owner. In this case, a lack of benefit sharing would aserves to redistribute this commercial potential to a commercial actor. 


the when a natural resource is expropriated.

s clearly pre-existent, and is transferred from the owner to some commercial entity

potentials one is ready to include 
in the owners' bundle of rights. 


 how one drawn the line between values that are created by society and values that are inherent in property itself.

that a well-functioning compensation regime will enhance the legitimacy of economic development takings.

In this article, I have discussed the issue of compensation for  economic developmment takngsIn this article, I have discussed how compensation should be calculated when commercial companies benefit from the property that is taken. I focused particularly on the no-scheme principle, which plays an important role in this regard in many jurisdictions, including in Norway. Moreover, I positioned my discussion relative to the US debate on legitimacy of takings, by asking about the feasibility of addressing this issue through reforms in relation to compensation assessment. 

I used recent developments in the UK to argue that the no-scheme principle might not be an insurmountable obstacle to such reforms, and then I looked to Norwegian law to consider the question of how an approach based on benefit sharing might play out in practice. I also noted a particular procedural feature of Norwegian appraisal courts, whereby they typically operate largely unconstrained by specific evaluation rules. I argued that this system was both flexible and capable of facilitating broad fairness considerations.

I then moved on to consider the case of waterfalls in more depth. I noted how the flexibility of the Norwegian system was used to great effect here, to ensure exactly the kind of benefit sharing that is now being considered by some US scholars. For over 80 years, the Norwegian courts happily deviated from the no-scheme rule entirely when awarding compensation for waterfalls.

Remarkably, this practice continued even after legislation was passed that provided much more specific guidelines to the appraisal courts, and which seemingly enforced a strict no-scheme principle in Norwegian law.\footnote{More generally, however, I noted how this legislation, and the Constitutional battles that followed it, has lead to a development whereby the appraisers are somewhat marginalized and the Supreme Court itself has assumed greater power in directing them, by providing their own interpretation of a body of legislation that contains many specific rules that are hard to apply to concrete cases in a uniform fashion.}

However, as the appraisal courts were marginalized by increasing levels of top-down control, first by the legislator and later by the Supreme Court, the method that was developed to compensate waterfalls would itself develop into a fixed and rigid rule. It was not adapted, in particular, to reflect technological and economic progress. Since these were particularly rapid and ground-breaking in the energy sector, the result was a very severe mismatch between the real value of waterfalls and the compensation paid following expropriation. 

In the final part of the article I then considered recent cases where the traditional method has been abandoned in favour of a market-based approach which is based on the general rules governing compensation today. I found that while this approach will in principle lead to payments that more closely reflect actual commercial values, severe problems arise in practice. Here, in particular, the no-scheme principle re-emerges on the scene with full force, becoming a very effective tool for those who seek to argue that hydropower development is not a fruit of property but belong to those who happen to obtain a license to expropriate.

If such arguments are successful, the market-value approach can lead to worse outcomes for local owners of waterfalls than what they would be entitled to under the traditional method. The deeper question that arises, of course, is the following: What is equitable benefit sharing in these cases, and how can it be ensured? 

After considering the Norwegian situation, it is my opinion that the best way to ensure benefit sharing under a compensatory approach is to revive the old system of an independent appraisal procedure relying on the discretion of lay people from the local area. The most important aspect of this, I believe, is that it enhances the democratic legitimacy of the compensatory approach. It is clear that there is a great deal of uncertainty in the kinds of calculations one must engage in to assess the commercial value of a waterfall.

Therefore, the temptation to rely blindly on experts and special rules that are not properly understood becomes great. This might reduce the uncertainty involved, but only to some extent. Moreover it also highly increased the risk of unfairness and opens up the possibility that powerful interests can surreptitiously usurp the procedure for their own interests. Compared to this, a system based on direct fairness assessments carried out by normal people, on the basis of (hopefully) neutral information provided by experts, might well be the best option.

However, I think the inherent difficulty in devising appropriate compensation mechanisms for commercial potentials suggest that the compensatory approach might be misguided altogether. In addition, as soon as one begins to look at the social function of property, and its role in human flourishing, it seems that any kind of financial compensation is going to provide an inadequate reply to deprivation of highly valuable property and natural resources. In light of this, I think it is appropriate to consider alternatives to expropriation in cases when economic rationales dictate economic development. I remark that this is a perspective that has also emerged in the US debate, particularly through the work of Heller and Hills, who have proposed a novel institution for collective action -- the {\it land assembly district} -- that is meant to obviate the need for expropriation in many cases of economic development.

I believe this is the way forward, and I will therefore finish also by pointing to a much more positive aspect of Norwegian law then that which I have considered so far. It is interesting, in particular, that Norway has an entire legal framework in place that can elegantly facilitate a shift away from expropriation towards collective action, also in cases when the latter requires the use of compulsion against recalcitrant owners. This is the system of {\it land consolidation courts}, which are empowered to organize development projects involving fragmented property rights. Importantly, the legal framework leaves room for {\it compelling} owners to participate, even if they are not inclined to do so. At the same time, it is a fundamental principle of land consolidation that it should leave all owners better off after the proceedings have concluded. 

In the case of hydropower development, the framework is already being put to the test in an increasing number of cases for substantial development projects, as local developers tend to shun away from outright expropriation of property belonging to unwilling neighbours. The land consolidation mechanisms that can be used to facilitate compulsory development in these situations form part of an ancient semi-juridical system of land management in Norway. In my opinion, this framework also points towards the future, as it provides a highly flexible approach for dealing with property and economic development under varying degrees of compulsion. This, however, is a topic for a future article.

\printbibliography

\end{document}


Instead, the Court states that a return to 

the traditional method is in order, meaning that they explicitly endorse a return to a method that seems completely at odds both with the reality of hydropower development and the body of legislation that has been introduced for compensation estimation after the natural horsepower method was introduced. But is there an


 However, they do not apply it in the traditional way. Rather, they sanction a modified version of it that moves away from compensation based on the level of stable effect towards compensation based on average effect.\footnote{That is, they replace the low water-flow by the average water-flow in the definition of Qref, c.f., Section \ref{sec:nathp}.}In addition, they also sanctioned the use of a significantly increased unit price compared to earlier times.

What to make of this? In fact, it seems hard indeed to make sense of since, effectively, by relying on the traditional method, the Supreme Court contradicts its own conclusion that compensations should be based on market value. Instead, they rely on a method that, in effect, is based on an attempt to quantify the value of the waterfall as it is being used by the expropriating party in his project. However, by relying on a technical method that has been completely outdated, it becomes difficult to assess the outcome properly, at least for a non-expert. This is so even after the modifications have been implemented, which make the method appear somewhat less irrational from a physical point of view.

But it is still noteworthy that the Supreme Court prefers the obscurity of the traditional method, as an established custom, over the explicit conclusion that it simply is not tenable to adopt the ``value to the owner'' principle in cases like this, as least not as that principle is construed in Norwegian law.

In any event, I think there is good reason to be critical of the Supreme Court for sanctioning the view that alternative development was unforeseeable in {\it Otra II}. Still, it is not possible to escape the fact that this reflects a general tendency in Norwegian law, whereby the positive dimension of the no-scheme rule is much weaker than the negative part. Even if it appears unreasonable, it might very well be a correct application of national law. Moreover, it could very well have been that alternative development was unforeseeable for \emph{some other reason}, for instance because the only commercially viable exploitation was the scheme planned by the expropriating party. In this case, the problem of how to compensate the owners in the absence of an alternative form of exploitation would still arise. It is this question, in particular, which seems entirely unsatisfactorily resolved under an application of a ``value to the owner'' principle.

This is witnessed by \emph{Otra II}, and, in fact, it appears that the Supreme Court, in their decision  \emph{not} to follow their own reasoning to its logical consequence, makes quite a powerful statement. For all intents and purposes, the Supreme Court \emph{rejects} the "value to the owner" principle, but they obscure this by wrapping it up in the traditional method, which is deeply flawed. However, the problem it attempts to solve appears significant, and it pertains directly to the question discussed more generally in Section \ref{sec:noscheme}, namely how to compensate owners that loose their land to commercial schemes. 

It seems that even the fiercest supporters of limiting owners' right to compensation tend to find it too offensive to apply this principle when it leaves the owners with no form of compensation in cases when they are forced to give up property to purely commercial undertakings. Indeed, such a practice would surely also be in breach of the human rights law. In these cases, the subjective aspect of the ``value to the owner'' principle is impossible to maintain. If the commercial value falls to be disregarded for no other reason than the fact that the state happens to have granted planning permission to the expropriating party rather than the owner, this is not only dubious with respect to human rights protecting property, but also appears to be a case of \emph{discrimination}, e.g., as prohibited by ECHR Article 14.

The problem does not arise when the buyer sees value in the property that is of a different \emph{kind} than that realizable by \emph{any} private owner. In this case, the rule simply states that the owner should not be able to demand that ``public value'' is transformed into commercial value just for him. This appears like a reasonable principle. But when there is commercial value already present on the "public" side of the transaction, it seems completely unwarranted that the public should be allowed to transfer this value from the owner to someone else without compensation. Thus, it seems that more accurately and acceptably, the ``value to the owner'' principle should be thought of as a ``commercial value'' principle. It seems, in particular, that the principle need to be stripped of any suggestion that a preferential financial position is to be awarded to whoever benefits from expropriation.\footnote{Exceptions might be possible to imagine, but, one would think, only when they can be construed as falling under the ``public value'' banner in some way.}

It seems unfortunate that this aspect has not been made explicit, and the difficulties that arise in the absence of this nuance are nicely illustrated by the case of Norwegian waterfalls. Still, as the case of \emph{Otra II} indicates, an interpretation of the ``value to the owner'' principle along less offensive lines is in reality already in place with regards to Norwegian hydro-power. Here it seems that ``value to the owner'' has in fact \emph{never} been applied in the traditional way. Hopefully, rather than obscuring this fact by relying on an unsatisfactory and artificial method for calculating the compensation, the future will see further developments that recognize the need for new principles. 

It should be recognized, in particular, that as the law has been applied for the last 80 years, despite its grave flaws and injustices, there has always been an implicit recognition in Norwegian law that the owners of waterfalls are \emph{entitled to their share} of the commercial benefits of hydropower. 



In this chapter, I provide a bird's eye view on my topic, by placing it in the theoretical landscape. My aim is to explain the key concepts that I will rely on to make sense of the empirical data considered in subsequent chapters. I will also present the main values that I will look to when I give normative assessments. In addition, I will relate my theoretical approach to current strands in property theory, focusing on those aspects of property theorizing that I regard as particularly relevant to the work done in this thesis.

I will strive to show that my approach to the empirical data is sound and informative, while focusing on principles of {\it legal} reasoning. I will not provide an extensive presentation of concepts or theoretical approaches developed in other fields, such as political science, sociology, economy, or psychology. However, I note that all these fields engage in interesting ways with the notion of property, and I think a multi-disciplinary approach can be illuminating.\footnote{For some examples of relevant work from economics, psychology and political science respectively, consider \cite{miceli11,nadler08,katz97,carruthers04}.} Hence, while I focus on legal and --  to some extent -- philosophical theories of property, I will try to make a note of specific questions I consider that are also analysed in related fields.

The crucial argument made in this chapter is that the category of {\it economic development takings} is relevant to legal reasoning about certain kinds of situations when private property is taken by the state. This is not {\it prima facie} clear. In fact, I am prepared to face critics who will argue that the category makes no legal sense at all. Fortunately, it makes perfect intuitive sense; it targets situations when property is, quite literally, taken for economic development. In most cases I will consider, this is even the explicitly stated aim used to justify eminent domain. Hence, the factual basis for our categorization can not be questioned.

The theoretical basis, on the other hand, can not be taken for granted. Indeed, a superficial look at dominant legal approaches to property would seem to indicate that in most property regimes, the nature of the project benefiting from a taking should not be in focus when assessing the legitimacy of interference. Rules aiming to protect property invariably focus on the rights of the affected owner, making clear that she enjoys some degree of protection against uncompensated state interference. But how can we, on this basis, justify having regard to the {\it purpose} of the taking? What bearing does this have for the question of legitimacy with respect to the owner's rights? At first sight, it might seem unwarranted to think that it should matter at all. Are not owners' rights  equally interfered with when property is taken for some uncontroversially public project, like a new public road, compared to the situation when it is taken for economic development? Is it not in fact a little small-minded, even short-sighted, to worry about the taker's gain, instead of concentrating on what the owner, if anything, stands to lose?

Much of the work in this chapter, albeit theoretical, is aimed at countering this very concrete objection. I believe it is important to do so thoroughly, since it is an objection that threatens to undermine the conceptual basis for the kind of study that I present in this thesis. Moreover, it is an objection that I think it is inappropriate to dismiss without further comment. In the context of US law, it might be possible to do so, since economic development takings have, as a matter of fact, gained recognition as an important category of legal reasoning.\footnote{See generally \cite{cohen06,somin07,malloy08}.}  In Europe, however, this has not yet happened, at least not to the same extent.

The reason for this difference is not that US law contains special rules that emphasize the importance of the distinguishing features of economic development takings.\footnote{In fact, many state laws now {\it do} contain such rules, following the backlash of the controversial decision in \cite{kelo05}. However, such rules were introduced only after the category of economic development takings first came to prominence in legal discourse. See generally \cite{eagle08,somin09,jacobs11}.} Rather, the difference must largely be attributed to the fact that economic development takings have resulted in great political controversy in the US, a controversy that has influenced both the law and legal scholars.\footnote{See, e.g., \cite[1190-1192]{somin08}.} Hence, in the absence of a similar political climate in Europe, a conceptual investigation into the very idea of an economic development taking is warranted, if not also required.

As I argue in this chapter, providing an adequate account of such takings forces us to broaden our theoretical outlook compared to most traditional strands of legal reasoning about property. However, I find considerable support for the necessary conceptual reconfiguration when I consider recent trends in property theory, particularly those that focus on the {\it social function} of property.\footnote{See generally \cite{alexander09a,foster11,singer00,underkuffler03,alexander06,alexander10,dagan11}.} Indeed, the crux of the main argument presented in this chapter is that this function allows us, even compels us, to pay attention to the special dynamics of power that tend to manifest in cases when private property is taken by the state for someone else's profit.

Some might argue that the most straightforward way of describing economic development takings is to say simply that they are {\it unfair}. Indeed, this has some merit also as a conceptual position, since it would seem to explain quite effectively why takings for profit have become so unpopular in the US, particularly when people's homes are taken.\footnote{See, e.g., \cite[742-748]{nadler08}.}  However, a more thoughtful assessment reveals that matters are not quite so simple. Indeed, it seems that economic development takings are an almost unavoidable consequence of any system that emphasize both state control over property and public-private partnerships in the economic sector. To condemn this political model of government is a possible response, but not one I will pursue in this thesis. Rather, I will focus on getting to the heart of what characterizes economic development takings, so that I may address also the question of how to deal with them, to ensure that their positive functions can be fulfilled in ways less offending.\footnote{Even those who support an outright ban on economic development takings should be interested in demarcating the category more closely, to arrive at a better understanding of what exactly will be banned.}

Therefore, the stark contrast between the intuitive response that a taking for profit is unfair, and the legal assessment that what matters is only the loss to the owner, needs to be considered further. A tentative first reconciliation can be achieved by arguing that the feeling of unfairness is in itself a loss that the owner incurs, so that the law had better take it into account. But, of course, not any subjective feeling should be given weight in a legal context. The question, therefore, is what exactly the feeling of unfairness can help us uncover about the nature of economic development takings. Does it uncover something legally significant?

In my view, the answer is yes, and in the following I will argue for this position in depth. To motivate this largely abstract analysis, I will begin by considering a concrete scenario which illustrates the need for contextual assessment and more fine-grained conceptual categories for reasoning about cases when demands for economic development come to pose a threat against established patterns of property.

\section{Donald Trump in Scotland}\label{sec:dts}

On the 10th of July 2010, the property magnate Donald Trump opened his first golf-course in Scotland, proudly announcing that it would be the ``best golf-course in the world''.\footnote{http://www.golf.com/courses-and-travel/donald-trump-scotland-golf-course-lives-hype (accessed 06 July 2014).} Impressed with the unspoilt and dramatic seaside landscape of Scotland's east coast, the New Yorker, who made his fortune as a real estate entrepreneur, had decided he wanted to develop a golf course in the village of Balmedie, close to Aberdeen.

To realize his plans, Trump purchased the Menie estate in 2006, with the intention of turning it into a large resort with a five-star hotel, 950 timeshare flats, and two 18-hole golf-courses. The local authorities were not particularly keen on the idea, and planning permission was initially denied by Aberdeenshire Council. Particularly worrying to the councilors was the fact that the proposed site for the development was declared to be of special scientific interest under EU conservation legislation. The frailty and richness of the sand dune ecosystem, in particular, suggested that the land should be left unspoilt for future generations.\footnote{See \url{http://en.wikipedia.org/wiki/Donald_Trump#Scottish_golf_course} (accessed 06 July 2014).} Trump was not deterred, however, and started lobbying Scottish politicians to gain support. In the end, he was able to convince Scottish ministers that he should be given the go-ahead on the prospect of boosting the economy by creating some 6000 new jobs.\footnote{See http://www.theguardian.com/world/2008/nov/04/donald-trump-scottish-golf-course (accessed 06 July 2014).}

Activists continued to fight the development, launching the ``Tripping up Trump'' campaign to back up local residents who refused to sell their properties.\footnote{See \url{http://www.trippinguptrump.co.uk/} (accessed 03 August 2014).} One of these, the farmer and quarry worker Michael Forbes, expressed his opposition in particularly clear terms, declaring at one point that Trump could ``shove his money up his arse''.\footnote{See \url{http://www.scotsman.com/news/donald-trump-s-plea-to-homeowners-on-the-menie-estate-1-1370270}. (accessed 03 August 2014)} Trump, on his part, had described Forbes as a ``village idiot'' that lived in a ``slum''.\footnote{See \url{http://www.bbc.co.uk/news/10205781} (accessed 08 July 2014).} Moreover, he had suggested that Forbes was keeping his property in a state of disrepair to pressure up the price of the property.\footnote{See \url{http://edition.cnn.com/2007/WORLD/europe/10/10/trump.golf/} (accessed 03 August 2014).} Forbes was offended. He proudly declared that he would never consider selling, as the issue had become personal.\footnote{See http://www.scotsman.com/news/scotland/top-stories/farmer-who-took-on-trump-triumphs-in-spirit-awards-1-2668649 (accessed 03 August 2014).}

At the height of the tensions, Trump considered his legal options, by asking the local council to consider issuing compulsory purchase orders (CPOs) that would allow him to take property from Forbes and other recalcitrant locals against their will.\footnote{See \url{http://www.thesundaytimes.co.uk/sto/news/uk_news/article184090.ece} (accessed 03 August 2014).} If carried out, this would have been an iconic example of an economic development taking. Moreover, it would not be the first time that the power of eminent domain had been used to the benefit of Donal Trump's business empire. In the 1990s, Trump famously succeeded in convincing Atlantic City to allow him to take the home of one Vera Coking, to facilitate further development of his casino facilities.\footcite[297-301]{jones00} This taking was eventually struck down by the New Jersey Superior Court, however, a result that was hailed as a milestone in the fight against ``eminent domain abuse'' in the US.\footnote{See \url{https://www.ij.org/cases/privateproperty} (accessed 12 August 2014). The case also caused a surge of attention directed at the issue, see \url{http://reason.com/archives/2008/03/03/litigating-for-liberty/4} (accessed 12 August 2014). For the decision itself, consult \cite{banin98}.}

In Scotland, Trump's plans were met with widespread outrage. The media coverage was wide, mostly negative, and an award-winning documentary was made which painted Trump's activities in Balmedie in a highly negative light.\footnote{See \url{http://www.youvebeentrumped.com}.} The controversy also made its way into UK property scholarship. Kevin Gray, in particular, a leading expert in property law, expressed his opposition by making clear that he thought the proposed taking would be an act of ``predation''.\footcite{gray11}

In fact, the case prompted Gray to formulate a number of key features that could be used to identify situations where compulsory purchase would be more likely to represent an abuse of power. He noted, in particular, that Trump's proposed takings would fall in line with a general tendency in the UK towards using compulsory purchase to benefit private enterprise, even in the absence of a clear and direct benefit to the public. Indeed, it was not unrealistic to think that CPOs might come to be used in Balmedie; if he had put his weight behind it, Trump might well have been able to make a successful case that existing statutory authorities could be used to justify takings of this kind.\footnote{In particular, the \cite{tcpsa97} contains a wide authority in s. 189, stating that local authorities has a general power to acquire land compulsorily in order to ``secure the carrying out of development, redevelopment or improvement''.} It would not be hard to argue that the public would benefit indirectly in terms of job-creation and increased tax revenues. Moreover, Scottish ministers had already gone far in expressing their support for the plans.

But then, in a surprise move, Trump announced he would not seek CPOs after all.\footnote{See http://news.stv.tv/north/224662-relief-for-residents-trump-lifts-threat-of-compulsory-purchase-orders (accessed 03 August 2014).} Quite possibly, he was discouraged by the negative press. But in addition, he had found another strategy, namely that of containment: He erected large fences, planted trees and created artificial sand dunes, all serving to prevent the properties he did not control from becoming a nuisance to his golfing guests. One local owner, Susan Monroe, was fenced in by a wall of sand some 8 meters high. ``I used to be able to see all the way to the other side of Aberdeen'', she said, ``but now I just look into that mound of sand''.\footnote{See http://www.theguardian.com/world/2012/jul/10/donald-trump-100m-golf-course (accessed 03 August 2014).} She also lamented the lack of support from the Scottish government, expressing surprise that nothing could be done to stop Trump.

But there was little left to do. As soon as Trump decided to build around them, the neighboring property owners found themselves completely marginalized. After all, Trump had the backing of the government, having been declared as a job-creator whose activities would boost the economy in the region. He had even received an honorary doctorate at the Robert Gordon University, a move that prompted the previous vice-chancellor, Dr David Kennedy, to hand his own honorific back in protest.\footnote{See \url{http://www.bbc.co.uk/news/uk-scotland-north-east-orkney-shetland-11421376}.}

But in the end, it was not by taking the land of others that Trump triumphed in Scotland. Rather, he succeeded by exercising ``despotic dominion' over his own.\footnote{To quote William Blackstone, \cite[2]{blackstone79b}.} This proved highly effective;  after he fenced them in, his neighbors were hard to see and hard to hear. The Balmedie controversy went quiet, the golfers came, Trump got his way. As he declared during the grand opening: ``Nothing will ever be built around this course because I own all the land around it. [...] It's nice to own land.''\footnote{See http://www.theguardian.com/world/2012/jul/10/donald-trump-100m-golf-course (accessed 06 July 2014).}

\subsubsection*{\ldots}

The tale of Trump coming to Scotland not only serves to illustrate the kind of scenario that I will be looking at in this thesis, it also puts the work into perspective. It shows, in particular, that what it means to protect property against undue interference can depend highly on the circumstances. For a while, it looked like Balmedie was about to become a canonical case of an economic development taking. But in the end, it became rather an illustration of something far more subtle, namely that the meaning of protecting property rights depends highly on context, our own perspective, and the values we aim to promote. 

Moreover, we are reminded that the function of property as such is deeply shaped by social, political and economic structures. It seems clear, in particular, that Donald Trump's ownership of the Menie estate has a vastly different meaning than does Michael Forbes' ownership of his small farm. To many observers, the former kind of ownership will represent some combination of power, privilege and profit, while the latter will be regarded as coming imbued with a mix of defiance, community and sustenance. Very different values are inherent in these two forms of ownership, and after Trump came to Balmedie, they clashed in a way that required the legal order to prioritize between them.

According to Trump and his supporters, protecting property rights against interference in Balmedie no doubt involves protecting the governmentally sanctioned golf resort plans from backwards locals who attempt to fight progress. In this narrative, ``protection'' can maybe even be used justify compulsory acquisition of property rights that are regarded as a hindrance to the full enjoyment of property for more resourceful members of the community. But for Michael Forbes and the other local owners, protecting property rights is likely to have a completely different meaning. To them, protecting property means above all else to protect a local community against what they see as a disruptive and damaging plan that will see both them and their properties turned into golfing props. Again, adequate protection might require an interference in property, to prevent Trump from using his land in a way that causes damage to his neighbors. Regardless of who we support, we are forced to recognize that protection implies interference and vice versa. 

This shows the conceptual inadequacy of a simplistic perspective whereby protecting property rights is seen as a black-and-white proposition, a call for limits on the state's power to do good, enforced to protect owners' right to do as they please. In reality, the situation is  more subtle, involving a number of additional dimensions. Importantly, how we assess concrete situations where property is under threat depends crucially on what we perceive as the ``normal'' state of property, the alignment of rights and responsibilities that we deem to be worthy of protection. Our stance in this regard clearly depends on our values. But values themselves are in turn influenced by the context of assessment within which they arise. More problematically still, they may be influenced by our \emph{perception} of this context, rather than by reality.

For example, property activists in the US tend to regard the value of autonomy as a fundamental aspect of property. But this must be understood in light of the idea that US society is founded on an egalitarian distribution of property, where ownership is meant to empower ordinary people by facilitating self-sufficiency and self-governance.\footnote{See, e.g., \cite[173]{ely07}.} Hence, the autonomy inherent in property ownership is not thought of as being bestowed on the few, but on the many. Protecting autonomy of owners against state interference is not about protecting the privileges of the rich and powerful, but is embraced as a way to protect {\it against} abuse by the privileged classes.\footnote{This narrative is enthusiastically embraced by US activists who fight economic development takings, see, e.g., \url{http://www.castlecoalition.org/}.} 

This, however, is only an {\it idea} of property protection. It might not correspond to the reality surrounding the rules that have been molded in its image. Indeed, it has been noted that despite the great pathos of the egalitarian property idea, egalitarianism has actually played a marginal role to the development of US property law.\footnote{\cite[361]{williams98} (``Why does the egalitarian strain of republicanism have such a substantial presence in American property rhetoric outside the law but so little influence within it?'')} More worryingly still, research indicates that land ownership in the US, while hard to track due to the idiosyncrasies of the land registration system, is not actually all that egalitarian.\footcite[246-247]{jacobs98} In this way, we are confronted with the danger of a disconnect between  values, reality and the law.

In Scotland, a similar story unfolds. Here, the traditional concern is that land rights are mainly held by the elites.\footnote{See generally \cite{wightman96,wightman13}.} As a result, Scottish property activists tend to focus on values such as equality and fairness, calling also on the state to regulate and implement measures to achieve more egalitarian control over the land. Indeed, reforms have been passed that sanction interference in established property rights on behalf of local communities.\footnote{See generally \cite{lovett11,hoffman13}.} At the same time, cases like Balmedie illustrate that the Scottish government, now with enhanced powers of land administration, may well choose to align themselves with the large landowners. Moreover, research indicates that recent reforms in Scottish planning law, which serve to enhance the power of the central government, has the effect of undermining local communities and their capacity for self-governance.\footnote{See generally \cite{pacione13,pacione14}.} Again, the danger of a disconnect between influential property narratives and reality is brought into focus.

On the other hand, it seems that grass root property activists in the US and Scotland may well be closer in spirit than they seem. Upon closer examination I cannot help thinking that they share many of the same concerns and aspirations, and that the differences arise mainly from the fact that they operate in different contexts and engage with different discourses of property. The challenge is to find categories of understanding that allow us to make sense of their shared spirit, as well as the spirit they oppose.

I think the example of Balmedie suggests a possible first step, by illustrating the need for a framework that will allow commentators to  deny that there is any inconsistency between opposing compulsory purchase orders while also supporting strict property regulation in the context of fighting a golf resort. Both of these positions, moreover, should be viewed as efforts to protect property. To the classical ``individual rights v state interference'' debate, such a dual position can be hard to make sense of. But in my opinion, this only points to the vacuity of such a conventional narrative.

In general, I think it is hard, close to impossible, to make sense of many contemporary disputes over property if we do not have the conceptual acumen to distinguish between egalitarian property held under a stewardship obligation by members of a local community, and feudal property held by businesses for investment. Moreover, there is no contradiction between promoting the value of autonomy for one of these, while emphasizing state control and redistribution for the other. The broader theme is the contextual nature of property, and its implications for protection of property rights. In the coming sections, I will locate a theoretical basis that will allow me to take advantage of this viewpoint in my legal analysis.

\section{Theories of property}\label{sec:top}

What is property? In common law jurisdictions, the standard answer is that property is a collection of individual rights, or more abstractly, {\it entitlements}.\footnote{The term ``entitlement'' was used to great effect in the seminal article \cite{calabresi72}.} Being an owner, it is often said, amounts to being entitled to one or more among a bundle of ``sticks'', streams of protected benefits associated with, or even serving to define, the property in question.\footcite[357-358]{merrill01} This point of view was first developed by legal realists in response to the natural law tradition, which conceptualized property in terms of the owner's dominion over the owned thing, particularly his right to exclude others from accessing it.\footcite[193-195]{klein11} In civil law jurisdictions, rooted in Roman law, a dominion perspective is still often taken as the theoretical foundation of property, although it is of course recognized that the owner's dominion is never absolute in practice.\footnote{For a comparison between civil and common law understanding of property, see generally \cite{chang12}.}

In modern society, the extent to which an owner may freely enjoy his property is highly sensitive to government's willingness to protect, as well as its desire to regulate. To civil law theorists, this sensitivity has been thought of as giving rise to various restrictions in property rights, but for common law theorists, overlooking a legal system with roots in a relatively stable feudal tradition, it has been thought of as {\it constitutive} of property itself.\footcite[7]{chang12} Indeed, the bundle of rights theory has long historical roots in common law. Arguably, it was distilled from the traditional estates system for real property, which was turned into a theoretical foundation for thinking about property in the abstract.\footnote{See \cite[7]{chang12}   
(``The ``bundle of rights'' is in a sense the theory implicit in the common law system taken to its extreme, with its inherently analytical tendency, in contrast to the dogged holism of the civil law.'').} 

However, during the 18th and 19th century, natural law thinking was also highly influential in common law. This is evidenced, for instance, by the works of William Blackstone and James Kent.\footnote{See generally \cite{blackstone79b,kent27}.} But towards the end of the 19th century, it became increasingly hard to reconcile such an approach to property with the reality of increasing state regulation. Hence, the bundle metaphor that gained prominence in the early 1900s can be seen to indicate a return to a more modest perspective.\footcite[195]{klein11}

Property rights under the bundle theory are thought to be directed primarily towards other people, not things.\footnote{See \cite[357-358]{merrill01} (``By and large, this view has become conventional wisdom among legal scholars: Property is a composite of legal relations that holds between persons and only secondarily or incidentally involves a ``thing''.'').} This underscores a second point about property in the real world, namely that the content of rights in property are necessarily relative to the totality of the legal order. For instance, relying on a bundle metaphor, it becomes perfectly natural that a farmer's property rights protects him against trespassing tourists, but not against the neighbor who has an established right of way. 

By contrast, the dominion theory suggests viewing such situations as exceptions to the general rule of ownership, which implies a right to exclusion at its core. In the case of property, exceptions no doubt make up the norm. But in civil law jurisdictions one lives happily with this. It takes the grandeur away from the dominion concept, but it retains a nice and simple structure to property law. In the civil law world, it is common to say that what the owner holds is the {\it remainder} after all positive rights, serving to restrict his dominion, have been deducted.\footcite[25]{chang12} Moreover, while there may be many limitations and additional benefits attached to property, they are all in principle carved out of one initial right, namely that of the owner. In this way, the system becomes more easily navigable.

An interested party may ask, ``who owns this land?'' Then, under the dominion theory, a clear answer is expected and will usually be adequate, even if it does not give a complete picture of all relevant property rights. Under the bundle theory, on the other hand, one might be inclined to respond, ``to which stick are you referring?'' Clearly, this narrative is more complex, perhaps unduly so. 

Some common law scholars have recently elaborated on this to develop a critique of the bundle theory, by suggesting that it should at least be complemented by a firm theory of {\it in rem} rights in property. This, they argue, would allow the law to operate more effectively, by relying on a simple and clear rule that, although defeasible, will generally suffice to inform people about their relevant rights and duties in relation to property.\footnote{\footcite[793]{merrill01b} (``The unique advantage of in rem rights -- the strategy of exclusion -- is that they conserve on information costs relative to in personam rights in situations where the number of potential claimants to resources is large, and the resource in question can be defined at relatively low cost.''); \footcite[389]{merrill01} (``The right to exclude allows the owner to control, plan, and invest, and permits this to happen with a minimum of information costs to others.''). See also \cite{ellickson11} (arguing that Merrill and Smith's analysis nicely complements and improves upon the bundle theory).} 

There are also other, less pragmatic, reasons why a dominion approach might be preferable, even if the bundle metaphor is arguably more accurate. In particular, some scholars point out that the bundle theory does not adequately reflect the sense in which property is a right to a {\it thing}, serving to create an attachment that is not easily reducible to a set of interpersonal legal relationships.\footnote{\cite[1862]{merrill07}. For a slightly different take on attachment, highlighting how the thingness of property marks its conditional nature and transferability, see \cite[799-818]{penner96}.} In the US, where the bundle theory has traditionally been dominant, critique like this seems to be gaining ground.\footnote{See generally \cite{foster10}.}

But in this thesis, the efficiency and clarity of different property concepts will not be a primary concern, nor will personal attachments to things in themselves play a particularly important role.\footnote{I mention, however, that the personhood-aspects of property that are sometimes highlighted in this regard will also be relevant to my analysis of economic development takings. However, this is not something that I think warrants extensive engagement with the bundle v dominion debate. I note, for instance, that in the work of Margaret Jane Radin, one of the main proponents of persoonhood accounts, the bundle theory is not challenged as much as it is readjusted, although in places it also seems to be the object of some implicit criticism, see, e.g., \cite[127-130]{radin93}.}
Hence, I will remain largely agnostic about this aspect of the debate between dominion and bundle theorists. In particular, the differences between civil and common law traditions in this regard do not cause special problems for my analysis of economic development takings. However, I am also more broadly interested in the values that are promoted by different ways of looking at property, particularly with regards to the question of when interference is legitimate under constitutional and human rights law. Hence, I  now turn to the question of whether or not there are any significant differences between dominion and bundle theories in this regard.

Intuitively, one might think that bundle theorists are likely to endorse greater room for state interference in property rights. Indeed, thinking about property as sticks in a bundle may lead one to think that property rights are intrinsically limited, so that subsequent changes to their content -- carried out by a competent body -- are mere reflections of their nature, not a cause for complaint. In particular, the theory conveys the impression that property is highly malleable. For the theorists that developed the bundle of sticks metaphor in the late 19th and early 20th century, this aspect was undoubtedly very important. By providing a highly flexible concept of property, they helped the state gain conceptual authority to control and regulate. Indeed, this was the clear intention of many early proponents of the bundle theory -- the ``progressives'' of their day.\footcite[195]{klein11} The early bundle theorists not only developed a theory to fit the law as they saw it, they also contributed to change.

In relation to takings law, the progressives succeeded in gaining acceptance for the use of eminent domain to benefit a wider range of public purposes than had so far been considered legitimate.\footnote{See generally \cite{yale49}.} In particular, they argued successfully that the so-called ``public use'' restriction, which had previously been enforced quite strictly, particularly by state courts, should be greatly relaxed. This change was important in creating the situation which led to economic development takings becoming a contentious issue in the US, and so provides important background to the main topic of my thesis.  I return to the public use debate in the US in much more depth later, in Chapter \ref{chap:2}, Section \ref{sec:hop}. Here I would like to stress that I think there can be little doubt that the increased scope given to eminent domain in the early 20th century was mutually conducive to the conceptual reorientation that took place during the same time.

In relation to the different, but related, issue of so-called regulatory takings, the bundle theory even  became directly implicated. A regulatory taking occurs when governmental control over the use of property becomes so severe that it must be classified as a taking in relation to the law of eminent domain. Particularly in the US, the question of when regulation amounts to a regulatory taking is highly controversial. The stakes are high because takings have to be compensated in accordance with the Fifth Amendment of the US constitution. At the same time, the law is unclear; a lack of statutory rules means that regulatory takings cases are often adjudicated directly against constitutional property clauses (often the relevant state constitution, in the first instance).

If property is thought of as a malleable bundle of entitlements that exists only because it is recognized by the law, it becomes more natural to argue that when government regulates the use of property, it does not deprive anyone of property rights, but merely restructures the bundle. In the case of {\it Andrus v Allard}, the Supreme Court adopted such an argument when it declared that ``where an owner possesses a full ``bundle'' of property rights, the destruction of one ``strand'' of the bundle is not a taking, because the aggregate must be viewed in its entirety''.\footcite[65--66]{andrus79}

Historically, therefore, it seems that bundle theorists have been largely aligned with those that favor a less restrictive approach to eminent domain. But I think it is wrong to conclude that the bundle theory {\it necessarily} implies such a stance on takings. Indeed, some prominent scholars have argued for an almost entirely opposite view. Professor Epstein, in particular, goes far in suggesting that as every stick in the property bundle represents a property right, government should not be permitted to remove any of them without paying compensation.\footcite[232-233]{epstein11} Moreover, Epstein does not believe that the bundle theory is responsible for the fact that his view of property has not been widely endorsed by US courts. Instead, he thinks the main (negative) impact of ``progressive'' thinkers stems from their tendency to adopt a ``top-down'' approach to property. That is, Epstein directs attention towards their tendency to view property rights as vested in, and arising from, the power of the state, not the possessions of individuals.\footnote{\cite[227-228]{epstein11} (``In my view, the nub of the difficulty with modern property law does not stem from the bundle-of-rights conception, but from the top-down view of property that treats all property as being granted by the state and therefore subject to whatever terms and conditions the state wishes to impose on its grantees'').} 

In my opinion, Epstein's argument shows that adoption of the bundle theory can hardly be considered a determinate factor for the kind of protection private property enjoys in a given legal system. Moreover, Epstein successfully demonstrates that as a rhetorical device, the theory may well be turned on its head. Unsurprisingly, the substance of the law, in the end, turns primarily on the values one adheres to, not the theoretical constructions one relies on when expressing those values.\footnote{To further underscore this point, it may be mentioned that while US courts do in fact recognize that a regulation can amount to a taking, this is practically unheard of in several other common law jurisdictions, including England and Australia, which nevertheless paint property in a similar conceptual light. Moreover, while the issue of regulatory takings is considered central to constitutional property law in the US, it is considered a fairly marginal issue in England, see \cite{purdue10}.}

In the civil law world, the relationship between property theorizing and property values is similarly hard to pin down at the conceptual level. To illustrate, I will again point to the question of regulatory takings. In some countries, like Germany and the Netherlands, the right to compensation is quite strong, but in other civil law countries, such as France and Greece, it is very weak.\footnote{See generally \cite{alterman10}.} In particular, the exclusive dominion understanding of property does not commit us to any particular kind of policy on this point. Indeed, the theory appears to cater comfortably to a range of different politically determined solutions to the problem of striking a balance between the interests of owners and the interests of the state. 

On the one hand, the undeniable fact of modern society is that property rights are enforced, and limited, by the power of government. Hanging on to the idea of dominion, then, necessarily forces us to embrace also the idea that dominion is not enjoyed absolutely and that government may interfere in property rights. In this way, the theory may serve as a conceptual basis upon which to argue for a more relaxed approach to protection of property rights. These rights are not absolute anyway, so why worry about interfering in them for the common good? But this story too may be turned on its head: A libertarian may well use the same image to tell a tale of property being ripped apart at its seams. Hence, he may argue, unless we want to completely lose our grasp of what property is, we had better enhance the level of protection offered to property owners.

To me, the upshot is that the differences between common law and civil law theorizing about property are not significant enough to 
make them crucial to the questions studied in this thesis. In particular, the differences between the bundle theory and the dominion idea do not appear to speak decisively in favor of any particular approach to economic development takings, nor does it provide any clear justification for regarding such takings as special in the first place. Property enjoys constitutional protection and is a recognized human right across the divide, but what this means in practice is hard to deduce from either account.

In terms of descriptive content, both theories are too bold and oversimplified. They provide a manner of speech, but they do little to enhance our understanding of the reality of property rights in modern society. In particular, they do not provide a functional account of what role property plays in relation to the social, economic and political structures within which it resides. In terms of normative content, on the other hand, they are both too bland and imprecise. They simply do not offer much clear guidance as to what norms and values the institution of property is meant to serve. They give neat explanations of what property is, but do not tell us {\it why} it should be protected. 

In the following, I will address both these shortcomings by considering property theories with a wider scope. There are many candidates that could be considered. In a recent book on property theory, Alexander and Pe\~{n}alver present five key theoretical branches: 
\begin{itemize}
\item {\it Utilitarian} theories, focusing on property's role in helping to maximize utility or welfare with respect to individual preferences and desires.\footnote{\cite[Chapter 1]{alexander10}.} 
\item {\it Libertarian} theories, focusing on property's role in furthering individual autonomy and liberty, as well as the importance of protecting property against state interference, particularly attempts at redistribution.\footnote{\cite[Chapter 2]{alexander10}.} 
\item {\it Hegelian} theories, focusing on the importance of property to the development of personhood and self-realization, particularly the expression and embodiment of free will through control and attachment to one's possessions.\footnote{\cite[Chapter 3]{alexander10}.}
\item {\it Kantian} theories, focusing on how property arises to protect freedom and autonomy in a coordinated fashion so that {\it everyone} may potentially enjoy it, through the development of the state.\footnote{\cite[Chapter 4]{alexander10}.}
\item {\it  Human flourishing} theories, focusing on property's role in facilitating participation in a community, particularly as a template allowing the individual to develop as a moral agent in a world of normative plurality.\footnote{\cite[Chapter 5]{alexander10}.}
\end{itemize}

It it beyond the scope of this thesis to give a detailed presentation and assessment of all these theoretical branches and the various ideas that have been discussed within each of them. However, in Section \ref{sec:hf} below, I will present the human flourishing theory in more detail. This is because I believe that if it is adopted, it suggests making a range of new policy recommendations regarding how the law {\it should} approach the question of economic development takings. 

First, however, I note that all the theories mentioned above are highly normative, used actively to promote the adoption of particular values associated with property. While I am not unwilling to take a stand in this debate, my main objective is to study economic development takings descriptively, by giving a case study of Norwegian waterfalls and discussing its significance in terms of comparative and human rights law. Hence, before I move on to consider other aspects, I first need a theoretical framework that allows me to meaningfully discuss those aspects that make economic development takings unique. I would like to do so, moreover, without thereby committing myself to any particular stance on how to normatively assess those aspects. 

To arrive at such a foundation, I will rely on the descriptive parts of the so-called {\it social function theory} of property.\footnote{See generally \cite{foster11,mirow10,alexander09a}. Be aware that some authors, particularly in the US, also speak of the {\it social obligation} theory, using it more or less as a synonym for the social function theory.} While this theory is often implicated in normative theories, including the human flourishing theory, I argue that it has a descriptive core which is also of great significance. Its importance to my work in this thesis is underscored by the fact that I will draw on the social function theory to answer the pressing problem of what makes economic development takings a legitimate and useful category of legal reasoning. 

Let me first reiterate that it is not {\it prima facie} clear that the category makes any legal sense at all, due to the fact that many jurisdictions lack rules that explicitly make the purpose of interference a relevant measuring stick for assessing legitimacy. To respond successfully to this potential objection, I believe it is necessary to look at property's social functions. In fact, property scholars are becoming increasingly aware of the need to do this in general, as they note that existing theories are overly focused on a narrative that revolves around individual entitlements. Many still reject that this necessitates conceptual reconfiguration, but the social function idea of property appears to be gaining ground, also as an important aid in making sense of how the law actually works. I believe this descriptive aspect of the theory provides the most appropriate way to argue that it is theoretically desirable to regard economic development takings as a special issue in property law, and I will argue for this in Section \ref{sec:edt}.

However, before making my specific point about takings, I will present the social function theory of property more generally. I will focus on showing that it captures aspects that are already highly relevant -- behind the scenes -- to how property rules are understood and applied in concrete situations. It seems, in particular, that socio-legal arguments play an important, yet often unacknowledged, role when courts interpret fundamental rules that are meant to protect private property. Bringing those aspects into the open is in itself a worthwhile project to pursue, irrespectively of any further normative stances that the social function theory might give the theorist occasion to adopt.

\section{The social function of property}

There is a growing feeling among property scholars that the notion of property has been drawn too narrowly by many of the traditionally dominant theories of property. Some have even gone as far as to challenge the idea that property is a meaningful and well-defined concept at all. These scholars point out that what counts as property in a given legal system, and what property entails in that system, depends largely on its social and political context, tradition, and even chance.\footnote{For a particularly inspiring exposition of property's elusive nature, see \cite{gray91}.} In the US, a utilitarian law-and-economics approach -- which largely takes the social and political underpinnings of property for granted -- has long been regard as standard, but even there the tide is turning. While most US scholars still regard property as a robust and meaningful category of legal thought, many are increasingly turning away from assessing property rules narrowly against their effectiveness in maximizing individual utility and social welfare. Instead, these scholars adopt a holistic approach, which allows property's social function to come into focus. One of the main proponents of this conceptual shift is Gregory S. Alexander, professor at Cornell University. In a recent article, he writes:

\begin{quote} Welfarism is no longer the only game in the town of property theory. In the last several years a number of property scholars have begun developing various versions of a general vision of property and ownership that, although consistent with welfarism in some respects, purports to provide an alternative to the still-dominant welfarist account.[...] These scholars emphasize the social obligations that are inherent in ownership, and they seek to develop a non-welfarist theory grounding those inherent social obligations.\footcite[1017]{alexander11}
\end{quote}

To scholars coming from political science, sociology or human geography, this trend will not raise many eyebrows, except perhaps for the fact that it is a recent one. After all, in fields such as these, property has never been understood merely as a set of individual entitlements that are meant to result in increased welfare. Rather, property is seen as a crucial part of the fabric of society, one that entrenches privileges and bestows power.\footnote{See generally \cite{carruthers04}.} Even scholars who believe that the institution of property is a force for good, recognize that being an owner carries with it political capital, social responsibility, and membership in a community. Those aspects, moreover, are often regarded as more important than entitlements explicitly recognized in positive legal terms. Crucially, they are important not only to the individual owners but also to society as a whole. How property rights are distributed among the population, for instance, has obvious political and economic implications, serving as a source of power and prosperity to some groups, while marginalizing others.\footnote{See, e.g., \cite[23]{carruthers04}. (``The right to control, govern, and exploit things entails the power to influence, govern, and exploit people'').}

But what is the relevance of this to property law? Usually, jurists approach property in isolation from such concerns, and often they do so because of practical necessity. The political question of what the law should be might require musings about the purpose and social context of property, but in the day-to-day workings of the law, the story goes, such considerations play a lesser role, with the importance of clear and simple rules outweighing the possible benefit that would result from contextual and holistic assessment. But at the same time, no functioning theory of  property would deny that social aspects such as expectations and obligations play a role in relation to property {\it in life}. The problem, rather, is that classical theories of property may be accused of taking the pragmatic view too far, by failing to recognize that many social functions are {\it intrinsic} to property, so that they may sometimes be impossible to disregard, also when the law is applied to concrete disputes.

This accusation can be raised against both bundle and dominion theorists. They both tend to leave little conceptual room for considering property as a social phenomena. It is recognized, of course, that rights in property -- bundled or otherwise -- serve to regulate social relations. But this effect is typically regarded as belonging to the periphery of property as a legal category, more relevant to sociologists than to property scholars. In addition, it is uncommon to observe that the causal relation between property rights and society is bidirectional, since the meaning and content of property itself is partly determined by the very same social structures that property helps establish and sustain. When this aspect of property is not recognized, the risk is that subtle dependencies between property and the political order are not brought into focus, even when they play an important role in practice.

This is particularly clear when it comes to socially defined obligations attached to property. Hardly anyone would protest that in practical life, what an owner will do with his property is as much constrained by the expectations of others as it is by law. But in addition to influencing the owner subjectively, expectations can take on an objective character by being embedded strongly in the social fabric. This, in turn, may give rise to a norm, or even a custom, which may be legally relevant, either because the law gives direct effect to it, or because it influences how we interpret rules relating to the use of property.\footnote{See generally \cite{penalver09,alexander09}.}

This seems hard to dispute as a descriptive assertion, but traditional property theorists have surprisingly little regard for it. According to Alexander, the classical theories of property convey the impression that ``property owners are rights-holders first and foremost; obligations are, with some few exceptions, assigned to non-owners''.\footcite[1023]{alexander11} The social function theorists attempt to redress this imbalance by developing theories that can naturally accommodate an account of social obligations that attaches greater weight to them as objects of property. As Alexander explains, ``social obligation theorists do not reverse this equation so much as they balance it. Of course property owners are rights-holders, but they are also duty-holders, and often more than minimally so.''\footcite[1023]{alexander11} 

It should be noted that while it lay dormant for some time, particularly in the US, this idea is by no means new. Its first modern expression is often attributed to Le{\'o}n Duguit, a French jurist active early in the 20th century. In a series of lectures he gave in Buenos Aires in 1911, Duguit challenged the classic liberal idea of property rights by pointing to their context-dependence, adopting a line of argument strikingly similar to how recent scholars have criticized the law-and-economics discourse of modern times.\footnote{See \cite[1004-1008]{foster11}. For more details about Duguit's work and the contemporaries that inspired him, see generally \cite{mirow10}.} In particular, Duguit also pointed to the notion of obligation, stressing the fact that individual autonomy only makes sense in a social context, wherein people are also dependent on each other and related through membership in communities. Hence, depending on the social circumstances of the owner, his property could entail as many obligations as it would entail entitlements and dominion. This, according to Duguit, was not only the reality of property ownership in life, it was also a normatively sound arrangement that should inspire the law, more so than the unrealistic visions of property evoked by the liberal tradition.\footnote{See \cite[1005]{foster11} (``The idea of the social function of property is based on a description of social reality that recognizes solidarity as one of its primary foundations'', discussing Duguit's work). It should also be noted that Duguit was particularly concerned with owners' obligations to make productive use of their property, to benefit society as a whole. This raises the question, however, to which we shall return in more depth later, who exactly should be granted the power to determine what counts as ``productive use''. In this way, Duguit's work also serves to underscore one of the main challenges of regulatory frameworks that seek to incorporate and draw on property's social dimension. How should decisions be made in such regimes?} 

Similar thoughts have been influential in Europe, particularly in the post-WW2 rebuilding period. For instance, as I discuss further in Chapter \ref{chap:2}, Section \ref{sec:germany}, the constitution of Germany -- her {\it Basic Law} -- contains a property clause that explicitly includes a provision stating that property entails obligations as well as rights. As argued by Alexander, this has had a significant effect on German property jurisprudence, creating a clear and interesting contrast with US law.\footnote{See \cite[338]{alexander03} (``The German Constitutional Court has adopted an approach that is both purposive and contextual, while the U.S. Supreme Court has not'').} 

A social perspective on property was also influential during the debate among the European states that first drafted the property clause in the First Protocol to the European Convention of Human Rights.\footnote{See \cite[1063-1065]{allen10}.} Later, however, the liberal conception of property gained ground also in Europe, causing jurisprudential developments that have been particularly clear in the case law from the European Court of Human Rights.\footnote{See generally \cite{allen10}.} Even so, property theorizing in Europe is still influenced by a social function view on property, more so than in the US. The European Court of Human Rights, for instance, stresses the importance of {\it proportionality} and {\it fairness} when adjudicating property cases, suggesting the importance of a contextual approach to the balancing of the many private and public interests involved.\footnote{See generally \cite[Chapter 5]{allen05}.}

I will return to possible normative implications of the social function theory later, but here I would like to stress that in the first instance it merely asks us to recognize an empirical truth. Property does not arise in a vacuum, but from within a society. As a philosophical proposition, this is obvious and hardly anyone denies it. But the social function theory asks us to consider something more, namely that property {\it law} continues to influence, and be influenced by, the social structures that surround it. Perhaps most importantly, property both reflects and shapes relations of power among members of a society.\footnote{This aspect of property's social function was stressed in a recent ``statement of purpose'' made by leading property scholars in support of the social function theory, see \cite{alexander09a}.} Moreover, it does not act uniformly in this way -- the actual effect of property on power depends on the circumstances.

An indebted farmer who is prevented by state regulation from making profitable use of his land might come to find that his property has become a burden rather than a privilege. As a consequence, someone who has already amassed power and wealth elsewhere might be able to purchase it from him cheaply. Indeed, this might provide an excellent opportunity for the outsider to consolidate his position. He can ensure that his privileges become further entrenched, both socially, politically and economically. By acquiring a farm and transforming it to recreational property, he symbolically and practically asserts his dominance and power, while also reaping a potential financial benefit resulting from his investments in a more ``modern'' pattern of use. In some cases, this dynamic can even become endemic in an area, resulting in a complete reshaping of the social fabric surrounding property.

The story might go like this: First, impoverished farmers and other locals sell homes to holiday dwellers. Then house prices soar. As a result, local people with agrarian-related incomes can't afford local homes, causing even more people to sell their land to the urban middle class. In this way, a causal cycle is established, the social consequences of which can be vicious, particularly to the low-income people who are displaced.\footnote{The general mechanism is well-documented and known as {\it gentrification} in human geography (often qualified as rural gentrification when it happens outside urban areas). See generally \cite{weesep94,phillips93,slater06}. For a case study demonstrating the role that state regulation can play (perhaps inadvertently) in causing rural gentrification, see \cite[1027-1030]{darling05}.} My theoretical contention is the following: Setting out to protect property in a situation like this -- when property rights pull in different directions -- requires taking some stance on whose property, and which of property's functions, one is aiming to protect. In particular, should the law protect the property rights of local people who face displacement, or should it protect the property rights of outsiders wishing to invest?

Some may shun away from this way of posing the question, by arguing that it would be better to rely on clear rules that can deliver justice to owners with a minimum level of dependence on the particular social processes that property is involved in at any given time. I am inclined to disagree with such a stance from the outset, since justice itself is a notion that largely seems to depend on social conditions. However, my main point here is that the prospect of such ``socially neutral'' rules is simply illusory when both sides of a conflict are in a position to adopt a property narrative to argue for their interests. For an excellent example of such a situation, it is enough to return to the story of Donald Trump coming to Scotland that we told in Section \ref{sec:dts}.

As long as Trump threatened to use compulsory purchase, the local people could adopt a traditional ``pro-property'' stance against Trump. But as soon as Trump decided to fence them in by relying on his own property rights, they had to adopt a seemingly contradictory view, {\it against} property, on the basis that Trump's rights should be limited out of concern for the community. So how do we classify the anti-Trump stance with regards to property? The answer is unclear under classical theories, but under the social function stance, it becomes easy to resolve. The locals sought to protect property, but not just any property. The property they wanted to protect was the property which served the social function of sustaining the existing community. The property they wanted to protect was the property that {\it meant} something to them.

Undoubtedly, this was also the sentiment of Trump and his supporters, who could also make a case based on property. Hence, in conflicts such as these the law will invariably have to take a stand regarding which social functions it wishes to promote. In all likelihood, such a stand must also sometimes be taken by whoever {\it interprets} the law, since it is exceedingly unlikely that the legislature will ever be able to provide clear rules for resolving all conflicts of this kind. Lastly, and most controversially, the courts may find occasion to curtail the power of government -- perhaps even the legislature -- if their power is usurped by powerful actors wishing to undermine property's proper functions to further their own interests. This, in particular, becomes the question of constitutional and human rights limits to interference in property.

Property shapes and reflects societies, but it also shapes and reflects social commitments and dependencies within those societies.\footnote{See generally \cite{alexander09}.} Again, this function of property is highly dependent on context. A small business owner, by virtue of being a member of the local community, is discouraged from becoming a nuisance to his neighbors. Everything from erecting bright neon signs to proposing condemnation of neighboring properties are actions that he will be socially deterred from taking. If the local shop owner does not conform to social expectations, he will pay a social price. Indeed, most likely even an economic price, especially if his customer-base is local. At the same time, the local connection would serve to make the business owner positively invested in the well-being of the community. This would encourage everything from sponsoring local events to hiring local youths as part-time helpers.

But at the same time, the local business owner might be discouraged from changing his business model to become more competitive, if this is perceived as a threat to other members of the community. Economic rationality might suggest that he should expand, say, by physically acquiring more space and targeting new groups of customers, but social rationality might make this an untenable proposal. This, however, might render the business economically unsustainable, particularly if it is facing fierce competition from businesses that are not similarly constrained by community ties. Moreover, even if the business is in fact viable as long as the community remains in place to support it, the perception that there is room for improvement might increase external pressures both on the business and the community. Importantly, in the age of regulation for commercial facilitation, the state itself might exert pressure of this kind.

Then, if our local shop owner goes out of business, for whatever reason, the new owner might fail to become integrated in the community in the same way, with obvious consequences for the property's function in that community. Indeed, if we imagine that the new owner is a large commercial actor who is hoping to raze the community in order to build a new shopping center, we are at once reminded of the stark contrasts that can arise between various social functions of property. The property rights of a shop owner can be the life nerve of a community, while the exact same rights in the hands of someone else can spell destruction. While this is an undeniable empirical fact of property ownership, it is far from clear what its legal ramifications are. Here, it is tempting to embrace a normative stance, and argue for particular social values that the law {\it should} promote. However, I would like to hold on to the descriptive mode of analysis a little further. For it is perfectly clear that regardless of whose interests win out in the end, assessments of the social function of property will have played an important role in brining about that outcome.

This is true not only when the law explicitly requires that this function is to be taken into account, such as in relation to the property clause of the basic law of Germany. It also commonly becomes true, as courts search for information to guide them in their interpretation and application of statutory rules that are seemingly not concerned with social aspects of property. The classical example from the US is the case of {\it State v Shack}.\footcite{shack71} The case concerned the right of a farmer to deny others access to his land, a basic exercise of the right to exclusion often regarded as fundamental to the very definition of property. The controversy arose after the two defendants, who worked for organizations that provided health-care and legal services to migrant farmworkers, entered the land of a farmer without permission. They were there to provide services to the farmers employees, and when the farmer asked them to leave, they refused.

In the first instance. they were convicted of trespassing in keeping with New Jersey state law, but on appeal the Supreme Court of New Jersey overturned the verdict. The court held that the dominion of the land owner did not extend to dominion over people who were rightfully on his land. Hence, as long as the defendants were there at the request of the workers, the owner had to tolerate this. Importantly, the court argued for this result -- which was not based on any natural reading of the New Jersey trespass statute -- by pointing also to the fact that the community of migrant workers was particularly fragile and in need of protection. Their right -- in property -- to receive visitors where they work and live, therefore, had to be recognized, in spite of this limiting the farmer's exclusion right.

The lesson to take from this is that the social function of property can play a role even when this does not explicitly follow from any property rules. This, in turn, may be used to argue that a shift towards a social function theory is desirable. In so far as the property rules we rely on explicitly directs us to take the social aspect of property into account when applying the law, it might be permissible for the practically minded jurist to conclude that there is little need for theorizing about property's social dimension. This dimension, in so far as it is relevant, is quantified inside the law itself, not by theories that encompass it. But as a matter of fact, cases like {\it State v Shack} show that the social dimension can be relevant even when it is not mentioned in any authority, even in relation to clear rules that would otherwise appear to leave little room for statutory interpretation. It arises as relevant, in such cases, because the social dimension is intrinsic to property itself. 

This might still be a radical claim, but it is primarily a descriptive one. Indeed, even if the case of {\it State v Shack} had gone the other way, I would be inclined to take from it the same lesson. If the owner's right to exclusion had received priority over the workers right to receive guests and the owner's obligation to respect this right, that too would be an outcome that would likely underscore the social function of property. To illustrate this, it is helpful to look to an article written by Eric Claeys, where he is critical both of the social function theory in general and {\it State v Shack} in particular.\footcite{claeys09} Given the basis on which that decision was made, he is led to argue, however, by also pointing to those aspects of the social context that speak in favor of the farmer.\footnote{\cite[941-942]{claeys09}.} Indeed, since he aims to engage with the social function theorists, he cannot simply declare  that the trespass rules are absolute and that the social circumstances are irrelevant. 

Instead, he argues that by considering the circumstances in {\it more} depth, a different outcome suggests itself.\footnote{\cite[941]{claeys09} (``there are good reasons for suspecting that there was more blame to go around in Shack than comes across in the case's statement of facts'').} But even if this is true, it is no argument against the descriptive content of the social function theory, merely an argument against those who think that particular values need to be endorsed by anyone willing to look to the social context of a property dispute. In this regard, it is not hard to agree with Claeys that normative fundamentalism is wrong. Indeed, he might even have a point in criticizing some social function theorists for normative naivety.\footnote{\cite[945]{claeys09} (``Judges might think they are doing what is equitable and prudent. In reality, however, maybe
they are appealing to a perfectionist theory of politics to restructure the law, to redistribute property, and ultimately to dispense justice in a manner encouraging all parties to become dependent on them.'')} 

I do not follow Claeys, however, when he takes this to be an argument {\it against} the form of legal reasoning that social function theories promote, and which he himself skillfully engages in.\footnote{In particular, I do not follow the leap Claeys makes when he suggests that it is beneficial to keep ``discretely submerged'' what he describes as ``culture war overtones'' in legal reasoning.\cite[947]{claeys09}.} In {\it State v Shack}, for instance, such reasoning was clearly in order. To engage in it was far {\it less} naive than to dismiss it on the basis that it would be irrelevant to the case. Indeed, if it the social function view had been dismissed, the entitlement-based idea of property would in effect do {\it unacknowledged} normative work, with no basis in anything more authoritative than a palpably oversimplified idea of the meaning of property. 

However, I agree with Claeys that prudence is in order. Moreover, I am not saying that the social function theory does not have normative consequences. It clearly does. Invariably so, simply because it provides a new way of talking about property and analysing conflicts, which will in turn influence our normative assessments. This is also illustrated by {\it State v Shack}. Despite Claeys skillful advocacy, many would no doubt fail to be convinced of the social merit of recognizing a right to exclusion in this case. But the crucial aspect of the social function narrative is that it makes such aspects clear, not that it commits us to, or promises to deliver, any morally superior stance on property that can deliver ``correct'' outcomes in cases such as this.

This challenges a common assumption, among both detractors and supporters of the social function theory, who argue that the theory commits us to a particular form of normative assessment, in pursuit of the ``good''. Some even argue that property law should be studied from the point of view of virtue ethics.\footnote{See generally \cite{penalver09}.} Unsurprisingly, critics such as Claeys use this to launch attacks on the social function theory and its supporters, by arguing that it represents a way of thinking that will invariably lead to lessened constitutional property protection and greater risk of abuse of state powers.\footnote{See \cite{claeys09} (``The more ``virtue'' is a dominant theme in property regulation, the less effective ``property'' is in politics, as a liberal metaphor steering religious, ethnic, or ideological extremism out of the public square'').} Indeed, increasing the room for state interference is often seen as the aim of conceptual reconfiguration; the social function view of property tends to be associated with social democratic and/or redistributive political projects, by which the notion of property is recast to justify greater interference in established rights.\footnote{Despite his commitment to ``value-pluralism'', this motivation is also clearly felt in the work of Gregory Alexander. He argues, for instance, that the social obligations inherent in property imply that the ``state should be empowered and may even be obligated to compel the wealthy to share their surplus with the poor'', see \cite[746]{alexander09}. For an assessment linking similar views on property in Europe to the dominance of social democratic thought in the post-WW2 period, see \cite{allen10}.}

It is important to note, however, that while social democratic policies may be easier to justify by emphasizing the social function of property, the mere recognition that property has an important social dimension does not in itself offer any justification for policies of this kind. For one, policy reasons must be tied to the prevailing social and economic circumstances, they will not automatically succeed merely by virtue of a conceptual shift. In addition, it seems to me that the most crucial premises used in arguments for greater state control and state-led redistribution projects concern the nature of the state, not the functions of property.

In particular, why should we believe that the state is the ultimate social institution to which property {\it should} answer? Is it not, for instance, equally possible to contend that property should continue to answer to less formal social structures that it is already embedded in by virtue of owners' membership in local communities? If so, one might as well want to limit the state's role to that of ensuring fair play among individuals and communities. A contentious question, then, might be to what extent the state should actively promote certain kinds of communities in accordance with political goals. Embracing more direct state control, on the other hand, would no longer seem very natural, at least not as a goal in itself. Indeed, on the social function view, the very idea of direct state control seems to depend on the claim that more low-level social structures fail to function properly and, crucially, that state control is {\it better}. In my opinion, this requires a separate argument. Hence, to move uncritically between talk of the ``community'' and talk of the state, as writers like Pe\~{n}alver and Alexander sometimes do, is in my opinion inappropriate.

In fact, I am inclined to believe that it is only appropriate to equate community with the state in highly special situations, for instance if it can be shown that owners insulate themselves from, and engage in exploitative practices towards, other people and communities. Importantly, to argue that such a situation obtains requires a case to be made that is compelling both empirically and politically. In this regard, I believe theory alone has little to offer. This is a reason to conclude that the social function view of property in fact tells us very little about how widely the state should intervene in property in a given society. It allows us to recognize the {\it possibility} that the state may have to intervene on behalf of certain property values, say those that aim to protect communities. But this is no argument in favour of the position that the state should intervene more or less often than it currently does. Importantly, the theory can still serve a crucial purpose in that it allows us to reason more clearly about {\it when} it is appropriate for the state to intervene. For instance, the social function theory will later be used by me to single out economic development takings for special consideration. But this will not commit me to a particular normative stance on such takings.

More generally, it does not follow from the recognition that property structures are social in nature that {\it any} institution should actively seek to neither change nor protect those structures. The Humean position, namely that the existing distribution of property rights represents a socially emergent equilibrium, remains plausible. Moreover, the normative stance that this equilibrium is a {\it good} one (or at least as good as it gets) remains as contentious -- and as arguable -- as ever. For this reason, I believe it is appropriate to approach the social function theory as a descriptive theory in the first instance.

It is worth emphasizing that in taking this view I depart from the stance taken by many of the contemporary scholars who advocate on behalf of social function theories, including some that reject social democratic ideals. Hanoch Dagan, for instance, is a self-confessed liberal, but still explicitly and strongly argues for a social function understanding on the basis that it is morally superior. ``A theory of property that excludes social responsibility is unjust'', he writes, and goes on to argue that ``erasing the social responsibility of ownership would undermine both the freedom-enhancing pluralism and the individuality-enhancing multiplicity that is crucial to the liberal ideal of justice''.\footcite[1259]{dagan07}

If this is true, then it is certainly a persuasive argument for those who believe in a ``liberal idea of justice''. But for those who do not, or believe that property law is -- or should be -- largely agnostic on this point, a normative justification for the social function theory along these lines can only discourage them from adopting it. Such a reader would be understandably suspicious that the {\it content} of the social function theory -- as Dagan understands it -- is biased towards a liberal world view. The reader might agree that property continuously interacts with social structures, but reject the theory on the basis that it seems to carry with it a normative commitment to promote liberalism.

Danach stands out somewhat in the literature by focusing on {\it liberal} values, but as I have already indicated, he is not alone in proposing highly normative social function theories. Indeed, most contemporary scholars endorsing a social function view on property base themselves on highly value-laden assessments of property institutions.\footnote{See, e.g.,  \cite{alexander09,crawford11,davidson11,singer09,penalver09}.} While they provide interesting insights into the nature of property, I am struck by a feeling that these writers all tend to overstate the desirable normative implications of adopting a social function view. In addition, they appear to believe that accepting this view on property requires us to embrace certain values and reject others. Moreover, one is left with the impression that the social function theory has little to offer beyond the values with which it is imbued, which can in turn push the law in the direction that these writers deem desirable. 

I disagree that this is the case, at least for the social function theory as I understand it. Dagan's theory of property might be conducive to ``liberal justice'', but this is because it involves far more than what follows analytically from the proper recognition that  social functions should be considered relevant when adjudicating on the rights and obligations attached to property. Indeed, it is Dagan's clearly stated aim to propose a theory that promotes specific liberal values. ``There is room to allow for the virtue of social responsibility and solidarity'', he writes, continuing by suggesting that ``those who endorse these values should seek to incorporate them -- alongside and in perpetual tension with the value of individual liberty -- into our conception of private property''.\footcite[802]{dagan99} This view is reflected further in the concrete policy recommendations he makes, for instance in relation to the question of when it is appropriate to award less than ``full'' (market value) compensation for property following a taking.\footnote{See generally \cite{dagan14b}.}

My objection is not that his proposals are necessarily wrong, but that they need not be accepted in order to conclude that the social function of property should be given a more prominent place in property theory. Importantly, I think the focus on normative reasons threatens to overshadow the most straightforward reason for awarding social structures a more prominent place in the analysis, namely that they are almost always crucially important behind the scenes, even if they go unacknowledged. The social function theory, rather than being ``good, period'', as Danach suggests, is nothing more or less than accurate, irrespectively of one's ethical or political inclinations. As such, it provides the foundation for a debate where different values and norms can be presented in a way that is conducive to meaningful debate, on the basis of a minimal number of hidden assumptions and implied commitments. Thus, the first reason to accept the social function theory, for me, is epistemic rather than deontic.

That is not to say that normative theories should not be formulated on the basis of the social function theory, it merely means that I believe it is useful to maintain at least a theoretical division between the descriptive and normative aspects of such theorizing. I return to normative aspects in the next section, arguing that the commitment to ``human flourishing'' endorsed by Professor Alexander is a particularly well-argued norm that arises from value-based assessment of the social function of property. This, I argue, is in large part also due to the value-pluralism inherent in this idea, suggesting as a positive normative claim that our notions of property {\it should} allow a divergence of opinions and values to influence the law and its application in this area.

Moreover, I believe the history of the social function theory lends support to my claim that it is useful to emphasize that the theory gives us important descriptive insights that carry few normative commitments. This is particularly important, I believe, in a time when property scholars are showing greater willingness to explore new theoretical ideas. Theories can hardly be entirely value-neutral, nor is this a goal in itself. But in my opinion, a good theory is one that can serve as a common ground for further discussion based on disagreement about values and priorities. According to Kevin Gray, ``the stuff of modern property theory involves a consonance of entitlement, obligation and mutual respect''.\footcite[37]{gray11} It is important, I think, that the same measured perspective is reflexively applied towards theory itself, to diminish the worry that a broader theoretical outlook is the first step towards unchecked state power and rule by ``judicial philosopher-kings''.\footcite[944]{claeys09}

In the next subsection, I will argue in some more detail why such a cautious perspective is warranted, by considering how the Italian fascists appropriated the social function theory in 1930s. Building on the highly inspiring work of di Robilant, I will also briefly track how non-fascist property scholars opposed this development by focusing on value-pluralism, local self-governance and freedom.\footcite{robilant13} Importantly, these scholars embraced the social function theory as a common ground from which to launch a meaningful attack on more radical ideas, without alienating those with divergent views. Instead of clinging to the old-style liberal discourse that the fascists had either flatly rejected or completely subverted, many Italian non-fascists were willing to engage in a discourse revolving around property's social function, by spelling out a more measured set of ideas based on this premise. Crucially, this set the stage for a form of intellectual resistance that did not reject those aspects of fascism that had great appeal to the public and which arguably also reflected true insights into the unfairness and lack of sustainability of the established legal order.

\subsection{Rooting out fascism: {T}he tree of property}

While the social function theory makes intuitive sense, it is also highly abstract. Therefore, its exact content has been notoriously hard to pin down. This is recognized by contemporary scholars endorsing a normative view, who attempt to address this by proposing lists of values that should be taken into account while giving examples of how they should be used to inform the law in concrete areas or cases.\footnote{See, e.g., \cite{alexander14,alexander11,dagan07}.} Unsurprisingly, however, views soon diverge regarding the concrete import of a social function view on legal reasoning. Even so, the contemporary debate appears to be based on a common ground that is quite stable, also with respect to the overall notion of what good the theory can do. But as history shows, this state of affairs is by no means guaranteed. 

In a recent article, Anna di Robilant illustrates this point exceptionally well by tracking the history of social function theorizing in Italy during the fascist era. The fascist property scholars, she notes, were happy to embrace the social function theory, since it provided them with a conceptual starting point from which to develop their idea that rights and obligations in property should be made to answer to one core value: the interests of the state.\footnote{See \cite[908-909]{robilant13} (``Fascist property scholars had also appropriated the social function formula. For the Fascists, the social function of property meant the superior interest of the Fascist state.'').} This stance was as effective as it was oversimplified. As di Robilant notes, ``earlier writers had been hopelessly evasive about the meaning and content of the social element of property''.\footcite[909]{robilant13} Hence, the fascist approach filled a need for clarity about the implications of the main idea, which was by now attracting increasing support both from the public and the academic community. Established property doctrine, it was widely felt, was both ineffective and unfair to ordinary people. Rather than securing productivity and a livelihood for all, property was used mainly as an instrument for maintaining the privileged position of the elites. By promising to change this state of affairs, the fascists attracted many to their cause.

As di Robilant notes, supporters of the fascist idea of property made clear that ``social function meant the productive needs of the Fascist nation''.\footcite[909]{robilant13} But at the same time, they cleverly denied that there was a ``contradiction between subordinating individual property rights to the larger interest of the Fascist state and the liberal language of autonomy, personhood, and labor''.\footcite[900]{robilant13} In this way, fascist scholars could claim that fascist liberalism was true liberalism, thereby subverting the conceptual basis for the traditional idea of liberal justice.\footcite[900]{robilant13} In this situation, there was reason to suspect that clinging to liberal dogma would be a largely ineffective response. Moreover, it seemed undeniable that fascism's appeal was rooted in real concerns about the fairness and effectiveness of the liberal legal order. 

Hence, many non-fascists shunned away from uncritical defense of traditional liberalism. Instead, they agreed that property's social function should come into focus, but emphasized the plurality of values that could potentially inform this function, not the interests of the state. In addition, they also noted that property rights were invariably associated with {\it control} over resources, and that the social functions of property depended on the resources in question. To own property, they argued, provides individuals with a source of privacy, power and freedom that is, as a matter of fact, highly valued. It is valued, moreover, for its implications in a social context. To capture these insights, Italian scholars adopted the metaphor of a ``tree'', by describing the core social function of property as the trunk, while referring to the various resource-specific values attached to property as branches.\footcite[894-916]{robilant13} As di Robilant notes regarding these theorists:

\begin{quote}
The rise of Fascism, they realized, was the
consequence of the crisis of liberalism. It was the consequence of liberals' insensibility to new ideas about the proper balance between individual rights and the interest of the collectivity.\footcite[907]{robilant13}
\end{quote}

In light of this, the tree-theorists concluded that continued insistence on the protection of the autonomy of owners was not a viable response. Instead, they adopted a theory that ``acknowledges and foregrounds the social dimension of property'', but without committing themselves to fascist ideas about the supreme moral authority of the state.\footcite[907]{robilant13} The value of autonomy was in turn recast in terms of property's social function. Arguably, this served to make the case far more compelling. Protecting autonomy could be seen as an aspect of protecting property's freedom-enhancing function, both at the individual level and as a way of ensuring a right to self-governance and sustenance for families and local communities. This, moreover, could not easily be derided as tantamount to protecting unfair privilege and entitlement. In fact, the suggestion was made in an effort to protect democracy itself.

I believe the story of fascist appropriation of the social function theory provides further weight to my claim that it is sensible to  maintain a descriptive perspective on its core features. Indeed, the readiness with which the fascists embraced social function theorizing serves as a reminder that we cannot easily predict what normative values may come to be promoted on its basis. Hence, it is also call for continuous vigilance when it comes to normative assessment and debate. At the same time, we are reminded of the danger of attaching too much normative prestige to a theory that is abstract and open to various interpretations.

In particular, it seems to me that failure to recognize the descriptive nature of the core idea can lead to unrealistic expectations of what the social function theory actually provides. In addition, it will make it harder for the theory to gain acceptance as a conceptual common ground from which to depart when engaging in debate. Indeed, if no division is recognized between normative and descriptive aspects, the historical record would allow detractors to make a {\it prima facie} plausible attack on the social function theory by arguing that it is fascism in disguise, or that fascism, rather than liberal justice, is where we end up in practice should we chose to adopt it.\footnote{This would echo the claim already made by Claeys, that the theory (when coupled with virtue ethics) might become a slippery slope towards the kind of extremism and revolt against oppression that gave rise to the Rwanda genocide in the early 1990s \cite[926-927]{claeys09}.}

In response, one might retort that this is cherry picking the historical facts, or that the fascists misunderstood or perverted the theory. That is certainly plausible, but the point I am trying to make here is that this kind of debate is in itself unhelpful. Unless the social function theory is rendered neutral enough to be acceptable as the conceptual premise of debate, it is likely going to fail -- in a purely epistemic sense -- as a template for negotiating conflicts about property. Those who oppose the norms associated with the theory will oppose also the core descriptive content, if they feel that the latter commits them to the former. I believe that this, in turn, suggests that those advocating on behalf of the social function theory should take care to avoid rhetorical hubris. The main point to convey, I believe, is that the theory is in fact more accurate, in a purely epistemic sense, than other conceptualizations of property.

The story of the fascist appropriation of the social function theory also points to the danger that often attach to abstract theories with normative implications: That they allow us to opportunistically recast whatever values we wish to promote, by providing qualifications for them in abstract terms that are hard to refute. The fascists did this, and the non-fascists responded. Hence, in the end one could do little more than hope that the fascists' vision of their state as an ``ethical state'' that ``every man holds in his heart'' would eventually prove less attractive then the promise of self-governing communities bustling with diversity in life and character.

\subsection{Towards a normative stance}

The social function theory can facilitate a new kind of normative reasoning, arising from how the theory allows us to recognize more subtle distinctions between different kinds of property and different kinds of circumstances. For instance, staunch entitlements-based approach to autonomy will leave us with little room to differentiate between the protection of investment property and the protection of a home, unless such a distinction is explicitly provided for in the law. But a social function approach compels us to notice the difference and to acknowledge that it might be legally, as well as ethically, relevant. Hence, if we seek to argue for protection of investment property, we must in principle be prepared to face counter-arguments that revolve around particulars of the investor's role in society and his relationship to the community of people that are affected by how he manages his property. Similarly, if someone argues against protecting home ownership, we can respond by drawing on additional arguments based on the importance of the home both to the owner, her family and friends, and her community. Under the social function theory, it becomes generally relevant to address how a home creates a sense of belonging and provides a basis for membership in social structures.

I believe normative assessments should aim to be as concrete as possible. That said, I still think it is worthwhile to provide more abstract forms of expression for core values, to clarify the ethical premises that provide the basis for concrete value-based conclusions. To me, therefore, normative theories should aim to be meta-ethical, not just ethical. They should provide a vocabulary and a conceptual framework tailored to advancing one's values. However, they should recognize that the ultimate expression of those values is given in relation to concrete facts. This, I believe, is prudent in light of how abstract ethical assertions are necessarily imprecise, and run the risk of being distorted or exaggerated, particularly as they gain influence.

Invariably, the most accurate information regarding the values I rely on when assessing cases will be conveyed by my assessment of the cases, not by my theorizing. On a deeper level, I am inclined to believe that value-systems are more or less unique to individuals, so that ethical theories are helpful primarily in that they provide an introduction to keywords and important lines of argument that will recur in different forms. As such, they enhance understanding, making it easier to communicate ideas and opinions in such a way that potential respondents are likely to enjoy a somewhat less inaccurate impression of what they are responding to. 

In short, I believe that ethics make moral judgments communicable, allowing new ideas to be created in the minds of individuals. It should come as no surprise to the reader, therefore, that I believe in ethical men and women, but not in ``ethical Man'' or -- God forbid --  the ``ethical State''. Luckily, I find some support for this world view in recent theories that have been proposed as normative extensions of the social function theory of property. These are the subject of my next section.

\section{Human flourishing}\label{sec:hf}

Taking the social function theory seriously forces us to recognize that a person's relation to property can be partly constitutive of that person's social and personal identity, including both its political and economic components. Hence, property influences people's preferences, as well as what paths lie open to them when they consider their life choices.\footnote{See generally \cite{alexander09}.} This effect is not limited to the owner, it comes into play for anyone who is socially or economically connected to property in some way. The life-significance of property might be clearly felt by a potentially large group of non-owners as well.\footcite[128-129]{alexander09d} The importance of property is obviously reduced if we move away from it in terms of social or economic distance. Hence, in many cases, property will be most important to its owner, simply because she is closest to it. This is not always the case, however, especially not if property rights are unevenly distributed, or in the possession of disinterested or negligent owners. Moreover, as mathematically oriented sociologists take pride in pointing out, social connections are ubiquitous  and the world is often smaller than it seems.\footnote{See generally \cite{schnettler09}.}

Hence, there is certainly potential for making wide-reaching socio-normative claims on the basis of this perspective on the meaning and content of property. But which such claims {\it should} we be making? According to some, we should adjust our moral compass by looking to the overriding norm of {\it human flourishing} as a guiding principle of property law. Colin Crawford, for example, explicitly argues that the social function theory of property should ``secure the goal of human flourishing for all citizens within any state''.\footcite[1089]{crawford11} In a recent article, Alexander goes even further, by declaring that human flourishing is the ``moral foundation of private property''.\footcite[1261]{alexander14} 

As I have already explained, I believe -- in contrast to both Crawford and Alexander -- that it is useful to decouple such normative claims from the descriptive core of the social function theory.\footnote{Crawford comments that the social function theory on its own  ``is not self-defining and invites many interpretations'', see \cite[1089]{crawford11}. The normative theory he proposes is clearly aimed at filing this perceived gap, by pinning down normative commitments that Crawford believes are intrinsic to the theory. However, as I have already argued, I reject this approach, since it unwisely downplays the fact that the social function theory can serve as a common ground among commentators with widely divergent normative views. Indeed, Crawford himself refers unfavorably to a writer who addresses the social function theory, but who, according to Crawford, proposes that ``property's social function is best served by focusing on overall economic production and efficiency in a given society, allowing the market's invisible hand to work its magic'', \cite[see][1089]{crawford11}. Against Crawford, I would argue that it is better to counter such a claim by arguing why it is normatively wrong than by suggesting that people with such values should be discouraged from attempting to argue for them on the basis of a social function understanding of property. Rather, by encouraging such an argument it should become easier to make the case why the values promoted are ultimately undesirable. This, at least, should follow if Crawford is otherwise largely correct (as I think he might be).} I therefore refer to the more distinctly normative aspects of their work as human flourishing theorizing. 

Human flourishing has a good ring to it, but what does it mean? According to Alexander, several values are implicated, both public and private.\footnote{See generally \cite{alexander14,alexander11}.} Importantly, Alexander stresses that human flourishing is {\it value pluralistic}.\footnote{\cite[750-751]{alexander09}.} There is not one core value that always guarantees a rewarding life. To flourish means to negotiate a range of different impulses, both internal and external. Importantly, these all act in a social context which influences their meaning and impact.\footcite[1035-1052]{alexander11}

For the family of a homeowner, the value of the ownership tends to be great; a home is a home for any non-owner living there, just as much as it is a home for the owner. This, in turn, creates both commitments and opportunities for the owner, which may or may not find recognition in the law and our legal reasoning. Regardless of this, it certainly carries significant importance both to her life and the lives of those that depend on her. If property is rented out as a home to someone else, the importance of ownership may be {\it greater} to a non-owner. Indeed, assuming a society where tenancy is a well-functioning social institution, the continuation of the established property pattern might well be of greater importance to the tenant than it is to the owner.

The effect on non-owners can also be restrictive in socially desirable ways. If an apartment has an owner, it discourages squatters, for instance. Moreover, this effect clearly depends also on {\it who} the owner is and the choices she makes in managing her property. If the owner lives in the apartment, squatting is hopefully not going to be an issue. But even the owner of an unoccupied apartment can discourage squatting by managing her property well. However, if owners mismanage their apartments, for instance because they seek to obtain demolition licenses, squatters can take opportunity of this. The risk, of course, increases if housing cannot be afforded by a large number of society's members. In this case, it is natural to argue that something is amiss with the prevailing property structures.

Now, the social function theory of property can also come into play, since it allows us to attach significance to this also when discussing the property rights of individual owners.  In particular, we are not compelled to pretend as though possible failures of property as a social institution are irrelevant when considering rights and responsibilities attached to it. As a matter of fact, they are not; actual squatters clearly affect the owner, influencing both the meaning and the value of her property, both to her, potential buyers, the local government, the state, and other interested parties. Even the mere {\it risk} of squatting can play such a role. But a property theory which does not recognize the social function of property might not allow us to take this into due regard. As long as the standard expectation of an owner is to be able to enjoy her apartment free of squatters, an entitlements-based view on property could easily force us into denial regarding actual (risk of) squatters.

In particular, we would be led to consider squatting as an interference with the owner's rights which the state can not, on pain of disrespecting property, recognize as a legitimate response to mismanagement and imbalances in the property structure. The normative significance of real life -- where squatting often happens due to badly managed property -- is discounted  because our conceptual glasses block it out. Then, the almost unavoidable consequence is that the state also recognizes a {\it positive} obligation to forbid squatting, and to forcibly remove squatters on behalf of owners. Under a narrow entitlements-based conception, this is the natural outcome, and must be classified as an act of protecting private property. Hence, under classical liberal values, it also becomes {\it good}. Here, however, the social function theory permits us to take a highly divergent view, to carry forward different value-judgments.

In particular, if squatting is recognized as creating new interests and obligations attaching to the property, it may now be argued that  it is the use of state power to evict that is the most severe act of interference. Not only interference in whatever housing rights the squatters may have, but in fact also as an interference in {\it property}. Hence, such state action might itself be morally suspect and held to be in need of further justification. In the Netherlands, the Supreme Court adopted a line of reasoning reflecting these insights, when it held that the right not to be disturbed in one's home life also applied to squatters. Hence, the property owner could not forcibly evict people who had taken up residence in her property.\footnote{See NJ 1971/38. The court held that the lower court had erred in taking it proven that the ``house in the original charge was ``in use'' by the owner of this house'', as required by the statute under which the squatters were tried. Instead, the Supreme Court held that ``art.138, in so far as it mentions houses, is specifically aimed at protecting home rights, in connection with which the words ``in use'' (differently than the court judged) can only be understood as ``actually in use as a house'' , as in accordance with ordinary use of language''. The upshot was that it was the squatters, not the owner, who enjoyed protection under the statute. In terms of the bundle theory, a right thought to be in the owner's bundle was deemed to actually belong to the bundle of the squatters, as this corresponded better to the circumstances of the case and the purposes meant to be served by the statute in question.}

In South Africa, a somewhat similar line of reasoning was adopted in the recent case of {\it Modderklip East Squatters v. Modderklip Boerdery (Pty) Ltd}, analysed in depth by Alexander and Pen\~{n}alver.\footcite[154-160]{alexander11} The case dealt with squatting on a massive scale: Some 400 people had taken up residence on land owned by Modderklip Farm, apparently under the belief that it belonged to the city of Johannesburg. The owner attempted to have them evicted and obtained an eviction order, but the local authorities refused to implement it. Eventually, the settlement grew to 40 000 people and Modderklip Farm complained that its constitutional property rights had not been respected.

The Supreme Court of Appeal concluded that Modderklip's property rights had indeed been violated, but noted that so had the rights of the squatters, since the state had failed to provide them with adequate housing.\footnote{See \cite{modderklip04}.} However, they upheld the eviction order and granted Modderklip Farm compensation for the state's failure to implement it. The Constitutional Court, on the other hand, while agreeing that the eviction order was valid, concluded that as long as the state failed in its obligations towards the squatters, the order should not be implemented.\footcite{modderklip05} The eviction of the squatters, in particular, was made contingent upon an adequate plan for relocation. In the meantime, Modderklip would receive monetary compensation, from the state rather than the squatters. In this way, the Court recognized the social function of property; they refused to give full effect to Modderklip's property rights as long as that meant putting other rights in jeopardy. The fact that the squatters had no place to go, in particular, was allowed to influence the content of Modderklip's right, making it impermissible to implement a standing eviction order.

It is possible to cast this outcome as an interference in property rights that was regarded as acceptable in the public interest. However, the reconceptualization in terms of property itself having a social function appears highly attractive. Moreover, it is also consistent with the South African constitution, which also focuses on property's social dimension.\footnote{See section 25 of the Constitution of the Republic of South Africa, Act 108 of 1996.} Thinking about cases like {\it Modderklip} in terms of property's social function allows us to remove the state as an intermediary between the owner and the other interested parties, in this case the squatters. As argued by Alexander and Pe\~{n}alver, it becomes possible to think of the Court as adjudging based on Modderklip's own responsibility, as an owner, towards other members of the community that have an interest in the property.\footnote{\cite[157]{alexander11} (``The courts' unwillingness to ratify Modderklip's desire to remove the squatters from its land illustrates the courts' willingness to take seriously the obligations of owners, not only as they concern owners' direct relationship with the state but also in relation to the needs of other citizens'').}

On this basis, it becomes easier to conclude that it is permissible for courts to take the social context into account even in the absence of any specific state action or legislation to indicate that this should be done, or that the public interest is at stake. Indeed, one of the problems in {\it Modderklip} was that the state had failed also in its responsibility towards the squatters. Moreover, while the local sheriff had refused to implement it, an eviction order had in fact been granted. Hence, thinking of the case as interference in the public interest becomes difficult.

More importantly, by taking into account the social function of property, it becomes possible to argue for the outcome in Modderklip positively on the basis of property values. In this way, property is no longer seen to stand in the way of justice in cases such as this. We need not ``interfere'' with rights to secure an appropriate outcome, we only need to apply property law. As Alexander puts it in another recent article: 

\begin{quote} The values that are
part of property's public dimension in many instances are necessary
to support, facilitate, and enable property's private ends.
Hence, any account of public and private values that depicts them as categorically
separate is grossly misleading. One important consequence of this
insight is that many legal disputes that appear to pose a conflict between
the private and public spheres or that seemingly
require the involvement of public law can and
should, in fact, be resolved on the basis of private law -- the law
of property alone.\footcite[1295-1296]{alexander14} \end{quote}

Protection of property, when property is understood in this way, becomes a potential source of justice, also for squatters. The basic values attached to property -- freedom, liberty, autonomy -- have not really changed, but are applied in a new way. In particular, they no longer only apply to the owner's interest in property, but also to that of other individuals closely connected to it. This normative turn, I argue, will potentially strengthen the institution of property itself, while also serving to loosen the compulsiveness of the idea that the ultimate expression of the public interest is found in the actions taken by the state. It suggests rather the view that the public interest manifests wherever the public may reside, including in property. This conclusion requires taking a normative stance, but a minimal one; we merely extend the scope of values traditionally attached to property.\footnote{Arguably, cases such as {\it Modderklip} might be taken to suggest that the social function theory, as soon as it is applied for the purposes of normative assessment, will systematically guide us to conclude that owners are not entitled to as many benefits as would otherwise follow from their property rights. It is fortunate, therefore, that the entire remainder of the thesis will focus on economic development takings, where it will typically appear more natural to conclude the opposite. In these cases, on a common- sense understanding of justice, applying the social function theory will allow us to recognize a sense in which owners should receive {\it increased} protection and more benefits, as a consequence of how such interferences can prove particularly damaging, both to the owner and to the social fabric of democracy.} 

That said, in the case of {\it Modderklip} the court was clearly faced with a value conflict that it is hard to resolve by looking to traditional liberal values. If these apply equally to the squatters, we are left with deadlock rather than resolution. Indeed, this was also reflected in the outcome of the case, which did not resolve the matter, but merely concluded that the state had failed in its obligations towards both of the parties. What should the solution be in the end? Should the squatters be allowed to stay, following condemnation of Modderklip's land, or should alternative housing be provided so that the eviction order can be carried out? The answer requires us to resolve a normative conflict, and how to do so might not be obvious. Moreover, value pluralism suggests that we must be prepared to engage with multiple ways of looking at the matter. In the interest of stability of property as an institution, allowing the squatters to succeed in establishing lasting title to the land might be considered unwise. Against this pragmatic and largely technical value, one would have to consider the values of community and belonging that now attach the squatters to their new homes. These two values are largely incommensurable, and it is not clear how to choose between them.

Still, Alexander maintains that human flourishing provides an ``objective'' standard on which to approach dilemmas such as these. Moreover, he ``rejects the view that what is good or valuable for a person is determined entirely by that person's own evaluation of the matter''.\footcite[1263]{alexander14} Some things are good for people, Alexander argues, irrespectively of whether or not people know so themselves. Hence, it may perhaps be argued that what is truly good for Modderklip is to come to an arrangement with the squatters and the state, to resolve the problem amicably. Moreover, failure to do so might entitle the state to take action that would otherwise seem to undermine the stability of property. This, then, would be partly due to this being conducive also to the flourishing of the people behind Modderklip, not only the squatters.

That, clearly, might be derided as an overly intrusive and moralistic way to approach property law. More generally, as Alexander notes, the exact content of goodness is ``necessarily contestable''. It consists of a list of different values which are all open to dispute, both as to their relevance and their precise meaning.\footcite[1263]{alexander14} Alexander goes on to list some key values that he believes are central, but the list is not meant to be exhaustive.\footcite[1764-1776]{alexander14}

Among the key values that Alexander discusses, we find many core private values that are commonly seen as important goals for the institution of property. This includes values such as autonomy and self-determination, both of which will feature heavily later in this thesis. However, Alexander also considers several public values, such as equality, inclusiveness and community. These too will be important later, as I will draw on them in my own normative analysis of economic development takings. I will be particularly concerned with the value of {\it participation}, understood, following Alexander, in terms of its broad social function.\footcite[1275-1276]{alexander14}

In my view, this value is closely related to the value of democracy. Participation in local decision-making processes is the root which enables democracy to come to fruition at the regional and national level. Moreover, participation is a value that will give me occasion to make particular policy suggestions regarding the correct way to approach economic development takings. Devoting some time to discussing this value in the abstract will therefore be helpful.

Alexander traces the value of participation back to Aristotle and the republican tradition. He notes, however, that this tradition involves a notion of participation that is somewhat narrowly drawn. For thinkers in the republican tradition, participation tends to mean public participation, meaning people's engagement with the formal affairs of the polity.\footcite[1275]{alexander14} For Alexander, participation has a broader meaning, involving also the value of being included in a community. He writes:

\begin{quote}
We can understand participation more broadly as an aspect of inclusion. In this sense participation means belonging or membership, in a robust respect. Whether or not one actively participates in the formal affairs of the polity, one nevertheless participates in the life of the community if one experiences a sense of belonging as a member of that community.\footcite[1275]{alexander14}
\end{quote}

Importantly, participation in a community can have a crucial influence also on people's preferences and desires. In this way, it is also invariably relevant -- behind the scenes -- to any assessment of property that focuses on welfare, utility or public participation in the classical sense. As Alexander and Pe\~{n}alver put it, drawing on the work of Amartya Sen and Martha Nussbaum:\footcite{sen84,sen85,sen99,nussbaum00,nussbaum02}
\begin{quote}
The communities in which we find ourselves play crucial roles in the formation of our preferences, the extent of our expectations and the scope of our aspirations.\footcite[140]{alexander09}
\end{quote}
Therefore, for anyone adhering to welfarism, rational choice theory, utilitarianism or the like, neglecting the importance of community is not only normatively undesirable, it is also unjustified in an epistemic sense. In particular, it should be recognized as a descriptive fact that community is highly relevant to {\it any} normative theory that attempts to take into account the preferences and desires of individuals.\footnote{Again, I think Alexander and other theorists attempting to incorporate such ideas in property law could benefit from making this descriptive point separately, so as to enable it to be considered in isolation from the more contentious normative arguments they construct on the basis of it.} But Alexander and Pe\~{n}alver go further, by arguing that participation in a community should also be seen as an independent, irreducibly social, value, not merely as a determinant of individual preferences and a precondition for rational choice. They write:

\begin{quote}
Beyond nurturing the individual capabilities necessary for flourishing, communities of all varieties serve another, equally important function. Community is necessary to create and foster a certain sort of society, one that is characterized above all by just social relations within it. By ``just social relations'', we mean a society in which individuals can interact with each other in a manner consistent with norms of equality, dignity, respect, and justice as well as freedom and autonomy. Communities foster just relations with societies by shaping social norms, not simply individual interests.\footcite[140]{alexander09}
\end{quote}

This, I think, is a crucial aspect of participation. Moreover, it is one that it is hard, if at all possible, to incorporate in theories that take  preferences and other attributes of individuals as the basis upon which to reason about property. For instance, if people in a community come under pressure to sell their homes to a large commercial company that wishes to raze them in order to construct a shopping mall, it may be appropriate to consider this as an unjustifiable attack on their property rights. Importantly, this may be so {\it irrespectively} of what the individual owners themselves think they should do. If they are offered generous financial compensations for their homes, or are threatened by eminent domain, economic incentives might trump the value of social inclusion and participation for all or a majority of these owners. As a consequence, the community might decide to sell.  

Even so, in light of the value of community, it would be in order for planning authorities, maybe even the judiciary, to view such an  agreement as an {\it attack on their property}. It is clear, in particular, that by the sale of the land, the ``just social relations'' inhering in the community will be destroyed. The members of the community -- including all the non-owners -- will lose their ability to participate in those relations. More concretely, the nature of the property rights that once contributed to sustaining ``just relations'' will now be transformed into property rights that serve different purposes. This includes aiding the concentration of power and wealth in the hands of commercially powerful actors. Such a change in the social function of property might have to be regarded -- objectively speaking -- as a threat to participation, community and democracy. Hence, on the human flourishing theory, it is also a threat to property. Our property institutions, therefore, should protect against it.

To demonstrate the general significance of such a line of normative reasoning, it is illustrative to mention a scenario -- not directly implicating property -- that is currently beginning to attract much attention in legal scholarship. This scenario arises in relation to the right to {\it privacy}. This right, of course, is increasingly perceived to be coming under threat in the information age. Crucially, it is beginning to become clear to legal theorists that viewing privacy merely as a private right is not going to provide a sustainable template for dealing with this challenge.\footnote{See generally \cite{schafer14}.} It seems, in particular, that people are simply too willing to give it up. This, in turn, contributes to the formation of potentially harmful social structures on the web. In particular, the lack of privacy becomes an impediment to dignity, freedom and respect in web societies. In this way, both individuals and society as a whole will eventually suffer, although this truth is not reflected in our individual preferences. Hence, it has been proposed that privacy should be considered also as a {\it common good}, so that protecting the privacy of individuals, in some cases, is an imperative irrespectively of what these individuals themselves desire and prefer. Privacy, in this way, becomes also an obligation, mirroring the similar phenomenon that we have observed with respect to the right to property.

There is a subtle issue that arises on the basis of this kind of normative reasoning about individual rights. Is it appropriate, in particular, to still think of such reasoning -- and the obligations it gives rise to -- as an aspect of protecting individuals? Is it not more accurate to say that this is an {\it interference} with individual rights, undertaken to further the public interest? Indeed, when the individual himself does not want his property or privacy to be ``protected'', is it not somewhat perverse to insists that this is what is happening? 

I am inclined to answer in the negative. In my opinion, we are still talking about protecting individual rights, even when this means imposing protections on people that they themselves do not want. Undoubtedly, this is {\it also} an interference in their rights, but just as different rights of different people can sometimes come into conflict, I am inclined to think that the same right, for the same person, can sometimes come into conflict with itself. This happens, in particular, when it is not possible to simultaneously protect all those functions that this right seeks to promote. 

For instance, if someone protests a taking on environmental grounds and also rejects financial compensation as immoral, the courts should still award just compensation for the land, if they find that the taking is valid. If the owner wishes, he can purge himself by making a donation to charity. Similarly, if someone attempts to commit suicide, the health services are still obliged to help, even against the patients wishes. This remains the case, moreover, even though suicide is no longer considered a criminal offense in the public interest. 

Protecting individuals against their will is condescending, no doubt, but it is still different, and often preferable, from subordinating their interests to that of the general public. If the justification for an act of interference is a vague proclamation of the ``public interest'', the individual is marginalized from the very start. A balancing act might be required, but this renders the individual relevant only to one side of the equation. On the other hand, if the act of interference is simultaneously rendered as protection, enforcement of an obligation, or a measure to enable participation, the individual occupies center stage. In so far as the public interests triumph, it is not because the individual loses, but because the public is deemed to know best how to secure the goal of human flourishing, both for the individual herself and other members of the social structures that surrounds her.

For instance, external interests of both a private and a public nature can dictate that owners should avoid becoming a nuisance to their neighbors. But under a human flourishing theory, we are also able to portray this as a case of protecting the individual's membership in the community. The public does not ``side with the neighbors'', but undertakes measures to protect the relationship between the owner and his fellows. In my opinion, a conceptual approach to property law that makes this portrayal plausible is highly desirable. 

For a second example, consider situations when environmental concerns suggest imposing restrictions on what an owner is permitted to do with his land. This too can be rendered as an act of protecting property. But doing so requires the regulatory body to relate the interference positively to the individual's interests and obligations, to ensure that they avoid adopting a narrative where the regulation is rendered as an act of enforcing the will of unnamed others against the will of specific owners. In this way, public values and the public interest can be given considerable weight, but will have to be rendered less abstract. In particular, these interests must be related concretely to the social functions of the rights protecting the individuals interfered with. The baseline for assessment remains actual persons and their well-being, not some abstract ideal of ``goodness''. Moreover, implementation of the collective will becomes a guide towards human flourishing for a society of individuals, not a goal in itself.

An individual might well be offended if the state adopts this narrative and implements behavioral restrictions by declaring ``it's for your own good''. But, I would argue, that is exactly as it should be. Any restriction of individual freedom is an offense, but one that is sometimes appropriate. If this is conveyed to people with a marginalizing ``your interests are not as important as ours'', the response might well be silence. But beneath the silence we may find disinterested apathy, or worse: contempt and despair. The interference is no longer an insult, but this is not because it is any more convincing to proclaim that interference is ``necessary for the greater good''. Rather, the interference is no longer an insult because it fails to properly engage the individual at all. The role of the person interfered with becomes passive -- she becomes an obstacle that needs to be removed. If such a dynamic of governance develops, the individual might take from this the lesson that she is unimportant in the greater scheme of things, that her interests are subordinate to those of ``the others'', and that her voice is not meant to be heard.

This is normatively undesirable. It represents a situation when the social effect of interference might become detrimental to society itself, particularly to the institution of democracy. It damages its roots, namely the ``just social structures'' that Alexander identifies as being at the core of the human flourishing theory. A better alternative, then, is to interfere in a way that constructively targets the individual, aiming to protect her by enabling her -- and compelling her -- to protect others and partake in social and political life. This can then become interference aimed at bringing the individual into the fold, making her play her part, by raising her to fruitful citizenship. Such a paternal (or maternal) state is one that cares, but one that may also be overprotective, unfair, or plain stupid. Hence, it becomes natural to resist and to revolt, but not without also carrying forward care and love for the social, political and legal structures within which this agency is (hopefully) permitted to take place.

The upshot, I believe, is that condescension in property law can be a good thing. To conceptualize an act of restriction as a means to empower the persons restricted is something they might well find offensive, but it also renders interference more meaningful to them. It provides both a reason to take a more active role in relation to the interfering power, and a possible cause for constructive resistance. Importantly, it does not force the conclusion that the public resides behind closed doors, disinterested in what the affected individual have to offer. Instead, it is an approach that encourages a response, by focusing always on the persons interfered with, whenever interference is deemed necessary. This is the vision of a bottom-up, rather than a top-down, approach to imposing the collective will on individuals. I believe it has merit. 

It will remain in the background as I now move on to apply the theories discussed in this and preceding sections. In the next section,  I return to the issue that will remain in focus for the remainder of this thesis. First, I will introduce economic development takings by considering the seminal case of {\it Kelo v City of New London}\footcite{kelo05}, which brought this category to prominence in the US discourse on property law. Then I will assess the unique aspects of such takings against the social function theory, to provide an argument that the category has significance for legal reasoning in takings law, as well as with respect to property as a constitutionally protected human right. Finally, I will provide an abstract presentation of the values that I believe should be considered important when normatively assessing the law in this area. In doing so, I will draw on the human flourishing theory, setting out the main values that will inform the concrete policy assessments I provide later. 

\section{Economic development takings}\label{sec:edt}

Constitutional property rules in many jurisdictions indicate, with varying degrees of clarity, that eminent domain should only be used to take property either for ``public use', in the ``public interest'', or for a ``public purpose''. Such a restriction can be regarded as an unwritten rule of constitutional law, as in the UK, or it can be explicitly stated, as in the basic law of Germany.\footnote{See Chapter \ref{chap:2}. Section \ref{sec:contrast} below.} In some jurisdictions, for instance in the US and in Norway, explicit property clauses exist, but are not formulated clearly.\footnote{See Chapter \ref{chap:2}, Section \ref{sec:us} and Chapter 3, Section \ref{sec:norexp} below.}

Both the Norwegian and the US property clauses appear to refer to public use only as a precondition for the duty to pay compensation. However, they are also universally read as expressing the {\it presupposition} that the power of eminent domain is only to be used in the public interest.\footnote{In the literature, it is rare to even note that a different interpretation is linguistically possible. But see \cite[205]{berger78}.} Indeed, in cases when one might say that private property is ``taken'' for a non-public use without compensation, for instance in a divorce settlement, it is not commonly regarded as an exercise of eminent domain. Rather, it is justified by making reference to a different category of rules, meant to ensure enforcement of obligations that arise between private parties independently of the state's power to single out and compulsorily acquire specific properties.

The exact boundary between eminent domain and other forms of state interference in property may not always be clear, but I will not worry too much about it in this thesis. I note, moreover, that most, if not all, legal scholars seem to agree that the power of eminent domain is meant to be exercised in the public interest. However, differences of opinion emerge when we turn to the question of whether the presupposed public use or public interest in the property taken serves also to restrict the power to take. In the US, most scholars agree that some restriction is intended, but there is great disagreement about its extent.\footcite[205]{berger78} In Norway, on the other hand, a consensus has developed that the public use limitation is so wide that it hardly amounts to a restriction at all.\footnote{See, e.g., \cite[368]{aall10}.} Moreover, the courts defer almost completely to the assessments made by the executive branch regarding the purposes that may be used to justify a taking.\footcite[368]{aall10}

Some US scholars adopt a similar stance, but others argue that the public use presupposition should be read as a strict requirement, forbidding the use of eminent domain unless the public will make actual use of the property that is taken.\footnote{Compare \cite{bell06,bell09,claeys04,sandefur06}.} Most scholars fall in between these two extremes. They regard the public use restriction as an important, practically relevant, limitation, but they also emphasize that courts should normally defer to the legislature's assessment of what counts as a public use.\footnote{See, e.g., \cite{merrill86,alexander05}. The fact that US jurists usually stress deference to the legislature, not the executive branch, should be noted as a further contrast with Norway.}

As I discuss in more depth in Chapter \ref{chap:2}, Section \ref{sec:hop}, the debate in the US has its roots in case law developed by state courts -- the federal property clause was for a long time not enforced against states. This has changed, however, and today the Supreme Court has a leading role also in this area of US law. It has developed a largely deferential doctrine, resembling the understanding of the public use limitation under Norwegian law.\footnote{See \cite{berman54,midkiff84,kelo05}.} The difference is that in the US, cases raising the issue  still regularly arise, and still prove controversial. The most important such case in recent times was {\it Kelo}, decided by the Supreme Court in 2005.\footnote{kelo05} This case saw the public use question reach new heights of controversy in the US.\footnote{See, e.g., \cite{somin09}.}

{\it Kelo} centered around the legitimacy of taking property to implement a redevelopment plan that involved construction of research facilities for the drug company Pfizer. The home of Suzanne Kelo stood in the way of this plan, and the city decided to use the power of eminent domain to condemn it. Kelo protested, arguing that making room for a private research facility was not a permissible  `public use''. She was represented by the libertarian legal firm {\it Institute for Justice}, which had previously succeeded in overturning similar instances of eminent domain at the state level.\footnote{See \url{https://www.ij.org/cases/privateproperty}.} Kelo lost the case before the state courts, but the Supreme Court decided to take it on, and they looked at it in great detail.

The precedent set by earlier federal cases was clear: As long as the decision to condemn was ``rationally related to a conceivable public purpose'', it was to be regarded as consistent with the public use restriction.\footcite[241]{midkiff84} Moreover, the role of the judiciary in determining whether a taking was for a public purpose was regarded as ``extremely narrow''.\footcite[32]{berman54} It had even been held that deference to the legislature's public use determination was required ``unless the use be palpably without reasonable foundation'' or involved an ``impossibility''.\footnote{See \cite[66]{dominion25}; \cite[680]{gettysburg96}.}

This understanding had also been reflected in the outcome of concrete cases resembling the situation in {\it Kelo}: In {\it Hawaii}, the Supreme Court had upheld a taking that would benefit private parties, with no direct benefit to the public.\footnote{\cite{midkiff84}. For a more detailed discussion, see Chapter \ref{chap:2}, Section \ref{sec:hop} below.} In {\it Berman}, it had upheld a taking for economic redevelopment of a blighted area, even though the property taken was not itself blighted.\footnote{\cite{berman54}. For a more detailed discussion, see Chapter \ref{chap:2}, Section \ref{sec:hop}.} But in the case of {\it Kelo}, the court hesitated.

Part of the reason was no doubt that takings similar to {\it Kelo} had been heavily criticized at state level, with an impression taking hold across the US that eminent domain ``abuse'' was becoming a real problem.\footnote{See, e.g., \cite[667-669]{sandefur05}.} A symbolic case that had contributed to this worry was the infamous \textcite{poletown81}. In this case, General Motors had been allowed to raze a town to build a car factory, a decision that provoked outrage across the political spectrum.\footnote{See generally \cite{sandefur05}.} The case was similar to {\it Kelo} in that the taker was a powerful commercial actor who wanted to take homes. This, in particular, served to set the case apart from  {\it Hawaii}, which involved a taking in favor of tenants, and to some extent also {\it Berman}, which involved a taking of businesses (and homes) in the interest of combating blight. Moreover, the Michigan Supreme Court had recently decided to overturn {\it Poletown} in the case of \textcite{wayne04}. Hence, it seemed that the time had come for the Supreme Court to reexamine the public use questions.\footnote{See, e.g., \cite{sandefur05,claeys04}.}

Eventually, in a 5-4 vote, the court decided to apply existing precedent and held against Suzanne Kelo. The majority also made clear that economic development takings were indeed permitted under the public use restriction, also when the public benefit was indirect and a private company would benefit commercially.\footcite[469-470]{kelo05} The backlash of this decision was severe. According to Ilya Somin, the case ranks among the most disliked decision that the Court has ever made.\footcite[2]{somin11} Some 80 - 90 \% of the US public expressed great disapproval, with critical voices coming from across the political spectrum\footcite[2108-2110]{somin09} Why did the case prove so controversial? No doubt, the discontent with the decision was fueled in large part by the fact that it was seen as a case of the government siding with the rich and powerful, against ordinary people.\footnote{\cite[630-634]{baron07}} Indeed, the party that appeared to benefit the most from the taking was Pfizer -- a multi-billion dollar company -- while Suzanne Kelo, who stood to lose, was a middle class homeowner. In this context, the taking of Kelo's home seemed morally suspect, an act of favoritism showing disregard for less influential members of society.\footnote{See, e.g., \cite{underkuffler06}.}

In addition, it is worth noting that many commentators conceptualized the {\it Kelo} case by thinking of it as belonging to a special category, by describing it as an economic development taking, a {\it taking for profit}, or, more bluntly, a case of {\it Robin Hood in reverse}.\footcite{somin05} Categories such as these had no clear basis in the property discourse before {\it Kelo}. Indeed, in terms of established legal doctrine, it would be more appropriate to say that the case revolved entirely around the notion of ``public use''. 

However, when we consider the most common reasons given for condemning the outcome in {\it Kelo}, we readily grasp why critics felt it was natural to classify the case along an additional dimension. A survey of the literature shows that many critical voices made use of a combination of substantive and procedural arguments to  paint a bleak picture of the {\it context} surrounding the decision to take Kelo's home. Important concrete factors that critics tend to stress include the imbalance of power between the commercial company and the owner, the incommensurable nature of the opposing interests, the lack of regard for the owner displayed by the decision makers, the close relationship between the company and the government, and the feeling that the public benefit -- while perhaps not insignificant -- was made conditional on, and rendered subservient to, the commercial benefit that would be bestowed on the commercial beneficiary.\footnote{See, for instance, \cite{underkuffler06,somin07,sandefur06,cohen06,hafetz09,hudson10}.}  This dynamic, in which public bodies no longer seem to be leading and pushing the process forward, but are also -- to quite some extent -- being led and being pushed, is regarded as particularly suspicious. This, in turn, is derided as a perversion of legitimate decision-making, used to argue more broadly that economic development takings such as {\it Kelo} suffer from what I will refer to here as a {\it democratic deficit}.

From a theoretical point of view, I take all of this to suggest that many critics of {\it Kelo} effectively adopted a social function view on property, by paying close attention to the wider social and political context of the taking.\footnote{For a particularly clear example of this, see \cite{underkuffler06}.} Importantly, if we now turn to the social function theory of property, we are placed in a position to engage more actively with this form of reasoning, as an integrated part of our assessment of the law. This may then in turn give us cues as to how we should reason -- within the law -- to justify a departure from the course laid down by previous cases on the ``public use'' requirement, where such a perspective was not adopted. Indeed, it seems to me that this is exactly what the minority of the Supreme Court did, particularly Justice O'Connor, who formulated a strongly worded dissent.\footnote{\cite[494-505]{kelo05}. Justice O'Connor was joined by the four other dissenters, but Justice Thomas also formulated his own dissent, taking a more narrow view and arguing for the revival of a strict reading of the public use requirement, see \cite[505-523]{kelo05}.} She writes as follows:

\begin{quote}
Any property may now be taken for the benefit of another private party, but the fallout from this decision will not be random. The beneficiaries are likely to be those citizens with disproportionate influence and power in the political process, including large corporations and development firms. As for the victims, the government now has license to transfer property from those with fewer resources to those with more. The Founders cannot have intended this perverse result.\footcite[505]{kelo05}
\end{quote}

It seems to me that the values Justice O'Connor rely on in her assessment are closely related to the idea of human flourishing presented by Alexander and others, particularly those pertaining to the political function of property as an anchor for community and democracy. Indeed, the danger of powerful groups gaining control of the power of eminent domain does not only affect the individual entitlements of owners. It also affects society, as the economic rationality used to justify interference comes to result in an implicit political statement to the effect that the property of the rich and powerful is better protected, and valued higher by the state, than property owned by regular citizens, who reside in ordinary communities.

The effect of a traditional economic development taking is that property rights are transferred from the many to the few, taken from ordinary people and given to the powerful. Hence, these cases represent a possibly pernicious redistribution of property, not necessarily in financial terms -- depending on the level of compensation -- but surely in terms of property's social function. The structural imbalances of the condemnation process itself find permanent expression in the new distribution of property. The social structures of a living community are dismantled in favor of a social structure that revolves around the commercial interest of a company. The political and social power of the community is diminished, perhaps lost in its entirety, while the political and social power of the company increases.

It seems clear that to Justice O'Connor, this too is a negative consequence of the taking. Again, we notice that recognizing this effect requires a social function approach to property. There is no clearly quantifiable individual loss -- no one particular ``stick'' in the property bundle that is not compensated. Rather, it is the community itself that is lost, a community that was not directly implicated in any ``entitlement'', but which played a crucial role in providing meaning to the totality of the bundle enjoyed by the owner. Even if we extend our perspective to account for indirect individual losses, we are not doing justice to the loss in this regard. The owner might relocate, acquire new property with a similar meaning in a new community somewhere else. But that does not make up for the fact that {\it this} community is lost forever, as {\it this} property takes on new meanings and functions. The loss to Suzanne Kelo, therefore, might  even be a significant loss to the City of New London.

Of course, the economic and social gains of development might outweigh such negative effects on community. But, arguably, the balancing of interests required in this regard can only be carried out by an institution that sufficiently recognizes the owners' and their community's right to participation and self-governance. The presence of a highly active commercial third party, in particular, means that public participation in the standard sense might be insufficient. In economic development takings, the commercial company typically appears alongside the government, as a more or less integrated part of the institutional structure making the decision to condemn. The owners, however, do not enjoy a corresponding level of participation.

In particular, their interests are only negatively defined. They are adversely effected and may object, but under standard administrative regimes they play no constructive role in the process. For instance, they are not called on to take part in the development itself, or to assess its merits more broadly than by being asked to respond based on their own individual entitlements. In fact, I think this is one of the main problems with economic development takings. I will argue for this in more depth later, but I remark here that an important reason to focus on this aspect is that it involves precisely those values that economic development takings are most likely to offend against. In particular, if the loss of community outweighs the positive effect of economic development, this is unlikely to be recognized by a process that relies mainly on the positive contribution of the developer and the expert planners.\footnote{A similar point is made in \cite{underkuffler06}.} 

The objections made by owners, moreover, may not only be given too little weight given the imbalance of power between owners and developers. As long as owners themselves focus only on the individual loss, they may not get to those issues that are the most important for property's social function. However, I do not think it is sufficient to theoretically proclaim that these aspects need to be considered. To address the democratic deficit of economic development takings, it seems likely that institutional changes will have to be made, to give those functions a voice in the decision-making process. This should ensure greater involvement by the local community (including, perhaps, even non-owners) in the decision-making process relating to development. Not only should they be asked if they have objections. They should be be included in a constructive way, perhaps even be compelled to assume an active role in relation to the proposed project.

This is a proposal that envisages owners engaging directly with both government and potential developers, consider alternative schemes, and make their own proposals. In short, this asks for a system where owners participate as a community. According to the human flourishing theory as I understand it, this is not only a right, but also an obligation. It gives a plausible basis on which to strike down economic development takings, and to do so without giving up the value of judicial deference. In addition, it is a call for institutional reform, to search for new governance frameworks that will empower owners and their communities.

It seems to me that Justice O'Connor's argument reflects some of these ideas. Indeed, she seems to believe strongly that the taking of Kelo's home would be a particularly harmful interference in the ``just social structures'' surrounding it. Importantly, a piece-by-piece entitlement-based approach to {\it Kelo} could hardly justify the degree of disapproval seen in Justice O'Connor's opinion. After all, Kelo had been offered generous compensation, there had been no clear breach of concrete procedural rules, and the claim that the taking was {\it only} a pretext to bestow a benefit on Pfizer did not seem supported by the facts.\footnote{See \cite{bell06}.} Rather, it was the overall character of the taking that could be used to argue that it was illegitimate. In this picture, moreover, the perceived lack of a clearly identifiable and direct public benefit becomes only one of several factors.

In addition, the institutional, social and political aspects of the case come into focus. The economic implications are less important to Justice O'Connor. Even the importance of homeownership to personhood does not receive the same attention as structural aspects. The problem which overshadows everything else is the concern that economic development takings represent a form of governmental interference in property that might come to systematically favor the rich and powerful to the detriment of the less resourceful. Hence, such takings may help establish and sustain patterns of inequality. Hardly anyone would openly regard this as desirable; it is not hard to agree that if Justice O'Connor's predictions about the fallout of {\it Kelo} are correct, then this is indeed be ``perverse''. 

The question, of course, is whether her predictions are warranted. This is a call for empirical and contextual assessment of economic development takings, to help us gain a better understanding of how they actual affect political, social and bureaucratic processes. In addition, it raises the question of how to {\it avoid} negative effects, that is, how to design rules and procedures that can reduce the democratic deficit of economic development takings. As I now move away from theory towards concrete assessment of economic development takings, both these questions will be in focus.

\section{Conclusion}

In this chapter, I have presented the core notion of my thesis, that of an economic development taking. I started by noting that while the notion is straightforward enough to define factually, it is far from obvious what implications it has for legal reasoning. I illustrated the subtleties involved by considering a concrete example of a commercial scheme that looked like it might well result in compulsory acquisition of land, namely Donald Trump's controversial plans to develop a golf course on a site of special scientific interest close to Aberdeen, Scotland. In the end, the plans did {\it not} require takings, as Trump was able to make creative use of property rights he acquired voluntarily, against the complaints of recalcitrant neighbors.

This turn of events made the example even more relevant to the points I have been trying to make in this chapter. It served to highlight, in particular, that the question studied in this thesis is not a black-and-white issue that sees privileged property rights enthusiasts on one side of the equation balanced against the good will of the regulatory state on the other. Rather, the example of Trump's golf course allowed me to emphasize the importance of context when assessing both the nature of property rights and the meaning of protecting them. In particular, to protect the property rights of those opposing Trump's golf course was not about protecting just any property, it was about protecting the property of members in a local community that felt it would be detrimental to this community, and to their lives, if Trump was allowed to redefine it. In particular, after Trump decided not to pursue compulsory purchase, protecting the property of these members of the community became a question of {\it restricting} the degree of dominion that Trump could exercise over his own property. Hence, under a conventional and overly simplistic way of looking at these matters, protecting property then became tantamount to restricting its use, a seeming paradox.

To resolve this paradox, and to arrive at a better conceptual understanding of economic development takings, I looked to various theories of property. I noted that there are differences between civil law and common law theorizing about property, but I concluded that these differences are not particularly relevant to the questions studied in this thesis. In particular, I observed that neither the bundle theory, dominant in the common law world, nor the dominion theory, used by civil law jurists, helped me clarify economic development takings as a category of legal thought.

I then went on to consider more sophisticated accounts of property, noting that a range of different {\it normative} theories have been proposed. These differ with respect to the values that they think the institution of property should promote, and as such they were also relevant to the question of assessing economic development takings. However, they do not allow us to zoom in on such takings in a more value-neutral way, to argue that regardless of one's normative persuasions, one should acknowledge that they deserve special attention.

I argued that in order to make this point successfully, the traditional entitlements-based perspective on property had to be abandoned. Instead, I looked to the social function theory of property, which encourages us to take a more contextual perspective on rights and obligations inherent in property. In particular, I noted that the social function theory compels us to recognize the importance of property in regulating social and political relations. Hence, economic development takings are special because they redefine the meaning of the property that is taken and cause a lasting disturbance to the established economic, social and political relationships that exist between owners, communities, state bodies, and commercial actors. The social function theory asks us to acknowledge that property rules are hardly ever neutral with regards to such effects. I identified this as the key observation that allowed me to make sense of economic development takings as category of legal reasoning.

After concluding that the social function theory allowed me to formulate a coherent conceptual basis for studying such takings, I went on to argue that in the first instance, the theory should be understood as giving us purely {\it descriptive} insights into the workings of property and its role in the legal order. In this, I advanced a different stance than many property scholars, by arguing that it would be better to decouple the more normative aspects of the theory, to allow the social function theory to serve as a common ground for further value-based debate.

I then went on to clarify my own starting point for engaging in such debate, by expressing support for the human flourishing theory proposed by Alexander and Pe\~{n}alver. I noted that this theory focuses on how property enables communities and individuals to  participate in social and political processes. I argued that protecting this function of property was good, and that this value should be considered fundamental in property law. Moreover, I noted that the human flourishing theory also contains a further important insight, concerning the scope of the state's power to protect. In particular, the theory asks us to recognize that protecting property against interference that is harmful to human flourishing is a responsibility that the state has even in cases when the individual owners themselves neglect to defend their property, for instance because of financial incentives to remain idle. In other words, some functions of property are such that owners have an obligation to preserve them, while the state has a duty to protect them, potentially even against the will of the owners.

After this, I went on to provide some introductory remarks on economic development takings, drawing on the theoretical insights collected from preceding sections. To make the discussion concrete, I considered the case of {\it Kelo}, which propelled the notion of an economic development taking to the front of the takings debate in the US. I focused particularly on the dissenting opinion of Justice O'Connor, and I argued that she approached the issue in a way that is consistent with the theoretical basis proposed in this chapter.

I will now go on to make my analysis of economic development takings more concrete, by considering how such takings are dealt with in Europe and the US respectively. I note that the category has yet to receive much attention in Europe, so the discussion focuses on the US. Here, the attention this issues has received after {\it Kelo} has been staggering. To get a broader basis upon which to asses all the various arguments that have been presented, I consider the historical background to the issue as it is discussed in the US. This involves giving a detailed presentation of the public use restriction, as it was developed in case law from the states during in the 19th and early 20th century. I then connect this discussion with recent proposals to deal with economic development takings, responding to the backlash of {\it Kelo}, by aiming to address the democratic deficit of such takings.

Later, when I begin to consider the law relating to Norwegian hydropower, I will look back at the theoretical basis provided in the present chapter to guide the analysis. In particular, I focus on certain decision-making mechanisms that have developed on the ground in Norway, as a practical response to the increased tendency for local owners to engage in hydropower development. I will argue that this shows the conceptual strength of the idea that property is irreducibly embedded in community, and that its meaning and function is not -- and should not -- be ordained from above, but should be allowed to arise from its grassroots through continuously evolving institutions of participatory democracy.

%If property rights, particularly rights to land, are distributed fairly in a local community, property is not a privilege. Even if most people do not hold land rights, as long as no one holds excessive amounts, there is no reason why owners and non-owners should not be on equal footing in the local community. They are mutually dependent on one another; non-owners need access to natural resources, while owners need access to services. Moreover, the bonds of community will tend to ensure that owners are deterred from engaging in exploitative practices towards non-owners in much the same way as non-owners are deterred from undermining property rights.

\chapter{Taking property for profit}\label{chap:2}

\section{Introduction}\label{sec:intro}

In the previous chapter, I argued that economic development takings should be considered a separate category of interference in private property. I also placed it in the theoretical landscape, by relating it to the social function theory of property. Economic development takings, I argued, raise questions about the overall effect of interference, questions that require contextual assessment. In particular, I argued that they require us to depart from the individualistic, entitlements-based approach that otherwise dominates in property law.

The significance of a new conceptual category should not be overstated. While I think the economic development label is a very helpful tool when thinking about certain takings cases, I am not suggesting that these cases should be approached uniformly and on the basis of mechanical legal assessment. Rather, the importance of context indicates that a concrete approach is in order. Moreover, while I have argued for a certain way of reasoning about economic development takings, I have so far said little about what the law has to say about them. In this chapter, I consider this question, by giving an overview of how economic development cases are dealt with in some representative jurisdictions.

First, I will comment briefly on the importance of economic development takings on the global stage.

\section{The ``underscrutinized'' language of economic development}\label{sec:lgppp}

Public-private partnerships are becoming increasingly important to the world economic order.\footnote{See generally \cite{saussier13}.} To some, they are the illegitimate children of privatization and deregulation, while others see them as efforts to make the public sector more efficient and accountable. Either way, their numbers are growing, and they appear to be here to stay.\footnote{Although their potentially pernicious effects on stability and accountability has also been noted. See, e.g., \cite{baker03} (arguing that ``the Enron scandal can be better understood as an American form of public private partnership rather than just another example of capitalism run amok'').} In this situation, it is inevitable that when eminent domain is used to acquire property for economic development, those who directly benefit will often be commercial companies rather than public bodies. In the previous chapter, I pointed out how indirect public benefits are typically used to justify such takings. Standard legitimizing reasons include the prospect of new jobs, increased tax revenues, and various other economic and social ripple effects. However, as I have indicated, economic development takings have a tendency to result in controversy.

In the US after {\it Kelo}, they have also been at the forefront of the constitutional property debate. In the rest of the world, a similar shift in academic outlook has yet to take place, but expropriation-for-profit situations are increasingly coming into focus here as well.\footnote{See, e.g., \cite{gray11,waring13,verstappen14}.} If we lift our perspective slightly, to consider commercially motivated interference more generally, it even seems appropriate to speak of a crisis of confidence in property law, particularly in relation to land rights. This is most clearly felt in the developing world, where egalitarian systems of property use and ownership are coming under increasing pressure. It has been noted, in particular, that large-scale commercial actors are assuming control over an increasing share of the world's land rights, a phenomenon known as {\it land grabbing}.\footnote{See generally \cite{borras11}.} 

So far, most research on land grabbing has looked at how commercial interests, often cooperating with nation states, exploit weaknesses of local property institutions, to acquire land voluntarily, or from those who lack formal title. However, the danger of {\it Kelo}-type reasoning has also been recognized. In particular, it has been noted how the purported public interest in economic development can be used to justify land grabs that would otherwise appear unjustifiable. In a recent article, Smita Narula cites {\it Kelo} directly and warns that procedural safeguards alone might not provide sufficient protection against abuse. She writes:
\begin{quote}
Procedural safeguards, however, can all too easily be co-opted by a state because its claims about what constitutes a public purpose may not be easy to contest. Particularly within the context of land investments, states could use the very general and under-scrutinized language of ``economic development'' to justify takings in the public interest.\footcite[157]{narula13}
\end{quote}

This underscores the broader relevance of the study of economic development takings. In addition, it reminds us that the question of what can be justified in the name of ``economic development'' is a general one, not confined to particular systems for organizing property rights. To address this, and to restore confidence in the institution of property more generally, many turn towards {\it human rights}. These scholars argue that a human right to land should be recognized on the international stage, a right that would apply even when those most affected by a land grab lack formal title.\footnote{See generally \cite{schutter10,schutter11,kunnerman13}.} If successful, this approach promises to deliver basic protection against interference in established patterns of property use independently of how particular jurisdictions approach property.

In Europe, a human rights perspective is already of great practical significance due to the European Convention of Human Rights (ECHR) and the court in Strasbourg (ECtHR). But, of course, in the context of land grabbing, protecting land rights is not primarily a question of protecting the civil law ideal of individual dominion. Rather, it is a question of providing protection against large-scale transactions that destabilize or destroy established patterns of land use, to the detriment of local communities. Nevertheless, the questions raised by the public interest  narrative -- and the notion of ``economic development'' in particular -- must be expected to arise in much the same way as in cases when formal title is acquired following a state-authorized taking.

Hence, it is somewhat surprising that the special category of for-profit takings has not received more attention from the point of view of human rights law. In human rights discourse, the focus tends to be rather on fairness and proportionality as broad benchmarks, in addition to specific values related to food security and protection of livelihoods that arise with particular urgency in the context of third-world land grabs. But how to achieve effective protection depends as much on the development of firm categories and enforcible legal principles as it does on broad benchmarks and good intentions. In this regard, I think Narula is right to stress that the lack of a convincing approach to the notion of ``economic development'' is a crucial challenge.

On the one hand, economic development is no doubt a sound overarching goal, particularly for poor nations. But at the same time, the risk of abuse is obvious when such a vague term is used to justify dramatic interferences in property. After all, interferences in property can cause severe disturbances in people's life. This, moreover, is true for middle-class US homeowner in much the same way as it is true for members of self-sustaining agrarian communities in Africa, although the stakes might be much higher for the latter.

As illustrated by {\it Kelo}, deep conflicts can arise in this regard also in developed democracies with long established systems of formal title. In the following, I will attempt to shed further light on the issue as it arises in such legal systems, without considering the additional complications that arise when property itself is -- formally speaking -- a more fluid concept. I note, however, that according to the social function view of property, there is no need to view formally recognized property rights as completely distinct from rights arising from property use. The two are intertwined and the difference between them is at most a matter of degree.\footnote{Moreover, if the human flourishing account of property values is successfully developed, there should even be hope that a unified normative treatment can be given at some point.}

However, my case study will look to Norwegian law, a prosperous European country with a long tradition of formal title to land. Hence, it is prudent to narrow down the discussion here by focusing on similar jurisdictions.\footnote{The relation with third-world land grabbing is a highly interesting question for future work.} I will do so now, beginning with a brief look at English and German law, to illustrate that there are great differences in how different European jurisdictions think about property in general, and takings in particular. Then I turn my attention to the ECHR and I focus on presenting the proportionality test that is now at the core of property adjudication at the ECtHR.

Following this, I move on to consider the US in greater depth, both the historical debate that led to {\it Kelo} and the suggestions for reform that have emerged following its backlash. A closer look is necessary because of the sheer magnitude of writing on this issue in the US. Moreover, while much of it is repetitive and coloured by the tense political climate, I believe some historical points, as well as some recent suggestions for reform, are highly relevant also to the international setting. To single out and analyse those aspects is the main aim of this part of the chapter. Indeed, the current debating climate in the US might be an indication of what is to come also in Europe, if concerns about the legitimacy of economic development takings are not taken seriously.

%I also highlight what I believe to be a connection between the situation in the US leading up to {\it Kelo} and the present situation in Europe, illustrated by the fact that the European Court of Human Rights is now explicitly endorsing ``stronger protection'' of property rights.  I attempt to identify the reasons behind calls for a stricter approach, arguing that it is connected to the fact that interferences in property under modern regulatory regimes is sanctioned in wide a range of different circumstances, serving to undermine their status as a necessary burden imposed on owner's according to the will of the greater public. In some cases, rather, takings appear to both owners and the public as improperly motivated and socially and politically unfair. I note that this happens particularly often in economic development cases, when commercial actors benefit to the detriment of local communities. I go on to list some concrete issues that arise with respect to such takings and that have been flagged as problematic in the literature.
%
%Following up on this, I consider various proposals that have been made to resolve tensions and limit the possibility of abuse in economic development cases. The differences of opinion that have been expressed in this regard have been quite substantial, and proposals have ranged from suggesting an outright ban on economic development takings  (Somin 2007; Cohen 2006) to suggesting that the best way forward is to reassess principles for awarding compensation in such cases (Householder 2007; Lehavi and Licht 2007).

%Much of the current theory focus on assessing traditional judicial safeguards that courts can rely on to prevent abuses, pertaining primarily to the material assessment of proportionality, public purpose, and compensation. 

%In the last part of the chapter, I will focus on a very interesting strand of recent work in the US, which shifts attention towards procedural rules that can help address the worry that economic development takings tend to suffer from a democratic deficit. The core concern is that the manner in which eminent domain decisions are typically made, and the way in which owners are compensated, might be unsuitable for economic development cases. Importantly, the need for special procedures has been noted, to restore legitimacy.\footnote{See generally \cite{lehavi07,heller08}.} This ties the US debate even closer to the European context, where proportionality, not public use, has become the key notion in property protection. Several recent suggestions from the US can be conceptualized as suggestions that aim to secure fairness and proportionality, while paying less attention to the formalistic question of what constitutes a ``public use''.
%
%%Also, it allows us to be very clear about a special concern that arises for economic takings cases: under current regulatory regimes, the government and the developer together often dominate the decision-making process completely, leaving the property owners marginalized. Hence, there is often a {\it democratic deficit} in such cases, resulting in discontent and a feeling that the taking is not in the public interest at all. Importantly, some recent writers hypothesize that if the proper balance can be restored in the decision-making process, so will the decision reached appear more legitimate, also with respect to the public use clause. In my opinion, this idea is crucial, and together with the question of compensation, which raises a similar structural problem, it will guide the rest of the work done in this thesis. 
%

In response to that worry, this chapter aims to  bring into focus the following key question: What principles can be used to ensure meaningful participation and just compensation in economic takings cases, without hindering socially and economically desirable development projects? The tentative answers provided in Section \ref{sec:ir} will set the stage for the remainder of the thesis, where they will be assessed in depth against the case study of Norwegian hydropower.

%In particular, I will consider two special semi-judicial procedural systems used in such cases in Norway, one targeting compensation following expropriation, and another used as an alternative to expropriation, particularly in cases when development requires cooperation among many owners.

%I conclude by arguing that approaches along procedural lines represent the best way forward in relation to addressing issues associated with economic development takings. This raises the following problem, however: what procedural principles can be used to ensure meaningful participation, without hindering socially and economically desirable development projects? This question sets the stage for the remainder of my thesis, where I conduct a case study of expropriation for the development of hydro-power in Norway. In particular, I will consider two special semi-judicial procedural systems used in such cases in Norway, one targeting compensation following expropriation, and another used as an alternative to expropriation, particularly in cases when development requires cooperation among many owners.

\section{A European contrast}\label{sec:contrast}

Economic development takings have not become as controversial in Europe as they are in the US, but there have been cases where the issue has come up, in several different jurisdictions.\footnote{For instance, in the UK, Ireland and Germany, as well as in Norway and Sweden. See \cite[466-483]{walt11}; \cite{stenseth10}.} The European Convention of Human Rights (ECHR) contains a property clause in Article 1 of Protocol No 1 (P1(1)), but the legitimacy of economic development takings has not yet been discussed in case law from the European Court of Human Rights (ECtHR). However, it is interesting to analyse cases like {\it Kelo} against P1(1), particularly since the ECtHR has developed a doctrine that focuses on ``proportionality'' and ``fairness'' rather than the purpose of interference.\footnote{See generally, \cite[Chapter 5]{allen05}. This approach may become even more significant as a source of property protection in the future, as the ECtHR have indicated that there are ``jurisprudential developments in the direction of a stronger protection under Article 1 of Protocol No. 1'', see \cite[135]{lindheim12}.}

The fundamental question raised by economic development takings can be formulated independently of specific property clauses as follows: When, if ever, is it permissible for governments to order compulsory transfer of property rights from citizens to for-profit legal persons in order to facilitate economic development?

In this section, I address economic development takings from the point of view of European sources. I first contrast English and German law, to show that there are significant differences between European jurisdictions in this regard. I then go on to give a more detailed presentation of the unifying property clause in P1(1) of the ECHR. The case law from the ECtHR is presented and analysed in some depth, in an effort to assess how the ECtHR would be likely to approach an economic development case such as {\it Kelo}. In particular, I argue that the ``proportionality'' doctrine offers an interesting approach to economic development cases. This doctrine stipulates that a ``fair balance'' must be struck  between the interests of the property owner and the public.\footcite[Chapter 5]{allen05} I argue that such a perspective could make it easier to get to the heart of why economic development takings are often seen as problematic, without getting lost in theoretical discussions about the meaning of  terms like ``public use'' or ``public purpose''. However, I also raise the concern that the ECtHR is not the appropriate institution for applying the proportionality test concretely. Its remoteness suggests that we should also look for more locally grounded legitimacy-enhancing institutions. Such institutions will likely be better able to assess the fairness of interference in context.

I go on to discuss whether existing government institutions can serve this purpose, arguing that local courts may well be the best candidates. However, I argue that active application of the ``proportionality''-doctrine in property cases has not yet developed fully at the local level. I also discuss possible shortcomings of local courts; as judicial bodies they are not intrinsically well-suited to carry out the kind of assessment that is required. Hence, I suggest that entirely new institutional proposals might be in order. I conclude by arguing that once the need for local grounding is recognized and met, the ECtHR has the potential to play an important and constrictive role in providing oversight and developing basic principles.

\subsection{England}\label{sec:england}

In England, the principle of parliamentary supremacy and the lack of a written constitutional property clause has led to expropriation being discussed mostly as a matter of administration and property law, not as a constitutional issue.\footcite{taggart98} Moreover, the use of compulsory purchase -- the term most often used to denote takings in the UK -- has not been restricted to particular purposes as a matter of principle. The uses that can warrant compulsory alienation of property are those that parliament regard as worthy of such consideration. However, as private property itself has long been recognized as a fundamental right, the power of compulsory purchase has typically been exercised with great caution. 

In his {\it Commentaries on English Law}, William Blackstone famously described property as the ``third absolute right'' that was ``inherent in every Englishman''.\footcite[134-135]{blackstone79}  Moreover, Blackstone expressed a very restrictive view on the possibility of expropriation, arguing that it was only for the legislature to interfere with property rights. He warned against the dangers of allowing private individuals, or even public tribunals, to be the judge of whether or not the ``common good'' could justify it. Blackstone went as far as to say that the public good was ``in nothing more invested'' than the protection of private property.\footcite[134-135]{blackstone79}

Historically, Blackstone's description conveys a largely accurate impression of takings practice in England. Indeed, Parliament itself would usually be the granting authority in expropriation cases, through so-called {\it private Acts}. Hence, compulsory purchase would not take place unless it had been discussed at the highest level of government. Moreover, the procedure followed by parliament in such cases strongly resembled a judicial procedure; the interested parties were given an opportunity to present their case to parliament committees that would then decide whether or not compulsion was warranted.\footnote{See \cite[13-16]{allen00}. While this procedure clearly reflected a protective attitude towards private property, recent scholarship has pointed out that expropriation was actually used more actively in Britain following the glorious revolution, see \cite{hoppit11}.} 

On the one hand, the direct involvement of parliament in the decision-making process reflected the fundamental respect for property rights that permeated the system. But at the same time, parliamentary supremacy also meant that the question of legitimacy was rendered mute as soon as compulsory purchase powers had been granted. The courts were not in a position to scrutinize takings at all, much less second-guess parliament as to whether or not the taking was for a legitimate purpose.

Eventually, an overworked parliament developed procedures for dealing more expeditiously with takings cases, and during the 19th Century, as an industrial economy developed, private Acts granting commercial companies the power to take land grew massively in scope and importance.\footnote{See \cite[204]{allen00}.} Private railway companies, in particular, regularly benefited from such Acts.\footnote{\cite[204]{allen00}. See generally \cite{kostal97}.} During this time, the expanding scope of private-to-private transfers for economic development lead to quite a bit of political debate and controversy. Usually, it would attract particular opposition from the House of Lords. Interestingly, this opposition was not only based on a desire to protect individual property owners. It also often reflected concerns about the cultural and social consequences of changed patterns of land use.\footcite[204]{allen00} 

Hence, the early debate on economic development takings in the UK shows some reflection of a contextual approach to property protection. However, as society itself changed dramatically following increasing industrialization, an expansive approach to compulsory purchase eventually emerged triumphant. At the same time, the idea that economic development could justify takings gradually became less controversial. 

Today, the law on compulsory purchase in England is regulated in statute and the role of courts is to a large extent limited to the application and interpretation of statutory rules. Some common law rules still play an important role, such as the {\it Pointe Gourde} rule discussed in more depth in Chapter \ref{chap:5}. With respect to the question of legitimacy, however, the starting point for English courts is that this is a matter of ordinary administrative law. 

More recently, the \cite{hra98} adds to this picture, since it incorporates the property clause in P1(1) into English law. But even so, the usual approach for English courts is to judge objections against compulsory purchase orders on the basis of the statutes that warrant them, rather than constitutional or human rights principles that protect property.\footnote{The important statutes are the \cite{ala81}, the \cite{lca61}, the \cite{tcpa90} and the \cite{pcpa04}. Acquisition of Land Act 1981, the Land Compensation Act 1961, the Town and Country Planning Act 1990 and the Planning and Compulsory Purchase Act 2004.} It is typical for statutory authorities to include standard reservations to the effect that some public benefit must be identified in order to justify a CPO, but the scope of what constitutes a legitimate purpose can be very wide. For instance, to warrant a taking under the \cite{tcpa90}, it is enough that it ``facilitates the carrying out of development, redevelopment and improvement on or in relation to the land''.\footcite[226]{tcpa90} 

While various governmental bodies are authorised to issue compulsory purchase orders (CPOs), a CPO typically has to be confirmed by a government minister. The affected owners are given a chance to comment and if there are objections, a public inquiry is typically held. The inspector responsible for the inquiry then reports to the relevant government minister, who makes the final decision about whether or not it should be granted, and on what terms. The CPO may then be challenged in court, but will usually only be scrutinized on the basis of whether or not it lies within the scope of the statute authorizing it. Hence, the discussion and evaluation at court is firmly grounded in statutory rules rather than constitutional principles.

That said, the idea that property may only be compulsorily acquired when the public stands to benefit permeates the system. Indeed, this has also been regarded as a constitutional principle, for instance by Lord Denning in {\it Prest v Secretary of State for Wales}.\footcite{prest82} He said:

\begin{quote}
It is clear that no minister or public authority can acquire any land compulsorily except the power to do so be given by Parliament: and Parliament only grants it, or should only grant it, when it is necessary in the public interest. In any case, therefore, where the scales are evenly balanced – for or against compulsory acquisition – the decision – by whomsoever it is made – should come down against compulsory acquisition. I regard it as a principle of our constitutional law that no citizen is to be deprived of his land by any public authority against his will, unless it is expressly authorised by Parliament and the public interest decisively so demands. If there is any reasonable doubt on the matter, the balance must be resolved in favour of the citizen.\footcite[198]{prest82}
\end{quote}

Lord Denning also supported the doctrine of necessity, as expressed by Forbes J in {\it Brown v Secretary for the Environment}:\footcite{brown78}

\begin{quote}It seems to me that there is a very long and respectable tradition for the view that an authority that seeks to dispossess a citizen of his land must do so by showing that it is necessary, in order to exercise the powers for the purposes of the Act under which the compulsory purchase order is made, that the acquiring authority should have authorisation to acquire the land in question.\footcite[291]{brown78}
\end{quote}

In practice, these principles are mostly implicit in legal reasoning, as a factor that influences the courts when they interpret statutory rules and carry out judicial review of administrative decisions. As Watkins LJ stated in {\it Prest}:

\begin{quote}
The taking of a person's land against his will is a serious invasion of his proprietary rights. The use of statutory authority for the destruction of those rights requires to be most carefully scrutinised. The courts must be vigilant to see to it that that authority is not abused. It must not be used unless it is clear that the Secretary of State has allowed those rights to be violated by a decision based upon the right legal principles, adequate evidence and proper consideration of the factor which sways his mind into confirmation of the order sought.\footcite[211-212]{prest82}
\end{quote}

In {\it R v Secretary of State for Transport, ex p de Rothschild}, Slade LJ referred to the judgment and made clear that he did not regard it as expressing a rule concerning the burden of proof in compulsory purchase cases, but rather as more general expressions about the severity of the interference and the importance of vigilance in such cases.\footnote{rothschild89} He said that they provided ``a warning that, in cases where a compulsory purchase order is under challenge, the draconian nature of the order will itself render it more vulnerable to successful challenge''.\footcite[938]{rothschild89}

A nice example of how these sentiments influence the assessment of legitimacy of takings, showing how it is applied in economic development cases, can be found in the recent case of {\it Regina (Sainsbury’s Supermarkets Ltd) v Wolverhampton City Council}.\footcite{sainsbury10} Here a CPO was granted to allow the company Tesco to acquire land from its competitor Sainsbury, in a situation when they were both competing for licenses to undertake commercial development on this land. The decisive factor that had led the local authorities to grant the CPO was that Tesco had offered to develop a different property in the same local area, which was currently in need of regeneration. 

Sainsbury protested, arguing that the local council could not strike such a deal on the use of its compulsory purchase power. It was argued, moreover, that taking the land for incidental benefits resulting from development in a different part of town was not legitimate under the Town and Country Planning Act 1990. The UK Supreme Court agreed 4-3, with Lord Walker in particular emphasizing the need for heightened judicial scrutiny in cases of private-to-private cases for economic development.\footcite[80-84]{sainsbury10} Lord Walker even cited {\it Kelo}, to further substantiate the need for a stricter standard in such cases.\footcite[81]{sainsbury10} 

However, the main line of reasoning adopted by the majority was based on an interpretation of the Town and Country Planning Act itself. In particular, the majority held that it was improper for the local council to take into consideration the development that Tesco had committed itself to carry out on a different site.\footcite[73-79]{sainsbury10} This, in particular, was not ``improvement on or in relation to the land'', as required by the Act.\footcite[336]{tcpa90} In addition, Lord Collins, who led the majority, said that ``the question of what is a material (or relevant) consideration is a question of law, but the weight to be given to it is a matter for the decision maker''.\footcite[70]{sainsbury10} Hence, the general importance of the decision for economic development cases is unclear.

Still, it is interesting to see how the purpose of the interference featured in the Supreme Court's interpretation and application of the statutory rules. The opinion of Lord Walker is particularly interesting, since he stresses that ``The land is to end up, not in public ownership and used for public purposes, but in private ownership and used for a variety of purposes, mainly retail and residential.''\footcite[81]{sainsbury10} He goes on to state that ``economic regeneration brought about by urban redevelopment is no doubt a public good, but ``private to private'' acquisitions by compulsory purchase may also produce large profits for powerful business interests, and courts rightly regard them as particularly sensitive.``\footcite[81]{sainsbury10}

Lord Walker then makes clear that he does not think it is impermissible, as such, for the local council to take into consideration positive effects on the local area, even when these do not directly result from the planned use of the land that is being acquired. Instead, he relies explicitly on the for-profit character of the taking, by arguing that ``the exercise of powers of compulsory acquisition, especially in a ``private to private'' acquisition, amounts to a serious invasion of the current owner's proprietary rights. The local authority has a direct financial interest in the matter, and not merely a general interest (as local planning authority) in the betterment and well-being of its area. A stricter approach is therefore called for.''\footcite[84]{sainsbury10} 

Lord Walker's opinion might indicate that the narrative of economic development takings is about to find its way into English case law. Moreover, a more critical approach might be adopted in the future, when compulsory purchase powers are made available to commercial companies wishing to undertake for-profit schemes. However, for schemes where the commercial aspect appears less dominant, English courts still appear very reluctant to quash CPOs, also when the purpose is economic development. This is so even in situations when the owners have requested a stricter standard of review on the basis of human rights law. 

For instance, in the case of {\it Smith \& Others v Secretary of State for Trade and Industry}, a caravan site was compulsorily acquired for development in connection with the London Olympic Games.\footcite{smith08} Some of the owners protested, including Romani Gypsies who used the caravans as their primary residence. A public inquiry was held, after which the inspector recommended that the CPO should not be confirmed until adequate relocation sites had been identified. However, due to the ``urgency, timing and importance '' of the project, the Secretary of State decided to go ahead before a relocation scheme was put in place (although he expressed commitment to ensuring satisfactory relocation).\footcite[10]{smith08} The owners argued that without satisfactory relocation plans, the interference in the property rights was not proportional and had to be struck down on the basis of human rights law, in particular Article 8 in the ECHR regarding respect for the home and private life.\footcite[27-51]{smith08}

The Court of Appeal considered the matter in great depth, applying the doctrine of proportionality developed at the ECtHR, which goes beyond the standard room for judicial review of administrative decisions under English law. However, the Court still concluded that the taking was proportional. This was largely based on the finding that ``the issue of proportionality has to be judged against the background that everyone accepts that an overwhelming case has been made out for compulsory acquisition of the sites for the stated objectives and that compulsory purchase is justified.''\footcite[42]{smith08} 

Justice Williams arrived at this conclusion after noting that the owners' {\it only} substantial objection against the CPO was that it was confirmed before adequate relocation measures had been agree on.\footcite[42]{smith08} Hence, the question as he saw did not concern the validity of using compulsory purchase powers, but merely the timing with which it had been ordered. On this basis, he framed the question of legitimacy as one relating to the ``necessity'' standard, according to which an infringement on Convention rights is only permissible when the public interest cannot be served in some other way.\footcite[43]{smith08} A strict reading of this standard holds that an interference must be the {\it least intrusive means} of achieving the stated aim.\footnote{Such a standard has been adopted in some Convention cases, for instance in \cite{samaroo01}.}

Justice Williams argued against such a strict reading, subscribing instead to a view expressed as an {\it obiter} in the case of {\it Pascoe v The First Secretary of State}. According to this view, an interference need not be the least intrusive means of achieving the public purpose, it is sufficient that the measure is ``reasonably necessary'' to achieve that aim.\footnote{See \cite[74-75]{pascoe06} (quoting \cite[25]{clay04}).} However, while noting his agreement with this approach, Justice Williams went on to apply the stronger necessity test, and found that even if this was applied the CPO in question would still be a proportional interference.\footcite[41-50]{smith08}

It seems clear that while the taking in question was for economic and recreational development purposes, the case was marked by the finding that the legitimacy of the aim of interference -- to facilitate the London Olympics -- was beyond reproach. Hence, there was no need for, or even room for, more detailed purposive reasoning of the kind that would later be applied by Lord Walker in {\it Sainsbury}. The fact that the taking was for economic development and recreation, not for a pressing public need, was not considered relevant, and was not held against the effects on the owners. This, in particular, was not how the issue of proportionality was conceptualized. Indeed, since the case was construed to be solely about the extent to which the CPO was ``necessary'' to further its stated aim, the proportionality test that was carried out, despite being detailed, was very narrow in scope. It concerned only proportionality of the means, not of the aim itself. The question of how to weigh the public interest in a multi-billion dollar sporting event against the security of someone's home was not considered.

In later cases, a dismissive attitude towards substantive review has been adopted also in situations when the owners have argued against takings by explicitly questioning the proportionality of the inference against the importance of the aim. In the case of {\it Alliance Spring Co Ltd v The First Secretary of State}, a large number of properties were expropriated to build a new football stadium for the football club Arsenal.\footcite{alliance06} Some owners who stood to lose their business premises as a result of the scheme protested the legitimacy of the order, pointing to the fact that the inspector in charge of the public inquiry had recommended against the takings.\footcite[6-7]{alliance06} As noted by Justice Collins, the main line of legal argument presented against the taking was that it did not serve a ``proper purpose''.\footcite[19]{alliance06} It is of note that in his evaluation of this argument, Justice Collins largely focuses on presenting the assessments carried out by the inspector and the Secretary of State, who went against the recommendation and confirmed the CPO. Finding that these assessments took all relevant matters into account and where not clearly unreasonable, Justice Collins goes on to conclude as follows: 

\begin{quote}
There is nothing in the material put before and accepted by the Inspector which persuades me that that decision was ill founded or was one which the Secretary of State was not entitled to reach. Developments which result in regeneration of an area are often led by private enterprise. Mr Horton perforce accepts that that is so, but submits that this is not the sort of situation where, for example, a private development is the anchor for a particular scheme. I disagree.\footcite[19]{alliance06}
\end{quote}

Hence, unlike the case of {\it Smith}, where the Court did in fact carry out its own assessment of proportionality, albeit only in relation to the question of necessity, the {\it Alliance} Court was content with deferring to the assessment carried out by the executive branch.\footnote{This has been criticized, e.g., by Kevin Grey who describes the reference to Convention Rights in Alliance as ``worryingly brief''. See \cite{gray11}.} As such, the case largely follows the set pattern of judicial review of CPOs from before the passing of the Human Rights Act 1998. This mean that it also stands in contrast to how English courts have approach the Convention in other kinds of cases, involving other rights, such as Article 8 in {\it Smith}. 

Whether the approach taken in {\it Alliance} is good law after {\it Sainsbury} is unclear; from Lord Walker's opinion, it seems that a more substantive assessment can be demanded for similar cases in the future. While this might not imply a different outcome for a case like {\it Alliance}, it would mean that courts would have to engage in independent review of the purpose and merits of contested CPOs that benefit commercial actors. In particular, English courts would have to change the way they approach such cases, by being more prepared to assess for themselves whether a fair balance is struck between the interests of the developer and the property owners. Hence, it is not unlikely that the category of economic development takings will become an important point of reference in the future, both for the law and those who study it.

\subsection{Germany}\label{sec:germany}

In German law we find an explicit constitutional property clause. In particular, Article 14 of the Basic Law ({\it Grundgesetz}) reads as follows:

\begin{quote}
(1) Property and the right of inheritance shall be guaranteed. Their content and limits shall be defined by the laws. \\
(2) Property entails obligations. Its use shall also serve the public good. \\
(3) Expropriation shall only be permissible for the public good. It may only be ordered by or pursuant to a law that determines the nature and extent of compensation. Such compensation shall be determined by establishing an equitable balance between the public interest and the interests of those affected. In case of dispute concerning the amount of compensation, recourse may be had to the ordinary courts.\footcite[14]{basic49}
\end{quote}

Apart from the fact that the property clause is explicit, I note two further characteristic features of the protection of property in Germany. First, the constitution explicitly stresses that property comes with social obligations as well as rights. The use of property should ``serve the public good''. On the other hand, it is also made clear that expropriation is only permissible when it is ``for the public good''. Hence, it follows immediately that the purpose of expropriation is a relevant factor when determining the legitimacy of a taking, irrespectively of the specific statute used to authorise it. Importantly, it is clear already from the outset that the question of legitimacy is a \emph{judicial} question, one which the courts can only answer if they form an opinion about that constitutes the ``public good''. 

This means that it is quite natural to approach the question of economic development takings from the point of view of constitutional law. Unlike in England, disputes over the legitimacy of such takings can be comfortably adjudicated directly against a ``public good'' restriction. While this sets Germany apart on the theoretical level, it is unclear how much of an effect it has had in practice. To shed some light on this question, we can look to the two major authorities on the legitimacy of economic development takings, the cases of {\it D\"{u}rkheimer Gondelbahn} and {\it Boxberg}.\footcite{durkheimer81,boxberg86} 

In both cases, the German Constitutional court found that expropriation to the benefit of commercial interests was illegitimate. However, the Court argued for this result on the basis that there was insufficient statutory authority for such takings in the concrete circumstances complained of. That is, the Court did not directly address the question of whether the relevant statutes were compliant with Article 14 of the basic law. Instead, they interpreted statutory authorities on the assumption that they had to be, following a pattern of reasoning that appears to be rather close to the approach followed by English courts in similar cases.\footnote{Although in {\it Dürkheimer Gondelbahn}, Böhmer J gave a separate concurring judgment where he argued for this result on the basis of the public good requirement of the basic law.} It seems, in particular, that even in Germany, the public purpose restriction is primarily relevant as a factor guiding the interpretation of statutory authorities.

That said, the cases of {\it D{\"u}rkheimer Gondelbahn} and {\it Boxberg} show that in situations when the public purpose of a taking is unclear, German courts seem inclined to favor a narrow interpretation of the relevant statute. In {\it Bloxberg}, several properties were expropriated in favor in favor of the car company Daimler Benz AG, for commercial purposes. The affected local communities suffered from high unemployment rates and a slow economy, so a {\it prima facie} reasonable cases could be made that allowing Daimler to acquire the land was in the public interest, as it would facilitate economic growth. However, the Federal Constitutional Court agreed with the owners that the expropriation was invalid. This, it held, was because the taking was outside the scope of the relevant statute, which authorised expropriations for ``planning purposes''. The owners had argued extensively using Article 14 of the Basic Law and the constitutional ``public good'' restriction clearly did play a role in the Court's reasoning. But at the same time, the Court stressed that private-to-private transfers that bestow financial benefit on the acquiring party may well satisfy the ``public good'' requirement. The important issue was whether a sufficiently strong public interest could be identified, irrespectively of any windfall benefits that might fall on private parties.

In light of this, I think it is wrong to exaggerate the importance of the explicit formulation of the public use test offered in the German constitution. Its importance seems to rest mainly in the fact that it provides a particularly authoritative expression guiding the national courts' application of statutory provisions regarding expropriation of property. But developments in common law, where the public use requirement is stressed as a guiding constitutional principle, might well point in the same direction. In principle, both German and English Courts are in a good position to respond to increased tension regarding economic development takings by developing a stricter standard of judicial review in such cases.

A different aspect of German law deserves special attention, however, since it does not appear to have any clear counterpart in the common law tradition. This is the  ``social-obligation'' norm in Article 14 (2), which points to a different conceptualization of property rights as such. As argued by Alexander, the distinguishing feature of the property clause in the German Constitution is that the value of property is thought to relate more strongly to its importance for human dignity and flourishing in a social context, rather than the protection of individual financial entitlements. As Alexander notes regarding the Germans' own conceptualization of their property clause:

\begin{quote}
This theory holds that the core purpose of property is not wealth maximization or the satisfaction of individual preferences, as the American economic theory of property holds, but self-realization, or self-development, in an objective, distinctly moral and civic sense. That is, property is fundamental insofar as it is necessary for individuals to develop fully both
as moral agents and participating members of the broader community.\footcite[745]{alexander03}
\end{quote}

With such a starting point, it is not surprising that in cases such as {\it Boxberg}, resembling {\it Kelo}, German Courts will tend to adopt a strict view on legitimacy. These are cases when the property rights infringed on serve a fundamentally different function for the two opposing private parties. To the owner, the property is a home, an important source of self-identity, autonomy, security and membership in a community. To the taker, it represents an obstacle to commercial development which needs to be removed. In such a situation, it is in keeping with the spirit of the social-obligation norm of property to offer enhanced protection to the homeowner. To this owner, the property serves a purpose which is fundamentally different, and arguably more worthy of protection, then the property's purpose for the developer. A taking in this situation might therefore, because of Article 14, require a particularly clear and strong public interest.

But unless there is an asymmetry between owner and taker, heightened scrutiny does not necessarily follow. Hence, it is interesting to speculate what German courts would have made of a case such as {\it Regina (Sainsbury’s Supermarkets Ltd) v Wolverhampton City Council}. Here, the interests of owner and taker were strictly commercial nature. Both owned part of the contested land and neither one could develop the land according to their plans without buying out the other. The enhanced protection of property offered under German law would probably not have much significance in such a case. 

In fact, it might well be that German courts would be {\it more} likely to accept such a taking. First, their conceptualization of property rights appears to allow greater flexibility to adapt the level of protection to the circumstances and the purposes of the property in question. So even if is correct that private-to-private transfers for commercial projects require a ``stricter approach'' in general, as argued by Lord Walker in \textcite{sainsbury10}, the fact that the interests of the owner were also purely commercial  might make this less relevant. Second, German courts might be more inclined to have regard to socially beneficial additional commitments entered into by the applicant, even if they do not concern the property that is taken. As a tie-breaker, looking to such commitments might be as good an approach as any other.\footnote{This was the view taken by the dissenting minority in \textcite{sainsbury10}.}

Of course, objections could still be raised on the basis of general administrative law. Indeed, some might see the case as an example of government ``auctioning'' off licenses to the highest bidder. This might well be regarded as an affront to good governance. I will not delve into German law to assess the case from this perspective. My point is simply that because of the purposive and contextual nature of Article 14, it seems unlikely that a case like \textcite{sainsbury10} would turn on constitutional property law.

To sum up, German constitutional law serves to create an interesting contrast with English law regarding the question of economic development takings. On the one hand, property appears to be better protected against such takings in Germany, but on the other hand, the extent to which increased protection is offered depends more closely on the social values involved. The German system appears to look more actively at the social function of property for guidance when resolving property disputes, thereby echoing some of the ideas discussed in Chapter \ref{chap:1}. 

In the next section, I will discuss the property clause in the ECHR, which explicitly serves to set up a minimum level of property protection that provides a common standard for all member states, including Germany and the UK.

\section{The Property Clause in the European Convention of Human Rights}

The starting point for property adjudication at the ECtHR is that States have a ``wide margin of appreciation'' with regards to the question of whether or not an interference in property rights is to be considered legitimate in pursuance of the public interest.\footcite[See][54]{james86} This question is thought to depend on democratically determined policies to such an extent that it is rarely appropriate for the Court to censor the assessments made by member states. At the same time, however, the Court has gradually come to take a more active role in assessing whether or not particular instances of interference are ``proportional'' and able to strike a ``fair balance'' between the interests of the public and the interests of the individual property owner.\footnote{See \cite[69]{sporrong82} and \cite[120]{james86}. The standard account of the protection against interference inherent in P1(1) describes it as consisting of three rules. First, there is the rule of {\it legality}, asserting that an interference needs to be authorized by statute. Second, there is the rule of {\it legitimacy}, making clear that interference should only take place in pursuance of a legitimate public purpose. Both of these rules are of little practical significance, however, as the margin of appreciation has been regarded as very wide in regards to both. The third rule is the ``fair balance'' principle, which is applied by the ECtHR in almost all cases when it finds that there has been a violation of P1(1). In the following, I focus only on this rule and on those aspects of it that I think are most relevant to the question of economic development takings. For a more detailed description of P1(1) generally, I refer to \cite{allen05}.} As argued by Tom Allen, this has caused P1(1) to attain a wider scope than what was originally intended by the signatories.\footcite[1055]{allen10}.

In the case law behind this development, the focus has predominantly been on the issue of compensation, with the Court gradually developing the principle that while P1(1) does not entitle owners to full compensation in all cases of interference, the fair balance will likely be upset unless at least some compensation is paid, based on the market value of the property in question.\footnote{See \cite[103]{scordino06}. The case also illustrates that the Court has come to adopt a fairly strict approach to the question of when it is legitimate to award less than full market value.} This focus on compensation has also been reflected in academic work on P1(1), which tends to address proportionality from an economic perspective, by investigating to what extent owners are entitled to compensation based on the market value of their property. Indeed, when considering the best known case law and literature on the subject, one is left with the impression that ``fair balance'' with regards to P1(1) is crucially linked to financial entitlements, primarily used as a standard that can justify a right to compensation that goes beyond what the wording of P1(1) might initially suggest.

In recent case law, however, it has become clear that the fair balance test encompass more than this, since it also gives the Court in Strasbourg occasion to reflect on the social context and purpose of interference, in a manner largely consistent with the social function approach to property. In {\it Chassagnou and others v France} the situation was that property owners were compelled to permit hunting on their land, following compulsory membership in a hunting association which was set up to manage hunting in the local area.\footcite{chassagnou99} They protested this on the grounds that they were ethically opposed to hunting, and the Court agreed that there had been a breach of P1(1). 

In the later case of {\it Hermann v Germany} the circumstances were similar, and the Court followed the precedent set in {\it Chassagnou}, commenting also that they had ``misgivings of principle'' about the argument that financial compensation could provide adequate protection in such a case.\footcite[See][91]{hermann12}  In this way, the hunting cases illustrate that to the ECtHR, the right to property is not seen as a mere financial entitlement. Moreover, the fair balance that must be struck could well pertain to other aspects, such as the owner's right to make use of his property in accordance with his convictions and to take part in decision-making processes regarding how it should be managed.\footnote{The assessment of proportionality should be concrete and contextual, and it is not based on a narrow or formalistic concept of property as dominion. This is demonstrated, for instance, by \cite{chabauty12}. Here the Court found no violation of P1(1) although the facts seemed close to those of {\it Chassagnou}. The case differed, however, in that the owner himself was not opposed to hunting, but wanted to withdraw his land from the hunters' association to enjoy exclusive hunting rights.}

In a different, but related, development, the Court has also adopted a contextual approach in recent cases involving rent control 
schemes and housing regulation. While there are obvious financial interests at stake in such cases, for both landlords and tenants, the Court has looked to the fairness of the underlying regulation more generally, by taking into account the local social, economic and political conditions. Moreover, the Court has not shunned away from using concrete cases as a starting point for providing an assessment of the sustainability of national law as such. In {\it Hutten-Czapska v Poland}, for instance, the Court concluded that the case demonstrated ``systemic violation of the right of property''.\footcite[239]{hutten06}

The case concerned a house that had been confiscated during WW2. After the war, the property was transferred back to the owners, but in the meantime, the ground floor had been assigned to an employee of the local city council. Moreover, the state implemented strict housing regulations during this time, which eventually meant that the applicant's house was placed under direct state management.\footcite[20-31]{hutten06} Following the end of communist rule in 1990, the owners were given back the right to manage their property, but it was still subject to strict regulation that protected the rights of the tenants.\footcite[31-53]{hutten06} In addition to rent control, rules were in place that made it hard to terminate the rental contracts. Hence, it became impossible for the owners to make use of the house themselves, as they wished to do.\footcite[20-53]{hutten06} 

After an in-depth assessment of the relevant parts of Polish law and administrative practice, the Grand Chamber of the ECtHR concluded that there had been a violation of P1(1). Importantly, they did not reach this conclusion by focusing on the house as a source of financial entitlements for the owners. Rather, they focused on the overall character of the Polish system for rent control and housing regulation, as it manifested in the concrete circumstances of the applicant's case. The financial consequences for the owners were considered, as was the financial situation of the tenants.\footcite[60-61]{hutten06} The Court was particularly concerned with the fact that the total rent that could be charged for the house was not sufficient to cover the running maintenance costs.\footcite[224]{hutten06} In particular, it was noted that the consequence of this would be ``inevitable deterioration of the property for lack of adequate investment and modernisation''.\footnote{\cite[224]{hutten06}.}

In the end, the Court concluded that the combination of a rigid rent control system, rules that made it hard for owners to terminate tenancy agreements, and the fact that the State itself had set up these agreements during the days of direct state management, meant that a fair balance had not been struck.\footcite[224-225]{hutten06} The contextual nature of the Court's reasoning is evidenced not only by the extent to which the concrete circumstances are assessed against the goal of fairness, but also by how the Court explicitly places the ``social rights'' of the tenants on equal footing with the property rights of the owners.\footcite[225]{hutten06} 

It is also of interest to note how the Court reasons towards the conclusion that the Polish legal order as such is at fault. In this regard, great weight is placed on the observation that the system suffers from a lack of adequate safeguards to protect owners against imbalances such as those identified in the present case. In particular, the Court comments on ``the absence of any legal ways and means making it possible for them either to offset or mitigate the losses incurred in connection with the maintenance of property or to have the necessary repairs subsidised by the State in justified cases''. Hence, the rent control scheme alone was not the whole problem, the Court also criticized what it saw as a defective way of implementing it.\footcite[224]{hutten06} Moreover, the Court did not censor the political reasoning that motivated Polish housing legislation, but concluded instead that the ``burden cannot, as in the present case, be placed on one particular social group, however important the interests of the other group or the community as a whole''. 

I think this is the most important aspect of the case, pointing to the core function that the ECtHR should embrace more generally. It seems to me, in particular, that objections can be raised against the appropriateness of having the Court in Strasbourg assess concretely what is fair regarding the relationship between owners and tenants in a specific house in Gdynia. Its remoteness to the local conditions, as well as its lack of sensitivity and accountability to locally grounded political processes, suggest that the Court is not ideally placed to carry out the kind of contextual assessment that it prescribes for such cases. In addition, the amount of resources and time needed to independently scrutinize these aspects convincingly risks undermining its ability to deal expediently with its case load. The ECtHR will hardly be able to protect human rights in Europe on a case-by-case basis.

Instead, the aim should always be to get at the systemic features that cause perceived imbalances. As in \textcite{hutten06}, the Court serves its function best when it is able to identify a sense in which the domestic legal order needs to be improved to better comply with human rights standards. This is particularly true when, as in that case, the Court notes that the applicants have insufficient options available for achieving a fair balance by appealing to institutions within the domestic legal order. By demanding {\it institutional} changes, in particular, the Court effectively delegates responsibility for ensuring the kind of fair balance that is required under the ECHR. Moreover, by scrutinizing the procedures and principles that the states apply when fulfilling this duty, it is likely that the Court will still be able to steer and unify the development of the case law. Importantly, they would then be able to do so without having to engage extensively in concrete assessments of fairness. 

Against this, one may argue that the judicial or administrative bodies of the signatory states can easily circumvent their obligations by giving a superficial or biased assessment of the facts in human rights cases, to avoid embarrassment for the state's political or bureaucratic elite. However, this might then be raised as a procedural complaint before the ECtHR, resulting in cases revolving around Articles 6 (fair trial) and 13 (effective remedy).\footnote{I note that this also fits with recent developments at the ECtHR, toward somewhat broader scrutiny under Article 6, see \cite{khamidov07}.}  In this way, the Court can streamline its functions, by always aiming to direct attention at issues that arise at a higher level of abstraction. This, in my view, is desirable. The ECtHR should not aim to micromanage the signatory states, particularly not in relation to a norm such a P1(1), which the Court itself regards as highly dependent on context.

However, the question arises as to what kind of institutions the Court should focus on in its effort to ensure fairness in relation to Convention rights such as property. It is not given, in particular, that directing attention towards domestic judicial bodies is the most appropriate approach. Rather, it seems logical to assume that those institutions most in need of reform will be those that are actually responsible for violations. A possible lack of an effective complaints procedure would be worrying, but hardly as problematic as possible systemic weaknesses that give rise to complaints in the first place. With this change in perspective, the Court can avoid getting stuck in deference to domestic judicial bodies, but still shift attention away from concrete assessment of alleged violations. They can do so, in particular, by concretely and critically assessing those rules and procedures that are identified as causally significant to individual complaints. \footnote{In the future, one might even encounter cases when the Court prefers to remain agnostic about whether a substantive violation occurred, focusing instead on the possible violation inherent in excessive systemic risks and a shortage of adequate safeguards.}

Indeed, I think the case of \textcite{hutten06} is suggestive of a move towards such a perspective. While the Court went into great detail about the facts of the case, it {\it also} looked at the case from an alternative perspective, more in line with the suggestion sketched above. In fact, I think it is likely that the Court will eventually veer even more towards such an approach, while deferring to national judicial bodies when it comes to concrete factual assessments. If not as a result of policy, I imagine this will happen from necessity, at least in relation to rights such as property, which now seem to flood the Court.

One might ask where this would leave the proportionality doctrine. In fact, I think this doctrine still makes good sense when framed in more abstract terms as the question of what kinds of rules, and what kinds of institutions, member states need to put in place to ensure fairness. In \textcite{hutten06}, the Court moved in this direction, when it explained the basic principle as follows:

\begin{quote}
In assessing compliance with Article 1 of Protocol No. 1, the Court must make an overall examination of the various interests in issue, bearing in mind that the Convention is intended to safeguard rights that are “practical and effective”. It must look behind appearances and investigate the realities of the situation complained of. In cases concerning the operation of wide-ranging housing legislation, that assessment may involve not only the conditions for reducing the rent received by individual landlords and the extent of the State’s interference with freedom of contract and contractual relations in the lease market, but also the existence of procedural and other safeguards ensuring that the operation of the system and its impact on a landlord’s property rights are neither arbitrary nor unforeseeable. Uncertainty – be it legislative, administrative or arising from practices applied by the authorities – is a factor to be taken into account in assessing the State’s conduct. Indeed, where an issue in the general interest is at stake, it is incumbent on the public authorities to act in good time, in an appropriate and consistent manner.\footcite[151]{hutten06} 
\end{quote}

I note how the Court builds on the earlier precedent set by cases such as \textcite{sporrong82} and \textcite{james86}. The first half of the quote, therefore, stresses that the Court itself must ``look to the realities of the situation''. However, in clarifying what is meant by this, the Court goes on to emphasize procedural aspects. In particular, it is made clear that the Court regards such aspects as an integral part of those ``realities'' that need to be assessed. Indeed, the Court even makes specific reference to the importance of several values that arise in the context of administrative law, such as predictability and effectiveness.

The passage above was subsequently quoted in {\it Lindheim and others v Norway}. In this case, the applicants complained that their rights had been violated by a recent Norwegian act that gave lessees the right to demand indefinite extensions of ground leases on pre-existing conditions.\footcite[119]{lindheim12}  In the end, the Court concluded that there had indeed been a breach of P1(1). Interestingly, they engaged in the same form of assessment that they had adopted in \textcite{hutten06}. They held, in particular, that it was the Act itself which was the underlying source of the violation, not merely its concrete application against the applicants. Hence, the Court did not only award compensation, it also ordered that general measures had to be taken by the Norwegian State to address the structural shortcomings that had been identified.

In this case, the Court also commented that its decision should be regarded in light of ``jurisprudential developments in the direction of a stronger protection under Article 1 of Protocol No. 1''.\footcite[135]{lindheim12} However, in light of the change in perspective that accompanies this development, it is interesting to ask in what sense exactly the protection is stronger. It is not {\it prima facie} clear, in particular, that the Court's remark should be read as a statement expressing a change in its understanding of the content of individual rights under P1(1). Rather, I am inclined to read it as a statement to the effect that the Court now assumes it has greater authority to address structural problems under that provision. This authority, in particular, is now seen to extend also to the fair balance requirement, not only the (much more narrowly drawn) legality and legitimacy rules. In effect, this also leads to stronger protection for individuals, since it allows the Court to conclude that a violation has occurred due to ``structural unfairness'', even when it is not possible to trace this back to any ``flawed'' decision that directly targets the applicants.

What is the relevance of all this to the issue of economic development takings? A great deal, I think. Indeed, I am struck by how the reasoning of the ECtHR in recent cases on hunting and rent control mirrors the kind of reasoning that Justice O'Connor engaged in when she considered {\it Kelo}. The emphasis is on structural aspects and fairness, but the considerations made in this regard are grounded on what the facts of the concrete case reveal about the rules and procedures involved. In this way, the contextual approach to property gains focus without losing its bite. Individual entitlements play a relatively minor role, the assessment is highly relational, focusing on where the system involved places different legal persons in relation to one another. The crux of arguments used to conclude violation is the observation that the system offends against the role that owners {\it should} occupy in order to be able to meet those obligations and exercise those freedoms that society normally regards as inherent to the form of property that they possess.

In this case, interference is not only unfair, it is also a failure of governance, and a structural inconsistency. In the case of \textcite{hutten06}, this boiled down to the observation that the system which had led to the complaint sought to resolve problems in the Polish housing sector in a manner that placed the burden ``on one particular social group'', namely the owners.\footcite[225]{hutten06} This conclusion was backed up by the concrete observation that the rules and procedures in place meant that owners who were expected to maintain their properties in good condition for their tenants were in fact prevented from doing so because they were not permitted to charge rents that would cover the costs.

In the case of {\it Kelo}, Justice O'Connor argued in  a similar fashion when she concluded that the system which had led to the decision to condemn Suzanne Kelo's house was likely to function so as to systematically ``transfer property from those with fewer resources to those with more''. This conclusion was backed up by the observation that the beneficiary in {\it Kelo} was a multi-billion dollar commercial company that had been allowed to take Kelo's home because this would lead to ``economic development''. To Justice O'Connor, there was little doubt that this could become a general pattern, if safeguards were not in place. Indeed, it must be presumed that a multi-million dollar company is always in a better position than a homeowner when it comes to arguing that  ``economic development'' will result from their ownership. More subtly, her opinion also hinted at the inconsistency involved in asserting abstractly that economic development would benefit the community indirectly, all the while the development would if fact require razing it.

To conclude, I think the ECtHR is more likely to approach a case like {\it Kelo} in the manner Justice O'Connor did. Whether they would reach the same result seems more uncertain, particularly since confidence in states' ability and willingness to regulate private-public partnerships might be higher in Europe. However, it seems unlikely that the ECtHR would follow the majority in {\it Kelo}, by simply deferring to the determinations made by the granting authority. Moreover, with the recent change in perspective towards a more structural assessment of property institutions at the ECtHR, it seems that Justice O'Connor's predictions about the ``fallout'' of the {\it Kelo} decision would likely strike a cord with the justices at Strasbourg. 

\section{The US perspective on economic takings}\label{sec:us}

I now consider US law in more depth. First, I track the development of the case law on the public use restriction in the Fifth Amendment and in various state constitutions, from the early 19th Century up to the present day.\footnote{The public use clause in the US constitution was not held to apply to state takings until the late 19th Century, see \cite{chicago97}.} Many writers assert that the 19th and early 20th Century was characterized by a ``narrow'' approach to public use which eventually gave way to a broader conception.\footnote{See, e.g., \cite[483]{walt11}; \cite[203-204]{allen00}. For a more in-depth argument asserting the same, see \cite{nichols40}.} Against this, I argue that it is more appropriate to think of this period as one when courts adopted a {\it broad} approach to judicial scrutiny of the takings purpose at state level. Importantly, I also argue that while different state courts expressed different theoretical views on the meaning of ``public use'', there was a growing consensus that the approach to judicial scrutiny should be contextual, focused on weighing the rationale of the taking against the concrete social, political and economic circumstances of the local area.\footnote{A summary of state case law that supports this view is given in the little discussed Supreme Court case of \cite{hairston08}.}  In particular, I argue that early state courts did not focus as much on the exact wording of the constitutional property clause as many later commentators have suggested.

I go on to show that the doctrine of deference that was developed by the Supreme Court early in the 20th Century was directed primarily at state courts, not state legislatures and administrative bodies.\footnote{See \cite{vester30} (echoing and citing \cite{hairston08}).} I then present the case of {\it Berman}, arguing that it was a significant departure from previous case law.\footcite{berman54} After {\it Berman}, deference was suddenly taken to mean deference to the (state) legislature, so there would be little or no room for judicial review of the takings purpose. I go on to present the subsequent developments at state level, characterized by increasing worry that the eminent domain power could be abused by powerful commercial actors. I discuss the case of {\it Poletown}, where a neighborhood of about 1000 homes was razed to provide General Motors with land to assemble a car factory.\footcite{poletown81} I link this to the subsequent controversy that arose over {\it Kelo}, suggesting that it should be seen as the eventual backlash of {\it Berman}, a consequence of abandoning the contextual approach to public use in favor of an almost absolute rule of deference.

After the historical overview, I go on to briefly present the vast amount of research that has targeted economic takings in the US after {\it Kelo}. I give special attention to writers that propose new legitimacy-enhancing institutions for facilitating economic development of jointly owned land. I focus on two proposals in particular, targeting compensation and participation respectively.\footcite{lehavi07,heller08} These proposals will serve as important reference points later on, when I consider the Norwegian appraisal  and land consolidation courts in Chapters 4 and 5.

\section{The history of the public use restriction}\label{sec:hop}

Going back to the time when the Fifth Amendment was introduced, there is not much historical evidence explaining why the takings clause was included in the bill of rights, and little in the way of guidance as to how it was originally understood. James Madison, who drafted it, commented that his proposals for constitutional amendments were intended to be uncontroversial to Congress.\footnote{See letters from Madison to Edmund Randolph dated 15 June 1789 and from Madison to Thomas Jefferson dated 20 June 1789, both included in \cite{madison79}.}  Hence, it is natural to regard it as a codification of an existing principle, rather than a novel proposal. Indeed, several State constitutions pre-dating the Bill of Rights also included takings clauses, and they all seem to be largely based on a codification of principles from English Common law.\footcite[See][299]{johnson11}

As we discussed in subsection \ref{sec:england} above, the typical English attitude from this time, which was also reflected in the law, held private property in very high regard. On this background it is not surprising that Madison regarded the property clause as an uncontroversial amendment.\footnote{Indeed, early American scholars also emphasized the importance of private property. For instance, in his famous {\it Commentaries}, James Kent described the sense of property as ``graciously implanted in the human breast'' and declared that the right of acquisition ``ought to be sacredly protected'', \cite[see][257]{kent27}.} Its importance may in fact have been greater as a legitimizing force, increasing confidence in the regulatory power of the newly established state by setting up clear parameters for the exercise of that power.  However, while the principle itself was regarded as self-evident, it was never clear what it would mean in practice, particularly in cases when takings where challenged on the basis that they were not for a ``public use'''.\footcite[See][317]{johnson11} 

There are two points that I would like to record about the early common law in the US  in this regard. First, the distinction between public use and public purpose does not appear to have been considered sharp. In his {\it Commentaries}, James Kent first makes clear that the power of eminent domain is for ``public use, and public use only", but then goes on to qualify this by stating that a taking which served a ``purpose not of a public nature'' would be unconstitutional.\footcite[See][275-276]{kent27}  He does not address this limitation in any detail, however, suggesting that it was not the subject of much debate at this time. To the founders, it seems that the right to compensation was considered the more important principle, something that is also reflected in the {\it Commentaries}.\footnote{James Kent held it to be  ``founded in natural equity'' and described it as an ``acknowledged principle of universal law'', \cite[see][276]{kent27}.} The public use limitation was probably taken for granted as a matter of principle, while it had not yet proved problematic as a matter of practical adjudication. Moreover, it appears to have been accepted that takings which clearly benefited the public would be legitimate regardless of whether or not the property was physically put to use by the public.\footcite{johnson11}

An interesting early illustration of how courts approached takings controversies at this time can be found in {\it Stowell v Flagg}, a Massachusetts case from 1814. In this case, a landowner complained that his land had been flooded by a mill and sought a remedy in common law. The mill owner protested, however, since he was entitled to flood the land according to a special mill act, which allowed him to exercise the power of eminent domain to gain the right to flood his neighbor (provided statutory compensation was paid). The focus in the case was on whether a common law claim for damages could still be made, irrespectively of the act's clear intention to deprive the affected neighbors of this opportunity. Hence, the court implicitly dealt with the legitimacy of the mill act itself, and they actively engaged with the public use requirement in the state constitution when making their assessment.\footcite{stowell14} In the end, they found that the act was legitimate, and they highlighted the purpose of the interference, commenting that ``these mills, early in the settlement of this country, were of great public necessity and utility''.\footcite[366]{stowell14} 

At the same time, however, the court had misgivings about how the act had come to be applied and expressed concern that ``the legislature, as well as the courts of law in this state, seem to have been disposed rather to enlarge, than to curtail, the power of mill owners''.\footcite[366]{stowell14} Still, after noting that affected land owners were entitled to compensation under the act, the court concluded that the act had to be observed and that it precluded any claims for damages under common law. Hence, the case is an early example of judicial deference to the legislature in takings cases, while also illustrating that the public use requirement was beginning to emerge as a potentially problematic issue in its own right. The presiding judge stated that he could not help thinking that the statute was ``incautiously copied from the ancient colonial and provincial acts'', but still held in favor of the mill owner,  concluding that ``as the law is, so must we declare it''.\footcite[368]{stowell14}

While judicial deference was recognized as a guiding principle early on in US takings law, it is important to note in this regard that eminent domain was seldom used in a way that would raise serious controversy. English common law, while lacking clearly defined constitutional safeguards, was, as we have already mentioned, based on a fundamentally cautious attitude, ensuring that the power would typically only be used as a last resort. As Professor Meidinger notes, the British were never really charged with abuse of eminent domain, and private property tended to be respected, also in the colonies.\footcite[17]{meidinger80} This undoubtedly influenced early US law. Indeed, the importance of constitutional limits on the taking power was made clear by the Supreme Court early on, as a matter of principle.\footnote{As reflected in {\it de dicta} comments from {\it Calder v Bull} and {\it Vanhorne’s Lessee v Dorrance}, see \cite[388]{calder98}; \cite[310]{vanhorne95}.} Hence, the relative lack of judicial interest in the question of legitimacy does not appear to have been due to a broad view on the scope of eminent domain, but an established practice of narrow use of that power, inherited from the English.

%The Legislature declare and enact, that such are the public exigencies, or necessities of the State, as to authorise them to take the land of A. and give it to B.; the dictates of reason and the eternal principles of justice, as well as the sacred principles of the social contract, and the Constitution, direct, and they accordingly declare and ordain, that A. shall receive compensation for the land. But here the Legislature must stop; they have run the full length of their authority, and can go no further: they cannot constitutionally determine upon the amount of the compensation, or value of the land. Public exigencies do not require, necessity does not demand, that the Legislature should, of themselves, without the participation of the proprietor, or intervention of a jury, assess the value of the thing, or ascertain the amount of the compensation to be paid for it. This can constitutionally be effected only in three ways.
%1. By the parties that is, by stipulation between the Legislature and proprietor of the land.
%2. By commissioners mutually elected by the parties.
%3. By the intervention of a Jury.

The traditional attitude to eminent domain would eventually give way to a more expansive approach, however. This development became particularly marked during the period of great economic expansion and industrialization in the mid to late 19th century, when eminent domain was increasingly used to benefit (privately operated) railroads, hydroelectric projects, and the mining industry.\footcite[23-33]{meidinger80} During this time, it also became increasingly common for landowners to challenge the legitimacy of takings in court, undoubtedly a consequence of the fact that eminent domain was now used more widely, for new kinds of projects.\footcite[24]{meidinger80} Controversy arose particularly often with respect to mill acts.\footnote{\cite[24]{meidinger80}. See also \cite[306-313]{johnson11} and \cite[251-252]{horwitz73}.} Such acts were found throughout the US, and many of them dated from pre-industrial times when mills were primarily used to serve the needs of self-sufficient agrarian communities.\footnote{A total of 29 states had passed mill acts, with 27 still in force, when a list of such acts was compiled in \cite[17]{head85}. According to Justice Gray, at pages 18-19 in the same, the ``principal objects'' for early mill acts had been grist mills typically serving local agrarian needs at tolls fixed by law, a purpose which was generally accepted to ensure that they were for public use.}  However, following economic and technological advances, acts that were once used to facilitate the construction of grist mills would increasingly also be relied on by developers wishing to harness hydropower for manufacturing, and eventually, for hydroelectric projects.\footnote{See, e.g., \cite[18-21]{head85} and \cite[449-452]{minn06}.}

The mill acts typically contained provisions that enabled the mill developer to condemn both property needed for the construction itself as well as the right to damage surrounding land by flooding or deprivation of water. Such takings became increasingly controversial, however, and many legitimacy cases came before state courts in the late 19th and early 20th century. In the next subsection I present some of these cases, to shed light on how states courts developed their own approach to the question of legitimacy of takings.

\subsection{Legitimacy in state courts}\label{subsec:state}

In the mil cases, we find the first clear evidence of how the public use requirement was put to use to enable state courts to scrutinize the legitimacy of takings. Generally speaking, when a court upheld an interference in private property, it would place decisive weight on the broader purpose of interference, typically by arguing that economic ripple effects ensured that the mill was in the public interest even if the public would not literally make use of it.\footnote{See, e.g., \cite{hazen53,scudder32,boston32}. A more comprehensive list of cases adopting a broad view can be found in \cite[617]{nichols40}.} By contrast, when a court decided that an interference was unconstitutional (with respect to state constitutions), it would often focus on the use made of the mill, arguing that it did not directly benefit the public in the sense required by the public use restriction.\footnote{See, e.g., \cite{sadler59,ryerson77,gaylord03,minn06}. A more comprehensive list can be found in {\it Public benefit or convenience as distinguished from use by the public as ground for the exercise of the power of eminent domain} 54 ALR 7 (American Law Reports, 1928).} For a time, a doctrine which sought to distinguish between takings for public use and takings for a public purpose, played quite a significant role in many states. Under this doctrine, only those takings that were deemed to qualify as public use takings under a narrow view of that term would be upheld.\footnote{Professor Nichols goes as far as to conclude that this emerged as the ``majority'' opinion on public use, see \footcite[617-618]{nichols40}. But contrast this with \cite{berger78} and \cite[24]{meidinger80}, who argue that the narrow view was only dominant in a handful of states, led by New York.}

%For instance, in the case of {\it Gaylord v. Sanitary Dist. of Chicago}, the Supreme Court of Illinois held the state Mill Act to be unconstitutional, as it was not limited to traditional flour mills. In doing so, the court observed that public use was ``something more than a mere benefit to the public''.\footcite[524]{gaylord03} Similar sentiments were expressed in other decisions striking down uses of eminent domain for mill construction, for instance in Vermont, Michigan and New York.\footnote{References.}

It is tempting to associate the narrow view on public use with a more restrictive attitude towards the use of eminent domain. Similarly, it is natural to assume that a broad view on public use suggests a more relaxed attitude. To some extent, the primary sources warrant this; unsurprisingly, those who endorsed a broad view on the public use question also often spoke in favor of judicial deference in legitimacy cases, while those endorsing a narrow view tended to emphasize the importance of constitutional safeguards against abuse of eminent domain. However, it seems that both groups were quite heterogeneous and that differences of opinion about the public use requirement did not necessarily reflect any deep ideological divisions.

It is clear, for instance, that many of the courts which favored a broad interpretation of public use still viewed the constitutional limitation on the takings power as an important safeguard, not only as a guarantee for compensation but also as a restriction on the purpose of takings. Indeed, it seems that most late 19th Century Courts, including those that upheld economic takings, were influenced by the growing body of case law across the US that actively scrutinized takings, sometimes striking them down. In particular, it seems that the strict deferential view was largely abandoned in economic takings cases during this period. Deference to the legislature still played an important role and was typically called on as an important argument in takings cases. However, it became much more common to discuss legitimacy also in terms of substantive arguments, by directly addressing the context and circumstances of the taking complained of. I believe this is an important insight to record about the case law from this period; despite differences of opinion about the meaning of public use, a consensus appears to have emerged that judicial review of legitimacy was appropriate and important in economic takings cases.

A good example is the case of {\it Dayton Gold \& Silver Mining Co. v. Seawell}, concerning a Nevada Act which stipulated that mining was a public use for which the power of eminent domain could be exercised to acquire additional rights needed to facilitate extraction.\footcite{seawell76} The Supreme Court of Nevada decided that the Act was constitutional and adopted a broad understanding of the property clause in the Nevada constitution.\footnote{Nev Const Art 8 § 1.} Interestingly, it argued for this interpretation partly on the basis that it would provide {\it better} protection for landowners:

\begin{quote}
If public occupation and enjoyment of the object for which land is to be condemned furnishes the only and true test for the right of eminent domain, then the legislature would certainly have the constitutional authority to condemn the lands of any private citizen for the purpose of building hotels and theaters. [...] Stage coaches and city hacks would also be proper objects for the legislature to make provision for, for these vehicles can, at any time, be used by the public upon paying a stipulated compensation. It is certain that this view, if literally carried out to the utmost extent, would lead to very absurd results, if it did not entirely destroy the security of the private rights of individuals. Now while it may be admitted that hotels, theaters, stage coaches, and city hacks, are a benefit to the public, it does not, by any means, necessarily follow that the right of eminent domain can be exercised in their favor.\footcite[410-411]{seawell76}
\end{quote}

The quote shows that a broad understanding of ``public use'' need not be synonymous with a less cautious attitude to abuse of the takings power. Indeed, while the Court decided to uphold the Act, it did so only after a very careful assessment of both legal arguments and factual circumstances. In particular, the Court considered the importance of mining, concluding that it was the ``greatest of the industrial pursuits'' in the state, and that all other interests were ``subservient'' to it.\footcite[409]{seawell76} Moreover, the Court commented that the benefits of the mining industry was ``distributed as much, and sometimes more, among the laboring classes than with the owners of the mines and mills''.\footcite[409]{seawell76}

This shows that the Court actively engaged with the purpose of the Act, thoughtfully assessing it against the constitution. Importantly, it did not do so in isolation, as a linguistic exercise or by attempting to recreate its ``original intent''. Rather, the court approached the constitutional safeguard by making detailed references to the prevailing social and economic conditions in the state of Nevada. The Court noted the importance of deference to the legislature on matters of policy, but it did so only after it had satisfied itself that the Act could be ``enforced by the courts so as to prevent its being used as an instrument of oppression to any one''.\footcite[412]{seawell76} More generally, the court commented as follows on the public purpose test that had to be performed in takings cases, elucidating on the principles on which it should be founded:

\begin{quote}
 Each case when presented must stand or fall upon its own merits, or want of merits. But the danger of an improper invasion of private rights is not, in my judgment, as great by following the construction we have given to the constitution as by a strict adherence to the principles contended for by respondent.\footcite[398]{seawell76}
\end{quote}

In light of this, {\it Dayton Gold \& Silver Mining Co. v. Seawell} must be regarded as an early example of a {\it contextual} approach to legitimacy, characterized by the willingness of the Court to engage in a fairly detailed analysis of the concrete circumstances and consequences of takings. A formalistic approach based on the phrase ``public use'' was abandoned, but not in favor of general deference. Rather, a more nuanced view was adopted, to respect the idea that the legislature should have the final say on policy while also recognizing that courts should play a crucial role in protecting citizens from abuse of the takings power. 

The case is not unique, but rather exemplifies the type of reasoning that was used in economic takings cases at this time. Interestingly, many common elements exist between courts that upheld and struck down such takings, irrespectively of whether or not they subscribed to a narrow or broad view on the public use test. One example is {\it Ryerson v. Brown}, a case often cited as an authority in favor of a narrow view.\footcite{ryerson77} Here the Supreme Court of Michigan explicitly qualifies its decision by stating that it is ``not disposed to say that incidental benefit to the public could not under any circumstances justify an exercise of the right of eminent domain'', hardly a clear endorsement of the narrow rule. The case concerned the constitutionality of a mill act, and while the court argues that public use should be taken to mean ``use in fact'', it is clear that ``use'' is understood rather loosely, not literally as physical use of the property that is taken.\footnote{The court explains its stance on the public use restriction by stating (emphasis added) ``it would be essential that the statute should require the use to be public in fact; in other words, that it should contain provisions entitling the public to {\it accommodations}.'' The court continues with an illustrative example: ``A flouring mill in this state may grind exclusively the wheat of Wisconsin, and sell the product exclusively in Europe; and it is manifest that in such a case the proprietor can have no valid claim to the interposition of the law to compel his neighbor to sell a business site to him, any more than could the manufacturer of shoes or the retailer of groceries. Indeed the two last named would have far higher claims, for they would subserve actual needs, while the former would at most only incidentally benefit the locality by furnishing employment and adding to the local trade''. See \cite[336]{ryerson77}.} Moreover, when clarifying its starting point for judicial scrutiny of mill acts, the court explains that ``in considering whether any public policy is to be subserved by such statutes, it is important to consider the subject from the standpoint of each of the parties''. Following up on this with regards to the act in question, the court finds that `` the power to make compulsory appropriation, if admitted, might be exercised under circumstances when the general voice of the people immediately concerned would condemn it''. After considering this and other possible consequences of mill development under the act, the court eventually declares it to be unconstitutional, summing up its assessment as follows: ``What seems conclusive to our minds is the fact that the questions involved are questions not of necessity, but of profit and relative convenience''.\footcite[336]{ryerson77}

Hence, far from nitpicking on the basis of the public use phrase, the court adopts a contextual approach to takings that is in fact rather similar to the approach of {\it Dayton Gold \& Silver Mining Co. v. Seawell}. The outcome it different, but it is also based on a different assessment of the context and the consequences of the takings complained of. Importantly, the case does not rest on any {\it a priori} assumption that economic takings of the kind in question could not meet a public use test -- no general rule is relied on at all. Hence, it is somewhat strange that later commentators have focused on the case for its comments on public use rather than its broad, albeit perhaps somewhat conservative, assessment of legitimacy. 

Many of the important cases from the late 19th Century, on both sides of the public use debate, shares many crucial features with the two cases discussed above.\footnote{See, e.g., \cite{scudder32} (Eminent domain power upheld, but said: ``The great principle remains that there must be a public use or benefit. That is indispensable. But what that shall consist of, or how extensive it shall be to authorize an appropriation of private property, is not easily reducible to a general rule. What may be considered a public use may depend somewhat on the situation and wants of the community for the time being.''), \cite{fallsburg03} (Eminent domain struck down, on holding that ``the private benefit too clearly dominates the public interest to find constitutional authority for the exercise of the power of eminent domain''), \cite[538]{board91} (Eminent domain struck down, qualified by ``not only must the purpose be one in which the public has an interest, but the state must have a voice in the manner in which the public may avail itself of that use'').} In my opinion, this points to an interesting alternative perspective on legitimacy adjudication from this time. Some commentators describe the case law as chaotic, with competing conceptions of constitutional limits competing for dominance.\footcite{berger78,meidinger80}. I think this is more accurate than saying that a narrow interpretation of public use developed as a general rule. However, I also find evidence that there was in fact a broad consensus in this period regarding the need for special judicial scrutiny of economic development cases. State courts widely engaged in contextual assessment of legitimacy, and they were conscious of the special challenges that arose in a time when eminent domain was being used to facilitate economic expansion that would benefit specific commercial actors. Differences of opinion about public use terminology was an important aspect of this, but it was rarely considered in isolation from other aspects. On a deeper lever, the fact that the public use debate was regarded as important in the first place clearly suggests that deference to the legislature was not held to be an exhaustive answer to the question of legitimacy. This, in my opinion, is an important observation which appears to have been somewhat overlooked in the literature. 

It is an observation that I think is relevant not only in relation to state law, but also when considering the takings doctrine that was later developed by the Supreme Court. While the narrow view of public use was indeed losing ground at the beginning of the 20th Century, the doctrine of extreme deference that was about to be adopted at the federal level represents a largely new development. The new deference was not originally directed at the legislature, in particular, but primarily towards the judiciary at the state level. Hence, it represent a development that is in some sense incomparable to the earlier case law from the states. The balance of power between states and the federal government also played an important role, which should not be overlooked.

\subsection{Legitimacy as discussed in the Supreme Court}\label{subsec:US}

Initially, the Supreme Court held that the takings clause in the US Constitution did not apply to state takings at all.\footcite{barron33} Federal takings, on the other hand, were of limited practical significance since the common practice was that the federal government would rely on the states to condemn property on their behalf.\footcite[30]{meidinger80}. This changed towards the end of the 19th Century, particularly following the decision in {\it Trombley v. Humphrey}, where the Supreme Court of Michigan struck down a taking that would benefit the federal government.\cite{trombley71} Not long after, in 1875, the first Supreme Court adjudication of a federal taking case occurred, marking the start of the development of the Supreme Court's own doctrine on public use and legitimacy.\footcite{kohl75} Eventually, in 1897, the Court would also hold that state takings could be scrutinized under the takings clause of the constitution.\footcite{chicago97} This was a development that can be traced to the passage of the Fourteenth Amendment to the Constitution after the civil war, concerning due process.\footcite{johnson11}. Indeed, some early Supreme Court cases dealing with state takings were adjudicated against the due process clause directly.\footnote{See, e.g., \cite{head85}.}

After the Supreme Court started developing its own case law on the legitimacy issue, the deferential stance soon became entrenched. As argued by Professor Horwitz, the mid to late 19th Century was the period in US history when control over property was transferred on a massive scale from agrarian communities to various agents of industrial expansion.\footcite{horwitz73} Moreover, it was a period of great optimism about the ability of {\it laissez faire} capitalism to ensure progress and economic growth. This was also reflected in the case law on eminent domain, particularly as developed by the Supreme Court. A particularly clear expression of this can be found in {\it Mt. Vernon-Woodberry Cotton Duck Co v Alabama Interstate Power Co}.\footcite{vernon16}  This case dealt with the legitimacy of a condemnation arising from the construction of a hydropower plant, which the Alabama Supreme Court had upheld against claims that it was unconstitutional under the constitution of Alabama. The presiding judge held that it was valid using quite brisk language:

\begin{quote}The principal argument presented that is open here, is that the purpose of the condemnation is not a public one. The purpose of the Power Company's incorporation, and that for which it seeks to condemn property of the plaintiff in error, is to manufacture, supply, and sell to the public, power produced by water as a motive force. In the organic relations of modern society it may sometimes be hard to draw the line that is supposed to limit the authority of the legislature to exercise or delegate the power of eminent domain. But to gather the streams from waste and to draw from them energy, labor without brains, and so to save mankind from toil that it can be spared, is to supply what, next to intellect, is the very foundation of all our achievements and all our welfare. If that purpose is not public, we should be at a loss to say what is. The inadequacy of use by the general public as a universal test is established. The respect due to the judgment of the state would have great weight if there were a doubt. But there is none.\footcite[]{vernon16}
\end{quote}

The quote serves as an indication of how deference was fast gaining ground, without yet being established doctrine. On the one hand, the Court stresses that deference to the {\it state} judgment (rather than the judgment of the legislature) should be given great weight in legitimacy cases. On the other hand, it prefers to conclude on the basis of its own assessment of the purpose of the taking. This assessment, however, is not particularly grounded in the circumstances on the ground in Alabama, being based rather on sweeping assertions about the ``organic relations of modern society'' and the desire to ``save mankind from toil that it can be spared''. 

This judgment, from 1916, was given during the so-called {\it Lochner} era of jurisprudence in the US, when the Supreme Court  would famously engage in active censorship of regulation that was meant to promote greater social and economic equality.\footcite{cohen08} In particular, much case law from this period witnesses to a general lack of deference. Hence, it is not unexpected to find that public use cases decided on the basis of substantive arguments. However, it is rather more surprising to find that deference actually played an increasingly important role in takings cases.\footnote{The {\it Lochner} era in general was characterized by courts engaging in censorship of state regulation, but this general tendency is not well reflected in how eminent domain law developed over the same period. This is interesting, as it points to the shortcoming of another commonly held view on property protection, namely that it largely serves the interests of property-owning elites, to the detriment of regulatory efforts to promote social equality. The cases through which {\it Lochner} era courts developed the deferential stance suggest a different interpretation; those who benefited most directly from takings in these cases were commercial interests, not vulnerable groups of society. Moreover, they benefited from acquiring land rights from members of agrarian communities, not from the elites. Hence allowing such takings to go ahead was no affront to the ideology of progress through {\it laissez faire} capitalism, quite the contrary. In particular, if it is true as many have argued, that the {\it Lochner} courts were ideologically committed to the promotion of unrestrained capitalism, there was little reason for them to oppose expansion of eminent domain into the commercial arena: those who would be likely to benefit were market actors who were proposing large scale commercial development projects. Indeed, the case law from this period makes it natural to argue that the deferential stance developed primarily to cater to the needs of the capitalists, under the perceived view that they represented the class which would bring progress and prosperity to the nation as a whole.} As early as { \it United States v. Gettysburg Electric Railway Co.}, a case from 1896, deference was described as a fundamental guiding principle, which should be adhered to except in very special circumstances.\footcite{gettysburg96} In particular, Justice Peckham lended his support to the following deferential stance on the public use test:

\begin{quote}
It is stated in the second volume of Judge Dillon's work on Municipal Corporations (4th Ed. § 600) that, when the legislature has declared the use or purpose to be a public one, its judgment will be respected by the courts, unless the use be palpably without reasonable foundation. Many authorities are cited in the note, and, indeed, the rule commends itself as a rational and proper one.\footcite[680]{gettysburg96}
\end{quote}

The case did not turn on the public use issue, however, as the condemned land would be used for battlefield memorials at Gettysburg, Pennsylvania, clearly a public use. In addition, the case concerned a federal takings, authorized by Congress. In later cases, the deferential stance was not adopted in cases originating from the states. As late as in 1930, in {\it Cincinatti v Vester}, the Supreme Court commented that the ``‘It is well established that, in considering the application of the Fourteenth Amendment to cases of expropriation of private property, the question what is a public use is a judicial one".\footcite[447]{vester30} In this judgment, Chief Justice Hughes also describes in more depth how the judicial assessment of the public use question should be carried out, echoing the contextual approach that had been developed in case law from the states.

\begin{quote}
In deciding such a question, the Court has appropriate regard to the diversity of local conditions and considers with great respect legislative declarations and in particular the judgments of state courts as to the uses considered to be public in the light of local exigencies. But the question remains a judicial one which this Court must decide in performing its duty of enforcing the provisions of the Federal Constitution.\footcite[447]{vester30}
\end{quote}

In {\it Hairston v. Danville \& W. R. Co.}, the same idea was expressed even more clearly by Justice Moody, who surveyed the state case law and declared that ``The one and only principle in which all courts seem to agree is that the nature of the uses, whether public or private, is ultimately a judicial question.''\footcite[606]{hairston08} He continued by describing in more depth the typical approach of the state courts in determining public use cases:

\begin{quote}
The determination of this question by the courts has been influenced in the different states by considerations touching the resources, the capacity of the soil, the relative importance of industries to the general public welfare, and the long-established methods and habits of the people. In all these respects conditions vary so much in the states and territories of the Union that different results might well be expected.\footcite[606]{hairston08}
\end{quote}

Justice Moody goes on to give a long list of cases illustrating this aspect of state case law, showing how assessments of the public use issue is inherently contextual and varies from state to state.\footcite[607]{hairston08} He then cites three further Supreme Court cases, pointing out that all of them express similar sentiments of support for state case law on this issue.\footnote{{\it Falbrook, Clark} and {\it Strickley}} Following up on this, he points out that ``no case is recalled'' in which the Supreme Court overturned ``a taking upheld by the state {\it court} as a taking for public uses in conformity with its laws'' (my emphasis). After making clear that situations might still arise where the Supreme Court would not follow state courts on the public use issue, Justice Moody goes on to conclude that the cases cited `` show how greatly we have deferred to the opinions of the state courts on this subject, which so closely concerns the welfare of their people''.\footcite[606]{hairston08}

I believe {\it Hairston} is an important case for two reasons. First, it makes clear that initially, the deferential stance in cases dealing with state takings was largely directed at the state courts rather than the state legislature. Second, it demonstrates federal recognition of the fact that a consensus had emerged in the states, whereby scrutiny of the public use determination was consistently regarded as a judicial task.\footnote{Indeed, {\it Hariston} provides the authority for {\it Vester} on this point. See \cite[606]{vester30}.} Moreover, the Court clearly looked favorably on the contextual approach adopted in such cases, whereby state courts would look to the concrete circumstances of the individual takings and acts complained of. The Court's approval of this tradition, in particular, is explicitly given as the reason for adopting a deferential stance. Put simply, the judicial test provided at state level was held to be of such high quality that there was little use for further scrutiny; a deferential stance was assumed, but made contingent on the fact that state courts would provide the required judicial scrutiny.

Despite this, {\it Hairston} would later be cited as an early authority in favor of almost unconditional deference in {\it US ex rel Tenn Valley Authority v Welch}.\footcite[552]{welch46} This case concerned a federal taking and it cited {\it US v Gettysburg Electric R Co} as an authority in favor of strong deference with regards to the public use limitation.\footcite{gettysburg96} However, the Court also paused to note that the later case of {\it City of Cincinnati v Vester} expressed the opposite view, that the public use test was a judicial responsibility.\footcite{vester30} In a very selective citation, the Court then purports to resolve this tension by quoting {\it Hairston} and the observation made there that the Supreme Court had never overruled the state courts in takings cases. Effectively, the importance of judicial scrutiny is thereby downplayed, although as we saw, the rationale behind {\it Hairston} was that state courts already offered high-quality judicial scrutiny of the public purpose.

{\it Welch} is particularly important because it is used as an authority in the later case of {\it Berman v Parker}, which endorses almost complete deference to the legislature regarding the public use issue.\footcite[32]{berman54} This case concerned condemnation for redevelopment of a partly blighted residential area in the District of Colombia, which would also condemn a non-blighted department store. In a key passage, the Court states that the role of the judiciary in scrutinizing the public purpose of a taking is ``extremely narrow''.\footcite[32]{berman54} The Court provides only two citations for this claim, one of them being {\it Welch}. The other case, {\it Old Dominion Land Co v US}, concerned a federal taking of land on which the military had already invested large sums in buildings.\footnote{The Court commented on the public use test by saying that ``there is nothing shown in the intentions or transactions of subordinates that is sufficient to overcome the declaration by Congress of what it had in mind. Its decision is entitled to deference until it is shown to involve an impossibility. But the military purposes mentioned at least may have been entertained and they clearly were for a public use''. See \cite[66]{dominion25} Hence, the Court took the view that courts should be cautious in second-guessing the intentions of Congress on the basis of what its subordinates had subsequently done and said. This is far from a general deferential stance on public use, and no cases are cited at all, suggesting further that the Court did not think its remarks would be of general significance. Still, a partial quote, used to substantiate  broad deference to the legislature (not only Congress, but also the states) except when it involves an ``impossibility'', has become commonplace. In particular, such a quote was used in the much discussed \cite[240]{midkiff84}.}
In my view, both cases are weak authorities for prescribing general deference regarding public use. Moreover, both cases are concerned with federal takings only, while in {\it Berman} the Court explicitly says that deference is due in equal measure to the state legislature.\footcite[32]{berman54} It is possible to see this as a {\it dictum}, since the District of Columbia is governed directly by Congress, but it is a passage that has had a great impact on future cases. In effect, {\it Berman} caused departure from a significant and consistent body of case law which recognized the important role of the judiciary, at state level, in assessing the purported public purpose of takings. It did so, moreover, without engaging with any of these cases at all.

In {\it Hawaii Housing Authority v Midkiff}, the Supreme Court further entrenched the principles of {\it Berman}, in a case where the state of Hawaii had made used of the takings power to break up an oligopoly in the housing sector.\footcite{midkiff84}  However, the fact that the case made it to the Supreme Court is perhaps suggestive of an increase in the level of worry and tension associated with eminent domain in the 1980s. Indeed, Justice Sandra Day O'Connor, joined by a unanimous Supreme Court, expressed general disapproval of private takings and she appears to have felt the need to provide further qualification for the deferential view, which she did in part by observing that ``judicial deference is required because, in our system of government, legislatures are better able to assess what public purposes should be advanced by an exercise of eminent domain''. Hence, judicial deference was not regarded as an absolute and systemic imperative, as in Berman, but made contingent on the fact that legislatures are ``better able'' than courts at conducting public purpose tests. Hence, some of the contextual ideas from earlier case law is echoed in the decision, but now with respect to the legislature. It should be noted that {\it Midkiff} follows {\it Berman} also in the authorities consulted, and does not consider the cases which had focused on the importance of judicial scrutiny at state level.

The purpose of interference in {\it Midkiff} was to break up an oligopoly to the benefit of tenants, not to further economic development by allowing commercial interests to take land. Hence, the rationale behind the interference is likely to have struck the Supreme Court as sound and just. Moreover, it seems that such an interference would be easy to uphold also under the doctrine of contextual judicial scrutiny of the public use determination. Indeed, Justice O'Connor partly relies on an assessment of the merits of the taking, pointing out that  ``regulating oligopoly and the evils associated with it is a classic exercise of a State's police powers''. In conclusion, the ``extremely narrow'' room for judicial review set up by {\it Berman} seems to have been replaced by a slightly more nuanced formulation, which nevertheless made clear that a legal precedent of deference had now become entrenched. Fine readings aside, {\it Midkiff} reaffirms the main principle:  the meaning of public use can be broad, and the room for judicial review of governmental assessments in this regard is narrow.

So far we have only commented on how the Supreme Court developed its own doctrine on the public use restriction in the early 20th Century. Given that its role in takings jurisprudence was limited up to this point, it is important to consider also the effect on state case law. In particular, what was the fallout of {\it Berman}, which failed to recognize the importance of the tradition for judicial scrutiny that had developed at the state level? A detailed assessment of this against primary sources will have to be left for future work. However, it seems clear that {\it Berman} had a significant effect, both conceptually and in practice. A clear indication of this can be found in the secondary literature. Indeed, most academics following WW2 seemed to converge towards the view that the public use requirement was of little or no judicial importance. Professor Merrill, in an influential paper from 1986, goes as far as to describe it as a ``dead letter''.\footcite{merrill86}. At the same time, eminent domain became more controversial in this period, as it was also put to use more aggressively by some states.

 Some concrete cases proved particularly controversial, and they were taken to illustrate the dangers of eminent domain, particularly in relation to economic development projects. While the takings power had traditionally been used mostly to condemn agrarian land rights, it was now regularly used to condemn middle class homes. The controversy surrounding the case of Poletown Neighborhood Council v. City of Detroit  illustrates this, and the case marks a watershed moment in the history of  economic development takings in the US.\footcite[See][380-381]{sandefur05} In {\it Poletown}, the Michigan Supreme Court held that it was not in violation of the public use requirement to allow General Motors to displace some 3500 people for the construction of a car assembly factory. The majority 5-2 cites {\it Berman}, commenting that its own room for review of the public use requirement is limited.\footcite[632-633]{poletown81}

The {\it Poletown} decision was controversial, and the minority, especially Justice Ryan, was highly critical of it. He objects both to the deferential stance in general and to the majority reading of {\it Berman} in particular, pointing out that the Supreme Court's doctrine of deference was in large part directed at the state courts.\footcite[668]{poletown81} Hence, he concludes, the majority's reliance on {\it Berman} is ``particularly disingenuous''.\footcite[668]{poletown81} 

Justice Ryan was not alone in his disapproval of {\it Poletown} and the case is widely regarded as the prelude to an era of increased tensions over economic development takings in the US. This would culminate with {\it Kelo} which, despite upholding an economic development taking, also signaled a move towards more active judicial review of the public use requirement. This effect of {\it Kelo} has become more clear over time, primarily due to state responses caused by widespread disapproval with the outcome. However, it has also been remarked that both the majority and minority opinions in {\it Kelo} indicate that the Supreme Court itself may not be entirely at ease with the doctrine of strict deference that developed after {\it Berman}. In the next subsection, I will give an overview of recent developments, particularly from the secondary literature.

\section{Economic development takings after Kelo}

The fact that {\it Kelo} was decided against the homeowner met with wide disapproval by the US public. In addition, many scholars expressed concern at what they saw as an ill advised ``abdication'' of the judiciary in takings cases. The minority opinions given in {\it Kelo}, particularly the opinion of Justice O'Connor, also proved influential, causing further attention to be directed at the perceived dangers of eminent domain abuse. A massive amount of literature has since appeared devoted to studying the ``problem'' of economic takings. Moreover,  many states have seen reforms aimed to curb the use of eminent domain for economic development.\footnote{For an overview and critical examination of the myriad of state reforms that have followed {\it Kelo}, I point to \cite{eagle08}. See also \cite{somin09}.} 

As of 2014, 44 states have passed post-{\it Kelo} legislation to curb the use of eminent domain for economic development.\footnote{According to the Castle Coalition, a property activist project associated with the Institute of Justice. See \url{http://www.castlecoalition.org/} for an up-to-date survey of state legislation on eminent domain.} Various legislative techniques have been adopted by the states to achieve this. Some states, including Alabama, Colorado, Michigan, enacted explicit bans on economic development takings and takings that would benefit private parties.\footcite[See][107-108]{eagle08} In South Dakota, the legislature went even further, banning the use of eminent domain  ``(1) For transfer to any private person, nongovernmental entity, or other public-private business entity; or (2) Primarily for enhancement of tax revenue''.\footnote{South Dakota Codified Laws § 11-7-22-1, amended by House Bill 1080, 2006 Leg, Reg Ses (2006).}

In other states, more indirect measures were also taken, such as in Florida, where the legislature enacted a rule whereby property taken by the government could not be transferred to a private party until 10 years after the date it was condemned.\footcite[809]{eagle08} Many states also offer inclusive, often lengthy, lists of uses that should count as public, allowing the states to restrict the eminent domain power while also allowing condemnations that are regarded as particularly important to the state.\footcite[804]{eagle08}
However, as argued by Somin, many of these legislative reforms are largely ineffective in preventing economic development takings.\footcite[2120]{somin09} Somin also points to another interesting trend, namely that state reforms enacted by the public through referendums tend to be far more restrictive and effective in preventing economic and private-to-private takings than reforms passed through the state legislature.\footcite[2143]{somin09} 

This is a further reflection of the extent to which the US public opposed the decision in {\it Kelo}. Surveys show that as many as 80-90 \% believe that it was wrongly decided, an opinion widely shared also among the political elite.\footcite[2109]{somin09} Indeed, {\it Kelo} has had a great effect on the discourse of eminent domain in the US, and this effect is perhaps of greater importance than the various state reforms that have been enacted. According to Somin, most of the reforms have in fact been ineffective, despite the overwhelming popular and political opposition against economic development takings.\footcite[2170-2171]{somin09} 

Somin is not alone in feeling that eminent domain reform has offered more than it could deliver, this is a sentiment that is expressed both by supporters and critics of {\it Kelo}. On the other hand, while practitioners have noted that it is largely business-as-usual in eminent domain law, they also report a greater feeling of unease regarding the public use requirement, expressing hope that the Supreme Court will soon revisit the issue.\footnote{See \cite{murakami13} (``Until the Supreme Court revisits the issue, we predict that this question will continue to plague the lower courts, property owners, and condemning authorities'').} In this way, the public backlash against {\it Kelo} has served as an influential reminder that the rationale behind eminent domain for economic development is largely out of sync with the sense of fairness and justice endorsed by most non-experts. 

The underlying cause of this, according to Somin, can be traced to the fact that people are ``rationally ignorant'' about the economic takings issue. For most people, it is unlikely that eminent domain will come to concern them personally or that they will be able to influence policy in this area. Hence, it makes little sense for them to devote much time to learn more about it. This, in turn, helps create a situation where experts can develop and sustain a system based on principles that, in fact, are opposed by a large majority of citizens.\footcite[2163-2171]{somin09} Indeed, Somin argues that surveys show how people tend to overestimate the effectiveness of eminent domain reform, possibly due to the fact that symbolic legislative measures are mistaken for materially significant changes in the law.\footcite{somin09}

I think Somin's analysis is on an interesting track, although it seems wrong to assume {\it a priori} that people's critical stance on economic development takings would necessarily remain in place if they educated themselves more on the issue. Rational ignorance, in particular, should be seen as a double-edged sword in disputes of this kind. But this does nothing to detract from the main message, which is that the {\it Kelo} backlash seems to have caused greater insecurity about what the law is, without being able to significantly curb those uses of eminent domain that have been deemed problematic. In my opinion, this shows that the static legislative approach to eminent domain reform, which has dominated the scene in the US so far, needs to be supplemented by more dynamic proposals. In particular, it seems important to target the decision-making processes surrounding planning and eminent domain, to look for principles by which this process can be imbued with legitimacy. 

In a country where the population expresses antagonism towards eminent domain for economic development, a more inclusive process will likely cause such takings to become more uncommon. On the other hand, if principles of good governance are put in place, it might also restore confidence in eminent domain as a procedure by which to implement democratically legitimate decisions about how to weigh the interests of landowners against the interests of the public. In the next subsection, I will consider two proposals for principles of this kind. The first targets specifically the question of how compensation is determined in economic development cases, a crucial aspect of legitimacy. The second proposal targets the decision-making process more broadly, by proposing a framework for land assembly that is meant to replace the use of eminent domain in certain circumstances.

%\noo{But it is not the general public that are the major stakeholders in such disputes, but rather the communities that are directly affected, including both the private property owners who will be burdened and those community members who stand to benefit. A good framework for balancing their interests relies on finding appropriate principles of good governance, so that governments can play an empowering role when such decisions are made. This is crucial for legitimacy of land use planning generally, but especially for eminent domain, where the gravity of the interference means that legitimacy is unlikely to arise unless the decision to condemn is firmly rooted in the interests of the main stakeholders. To the greatest possible extent, it also seems crucial to emphasize local conditions and ensure that the decision enjoys broad local support. 
%
%Shortly after {\it Poletown} was overturned, the case of Kelo saw the legitimacy of economic takings brought before the Supreme Court once again. This time there was real doubt and disagreement among the justices regarding the scope of the public use limitation. The case revolved around the legitimacy of condemning a home in favour of a research facility for the drug company Pfizer, which was part of a development plan for the City of New London.  The owner, Suzanne Kelo, argued that the condemnation of her home was in breach of the constitution, since it was a private-to-private taking ostensibly to the benefit of Pfizer rather than any clearly defined public use or interest.
%
%In Kelo, Justice Thomas adopted the strictest view on the public use test. He entirely disregarded  the precedent set by Berman and Midkiff in favour of constitutional originalism, the doctrine which asserts that direct assessment of the wording in the Constitution, and the intentions of the founding fathers, is the approach that should be used to decide constitutional cases. Following up on this he held that actual right of use for the public was the test that had to be applied in takings cases. The hundred years of precedent preceding Kelo was described as “wholly divorced from the text, history, and structure of our founding document", and thus Justice Thomas concluded that it had to be abandoned. 
%
%Justice O'Connor, in an expression of dissent joined by Chief Justice Rehnquist and Justices Scalia
%and Thomas, argued against legitimacy on less theoretical grounds, based on the facts of the case and the precedent that would be set for similar cases in the future. Her main legal argument was that while public use should be interpreted broadly, the possibility of positive ripple effects was not enough to justify private-to-private takings. In particular, Justice O'Connor took a very bleak view on the practical consequences that would arise from allowing economic takings that could be justified only by pointing only to indirect positive consequences for the public. She commented on the majority decision to uphold the taking as follows: 
%
%Any property may now be taken for the benefit of another private party, but the fallout from this decision will not be random. The beneficiaries are likely to be those citizens with disproportionate influence and power in the political process, including large corporations and development firms. As for the victims, the government now has license to transfer property from those with fewer resources to those with more. The Founders cannot have intended this perverse result.
%
%It seems that a major point of contention among the judges in the Supreme Court was whether or not these grim predictions was a realistic assessment of what the consequences of the decision would be. Surely, anyone who agrees with Justice O'Connor in her prediction of the fallout would also agree with here conclusion that it is perverse. But the majority in Kelo, in an opinion written by Justice Stevens, disagreed with her assessment, observing instead that a more restrictive view on economic takings would make it more difficult to cater to the "diverse and always evolving needs of society". 
%
%But the majority opinion also stressed that purely private takings where not permissible, and they attached great significance to the substantive assessment that the actual taking of Suzanne Kelo's home formed part of a comprehensive development plan that would not bestow special benefit on any particular group of individuals. Moreover, Justice Kennedy, in his concurring opinion, emphasised that states should not use public purpose as a pretext for interfering in property rights to the benefit of commercial actors.
%Hence the overall impression one is left with when considering Kelo in its historical and legal context is that it reflects an increasingly cautious attitude to economic takings. The precedent of virtually unlimited deference that was set in case law from the mid-to-late 19th Century was eschewed in favour of a more contextual approach where the merits and deeper purpose of the plans underlying a taking is not axiomatically beyond the scrutiny of the courts.
%
%From considering the reception of the case by the general public, we see even more clearly how Kelo in effect marks a change in the US towards greater scrutiny. 
%
%Indeed, the voices that have dominated in the aftermath of Kelo were critical of the decision and criticized the court for not offering better protection to property owners. The case also led to an a surge of academic interest in the pubic use restriction, with many arguing for further restrictions on the scope of the takings power. 
%Hence it seems that Justice O'Connor's opinion largely reflects contemporary worries about takings in the US, worries that are now also becoming increasingly relevant to how the law develops and is understood. Many states have changed their own eminent domain codes  following Kelo, to make it harder to undertake economic takings. Moreover, the federal government also banned such takings from taking place on the basis of federal takings powers.
%It will lead us astray to delve deeply into the question of what caused this change in perspective on economic takings in the US, but we can offer a few hypothesis. First, it seems that cases such as Poletown illustrates the potential danger inherent in making the power of eminent domain available to market players. In particular, the main worry that has been raised is that the pretext of public purpose may be in the process of becoming a powerful instrument for influential market actors to gain access to regulatory powers of government. As these powers has massively expanded in the post-WW2 period, so has the potential for abuse. In addition, it seems that while those who were adversely affected by eminent domain tended to be less privileged and resourceful groups of society, the takings power is now increasingly brought to bear also against members of the middle class, who are in a better position to fight it, both legally and on the political scene.
%
%While opinions differ greatly both regarding the extent of the problem and the causes of recent controversy, there is something near consensus in the US after Kelo that economic development takings raise special problems under the current system of eminent domain, and that these need to be addressed with a view to reducing tensions and restoring faith in the system. Indeed, even the majority in Kelo hint strongly at this when they say that  
%Some have argued forcefully that a strict reading of the public use requirement is the way forward, if not by strict interpretation then by an explicit ban on economic development takings.  However, it is tempting here to echo the worries expressed in Seawell, that a strict formalistic approach to legitimacy runs the risk not only of being inflexible, but also, eventually, of offering less  protection to property owners. How, then, should we reduce the risk of abuses?
%While many have focused on the question of banning economic taking, or reconsidering the public use clause, some have addressed this question from such a broader angle. In my opinion, this is the way forward. It seems, in particular, that a complete ban on economic development takings will leave a vacuum in the current economic system, which presupposes a great deal of cooperation between commercial and public interest. Particularly when it comes to economic development, the private-public partnership model has gained influence to the point that a ban on economic development takings would likely prove impossible to implement in a satisfactory manner. 
%More generally, it seems hard to address the problem of economic takings without considering the role they play in the larger economic context within which current rules and practices have developed. Based on such considerations, I believe the procedural approach to economic takings is the appropriate one. This perspective asks us to take a closer look at judicial safeguards for protecting the role of property owners in the decision-making processes that lead up to the use of eminent domain. To some extent one might approach this on the basis of existing legal principles, asking for better scrutiny of procedural aspects, or by making it easier to bring pretext claims before the courts. However, it might also require new ideas, and, in particular, the introduction of new institutions for decision-making and administration of the eminent domain process.
%
%In the next section, I will look at two concrete proposals in more detail, one concerning the decision-making step and the other concerning the calculation of compensation. 
%They will be important because they serve as starting points for the case study that is to follow, addressing mechanisms that we will return to in Chapters x and y when we look more closely at two Norwegian legal institutions that share many features with the theoretical roposals discussed in the next section.
%}

\section{Institutional proposals for increased legitimacy}\label{sec:ir}

In this subsection, I first present the Special Purpose Development Companies proposed by Lehavi and Licht.\footcite{lehavi07} I relate this proposals to theoretical approaches to the issue of compensation, before I go on to note some shortcomings and open questions that I will later address in my case study. I then go on to consider the Land Assembly Districts proposed by Heller and Hills.\footcite{heller08} I consider this proposal in light of the stated motivation, which is to design an effective mechanism of self-governance that can replace eminent domain in economic development cases. I present some unresolved questions and argue that there is a tension in the proposal between its narrow scope, imposed to prevent majority tyranny and other forms of abuse, and its broad goal of empowering local communities. 

\subsection{Special Purpose Development Companies}

The primary distinguishing feature of economic development takings is that they give the taker an opportunity to profit commercially from the development. This may even be the primary aim of the project, with the public benefiting only indirectly through potential economic and social ripple effects. Property owners facing condemnation in such circumstances might expect to take a share in the profit resulting from the use of their land. However, in many jurisdictions, including the US, the rules used to calculate compensation prevents owners from getting any share in the commercial surplus resulting from development.\footnote{See, e.g., \cite[965-966]{fennell04}.} In particular, various no-scheme rules are typically in place to ensure that compensation is based on the pre-project value of the land that is being taken.\footnote{See, e.g., \cite[81]{freilich06} The policy reasons for such rules is that they ensure that the public does not have to pay extra due to its own special want of the property. After all, this is one of the main purposes of using eminent domain in the first place; to ensure that the public does not have to pay extortionate prices for land needed for important projects. However, when the purpose of the project is itself commercial in nature, there appears to be a shortage of good policy reasons for excluding this value from consideration when compensation is calculated. This is especially true when, as in the US, compensation tends to be based on the market value of the land taken. Why should a commercial condemner's prospect of carrying out economic development with a profit be disregarded from the assessment of market value? In any fair and friendly transaction among rational agents, one would expect benefit sharing in a case like this. Yet for economic development backed up by eminent domain, the application of elimination rules ensures that all the profit goes to the developer. 

Some authors have argued that failures of compensation is at the heart of the economic takings issue and that worry over the public use restriction is in large part only a response to deeper concerns about the ``uncompensated increment'' of such takings.\footcite[See][962]{fennell04} In addition to the lack of benefit sharing, previous work has identified two further problems of compensation that also tend to become exasperated in economic development cases. First, the problem of ``subjective premium'' has been raised, pointing to the fact that property owners often value their own land higher than the market value, for personal reasons.\footcite[963]{fennell04} For instance, if a home is condemned, the homeowner will typically suffer costs not covered by market value, such as the cost of moving, including both the immediate ``objective'' logistic costs as well as more subtle costs, such as having to familiarize oneself with a new local community. Second, the problem of ``autonomy'' has been discussed, arising from the fact that an exercise of eminent domain deprives the landowner of her right to decide how to manage her property.\footnote{Discussed in \cite[966-967]{fennell04}. For a general personhood building theory of property law, see \cite{radin93}. For a general economic theory of the subjective value of independence, see \cite{benz08}.}

In \footcite{lehavi07}, the authors propose a novel approach for addressing the ``uncompensated increment'' in economic takings cases. Their proposal is based on a new kind of structure that they dub a {\it Special Purpose Development Corporation} (SPDC). The idea is that owners affected by eminent domain will be given a choice between standard pre-project market value and shares in a special company. This company will exist only to implement a specific step in the implementation of the development project: the transaction of the land-rights. The SPDC may choose either to offer their rights on an auction or else negotiate a deal with a designated developer.\footcite[1735]{lehavi07} Hence, the idea is to ensure that the owners are paid a value that reflects the post-project value of the land, but in such a way that the holdout problem is avoided. In particular, the SPDC will have a single task: to sell the land for the highest possible price within a given time frame.\footcite[1741]{lehavi07} After the sale is completed, the SPDC will divide the proceeds as dividends and be wound up.\footcite[1741]{lehavi07}

Other suggestions have taken a more static approach to compensation reform, such as proposing to give owners a fixed premium in cases of economic development, or developing mechanisms of self-assessment to ensure that compensation is based on the true value the owner attributes to his own land.\footnote{A range of static proposals have been proposed in the literature: Merrill proposes 150 \% of market value for takings that are deemed to be ``suspect'', including takings for which the nature of the public use is unclear, see \cite[90-93]{merrill86}. Krier and Serkin propose a system that provide compensation for a property's special suitability to its owner, or a system where compensation is based on the court's assessment of post-project value, see \cite[865-873]{krier04}. Fennell proposes a system of self-evaluation of property for takings purposes with tax-breaks given to those who value their property close to market value (to avoid overestimation), see \cite[995-996]{fennell04}. Bell and Parchomovsky also propose self-evaluation, but rely on a different mechanism to prevent overestimation; tax liability is based on the self-reported value and no property can be sold by its owner for less than his reported value, see \cite[890-900]{bell07}.} Compared to such proposals, the idea of SPDCs is more sophisticated and should be looked at in more depth. 

The conceptual premise for the proposal is that takings for economic development can be seen as compulsory incorporation, a pooling of resources useful in overcoming market failures.\footcite[1732-1733]{lehavi07} Just as the corporation is formed to consolidate assets in order to facilitate effective management, so is eminent domain used to assemble property rights in order to facilitate efficient organization of development. According to Lehavi and Licht, this also provides a viable approach to problems of ``opportunistic behavior''; hierarchical governance after assembly ensures that order and unity can be regained even if interests in the land are distributed among a large and heterogeneous group of potentially mischievous shareholders.\footcite[1733]{lehavi07} In the words of Lehavi and Licht:

\begin{quote}
The exercise of eminent domain powers thus resembles an incorporation by the government of all landowners with a view to brining all the critical assets under hierarchical governance. Establishing a corporation for this purpose and transferring land parcels to it thus would be merely a procedural manifestation of the substantive economic reality that already takes place in eminent domain cases.
\end{quote}

As soon as we look at the rationale behind economic development takings in this way, any remnant of good policy reasons for ensuring that the developer gets all the profit seems to disappear. Rather, we are led to consider compensation as an issue entirely separate from the exercise of the takings power. After the land has been reorganized by eminent domain and an SPDC has been formed, the land rights might as well be sold {\it freely} to a developer. In this way, the land will be sold for a price that is closer to an actual market value, on the market where the land is destined for development.\footcite[1735-1736]{lehavi07} More generally, the SPDC becomes an aid that the government can use to create more favorable market conditions for transferring land that has commercial potential in its public use. Due to the compulsory pooling of resources, no owner can exercise monopoly power by holding out, but due to decoupling of compensation from assembly, the owners can now negotiate with potential developers for a share of the resulting profit. Moreover, the fact that the SPDC offers its rights on an actual market can also help ensure that more information become available regarding the true economic value of the development, something that may in turn help ensure that only the good projects will be successful in acquiring land. Hence, according to Lehavi and Licht, an additional positive effect of SPDCs is that developers and governments will shun away from using the eminent domain power to benefit projects that are not truly welfare-enhancing.\footcite[1735-1736]{lehavi07}

In addition to these substantive consequences, the SPDC-proposal also stands out because it has a significant institutional component, pointing to its potential for restoring procedural legitimacy as well as substantive fairness. Lehavi and Licht discuss corporate governance issues at some length, but without committing themselves to definite answers about how the operations of the SPDC should be organized.\footcite[1040-1048]{lehavi07} Indeed, while their proposal is perhaps most interesting because of its procedural aspects, it also appears to be rather preliminary in this regard. The main idea is to let the SPDC structure piggyback on existing corporative structures, particularly those developed for securitization of assets.\footnote{See generally \cite{schwarcz94}. For an up-to-date overview, targeting special challenges that became apparent during the 2008 financial crisis, see \cite{schwarcz13}.} The basic idea is that the corporate structure should be insulated from the original landowners to the greatest possible extent; it should have a narrow scope, it should be managed by neutral administrators, and it should entrust a third party with its voting rights.\footcite[1742]{lehavi07} This is meant to prevent failures of governance within the SPDC itself, making it harder for majority shareholders and self-interested managers to co-opt the process. For instance, if a possible developer already holds a majority of the shares in an SPDC, this structure would prevent him from using this position to acquire the remaining land on favorable terms. 

Lehavi and Licht observe that under US law, the government would often be required to make shares in an SPDC available to the landowners as a public offering.\footcite[1745]{lehavi07} Lehavi and Licht deem this to be desirable, arguing that full disclosure will provide owners with a better basis on which to decide whether or not to accept SPDC shares in place of pre-project market value. It will also facilitate trading in such shares, so that they will become more liquid and therefore, presumably, more valuable.\footcite[1746]{lehavi07} 

Lehavi and Licht's proposal is interesting, but I think a fundamental objection can be raised against it. In particular, it seems that their governance model more or less completely alienate property owners from the decision-making process after SPDC formation. Limiting the participation of owners is to a large extent an explicit aim, since governance by experts is held to increase the chances of ensuring good governance. But is expert rule really the answer?

It seems that from the owners' point of view, Lehavi and Licht's proposals for governance reduces the SPDC to a mechanism whereby they can acquire certain financial entitlements. These may exceed those that would follow from standard compensation rules, but they do not directly empower owners vis-{\'a}-vis developers and the government. Instead, a largely independent structure will be introduced. It is this new organizational structure, rather than the owners, that will now become an important actor in the eminent domain process. In principle, it is meant to represent owners, but to what extent can it do so effectively? After all, it is specifically intended to operate as neutral player, charged with maximizing the price, nothing more. Hence, it appears that the SPDC will not be able to give owners an arena to negotiate on the basis of the personal and social importance they attribute to their land rights. How the problem of ``autonomy'' is addressed by the proposal is therefore hard to see and the ``subjective premium'' also appears to be in danger, unless it can be objectively quantified and covered by the surplus from a voluntary sale. But if such quantification is possible, then why not simply tell the appraiser to award some premium under standard compensation rules?

More generally, it seems to me that while all three categories of ``uncompensated increments'' are interesting to study from a financial viewpoint, severe doubts can be raised regarding the feasibility of addressing the subjective aspects of this as a question of compensation. It may be that issues related to ``subjective premium'' and ``autonomy'' are seen as public use issues for good reason; they are hard to quantify otherwise. Moreover, attempting to do so might do more harm than good. On the one hand, it might skew the political process, since owners that have been ``bought off'' don't object to ill-advised development projects, as long as they generate financial revenue. But what about projects that are undesirable for other reasons, for instance because they completely change the character of a neighborhood, or because they are harmful to the environment? On the other hand, the very idea that money can compensate for the subjective importance of property and autonomy can itself prove offensive. At least it seems likely that it would often come to be seen as inadequate and inefficient.\footnote{For more detailed criticism of the compensation approach to the public use issue, see \cite{garnett06}.} Moreover, an owner that is compelled to give up his home after an inclusive process where the public interest has been debated and clearly communicated is likely to feel like he incurs less costs related both to his subjective premium and his autonomy. Hence, the lack of participation in the decision-making process can in itself increase the uncompensated loss. Clearly, no externally managed ``bargain-oriented'' SPDC will be able to resolve this problem. Of course, some ``objective'' elements of, such as relocation costs or cost for juridical assistance, can still be addressed under the banner of compensation. But in most jurisdictions, they already are.\footnote{See, e.g., \cite[121-126]{garnett06}.} For more subtle aspects, the aftermath of {\it Kelo} itself can serve as an illustration of how a compensatory approach is unsatisfactory:

After the case, Suzanne Kelo remained defiant, until she eventually decided to settle in 2006, for an offer of \$ 442 155, more than \$ 319 000 above the appraised value.\footcite[1709]{lehavi07} Apparently, the other owners affected by the same taking were not particularly pleased, arguing that recalcitrant owners were actually rewarded for holding out.\footcite[1709]{lehavi07} On the other hand, there is no indication that Suzanne Kelo was not genuine in her opposition to the taking. Indeed, after the long struggle she had taken part in, it is easy to imagine that financial compensation, if it was to be an effective remedy at all, would have to be very high. Even after she had settled, Kelo apparently toured the country speaking out against economic takings. This, too, is a statement to the inadequacy of a purely financial approach to legitimacy. 

I conclude that SPDCs have serious shortcoming with regards to the subjective aspects of undercompensation, aspects that can only be addressed if the focus turns towards participation. However, SPDCs do seem promising when it comes to profit-sharing. This, after all, is what the structure is specifically aiming to achieve. In addition, I agree that SPDCs will likely have a positive effect on the other actors in the eminent domain process. In particular, I agree with Lehavi and Licht that greater openness is likely to result, revealing the true merits of development projects, at least in so far as these are translatable into financial terms. The fact that developers must negotiate with an SPDC who can threaten to make the land available an an open auction will likely deter developers and government from pursuing fiscally inefficient projects. Hence, the risk that governments will subsidized such projects by giving them cheap access to land will also be reduced. In addition, the presence of a third voice, speaking on behalf of owners, is likely to help achieve a better balance of power in development takings. 

Even if the individual landowners do not have a voice in this process, the fact that the landowners are better represented as a group is then still likely to have a positive effect on legitimacy. On the other hand, as long as the power of the SPDC is limited to choosing the best offer and negotiating over price, it seems that SPDCs will easily end up being dominated by developers and government. This is a particular concern in cases when competition fails to arise after SPDC formation. To ensure that there are other interested parties, in particular, sems like an important precondition for the proposal to work in practice. In this regard, it is important to realize that a lack of interest from other developers may not be due to the superiority of the original developer's plans. It might rather be due to the fact that the scope of the assembly giving rise to the SPDC is so defined as to make alternatives unfeasible. The danger of abuse in this regard seems significant, particularly when developers themselves participate in coming up with the plans that give rise to SPDC formation. 

Moreover, as long as owners remain marginalized in the planning phase, it is easy to imagine situations where the plan itself will be formulated in such a way that only one developer is in a position to successfully implement it. A simple example would be if a prospective developer already owns some of the land that is critical to the plan, and is able to ensure that this land is kept out of the scope of the SPDC. Clearly, if SPDCs are to operate effectively, such instances of manipulation need to be avoided, suggesting that the proposal as it stands needs to be fleshed out in greater detail.

The problems addressed here both seem to point to the fact that the SPDCs, while more flexible than other suggestions, are still too static to achieve many of their objectives. In particular, to arrive at genuine market conditions for assessing post-project value, there is still a need for changes in the dynamics of the planning process underlying the taking. Moreover, to ensure legitimacy, there is a need for a mechanism that goes beyond expert bargaining and provides owners with better access to the decisionmaking process. In the next subsection, I will consider a proposal that aims to address this, by proposing a framework for self-governance. 

\subsection{Land assembly districts}

In a recent article, Heller and Hills propose a new institutional framework for carrying out land assembly for economic development. Interestingly, it is meant to replace eminent domain altogether. The goal is to ensure democratic legitimacy while also creating a template for collective decision-making that will prevent inefficient gridlock and holdouts. The core idea is to introduce {\it Land Assembly Districts} (LADs), institutions that will enable property owners in a specific area to make a collective decision about whether or not to sell the land to a developer or a municipality.\footcite[1469-1470]{heller08} Anyone can propose and promote the formation of a LAD, but both the official planning authorities and the owners themselves must consent before it is formed.\footcite[1488-1489]{heller08} Clearly, some form of collective action mechanism is required to allow the owners to make such a decision. Hiller and Hill suggest that voting under the majority rule will be adequate in this regard, at least in most cases.\footnote{See \cite[1496]{heller08}. However, when many of the owners are non-residents who only see their land as an investment, Heller and Hills note that it might be necessary to consider more complicated voting procedure, for instance by requiring separate majorities from different groups of owners. See \cite[1523-1524]{heller08}.} How to allocate voting rights in the LAD is another issue that require careful consideration, but Heller and Hills land on the proposal that they should in principle be given to owners in proportion to their share in the land belonging to the LAD.\footnote{See \cite[1492]{heller08}. For a discussion of the constitutional one-person-one-vote principle and a more detailed argument in favor of the property-based proposal, see \cite[1503-1507]{heller08}.} Owners can opt out of the LAD, but in this case eminent domain can be used to transfer the land to the LAD using a conventional eminent domain procedure.\footcite[1496]{heller08}

Heller and Hills envision an important role for governmental planning agencies in approving, overseeing and facilitating the LAD process. Their role will be most important early on, in approving and spelling out the parameters within which the LAD is called to function.\footcite[1489-1491]{heller08} Hence, it appears to be assumed that the planning authorities will define the scope of the LAD by specifying the nature of the development it can pursue. A possible challenge that arises, and which Heller and Hills do not address at any length, is that the scope of the LAD needs to be broad enough to allow for meaningful competition and negotiation after LAD formation. At the same time, however, there will probably be a push, both by governments and initiating developers, to ensure that the scope is defined narrowly enough to give confidence that rezoning permissions will not be denied at a later stage. Another potential challenge is that the planning authorities might have an incentive to refuse granting approval for LAD formation, since it effectively entails that they give up the power of eminent domain for the land in question. For this reason, Heller and Hills propose that a procedure of judicial review should exist whereby a decision to deny approval for LAD formation can be scrutinized.\footcite[1490]{heller08} 

After the formation of the LAD, the government will have a less important position, but the planning authorities will still occupy an important facilitating role. Heller and Hills envision a system of public hearings, possibly organized by the planning authorities, where potential developers meet with owners and other interested parties to discuss  plans for development.\footnote{See \cite[1490-1491]{heller08}.} In this process, it is assumed that also other voices will be represented, such as owners of adjoining land, who can use this opportunity to express objections against the project. Their role in the process is not clarified, but presumably the planning authorities would be able to offer this group some protection, if not in relation to the LADs own operations, then later in relation to the decision whether or not to grant the licenses needed for the development project.

Importantly, if the owners do not agree to forming a LAD, or if they refuse to sell to any developer, the government will be precluded from using eminent domain against them to assemble the land.\footcite[1491]{heller08} This is the crucial novel idea that sets the suggestion apart from other proposals for institutional reform that have appeared after {\it Kelo}. LADs will not only ensure that the owners get to bargain with the developers over compensation, it will also give them an opportunity to refuse any development to go ahead, if they should so decide. Hence, the proposal shifts the balance of power in economic development cases, giving owners a greater role also in preparing the decision whether or not to develop, and on what terms. In my opinion, this makes the proposal stand out as particularly interesting in the recent literature on economic takings. It is the first concrete suggestion that addresses the democratic deficit in a dynamic, procedural manner, without failing to recognize that the danger of holdouts is real and that institutions are needed to avoid it, also in economic development cases.

There are some problems with the model, however. Kelly points out that the basic mechanism of majority voting is itself imperfect, and can lead both to overassembly and underassembly, depending on the circumstances.\footcite{kelly09} He points out, in particular, that if different owners value their property differently, majority voting will tend to disfavor those with the most extreme viewpoints, either in favor of, or against, assembly. If these viewpoints are assumed to be non-strategic and genuine reflections of the welfare associated with the land, the result can be inefficiency. In a nutshell, the problem is that a majority can often be found that does not take due account of minority interests. For instance, if some owners are planning alternative development, leading them to attribute a high {\it hope}-value to their land, they can safely be ignored as long as the majority have no such plans. This could become particularly bad in cases when the alternative development itself is more socially desirable than the development that will benefit from assembly. The role of the LAD in such cases will not improve the quality of the decision to develop, since it pushes the decision-making process into a track where those interests that {\it should} prevail are voiced only by a marginalized minority inside the new institution.\footnote{Of course, one might imaging these landowners opting out of the LAD, or pursuing their own interests independently of it. However, they are then unlikely to be better off than they would be in a no-LAD regime. In fact, it is easy to imagine that they could come to be further marginalized, since the existence of the LAD, acting ``on behalf of the owners'', might detract from any dissenting voices on the owner-side.}

More generally, the lack of clarity regarding the role of LADs in the planning process is a problem. As it stands, the proposal leaves it uncertain how LADs will affect the decision-making process regarding development. But the ideal is clearly stated: LADs should help to establish self-governance in land assembly cases. In particular, Heller and Hills argue that LADs should have ``broad discretion to choose any proposal to redevelop the neighborhood -- or reject all such proposals''.\footcite[See][1496]{heller08} As they put it, two of the main goals of LAD formation is to ensure `` preservation of the sense of individual autonomy implicit in the right of private property and preservation of the larger community's right to self-government''.\footcite[See][1498]{heller08} Unfortunately, these ideals are somewhat at odds with the concrete rules that Heller and Hills propose, particularly those aiming to ensure good governance of the LAD itself. 

In relation to the governance issue, Heller and Hills echo many of the ``corporate governance''-ideas that also feature heavily in Lehavi and Licht's proposal. Indeed, in direct contrast to their comments about ``broad discretion'' and ``self-governance'', Hiller and Hills also state that ``LADs exist for a single narrow purpose -- to consider whether to sell a neighborhood''.\footcite[See][1500]{heller08} This is a good thing, according to Heller and Hills, since it provides a safe-guard against mismanagement, serving to prevent LADs from becoming battle grounds where different groups attempt to co-opt the community voice to further their own interests. As Heller and Hills puts it, the narrow scope of LADs will ensure that ``all differences of interest based on the constituents' different activities and investments, therefore, merge into the single question: is the price offered by the assembler sufficient to induce the constituents to sell?''.\footcite[1500]{heller08}

But this means that there is an internal tension in the LAD proposal, between the broad goal of self-governance on the one hand and the fear of neighborhood bickering and majority tyranny on the other. It is hard to see, in particular, how the idea of LADs with a ``narrow purpose'' is compatible with a scenario where the LAD has ``broad discretion'' to choose between competing proposals for development. If such discretion may indeed be exercised, what is to prevent special interest groups among the landowners from promoting development projects that will be particularly favorable to them, rather than to the landowners as a group? And what is to prevent landowners from making behind-the-scene deals with favored developers at the expense of their neighbors? It seems like a great challenge to come up with rules that prevent mechanisms of this kind, without also constraining the landowners so much that meaningful ``self-governance' becomes an impossibility. If a LAD is obliged to only look at the price, this might prevent abuse. But it will not give owners broad discretion to choose among development proposal. Effectively, it will render LADs as little more than a variant of SPDCs, where the owners are awarded an extra bargaining-chip: the option to refuse all offers. 

In my view, such a restriction on the operations of LADs is not desirable. It is easy to imagine cases where competing proposals, perhaps emerging from within the community of owners themselves, will emerge in response to the formation of a LAD. Such proposals may involve novel solutions that are superior to the original development plans, in which case it is hard to see any good reason why they should not be taken into account, even if they are less commercially attractive. In particular, the formation of a LAD and the competition for development that ensues creates an opportunity for tapping into a greater pool of ideas for redevelopment, ideas which may then also be rooted more firmly in the local community. Surely, getting such proposals to the table would be desirable and it would take us to the heart of self-governance. At the same time, it is easy to acknowledge that problematic situations may arise, for instance if a majority forms in favor of a scheme that involves razing only the homes of the minority, maybe on the rationale that these are ``more blighted''. That would likely give rise to accusations of unfair play, which may or may not be warranted. But irrespectively of this, an alternative project of this kind might well be a better use of the land in question, also from the point of view of the public. Hence, it would seem that the planning authorities would be obliged to give it some serious consideration. Then, however, the LAD has truly become an arena for a new kind of power play among different interest, and a potential vehicle of force for whomever secures support from a majority of owners within the district.

In their proposal, Heller and Hills are aware of this potential problem, which they propose to resolve by strict regulation. In particular, they argue that ``LAD-enabling legislation should require especially stringent disclosure requirements and bar any landowner from voting in a LAD if that landowner has any affiliation with the assembler''.\footcite{heller08} But this raises further questions. For one, what is meant by ``affiliation'' here? Say that a land owner happens to own shares in some of the companies proposing development. Should he then be barred from voting? If so, should he be barred from voting on all proposals, or just those involving companies in which he is a shareholder? If the answer is yes, how would this be justified? Would it not be easy to construe such a rule as discrimination against landowners who happen to own shares in development companies? On the other hand, if the landowner in question is allowed to vote on all other proposals, would it not be natural to suspect that his vote is biased against assembly that would benefit a competing company? Or what about the case when some of the land owners are employed by some of the development companies? Should such owners be barred from voting on proposals that could benefit their employers? This seems quite unfair as a general rule, especially if a low-level employment relationship has such a dramatic effect. But in some cases even low-level ties could play a decisive factor. This might happen, for instance, if an important local employer proposes development in a neighborhood where it has a large number of employees.

Of course, the most pressing issue that arises is the following: who exactly should be empowered to make the determination of when an affiliation is such that an owner should be deprived of his voting rights? Heller and Hills give no answer, but it is easy to imagine that whoever is given this task in the first instance, the courts would soon enough be asked to consider the question. At this point, the circle has in some sense closed in on the proposal. In particular, one might ask: why is it easier to determine if someone can be deprived of his voting rights due to an ``affiliation'', than it is to determine if someone can be deprived of his land due to some planned ``public use''?

In any event, to come up with a set of rules ensuring that LADs can deliver both self-governance and good governance largely remains an open problem. This is acknowledged by Hiller and Hills themselves, who point out that further work is needed and that only a limited assessment of their proposal can be made in the absence of empirical data. Later in the thesis, I will shed light on this challenge when I consider the Norwegian rules relating to land consolidation, showing how these can be looked at as a highly developed institutional embedding of many of the central ideas of LADs. The assessment of how they function in cases of economic development, and how they are increasingly used as an alternative to expropriation in cases of hydro-power development, will allow me to shed further light on the issues that are left open by Heller and Hills' important article.

\section{Conclusion}

In this chapter, I have given a more in-depth presentation of economic development takings in law. I began by noting that the issue is particularly pressing for land users that are not regarded as bringing about economic growth. Hence, I argued that the issue is closely related to that of land grabbing, which is currently receiving much attention, both academic and political. Under the social function understanding of property there is in principle no difference between protecting property rights arising from formal title and property rights arising from use. That said, special issues arise in the latter case, not least because it is unclear how the law should deal with rights resulting from cultural practices that western property regimes are not designed to handle. In addition, I noted that special issues related to food security and poverty arise with particular urgency in relation to land grabbing.

Moreover, the nature of my case study makes it more natural for me to focus on traditional western systems of property law. Hence, I went on to shed more light on discuss how economic development takings are dealt with in such legal systems, focusing on Europe and the US respectively. For the case of Europe, this assessment was made more difficult by the fact that the category is not an established part of legal discourse. However, by looking to England and Germany as concrete examples, I noted that such cases do arise and that they are increasingly seen as controversial. I also noted that there is a contrast between how England and Germany approach such cases, as well as how they approach property more generally. Germany, in particular, goes further in explicitly recognizing the social functions of property, by actively looking to social and political values when assessing whether interferences are legitimate. In England, similar reasoning is at most applied indirectly, as takings are approached almost entirely as an issue of administrative law. 

I then went on to consider the property protection offered by P1(1) of the ECHR, and how it is applied by the Court in Strasbourg. I zoomed in on those aspect that I believe to be the most relevant for economic development takings. While I noted that this category has yet to be discussed by the ECtHR, I argued that a recent shift in the Court's property adjudication is suggestive of the fact that it would likely approach such cases similarly to how Justice O'Connor approached {\it Kelo}. In particular, I noted how the Court has recently adopted a stricter standard of assessment. This standard, I argued, is characterized primarily by increased sensitivity to systemic imbalances causing alleged P1(1) violations. Hence, to regard economic development takings as a special category appears to fit well with recent jurisprudential developments at the Court in Strasbourg.

I went on to consider US sources on economic development takings, noting that the issue has receive an extraordinary amount of attention in recent years. I adopted an historical approach to the material, by tracing the case law surrounding the public use restriction in the fifth amendment to the US constitution, which was much debated even before the specific issue of economic development takings rose to prominence. I focused particularly on case law developed by state courts, and I argued that it shows great sensitivity to the need for contextual assessment. Indeed, I argued that originally, many state courts implicitly adopted a social function view on property when they assessed such cases.

I then looked at the history of Supreme Court adjudication of public use cases. I noted that the doctrine of deference was developed early on, but that it was initially directed mainly at state courts. In fact, I showed that the Supreme Court itself noted with approval the contextual and in-depth approach these courts would rely on when dealing with this issue.

The shift, I argued, came with \textcite{berman54}, in which the Supreme Court adopted a deferential doctrine that was directed specifically at the {\it state legislature}. This, I argued, was quite a dramatic departure from the Court's previous attitude towards {\it state} takings. In fact, it was almost entirely backed up by precedent set in cases when {\it federal} takings had been ordered by Congress. I went on to consider the fallout of \textcite{berman84} at state level, which culminated with the infamous {\it Poletown} case. This case prompted wide-spread accusations of eminent domain abuse and thus it set the stage for {\it Kelo}.

After completing the historical overview, I went on to consider the literature after {\it Kelo}. I expressed particular support for those responses that focus on the need for {\it institutional} reform, to address precisely those dangers that Justice O'Connor pointed to in her minority opinion. As a shorthand, I proposed referring to the mechanisms she identified as the {\it democratic deficit} of economic development takings. 
% I zoomed in on two of those in particular, the Special Purpose Development Companies proposed by Lehavi and Licht, and the Land Assembly Districts suggested by Heller and Hills. I gave an in-depth presentation of these two proposals, pointing out strengths and weaknesses. 
%%In coming chapters, I will refer back to this as I consider similar institutions and mechanisms that are currently operating in Norwegian law relating to hydropower development.

I then gave a thorough presentation of two recent reform suggestions that might help address this deficit. Both are institutional in nature, based on setting up formally recognized coalitions of land owners that can act as a counterweight to the disproportional power of commercial beneficiaries. The first suggestion, by Lehavi and Licht, limited its attention to compensation, recognizing the need for a system whereby the land owners are compensated based on post-project value.  This idea in itself represents a fairly dramatic break with the currently dominant doctrine in takings law, where compensation is almost always, and in almost all jurisdictions, based on the pre-project value of the land, or the {\it value to the owner}.

In Chapter \ref{chap:5}, I will return to this principle in more depth, looking at how it developed internationally, and in Norway. I will also look at how it has now been abandoned in Norwegian law, for some case types involving hydro-power development. I relate this to the special role played by the appraisal courts in Norway. The local grounding of these courts, involving lay people sitting as court appointed appraisers, allows the law to be applied in a way that adapts to the concrete circumstance in a way that may enhance the perceived fairness and legitimacy of the taking. At the same time, however, the judicial procedure, with a (limited) possibility for appeal, puts in place safeguards against abuse.

The second suggestion I looked at in depth, proposed by Heller and Hills, focused not on compensation, but on the decision-making process leading to an economic development project being implemented. This proposal is based on the idea that local communities should be entitled to greater self-governance in such cases. At the same time, it recognizes the need for a mechanism to avoid inefficient and socially harmful gridlock due to holdouts among unwilling owners. Instead of eminent domain, however, a different mechanism is proposed: a land assembly district (LAD). 

This is also a new class of institutions, and I pointed out some problems and seeming inconsistencies in the proposal, regarding the exact role they will have in the planning process. I argued that while the risk of abuse and failure increases with the level of participation, so does the positive effect on legitimacy. I concluded that to reduce the democratic deficit in economic development cases, a wide power of participation must be granted to the land owners and their (immediate) communities. This is needed, in particular, to restore balance in the relationship between owners and others directly connected with the land, the planning authorities, and the commercial actors interested in development for profit. The question that is as of yet unresolved is how to organize such participation in a way that avoids obvious pitfalls, such as administrative inefficiency and tyranny by majorities or elites that gain control of the local agenda.

In Chapter \ref{chap:6}, I will shed light on this question by considering the Norwegian institution of land consolidation, which has very long traditions. It is a flexible frameworks which includes, among other things, a a template for establishing institutions that can function as a LAD. I will focus on how land consolidation functions in cases of economic development that would otherwise likely be pursued by eminent domain. Norwegian hydro-power development will again be in focus, but I will also discuss planning law and development more generally, as the Norwegian government is now considering making consolidation, traditionally a rural institution, a primary mechanism for land development also in urban areas.

Before I delve into this, I will use the next chapter to present an overview of Norwegian hydropower and the role of waterfalls as private property.

\printbibliography

\end{document}

