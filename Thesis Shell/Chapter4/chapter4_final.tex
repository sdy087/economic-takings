\chapter{Just compensation}\label{chap:5}

\section{Introduction}\label{sec:into5}

In this Chapter, I consider the question of compensation for waterfalls in more depth. The main issue that arises is whether or not owners should be compensated for the loss of a commercial hydropower potential. If so, the compensation payments can be very large, so large that expropriation will no longer be a feasible option. Traditionally, however, no such compensation was awarded and the amounts paid to owners were negligible. In fact, owners would often have been left with nothing at all, were it not for the fact that a theoretical compensation formula was developed which avoided this outcome, by ensuring some degree of benefit sharing.

The question of whether or not to base compensation on the loss of a commercial hydropower potential is closely related to the so-called ``no scheme'' principle, according to which compensation is to be based on the value of the property such as it would have been if the expropriation scheme had not been authorised. If one takes the view that hydropower development is the prerogative of the party that obtained such an authorisation, it follows from the principle that no compensation is payable to the owners of the waterfalls, at least not for the hydropower potential. The value of hydropower, in particular, is then regarded as being due to the scheme, not due to the natural resource that the waterfall represents. This perspective was implicitly adopted in Norwegian law from the early 20th to the early 21st century, when it began to loose ground due to the liberalization of the energy sector.

The structure of this chapter is as follows: In Section \ref{sec:nsp}, I provide a comparative and theoretical context for the case study that is to follow. I do so by discussing the origin and current status of the no-scheme principle in UK law. This facilitates a broader perspective on the data presented on Norwegian law in subsequent sections.  It also allows me to make a more general point, namely that the need to distinguish between commercial/private and public values inherent in a development project arises with great force when one attempts to apply the no-scheme principle in the context of an economic development taking. I identify the lack of a well-developed framework for making such a distinction as one of the main problems associated with such takings. The main worry is that the principle, when applied to commercial values associated with a development scheme, results in {\it discrimination}. Some categories of owners are entitled to commercial benefits that other categories of owners are effectively deprived of by an application of the no-scheme principle.

In Section \ref{sec:norcom}, I go on to present Norwegian compensation law, with a focus on various manifestations of the no-scheme principle. I also present the special judicial procedure used to award compensation following expropriation. I pay particular attention to the fact that it relies on the use of lay appraisers, who also have considerable influence over the application of the law in such cases. This system, I argue, is potentially very flexible, allowing compensation awards to be based on broad and contextual fairness considerations, rather than static application of special rules. It means, in particular, that the no-scheme principle is not applied without exception, even if it does have status as a general principle. I finish this section by noting that the traditional system has been somewhat undermined since WW2, following legislation specifically aimed at reducing compensation payments and narrowing the room for lay discretion in appraisal disputes.

In Section \ref{sec:nathp}, I move on to consider compensation for waterfalls. The first thing I note is that the no-scheme principle was not traditionally applied, since it would have led to little or no compensation for owners during the monopoly era. Instead, a theoretical method was used, based on the notion of natural horsepower. This method was meant to give owners a share of the benefit in hydropower development and was developed by the appraisal courts early in the 20th century. It was modelled on the market for waterfalls that had existed prior to monopolization, but as the years went by, the method became farther and farther removed from the physical and economic realities of the hydropower industry. Increasingly, it resulted in no more than a symbolic form of benefit sharing with owners.

Following liberalization the traditional method has been abandoned for several categories of cases, a development that I discuss in Section \ref{sec:fa}. The crucial condition for applying the new method, based on market-value assessment, is that an alternative development scheme would have been ``foreseeable'' in the absence of the expropriation project. How to interpret the meaning of ``foreseeable'', and how to determine the scope of the ``expropriation project'', are crucial issues currently being worked out in Norwegian case law. I link the reform in this area with the institutional framework surrounding appraisal disputes. I note, in particular, that the natural horsepower method was first abandoned by the appraisal courts, with the lay appraisers assuming a leading role. 

Unfortunately, the market-based method also raises problems, the most severe of which are related to the scope of the no-scheme principle. The lack of clarity about its scope and implication has created a situation where it appears that if owners are lucky, or employ skilled arguers, they can collect a very substantial sum of money with little or no effort and with no social responsibilities attached. On the other hand, if they are unlucky, they are forced to give up what is often the most valuable asset of their local community for nothing but a symbolic payment. I conclude by arguing that a much better approach would be to try and get owners involved in sustainable hydropower in a way that can remove the need for expropriation altogether. 

This sets the stage for the last chapter of this thesis, where I return to the question of how to replace expropriation by mechanisms of participatory democracy, referring back also to the discussion in Chapter \ref{chap:1}.

%As development is now organized as a commercial pursuit, this should in principle be possible, since the owners {\it do} have an incentive to get involved, also in cases when the public dictate the set of possible terms through strict regulation. In practice, however, what is needed is a mechanism for organizing such owner-involvement. This mechanism will undoubtedly also need to be endowed with powers of coercion if it is to be effective.
%
%The Supreme Court struck down their judgement on a technicality, but refused to reject the principle that lay people were free to adopt a new method in cases when the traditional method would not adequately reflect the value of ``foreseeable'' use. I argue that this shows the strength of a long tradition of respecting the discretion of lay people in appraisement disputes. Many legal scholars, in particular, had previously regarded the natural horsepower method as a {\it rule of law}, set by precedent.
%
%The method has not been abandoned as a matter of principle, however. As made clear recently by the Supreme Court, it is still to be applied in cases when a calculation based on ``foreseeable use'' does not lead to higher compensation payments. The crucial question becomes what exactly is meant by this notion. I address this in some depth, by pointing to how Norwegian law in general is  marked by a tendency to disregard any use that is not sanctioned by public plans, including in cases when these plans themselves provide the rationale for expropriation. This appears to be contrary to the no-scheme principle, demonstrating more generally that only ``one half'' of the principle tend to apply in a Norwegian setting. In so far as the principle precludes giving the owner a share of the expropriation surplus, it is applied, but in so far as it entitles him to compensation based on a future use that is rendered unforeseeable by the planning underlying the expropriation license, it is not.
%
%There are some exceptions to this, however, and the Supreme Court has indicated that one of them applies to hydropower cases. At the same time, however, it has been stressed that even if the expropriation plans themselves are not binding for the compensation assessment, the ``public rationale'' underlying these plans must be taken into consideration when awarding compensation. In effect, this means that compensation is not offered for alternative uses in so far as the project proposed by the expropriating party is superior and could not be undertaken by the current owner. In effect, it seems that a partly {\it subjective} standard is introduced into compensation law, whereby local owners are denied compensation for a commercial value that is deemed to be such that it is only realizable by the expropriating party.
%
%The Supreme Court has not been entirely consistent about the scope and exact content of the ``public rationale'' principle, however,   and the issue is still very much contested in Norwegian courts. In Section \ref{sec:ko}, I illustrate the current unclear state of the law by contrasting two recent Supreme Court cases. In the first, the court embraced an objective version of the ``public rationale'' principle by holding that as the expropriating party's project resulted in more public benefits, compensation could be based on the premise that the owners' foreseeable use of the waterfalls was to cooperate with the large energy company in realizing the plans, to take their share of its commercial potential. 
%
%In {\it Otra II} on the other hand, the Court held that this should not be the conclusion in so far as cooperation was deemed to be ``impractical'', following a concrete assessment of the facts. It seems quite clear that the notion of ``impracticality'', as it was used here, serves to introduce a subjective assessment standard, contrary to what otherwise dictated by Norwegian compensation law. 
%I go on to consider the merits of {\it Otra II} against human rights law, anticipating also the outcome of the appeal currently lodged with the ECtHR in Strasbourg.
%
%Finally, I conclude that the case law on compensation demonstrates the intrinsic inadequacy of a narrow perspective on takings for profit. It seems clear, in particular, that all of the approaches currently in use to calculate compensation for waterfalls leave great room for bickering, manipulation and long-winded court battles. Moreover, the factual premise for the calculation is typically extremely uncertain, meaning that the whole procedure appears as something of a gamble, for both owners and developers. Hence, the developers favour the use of the natural horsepower method, which is completely removed from the reality of hydropower, but deliver predictably low compensation payments that will not prove too damaging to the profit-margin of the development company. On the other hand, owners have an incentive to push for compensation mechanisms that will allow them to collect the entire financial potential of hydropower development without actually investing any effort in planning or administerting such development, and without subjecting themselves to any of the risks involved. 

\section{The ``no scheme'' principle}\label{sec:nsp}

In most jurisdictions, a fundamental principle relating to compensation following expropriation is that compensation should be calculated without taking into account changes in the property's value that are due to the expropriation, or the scheme underlying it. In short, compensation should be based on the owner's loss, not the taker's gain. In a recent Law Commission consultation paper, this principle is referred to as the \emph{no-scheme} rule, a terminology I will also adopt here, noting that while the exact details of the rule might differ between jurisdictions, the underlying principle appears to play a crucial role in both civil and common law traditions for regulating compensation following expropriation.\footnote{I am not aware of a single jurisdiction that does not include some rule corresponding to (aspects of) the no-scheme principle. I mention that in addition to the jurisdictions discussed in this section, no-scheme rules are also found in pure civil law jurisdictions like Germany and the Netherlands, see \cite[5,21]{sluysmans14}.}

While the no-scheme principle is easy enough to comprehend when it is stated in general terms, it raises many difficult questions when it is to be applied in concrete cases. What the rule asks of the valuers, in particular, is quite daunting; they are forced to consider a counterfactual ``no-scheme world'', and they must calculate the value of the property based on the workings of such an imaginary world. The crucial question that arises, of course, is the question of what exactly this world should be taken to look like.

In the first instance, it might be tempting to state simply that this is a ``question of fact for the arbitrator in each case'', as expressed by the Privy Council in \emph{Fraser}, a Canadian case from 1917.\footnote{\cite[194]{fraser17}.} However, as the history of the no-scheme rule has shown, this point of view is not tenable.\footnote{For a history of the rule in UK law, clearly illustrating the difficulty in interpreting it and applying it to concrete cases, I point to Appendix D of \cite{lawcom03}. See also \cite{lawcom01}.}  The problem is that the nature of the no-scheme world cannot be determined without making a vast range of assumptions, many of which appear to depend on how one understands the law. The challenges that arise were discussed in great detail by Lord Nicholls in the recent case of \emph{Waters}. He described the task as ``daunting'', noting also that some of the more recent statutory provisions ``defy ready comprehension''.\footnote{\cite[19]{waters04}.}

\noo{
\begin{quote}
The extreme complexity of the issues that I have had to consider, the
uncertainty in the law, the obscurity of the statutory provisions, and
the difficulties of looking back over a long period of time in order to
decide what would have happened in the no-scheme world
demonstrate, in my view, that legislation is badly needed in order to
produce a simpler and clearer compensation regime. I believe that
fairness, both to claimants and to acquiring authorities, requires
this
\end{quote}
}
The Lords clearly saw \emph{Waters} as an opportunity to offer a clarification on the no-scheme rule and how to interpret it. In particular, their judgement went into more detail than what seemed necessary for the case at hand. Even if it was not needed for the result, the Lords also addressed many of the issues raised by the Law Commission in their recent report, focusing particularly on resolving the tension which was identified there between the principle relied on in the \emph{Pointe Gourde} case and the reasoning adopted in the so-called \emph{Indian} case from 1939.\footnote{\cite{indian39,gourde47}.} In the \emph{Indian} case, the scheme was given a very narrow interpretation, with Lord Romer interpreting the scope as follows.\footcite[319]{indian39}

\begin{quote}
The only difference that the scheme has made is that the acquiring
authority, who before the scheme were possible purchasers only, have
become purchasers who are under a pressing need to acquire the
land; and that is a circumstance that is never allowed to enhance the
value.
\end{quote}

Importantly, this did not entail that the purchaser's demand for the property was to be disregarded, since, as Lord Romer puts it:\footcite[316-317]{indian39}

\begin{quote}
[...] The fact is that the only possible purchaser of a potentiality is
usually quite willing to pay for it […]
\end{quote}

In \emph{Pointe Gourde}, a different stance appears to have been adopted.\footcite{gourde47} The case concerned a quarry that was expropriated for the construction of a US naval base in Trinidad. The quarry had value to the owner as a business, and the valuer had found that if the quarry had not been forcibly acquired, it could also have supplied the US navel base on a voluntary basis, thereby increasing its profits. However, the value of this potential fell to be disregarded, with Lord MacDermott describing the no-scheme rule as follows:\footcite[572]{gourde47}

\begin{quote}
It is well settled that compensation for the compulsory acquisition of
land cannot include an increase in value, which is entirely due to the
scheme underlying the acquisition
\end{quote}

Seemingly, this is at odds with the position taken by Lord Romer in the {\it Indian} case. It seems clear that in the absence of a compulsory purchase order, the US would have been ``quite willing'' to pay for the quarry's services. Still, this potential had to be disregarded. 

In \emph{Waters}, both Lord Nicholls and Lord Scott addressed the tension between the two decisions in great detail. They then offered a reconciliatory interpretation, one which seems to narrow the no-scheme rule compared to how it has most commonly been understood following \emph{Pointe Gourde}. Moreover, the House of Lords also noted the need for reform and legislation, with Lord Scott describing the current state of the law as ``highly unsatisfactory''.\footcite[164]{waters04}

To explain how a seemingly simple principle could become so troubling in practice, I think it is important to start by noting that after the introduction of extensive planning legislation in the 20th century, development of property tends to be contingent on governmental licenses and plans. Moreover, the power to expropriate is often granted as a result of comprehensive regulation of the property-use in an area, often following public plans that encompass more than the particular project that will benefit from compulsory purchase. As a result, it has become increasingly difficult to ascertain what is meant by the ``scheme'' in compensation cases. Does it include the whole planning history leading to expropriation, does it only refer to the power to expropriate, or is it something in between?

A fine balancing act must be made when attempting to answer this question. Under a wide interpretation of ``the scheme'', forcing the valuer to entertain many counterfactual assumptions, the property owner might come to feel that he is not compensated for his true loss, but rather an imaginary one. Indeed, the no-scheme world that the valuer must consider can end up being far removed from the actual one, forcing him to go back many years, perhaps decades, to establish what would have been the status of the property in question if the sequence of planning steps eventually leading to expropriation had not taken place. 

This can leave the property owner in an unpredictable and very weak position. Taken to extremes, the no-scheme principle can then also come to run amiss with respect to human rights law and constitutional provisions protecting private property. On the other hand, if the scheme is interpreted too narrowly, one runs the risk of endangering important public schemes by compelling the public to pay extortionate amounts. In many cases, it is undoubtedly true that the value of property is increased by public investments and plans for the area in which the property is found. Moreover, one may ask if it is right to pay compensation based on increases in value that result from investments and plans that would not have materialised unless the power to expropriate had been anticipated. This, it may be argued, would be a form of double payment that should be avoided.

As noted by the Law Commission, it is important to keep in mind that the no-scheme rule serves at two distinct purposes.\footcite[69-70]{lawcom03} First, the rule has an important \emph{positive} dimension, enhancing compensation payments. Property owners are not only compensated for the direct loss of their property, but also for the possible depreciation of their property's value following the decision to carry out a scheme which requires expropriation. Seemingly, this is easy to justify: It seems intuitively unreasonable if the deleterious effects of a threat of compulsion is permitted to result in reduced compensation payments.

However, under the extensive planning regimes common today, it is not clear where to draw the line. When is the regulation leading up to the scheme to be regarded as reflecting general public control over property use, and when is it to be regarded as a measure specifically aimed at compelling private owners to give up their property? As we will see when we consider the role of the no-scheme rule in Norwegian law, this question can easily become highly controversial, especially when it is linked with the more general question of whether or not the state should be liable to pay compensation for regulation that adversely affects the potential for future development. In jurisdictions that do not recognize owners' right to such compensation, like Norway and England, it is easily argued that the positive aspect of the no-scheme rule must be limited correspondingly. Why should a depreciation of value following regulation imply compensation when the property is eventually expropriated, but not otherwise?

In addition to its positive dimension, the no-scheme rule also has an important \emph{negative} dimension, expressed in {\it Pointe Gourde} as the principle that an {\it increase} in value should be disregarded when it is ``entirely due to the scheme''. The negative dimension has attracted more interest and controversy than the positive dimension, especially in the UK. This is also the aspect of the rule that was at the center of attention in {\it Waters}.

It is not surprising that the negative aspect of the no-scheme principle more often results in complaints, as property owners stand to loose whenever it is applied. However, on a traditional understanding of the public purpose of expropriation, the negative aspect of the rule is also seemingly easy to justify. In \emph{Waters}, Lord Nicholls describes the most important policy reasons as follows:\footcite[18]{waters04}

\begin{quote}
When granting a power to acquire land compulsorily for a particular purpose Parliament cannot have intended thereby to increase the value of the subject land. Parliament cannot have intended that the acquiring authority should pay as compensation a larger amount than the owner could reasonably have obtained for his land in the absence of the power. For the same reason there should also be disregarded the ``special want'' of an acquiring authority for a particular site which arises from the authority having been authorised to acquire it.
\end{quote}

This appears like a reasonable justification. Notice, however, that Lord Nicholls avoids using the word ``scheme''. In particular, he does not identify the scheme's absence as the measuring stick for ascertaining on what basis parliament intends compensation to be based. Rather, Lord Nicholls speaks of what the owner could reasonably have obtained in the \emph{absence of the power} to acquire the land compulsory. In this way, he seems to prescribe a rather narrow interpretation of the negative dimension of the no-scheme rule.\footnote{See also the commentary offered in \cite{newuk}.} It is the power to expropriate that should not give rise to an increased value, nothing at all is said at this stage about the scheme that benefits from it.

It would appear, therefore, that there is nothing in principle that prevents the property from being compensated on the basis of its value in a scheme that differs from the scheme underlying expropriation only in that it does not have such powers. Indeed, this subtle caveat appears to be rather crucial for the remainder of Lord Nicholls' arguments, when he attempts to reconcile the principle adopted in the \emph{Indian} case with the \emph{Pointe Gourde} case.

It would lead me too far astray to go into all the subtle details about the interpretation of the no-scheme rule in UK law and the possible implications of \emph{Waters}. Rather, I would like to focus on one specific aspect, namely the application of the principle when the scheme in question is a commercial enterprise. The UK Supreme Court touched on this issue in the recent case of  \emph{Bocardo}.\footnote{\cite{bocardo10}.} The case was decided under dissent, suggesting that the clarifications offered in \emph{Waters} have not been as conclusive as one might have hoped.

\emph{Bocardo} concerned a reservoir of petroleum that extended beneath the appellant's estate. The petroleum could not be extracted without carrying out works beneath their land. The first question that arose was whether or not extraction of the petroleum amounted to an infringement of property rights. This was answered in the affirmative. The second question that arose was what principle of compensation should be adopted to compensate the owner. The Supreme Court, following some deliberation, found that the general rules applied, meaning that the case should be decided on the basis of an application of the no-scheme principle.

However, opinions differed as to the correct interpretation of this principle, as well as how the facts should be held against the law. The crucial point of disagreement arose with respect to whether or not the special suitability, or \emph{key value}, of the appellant's land, \emph{pre-existed} the petroleum scheme.

In \emph{Waters}, the House of Lords had cited and expressed support for the following passage, taken from Mann LJ's judgement in \emph{Batchelor}.\footnote{\cite[361]{batchelor89}. Cited by Lord Nicholls at \cite[65]{waters04}.}

\begin{quote}
If a premium value is ``entirely due to the scheme underlying the acquisition'' then it must be disregarded. If it was pre-existent to the acquisition it must in my judgement be regarded. To ignore the pre-existent value would be to expropriate it without compensation and would be to contravene the fundamental principle of equivalence.
\end{quote}

%(see \emph{Horn v Sunderland Corporation})
Relying on this distinction between the potentialities that are ``pre-existing'' and those that are due to the scheme, the minority in \emph{Bocardo}, led by Lord Clarke, made the following observation.\footcite[42]{bocardo10}

\begin{quote}
Anyone who had obtained a licence to search, bore for and get the petroleum under Bocardo’s
land would have had precisely the same need to obtain a wayleave to obtain access
to it if it was not to commit a trespass. So it was not the respondents' scheme that
gave the relevant strata beneath Bocardo’s land its peculiar and unusual value. It
was the geographical position that its land occupies above the apex of the
reservoir, coupled with the fact that it was only by drilling through Bocardo’s land
that any licence holder could obtain access to that part of the reservoir that gives it
its key value.
\end{quote}

This view was rejected by the majority, led by Lord Brown, who interpreted the no-scheme rule quite differently:\footcite[83]{bocardo10}

\begin{quote}To my mind it is impossible to characterise the key value in the ancillary
right being granted here as ``pre-existent'' to the scheme. There is, of course,
always the chance that a statutory body with compulsory purchase powers may
need to acquire land or rights over land to accomplish a statutory purpose for
which these powers have been accorded to them. But that does not mean that upon
the materialisation of such a scheme, the ``key'' value of the land or rights which
now are required is to be regarded as “pre-existent”.
\end{quote}

While the case was resolved in keeping with this view, the dissent suggests that the clarification in \emph{Waters} has not resolved all issues. Moreover, it suggests that special questions arise when the expropriation scheme itself involves the realisation of a commercial potential inherent in the land that is taken. Is it permissible for government to grant the value of this potential to the taker -- by granting him the necessary licenses -- without subsequently recognizing the potential as having been taken from the owner? 

This issue does not \emph{not} primarily depend on the scope of the scheme as such. In {\it Bocardo}, for instance, it was obvious that the scheme was the entire project aimed at extracting petroleum from the reserve, including the necessary works beneath the appellant's estate. But even so, it was still unclear whether the special value of the appellant's land could be said to have been {\it caused} by the scheme. The issue that arises in these kinds of situations is ontological: When should we attribute a given value to an act of government, and when should we attribute it to nature, as a fruit of the land? Or in more practical legal terms: When is a given property value that is unlocked by a development scheme part of the original owner's bundle of rights?

To answer this question, it is tempting to look for a causal link between scheme and value, to substantiate the claim that the value was not in fact pre-existent. But as \emph{Bocardo} illustrates, it is not always obvious what should be taken as good evidence for such a link. It seems that one's perspective on this will tend to depend also on one's point of view on the much more general question of what values one recognize as inherent in property rights.

When Lord Clarke remarked that the state, following nationalisation in 1934, could have given the right to extract the petroleum to \emph{someone else}, he was certainly correct. Hence, I also agree with him that ``the key value was not created by the 1934 Act or the grant of the petroleum licence to Star''.\footnote{See \cite[163]{bocardo10}.} But whose value was it, and was it a commercially realisable value? Here, Lord Clarke appears to assume that the value must belong to the property owner and that this owner would also have been able to make a profit from it in the absence of the expropriation scheme. This, I believe, is a leap that requires further justification. Just because some property has key value does not mean that the owner of the property is entitled to that value, or that it can ever be translated into a financial profit.

On the one hand, it is easy to agree with Lord Clarke that compulsory acquisition of a wayleave is no precondition for an extraction scheme. The project could well have been carried out by a developer who was willing to pay the owner for the special suitability of his land. But on the other hand, it does not seem obvious that the owner is meant to be able to demand such payment under the regulatory system currently in place. Hence, even in the absence of a causal link between scheme and value, one might be entitled to conclude that the special value falls to be disregarded because it has already effectively been removed from the owner's bundle.

In the case of {\it Bocardo}, I think this perspective would have been particularly helpful to Lord Brown, who argued that the value of the strata was not pre-existent. As it stands, his argument seems rather strained. After all, it was the physical conditions that gave the land its value, not the abstract fact that a development license had been granted. However, by looking at his argument in more depth, it is tempting to rephrase his conclusion by saying that he regarded the special suitability of the strata as having no commercial value under the prevailing regulatory regime.

In the end, I am agnostic about the correct way to judge {\it Bocardo}, but I think the crucial question that it raised was the following: did parliament intend to give petroleum developers a right to extract substrata resources without sharing the profits with affected surface owners? If no clear answer is available, conflicts can result, particularly if the question itself is obfuscated, as I think it was in {\it Bocardo}. It seems to me, in particular, that the focus on causality and the notion of ``pre-existence'' was not very helpful. Rather, I think the crucial keyword should have been benefit sharing.

The first question to ask in this regard is what parliament intended when it set up the current regulatory framework. If this is unclear or the evidence suggests that benefit sharing was not intended, the question becomes whether or not benefit sharing is nevertheless required on the basis of constitutional or human rights law. In a case like {\it Bocardo}, the latter question is unlikely to arise with any great force. It seems to me, in particular, that the question of how to deal with a property's ``key value'' in relation to other property is usually a question that can be resolved merely by pointing to the legitimate public interest in avoiding unwanted holdouts.

Even so, if the courts engage with the question of benefit sharing without being explicit about it, the lack of democratic accountability can become a worry. I think it is important to emphasize the political sensitivity of the range of complex rules found in compensation law. If not, a crisp political question risks becoming obfuscated to the extent that it can only be engaged with in a meaningful way by legal professionals. This, in turn, increases the chance of abuse and undue influence of special interest groups. While most people remain ignorant of the political work done by the courts in this regard, those who stand to gain the most are free to lobby and argue on technical points to gradually shape the law of benefit sharing according to their own interests. A conceptual shift might be needed to prevent this development from becoming precarious to the legitimacy of compensation law in general, and the no-scheme rule in particular. 

In addition, the question becomes much more pressing in cases when the development potential as such is subject to expropriation. An extreme case arises when natural resources are expropriated. For an illustration which also links up to my case study, I mention particularly the cases of \emph{Cedars} (1914) and \emph{Fraser} (1917), two Canadian compensation disputes regarding expropriation for hydropower. They were cited as important authorities by both the Law Commission and the House of Lords in \emph{Waters}.\footnote{\cite{cedars14,fraser17}.} 

In \emph{Fraser}, it was the waterfalls themselves that were subject to expropriation, yet the Privy Council still found that the value of the potential for hydropower exploitation of these falls should be disregarded when compensating them. The reasoning adopted seems to follow a standard ``value to the owner'' approach. However, reflecting back on {\it Bocardo}, it is hard to see how anyone could think that the value of the waterfalls were not ``pre-existent'' to the scheme to develop them. Surely, as a natural resource, a waterfall has significant value in itself, independently of any particular ``scheme''? 

Not so, according to the Privy Council, who found that the owners of waterfalls could not themselves have developed hydropower. Here, a subjective standard was in effect employed, whereby the bundle of rights associated with a property depended not only on the property itself but also on the nature of its owner. This unequal treatment of owners is such that is could, in my opinion, now be attacked from the point of view of human rights and constitutional law.\footnote{Although such an approach might not be required to overrule them, as the Canadian cases already appear to be at odds with both {\it Waters} and {\it Bocardo}.}

However, in order to make such an attack, it is necessary to use a working distinction between commercial and non-commercial aspects of a development scheme. The pre-existence test is inadequate. For instance, there can be no doubt that the energy inherent in water pre-exists any scheme seeking to harness it. Moreover, it seems clear that energy has great value, meaning that the value of a waterfall pre-exists any scheme for hydropower exploitation. However, we must also ask: what \emph{kind} of value is it?

To illustrate why this is a relevant consideration, consider a case where the property value is enhanced for the owner because of a personal attachment. In this case, it seems fair to differentiate, so that the owner's subjective attachment to the property is taken into account, potentially leading to a higher compensation payment then any other owner would receive. It is irrelevant, moreover, whether or not the particular aspect of the property to which the owner is attached is pre-existing. The relevant consideration is simply whether or not the value in question is such that one thinks it {\it should} be compensated. The value is {\it not} commercial, however, but personal (and, in so far as it receives recognition, also public). This is {\it why} differential treatment becomes justifiable. 

Similarly, in so far as a piece of land is particularly suited for building a school, it seems unproblematic to deny benefit sharing with the owner. In this case, the suitability is pre-existent, but it reflects a value to the public, not to commerce. Hence, a disregard rule can safely be applied, even though the public would been willing to pay large amounts in friendly negotiations. But what if the land was not particularly suited for a school, but for a shopping mall? Here I believe a different standard is needed. It seems, in particular, that benefit sharing is required in this case since one would otherwise illegitimately discriminate between owners. Why should the owners of shares in a shopping mall be allowed to profit, when the owners of the suitable land are not?

As a practical test, I propose the heuristic whereby one regards the commercial value of the development as evidence that disregard rules like the no-scheme principle should not be applied. The underlying rationale behind this heuristic is based on the public interest requirement. It seems to me, in particular, that disregard rules are also in need of justification based on the needs of the public.
In my opinion, the public interest/purpose requirement extends to compensation in such a way that a value needs to be identified as a public value in order for it to be legitimate to disregard this value when compensating the owner. 

More generally, I fail to see how it could ever be legitimate to apply a no-scheme principle unless it serves the public good. If the principle is applied in a way that results in a commercial benefit to the taker and a commercial loss to the owner, I would argue that it renders the expropriation as a whole unsafe in relation to the public interest requirement. One aspect of the interference, at least, then lacks proper motivation. From this I arrive at the general conclusion that values which are recognized as commercial should never be disregarded.

The distinction between commercial and public values is obviously not written in stone, but is down to a political decision. Moreover, it can hardly be regarded as permanent. In addition, it can often be difficult to assess where the line is to be drawn, especially in cases when public-private partnerships are relied on to provide public services. Nevertheless, it seems to me that the public interest requirement in constitutional and human rights law makes it necessary to be explicit about private and public values also in relation to compensation. Moreover, it seems like doing so could be very helpful in many cases, such as {\it Bocardo} and {\it Fraser}.

For instance, even if the public value of hydropower pre-exists an hydropower scheme, this does \emph{not} necessarily mean that there is any pre-existent commercial value in hydropower. What counts as {\it commercial} value, in particular, must first be answered. This, moreover, depends entirely on whether or not the public has settled on a regulatory regime that allows commercial exploitation.
Hence, I arrive at the following suggestion for a modified version of the ``pre-existence'' test: An owner should always be compensated for the value of any pre-existent \emph{commercial} value that his property has.\footnote{Certainly, a clarification along these line would not resolve all issues. It would not, for instance, offer any conclusive guidance with respect to the specific issues related to "key value" raised in \emph{Bocardo}.} 

To answer the question of what should be regarded as a pre-existing commercial value, one must take a broad look at the prevailing regulatory regime. Moreover, one must expect that the assessment will depend on the context of regulation, in particular the extent to which the state \emph{allows} the disputed value to be commercially realized. The law relating to compensation should be such that it can tolerate significant changes in these parameters. The theoretical question that arises concerns only the conceptual foundation for the assessment. The actual lines that must be drawn are all drawn in the sand, as usual.

In the next section, I will address Norwegian compensation law to shed light on some such lines that have been drawn in relation to waterfalls, which have recently been washed away and redrawn following liberalization of the hydropower sector. This will allow me to shed light both on the no-scheme rule and alternatives to it. 

%Moreover, I note how the Norwegian system was originally based on a rejection of the idea that all disputes had to be resolved uniformly on the basis of a battery of specific rules. Instead, great emphasis was placed on the discretion of lay people. In later years, however, Norwegian compensation law has developed along a similar trajectory to that of the UK. The no-scheme principle, in particular, has now been addressed in so many different ways and by some many different sources of authority that it appears just as much in defiance of ready comprehension as in the UK before {\it Waters}. I will now try to untangle the web somewhat, while moving towards the special points I would like to make based on the idiosyncratic case of waterfalls. 
%
%But the assessment itself was above all else discretionary. Legitimacy of the process was ensured in a bottom-up fashion, by the involvement of lay people sitting as appraisers, alongside a regular judge.
%
%I note, however, that this system has largely been modified so that, today, the appraisal courts are far more constrained from the top down, by legislation and the Supreme Court. I argue that this has been a source of difficulty for the system, particularly in relation to the no-scheme principle which, as I argued above, necessitates a concrete assessment, and can not -- should not -- be resolved by all-encompassing principles. I note, in particular, how the increasingly constrained room for discretion by lay people means that the distinction between commercial and public value -- which must now be determined centrally -- becomes muddled. In many cases, the idea acting as a premise for the general rules applied simply does not correspond to reality. This often leads to unacknowledged commercial windfalls for takers, arising when owners are denied compensation for commercially valuable rights that the law presupposes to be wholly public, even though they are not. 

\section{Appraisal courts and ``foreseeable alternatives''}

The owner's right to compensation following expropriation of property is enshrined in very simple terms in Section 105 of the Norwegian Constitution.\footnote{\cite[105]{grunnloven14}.} This section states simply that \emph{full compensation} is to be paid in all cases when the public interest warrants the compulsory acquisition of property. For more than 150 years, this was the sole legislative basis for compensation rules in Norway. The methods used to calculate full compensation in different scenarios developed entirely through case law.

According to a long legal tradition in Norway, the discretionary aspects of property valuation is regulated by a special procedure, with a significant reliance on so called \emph{unwilling appraisers}. These are members of the public who have no interests in the case at hand. They may be chosen, however,  specifically for their suitability in judging the value of the contested property, either because they are resident in the local area or because they have special expertise.

The appraisal procedure has a long history, going back to customary law that pre-dates the constitution. The rules regulating it today are found in the \cite{aa17}.\footnote{Act no 1 of 1 June 1917 relating to Appraisal Disputes and Expropriation Cases.} Appraisal cases are organised similarly to civil disputes, and the procedure is administered by the district courts.\footnote{\cite[5]{aa17}.} Appraisal courts are usually composed of a panel consisting of one professional judge and four appraisers, with no special juridical qualifications. 

The standard arrangement is that appraisers are chosen from the general public in the district where the property in question is located. But the Act opens up for the possibility that appraisers may also be chosen for their special technical expertise.\footnote{See \cite[11|12]{aa17}.} Their role in the procedure is on par with the judge, and the panel decides jointly both the legal and the technical questions, usually on the basis of technical reports put forth by the parties. These reports are presented during the main hearing, and may be challenged by the parties, in more or less the same way as the district court hears evidence in a regular civil dispute.\footnote{See particularly \cite[22|27]{aa17}, with further references to the \cite{da05} (Act No 90 of 17 June 2005 relating to the Mediation and Procedure in Civil Disputes).} 

There is a possibility for appeal to the appraisal Court of Appeal, which is the regular Court of Appeal sitting as an appraisal court in accordance with the rules of the \cite{aa17}. The right to having an appeal heard is not absolute; whether the appraisal Court of Appeal will hear the case depends on its importance, according to rules that correspond to those in place for regular civil disputes.\footnote{See \cite[32]{aa17}.} The procedure closely corresponds to the procedure followed in appraisal disputes at the district level.\footnote{See \cite[38]{aa17}.} However, the decision made by the appraisal Court of Appeal is final as far the appraisal assessment is concerned. An appeal to the Supreme Court can only be accepted on legal grounds.

As a consequence of this, the appraisal courts have been very important in interpreting and developing the law relating to compensation in Norway. Their importance was particularly great all the while the meaning of ``full compensation'' was not clarified further in statute. The presence of lay people sitting as judges is consistent with how many civil disputes are resolved. But what makes the appraisal courts unique is that in these cases the lay people where traditionally allowed to engage with the issue under very few restraints, beyond procedural rules and the words of the Constitution.

At the same time, the practical viewpoint enforced by the procedural form meant that legal questions would often remain in the background in such cases. Typically, the legal issues would only come to the forefront if the Supreme Court decided to hear the case as a matter of principle. 

The primary criticism voiced against this system, particularly following the Second World War, was that it gave the appraisal courts too much discretionary power. Hence, the argument went, legislation was needed to make the outcome of appraisal cases more predictable.\footnote{See, for instance, Part 2, Chapter 1 of \cite{nut69}, handed over to the Ministry of Justice by the so called Husaas committee, appointed by the King in Council 6 Aug 1965.} However, while the law regarding compensation was not formalized in written form, there were legal scholars who developed theories and aimed to explicate its content based on the body of case law that was available.

Also, the Supreme Court did regularly hear cases concerning legal arguments regarding compensation, and they developed a consistent position on at least some of the more critical and recurring legal issues. At this time, the central source of legal reasoning regarding appraisal was still to be found in the constitution itself. As a result, theories of compensation law tended to be \emph{absolutist}, in the sense that they looked directly to wording in Section 105, also when tackling specific problems of interpretation. 

\subsection{Constitutional absolutism}

Absolutism was widely endorsed by Norwegian legal scholars as late as in the 1940s. The well-known legal scholar Magne Schjødt summed it up as follows:\footcite[177]{schjodt47}

\begin{quote}
When an owner is entitled to compensation, he is entitled to have his full economic loss covered. He should receive full compensation, see p 42 ff. This is the great principle that remains absolute and any dispute must be resolved on its basis.
\end{quote}

A typical example of the style of legal reasoning that this view gave rise to can be found in thes writings of the prominent legal scholar Frede Castberg. He specifically addressed also the no-scheme principle, by asking about the extent to which increases in value due to the scheme underlying expropriation was to be taken into account when calculating compensation. His reasoning in this regard was based directly on a reading of the Constitution. Moreover, it was based on the principle of \emph{equality}, which was at that time considered particularly crucial in understanding constitutional law. The following quote serves to sum up Castberg's position on the no-scheme principle:\footcite[268]{castberg64b}

\begin{quote}
The owner is entitled to full compensation. The expropriation should not leave him worse off economically than other owners. Hence if the public has knowledge that an industrial undertaking is being planned, that a railway will be built etc, and this affects the value of property generally in a district, then the increased value of the property that will be expropriated must be taken into account. If not, the owners of such property will be worse off than other owners from the same district. On the other hand, if the expectation of the scheme underlying expropriation leads to a general depreciation of value, then it is this new value -- not the original value -- that is relevant for calculating compensation. The crucial question is what the actual value is, when expropriation takes place.
\end{quote}

% e mention that the problem analyzed by Castberg in this passage has been considered in many jurisdiction, and is dealt with in common law by the so called \emph{no-scheme} rule. This is more a principle than a single rule, and it is typically understood as a mechanism that is meant to ensure that changes in value due to the scheme underlying expropriation are disregarded.\footnote{For an history of the rule in common law (primarily the UK), which also illustrates the difficulty in interpreting it and applying it to concrete cases, we point to Appendix D of Law Commission Report No 286, 2003} In comparative terms, Castberg appears to favor a \emph{narrow} interpretation of the principle -- a restrictive view on when additional value due to the scheme should be disregarded -- quite close in spirit to the so called \emph{Indian} case from 1939\footnote{\emph{Vyricherla Narayana Gajapatiraju v Revenue Divisional Officer, Vizagapatam} [1939] AC 302.}, which was been much discussed in common law and was dealt with extensively by the House of Lords as late as in 2004.\footnote{In the case of \emph{Waters and other v Welsh National Assembly} [2004] UKHL 19. 
%The primary precedent for a broader interpretation of the non-statutory no-scheme rule, on the other hand, is \emph{Pointe Gourde}, \emph{Pointe Gourde Quarrying and Transport Co v Sub-Intendent of Crown Lands} [1947] AC 565, PC, 572, per Lord MacDermott. This case proved highly influential for the understanding of compensation rules in the post-war period, in many common law jurisdictions, but has recently been challenged by a renewed interest in more narrow viewpoints such as that expressed in the \emph{Indian} case, see  \cite{newuk} and also the case of \emph{Star Energy Weald Basin Limited and another (Respondents) v Bocardo SA (Appellant) [2010] UKSC 35}.}In the context of Norwegian law, it is of particular interest to note how Castberg's views in this regard is arrived at through considering the constitution itself, founded on the principle of equality.\footnote{In this way, he arrives at a narrow no-scheme rule quite abstractly, and through a different route than the one adopted in the \emph{Indian} case, where the outcome appears to have turned crucially on the particular facts in the case, a close reading of precedent, as well as the perceived fairness of the result.}

As Castberg bases his analysis on the exact wording of the Constitution, he does not engage in any reasoning based on the extent to which it can be regarded as socially fair for the public to pay compensation for value that arise due to the beneficial consequences of the public project itself. Crucially, he does not address the concern that this can be seen as a form of double payment. Such pragmatic and utilitarian points were not widely used to interpret the law in the legal tradition Castberg was part of. This, in particular, is why I think it is appropriate to use the label of constitutional absolutism to describe this kind of reasoning.

However, it is not correct to think that such reasoning is necessarily ``owner-friendly''. To see this, it is enough to note that Castberg, in the quote above, explicitly states that depreciation of value due to the scheme should not be disregarded. In addition, Castberg did not intend to reject the no-scheme principle altogether. In particular, he explicitly denied that owners of expropriated property should ever be able to claim compensation based on the special want of the acquiring party:\footcite[268]{castberg64b}

\begin{quote}
The situation is different if the property has increased value due to the expectation that it will be expropriated. The owner can not demand that this increase is compensated since that would be the same as giving him a special advantage compared to those from whom no property is expropriated.
\end{quote}

Hence, Castberg accepts a narrow version of the no-scheme principle, similar in spirit to that presented by Lord Romer in the {\it Indian} case. Castberg's view appears to have been shared by many academics of his day, and it was also largely reflected in case law from the Supreme Court.\footnote{See below.} At the same time, the very nature of the system for deciding appraisal disputes gave the local appraisers great freedom in adapting the principles in a way that suited the concrete circumstances.

To some extent, this would also involve making an assessment of what was regarded as a fair and just outcome. Hence, while the theory of the time was absolutist, case law was more multi-faceted. Importantly, fairness was seen as a concrete issue that had to be addressed on a case by case basis, an approach that would not necessarily lead to general rules. The Supreme Court largely sanctioned this approach, by passively respecting the discretion of the appraisal courts, as vested in them within an absolutist theoretical framework.

But as long as they did not cross the line with regards to the constitution, the appraisal courts were largely allowed to adopt broader viewpoints as well. The point was, however, that such viewpoints were \emph{not} extensively codified in terms of special principles used to deal with special case types or issues. Rather, they arose as a logical consequence of the way in which appraisal disputes were organized. Social justice and fairness perspectives were not excluded, but could in fact play an important role in practice.. However, such perspectives arose \emph{indirectly} through a \emph{decentralized} system which gave local courts great freedom when applying the law.

The way in which the no-scheme principle was applied serves as a nice illustration of this. On the one hand, the theoretical views of Castberg were widely accepted, but at the same time they were regarded as no more than guidelines that had to be adapted to the circumstances. Moreover, it was not unheard of for the appraisers to disagree with the judge about how this should be done, and to award compensation according to a different understanding of the law than that favoured by the judge. 

This happened, for instance, in the case of \emph{Tuddal}, where land was expropriated for construction of a power grid.\footcite{tuddal56} In relation to this, the expropriating party also acquired the right to use a private road. According to the juridical judge in the appraisal court of appeal, and consistent with the teaching of Castberg, compensation should be awarded solely on the basis of what the owners stood to lose. In this case, that would mean compensation based on the increased cost in maintaining the road resulting from increased use. However, the lay appraisers found this result unreasonable and awarded compensation also for the special value the use of the road would have for the acquiring party. The Supreme Court, although they found fault with the reasons given by the law appraisers, agreed that such compensation was possible in principle. The presiding judge offered the following perspective:\footcite[111]{tuddal56}

\begin{quote}
Since they were the private owners of the road, A/S Tuddal could, before the expropriation, refuse to let the Water Authorities make use of it. Hence it might be possible for A/S Tuddal, through negotiation and voluntary agreement with the Water Authorities or others with a similar interest, to demand a reasonable fee, and in this way achieve a greater total benefit than full compensation for damages and disadvantages. Following the expropriation, it is no longer possible for A/S Tuddal, in its dealings with the Water Authorities, to economically benefit from their ownership of the road in this way. If the company suffer an economic loss as a result of this, I believe they are entitled to compensation. Whether or not such an opportunity as I have mentioned -- all things considered -- was present at the time of the expropriation, falls to the appraisal court to decide, on the basis of whether or not an economic loss is suffered beyond that which follows from damages and disadvantages. On this basis, I assume that the appraisal court of appeal's decision to awarded compensation for the value of the right of way that is acquired can not -- in and of itself -- be regarded as an erroneous application of the law.
\end{quote}

The Supreme Court's reasoning illustrates two points. First, we see how the Supreme Court adopts absolutism in its interpretation of the law. Through careful use of wording, the compensation premium is not conceptualized as compensation based on the value of the road to the acquiring authority, but rather as compensation for the loss of potential profit following from a voluntary agreement. Hence, a seeming contradiction with the no-scheme principle is avoided.\footnote{This particular interpretation of full compensation led to arguments in the post-war period, regarding whether or not owners had a right to compensation based on the loss of profit from hypothetical voluntary agreements with the acquiring party. In the end, a consensus formed that this type of compensation should not in general be awarded. See \cite{nut69},Part 2, Chapter 4, Section 2.E.}

But {\it Tuddal} also illustrates a second important point, namely that the Supreme Court was prepared to defer greatly to the judgement of the appraisal court. It is stated explicitly that it falls for this court to decide whether or not the opportunity to profit from the road by negotiating with the expropriating party was present at the time of expropriation. This is particularly noteworthy in light of the dissent of the juridical judge in the appraisal court of appeal, and in light of the dominant legal theorizing of the day, which did  not seem to support the idea that a premium should ever be paid in a situation like this. Hence, the decision tells us that the Supreme Court went far in defending the discretion of the laypeople, as a \emph{systemic} feature. They tested with great caution whether it was truly outside the permissible legal boundary, but concluded that it should simply be regarded as an exercise of the lay judgement that the system presupposed.

This impression of the case is accentuated when considering other cases dealing with similar issues. Across the board, I note a strong  tendency to defend the role of the laypeople in the appraisal process. A particularly clear expression of this can be found in \emph{Marmor}, also from 1956, where the Supreme Court overturned a decision made by the appraisal court of appeal on the grounds that the court had been {\it too} reliant on general principles.\footnote{\cite{marmor56}.} This, the Supreme Court held, offend against both the principle of full compensation and the principle of discretionary evaluation by laymen.

The case involved expropriation of a private railway track, for the construction of a public railway. It was clear that the track which was being expropriated did not have market value in general, so the expropriating party argued that the value of these tracks to the public railway should not be taken into account when calculating compensation. The appraisal court of appeal agreed, pointing to the standard teaching of the day. The Supreme Court, on the other hand, struck down the decision because they felt that a standardized approach to the case was inappropriate given the circumstances. The presiding judge argued as follows:\footcite[498-499]{marmor56}

\begin{quote}
In my opinion one can not simply assume that a property does not have market value when it has no value for anyone other than the expropriating party. The question needs to be assessed concretely. I agree with the expropriating party -- as has also been confirmed on several occasions by the Supreme Court -- that in general one should not take into consideration the special value that the purpose of expropriation gives the property. This should not lead to a spike in compensation payments. On the other hand, I can not agree that it is automatically reasonable, or in keeping with Section 105 of the constitution, if the expropriating party in cases like the present one could acquire property at a price that is below what it would be natural and commercially appropriate to pay in a voluntary purchase.
\end{quote}

Again, I note the two main building blocks used in the argument: First, the standard reference to the constitution, and secondly, a reference to the need for \emph{concrete assessment}. This further reflects the strong confidence that the Supreme Court had in the integrity and autonomy the appraisal procedure. Moreover, I notice how, in this case, absolutism regarding the constitutional protection of property is \emph{not} used to argue for specific rules or principles, but rather to back up the argument that compensation should result from real assessment, and not be overly reliant on such rules. In the case of {\it Marmor}, this was the outcome even if the rules in question had the status of valid guidelines that had also been backed up by a series of Supreme Court decisions.

In addition to making the overreaching remarks quoted above, the Supreme Court also gave pointers as to the kinds of facts that should be considered. For instance, the presiding judge paid particular attention to the wider \emph{context} of expropriation, and the manner in which expropriation was used to benefit certain interests. He also noted how expropriation had come to replace voluntary agreement as the standard means of acquisition for this type of development. Therefore, the practice of using expropriation effectively prevented a market from developing, a market that might otherwise have appeared naturally:\footcite[499]{marmor56}

\begin{quote}
I also point to the fact that the case concerns an area of activity where the expropriating party has a {\it de facto} monopoly which makes it impossible for anyone else to make use of the property for the same purpose. This in itself makes it questionable to simply assume that the lack of financial value for other purchasers provides the appropriate basis for calculating compensation. When considering this question, it is also appropriate to take into account that we have lately seen a great increase in the use of expropriation to undertake projects such as this. Compulsion is becoming the primary mode for acquisition of property -- not voluntary sale following friendly negotiations.
\end{quote} 

In my opinion, the importance of this decision, which makes it highly relevant even today, is not that it seems to favour a narrow interpretation of the no-scheme principle. In fact, I think it is erroneous to read the judgement as expressing general support for any particular interpretation. In addition, I do not think the decision can be read as supporting a general principle by which compensation can always be based on the value of hypothetical agreements that could have been made with the expropriating party. Rather, I take the judgement to be an expression of scepticism towards blind obedience to \emph{any} set of detailed rules for calculating compensation that serve to limit the room for lay discretion.

At the very least, it seems clear upon closer inspection of the argument that the main objective of the Court was not to express any particular view regarding the content of the no-scheme principle, but rather to instil to the appraisal courts that they could not use this rule as an excuse not to engage in concrete assessment to ensure a reasonable outcome in keeping with the constitution.

I believe this point is important to stress. It illustrates how absolutism need not, and did not, result in a rigid system with little room for assessment based on justice and fairness, broadly conceived. Quite the contrary, the absolutism endorsed by the Supreme Court, and inherent in the Norwegian system of appraisal courts, was not characterized by blind obedience to specific rules, like those proposed by Castberg. Rather, the system was flexible, and it was explicitly intended to function such that fairness assessments based on concrete circumstances could be accommodated.\footnote{Going back to even older legal scholarship, we see that this view on the meaning of absolutism has a long history in Norway. It is present, for instance, in the work of the famous 19th century scholar Aschehough, who stressed the link between the constitution and the appraisal procedure when he considered the (then) hypothetical situation that legislation would be introduced with the specific aim of reducing the level of compensation payments following expropriation. See \cite[48]{aschehough93} 

%\begin{quote}
%If it becomes common practice to award compensation payments that are unreasonably high, this would make important public projects more expensive and difficult to carry out, greatly to the detriment of society. In many cases it might not be possible to rely on legislation to prevent such excessive compensation payments, since this would restrict the appraisers too much. To some extent this might be possible, however, and as far as it goes, parliament must be permitted to do so. However, if enacted rules clearly lead to less than full compensation in an individual case, they will be overruled by Section 105 of the constitution, and fall to be disregarded in that particular case.
%\end{quote}
%
%This quote is important because it does not rely on any particular interpretation of the constitutional demand for full compensation, but sees this inherently as an issue that needs to be resolved by concrete assessment of individual cases. Absolutism to Aschehough implies freedom and responsibility for the appraisers; freedom to judge individual cases by its merits, and a responsibility to award full compensation, irrespectively of any specific rules that might be in place to curtail excessive payments. The important point is that Aschehough here sees absolutism as a principle that should be applied to cases, not to principles. He does not argue that rules introduced to limit compensation payments would be inadmissible merely because they might sometimes suggest less than full compensation. Rather, he takes it for granted that it falls to the appraisal courts to apply the rules in a way that would prevent such outcomes. As long as the appraisal courts remain free to apply the rules in such a way that full compensation is awarded, specific rules intending to prevent excessive payments can happily coexist with absolutism.
%
%The subtle view taken by Aschehough was largely overlooked in debates following the introduction of the Compensation Act 1973, which served to introduce radical rules of exactly the kind he had predicted and considered 80 years earlier. The consequence was, as I will discuss in more depth in the next section, that the Supreme Court was forced to actively steer the interpretation of the Act to ensure that section 105 would not be violated in concrete cases. Hence, the introduction of legislation served to destabilize the system, by narrowing the room for lay judgements and increasing the reliance on legislation and special principles developed by the Supreme Court. This development is the subject of the next subsection.

%More generally, the 60s and 70s appears to be a period when the crucial role of the appraisal procedure was to some extent forgotten, and also undermined, following a heated political and ideological debate regarding the appropriateness and admissibility of introducing rules to ensure that compensation payments were brought down to a lower level. This had deep and lasting effects on Norwegian compensation law, and it is popularly described as a period when the social democrats won recognition for the principle that social fairness suggested the introduction of compensation rules and disregards that were more extensive than what had previously been considered appropriate. 
%
%This was conceived of as a fight for social justice against outdated and conservative ideas of constitutional absolutism. But it seems to us that this view of the history of Norwegian compensation law is erroneous, and largely unhelpful. The approach taken by Aschehough, in particular, placing emphasis on the important role played by the appraisers in achieving fairness and justice in concrete cases, does not appear to contradict social democratic goals at all. In fact, it seems that his approach might be better suited to serve such goals, and to accommodate a variety of different political opinions and ideas, than an approach which is based on attempting to flesh out in painstaking detail how the appraisal courts should go about achieving the balance between social fairness and owners' rights. We will return to this point later, but first we will take a closer look at the history of the radical Compensation Act 1973 and the censorship to which it was subjected by the Supreme Court, leading to the Compensation Act 1984, currently in place.

\subsection{Pragmatism}\label{sec:prag}

Following the Second World War, the social democratic \emph{Labour Party} gained a secure grip on political power in Norway. As a result, many reforms were carried out that would reshape Norwegian society. One of the most important reforms concerned the introduction of extensive planning law to ensure that land use was put under public control.\footnote{See generally \cite{thomassen97,kleven11}.} As a result of this, the period also saw expropriation being used more extensively to further public projects, such as the large scale construction of hydropower to ensure general supply of electricity.\footnote{See generally \cite{skjold06,thue06b}.} As a result of these changes, the opinion was soon voiced that there was a need for a more uniform approach to compensation, which collected some basic principles in a common body of written law. In addition, it was an explicitly stated political goal to bring compensation payments down.

In 1965, the so called \emph{Husaas committee} was appointed by the King and charged with the task of assessing the compensation rules currently in place.\footnote{Appointed by the King in Council on 6. Aug 1965.} The committee was also ordered to make a concrete suggestion regarding the need for additional principles of compensation, and to consider if these should be given in the form of a special compensation act. Initially there was some doubt as to the extent to which is was at all permissible to give rules regulating compensation, as the constitution itself addressed the matter. 

However, the committee noted that some rules had already been introduced for specific case types, for instance in relation to expropriation for hydropower development.\footnote{As discussed in Chapter \ref{chap:..}, Section \ref{sec:...} in relation to the \cite[16]{wra17}.} In addition, legal scholars of the day were generally of the opinion that compensation rules could be given, on the understanding that the courts would deviate from them in so far as they seemed to go against the Constitution. Hence, the Constitution was not understood to stand in the way of more specific rules.\footnote{\cite[136-137]{nut69}.} According to the minority of five, no such rules were actually needed, but the majority of ten disagreed.\footcite[137]{nut69} }

When considering the question of what kind of rules should be introduced, the Committee looked to case law as well as existing literature on compensation. They were faced with highly divergent opinions on the subject. Since WW2, in particular, a pragmatic view on property rights had taken hold, whereby an absolute right to property was increasingly felt to stand in the way of efforts to rebuild the country and ensure development following the great war. The Labour party had secured a firm grip on government at this point, so there was also an ideological shift taking place that emphasised the importance of building a welfare state over protecting the entitlements of individuals.

This was by no means a consensus view among legal scholars, however, and it was particularly contentious with regards to property. 

%As a result, some disagreed strongly with the very idea of legislation regarding compensation, and tensions arose that have led to much legal controversy and are still important in the law today.

%The majority pointed out that a vague general principle such as that provided by the constitution would by necessity have to be interpreted in order to be applied to concrete cases.\footnote[137]{nut69} Hence, it was not only permissible, but also desirable, for parliament to give more detailed instructions as to how is should be applied and understood by the courts and the appraisal courts. Leaving it to the judiciary to flesh out the exact meaning of full compensation through case law, it was felt, was not appropriate in a regulatory regime where expropriation had become increasingly important as a means to ensure modernization and development of critical infrastructure.

%In addition to this, the Supreme Court itself had recently expressed its support for a new view on regulation of property use, supported by contemporary legal scholars and politicians, whereby the State was regarded as having wide discretionary powers to determine how property should be used. This right to regulate, in particular, was increasingly coming to be seen as a right that did not infringe on property rights, so that the State would not have to compensate owners if they exercised it, except in special cases.\footnote{See, in particular, Rt. 1970 p. 67.}.

This problem area was mapped out in some detail by the Husaas committee, who traced the pragmatic view on compensation, identifying it using the following quote by the leading scholar Knoph from \cite[113]{knoph39}.

\begin{quote}
Since Section 105 is a rule prescribing practical justice, directed at parliament, and not an ethical postulate of absolute validity, it must be permitted to make technical legal considerations, so that one accepts compensation rules that lead to correct and just results on average, even if it does not grant the owner full individual justice in every case.
\end{quote}

Many disagreed vehemently with this perspective, based on absolutist principles.\footnote{See, e.g., \cite[20-22]{robberstad57};\cite[44]{schjodt47}.} The prominent legal scholar Schjødt, for instance, describes Knoph's reading of the law scathingly as follows:\footcite[44]{schjodt47}

\begin{quote}Luckily it has not had any effect on judicial practice whatsoever. No court of law would accept that compensation should be set according to a norm that may be practical and just in general, but does not grant the owner full compensation in all individual cases.
\end{quote}

By the late 1960s, however, Knoph's view was beginning to find favour among legal scholars.\footnote{See \cite[17]{fleischer68};\cite[41]{opshal68}.} When assessing the writings on the subject, the Husaas committee noted this tension. In response to it, they proposed a set of general principles for compensation which are still largely with us today. They were moving in a pragmatic direction, but rather cautiously. Hence, they refrained from encoding principles that would appear too offensive to the absolutists, even if the pervading political sentiment was that compensation rights had to be limited to ensure more effective state regulation of property use.

Importantly, the Husaas committee distilled from case law the principle that owners could only demand compensation based on the value of a specific use of the property when this use was ``foreseeable''.
The committee sought to codify this idea, which they saw as expressing an interpretation of Section 105 that was already largely entrenched in case law.\footcite[134]{nut69} This led to the following conclusion:\footnote{\cite[142]{nut69}.}

\begin{quote}
It is the view of the committee that it is correct to encode in the act the principle that the owner is entitled to compensation based on the value that results from taking into account the foreseeable and natural use of the property, given its location and the surrounding conditions. The exact meaning of ``natural and foreseeable'' use must be decided after a concrete assessment in individual cases. By encoding this general principle, however, it will become clear that compensation should not be based on private or public plans unless these plans coincide with the use of the property that is natural and foreseeable, independently of the scheme underlying expropriation.
\end{quote}

Importantly, I note how the committee actually does more than just encode a foreseeability constraint. They also state outright that this constraint is taken to imply the no-scheme principle, since they stipulate that the assessment of what counts as foreseeable and natural must be made independently of the scheme underlying expropriation. Since this statement is made quite generally, it also seems that the committee expresses a broader view on the no-scheme principle than that endorsed by Castberg.\footcite[268]{castberg64b} It is no longer only the special want of the expropriating party that should not be taken into account, the entire scheme ``underlying'' expropriation should be disregarded.

But in fact, this view was not in keeping with the political motivation for an act regarding compensation. It was too owner-friendly. Hence, the Ministry of Justice deviated from it in their final proposition to parliament. Instead of encoding existing principles, they sought a more aggressively pragmatic system whereby compensation would in general be based on the value of the \emph{current use} of the property.\footnote{\cite[19-20]{otprp59}.} In this way, the argument went, the public no longer had to pay a financial premium to owners based on possible future uses that would in any event, in most cases, be reliant on public development permissions.\footnote{\cite[17-20]{otprp59}.} 

%Such permissions, it was argued, could never be foreseeable in circumstances when it was in the public interest that the property should be expropriated, and hence all future development potential should in principle fall to be disregarded.

%The Ministry commented on this as follows:\footnote{\cite[19-20]{otprp59}.}
%
%\begin{quote}
%The Ministry is of the opinion that it is particularly important to arrive at a rule that can bring the assessment of property value down to a realistic level, and believes that the natural starting point for such an assessment must be the current use of the property, especially for expropriation of real property. As mentioned, it is the opinion of the Ministry that a practice has developed that gives too much weight to more or less uncertain future possibilities for the property, something that has led to a sharp rise in compensation payments.
%\end{quote}

After intense debate in parliament, where the minority center-right parties all opposed its introduction, the current use rule was eventually encoded in section 4, no 1 of the \cite{ca73}.\footnote{Act No 4 of 26 March 1973 Regarding Compensation following Expropriation of Real Property.} This was largely seen as a social democratic victory and a clear indication that the pragmatic approach to property protection was gaining ground. When clarifying their principled starting point regarding what should count as \emph{realistic}, the Ministry made the following assertion regarding the scope of the constitutional protection offered in Section 105, showing the ideological underpinnings of the new Act:\footcite[17]{otprp70}

\begin{quote}
However, a right to complete -- or almost complete -- equality can not be derived from the constitution. It must be taken into account that we are here discussing equality with regards to increases in property value that are, in themselves, undeserved. [...]  %  The starting point must be that it is not, in and of itself, contrary to the constitution that one property owner do not benefit from the same increase in value as another, when the increase in value, for both of them, is due to public investment and does not stem from their own efforts. \\ \\
Certainly, it would be best to avoid any kind of inequality, if it was possible. But the examples we have considered illustrate that, today, inequality between property owners is tolerated with regards to public investments and regulation, and that, moreover, practical and economic considerations dictate that we \emph{should} make use of differential treatment in this regard.
\end{quote}

This echoes Knoph, but also goes much further. In particular, the Ministry explicitly states that differential treatment is appropriate in the context of expropriation, and, by implication, that this should be done precisely to avoid compensation payments that include compensation for ``undeserved'' increases in value. Also, in proposing that compensation payments should be based on current use, the scope of ``undeserved value'' is made very wide. In principle it would seem to include \emph{any} value that could be attributed to an as of yet unrealized potential that the property in question might have. The question of whether or not this value was reflected in the market value of the property, in particular, was not regarded as relevant. This was in itself radical, since market value based on the likely use of an ``average buyer'' had previously been the dominant starting point for appraissal courts when awarding compensation.\footcite[112-113]{nut69}

The conceptual significance of this change in perspective should not be underestimated. Here the Ministry stood firmly behind a pragmatic view. Perceived social fairness was the overriding constraint, also with respect to constitutional property protection. However, on taking this view to its logical conclusion, it was recognized that any general compensation rules that might be introduced should themselves be subject to a fairness test, so that, for instance, the current use principle could not itself be absolute or without exception. 

Rather, it could only be applied in so far as it served the overreaching goal of social justice and fairness which was regarded as the fundamental component of property protection that made such a rule possible. This, in particular, seems like a crucial observation, and one that has in my opinion been overlooked, with unfortunate consequence for the subsequent debate and development of the law. Indeed, it echoes the sentiment behind the age-old procedural arrangement that placed high value on the free discretion of the appraisal courts. Hence, it points to the possibility of finding some \emph{common ground} between absolutist and pragmatist views on compensation.

Sensible voices from both camps seemed to arrive at the conclusion that in the end, there was no way around a concrete and contextual assessment, where social fairness values are (hopefully) used as a guide. In an attempt to translate aspects of such a perspective into legislation, the Ministry set out two exceptions to the current use rule. The first, which received by far the most attention, was based on the notion of equality between owners in same local area.\footcite[19]{otprp70} It stipulated that the appraisal courts should be free to deviate from the current use rule in so far as it felt that it was reasonable to do so in order to ensure a reasonable degree of equality between neighbouring owners.\footnote{This principle was eventually encoded in section 5, no 1-3 of the \cite{ca73}. It would prove highly controversial, since it was only formulated as a rules that ``could'' be used to increase the compensation. In \emph{Kløfta}, the Supreme Court eventually deviated from this and overruled the Act by making clear that additional compensation was \emph{obligatory} in a range of cases when the intention had clearly been that the rule should be used sparingly. In this way, and possibly inadvertently, the Supreme Court ended up defending owners' interest by \emph{limiting} the power of the appraisal courts.}

However, the Ministry also noted the need for a second exception, which is in my opinion far more important and interesting. This exception pointed to the need to ensure equality between the taker and the owner, in so far as the taker could not be regarded as the embodiment of purely public values.

%
%\begin{quote}
%One is aware that the principle of current use compensation cannot be without exception. Even though this rule will be fair in general it can, in some cases, disproportionately disadvantage property owners. One has therefore suggested rules that modify the principle to some extent. These are given for somewhat different reasons. \\ \\
%
%One case addresses the situation when current use compensation means that a property owner will be significantly worse off that other owners of similar property in the same district, according to how these properties are normally used. In these cases, the principle of equality suggest that the owner receives some -- but not necessarily full -- compensation for the inequality that would otherwise arise from the fact that his property was made subject to expropriation. %Etter departementets oppfatning har en ekspropriat etter grunnloven ikke noe krav på å bli satt helt i samme stilling som om ekspropriasjonen ikke var skjedd, en forskjellbehandling innen rimelige grenser må grunnloven tillate når dette tilsies av tungtveiende samfunnsmessige grunner. 
%\end{quote}

Importantly, the rule sought to address precisely the situation that arises when the taker benefits commercially from the expropriation. Moreover, it addressed the question of the \emph{power balance} between the expropriating party and the owner. In the words of the Ministry:\footcite[19]{otprp70}

\begin{quote}
The second modification we make has to do with the relationship between the property owner and the expropriating party. If the use of the property that the expropriation presupposes gives the property a value that is significantly higher than the value suggested by current use, this will entail a transfer of value from the property owner to the acquiring party. In some cases this might be unreasonable. As an example of when this can become an issue, we mention an agricultural property that is expropriation for the purposes of industrial production. In such a case it might be natural that the owner receives a certain share in the increased value that the new use of the property will lead to.[...] %This would be different than, say, a situation where an agricultural property is expropriated for constructing a road or for setting up recreational outdoor grounds. In such cases, the expropriation will not lead to any such economically advantageous use of the property that will give the expropriating party an economic advantage. 

To establish a flexible system, the Ministry has concluded that it is practical that the King gives rules concerning the cases where an enhanced compensation payment, based on these principles, might be appropriate. This should not be decided by individual assessment, but governed by rules for special case types. Hence, the proposed Act states that the King can pass regulation concerning this matter.
\end{quote}

This quote goes right to the heart of one of the main problems of economic development takings, and proposes a possible remedy. However, the Ministry took the view that this remedy should {\it not} be administered by the appraisal courts, but should be left in the hands of the executive. Already here I note a reason worry whether this could then ever become an ineffective way of achieving fairness in practice. Indeed, the fact that this aspect of the 1973 Act has been largely overlooked and forgotten seems to prove my point. No rules such as those proposed by the Ministry as a possibility have in fact been introduced, the entire profit still goes to the taker in cases when commercial schemes benefit from expropriation.

%More broadly, it is hard to disagree that the context of expropriation must by necessity come to play a crucial role for any approach based on compensating the ``deserved'' value. What this value should be taken to be, in particular, can hardly be determined once and for all and in general terms, but must rather be subject to continuous revision depending on how expropriation is \emph{actually used} in society. This includes looking to the purpose it is meant to serve, the parties who stand to benefit, and the groups who tend to loose their land.

%Indeed, stipulating that compensation should be ``deserved'' appears to provide a benchmark that is just as unclear as the stipulation that compensation should be ``full''. It seems, in particular, that the inherent ambiguity of these terms allows us to draw two conclusions: first, that they might very well have the same meaning, and second, that they cannot possibly be defined once 
%and for all by any act of parliament, or by any decision in the Supreme Court.
%
%But this suggest, against the Ministry and the overall spirit of the 1973 Act, that the system of appraisal courts has an important role to play in ensuring fairness in individual cases. It is hard to see how the objective of social fairness and justice for the individual can be reached without making heavy use of appraisers with discretionary competence. 

The procedural and contextual aspect of fairness seems to have been overlooked by those pushing for the 1973 Act. Since the appraisal courts were regarded as compensating owners too generously, their freedom of discretion was seen as a problem rather than as a path towards a solution. I think this regrettable. If the new Act had been slightly more temperate in its approach, by encouraging the appraisal courts to take a broader view on fairness, rather than to force them to adopt current use value as a baseline, it might have been a success. Instead, it caused an outcry, with attention shifting away from practical matters towards doctrinal issues. The primary such issue, and the most serious one, concerned the question of whether the Act as such was in breach of the constitution. This was eventually considered by the Supreme Court in the case of \emph{Kløfta} in 1976.\footnote{\cite{klofta76}.}. 

Following this decision, the 1973 Act would be significantly reinterpreted to make it appear less offensive to the constitutional standard of full compensation. However, it seems to me that the Supreme Court largely accepted that the intention behind the Act should be respected and that appraisal practice needed to be adjusted accordingly. In this, the Supreme Court signalled loyalty to the political system and the democratic process. However, in implementing this adjustment in practice, they also, possibly inadvertently, set up a system where the role of the local appraisal courts were undermined even further.

Not only were they constrained by an Act that seemed to run counter to the Constitution, they were now also ordered from above to openly deviate from its exact wording, but only for a select group of cases meeting certain pre-defined criteria. In essence, the Supreme Court itself assumed greater control over how compensation law was to be applied, no longer merely in broad strokes, but increasingly also by developing special rules for specific case types.\footnote{The clearest indication of this shift is found in recent case law wherein the Supreme Court has provided a myriad of detailed rules and directions regarding how appraisal courts should decide on the thorny issue of whether to consider public plans binding for the compensation award or to disregard them under a no-scheme rule. See generally \cite[7-9]{nou03}.} In the following section, I describe this in more detail.

\subsection{The top-down approach}\label{sec:regab}

Following the decision in {\it Kløfta}, the \cite{84} was introduced. It reverted back to the ``foreseeability'' test proposed by the Husaas committee. In section 4, it is stated that financial compensation (as opposed to compensation in kind) is to be based on either value of use or value of sale, whichever is highest.\footcite[4]{ca84} Sections 5 and 6 describes in more detail how the calculation should be carried out.

In both regards, the principal requirement is that the value is calculated based on a use of the property that is foreseeable and natural given the surrounding conditions. In relation to the value of sale, there is an additional requirement, namely that the use must be one that an ``average'' buyer would be likely to make of the property. Hence, the value of sale should be set as a general market value, not a value arising from selling the property to a specially interested party.

The extent to which the foreseeability requirement entails that the use in question has to be in accordance with public plans currently in place has been disputed. In general, compensation is only based either on uses permitted by public plans currently in place or uses that seem likely to be permitted in the future. In Norwegian law, whether a use is foreseeable is an ``either/or'' question. 

No compensation is given to reflect the so-called ``hope'' value, namely the part of a property's value that depends on the perceived likelihood of a change in planning status and future possibilities.\footnote{By contrast, compensation tends to include such an element in the UK.} If a permission for future use is deemed likely, it is subsequently regarded as a certainty for purposes of compensation, although the present-day value of a future possibility is usually calculated in a way that takes interest and inflation into account.\footnote{These calculations tend to be notoriously schematic, however, quite far removed from the realities of the financial system.} Similarly, if a future possibility is deemed unlikely, no compensation is paid for it whatsoever.

In effect, the best an owner can hope for in Norway is that the likelihood of having received more than full compensation is greater than the likelihood of having received less.

Tensions and disputes tend to arise either directly in relation to the foreseeability test, or else in relation to one of the disregard rules that encode aspects of the no-scheme principle. The disregard rules included in statute are all formulated in relation to the value of sale, although they are also regarded as applying to value of use assessments. To some extent, they may also be redundant, in so far as they already follow from the foreseeability test.\footnote{I recall that the Husaas committee itself thought that a rather wide no-scheme principle would follow already from the foreseeability test. See above.}

The main disregard rule included explicitly in the \cite{ca84} is formulated very similarly to the no-scheme rule in the UK. It states that one should not take into account changes in value that can be attributed to the ``expropriation measure''.\footcite[5]{ca84} Interestingly, the notion of an ``expropriation measure'' is defined in the Act. In section 2, the expropriation measure is said to be the ``activity, installation, or purpose'' benefiting from expropriation. Hence, while there is a definition, it is (as expected) very vague. 

If the definition had only included the second item -- that of an ``installation'' -- it would amount to a meaningful restriction. However, as it stands, an ``expropriation measure'' seems like it could include just about anything that stands in some kind of relationship to the expropriation order.
The purpose of the expropriation is included in the list, which, if taken literally, would lead to rather absurd results. 

For instance, if houses next to a small public road are expropriated for the construction of a motorway, the wording suggests that even the presence of the public road must be disregarded when assessing their value. This would seem to follow, in particular, in so far as the pubic road was built in pursuance of the same purpose as the motorway now being constructed. Luckily, the rule is not understood in this way in practice. 

However, a second rule expressed in section 5 of the \cite{ca84} states that changes in value due to other investments that the expropriating party has carried out, or plans to carry out, also falls to be disregarded. The condition is that they have {\it either} been carried out in relation to the expropriation measure {\it or} during the last 10 years.\footnote{See the third and fourth paragraph of \cite[5]{ca84}.} Hence, the disregard rule in section 5 is stronger than most no-scheme rules, in that previous or planned investments must sometimes be disregarded even if they stand in no relation between the expropriation scheme besides being carried out by the same party. In so far as the expropriating party is a public body, even this is relaxed, since all investments carried out by {\it any} public body is then to be disregarded, limited only by the 10 years rule.\footcite[5]{ca84} 

%When the constitutionality of the Compensation Act 1973 came before the Supreme Court in \emph{Kløfta}, they chose to sit as a grand chamber and they reached a decision under dissent, being divided into two fractions, consisting of 9 and 8 supreme judges respectively. However, both fractions approached the problem of constitutionality by endorsing an interpretation of Section 5 nr. 1 in the Compensation Act 1973 that gave the exception to the current use much wider scope than what had been intended by parliament. The majority went farthest, and unlike the minority they also regarded the compensation payment in the concrete case to be insufficient. The first voter for the majority commented as follows on the constitutional aspect of the case.\footcite[7-8]{klofta76}
%
%\begin{quote}
%[...] But the main question in this case, is whether or not it is in keeping with Section 105 to generally award compensation at a level below the market value that could legally be estimated, and that the owner could actually have achieved, if expropriation had not taken place. In my view, this involves allowing expropriation to transfer a right that the owner had, with a value to which he was entitled. If he is refused compensation for this value, he would, depending on the circumstances, be left significantly worse off than others in a similar position, who owns property that is not expropriated. Such a result I cannot accept. It would be a breach of established customary law and a practice that has been established throughout the years both by the appraisal courts and the Supreme Court. I refer particularly to Rt 1951 s. 87 (particularly p. 89, Opdahl). This practice is in itself a significant contribution to interpreting Section 105 on this point.
%\end{quote}
%
%I note the emphasis placed on \emph{market value} in the majority's reasoning. This may appear to be in keeping with an absolutist doctrine, but as I have mentioned, it can have unfortunate, possibly unintended, consequences for property owners, especially when combined with a restrictive view on what counts as foreseeable future development. I note, however, a technical point that might be of some significance for the interpretation of \emph{Klofta}: Instead of stating outright that a market value rule follows from the wording of the Constitution as such, the majority takes the view that this interpretation suggests itself based on the compensation practice that had currently been established. This might limit the scope of the majority's remarks in this regard, but it also serves to give further support to the claim that the role of the appraisal courts, and their assessments, still had a strong position in Norwegian compensation law at the time of \emph{Kløfta}. 
%
%I remark that the minority disagreed on the constitutional status of the market value rule. Indeed, it was in this regard that the difference of opinion between the minority and the majority was most clearly felt. The minority, in particular, explicitly rejected the view that this rule could be derived from the constitution itself, and they also disagreed with the understanding that it would have status as a constitutional rule simply because it had been adopted in practice. This bestowed merely the status of ordinary legal precedent. As expressed by the minority:\footcite[23-24]{klofta76}
%
%\begin{quote}
%Case law from this area cannot be understood as preventing parliament from changing the rules in accordance with what they regard as necessary. That would prevent a reasonable and natural development and would not be in keeping with the consensus view that Section 105 of the constitution is a rule that must be interpreted in light of, and adapted to, how society has developed and how the law is viewed. I believe the practice that have evolved cannot be decisive if a new situation and new needs require a different solution. Whether the Compensation Act is in breach of the right to full compensation enshrined in the constitution, must depend on an interpretation of the wording in the constitution itself.[...] \\ \\
%In my opinion, neither the intentions of parliament nor the way they are sought implemented through Sections 4 and 5 are in breach of the equality principle upon which Section 105 of the constitution is based. It does not follow from the constitution that an owner is in all circumstances -- and irrespectively of the economic forces from which the market value results -- entitled to compensation that is at least as great as the greatest legal value that the property could represent on a free market. A different matter is that Section 105 of the constitution could be important to the interpretation and application of the rules.
%\end{quote} 
%
%Hence, the market value rule was explicitly renounced as a constitutional principle by the minority, who nevertheless conceded that the constitution could be used to interpret Sections 4 and 5 of the Compensation Act 1973. Both the minority and the majority agreed, however, that  it would be wrong to go on to consider Section 4 of the Compensation Act 1973 in isolation. For the majority, this would clearly have led to the Compensation Act 1973 being held to be in breach of the constitution, something that was avoided since the Supreme Court chose to consider the law as a whole, with the majority using the reasoning detailed above to argue for a new interpretation of Section 5, rather than as a means to undermine Section 4. Still, their interpretation of Section 5 went well beyond what it seemed that parliament had intended, leading some scholars to claim that \emph{Kløfta} should be read as holding that the Compensation Act 1973 was unconstitutional.\footcite[477]{andenes86} In the words of the majority:\footcite[12-13]{klofta76}
%
%\begin{quote}
%The purpose of this rule is to award compensation beyond current use in cases where valuations according to section 4 could be in breach with section 105 of the Constitution. As it stands, section 5 no 1 is not sufficiently suited for this purpose. By its wording it gives the appraisal courts an opportunity to assess whether or not it is reasonable to award additional compensation, even when the conditions for this is otherwise met, and even then with the limitation that the compensation would otherwise be significantly unreasonable. Such a free position for the individual appraisal courts -- without possibility of legal appeal -- would not be in keeping with the purpose of the rule and the demand for full compensation set out in the Constitution.
%\end{quote}
%
%On this basis, the Supreme Court chose to interpret section 5 no 1 in such a way that whenever the conditions were fulfilled, the appraisal courts were \emph{obliged} to award additional compensation, On this basis they found that the property owners in \emph{Kløfta} was entitled to have their compensation looked at again, in a new round before the appraisal courts. The minority agreed in principle, yet did not go as far as the majority, concluding that based on the particular facts at hand section 5 had been adequately considered by the appraisal court in this particular case.\footcite[22]{klofta76} In addition, the majority went quite far in suggesting that ``full compensation'' entitled the owner to {\it market value} compensation, whenever this would result in a higher award than a ``value of use'' approach.\footcite[14]{klofta76} Moreover, they adopted a more narrow interpretation of the (negative) no-scheme rule, whereby public plans not closely related to the expropriation project should not be disregarded.\footcite[15-16]{klofta76} In these matters, the minority took a different view, arguing against market value as a general benchmark and in favour of a broader no-scheme rule.\footnote{\cite[22-23|30-31]{klofta16}.}
%
%The upshot of \emph{Kløfta} was that section 5 no 1 came to be seen as an obligatory rule, leading to compensation having to be enhanced whenever the current use rule led to payments that did not reflect the market value of comparable properties. However, the conditions stated in section 5 no 2 and no 3 were still regarded as relevant, and in interpreting these conditions, a body of law developed whereby the market value rule was applied in a way that would come to involve significant reduction in compensation compared to what would result from practice as it had been prior to the Compensation Act 1973. In this way, the pragmatic approach proved triumphant, not because current use value was introduced as the general starting point, on the contrary, but because a range of new disregards were introduced to reduce the level of compensation in a range of different circumstances. After \emph{Kløfta}, in particular, the following rules were all considered legitimate ways to decrease the level of compensation.

%In section 5 no 3 and no 4, the Expropriation Compensation Act 1973 encoded the following three disregard principles that are all, to varying degrees, still important in compensation law today. 
%
%\begin{enumerate}
%\item Changes in value that are due to the expropriation scheme should be disregarded, both when these are already carried out as well as when they are planned, c.f., section 5 no 2 of the \cite{ca73}.
%\item To the extent that it is regarded reasonable, \emph{increases} in value that are due to public plans or investments should be disregarded, irrespectively of whether or not they have already been carried out, c.f., section 5 no 2 of the \cite{ca73}.
%\item An increased value falls to be disregarded if it results from considering a use of the property which is not in accordance with public plans, c.f., section 5 no 3 of the \cite{ca73}.
%\end{enumerate}
%
%While the \cite{ca73} has now been replaced by the \cite{ca84}, the formulation given in the 

These rules severely limits the level of compensation payments, and in many cases it appears to make the principle of full compensation based on market value rather illusory, even if this was the principle endorsed by the Supreme Court in {\it Kløfta}. On the one hand, the foreseeability test can serve to rule of value arising from any use of the property that is not in keeping with the current public plan. At the same time, the no-scheme rule explicitly encoded in section 5 can be used to also disregard values that are due to this plan, particularly if they are regarded as standing in some relation to the expropriation measure. 

The outcome could easily become, logically speaking, that no compensation can be awarded whatsoever. However, the system tends to revert back to the current use compensation in such cases. For instance, if agricultural land is expropriated for the purpose of a motorway, and it would otherwise appear foreseeable that it might be used for housing in the future, the compensation will usually be based on agricultural use because the value for housing is disregarded under a foreseeability test while possible increases in value due to the motorway plan itself is disregarded under the no-scheme rule.

In practice, with virtually all novel economic activity making use of land is dependent on acquiring new planning permissions, the current use rule will typically be applied as intended by the \cite{ca73}. The main difference is that the rule is not thought of, or described, as an absolute. It rather tends to arise merely as a side effect of other rules.\footnote{A similar point was made in \cite{stordrange94}.} Outcomes that are in keeping with current use thinking will typically be designated as ``full compensation based on market value'' -- the standard phrase adopted in most appraisal judgements -- notwithstanding the fact that the accuracy of such a description depends on the disregards that have been applied.

%The \cite{ca84} was eventually introduced to reflect the principles laid down in \emph{Kløfta}, but it did not in any essentially way change or influence the course of the law that had already been set. Its main purpose was to bring the wording of the legislation more into keeping with how the law was interpreted by the Supreme Court. It explicitly returned to the starting point of the Husaas committee, namely that the compensation should be based on the value of the "foreseeable use" that the owner himself, or an average buyer, might make of the property. But it maintained and endorsed disregard rules no 1-3, except for restricting disregard no 2 to public investments, such that increased value due to public plans currently in place could not be disregarded.\footnote{In this way, the paradox mentioned above, that compensation could become impossible to award because there was no possible basis upon which to calculate it, was avoided.}

The statutory rules do not provide clear guidance as to how the disregard rules should be understood or applied, nor do they consider or resolve the question of when, if ever, they would need to be applied with caution in order not to go against the constitution. However, following {\it Kløfta}, there has been a growing expectation that cases where such issues arise should be resolved by crisp rules, not by the discretion of the appraisal courts. The age when the appraisal courts were considered free to assess cases directly against the Constitution is gone. Rather, an ethos had taken hold where the need to curb the freedom of appraisers, in the interest of ensuring predictability and centralized control, is emphasized.

As a result, difficult cases now routinely end up in the Supreme Court. Here, difficult circumstances are used as the basis for formulating more and more specific rules for special case types. As an example of this mechanism, it is enlightening to consider the case law surrounding the question of whether public plans currently in place are binding when calculating compensation. This rule cannot apply without exception, as recognized already by the \cite{ca73}. But when is it permissible to deviate from it?

The question has arisen in many Supreme Court cases following {\it Kløfta}. \emph{Østensjø} concerned land that was being expropriated for housing purposes, but such that one unlucky owner would only contribute land used for infrastructure that would serve the larger housing project.\footnote{\cite{ostensjo77}.} In this case, the Supreme Court agreed that he was entitled to compensation based on value of his land for housing purposes, irrespectively of the fact that \emph{his} land could not be used in this way according to the plan. However, in many other cases, the disregard rule is upheld even when it is hard to see it as either fair or just, simply on account of it having status as a general rule.\footnote{For instance in \cite{malvik93}. In this case, owners of property used for a motorway were only entitled to compensation based on current agricultural use because the planned motorway-use was assumed binding for the compensation assessment under the market value approach.}

One example is found in \emph{Sea Farm} which dealt with the issue of whether or not the owner of a commercial property should be awarded compensation for the value of investments carried out by the previous tenant.\footcite{seafarm08} There was no doubt that the owner was entitled to these investments, but since the acquiring authority was the only purchaser who was likely to benefit commercially from them, no compensation was awarded for the loss of these investments. This, in particular, followed from a strict reading of the requirement that compensation should be based on the foreseeable use that an "average" buyer could make of the property, encoded in Section 5 of the Compensation Act 1984. Adherence to the wording used in the act seems to have taken priority over an assessment based on the facts of the case. It seems difficult to argue that it would be either unjust or unreasonable, in particular, to compensate the owner for investments that would prove commercially valuable to the acquiring party.\footnote{The decision was sharply criticized by a former supreme judge. See \cite{skoghoy08}.}

In my opinion, this example illustrates how the development of compensation law towards greater reliance on specific rules rather than concrete assessment based on general principles can be harmful. I also threatens to undermine the idea behind the special procedure used to decide appraisal disputes, which has a long history in Norwegian law.\footnote{One might ask if it has status of constitutional customary law, especially since it concerns the mechanism by which a constitutional rule is meant to be upheld.} It also seems to severely underestimate the extent to which compensation rules, when applied to concrete cases, must and should be interpreted based on the context of the case. It seems difficult, if not completely impossible, to achieve social fairness and individual justice by a set of specific rules on the basis of which all legal issues can be resolved mechanically by blind application of such rules. %Moreover, it would be wrong to think that Section ... of the Appraisal Act 1917, encoding the principle that laymen should take part in the decision-making both with regards to legal and technical matters that arose in appraisal disputes.

In the following section, I will turn to waterfalls and hydropower. Interestingly, the compensation practices developed in this regard often deviate significantly from the general approach to compensation. The special approach developed, in particular, as a result of the perceived unfairness of denying benefit sharing altogether in such cases. Hence, looking to waterfalls serves to underscore my point about the importance of a flexible system. 

%The main benefit sharing principle that was developed was known as the {\it natural horsepower method} for calculating compensation for waterfalls following expropriation. It was initially developed by the appraisal courts as an ad hoc approach to ensuring some benefit sharing in hydropower cases. Later, however, many came to regard it as a binding principle of customary law. Gradually, it came to be applied by the courts with little or no regard for how well it suited the circumstances of the case and the changing realities of the hydropower sector. As a result, the method became hopelessly outdated, leading to compensation payments that had little or nothing to do with the actual value of waterfalls for hydropower. 

%Today, while the method has been abandoned for certain case types, it is still applied as the default rule for compensation waterfalls.
%
%
% address this issue in more detail, and we will argue for a different conceptual approach to compensation law, grounded both in the procedural tradition of appraisal courts and the more subtle parts of the absolutist and pragmatic theoretical traditions. It seems to me that the most striking lesson that should be drawn from considering the history of Norwegian compensation law is that a \emph{contextual} view of compensation has been a common denominator that both the absolutist and pragmatist camps have endorsed. Unfortunately, this common element was overshadowed by political conflict regarding the weighing of different values. However, there can be little doubt that social fairness and individual justice should \emph{both} to be regarded as important objectives for compensation rules. Moreover, while they may sometimes be opposing, they need not be, and their exact relationship depends largely on the circumstances. It seems to us that it is simply inappropriate to let particular political sentiments regarding their relationship and relative importance, sentiments that are usually dependent on the particulars of the prevailing political, social and economic conditions, dictate the development of the legal framework for resolving compensation disputes.
%
%Considering current trends and recent issues in expropriation law, particularly related to commercial expropriation, further suggests that a different perspective is needed on this matter. In particular, we believe it is time to recall the idea of the independent and impartial discretion of the appraisal court, relying on the good common sense of laymen as well as the legal expertise of judges. The appraisal courts should in our opinion be set with the task of more actively evaluating how fairness and justice is best served in individual cases, at least if the overall goal is truly to arrive at a socially fair and individually just compensation system. We discuss this idea in more detail in the final section below.

\section{``Natural horsepowers''}

Following the introduction of a general expropriation authority covering waterfalls in the early 20th century, the question of how to value waterfalls came before the appraisal courts. The regulatory regime that was established made private commercial development difficult or impossible, and this in turn meant that the commercial market for waterfalls all but disappeared. Hence, a strict application of the no-scheme rule could lead to no compensation being paid at all. Arguably, a waterfall had no value to anyone except the acquiring authority, since no alternative development scheme could be regarded as foreseeable.

The appraisal courts did not follow this point of view to its logical conclusion. Instead, they introduced a theoretical formula for calculating waterfall compensation. In effect, this method served to create an artificial market for waterfalls, controlled by the appraisal courts. Initially, this artificial market was modelled on the actual market that had existed prior to the regulatory reform. Over time, however, the waterfall ``market'' would slide further and further into the legal sphere, away from the physical and commercial reality of hydropower development.

The key notion used to determine the price of a waterfall on this market was that of a {\it natural horsepower}, a gross measure of electric effect.\footnote{A horsepower, of course, is an old-fashioned unit of effect which is still sometimes used, e.g., in relation to cars. In the context of electricity, it is replaced by {\it Watts}, such that 1 horespower (hp) = 745.69 Watts.} As I mentioned in Chapter \ref{chap:4}, the lack of a national grid at this time meant that the value of a hydropower plant was largely determined by the stable effect that the plant could deliver, not the total amount of electricity that could be produced. This, in turn, was a function of the degree of water regulation implemented by the hydropower developer. 

To simplify the calculation, the natural horsepower of a waterfall was introduced as a gross estimate of the stable effect that could be ensured given a choice regarding the level of regulation of the watercourse. The value of the waterfall itself was then determined by fixing a price per natural horsepower. This price was set on the basis of prices paid for other waterfalls, with some adjustments typically carried to take into account the level of cost and benefits associated with the hydropower project in question.

As I remarked in Chapter \ref{chap:4}, the notion of a natural horsepower is used in other contexts as well, for instance to determine what kind of licenses a development project requires. The use made of it to calculate compensation for waterfalls had no legislative basis, but arose as a result of the appraisal courts' efforts to calculate market prices. After the actual market based on the natural horsepower method disappeared, the method stuck and was applied as a matter of custom.\footnote{See generally the description of the history of the method given by the Supreme Court in \cite{uleberg08}.}
%
%
%prove shockingly unfair to owners of waterfalls. Presumably, since waterfalls could not be exploited for any significant commercial gain except through hydro-power exploitation, disregarding the hydro-power scheme when calculating compensation could lead to nil or close to nil being awarded to the owner. But this was not seen as an acceptable outcome, and instead the Norwegian courts introduced a special method to compensate waterfalls that gave the owner a \emph{share in the value of the hydro-power scheme} for which expropriation was taking place.
%
%Norway did not at this time have any legislation specifically aimed at regulating compensation following expropriation, and when formulating the special rules for compensation of waterfalls, the Norwegian courts seems to have relied on an analogical application of the gross valuation techniques introduced in the Industrial Concession Act 1917 and the Watercourse Regulation Act 1917.\footnote{Act No. 17 of 14 December 1917 relating to Regulations of Watercourses and Act No. 16 of 14 December 1917 relating to Acquisition of Waterfalls, Mines and other Real Property}. Neither of these acts were aimed at compensating owners, but they relied on methods for assessing the potential and significance of hydro-power projects with respect to the question of whether or not a special concession from the State was required.\footnote{To acquire the waterfall and the right to regulate the water-flow respectively.} In effect, by relying on the methods of valuation introduced there, the compensation mechanism that was introduced deviated completely from the "value to the owner" principle. On the other hand, it also closely mimicked the manner in which owners of waterfalls would be compensated on the market in the early days, prior to the introduction of our concession laws, when speculators would pay for waterfalls on the basis of what they assumed to get out of them in subsequent hydro-power projects.

In the Supreme Court case of \emph{Hellandsfoss}, some 80 years after it was first introduced, the natural horsepower method was described and put into context as follows:\footcite[1599]{hellandsfoss97}
\begin{quote}
The principle set out in the Compensation Act, Section 5, is that compensation should be determined on the basis of an estimation of what ordinary buyers would pay for the property in a voluntary sale, taking into account such use of the property as could reasonably be anticipated. For waterfalls, however, this often offers little guidance, and the value of waterfall rights have traditionally been determined based on the number of natural horsepowers in the fall, which are then multiplied by a price per unit. The unit price is determined after an overall assessment of the waterfall, including the cost of the scheme, its location, and levels of compensation paid for similar types of waterfalls in the past. The number of natural horsepowers is calculated by the formula ``natural horsepower = $13.33 \ \times \ Qreg \ \times \ Hbr$'', where $Qreg$ is the regulated water flow and $Hbr$ is the height of the waterfall.
\end{quote}

In this formula, $Qreg$ represents a quantity of water, measured in cubic meters per second (m3/sec), while $Hbr$ represents height measured in meters. The number $13.33$ is the force of the gravitational pull on earth measured in horsepower. 

In the standard account of the natural horsepower method, it is often said that the number of natural horsepower in a waterfall is a measure of gross effect, giving us the amount of ``raw'' power in the waterfall.\footnote{See \cite{vislie02}.} This is not accurate. Indeed, from the quote given above it is clear that the natural horsepower does {\it not} depend only on the nature of the waterfall. It also depends on the specific plans for development presented by the expropriating party. In particular, the quantity $Qreg$ is entirely a function of how the developer {\it chooses} to develop the waterfall, in that it measures the ``regulated water flow''.\footnote{In addition, the quantity $Hbr$ depends on the height over which the developer plans to make use of the water. The development potential that the owner is deprived of can amount to either more or less than this, depending on the nature of alternative schemes.}

In {\it Hellandsfoss}, the Supreme Court itself glosses over this point when it speaks of the ``natural horsepower in the fall''. It would be more accurate to speak of the natural horespower of the particular development scheme benefiting from expropriation.\footnote{Regulation of a watercourse can involve building a reservoir and/or installations that transfer water from one river to another. Then, if there is excess water, for instance due to flooding, water can be stored in the dam for later use. When there is no drought, the stored water can be released. In this way, it becomes possible to even out the water-flow over the year. Today, however, many hydropower plants, particularly smaller ones, involve little or not regulation. Instead, such run-of-river scheme operate by harnessing energy from whatever water is present in the river at any given time.}

But how exactly is the regulated water flow determined for the purposes of compensation estimation? In section 2 of the \cite{ica17}, it is said that the regulated water flow is to be determined ``on the basis of the increase of the low water flow of the watercourse, which the regulation is supposed to cause beyond the water flow which is considered foreseeable for 350 days a year.'' Hence, the idea is that only the {\it increase} in water flow is to be measured. This means that if the developer proposes a run-of-river project with no regulation, then the natural horsepower of the project will automatically be $0$.\footnote{In fact, things could become even worse for the owner, since the proposed project might lead to $Qreg$ becoming a {\it negative} number. This follows from section 10 of the \cite{wra00}. Here the NVE is given the power to compel the owner of a hydropower scheme to ensure that a certain quantity of water is always allowed to pass through the intake of the plant. This flow of water is typically referred to as the {\it minimum water flow}, but is sometimes used in place of the low water-flow before regulation when calculating the natural horsepower of a project. The idea behind imposing a minimum water-flow is to reduce the negative environmental impact. For many run-of-river schemes, the minimum water flow ordered by the NVE is higher than the low water flow after regulation. Hence, if the minimum water flow is subtracted from the low water flow after regulation, the result is a negative number. That is, one might end up with a {\it negative} $Qreg$.} But this outcome was averted in practice by an {\it ad hoc} adaptation of the traditional method. In relation to compensation, it became established practice to omit the deduction of the previous water flow, so that one would use the entire low water-flow after regulation as $Qreg$. That is, the quantity used for $Qreg$ when computing the natural horsepower of a waterfall for purposes of compensation is the estimated amount of water that is present in the river for at least 350 days a year after regulation.

This means that the natural horsepower of a development scheme has little bearing on the amount of energy that will actually be harnessed from it. Today, modern electricity generators can produce electricity at varying levels of effect, depending on the water-flow of the river. But the water-flow is not a constant as assumed by the natural horsepower formula. Rather, it varies considerably over the year.

As a result, the natural horsepower of a regulation does not have much to do with the value of neither waterfalls nor hydroelectric plants.\footnote{See generally \cite{sofienlund08}.} Indeed, the annual income of a hydroelectric plant has nothing to do with natural horsepower, it is solely a function of the price paid per kilowatthour and the total number of kilowatthours harnessed over the year (kWh/year).\footcite{sofienlund08} The amount of energy generated in a power plant could be measured in other units than kWh, e.g. in terms of the amount of horsepower-hours per year. But the important point to keep in mind is that an energy producer gets paid for the amount of energy he can deliver, \emph{not} the effect he can maintain in his station over a long duration of time. %Hence, even if if we uwould se kilowatt instead of horsepower and talk of the natural kilowatt of a hydropower plant, the quantity we are discussing is the same, and still has little or no bearing on the value of the waterfall.

Talking of natural horsepower therefore serves to give a skewed picture of the potential of a waterfall, especially for run-of-river projects. It is not unusual that the low water-flow in a river amounts to only about 3-5 \% of the average water supply. In modern hydropower projects, one would expect 70-80 \% of this water-flow to be harnessed for energy production even in the absence of any regulation. Hence, in these cases, the natural horsepower method, as it was traditionally applied, would only compensates the owners for about 5 \% of the energy that would actually be harnessed from their waterfalls.\footnote{sofienlund08}

This observation, which is trivial given a rudimentary understanding of the energy business, was not made in the context of expropriation until late in the 1990s. Moreover, the point was raised against the advice of legal experts who regarded the established method as a principle of customary law.\footnote{In the aforementioned case of {\it Hellandsfoss}, for instance, a local owner raised the issue with his legal council, who advised against raising it as an issue before the appraisal courts. The owner listened to his legal council, resulting in a compensation payment that is only a small fraction of what he would be entitled to under the method used in some more recent cases, e.g., in \cite{sauda08}. Source: Private correspondence.} At the same time, both engineers and government officials were well aware of the inadequacies of the method, as illustrated for instance by the following passage from a governmental report made in 1991:\footnote{\cite[19]{otprp50}, discussing the notion of natural horsepower in connection to the uses made of that term in other parts of the law.}

\begin{quote}
The Ministry of Petroleum and Energy has considered moving a proposition for changing the hydrological definitions in the Industrial Concession Act 1917 and the Watercourse Regulation Act 1917. Today the act uses a calculation method based on an increase in regulated water-flow, i.e. that of natural horsepower.[.......] The hydrological definitions of these acts, supposed to indicate how much electricity can be generated, were made on the basis of technical and operative conditions differing very much from contemporary circumstances. In implementing the definitions referred to above one has tried to adapt to the new technological realities of the present day. Therefore, in practice, a calculation based on current production is used instead. From several quarters, particularly the Association of Waterfall Regulators, there has been raised a strong wish to authorize this practice by altering the definitions of the relevant laws. The Department of Oil and Energy agree, but have not as yet made a sufficient elucidation of the issues to be able to move a proposition of alteration of these acts.
\end{quote}

The quote shows that in administrative practice, it had become common to deviate from the definition of a natural horsepower, since it no longer reflected a relevant figure. A similar move would not be made in the context of expropriation for another 20 years.\footnote{A ``natural horsepower'' calculation modified along the lines described by the Ministry in 1991 is now sometimes used also in compensation cases, following its adoption for some of the waterfalls that were expropriated in the case of \cite{sauda08}.}

Within the ranks of the specialized water authorities, the inadequacies of the natural horsepower method had been known even longer. Here it had also been noted that the method did not give rise to realistic estimates of the value of waterfalls. The first record I can find of such an admission dates back to 1957, from an article written by the director at the NVE which was published in their internal newsletter.\footnote{See \cite{....}. The director even went as far as to illustrate a different method, which would also be outdated given today's regulatory regime, but which would reflect contemporary \emph{actual} valuations, used by the NVE itself.}

Considering the physics behind the traditional method is enough to reveal that it fails to give rise to valuations that reflect the value of waterfalls, under any reasonable set of assumptions about the correct general compensation principles one should adopt. Important in this regard is the fact that  the method relies on data that depends entirely on the expropriating party's project. The compensation to the owner depends not on their loss, but on the technical details of the project that the expropriating party proposes. This clearly deviates from even a narrow interpretation of the no-scheme principle.

However, while the idea of compensating the owner of waterfalls by a price per natural horsepower is fundamentally flawed at the theoretical level, there are even more serious concerns that arise when one begins to consider the way in which the unit price has been determined {\it in practice}. The traditional approach to this question has had a particularly dramatic effect on the level of compensation payments. 

In case law based on the traditional method, it is often said that the price set per natural horsepower is set according to ``market price'' for waterfalls. But for the most part, what this means is that the court looks to prices awarded in earlier compensation cases. This practice gave rise to a price level that was entirely artificial. It reflected, more than anything else, the power balance between buyer and seller in the courtroom. It was certainly no genuine market value, even if it was described as such. This has become very clear after the adoption of new, genuinely market-based, methods in recent years.\footnote{See generally \cite{larsen08}.}

Indeed, while the unit price for a natural horsepower did increase somewhat during the first 80 years that the traditional method was used, this increase neither reflected the value of hydropower in particularly nor the level of inflation in general.\footnote{See \cite{sofienlund08}.} Moreover, while the price-level was determined by the courts, some voluntary agreements were also made on the basis of the same method. These could then in turn be used to back up the claim that this was a genuine market-based valuation principle. In this way, it became possible to legitimize an increasing imbalance of power between owners and purchasers. In the end, this imbalance became extreme.

For instance, in 2002 a waterfall belonging to local landowners in the rural community of Måren, located in south-western Norway, was sold for the sum of kr 45 000 (roughly £ 4500), based on traditional calculations.\footnote{Source: private correspondence.} The waterfall has now been exploited in a small-scale hydro-power plant belonging to the large energy company BKK, with annual energy output of 21 GWh.\footnote{$http://www.bkk.no/om_oss/anlegg-utbygging/Kraftverk_og_vassdrag/andre-vassdrag/article29899.ece$} For comparison, I mention that in the case of \emph{Sauda}, where a more realistic market-based method was used, the owners received a compensation which totalled about 1 kr/kWh annual production.\footnote{LG-2007-176723 (I acted as council for some of the owners in this case).} Applied to the Måren case, this would take the compensation from kr 45 000 to kr 21 000 000. That is, the price would have been almost 500 times higher.\footnote{In fact, the Måren waterfalls were cheaper to exploit, so in reality, one would expect that the new method applied to Måren would yield even greater compensation per kWh. I also remark that the value awarded in \emph{Sauda} was market-value, not value of use. It was assumed, in particular, that the owners would have to cooperate with a ``professional'' energy company to develop hydropower. This, in effect, halved the compensation awarded, since the Court's decision was based on the premise that the professional company was willing to pay about 50\% of the profit as rent to the owners.}

The case of Måren illustrates an important point, namely that when the traditional method was used, and described as the ``market value'' of waterfalls by the courts, this became a self-fulfilling prophecy. The prices paid in voluntary transactions were influenced by the practice adopted by the courts far more than the other way around. This, indeed, appears to be a general danger in cases when expropriation is widely used for some particular purpose. The prices paid can easily be kept artificially low by developers making use of expropriation as soon as prices begin to rise. In that way, by relying on what is ostensibly ``market value'' compensation, an artificial price level can be established and maintained. 

I mention that in a setting where the owners are politically powerful and can exert undue influence on the compensation process, the effect can be reversed, so that the ``market based'' approach leads to inflated compensation levels, including elements of holdout value. The general point is that the market approach can be turned to the advantage of the most resourceful and powerful groups, particularly in situations when expropriation is widely used for a particular kind of development. In such cases, a market-based approach is not as politically neutral and ``objective'' as its proponents tend to argue.


The potential severity of this mechanism is nicely illustrated by the case of Norwegian waterfalls. In my opinion, preventing such a mechanism from undermining the fairness of a compensation regime is a main challenge associated with regulatory systems that presuppose extensive use of expropriation. Moreover, in case expropriation is used to further economic development by commercial actors, it is likely that the effect will be detrimental to owners, while creating increased financial incentives for developers to favour expropriation. In this way, a vicious circle is established which can make it hard to break out of the ``expropriation loop'', even though alternatives exist that fulfil the same public interests while ensuring far more equitable forms of benefit sharing and participation.

\section{{\it Kløvtveit} and {\it Otra Kraft}}

Following the liberalization of the Norwegian energy sector in the 1990s, the traditional method came under increasing pressure. It was argued to be unjust by owners and it was held to be illogical by engineers working on developing small-scale hydropower.\footnote{See generally \cite{dyrkolbotn96}.} Eventually, legal professionals followed suit and came to the realization that established compensation  rules based on market value could be applied.\footnote{See generally \cite{larsen06}.} 

Indeed, a new market for waterfalls had begun to develop at this point, following the increased interest in small-scale hydropower and the formation of new companies specializing in cooperating with local owners. For transactions of rights to waterfalls taking place in this market, the traditional method of valuation was not used. In fact, waterfalls were rarely sold at all, but rather leased to the development company for an annual fee. Typically, this fee was calculated by fixing a percentage of the energy produced during the year, and compensating the owners of the waterfall by multiplying this with the market price for electricity obtained throughout the year, possibly deducting production specific taxes, but with no deduction of other cost. In effect, owners would get a fee corresponding to a set percentage of annual gross income in the hydro-power plant.\footnote{See \cite{larsen06}.}

Usually, such a fee entitles the owners to 10-20\% of the income from sale of electricity, depending on the cost of the project. Moreover, it is common that the owners are entitled to up to 50\% of the income derived from so-called \emph{green certificates}, a support mechanism for new renewable energy projects, corresponding to the Renewables Obligation in the UK.\footnote{See http://www.ofgem.gov.uk/Sustainability/Environment/RenewablObl/ for further details.} Essentially, and somewhat simplified, the scheme allows the energy producer to collect a premium on his sale of electricity, which, owning to its ``green'' status, is valued more highly by buyers (usually electricity suppliers), who are required to ensure that a certain proportion of the energy they offer to their customers is considered green. In Norway, such a scheme has been talked about for years, but was only put in force in 2012.\footnote{http://www.regjeringen.no/en/dep/oed/Subject/energy-in-norway/electricity-certificates.html?id=517462} Currently, energy producers can claim a premium of about 2 pp per KWh per year, meaning that about a third of the annual income for new renewable energy projects comes from the sale of green certificates.\footnote{While the premium must be expected to go down somewhat as the certificate market matures and more energy producers acquire "green" status, it will certainly remain an important source of extra income for renewable energy producers also in the future.}

Since these leasehold agreements tie compensation to the fate of the hydropower project, several questions arise when attempting to estimate a present-day value of a waterfall on this market. The valuers first have to determine what the most likely project looks like. Then they have to determine what the annual production will be. After this, they must assess the cost of constructing the plant, something that will in turn make it possible to estimate the level of rent likely to be paid to the waterfall owners. Then, since this rent is set as a percentage of the income from sale of electricity and energy certificates, the need arises to stipulate future prices, usually for as long as 40 years (the usual length of a leasehold). Finally, a present-day value can be calculated based on this cash flow.

The appraisal courts began to use just such a model around 2005. The first case of this kind to reach the Supreme Court was \emph{Uleberg}. In the appraisal Court of Appeal, the lay appraisers overruled the juridical judge and awarded compensation based on the new method. The Supreme Court ordered a retrial on a technicality, but it also commented that it supported the adoption of the new method in cases when \emph{alternative} small-scale development was deemed a \emph{foreseeable} use of the waterfall in the absence of the expropriation scheme.\footnote{\cite{uleberg08}.} Since \emph{Uleberg}, the new method has continued to be used in many cases before appraisal courts.\footnote{See generally \cite{larsen06,larsen08,larsen11}, a series of Norwegian papers discussing the new method.}

It is important to note that it was the lay appraisers that pushed for a new method initially, against the judgement of the legal professionals. This shows, in my opinion, that the old system of lay judgement in appraisal disputes still plays a role in Norway. Moreover, it demonstrates that it has positive qualities that should be preserved in the future. However, the new method is certainly not without its own problems. 

Unsurprisingly, it tends to lead to a rather protracted process of valuation, mostly dominated by experts. Moreover, given all the uncertain elements of the calculation, it is typical that the opposing parties produce expert witnesses that diverge significantly in their valuations. While this can be problematic, the fundamental \emph{legal} challenge arises with respect to the no-scheme rule. In particular, what hydropower scheme should the compensation be based on? Several questions arise, as listed below.

\begin{itemize}
\item (1) Is it foreseeable that the waterfall could be used in a hydropower project in the absence of a power to expropriate?
\item (2) If the answer to question (1) is yes, what would such a scheme look like?
\item (3) Is it foreseeable that such a scheme would obtain the necessary licenses?
\item (4) Does the no-scheme rule imply that the project benefiting from expropriation cannot be regarded as a foreseeable scheme for the purpose of compensation?
\item (5) Is the fact that the scheme underlying expropriation obtained a development license to be regarded as evidence that no other scheme would be likely to obtain such a license?
\item (6) How should compensation be calculated if it is determined that no hydropower scheme would have been foreseeable in the absence of the power to expropriate? 
\end{itemize}

In some cases, for instance when the project benefiting from expropriation is not commercially viable but is carried out for public purposes with the help of special state funding, the answer to question (1) might be no. However, in most cases, the question will be answered in the affirmative, since the scheme benefiting from expropriation already serves as an indication that the waterfall can be commercially harnessed for energy. However, here the no-scheme rule comes into play and creates severe difficulty once we reach question (2). For what kind of scheme can be assumed foreseeable all the while we are obliged to disregard the scheme underlying expropriation? 

In most cases so far, the owners have claimed that compensation should be based on the value of a small-scale hydropower scheme. Since such a scheme is likely to be clearly distinct from the expropriation scheme, one might think that the no-scheme rule will not come into play. This, however, is not necessarily the case. It appears, in particular, that the answer to question (3), asking about the likelihood of obtaining licenses, will still depend on how one views the no-scheme rule. It seems, in particular, that anyone who answers question (5) in the affirmative, will be inclined to say that the alternative project could not expect to get planning permission. This is so, such a person might argue, precisely \emph{because} licenses were granted to the expropriating party. This line of reasoning has been consistently advocated by the large energy companies, ever since the new method emerged.\footnote{See, e.g., \cite{klovtveit11,otra11,otra13}. The argument is often sugar-coated by pointing to the reasons underlying the decision to grant a license -- typically energy efficiency -- rather than by focusing on the formal license itself. In this way, one arrives at an interpretation of the no-scheme rule whereby the scheme can perhaps be said to have been disregarded even though one still takes into account reasons why it should be preferred over other schemes.}

Then the question arises: Is someone who reasons like this at odds with the no-scheme rule? It would seem so, but remember the earlier discussion on the no-scheme rule in Norwegian law, where I noted that the rule has tended to be applied much more narrowly along its positive dimension. Following up on this, it can be argued that while the expropriation scheme is to be disregarded for the purpose of compensation valuation, the regulation underlying the scheme -- or at least the rationale behind this regulation -- is nevertheless to be taken into account. If this point of view is adopted, then the conclusion can easily become that alternative development is to be regarded as unforeseeable. The reason, moreover, will be precisely the fact that the expropriation scheme received a development license. 

Indeed, this line of reasoning was given a stamp of approval in the recent Supreme Court case of \emph{Otra II}.\footcite{otra13} Here, the presiding judge made the following remarks, quoting Gulating Lagmannsrett (the appraisal Court of Appeal), expressing his support, and adding a few comments of his own.\footcite[]{otra13}

\begin{quote}
"[....] The Court of Appeal finds it difficult to distinguish this case from other cases when it has been established that alternative development is not foreseeable. It does not seem relevant whether this is the case because the alternative is not commercially viable or because the alternative must yield to a different exploitation of the waterfall" 
I agree with the Court of Appeal, and I would like to add the following: As the survey of the general principles have shown, it is assumed, both in the Expropriation Act, Sections 5 and 6, and in case-law, that only the value of a foreseeable alternative should be compensated. This starting point means that it would be in breach of the general arrangement if a waterfall that can not be used in foreseeable small-scale hydro-power was to be compensated as if it could be put to such use.
\end{quote}

Having used the development license granted to the expropriating party as evidence that alternative development was unforeseeable, the Court needed to answer question (6) by coming up with some alternative way of compensating the owners.  To do so, the Court was again faced with considering the implications of the expropriation scheme. One possibility would be to ensure that the negative and positive dimensions of the no-scheme rule came to be aligned with one another. That is, as the expropriation scheme was used to rule out alternatives, one might then proceed to use it also as the basis for valuation. Indeed, this is what the Supreme Court did. But at this point, the adherence to the no-scheme rule and a market-based approach spelled doom for the waterfall owners. As the presiding judge reasoned:\footcite[]{otra13}

\begin{quote}
Based on the arguments presented to the Supreme Court, I find it safe to assume that there does not today exist any market for the sale and leasing of waterfalls for which alternative development is not foreseeable, but where the waterfalls can be used in more complex hydro-power schemes. The appellants have not been able to produce documents or prices to document the existence of such a market
\end{quote}

The implicit assumption is that in order to value the waterfall according to its potential for hydropower production, a market needs to be identified. It is \emph{not} considered sufficient that the scheme for which expropriation takes place is itself a hydropower project, on the basis of which the  waterfall value could be assessed following exactly the same steps as in the new method. I also remark that it is very hard to imagine how a market of the kind asked for here could ever develop. After all, any alternative buyers are, by the Court's reasoning a few lines earlier, effectively excluded from being taken into account. In this case, if there was to be a market, it would presumably have to be one that emerged entirely out of the benevolence of the expropriating party.

In fact, the Supreme Court's reasoning in \emph{Otra II} serve as an excellent example of the type of reasoning that makes the no-scheme rule highly problematic for cases of expropriation that benefit commercial schemes. On the one hand, the rule can be used to argue that the inherent value of the scheme itself should not provide a basis for calculating the compensation. On the other hand, it can be used to argue that alternatives must be disregarded in so far as they represent the same kind of exploitation as the expropriation scheme, because they are inferior to it according to the state.

When taken to its logical conclusion, this line of reasoning leads to an offensive result; The commercial value of the property is not to be compensated because the optimal commercial use is the use that the expropriating party aims to make of it. Note that the conclusion is not just that this optimal value, inherent in the scheme, should not be compensated. No, the conclusion in \emph{Otra II} was that \emph{no} compensation could be estimated for any use of the same \emph{kind}, since such use was not foreseeable, owing to the absence of a market.

It is certainly possible to argue that this decision represent a misguided application of the no-scheme rule. In effect, the Supreme Court allowed the licences given to the expropriating party to act as evidence that alternative development was unforeseeable, while it used the no-scheme rule to argue that the hydropower scheme for which this planning permission was given could not itself form basis for compensation payment based on market value.  On the other hand, it seems that even if we disregard the scheme completely, it is unnatural to base the compensation payment on the value of a hydropower scheme that is less beneficial, both commercially and in terms of resource efficiency, than the scheme for which expropriation takes place. Such a scheme would not, one must presume, \emph{actually} have been carried out, regardless of the questions of whether or not it would have been given licenses in the absence of a preferable scheme. 

However, it is not seem particularly difficult to determine what would have been a foreseeable use in these cases, if one assumed only that the power to expropriate had not been granted. If so, it would seem all but certain that a scheme corresponding closely to that underlying expropriation would be implemented. This scheme, however, would be carried out on the basis of sharing the commercial benefit with the owners, not on the basis of expropriation. 

But in \emph{Otra II}, this line of thought was also rejected.\footnote{Although this was in part due to the point not having been argued before the Court of Appeal.} Instead, the Court states that a return to the traditional method is in order. However, they do not apply it in the traditional way. Rather, they sanction a modified version of it that moves away from compensation based on the level of stable effect towards compensation based on average effect.\footnote{That is, they replace the low water-flow by the average water-flow in the definition of Qref, c.f., Section \ref{sec:nathp}.}In addition, they also sanctioned the use of a significantly increased unit price compared to earlier times.

What to make of this? In fact, it seems hard indeed to make sense of since, effectively, by relying on the traditional method, the Supreme Court contradicts its own conclusion that compensations should be based on market value. Instead, they rely on a method that, in effect, is based on an attempt to quantify the value of the waterfall as it is being used by the expropriating party in his project. However, by relying on a technical method that has been completely outdated, it becomes difficult to assess the outcome properly, at least for a non-expert. This is so even after the modifications have been implemented, which make the method appear somewhat less irrational from a physical point of view.

But it is still noteworthy that the Supreme Court prefers the obscurity of the traditional method, as an established custom, over the explicit conclusion that it simply is not tenable to adopt the ``value to the owner'' principle in cases like this, as least not as that principle is construed in Norwegian law.

In any event, I think there is good reason to be critical of the Supreme Court for sanctioning the view that alternative development was unforeseeable in {\it Otra II}. Still, it is not possible to escape the fact that this reflects a general tendency in Norwegian law, whereby the positive dimension of the no-scheme rule is much weaker than the negative part. Even if it appears unreasonable, it might very well be a correct application of national law. Moreover, it could very well have been that alternative development was unforeseeable for \emph{some other reason}, for instance because the only commercially viable exploitation was the scheme planned by the expropriating party. In this case, the problem of how to compensate the owners in the absence of an alternative form of exploitation would still arise. It is this question, in particular, which seems entirely unsatisfactorily resolved under an application of a ``value to the owner'' principle.

This is witnessed by \emph{Otra II}, and, in fact, it appears that the Supreme Court, in their decision  \emph{not} to follow their own reasoning to its logical consequence, makes quite a powerful statement. For all intents and purposes, the Supreme Court \emph{rejects} the "value to the owner" principle, but they obscure this by wrapping it up in the traditional method, which is deeply flawed. However, the problem it attempts to solve appears significant, and it pertains directly to the question discussed more generally in Section \ref{sec:noscheme}, namely how to compensate owners that loose their land to commercial schemes. 

It seems that even the fiercest supporters of limiting owners' right to compensation tend to find it too offensive to apply this principle when it leaves the owners with no form of compensation in cases when they are forced to give up property to purely commercial undertakings. Indeed, such a practice would surely also be in breach of the human rights law. In these cases, the subjective aspect of the ``value to the owner'' principle is impossible to maintain. If the commercial value falls to be disregarded for no other reason than the fact that the state happens to have granted planning permission to the expropriating party rather than the owner, this is not only dubious with respect to human rights protecting property, but also appears to be a case of \emph{discrimination}, e.g., as prohibited by ECHR Article 14.

The problem does not arise when the buyer sees value in the property that is of a different \emph{kind} than that realizable by \emph{any} private owner. In this case, the rule simply states that the owner should not be able to demand that ``public value'' is transformed into commercial value just for him. This appears like a reasonable principle. But when there is commercial value already present on the "public" side of the transaction, it seems completely unwarranted that the public should be allowed to transfer this value from the owner to someone else without compensation. Thus, it seems that more accurately and acceptably, the ``value to the owner'' principle should be thought of as a ``commercial value'' principle. It seems, in particular, that the principle need to be stripped of any suggestion that a preferential financial position is to be awarded to whoever benefits from expropriation.\footnote{Exceptions might be possible to imagine, but, one would think, only when they can be construed as falling under the ``public value'' banner in some way.}

It seems unfortunate that this aspect has not been made explicit, and the difficulties that arise in the absence of this nuance are nicely illustrated by the case of Norwegian waterfalls. Still, as the case of \emph{Otra II} indicates, an interpretation of the ``value to the owner'' principle along less offensive lines is in reality already in place with regards to Norwegian hydro-power. Here it seems that ``value to the owner'' has in fact \emph{never} been applied in the traditional way. Hopefully, rather than obscuring this fact by relying on an unsatisfactory and artificial method for calculating the compensation, the future will see further developments that recognize the need for new principles. 

It should be recognized, in particular, that as the law has been applied for the last 80 years, despite its grave flaws and injustices, there has always been an implicit recognition in Norwegian law that the owners of waterfalls are \emph{entitled to their share} of the commercial benefits of hydropower. 
In fact, in the recent Supreme Court case of \emph{Kløvtveit}, a novel approach along the lines I am advocating was applied in circumstances similar to that of {\it Otra II}.\footcite{klovtveit11} The conclusion here too was that alternative development was not foreseeable. However, unlike in \emph{Otra II}, the lay appraisers in the Court of Appeal had compensated the owners based on the fact that they regarded it as foreseeable that in the absence of the scheme, the waterfalls would have been exploited in exactly the same way, except that it would have happened in the form of \emph{cooperation} between the owners and the expropriating party. By this line of reasoning, the Court effectively seems to have adopted a more modern ``commercial value'' principle, to replace the traditional method. 

For commercial projects, it seems that in the absence of a power to expropriate, any rational buyer would look to cooperate with the owners. This would not necessarily be a safe assumption to make for non-commercial projects. Such projects may fail to provide the necessary incentives for cooperation, even though they should nevertheless be carried out in the public interest.

I mention that \emph{Kløvtveit} was discussed in \emph{Otra II}. But the presiding judge chose to focus on what he regarded as the ``practical problems'' associated with the prospect of cooperation and a compensation award calculated on this premise. The cooperation model was not the center of attention in the case, however, so one can only hope that \emph{Kløvtveit}, rather than \emph{Otra II}, will become the influential precedent for future cases.

\section{Conclusion}

In this chapter I have presented the current compensation regime associated with waterfalls and I have related it to the broader question of how compensation should be calculated when commercial companies benefit from the property that is taken. I focused particularly on the no-scheme principle, which plays an important role in this regard in many jurisdictions, including in Norway. But I also emphasized another aspect particular to the Norwegian system, namely its reliance on the judgement of lay appraisers. 

I noted that the appraisal courts would typically operate largely unconstrained by specific evaluation rules, as they were directly guided by the Constitution and its requirement that ``full compensation'' had to be paid. I argued that this system was both flexible and capable of facilitating broad fairness considerations. The notion that constitutional absolutism was a rigid system, I argued, is largely unfounded. The system had a procedural flexibility that should not be underestimated and which served as a counterbalance to its seeming adherence to a strict dogma.

In fact, when moving on to consider the case of waterfalls in more depth, I noted how this flexibility was used to great effect. It allowed the appraisal courts to ensure that some benefit sharing was maintained in hydropower cases even after the regulatory system was transformed so that benefit sharing following expropriation would be hard or impossible to achieve in a system based on a legal formalization of the no-scheme principle. For over 80 years, the courts happily deviated from it entirely when awarding compensation for waterfalls. Remarkably, this practice continued even after legislation was passed that provided much more specific guidelines to the appraisal courts, and which seemingly enforced a strict no-scheme principle in Norwegian law.\footnote{More generally, however, I noted how this legislation, and the Constitutional battles that followed it, has lead to a development whereby the appraisers are somewhat marginalized and the Supreme Court itself has assumed greater power in directing them, by providing their own interpretation of a body of legislation that contains many specific rules that are hard to apply to concrete cases in a uniform fashion.}

However, as the appraisal courts were marginalized by increasing levels of top-down control, first by the legislator and later by the Supreme Court, the method that was developed to compensate waterfalls would itself develop into a fixed and rigid rule. It was not adapted, in particular, to reflect technological and economic progress. Since these were particularly rapid and ground-breaking in the energy sector, the result was a very severe mismatch between the real value of waterfalls and the compensation paid following expropriation. 

In the final part of the chapter I then considered recent cases where the traditional method has been abandoned in favour of a market-based approach which is based on the general rules governing compensation today. I found that while these cases tend to result in payments that more closely reflect actual commercial values, they raise severe problems of their own. Here, in particular, the no-scheme principle re-emerges on the scene with full force, becoming a very effective tool for those who seek to argue that hydropower development is not a fruit of property but belong to those who obtain expropriation and development licenses from the state. 

If such arguments are successful, the market-value approach can lead to worse outcomes for local owners of waterfalls than what they would be entitled to under the traditional method. The deeper question that arises, of course, is the following: what is equitable benefit sharing in these cases, and how can it be ensured? A second question is whether owners can in fact {\it demand} some level of benefit sharing on the basis of human rights law. This question is now coming into focus in Norway, as the Supreme Court's decision in {\it Otra II} has been brought before the ECtHR.
 
It is my opinion that the best way to ensure benefit sharing under a compensatory approach is to revive the old system of an independent appraisal procedure relying on the discretion of lay people from the local area. The most important aspect of this, I believe, is that it enhances the democratic legitimacy of the compensatory approach. It is clear that there is a great deal of uncertainty in the kinds of calculations one must engage in to assess the commercial value of a waterfall. Therefore, the temptation to rely blindly on experts and special rules that are not properly understood becomes great. This might reduce the uncertainty involved, but only to some extent. Moreover it also highly increased the risk of unfairness and opens up the possibility that powerful interests can sereptisiously usurp the procedure for their own interests. Compared to this, a system based on direct fairness assesments carried out by noraml people, on the basis of (hopefully) neutral information provided by experts, might well be the best option.

However, I think the inherent difficulty in devising appropriate compensation mechanisms for commercial potentials suggest that the compensatory approach might be misguided altogether. In addition, as soon as one begins to look at the social function of property, and its role in human flourishing, it seems that any kind of financial compensation is going to provide an inadequate reply to deprivation of commercially interesting property. Such a system speaks volumes about {\it who} society deems capable of carrying out commercial projects. The discrimination suffered by property owners of the {\it wrong kind}, should also not be underestimated. 

In light of this, I think it is appropriate to consider alternatives to expropriation in cases when economic rationales dictate economic development. Interestingly, the Norwegian system has an entire legal framework in place that can elegantly facilitate such a shift, should enough political be mustered to compel government and developers to make use of it. In the case of hydropower development, it already being put to the test in an increasing number of cases for substantial development projects, as local developers tend to shun away from outright expropriation of property belonging to unwilling neighbours. 

The land consolidation mechanisms that can be used to facilitate compulsory development in these situations form part of an ancient semi-juridical system of land management in Norway. In my opinion, this framework also points towards the future, as it provides a highly flexible approach for dealing with property and economic development under varying degrees of compulsion. In my next and final chapter I will present it in more depth and argue that it can often provide solutions that are both more effective and more equitable than solutions arrived at in a system that relies on expropriation. The compensation issue, in particular, is resolved simply by giving unwilling owners low-risk financial instruments tied to the development that is ordered to take place on their property.

