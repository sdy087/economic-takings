\chapter{Just compensation}\label{chap:5}

\section{Introduction}\label{sec:into5}

In this Chapter, I consider the question of compensation following expropriation of waterfalls. The main question that arises is whether or not owners should be compensated for the loss of a commercial hydropower potential. If so, the compensation payments can be very large, so large that expropriation is no longer a feasible option. However, traditionally, no such compensation was awarded and the amounts paid to were negligible. In fact, no compensation would have been paid at all, were it not for the fact that a theoretical compensation formula was developed to avoid reaching the conclusion that waterfalls could be expropriated with no compensation payable to owners.

The question of whether or not to base compensation on the loss of a commercial hydropower potential is closely related to the so-called ``no scheme'' principle, by which compensation is to be based on the value of the property taken such as it would have been if the expropriation scheme had not been proposed. If one takes the view that hydropower development is the prerogative of the party that has been granted a development license, it might then follow that no compensation is payable to the owners of the waterfalls, at least not for the hydropower potential. The value of hydropower, in particular, may be regarded as being due to ``the scheme'', not due to the natural resource that the waterfall represents. 

In Section \ref{sec:nsp}, I describe the no-scheme principle in more depth, by comparing its origin and current status in English law with that of Norwegian law. It is of independent interest to note the close correspondence with common law. Moreover, considering the English debate in particular will also serve to bring out some concrete idiosyncrasies regarding how the principle is understood in Norway. The debate about how to apply it in waterfall cases is ongoing in Norway, and the Supreme Court has dealt with the issue in a range of recent cases. 

Traditionally, the principle was {\it not} applied, however, since it would have led to little or no compensation for owners during the monopoly era. Instead, a theoretical method was applied that was meant to give owners a share of the benefit in hydropower development. This method was developed early in the 20th century, however, so it gradually became more and more removed from the realities of hydropower development. I describe the method in some depth in Section \ref{sec:nathp}, culmination in a number of reasons why it no longer succeeds in achieving a meaningful form of benefit sharing.

Then, in Section \ref{sec:fa}, I go on to consider the processes that eventually led to the method being abandoned in cases when alternative owner-led development schemes would have been ``foreseeable'', assuming the expropriation project had not been proposed. I link the reform in this area with the institutional framework used to determine compensation payments under Norwegian law. I note, in particular, that the natural horsepower method was first abandoned by the appraisal courts, where lay people sit as appraisers alongside a regular judge. In the first case, moreover, the lay people eventually adopted the new method against the dissenting opinion of the regular judge. 

The Supreme Court struck down their judgment on a technicality, but refused to reject the principle that lay people were free to adopt a new method in cases when the traditional method would not adequately reflect the value of ``foreseeable'' use. I argue that this shows the strength of a long tradition of respecting the discretion of lay people in appraisement disputes. Many legal scholars, in particular, had previously regarded the natural horsepower method as a {\it rule of law}, fixed by precedent.

The method has not been abandoned as a matter of principle, however. As made clear recently by the Supreme Court, it is still to be applied in cases when a calculation based on ``foreseeable use'' does not lead to higher compensation payments. The crucial question becomes what exactly is meant by this notion. I address this in some depth, by pointing to how Norwegian law in general is  marked by a tendency to disregard any use that is not sanctioned by public plans, including in cases when these plans themselves provide the rationale for expropriation. This appears to be contrary to the no-scheme principle, demonstrating more generally that only ``one half'' of the principle tend to apply in a Norwegian setting. In so far as the principle precludes giving the owner a share of the expropriation surplus, it is applied, but in so far as it entitles him to compensation based on a future use that is rendered unforeseeable by the planning underlying the expropriation license, it is not.

There are some exceptions to this, however, and the Supreme Court has indicated that one of them applies to hydropower cases. At the same time, however, it has been stressed that even if the expropriation plans themselves are not binding for the compensation assessment, the ``public rationale'' underlying these plans must be taken into consideration when awarding compensation. In effect, this means that compensation is not offered for alternative uses in so far as the project proposed by the expropriating party is superior and could not be undertaken by the current owner. In effect, it seems that a partly {\it subjective} standard is introduced into compensation law, whereby local owners are denied compensation for a commercial value that is deemed to be such that it is only realizable by the expropriating party.

The Supreme Court has not been entirely consistent about the scope and exact content of the ``public rationale'' principle, however,   and the issue is still very much contested in Norwegian courts. In Section \ref{sec:ko}, I illustrate the current unclear state of the law by contrasting two recent Supreme Court cases. In the first, the court embraced an objective version of the ``public rationale'' principle by holding that as the expropriating party's project resulted in more public benefits, compensation could be based on the premise that the owners' foreseeable use of the waterfalls was to cooperate with the large energy company in realizing the plans, to take their share of its commercial potential. 

In {\it Otra II} on the other hand, the Court held that this should not be the conclusion in so far as cooperation was deemed to be ``impractical'', following a concrete assessment of the facts. It seems quite clear that the notion of ``impracticality'', as it was used here, serves to introduce a subjective assessment standard, contrary to what otherwise dictated by Norwegian compensation law. 
I go on to consider the merits of {\it Otra II} against human rights law, anticipating also the outcome of the appeal currently lodged with the ECtHR in Strasbourg.

Finally, I conclude that the case law on compensation demonstrates the intrinsic inadequacy of a narrow perspective on takings for profit. It seems clear, in particular, that all of the approaches currently in use to calculate compensation for waterfalls leave great room for bickering, manipulation and long-winded court battles. Moreover, the factual premise for the calculation is typically extremely uncertain, meaning that the whole procedure appears as something of a gamble, for both owners and developers. Hence, the developers favour the use of the natural horsepower method, which is completely removed from the reality of hydropower, but deliver predictably low compensation payments that will not prove too damaging to the profit-margin of the development company. On the other hand, owners have an incentive to push for compensation mechanisms that will allow them to collect the entire financial potential of hydropower development without actually investing any effort in planning or administerting such development, and without subjecting themselves to any of the risks involved. 

The current system means that if they are lucky, or employ skilled arguers, they can collect a very substantial sum of money for little or no efforts and with no social responsibilities attached to it. On the other hand, if they are unlucky, they are forced to give up their most valuable asset for nothing but a symbolic payment. I conclude by arguing that a much better approach would be to try and get owners involved in sustainable hydropower in a way that can remove the need for expropriation altogether. As development is now organized as a commercial pursuit, this should in principle be possible, since the owners {\it do} have an incentive to get involved in  development, also in cases when the public severely restrict the set of possible terms through regulation.  In practice, however, what is needed is a mechanism for organizing such owner-involvement. This mechanism, moreover, will undoubtedly also need to be endowed with powers of coercion if it is to be effective.

This, then, sets the stage for the last chapter, where I return to the question of how to replace expropriation by mechanisms of participatory democracy, referring back also to the discussion in Chapter \ref{chap:1}.

\section{The ``no scheme'' principle}\label{sec:nsp}

In most jurisdictions, a fundamental principle relating to compensation following expropriation is that the owner's loss should be calculated without taking into account changes in the value of their property that are due to the expropriation itself, or the scheme underlying it. In a recent Law Commission consultation paper, this principle is referred to as the \emph{no-scheme} rule, a terminology we will also adopt here, noting that while the exact details of the rule might differ between jurisdictions, the underlying principle appears to play a crucial role both in civil and common law traditions for regulating compensation following expropriation.\footnote{Need a good reference for this...}

While the no-scheme rule is easy enough to comprehend when it is stated in general terms, it raises many difficult questions when it is to be applied in concrete cases. What the rule asks of the valuer, in particular, is quite daunting; he is forced to consider the counterfactual "no-scheme world", and he must calculate the value of the property based on the workings of this imaginary world. One crucial question that arises, and which has traditionally proved to be highly contentious, is the question of what exactly this world looks like.

In the first instance, it might be tempting to take the view that this is a ``question of fact for the arbitrator in each case'', as expressed by the Privy Council in \emph{Fraser}, an influential Canadian case from 1917.\footnote{\emph{Fraser v City of Fraserville}, [1917] AC 187, p. 194} However, as the history of the no-scheme rule has shown, this point of view is not tenable.\footnote{For an history of the rule in UK law, clearly illustrating the difficulty in interpreting it and applying it to concrete cases, we point to Appendix D of Law Commission Report No 286, 2003} The problems associated with the rule were discussed in great detail by Lord Nicholls in the recent case of \emph{Waters}, who summarized the state of the law as follows.\footnote{\emph{Waters and other v Welsh National Assembly} [2004] UKHL 19}

\begin{quote}
Unhappily the law in this country on this important subject is fraught with complexity and obscurity. To understand the present state of the law it is necessary to go back 150 years to the Lands Clauses Consolidation Act 1845. From there a path must be traced, not always easily, through piecemeal development of the law by judicial exposition and statutory provision. Some of the more recent statutory provisions defy ready comprehension. Difficulties and uncertainties abound. One of the most intractable problems concerns the 'Pointe Gourde principle' or, as it is sometimes known, the 'no scheme rule'. On this appeal your Lordships' House has the daunting task of considering the content and application of this principle.
\end{quote}

\noo{
\begin{quote}
The extreme complexity of the issues that I have had to consider, the
uncertainty in the law, the obscurity of the statutory provisions, and
the difficulties of looking back over a long period of time in order to
decide what would have happened in the no-scheme world
demonstrate, in my view, that legislation is badly needed in order to
produce a simpler and clearer compensation regime. I believe that
fairness, both to claimants and to acquiring authorities, requires
this
\end{quote}
}
In the case of \emph{Waters}, the House of Lords seems to have made it an explicit aim to offer a clarification of the no-scheme rule and how to interpret it, and their judgement went into much more detail than warranted by the concrete case at hand, which seems to have been fairly straightforward. Even if it was not needed for the result, the House of Lords addressed many of the issues raised by the Law Commission in their report, focusing particularly on resolving the tension which was identified there between the principle relied on in \emph{Pointe Gourde} and the reasoning adopted in the so-called \emph{Indian} case from 1939.\footnote{\emph{Vyricherla Narayana Gajapatiraju v Revenue Divisional
Officer, Vizagapatam} [1939] AC 302.} In the \emph{Indian} case, the scheme was given a very narrow interpretation, with Lord Romer interpreting the scope as follows.

\begin{quote}
The only difference that the scheme has made is that the acquiring
authority, who before the scheme were possible purchasers only, have
become purchasers who are under a pressing need to acquire the
land; and that is a circumstance that is never allowed to enhance the
value.
\end{quote}

This, however, did not entail that the purchaser's demand for the property was to be disregarded, since, as Lord Romer puts it:

\begin{quote}
[...] The fact is that the only possible purchaser of a potentiality is
usually quite willing to pay for it […]
\end{quote}

In \emph{Pointe Gourde}, a different stance appears to have been adopted. The case concerned a quarry that was expropriated for the construction of a US naval base in Trinidad. The quarry had value to the owner as a business, and the valuer had found that if the quarry had not been forcibly acquired, it could also have supplied the US navel base for a profit. However, the value of this potential fell to be disregarded, with Lord MacDermott describing the no-scheme rule as follows:

\begin{quote}
It is well settled that compensation for the compulsory acquisition of
land cannot include an increase in value, which is entirely due to the
scheme underlying the acquisition
\end{quote}

Seemingly, this is at odds with the position taken by Lord Romer in the {\it Indian} case. In \emph{Waters}, both Lord Nicholls and Lord Scott addressed this tension in great detail, and offered a reconciliatory interpretation, which seems to narrow the no-scheme rule compared to how it has most commonly been understood following \emph{Pointe Gourde}. Moreover, the House of Lords also noted the need for reform and legislation, with Lord Scott going as far as referring to what he described as the present ``highly unsatisfactory state of the law''.

To understand how a seemingly simple principle could become so troubling in practice, it is crucial to keep in mind that after the introduction of extensive planning legislation in the 20th century, commercial development of property now tends to be contingent on governmental licenses and subject to extensive oversight. Moreover, the power to expropriate is often granted as a result of comprehensive regulation of the property-use in an area, often following public plans that are wider and encompass more than the particular project that will benefit from such a power. As a result, it has become increasingly difficult to ascertain what is meant by the scheme. Does it include the whole planning history leading to expropriation, does it only refer to the power to expropriate, or is it something in between?

The policy reasons motivating the no-scheme principle suggest that a fine balancing act must be made when addressing this issue. Under a wide interpretation of ``the scheme'', forcing the valuer to entertain many counterfactual assumptions, the property owner might come to feel that he is not compensated for his true loss, but rather an imaginary one. Indeed, the no-scheme world that the valuer must consider can end up being far removed from the actual one, forcing him to go back many years, perhaps decades, to establish what would have been the status of the property in question if the sequence of planning steps eventually leading to expropriation had not taken place. 

This can leave the property owner in a perilous situation, particularly if it makes the outcome seem arbitrary. Taken to extremes, the no-scheme principle can then also come to run amiss with respect to human rights law and constitutional provisions protecting private property. On the other hand, if the scheme is interpreted too narrowly, one runs the risk of endangering important public schemes by compelling the public to pay extortionate amounts. In many cases, it is undoubtedly also true that the value of property is increased by public investments and plans for the area in which the property is found, investments and plans that would not have materialised unless the power to expropriate had been anticipated.

It is important to keep in mind here, as noted by the Law Commission, that the no-scheme rule serves at two distinct purposes.\footnote{Ibid ....} First, it should be noted that the rule has an important \emph{positive} dimension: Property owners are not only compensated for the direct loss of their property, but also for the possible depreciation of their property's value following the decision to carry out a scheme which requires expropriation. Seemingly, this is easy enough to justify: It seems intuitively unreasonable if the deleterious effects of a threat of compulsion is permitted to result in reduced compensation payments.

However, under the extensive planning regimes common today, it is not clear where to draw the line: When is the regulation leading up to the scheme to be regarded as reflecting general public control over property use, and when is it to be regarded as a measure specifically aimed at compelling private owners to give up their property? As we will see when we consider the role of the no-scheme rule in Norwegian law, this question can easily become highly controversial, especially when it is linked with the more general question of whether or not the state should be liable to pay compensation for regulation that adversely affects the potential for future development. In jurisdictions that do not recognize owners' right to such compensation, like Norway and the UK, it is easily argued that the positive aspect of the no-scheme rule must be limited correspondingly. Why would a depreciation of value following regulation imply compensation when the property is eventually expropriated, but not otherwise?

In addition to its positive dimension, the no-scheme rule also has an important \emph{negative} dimension, expressed in {\it Pointe Gourde} as the principle that an {\it increase} in value should be disregarded when it is ``entirely due to the scheme''. The negative dimension has attracted more interest and controversy than the positive dimension, especially in the UK, and it was also at the center of attention in {\it Waters}. This is understandable all the while the negative aspect of the principle is more likely to result in protests from property owners. However, on a traditional understanding of the public purpose of expropriation, the negative aspect of the rule is also seemingly easy to justify. In \emph{Waters}, Lord Nicholls describes the most important policy reasons as follows:

\begin{quote}
When granting a power to acquire land compulsorily for a particular purpose Parliament cannot have intended thereby to increase the value of the subject land. Parliament cannot have intended that the acquiring authority should pay as compensation a larger amount than the owner could reasonably have obtained for his land in the absence of the power. For the same reason there should also be disregarded the ``special want'' of an acquiring authority for a particular site which arises from the authority having been authorised to acquire it.
\end{quote}

This appears like a reasonable justification. Notice, however, that Lord Nicholls avoids using the word ``scheme''. In particular, he does not identify the scheme's absence as the measuring stick for ascertaining on what basis parliament intends for compensation to be based. Rather, Lord Nicholls speaks of what the owner could reasonably have obtained in \emph{the absence of the power} to acquire the land compulsory. In this way, he seems to prescribe a rather narrow interpretation of the negative dimension of the no-scheme rule.\footnote{I mention that this interpretation of \emph{Waters} is also argued for in \cite{newuk}.} It is the power to expropriate that should not give rise to an increased value -- nothing is said at this stage about the scheme that benefits from it.

It would appear, therefore, that there is nothing in principle that prevents the property from being compensated on the basis of its value in a scheme that differs from the scheme underlying expropriation only in that it does not have such powers. Indeed, this subtle distinction appears to have been rather crucial for the remainder of Lord Nicholls' reasoning, where he attempts to reconcile the principle adopted in the \emph{Indian} case with the \emph{Pointe Gourde} case.

It will lead us to far astray to go into further details about the interpretation of the no-scheme rule in UK law and the possible implications of \emph{Waters}. Rather, we would like to turn our attention to the recent UK Supreme Court case of \emph{Bocardo}.\footnote{\emph{Star Energy Weald Basin Limited and another (Respondents) v Bocardo SA (Appellant) [2010] UKSC 35}} This case was decided under dissent, and it suggests that the clarification offered in \emph{Waters} might not have been as conclusive as one  had hoped. This worry arises, as we will see, particularly in those cases when expropriation benefits commercial schemes.\noo{ and for which the conceptual framework surrounding expropriation is, in our opinion, in need of refinement.}

\emph{Bocardo} was such a case. In short, it concerned a reservoir of petroleum that extended beneath the appellant's estate, and could not be exploited without carrying out works beneath their land. The first question that arose was whether or not extraction of the petroleum amounted to an infringement on property rights. This was answered in the affirmative. The second question that arose was what principle of compensation should be adopted to compensate the owner. The Supreme Court, following some deliberation, found that the general rules applied, and that the case should be decided on the basis of an application of the no-scheme rule. 

However, opinions differed as to the correct interpretation of the law, as well as how the facts should be held against the law. The crucial point of disagreement arose with respect to whether or not the special suitability, or \emph{key value}, of the appellant's land for the purpose of petroleum exploitation was to be regarded as \emph{pre-existing} with respect to the petroleum scheme.

In \emph{Waters}, the House of Lords had cited and expressed support for the following passage, taken from Mann LJ's judgment in \emph{Batchelor}.\footnote{\emph{Batchelor v Kent County Council} 59 P \& CR 357 p. 361}

\begin{quote}
If a premium value is ``entirely due to the scheme underlying the acquisition'' then it must be disregarded. If it was pre-existent to the acquisition it must in my judgment be regarded. To ignore the pre-existent value would be to expropriate it without compensation and would be to contravene the fundamental principle of equivalence (see \emph{Horn v Sunderland Corporation}).
\end{quote}

Relying on this distinction between the potentialities that are ``pre-existing'' and those that are due to the scheme, the minority in \emph{Bocardo}, led by Lord Clarke, made the following observation.

\begin{quote}
Anyone who had obtained a licence to search, bore for and get the petroleum under Bocardo’s
land would have had precisely the same need to obtain a wayleave to obtain access
to it if it was not to commit a trespass. So it was not the respondents' scheme that
gave the relevant strata beneath Bocardo’s land its peculiar and unusual value. It
was the geographical position that its land occupies above the apex of the
reservoir, coupled with the fact that it was only by drilling through Bocardo’s land
that any licence holder could obtain access to that part of the reservoir that gives it
its key value.
\end{quote}

This, however, was rejected by the majority, led by Lord Brown, who interpreted the no-scheme rule quite differently in this respect. He made the following comments regarding the issue of whether or not the value of the appellants land for petroleum extraction existed prior to the scheme.

\begin{quote}To my mind it is impossible to characterise the key value in the ancillary
right being granted here as ``pre-existent'' to the scheme. There is, of course,
always the chance that a statutory body with compulsory purchase powers may
need to acquire land or rights over land to accomplish a statutory purpose for
which these powers have been accorded to them. But that does not mean that upon
the materialisation of such a scheme, the ``key'' value of the land or rights which
now are required is to be regarded as “pre-existent”.
\end{quote}

While the case was resolved in keeping with this view, the dissent suggests that the clarification in \emph{Waters} has not resolved all issues. Moreover, it suggests that special questions arise with respect to the question of what potentials for development should be taken into account when evaluating a property. Crucially, the question raised in \emph{Bocardo} does \emph{not} relate to the scope of the scheme -- it was obvious that the scheme was the entire project aimed at extracting petroleum from the reserve. However, even when the scheme was unambiguously circumscribed, significant questions arose as to what ``value to the owner'' actually meant in this situation. 

In fact, it seems to us that \emph{Bocardo} serves to take the debate regarding compensation and the no-scheme rule one step further, and in a somewhat different direction compared to the debate revolving around the ``classic'' problem of determining the extent of the scheme. In some sense, it seems that the question raised by \emph{Bocardo} goes deeper, to the very core of the idea underlying the negative aspect of the no-scheme rule. When is it appropriate to say that some particular value is \emph{due to} the scheme and therefore not part of the owner's bundle of entitlements?

This asks us to establish a causal link between scheme and value. But as \emph{Bocardo} illustrates, it is by no means obvious what should be taken to constitute evidence for such a link. Moreover, it seems that the answer can depend largely on the point of view with which you \emph{choose} to analyze the matter at hand. For instance, when Lord Clarke went on to point out that the state, as owners of the Petroleum following nationalization in 1937, could have given the right to extract it to \emph{someone else}, he was certainly not incorrect.\footnote{References.} 

Moreover, it seems that this fact does in some sense break the causal link between scheme and value, although weakly so, since the difference between all schemes so conceived would only relate to \emph{who} the developers are, not the nature of the schemes as such. Consider, however, a scheme that is conceived of slightly differently, based on cooperation with the owner rather than expropriation. Would it not follow that this scheme would have \emph{precisely the same need to obtain a wayleave}, as Lord Clarke puts it? Moreover, is it still not also true that those behind it might be \emph{quite willing to pay}, as Lord Romer expressed it in the \emph{Indian} case? If so, the scheme does not cause the value. The fact that the wayleave is acquired compulsorily, in particular, is no precondition for the scheme. The project could equally well have been carried out by a developer who was willing to pay the owner for the special suitability of his land. If this is true, it appears that the justification for disregarding the special suitability of the land disappears.

On the other hand, it is also possible to adopt the point of view adopted by Lord Brown, which I choose to interpret as follows: Since the relevant strata did not have any special value except in relation to a petroleum-scheme requiring access, its value was causally dependent on the existence of \emph{some} such scheme. Hence, since the legislature {\it had} in fact made powers of compulsory acquisition available for developers of such schemes, this value was not, under the current regulatory system, regarded as pre-existent. The special value had, pursuant to political determination, been apportioned wholly to the developers of petroleum schemes. If this is true, then there is little doubt that the special value should be disregarded for the purposes of compensation. It seems clear, in particular, that under English law, no compensation claims for owners arise from such regulation.\footnote{I mention that the dispute in {\it Bocardo} also addressed more generally the question of how to appropriately compensate property that has ``key value'' with respect to the development of other property. It seems, in particular, that the strata, in its absence of any inherent value, more easily fell to be disregarded, and that Lord Brown's arguments in particular relies on establishing such a lack of intrinsic value. However, the question of when a particular aspect of value is to be regarded as pre-existent tend to arise in many other cases as well, and can be expected to arise particularly often with respect to commercial schemes. An extreme case obtains when we consider expropriation of \emph{natural resources}. Surely, if what was subject to expropriation in \emph{Bocardo} had been the petroleum itself, and not a right to access it, then even Lord Brown would have concluded that its value was pre-existent? This seems likely indeed, and then it appears to be good law in the UK after \emph{Waters} that it should also be compensated, irrespectively of whether or not the expropriating party is the only potential buyer.}

I conclude that the underlying question in {\it Bocardo} became the following: Did parliament really intend to let the entire value go to the petroleum-developer in such cases? This, in turn, seems to turn on the question of whether or not the use of compulsion in such cases was meant to be the norm or the exception. If compulsion is used in most or all cases, one can hardly claim that an alternative scheme based on cooperation is a realistic scenario that can provide the basis for calculating compensation. If, on the other hand, the power of compulsion is only made available to prevent holdouts and extortion, the case could still well be made that the value to the owner that would result from an equitable non-compulsion scheme should be compensated.

In this way, the case of \emph{Bocardo} highlights what I believe to be the crucial keyword underlying many of the most intractable problems associated with the negative aspect of the no-scheme principle, namely {\it benefit sharing}. In particular, disagreements about how to apply the principle invariably tends to boil down to the question of whether or not benefit sharing is considered desirable, or at least possible, under the current regulatory regime. In the first instance, this is a political question. But as cases regarding the no-scheme principle shows, it is one that the legislature often fails to address explicitly. 

Hence, it often falls to the courts to make determinations that are politically sensitive. This is in itself problematic. It seems, in particular, that in so far the courts' application of the no-scheme principle implements specific policies on benefit sharing, the lack of democratic accountability is worrying. This worry is exasperated by the fact that the law relating to the principle is notoriously unclear and complex. The obscurity of the law means that what is initially a clear and straightforward political issue runs the risk of becoming an impenetrable subject that only specialist jurists can hope to engage with in a meaningful way. This, in turn, opens up the possibility of abuse, as the courts do important work in shaping the politics of benefit sharing, work that is unacknowledged by many, but not necessarily by the wealthy and powerful.

An additional issue that arises is to what extent constitutional and human rights law makes it possible for owners to {\it demand} benefit sharing. This question arises most clearly in situations when a lack of benefit sharing points to a possible {\it unequal} treatment of opposing commercial interests. In so far as public interests in a specific form of activity dictates general limitations to the possible commercial benefits that can result from this activity, constitutional and human rights issues are unlikely to arise. However, in so far as commercial benefits accrue to {\it some}, but not to the affected owners, it would seem much more likely that fundamental principles could come to be violated.

This issue comes into focus when the scheme underlying expropriation itself has a commercial nature. In several of the ``classical'' cases that are cited as the foundation for the original no-scheme rule, this issue arose with great urgency. I mention, in particular, the cases of \emph{Cedars} (1914) and \emph{Fraser} (1917), two important Canadian compensation disputes regarding expropriation for hydropower, cited both by the Law Commission and the House of Lords in \emph{Waters}.\footnote{\emph{Cedars Rapids Manufacturing and Power Co v Lacoste}, [1914] AC 569 and \emph{Fraser v City of Fraserville} [1917] AC 187.} In \emph{Fraser}, it was the waterfalls themselves that were subject to expropriation, yet the Privy Council still found that the value of the potential for hydropower exploitation of these falls should be disregarded when compensating them.

The reasoning adopted seems to follow a standard ''value to the owner`` approach. However, reflecting back on {\it Bocardo}, it is hard to see how anyone could think that the value of the waterfalls is not ``pre-existent'' to the scheme to develop them. Surely, as a natural resource a waterfall has significant value in and of itself? This, however, was not the view taken by the Privy Council, which found that owners of waterfalls were not in fact capable of developing hydropower themselves. Hence, a subjective standard was in effect employed, whereby the entitlements regarded as arising from the professional company's  ownership of the waterfalls far exceeded the entitlements regarded to arise from the farmers' ownership of the same. This unequal treatment of owners, it seems, is such that is could, in the present day, be attacked from the point of view of human rights and constitutional law. But such an approach might not be required, as the Canadian cases already appear to be at offs with both {\it Waters} and {\it Bocardo}.

Still, they can serve as great examples of the type of situation where the need for a distinction between commercial and non-commercial aspects arise most forcefully. On the one hand, there can be no doubt that the energy inherent in water pre-exists any scheme seeking to harness it. Moreover, it seems clear that energy has great value, meaning that the value of a waterfall pre-exists any scheme for hydro-power exploitation. However, we must also ask: what \emph{kind} of value is it?

In fact, it seems that any value resulting in compensation to the owner must by the nature of things either be \emph{personal}, related to claims for disturbance etc, or else \emph{commercial}, namely such a kind of value that can be realized by a company or an individual operating for profit -- possibly the owner, possibly some buyer of their property. A different kind of value altogether is the \emph{public value}, which can not be realized for profit by \emph{anyone}. 

The distinction between commercial and public value is, obviously, down to a political decision. Moreover, it can hardly be regarded as permanent. In addition, it can often be difficult to assess where the line is to be drawn, especially in cases when public-private partnerships are relied on to provide public services. Nevertheless, it seems perfectly legitimate to make this distinction, and it seems like it can be very helpful in many cases. It seems, in particular, that both the political desirability and the constitutional necessity of benefit sharing is greatest in relation to what one would refer to as commercial value. 

This perspective can hopefully also help clarify the distinction between the concrete judicial application of the no-scheme principle and its political determined scope. In particular, I think that a slight departure from previous case law suggests itself in this regard. For instance, even if the public value of hydropower pre-exists the hydro-power scheme, this does \emph{not} necessarily mean that there is any pre-existent commercial value in hydro-power. Hence, to structure the discussion around what counts as pre-existent value, as done in {\it Bocardo}, might not lead to a sufficiently fine-grained approach.  What counts as {\it commercial} value, in particular, must first be answered. This, moreover, depends entirely on whether or not the public has settled on a regulatory regime that allows commercial exploitation.

In the case of {\it Bocardo}, I think this perspective would have been particularly helpful to Lord Brown, who argued that the value of the strata was not pre-existent. From an intuitive point of view, his argument seems rather strained. After all, it was the physical conditions that gave the land its value, not the abstract fact that a development license had been granted. However, by looking at his argument in more depth, it is tempting to rephrase his conclusion by saying that he regarded the value of the strata as having no commercial value under the prevailing regulatory regime.

Hence, I arrive at a modified version of the ``pre-existence'' test used in \emph{Bocardo}: An owner should always be compensated for the value of any pre-existent \emph{commercial} value that his property has.\footnote{Certainly, a clarification along these line would not resolve all issues. It would not, for instance, offer any conclusive guidance with respect to the specific issues related to "key value" raised in \emph{Bocardo}.} I remark that the question of what commercial value can be said to pre-exist a scheme might turn rather more on facts than on law. Hence, this way of posing the problem is also apt to bring out how the determination is highly sensitive to the political context in which expropriation takes place.

It seems quite clear, in particular, that in order to answer the question of what should be counted as a pre-existing commercial value, one must take a broad look at the prevailing regulatory regime. Moreover, one must expect that the correct assessment of this question will depend on the context of regulation, in particular the extent to which the state \emph{allows} the disputed value to be commercially realized. The law relating to compensation should be such that it can tolerate significant changes in these parameters. It  seems, therefore, that the important legal question in this regard is to provide a sound conceptual foundation that can enable sound and equitable assessment across a range of different scenarios. Moreover, it seems that the courts, in light also of human rights law, has an important supervisory role to play in this regard.

In the next section, I will address Norwegian compensation law. I will observe, in particular, how the system was originally based on an approach to compensation that was not overly constrained by the idea that all disputes had to be resolved uniformly on the basis of clear and precise rules. The no-scheme principle, while clearly relevant, was understood broadly, as a guideline for assessment. But the assessment itself was above all else discretionary. Moreover, legitimacy of the process was ensured by the involvement of lay people sitting as appraisers, alongside a regular judge. 

I note, however, that this system has largely been modified so that, today, the appraisal courts are far more constrained from the top down, by legislation and the Supreme Court. I argue that this has been a source of difficulty for the system, particularly in relation to the no-scheme principle which, as I argued above, necessitates a concrete assessment, and can not -- should not -- be resolved by all-encompassing principles. I note, in particular, how the increasingly constrained room for discretion by lay people means that the distinction between commercial and public value -- which must now be determined centrally -- becomes muddled. In many cases, the idea acting as a premise for the general rules applied simply does not correspond to reality. This often leads to unacknowledged commercial windfalls for takers, arising when owners are denied compensation for commercially valuable rights that the law presupposes to be wholly public, even though they are not. 

\section{Appraisal courts and ``foreseeable alternatives''}

The owner's right to compensation following expropriation of property is enshrined in very simple terms in Section 105 of the Norwegian Constitution. It states simply that \emph{full compensation} is to be paid, in all cases when the public interest warrants the compulsory acquisition of property. For more than 150 years, this was the sole legislative basis for compensation rules in Norway. In particular, the concrete methods employed to calculate full compensation in different scenarios developed entirely through case law.

According to a long legal tradition in Norway, the discretionary aspects of property valuation is regulated by special procedure, with a significant reliance on so called \emph{unwilling appraisers}. These are members of the public who have no interests in the case at hand. They may be chosen specifically because they are deemed to be in a good position to judge the value of property, either because they are resident in the local area, or because they have special expertise.

The appraisal procedure has a long history, going back to customary law that predates the constitution. The rules regulating it today are found in the Appraisal Act 1917.\footnote{Act no 1 of 1. June 1917 relating to Appraisal Disputes and Expropriation Cases.} Appraisal cases are organised similarly to civil disputes, and the procedure is administered by the district courts.\footnote{See Section 5 of the Appraisal Act 1917.} The importance of lay discretion is the main distinguishing feature: The appraisal court is usually composed of a panel consisting of one professional judge and four appraisers, with no special juridical qualifications. 

The standard arrangement is that appraisers are chosen from the general public in the district where the property in question is located. But the Act opens up for the possibility that appraisers may also be chosen for their special technical expertise.\footnote{See Sections 11 and 12 of the Appraisal Act 1917.} Their role in the procedure is on par with the judge, and the panel decides jointly both the legal and the technical questions, usually on the basis of technical reports put forth by the parties. These reports are presented during the main hearing, and may be challenged by the parties, in more or less the same way as the district court hears evidence in a regular civil dispute.\footnote{See particularly Section 27 and Section 22 of the Appraisal Act 1917, with further references to the Dispute Act 2005 (Act No 90 of 17 June 2005 relating to the Mediation and Procedure in Civil Disputes).}

There is a possibility for appeal to the appraisal Court of Appeal, which is the regular Court of Appeal sitting as an appraisal court in accordance with the rules of the \cite{aa17}. The right to having an appeal heard is not absolute; whether the appraisal Court of Appeal will hear the case depends on its importance, according to rules that correspond to those in place for regular civil disputes.\footnote{See Section 32 of the Appraisal Act 1917.} The procedure followed closely corresponds to the procedure followed in appraisal disputes at the district level.\footnote{See Section 38 of the Appraisal Act 1917.} However, the decision made by the appraisal Court of Appeal is final as far the appraisal assessment is concerned. An appeal to the Supreme Court can only be accepted on legal grounds.

As a consequence of this, the appraisal courts have been very important in interpreting and developing the law relating to compensation in Norway. Their importance was particularly great all the while the meaning of ``full compensation'' was not clarified further in statute. At the same time, the practical viewpoint enforced by the procedural form meant that legal questions would often remain in the background in such cases. Typically, the legal issues would only come to the forefront if the Supreme Court decided to hear the case as a matter of principle. 

This primary criticism voiced against this system, particularly following the Second World War, was that it gave the appraisal courts too much discretionary power. Hence, the argument went, legislation was needed to make the outcome of appraisal cases more predictable.\footnote{See, for instance, Part 2, Chapter 1 of the \emph{Report Regarding Appraisal Procedures and Compensation following Expropriation}, NUT 1969 nr. 2 (Norwegian governmental reports), handed over to the Department of Justice by the so called Husaas committee, appointed by the King in Council 6. Aug 1965.} However, while the law regarding compensation was not formalized in written form, there were legal scholars who developed theories and aimed to explicate its content based on the body of case law that was available. 

Also, the Supreme Court did regularly hear cases concerning legal arguments regarding compensation, and they developed a consistent position on at least some of the more critical and recurring legal issues. At this time, the central source of legal reasoning regarding appraisal was still to be found in the constitution itself. As a result, theories of compensation law tended to be \emph{absolutist}, in the sense that they looked directly to wording in Section 105, also when tackling specific problems of interpretation. 

\subsection{Constitutional absolutism}

Absolutism was widely endorsed by Norwegian legal scholars as late as in the 1940s. In \cite[p. 177]{schj}, it was summed up as follows:

\begin{quote}
When an owner is entitled to compensation, he is entitled to have his full economic loss covered. He should receive full compensation, see p. 42 ff. This is the great principle that remains absolute and any dispute must be resolved on its basis.
\end{quote}

A typical example of the style of legal reasoning that this view gave rise to can be found in the writings of the prominent legal scholar Frede Castberg. He is particularly interesting since he specifically addressed the no-scheme principle, by asking about the extent to which increases in value due to the scheme underlying expropriation was to be taken into account when calculating compensation. His reasoning in this regard was based directly on a reading of the Constitution. Moreover, it was based on the principle of \emph{equality}, which was at that time considered particularly crucial in understanding constitutional law. The following quote serves to sum up Castberg's position on the no-scheme principle:\footcite[Volume 2, p. 268]{castberg}

\begin{quote}
The owner is entitled to full compensation. The expropriation should not leave him worse off economically than other owners. Hence if the public has knowledge that an industrial undertaking is being planned, that a railway will be built etc, and this affects the value of property generally in a district, then the increased value of the property that will be expropriated must be taken into account. If not, the owners of such property will be worse off than other owners from the same district. On the other hand, if the expectation of the scheme underlying expropriation leads to a general depreciation of value, then it is this new value -- not the original value -- that is relevant for calculating compensation. The crucial question is what the actual value is, when expropriation takes place.
\end{quote}

% e mention that the problem analyzed by Castberg in this passage has been considered in many jurisdiction, and is dealt with in common law by the so called \emph{no-scheme} rule. This is more a principle than a single rule, and it is typically understood as a mechanism that is meant to ensure that changes in value due to the scheme underlying expropriation are disregarded.\footnote{For an history of the rule in common law (primarily the UK), which also illustrates the difficulty in interpreting it and applying it to concrete cases, we point to Appendix D of Law Commission Report No 286, 2003} In comparative terms, Castberg appears to favor a \emph{narrow} interpretation of the principle -- a restrictive view on when additional value due to the scheme should be disregarded -- quite close in spirit to the so called \emph{Indian} case from 1939\footnote{\emph{Vyricherla Narayana Gajapatiraju v Revenue Divisional Officer, Vizagapatam} [1939] AC 302.}, which was been much discussed in common law and was dealt with extensively by the House of Lords as late as in 2004.\footnote{In the case of \emph{Waters and other v Welsh National Assembly} [2004] UKHL 19. 
%The primary precedent for a broader interpretation of the non-statutory no-scheme rule, on the other hand, is \emph{Pointe Gourde}, \emph{Pointe Gourde Quarrying and Transport Co v Sub-Intendent of Crown Lands} [1947] AC 565, PC, 572, per Lord MacDermott. This case proved highly influential for the understanding of compensation rules in the post-war period, in many common law jurisdictions, but has recently been challenged by a renewed interest in more narrow viewpoints such as that expressed in the \emph{Indian} case, see  \cite{newuk} and also the case of \emph{Star Energy Weald Basin Limited and another (Respondents) v Bocardo SA (Appellant) [2010] UKSC 35}.}In the context of Norwegian law, it is of particular interest to note how Castberg's views in this regard is arrived at through considering the constitution itself, founded on the principle of equality.\footnote{In this way, he arrives at a narrow no-scheme rule quite abstractly, and through a different route than the one adopted in the \emph{Indian} case, where the outcome appears to have turned crucially on the particular facts in the case, a close reading of precedent, as well as the perceived fairness of the result.}

As Castberg bases his analysis on the exact wording of the Constitution, he does not engage in any reasoning based on the extent to which it can be regarded as socially fair for the public to pay compensation for value that arise due to the beneficial consequences of the public project itself. Crucially, he does not address the concern that this can be seen as a form of double payment. Such pragmatic, utilitarian, reasoning was not widely adopted in the legal tradition Castberg was part of. This, in particular, is why I think it is appropriate to use the label of constitutional absolutism to describe this kind of reasoning. 

However, it is not correct to think that such reasoning is necessarily ``owner friendly''. To see this, it is enough to note that Castberg, in the quote above, explicitly states that depreciation of value due to the scheme should not be disregarded. In addition, Castberg did not intend to reject the no-scheme principle altogether. In particular, he denied that owners of expropriated property should ever be able to claim compensation based on the special want of the acquiring party:

\begin{quote}
The situation is different if the property has increased value due to the expectation that it will be expropriated. The owner can not demand that this increase is compensated since that would be the same as giving him a special advantage compared to those from whom no property is expropriated.
\end{quote}

Hence, I conclude that Castberg accepts a narrow version of the no-scheme principle, similar in spirit to that presented by Lord Romer in the {\it Indian} case. Castberg's view appears to have been shared by many academics of his day, and was also, to some extent reflected in case law from the Supreme Court.\footnote{References needed.} At the same time, the very nature of the system for deciding appraisal disputes gave the local appraisers great freedom in adapting the principles in a way that suited the concrete circumstances.

To quite some extent, this would also involve making an assessment of what was regarded as a fair and just outcome. Hence, while the theory of the time was absolutist, case law was more multi-faceted. Importantly, fairness was seen as a concrete issue that had to be addressed on a case by case basis, an approach that would not necessarily lead to general rules. The Supreme Court largely sanctioned this approach, by displaying great respect for the discretion of the appraisal courts, as vested in them within an absolutist theoretical framework.

As long as they did not cross the line with regards to the constitution, the appraisal courts were largely allowed to adopt broader viewpoints. But such viewpoints were \emph{not} extensively codified in terms of special principles used to deal with special case types or issues. Rather, they arose as a logical consequence of the way in which appraisal disputes were organized. Social justice and fairness perspectives were not excluded, but could in fact play an important role in practice.. However, such perspectives arose \emph{indirectly} through a \emph{decentralized} system which gave local courts great freedom when applying the law.

The way in which the no-scheme principle was applied serves as an excellent illustration of this. On the one hand, the theoretical views of Castberg were widely accepted, but at the same time they were regarded as no more than guidelines that had to be adapted to the circumstances. Moreover, it was not unheard of for the appraisers to disagree with the judge about how this should be done, and to award compensation according to a different understanding of the law than that favored by the judge. 

This happened, for instance, in the case of \emph{Tuddal}, where land was expropriated for construction of a power grid. In relation to this, the expropriating party also acquired the right to use a private road.\footnote{Rt. 1956 p. 109}. According to the juridical judge in the appraisal court of appeal, following the teaching of Castberg, compensation should be awarded solely on the basis of what the owners stood to lose. In this case, that would mean compensation based on the increased cost in maintaining the road resulting from increased use. However, the lay appraisers found this result unreasonable and awarded compensation also for the special value the use of the road would have for the acquiring party. The Supreme Court, although they found fault with the argumentation relied on by the appraisers, agreed that such compensation was possible in principle. The presiding judge offered the following perspective:

\begin{quote}
Since they were the private owners of the road, A/S Tuddal could, before the expropriation, refuse to let the Water Authorities make use of it. Hence it might be possible for A/S Tuddal, through negotiation and voluntary agreement with the Water Authorities or others with a similar interest, to demand a reasonable fee, and in this way achieve a greater total benefit than full compensation for damages and disadvantages. Following the expropriation, it is no longer possible for A/S Tuddal, in its dealings with the Water Authorities, to economically benefit from their ownership of the road in this way. If the company suffer an economic loss as a result of this, I believe they are entitled to compensation. Whether or not such an opportunity as I have mentioned -- all things considered -- was present at the time of the expropriation, falls to the appraisal court to decide, on the basis of whether or not an economic loss is suffered beyond that which follows from damages and disadvantages. On this basis, I assume that the appraisal court of appeal's decision to awarded compensation for the value of the right of way that is acquired can not -- in and of itself -- be regarded as an erroneous application of the law.
\end{quote}

The Supreme Court's reasoning illustrates two points. First, we see how the Supreme Court adopts absolutism in its interpretation of the law. Through careful use of wording, the compensation premium is not conceptualized as compensation based on the value of the road to the acquiring authority, but rather as compensation for the loss of potential profit following from a voluntary agreement. Hence, a seeming contradiction with the no-scheme principle is avoided.\footnote{This particular interpretation of full compensation led to arguments in the post-war period, regarding whether or not owners had a right to compensation based on the loss of profit from hypothetical voluntary agreements with the acquiring party. In the end, a consensus formed that this type of compensation should not in general be awarded. See NUT 1969 nr. 2, Part 2, Chapter 4, Section 2.E.}

But {\it Tuddal} also illustrates a second important point, namely that the Supreme Court was prepared to defer greatly to the judgment of the appraisal court. It is stated explicitly that it falls for this court to decide whether or not the opportunity to profit from the road by negotiating with the expropriating party was present at the time of expropriation. This is particularly noteworthy in light of the dissent of the juridical judge in the appraisal court of appeal, and in light of the dominant legal theorizing of the day, which did  not seem to support the idea that a premium should ever be paid in a situation like this. Hence, the decision tells us that the Supreme Court went far in defending the discretion of the laypeople, as a \emph{systemic} feature. They tested with great caution whether it was truly outside the permissible legal boundary, but concluded that it should simply be regarded as an exercise of the lay judgment that the system presupposed.

This impression of the case is accentuated when considering other cases dealing with similar issues. Across the board, I note a strong  tendency to defend the role of the laypeople in the appraisal process. A particularly clear expression of this can be found in \emph{Marmor}, also from 1956, where the Supreme Court overturned a decision made by the appraisal court of appeal on the grounds that the court had been {\it too} reliant on general principles, leading it to offend against both the principle of full compensation and the principle of evaluation by impartial laymen.\footnote{Rt. 1956 s. 493.} 

The case involved expropriation of a private railway track, for the construction of a public railway. It was clear that the track which was being expropriated did not have market value in general, so the expropriating party argued that the value of these tracks to the public railway should not be taken into account when calculating compensation. The appraisal court of appeal agreed, pointing to the standard teaching of the day. The Supreme Court, on the other hand, struck down the decision because they felt that a standardized approach to the case was inappropriate given the circumstances. The presiding judge argued as follows:

\begin{quote}
In my opinion one can not simply assume that a property does not have market value when it has no value for anyone other than the expropriating party. The question needs to be assessed concretely. I agree with the expropriating party -- as has also been confirmed on several occasions by the Supreme Court -- that in general one should not take into consideration the special value that the purpose of expropriation gives the property. This should not lead to a spike in compensation payments. On the other hand, I can not agree that it is automatically reasonable, or in keeping with Section 105 of the constitution, if the expropriating party in cases like the present one could acquire property at a price that is below what it would be natural and commercially appropriate to pay in a voluntary purchase.
\end{quote}

Again, I note the two main building blocks used in the argument: First, the standard reference to the constitution, and secondly, a reference to the need for \emph{concrete assessment}. This further reflects the strong confidence that the Supreme Court had in the integrity and autonomy the appraisal procedure. Moreover, I notice how, in this case, absolutism regarding the constitutional protection of property is \emph{not} used to argue for specific rules or principles, but rather to back up the argument that compensation should result from real assessment, and not be overly reliant on such rules. In the case of {\it Marmor}, this was the outcome even if the rules in question had the status of valid guidelines that had also been backed up by a series of Supreme Court decisions.

In addition to making the overreaching remarks quoted above, the Supreme Court also gave pointers as to the kinds of facts that should be considered. For instance, the presiding judge paid particular attention to the wider \emph{context} of expropriation, and the manner in which expropriation was used to benefit certain interests. He also noted how expropriation had come to replace voluntary agreement as the standard means of acquisition for this type of development. Therefore, the practice of using expropriation effectively prevented a market from developing, a market that might otherwise have appeared naturally.

\begin{quote}
I also point to the fact that the case concerns an area of activity where the expropriating party has a {\it de facto} monopoly which makes it impossible for anyone else to make use of the property for the same purpose. This in itself makes it questionable to simply assume that the lack of financial value for other purchasers provides the appropriate basis for calculating compensation. When considering this question, it is also appropriate to take into account that we have lately seen a great increase in the use of expropriation to undertake projects such as this. Compulsion is becoming the primary mode for acquisition of property -- not voluntary sale following friendly negotiations.
\end{quote} 

In my opinion, the importance of this decision, which makes it highly relevant even today, is not that it seems to favour a narrow interpretation of the no-scheme principle. In fact, I think it is erroneous to read the judgment as expressing general support for any particular interpretation. In addition, I do not think the decision can be read as supporting a general principle by which compensation can always be based on the value of hypothetical agreements that could have been made with the expropriating party. Rather, I take the judgment to be an important and, as I will show later, timely warning against blind obedience to \emph{any} set of detailed rules for calculating compensation that serve to limit the room for lay discretion. 

At the very least, it seems clear upon closer inspection of the argument that the main objective of the Court was not to express any particular view regarding the content of the no-scheme principle, but rather to instill to the appraisal courts that they could not use this rule as an excuse not to engage in concrete assessment to ensure a reasonable outcome in keeping with the constitution.

I believe this point is important to stress. It illustrates how absolutism need not, and did not, result in a rigid system with little room for assessment based on justice and fairness, broadly conceived. Quite the contrary, the absolutism endorsed by the Supreme Court, and inherent in the Norwegian system of appraisal courts, was not characterized by blind obedience to specific rules, like those proposed by Castberg. Rather, the system was flexible, and it was explicitly intended to function such that fairness assessments based on concrete circumstances could be accommodated. 

Going back to even older legal scholarship, we see that this view on the meaning of absolutism has a long history in Norway. It is present, for instance, in the work of the famous 19th century scholar Aschehough, who stressed the link between the constitution and the appraisal procedure when he considered the (then) hypothetical situation that legislation would be introduced with the specific aim of reducing the level of compensation payments following expropriation:\footcite[p.48]{asch} 

\begin{quote}
If it becomes common practice to award compensation payments that are unreasonably high, this would make important public projects more expensive and difficult to carry out, greatly to the detriment of society. In many cases it might not be possible to rely on legislation to prevent such excessive compensation payments, since this would restrict the appraisers too much. To some extent this might be possible, however, and as far as it goes, parliament must be permitted to do so. However, if enacted rules clearly lead to less than full compensation in an individual case, they will be overruled by Section 105 of the constitution, and fall to be disregarded in that particular case.
\end{quote}

This quote is important because it does not rely on any particular interpretation of the constitutional demand for full compensation, but sees this inherently as an issue that needs to be resolved by concrete assessment of individual cases. Absolutism to Aschehough implies freedom and responsibility for the appraisers, freedom to judge individual cases by its merits, and a responsibility to award full compensation, irrespectively of any specific rules that might be in place to curtail excessive payments. The important point is that Aschehough here sees absolutism as a principle that should be applied to cases, not to principles. He does not argue that rules introduced to limit compensation payments would be inadmissible merely because they might sometimes suggest less than full compensation. Rather, he takes it for granted that it falls to the appraisal courts to apply the rules in a way that would prevent such outcomes. As long as the appraisal courts remain free to apply the rules in such a way that full compensation is awarded, specific rules intending to prevent excessive payments can happily coexist with absolutism.

The subtle view taken by Aschehough was largely overlooked in debates following the introduction of the Compensation Act 1973, which served to introduce radical rules of exactly the kind he had predicted and considered almost 90 years earlier. The consequence was, as I will discuss in more depth in the next section, that the Supreme Court was forced to actively steer the interpretation of the Act to ensure that section 105 would not be violated in concrete cases. Hence, the introduction of legislation served to destabilize the system, by narrowing the room for lay judgments and increasing the reliance on legislation and principles developed at the Supreme Court. This development is the subject of the next subsection.

%More generally, the 60s and 70s appears to be a period when the crucial role of the appraisal procedure was to some extent forgotten, and also undermined, following a heated political and ideological debate regarding the appropriateness and admissibility of introducing rules to ensure that compensation payments were brought down to a lower level. This had deep and lasting effects on Norwegian compensation law, and it is popularly described as a period when the social democrats won recognition for the principle that social fairness suggested the introduction of compensation rules and disregards that were more extensive than what had previously been considered appropriate. 
%
%This was conceived of as a fight for social justice against outdated and conservative ideas of constitutional absolutism. But it seems to us that this view of the history of Norwegian compensation law is erroneous, and largely unhelpful. The approach taken by Aschehough, in particular, placing emphasis on the important role played by the appraisers in achieving fairness and justice in concrete cases, does not appear to contradict social democratic goals at all. In fact, it seems that his approach might be better suited to serve such goals, and to accommodate a variety of different political opinions and ideas, than an approach which is based on attempting to flesh out in painstaking detail how the appraisal courts should go about achieving the balance between social fairness and owners' rights. We will return to this point later, but first we will take a closer look at the history of the radical Compensation Act 1973 and the censorship to which it was subjected by the Supreme Court, leading to the Compensation Act 1984, currently in place.

\subsection{Pragmatism}\label{sec:prag}

Following the Second World War, the social democratic \emph{Labour Party} gained a secure grip on political power in Norway. As a result, many reforms were carried out that would reshape Norwegian society. One of the most important reforms concerned the introduction of extensive planning law to ensure that land use was put under public control. As a result of this, the period also saw expropriation being used more extensively to further public projects, such as the large scale construction of hydro-power to ensure general supply of electricity.\footnote{References.} As a result of these changes, the opinion was soon voiced that there was a need for a more uniform approach to compensation, which collected some basic principles in a common body of written law. In addition, it was an explicitly stated political goal to bring compensation payments down.

In 1965, the so called \emph{Husaas committee} was appointed by the King and charged with the task of assessing the compensation rules currently in place.\footnote{Appointed by the King in Council on 6. Aug 1965.} The committee was also ordered to make a concrete suggestion regarding the need for additional principles of compensation, and to conisder if these should be given in the form of a special compensation act. Initially there was some doubt as to the extent to which is was at all permissible to give rules regulating compensation, as the constitution itself addressed the matter. However, the committee noted that existing legal scholarship seemed to suggest that such rules could be given, and that, moreover, specific rules had already been introduced, for instance in relation to expropriation for hydro-power development.\footnote{See Section 16 of the Watercourse Regulation Act 1917 (Act No. 17 of 14 December 1917 relating to Regulations of Watercourses).} Hence, the majority of the committee concluded that the constitution did not stand in the way of more detailed legislation. The Constitution provided an important outer barrier, serving also to protect owners' rights against acts of parliament, but it did not prevent legislation providing legally binding guidance as to how the notion of full compensation should be understood and applied by the courts in appraisal disputes.

Moreover, the majority pointed out that a vague general principle such as that provided by the constitution would by necessity have to be interpreted in order to be applied to concrete cases. Hence, it was not only permissible, but also desirable, for parliament to give more detailed instructions as to how is should be applied and understood by the courts and the appraisal courts. Leaving it to the judiciary to flesh out the exact meaning of full compensation through case law, it was felt, was not appropriate in a regulatory regime where expropriation had become increasingly important as a means to ensure modernization and development of critical infrastructure.

%In addition to this, the Supreme Court itself had recently expressed its support for a new view on regulation of property use, supported by contemporary legal scholars and politicians, whereby the State was regarded as having wide discretionary powers to determine how property should be used. This right to regulate, in particular, was increasingly coming to be seen as a right that did not infringe on property rights, so that the State would not have to compensate owners if they exercised it, except in special cases.\footnote{See, in particular, Rt. 1970 p. 67.}.

More generally, constitutional law was increasingly seen from a pragmatic utilitarian point of view, as a collection of foundational principles that aimed to stress the importance of ensuring social fairness just as much as individual justice. This was by no means a consensus view among legal scholars, however, and it was particularly contentious with regards to property. As a result, some disagreed strongly with the very idea of legislation regarding compensation, and tensions arose that have led to much legal controversy and are still important in the law today. 

This problem area was mapped out in some detail by the Husaas committee, who traced the pragmatic view on compensation, identifying it using the following quote by the leading scholar Knoph from \cite[p. 113]{knoph}.

\begin{quote}
Since Section 105 is a rule prescribing practical justice, directed at parliament, and not an ethical postulate of absolute validity, it must be permitted to make technical legal considerations, so that one accepts compensation rules that lead to correct and just results on average, even if it does not grant the owner full individual justice in every case.
\end{quote}

This view was becoming influential in the 1960s, as evidence for instance by the work of scholars such as .... and Opsahl.\footnote{See generally \cite{grunn,opshal}}. But many disagreed vehemently, based on absolutist principles.\footcite{robb2,schj}. The prominent legal scholar ... Scjhødt, for instance, describes Knoph's reading of the law scathingly as follows:\footcite[44]{schj}

\begin{quote}Luckily it has not had any effect on judicial practice whatsoever. No court of law would accept that compensation should be set according to a norm that may be practical and just in general, but does not grant the owner full compensation in all individual cases.
\end{quote}

When assessing the current state of the law, the Husaas committee encountered many manifestations of the tension between a pragmatic and an absolutist understanding of the protection of property. They proposed a set of general principles for compensation which are still largely with us today, and they engaged in a fine balancing act. They were moving in a pragmatic direction, but cautiously. Hence, they refrained from encoding principles that would appear too offensive to the absolutists, even if the pervading political sentiment was that more was needed to ensure a more effective state regulation of property use.

Importantly, the Husaas committee distilled from case law the principle that owners could only demand compensation based on the value of a specific use of the property when this use was ``foreseeable''. To determine what counts as foreseeable one could look either to the owner's own planned or actual use, or else one could look to the market value of the property, as this would reflect the value of such use as an average buyer might make of the property.\footnote{See, for instance, Rt. 1925 s. 47 and Rt. 1926 p. 669.} The committee took the view that deviating from this starting point would not be constitutionally admissible. Hence, instead of introducing altogether new principles, they sought to codify what they saw as the existing interpretation of Section 105 in this regard. This led to the following conclusion:\footnote{NUT 1969 nr. 2, Part 2, Chapter 5, Section 3.}

\begin{quote}
It is the view of the committee that it is correct to encode in the act the principle that the owner is entitled to compensation based on the value that results from taking into account the foreseeable and natural use of the property, given its location and the surrounding conditions. The exact meaning of ``natural and foreseeable'' use must be decided after a concrete assessment in individual cases. By encoding this general principle, however, it will become clear that compensation should not be based on private or public plans unless these plans coincide with the use of the property that is natural and foreseeable, independently of the scheme underlying expropriation.
\end{quote}

I note how the committee here also express support for the no-scheme principle, by remarking that the assessment of what counts as foreseeable and natural must be made independently of the scheme underlying expropriation. Since this statement is made quite generally, it also seems that the committee expresses a broader view on the no-scheme principle than that endorsed by Castberg. It is not longer only the special want of the expropriating party that should not be taken into account, the entire scheme ``underlying'' expropriation should be disregarded.

This view, however, was not in keeping with the political motivation for an act regarding compensation, and the Ministry of Justice deviated from it in their final proposition to parliament. Instead of encoding existing  principles, they sought a pragmatic system whereby compensation would in general be based on the value of the \emph{current use} of the property, thus preventing the public from having to pay a financial premium to owners based on possible future value that would in any event, in most cases, be reliant on public development permissions. %Such permissions, it was argued, could never be foreseeable in circumstances when it was in the public interest that the property should be expropriated, and hence all future development potential should in principle fall to be disregarded.

The Ministry commented on this as follows:\footnote{Ot.prp.nr.56 (1970-1971), p.19-20}

\begin{quote}
The Ministry is of the opinion that it is particularly important to arrive at a rule that can bring the assessment of property value down to a realistic level, and believes that the natural starting point for such an assessment must be the current use of the property, especially for expropriation of real property. As mentioned, it is the opinion of the Ministry that a practice has developed that gives too much weight to more or less uncertain future possibilities for the property, something that has led to a sharp rise in compensation payments.
\end{quote}

After intense debate in parliament, where the minority center-right parties all opposed its introduction, the current use rule was eventually encoded in section 4, no 1 of the \cite{ca73}.\footnote{Act No 4 of 26 March 1973 Regarding Compensation following Expropriation of Real Property.} This was largely seen as a social democratic victory, and a clear indication that the absolutist view on property protection was increasingly losing ground to a more pragmatic approach. When clarifying their principled starting point regarding what should count as \emph{realistic}, the Ministrys made the following assertion regarding the scope of the constitutional protection offered in Section 105, showing the ideological underpinnings of the act:\footnote{Reference}

\begin{quote}
However, a right to complete -- or almost complete -- equality can not be derived from the constitution. It must be taken into account that we are here discussing equality with regards to increases in property value that are, in themselves, undeserved. [...]  %  The starting point must be that it is not, in and of itself, contrary to the constitution that one property owner do not benefit from the same increase in value as another, when the increase in value, for both of them, is due to public investment and does not stem from their own efforts. \\ \\
Certainly, it would be best to avoid any kind of inequality, if it was possible. But the examples we have considered illustrate that, today, inequality between property owners is tolerated with regards to public investments and regulation, and that, moreover, practical and economic considerations dictate that we \emph{should} make use of differential treatment in this regard.
\end{quote}

Here we see a clear expression of support for a pragmatic view of property rights, echoing Knoph, but going much further. In particular, the Ministry explicitly states that differential treatment is appropriate in the context of expropriation, and, by implication, that this should be done precisely to avoid compensation payments that include compensation for ``undeserved'' increases in value. Also, in proposing that compensation payments should be based on current use, the scope of ``undeserved value'' is made very wide -- in principle it would seem to include \emph{any} value that could be attributed to an as of yet unrealized potential that the property in question might have. The question of whether or not this value was reflected in the market value of the property, in particular, was not regarded as relevant. This was in itself radical, since market value compensation had been the dominant starting point for reasoning about compensation following expropriation.\footnote{References.}

The conceptual significance of this change in perspective should not be underestimated. Here the Ministry stood firmly behind a pragmatic view, where social fairness was the overriding constraint, also with respect to constitutional property protection. However, on taking this view to its logical conclusion, it was recognized that any general compensation rules that might be introduced should themselves be subject to a fairness test, so that, for instance, the current use principle could not itself be absolute or without exception. 

Rather, it could only be applied in so far as it served the overreaching goal of social justice and fairness which was regarded as the fundamental component of property protection that made such a rule possible. This, in particular, seems like a crucial observation, and one that has in my opinion been overlooked, with unfortunate consequence for the subsequent debate and development of the law. Indeed, it echoes the sentiment expressed by Aschehough that we quoted above -- similarly overlooked -- and thus it points to the existence of possible \emph{common ground} between absolutist and pragmatist views on compensation. 

Sensible voices from both camps, in particular, seem to arrive at the conclusion that in the end, there is no way around a concrete and contextual assessment, where social fairness values are used to guide the assessment of individual cases. I find some support for this perspective also in the Ministry's own assessment of their radical proposal, as expressed in the exceptions that they introduced to the main rule. In this regard, they commented as follows:\footnote{References.}

\begin{quote}
One is aware that the principle of current use compensation cannot be without exception. Even though this rule will be fair in general it can, in some cases, disproportionately disadvantage property owners. One has therefore suggested rules that modify the principle to some extent. These are given for somewhat different  reasons. \\ \\

One case addresses the situation when current use compensation means that a property owner will be significantly worse off that other owners of similar property in the same district, according to how these properties are normally used. In these cases, the principle of equality suggest that the owner receives some -- but not necessarily full -- compensation for the inequality that would otherwise arise from the fact that his property was made subject to expropriation. %Etter departementets oppfatning har en ekspropriat etter grunnloven ikke noe krav på å bli satt helt i samme stilling som om ekspropriasjonen ikke var skjedd, en forskjellbehandling innen rimelige grenser må grunnloven tillate når dette tilsies av tungtveiende samfunnsmessige grunner. 
\end{quote}

This principle was eventually encoded in section 5, no 1-3 of the \cite{ca73}. It would prove highly controversial, since it was only formulated as a rules that ``could'' be used to increase the compensation. In \emph{Kløfta}, the Supreme Court eventually deviated from this and overruled the Act by making clear that additional compensation was \emph{obligatory} in a range of cases when the intention had clearly been that the rule should be used sparingly. In this way, and possibly inadvertently, the Supreme Court ended up defending owners' interest by \emph{limiting} the power of the appraisal courts.

Before moving on to consider this in more detail, I mention a second exception to the current use rule. In some sense, it is the more interesting of the two, even if it has been largely forgotten. In fact, I think directing attention to it, and to the idea that it captures, is highly relevant, particularly in relation to economic development takings. The rule sought to address precisely the situation that arose when the taker would benefit commercially from the expropriation and it addressed also the question of the \emph{power balance} between the expropriating party and the owner. In the words of the Ministry:

\begin{quote}
The second modification we make has to do with the relationship between the property owner and the expropriating party. If the use of the property that the expropriation presupposes gives the property a value that is significantly higher than the value suggested by current use, this will entail a transfer of value from the property owner to the acquiring party. In some cases this might be unreasonable. As an example of when this can become an issue, we mention an agricultural property that is expropriation for the purposes of industrial production. In such a case it might be natural that the owner receives a certain share in the increased value that the new use of the property will lead to.[...] %This would be different than, say, a situation where an agricultural property is expropriated for constructing a road or for setting up recreational outdoor grounds. In such cases, the expropriation will not lead to any such economically advantageous use of the property that will give the expropriating party an economic advantage. 

To establish a flexible system, the Ministry has concluded that it is practical that the King gives rules concerning the cases where an enhanced compensation payment, based on these principles, might be appropriate. This should not be decided by individual assessment, but governed by rules for special case types. Hence, the proposed Act states that the King can pass regulation concerning this matter.
\end{quote}

Again, this quote expresses the crucial insight that fairness with regards to compensation following expropriation can not be arrived at without adapting the rules to the circumstances. For any pragmatic approach to compensation, the \emph{context} of expropriation must by necessity come to play a crucial role, especially if the starting point is explicitly taken to be that compensation should only encompass the ``deserved'' value. What this value should be taken to be, in particular, can hardly be determined once and for all and in general terms, but must rather be subject to continuous revision depending on how expropriation is \emph{actually used} in society, the purpose it is meant to serve, the parties who benefit, and the degree of commercial economic benefit that results for individual parties other than the original owners.

Indeed, stipulating that compensation should be ``deserved'' appears to provide a benchmark that is just as unclear as the stipulation that compensation should be ``full''. It seems, in particular, that the inherent ambiguity of these terms allows us to draw two conclusions: First, that they might very well have the same meaning, and second, that they cannot possibly be defined once 
and for all by any act of parliament, or by any decision in the Supreme Court.

This, in turn, suggests that the Norwegian system of appraisal courts, and the presence of laymen in the decision-making processes of these courts, bears crucial influence on how well the Norwegian system is able to meet both the requirement of social fairness and justice for the individual. Unfortunately, however, the procedural, contextual aspect of fairness and justice was not recognized following the passing of the 1973 Act, with attention shifting towards issues of legal interpretation that arose from it. The primary such issue, and the most serious one, concerned the question of whether the law as such was in breach of the constitution. This was eventually resolved by the Supreme Court in the case of \emph{Kløfta} in 1976.\footnote{Rt. 1976 p. 1}. 

Here the 1973 Act would be significantly reinterpreted to make it appear less offensive to the constitutional standard of full compensation. However, taking a broader perspective, it seems to us that \emph{Kløfta} largely accepted that the intention behind the act should be respected and that appraisal practice needed to be adjusted accordingly. In this, the Supreme Court signalled loyalty to the political system and the democratic process. However, in implementing this adjustment in practice, they also, possibly inadvertently, set up a system where the role of the local appraisal courts appears to have been undermined, and where the Supreme Court itself assumed greater control over how the compensation law was to be applied in concrete cases. This characterizes the current state of the law, which I describe in more detail in the following section.

\subsection{The top-down approach}\label{sec:regab}

When the constitutionality of the Compensation Act 1973 came before the Supreme Court in \emph{Kløfta}, they chose to sit as a grand chamber and they reached a decision under dissent, being divided into two fractions, consisting of 9 and 8 supreme judges respectively. However, both fractions approached the problem of constitutionality by endorsing an interpretation of Section 5 nr. 1 in the Compensation Act 1973 that gave the exception to the current use much wider scope than what had been intended by parliament. The majority went farthest, and unlike the minority they also regarded the compensation payment in the concrete case to be insufficient. The first voter for the majority commented as follows on the constitutional aspect of the case.

\begin{quote}
[...] But the main question in this case, is whether or not it is in keeping with Section 105 to generally award compensation at a level below the market value that could legally be estimated, and that the owner could actually have achieved, if expropriation had not taken place. In my view, this involves allowing expropriation to transfer a right that the owner had, with a value to which he was entitled. If he is refused compensation for this value, he would, depending on the circumstances, be left significantly worse off than others in a similar position, who owns property that is not expropriated. Such a result I cannot accept. It would be a breach of established customary law and a practice that has been established throughout the years both by the appraisal courts and the Supreme Court. I refer particularly to Rt 1951 s. 87 (particularly p. 89, Opdahl). This practice is in itself a significant contribution to interpreting Section 105 on this point.
\end{quote}

I note the emphasis placed on \emph{market value} in the majority's reasoning. This may appear to be in keeping with an absolutist doctrine, but as I have mentioned, it can have unfortunate, possibly unintended, consequences for property owners, especially when combined with a restrictive view on what counts as foreseeable future development. I note, however, a technical point that might be of some significance for the interpretation of \emph{Kløfta}: Instead of stating outright that a market value rule follows from the wording of the Constitution as such, the majority takes the view that this interpretation suggests itself based on the compensation practice that had currently been established. This might limit the scope of the majority's remarks in this regard, but it also serves to give further support to the claim that the role of the appraisal courts, and their assessments, still had a strong position in Norwegian compensation law at the time of \emph{Kløfta}. 

I remark that the minority disagreed on the constitutional status of the market value rule. Indeed, it was in this regard that the difference of opinion between the minority and the majority was most clearly felt. The minority, in particular, explicitly rejected the view that this rule could be derived from the constitution itself, and they also disagreed with the understanding that it would have status as a constitutional rule simply because it had been adopted in practice. This bestowed merely the status of ordinary legal precedent. As expressed by the minority:

\begin{quote}
Case-law in this area cannot be understood as preventing parliament from changing the rules in accordance with what they regard as necessary. That would prevent a reasonable and natural development and would not be in keeping with the consensus view that Section 105 of the constitution is a rule that must be interpreted in light of, and adapted to, how society has developed and how the law is viewed. I believe the practice that have evolved cannot be decisive if a new situation and new needs require a different solution. Whether the Compensation Act is in breach of the right to full compensation enshrined in the constitution, must depend on an interpretation of the wording in the constitution itself.[...] \\ \\
In my opinion, neither the intentions of parliament nor the way they are sought implemented through Sections 4 and 5 are in breach of the equality principle upon which Section 105 of the constitution is based. It does not follow from the constitution that an owner is in all circumstances -- and irrespectively of the economic forces from which the market value results -- entitled to compensation that is at least as great as the greatest legal value that the property could represent on a free market. A different matter is that Section 105 of the constitution could be important to the interpretation and application of the rules.
\end{quote} 

Hence, the market value rule was explicitly renounced as a constitutional principle by the minority, who nevertheless conceded that the constitution could be used to interpret Sections 4 and 5 of the Compensation Act 1973. Both the minority and the majority agreed, however, that  it would be wrong to go on to consider Section 4 of the Compensation Act 1973 in isolation. For the majority, this would clearly have led to the Compensation Act 1973 being held to be in breach of the constitution, something that was avoided since the Supreme Court chose to consider the law as a whole, with the majority using the reasoning detailed above to argue for a new interpretation of Section 5, rather than as a means to undermine Section 4. Still, their interpretation of Section 5 went well beyond what parliament had intended, leading some scholars to claim that \emph{Kløfta} should be read as holding that the Compensation Act 1973 was unconstitutional.\footnote{References.} In the words of the majority:

\begin{quote}
The purpose of this rule is to award compensation beyond current use in cases where valuations according to Section 4 could be in breach with Section 105 of the Constitution. As it stands, Section 5 nr. 1 is not sufficiently suited for this purpose. By its wording it gives the appraisal courts an opportunity to assess whether or not it is reasonable to award additional compensation, even when the conditions for this is otherwise met, and even then with the limitation that the compensation would otherwise be significantly unreasonable. Such a free position for the individual appraisal courts -- without possibility of legal appeal -- would not be in keeping neither with the purpose of the rule nor the demand for full compensation set out in the constitution.
\end{quote}

On this basis, the Supreme Court chose to interpret Section 5 nr. 1 in such a way that whenever the conditions were fulfilled, the appraisal courts were \emph{obliged} to award additional compensation, and on this basis they found that the property owners in \emph{Kløfta} was entitled to have their compensation looked at again, in a new round before the appraisal courts. The minority agreed in principle, yet did not go as far as the majority, concluding that based on the particular facts at hand Section 5 had been adequately considered by the appraisal court in this particular case.

The upshot of \emph{Kløfta} was that Section 5 nr. 1 came to be seen as an obligatory rule, leading to compensation having to be enhanced whenever the current use rule led to payments that did not reflect the market value of comparable properties. However, the conditions stated in Section 5 nr. 2 and nr. 3 were still regarded as relevant, and in interpreting these conditions, a body of law developed whereby the market value rule was applied in a way that would come to involve significant reduction in compensation compared to what would result from practice as it had been prior to the Compensation Act 1973. In this way, the pragmatic approach proved triumphant, not because current use value was introduced as the general starting point, on the contrary, but because a range of new disregards were introduced to reduce the level of compensation in a range of different circumstances. After \emph{Kløfta}, in particular, the following rules were all considered legitimate ways to decrease the level of compensation.

In Section 5 nr. 2 and nr. 3, the following three disregard principles are encoded, all of which are, to varying degrees, still important in compensation law today.

\begin{enumerate}
\item Changes in value that are due to the expropriation scheme or investments or other activities should be disregarded, both when these are already carried out as well as when they are planned, c.f., Section 5 nr. 2 of the Compensation Act 1973.
\item To the extent that it is regarded reasonable, \emph{increases} in value that are due to public plans or investments should be disregarded, irrespectively of whether or not they have already been carried out, c.f., Section 5 nr.2 of the Compensation Act 1973.
\item An increased value falls to be disregarded if it results from considering a use of the property which is not in accordance with public plans, c.f., Section 5 nr. 3 of the Compensation Act 1973.
\end{enumerate}

These rules severely limits the level of compensation payments, and in many cases it appears to make the principle of full compensation based on market value rather illusory. Notice, in particular, that on the one hand, disregard rule nr. 2 can be applied to disregard the value arising from any use of the property that is not in keeping with the current public plan, whereas disregard rule nr. 3 can be used to also disregard any value that is due to this plan. While the outcome, logically speaking, should then be that no compensation can be awarded whatsoever, the disregard rule nr. 3 is usually seen to revert back to the current use compensation in such cases. For instance, if agricultural land is expropriated for the purpose of a motorway, and it would otherwise appear foreseeable that it could be used for housing, the compensation will be based on agricultural use because the value for housing is disregarded according to disregard rule nr. 3.

In practice, then, with virtually all novel economic activity making use of land is dependent on acquiring new planning permissions, the current use rule will typically be applied as intended by the Compensation Act 1973, with the only difference being that it is not thought of or described as such.\footnote{A similar point was made in \cite{stor}.} Rather, outcomes that are basically in keeping with current use thinking will be designated as "full compensation based on market value" -- the standard phrase adopted in most appraisal judgments -- and the fact that the outcome is equivalent to current use compensation remains unclear until one considers the range of disregards that have been applied. In this way, the state of law that followed \emph{Kløfta}, and which has largely been upheld and codified in later case-law, is greatly influenced by, and largely in keeping with the intentions behind the Compensation Act 1973. 

The Compensation Act 1984 was eventually introduced to reflect the principles laid down in \emph{Kløfta}, but it did not in any essentially way change or influence the course of the law that had already been set. Its main purpose was to bring the wording of the legislation more into keeping with how the law was interpreted by the Supreme Court. It explicitly returned to the starting point of the Husaas committee, namely that the compensation should be based on the value of the "foreseeable use" that the owner himself, or an average buyer, might make of the property. But it maintained and endorsed disregard rules nr. 1-3, except for restricting disregard nr. 2 to public investments, such that increased value due to public plans currently in place could not be disregarded.\footnote{In this way, the paradox mentioned above, that compensation could become impossible to award because there was no possible basis upon which to calculate it, was avoided.}

Beyond this, it did not give any further guidance as to how the disregard rules should be understood or applied, nor did it consider or resolve the question of when, if ever, they would need to be applied with caution in order not to go against the constitution. However, it was expected that cases where such issues arose would be resolved by strict adherence to firm principles, and that unless these principles could be derived from the Compensation Act 1984 itself, they should be laid down by the Supreme Court. Deciding on the law in such matters should not, in particular, be left to the discretion of the appraisers. The age when the appraisal courts were considered free to assess the cases based on their merits and directly against the overriding goal of achieving justice and fairness grounded in the constitution was over. Rather, an ethos had taken hold where the need to curb their freedom, in the interest of ensuring predictability and centralized control, was considered more important than upholding the system of lay judgment. 

As a result, difficult cases now routinely end up in the Supreme Court, who attempt to stick to established standardized rules as much as possible, but who will formulate new such rules for compensation of specific case types, if this proves unavoidable. As an example of this mechanism, it is enlightening to consider the case-law based on disregard rule nr. 3, which states that public plans currently in place are binding when calculating compensation. This rule cannot apply without exception, as recognized already by the Compensation Act 1973, since it may lead to outcomes that run counter to both the constitution and a common, rudimentary sense of fairness. 

One case which was considered by the Supreme Court in \emph{Østensjø} concerned land that was being expropriated for housing purposes, but such that one unlucky owner would only contribute land used for infrastructure that would serve the larger housing project.\footnote{Rt. 1977 p. 24} In this case, the Supreme Court agreed that he was entitled to compensation based on value of his land for housing purposes, irrespectively of the fact that \emph{his} land could not be used in this way according to the plan. However, in many other cases, the disregard rule is upheld even when it is hard to see it as either fair or just, simply on account of it having status as a general rule.\footnote{For instance in \emph{Malvik}, Rt. 1993 p. 409, where owners of property used for a motorway were only entitled to compensation based on current agricultural use because the regulation for motorway use was assumed binding for the compensation assessment.} 

One example is found in \emph{Sea Farm} which dealt with the issue of whether or not the owner of a commercial property should be awarded compensation for the value of investments carried out by the previous tenant.\footnote{Rt. 2008 p. 240} There was no doubt that the owner was entitled to these investments, but since the acquiring authority was the only purchaser who was likely to benefit commercially from them, no compensation was awarded for the loss of these investments. This, in particular, followed from a strict reading of the requirement that compensation should be based on the foreseeable use that an "average" buyer could make of the property, encoded in Section 5 of the Compensation Act 1984. Adherence to the wording used in the act seems to have taken priority over an assessment based on the facts of the case. It seems difficult to argue that it would be either unjust or unreasonable, in particular, to compensate the owner for investments that would prove commercially valuable to the acquiring party.\footnote{The decision was sharply criticized by a former supreme judge \cite{skog}.}

In my opinion, this example illustrates how the development of compensation law towards greater reliance on specific rules rather than concrete assessment based on general principles can be harmful, and how it also threatens to undermine the idea behind the special procedure used to decide appraisal disputes, which has a long history in Norwegian law.\footnote{One might ask if it has status of constitutional customary law, especially since it concerns the mechanism by which a constitutional rule is meant to be upheld.} It also seems to severely underestimate the extent to which compensation rules, when applied to concrete cases, must and should be interpreted based on the context of the case. It seems difficult, if not completely impossible, to achieve social fairness and individual justice by a set of specific rules on the basis of which all legal issues can be resolved mechanically by blind application of such rules. %Moreover, it would be wrong to think that Section ... of the Appraisal Act 1917, encoding the principle that laymen should take part in the decision-making both with regards to legal and technical matters that arose in appraisal disputes.

In the following section, I will present a compensation principle that serves to underscore this point while also allowing me to turn my attention specifically towards waterfalls and hydropower schemes. The principle in question is known as the {\it natural horsepower method} for calculating compensation for waterfalls following expropriation. It was initially developed by the appraisal courts as an ad hoc approach to ensuring some benefit sharing in hydropower cases. Later, however, many came to regard it as a binding principle of customary law. Gradually, it came to be applied by the courts with little or no regard for how well it suited the circumstances of the case and the changing realities of the hydropower sector. As a result, the method became hopelessly outdated, leading to compensation payments that had little or nothing to do with the actual value of waterfalls for hydropower. 

%Today, while the method has been abandoned for certain case types, it is still applied as the default rule for compensation waterfalls.
%
%
% address this issue in more detail, and we will argue for a different conceptual approach to compensation law, grounded both in the procedural tradition of appraisal courts and the more subtle parts of the absolutist and pragmatic theoretical traditions. It seems to me that the most striking lesson that should be drawn from considering the history of Norwegian compensation law is that a \emph{contextual} view of compensation has been a common denominator that both the absolutist and pragmatist camps have endorsed. Unfortunately, this common element was overshadowed by political conflict regarding the weighing of different values. However, there can be little doubt that social fairness and individual justice should \emph{both} to be regarded as important objectives for compensation rules. Moreover, while they may sometimes be opposing, they need not be, and their exact relationship depends largely on the circumstances. It seems to us that it is simply inappropriate to let particular political sentiments regarding their relationship and relative importance, sentiments that are usually dependent on the particulars of the prevailing political, social and economic conditions, dictate the development of the legal framework for resolving compensation disputes.
%
%Considering current trends and recent issues in expropriation law, particularly related to commercial expropriation, further suggests that a different perspective is needed on this matter. In particular, we believe it is time to recall the idea of the independent and impartial discretion of the appraisal court, relying on the good common sense of laymen as well as the legal expertise of judges. The appraisal courts should in our opinion be set with the task of more actively evaluating how fairness and justice is best served in individual cases, at least if the overall goal is truly to arrive at a socially fair and individually just compensation system. We discuss this idea in more detail in the final section below.

\section{``Natural horsepowers''}

As the market for waterfalls dwindled after the introduction of the tight regulatory regime in the early 20th century, the question arose as to how compensation should be calculated following expropriation in favour of public hydropower projects. This question, if resolved by a standard no-scheme approach, could easily result in no substantial compensation being paid at all, since commercial hydropower development was no longer regarded as a foreseeable use of a waterfall. This was not seen as an acceptable outcome, however, and instead the courts decided to introduce benefit sharing by constructing an artificial market for waterfalls.

This market was modeled after the actual market that had existed prior to the licensing acts and the subsequent monopolization of the electricity sector. As I mentioned in Chapter \ref{chap:4}, the lack of a national grid at this time meant that the value of a hydropower plant was measured in terms of the stable effect that the plant could deliver, not the total amount of electricity that could be produced. Hence, on the market for waterfalls, the price a purchaser was willing to pay was heavily influenced by the degree of regulation they thought could be implemented in the water course. To simplify the valuation process, a practice developed whereby waterfalls would be valued at a certain price per {\it natural horsepower}.

As I remarked in Chapter \ref{chap:4}, the notion of a natural horsepower is  used in many contexts, for instance to determine what kind of licenses a development project requires. The use made of it to calculate compensation for waterfalls had no legislative basis, but arose as a result of the appraisal courts' efforts to calculate a realistic market price. After the natural market disappeared, method stuck and was applied as a matter of customary law.
%
%
%prove shockingly unfair to owners of waterfalls. Presumably, since waterfalls could not be exploited for any significant commercial gain except through hydro-power exploitation, disregarding the hydro-power scheme when calculating compensation could lead to nil or close to nil being awarded to the owner. But this was not seen as an acceptable outcome, and instead the Norwegian courts introduced a special method to compensate waterfalls that gave the owner a \emph{share in the value of the hydro-power scheme} for which expropriation was taking place.
%
%Norway did not at this time have any legislation specifically aimed at regulating compensation following expropriation, and when formulating the special rules for compensation of waterfalls, the Norwegian courts seems to have relied on an analogical application of the gross valuation techniques introduced in the Industrial Concession Act 1917 and the Watercourse Regulation Act 1917.\footnote{Act No. 17 of 14 December 1917 relating to Regulations of Watercourses and Act No. 16 of 14 December 1917 relating to Acquisition of Waterfalls, Mines and other Real Property}. Neither of these acts were aimed at compensating owners, but they relied on methods for assessing the potential and significance of hydro-power projects with respect to the question of whether or not a special concession from the State was required.\footnote{To acquire the waterfall and the right to regulate the water-flow respectively.} In effect, by relying on the methods of valuation introduced there, the compensation mechanism that was introduced deviated completely from the "value to the owner" principle. On the other hand, it also closely mimicked the manner in which owners of waterfalls would be compensated on the market in the early days, prior to the introduction of our concession laws, when speculators would pay for waterfalls on the basis of what they assumed to get out of them in subsequent hydro-power projects.
\footnote{References.}

In the Supreme Court case of \emph{Hellandsfoss}, some 80 years after it was first introduced, the natural horsepower method for compensation was described as follows:\footnote{Rt. 1997 s. 1594.} 
\begin{quote}
The principle set out in the Compensation Act, Section 5, is that compensation should be determined on the basis of an estimation of what ordinary buyers would pay for the property in a voluntary sale, taking into account such use of the property as could reasonably be anticipated. For waterfalls, however, this often offers little guidance, and the value of waterfall rights have traditionally been determined based on the number of natural horsepowers in the fall, which are then multiplied by a price per unit. The unit price is determined after an overall assessment of the waterfall, including the cost of the scheme, its location, and levels of compensation paid for similar types of waterfalls in the past. The number of natural horsepowers is calculated by the formula ``natural horsepower = $13.33 \ \times \ Qreg \ \times \ H$'', where $Qreg$ is the regulated water flow and $Hbr$ is the height of the waterfall.
\end{quote}

In this formula, $Qreg$ represents a quantity of water, measured in cubic meters per second (m3/sec), while $H$ represents height  measured in meters. Horsepower is an old-fashioned measure of effect, and in the standard account of the natural horsepower method, it is said that the number of natural horsepowers in a waterfall is a measure of gross effect, giving us the amount of ``raw'' power in the waterfall.\footnote{See \cite{Falk}(in Norwegian)}

However, it is not hard to see that the natural horsepower is not calculated based on the qualities of the waterfall as such, but depends on the level of water regulation that a given hydropower project entails. The Supreme Court glosses over this point when it speaks of the ``natural horsepowers in the fall''. But on a closer reading, one notices the significance of the fact that the formula for calculating natural horsepower depends on the variable Qreg, which is said to measure the ``regulated water flow''. Hence, the notion of natural horsepower is not intrinsic to waterfalls, but arises only relative to concrete development plans.

It is more correct, therefore, to speak of the natural horsepower of a regulation project, rather than a waterfall. The notion essentially takes the level of proposed regulation as input and produces as output the gross level of electric effect that this regulation would result in if harnessed by a hydropower plant in the waterfall in question. As such, it does not reflect the total amount of energy that may be harnessed from the waterfall. The electricity generator can run at different effect levels depending on the actual water flow, meaning that far more electricity may be produced than indicated by the natural horsepower of the regulation. Hence, the notion of natural horsepower has little relevance also to the performance of the hydropower plant that is being proposed. It only measures the theoretically maximum level of stable electric effect that this plant can output, irrespectively of the weather conditions.

As mentioned in Chapter \ref{chap:4}, this quantity was important 100 years ago, when there was not electricity grid and energy producers were paid based on the stable effect they could deliver. Today, however, the natural horsepower of a regulation does not have much to do with the value of neither waterfalls nor hydroelectric plants. An additional important observation is that horsepower is no longer used as a measure of effect in the energy business. 

Instead, kilowatt (kW) is used, and electricity producers are paid per kilowatthour (kWh) they are able to produce.\footnote{1 Kilowatt(Kw) = 1000 Watt(W)} Hence, the annual income of a hydroelectric plant is a function of the price paid per kilowatthour and the total number of kilowatthours harnessed over the year (kWh/year). The amount of energy generated in a power plant could be measured in other units than kWh, e.g. in terms of the amount of horsepower-hours per year. The important point to keep in mind is that an energy producer gets paid for the amount of energy he can deliver, \emph{not} the effect he can maintain in his station constantly. %Hence, even if if we uwould se kilowatt instead of horsepower and talk of the natural kilowatt of a hydropower plant, the quantity we are discussing is the same, and still has little or no bearing on the value of the waterfall.

Another problematic aspect of the natural horsepower method has to do with how the regulated water flow, Qreg, is measured.
First, I note that the licensing condition of the \cite{wra17}, which depends on the notion of natural horsepower, suggests a definition of $Qreg$:\footcite[2]{wra17}

\begin{quote}
Waterfall regulation for the production of electric energy which increases hydro power:
\begin{itemize}
\item[a)]            by at least 500 natural horsepower in one or several waterfalls which can be developed 
                 collectedly, or

\item [b)]             by at least 3.000 natural horsepowers in the whole watercourse, or

\item [c)]              which alone or together with earlier regulations significantly affects the environmental
                 conditions or other public interests, 
\end{itemize}
can only be exploited by the State or a developer who obtains permission from the King. \\ 
If regulation of a watercourse increases the water-power in the river by at least 20.000 natural horsepowers, or if there are essential conflicting interests, then the case should be submitted to Parliament before license is given, unless the Department finds it unnecessary.

The increase of the hydro-power according to the first and second point is calculated on the basis of the increase of the low water-flow of the watercourse, which the regulation is supposed to cause beyond the water flow which is considered foreseeable for 350 days a year. When making the calculation it is to be assumed that the regulation is operated in such a way that the water-flow during the low water periods becomes as even and regular as possible.  
\end{quote}

In the third paragraph, $Qreg$ is defined as the {\it increase} of the low water flow, the amount of water present in the waterfall for at least 350 days every year. That is, $Qreg$ is found by first estimating the amount of water that one can expect to be present for at least 350 days of the year and then subtracting the water that is expected for 350 days without regulation.

Regulation of a watercourse can involve building a reservoir and/or installations that transfer water from one river to another. Then, if there is excess water, for instance due to flooding, water can be stored in the dam for later use. Hence, if there is drought, the stored water can be released. In this way, it becomes possible to even out the water-flow over the year. This means that the water which is guaranteed to be present for at least 350 days a year will typically increase. In light of the definition of $Qreg$, it is clear that the definition of natural horsepower depends crucially on the level of regulation involved in the planned hydro-power project. 
But the definition only takes into account the gross effect resulting from the \emph{increase} in low water-flow following regulation. 
It follows that if the planned project is a run-of-river project not involving regulation, a type of project that is commonly seen today, the number $Qreg$ will by necessity be $0$ and the waterfall will be deemed not to posses any natural horsepower at all. As a consequence, if this definition is used to calculated compensation, no compensation will be paid to the owner at all if the developer proposes a run-of-river project.\footnote{In fact, things could become even worse for the owner, since the proposed project might lead to $Qreg$ becoming a {\it negative} number. This follows from section 10 of the \cite{wra00}. Here the NVE is given the power to compel the owner of a hydropower scheme to ensure that a certain quantity of water is always allowed to pass through the intake of the hydropower plant. This flow of water is typically referred to as the {\it minimum water flow}. It is typically imposed to reduce the environmental damage of the hydropower plant. For many run-of-river schemes, the minimum water flow ordered by the NVE is higher than the low water flow before regulation. For schemes involving a high degree of regulation, it is common for the minimum water flow to be subtracted from the regulated water flow, to account for the fact that this water cannot actually be used for hydropower production. This was done, for instance, in the case of {\it Sauda}. However, if the same logic is applied to run-of-river schemes, it will give rise to a {\it negative} $Qreg$. Hence, if the natural horsepower method is applied slavishly for such schemes, the outcome would be that the owner should pay the developer who expropriates his waterfalls, not vice versa.}

In practice, however, the traditional method not been applied in this way. Instead, it has been common to take $Qreg$ to be simply the regulated flow of water after regulation, without subtracting the low water flow that existed prior to development. But even after this modification, talk of natural horsepower serves to give a drastically skewed picture of the potential of a waterfall, especially for run-of-river projects. It is not unusual, in particular, that the low water flow in a river amounts to only about 3-5 \% of the average water supply. In modern hydropower projects, one would expect 70-80 \% of this water-flow to be harnessed for energy production even in the absence of any regulation. Hence, in these cases, the natural horsepower method only compensates the owners for about 5 \% of the energy that can actually be harnessed from their waterfall.

This observation, which is trivial given a basic understanding of physics and the hydropower industry, was not made in a legal context until the early 21th century. The inadequacies of the natural horsepower method were noted only when growing interest in small-scale hydropower led to conflicts in which local owners themselves protested its use, against the advice of legal experts.\footnote{In the aforementioned case of {\it Hellandsfoss}, for instance, a local owner raised the issue with his legal council, who advised against raising it as an issue before the appraisal courts. Source: Private correspondence.}
At the same time, however, the Norwegian government has certainly been aware of the shortcomings of the method, as illustrated for instance by the following passage, taken from a report presented to parliament in 1991-1992.\footnote{Ot.prp. No 50 (1991-1992) p 19, discussing the notion of natural horsepower in connection to the uses made of that term in other parts of the law.}

\begin{quote}
The Ministry of Petroleum and Energy has considered moving a proposition for changing the hydrological definitions in the Industrial Concession Act 1917 and the Watercourse Regulation Act 1917. Today the act uses a calculation method based on an increase in regulated water flow, i.e. that of natural horsepower.[.......] The hydrological definitions of these acts, supposed to indicate how much electricity can be generated, were made on the basis of technical and operative conditions differing very much from contemporary circumstances. In implementing the definitions referred to above one has tried to adapt to the new technological realities of the present day. Therefore, in practice, a calculation based on current production is used instead. From several quarters, particularly the Association of Waterfall Regulators, there has been raised a strong wish to authorize this practice by altering the definitions of the relevant laws. The Department of Oil and Energy agree, but have not as yet made a sufficient elucidation of the issues to be able to move a proposition of alteration of these acts.
\end{quote}

Within the ranks of the water authorities, it has actually been well-known for decades that the notion of a natural horsepower fails to give an adequate picture of the potential that a waterfall has for hydro-power. The development of new technology had made this apparent already in the 1950s, when it was also raised as an issue, specifically with respect to compensation following expropriation, by a director at the NVE, who commented, in 1957, that he failed to see how the traditional method could be an adequate means for valuating waterfalls.\footnote{See \cite{....}. The director even went as far as to illustrate a different method, which would also be outdated given today's regulatory regime, but which would reflect contemporary \emph{actual} valuations, used by the NVE itself.}

Considering the physics behind the traditional method is enough to reveal that it fails to give rise to valuations that reflect the value of waterfalls, under any reasonable set of assumptions about the correct general compensation principles one should adopt. In addition, the method relies on data from the expropriating party's project, meaning that the compensation to the owner depends not on their loss, but on the technical details of the project that the expropriating party proposes. This clearly deviates from even a narrow interpretation of the no-scheme principle. 

However, while the idea of compensating the owner of waterfalls by a price per natural horsepower is fundamentally flawed already at the level of physics, there are even more serious concerns that arise when one begins to consider the way in which the \emph{unit price} has been determined. The traditional approach to this question has had a particularly dramatic effect on the level of compensation payments. In case law based on the traditional method, it is often said that the price set per natural horsepower is set according to ``market price'' for waterfalls. But for the most part, what this means is that the court looks to prices awarded in earlier compensation cases, not to prices obtained in voluntary sales.

This, in turn, gave rise to a price level that is entirely artificial, reflecting, more than anything else, the power balance between buyer and seller in the courtroom. It was certainly no genuine market value, even if it was described as such. Indeed, while the prices paid did see some increase during the first 80 years that the traditional method was used, this hardly reflected neither the increase in value of hydropower nor the general level of inflation.\footnote{References needed.} Moreover, while the price-level was determined by the courts, there were also some cases of voluntary agreements that used the same method, and could thus be used to justify it's status as a genuine market-based valuation principle. But these cases tend to reveal, more than anything else, the imbalance of power between owner and purchaser that developed over time, as a result of the natural horsepower method.

For instance, as late as in 2002 a waterfall belonging to local landowners in the rural community of Måren, in Western Norway, was sold for the sum of kr 45 000 (roughly £ 4500), based on traditional calculations. The waterfall has now been exploited in a small-scale hydro-power plant belonging to the large energy company BKK, with annual energy output of 21 GWh.\footnote{$http://www.bkk.no/om_oss/anlegg-utbygging/Kraftverk_og_vassdrag/andre-vassdrag/article29899.ece$} For comparison, I mention that in the case of \emph{Sauda}, where a more realistic market-based method was used, the owners received a compensation which totaled about 1 kr/kWh annual production.\footnote{LG-2007-176723 (I acted as council for some of the owners in this case).} Applied to the Måren case, this would take the compensation from kr 45 000 to kr 21 000 000. That is, almost 500 times more would have had to be paid.\footnote{In fact, the Måren waterfalls were cheaper to exploit, so in reality, one would expect that the new method applied to Måren would yield even greater compensation per kWh. I also remark that the value awarded in \emph{Sauda} was market-value, not value of use, since it was assumed that the owners would have to cooperate with a ``professional'' energy company to develop hydropower. This, in effect, halved the compensation awarded, since the premise for the Court was that the professional company was willing to pay about 50 \% of the profit as rent to the owners.}

The case of Måren illustrates an important point, namely that when the traditional method was used, and described as the ``market value'' of waterfalls by the courts, this could become a self-fulfilling prophecy. The prices paid in voluntary transactions were influenced by the practice adopted by the courts far more than the other way around. This, indeed, appears to be a general danger in cases when expropriation is widely used for the purpose of commercial development. Then, it seems, prices paid can easily be kept artificially low by developers making use of expropriation as soon as prices begin to rise. In that way, by relying on what is ostensibly ``market value'' compensation, an artificial price level can be established and maintained. That this mechanism can be severe is nicely illustrated by the case of Norwegian waterfalls. In my opinion, preventing this mechanism is a main challenge associated with regulatory systems that enables developers to make extensive use of expropriation to further economic development.



\section{{\it Kløvtveit} and {\it Otra Kraft}}

Following the liberalization of the Norwegian energy sector in the 1990s, the traditional method came under increasing pressure. It was argued to be unjust by owners who felt that they were being deprived of a valuable commercial assets, and it was held to be illogical by engineers working on developing small-scale hydro-power.\footnote{References} Eventually, legal professionals followed suit, and came to the realization that established rules based on market value could now be applied. Indeed, a new market for waterfalls had begun to develop at this point, following the increased interest in small-scale hydro-power and the formation of new companies specializing in cooperating with local owners. For transactions of rights to waterfalls taking place in this market, the traditional method of valuation was not used, and waterfalls were rarely sold at all, but rather leased to the development company for an annual fee. Typically, this fee was calculated by fixing a percentage of the energy produced during the year, and compensating the owners of the waterfall by multiplying this with the market price for electricity obtained throughout the year, possibly deducting production specific taxes, but with no deduction of other cost. In effect, owners would get a fee corresponding to a set percentage of annual gross income in the hydro-power plant.\footnote{References}

Usually, the fee entitles the owners to 10-20 \% of the income from sale of electricity, depending on the cost of the project. Moreover, it is common that the owners are entitled to up to 50 \% of the income derived from so-called \emph{green certificates}, a support mechanism for new renewable energy projects, corresponding to the Renewables Obligation in the UK.\footnote{See http://www.ofgem.gov.uk/Sustainability/Environment/RenewablObl/ for further details.} Essentially, and somewhat simplified, the scheme allows the energy producer to collect a premium on his sale of electricity, which, owning to its "green" status, is valued more highly by buyers (usually electricity suppliers), who are required to ensure that a certain proportion of the energy they offer to their customers (usually consumers, like you and me) is considered green. In Norway, such a scheme has been talked about for years, but was only put in force in 2012.\footnote{http://www.regjeringen.no/en/dep/oed/Subject/energy-in-norway/electricity-certificates.html?id=517462} Currently, energy producers can claim a premium of about 2 pp per KWh per year, meaning that about a third of the annual income for new renewable energy projects comes from the sale of green certificates.\footnote{While the premium must be expected to go down somewhat as the certificate market matures and more energy producers acquire "green" status, it will certainly remain an important source of extra income for renewable energy producers also in the future.}

In light of the fact that the agreements on the new market are based on leases that tie compensation to the fate of the particular hydro-power project that is being undertaken, several questions arise when attempting to value waterfalls by looking to this market. If a given project has been identified as providing the basis for valuation, the task is difficult, but mainly a question of factual assessment. The valuer have to determine first what the annual production would be and also determine the costs of carrying out the project. Then, on this basis, he must move on to determine what the annual fee would be, and then, in order to complete the process, he must stipulate what the price of electricity, and of green certificates, is likely to be for the next 20 years or so. On the basis of this information, it becomes possible to determine the annual income to the owner of the waterfall over a period of 20 years, and then one would also have a basis upon which to calculate a reasonable present-day value of the scheme to the owner of the waterfall. 

Indeed, this is the model that has been used in the cases that have been before the courts and where the traditional method has not been adopted. The first such case was \emph{Møllen}, and while the Supreme Court rejected the method as it was applied in this case, because it was found that the date of valuation was to be based on prices obtained for waterfalls in the 1960s, they commented that they supported the adoption of the new method in cases when \emph{alternative} small-scale development was deemed a \emph{foreseeable} use of the waterfall in the absence of the scheme.\footnote{Rt. 2008 s. 82.} 
Since \emph{Møllen} the new method has been used in many cases before the lower courts and the Lands Tribunal.\footnote{See for instance \cite{tf1,tf2,tf3}, a series of academic papers discussing the new method (in Norwegian).}

Unsurprisingly, the new method tends to lead to a rather protracted process of valuation, mostly dominated by experts. Moreover, given all the uncertain elements of the calculation, it is typical that the opposing parties produce expert witnesses that diverge significantly in their valuations. While this is problematic enough, the fundamental \emph{legal} challenge arises with respect to the choice made about what scheme the compensation should be based on. This becomes especially tricky if one attempts to follow a standard no-scheme approach. In the following, we summarize the main issues that arise.

\begin{enumerate}
\item In the absence of the hydro-power scheme benefiting from expropriation, is it foreseeable that the waterfall would nevertheless be used in a hydro-power project?
\item If the answer to Question (1) is yes, what would such a foreseeable project look like?
\item Is it foreseeable that an alternative project would get planning permission?
\item Does the no-scheme rule imply that the project benefiting from expropriation cannot be regarded as the foreseeable use for the purpose of compensation?
\item Can the fact that the scheme underlying expropriation obtained planning permission be taken as evidence to support that alternative uses of the waterfall would not be given planning permission?
\item How should compensation be calculated if it is determined that no alternative project is foreseeable? 
\end{enumerate}

In some cases, for instance when the project benefiting from expropriation is not commercially viable but is carried out for public purposes with the help of special State funding, the answer to Question (1) might be no. However in most cases, the question will be answered in the affirmative, since the scheme benefiting from expropriation already serves as an indication that the waterfall can be harnessed for energy. However, here the no-scheme rule comes into play and creates severe difficulty once we reach Question (2). For what kind of scheme can be assumed foreseeable all the while we are obliged to disregard the scheme underlying expropriation? In most cases that have come before the courts so far, the owners have claimed that alternative development in small-scale hydro-power should serve as the basis for compensation, and such cases the problem of the no-scheme rule appears to have been circumvented. This, however, is not necessarily the case. It appears, in particular, that the answer to Question (3), asking about the likelihood of planning permission, might again depend on how you view the no-scheme rule. It seems, in particular, that anyone who answers Question (5) in the affirmative, looking to the planning permission actually given to the expropriating scheme for evidence, might be inclined to say that the alternative project could not expect to get planning permission, and that this is so \emph{because} planning permission was granted to the expropriating party (or this line of reasoning might be sugarcoated by pointing to whatever underlying reasons the authorities had for considering the scheme underlying expropriation the optimal use of the waterfall).

Then the question arises: Is someone who reasons like this at odds with the no-scheme rule? It would seem so, but remember the earlier discussion on the no-scheme rule in Norwegian law, where we noted that the rule has tended to be applied much more narrowly along its positive dimension. Following up on this, it can be argued, in keeping with the general tendency in how the law is applied in Norway, that while the expropriation scheme is to be disregarded for the purpose of compensation valuation, the regulation underlying the scheme -- or at least the rationale behind this regulation -- is nevertheless to be taken into account. If this point of view is taken, then the conclusion can easily become that alternative development is to be regarded as unforeseeable, and that the reason why this must be the case is precisely the fact that the expropriation scheme received planning permission. Indeed, this line of reasoning was given a stamp of approval in the recent Supreme Court case of \emph{Otra II}. Here, the presiding judge made the following remarks, quoting Gulating Lagmannsrett (the regional Court of Appeal), expressing his support, and adding a few comments of his own.

\begin{quote}
"[....] The Court of Appeal finds it difficult to distinguish this case from other cases when it has been established that alternative development is not foreseeable. It does not seem relevant whether this is the case because the alternative is not commercially viable or because the alternative must yield to a different exploitation of the waterfall" 
I agree with the Court of Appeal, and I would like to add the following: As the survey of the general principles have shown, it is assumed, both in the Expropriation Act, Sections 5 and 6, and in case-law, that only the value of a foreseeable alternative should be compensated. This starting point means that it would be in breach of the general arrangement if a waterfall that can not be used in foreseeable small-scale hydro-power was to be compensated as if it could be put to such use.
\end{quote}

Having used the planning permission granted to the expropriating party as evidence that alternative development was unforeseeable, the Court needed to answer Question (6) by coming up with some alternative way of compensating the owners.  To do so, the Court also had to answer Question (4), however, asking whether or not the scheme underlying expropriation could be taken into account at this stage. Moreover, one would think, given how the expropriation scheme was used as an argument when answering Question (5) in the negative, that it \emph{could} be taken into account. This, however, was answered in the negative by the Supreme Court in \emph{Otra II}, where the presiding judge reasoned as follows.

\begin{quote}
Based on the arguments presented to the Supreme Court, I find it safe to assume that there does not today exist any market for the sale and leasing of waterfalls for which alternative development is not foreseeable, but where the waterfalls can be used in more complex hydro-power schemes. The appellants have not been able to produce documents or prices to document the existence of such a market
\end{quote}

Let us first remark that it is hard to imagine how a market such as that asked for here could ever develop, all the while alternative buyers, by the courts own reasoning, are excluded from being taken into account. In this case, if there was to be a market, it would have to be down either to the regular benevolence of the expropriating parties in cases like these, or to the government compelling them to enter friendly negotiations. In the Norwegian context, both seem unlikely. More important, however, is the implicit assumption that in order to value the waterfall according to its potential for hydro-power production, a market needs to be identified. It is \emph{not} considered sufficient that the scheme for which expropriation takes place is itself a hydro-power project, on the basis of which the value of the waterfall could be assessed following exactly the same steps as in the new method.

In fact, the Supreme Court's reasoning in \emph{Otra II} serve as an excellent example of the type of reasoning that makes the no-scheme rule highly problematic for cases of expropriation that benefit commercial schemes. Indeed, it seems to follow that the scheme itself should not provide a basis for calculating the compensation, but then, on other hand, it also appears that there really is no other way to calculate it, seeing as the State, by giving permission to carry out the scheme, have effectively acted in such a way as to make an alternative use \emph{of the same kind} seem inherently unforeseeable. This is not so much of a problem when the use that the owner could make of the property has a different character than the scheme, since in this case, if we disregard the scheme, this other use might seem foreseeable. However, when the alternative use of the property is of \emph{exactly the same kind} as the use made of it by the scheme, it does seem counterintuitive, as noted by the Court of Appeal in \emph{Otra II}, to regard it as foreseeable, all the while the scheme will tend to appear the more rational form of exploitation.

It seems, however, that when taken to its logical conclusion, this line of reasoning, based on the no-scheme rule, leads to an offensive results; the commercial value of the property is not to be compensated because the optimal commercial use of the property is the use that the expropriating party aims to make of it in the scheme underlying expropriation. Note that the conclusion is not just that this optimal value, inherent in the scheme, should not be compensated. No, the conclusion in \emph{Otra II} was that \emph{no} compensation could be estimated for any use of the same \emph{kind}, since such use was not foreseeable, owing to the absence of a market.

It is certainly possible to argue that this decision represent a misguided application of the no-scheme rule. In effect, the Supreme Court allowed the planning permission given to the expropriating party to act as evidence that alternative development was unforeseeable, while it used the no-scheme rule to argue that the hydro-power scheme for which this planning permission was given could not itself form basis for compensation payment based on market value.  On the other hand, it seems that even if we disregard the scheme completely, it is unnatural to base the compensation payment on the value of a hydro-power scheme that is less beneficial, both commercially and in terms of resource efficiency, than the scheme for which expropriation takes place. Such a scheme would not, one must presume, \emph{actually} have been carried out, regardless of the questions of whether or not it would have been given planning permission. But it does seem particularly difficult, intuitively speaking, to predict what use would have been made of the waterfall if these were the facts: more or less exactly the same scheme as that underlying expropriation would be implemented, but by some from of voluntary agreement with the owners, not by means of expropriation. 

In \emph{Otra II}, however, this line of thought was also rejected, although this was, in part, due to the point not having been argued before the Court of Appeal. But then the question arose as to how exactly compensation should be calculated. The answer, following up on the "value to the owner" principle so forcefully adopted elsewhere in the judgment, would appear to be that no compensation should be paid at all, save perhaps for loss of fishing rights and the like. This, however, was \emph{not} the conclusion reached by the Supreme Court. Instead, the Court states that a return to the traditional method is in order. However, they do not apply it in the traditional way. Rather, they casually sanction the replacement of regulated low water-flow in the definition of $Qreg$ by the average water-flow, thus moving away from compensation based on the level of regulation, to compensation based on average effect in the project. Moreover, they also sanctioned the use of a significantly increased unit price compared to earlier times.

What to make of this? It seems hard indeed to make sense of indeed, since effectively, by relying on the traditional method, the Supreme Court contradicts its own conclusion that compensations should be based on market value. Instead, they rely on a method that, in effect, is based on an attempt to quantify the value of the waterfall as it is being used by the expropriating party in his project. However, by relying on a technical method that has been completely outdated, and have lived its own life in the courts, it becomes difficult to assess the outcome properly, at least for a non-expert. It seems, in particular, that the Supreme Court prefers the obscurity of the traditional method, and its status as an established principle, over the possibly radical conclusion that, in cases such as this, it simply is not tenable to adopt the "value to the owner" principle, as least not as construed in Norwegian law.

Indeed, it is simple enough to be critical of the Supreme Court based on the fact that they regarded alternative development as unforeseeable in this case, when the planning permission granted to the expropriating scheme itself appears to have provided the decisive bit of evidence. Still, it is not possible to escape the fact that this reflects a general tendency in Norwegian law, and so, even if it appears unreasonable, it might very well be a correct application of national law. Moreover, it could very well have been that alternative development was unforeseeable for \emph{some other reason}, for instance because the only commercially viable exploitation was the scheme planned by the expropriating party. In this case, the problem of how to compensate the owners in the absence of an alternative form of exploitation would still arise. It is this question, in particular, which seems entirely unsatisfactorily resolved under an application of a "value to the owner" principle.

This is witnessed by \emph{Otra II}, and, in fact, it appears that the Supreme Courts decision \emph{not} to follow their own reasoning to its logical consequence is the main lesson to be learned from the case. For all intents and purposes, the Supreme Court \emph{rejects} the "value to the owner" principle, but they obscure this by wrapping it up in the traditional method, which is deeply flawed. However, the problem it attempts to solve appears significant, and it pertains directly to the question discussed more generally in Section \ref{sec:noscheme}, namely how to apply the "value to the owner" principle with respect to commercial schemes. It seems that even the fiercest supporters of limiting owners' right to compensation tend to find it too offensive to apply this principle when it leaves the owners with no form of compensation for giving up property to multi-million, purely commercial undertakings. Indeed, such a practice would surely also be in breach of the human rights law. It seems, in particular, that the subjective aspect of the "value to the owner" principle is impossible to maintain. Indeed, if the commercial value falls to be disregarded for no other reason than the fact that the State happens to have granted planning permission to the expropriating party rather than the owner, this is not only dubious with respect to human rights protecting property, but also appears to be a case of \emph{discrimination}, e.g., as prohibited by ECHR Article 14.

The problem does not arise when the buyer sees value in the property that is of a different \emph{kind} than that realizable by \emph{any} private owner. In this case, the rule simply states that the owner should not be able to demand that "public value" is transformed into commercial value just for him. This appears like a reasonable principle. But when there is commercial value already present on the "public" side of the transaction, it seems completely unwarranted that the public should be allowed to transfer this value from the owner to someone else without compensation. Thus, it seems that more accurately and acceptably, the "value to the owner" principle should be thought of as a "commercial value" principle. It seems, in particular, that the principle need to be stripped of any suggestion that a preferential financial position is to be awarded to whoever benefits from expropriation.\footnote{Exceptions might be possible to imagine, but, one would think, only when they can be construed as falling under the "public value" banner in some way.}

It seems unfortunate that this aspect has not been made explicit, and the difficulties that arise in the absence of this nuance seems nicely illustrated by the case of Norwegian waterfalls. Still, as the case of \emph{Otra II} seems to indicate, an interpretation of the "value to the owner" principle along less offensive lines is in reality already in place with regards to Norwegian hydro-power. Here, it seems that "value to the owner" has in fact \emph{never} been applied in the traditional way. Hopefully, rather than obscuring this fact by relying on an unsatisfactory and artificial method for calculating the compensation, the future will see further developments that recognize the need for new principles. It should be recognized, in particular, that as the law has been applied for the last 80 years, despite its grave flaws and injustices, there has always been an implicit recognition in Norwegian law that the owners of waterfalls are \emph{entitled to their share} of the commercial benefits of hydro-power. 

In fact, in the recent Supreme Court case of \emph{Kløvtveit}, a further illustration of this is found. The conclusion here was also that alternative development was not foreseeable. However, unlike in \emph{Otra II}, the Court of Appeal had compensated the owners based on the fact that they regarded it foreseeable that in the absence of the scheme, the waterfalls would have been exploited in exactly the same way, except that it would have happened in the form of \emph{cooperation} between the owners and the expropriating party. By this line of reasoning, the Court effectively seems to have adopted a more rational "commercial value" principle, to replace the traditional method. 

Indeed, is it not always the case, at least under objective standards of assessment, that when alternative development is unforeseeable, then a rational alternative buyer -- assumed to operate in a world where there are no "powers of compulsion", to paraphrase Lord Nicholls in \emph{Waters} -- would look precisely to the likely possibility of cooperating with the expropriating party? This, on the other hand, would \emph{never} be a safe assumption to make for non-commercial aspects which, in the absence of commercial potential would not give an alternative buyer financial incentive to do so.

We mention that \emph{Kløvtveit} was discussed in \emph{Otra II}, and that the presiding judge made some reflections, focusing on what he regarded to be "practical problems" with cooperation. However, this was not crucial to the decision, since the cooperation model was not argued for by council. In light of this, one can only hope that \emph{Kløvtveit}, rather than \emph{Otra II}, will become the influential precedent for future cases.

\section{Conclusion}

We have presented an overview of Norwegian law relating to compensation following expropriation. First, we identified two different strands of thought regarding this matter, which we referred to as absolutist and pragmatist respectively. We noted that the tension between these two perspective became aggravated in the 60s and 70s, when legislation was passed with the explicit intent of bringing compensation payments down and to enforce a more pragmatic approach. The legacy of this era was a lasting pragmatist turn in compensation law, but also a greater centralization of power regarding the assessment of appraisal disputes. In \emph{Kløfta}, the Supreme Court modified some pragmatist rules introduced by parliament, but they also sanctioned a range of disregards that reflected the pragmatic intent behind these rules. Moreover, they assumed a greater role in providing special rules for the appraisal courts to follow in these matters, hence limiting the role of the laypeople in the appraisal process, and thus also changing the character of this process, which has long roots in the Norwegian legal tradition.

We focused particular attention on this latter change in the law, and we argued that it has resulted in an overly simplistic and often unhelpful narrative regarding compensation. Moreover, we argued that it inadvertently went against one crucial principle that more subtle thinkers from both the absolutist and pragmatist camps agreed on: the need for concrete fairness assessment. We went on to suggest that the importance of this principle is further accentuated today, when the context of expropriation is often quite different from the standard assumption of property taken for the public good. Often, the economic system currently in place, and the widespread use of expropriation that has followed the advent of extensive planning law, leads to expropriation appearing primarily as a means for commercial actors to make a profit. It might be hard to directly address the legitimacy of this in legal terms, by demanding that courts take an active role in interpreting the public interest requirement. But then the nature of compensation rules applied to such cases becomes a crucial special question. If dealt with in the right way, compensation can be used to achieve greater fairness in such cases, and also, more importantly, can serve as an effective safeguard against excesses. 

Commercial companies, presumably, only want to use expropriation as long as this is the most \emph{profitable} or \emph{practical} manner in which to acquire property. Moreover, it seems that a system where expropriation regularly comes to be used for this reason would have to be regarded as inherently flawed by both pragmatists and absolutists. Hence there should be cause for reaching common ground on the principle that compensation rules needs to be such that they prevent commercial companies from profiting merely from being able to use compulsion against other members of society. Achieving success in this regard, however, might not be so easy if one is committed to a top-down approach relying on the introduction of yet more special rules. Rather, justice and fairness might be better served by taking note of the potential inherent in the special way that Norway organizes appraisal disputes. By focusing on the need for concrete fairness assessment, and demanding that appraisers look to the power balance between the parties, the purpose of the expropriation, and the possible commercial interests involved, it seems that much can be achieved. The recent developments in case-law regarding waterfalls illustrates this. One can only hope that the powers that be are not too invested in the idea that \emph{they} are the ultimate authority on fairness, to allow these 
encouraging trends to develop further.

\section{The no-scheme rule in Norwegian law}\label{sec:nonor}

Before 1973, the Norwegian law relating to compensation for expropriation of property was based on case law. The courts would interpret Section 105 in the Norwegian constitution which demands that ``full'' compensation is to be paid. The no-scheme rule was typically applied, such that when assessing the value of the property, the element of compulsion was disregarded, and changes in value that could be attributed to the underlying scheme tended not to be taken into account. As in the UK, difficult questions would arise for comprehensive schemes based on public plans for the use of the property, and it proved difficult to identify any clear rule concerning the distinction between the scheme itself and the regulation of property-use that preceded it.\footnote{References.}

Following the Second World War, there was an increasing trend that effects of regulation \emph{would} be taken into account, but only with regards to the \emph{positive} aspect of the no-scheme rule. That is, the scheme was taken into account in so far as it could be used to argue against alternative development, but not in such a way that it could lead to an increase in the value of the property. In some cases, even the regulation directly preceding, and providing the basis for, the use of compulsion, would be taken into account.\footnote{Such as in Rt. 1970 s. 1028, where a property which had been used by a local business owner was expropriated to implement a public plan that regulated the property for use as a public road with parking spaces. The owner was not compensated for the loss of business revenue, since, according to the majority in the Supreme Court, the regulation of the property had to be taken into account (the decision was given under dissent). The case was somewhat special, however, since the business value could be realized by the owner only if he had been given the opportunity to rebuild his store, following a fire. Moreover, he was already ensured compensation based on the value of the property as a plot for housing, for independent reasons.}

The general picture, however, was that a no-scheme rule applied to underlying regulation of property use as long as there was a causal link between this regulation and the subsequent expropriation.\footnote{References. \noo{Husaas-komiteen}} To apply this in concrete cases often proved problematic, however, as illustrated by the many conflicting opinions voiced about the current law during the preparation of the original Compensation Act from 1973.\footnote{See, for instance, the historical overview given in NOU 2003:29}

In the 1973 Act, a radical rule was put in place to resolve all outstanding issues: valuation should be based on the  \emph{existing use} of the property at the time of expropriation.\footnote{So the Norwegian law mirrored the rule introduced in the UK Town and Country Planning Act 1947 which was later replaced by the current Land Compensation Act 1961.} The rule went further than the no-scheme rule in that it prescribed that compensation should disregard \emph{all} kinds of hypothetical development of the property, notwithstanding their status with respect to existing plans and regulations. But it also involved a break with it, since, on the face of it, the implication would be that any kind of regulation predating the scheme would be taken into account when it provided the basis for the "existing use".

However, in Section 4, nr. 3 of the Act, this aspect of the "existing use" principle was limited by a \emph{separate} provision implementing the \emph{negative} aspect of the no-scheme rule; the value of existing use due to public regulation underlying the expropriation should be \emph{deducted} from the compensation payment. As such, the Compensation Act 1973 implemented a system whereby the positive part of the no-scheme rule would be given a very narrow interpretation -- any scheme or regulation limiting current use was to be be taken into account -- while the negative aspect was explicitly provided for in statute -- the value of existing use that could be attributed to public regulation underlying the scheme was to be deducted.

This new rule was quite controversial, and to make the system more flexible, the 1973 Act included a rule that allowed the Lands Tribunal to increase compensation, on a discretionary basis, by taking into account that value of comparable properties in the district where expropriation took place. Still, property owners felt that the new Act went too far in depriving them of the right to compensation, and the matter came before the Supreme Court in \emph{Kløfta}.\footnote{Rt. 1976 s. 1} After deliberating in plenum, the Court presented a revisionary interpretation of the new Act, essentially judging the intention behind the main rule as being incompatible with the protection of property encoded in the Norwegian Constitution, Section 105. The main step taken by the Court was to regard the discretionary increase of compensation as a \emph{mandatory} step, one that had to be carried out whenever certain conditions were fulfilled. What exactly these conditions amounted to, and how they should be interpreted, was not conclusively resolved, however. Indeed, the confusion that arose after \emph{Kløfta} led to a heated academic debate in Norway, and a long line of Supreme Court cases has since attempted to clarify the current state of the law. \footnote{References.}

In 1984, taking into account the ruling in \emph{Kløfta}, a new Compensation Act was passed, which is still in force today.\footnote{Act No. 17 of 06. April 1984 relating to Compensation following Expropriation of Real Property} According to Section 4 of the Compensation Act, compensation is to be calculated as the highest of either the value of the property as it could be put to use by the owner, his \emph{value of use}, or the \emph{market value}, the payment he could expect to receive from a typical willing buyer. In both cases, a no-scheme rule applies: in Section 5, Paragraph 3, it is stated that when calculating the market value, changes in value due to the scheme is to be disregarded, while in Section 6, it is stated that the value of use should be based on \emph{foreseeable} use of the property. In practice, this has been interpreted as referring to such use as it is reasonable to expect would have occurred in the absence of the scheme.\footnote{References.} 

However, the spirit of the 1973 Act is still clearly felt in Norwegian law, and the no-scheme rule is thought of somewhat differently than in the UK. Firstly, it is common to distinguish more sharply between the positive and the negative aspects of the rule, and unlike in the UK, much, if not most, attention has been devoted to the former aspect, when the scheme is used to justify decreased levels of compensation. Secondly, and specifically as it relates to the positive aspect of the rule, the tendency has been to give "the scheme" a narrow interpretation, regarding public plans and regulation as binding for valuation, even when they are intimately related to the undertaking for which a right to expropriate is granted. Simultaneously, if the plan leads to an increased value that \emph{can not be realized by the current owner}, a "value to the owner" principle typically applies directly, such that this value is not compensated, irrespectively of what the scheme is taken to be.\footnote{References.}

As we mentioned, much attention in Norway has been directed at the positive aspect of the no-scheme rule, and there are some important exceptions to the main principle of regarding public plans as binding for the evaluation. The main exception is that a plan tends to be disregarded when it has no other purpose than to facilitate the scheme for which expropriation is needed. The important precedent in this regard is the influential Supreme Court case of \emph{Lena}.\footnote{Rt. 1996 s. 521.} However, the line of reasoning adopted by the Supreme Court in this case has so far been called on almost exclusively in cases when compulsory acquisition takes place to implement plans for public buildings or other kinds of public installations, like playgrounds or parking spaces.

In case of expropriation taking place to implement such plans, if a valuation were to be made on the basis of the use prescribed by the plan itself, one would expect the market value of the property to come out as nil or close to nil, such that one would be forced to return to existing use as the basis for valuation.\footnote{In fact, a logical continuation of the line of reasoning prescribed by the main rule would suggest that even existing use would be inadmissible as a basis for compensation since continuation of this use would not be in accordance with the plan, and hence unforeseeable. However, this has not, as far as we are aware, ever been argued.} It is not hard to understand how this could come to be felt as unfair to property owners. Moreover, it seems that it would easily come to run counter to Section 105 of the Norwegian Constitution. In some sense, it would represent a watering down of this provision, allowing the State to deprive property owners of value by using unfavorable regulation as an explicit means to later acquire properties cheaply by use of expropriation.

On the other hand, it can be argued that the question raised by cases such as these is not really a question of compensation for expropriation, but rather a question of whether or not property owners should have a claim of compensation for losses incurred due to \emph{regulation}. This is not generally granted under Norwegian law, and so the special rules that entitles the owner to such compensation in cases of regulation leading to expropriation can be seen as insufficiently justified within the broader context of Norwegian planning law. Indeed, some scholars have voiced this opinion forcefully, and the current state of the law is unclear at best, with the special rule introduced in \emph{Lena} being hard to apply to other cases, and giving rise to further Supreme Court decisions attempting to map out in more detail when exactly they come into play. Moreover, attempts to reform the law on this point have so far stranded in controversy.\footnote{NOU 2003:29 and further references.}

It seems to us, however, that while this debate is interesting, the truly fundamental questions about the current state of Norwegian law do not arise in this regard, but with regards to the other principle we mentioned, namely that no compensation is offered for value that \emph{the owner can not realize}.\footnote{By "realize" here, we mean realizable either as the owner's "value of use" or else realizable by selling the property at "market value", as prescribed by the Compensation Act.} This rule seemingly applies without reservation, and, as in the UK context, it appears like a logical consequence of the \emph{value to the owner} principle. However, as we argued for in the general discussion on the no-scheme rule in Section \ref{sec:noscheme}, it seems pertinent to distinguish between the \emph{subjective} and \emph{objective} aspect of this principle. In particular, if the owner can not realize the value because the value is not of a \emph{kind} that is available for commercial realization, for instance because it only represents non-commercial value to the public, then the rule appears easy to defend along traditional lines. On the other hand, if the owner can not realize the value because the State desires that \emph{someone else} be allowed to realize it, then the principle appears highly problematic.

While the general debate on Norwegian compensation law has completely neglected to consider this aspect, it features extensively, albeit implicitly, in one particular branch, namely the law relating to compensation for waterfalls. In the remainder of this paper, we turn to this particular area of Norwegian law, offering a detailed analysis of the problems that arise, and how they severely challenge the traditional "value to the owner" reasoning about compensation for commercial undertakings. 

