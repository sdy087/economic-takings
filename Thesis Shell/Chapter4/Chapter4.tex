\chapter{Just compensation}\label{chap:5}

\section{Introduction}\label{sec:into5}

\section{The ``no scheme'' principle}

\section{``Natural horsepowers''}

\section{Appraisal courts and ``foreseeable alternatives''}

\section{{\it Kløvtveit} and {\it Otra Kraft}}

\section{Conclusion}

Old stuff:

\section{Introduction}\label{intro}

We survey the history of Norwegian law regarding compensation for expropriation, focusing on the processes that led to the current regime codified in the Compensation Act 1984.\footnote{....} We identify two opposing strands of thought that we refer to as \emph{absolutism} and \emph{pragmatism}. This abstraction appears useful, and it also seems to be largely adopted as a premise in contemporary academic debates, as well as in case-law from the Supreme Court and in reports and proposals from governmental bodies. We explore absolutism in Section \ref{sec:ab}, which was the dominating perspective until the Second World War. Roughly speaking, absolutists favor strict adherence to the wording of Section 105 of the constitution, which they take as entitling property owners to protection and \emph{individual justice}.  

In Section \ref{sec:pra} we explore pragmatism, which became more dominant in the post-war period. Pragmatists take a broader perspective, and they tend to regard the overriding principle encoded in Section 105 as encompassing also \emph{social fairness}, the exact content of which can only be determined by looking to the current political climate and the prevailing social conditions.

However, as we note in both Sections \ref{sec:ab} and \ref{sec:pra}, there are inherent limitations in abstractions such as these. Moreover, we also direct attention to a particular aspect of Norwegian law that has in our opinion not received enough attention: the \emph{procedure} used to resolve appraisal disputes. This procedure is special in that it relies on the judgment of laymen. Moreover, it does not only rely on them to assess the value of property under given assumptions, but also to apply the law in order to determine what those assumptions should be. We argue that  this system has great potential for accommodating approaches to compensation based on rules that are derived from a concrete assessment of fairness and justice against the facts of individual cases. We also argue that this approach has been favored by thinkers from both the absolutist and the pragmatist camp, and might hence represent a common element to both.

In Section \ref{sec:regab} we first discuss how the debate between pragmatism and absolutism became particularly heated and political in the 60s. Then we describe the process that led to the passing of the first act relating to compensation, introduced in 1973 as a distinctly pragmatist project.\footnote{....} It proved highly controversial, and a compromise between absolutists and pragmatists eventually had to be enforced by the Supreme Court. This was done in \emph{Kløfta}, where the Supreme Court explicitly rejected the intended reading of the Compensation Act 1973, but at same time provided support for many of the pragmatic rules it sought to introduce.\footnote{Rt. 1976 p. 1} We then observe how there has been a tendency, following \emph{Kløfta}, towards greater reliance on a range of special rules for awarding compensation, limiting the importance of discretion in the appraisal courts. 

We argue that this has been largely detrimental to the development of the law in this area, and in Section \ref{sec:context} we direct special attention to the problem of expropriation that is used as an instrument for economic development through public-private partnerships. Here it will often be the case that the expropriating party benefits commercially from expropriating, and those who stand to lose their property are only rarely entitled to a share in this commercial benefit, due to the special rules currently in place to regulate compensation. We argue that a more flexible approach, placing more emphasis on fairness assessment in the appraisal courts, based on the circumstances of the case and particularly the balance of power between the parties, is needed. We argue, moreover, that adequate compensation rules can act as a safeguard against excessive use of expropriation in cases when it appears dubious if compulsion can at all be justified by public interests. We look to the case of waterfalls expropriated for hydro-power for an example of these mechanisms, and we note that the appraisal courts have, despite being marginalized, contributed to a complete revision of the principles used to awarded compensation for expropriation that benefits commercial hydro-power. In Section \ref{sec:conc}, we provide a summary of our main points and we offer a conclusion.

%\section{Background and motivation}\label{sec:back}
%
%The rules regarding the right to compensation following expropriation under Norwegian law are usually presented to an academic audience in one of the following ways
%\begin{enumerate}
%\item By giving a concise overview, directed at a general audience of lawyers and academic legal scholars, focusing on the constitutional protection of property provided in Section 105 of the Norwegian constitution, demanding that "full compensation" should be paid. This is typically followed up by a brief explanation of the rules encoded in the Compensation Act 1984, which sets out how this is to be achieved through awards based on either market value or value of use, such that the owner is entitled to whichever will yield the greatest level of compensation.
%\item In painstaking detail, directed at a very specialized audience of property lawyers and legal scholars specializing in property, focusing on special issues and problems of interpretation arising with respect to specific provisions, largely drawing on a vast and intricate body of case-law. In such expositions, the focus is typically directed at the requirement which appears seemingly innocuously as the second paragraph of Section 5 in the Compensation Act 1984, namely that the value should always be be calculated based on the value of such use of the property that the court regards as "foreseeable".
%\end{enumerate}
%
%One striking feature of legal discourse in Norway is that the description given along the lines of (1) and the impression it creates of Norwegian law, appears to largely contradict the impression one is left with if one  delves into the details of how exactly the rules are applied, especially in light of the requirement that the value should be calculated based on foreseeable use. On the one hand, we have the standard descriptions of Norwegian compensation law, such as that provided in \cite{Aall}, a textbook on human rights. Here, it is simply stated that as the right to full compensation is enshrined in the Norwegian constitution, full market value is usually paid and hence the constitutional protection of Norwegian property owners is strong, going well beyond other instruments that provide property protection, such as the European Convention of Human Rights, Protocol 1, Section 1. But on the other hand, we also have peculiar decisions such as that of Rt. 1.... p. ...., a case before the Supreme Court, where a property owner was deprived of a commercial property and was not entitled to be compensated for the value of investments that had been carried out by his tenant, even though it was clear that they belonged to him, and that the expropriating party would himself benefit commercially from them.\footnote{The decision was a rather extreme manifestation of certain aspects of principle that the use which forms basis for compensation must be "foreseeable", and it was sharply criticized, for instance by former Supreme Court Judge Skoghøy in ...} Such decisions and the mechanisms that underlies them are rarely mentioned when the state of Norwegian property law is summed up, however, but rather relegated to technical texts that deal with specific legal problems related to compensation. % Here, however, a view of constitutional safe-guards have long since developed that might prove surprising to academics not well versed in Norwegian property law. In the words of the prominent property rights scholar Professor Fleischer:
%
%In light of this, it seems that an intermediate view on compensation rules can be fruitful, allowing us to present some main threads in the web of intricate special rules, and to relate it to a broader perspective on Norwegian property law, and the protection of property encoded in the constitution. Below we do so, and we give a chronological overview of how the law developed, from a situation where there was no legislation to regulate compensation apart from the constitution itself, via a first Compensation Act of 1973 that was widely regarded to have been deemed unconstitutional by the Supreme Court in \emph{Kløfta} Rt. 1976 p.1, to the current Compensation Act 1984 which removes some of the more radical features of the 1973 Act, yet retains important aspects of its general disregard for value due to potential future development, entitling the owner to market value, but only based on such a market as the court (or, indeed, the excutive branch of government) deems foreseeable.
%
%We then link this chronological overview to a particularly pressing issue in contemporary expropriation law, namely the question of how to deal with commercial expropriation, where commercial interests benefit from compulsory acquisition of property. In many cases, the current compensation rules appear to give commercial actors an incentive to use expropriation in this way, and in many cases, the excetuive branched of government seem inclinded to let them. Hence expropriation becomes a means for transferring commercial interests in property from some groups of original owners, usually poor and socially disadvantaged, to other groups, usually powerful and with more financial strenght. This unfortunate mechanism, we argue, is in large part due to how the rules for Comepnsation actually work, showing both that they hardly offer property owners the level of protection forespeilet by many overviews of Norweigan law, and also suggesting the need for a new, brouder look at some of the more subtle mechanisms at play.

\section{Look to the constitution! The era of absolutism}\label{sec:ab}

The right to compensation following expropriation of property is enshrined in Section 105 of the Norwegian Constitution of 1814, in very simply terms. The constitution simply demands that \emph{full compensation} is to be paid, in all cases when the public interest warrants the compulsory acquisition of property. For more than 150 years, until the introduction of the Compensation Act 1973, this was the sole legislative basis for compensation rules in Norway. The concrete methods employed to calculate full compensation for different types of property, in particular, developed through case-law. However, according to a long legal tradition in Norway, going back even further than our constitution, the discretionary aspects of property valuation is regulated by special procedure, with a significant reliance on so called \emph{unwilling appraisers}, members of the general public, or, in some cases, technical experts, who have no interests in the case at hand, but who are regarded as being in a better position to judge the value of property than legal professionals.

This special legal procedure has a long history, going back to customary law that predates even the constitution, and the rules regulating it were revised and codified in their current form by the Appraisal Act of 1917.\footnote{Act no 1 of 1. June 1917 relating to Appraisal Disputes and Expropriation Cases.} In short, the Norwegian system now organizes these disputes similarly to regular civil disputes, and the procedure is administered by the district courts.\footnote{See Section 5 of the Appraisal Act 1917.} The presence of laymen is the major distinguishing feature: the court is composed of a panel consisting of one judge and normally four appraisers, who do not have any special legal competence. The standard arrangement is that they are chosen from the general public in the district where the property in question is located, but the Act opens up for the possibility that they may also be chosen for their special technical expertise.\footnote{See Sections 11 and 12 of the Appraisal Act 1917.}

Their role in the procedure is on par with the judge, however, and the panel decides both the legal and the technical questions together, usually following technical reports assembled by the acquiring party, which the property owner might then challenge more or less as if it was presented as evidence in a standard legal dispute.\footnote{See particularly Section 27 and Section 22 of the Appraisal Act 1917, with further references to the Dispute Act 2005 (Act No 90 of 17 June 2005 relating to the Mediation and Procedure in Civil Disputes).}
There is a possibility for appeal to the high appraisal court, which is organized alongside the regular regional high courts, and the possibility of getting the appeal heard depends on the importance of the case, following rules that correspond to those in place for regular civil disputes.\footnote{See Section 32 of the Appraisal Act 1917.} The procedure followed is an adaptation of those used for appraisal disputes at the district level, again according to the standard adaptations used for appeal procedures in civil cases.\footnote{See Section 38 of the Appraisal Act 1917.} However, the decision made by the high appraisal court is final as far the appraisal assessment is concerned, an appeal to the Supreme Court can only be accepted on legal grounds.

As a consequence of this system, and the lack of legislation regarding the meaning of "full compensation", the appraisal courts have been very important in interpreting and developing the law relating to compensation in Norway. At the same time, the practical viewpoint and emphasis suggested by the special procedural form led to legal aspects often being situated in the background in such cases, only coming to the forefront if and when the legal aspects of the case reached the Supreme Court. Indeed, the primary criticism voiced against the system, particularly following the Second World War, was that it gave the appraisal courts too much discretionary power and that legislation was needed to make the outcome of appraisal cases more predictable.\footnote{See, for instance, Part 2, Chapter 1 of the \emph{Report Regarding Appraisal Procedures and Compensation following Expropriation}, NUT 1969 nr. 2 (Norwegian governmental reports), handed over to the Department of Justice by the so called Husaas committee, appointed by the King in Council 6. Aug 1965.}

However, while the law regarding compensation was not formalized in written form, and also opened up for considerable discretion on part of the appraisal courts, there were legal scholars who developed theories and aimed to explicate its content based on the body of case-law that was available. Also, the Supreme Court did regularly hear cases concerning legal arguments regarding compensation, and they developed a consistent position on at least some of the more critical and recurring legal issues. The central source of legal reasoning regarding appraisal at this point was still to be found in the constitution itself, and the theories regarding compensation law that were \emph{absolutist} in the sense that they looked directly to wording in Section 105, also when tackling specific problems of interpretation. This general starting point was widely accepted as late in the 1940s, and in \cite[p. 177]{schj} it was summed up as follows.

\begin{quote}
When an owner is entitled to compensation, he is entitled to have his full economic loss covered. He should receive full compensation, see p. 42 ff. This is the great principle that remains absolute and any dispute must be resolved on its basis.
\end{quote}

A typical example of the style of legal reasoning that this view gave rise to can be found in the writings of the prominent legal scholar Frede Castberg. One of the problems he addressed was the extent to which increases in value due to the scheme underlying expropriation was to be taken into account when calculating compensation, and he based his reasoning in this regard directly on a reading of the constitution. His interpretation, moreover, was based on the principle of \emph{equality}, which was considered particularly crucial in understanding constitutional law. He wrote as follows, in \cite[Volume 2, p. 268]{castberg}.

\begin{quote}
The owner is entitled to full compensation. The expropriation should not leave him worse off economically than other owners. Hence if the public has knowledge that an industrial undertaking is being planned, that a railway will be built etc, and this affects the value of property generally in a district, then the increased value of the property that will be expropriated must be taken into account. If not, the owners of such property will be worse off than other owners from the same district. On the other hand, if the expectation of the scheme underlying expropriation leads to a general depreciation of value, then it is this new value -- not the original value -- that is relevant for calculating compensation. The crucial question is what the actual value is, when expropriation takes place.
\end{quote}

We mention that the problem analyzed by Castberg in this passage has been considered in many jurisdiction, and is dealt with in common law by the so called \emph{no-scheme} rule. This is more a principle than a single rule, and it is typically understood as a mechanism that is meant to ensure that changes in value due to the scheme underlying expropriation are disregarded.\footnote{For an history of the rule in common law (primarily the UK), which also illustrates the difficulty in interpreting it and applying it to concrete cases, we point to Appendix D of Law Commission Report No 286, 2003} In comparative terms, Castberg appears to favor a \emph{narrow} interpretation of the principle -- a restrictive view on when additional value due to the scheme should be disregarded -- quite close in spirit to the so called \emph{Indian} case from 1939\footnote{\emph{Vyricherla Narayana Gajapatiraju v Revenue Divisional
Officer, Vizagapatam} [1939] AC 302.}, which was been much discussed in common law and was dealt with extensively by the House of Lords as late as in 2004.\footnote{In the case of \emph{Waters and other v Welsh National Assembly} [2004] UKHL 19. The primary precedent for a broader interpretation of the non-statutory no-scheme rule, on the other hand, is \emph{Pointe Gourde}, \emph{Pointe Gourde Quarrying and Transport Co v Sub-Intendent of Crown Lands} [1947] AC 565, PC, 572, per Lord MacDermott. This case proved highly influential for the understanding of compensation rules in the post-war period, in many common law jurisdictions, but has recently been challenged by a renewed interest in more narrow viewpoints such as that expressed in the \emph{Indian} case, see  \cite{newuk} and also the case of \emph{Star Energy Weald Basin Limited and another (Respondents) v Bocardo SA (Appellant) [2010] UKSC 35}.}

In the context of Norwegian law, it is of particular interest to note how Castberg's views in this regard is arrived at through considering the constitution itself, founded on the principle of equality.\footnote{In this way, he arrives at a narrow no-scheme rule quite abstractly, and through a different route than the one adopted in the \emph{Indian} case, where the outcome appears to have turned crucially on the particular facts in the case, a close reading of precedent, as well as the perceived fairness of the result.} He does not, therefore, engage in any reasoning based on the extent to which it can be regarded as socially fair for the public to pay compensation for value that encompass the beneficial consequences of the project itself, and does not address the concern that this can be seen as a form of double payment. Such pragmatic, utilitarian reasoning was not widely adopted in the legal tradition Castberg was part of and his theory appears as an example of constitutional absolutism. But against the idea that this style of reasoning is necessarily "owner friendly", his work also serves to illustrate that absolutism based on the principle of equality can lead to rigid interpretations that disfavor property owners. For instance, it was regarded as beyond doubt by Castberg that owners of expropriated property could not claim compensation based on the special want of the acquiring party. This, apparently, should also apply quite generally. He continues as follows, immediately after the passage quoted above.

\begin{quote}
The situation is different if the property has increased value due to the expectation that it will be expropriated. The owner can not demand that this increase is compensated since that would be the same as giving him a special advantage compared to those from whom no property is expropriated.
\end{quote}

While Castberg's view appears to have been shared by many academics of his day, and was also, to some extent reflected in case law from the Supreme Court, the very nature of the system for deciding appraisal disputes gave the local appraisers great freedom in adapting the rules to suit the concrete circumstances of the case. To quite some extent, this would also involve making an assessment of what was regarded as a fair and just outcome, but on a case by case basis, not necessarily leading to special rules for specific types of cases. Indeed, when one looks more closely at case-law from the Supreme Court, one sees that there was  great tolerance for the use of discretion in the appraisal courts, vested within an absolutist theoretical framework.

As long as appraisal courts did not cross the line with regards to the constitution, they were largely allowed to adapt more pragmatic viewpoints. But such viewpoints were \emph{not} extensively codified in terms of special principles used to deal with special case types or issues, which the local courts where then obliged to follow in future cases. Rather, it arose as a logical consequence of the way in which appraisal disputes were organized, giving room for discretion, demanding consultation with laymen from the local communities, also on matters of legal interpretation. Hence, with absolutism as the theoretical underpinning of the system, a pragmatic approach to compensation was largely achieved \emph{indirectly} through a \emph{decentralized} system which gave local courts great freedom when applying the law. 

Again, the way in which the no-scheme rule was applied serves as an excellent illustration. On the one hand, the theoretical views of Castberg were widely accepted, but at the same time they were regarded as general guidelines that would necessarily have to be adapted to the circumstances. Moreover, it was not unheard of for the appraisers to disagree with the judge about how this should be done, and to award compensation according to a different understanding of the law than that favored by the judge. 

This happened, for instance, in the case of \emph{Tuddal}, where land was expropriated for construction of a power grid, and the expropriating party also acquired the right to use a private road.\footnote{Rt. 1956 p. 109}. According to the judge in the high appraisal court, who seems to have followed the teaching of Castberg, compensation should be awarded solely on the basis of what the owners stood to lose, calculated in this case based on the increased cost in maintaining the road resulting from increased use. However, the lay appraisers found this result unreasonable and awarded compensation also for the special value the use of the road would have for the acquiring party. The Supreme Court, although they found fault with the argumentation relied on by the appraisers, agreed that such compensation was possible in principle. The first voter offered the following perspective.

\begin{quote}
Since they were the private owners of the road, A/S Tuddal could, before the expropriation, refuse to let the Water Authorities to make use of it. Hence it might be possible for A/S Tuddal, through negotiation and voluntary agreement with the Water Authorities or others with a similar interest, to demand a reasonable fee, and in this way achieve a greater total benefit than full compensation for damages and disadvantages. Following the expropriation, it is no longer possible for A/S Tuddal, in its dealings with the Water Authorities, to economically benefit from their ownership of the road in this way. If the company suffer an economic loss as a result of this, I believe they are entitled to compensation. Whether or not such an opportunity as I have mentioned -- all things considered -- was present at the time of the expropriation, falls to the appraisal court to decide, on the basis of whether or not an economic loss is suffered beyond that which follows from damages and disadvantages. On this basis, I assume that the high appraisal court's decision to awarded compensation for the value of the right of way that is acquired can not -- in and of itself -- be regarded as an erroneous application of the law.
\end{quote}

The Supreme Court's reasoning illustrates two main points. First and most notably, we see how the Supreme Court adopts absolutism in its interpretation of the law, and makes sure, through careful use of wording, that the compensation for the value of the use of the road is not conceptualized as compensation based on the value of the road to the acquiring authority, but rather as compensation for the loss of potential profit following from a voluntary agreement. Hence the appropriateness of this form of compensation follows from the requirement that full compensation should be paid, based on the owners' loss. This particular interpretation of full compensation led to arguments in the post-war period, regarding whether or not owners had a right to compensation based on the loss of profit from hypothetical voluntary agreements with the acquiring party. In the end, a consensus formed that this type of compensation should not in general be awarded.\footnote{NUT 1969 nr. 2, Part 2, Chapter 4, Section 2.E.}

Despite this, we think \emph{Tuddal} is very interesting, also for the law as it stands today. It illustrates a second point, in particular, which also seems more relevant for our paper. We notice, in particular, the clear sense of commitment and loyalty to the procedural system displayed by the Supreme Court in its reasoning. This sentiment might be mostly implicit, but there can be no doubt, especially in light of the dissent from the judge in the high appraisal court and the legal theorizing of the day, that the Supreme Court went far in defending the discretion of the laypeople, as a \emph{systemic} feature. They seem to have actively sought out ways in which to legally justify the decision reached by the laymen, and to test with great caution whether it was truly outside the permissible legal boundary, or simply an exercise of the lay judgment that the system presupposed. 


This impression of the case is accentuated when we consider other cases dealing with the same and similar issues, and where a similar tendency to defend the role of the laypeople in the appraisal process can also be identified. A particularly clear expression of this can be found in \emph{Marmor}, a different case from 1956, where the Supreme Court overturned a decision made by the high appraisal court on the grounds that the court had not engaged in an assessment that had wide enough scope to do justice to the constitutional principle of full compensation, and the principle of evaluation by impartial laymen.\footnote{Rt. 1956 s. 493.} The case involved expropriation of a private railway track, for the construction of a public railway, and it was clear that the track which was being expropriated did not have market value in general. The expropriating party hence argued that the value of these tracks to the public railway should not be taken into account when calculating compensation, and the high appraisal court agreed with this, pointing to the standard teaching of the day. The Supreme Court disagreed, however, and felt that a standardized approach to the case was inappropriate given the circumstances. The first voter, in particular, made the following remarks.

\begin{quote}
In my opinion one can not simply assume that a property does not have market value when it has no value for anyone other than the expropriating party. The question needs to be assessed concretely. I agree with the expropriating party -- as has also been confirmed on several occasions by the Supreme Court -- that in general one should not take into consideration the special value that the purpose of expropriation gives the property. This should not lead to a spike in compensation payments. On the other hand, I can not agree that it is automatically reasonable, or in keeping with Section 105 of the constitution, if the expropriating party in cases like this one could acquire property at a price that is below what it would be natural and commercially appropriate to pay in a voluntary purchase.
\end{quote}

Again we notice that there are two main building blocks used in the argument; firstly, a reference is made to the constitution, reflecting the absolutism of the day, and secondly, a reference is made to the need for \emph{concrete assessment}, reflecting strong confidence in the integrity and autonomy the appraisal procedure. Moreover, we notice how absolutism regarding the constitutional protection of property owners is \emph{not} used to argue for specific rules or principles that should be adopted, but rather to back up the argument that compensation should result from real assessment, and not be overly reliant on such rules, not even when these rules appear sound in general, and have been backed up by a series of Supreme Court decisions.

In addition to making these overreaching remarks, the Supreme Court also gave pointers as to the kinds of facts that should be considered. For instance, they paid particular attention to the wider \emph{context} of expropriation, and the manner in which expropriation was used to benefit certain interests. They also noted how it had come to replace voluntary agreement as the standard means of acquisition for this type of development, therefore effectively preventing a market from developing. In the word of the first voter, below.

\begin{quote}
I also point to the fact that the case concerns an area of activity where the expropriating party has a de facto monopoly which makes it impossible for anyone else to make use of the property for the same purpose. This in itself makes it questionable to simply assume that the lack of financial value for other purchasers provides the appropriate basis for calculating compensation. When considering this question, it is also appropriate to take into account that we have lately seen a great increase in the use of expropriation to undertake projects such as this. Compulsion is becoming the primary mode for acquisition of property -- not voluntary sale following friendly negotiations.
\end{quote} 

In our opinion, the primary historical importance of this decision, which we think makes it highly relevant even today, is not to be found with regards to the particular legal interpretation of the no-scheme rule that the Supreme Court appears to endorse. Indeed, it seems to us that it would be an \emph{erroneous} reading of this judgment to take it as expressing support for a general principle that compensation can always be based on the value of hypothetical agreements that could have been made with the expropriating party. Rather, we believe that the judgment should be read as arguing against the blind obedience to \emph{any} such general rules for calculating compensation. At the very least, it seems clear upon closer inspection of the argument that the main objective of the court was not to express any particular view regarding the content of the no-scheme rule, but to instill to the appraisal courts that they could not use this rule as an excuse not to engage in concrete assessment to ensure a reasonable outcome in keeping with the constitution.

We believe this point is important to stress. It illustrates how absolutism need not, and did not, result in a rigid system with little room for assessment based on justice and fairness, broadly conceived. Quite the contrary, the absolutism endorsed by the Supreme Court, and inherent in the Norwegian system of appraisal courts, was not characterized by blind obedience to specific rules, like those proposed by Castberg. Rather, the system was flexible, and it was explicitly intended to function such that fairness assessments based on concrete circumstances could be accommodated. 

Going back to even older legal scholarship, we see that this view on the meaning of absolutism has a long history in Norway. For instance in the work of the famous 19th Century scholar Aschehough, who stressed the link between the constitution and the appraisal procedure when he considered the (then) hypothetical situation that legislation was introduced with the specific aim of reducing the level of compensation payments following expropriation. We quote from \cite[p.48]{asch} 

\begin{quote}
If it becomes common practice to award compensation payments that are unreasonably high, this would make important public projects more expensive and difficult to carry out, greatly to the detriment of society. In many cases it might not be possible to rely on legislation to prevent such excessive compensation payments, since this would restrict the appraisers too much. To some extent this might be possible, however, and as far as it goes, parliament must be permitted to do so. However, if enacted rules clearly lead to less than full compensation in an individual case, they will be overruled by Section 105 of the constitution, and fall to be disregarded in that particular case.
\end{quote}

This quote is important because it does not rely on any particular interpretation of the constitutional demand for full compensation, but sees this inherently as an issue that needs to be resolved by concrete assessment of individual cases. Absolutism to Aschehough implies freedom and responsibility for the appraisers; freedom to judge individual cases by its merits, and a responsibility to award full compensation, irrespectively of any specific rules that might be in place to curtail excessive payments. The important point is that Aschehough here sees absolutism as a principle that should be applied to cases, not to principles. He does \emph{not} argue that rules introduced to limit compensation payments would be inadmissible merely because they might sometimes suggest less than full compensation. Rather, he takes it for granted that it falls to the appraisal courts to \emph{apply} the rules in a way that would prevent such outcomes. As long as the appraisal courts remain free to apply the rules in such a way that full compensation is awarded, specific rules intending to prevent excessive payments can happily coexist with absolutism.

The subtle view taken by Aschehough was largely overlooked in debates following the introduction of the Compensation Act 1973, however, even though this act introduced radical rules of exactly the kind he had predicted and considered almost 90 years earlier. More generally, and as we will discuss in more detail below, the 60s and 70s appears to be a period when the crucial role of the appraisal procedure was to some extent forgotten, and also undermined, following a heated political and ideological debate regarding the appropriateness and admissibility of introducing rules to ensure that compensation payments were brought down to a lower level. This had deep and lasting effects on Norwegian compensation law, and it is popularly described as a period when the social democrats won recognition for the principle that social fairness suggested the introduction of compensation rules and disregards that were more extensive than what had previously been considered appropriate. 

This was conceived of as a fight for social justice against outdated and conservative ideas of constitutional absolutism. But it seems to us that this view of the history of Norwegian compensation law is erroneous, and largely unhelpful. The approach taken by Aschehough, in particular, placing emphasis on the important role played by the appraisers in achieving fairness and justice in concrete cases, does not appear to contradict social democratic goals at all. In fact, it seems that his approach might be better suited to serve such goals, and to accommodate a variety of different political opinions and ideas, than an approach which is based on attempting to flesh out in painstaking detail how the appraisal courts should go about achieving the balance between social fairness and owners' rights. We will return to this point later, but first we will take a closer look at the history of the radical Compensation Act 1973 and the censorship to which it was subjected by the Supreme Court, leading to the Compensation Act 1984, currently in place.

\section{Give them what they deserve! The era of pragmatic, utilitarian reform}\label{sec:pra}

Following the Second World War, the social democratic \emph{Labour Party} gained a secure grip on political power in Norway, and many reforms were carried out that would reshape Norwegian society. One of the most important reforms concerned the introduction of extensive planning law to ensure that land use was put under public control, and in this period expropriation was also becoming used more extensively to further public projects, such as the large scale construction of hydro-power to ensure general supply of electricity.\footnote{References.} As a result of these changes, the opinion was soon voiced that there was a need for a more uniform approach to compensation, which collected some basic principles in a common body of written law, and which could serve to bring compensation payments down. This, it was felt, should be done in order to facilitate more efficient implementation of public policies. 

In 1965, as a result of these new ideas, the so called \emph{Husaas committee} was appointed by the King and charged with the task of assessing the compensation rules currently in place.\footnote{Appointed by the King in Council on 6. Aug 1965.} They were also ordered to make a concrete suggestion regarding the need for additional principles of compensation, and if these should be given in the form of a special compensation act. Initially there was some doubt as to the extent to which is was at all permissible to give rules regulating compensation, as the constitution itself addressed the matter. However, the committee noted that existing legal scholarship seemed to suggest that such rules could be given, and that, moreover, specific rules had already been introduced, for instance in relation to expropriation for hydro-power development.\footnote{See Section 16 of the Watercourse Regulation Act 1917 (Act No. 17 of 14 December 1917 relating to Regulations of Watercourses).} Hence the majority of the committee concluded that while the constitution provided an important outer barrier, serving also to protect owners' rights against acts of parliament, it did not prevent legislation providing legally binding guidance as to how the notion of full compensation should be understood and applied by the courts in appraisal disputes.

Moreover, the majority pointed out that a vague general principle such as that provided by the constitution would by necessity have to be interpreted in order to be applied to concrete cases. Hence it was not only permissible, but also desirable, for parliament to give more detailed instructions as to how is should be applied and understood by the courts and the appraisal courts. Leaving it to the judiciary to flesh out the exact meaning of full compensation through case law, it was felt, was not appropriate in a regulatory regime where expropriation had become increasingly important as a means to ensure modernization and development of critical infrastructure. 

%In addition to this, the Supreme Court itself had recently expressed its support for a new view on regulation of property use, supported by contemporary legal scholars and politicians, whereby the State was regarded as having wide discretionary powers to determine how property should be used. This right to regulate, in particular, was increasingly coming to be seen as a right that did not infringe on property rights, so that the State would not have to compensate owners if they exercised it, except in special cases.\footnote{See, in particular, Rt. 1970 p. 67.}.

More generally, constitutional law was increasingly seen from a pragmatic utilitarian point of view, as a collection of foundational principles that aimed to stress the importance of ensuring social fairness just as much as individual justice. This was by no means a consensus view among legal scholars, however, and it was particularly contentious with regards to property. As a result, some disagreed strongly with the very idea of legislation regarding compensation, and tensions arose that have led to much legal controversy and are still important in the law today. 

This problem area was mapped out in some detail by the Husaas committee, who traced the pragmatic view on compensation, identifying it using the following quote by the leading scholar Knoph from \cite[p. 113]{knoph}.

\begin{quote}
Since Section 105 is a rule prescribing practical justice, directed at parliament, and not an ethical postulate of absolute validity, it must be permitted to make technical legal considerations, so that one accepts compensation rules that lead to correct and just results on average, even if it does not grant the owner full individual justice in every case.
\end{quote}

This view was becoming influential in the 60s, see for instance \cite{grunn,opshal}. However, there were many that disagreed vehemently, based on absolutism principles \cite{robb2,schj}. The latter describes Knoph's reading of the law scathingly as follows, on p. 44.

\begin{quote}Luckily it has not had any effect on judicial practice whatsoever. No court of law would accept that compensation should be set according to a norm that may be practical and just in general, but does not grant the owner full compensation in all individual cases.
\end{quote}

When assessing the current state of the law, the Husaas committee encountered many manifestations of the tension between a pragmatic and principled understanding of the protection of property, and in proposing a set of general principles for compensation which are still, in a modified form, with us today, they engaged in a fine balancing act. While they were clearly aiming to move in the pragmatic direction, the were nevertheless cautious, and they refrained from encoding principles that would appear too offensive to the absolutists, even if the pervading political sentiment was that more was needed to ensure a more effective State regulation of property use. 

The basic starting point for compensation that the Husaas committee identified in case-law was that of compensation based on loss of value according to "foreseeable use", as this could be reflected either in the owner's own use, or the market value, reflecting the value of such use as an average buyer might make of the property.\footnote{See, for instance, Rt. 1925 s. 47 and Rt. 1926 p. 669.} Moreover, the committee took the view that deviating from this starting point would not be constitutionally admissible, and instead they sought to codify what they saw as the existing interpretation of Section 105 in this regard. They concluded as follows.\footnote{NUT 1969 nr. 2, Part 2, Chapter 5, Section 3.}

\begin{quote}
It is the view of the committee that it is correct to encode in the act the principle that the owner is entitled to compensation based on the value that results from taking into account the foreseeable and natural use of the property, given its location and the surrounding conditions. The exact meaning of "natural and foreseeable" use must be decided after a concrete assessment in individual cases. By encoding this general principle, however, it will become clear that compensation should not be based on private or public plans unless these plans coincide with the use of the property that is natural and foreseeable, independently of the scheme underlying expropriation.
\end{quote}

This view, however, was not in keeping with the political motivation for an act regarding compensation, and the Department of Justice deviated from it in their final proposition to parliament. Instead of encoding existing  principles, they sought a pragmatic system whereby compensation would in general be based on the value of the \emph{current use} of the property, thus preventing the public from having to pay a financial premium to owners based on possible future value that would in any event, in most cases, be reliant on public development permissions. %Such permissions, it was argued, could never be foreseeable in circumstances when it was in the public interest that the property should be expropriated, and hence all future development potential should in principle fall to be disregarded.

The Department commented on this as follows.\footnote{Ot.prp.nr.56 (1970-1971), p.19-20}

\begin{quote}
The Department is of the opinion that it is particularly important to arrive at a rule that can bring the assessment of property value down to a realistic level, and believes that the natural starting point for such an assessment must be the current use of the property, especially for expropriation of real property. As mentioned, it is the opinion of the Department that a practice has developed that gives too much weight to more or less uncertain future possibilities for the property, something that has led to a sharp rise in compensation payments.
\end{quote}

After intense debate in parliament, where the minority center-right parties all opposed its introduction, the current use rule was eventually encoded in the Compensation Act of 1973, in Section 4 nr. 1.\footnote{Act No 4 of 26 March 1973 Regarding Compensation following Expropriation of Real Property.} This, moreover, was largely seen as a social democratic victory, and a clear indication that the absolutist view on property protection was increasingly losing ground to a more pragmatic approach. In particular, when clarifying their principled starting point regarding what should count as \emph{realistic}, the Department made the following assertion regarding the scope of the constitutional protection offered in Section 105, showing the ideological underpinnings of the act.

\begin{quote}
However, a right to complete -- or almost complete -- equality can not be derived from the constitution. It must be taken into account that we are here discussing equality with regards to increases in property value that are, in themselves, undeserved. [...]  %  The starting point must be that it is not, in and of itself, contrary to the constitution that one property owner do not benefit from the same increase in value as another, when the increase in value, for both of them, is due to public investment and does not stem from their own efforts. \\ \\
Certainly, it would be best to avoid any kind of inequality, if it was possible. But the examples we have considered illustrate that, today, inequality between property owners is tolerated with regards to public investments and regulation, and that, moreover, practical and economic considerations dictate that we \emph{should} make use of differential treatment in this regard.
\end{quote}

Here we see a clear expression of support for a pragmatic view of property rights, echoing Knoph, but going much further. In particular, the Department explicitly states that differential treatment is appropriate in the context of expropriation, and, by implication, that this should be done precisely to avoid compensation payments that include compensation for "undeserved" increases in value. Also, in proposing that compensation payments should be based on current use, the scope of "undeserved value" was made extremely wide -- in principle it would seem to include \emph{any} value that could be attributed to an as of yet unrealized potential that the property in question might have. The question of whether or not this value was reflected in the market value of the property, in particular, was not regarded as relevant. This was in itself very radical, since market value compensation had been the dominant starting point for reasoning about compensation following expropriation.\footnote{References.}

It seems to us that we should not underestimate the conceptual significance of this change in perspective. Here, the Department stood firmly behind a pragmatic view, where social fairness was the overriding constraint, also with respect to constitutional property protection. However, on taking this view to its logical conclusion, it was recognized that any general compensation rules that might be introduced should themselves be subject to a fairness test, so that, for instance, the current use principle could not itself be absolute or without exception. Rather, it could only be applied in so far as it served the overreaching goal of social justice and fairness which was, after all, regarded as the fundamental component of property protection that made such a rule possible. This, in particular, seems like a crucial observation, and one that has in our opinion been overlooked, with unfortunate consequence for the subsequent debate and development of the law. Indeed, it echoes the sentiment expressed by Aschehough that we quoted above -- similarly overlooked -- and thus it points to the existence of possible \emph{common ground} between absolutist and pragmatist views on compensation. Sensible voices from both camps, in particular, seem to arrive at the conclusion that in the end, there is no way around a \emph{concrete, contextual} assessment, where the assessment of social fairness and justice is held against the concrete circumstances of individual cases. Approaching such a view, from a pragmatist angle, the Department commented as follows.  

\begin{quote}
One is aware that the principle of current use compensation cannot be without exception. Even though this rule will be fair in general it can, in some cases, disproportionately disadvantage property owners. One has therefore suggested rules that modify the principle to some extent. These are given for somewhat different  reasons. \\ \\

One case addresses the situation when current use compensation means that a property owner will be significantly worse off that other owners of similar property in the same district, according to how these properties are normally used. In these cases, the principle of equality suggest that the owner receives some -- but not necessarily full -- compensation for the inequality that would otherwise arise from the fact that his property was made subject to expropriation. %Etter departementets oppfatning har en ekspropriat etter grunnloven ikke noe krav på å bli satt helt i samme stilling som om ekspropriasjonen ikke var skjedd, en forskjellbehandling innen rimelige grenser må grunnloven tillate når dette tilsies av tungtveiende samfunnsmessige grunner. 
\end{quote}

This principle was eventually encoded in the Compensation Act 1973 Section 5 nr. 1-3, and they would prove highly controversial. In \emph{Kløfta}, in particular, the Supreme Court interpreted additional compensation according to Section 5 nr. 1 as being \emph{obligatory} in a range of cases when the intention had clearly been that the rule should be used sparingly, and only when the courts considered it reasonable to do so. In this way, and possibly inadvertently, the Supreme Court defended owners' interest by \emph{limiting} the power of the appraisal courts. This, however, led to a change of perspective on the law, with the role of direct guidance from the Supreme Court becoming increasingly important, and the role of the appraisal courts in interpreting the law becoming increasingly narrow.

Before moving on to consider this in more detail, we should not forget the second exception to the current use rule that the Department introduced. In some sense, it is the more interesting of the two, even if it has been largely forgotten. In fact, we think directing attention to it, and to the idea that it captures, is highly relevant for one of the most pressing issues regarding compensation today, namely the case of \emph{commercial expropriation}, i.e., when expropriation is used as a tool by commercial actors who with to acquire property and who enjoy a financial benefit from being allowed to employ compulsion when doing so. 

The second exception rule from the Compensation Act 1973 sought to address precisely circumstances such as this, as it addressed the question of the \emph{power balance} between the expropriating party and the owner and the \emph{purpose} of the expropriation. In the words of the Department:

\begin{quote}
The second modification we make has to do with the relationship between the property owner and the expropriating party. If the use of the property that the expropriation presupposes gives the property a value that is significantly higher than the value suggested by current use, this will entail a transfer of value from the property owner to the acquiring party. In some cases this might be unreasonable. As an example of when this can become an issue, we mention an agricultural property that is expropriation for the purposes of industrial production. In such a case it might be natural that the owner receives a certain share in the increased value that the new use of the property will lead to.[...] %This would be different than, say, a situation where an agricultural property is expropriated for constructing a road or for setting up recreational outdoor grounds. In such cases, the expropriation will not lead to any such economically advantageous use of the property that will give the expropriating party an economic advantage. 

To establish a flexible system, the Department has concluded that it is practical that the King gives rules concerning the cases where an enhanced compensation payment, based on these principles, might be appropriate. This should not be decided by individual assessment, but governed by rules for special case types. Hence, the proposed Act states that the King can pass regulation concerning this matter.
\end{quote}

Again, this quote expresses the crucial insight that fairness with regards to compensation following expropriation can not be arrived at without adapting the rules to the circumstances. For any pragmatic approach to compensation, the \emph{context} of expropriation must by necessity come to play a crucial role, especially if the starting point is explicitly taken to be that compensation should only encompass the "deserved" value. What this value should be taken to be, in particular, can hardly be determined once and for all and in general terms, but must rather be subject to continuous revision depending on how expropriation is \emph{actually used} in society, the purpose it is meant to serve, the parties who benefit, and the degree of commercial economic benefit that results for individual parties other than the original owners. Indeed, stipulating that compensation should be "deserved" appears to provide a benchmark that is just as unclear as the stipulation that compensation should be "full". This, however, might be a source of inspiration rather than despair. It seems, in particular, that the inherent ambiguity of these terms allows us to draw two conclusions: first, that they might very well have the same meaning, and second, that they cannot possibly be defined once 
and for all by any act of parliament, or by any decision in the Supreme Court.

This, in turn, suggests that the Norwegian system of appraisal courts, and the presence of laymen in the decision-making processes of these courts, bears crucial influence on how well the Norwegian system is able to meet both the requirement of social fairness and justice for the individual. Unfortunately, however, the procedural, contextual aspect of fairness and justice was not recognized following the passing of the 1973 Act, with attention shifting towards issues of legal interpretation that arose from it. The primary such issue, and the most serious one, concerned the question of whether the law as such was in breach of the constitution. This  was eventually resolved by the Supreme Court in the case of \emph{Kløfta} in 1976.\footnote{Rt. 1976 p. 1}. 

Here the 1973 Act would be significantly reinterpreted to make it appear less offensive to the constitutional standard of full compensation. However, taking a broader perspective, it seems to us that \emph{Kløfta} largely accepted that the intention behind the act should be respected, and that appraisal practice needed to be adjusted accordingly. In this, the Supreme Court signaled loyalty to the political system and the democratic process. However, in implementing this adjustment in practice, they also, possibly inadvertently, set up a system where the role of the local appraisal courts appears to have been undermined, and where the Supreme Court itself assumed greater control over how the compensation law was to be applied in concrete cases. This characterizes the current state of the law, which we describe in more detail in the following section.

\section{Market value, but \emph{we} determine the market! The era of centralization and pragmatism by regulation}\label{sec:regab}

When the constitutionality of the Compensation Act 1973 came before the Supreme Court in \emph{Kløfta}, they chose to sit as a grand chamber and they reached a decision under dissent, being divided into two fractions, consisting of 9 and 8 supreme judges respectively. However, both fractions approached the problem of constitutionality by endorsing an interpretation of Section 5 nr. 1 in the Compensation Act 1973 that gave the exception to the current use much wider scope than what had been intended by parliament. The majority went farthest, and unlike the minority they also regarded the compensation payment in the concrete case to be insufficient. The first voter for the majority commented as follows on the constitutional aspect of the case.

\begin{quote}
[...] But the main question in this case, is whether or not it is in keeping with Section 105 to generally award compensation at a level below the market value that could legally be estimated, and that the owner could actually have achieved, if expropriation had not taken place. In my view, this involves allowing expropriation to transfer a right that the owner had, with a value to which he was entitled. If he is refused compensation for this value, he would, depending on the circumstances, be left significantly worse off than others in a similar position, who owns property that is not expropriated. Such a result I cannot accept. It would be a breach of established customary law and a practice that has been established throughout the years both by the appraisal courts and the Supreme Court. I refer particularly to Rt 1951 s. 87 (particularly p. 89, Opdahl). This practice is in itself a significant contribution to interpreting Section 105 on this point.
\end{quote}

We notice in particular the emphasis placed on \emph{market value} in the majority's reasoning. This may appear to be in keeping with an absolutist doctrine, but as we have mentioned, and will argue in more detail below, it can have unfortunate, possibly unintended, consequences for property owners, especially when combined with a restrictive view on what counts as foreseeable future development. We note, however, a technical point that might be of some significance for the interpretation of \emph{Kløfta}; instead of stating outright that a market value rule follows from the wording of the Constitution as such, the majority takes the view that this interpretation suggests itself based on the compensation practice that had currently been established. This might limit the scope of the majority's remarks in this regard, but it also serves to give further support to the claim that the role of the appraisal courts, and their assessments, still had a strong position in Norwegian compensation law at the time of \emph{Kløfta}. 

We remark that the minority disagreed on the constitutional status of the market value rule. Indeed, it was in this regard that the difference of opinion between the minority and the majority was most clearly felt. The minority, in particular, explicitly rejected the view that this rule could be derived from the constitution itself, and they also disagreed with the understanding that it would have status as a constitutional rule simply because it had been adopted in practice. This bestowed merely the status of ordinary legal precedent. In the words of the first voter for the minority:

\begin{quote}
Case-law in this area cannot be understood as preventing parliament from changing the rules in accordance with what they regard as necessary. That would prevent a reasonable and natural development and would not be in keeping with the consensus view that Section 105 of the constitution is a rule that must be interpreted in light of, and adapted to, how society has developed and how the law is viewed. I believe the practice that have evolved cannot be decisive if a new situation and new needs require a different solution. Whether the Compensation Act is in breach of the right to full compensation enshrined in the constitution, must depend on an interpretation of the wording in the constitution itself.[...] \\ \\
In my opinion, neither the intentions of parliament nor the way they are sought implemented through Sections 4 and 5 are in breach of the equality principle upon which Section 105 of the constitution is based. It does not follow from the constitution that an owner is in all circumstances -- and irrespectively of the economic forces from which the market value results -- entitled to compensation that is at least as great as the greatest legal value that the property could represent on a free market. A different matter is that Section 105 of the constitution could be important to the interpretation and application of the rules.
\end{quote} 

Hence the market value rule was explicitly renounced as a constitutional principle by the minority, who nevertheless conceded that the constitution could be used to interpret Sections 4 and 5 of the Compensation Act 1973. Both the minority and the majority agreed, however, that  it would be wrong to go on to consider Section 4 of the Compensation Act 1973 in isolation. For the majority, this would clearly have led to the Compensation Act 1973 being held to be in breach of the constitution, something that was avoided since the Supreme Court chose to consider the law as a whole, with the majority using the reasoning detailed above to argue for a new interpretation of Section 5, rather than as a means to undermine Section 4. Still, their interpretation of Section 5 went well beyond what parliament had intended, leading some scholars to claim that \emph{Kløfta} should be read as holding that the Compensation Act 1973 was unconstitutional.\footnote{References.} In the words of the majority:

\begin{quote}
The purpose of this rule is to award compensation beyond current use in cases where valuations according to Section 4 could be in breach with Section 105 of the Constitution. As it stands, Section 5 nr. 1 is not sufficiently suited for this purpose. By its wording it gives the appraisal courts an opportunity to assess whether or not it is reasonable to award additional compensation, even when the conditions for this is otherwise met, and even then with the limitation that the compensation would otherwise be significantly unreasonable. Such a free position for the individual appraisal courts -- without possibility of legal appeal -- would not be in keeping neither with the purpose of the rule nor the demand for full compensation set out in the constitution.
\end{quote}

On this basis, the Supreme Court chose to interpret Section 5 nr. 1 in such a way that whenever the conditions were fulfilled, the appraisal courts were \emph{obliged} to award additional compensation, and on this basis they found that the property owners in \emph{Kløfta} was entitled to have their compensation looked at again, in a new round before the appraisal courts. The minority agreed in principle, yet did not go as far as the majority, concluding that based on the particular facts at hand Section 5 had been adequately considered by the appraisal court in this particular case.

The upshot of \emph{Kløfta} was that Section 5 nr. 1 came to be seen as an obligatory rule, leading to compensation having to be enhanced whenever the current use rule led to payments that did not reflect the market value of comparable properties. However, the conditions stated in Section 5 nr. 2 and nr. 3 were still regarded as relevant, and in interpreting these conditions, a body of law developed whereby the market value rule was applied in a way that would come to involve significant reduction in compensation compared to what would result from practice as it had been prior to the Compensation Act 1973. In this way, the pragmatic approach proved triumphant, not because current use value was introduced as the general starting point, on the contrary, but because a range of new disregards were introduced to reduce the level of compensation in a range of different circumstances. After \emph{Kløfta}, in particular, the following rules were all considered legitimate ways to decrease the level of compensation.

In Section 5 nr. 2 and nr. 3, the following three disregard principles are encoded, all of which are, to varying degrees, still important in compensation law today.

\begin{enumerate}
\item Changes in value that are due to the expropriation scheme or investments or other activities should be disregarded, both when these are already carried out as well as when they are planned, c.f., Section 5 nr. 2 of the Compensation Act 1973.
\item To the extent that it is regarded reasonable, \emph{increases} in value that are due to public plans or investments should be disregarded, irrespectively of whether or not they have already been carried out, c.f., Section 5 nr.2 of the Compensation Act 1973.
\item An increased value falls to be disregarded if it results from considering a use of the property which is not in accordance with public plans, c.f., Section 5 nr. 3 of the Compensation Act 1973.
\end{enumerate}

These rules severely limits the level of compensation payments, and in many cases it appears to make the principle of full compensation based on market value rather illusory. Notice, in particular, that on the one hand, disregard rule nr. 2 can be applied to disregard the value arising from any use of the property that is not in keeping with the current public plan, whereas disregard rule nr. 3 can be used to also disregard any value that is due to this plan. While the outcome, logically speaking, should then be that no compensation can be awarded whatsoever, the disregard rule nr. 3 is usually seen to revert back to the current use compensation in such cases. For instance, if agricultural land is expropriated for the purpose of a motorway, and it would otherwise appear foreseeable that it could be used for housing, the compensation will be based on agricultural use because the value for housing is disregarded according to disregard rule nr. 3.

In practice, then, with virtually all novel economic activity making use of land is dependent on acquiring new planning permissions, the current use rule will typically be applied as intended by the Compensation Act 1973, with the only difference being that it is not thought of or described as such.\footnote{A similar point was made in \cite{stor}.} Rather, outcomes that are basically in keeping with current use thinking will be designated as "full compensation based on market value" -- the standard phrase adopted in most appraisal judgments -- and the fact that the outcome is equivalent to current use compensation remains unclear until one considers the range of disregards that have been applied. In this way, the state of law that followed \emph{Kløfta}, and which has largely been upheld and codified in later case-law, is greatly influenced by, and largely in keeping with the intentions behind the Compensation Act 1973. 

The Compensation Act 1984 was eventually introduced to reflect the principles laid down in \emph{Kløfta}, but it did not in any essentially way change or influence the course of the law that had already been set. Its main purpose was to bring the wording of the legislation more into keeping with how the law was interpreted by the Supreme Court. It explicitly returned to the starting point of the Husaas committee, namely that the compensation should be based on the value of the "foreseeable use" that the owner himself, or an average buyer, might make of the property. But it maintained and endorsed disregard rules nr. 1-3, except for restricting disregard nr. 2 to public investments, such that increased value due to public plans currently in place could not be disregarded.\footnote{In this way, the paradox mentioned above, that compensation could become impossible to award because there was no possible basis upon which to calculate it, was avoided.}

Beyond this, it did not give any further guidance as to how the disregard rules should be understood or applied, nor did it consider or resolve the question of when, if ever, they would need to be applied with caution in order not to go against the constitution. However, it was expected that cases where such issues arose would be resolved by strict adherence to firm principles, and that unless these principles could be derived from the Compensation Act 1984 itself, they should be laid down by the Supreme Court. Deciding on the law in such matters should not, in particular, be left to the discretion of the appraisers. The age when the appraisal courts were considered free to assess the cases based on their merits and directly against the overriding goal of achieving justice and fairness grounded in the constitution was over. Rather, an ethos had taken hold where the need to curb their freedom, in the interest of ensuring predictability and centralized control, was considered more important than upholding the system of lay judgment. 

As a result, difficult cases now routinely end up in the Supreme Court, who attempt to stick to established standardized rules as much as possible, but who will formulate new such rules for compensation of specific case types, if this proves unavoidable. As an example of this mechanism, it is enlightening to consider the case-law based on disregard rule nr. 3, which states that public plans currently in place are binding when calculating compensation. This rule cannot apply without exception, as recognized already by the Compensation Act 1973, since it may lead to outcomes that run counter to both the constitution and a common, rudimentary sense of fairness. 

One case which was considered by the Supreme Court in \emph{Østensjø} concerned land that was being expropriated for housing purposes, but such that one unlucky owner would only contribute land used for infrastructure that would serve the larger housing project.\footnote{Rt. 1977 p. 24} In this case, the Supreme Court agreed that he was entitled to compensation based on value of his land for housing purposes, irrespectively of the fact that \emph{his} land could not be used in this way according to the plan. However, in many other cases, the disregard rule is upheld even when it is hard to see it as either fair or just, simply on account of it having status as a general rule.\footnote{For instance in \emph{Malvik}, Rt. 1993 p. 409, where owners of property used for a motorway were only entitled to compensation based on current agricultural use because the regulation for motorway use was assumed binding for the compensation assessment.} One example is found in \emph{Sea Farm} which dealt with the issue of whether or not the owner of a commercial property should be awarded compensation for the value of investments carried out by the previous tenant.\footnote{Rt. 2008 p. 240} There was no doubt that the owner was entitled to these investments, but since the acquiring authority was the only purchaser who was likely to benefit commercially from them, no compensation was awarded for the loss of these investments. This, in particular, followed from a strict reading of the requirement that compensation should be based on the foreseeable use that an "average" buyer could make of the property, encoded in Section 5 of the Compensation Act 1984. Adherence to the wording used in the act seems to have taken priority over an assessment based on the facts of the case. It seems difficult to argue that it would be either unjust or unreasonable, in particular, to compensate the owner for investments that would prove commercially valuable to the acquiring party.\footnote{The decision was sharply criticized by a former supreme judge \cite{skog}.}

In our opinion, this example illustrates how the development of compensation law towards greater reliance on specific rules rather than concrete assessment based on general principles can be harmful, and how it also threatens to undermine the idea behind the special procedure used to decide appraisal disputes, which has a long history in Norwegian law.\footnote{One might ask if it has status of constitutional customary law, especially since it concerns the mechanism by which a constitutional rule is meant to be upheld.} It also seems to severely underestimate the extent to which compensation rules, when applied to concrete cases, must and should be interpreted based on the context of the case. It seems difficult indeed, if not completely impossible, to achieve social fairness and individual justice by a set of specific rules on the basis of which all legal issues can be resolved mechanically by blind application of such rules. %Moreover, it would be wrong to think that Section ... of the Appraisal Act 1917, encoding the principle that laymen should take part in the decision-making both with regards to legal and technical matters that arose in appraisal disputes.

In the following section we will address this issue in more detail, and we will argue for a different conceptual approach to compensation law, grounded both in the procedural tradition of appraisal courts and the more subtle parts of the absolutist and pragmatic theoretical traditions. It seems to use, in particular, that the most striking lesson that should be drawn from considering the history of Norwegian compensation law is that a \emph{contextual} view of compensation has been a common denominator that both the absolutist and pragmatist camps have endorsed. Unfortunately, this common element was overshadowed by political conflict regarding the weighing of different values. However, there can be little doubt that social fairness and individual justice should \emph{both} to be regarded as important objectives for compensation rules. Moreover, while they may sometimes be opposing, they need not be, and their exact relationship depends largely on the circumstances. It seems to us that it is simply inappropriate to let particular political sentiments regarding their relationship and relative importance, sentiments that are usually dependent on the particulars of the prevailing political, social and economic conditions, dictate the development of the legal framework for resolving compensation disputes.

Considering current trends and recent issues in expropriation law, particularly related to commercial expropriation, further suggests that a different perspective is needed on this matter. In particular, we believe it is time to recall the idea of the independent and impartial discretion of the appraisal court, relying on the good common sense of laymen as well as the legal expertise of judges. The appraisal courts should in our opinion be set with the task of more actively evaluating how fairness and justice is best served in individual cases, at least if the overall goal is truly to arrive at a socially fair and individually just compensation system. We discuss this idea in more detail in the final section below.

\section{Let their peers decide! Suggestions for an era of contextual compensation assessment}\label{sec:context}

In recent years, there has been a shift of attention away from the old narratives regarding compensation, especially with regards to expropriation that benefits \emph{commercial} interests, for instance because expropriation is used to redistribute the ownership of natural resources, transferring rights to commercial development of these resources from rural communities to commercial companies.\footnote{Larsen et al.} This shift of attention towards the special questions that arise with respect to commercial and private-to-private expropriations is clearly felt also on the international stage.\footnote{References} However, unlike the situation in many other jurisdictions, for instance in the US, the question in Norway does not revolve so much around the interpretation and nature of the public interest requirement for expropriation, but is rather focused on how the context of expropriation might affect the issue of compensation, and how it might require us to look at established principles in a new light. Increasingly, it is becoming clear that commercial expropriation calls for special rules, both in order to achieve fairness and justice, but also in order to put in place effective safeguards against excessive use of expropriation in circumstances when it might come to appear increasingly illegitimate.

In this regard, moreover, the situation in Norway is quite different today than it was in the 60s and 70s when the currently predominant conceptualization of compensation rules was established. Processes of commercialization, privatization and public-private partnerships have led to new contexts of expropriation where commercial interests have come to regularly derive significant financial benefit from measures that involve the compulsory acquisition of property from owners that are generally low-income and low-status members of society. This, indeed, is also a trend internationally, and it increasingly leads to controversy, something that has also resulted in increased recognition that it should be addressed critically, both by academics and the legislative authorities, in a range of different jurisdictions.\footnote{References.}

It is interesting to note, however, that just as Aschehough, the absolutist, saw that there might arise a need for legislation to curb excessive compensation payments, so did the pragmatists behind the Compensation Act 1973 see the problems that might arise from such trends regarding how expropriation is actually used. Indeed, their reflections on this point, quoted above, and leading to Section 5 nr. 4 in the Compensation Act 1973, appear even more relevant today than it did in 1973. We recall that this provision gave the King in Council the right to establish specific rules that additional compensation should be paid in some cases, when the balance of power and economic interests that existed between the parties made this appear reasonable. No regulation on this point was ever passed however, and when the Compensation Act 1984 was introduced, this rule was removed, ostensibly based on the assumption that there was no need for such a rule now that the general system had been reverted to compensation based on full market value.\footnote{References.}

This shows the negative effects of the artificial and simplified division lines that became prominent in the 60s and 70s. The debate became very politically charged and soon became clearly demarcated with absolutists on the one hand, arguing in favor of equality and individual justice for property owners, and pragmatists on the other, arguing on the basic assumption that expropriation took place to benefit non-commercial public interest, and that fairness therefore dictated that compensation should be limited to facilitate efficient implementation of public policies. To simplify matters even further, those who held the former view would typically be described as "owner friendly", belonging to the political right, while those endorsing the latter view were regarded as more "community friendly", and belonging to the political left.\footnote{This terminology was then even put to use by academics working on compensation law, see \cite{stor}, and \cite{regeksp} for sharp criticism of this (by a prominent pragmatist).} These two camps would naturally clash over the current use rule, which explicitly departed from the market value approach -- which had by then become a primary tenet for the "owner friendly" camp -- and it did so, it was perceived, to the detriment of property owners, but for the benefit of greater society.

Hopefully, the historical overview we have given in Sections \ref{sec:ab}-\ref{sec:regab} show that a more subtle view is in order, and could very well have been adopted already in the 60s. Moreover, But the same conclusion should be arrived at, we think, also by considering the actual content of the rules in question. For instance, why is  market value compensation necessarily owner friendly? This, surely, will depend on the nature of the market? It would not, to consider an example highly relevant to contemporary issues, follow from the market value approach to compensation that owners should get compensation based on the fact that the expropriating party stands to benefit commercially. This most certainly does not follow from the market value approach currently in place, which is in many cases little more than a current use rule in disguise. However, it would not even follow from the classical market value approach, such as that argued for by Schjødt, who we quoted above in Section \ref{sec:ab}. Under any market value paradigm, it would be necessary to demonstrate that the commercial value for the expropriating party was somehow reflected in a market price. 

This could only rarely be assumed, however, since the planned commercial activity would typically be reliant on public permission that an average buyer could not realistically expect. Of course, in such cases one might nevertheless argue for compensation based on the value that a special buyer, who could get the necessary permission, might be willing to pay. But this would then \emph{not} be in keeping with a standard market value approach. It might, however, come to be perceived as both individually just \emph{and} socially fair to do so, in which case it is hard to see why the appraisal courts should not be permitted to award compensation on this basis. Indeed, we recall the decisions from the Supreme Court in the 50s, discussed in Section \ref{sec:ab}, where the court took precisely this approach, even going so far as to demand that an assessment along these line \emph{had} to be carried out by the appraisal court, even if the standard rule did not warrant additional compensation.

These decisions are typically regarded as outdated, however, with the firm conclusion in contemporary scholarship being that there is no non-statutory rule in Norwegian law according to which an owner can demand compensation on the basis of his loss of opportunity for making a profit from a voluntary agreement with the expropriating party. This is probably beyond doubt, as far as it goes, but in our opinion it misses the point of the decisions from the 50s, and argues against the existence of a rule that has \emph{never} really been endorsed by anyone. Certainly, the Supreme Court did not rely on any such special rule, nor did they introduce one. Their point was not that an owner could \emph{always} demand compensation on the basis of lost profit from hypothetical agreements, merely that this could be way in which to ensure that full compensation was paid, \emph{if the context of the case dictated it}. Moreover, their main point seems to have been against the blind obedience to any specific interpretation of special rules; the matter needed to be considered concretely, and while special principles might provide a starting point and aid in the assessment, it fell to the appraisal court to apply these rules to the facts so that justice was served.

Hence, just as it is mistaken to think that compensation based on a hypothetical \emph{fair price}, not based on the general market value, always needs to be considered, it is also not correct to rule this out completely, even if a market value approach is to be adopted in most cases. This was the law in the 50s, according to the Supreme Court, and it is hard to see why it should not be just as valid today. Indeed, it was also in keeping with the Departments comments regarding the appropriate interpretation of the 1973 Act, although the Department, probably unwisely, sought to transfer authority in this regard from the appraisal courts to the King in Council. However, as this rule was not used, and subsequently removed, it appears that we must again look to the appraisers in this matter.

Moreover, it seems that the particular possibility of awarding compensation based on a "fair price" might be much more important today than it was in the 50s. This is because it can be used to address problems arising when awarding compensation for cases of commercial expropriation, in situations when the expropriating party is the only one who is likely to be able to carry out the planned project, either because he has acquired extensive public permissions, or because he is in possession of special knowledge or property needed for the scheme. Considering some contemporary issues regarding the expropriation of waterfalls can serve to illustrate this point. With regards to waterfalls, in particular, both the positive effect of the reliance on laymen in appraisal courts, as well as the negative effect of simplistic approaches based on a range of specific rules, has been demonstrated by recent case-law.

\sjur{Insert brief description of the case-law on waterfalls and relate it to the above.}

\section{Conclusion}\label{sec:conc}

We have presented an overview of Norwegian law relating to compensation following expropriation. First, we identified two different strands of thought regarding this matter, which we referred to as absolutist and pragmatist respectively. We noted that the tension between these two perspective became aggravated in the 60s and 70s, when legislation was passed with the explicit intent of bringing compensation payments down and to enforce a more pragmatic approach. The legacy of this era was a lasting pragmatist turn in compensation law, but also a greater centralization of power regarding the assessment of appraisal disputes. In \emph{Kløfta}, the Supreme Court modified some pragmatist rules introduced by parliament, but they also sanctioned a range of disregards that reflected the pragmatic intent behind these rules. Moreover, they assumed a greater role in providing special rules for the appraisal courts to follow in these matters, hence limiting the role of the laypeople in the appraisal process, and thus also changing the character of this process, which has long roots in the Norwegian legal tradition.

We focused particular attention on this latter change in the law, and we argued that it has resulted in an overly simplistic and often unhelpful narrative regarding compensation. Moreover, we argued that it inadvertently went against one crucial principle that more subtle thinkers from both the absolutist and pragmatist camps agreed on: the need for concrete fairness assessment. We went on to suggest that the importance of this principle is further accentuated today, when the context of expropriation is often quite different from the standard assumption of property taken for the public good. Often, the economic system currently in place, and the widespread use of expropriation that has followed the advent of extensive planning law, leads to expropriation appearing primarily as a means for commercial actors to make a profit. It might be hard to directly address the legitimacy of this in legal terms, by demanding that courts take an active role in interpreting the public interest requirement. But then the nature of compensation rules applied to such cases becomes a crucial special question. If dealt with in the right way, compensation can be used to achieve greater fairness in such cases, and also, more importantly, can serve as an effective safeguard against excesses. 

Commercial companies, presumably, only want to use expropriation as long as this is the most \emph{profitable} or \emph{practical} manner in which to acquire property. Moreover, it seems that a system where expropriation regularly comes to be used for this reason would have to be regarded as inherently flawed by both pragmatists and absolutists. Hence there should be cause for reaching common ground on the principle that compensation rules needs to be such that they prevent commercial companies from profiting merely from being able to use compulsion against other members of society. Achieving success in this regard, however, might not be so easy if one is committed to a top-down approach relying on the introduction of yet more special rules. Rather, justice and fairness might be better served by taking note of the potential inherent in the special way that Norway organizes appraisal disputes. By focusing on the need for concrete fairness assessment, and demanding that appraisers look to the power balance between the parties, the purpose of the expropriation, and the possible commercial interests involved, it seems that much can be achieved. The recent developments in case-law regarding waterfalls illustrates this. One can only hope that the powers that be are not too invested in the idea that \emph{they} are the ultimate authority on fairness, to allow these 
encouraging trends to develop further.

