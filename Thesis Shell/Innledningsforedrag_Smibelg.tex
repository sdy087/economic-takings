%\documentclass[12pt,a4paper]{memoir} % for a long document
\documentclass[12pt,a4paper]{article} % for a short document

\usepackage[utf8]{inputenc} % set input encoding to utf8
\usepackage[style = oscola]{biblatex}

% Don't forget to read the Memoir manual: memman.pdf

\title{{\huge Summary and progress report:} \\ ``On the Legitimacy of Economic Takings: A case study of Norwegian waterfalls taken for hydropower development''}
\author{Sjur K Dyrkolbotn}
\date{} % Delete this line to display the current date

\newcommand{\noo}[1]{}

\addbibresource{thesis.bib}

%%% BEGIN DOCUMENT
\begin{document}

\title{The US perspective on economic takings}

Hålogaland Lagmannsrett sak nr. 14.092631SKJ-HALO

Disposisjon til innledningsforedrag
Advokat Sjur Dyrkolbotn

I	Innledning

1.	Sammendrag av foredraget

2.	Oversikt

	Representerer Odd Kristian Pedersen, eier av fallrettene i Mangåga

	SKS Produksjon er gitt tillatelse til overføring, ikke ekspropriasjon av fallrettigheter, jf KU s. 3

	Et inngrep som nyter ekspropriasjonsrettslig vern tilsvarende fallretter, jf vassdragsreguleringsloven § 16

	Kart over området
	
	Tvist om fallrettene i Mangåga
	
	Fordeling skjer i eget skjønn, samlet erstatning fastsettes her

	Rettskraft for Pedersen og de som avleder sin rett fra hans

3.	Grunnlag for å begjære overskjønn 

	Det skal utmåles "full erstatning"

	Tre mulige utmålingsprinsipp:

a)	Småkraft

	Nåverdien av småkraft i Mangåga, slik saken stod for skjønnsretten: 

	I følge Småkraftkonsult AS: 		kr 132 millioner, jf DU s. 14 (nå SU s. 343)
	
	I følge skjønnsretten: 			kr – 3 millioner, jf DU s. 38

	Et avvik på over 130 millioner

	Hva som er rett er langt på vei et skjønnsspørsmål

	Skjønnsrettens vurdering er så langt ikke gjort gjenstand for kontradiksjon

	Den bygger på uklare premisser, og begrunnelsen er mangelfull

	Mange av premissene fremstår som feilaktige

b)	Samarbeid

	Erstatning basert på verdien av et stort samarbeidsprosjekt
	
	Tingretten legger feilaktig til grunn en subjektiv vurderingsstandard, jf DU. s. 41 - 42

	Spørsmålet skal vurderes objektivt, mer om dette i prosedyren

	De avgjørende faktiske forhold er av bedriftsøkonomisk og markedsøkonomisk natur
	
	En økonomisk rasjonell aktør må alltid antas å ville samarbeide om en lønnsom utbygging

	«Ingen må få mulighet til å utnytte sin monopolsituasjon», jf Rt. 2011 s. …

c)	Naturhestekraftmetoden

	Tingretten sin vurdering og begrunnelse av pris per naturhestkraft er utilstrekkelig, jf DU s. 44

	Resultatet er at prisen er for lav

	Høyesterett presiserer behovet for tilpasning og bruk av skjønn når metoden anvendes, jf Rt. ....

	Metoden er et utslag av et prinsipp om gevinstfordeling med grunneier
	
	I dag er forholdet til andre aktører på markedet også sentralt 

	Erstatningens rimelighet må vurderes også opp mot samfunnets interesse i "fair play"

	Ekspropriasjon skal ikke være et redskap for å oppnå kommersiell gevinst!	
	
	Tingretten begrenser seg til en meget kort vurdering av priser gitt i tidligere skjønn, opp mot lønnsomheten av storprosjektet

	Dersom det utmåles erstatning etter naturhestekraftmetoden må overskjønnet også vurdere rimeligheten opp mot gevinstdeling og markedsvirkninger

4.	Saken i et mer overordent perspektiv

	Markedsregulering av kraftbransjen

	Kommersielle utbyggingsselskaper, ikke lenger offentlige tjenesteytere

	Hva med lokale grunneiere og lokalbefolkningen forøvrig?

	Ekspropriasjon for å fremme felleskapets interesser 

	Ekspropriasjon skal ikke kunne brukes av mektige markedsaktører for å oppnå kommersiell gevinst

	Forholdet til konkurrerende kraftselskaper

	Konklusjon: Erstatningene for fallretter må reflektere reelle kommersielle verdier

	Høyesterett sin prinsippavgjørelse i Rt. 2008 s. 82	

	Flere prinsippielle uttalleser av stor betydning i Kløvtveit, Rt. 2011 s. ....

II.	Nærmere om småkraftvurderingen

1	Oversikt

a)	Det rettslige utgangpunkt

	Hva ville verdien vært dersom SKS Produksjon ikke fikk ekspropriere?

	Hvilken verdi er det "påregnelig" at fallrettene da ville hatt på grunneier sin hånd?
	
	SKS Produksjon sitt prosjekt må tenkes borte, siden det forutsetter ekspropriasjon	
	
	Den form for ressursutnyttelse som deres prosjekt baserer seg på er fortsatt relevant

	Kløvtveit og Otra II

	Mer om dette i prosedyren	
	
b)	Skjønnsspørsmålet

	Ville småkraft i Mangåga vært påregenelig dersom SKS Produksjon sitt prosjekt tenkes borte?

	I så fall, hvilken verdi ville denne utnyttelsesformen ha hatt på grunneierens hånd i dag?

	Prinsipal anførsel: Det ville stått ferdig utbygd småkraft i dag, jf konsesjonssøknad fra 2005

	Subsidiær anførsel: Det ville blitt påbegynt småkraftutbygging i dag

	Utbyggingskostnader og verdier vurdert av Småkraftkonsult, jf SU s. 265-346

	Spesifisert for Mangåga, SU s. 337-346

	Skjønnsretten tok utgangspunkt i forrige utgave av denne rapporten, jf DU s. 27

	Men det ble foretatt meget dramatiske justeringer, jf DU s. 38

	«I de beregninger som Småkraftkonsult har foretatt er utbyggingskostnad satt til 67,27 millioner kroner. 
	Etter et prosentvis påslag for rigg og drift, for nettkostnader i form av anleggsbidrag, for tunnel og rør/grøft samt for usikkerhet, 
	har skjønnsretten kommet frem til at antatt utbyggingskostnad skal justeres til 91 millioner kroner. 
	Med korrigert utnyttelsesgrad fra 86 % til 77 % blir den årlige produksjonen 17,45 GWh. Dette vil igjen
	gi en utbyggingspris på 5,21 kr/kWh. Med en kapitaliseringsrente på 7 % er beregnet 
	nåverdi basert på deling av et eventuelt overskudd negativ med -3,06 millioner kroner.”

	Kostnadsjustering på over 20 millioner

	Produksjonsjustering på over 2 GWh/år

	Rentejustert fra 5 til 7 %

	Inntektsjustert, strømpris fra 38 til 32 øre per KWh 

	Hva er begrunnelsen for justeringene?

	Ikke konkretisert for Mangåga, bortsett fra (delvis) spørsmålet om produksjonsvolum, jf DU s. 28-29

	Begrunnelsen forøvrig er generell og lite konkret

	14 prosjekter behandles under ett, jf DU s. 27-35

	Ikke forsvarlig, en saksbehandlingsfeil

	Kostnads- og produksjonsestimat spesielt viktig i denne sesjonen

	Inntekts- og verdivurderingen kommer mer i fokus i neste sesjon

	Det tas forbehold om supplering av bevistilbudet

2.	Kostnadsberegningen

	Småkraftkonsult bygger på NVE sine standard, erfaringstall og befaring

	Tingrettens justeringer er umulige å forstå i lys av begrunnelsen (felles for alle anleggene):

2.1 	Hva har skjønnsretten egentlig gjort?

a)	Rigg og anlegg	

	"Heves med mellom 15 % og 35 %", jf DU s. 31
	
	Hva betyr dette? 
	
	En økning på mellom 15 % og 35 %, eller en økning (fra 20 %) på mellom 75 % og 175 %?
	
	I begge tilfeller: Hvor mye for Mangåga, og med hvilken begrunnelse?
	
	Umulig å vite

	La oss anta at det foretas økning av kostnaden på 35 %

	Da får vi for Mangåga (SU s. 230):

	Oppjustering rigg og drift = Andre Kostnader (?) = kr 13.02 mill x 1.35 = kr 17.57 mill kr

	Forskjell = kr 4.55 mill
	
b)	Tunnelarbeider

	Justert opp med 15 %, jf DU. s. 31

	Gjelder formodentlig ikke boret tunnell, jf "relativt korte tunnellarbeider" og drøftelse DU s. 30

	Da får vi for Mangåga (SU s. 231):

	Oppjustering sprengt tunnell = kr 5.84 mill x 1.15 = kr 6.71 mill	

	Forskjell = kr 0.87 mill

	Total justering så langt = kr 5.42 mill

c)	Uforutsettposten

	Justert opp med mellom 10 % og 20 %, jf DU s. 31	

	Betyr det 10-20 % økning eller menes det egentlig her 100-200 %?

	La oss anta 20 % for Mangåga

	Da får vi:

	Oppjustering usikkerhet = kr 5.42 mill x 1.2 = kr 6.5 mill

	Forskjell = kr 1.08 mill

	Total justering så langt = kr 6.5 mill

d)	Anleggsbidrag

	Ikke spesifisert hvilken justering og fordeling som er gjort

	Det fremkommer at Tjørhom har regnet med kr 49 mill, jf DU s. 33

	Retten legger til grunn 80 % av RLK sitt estimat på kr 102.2 mill, altså kr 81.76 mill, jf DU s. 35 

	Dette er en økning på 81.76 / 49 = 1.66, det vil si 66 %

	Ikke spesifisert hvor stor andel av økningen som faller på Mangåga, jf DU s. 35

	La oss anta 66 % oppjustering

	Da får vi:

	Oppjustering anleggsbidrag = kr 4.47 mill x 1.66 = kr 7.42 mill

	Forskjell = kr 2.95 mill

	Total justering så langt = kr 9.45 mill

	Ingen flere kostnadsjusteringer nevnt av retten noe sted

	Mer enn 10 millioner er umulig å spore

	Regnefeil?

	Har skjønnsretten egentlig ment at man burde øke drift og riggkostnadene med 75 - 175 %?

	Hva med uforutsettposten, ble den da egentlig doblet, ikke økt med 10-20 % som oppgitt? 

	Hva med tunnelarbeidene, som ikke er oppgitt som prosent i utgangspunktet? 

	Kanskje har retten et avvikende prosentbegrep, men regnestykket ser uansett ikke ut til å gå opp

	Konklusjon: Det er umulig å vite hva skjønnet har lagt til grunn som premiss for kostnadsvurderingen

	Spørsmålet må vurderes helt på nytt

	Dette er nødvendig allerede av saksbehandlingshensyn

2.2	Hvilke kostnader bør overskjønnsretten legge til grunn?

	Vår anførsel: Tjørhom sin kostnadsvurdering er god, kanskje noe for høy

	Betydningen av at det brukes lokal prosjektledelse, lokale entreprenører og lokal arbeidskraft	

	Rigg og drift

	Småkraftkonsult legger til grunn 20 %, jf DU s. 31
	
	Erfaring viser ofte enda mindre for denne typen prosjekter, jf vitnet Brattland

	En boreløsning reduserer for eksempel riggkostnadene dramatisk, jf Brattland og Tonstad

	Naturforholdene og værforholdene ikke spesielt vanskelige sammenlignet med andre prosjekter, jf vitnet Brattland og befaring

	Uforutsette kostnader	

	Tunnelarbeider

	Anleggsbidrag og kraftlinjer

	Konklusjon: Kostnadene er i alle fall ikke høyere enn det Tjørhom har beregnet

	For alle disse forhold må vurderingen (og begrunnelsen) være konkret opp mot Mangåga

3.	Produksjonsestimat

	Vassvatn som vannmerke, DU s. 337

	Mer snaufjell, men også mer effektiv sjø

	Lavere minimumshøyde, men også noe høyere maksimumshøyde

	Uansett, det avgjørende er den vektede gjennomsnittshøyden

	Vassvatn noe lavere, men det mest representative kjente vannmerke

	Usikkerheten er større når det gjelder volum enn kurve		

	Usikkerhet finnes, men ulike momenter trekker i ulike retninger

	Det avgjørende er hva som er sannsynlig

	Vassvatn gir et sannsynlig estimat og bør legges til grunn uten justering

	Virkningen av valget uansett redusert av regulering i Mangåga

	Slukeevne

4.	Inntekter

	Tjørhom sin rapport gir et godt bilde på langsiktige priser

	Grønne sertifikater er innført og småkraft ville nyte godt av disse

	Mer om dette i neste sesjon

5.	Rente

	Forholdet mellom bruksverdi og salgsverdi

a)	Rentevurdering for neddiskontering av bruksverdi i erstatningssak: Hvilken avkastning vil grunneier få på erstatningssummen?
	
	Hvor mye risiko kan man kreve at grunneier tar når han investerer sin erstatning?
	
	Investere i it-aksjer, eller sette pengene i banken?

	Mer konkret: Hvordan skal grunneier kunne gjenskape sannsynlig kontantstrøm fra småkraftverk?

	"Prejudikatsrenten" på 5 %

	Må den settes ned?

b)	Rentevurdering for å finne salgsverdi på markedet: Hvilket avkastningskrav stiller aktørene?

	Hvor mye ekstrabetalt vil markedsaktørene ha for å ta risikoen ved å investere i et kraftprosjekt?

	Fasit i dag, ca 7 % dersom man regner med inflasjon, ca 5 % uten

	Tjørhom sin verdivurdering er basert på bruksverdibetraktning og prejudikatsrente, derfor meget høy

	Hva med ulike salgsverdibetraktninger, dersom dette fremstår som det påregnelige?

	Det enkleste er å benytte brutto leiemodell, velkjent fra markedet og skjønnspraksis

	Dette vil føre til erstatning i størrelsesorden 20-40 millioner for Mangåga

	Hva som er påregnelig her er et spørsmål for neste sesjon, det bebudes supplerende bevisførsel på dette punkt
	
6.	Konklusjon

	Kostnader:

	Tjørhom, som vil bli supplert på noen punkter

	Inntekter:

	Tjørhom, som vil bli supplert på noen punkter

	Verdier:

	Tjørhom sin beregning, men andre beregningsprinsipp vil også belyses

III	Samarbeidsmetoden

	Et spørsmål for neste sesjon

	Det sentrale skjønnsspørsmål: Hva er marginalkostnader og marginallønnsomhet av SKS Produksjon sin overføring?

	Generelt, usikkert, estimat for begge utbyggingene på 5.5 kr/kWh DU s. 181

	Usikkerhet må aksepteres	

	Men tallene er ikke spesisfisert opp mot Mangåga
	
	Dokumentasjon ut fra en marignalbetraktning bes fremlagt, jf DU s. 240-241

IV	Naturhestekraftmetoden

	Et spørsmål for neste sesjon

	Også her trengs den samme dokumentasjonen som nevnt over

V	Konklusjon


\end{document}

