\part{A Case Study of Norwegian Waterfalls}

\chapter{Norwegian Waterfalls and Hydropower}\label{chap:3}

\section{Introduction}\label{sec:into3}

Norway is country of mountains, fjords and rivers, where around 95 \% of the annual domestic electricity supply comes from hydropower.\footnote{See Statistics Norway, data from the year 2011, http://www.ssb.no/en/elektrisitetaar/.} The right to harness energy from rivers, streams and waterfalls generally belongs to local landowners under a riparian system.\footnote{This arrangement is rooted in the first known legal sources in Norway, the so-called ``Gulating'' laws, thought to have been in force well before AD 1000. See \cite[111-112,120]{robberstad81}.}  Historically, waterfalls were very important to local communities, particularly as a source of power for flour mills and saw mills.\footnote{See \cite[121]{tvedt13}.} %%Indeed, the fact that peasants in Norway controlled local water resources can help explain why they were relatively free, both economically and socially, compared to many other places in Europe.\footnote{See \cite[121]{tvedt13}.}

Following the industrial revolution, local ownership and management came under increasing pressure. At the beginning, this pressure was exerted by private commercial interests, often foreign investors, who saw the industrial potential in hydropower and started speculating in Norwegian water resources.\footnote{See \cite[30-31]{nou04}.} Later, the pressure on local self-governance was exerted mainly by the government, following the introduction of new legislation to regulate the development of hydroelectric power.\footnote{See \cite[41-57]{thue96} (describing the  regulatory system set up during this time).} This legislation set up a system that gave highly preferential treatment to public utilities over private actors, including local owners.\footnote{See \cite[46]{thue96} (describing legislation introduced to promote public utilities, including new expropriation authorities directed at local owners of waterfall).} At first, the motivation behind this reform was to facilitate a decentralised form of government control, led by public utilities controlled by the municipality governments.\footnote{See \cite[44-47]{thue96}.} However, the hydroelectric sector underwent gradual centralisation, a process that gained momentum after the Second World War when the state itself assumed a leading role.\footnote{See \cite[59-85]{thue96}. For the history of the state's involvement with hydropower generally, see \cite{thue06, skjold06,thue06b}.} At this time, local communities and local riparian owners  became increasingly marginalised. In particular, they were forced to shut down their hydroelectric plants in order to  connect the national, monopolised, electricity grid.\footnote{See \cite[p.111]{hindrum94}.}

Then, in the early 1990s, the electricity sector was reformed once again, largely inspired by the market-orientation and privatisation of the public sector in the UK under Thatcher.\footnote{See generally \cite{midttun98}.} The production sector was liberalised, while public utilities where reorganised as commercial companies.\footnote{See \cite[86]{efta07} (describing how Norwegian electricity companies, most of which are still (partly) publicly owned, now operate as for-profit, limited liability companies).} At the same time, the regulatory system was decoupled from both political and commercial decision-making processes, to become more expert-based.\footnote{[26-27]{brekke12}.} Moreover, the sector underwent additional centralisation, as a result of mergers and acquisitions among former public utilities.\footnote{See \cite[583]{bibow03}. I mention that despite significant continuous centralisation from the Second World War to this day, the Norwegian hydroelectric sector is still relatively decentralised compared to other countries, e.g., the UK, see \cite[181]\cite{midttun98}. Arguably, this is a lasting influence of a tradition based on local, egalitarian, ownership of water resources.}

Following the reform, access rights to the national grid are meant to be granted equally to all potential actors on the energy market, including private companies.\footnote{See generally \cite{hammer96}. For an interesting presentation and analysis of grid-based markets in general, see \cite{falch04}.} After the passage of the \cite{ea90}, the energy companies controlling the local grids were no longer authorised to shut out competitors.\footnote{See the \cite[3-4]{ea90}.} A side-effect of this is that it has become possible for local landowners to undertake their own hydropower projects. Local owners can now access the grid to sell the electricity they produce on Nord Pool, the largest electrical energy market in Europe.\footnote{See generally \cite{larsen06,larsen08,larsen12}.} This has led to increased tension between local interests and established hydropower companies. The following fundamental question has arisen: who is entitled to benefit from rivers and waterfalls, and who is entitled to a say in decision-making processes concerning their use?

In this chapter, I set the stage for discussing this question in more depth, by detailing how the hydropower sector is organised. I look to the law as well as to commercial and administrative practices and I focus on those aspects that have changed following the reform of the early 1990s. I pay particular attention to the growing importance and competitiveness of so-called {\it small-scale} hydropower, including development projects that are undertaken by local owners, or in cooperation with them. Several commercial actors have emerged who now specialise in such cooperation, by offering waterfall owners a significant share of the commercial value resulting from development. 

To bring out the multi-faceted character of the current debate on hydropower in Norway, I begin in Section \ref{sec:nutshell} by offering a basic overview of the Norwegian political system. I focus on the role that property has played in the history of Norwegian democracy. In Section \ref{sec:hl}, I move on to describe the law relating to hydropower development. I first identify a basic tension in the law -- some would call it an inconsistency -- between hydropower as a private right on the one hand, and a public good on the other. I then go on to explain how this tension permeates the law, by presenting the statutory regulation of hydropower development in more depth.

I follow this up by considering hydropower in practice, especially practices that relate to small-scale development in cooperation with local owners. I trace the history of the dominant model used to organise this form of development, going back to the first expression I could find of the core principles, given in the so-called {\it Nordhordlandsmodellen}, from 1996.\footnote{See \cite{dyrkolbotn96}.} This model presented a financial mechanism for benefit sharing that was later adopted by the market for small-scale hydropower development. In addition, the model expressed broader governance principles pertaining to the importance of sustainability, the involvement of non-owners, and the desirability of long-term planning based on local conditions and local participation in decision-making processes.

However, as I discuss in Section \ref{sec:future}, these aspects of the model have largely failed to make an impact. I go on to argue that a lack of social awareness  might be part of the reason why small-scale hydropower now appears to be falling out of favour. I note, moreover, how a recent change in the perception of small-scale hydropower has also lead to a resurgence of large-scale development. This is threatening to undermine the position of local communities and owners, and has led to a series of controversial cases before the courts.

This underscores the importance of maintaining a social function perspective on local ownership of riparian rights. Moreover, it provides important background and context for my study in the two chapters that follow, wherein I specifically address the use of expropriation -- and alternatives to it --  to facilitate hydropower development.

\section{Norway in a Nutshell}\label{sec:nutshell}

Norway is a constitutional monarchy, based on a representative system of government.\footnote{For Norwegian constitutional law generally, see \cite{andenes06}.} The executive branch is led by the King in Council, the Cabinet, headed by the Prime Minister. Legislative power is vested in the Storting, the Norwegian parliament, elected by popular vote in a multi-party setting.\footnote{It should be noted that the executive branch also enjoys considerable legislative power under Norwegian law. Both informally, because it prepares new legislation, and also formally, because it has wide delegated powers to issue so-called {\it directives} (forskrifter). Indeed, it is typical for acts of parliament to include a general delegation rule which permits the executive to legislate further on the matters dealt with in the act, by clarifying and filling in the gaps left open by it.} In 1884, the parliamentary system first triumphed in Norway, as the cabinet was forced to resign after it lost the confidence of parliament. The principle has since obtained the status of a constitutional custom. In particular, the cabinet can not continue to sit if parliament expresses mistrust against it. However, an express vote of confidence is not required. In practice, due to the multi-party nature of Norwegian politics, minority cabinets are quite common. These can sustain themselves by making long-term deals with supporting parties, or by looking for a majority on a case-by-case basis.

The judiciary is organised in three levels, with 70 district courts, 6 courts of appeal, and the Supreme Court. The district courts have general jurisdiction over most legal matters; there is no division between constitutional, administrative, civil, criminal courts. \footnote{However, there are distinct procedural rules for civil and criminal cases and a special court exists for {\it land consolidation}. See the \cite{lca79}. Moreover, both the district courts and the courts of appeal follow special procedural rules in {\it appraisement disputes}, for instance when compensation is awarded following expropriation. See the \cite{aa17} respectively, discussed in more detail in later chapters.} The courts of appeal have a similarly broad scope. Moreover, the right to appeal is ensured in most cases.\footnote{The right to an appeal is not absolute. In civil cases, it is generally required that the stakes are above a certain lower threshold, measured in terms of the appellants' financial interest in the outcome. See \cite[29-13]{da05}.} The Supreme Court, on the other hand, operates a very strict restriction on the appeals it will allow.\footnote{See the \cite[30-4]{da05}.} It typically only hears cases if a matter of principle is at stake, or if the law is thought to be in need of clarification.\footnote{See, generally, \cite{skoghoy08}.}

The Norwegian legal system is often said to be based on a special ``Scandinavian'' variety of civil law, which includes strong common law elements: legislation is not as detailed as elsewhere in continental Europe, some legal areas lack a firm legislative basis, it is generally accepted that courts develop the law, and the opinions of the Supreme Court are considered crucial to the legislative interpretation at the lower courts.\footnote{See, generally, \cite{bernitz07}.} At the same time, legislation remains the primary source used to resolve most legal disputes. Moreover, when applying the law, the courts tend to place great weight on preparatory documents procured by the executive branch. These documents are widely regarded as expressions of legislative intent, even though parliament is not usually actively involved in the process during the preparatory stages.

The Constitution of Norway dates back to 1814 and was heavily influenced by contemporaneous political movements, particularly in the US and France.\footnote{See generally \cite{mestad14}.} Moreover, it was influenced by a desire for self-determination, as Noway was at that time a part of Denmark-Norway, largely controlled by the Danish elite.\footnote{See generally \cite{ }.} Following the Napoleonic wars, Norwegian politicians sought to take advantage of Denmark's weak position to gain independence. In the end, Norway was instead forced to enter into a union with Sweden, but the Constitution remained in place. Moreover, after the triumph of the parliamentary system in 1884, Norway would also eventually gain independence, in 1905, following a peaceful and democratic transition process.\footnote{See generally \cite{sejersted15}.}

During the 19th century, farmers and peasants emerged as a powerful group in Norwegian politics. This, it is commonly held, was in large part due to the fact that they were also landowners, whose rights and contributions were not limited to traditional farming.\footnote{The ``classic'' presentation of the political influence of farmers in Norway is \cite{koht26}.} Importantly, Norwegian tenant farmers and small-holders had a significant degree of influence over the management of the land and its natural resources. The feudal tradition was never as strong in Norway as elsewhere in Europe.\footnote{See \cite[59-60]{pryser99}.}

\noo{ The majority of Norwegian peasants were tenants in the 17th century, but they generally enjoyed better protection against abuse. In addition, the remoteness of many rural communities and the challenging natural environment meant that large feudal estates could hardly operate effectively without granting much autonomy to local peasants. Moreover, the black death had severely taken its toll on the population in Norway, wiping out entire communities, including the feudal elites and the social and physical infrastructure that sustained them. It should be noted that this is also considered an important reason why Denmark could dominate Norway.\footnote{The origins of Denmark-Norway was the so-called Kalmar union, which also included Sweden. Initially, Norway took part on relatively equal footing with Sweden and Denmark. Later, however, after Sweden left the union, Denmark developed a much stronger position than Norway. See generally \cite{....}.}
}

The Danish-Norwegian nobility had fallen into a fiscal crisis in the 18th century, weakening their influence further. This had in turn made it possible for tenant farmers in Norway to buy land from their landowners,  including grazing grounds and non-arable land.\footnote{See \cite[59-60]{pryser99}.} As a result, the distribution of land ownership in Norway had already become highly egalitarian at the time of the Constitution. \noo{It is worth noting that farmers would typically purchase shares of larger estates. They would then acquire sole ownership over their own house and cultivated ground, while becoming co-owners of the surrounding land, alongside other local farmers.} Moreover, many resources attached to land were owned jointly by several members of the local community, as larger estates were partitioned into several individual smallholdings. As a result, Norway became a society were land ownership was not a privilege for the few, but held by the many, particularly compared to feudal Europe.\footnote{For a comparative discussion of this, focusing on how it influenced the industrialisation process in Norway, setting it apart from the industrialisation process in the UK, see \cite{brox13}.}

In 1814, the landed nobility in Norway was further marginalised. Indeed, the Constitution itself prohibited the establishment of new noble titles and estates.\footcite[23|118]{c14} Then, in 1821, all hereditary titles were abolished (although existing nobles kept their titles for their lifetimes).\footnote{See `Lov, angaaende Modificationer og nærmere Bestemmelser af den Norske Adels Rettigheder' (Act of August 1, 1821).} By the middle of the 19th century, ordinary farmers had gained even greater political influence. In fact, they emerged as the leading political class, alongside the city bureaucrats.\footnote{See generally \cite{hommerstad14}.} During this time, Norway also introduced a system of powerful local municipalities. These were organised as representative democracies, becoming miniature versions of the cherished, as of yet unfulfilled, nation state (Norway was still in a union with Sweden at this time). Even today, municipalities retain a great deal of power in Norway, particular in relation to land use planning.\footnote{They are the primary decision-makers for spatial planning, as pursuant to \cite{pb08}.} They represent a highly decentralised political structure, with a total of 428 municipalities in force as of 01 January 2013. \footnote{This is down from the all-time high of 747 in 1930. There have long been proposals to reduce the number of municipalities further, but so far the political resistance against this has prevented major reforms. See \cite{kommuner14} (report to the Ministry from an expert committee on municipality reform, 2014).}

Local control of water resources, ensured through property rights, was very important to farmers and rural communities in pre-industrial Norway. According to Terje Tvedt, 10 000 - 30 000 mills were in operation in Norway in the 1830s.\footnote{See \cite[121]{tvedt13}.} As Tvedt argues, the fact that these mills were under local control was particularly important because it helped ensure self-sufficiency. In addition, saw mills became an important source of extra income for Norwegian farming communities. \noo{While some of the larger mills were controlled and operated on behalf of non-local owners, most of them were run by the farmers themselves.} %Indeed, even during feudal times, tenant farmers often successfully argued that their tenancy entitled them to engage freely in the saw mill and timber industry, although the Danish Crown also put in place a concession system that gave them some power (until the system was abolised in 1830s).\footnote{...}

Today, the importance of water is clearly felt throughout Norwegian society. This is not because water is scarce, but rather because it is so plentiful. Not only is water power the main source of domestic energy. It also occupies a special place in Norwegian culture. It is important to the identity of many communities, particularly in the western part of the country, where majestic waterfalls are important symbols both of the hardship of the natural conditions and the sturdiness of local people. One particular aspect of this, with significant economic implications, is that waterfalls are an important asset to Norwegian tourism. The so-called ``Norway in a nutshell'' tours, for instance, have become greatly popular, based on delivering access to wild and unspoilt nature, with fjords, waterfalls, idyllic villages, and railway lines that seem to defy gravity.\footnote{See \cite{nutshell}.}

Another aspect of the same is the great tradition in Norway for local resistance against large-scale development that is considered damaging to the environment. In the 1960s and 70s, when the state embarked on their most ambitious projects, this led to a general political movement in Norway which saw progressive leftist protest groups join forces with local opposition groups in a fight against centralisation, exploitation of weaker groups, and environmental destruction.\footnote{See \cite{....}.}

This all speaks to the fact that water resources are embedded in the social fabric in Norway in such a way that an entitlements-based account of property rights to such resources would be largely inappropriate. Rather, the case of Norwegian streams and waterfalls seems to be particularly suited for an investigation based on a social function view on property. As I show in this and the following two chapters, rivers and waterfalls serve to bring out tensions between rights and obligations in property, while also shedding light on the question of how to organise decision-making processes regarding economic development.

In the next section, I argue that the present law on hydropower in Norway tends to recognise only a small part of the relevant picture. On the one hand, it recognises the financial entitlements of individual owners, which it tries to balance against the regulatory needs of the state. But it largely fails to take into account that owners have broader interests, even obligations, relating to the sustainable management of their streams and their waterfalls. Moreover, it also seems that the law is increasingly failing to take into account that commercial interests can exert a strong pull on various state bodies, particularly those that are only weakly grounded in processes of democratic decision-making.

As a result, the current narrative on water resource management in Norway appears to be based on a false dichotomy that sees the interests of ``profit-maximising'' owners, acting out of self-interest, pegged against the interests of a ``benevolent'' state, acting for the common good. In the following, I shed further light on this narrative and argue that it is deeply flawed.

\section{Hydropower in the Law}\label{sec:hl}

Under Norwegian law, rights to harness power from rivers and waterfalls are regarded as private property.\footnote{Historically, the law emphasised ownership of traditional agrarian water resources, such as fishing rights. However, new sticks were added to the waterfall bundle over the years, including the right to develop hydropower, see \cite[14-32]{vislie44}. For a detailed presentation of the history of water law in pre-industrial times, I refer to \cite{motzfeld08}.} The system is riparian, so by default, a stream belongs to the owner of the land over which the water flows.\footnote{See the \cite[13]{wra00}.} The landowners do not own the water as such -- freely running water is not subject to ownership -- and the riparian owners' right to withhold or divert water is limited.\footnote{See \cite[8|15]{wra00}.} It is common in Norway to refer to owners of hydropower rights as {\it waterfall owners} (`falleiere'), a terminology I will also adopt-\footnote{The Norwegian term `fall' has a somewhat broader meaning than its English counterpart, `waterfall'. The word `fall' is used to describe a continuous section of any stream or river, typically identified by giving the total difference in altitude over the relevant stretch of riverbed. Furthermore, the Norwegian term `falleier' refers to a legal person who possesses the rights to the hydropower over such a section. In this thesis, I will typically refer to the owners of waterfalls, streams and rivers with the intended reading being the same as the Norwegian notion of a `falleier'. If special qualification is needed, for instance to distinguish between different classes of riparian owners, I will make a note of this explicitly.}

The waterfall owners have the exclusive right to harness the potential energy in the water over the stretch of riverbed belonging to them. This right can be partitioned off from rights in the surrounding land, and large-scale hydropower schemes typically involve such a separation of water rights from land rights. In this way, the energy company acquires the right to harness the energy, while the local landowners retain ownership of the surrounding land.

Norwegian rivers, and especially rivers suitable for hydropower schemes, tend to run across grazing land and non-arable land that is owned jointly by local farmers. Hence, rights to streams and waterfalls are typically held among several members of the rural community.\footnote{Rivers tend to run through land that has not to been enclosed. Moreover, in places where there has been a land enclosure, water rights are often explicitly left out, such that they are still considered jointly owned rights belonging to the community of local farmers. For more details on (forms of) joint ownership among Norwegian farmers, see, e.g., \cite[570]{stenseth07}.} Local owners might not be willing to give up their ownership to facilitate development, especially not on terms proposed by external developers. Hence, the authority to expropriate has become an important legal instrument for Norwegian hydropower companies.

This has resulted in a tension where, on the one hand, rights to harness hydropower from streams and waterfalls are considered private property, while on the other hand, it has become common to speak of hydropower as a resource belonging to the public. Since the \cite{ica17} was amended in 2008, this ambivalence in the discourse surrounding hydropower has also been part of the statutory provisions regulating hydropower development. I quote the two relevant sections side by side below:\footnote{The first quote is taken from the general water law, with roots going back a thousand years to the so-called ``Gulating'' laws mentioned in Section \ref{sec:into3}. The second quote is taken from a law directed specifically at large-scale hydropower, introduced during the early days of the hydropower industry.}

{\begin{minipage}[t]{16em}
 \begin{aquote}{\tiny \cite[13]{wra00}} \footnotesize A river system belongs to the owner of the land it covers, unless otherwise dictated by special legal status. [...]

The owners on each side of a river system have equal rights in exploiting its hydropower.
\end{aquote}  
\end{minipage}}
{\begin{minipage}[t]{22em}
\begin{aquote}{\tiny \cite[1]{ica17} (after amendment in 2008)} \footnotesize Norwegian water resources belong to the general public and are to be managed in their interest. This is to be ensured by public ownership.
\end{aquote}
\end{minipage}} \\

The intended reading of section 1 of the \cite{ica17}, quoted on the right above, is that it expresses a ``general starting point''.\footnote{See \cite[72]{otprp61}.} According to the Ministry, it expresses no more than what has always been the purpose of the special licensing requirements for large-scale hydropower.\footnote{See \cite[72]{otprp61}.}

Despite appearances, it would be wrong to regard this as an attempt to explicitly confront the principle of private property expressed in section 13 of the \cite{wra00}, quoted on the left above. At least, such a confrontation does not appear to have been intended by the Ministry.\footnote{There are no indications in the preparatory materials that the Ministry sought to confront the principles of ownership encoded in the \cite{wra00}.} However, the Ministry's comment underscores the extent to which the government regards it as natural to interfere with private rights to waterfalls, to pursue policies that it regards to be in the public interest. Taken in this light, section 1 of the \cite{ica17} reflects the prevailing opinion that there are few, if any, recognised limits on the state's power to manage privately owned water resources.\footnote{For a reflection of the same attitude, citing the state's broad regulatory competence as the main reason not 
to nationalise Norwegian water power rights, I refer to the preparatory documents underlying the \cite{wra00}. See \cite[152-153]{nou94}.}

This aspect of the Norwegian system has become particularly significant following the liberalisation of the electricity sector in the early 1990s.\footnote{See, e.g., \cite{larsen06}.} Since then, there have been an increasing number of cases where owners who are interested in undertaking their own development schemes attempt to fend off commercial energy companies wishing to expropriate.\footnote{See, e.g., \cite{sofienlund07}.} Importantly, the state has tended to side with the commercial companies in these cases, granting them the authority to expropriate for economic development. This has resulted in several Supreme Court decisions on hydropower and expropriation in the past few years.\footnote{See \cite{uleberg08,otra10,jorpeland11,klovtveit11,otra13}.} Before discussing the case law in more detail in the next chapter, I present the most important legislation regarding hydropower development and provide a step-by-step overview of the licensing procedure. %I focus on those aspects that are particularly relevant to the position of local owners and their communities.

%Before we delve into the details, we will elaborate a bit further on the context in which the law was called upon to function in this case. Importantly, the economic, social and political context of expropriation has changed rather dramatically in recent years.

%There are two developments that have been particularly important. First, there has been a general shift from viewing electricity production as a public service to viewing it as a commercial enterprise. This has made the legitimacy of expropriation appear more controversial, and the argument is often voiced that expropriation does not happen in the interest of the public at all, but \emph{solely} in order to benefit the commercial interests of particular companies.\footnote{This has been a recurring theme in articles appearing in "Småkraftnytt", the newsletter for "Småkraftforeninga", an interest organization for owners of small-scale hydro-power, which currently have 236 associated small scale hydro-power plants, see http://kraftverk.net/ (in Norwegian). In addition to the case of Måland, the question has also been brought before the (lower) national courts in some other cases, such as \emph{Sauda}, LG-2007-176723 (Gulating Lagmannsrett, regional high court), and \emph{Durmålskraft}, see http://www.ranablad.no/nyheter/article5583405.ece (decision from the district court, as reported in a Norwegian newspaper). In both cases, the outcome was generally more favorable to the expropriating party than the local owners, and the reasoning adopted by the courts appears similar to that of \emph{Måland}.}

%In this way, expropriation of Norwegian waterfalls raises issues that have become increasingly important also in a global setting, and which seem to arise naturally in systems where economic activities are organized based on public-private partnerships. In such systems, it seems practically inevitable that cases of expropriation -- undertaken to benefit the public -- will also often come to benefit developers that are motivated by purely commercial interests. While this in itself might not be problematic, it will easily lead to the concern that the commercial interests of powerful companies is the main reason why expropriation is permitted, and that expropriation is being used as a commercial tool for powerful market forces, to the detriment of less powerful actors.

%For the case of Norwegian waterfalls, however, liberalization of the energy sector has also had a positive effect for local communities, in that it has served to make local owners more active. It has become increasingly common that they exploit their hydro-power resources themselves, often in small scale projects, and often in cooperation with companies that specialize in such development.\footnote{In 2012, the NVE granted 125 new licenses for small scale hydro-power, and at the end of the year they had 859 applications still under consideration. Source: report made by the NVE, available at http://www.nve.no/Global/Energi/Q412\_ny\_energi\_tillatelser\_og\_utbygging.pdf (in Norwegian). } This, of course, only adds to the controversy surrounding expropriation of waterfalls, especially when local owners are deprived of the opportunity for small scale development.

\subsection{The Water Resources Act}\label{sec:wra00}

The \cite{wra00} contains the basic rules regarding water management in Norway.\footnote{Act No 82 of 24 November 2000 relating to river systems and groundwater (unofficial translation provided by the University of Oslo, \url{http://www.ub.uio.no/ujur/ulovdata/lov-20001124-082-eng.pdf}). I also mention the Water Framework Directive of the European Union, \cite{water00}. It has been implemented in Norwegian law as the Directive Regarding Frameworks for Water Management, FOR-2006-12-15-1446. It does not directly impact on the hydropower licensing procedure, so I will not say much about it in this thesis. However, I mention that there is some concern that the Norwegian implementation of the directive has not sufficiently recognized the need for structural reforms, preferring to rely on the established approach to water management, which is centralised and sector-based. See \cite{hanssen14}.} This act is not only concerned with hydropower, but regulates the use of river systems and groundwater generally.\footnote{See the \cite[1]{wra00}. A river system is defined as ``all stagnant or flowing surface water with a perennial flow, with appurtenant bottom and banks up to the highest ordinary floodwater level'', see \cite[2]{wra00}. Artificial watercourses with a perennial flow are also covered (excluding pipelines and tunnels), along with artificial reservoirs, in so far as they are directly connected to groundwater or a river system, see the \cite[2a-2b]{wra00}.} In section 8, the Act sets out the basic license requirement for anyone wishing to undertake measures in a river system.\footnote{Measures in a river system are defined as interventions that ``by their nature are apt to affect the rate of flow, water level, the bed of a river or direction or speed of the current or the physical or chemical water quality in a manner other than by pollution'', see the \cite[3a]{wra00}.} The main rule is that if such measures may be of ``appreciable harm or nuisance''  to public interests, then a license is required.\footnote{See the \cite[8]{wra00}. There are two exceptions, concerning measures to restore the course or depth of a river, and concerning the landowner's reasonable use of water for his permanent household or domestic animals, see the \cite[12|15]{wra00}.} The water authorities themselves decide if this condition is met.\footnote{See \cite[18]{wra00}.} In relation to hydropower development, it is established practice that most hydropower projects over 1000 KW will be deemed to require a license.\footnote{See, e.g., \url{http://www.nve.no/no/Konsesjoner/Vannkraft/Konsesjonspliktvurdering/} (accessed 16 August 2014). Exceptions are possible, for instance projects that upgrade existing plants, or which utilise water flowing between artificial reservoirs.}

The basic assessment criterion is that a license ``may be granted only if the benefits of the measure outweigh the harm and nuisances to public and private interests affected in the river system or catchment area''.\footnote{See \cite[25]{wra00}.} Hence, the water authorities are empowered to decide whether a licence {\it should} be granted, if they find that the benefits outweigh the harms. The courts are very reluctant to censor the discretion of the administrative decision-makers on this point.\footnote{This is an expression of the principle of ``freedom of discretion'' for the administrative branch, a fundamental tenet of Norwegian administrative law. See generally \cite[71-74]{eckhoff14}.}

The Ministry of Petroleum and Energy maintains indirect control over the assessment process by issuing directives regarding the administrative procedure in licensing cases.\footnote{See section 65 of the \cite{wra00}.} In addition, the procedure is determined in large part by administrative practices developed by the water authorities themselves.\footnote{I return to a presentation of administrative practice in Section \ref{sec:step}.}.\footnote{In principle, many of the rules in the \cite{paa67} also apply. However, in practice, these rules are of limited practical relevance compared to sector-specific practices. This has raised controversy in recent years, particularly in cases involving expropriation, as discussed in Chapter \ref{chap:4}, Section \ref{sec:jorpeland}.} 

A few basic procedural rules are encoded directly in the \cite{wra00}. This includes rules to ensure that the application is sufficiently documented, so that the authorities have enough information to assess its merits.\footnote{See \cite[23]{wra00}.} Moreover, a basic publication requirement is expressed, stating that applications are public documents and that the applicant is responsible for giving public notice. The intention is that interested parties should be given an opportunity to comment on the plans.\footnote{See \cite[24]{wra00}. There are some exceptions to the requirement to give public notice, however. It may be dropped in case it appears superfluous, or if the application must be rejected or postponed, see \cite[24a-24c]{wra00}.} More detailed rules for public notice of applications are given in section 27-1 of the \cite{pb08}, which also applies to licensing applications under section 8 of the \cite{wra00}.\footnote{In addition, I mention section 22, which regulates the relationship between licensing and planning in relation to water resources. In essence, the section stipulates that the water authorities may prioritise planning over assessment of individual licensing cases, e.g., by refusing to take applications under consideration if they interfere with ongoing planning procedures. However, the section leaves significant room for discretion in this regard. It also bears noting that watercourse planning is placed under centralised government control. This contrasts with land use planning in general, which is mainly the responsibility of the local municipality governments. See generally \cite{sp}.}

\noo{Furthermore, an important rule of principle is given in section 22, regarding the relationship between applications for licenses and governmental ``master plans'' for the use or protection of river systems in a greater area. These plans have no clear legislative basis, but were introduced through parliamentary action in the 1980s, when the parliament decided to initiate such planning in an effort to introduce a more holistic basis for assessment of licensing applications.\footnote{Today, the planning authority is delegated to the Directorate of Natural Preservation and the NVE. See \cite{sp}.} %Moreover, the system has undergone reform as a consequence of Norway's implementation of the water directive of the European Union. See .......?!?} 
According to section 22 of the \cite{wra00}, if a river system falls within the scope of a master plan that is under preparation, an application to undertake measures in this river system may be delayed or rejected without further consideration.\footnote{See \cite[22]{wra00}, para 1.} Moreover, a license may only be granted if the measure is without appreciable importance to the plan.\footnote{See \cite[22]{wra00}, para 1.} In addition, once a plan has been completed, the processing of applications is to be based on it, meaning that an application which is at odds with some master plan may be rejected without further consideration.\footnote{See \cite[22]{wra00}, para 2.} It is still possible to obtain a license for such a project, but if it harnesses less hydropower than the project indicated by the plan, section 22 states that only the Ministry may grant it.\footnote{See \cite[22]{wra00}, para 2.}
}

The rules considered so far apply to any measures in river systems, not only hydropower projects. However, special procedures that apply to hydropower cases are described in other statutory provisions. The most important is the \cite{wra17}, which is specifically aimed at a certain subgroup of hydropower schemes, namely those that involve regulation of the flow of water in a river system.\footnote{See Section \ref{sec:wra17} below.} However, according to section 19 of the \cite{wra00}, many provisions from the \cite{wra17} also apply to unregulated, run-of-river, schemes, if they generate more than 40 GWh per annum.\footnote{See \cite[19]{wra00}.} %In the next section, I will present the \cite{wra17} in more detail.

\subsection{The Watercourse Regulation Act}\label{sec:wra17}

In order to maximise the output of a hydropower scheme, the flow of water may be regulated using dams or diversions. Regulation was particularly important in the early days of hydropower, before the national electricity grid was developed.\footnote{See \cite[83]{uleberg08}.} %Consumers did not want to pay for more energy than they needed when the flow was high, and they wanted to avoid power cuts in periods of drought. At this time, a few local hydropower plants were typically the only sources of electricity in any given area. This meant that regulation of the waterflow was needed to even out the level of electric output, otherwise the electricity supply would be unstable. 
Indeed, in the early days, it was common for electricity producers to get paid based on the stable effect they were able to deliver, rather than the total amount of energy they harnessed.\footnote{See \cite{sofienlund07}.}

%This changed with the development of a wide-ranging electricity grid, which allowed electricity to be imported and exported between different geographical areas depending on the levels of output from those areas.\footnote{See \cite[83-84]{uleberg08}.} 
Today, this has changed, as producers get paid based on the total amount of electricity they deliver,  measured in kilowatt hours (KWh). The price fluctuates over the year, and the supply-side is still influenced by instability in the waterflow in Norwegian rivers. However, the smoothing effect of the national grid means that run-of-river schemes can be carried out profitably, even if most of the electricity from the plant is produced during peak periods. %In addition, due to technological advances, the kinds of generators needed to exploit fluctuating levels of water have become much cheaper.

Despite the growing importance of run-of-river schemes, many key rules regarding hydropower development are still found in the \cite{wra17}.\footnote{Act relating to the regulation of watercourses of 14 December 1917 No. 17.} This act defines regulations as ``installations or other measures for regulating a watercourse's rate of flow''. It also explicitly states that this covers installations that ``increase the rate of flow by diverting water''.\footnote{See \cite[1]{wra17}.} The core rule of the act is that watercourse regulations that affect the rate of flow of water above a certain threshold are subject to a special licensing requirement.\footnote{See \cite[2]{wra17}.}

The threshold is defined in terms of the notion of a ``natural horsepower'', such that a license is required if the regulation yields an increase of at least 400 natural horsepower in the river. Natural horsepower is a measure of the gross estimate of the power that can be harnessed from a river stably for at least 350 days a year.\footnote{See \cite[2]{wra17}.} The definition is a simple mathematical expression, given below:

$$
nat.hp(Q,H) = 13.33 \times H \times Q
$$
This formula states that the natural horsepower of a regulation project ($nat.hp(Q,H)$) is a function of two variables, $H$ and $Q$. The constant factor $13.33$ is the force of gravity of Earth exerted on a mass of 1 kg (or, approximately, 1 litre of water). The variable $H$ is the difference in altitude (measured in metre) from the intake dam to the power generator. The variable $Q$ is the amount of water (measured in litre) stably available every second for at least 350 days per year. The result is then a gross estimate (assuming no energy loss) of the stable horsepower output of the hydroelectric plant that harnesses the power of $Q$ litres of water per second over a difference in altitude of $H$ metres. 

Section 2 of the \cite{wra17} asks us for the {\it increase} of this figure after regulation. To arrive at this number, one first uses the formula with $Q$ taken to be $Q_1$, the stable water flow prior to regulation, before calculating it with $Q$ taken to be $Q_2$, the stable water flow after regulation. The difference between the second and the first figure ($nat.hp(Q_2,H) - nat.hp(Q_1,H)$) is the increase of natural horsepower resulting from regulation.

Effectively, at a time when electricity had to be produced at a stable effect, from a stable source of power, this increase in natural horsepower was a gross estimate of the value added to the river by regulation.
%For this reasons, the courts 
%also turned to the notion of a natural horsepower to award compensation following expropriation of hydropower as such, a curious compensation practice that never had any legislative basis.\footnote{See \cite[82-83]{uleberg08}.} I return to this in more depth in the next chapter.
In the present context, suffice it to say that if a hydropower project involves regulation at all (i.e., if it is not a run-of-river scheme), it will indeed yield 400 natural horsepower or more. Hence, a special license will be required pursuant to section 2 of the \cite{wra17}. 

%In addition to these regulation projects, the \cite{wra00} stipulates that many rules in the \cite{wra17} apply to any hydropower scheme that will generate more than 40 GWh annually.\footnote{See \cite[19]{wra00}.}

The criteria for granting a regulation license are similar to those for granting a license pursuant to the \cite{wra00}. In particular, section 8 of the \cite{wra17} states that a license should ordinarily be issued only if the benefits of the regulation are deemed to outweigh the harm or inconvenience to public or private interests.\footnote{See \cite[8]{wra17}.} In addition, it is made clear that other deleterious or beneficial effects of importance to society should be taken into account.\footnote{See \cite[8]{wra17}.} Finally, if an application is rejected, the applicant can demand that the decision is submitted for review by parliament.\footnote{See \cite[8]{wra17}.}

The \cite{wra17} contains more detailed rules regarding the procedure for dealing with license applications. The most practically important is that the applicant is obliged to carry out an impact assessment pursuant to the \cite{pb08}.\footnote{Act no 71 of 27 June 2008 relating to Planning and Building Applications.} This means that the applicant must organise a hearing and submit a detailed report on positive and negative effects of the development, prior to submitting a formal application for a licence. Effectively, at least {\it two} detailed rounds of assessment are therefore required before a license is granted. %As I discuss in more depth in Section \ref{sec:step} below, impact assessments tend to focus on environmental issues as well as general societal consequences. The local perspective, particularly the effects of development on local owners, is usually not a primary concern. 

In addition to prescribing impact assessments, the \cite{wra17} contains more specific rules concerning the second public hearing that should take place, when the application as such is processed. First, the applicant should make sure that the application is submitted to the affected municipalities and other interested government bodies.\footcite[6]{wra17} Second, the applicant should send the application to organisations, associations and the like whose interests are ``particularly affected''.\footcite[6]{wra17} Along with the application, these interested parties should be given notification of the deadline for submitting comments, which should not be less than three months.\footnote{See \cite[6]{wra17}.} The applicant is also obliged to announce the plans, along with information about the deadline for comments, in at least one commonly read newspaper, as well as the Norwegian Official Journal.\footnote{\cite[6]{wra17}. The Norwegian Official Journal is the state's own announcement periodical.} %In so far as the water authorities find it ``reasonable'', the applicant is obliged to compensate landowners and other interested parties for expenses accrued in relation to legal and expert assistance sought in relation to the application.\footcite[6]{wra17}

A license pursuant to the \cite{wra17} might be cumbersome to obtain, but a successful application also results in a significant benefit. Most importantly, the license holder then automatically has a right to expropriate the necessary rights needed to undertake the project, including the right to inconvenience other owners.\footnote{See \cite[16]{wra17}.} Hence, expropriation is a side-effect of a regulation license. Even so, the issue of expropriation rarely receives any special consideration in regulation cases. In particular, the assessment undertaken by the water authorities is focused on the licensing issue, which does not compel them to direct any special attention towards owners' interests.\footnote{I demonstrate this, and discuss it in much more depth, in Chapter \ref{chap:4}, Section \ref{sec:jorpeland}.}

In general, the issue of who owns and controls the water resources in question receives little attention in relation to licensing applications, both pursuant to the \cite{wra17} and the \cite{wra00}. Instead, the focus is on weighing environmental interests against the interest of increasing the electricity supply and facilitating economic development. The issue of resource ownership is more prominent in relation to a third important statute, namely the \cite{ica17}.

\subsection{The Industrial Licensing Act}\label{sec:ica17}

In the early 20th century, industrial advances meant that Norwegian waterfalls became increasingly interesting as objects of foreign investment. To maintain national control of water resources, parliament passed an act in 1909 that made it impossible to purchase valuable waterfalls without a special license.\footnote{See \cite{....}.} The follow-up to this act is the \cite{ica17}, which is still in force.\footnote{Act relating to acquisition of waterfalls, mines, etc. of 14 December 1917 No. 16.} It applies to potential purchasers and leaseholders of rivers that may be exploited so that they yield more than 4000 natural horsepower.\footnote{Unlike section 2 of the \cite{wra17}, this asks only for the number of horsepower in the river (after regulation), not the {\it increase} of this number.}

In practice, this means that the act does not apply to many run-of-river hydropower schemes, even large-scale projects. Even some regulation schemes fall outside the scope of the \cite{ica17}, although most large-scale regulation schemes will be covered. Originally, the main rule in the \cite{ica17} stated that all licenses granted to private parties were time-limited, and that the waterfalls would become state property without compensation when they expired, after at most 60 years.\footnote{See the old \cite[2]{ica17}, in force before the amendment on 26 September 2008.} This was known as the rule of {\it reversion} in Norwegian law.\footnote{This is a misnomer, however, in light of how most rivers and waterfalls were originally owned by local peasants, not the state.}

In a famous Supreme Court case from 1918, the rule was upheld after having been challenged by owners on constitutional grounds.\footnote{See \cite{johansen18}.} This was based on the finding that reversion represented a form of regulation of property, not expropriation. Hence, it could not be challenged on the basis of section 105 of the Constitution, even though the owners were not awarded any compensation. 

While the rule of reversion withstood internal challenges, it was eventually struck down by the EFTA Court in 2007, as a breach of the EEA agreement.\footnote{See \cite{efta07}. The EEA (European Economic Area) agreement sets up a framework for the free movement of goods, persons, services and capital between Norway, Iceland, Lichtenstein and the European Union. The EFTA (European Free Trade Association) oversees the implementation of the EEA for those members of EFTA that are also members of the EEA (all except Switzerland). For further details, see generally \cite{bull94,magnussen02,fredriksen09}.} This conclusion was based on the fact that reversion only applied to privately owned companies, which the Court regarded as an illegitimate form of discrimination. After this ruling, the \cite{ica17} was amended. Today, only companies where the state controls more than 2/3 of the shares may purchase waterfalls or rivers to which the act applies.\footnote{See the \cite[2]{ica17}.}

This means that such rivers and waterfalls can only be bought, leased or expropriated by companies in which the state is a majority shareholder. In practice, however, landowners are still able to sell the land from which the right to a waterfall originates, even if this also means transferring the waterfall to a new owner. The rule is only enforced when riparian rights as such are transferred, specifically for the purpose of large-scale hydropower development. In particular, small-scale development and large run-of-river schemes can still be carried out by local owners. %Moreover, local owners may in theory still develop hydropower in rivers and waterfalls that fall under the act, since they already own them. But this would be difficult in practice if they are denied permission to partition the water rights off from the surrounding land, to make them available as stand-alone security for debt commitments. In effect, local owners would have a hard time acquiring financing for projects in these rivers, particularly if they do not wish to put their entire land holdings down as security.

%Moreover, if they succeed in acquiring financing, a development license would likely be hard to obtain. It is quite clear, in particular, that the Norwegian government takes the view that hydropower projects in waterfalls falling under the \cite{ica17} should only be undertaken by companies in which the state has at least 2/3 of the shares. Moreover, the state does not seem willing to differentiate between development by external commercial interests and development by local owners.
%In these cases, however, the rule is used to deprive local owners of their resources, in what seems to be a {\it de facto} expropriation of property rights. For instance, in the ongoing case of {\it Sauland}, the local owners of a large waterfall wish to develop a project that would fall under the \cite{ica17}. At the same time, a large-scale development involving the same waterfall is planned by a company in which the state owns more than 2/3 of the shares. The case is still pending a final decision, but the water authorities have stated their unwillingness to assess any licensing application from the local owners, unless they ensure that the state is granted at least a 2/3 stake in the development.\footnote{Source: NVE (\url{www.nve.no}).}

%In this case, the authorities use section 2 of the \cite{ica17} to deprive the owners of control over the development of their waterfalls. This, moreover, is not regarded as expropriation under Norwegian law. Hence, it would be unlikely to result in an obligation to pay compensation to original owners.\footnote{Instead, the question may be raised whether the government acted in accordance with the \cite{ica17} in this case, or extended its scope in an illegitimate way. After all, the project that the water authorities refused to consider did not in fact involve transferring control over the waterfalls to any new, non-local, owners.} The question of how to apply the \cite{ica17} in this situation has not yet been clarified by the courts. If local owners may be deprived of the development potential under the act, it also raises the further question of what the consequences will be for the level of compensation when this development potential is subsequently transferred to a company in which the state controls more than 2/3 of the shares. If the initial act of deprivation is regarded as following from regulation rather than expropriation, it would seem to follow from general expropriation law that no compensation is payable when the hydropower potential is subsequently taken by someone to whom all the necessary licenses may be granted.\footnote{I note the parallel with the {\it Agri} case in South Africa, which concerned similar mechanisms in the context of mineral rights, as discussed in Chapter \ref{chap:1}, Section \ref{sec:esr}.} In this situation, the hydropower potential might not need to be compensated, since it can no longer be said to represent a foreseeable source of income for the original owners.

The policy justification for the (amended) \cite{ica17} is based on the idea that giving preference to state-owned actors will protect the public interest in Norwegian hydropower. However, this perspective clashes with the fact that the electricity sector itself has been liberalised. The state may be a majority shareholder in the most powerful companies, but these companies are now run according to 
commercial principles, with little or no direct political involvement.\footnote{See \cite[86]{efta07}.}

Hence, as the EFTA court highlights in its judgement on reversion, there appears to be a lack of convincing policy reasons why state-owned companies should be given preferential treatment.\footnote{See \cite[84-87]{efta07}.} In light of this, Norway's response to the Court's decision is a curious one: instead of creating a level playing field, the preference given to state-owned commercial companies is made even more marked, as privately owned companies are now excluded from one segment of the hydropower market altogether.

Of course, the public benefits indirectly from the fact that public bodies, as shareholders, are entitled to dividends. But it is not clear why this benefit should be considered in a different light than other indirect financial benefits which might as well be extracted from private companies, e.g., through taxation. Moreover, public-private partnerships are still permitted, as private actors may own up to two-thirds of ``state-owned'' companies. What this means is that the preferential treatment given to state actors is in fact also extended to those private actors that the state happen to prefer. Interestingly, this style of regulation contrasts quite sharply with some of the key ideas behind the basic building block of the liberalised electricity market, namely the \cite{ea90}.

\subsection{The Energy Act}\label{sec:ea}

Before 1990, the Norwegian electricity sector was tightly regulated by the government.\footnote{See generally \cite{bye05,skjold07}.} The responsibility for the national grid was divided between various public utilities that would also typically engage in electricity production, wielding monopoly power within their districts. The most powerful utilities were controlled by the state, who also developed large-scale hydropower to supply the metallurgical industry with cheap electricity.\footnote{See \cite[67-71]{thue96}.} However, the county councils and the municipalities maintained a significant stake in the hydroelectric sector, as they often controlled the utilities responsible for the electricity supply in their own local area.\footnote{See \cite[85]{thue96}.} 
Prior to 1990, there was no real competition on the electricity market, and the local monopolists could deny other energy producers access to their segment of the distribution grid.\footnote{See \cite[83-84]{uleberg08}.}

%At the same time, the system ensured that energy companies were more or less directly politically accountable. They were typically run by politically appointed boards, often organised as administrative bodies of local government.\footnote{.....} Moreover, they were not subject to commercial management principles that insulated them from the political discourse of the local area in which they operated.\footnote{.....}

This system was abandoned following the passage of the \cite{ea90}.\footnote{See generally \cite{bibow11}.} This act set up a new regulatory framework, where management of the grid was decoupled from the hydropower production sector.\footnote{See generally \cite{bye05}.} In particular, the act established a system whereby consumers could choose their electricity supplier freely. At the same time, the act aimed to ensure that producers were granted non-discriminatory access to the electricity grid. This laid the groundwork for what has today become an international market for the sale of electricity, namely the Nord Pool.\footnote{See \url{http://www.nordpoolspot.com/About-us/}. See generally \cite{skjold07,galtung07}.}

In response to this, monopoly companies were reorganised, becoming commercial companies that were meant to compete against each other, and against new actors that entered the market.\footnote{See \cite{claes11}.} %The Norwegian state retained a significant stake as shareholders in energy companies, now often alongside private investors. Moreover, as seen in the previous section, Norwegian law continue to favour companies where a majority of the shares are held by the state. To this day, the largest and most influential companies remain under majority public ownership. 
%the recent EFTA Court case, Case E-2/06, \emph{EFTA Surveillance Authority v. The Kingdom of Norway}, EFTA Court Report 2007, p.164. Here, the Court considered the old Norwegian rule of \emph{reversion}, whereby a license to undertake certain large scale hydro-power schemes (strictly speaking, a license to acquire the waterfalls needed to undertake it) came with a special clause that the private developer had to give up ownership to the State after a fixed period of time. This clause was held to be in breach of the EEA agreement since it only applied to private companies. We remark that the Norwegian government responded to this with an amendment after which reversion no longer applies, but which stated that a license to acquire waterfalls for the purpose of such large scale schemes can not be given at all to any company in which private parties own more than 1/3 of the shares.} %The aim of liberalization in Norway was not to minimize state entitlements arising from the hydropower sector. Rather, the intention was restricted to giving consumers greater freedom in choosing their energy-supplier, as well as to enhance efficiency in the sector by introducing competition.\footnote{See for instance \cite{liberal}, which offers a comparative study of the liberalization of the energy sectors in Norway and the UK.} In practice, however, this means that the state is now expected to function as a regular shareholder, which directs its companies to be run according to principles of commercial governance. Indeed, this is to some extent even a legally enforced requirement, e.g., pursuant to the free trade and competition rules of the EEA agreement. 
In addition to commercialisation, the market-orientation of the sector has also lead to centralisation, as many of the locally grounded municipality companies have disappeared as a result of mergers and acquisitions.\footnote{Today, the 15 largest companies, largely controlled by the state and some prosperous city municipalities, own roughly 80\% of Norwegian hydropower, measured in terms of annual output. See \cite[28]{otprp61}. I remark that the process of consolidation started even before the market-oriented reform of the sector. In particular, from 1960 onwards there was a significant push towards centralisation, as the state became a more dominant actor in the hydropower sector. For the state's increasing influence on the sector generally, see \cite{skjold06,thue06b}.} As a result, the local and political grounding of the electricity sector, which used to be ensured through decentralised municipal ownership, has been significantly weakened.

At the same time, the fact that any developer of hydropower is now entitled to connect to the national grid gives private actors a possibility of entering the Norwegian electricity market. They may do so not merely as (minority) shareholders in former utilities, but also as {\it competitors}, as long as they stick to run-of-river or small-scale hydropower.\footnote{See generally \cite{larsen06,larsen08,larsen12}.} In the next section, I give a step-by-step presentation of the licensing procedure for hydropower, which serves to summarise the legislative framework and provide information about the institutional framework within which it is called to function.

\noo{ The \cite{ea90} contains provisions that lay down basic rules regarding various kinds of licenses needed to construct and operate electric installations, including those needed for the development of hydropower. Usually, however, these licenses are not as controversial or hard to obtain as those pertaining directly to the hydropower plant itself. In addition, they raise no particular issues relating to expropriation or control over water resources, so I will not discuss them further in this thesis. Instead, I will now give a step-by-step description of the licensing procedure for hydropower development, to summarise and elucidate on the rules presented in this and preceding sections.}

%By far the most important aspect of the act for hydropower development, particularly in cases when different parties are interested in carrying out development, concern the right to connect to the grid. However, this right can become illusory due to the fact that the local grid company -- a monopolist -- has the right to demand contributions from hydropower developers, in so far as the grid needs to be improved in order to handle the output from their hydropower schemes. 
%
%The size of the contribution required, and the technical basis for calculating it, is largely determined by the grid company itself. In many cases, the costs can become so great as to prohibit  profitable development. This can result in conflict, particularly in cases when the grid company is affiliated with a competing hydropower developer who has a commercial interest in preventing other developers from connecting to the local grid. This issue arises in many cases involving expropriation, as the beneficiary is often also the local grid monopolist. In effect, the taker of the waterfall is tasked with calculating the cost of connecting alternative, owner-led, projects to the grid, a crucial factor in determining the  compensation payable. I return to this issue in Chapter \ref{chap:5}. 
%
%In the next section, I consider a recent piece of legislation that creates an additional financial incentive for carrying out hydropower development, by introducing a certificate system to subsidize 
%environmentally friendly energy.
%
%%The most significant step towards liberalization of the Norwegian energy sector was made in 1990 when the Energy Act was passed, an important new piece of statute reorganizing the system for the distribution of electricity.\footnote{Act nr. 50 of 29 of June 1990 relating to the generation, conversion, transmission, trading, distribution and use of electricity.} 
%
%\subsection{Electricity Certificate Act}\label{eca11}
%
%In the 1960s and 70s, hydropower projects in Norway often sparked great controversy, with environmental groups in particular protesting what they saw as unjustifiable destruction of nature in the interest of economic development. Today, however, the environmental interests in hydropower are more divided. On the one hand, many still regard hydropower skeptically as destruction of nature. On the other hand, the increased focus on global warming has led many environmentalists to embrace hydropower as a renewable energy source.
%
%In 2011, the \cite{eca11} was passed, to set up a market for trade in so-called ``green energy'', which would effectively subsidize the further development of renewable energy, including hydropower.\footnote{Act relating to electricity certificates of
%24 June 2011 No. 39.} The basic building block of the new market is the electricity certificate, which will be issued to all renewable energy projects completed before 2020. The demand for such certificates is then created artificially, as the Act stipulates that energy suppliers and certain categories of end-users are required to purchase certificates based on their electricity consumption. 
%
%The Act sets up an incentive to develop hydropower, and it contributes significantly to the profitability of hydropower development.

\subsection{The Licensing Procedure}\label{sec:step}

The water authorities in Norway are centrally organised. The most important body is the Norwegian Water Resources and Energy Directorate (NVE), based in Oslo.\footnote{See \url{www.nvn.no}.} In many cases, the NVE have been delegated authority to grant development licenses themselves, but in case of large-scale development, they only prepare the case, then hand it over to the Ministry of Petroleum and Energy.\footnote{See delegation of 19 December 2000, from the Ministry of Petroleum and Energy (FOR-2000-12-19-1705) and directive of 15 December 2000, from the King in Council (FOR-2000-12-15-1270), pursuant to \cite[64]{wra00}.} The Ministry, in turn, gives its recommendation to the King in Council, who makes the final decision.\footnote{See directive of 15 December 2000, from the King in Council (FOR-2000-12-15-1270).} Parliament must also be consulted for regulations that will yield more than 20 000 natural horsepower.\footnote{See \cite[2]{wra17}.}

%The local municipalities are becoming increasingly marginalised in relation to hydropower management. Their role is usually limited to commenting on the plans, alongside other stakeholders. %At the same time, the \cite{pb08} does play an important role, but not in so far as , which normally grants significant authority to local %municipalities does play an important role. This is because projects over a given threshold requires an impact assessment (IA) to be carried out pursuant to section ..... 

As indicated by the survey of relevant legislation given in previous sections, there are many categories of hydropower projects. Moreover, different categories call for different licenses. Hence, the first step in the application process is for the developer to determine exactly what kind of license they require. This is further complicated by the fact that some categories overlap, since they are based on different measuring sticks for assessing the scale of an hydropower project. 

One important parameter is the power of the hydropower generator, measured in MW (Megawatts). There are four categories of hydropower formulated on this basis: the micro plants (less than $0.1$ MW), the mini plants (less than $1$ MW), the small-scale plants (less than $10$ MW), and the large-scale plants (more than $10$ MW). In practice, one tends to use small-scale hydropower more loosely, to refer to all projects less than 10 MW. Still, a further qualification is sometimes required. For example, the authority to grant a license for a micro or mini plant has been delegated to the regional county councils since 2010, in an effort to reduce the queue of small-scale applications at the NVE.\footnote{See delegation letter from the Ministry of Petroleum and Energy, dated 07 December 2009, available at \url{http://www.nve.no} (accessed 24 August 2014). The county council is an elected regional government institution situated between the municipalities and the central government. There are 19 county councils in Norway as of 01 January 2015. They are comparatively less important than both the municipalities and the central government, but have several  responsibilities, particularly in relation to infrastructure, education and resource management. See generally \cite{berg15}.} The council's decision is based on a (simplified) assessment made by the regional office of the NVE. In addition, licenses for micro and mini plants may be granted even in watercourses that have protected status pursuant to environmental law.\footnote{See Decision no 240, Stortinget (2004-2005), St.prp.nr.75 (2003-2004) and Innst.S.nr.116 (2004-2005).}

For small-scale plants proper, the authority to grant a license is delegated to the NVE, with the Ministry serving as the instance of appeal.\footnote{See delegation of 19 December 2000, from the Ministry of Petroleum and Energy (FOR-2000-12-19-1705).} For large-scale plants, the granting authority is the King in Council, based on a recommendation from the Ministry.\footnote{See directive of 15 December 2000, from the King in Council (FOR-2000-12-15-1270).} However, in practice, the decision is usually closely based on assessments and recommendations provided by the NVE.\footnote{For a detailed guide to the administrative process for large-scale applications, published by the NVE, see \cite{stokker10}.}

While the relevant licensing authority depends on the effect of the planned plant, the kind of license required depends on a different categorisation, relating to the level of planned water regulation, measured in natural horsepower. Here, there are three categories: run-of-river schemes  (less than $500$ natural horsepower), non-industrial regulations (less than $4000$ natural horsepower), and industrial regulations (more than $4000$ natural horsepower).\footnote{See \cite[2]{wra17} and \cite[1,2]{ica17}.} % 20 000 NatHp), and the very large regulations (> 20 000 NatHp). 

Almost all hydropower schemes require a license pursuant to section 8 of the \cite{wra00}.\footnote{As mentioned in Section \ref{sec:hl}, the exceptions are very small schemes (usually mini or micro) that are deemed to be relatively uncontroversial. Such schemes only require a license pursuant to the \cite{pb08}.} For run-of-river schemes, no further licenses are required for the development itself, although an operating license pursuant to the \cite{ea90} is typically required for the electrical installations.\footnote{See \cite[3-1]{ea90}.} For schemes involving a non-industrial regulation, an additional license pursuant to section 8 of the \cite{wra17} is required. Industrial regulation schemes require yet another license, pursuant to section 2 of the \cite{ica17}.

As is to be expected, the complexity of the licensing procedure tends to increase with the number of different licenses required. However, the licensing applications tend to be dealt with in parallel, so that all licenses are granted at the same time, following a unified assessment. In practice, when the \cite{wra17} applies, it structures the procedure as a whole, also those aspects that pertain to other licenses. 

In addition, yet another categorisation of hydropower schemes is used to determine the relevant application procedure. This categorisation is based on the annual production of the proposed plant, measured in GWh/year. There are three categories: simple schemes (less than $30$ GWh/year), intermediate schemes (less than $40$ GWh/year), and complicated schemes (more than $40$ GWh/year). As mentioned in Section \ref{sec:wra17}, the most important rules in the \cite{wra17} applies to complicated schemes, regardless of whether or not the scheme involves a regulation.\footnote{See \cite[19]{wra00}.} In addition, applications for such schemes must be accompanied by an impact assessment pursuant to section 14-6 of the \cite{pb08}.

This means that the applicant is required to organise a public hearing prior to submitting their formal application, to collect opinions on the project and provide an overview of benefits and negative effects of the plans, particularly as they relate to environmental concerns.\footnote{See directive of 19 December 2014 (FOR-2014-12-19-1758), pursuant to the \cite[1-2,14-6]{pb08}.} In practice, if an impact assessment is required this significantly increased the scope and complexity of the application processing.

For intermediate schemes that do not involve regulation, the rules in the \cite{wra17} do not apply. However, impact assessments {\it may} still be required.\footnote{See \cite[20]{stokker10}.} Here the threshold of 30 GWh/year has been set as an additional threshold by the NVE, who have been delegated authority to require impact assessments for hydropower projects even when these yield less than 40 GWh/year.\footnote{See directive of 19 December 2014 (FOR-2014-12-19-1758).} For the intermediate schemes, NVE decides whether an impact assessment is required on a case-by-case basis. For simple schemes, on the other hand, impact assessments will not be required. Such schemes make up the core of what is described as small-scale hydropower in daily language.

The time from application to decision can vary widely, depending on the complexity of the case, the level of controversy it raises, and the priority it receives by the licensing authority. Usually, the assessment stage itself will last 1-3 years, sometimes longer.\footnote{See \cite[84-85]{nou129}.} While large-scale schemes involve more complicated procedures, they are also typically given higher priority than small-scale schemes. In recent years, following the surge of interest of small-scale development, a processing queue has formed at the NVE.\footnote{See \cite[84]{nou129}.} This means that small-scale applications typically have to wait a long time, sometimes several years, before the NVE begins processing them.\footnote{See \cite[84]{nou129}.}

As I will discuss in more depth in the next chapter, the issue of expropriation is rarely given special attention during the application assessment. This is so even in cases when an application to expropriate waterfalls is submitted alongside the licensing applications. The issue of expropriation is rarely singled out for special treatment, at least not in cases of large-scale development. Moreover, as mentioned in Section \ref{sec:hl}, an automatic right to expropriate follows from section 16 of the \cite{wra17}.

%This rule is not understood to cover the right to harness hydropower as such, but it {\it is} understood to cover the right to divert water away from river systems where the applicant has no previous riparian rights.\footnote{See \cite{jorpeland11}.} This is a {\it de facto} expropriation of a riparian right, and it is recognised as such in relation to the issue of compensation. However, it does not count as expropriation of a right to harness hydropower. Recently, the Supreme Court held that because of this,  many of the procedural rules that ordinarily apply to expropriation of riparian rights are not relevant.\footnote{See \cite{jorpeland11}.} Rather, the procedural rules and practices related to the licensing procedure are considered exhaustive.

%These rules and practices pay little attention to the interests of local owners and the immediate local community. Usually, the only locally grounded actor that is recognised as playing an active role in the process is the municipality. However, even the role of the municipality is limited. Once a license is granted according to a sector-specific statute, no regular planning license needs to be obtained from the municipality government.\footnote{See \cite[12-1]{pb08}.} However, the municipalities must be notified of any application that might affect their interests, and they are expected to express their views.\footnote{See \cite[8]{wra17} and \cite[24]{wra00}.} In addition, they may protest against the plans using a form of objection that requires the NVE to enter into dialogue with them about possible changes and improvements.\footnote{See \cite[5-4,5-6]{pb08}, c.f., \cite[24]{wra00}.} If an agreement is not reached, a license can still be approved, although then always by the Ministry, not the NVE.\footnote{See \cite[5-6]{pb08}.}

The procedural protection offered to local owners is very limited, particularly in relation to large-scale schemes. The NVE is not even obliged to notify the owners of licensing applications concerning their riparian rights. Typically, it is expected that the applicant notifies affected owners when submitting a license application. However, it is not established practice for the NVE to check that the applicant has fulfilled their obligation in this regard.\footnote{See \cite{jorpeland11}, and the discussion in Chapter \ref{chap:4}, Section \ref{sec:jorpeland}.} 

Interestingly, the applicant has been given responsibility for many aspects of the assessment process in licensing cases, including the assessment of possible alternatives.\footnote{See \cite{stokker10}. For a concrete example of its effect in expropriation cases, I refer to \cite{jorpeland11} and the detailed assessment of this case offered in the next chapter.} This would seem to raise competency questions, particularly in cases where the owners themselves propose alternatives.
However, even for such cases it is established practice for the NVE to rely on information supplied by the applicant, a practice that the Supreme Court has accepted.\footnote{See \cite{jorpeland11}.}

In cases that fall under the \cite{wra17}, the NVE must send its final report and recommendation to the interested parties for comments.\footnote{See \cite[6]{wra17}.} It is established practice that local owners do {\it not} count as interested parties in this regard.\footnote{See \cite{jorpeland11}.} This includes the owners of those rivers and waterfalls that the applicant wishes to expropriate. Hence, while the municipalities and various environmental interest groups are informed of how the case progresses and asked to comment prior to the final decision, the owners must inquire on their own accord if they wish to be kept up to date on the application process.

\noo{The lack of procedural safeguards protecting the interests of local owners is reflected in the kind of assessments that tend to be carried out. It is typical for assessments to focus primarily on the benefit of increased electricity production weighed against the negative effects on the natural environment. This, indeed, is the perspective that permeates the whole system, from the rules setting out the expected content of applications, through to the procedures followed when assessing cases, towards the criteria used to determine if a license should be granted.\footnote{See \cite{stokker10}.} 

As a result, the opinions of environmental groups and expert agencies are typically taken seriously, while the owners typically struggle to make an impact.\footnote{See \cite{jorpeland11}.}


To summarise, hydropower cases are assessed from within an expert-based system of governance, which also relies heavily on data that is collected and presented by the applicant. Political voices tend to remain fairly distant, although special interest groups can still play an important role. Municipality companies have been replaced by commercial actors on the applicant side, while the most important administrative decision-maker, the NVE, is a centralised, expert-based, directorate.
}

In summary, the procedural framework surrounding licensing of hydropower development leaves local owners in a highly precarious position. Given that expropriation is often an automatic side-effect of a development license, this already suggests that legitimacy issues are likely to likely to arise when waterfalls are taken for hydropower. I return to this issue specifically in the next chapter. First, I will discuss market practices in more depth, focusing on the changes that resulted from the liberalisation reform of the early 1990s. 

%I begin by considering the established part of the sector, by presenting the ownership and management structures surrounding large-scale plants and the management of the national grid. Then I go on to consider specifically the surge of interest in small-scale hydropower, which represents an important counterweight to the process of centralisation that has followed in the wake of the reform.

%
%
%
%
%of unit of power 
%
%
%For the local owners of waterfalls, the situation is worse, since they are not identified as stakeholders in large scale projects. They are not, in particular, mentioned in the Watercourse Regulation Act, Section 6, which regulates the steps that must be taken when preparing such cases.\footnote{Nor do the seem to be mentioned  in any of the documents setting out how the authorities deal with such cases in practice. See, for instance, the guide published by NVE \cite{rettleiar} (in Norwegian), directed at applicants, and setting out how NVE deals with cases involving large scale hydro-power.} Consequently, it is hardly surprising that in administrative practice, it has been uncommon to devote particular attention to local owners. Rather, the focus has typically been on environmental issues and the opinions of various interest groups, such as hunter or fishermen's associations.\footnote{For a more in depth account of the process, we point to the standard legal reference on Norwegian water law \cite{falk}(in Norwegian).}

\section{Hydropower in Practice}

The history of hydropower in Norway can be roughly divided into four stages. The first stage was the development that took place prior to 1909. During this time, private actors dominated, with public ownership playing a minor role.\footnote{See \cite{otprp61}.} Moreover, there were many private interests speculating in acquiring Norwegian waterfalls, anticipating the value that these would have for industrial development.\footnote{See \cite[30-31]{nou04}.}

After 1909, the introduction of licensing obligations and the rule of reversion made it much harder for private companied to acquire waterfalls that were suitable large-scale industrial development. At the same time, local municipalities began to invest in hydropower to provide electricity to its citizens, a service they were increasingly being obliged to provide.\footnote{See \cite{otprp61}.} This marked the start of the second stage of hydropower development, which saw the development of a more strictly regulated sector. However, this sector was also highly decentralised, for a large part dominated by local actors.

In fact, throughout the first half of the 20th century, most hydroelectric plants were small-scale plants that supplied local communities with electricity.\footnote{See \cite[11]{utbygd46}. This is a report from the water directorate published in 1946, showing that as of 31 December 1943, $97.8 \%$ of all hydroelectric plants in Norway were small-scale plants. However, these plants contributed only $28 \%$ of the total hydroelectric power installed at that time.} Moreover, as late as in 1943, $89 \%$ of all hydroelectric power stations in Norway were still private, many of which were mini and micro plants that were owned and operated by the local community.\footnote{See \cite[6]{utbygd46}. See also \cite[111]{hindrum94}.} However, many bigger plants were also under private ownership, and $57 \%$ of the total hydroelectric power available at this time was supplied by the private sector. Interestingly,  while the micro and mini plants accounted for $72.9 \%$ of the total number of plants, they only accounted for $1.6 \%$ of the total electricity supply.\footnote{See \cite[7]{utbygd46}.} 

This clearly illustrates the importance of smaller, local initiatives, in the process of providing Norway with electricity, particularly in rural areas. By the end of 1943, $80 \%$ of the Norwegian population had access to electricity at home. In scarcely populated rural areas, the corresponding figure was $70 \%$.\footnote{See \cite[7]{utbygd46}.} Hence, the decentralised approach to hydropower development, based on private ownership and local control, succeeded in supplying electricity to most of the country's population.

However, the regulatory regime was soon to undergo a significant change, designed to facilitate industrial development and increased state control. This change came quite rapidly after the Second World War, when the central government began to invest heavily in hydropower, often to ensure economic development by subsidising the metallurgical industry.\footnote{See \cite[59-65]{thue96}.} This period saw increased marginalisation of small private electricity companies, as well as local owners.\footnote{At the same time, powerful (private) metallurgical interests benefited greatly, sometimes also at the expense of the general supply of electricity. See \cite[65-71]{tvedt96}.} Indeed, it was often demanded, as a condition for allowing local communities access to the national electricity grid, that local hydroelectric  plants had to be shut down.\footnote{See \cite[p.111]{hindrum94}.} During this time, the development of hydropower was seen as an important aspect of rebuilding the nation, a task carried out in the public interest, not primarily to supply the public with electricity, but rather to facilitate a specific kind of economic development that the central government regarded as desirable.\footnote{See \cite[59]{thue96}.}

%At the same time, the typical development project had grown both in scale and complexity, making environmental worries and local demands for increased benefit sharing more convincing.\footnote{See \cite[73]{thue96}.} 
The state-dominated system set up on this 
basis remained in place until the 1970s, when environmental concerns and discontent among local populations led to new reforms.\footnote{See \cite[71-75]{thue96}.} As the scale of typical development projects had grown significantly, new projects would tend to meet with significant opposition from various stakeholders, including environmental interest groups, local communities, as well as municipal and regional government institutions.\footnote{See \cite[71-72]{thue96}.} The typical response from the state was to introduce measures that sought to pacify the regional and municipal government opposition, which was considered more serious than opposition from local people and environmental groups. The standard approach was to grant an increased share of the financial benefit to local and regional institutions of government, to instil support for state-led development plans.\footnote{See \cite[73-76]{nilsen08}.} This generally worked quite well, and to some extent it limited the centralisation process and the state's power over the hydroelectric sector.\footnote{See \cite[85]{thue96}.}

The fourth stage of hydropower development began in 1990 after the passage of the \cite{ea90}. The liberalisation that followed saw the transformation of the hydropower sector into a commercial market, based on profit-maximising and competition. As a result, the structure of decentralised management withered away further, as many municipality companies were either bought up by more commercially aggressive actors or forced to merge and change their business practices in order to remain competitive.\footnote{See \cite[583]{bibow03} (commenting on the increased consolidation of power on the electricity market, following acquisitions and mergers after 1990).} At the same time, a new decentralised force emerged in the sector, in the form of local owner-led projects.\footnote{See Section \ref{sec:small} below.}

%While private, these actors are not comparable to the industrial speculators that prompted the legislative response in the early 20th century. Rather, they arose as a reflection of the egalitarian system of land ownership in Norway, carrying forward an ancient tradition of local control over water resources that had been temporarily superseded by the political legitimacy of local municipalities.  
%As mentioned, owners of waterfalls are typically groups of ordinary local residents, most often farmers. With farming on the decline in Norway, particularly in those parts rich in water power, this development represents a chance for rejuvenation for rural communities. 

The core idea behind the \cite{ea90} was that the electricity sector should be restructured in such a way that production and sale of electricity, activities deemed suitable for market regulation, would be kept organisationally separate from electricity distribution over the national grid, a natural monopoly. However, the act itself does not explain in any depth how this is to be achieved. In practice, the divide has not been strictly implemented. Most of the large energy companies in Norway continue to maintain interests in both distribution, production and sale of electricity, a phenomenon known as ``vertical integration''.\footnote{See \cite[580-583]{bibow03}.} In fact, the degree of vertical integration in the electricity sector initially increased after the passage of the \cite{ea90}.\footnote{See \cite[583]{bibow03}.}

To some extent, the water authorities have responded to this by making use of their authority to give organisational directives when they grant distribution licenses.\footnote{See \cite[4-1]{ea90}, para 2, no 1.} For instance, electricity companies are now required to keep separate accounts for production, distribution and sale of electricity.\footnote{See directive of 11 March 1999 (FOR-1999-03-11-302), s 4-4 a and s 2-6, issued by the NVE pursuant to directive of 7 December 1990 (FOR-1990-12-07-959), s 9-1, cf., \cite[10-6]{ea90}.} It is also required that transactions across these functional divides are clearly marked, and that they are based on market prices.\footnote{See directive of 11 March 1999 (FOR-1999-03-11-302), s 2-8.} %Moreover, the NVE serves a control function in this regard, as they review the accounts of distributors on an annual basis.\footnote{See directive of 11 March 1999 (FOR-1999-03-11-302), s 2-1.}

The water authorities have sometimes gone further, by requiring that a separate company is set up to manage distribution activities.\footnote{See \cite[581-582]{bibow03}.} However, it is permitted for this reorganisation to take place through the formation of a conglomerate, under a single parent company that controls both the distribution company, the production company and the sales company. Indeed, this model has now been implemented by most of the large energy companies in Norway.\footcite[582]{bibow03}

It seems unclear whether this approach really achieves the stated objective. By adopting the conglomerate model of organisation, the major players on the market have successfully gained control over a larger share of both the production and distribution facilities for electricity. Hence, these actors effectively control the core infrastructure that makes up the backbone of the Norwegian electricity sector. The {\it intention} is that monopoly power should only be exercised with respect to the distribution grid on non-discriminatory terms. But is this realistic when the conglomerate controlling the grid operator has significant stakes also in production and the trade of electricity?

This question calls for a separate study, and I will not be able to address it in any depth here. However, I will direct attention at one aspect that arises with particular urgency for small-scale development of hydropower, concerning access to the grid. It is quite common, in particular, that small-scale projects remain unrealised because the grid is regarded to lack sufficient capacity to accommodate new electricity.\footnote{See, e.g., \cite[84,161-162]{nou129}.}

Following an amendment of the Energy Act in 2009, grid companies are now obliged to facilitate access for producers, even when this necessitates new investments.\footnote{See Act no 105 of 19 June 2009 regarding changes in the \cite{ea90}.} However, the energy producer seeking access is typically required to reimburse the grid company for the cost of new investments, as determined in the first instance by the grid company itself (the NVE serves a supervisory function).\footnote{See directive of 7 December 1990 (FOR-1990-12-07-959), s 3-4.} In addition, grid companies may still deny access in cases when the needed investments are not ``socio-economically rational''.\footnote{See \cite[3-4]{ea90}. The authority to decide whether this requirement is fulfilled is vested with the Ministry.} %Hence, hundreds of small-scale development licenses remain unrealised because the license holders are still unable to access the grid.

%The grid company is authorised to deny access in such cases.\footnote{See \cite[3-4]{ea90}, c.f., directive of 7 December 1990 (FOR-1990-12-07-959), s 3-4.}
Often, the relevant grid company will be a sister company of an energy producer operating in direct competition with the company seeking access. This can raise questions about the impartiality of the assessments carried out by the grid company. In expropriation cases, this becomes a particularly thorny issue, especially in relation to the assessment of the cost of undertaking an alternative development schemes.\footnote{This assessment is often crucial, because it provides information about the value of the development potential that the owners stand to loose.} Riparian owners are rarely pleased when they realise that the expropriating party is part of the same conglomerate as the grid company that estimates the grid connection costs associated with owner-led development.\footnote{...}

%\noo{ The water authorities themselves have recognised that access rights soon become illusory if it is too easy for the grid companies to deny access based on efficiency considerations.\footnote{See \cite{otprp62}.} At the same time, they point to the need for responsible management of the national grid, which, as they see it, requires delegation of authority to the grid companies. Hence, the authorities are left with a dilemma. So far, they have responded to this mainly by issuing more regulation, not by attempting to reduce the level of power-concentration in the electricity sector.}

%Hence, it seems that these new rules only push the question of fairness and accountabil%ity further into the details of the decision-making process, without addressing the underlying problem of power concentration. %Unsurprisingly, controversies continue to arise, particularly when owner-led and small-scale projects remain unfulfilled due to grid constraints.

%Functional division was only one of two core intentions behind the liberalization of the 1990s. The other aim, which has been fulfilled to a greater degree, was for centralization. Indeed, since 1990 there has been an unprecedented consolidation of power compared to the earlier days when municipality-owned companies dominated the sector. While this was an aim of the reform, intended to bring about greater efficiency, enhanced expertise and increased competitiveness, it has later been criticized. It has been pointed out, in particular, that the practical consequence of liberalization has been that the local accountability of the electricity sector has been lost, both organizationally and politically.\footnote{See \cite{agnell11}. 

Meanwhile, the market-orientation of the electricity sector has reduced the level of political control and accountability. Today, a management model based on economic rationality and expert-rule has become dominant. According to Brekke and Sataøen, this serves to set the reform that took place in Norway apart from similar energy reforms in Sweden and the UK.\footnote{See \cite{brekke12}.} Moreover, Brekke and Sataøen argue that this has resulted in a lack of legitimacy that has been a significant contributory cause of recent national-scale controversies, particularly with regards to the development of the national grid.\footnote{The most serious case so far is that of {\it Sima - Samnanger}, concerning a new distribution line that will cut through the area known as {\it Hardanger}, a scenic part of south-western Norway. The plans met with significant resistance at both the national and the local level, but the government pushed ahead, leading to confrontations that also involved some acts of civil disobedience. See \cite[22-23]{brekke12}.} %It would destroy a valuable part of Norwegian nature, they argued, without providing the people living there with anything in return.\footcite[26-27]{brekke12} Despite significant opposition to the plans voiced both at the national and the local level, the government pushed ahead, resulting in a serious confrontation with local and environmental interests, involving acts of civil disobedience. Arguably, the case demonstrates how the increasingly centralised, sector-based, management framework results in a lack of democratic legitimacy.\footcite[27]{brekke12}

At the same time, the growth of the small-scale hydropower sector gives local communities a new voice, as market participants, thereby acting as a counterweight to centralisation and expert-rule. Since the mid- to late 1990s, the small-scale sector has grown significantly. It has been estimated that about one third of the remaining potential for hydropower in Norway, measured in annual energy output, will come from small-scale projects.\footnote{See \cite[231]{nou129}.}

Many established energy companies have entered into the small-scale market, but they are facing serious competition from new actors, several of which are  owner-controlled and locally based. This development has been a counterweight to the increasingly centralised ownership pattern in the hydroelectric sector. In many ways, owner-led and owner-cooperating companies have replaced the municipality companies as the local anchors of Norwegian hydropower.

In a recent report, the potential for profitable small-scale hydropower projects was estimated to be around 20 TWh/year.\footnote{See \cite{aanesland09}. For comparison, suggesting the scale of this potential, I mention that the total consumption of electricity in Norway in 2011 amounted to 114 TWh, see \url{http://www.ssb.no/en/energi-og-industri/statistikker/elektrisitetaar}.} On this basis, the authors of the report estimate that the total present-day value of all waterfalls suitable for small-scale hydropower is about 35 billion Norwegian kroner, i.e., about 3.5 billion pounds.\footnote{See \cite[1]{aanesland09}.} This calculation is based on a model where the waterfalls are exploited in cooperation with an external commercial company, inspired by existing agreements between owners and the limited company {\it Småkraft AS}. Hence, the calculation might be an underestimate of what small-scale hydropower could represent for local communities if they remain in charge of development themselves.

Small-scale hydropower has become socially and political significant in Norway. In the report mentioned above, it is estimated that the value of rivers and waterfalls amount to just under 50 \% of the total equity in Norwegian agriculture.\footcite[1]{aanesland09} Moreover, hydropower is increasingly seen as a possibility for declining regions to counter depopulation and poverty. In some communities, small-scale hydropower is the only growth industry. For these communities, pursuing hydropower development at the local level also provides a way to regain some autonomy with respect to how local natural resources should be managed. Hence, small-scale hydropower takes on great political and social importance, not just for the owners of waterfalls, but for the community as a whole.

For an example of a community where small-scale hydropower has played such a role, I point to Gloppen, a municipality in the county of Sogn og Fjordane, in the western part of Norway. 19 hydropower schemes have already been carried out, all except one by local owners, amounting to a total production of over 250 GWh/year. This prompted the mayor to comment that ``small scale hydro-power is in our blood''.\footnote{See \cite{starheim12}.} When interviewed, he also directed attention at the fact that hydropower had many positive ripple effects, since it significantly increased local investment in other industries, particularly agriculture, which had been severely on the decline.

To achieve such effects, it is important to organise development in an appropriate manner. Moreover, to explain how waterfalls came to be as valuable as they are today, it is crucial to direct attention to the way in which waterfall owners initially asserted themselves on the market. In the following, I do this by giving an in-depth presentation of an early model for local involvement in hydropower development, presented at a seminar in 1996.\footnote{See \cite{dyrkolbotn96}.} This model contains an early expression of several ideas that would prove influential to the development of the small-scale hydropower sector.

However, certain other aspects of the model have not been widely adopted. These are aspects that pertain to the balance of power between owners and developers, as well as the relationship that should be established with larger communities of non-owners, including environmental groups and other water users. Hence, considering the model in some depth, and assessing its impact, will allow me to shed light on desirable social functions of waterfall ownership, and the extent to which such functions are fulfilled on the market today.

%In the following I will discuss this part of the sector in more depth, by tracking the development of the current practices that characterize their operations on the market. 

%In this section, I will present the typical steps involved in obtaining a license to develop hydropower. As I have already discussed, different rules apply to different kinds of projects. Here I will focus on the common elements, particularly those that relate to the role of local owners and communities. In this respect, the various procedures do not differ significantly, although the more extensive application process required for large schemes will obviously tend to result in greater attention directed at the plans, something that  might in turn result in greater local involvement in the decision-making process.\footnote{The most important procedural distinction is made between those applications that require an impact assessment (IA) and those that do not. The current rule is that projects that will produce more than 40 GWh/year require an impact assessment, while projects that will produce more than 30 GWh/year might require an impact assessment, as determined by the NVE. See {\it Directive Regarding Impact Assessment} of 26 June 2009, issued by the Ministry of the Environment pursuant to sections 4-2 and 14-6 of the \cite{pb08}.}

%Before the introduction of a centralized system of management of water resources, local farmers made extensive use of the power inherent in water. According to Terje Tvedt, there were 20 -- 30 000 watermills in Norway by the 1830s.\footnote{References} These were not used to produce electricity, but as grist mills, to create flour. This pattern of use rendered waterfalls of limited interest to outside investors or the state, so no significant pressures were placed on the local communities and their exercise of self-governance. On the other hand, Tvedt also stresses the importance of water, including its political implications, and how they gave local communities a better basis on which to claim a right to self-governance also in other matters. Indeed, the control over water, emerging as a side-effect of the control over land, played an important role in empowering Norwegian farmers and their communities in the 19th century. 
%
%Towards the end of the 19th century, local self-governance of water came under increasing pressure from outside investors. This happened in response to increased interest in hydropower as a potential source of energy for industrial exploits. Indeed, more and more waterfalls were purchased as mere objects of speculation, often by foreigners. As a result, the Norwegian government felt compelled to introduce regulation to protect national interests. This led to the passing of the predecessors of the \cite{ica17} and \cite{wra17}, referred to in Norway as the ``panic acts'' of 1909. They set up the basic license requirement for purchase of waterfalls and development of large-scale hydropower involving regulation, while also creating a space for the state as a marker player, by introducing the rule of reversion. 
%
%Still, however, development of hydropower was mostly undertaken by private companies, either large industrial companies like Norsk Hydro AS, or else community-owned companies that wanted to supply electricity locally. It was not until after WW2 that the state assumed the role as the leading hydropower developer in Norway. This was also when the national electricity grid was established and put under direct state control. Previously, the grid had been operated by a mix of state and private actors, who had organized themselves in a joint umbrella organization. Now, however, hydropower development for electricity production was linked with management of the grid, by the establishment of municipal and state-controlled electricity providers, organized as public bodies rather than commercial companies. 
%
%This area saw the extensive use of expropriation to facilitate hydropower. In1940, an act was passed which provided the necessary authority to compulsorily acquire waterfalls for hydropower. This act explicitly stipulated that expropriation should only take place in favor of projects that would serve the electricity supply in the local area. Hence, at this time, the use of expropriation to further the development of hydropower was specifically linked to the idea that the electricity supply should be organized as a non-commercial public service.
%
%During this time, many local hydropower plants that had previously supplied electricity locally were shut down. In particular, they were not allowed to connect to the national grid. Moreover, the grid authorities would often explicitly require these plants to shut down as a precondition to allow the local community to connect to the national grid. The waterfall owners were left marginalized during this time, as they had no real opportunity to develop hydropower themselves, or sell their rights to anyone else than the local monopolist. 
%
%However, as I will discuss in more detail in Chapter \ref{chap:5}, a system was developed -- not through legislation, but by the appraisal courts -- to ensure that the owners would receive at least some compensation for their waterfalls. This was despite the fact that under the prevailing regulatory regime, waterfalls were practically worthless to local owners. Hence, the compensation measures adopted reflected the tradition for local control over waterfalls. If such measures had not been introduced, in particular, local owners would hardly be entitled to any compensation at all under ordinary Norwegian expropriation law.
%
%Following liberalization of the energy sector, the basic premise for this way of organizing hydropower development was lost. In particular, hydropower development was now to be regarded as a commercial enterprise, within a market-based system. This removed the conceptual legitimacy of a system that marginalized local owners. Moreover, the most important practical impediment to owner-led development was also removed, as non-discriminatory access to the grid was provided for in statute. 
%
%As a result, there has been a surge of interest in the exploitation of waterfalls in owner-led hydropower schemes. In addition, the established energy companies have, to some extent, embraced this new situation by competing also in attempting to strike deals with the local owners. In this way, they recognize the right of self-governance and owner-management of hydropower, thereby helping to further undermine the rationale behind using expropriation. 
%
%But not all companies have responded in this way. Many of the largest and most influential companies ahve refused to alter their practices, and expect that they will still be able to enjoy the use of expropriation. These actors, which are often partly state-owned, refuse to recognize the right of waterfall owners to assume a decisive role in the mangement of their resource. Moreover, the are unwilling to enter the market for waterfalls in competition with actors that wish to collaborate with local owners. This has created considerable tension in recent years. 

%As we have mentioned, the typical owners of Norwegian waterfalls are communities of farmers and smallholders. Historically, the right to land-based resources, especially in the mountainous areas of the west and the north, where most valuable waterfalls are located, was held by local people, the same people who made use of it on a day to day basis. The main reason for this, which by European standards stands out as quite unusual, was that Norway never really had a separate class of landed nobility. Consequently, the Norwegian farmer occupied a position of relative autonomy and freedom, even to the point of exercising significant political influence, especially in the early days of Norwegian democracy.\footnote{During the 19th century the two dominant group in Norwegian politics were the farmers and the civil servants, and the former group exercised great influence in the Norwegian parliament, with the 1833 election leading to what became known as the farmers parliament. The "classic" academic treatment of farmers' influence over 19th century Norwegian politics is \cite{Koht} (in Norwegian). More on the author and a summary of his views can be found here, http://en.wikipedia.org/wiki/Halvdan\_Koht.} Following industrialization, however, their role became much more marginal, and farming has steadily become more and more unprofitable, with many farming communities having already disappeared, and many others threatened by depopulation. In light of this, the possibility of undertaking small scale hydro-power is often seen as being important to the survival of rural communities themselves, not just as a means for individual members of such communities to make a profit.

\section{{\it Nordhordlandsmodellen}}

In five brief points, the {\it Nordhordlandsmodellen} sets out a framework for cooperation between waterfall owners, professional hydroelectricity companies, local communities, and greater society.\footnote{See \cite{dyrkolbotn96}. The model was presented at a seminar in 1996, as the result of a collaboration between Otto Dyrkolbotn, a farmer and a lawyer, and Arne Steen, the director of {\it Nordhordland Kraftlag}, a municipality-owned energy company.} %The first three of these points would later provide the blueprint for many commercial actors hoping to cooperate with local owner. The last two points, which seek to address environmental concerns and the interests of the community more broadly, have largely been ignored.
The first point makes clear that the aim of cooperation should be to ensure local ownership and control: external interests should never be allowed to hold more than 50 \% of the shares in the development company. If the company is organised as a limited liability enterprise, then the plan stipulates that local residents -- not necessarily owners -- are to be given a right of preemption in the event that shares come up for sale. The possibility of organising the development company as a local cooperative is also mentioned.\footnote{References needed.}

The second point of the model sets out a method for valuing the riparian rights prior to development. It stipulates that the appraisement should reflect the real value of such rights, normally estimated on the basis of lease capitalisation. More concretely, the valuation should be based on the premise that the riparian owners will be entitled to rent based on the level of annual production in the planned hydropower project. Then, for the purpose of appraisal, the expected rent per annum is capitalised to find the present value of the riparian rights, relative to the development project in question.\footnote{This approach stands in stark contrast to the earlier valuation method used in the electricity sector, which relied on a purely theoretical assessment based on the aforementioned notion of a natural horsepower. See \cite{dyrkolbotn15,hellandsfoss97}.}

After such a value has been calculated, the model stipulates that owners are to be given a choice of either leasing out their water rights to receive rent, or to use the capitalised value of (part of) this rent as equity to acquire shares in the development company. The third point in the model then offers a clarification, by stating that the development company should not in any event acquire ownership of riparian rights, but only a time-limited right of use. After 25-35 years, this usufruct should fall away and the waterfall should revert back to the owner of the surrounding land, free of charge. This is the proposed rule even in cases when the landowners themselves initially control the majority of the shares in the development company. Hence, the rule places a limit on alienation; no separation of water rights from land rights is allowed to last for more than 35 years. The model demonstrates the commercial viability of this organisational model by pointing to a concrete municipality-owned energy company that has stated its willingness to cooperate with owners on such terms, to help with financing and share the risk.\footnote{The company in question is Nordhordland Kraftlag, where one of the authors of the model, Arne Steen, was a director.} 

Following up on this organisational blueprint, the fourth and fifth points of the model describe the intended role of the local development company in society, by stressing the relationship between hydropower and other interests and potential uses of the affected river. Importantly, the model stipulates that potential developers should be willing to take on formal obligations towards other user groups. Moreover, obligations should not only be negatively defined, as duties to minimise or avoid harms. Positive obligations should also be introduced, such as duties to improve other qualities of the river system, and to engage in active cooperation with other users. %The importance of environmental interests is emphasised especially, and it is made clear that such interests should be taken into regard even in circumstances when this is not prescribed by the authorities. Moreover, fishing and tourism are mentioned as concrete examples of other water uses that the hydropower company should actively seek to promote.

The overall aim, it is made clear, is to ensure sustainable management of the river system as a whole. Interestingly, the model predicts that active local ownership along these lines is likely to make a structural contribution to sustainability that exceeds what can be achieved through governmental regulation alone. This claim is illustrated by a concrete example of a case in which the local owners decided to pursue a scheme that was less environmentally invasive than the project endorsed by the water authorities.\footnote{Today, this project has become Svartdalen Kraftverk, finalised in 2006. It produces 30 GWh annually, enough electricity for about 1500 households, see \url{http://no.wikipedia.org/wiki/Svartdalen_kraftverk}.}

The model goes on to emphasise the need for integrated processes of resource planning and decision-making, to ensure that hydropower development is not approached as an isolated economic and environmental concern, but looked at in a broader social and political context. To achieve this, it is argued that local communities need to play an important role in the management of water resources. Another concrete example follows, regarding {\it Romarheimsvassdraget}, a river system in the municipality of Lindås, in the county of Hordaland.

This river system was originally intended for large-scale development undertaken by BKK AS, without the participation of local owners.\footnote{BKK AS is one of the 15 biggest hydropower companies in Norway, and would later also purchase Nordhordland Kraftlag.} The project would involve a total of three river systems, such that the water from {\it Romarheimselva} and another river would be diverted to a neighbouring municipality for hydropower development there. The local owners argued against these plans by proposing a number of smaller development schemes. Eventually, they were successful, as the NVE agreed to endorse an alternative consisting of 7 distinct run-of-river projects undertaken in cooperation with local owners.\footnote{See {\it Vassragsrapport nr. 25}, Direktoratet for Naturforvaltning, 1999.}

It is important to note that when {\it Nordhordlandsmodellen} was formulated, owner-led development of hydropower was still a recent phenomenon, driven forward by individual owners and local groups that saw the potential and had enough know-how to get organised. Later, commercial companies emerged that specialised in cooperating with local owners.\footnote{For a good survey of the later developments, I point to \cite{larsen06,larsen08,larsen12}.} This has made it relatively easy to initiate a process of owner-led development. Moreover, owners that are not themselves aware of the potential inherent in their riparian rights may be approached by interested commercial actors. These actors will then tend to compete for the chance of striking a deal with the owners. Most of them rely on cooperation on terms that reflect the main ideas expressed in the first three points of {\it Nordhordlandsmodellen}.

However, several adjustments have become standard, and these systematically benefit the external partner: the requirement that locals should at all times control a majority of the shares is dropped, the period of usufruct is typically longer than 35 years, the reversion to the landowners after this time is made conditional on payment for machines and installations, and no preemption rights are granted to local residents. However, the core idea that riparian rights are to be valued based on a capitalisation of future rent is accepted. This means, in turn, that local owners rarely need to raise any additional capital to acquire shares in the development company. Moreover, the rent itself can become a significant source of income.

There are two main approaches to calculating this rent. The first approach, introduced already in {\it Nordhordlandsmodellen}, specifies the rent as a percentage of the gross income from sale of electricity, today often around 10-20 \%.\footnote{Source: contracts presented to the court in \cite{sauda09} (available from the author upon request). See also \cite[55-57]{hauge15}.} In this way, passive owners need not take on any risk related to the performance of the hydropower company. The second approach has been developed by the company Småkraft AS, which is now the leading market actor specialising in cooperation with local owners.\footnote{It is owned by several large-scale actors on the energy market, see \url{www.smaakraft.no}.} According to their model, riparian owners are paid a share of the annual {\it surplus} from hydropower generation.\footnote{See \cite[57-60]{hauge15} (also discussing variants of this contractual idea, based on how the surplus is actually defined in the contract).}

This share is usually higher than the rent payable based on the net income; often, the owners are entitled to $50 \%$ of the profit.\footnote{Source: contracts presented to the court in \cite{sauda09} (available from the author upon request). See also \cite[58]{hauge15}.} Hence, if the project is a success, the riparian owners might be better compensated. However, the owners have to accept some risks as though they were shareholders, and they do so even though they might not have much of a say in how the company is run.\footnote{To limit the risk for owners, companies such as Småkraft AS also operates a system of ``guaranteed'' rent, but this rent is usually quite a lot less than what the owners could expect from an agreement based solely on rent based on gross income. Source: contracts presented to the court in \cite{sauda09} (available from the author upon request).}

To illustrate the financial scale of the rent agreements that have now become standard, let us consider a typical small-scale hydropower plant that produces 10 GWh annually. With an electricity price of NOK 0.3 per KWh, this gives the hydropower plant an annual gross income of NOK 3 million. If the rent payable is 20 \%, the waterfall owners will receive NOK 600 000 annually, approximately GBP 60 000. By contrast, if the rights were expropriated, the traditional method of calculating compensation would be unlikely to result in more than NOK 600 000 as a {\it one-time payment} for a waterfall that yields 10 GWh per annum.\footnote{For further details on the compensation issue, see \cite{dyrkolbotn14,dyrkolbotn15,dyrkolbotn15a}. Sometimes, the difference in valuation would be even greater, since the natural horsepower of a development project is highly sensitive to the level of regulation of the waterfall, much more so than the value of the development. For an demonstration of how this affected compensation according to the natural horsepower method, one may consider the case \cite{hellandsfoss97}, which went to the Supreme Court. Here the owners were paid just over NOK 1 million for a waterfall that would yield 100 GWh per annum.}

Hence, the financial consequences of the ideas expressed in {\it Norhordlandsmodellen} have been dramatic. At the same time, it is clear that the latter two points of the model, addressing the importance of responsible and inclusive management of river systems, have not had the same degree of influence on the market. In the next section, I address this in more depth and comment on some recent developments that threaten to undermine the status of small-scale development as a sustainable alternative to large-scale exploitation. I argue, in particular, that the future of hydropower will likely leave local owners and their communities marginalised once again, unless a social function approach to small-scale development is adopted and entrenched in the law.

\section{The Future of Hydropower}\label{sec:future}

In recent years, there has been a growing tension between the small-scale hydropower sector and environmental groups. There is talk of a brewing ``hydropower battle'', as environmentalists grow increasingly critical of what they regard as predatory practices.\footnote{See \cite{haltbrekken12}.}
Reports on small-scale producers who violate regulations help fuel the negative impression of the industry.\footnote{In 2010, the NVE conducted randomised inspections and announced that 4 out of 5 mini and micro plants operated in violation of regulations pertaining to the amount of water they may use at any given time. See \cite{ulovlig10}. In the largest newspaper in Norway, this was reported under the heading that four out of five small-scale plants break the law, see \cite{ulovlig10b}. This is misleading, since mini and micro plants are distinct from small-scale plants proper. Most importantly, the former kinds of plants do not usually require a sector-specific development license. Because of this, it also seems plausible that the reported violations might in large part be due to a lack of knowledge and professionalism, not predation. I remark that questions later emerged regarding the accuracy of the report itself. Apparently, one of the plants that was reported to have violated regulations did not even exist, see \cite{tvilsom10}.} At the same time, the price of electricity has been much lower in recent years than what had previously been forecast, causing severe financial difficulty for many small-scale developers.\footnote{See \cite{sunde14}.} This has also revealed that some of the investors on the market have engaged in speculative practices, by aggressively entering into agreements with local owners, without carrying out much hydropower development.\footnote{See \cite{endresen14}.}

On the regulatory side, the water authorities have now adopted much stricter procedures to assess licenses for small-scale hydropower.\footnote{See \cite{lie12}.} In addition, different planning routines have been adopted to ensure that small-scale schemes are no longer considered individually, but in so-called ``packages'', collecting together applications from the same area. As a consequence of these changes, the number of rejected small-scale applications have increased dramatically in recent years.\footnote{In 2013, the number of rejections tripled compared to previous years, while the number of accepted applications remained stable. See \cite{sunde14b}.}

At the same time, powerful market actors who favour a traditional mode of exploitation have seized the opportunity to lobby more aggressively against small-scale hydropower, in favour of large-scale projects.\footnote{See, e.g., \cite{alexandersen14}.} Such projects, they argue, are preferable also from an environmental point of view. In recent years, this point of view has proven influential in many quarters, particularly among state agencies, such as the NVE and the Norwegian Environmental Agency.\footnote{See \cite{nilsen11}.} It has also been claimed that this perspective is backed up by research done on environmental effects of small-scale and large-scale projects.\footnote{See generally \cite{bakken12,bakken14}.} 

The core argument at work here has a very simply structure: small-scale plants indirectly affect a greater total area of land per energy unit produced, therefore they are considered more environmentally intrusive than large-scale schemes.\footcite[96-99]{bakken14} The premise of this argument is no doubt correct, since small-scale development is a decentralised approach to hydropower. In particular, several small-scale plants, at many different locations, are required to match the energy produced by a single larger plant. However, this quantitative observation has no bearing on the issue of how small-scale plants qualitatively effect the surrounding environment, compared to large-scale projects. Hence, the conclusion that large-scale projects are more environmentally friendly, appears to be  unsubstantiated.

Plainly, the research done on this so far has provided little or no information to shed light on the qualitative aspect of the debate. In particular, the parameters used to compare small-scale and large-scale developments are defined in terms of generic buffer zones that do not take into account differences in the severity of different kinds of environmental intrusions. For instance, as long as both installations are observable by passers by, a hut with a turbine inside is considered to have the same ``scenic impact'' as a concrete dam that distorts the water level in a lake by several meters. See \footnote{See \cite[95]{bakken14}.}

\noo{The only buffer zone that is not defined in this way is the {\it scenic} buffer, the area from which some installation can be seen. Here the model takes into account that a large installation should be assessed using a larger buffer zone than a small one, since the former is visible over a greater area. But even for this parameter, no distinction is made based on the actual visual impression; a large dam that dries up a river and makes it possible to regulate the water level in a lake by several meters counts the same as a small cabin with a generator inside, as long as both can be seen.\footnote{See \cite[95]{bakken14}.} For the other parameters, the data analysis is even more dubious, since the buffers are set uniformly according to general rules of thumb.\footnote{See \cite[95]{bakken14}.} For instance, a conflict with a threatened species is assumed to arise whenever a technical installation occurs within a certain distance from its natural habitat.\footnote{See \cite[95]{bakken14}.} Importantly, nothing is said about the severity of conflict, and no distinction is made between a minor installation and a massive disturbance.}

Despite the lack of a qualitative argument in favour of large-scale hydropower, the idea that this kind of development is better for the environment now appears to be gaining ground in Norway. This represents a complete reversal compared to the political narrative that has dominated for the last 15-20 years. Indeed, the merits of small-scale development was strongly emphasised by political leaders around the turn of the century. In his New Year's speech 01 January 2001, the Prime Minister went as far as to declare that the age of large-scale development was over.\footnote{See, e.g., \cite[34]{haltbrekken12}.} The same phrase was then repeated in the policy platforms of two successive national governments, in 2005 and 2009 respectively.\footnote{See the ``Soria Moria'' declaration from 2005, p 57, and ``Soria Moria II'', from 2009, p 52 (available at \url{www.regjeringen.no}).} 

However, as administrative practices and case law on hydropower shows, the end of large-scale exploitation has proved impossible to implement. Despite being official policy at the highest level of government for almost 15 years, large-scale development interests continue to dominate in the hydropower sector.\footnote{I believe the material presented in this thesis warrants making this claim. Moreover, it is underscored by the two recent Supreme Court decisions in \cite{jorpeland11} and \cite{otra13}.} Interestingly, the leading national politicians are now changing their position as well.\footnote{See \cite{liemin14} (reporting on recent public statements made by the Minister in support of large-scale development).} Arguably, this demonstrates how the politicians have yielded to pressure exerted by expert planners and commercial interests in this matter.

%Despite this, the legal position of owners and local communities has been weakened in recent years, to make room for large-scale development interests. The best example of this is the case of {\it Otra}, concerning a large-scale development project in the southern part of Norway.

%The developer of this project was granted permission to expropriate riparian rights, resulting in a legal conflict that went to the Supreme Court twice.\footnote{See \cite{otra10,otra13}.} The waterfall owners argued that they should be compensated for the loss of a small-scale development potential, but the developer disagreed. First, the owners were successful in their main claim, but the Supreme Court sent the case back to the appraisal court of appeal on a technicality.\footnote{See \cite{otra10}.} Then, the second time the case came before this court, it was decided that compensation for the lost small-scale potential should not be awarded. The reason given was that a small-scale scheme could not expect to obtain a development license, since large-scale schemes were preferred by the licensing authorities.\footnote{See \cite{otra13}. I note that the court's reasoning on this point is at odds with what is known as the ``no-scheme'' principle in the UK, see e.g., \cite{lawcom01}. This principle states, roughly, that compensation following expropriation should not reflect changes in value that are due to the expropriation scheme. A no-scheme principle is typically observed in Norway as well, but it is a general feature of Norwegian expropriation law that this principle is applied rather narrowly in case the expropriation scheme follows from public planning (in which case applying a no-scheme rule tends to result in higher compensation). See, e.g., \cite{stordrange07}.} The consequence was that the waterfalls were compensated based on the traditional method, resulting in a fraction of the compensation originally awarded. Only a few years earlier, leading hydropower lawyers had predicted that this method was a thing of the past.\footnote{See \cite{larsen12}.} After the decision in {\it Otra II}, there is reason to believe that the future will hold the opposite scenario in store; compensation for small-scale potentials will be a thing of the past, at least in all cases when expropriation takes place to benefit large-scale development. 

%Indeed, one of the first clear renunciations of the small-scale narrative came during the opening of the {\it Otra} plant, in 2014. The Minister himself presided over the festivities, and used the occasion to explicitly reject the previous political line, by publicly declaring that he was in favour of more large-scale development.\footnote{See \url{http://www.tu.no/kraft/2014/10/28/energiministeren-etterlyser-mer-regulerbar-vannkraft}.}

The political shift observed at present is likely to result in a further weakening of property and the rights of local communities. For example, it provides indirect political legitimacy to the NVE, who now pursue an explicit policy of prioritising applications for large-scale projects when these come into conflict with small-scale schemes in the same rivers.\footnote{See letter from the NVE of 21 March 2012 regarding new routines for the assessment of hydropower applications.} In many cases, the NVE will refuse to consider applications from owners as long as there are applications pending that might result in the expropriation of their property.\footnote{....}
%\footnote{The NVE have tended to apply such a priority rule for many years, but the legality of such an approach has been somewhat unclear (it depends on whether or not giving priority to takers is necessary to facilitate public planning, see \cite[21]{wra00}). However, after the decision in \cite{jorpeland11}, where a waterfall taking was unsuccessfully challenged on procedural grounds, the NVE seems less hesitant in general to adopt a ``strict'' line when dealing with recalcitrant owners.}

\noo{All in all, it seems that small-scale hydropower is currently loosing both commercial force and political credibility as a sustainable alternative for development. The underlying causes of this deserve more attention than I can devote to them in this thesis. It would be particularly interesting to conduct a further examination into the effects of lobbying and the relationship between commercial interests and bureaucratic elites.}

To conclude this section, I would like to emphasise a different aspect of the recent decline of the small-scale sector. Specifically, it seems that the small-scale sector itself needs to be challenged with its seeming failure to comprehensively address social and environmental concerns. Relying on the observation that large-scale development appears to be even worse, in both respects, is hardly adequate. Indeed, it seems plausible to hypothesise that part of the reason why the small-scale industry has been so easily undermined can be explained by the fact that the industry has failed to broadly mobilise property owners and local communities in decision-making processes.

For instance, the small-scale industry has on occasion actively sought to undermine property rights, possibly in an effort to mimic the successes of their large-scale competitors. The industry has argued, in particular, that expropriation should be made more easily available as a tool for small-scale developers and owners who wish to take property from reluctant neighbours.\footnote{See \cite{brekken07,brekken08}. The articles are written by a leading Norwegian energy lawyer, apparently in his capacity as legal representative of ``Småkraftforeningen'', an interest organisation for small-scale hydropower (the articles are published in the newsletter of this organisation).} The argument rests on a peculiar form of anti-discrimination reasoning; as long as large-scale developers are allowed to take property by force, small-scale developers should be allowed to do the same. In a world where takings are endemic, this might make some sense. However, it is hardly an attitude that helps the small-scale industry preserve its image as the more sustainable hydropower option.

%At the same time, one should not disregard the fact that many power companies appae who lobby may feel threatened by the surge of interest in small-scale hydropower. As I will discuss in the next chapter, recent case law on expropriation shows how the largest actors are working very hard to regain control over the market. Recently, they have been successful in court, and this too might be a reason behind the shift in political climate. Nevertheless, it strikes me as appropriate to direct a critical eye towards the small-scale industry itself. It seems, in particular, that the objective of profit-maximising has taken center stage to an extent that might harm the sector. %According to Norhordlandsmodellen, the environmentalists were natural partners of the small-scale hydropower movement. But now, they are fast becoming its sworn enemies.\footnote{See \cite{haltebrekken12}.} I think this aspect, in particular, is dangerous to the future of the industry, as it previously benefited greatly precisely because of its image as a more environmentally friendly alternative.

These critical remarks should not detract from the fact that the growth in small-scale hydropower has led to dramatically increased benefit sharing with many local owners of rivers and waterfalls. However, recent events indicate that it is inappropriate to look at this development in isolation from other concerns. When assessing the future of small-scale hydropower and local property rights to waterfalls, it seems important to also take into account the broader societal consequences of new commercial practices. If one fails in this regard, the pernicious image of owners as socially passive ``profit-maximisers'' gains a firmer hold both on the political and legal narrative. The negative consequences of this for property as an institution is already apparent in Norway, as I will discuss in the next chapter.

The call for a broader understanding of the role of small-scale hydropower and owner-led development echoes the theoretical discussion presented in Chapter \ref{chap:1}. There I argued that an entitlements-based perspective on property rights fails to do justice to the issues that arise in the context of economic development. In relation to hydropower development, this insight is strongly implicit in {\it Nordhordlandsmodellen}. However, in the current debating climate in Norway, it seems to be at risk of disappearing from view.

To counter this, I believe the social function view of property must be developed further, so that concrete policy recommendations can be formulated on its basis. The aim, I believe, should be to arrive at frameworks for participatory decision-making regarding hydropower that allows local owners and communities to contribute constructively when society desires commercial development based on  their water rights.

I return to this issue in Chapter \ref{chap:6}, where I argue that the Norwegian institution of land consolidation can be used to achieve this. First, I will zoom in on the issue of expropriation, where the mechanisms identified in this section often lead to concrete legal disputes. This will bring into focus important issues surrounding the status of economic development takings under Norwegian law.

\noo{ \section{Conclusion}\label{sec:conc3}

In this Chapter, I introduced my case study and provided background information that places it in a broader context with respect to Norwegian law. I presented the legal and regulatory framework surrounding hydropower development, while also tracing its history back to pre-industrial times. I noted that local rights to hydropower has a long tradition in Norway. However, I also observed that after the advent of the industrial age, and particularly following the Second World War, the state took the view that hydropower was a public good that should be exploited for industrial development in the public interest.

The tension that followed now permeates the law on hydropower, particularly following the liberalising reform of the early 1990s. This reform reorganised hydropower development as a commercial pursuit. At the same time, local owners were empowered by the reform, as they were now able to engage in commercial hydropower development themselves. This was made possible by the fact that a market for electricity was set up, founded on the idea that all actors should have access to the electricity grid on non-discriminatory terms. 

I discussed the resulting system in some depth, addressing also the question of whether or not the market functions as intended. I noted that the energy reform led to increased concentration of power in the electricity sector, where commercial companies partly owned by the state now wield more power than before. This, I argued, threatens to undermine the intentions behind the reform. I also looked at the extent to which the regulatory framework is able to accommodate new actors and true competition on non-discriminatory terms. I focused particularly on the status of locally led projects as well as the companies that specialise in cooperating with owners. I also discussed controversies that have resulted, particularly relating to the perceived discrimination of smaller actors on the market.

Then I went on to present a prototype for the model by which the smaller actors now tend to organise themselves. I observed that they too appear to have adopted a strongly commercial outlook on the meaning of local hydropower development. I discussed how this departs from earlier ideas, which were based on seeing local development as an expression of local democracy and local management of resources. This earlier vision actively sought to ensure sustainability and incorporate other water interests in the decision-making, a perspective that now seems to be largely missing.

I concluded by arguing that this might be a contributing reason why small-scale development is now falling out of favour. Today, critical voices claim that large-scale development is better, not only because it is more commercially optimal, but also because it is more environmentally friendly. Moreover, issues relating to ownership, control, benefit sharing and local participation, appear only at the fringes, both of the current debate and the current regulatory framework.

This state of affairs, I think, foreshadows many of the issues that will be brought into focus in the next chapter. There, I will look specifically at expropriation of waterfalls, by tracking the position of owners under the current regime. I will argue that the law as it stands is based on a perspective that blocks out both the significant commercial interests of the taker, as well as the significant social functions and obligations of the original owners. The issue of expropriation, in particular, will invariably raise questions that seem difficult to address without adopting a broader view, which also takes into account the owners' communities and their role within it. }