\chapter{Norwegian waterfalls and hydropower}\label{chap:3}

\section{Introduction}\label{intro}

Norway is country of mountains, fjords and rivers, and about 95 \% of the annual domestic electricity supply comes from hydro-power.\footnote{See Statistics Norway, data from the year 2011, http://www.ssb.no/en/elektrisitetaar/.} The right to harness rivers for hydro-power is held by local landowners, but historically, this right has not been of much use to them, since the Norwegian electricity sector has been organized as a regulated monopoly, with most hydro-power schemes carried out by non-commercial companies controlled by the State, or local governmental bodies. In the early 1990's, however, the sector was liberalized, and it has become increasingly common for local landowners to undertake their own hydro-power projects. This has led to increased tension between local interests and established energy companies. Following liberalization, these companies are now organized for profit, and this raises the question: who is entitled to benefit from Norwegian hydro-power? 

Increasingly, it is becoming clear that this is not merely a question of the opposing commercial interests of individuals and companies, but also crucially involves the local communities directly affected by development. The original owners of hydro-power tend to be farmers residing in the local communities where the resources are found, and therefore, the question of who should benefit also encompasses the question of who should be allowed to participate in decision-making process, and what degree of autonomy local communities are to be granted in this regard. Are local owners and their communities entitled to a say in how the hydro-resource is to be exploited, or must they accept to remain passive, as they were rendered by the monopoly which used to be in place?

In this chapter, we address some recent demands made by local people, to the effect that they should be allowed to partake more actively in decision-making processes regarding local waterfalls. We focus on the legal status of such demands under Norwegian law, and we do so by considering the recent Supreme Court Case of \emph{Ola Måland and others v. Jørpeland Kraft AS}.\footnote{Ola M{\aa}land and others v. J{\o}rpeland Kraft AS, Rt 2011 s. 1393. I mention that I represented the local owners in this case, as a trainee lawyer in the district and regional courts, and as the responsible lawyer before the Supreme Court.} In this case, the local owners protested the legality of a license that granted the developer, Jørpeland Kraft AS, a right to divert water away from their waterfalls, thereby reducing the potential for local hydro-power. The owners argued that consultation had been insufficient, and they also contended that the assessment of the case made by the water authorities had been inadequate, and that the decision has been based on an erroneous account of the facts. They won the case in the district Court, Stavanger Tingrett, but lost under appeal to the regional Court, Gulating Lagmannsrett. The Norwegian Supreme Court also found in favor of the developer, and the argument they gave to support this conclusion goes far in suggesting that while the commercial interests of local owners need to be compensated, the presence of such interests do not entitle local communities to a greater say in decision-making processes. Importantly, the Court held that the presence of local interests does not necessitate the adoption of different administrative practices, and that the procedures developed during the decades of direct, non-commercial, administration of the energy sector, could still be followed.

The case is significant, since the traditional hydro-electric scheme in Norway typically involves expropriation, often interfering with the property rights of hundreds of local individuals. Traditionally, owners of waterfalls would be compensated according to a standardized mathematical method that was based on the assumption that they had no interest in hydro-power themselves.\footnote{The method consists in calculating the number of \emph{natural horsepowers} in the waterfall, and then multiplying this number by a price pr. natural horsepower, determined by the discretion of the Court, but in practice based almost solely on what has been awarded in previous cases. For a description of the traditional method, we point to \cite{falk} Chapter 7, page 521-522 (in Norwegian).} In a landmark case from 2008, however, the Norwegian Supreme Court commented, in an \emph{obiter dicta}, that the traditional method for calculating compensation for waterfalls was no longer appropriate, at least not in cases when it can be demonstrated that the original owners could have exploited the resource themselves, if the expropriation had not taken place.\footnote{The case of Agder Energi Produksjon AS vs. Magne Møllen, Rt. 2008 s. 81. The local owner lost the case, the reason being that the Supreme Court held that compensation should not be based on the present day value of the waterfall, but the value it had when the original transferral of rights took place, in the 1960's. The \emph{obiter dicta} has been used as an authority for subsequent Supreme Court decisions, however, see, for instance, Rt. 2010 s. 1056 and Rt. 2011 s. 1683. It has received quite a lot of scholarly attention as well, see \cite{Tf1,Tf2,Tf3}.} This decision has had a profound impact on the level of compensation awarded for waterfall rights, leading to payments that can will typically be ten to a hundred times higher than that which would have been awarded according to the traditional method.\footnote{So far, in cases that have come before the court, there has been about a twenty-fold increase in compensation, see \cite{Tf1}, but the new method will, when applied to certain kinds of projects (cheap to build, and involving little or not regulation of the water-flow), result in compensation having to be paid that amounts to at least a hundred times more than what could be expected if the traditional method had been applied.} More generally, it also served to shift the balance of power in favor of local owners and their communities, who increasingly expect to have their voices heard, and to get a more direct say over how their energy resources are managed and exploited.

After \emph{Måland}, however, it has become unclear to what extent the presence of local, commercial interests will continue to influence the Norwegian energy sector, and if we will see more active participation by local people in the future. In fact, recent statements made by the Norwegian water authorities seem to suggest that this is becoming increasingly unlikely, as a shift in policy seems to have taken place, whereby local, small scale projects, are now to be given lower priority than large scale projects undertaken by established energy companies.\footnote{These statements were not linked to \emph{Måland}, but were made in a more general context, ostensibly motivated by the desire to increase the efficiency of the administrative process, see http://www.nve.no/no/Konsesjoner/Vannkraft/Smaakraft/ where the new policy was announced. It also received some attention from the press, see, for instance, http://www.tu.no/energi/2012/01/18/nve-varsler-flere-smakraft-avslag.}

Taking a broader view on Norwegian law, we believe that recent experiences regarding hydro-power provides an interesting case to study, and one that will shed light on how property rights function in a social, economic and political context. It seems, in particular, that the view of property rights to waterfalls adopted by the Supreme Court rests on a narrow interpretation, seeing such rights merely as bestowing financial interests on certain individuals. This was clearly felt in \emph{Måland}, and we think the case also serves to illuminate certain consequences of such a view, suggesting, in particular, that it can have detrimental social and political consequences, and can very easily lead to perceived injustices. We also think it is pertinent to ask if the narrow view of property which seems to have been adopted for waterfalls in Norway is adequate with respect to human rights law, or if the right to property should also be considered a right to participate, and a right to be heard, in decision-making processes.

In the following, we first give the reader some further background on Norwegian hydro-power, and then 
we present \emph{Måland} in some detail, focusing on giving the reader an impression of current administrative practices, by describing how they played out in this particular case, and by detailing how they came to result in a decision that the original owners felt to be fundamentally unjust. We also address the legal arguments given by the opposing sides and the arguments relied upon by the national courts. We conclude by presenting some overreaching issues that we believe the case raises, regarding both the social context of property rights, the content of property as human right, and the question of whether or not the protection awarded under Norwegian law currently meets the standard set by the European Convention of Human Rights, as interpreted by the Court in Strasbourg. 

\section{Background: local owners making their voices heard by suggesting small scale hydro-power}\label{context}

As we mentioned in the Introduction, Norwegian law regards waterfalls as private property, and by default, a waterfall belongs to the owner of the land over which the water flows.\footnote{See Section 13 of Act No. 82 of 24 November 2000 relating to River Systems and Groundwater.} This does not mean that the landowner owns the water as such -- freely running water is not subject to ownership -- but it entitles the owner of the waterfall to harness the potential energy in the water over the stretch of riverbed belonging to him. This right can be partitioned off from any rights in the surrounding land, and large scale hydro-power schemes typically involve such a separation of water-rights from land-rights, giving the energy company the right to harness the energy, while the local landowner retains the rights in the surrounding land.

Norwegian rivers, and especially rivers suitable for hydro-power schemes, tend to run across grazing land owned jointly by farmers, so rights to waterfalls are typically held among several members of the local, rural community.\footnote{The land in question tend not to be enclosed, in particular, and in cases where there has been a land enclosure, water-rights have often explicitly been left out, such that they are still considered common rights, belonging to the community of local farmers.} They might not always be willing to give them up, especially not on the terms proposed by the developer, so the use of expropriation has played an important role in the history of Norwegian hydro-power. This has meant that the terms governing separation of water-rights from land-rights, including the level of compensation paid to landowners, and the influence they are granted in the decision-making process, has been determined by the law. Following legislation in the early 20th century, a regulatory system was put in place that centralized the management of Norwegian water resources. It clearly favored exploitation by the State or by companies owned by local governmental bodies, and the local landowners were severely marginalized. In most cases, they would have to accept the terms presented to them by the developer, or else argue the matter in Court, after the developer had already been granted a license to expropriate. Landowners were not in a good position to negotiate the terms of the development, and their property rights appeared increasingly nominal, the prevailing political attitude being that waterfalls formed part of the common heritage of the Norwegian people, and should be managed in their interest.\footnote{While some of the claims made here will be further qualified by what is to follow, the general picture we paint here is communicated also by the standard work on Norwegian water law \cite{falk}.}

This created a legal tension where, on the one hand, waterfalls were still considered private property under land law, yet, on the other hand, were considered as belonging to the public as far as large scale hydro-power development was concerned. The following two quotes, the first from the general water law, with roots going back at least to the 19th Century, and the second  from a law directed specifically at large scale hydro-power, illustrates this ambivalence.

{\begin{minipage}[t]{16em}
 \begin{aquote}{\tiny Section 13, Water Resources Act 2000} \footnotesize A river system belongs to the owner of the land it covers, unless otherwise dictated by special legal status. [...]

The owners on each side of a river system have equal rights in exploiting its hydro-power...
\end{aquote}  
\end{minipage}}
{\begin{minipage}[t]{22em}
\begin{aquote}{\tiny Section 1, Industrial Concession Act 1917 (amended 2008)} \footnotesize Norwegian water resources belong to the general public and are to be managed in their interest. This is to be ensured by public ownership...
\end{aquote}
\end{minipage}} \\

Following liberalization of the Norwegian energy sector in the early 1990's, this legal tension in statute has increasingly also become a tension in politics, where the interests of local communities and landowners stand in opposition to the interests of large energy companies, often owned by the State, seeking to harness locally owned resources for commercial gain. The question of how the Norwegian legal framework is actually applied in this regard is therefore a matter that has come under increased scrutiny. This was the question that went before the courts in the case of \emph{Måland}, where local owners protested the legality of expropriation on the grounds that they could harness the water in their own small scale hydro-power scheme. Before we delve into the details, we will elaborate a bit further on the context in which the law was called upon to function in this case. Importantly, the economic, social and political context of expropriation has changed rather dramatically in recent years, and we do not think it is possible to understand the case and the issues it raised except in the context of these changes. 

There are two developments that have been particularly important. First, there has been a general shift from viewing electricity production as a public service to viewing it as a commercial enterprise. This has made the legitimacy of expropriation appear more controversial, and the argument is often voiced that expropriation does not happen in the interest of the public at all, but \emph{solely} in order to benefit the commercial interests of particular companies.\footnote{This has been a recurring theme in articles appearing in "Småkraftnytt", the newsletter for "Småkraftforeninga", an interest organization for owners of small-scale hydro-power, which currently have 236 associated small scale hydro-power plants, see http://kraftverk.net/ (in Norwegian). In addition to the case of Måland, the question has also been brought before the (lower) national courts in some other cases, such as \emph{Sauda}, LG-2007-176723 (Gulating Lagmannsrett, regional high court), and \emph{Durmålskraft}, see http://www.ranablad.no/nyheter/article5583405.ece (decision from the district court, as reported in a Norwegian newspaper). In both cases, the outcome was generally more favorable to the expropriating party than the local owners, and the reasoning adopted by the courts appears similar to that of \emph{Måland}.}

In this way, expropriation of Norwegian waterfalls raises issues that have become increasingly important also in a global setting, and which seem to arise naturally in systems where economic activities are organized according to a mix of socialist and free market principles. In such systems, it seems practically inevitable that cases of expropriation -- undertaken to benefit the public -- will also often come to benefit developers that are motivated by purely commercial interests. While this in itself might not be problematic, it will easily lead to the concern that the commercial interests of powerful companies is the \emph{only} reason why expropriation is permitted, and that expropriation is being used as a commercial tool for powerful market forces, to the detriment of less powerful actors. That this can be highly controversial is illustrated in the US case of \emph{Kelo v. City of New London}, which divided the US Supreme Court and has also attracted great attention, both from legal scholars and in the general public, as a political issue.\footnote{\emph{Kelo v. City of New London}, 545 U.S. 469 (2005).}

For the case of Norwegian waterfalls, however, liberalization of the energy sector has also had a positive effect for local communities, in that it has served to make local owners more active. It has become increasingly common that they exploit their hydro-power resources themselves, often in small scale projects, and often in cooperation with companies that specialize in such development.\footnote{In 2012, the NVE granted 125 new licenses for small scale hydro-power, and at the end of the year they had 859 applications still under consideration. Source: report made by the NVE, available at http://www.nve.no/Global/Energi/Q412\_ny\_energi\_tillatelser\_og\_utbygging.pdf (in Norwegian). } This, of course, only adds to the controversy surrounding expropriation of waterfalls, especially when local owners are deprived of the opportunity for small scale development.

The most significant step towards liberalization of the Norwegian energy sector was made in 1990 when the Energy Act was passed, an important new piece of statute reorganizing the system for the distribution of electricity.\footnote{Act nr. 50 of 29 of June 1990 relating to the generation, conversion, transmission, trading, distribution and use of electricity.} The Energy Act introduced the principle that energy consumers and producers should have non-discriminatory access to the national electricity grid, thereby creating a market where any actor, privately owned or otherwise, could supply electricity to the grid, and profit commercially from hydro-power. In the same period of time, monopoly companies were reorganized, becoming commercial companies that were meant to compete against each other, and against new commercial actors that entered the market.\footnote{For a short English summary of how the system is administered, see for instance \cite[p.29-30]{ar2010}, and for more detail, we point to \cite{Hammer2}.}

The Norwegian State retained a significant stake as shareholders in energy companies, however, now often alongside private investors. Moreover, many rules in Norwegian law favor companies where a majority of the shares are held by the State, and to this day the largest and most influential Norwegian energy companies remain under public ownership.\footnote{The fact that publicly owned companies are favored in this way is often seen as a questionable practice with regards to competition law, see for instance the recent EFTA Court case, Case E-2/06, \emph{EFTA Surveillance Authority v. The Kingdom of Norway}, EFTA Court Report 2007, p.164. Here, the Court considered the old Norwegian rule of \emph{reversion}, whereby a license to undertake certain large scale hydro-power schemes (strictly speaking, a license to acquire the waterfalls needed to undertake it) came with a special clause that the private developer had to give up ownership to the State after a fixed period of time. This clause was held to be in breach of the EEA agreement since it only applied to private companies. We remark that the Norwegian government responded to this with an amendment after which reversion no longer applies, but which stated that a license to acquire waterfalls for the purpose of such large scale schemes can not be given at all to any company in which private parties own more than 1/3 of the shares.}

It seems, in particular, that the aim of liberalization in Norway has never been to minimize State control over hydro-power, but rather to give consumers greater freedom in choosing their energy-supplier, and to enhance efficiency in the sector by introducing competition.\footnote{See for instance \cite{liberal}, which offers a comparative study of the liberalization of the energy sectors in Norway and the UK.} Still, the fact that any developer of hydro-power is now legally entitled to connect to the national grid has proved important in giving actors that are not owned by the State a fighting chance on the Norwegian energy market. It has been especially important for local owners of waterfalls, since it means that if they undertake hydro-power projects themselves, they can no longer be refused access to the grid, but will be in a position to benefit commercially.

It should be noted that the Norwegian grid is operated by regional companies, responsible for the supply and distribution of electricity in their region. These will typically also be energy producers themselves, and historically, they would prevent other hydro-power initiatives by refusing them access to the grid. In fact, in the early days on Norwegian hydro-power, in the first half of the 20th century, there were quite a few locally owned and operated power plants, often providing local communities with electricity. When the national grid was established, most of them were closed down, often as a result of an explicit policy on part of the authorities. To increase the cost-effectiveness of the companies responsible for providing the national service, these companies were often allowed to demand, as a condition for allowing local communities access to the grid, that local hydro-power plants had to be shut down.\footnote{See \cite[p.111]{Hindrum} (in Norwegian).} 

Following legislation whereby access to the grid is provided for in statute, we have seen a surge of interest in the exploitation of waterfalls in small scale hydro-electric schemes, and these schemes are often initiated by local owners. As we have mentioned, the typical owners of Norwegian waterfalls are communities of farmers and smallholders. Historically, the right to land-based resources, especially in the mountainous areas of the west and the north, where most valuable waterfalls are located, was held by local people, the same people who made use of it on a day to day basis. The main reason for this, which by European standards stands out as quite unusual, was that Norway never really had a separate class of landed nobility. Consequently, the Norwegian farmer occupied a position of relative autonomy and freedom, even to the point of exercising significant political influence, especially in the early days of Norwegian democracy.\footnote{During the 19th century the two dominant group in Norwegian politics were the farmers and the civil servants, and the former group exercised great influence in the Norwegian parliament, with the 1833 election leading to what became known as the farmers parliament. The "classic" academic treatment of farmers' influence over 19th century Norwegian politics is \cite{Koht} (in Norwegian). More on the author and a summary of his views can be found here, http://en.wikipedia.org/wiki/Halvdan\_Koht.} Following industrialization, however, their role became much more marginal, and farming has steadily become more and more unprofitable, with many farming communities having already disappeared, and many others threatened by depopulation. In light of this, the possibility of undertaking small scale hydro-power is often seen as being important to the survival of rural communities themselves, not just as a means for individual members of such communities to make a profit.

As local owners started to harness their waterfalls themselves, commercial companies also emerged, specializing in cooperating with them. In a recent report, it was estimated that there is a potential for profitable small scale hydro-power of about 20 TWh/year \cite{Aanesland}, with a total value, before investment, of about 70 billion Norwegian kroner, i.e., about 8 billion pounds.\footnote{For comparison, suggesting the scale of this potential, we mention that the total consumption of electricity in Norway in 2011 amounted to 114 TWh, see http://www.ssb.no/en/energi-og-industri/statistikker/elektrisitetaar.}  This report was based on a particular model of cooperation with a commercial company, Småkraft AS, and might be an underestimate of what small scale hydro-power could represent for local communities if they take a more independent role in developing the resource. Thus, small scale development of hydro-power has become socially and political significant, and it is increasingly seen as a possibility for these regions to counter depopulation and poverty, while also regaining some of their autonomy and influence with respect to how the natural resources found locally are to be managed. In many cases, small-scale hydro-power appears to be the only growth industry, and takes on great political and social importance for the community as a whole, not just the owners of waterfalls.\footnote{For an example of a community where small scale hydro-power has played such a role, we can point to Gloppen, a municipality in the county of Sogn og Fjordane, in the western part of Norway. 19 schemes have already having been successfully carried out, all except one by local owners themselves, amounting to a total production of over 250 GWh/year. This prompted the mayor to comment that "small scale hydro-power is in our blood", see \cite{Gloppen}. When interviewed, he also directed attention at the fact that hydro-power had many positive ripple effects, since it significantly increased local investment in other industries, particularly agriculture, which had been severely on the decline.}

Summing up, we can conclude that expropriation of waterfalls has become more politically and social controversial, and we believe that the case of \emph{Måland}, to which we now turn, must be understood in this context. The case did not attract the same public attention as the cases relating to the revision of the traditional method for awarding compensation, and has, as far as we are aware, not previously received any scholarly attention either. It seems important, however, in that it clarifies the stance that Norwegian Courts take with respect to the question of the extent to which local owners and communities are entitled to take part in the decision-making processes concerning commercial development of the waterfalls they own. Moreover, while local interests have claimed a significant victory with respect to compensation, \emph{Måland} limits its impact since it suggest that there is still very limited legal protection of local owners' right to have their voices heard regarding how Norwegian power is to be managed.

In the following, we give a presentation of the case. We start by presenting the facts, and we do so going back to original sources, not merely looking to the brief presentation provided by the courts in their judgments, but to the preparatory documents assembled by the water authorities, taking special note of both the arguments presented by the expropriating party, and the objections raised by original owners and their representatives. Building on this, we the present the legal arguments that were raised by both sides, aiming to provide a more in depth presentation than the review given by the courts. We then present and compare the various arguments relied upon by the courts, and we offer our own analysis of how to understand the outcome of the case in the context of Norwegian law. We continue by addressing what the decision tells us about the Norwegian legal framework for hydro-power exploitation more broadly, and the questions it raises with respect to the social implications of expropriation of waterfalls, and with respect to human rights law.

\section{The facts of the case}\label{sum}

The case started 10 September 2004 with Jørpeland Kraft AS submitting an application to undertake a watercourse regulation, as provided for in the Watercourse Regulation Act of 1917, Section 8.\footnote{Act No. 17 of 14 December 1917 relating to Regulations of Watercourses.} As is customary, the application also included an application for a license to acquire waterfalls, as set out in the Industrial Concession Act, and a right to expropriate necessary rights from local owners, as provided for in the Water Resources Act, Section 51 and the Expropriation Act, Section 2 nr. 51.\footnote{Act No. 16 of 14 December 1917 relating to Acquisition of Waterfalls, Mines and other Real Property, Act No. 82 of 24 November 2000 relating to River Systems and Groundwater and Act No. 3 of 23 November 1959 relating to Expropriation.} In practice, it has not been common to consider such applications separately, but to consider the project as a whole, and to raise issues with respect to special provisions, and particular licenses, only in so far as they arise in connection with assessing the application for a development license, which is considered the main issue.

The management of water resources in Norway is centralized, and at the lowest level of authority we find the Norwegian Water Resources and Energy Directorate (NVE), which is a national body, based in Oslo. In some cases they have been delegated authority to grant development licenses themselves, but in most cases of large scale development, they only prepare the case, then hand it over to the Ministry of Petroleum and Energy which then gives its recommendation to the King in Council, who makes the final decision. The local municipalities, while generally quite powerful under Norwegian law, are completely sidelined when it comes to management of water resources, and their role is mostly limited to commenting on the plans, alongside other stakeholders.\footnote{Although there seems to be a good case to be made that they could exert greater influence over the process, based both on general planning law, which empowers them a great deal, or on special rules set out in agricultural law, which requires them to approve, on a case by case basis, any shifts in the property structure of agricultural land. In practice, however, they almost never exercise any of these powers with respect to water resources. If the developer has a license to undertake the scheme itself granted by the King, then it seems that municipalities most often take it to mean that they are obliged to follow suit, by granting the (relatively speaking) minor licenses that might be required with respect to general planning law and agricultural law.}

For the local owners of waterfalls, the situation is worse, since they are not identified as stakeholders in large scale projects. They are not, in particular, mentioned in the Watercourse Regulation Act, Section 6, which regulates the steps that must be taken when preparing such cases.\footnote{Nor do the seem to be mentioned  in any of the documents setting out how the authorities deal with such cases in practice. See, for instance, the guide published by NVE \cite{rettleiar} (in Norwegian), directed at applicants, and setting out how NVE deals with cases involving large scale hydro-power.} Consequently, it is hardly surprising that in administrative practice, it has been uncommon to devote particular attention to local owners. Rather, the focus has typically been on environmental issues and the opinions of various interest groups, such as hunter or fishermen's associations.\footnote{For a more in depth account of the process, we point to the standard legal reference on Norwegian water law \cite{falk}(in Norwegian).}

The applicant in Måland, Jørpeland Kraft AS, is a company jointly owned by Scana Steel Stavanger AS, who own 1/3 of the shares, and Lyse Kraft AS, who is the majority shareholder holding the remaining shares. The former is a steelworks company located in the small town of Jørpeland in Rogaland county, southwestern Norway. Historically, this company has been a major employer in Jørpeland, which is located by the sea, next to a mountainous area. The main source of energy for the steel industry in Norway has been hydro-power, and Scana Steel Stavanger AS is no exception. The company uses energy harnessed from the rivers in the area, and while the primary river runs through the town of Jørpeland itself, it is supplemented by water from other rivers in the area that are diverted so that they can be exploited more efficiently along with the water from the Jørpeland river.

Recently, Norwegian steel companies have become less profitable, due in great part to increased foreign competition and a significant increase in cost of operation associated with this type of industry in Norway, particularly salary costs.\footnote{For a reference on this, see \emph{Information Booklet about Norwegian Trade and Industry}, published by the Ministry of Trade and Industry in 2005.} This has led to many such companies shifting their attention away from labor-intensive steel production, and focusing instead on producing electricity, selling it directly on the national grid. Jørpeland Kraft AS was established as part of such a move being made with regards to the energy resources in Jørpeland, and the role played by Lyse Kraft AS is an important one. As we mentioned, Norwegian law favors companies where the majority of the shares are held by public bodies, and Lyse Kraft AS, being publicly owned, with the city of Stavanger as the main shareholder, is therefore a valuable partner. Moreover, Lyse Kraft AS, while being a commercial company, is also responsible for the electricity grid in the region. It was established as a merger between several local monopoly companies in the Stavanger region which were reorganized following liberalizaion of the sector in the early 1990's. As discussed in Section \ref{context}, there is little doubt that old monopolists still enjoy considerable power and influence.\footnote{In fact, Lyse Kraft AS is good example suggesting that their power might in some cases have \emph{increased}. Since liberalization, the restraints imposed both by the non-commercial nature of former monopolists, and the local, political, anchoring of such companies, have disappeared.} This is another reason why they can serve as valuable partners for private companies wishing to make a profit from Norwegian hydro-power.

With attention shifting from harnessing rivers for the purpose of industrial production to the purpose of producing electricity to sell on the national grid, the main variables that determines the profitability of the undertaking also changes. On the cost side, what matters becomes only the cost of producing the electricity itself, and this is typically determined, for the most part, by the investments required for the original construction works.\footnote{For an overview of the considerations made when assessing the commercial value of small scale hydro-power, we point to \cite{kartlegging}. In fact, due to the importance that small scale hydro-power has assumed in recent years, investigating models for investing in such projects has become an active field of research in Norway, see for instance \cite{investment}.} Running and maintaining a hydro-power station tends to be comparatively inexpensive. On the income side, what matters is the price of energy on the electricity market, a market that is no longer anchored in the local conditions of supply and demand.

Importantly, as long as energy production is the sole focus, the business no longer depends in any significant way on the local labor force, and as a result, it is typical that large scale exploitation becomes much more profitable, compared to the medium or small scale power plants typically needed to facilitate local industrial exploits. Hence, it was in keeping with a general trend in Norway when Jørpeland Kraft AS, following their shift in commercial strategy, proposed to undertake measures to increase their energy output. This could be achieved relatively cheaply, by further constructions aimed at channeling water from nearby waterfalls into dams that were already built to collect the water from the Jørpeland river.

One relatively small waterfall from which Jørpeland Kraft AS suggested to extract water was owned by Ola Måland and five other local farmers. This waterfall is not located in Jørpeland kommune, and does not reach the sea at Jørpeland, but runs through the neighboring municipality of Hjelmeland, on the other side of a mountain range, until it eventually reaches the sea at Tau, another neighboring municipality. The plans to divert this water would deprive original owners of water along some 15 km of riverbed, all the way from the mountains on the border between Hjelmeland and Jørpeland, to the sea at Tau. Far from all the water would be removed, but the water-flow would be greatly reduced in the upper part of the river known as "Sagåna", the rights to which is held jointly by Ola Måland and five other local farmers from Hjelmeland. 

The water in question stems from the \emph{Brokavatn}, located 646 meters above sea level, where altitude soon drops rapidly so that hydro-power is a particularly well-suited form of exploitation for this water. Plans were already in place for making such use of it, from about the altitude of Brokavatn, to the valley in which the original owners' farms are located, at about 80 meters above sea level. In fact, a rough estimate of the potential was originally made by the NVE, and estimated to yield gross annual production of 7.49 GWh pr. annum, about five times more than the water from Brokavatn would contribute to the project proposed by Jørpeland Kraft AS. This estimate was not made in relation to the case, but as part of a national project to survey the remaining energy potential in Norwegian rivers.\footnote{The survey was carried out in 2004, and its results are summarized in \cite{kartlegging}.} \noo{More recent calculations, made by several different experts, acting both on behalf of Jørpeland Kraft AS and original owners, suggests that the water which would be lost would in fact be crucial to the commercial potential of hydro-power for the original owners. Having the water available would take such a project from being somewhat marginal to being a highly profitable endeavor. The owners were not aware of this at the time when the case was being prepared by the water authorities, nor where they informed of this as part of the process.} 

Despite holding the relevant property rights, and despite having considerable commercial interests that would be effected, original owners were not identified as significant stakeholders in the project. Rather, the approach to the case was the traditional one, with focus being directed at the environmental impact, with relevant interests groups being called upon to comment on consequences in this regard, and quite some public debate arising with respect to the balancing of commercial interests and the desire to preserve wildlife and nature.

Nevertheless, one of the owners, Arne Ritland, commented on the proposed project, in an informal letter sent directly to Scana Steel Stavanger AS. In this letter he inquired for further information, and he protested the transferral of water from Brokavatn. He also mentioned the possibility that an alternative hydro-power project could be undertaken by original owners, but he did not go into any details regarding this, stating only that such a locally owned hydro-power plant had previously been in operation in the area. The plant he was referring to dates back to the time before we had a national grid, and was only directed at local supply of electricity. It has since been shut down.

Arne Ritland received a reply stating that more information on the project and its consequences would soon be provided, and he did not pursue the matter further at this time. Meanwhile, Scana Steel Stavanger AS submitted his letter to the water authorities, who in turn presented it to the NVE as a formal comment directed at the application. This prompted Jørpeland Kraft AS to undertake their own survey of alternative hydro-power in Sagåna, and the conclusion, but not the report itself, was sent to the water authorities. The original owners were not informed, and they were not asked to comment on it, or even told that such an investigation of the commercial potential in their waterfalls was being considered by the expropriating party, as a response to Ritland's letter.

Despite being presented with the issue, the water authorities did not take steps to investigate the commercial potential of local hydro power on their own accord. Moreover, the conclusion presented by Jørpeland Kraft AS did not go into details, but merely stated that if the local owners decided to build two hydro-power plants in Sagåna, then one of them, in the upper part of the river, close to Brokavatn, would not be profitable, neither with nor without the water in question. The other project, on the other hand, in the lower part, could still be carried out profitably even after the transferral. No mention was made as to what the original owners actually stood to loose, nor was there any argument given as to why it made sense to build two separate small-scale power plants in Sagåna. In their final report, the NVE handed these findings over to the Ministry, but did not inform the original owners. 

In addition to the report made by Jørpeland Kraft AS themselves, Hjelmeland kommune, the local municipality government, also commented on the possibility of local hydro-power. In their statement to the NVE, they directed attention to the data in the NVE's own national survey, which suggested that a single hydro-power plant in Sagåna would be a highly profitable undertaking. On this basis, they protested the transferral, arguing that original owners should be given the possibility of undertaking such a project. This statement was not communicated to the original owners, and in their final report it was dismissed by the NVE, who stated that the most energy efficient use of the water would be to transfer it and harness it at Jørpeland.

In addition to the statement made by Ritland, one other property owner, Ola Måland, commented on transferral. He did so without having any knowledge of the commercial potential the water held for him and his co-owners, and without having been informed of the statement made by Hjelmeland Kommune. On this basis, he expressed his support for the transferral, citing that the risk of flooding in Sagåna would be reduced. He also phrased his letter in such a way as to suggest he was speaking on behalf of other owners, but he was the only person to sign it. In the final report to the Ministry, the NVE, in their own conclusion, use this as an argument in favor of transferral, stating that the original owners were in favor of it, and that the opinion of Hjelmeland Kommune should therefore not be given any weight. They neglect to mention Arne Ritland's statement in this regard, and earlier in the report, where his statement is referred to along with many others, Ritland is referred to as a private individual, while Ola Måland is referred to as a property owner, and taken to speak on behalf of the others. The report made by the NVE, while it was not communicated to the affected local owners, it was sent to many other stakeholders, including Hjelmeland Kommune. In light of NVE's conclusions, they changed their original position, informing the Ministry that they would not press any further for local hydro-power, since this was not what the original owners wanted themselves. 

While the case was being prepared by the water authorities, the original owners had begun to consider the potential for hydro-power on their own accord, and in late 2006, when the case reached the Ministry, they where not aware that a decision was imminent. Rather, they were under the impression that they would receive further information before the case went further. Still, as they came to realize the commercial value of the water from Brokavatn in their own project, they approached the NVE, inquiring about the status of the plans proposed by Jørpeland Kraft AS. They were subsequently informed that an opinion in support of transferral had already been offered to the Ministry, and that a final decision would soon be made. This communication took place in late November 2006, summarized in minutes from meetings between local owners, dated 21 and 29 of November. On 15 of December 2006, the King in Council granted a concession for Jørpeland Kraft AS to transfer the water from Brokavatn to Jørpeland.

At this point, it was becoming increasingly clear to the original owners that the water from Brokavatn would be crucial to the commercial potential of their own project, and they also retrieved expert opinions suggesting that the NVE was wrong in concluding that transferral would be the most efficient use of the water. In light of this, they decided to question the legality of the transferral, arguing that the decision was invalid.

The license given to Jørpeland Kraft AS was challenged by the original owners on the grounds that the expropriation was materially unjustified, and that the administrative process leading up to the permission to expropriate did not fulfill procedural requirements. The local court, Stavanger Tingrett, held that the original owners were right in protesting the transfer, with the court emphasizing that the preparatory steps taken in cases such as these needed to provide adequate guarantee that the authorities had also considered the fact that the waterfalls could have been exploited commercially by the original owners themselves.\footnote{Stavanger Tingrett 20.05.2009, case nr. 07-185495SKJ-STAV.}

This view was rejected by the regional court, Gulating Lagmannsrett, which held that sufficient steps had been taken to clarify the commercial interests of the owners, and, moreover, that established practice regarding the preparation and evaluation of such cases -- dating from a time when it was not feasible for original owners to undertake hydro-power schemes -- still provided adequate protection.\footnote{Gulating Lagmannsrett 10.01.2011, case nr. 09-138108ASD-GULA/AVD2.} The Supreme Court also held in favor of Jørpeland Kraft AS, and they went even further in stating that established practice was beyond reproach.

In the following section, we present the main legal arguments relied on by the parties, as well as a summary of how the three national courts approached the case, and how they argued for their respective decisions.

\section{The legal arguments, and the view taken by the national courts}\label{view}

The original owners had several arguments in support of their claim that the concession was invalid. Firstly, they argued that procedural mistakes had been made in preparing the case; secondly, they argued that according to Norwegian expropriation law, it was not permissible to expropriate in a situation such as this, when the loss of energy and commercial potential would outweigh the gain to those same interests, which, ostensibly, were the only interests identified in favor of transferral. It seemed to the original owners that expropriation in this case would only serve to benefit the commercial interests of Jørpeland Kraft AS, and that it would do so to the detriment of both local and public interests. For this reason, the owners held that the concession should be regarded as an abuse of power, a manifestly ill-founded decision which could not be upheld.\footnote{There are at least two different ways in which to argue such a point under Norwegian law. One is with respect to water law and general administrative law, whereby clearly ill-founded decisions can be overturned by the courts, even when they involve discretion on part of the executive, which is otherwise not subject to review by the courts. Secondly, an argument can be made with respect to the Norwegian Constitution, Section 105, which gives property a protected status. The former is usually more effective, but in both cases, quite a severe transgression will have to be established before courts consider it within their competence to overturn discretionary decisions. A scholarly examination of these two sets of provisions are given in \cite{Efvl} and \cite{flei} respectively (both in Norwegian).} The owners argued, moreover, that the government had not fulfilled its duty to consider the case with due care, and that the assessment made with respect to the interests of the local community at Hjelmeland, and the local owners residing there, was not adequate. Particular attention was directed at the fact that local owners had not been informed about the progress of the case, and had not been told of, or asked to comment on, those preparatory steps that were being made explicitly with regards to assessing their interests. 

In addition, owners also argued that irrespectively of how the matter stood with respect to national law, the expropriation was unlawful because it would be in breach of the provisions in the ECHR TP1-1 regarding the protection of property.\footnote{European Convention of Human Rights Article 1 of Protocol 1.}\noo{An argument was also made to the effect that expropriation would be in breach of provisions in the EEA agreement regarding unlawful state support for the commercial interests of specific companies.}

Jørpeland Kraft AS protested all these objections to the expropriation, arguing that it was the responsibility of the owners themselves to provide information about possible objections against the project, and that the process had therefore been in accordance with the law. Unfortunate misunderstandings, if any, should be attributed to the fact that original owners had neglected their responsibilities in this regard. Moreover, Jørpeland Kraft AS argued that it was not for the courts to subject the assessment of public and private interests to any further scrutiny, since this was a matter for the government to decide. 

Indeed, according to Norwegian national law, it is traditionally held that unless the exercise of power it clearly unjustified, the courts do not have the authority to overturn decisions based on discretion, unless it can be demonstrated that the government has made procedural mistakes. While this view has become somewhat more relaxed in recent years, with a standard of \emph{reasonableness} increasingly being imposed by courts in similar cases, the inadmissibility of court interference in administrative discretionary decisions is still very much a part of Norwegian national law.\footnote{See \cite{Efvl}, in particular, chapters 24 and 29.}

Finally, Jørpeland Kraft AS argued that there was no issue of human rights at stake in the case. While they argued for this by stating that as the procedural rules had been followed and that the material decision was beyond reproach, they also went far in suggesting that as the owners would be compensated financially by the courts for whatever loss they would incur, no human rights issues could possibly arise in the case. \noo{ They also rejected the view that the case could be seen as an instance of illegitimate state support for Jørpeland Kraft, but failed to provide specific arguments in this regard.}

The matter went before Stavanger Tingrett who gave their judgment on 20 May 2009. In the following, we offer a presentation of the reasons given by this court, leading to the conclusion that the expropriation was unlawful and that the transferral could not be carried out. 

Stavanger Tingrett agreed with the original owners that the decision to grant concession was based on an erroneous account of the relevant facts, and they concluded that it was evident, from the NVE's own figures, that allowing the applicants to use the water from Brokavatn in their own hydro-electric scheme would be the most efficient way of harnessing the potential for hydroelectric production, directly contradicting what the NVE stated in their report. Moreover, they noted that these were the same estimates as those referred to by  Hjelmeland Kommune in their initial objection, and found it to be in breach of procedural rules that this was not considered further by the authorities.

The Court substantiated their decision by giving direct quotes from the report made by the NVE. For instance, in the report, on p. 199, it says, as quoted by Stavanger Tingrett (my translation):
%\begin{quote}Hjelmeland kommune ser helst at kraftressursene i vassdraget blir utnyttet av lokale %grunneiere. 
%Dette står i kontrast til uttalelsen fra grunneierne selv som ønsker at overføring blir gjennomført, 
%slik at flom og erosjonsskader kan bli noe redusert. NVE mener at den beste utnyttelsen med tanke 
%på kraftproduksjon vil være å tillate overføringen da en slik løsning vil innebære at vannet utnittes i 
%størst fallhøyde. Når dette samtidig er grunneiernes eget ønske har vi ikke tillagt Hjelmeland 
%kommunes synspunkt på dette noen vekt
%\end{quote}
%Our own translation follows below: 
\begin{quote}
Hjelmeland kommune would like the hydro-electric potential in the waterfall to be exploited by 
local property owners. This stands in contrast to the statement given by the property owners 
themselves, who wish that the transfer of water takes place, so that damage due to flooding can be 
somewhat reduced. NVE thinks that the best use of the water with respect to hydro-electric 
production is to allow a transfer, since this means that the water can be exploited over the greatest
distance in elevation. When this is also the property owners' own wish, we will not attribute any 
weight to the views of Hjelmeland kommune.
\end{quote}

Stavanger Tingrett concluded that as this was a factually erroneous account of the situation, the decision made to allow transferral of the water could not be upheld. Summing up, the Court offered the following assessment of the case (my translation):

\begin{quote}
It is the opinion of the court, having considered how the case was prepared by the authorities, that the factual basis for the decision made by the government suffers from several significant mistakes and is also incomplete.
\end{quote}

In light of this, Stavanger Tingrett concluded that the decision to grant concession for transfer of water was invalid. As to the legal basis of this, the court relied on the recognized principle of Norwegian public law that while the exercise of discretionary powers is usually not subject to review by court, a decision based on factual mistakes is nevertheless invalid if it can be shown that the mistakes in question were such that they could have affected the outcome. This is not provided for explicitly in statue, but it is one of the core unwritten legal principles of Norwegian public law.\footnote{See \cite{Efvl}}

Concerning the second requirement, that the factual mistakes could have affected the outcome, Stavanger Tingerett found that it was clearly fulfilled in this case since, in fact, the hydro-power suggested by original owners was, based on data available to the government at the time of decision, an objectively speaking \emph{better} use of the resource, even with respect to public interest. In any event, the requirement with regards to factual and procedural mistakes is only that the mistakes \emph{could} have affected the outcome; in the presence of mistakes, the burden of proof is shifted over to the party seeking to defend the decision.

Since Stavanger Tingrett agreed with the original owners that the decision was invalid due to being based on incorrect facts, there was no need to consider further the claims regarding the legitimacy of the decision with respect to human rights law. Stavanger Tingrett did conclude, however, making a more overreaching assessment of the case, that the procedure followed in preparing the case had not taken sufficient regard of owners' interests, and that this was the likely cause of the mistakes that had been made. The Court also argued that the standard of protection for interest of original owners had to interpreted as being more strict now that local hydro-power was an option available to original owners. 

\noo{In this regard, t also seems that Stavanger Tingett found some additional support in its interpretation of Norwegian law that was based on human rights concerns, especially the fact that expropriation, in circumstances such as those of this case, appeared to be a major interference in the rights of owners, and that established practice developed under a different regulatory regime was therefore no longer able to provide adequate protection.}

Jøpeland Kraft AS appealed the decision, and the case then went before the regional court, Gulating Lagmannsrett. They overruled the decision made by Stavanger Tingrett. In their argument, they do not rely on direct assessment of the report made by NVE, nor do they mention the expert statements retrieved by the opposing sides. Instead, they base their decision on general considerations concerning the need for efficient procedures in cases such as these. Such reasoning provides the apparent grounds for making the following rather crucial observation concerning the facts:

\begin{quote}... It was not a mistake to take Ola Måland's statement into consideration, as he was, and still is, a significant property owner. NVE's statement to the effect that granting the concession will facilitate 
a more effective use of the water seems appropriate, as it refers to a current hydro-electric plant that 
exploits a waterfall of 13.5 meters.
\end{quote}

Nowhere in their decision do they mention the statement made by Hjelmeland kommune, nor do they comment on the fact that alternative hydro-power, as suggested by the NVE itself, and pointed to in this statement, amounts to exploiting the waterfall over a difference in altitude of some 550 meters. In fact, the hydroelectric plant that they do mention has nothing to do with Ola Måland and the other owners, but exploits the same water further downstream. It was brought up in the testimony made by a representative from NVE, who, when pressed on the matter, claimed that the reasonable way to interpret the paragraph that Stavanger Tingrett quoted, and to which Gulating Lagmannsrett implicitly refer, was to see it as a statement regarding this hydro- electric plant. In light of the statement provided by Hjelmeland kommune, to which the report explicitly refers, this appears to be a manifestly ill-founded interpretation. But the regional court adopted it, without further comment.

As far as the legal basis of their decision is concerned, it seems that Gulating Lagmannsrett holds, quite generally, that the practice adopted by the water authorities in cases like these still provide adequate protection for original owners, and that it is not for the courts to subject it to critical review. As mentioned, they seem to base their stance in this regard on an overreaching appeal to the need for efficient procedures to deal with cases such as these.

The decision was appealed by Ola Måland and other, and the Norwegian Supreme Court decided to consider the juridical aspects of the case. The appeal concerning the assessment of the facts made by Gulating Lagmannsrett would not be considered, but was to be taken as correct. Since Gulating Lagmannsrett decided to regard as inessential several facts that were seemingly apparent, even from the report made by NVE itself, the appellants presented these facts to the Supreme Court and argued that Stavanger Tingrett was right regarding their consequences. \noo{In addition to this, written statements were retrieved from the Øystein Grundt, the public officer from the NVE that had been responsible for the preparation of the case, and Harald Sollie, }

The Supreme Court ruled in favor of Jørpeland Kraft AS. They comment on the relevant facts on 
p. 9 of their decision. There, they mention that Jørpeland Kraft AS had considered the possibility that a hydro-electric scheme could be undertaken by local property owners. As we mentioned in Section \ref{sum}, a statement was provided to the NVE by Jørpeland Kraft AS themselves -- the parties who stood to benefit from the transferral -- addressing one possible project that was deemed not to be commercially viable. Recall that in the same statement another project was also identified -- in the same river, using the same water -- that they claimed was such a good project that it could be carried out even after the transferral. As we mentioned, the statement does not say anything about what the property owners stand to loose when the water from Brokavatn disappears, and the Supreme Court is also silent on this. Nor do they mention that the statement was never handed over to the applicants, and that the details of the calculations were never handed over to, or considered by, the NVE. In fact, the full report first appeared during the hearing at Gulating Lagmannsrett, but this fact was not considered relevant by the Supreme Court.

Moreover, the Supreme Court remains silent on the fact that the conclusion concerning efficiency of exploitation contradicts both the NVE's own assessment, the statement made by Hjelmeland Kommune, and also all subsequent assessments made both on behalf of the applicants and on behalf of Jørpeland Kraft AS. We mention that all of the above were presented to all national courts, including the Supreme Court.

As to the legal questions raised by the case, the Supreme Court makes a more detailed argument than the regional court, culminating in the conclusion that established practice still provides adequate protection. Interestingly, the Supreme Court base their arguments in this regard on the premise that the case does \emph{not} involve expropriation of waterfalls. A similar sentiment is expressed by Gulating Lagmannsrett, and it was also argued for by Jørpeland Kraft AS, but the true force of this point of view did not become apparent until the case reached the Supreme Court. 

The Court first concludes that a legal basis for the concession to transfer the water is to be found in the Watercourse Regulation Act, Section 16. Moreover, they conclude that while this provision alone does not provide a right to expropriate the waterfall, it does give the applicant a right to divert the water away from it. While the Supreme Court notes that this amounts to an interference in property rights, they take it as an argument in favor of regarding the rules in the Watercourse Regulation Act as the primary source of guidance concerning what should be considered when preparing such cases. The hold, in particular, that the provisions in the Expropriation Act applies only so far as they supplement, and are not in conflict with, the rules of the Watercourse Regulation Act and established practice with respect to the provisions in this Act. Moreover, the main reason they give for this is that the diversion of water is \emph{not} to be considered as an expropriation of a waterfall.

There is, as we mentioned, no rule in the Watercourse Regulation Act which states that the authorities are required to consider specifically the question of how the regulation affects the interests of property owners. Such a rule is found in the Expropriation Act, Section 2, but according to the Supreme Court, it does not apply in cases where water is being diverted away from a river. This is so, according to the Supreme Court, because transferral of water is not regarded as a case of expropriation of a right to the waterfall, but merely an expropriation of a right to deprive the waterfall of water.

This is significant in two ways. First, it is important with respect to the legal status of owners who are affected by projects involving transferral of water. In Norwegian law after Måland, it seems that established practice with respect to the assessment of such cases, focusing on environmental aspects and the positions taken by various interest groups, is beyond reproach already because such cases do not involve expropriation of waterfalls. However, considering that the Norwegian water authorities seem to follow these practices generally, and not just in cases where water is transferred, it remains to be seen if this is a practically significant difference in the level of protection. Is the conclusion regarding the admissibility of current administrative practices supposed to apply only to those cases when water is subject to transferral? If it is, then it leads to the peculiar situation that the level of protection for owners depend solely on the way in which the developer propose to gain control over the water. The difference appears completely arbitrary, however, at least from the point of view of owners. But of course, it will soon cease to be arbitrary for developers, who must be expected to favor gutter projects, collecting water from many small rivers and diverting it, since this mode of exploitation makes it easier to acquire necessary rights. On the other hand, if the Supreme Court is to be understood as saying that traditional practices are adequate in general, the consequences of the decision seem fairly dramatic for local owners. It appears that it is not possible, in cases involving expropriation of waterfalls, to solicit any kind of judicial review, not even in circumstances when the factual basis of the decision is manifestly erroneous, and not even if this appears to be the consequence of the authorities neglecting to keep local owners informed about the assessments made regarding their interests.

To illustrate that a lack of consultation is a general problem, and not confined to the particular case of \emph{Måland}, we will conclude by offering a quote from Harald Solli, director of the Section for Concessions at the Ministry of Petroleum and Energy, who submitted written evidence to the Supreme Court regarding the practices followed in cases involving expropriation of waterfalls. Below, we give one of several exchanges that seem to indicate that under current practices, local owners are left in a rather precarious position (my translation).

\begin{quote}
Q: In cases such as this, should owners affected by a loss of small scale hydro-power potential be kept informed about the factual basis on which the authorities plan to base their decision? I am thinking especially about those cases in which the authorities make an assessment regarding the potential for small scale hydro-power on affected properties. \\
A: Affected owners must look after their own interests. The assessments made by the NVE in their report is a public document, and it can be accessed online through the home page of the NVE.
\end{quote}

By their reasoning in \emph{Måland}, it appears that the Supreme Court gave this dismissive attitude towards local owners a stamp of approval. In light of this, we believe the study of the law in a socio-legal setting becomes all the more relevant. For while this attitude might be a reflection of correct national law, as decided in the final instance by the Supreme Court, it seems pertinent to ask if it is \emph{reasonable} law. Also, it seems that one must ask if a case can not be made with respect to human rights, by arguing that the protection awarded is insufficient in this regard. This point, while it was raised by the original owners in \emph{Måland}, did not receive any separate treatment in the Supreme Court. In the following section, we briefly describe some more questions we think the case raises and which we will address further in subsequent chapters.

\section{Consequences of the case and the questions it raises}\label{cons}

Following \emph{Måland}, it seems we must conclude that the development which has taken place in the energy sector, and has lead to small scale hydro-power becoming profitable and possible for local owners to carry out themselves, does not imply that original owners are entitled to increased participation in decision-making processes under national law. Even if this is the view held by the Norwegian judiciary, we should of course not overlook the possibility that the water authorities themselves will eventually adopt new practices regarding the assessment of such cases. So far, however, it seems that they stick quite closely to the established routine. 

Since the outcome in Norwegian Courts was that established practices were not found to be in breach of principles of Norwegian expropriation law, it seems reasonable to ask instead about the sustainability of these practices. In fact, the case of \emph{Måland} seems to illustrate precisely why the current system is inadequate, and how it can lead to decisions that appear ill-founded and leave the affected communities feeling marginalized. The likelihood of \emph{factual mistakes}, in particular, seems to increase greatly when the involvement of the local population is not ensured in the preparatory stages.

More importantly, it seems that decisions reached following a traditional process can easily lead to takings for which it is difficult to see any legitimate reason why the project proposed by the developer would be a better form of exploitation than allowing the local owners to carry out their own projects. Indeed, in the case of \emph{Måland}, it seemed that small-scale hydro-power would be a better way of harnessing the water in question, even in the sense that it would be more efficient, and would provide the public with more electricity at a lower cost. More generally, unless the issue of alternative exploitation in small scale hydro-power is considered during the assessment made by the water authorities, one risks making decisions that are not in the public interest at all. 

Even worse, it can send out the signal that expropriation of owners' rights is undertaken solely in order to benefit the commercial interests of the energy company applying for a development license. We mentioned in Section \ref{context} that this mechanism, whereby expropriation appears to benefit commercial interests rather than the public, is becoming increasingly important in the international context as well. It is particularly in this regard that we think the case of Norwegian waterfalls warrants attention from the perspective of human rights. At this point, it seems appropriate to recall some concerns expressed by US Justice O'Connor, taken from her dissenting opinion in \emph{Kelo}.

\begin{quote}
Any property may now be taken for the benefit of another private party, but the fallout from this decision will not be random. The beneficiaries are likely to be those citizens with disproportionate influence and power in the political process, including large corporations and development firms. As for the victims, the government now has license to transfer property from those with fewer resources to those with more. The Founders cannot have intended this perverse result.
\end{quote}

In \emph{Kelo}, it seemed that a major point of contention was whether or not these grim predictions did indeed reflect a realistic analysis of the fallout of the decision. Surely, anyone who agrees with Justice O'Connor in her prediction, would also agree with here conclusion that it is perverse. However, whether her pessimism is warranted by empirical fact seems less clear. In this context, we believe the case of Norwegian waterfalls can serve an important broader purpose, as a means towards shedding more light on the hypothesis that a loose interpretation of the public interest requirement will indeed lead to a transfer of property from those with fewer resources to those with more. The \emph{Måland} case, and the current tensions regarding expropriation for the benefit of Norwegian hydro-power, seems to suggest that her concern should indeed be taken seriously. Also, the Norwegian experience seems to show that we need to be clear about the fact that property has a social and political function that goes beyond the financial interests of individuals. For the Norwegian case at least, it seems particularly relevant to ask if local people, by virtue of their right to property and their original attachment to the land, have a legitimate expectation \emph{both} that their commercial interests should be protected, \emph{and} that they should be granted a say in decision-making processes. Financial protection does not necessarily imply social protection, and the right to participate and be heard might be both more significant, and harder won, than the right to be compensated according to whatever the powers that be come to regard as the market value of the property in question.

Another perspective, which we will also pursue further in subsequent chapter, is the question of how property rights relates to the overreaching goal of sustainable development of natural resources. Rather than seeing property rights as a means towards securing sustainable development, it seems more common to see it as an impediment. This, indeed, has shaped much of the Norwegian discourse regarding environmental law and policy, including that which relates to waterfalls.\footnote{For example, such a skeptical view of property rights appear to provide an overriding perspective in \cite{backer1} (in Norwegian), which is a widely used textbook on environmental law in Norway.} Moreover, a typical justification given for interference in property is that an equitable and responsible management of natural resources requires it. It seems to us, however, that an egalitarian system of private ownership of resources -- as we find in Norway for the case of waterfalls -- could itself serve as a sustainable basis for management of these resources. It seems plausible for us to suggest that private property rights is one of the most robust ways in which local communities can be given a degree of self-determination concerning how to manage local resources. This is typically considered desirable also from the point of view of sustainability, but perhaps even more importantly, when property is in the hands of the many rather than the few, is it not also reasonable to expect that the state will be able to more effectively and rationally exercise its regulatory powers? Otherwise, the danger is that the government is being intimidated by large commercial enterprises, perhaps partly owned by the State itself, that command political influence and might not take lightly to what they perceive as undue political interference in their business practices. Such a position might be tenable if you are one of the worlds leading energy companies, but hardly if you are a farmer. 

We think the case of \emph{Måland} suggests that we should investigate these questions in more depth. It seems, in particular, that we must ask about the extent to which commercial companies have succeeded in usurping the notions of sustainable development and public interest, putting the power of these ideas to use in order to secure control over resources and to enlist governmental support, and favorable treatment, for their own commercial undertakings. The extent to which such a mechanism influences the Norwegian energy sector, and the possible implications this might have, both legally and socially, remains to be worked out.

In subsequent chapters, two questions arising from this will receive particular focus. First, we will aim to clarify the importance of the conflict between large scale hydro-power and small scale development by surveying recent and current hydro-power projects in Norway, not in any depth, but by taking note of whether the issue arose. Secondly, we will aim to shed light on the importance of small scale hydro-power to the communities in which local owners reside. As we mentioned, they are usually farmers, and most often in areas were farming is becoming increasingly unprofitable. From the socio-legal point of view it seems highly relevant to ask who the people who loose their resources are, and in what social context we find them. Moreover, while it is clear that hydro-power has become an important source of income in many small and relatively impoverished farming communities, the exact implications of this development, financially and socially, remains to be mapped out.

Following this, it seems natural to return to the legal question of the legitimacy of interference, not from the point of view of national law, but from the point of view of property as a human right. Importantly, it seems to us that property has a clear social dimension, and that mapping out the socio-legal function of specific property rights should inform the judgment we make regarding the level of protection to which owners are entitled. Also, while property is an individual right, it can also be a communal one, and, as such, it can serve to empower local communities that would otherwise be marginalized. The protection of an egalitarian structure of ownership, then, does not appear to be subsumed by, or even conceptually the same as, protecting against individual transgressions. We believe that the case of Norwegian waterfalls demonstrates that this should be kept in mind when analyzing the legitimacy of interference in property for the benefit of commercial undertakings.

\noo{current ownership structure of waterfalls is therefore not simply a question of protecting the commercial interests of individuals who happen to own valuable resources, but also a question of protecting the local communities where these resources are found, giving them the possibility of influencing the way in which the resources are to be harnessed. It seems, however, that local people are often in danger of being seen as an hindrance, both to sustainable development and economic growth, because the commercial companies, along with the environmental interests groups, have claimed this stage as their own. Such, it seems, is the case for Norwegian waterfall. Despite an explosion of interest in small scale hydro-power in recent years, there still seems to be little room left for local communities in the Norwegian discourse concerning hydro-power. It will be an important aim of our work in following chapters to map our in more detail how this influences the law and the administrative policies that are adopted.
}

\section{Conclusion}\label{conc}

As we have shown, \emph{Måland} serves to illustrate many of the current tensions and issues surrounding expropriation of waterfalls in Norway. It also serves to clarify the extent to which local owners are 
marginalized under the regulatory practices currently in place, and shows that the regulatory system does not clearly separate the question of how to judge an application to undertake development from the question of whether or not expropriation should take place. Moreover, the case seems to suggest that this will tend to lead to the emphasis being on issues that have to do with development, while issues relating to expropriation, and owners' interests, will be overlooked. Summing up, the case seems to show that the current regulatory system in Norway functions in such a way that it is bound to give rise to conflicts between local interests and the interests of commercial companies and the State.

The case also sheds new light on the legitimacy of using expropriation in order to benefit commercial interests. In this way, it takes on broader significance, by lending empirical support to the prediction offered by Justice O'Connor with respect to \emph{Kelo}, regarding the fallout of a loose interpretation of the public interest requirement for expropriation.

In our opinion, this contributes to making Norwegian waterfalls an interesting case study on expropriation,  and one that warrants further consideration with respect to human rights. In subsequent chapters, we will offer such an analysis, by addressing the question of whether or not local owners and communities can claim that they are entitled to greater protection than that which is currently provided under Norwegian law.

