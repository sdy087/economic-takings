%\newcommand{\isr}[1]{{#1}}

\chapter{Possible Approaches to the Legitimacy Question}\label{chap:3}

\section{Introduction}\label{sec:3:1}

%In the previous chapter, I introduced the social function perspective of property and argued in favour of a normative approach to property based on the notion of human flourishing. Moreover, I argued that economic development takings make up a separate category of interference with private property, deserving of special attention. I also placed this category in the theoretical landscape, by relating it to the theory of property presented in the first part of the chapter. Specifically, I argued that economic development takings raise questions that require us to depart from the individualistic, entitlements-based narrative that has tended to dominate in property theory.

There are many ways of thinking about the legitimacy of takings. Moreover, how one chooses to think about it is likely to depend not only on legal training, but also on more overarching visions of society. Specifically, it seems that one's approach to the legitimacy question will invariably depend also on a view of the relationship between the government, the law, and the institution of private property. To ask what imbues an act of taking with legitimacy, is to ask how this relationship should be regulated.

This chapter considers the question of legitimacy of economic development takings on the basis of a social function understanding of property inspired by the norm of human flourishing. To move towards a justiciable legitimacy standard on this basis, the chapter first considers existing approaches to judicial review, based on evidence from England and Wales, the United States, and the European Court of Human Rights.

Specifically, the chapter starts by considering the idea that the legislature itself imbues each instance of a taking with legitimacy, as the result of a decision made in a legitimate manner within a democracy. This narrative has long held sway in England and Wales, giving rise to a focus on procedural aspects, leaving little room for substantive judicial scrutiny of takings. The doctrine of deference has developed as an overarching norm that guide the courts when faced with controversial takings. %In England and Wales, this perspective carries great weight, particularly historically, when parliament itself would authorise most takings directly through so-called private Acts. 

The sheer size and complexity of the modern state, with its ever growing presence in the private sphere, puts this perspective on legitimacy under strain. To some extent, it can be upheld by a well-organised executive, compelled to remain faithful to Parliament and the ideas of democracy. However, a threat to the stability and success of such a procedural approach can arise from the lack of clear safeguards to protect against institutional failure and substantive abuse. If property as an institution begins to falter, for instance because takings for profit become too prevalent, the courts might find themselves unable to intervene on behalf of those democratic ideals that motivate the principle of deference in the first place. This chapter will argue that recent cases of economic development takings in England illustrate this worry, suggesting the need for substantive approaches to legitimacy in the law.

Following up on this, the chapter goes on to consider the US, where the public use restriction is considered to be an important substantive limitation on the government's power to take property. I argue that a contextual approach to the public use requirement, sensitive to local conditions and social functions, was prevalent among the state courts in early public use cases. Moreover, I note that there has been a resurgence of more extensive public use scrutiny after {\it Kelo}, particularly at the state level. However, this change appears to have been largely ineffective at curbing dubious uses of eminent domain. Specifically, it has been argued that recent reforms have been largely symbolic: hidden within the complex arrangements of modern government, it is business more or less as usual.\footnote{See generally \cite{somin09}.}

%I track the history of public use scrutiny in some depth, showing that it was widespread and extensive at the state level, especially until a contrasting position of almost unconditional deference to the legislature was adopted by the Supreme Court in the case of {\it Berman}. After this, at the federal level at least, the public use restriction was effectively stripped of its content.\footnote{See \cite{berman54}.} The eventual backlash of this came with {\it Kelo}, which was decided in keeping with precedent, but which gave rise to severe doubts among the justices, particularly those who looked at the history of the public use doctrine and how it had worked prior to {\it Berman}.

This in turn raises the issue of how to combine the procedural and the substantive perspective on legitimacy, to ensure that quality standards translate into effective protection. This brings me to the third approach to legitimacy, which I call the institutional fairness approach. I argue that this approach has been adopted recently by the ECtHR in Strasbourg, as the Court has developed a system of pilot judgements to deal with its vastly increasing case load. The idea of such judgements is that they will give the Court an opportunity to address systemic problems, to determine whether it should order the state to take general measures to improve national institutions.

Quite apart from the practical motivation behind this development, I argue that the institutional path is the way forward towards better testing for legitimacy in takings cases. It should work well because it allows courts to adopt a middle ground between the procedural and the substantive approach. The chapter elaborates on this claim by giving a more detailed presentation of the case of {\it Hutten-Czapska v Poland}.\footnote{See \cite{hutten06}.}

Following up on this, the chapter considers the question of how the courts should proceed to assess the legitimacy of economic development takings against a standard of institutional fairness. Building on a list of conditions due to Kevin Gray, I propose a concrete heuristic for this purpose. In addition to the original points made by Gray, I add three of my own, inspired by the discussion in this and the previous chapter. 

Admittedly, a legitimacy test can never provide more than a partial solution to the legitimacy problem. Specifically, if the desire for economic development is a genuine reflection of democratic decision-making, the follow-up question should be how to better enable the collective to communicate this desire to private owners, preferably without resorting to eminent domain. I address that question by looking to the governance theory of common pool resources, developed by Elinor Ostrom and others. I argue that the connection between their theories and property law suggests a need for new institutions that will allow the collective to push for economic development on private land without negating property rights.

Such a proposal has already been made by Heller and Hills, who recommends that so-called {\it Land Assembly Districts} should be used to replace eminent domain in many economic development scenarios.\footnote{See \cite{heller08}.} I analyse the proposal in some depth and argue that Land Assembly Districts are not the final answer to the legitimacy question for economic development takings. Specifically, I note the underlying tension between the ideal of self-governance and the fear of abuse by local elites, highlighting the need for institutions that are more closely matched to local conditions.
%es, and on what one regards as property's proper functions. %In light of this, I believe the critical examination of Land Assembly Districts marks a natural end to this chapter, as well as to the theoretical part of this thesis as a whole.

\section{England and Wales: Legitimacy through Parliament}\label{sec:3:2}

In England and Wales, the principle of parliamentary sovereignty and the lack of a written constitutional property clause has led to expropriation being discussed mostly as a matter of administrative law and property law, not as a constitutional issue.\footnote{See generally \cite{taggart98}.} Moreover, the use of compulsory purchase -- the term used to denote takings in the UK -- is not restricted to particular purposes as a matter of principle.\footnote{See \cite[201]{allen00}; \cite[48-49]{waring09}.} The uses that can justify taking property by compulsion are those uses that Parliament regard as worthy of such consideration.\footnote{According to Reynolds, the idea that Parliament (as opposed to various local authorities) was the key legitimising force behind takings gained ground in England from the 16th century onwards, see \cite[41-42]{reynolds10}.} However, as private property has typically been held in high regard, the power of compulsory purchase has traditionally been exercised with caution.\footnote{See \cite[15]{allen00}; \cite[47-48]{waring09}. However, there has never been anything like a complete aversion to expropriation, see \cite[34-46]{reynolds10}; \cite[126-128]{hoppit11} (pointing out that compulsory purchase grew dramatically in scope after the glorious revolution in 1688).}

In his {\it Commentaries}, William Blackstone famously described property as the ``third absolute right, inherent in every Englishman''.\footnote{See \cite[134-135]{blackstone79}. The first right, according to Blackstone, is security, while the second is liberty.} Moreover, Blackstone expressed a very restrictive view on the appropriateness of expropriation, pointing out that it was only the legislature that could legitimately interfere with property rights. He warned against the dangers of allowing private individuals, or even public tribunals, to be the judge of whether or not the common good could justify takings. Blackstone went as far as to say that the public good was ``in nothing more essentially invested'' than the protection of private property.\footcite[134-135]{blackstone79}

In terms of historical accuracy, Blackstone's claims about the sanctity of property in England and Wales appear questionable. Specifically, it has been argued that his description of property might be shaped not so much by evidence as by political values gaining ground among the bourgeoisie after the decline of the feudal system.\footnote{See \cite[34-35]{waring09} (describing Blackstone's account as giving rise to a ``myth''); \cite[121]{hoppit11} (noting the tension in Blackstone's own account of property, which also acknowledged that parliament frequently interfered with private property).} That said, it should not be forgotten that the conferral of compulsory purchase powers did require parliamentary involvement on a case-by-case basis, often through so-called {\it private} Acts, granting compulsory purchase powers to specific legal persons.\footnote{See \cite[43-46]{nulty12}; \cite[204]{allen00}; \cite[104-116]{waring09}.} Arguably, this practice reflects that takings of private property, although far from unheard of, were indeed considered draconian.\footnote{Moreover, it is noteworthy how much weight was consistently placed on the right to compensation, also when Parliament sanctioned the taking, see \cite[15]{allen00}.}

Interestingly, the procedure followed by Parliament in takings cases often resembled a judicial procedure; the interested parties were given an opportunity to present their case to Parliament committees that would then effectively decide whether or not compulsion was warranted.\footnote{See \cite[13-16]{allen00}; \cite[105-106]{waring09}.} On the one hand, the direct involvement of Parliament in the decision-making arguably marks a fundamental, albeit conditional, respect for property rights.\footnote{See \cite[103]{hoppit11}.} But at the same time, parliamentary sovereignty meant that the question of legitimacy was rendered mute as soon as compulsory purchase powers had been granted. The courts were not in a position to scrutinize takings at all, much less second-guess Parliament as to whether or not a taking was for a legitimate purpose.\footnote{See, e.g., \cite[643]{nulty12}; \cite[107]{waring09}.}

During the late 18th and and early 19th century, as an industrial economy developed, private Acts grew massively in scope and importance.\footnote{See \cite[108-112]{waring09}; \cite[100-101]{hoppit11}.} Railway companies, in particular, regularly benefited from such Acts.\footnote{See \cite[144]{kostal97}.} During this time, the expanding scope of private-to-private transfers for economic development led to high-level political debate and controversy.\footnote{See \cite[144]{kostal97}; \cite[204]{allen00}.} There would often be particular opposition coming from the House of Lords. This opposition was not only based on a desire to protect individual property owners, but also tended to reflect concerns about the cultural and social consequences of changed patterns of land use.\footcite[204-205]{allen00}

Hence, the early {\it political} debate on economic development takings in the UK shows some reflection of a social function approach to property protection. At the same time, as society changed following increasing industrialisation, a more expansive approach to compulsory purchase would eventually emerge as the norm.\footnote{Arguably, the social function perspective helps explain why this happened. Indeed, the expanded use of private takings in England during the 19th century, particularly in connection with the railways, might have served a more easily justifiable social function than that commonly associated with economic development takings today. Waring, in particular, notes how railway takings tended to affect aristocratic landowners rather than marginalised groups (``unlike private takings today, the railway legislation was most likely to affect those who could best defend their property rights from attack''), see \cite[111]{waring09}.} The idea that economic development could justify takings became less controversial.

Today, the law of compulsory purchase is regulated in statute. Hence, Parliament rarely gets involved on a case-by-case basis and the role of the courts is largely limited to the application and interpretation of statutory rules.\footnote{See \cite[116-121]{waring09}. Some common law rules still play an important role, such as the {\it Pointe Gourde} rule, which stipulates that changes in value due to the compensation scheme itself should be disregarded when calculating compensation to the owner. The rule takes it name the case of \cite{gourde47}. See also the discussion in chapter \ref{chap:5}, section \ref{sec:5:5:3}.} With respect to the question of legitimacy, the starting point for the courts is that this is a matter of ordinary administrative law.\footnote{See \cite{taggart98}.} More recently, the \cite{hra98} adds to this picture, since it incorporates the property clause in P1(1) into the law.\indexonly{echr} Even so, the usual approach in England and Wales is to judge objections against compulsory purchase orders on the basis of the statutes that warrant them, rather than constitutional principles or human rights provisions that protect property.\footnote{See \cite[121-132]{waring09}.} It is typical for statutory authorities to include standard reservations to the effect that some societal benefit must be identified in order to justify a compulsory purchase order, but the scope of what constitutes a legitimate purpose can be very wide. For instance, to justify a taking under the \cite{tcpa90}, it will generally suffice to argue that it will ``facilitate the carrying out of development, redevelopment or improvement on or in relation to the land''.\indexonly{tcpa90}\dni\footcite[226]{tcpa90}

While various governmental bodies are authorised to issue compulsory purchase orders (CPOs), a CPO typically has to be confirmed by a government minister.\footnote{See \cite[48]{waring09}.} The affected owners are given a chance to comment, and if there are objections, a public inquiry is typically held. The inspector responsible for the inquiry then reports to the relevant government minister, who makes the final decision about whether or not it should be granted, and on what terms. The CPO may later be challenged in court, but then on the basis of the statute authorising it, not on the basis of whether or not the purpose is legitimate as such.\footnote{See \cite[48-49]{waring09}. The typical way to launch an attack on a taking would be to argue that it serves a purpose that falls outside the scope of the statute authorising it, or, more subtly, that the administrative decision-maker took irrelevant purposes into account when granting the power.}

That said, the idea that property may only be compulsorily acquired when the public stands to benefit permeates the system.\footnote{It should be emphasised that this idea is not traditionally anchored in any fixed interpretation of terms like ``public interest'' or ``public purpose'', much less ``public use''. The starting point is statutory interpretation, not independent judicial scrutiny of whether the purpose of compulsory purchase is sufficiently ``public'' according to some general standard, see \cite[20-24]{allen00}.} Indeed, this has also been regarded as a constitutional principle, for instance by Lord Denning in {\it Prest v Secretary of State for Wales}.\footnote{See \cite[198]{prest82} (``I regard it as a principle of our constitutional law that no citizen is to be deprived of his land by any public authority against his will, unless it is expressly authorised by Parliament and the public interest decisively so demands.'').} Moreover, in {\it R v Secretary of State for Transport, ex p de Rothschild}, Slade LJ spoke of ``a warning that, in cases where a compulsory purchase order is under challenge, the draconian nature of the order will itself render it more vulnerable to successful challenge''.\footcite[938]{rothschild89}

In keeping with the principle of parliamentary sovereignty, this warning targets judicial review of administrative decision-making, not legislation. Despite this limitation, the English approach to legitimacy has traditionally proved quite effective in preventing controversy from arising with respect to the use of eminent domain.\footnote{See generally \cite{allen10}.} An underlying respect for private property, superseded only by respect for the authority of Parliament, appears to have influenced the decision-making framework and the surrounding administrative practices. Specifically, legitimacy has been pursued through legislation, regulation, and administrative practice, leaving less room for substantive judicial scrutiny.\footnote{For a more detailed analysis of how this works, noting, among other things, that higher levels within the executive are also meant to act as safeguards of private property, filling -- to some extent -- the possible role of courts in this regard, see \cite[85-100]{allen08}.}

However, England and Wales have also seen controversial economic development takings challenged in the courts. Indeed, such cases appear to have become more frequent.\footnote{See generally \cite{gray11}.} For instance, in the case of {\it Alliance}, many properties were taken in order to facilitate the construction of a new stadium for the football club Arsenal.\footcite{alliance06} Some owners who stood to lose their business premises protested on the basis that the purpose was dubious, pointing also to the fact that the inspector in charge of the public inquiry had recommended against the takings.\footcite[6-7]{alliance06} Their arguments also invoked P1(1) of the ECHR, to overcome the limitations of traditional judicial review in England and Wales. However, these argument were all quite summarily rejected by the Court.\footnote{See \cite[6-7]{alliance06}. For a critical discussion, describing the Court's assessment against P1(1) as ``worryingly brief'', see \cite[26]{gray11}.\indexonly{echr}}

Arguably, the {\it Alliance} case reflects a weakness of the English approach to legitimacy. This weakness, moreover, appears to go beyond whatever doubts one might have about the principle of parliamentary sovereignty applied to property as a constitutional and human right. Specifically, if the framework laid down by Parliament greatly empowers the administrative branch, while failing to appropriately regulate administrative practices, the deference due to Parliament might effectively become undue deference to the executive branch. If the {\it practice} of using compulsory purchase continues to expand in relation to for-profit undertakings, there appears to be a significant risk of abuse associated with broad powers granted to the executive to take property for economic development. If there is also a change of perspective on the role of private property in society, moving away from the reverent attitude expressed by Blackstone, there appears to be a lack of other sources for legitimacy in a system so reliant on a narrative of pure procedure.

To some extent, it would be possible for the Supreme Court to develop a more restrictive stance on compulsory purchase to address this, within the established constitutional order. In fact, there are some signs that this might be about to happen, specifically with respect to the broad powers granted under section 226 of the \cite{tcpa90}. In the case of {\it R (Sainsbury's Supermarkets Ltd) v Wolverhampton City Council}, Lord Walker cited {\it Kelo} and went on to comment that ``economic regeneration brought about by urban redevelopment is no doubt a public good, but ``private to private'' acquisitions by compulsory purchase may also produce large profits for powerful business interests, and courts rightly regard them as particularly sensitive''.\footnote{See \cite[82]{sainsbury10}.}

However, the outcome of {\it Sainsbury} also underscores the weaknesses of an indirect approach to legitimacy through administrative law. Instead of relying on Lord Walker's observations about the sensitivity of economic development takings, the majority of the Court quashed the CPO on the basis of a more conventional approach to statutory interpretation. Specifically, the majority found that the local government had erred when it took into account promises that the taker had made regarding a regeneration project in a different part of town. This was regarded as contravening section 226 of the \cite{tcpa90}, which only directs attention at the potential for improvements on or in relation to ``the land'', i.e., the land that is subject to compulsory purchase. The reasoning behind the decision, therefore, rests largely on a technicality, not any substantive assessment of legitimacy.

On a more purposive assessment, the taking in {\it Sainsbury} should arguably have been upheld: the owner and the taker were both large commercial companies, they had shared ownership of the disputed development site, they both wanted to develop at the expense of the other party, and the taker appeared to have the best overall plan for the community. Ironically, the English approach resulted in such a taking being struck down as illegitimate, while the taking in {\it Alliance}, involving the displacement of local people in favour of a football club, received little or no scrutiny at all. In light of this, it seems that alternatives to the traditional idea of legitimacy should be considered, at least if one agrees with Lord Walker's characterisation of economic development takings as ``particularly sensitive''.\footnote{See \cite[82]{sainsbury10}.}

%\footnote{Similarly in the case of \cite{margate13} (the owners of an amusement park engaged extensively with the local authorities to come up with a regeneration plan, but their property was compulsorily acquired and handed over to a different developer when the negotiating climate soured).}

\section{The US: Legitimacy through Public Use}\label{sec:3:3}

By contrast to the situation in England and Wales, the US Constitution is a basis for judicial review also with respect to the federal and state legislatures. Considering its status as a basis for potentially extensive review, the Constitution is remarkably terse. The takings clause, arriving as the final clause of the Fifth Amendment, reads simply ``nor shall private property be taken for public use, without just compensation''.\footnote{See the \cite{fifth}.}

The compensation requirement is clearly stated, if embryonic, but the takings clause is also understood to include the requirement that property may only be taken for ``public use''. This is the aspect of the clause that interests me in this thesis, since it provides an anchor for legitimacy that is particularly relevant -- and contentious -- in relation to economic development takings.\footnote{At least this is so on a social function understanding of legitimacy. For an in-depth discussion of the compensation requirement applied to economic development takings, arguing that the compensatory viewpoint does not get to the heart of the legitimacy question, see \cite{dyrkolbotn15a}.} Specifically, the question is to what extent such takings offend against the clause: is a taking for economic development by a commercial company really a taking for ``public use''?

Going back to the time when the Fifth Amendment was introduced, there is not much historical evidence explaining why the takings clause was included in the Bill of Rights. Moreover, there is little in the way of guidance as to how the takings clause was originally understood. James Madison, who drafted it, commented that his proposals for constitutional amendments were intended to be uncontroversial.\footnote{See letters from Madison to Edmund Randolph dated 15 June 1789 and from Madison to Thomas Jefferson dated 20 June 1789, both included in \cite{madison79}.} Hence, it is natural to regard the takings clause as a codification of an existing principle, not a novel proposal. Indeed, several state constitutions pre-dating the Bill of Rights also included takings clauses, seemingly based on codifying principles from English common law.\footcite[See][299]{johnson11} 

%As Meidinger notes, the Americans had never really charged the British with abuse of eminent domain, and private property had tended to be respected, also in the colonies.\footcite[17]{meidinger80} This undoubtedly influenced early US law.

Just like English scholars at the time, early American scholars emphasised the importance of private property. James Kent, for instance, described the sense of property as ``graciously implanted in the human breast'' and declared that the right of acquisition ``ought to be sacredly protected''.\footnote{See \cite[see][257]{kent27}.} Indeed, the Supreme Court itself expressed similar sentiments early on, when it spoke of the impossibility of passing a law that ``takes property from A and gives it to B''.\footnote{This was a {\it de dicta} in \cite[388]{calder98}. See also \cite[310]{vanhorne95}.}

However, just as in England, this early US attitude changed in response to industrial advances and a desire for economic development. As the 19th century progressed, eminent domain was used more frequently, now also to benefit (privately operated) railroad operations, hydroelectric projects, and the mining industry.\footcite[23-33]{meidinger80} During this time, it also became increasingly common for landowners to challenge the legitimacy of takings in court, undoubtedly a consequence of the fact that eminent domain was used more widely, for new kinds of projects.\footcite[24]{meidinger80} 

Controversy over the public use requirement arose particularly often with respect to the so-called mill Acts.\footnote{\cite[24]{meidinger80}. See also \cite[306-313]{johnson11} and \cite[251-252]{horwitz73}.} Such Acts were found throughout the US, many of them dating from pre-industrial times when mills were primarily used to serve the farming needs of agrarian communities.\footnote{A total of 29 states had passed mill Acts, with 27 still in force, when a list of such Acts was compiled in \cite[17]{head85}. According to Justice Gray, at pages 18-19 in the same, the ``principal objects'' for early mill Acts had been grist mills typically serving local agrarian needs at tolls fixed by law, a purpose which was generally accepted to ensure that they were for public use.} Following economic and technological advances, provisions originally enacted to serve local farming purposes were now being used by developers wishing to harness hydropower for manufacturing and hydroelectric plants.\footnote{See, e.g., \cite[18-21]{head85} and \cite[449-452]{minn06}.}

It is important to note, however, that mill Acts could not be used to authorise large-scale compulsory transfers of natural resources from owners to non-owners.\footnote{See the discussion in \cite{head85}.} Rather, mill Acts provided management tools that could be used to ensure that owners of water resources could make better use of their rights. This would sometimes involve allowing riparian owners to interfere with, or take a necessary part of, the property of their neighbours, e.g., by constructing dams that would flood neighbouring land.\footnote{See, e.g., \cite[265]{staples03}.} However, the primary purpose was to facilitate rational coordination among owners, to the benefit of their community as a whole. This point was frequently made by courts that upheld mill Act takings, also when such takings would benefit the manufacturing industry.\footnote{See, e.g., \cite{fiske31}. See also the discussion (including references to other cases) in \cite{head85}.}

%As the industrial use of mill acts increased in scope, the original aim of these acts gradually became overshadowed by the strength of the commercial interests involved. 
%This, in turn, lead to public use controversies arising in relation to provisions that had not previously raised any doubts.\footnote{See \cite{head86}.} The case law on
More generally, case law on public use from the state courts at this time was characterised by a highly contextual understanding of property protection and the meaning of public use.\footnote{See, e.g, \cite{scudder32} (taking upheld, but said that ``the great principle remains that there must be a public use or benefit. That is indispensable. But what that shall consist of, or how extensive it shall be to authorize an appropriation of private property, is not easily reducible to a general rule.'');  \cite[409]{seawell76} (taking for a mineral company upheld on the basis that mining was the ``greatest of the industrial pursuits'' in the state of Nevada and that the benefits of the industry were ``distributed as much, and sometimes more, among the laboring classes than with the owners of the mines and mills.''); \cite[337]{ryerson77} (taking struck down, by a state court that was ``not disposed to say that incidental benefit to the public could not under any circumstances justify an exercise of the right of eminent domain.''. See also \cite{gray11} (with many references to state courts striking down takings as impermissible).} Arguably, the case law on public use from the states even deserves to be categorised as an early example of a legitimacy approach based on a social function understanding of property. Initially, it was also well received by the Supreme Court, as discussed below.

\subsection{Legitimacy as Discussed by the Supreme Court}\label{sec:3:3:1}

\noo{ The early 20th century was a period of great optimism about the ability of {\it laissez faire} capitalism to ensure progress and economic growth, a sentiment that was reflected in the federal case law on eminent domain. A particularly clear expression of this can be found in {\it Mt Vernon-Woodberry Cotton Duck Co v Alabama Interstate Power Co}.\footcite{vernon16}  This case dealt with the legitimacy of condemnation arising from the construction of a hydropower plant. The Supreme Court held that it was legitimate, with the presiding judge arguing briskly as follows:

\begin{quote}The principal argument presented that is open here, is that the purpose of the condemnation is not a public one. The purpose of the Power Company's incorporation, and that for which it seeks to condemn property of the plaintiff in error, is to manufacture, supply, and sell to the public, power produced by water as a motive force. In the organic relations of modern society it may sometimes be hard to draw the line that is supposed to limit the authority of the legislature to exercise or delegate the power of eminent domain. But to gather the streams from waste and to draw from them energy, labor without brains, and so to save mankind from toil that it can be spared, is to supply what, next to intellect, is the very foundation of all our achievements and all our welfare. If that purpose is not public, we should be at a loss to say what is. The inadequacy of use by the general public as a universal test is established. The respect due to the judgment of the state would have great weight if there were a doubt. But there is none.\footcite[32]{vernon16}
\end{quote}

On the one hand, the Court notes the importance of deference to the {\it state} judgement (not specifically the judgement of the state legislature). On the other hand, it prefers to conclude on the basis of its own assessment of the purpose of the taking. This assessment, however, is not grounded in the facts of the case or the circumstances in Alabama. Rather, it is based on sweeping assertions about ``all our welfare'' and the desire to ``save mankind from toil that it can be spared''. This marks a contrast with the approach of state courts, as discussed in the previous subsection.
}
With respect to takings ordered by the federal government, the Supreme Court never showed much willingness to enforce a strict public use requirement.\footnote{However, for a long time, federal takings were of limited practical significance since the common practice was that the federal government would rely on the states to condemn property on its behalf, see \cite[30]{meidinger80}. This changed towards the end of the 19th century, particularly following the decision in {\it Trombley v Humphrey}, where the Supreme Court of Michigan struck down a taking that would benefit the federal government, see \cite{trombley71}.} In {\it United States v Gettysburg Electric Railway Co}, a case from 1896, deference to the legislature in federal takings cases was referred to as a principle that should be observed unless the judgement of the legislature was ``palpably without reasonable foundation''.\footcite[680]{gettysburg96}

Importantly, however, such a deferential stance was not adopted in cases originating from the states.\footnote{Originally, the Supreme Court had held that the takings clause in the US Constitution did not apply to state takings at all, see \cite{barron33}. However, this changed after the due process clause of the Fourteenth Amendment was introduced after the Civil War, see, e.g., \cite{head85}. Later, in 1897, the Supreme Court held that state takings could be scrutinized directly against the takings clause of the Fifth Amendment, see \cite{chicago97}.} In {\it Cincinatti v Vester}, a case from 1930, the Supreme Court commented that ``it is well established that, in considering the application of the Fourteenth Amendment to cases of expropriation of private property, the question what is a public use is a judicial one''.\footcite[447]{vester30}

In the earlier case of {\it Hairston v Danville \& W R Co}, from 1908, the same was expressed by Justice Moody, who surveyed the state case law and declared that ``the one and only principle in which all courts seem to agree is that the nature of the uses, whether public or private, is ultimately a judicial question.''\footcite[606]{hairston08} Justice Moody continued by describing in more depth the typical approach of the state courts in determining public use cases:

\begin{quote}
The determination of this question by the courts has been influenced in the different states by considerations touching the resources, the capacity of the soil, the relative importance of industries to the general public welfare, and the long-established methods and habits of the people. In all these respects conditions vary so much in the states and territories of the Union that different results might well be expected.\footcite[606]{hairston08}
\end{quote}

Justice Moody goes on to give a long list of cases illustrating this aspect of state case law, showing how assessments of the public use issue had been inherently contextual.\footcite[607]{hairston08} Following up on this, he points out that ``no case is recalled'' in which the Supreme Court overturned ``a taking upheld by the state {\it court} as a taking for public uses in conformity with its laws'' (my emphasis). After making clear that situations might still arise where the Supreme Court would not follow state courts on the public use issue, Justice Moody goes on to conclude that the cases cited ``show how greatly we have deferred to the opinions of the state courts on this subject, which so closely concerns the welfare of their people''.\footcite[606]{hairston08}

{\it Hairston} is important for three reasons. First, it makes clear that initially, the deferential stance in cases dealing with state takings was primarily directed at state courts rather than legislatures and administrative bodies. Second, it demonstrates federal recognition of the fact that a consensus had emerged in the states, whereby scrutiny of the public use determination was consistently regarded as a judicial task.\footnote{Indeed, {\it Hairston} provides the authority for {\it Vester} on this point. See \cite[606]{vester30}.} Third, it provides a valuable summary of the contextual approach to the public use test that had developed at the state level. 

The {\it Hairston} Court clearly looked favourably on the case law from state courts. Importantly, when a deferential stance was adopted, this was clearly contingent on the assumption that state courts would continue to administer the public use test with the required vigour. Despite this, {\it Hairston} would later be cited as an early authority in favour of almost unconditional deference to legislators.\footnote{In fact, it was cited in this way also by the majority in {\it Kelo}, see \cite[482-483]{kelo05}.} 

This happened in {\it US ex rel Tenn Valley Authority v Welch}, concerning a federal taking.\footcite[552]{welch46} The Court first cited {\it US v Gettysburg Electric R Co} as an authority in favour of deference with regards to the public use limitation.\footcite{gettysburg96} The Court then paused to note that {\it Vester} later relied on a conflicting view, namely that the public use test was a judicial responsibility.\footcite{vester30} The Court then attempts to undercut this by setting up a contrast between {\it Vester} and {\it Hairston}, by selectively quoting the observation made in the latter case that the Supreme Court had never overruled the state courts on the public use issue.\footnote{See \cite[552]{welch46}.} Hence, {\it Hairston} is effectively used to argue against judicial scrutiny, in a manner that is quite incommensurate with the full rationale behind the Court's decision in that case.

Later, {\it Welch} was used as an authority in the case of {\it Berman v Parker}.\footcite{berman54} This case concerned condemnation for redevelopment of a partly blighted residential area in the District of Colombia, which would also condemn a non-blighted department store. In a key passage, the Court states that the role of the judiciary in scrutinizing the public purpose of a taking is ``extremely narrow''.\footcite[32]{berman54} The Court provides only two references to previous cases to back up this claim, one of them being {\it Welch}.\footnote{The other case, {\it Old Dominion Land Co v US}, concerned a federal taking of land on which the military had already invested large sums in buildings. The Court commented on the public use test by saying that ``there is nothing shown in the intentions or transactions of subordinates that is sufficient to overcome the declaration by Congress of what it had in mind. Its decision is entitled to deference until it is shown to involve an impossibility. But the military purposes mentioned at least may have been entertained and they clearly were for a public use'', see \cite[66]{dominion25}. A partial quote, to the effect that deference to the legislature is in order except when it involves an ``impossibility'', was used to justify the decision in \cite[240]{midkiff84}.}

Moreover, both of the cases cited were concerned with federal takings, while in {\it Berman} the Court explicitly says that deference is due in equal measure to the state legislature.\footcite[32]{berman54} It is possible to regard this merely as a {\it dictum}, since the District of Columbia is governed directly by Congress. However, {\it Berman} was to have a great impact on future cases. In effect, it undermined a large body of case law on judicial review of takings without engaging with it at all.

In {\it Hawaii Housing Authority v Midkiff}, the Supreme Court further entrenched the principle expressed in {\it Berman}.\footcite{midkiff84} Here the state of Hawaii had made use of eminent domain  to break up an oligopoly in the housing sector. Given the circumstances of the case, it would have been natural to argue in favour of this taking on the basis that it served a proper public purpose.

However, the Court instead decided to rely on the doctrine of deference, shunning away from scrutinizing the takings purpose. Justice O'Connor, in particular, observed that ``judicial deference is required because, in our system of government, legislatures are better able to assess what public purposes should be advanced by an exercise of eminent domain''.\footcite[244]{midkiff84}

Effectively, what had been a doctrine of deference to state courts had now transformed into a doctrine of deference to state legislatures. In light of this, it had to be expected that {\it Kelo} would be decided in favour of the taker.\footnote{In fact, as pointed out by Somin, the {\it Kelo} case represents a slight tightening of the earlier line on public use. See \cite{somin07}.} However, the history of the public use requirement tells us that this was not inevitable. Hence, the question arises whether legitimacy can be increased by reviving the public use test. The next section sheds some light on this, on the basis of legislative developments in the US after {\it Kelo}.

\subsection{Economic Development Takings after {\it Kelo}}\label{sec:3:3:2}

Following {\it Kelo}, much attention was directed at the danger of eminent domain abuse in the US.\footnote{See generally \cite{somin09}.} Moreover, the {\it Kelo} decision itself proved extremely unpopular. Surveys suggest that about 80-90 \% of the general population believe that it was wrongly decided, an opinion widely shared also among members of the political elite.\footcite[2109]{somin09}

Many states responded by introducing reforms aimed at limiting the use of eminent domain for economic development.\footnote{For an overview and critical examination of the myriad of state reforms that have followed {\it Kelo}, I point to \cite{eagle08}. See also \cite{somin09}.} Within two years, 44 states had passed post-{\it Kelo} legislation in an attempt to achieve this.\footnote{See \cite{castle}.} Various legislative techniques were adopted. Some states, including Alabama, Colorado and Michigan, enacted explicit bans on economic development takings and takings that would benefit private parties.\footcite[See][107-108]{eagle08} In South Dakota, the legislature went even further, banning the use of eminent domain ``(1) For transfer to any private person, nongovernmental entity, or other public-private business entity; or (2) Primarily for enhancement of tax revenue''.\footnote{South Dakota Codified Laws § 11-7-22-1, amended by House Bill 1080, 2006 Leg, Reg Ses (2006).}

In other states, more indirect measures were taken, such as in Florida, where the legislature enacted a rule whereby property taken by the government could not be transferred to a private party until 10 years after the date it was condemned.\footcite[809]{eagle08} Many states also introduced lists of uses that were to count as public, designed to restrict the room for administrative discretion while allowing condemnations for purposes that were regarded as particularly important.\footcite[804]{eagle08}

%As Somin has pointed out, state reforms enacted by the public through referendums tend to be more restrictive than reforms passed through the state legislature.\footcite[2143]{somin09} Many of the more radical reform proposals, moreover, were not endorsed by any of the branches of government, but were initiated by activist groups as ballot measures.\footnote{In some US states, initiative processes make it possible for activist groups to put measures on the ballot without prior approval by the state legislature. See \cite[2148]{somin09}.} As Somin observes, the reforms taking place via this route would be comparatively strict, testifying to the power of direct democracy.\footnote{See \cite[2143-2149]{somin09}.}

These reforms show that {\it Kelo} had a great effect on the discourse of eminent domain in the US. However, the effect on the law has been less clear. According to Somin, most state reforms have been ineffective.\footcite[2170-2171]{somin09} Even when seemingly strict rules have been introduced, it is typically easy for the government to carry out economic development takings as before, provided these takings are not referred to as economic development takings, but described in some other vague manner, e.g., as removal of ``blight''.\footnote{See \cite[2170-2171]{somin09}.} At the same time, property lawyers report a greater feeling of unease regarding the correct way to approach the public use requirement, expressing hope that the Supreme Court will soon revisit the issue.\footnote{See \cite{murakami13} (``Until the Supreme Court revisits the issue, we predict that this question will continue to plague the lower courts, property owners, and condemning authorities.'').} 

Why have legislative reforms proved inadequate? Part of the reason, according to Somin, is that people are ``rationally ignorant'' about the economic takings issue.\footnote{See \cite[2170]{somin09}.} For most people, it is unlikely that eminent domain will come to concern them personally or that they will be able to influence policy in this area. Hence, it makes little sense for them to devote much time to learn more about it. This, in turn, helps create a situation where experts can develop and sustain a system based on practices that a majority of citizens actually oppose.\footcite[2163-2171]{somin09} To back up this analysis, Somin points out that surveys seem to show that people generally overestimate the effectiveness of eminent domain reform, by mistaking symbolic measures for materially significant changes in the law.\footcite[2155-2157]{somin09}

Arguably, this also shows that the legislative approach so far, which has focused on introducing more elaborate and detailed versions of the public use restriction, need to be supplemented by different kinds of proposals. Specifically, it seems important to also target the structural processes that result in the taking of private land for economic development. After all, it is when we direct attention at the decision-making involved in bringing about actual takings that we will locate those stakeholders who cannot afford to remain rationally ignorant about eminent domain. %These processes, it seems, need to be imbued with greater legitimacy. %In particular, it might be that owners themselves should be granted a better chance to participate in the management of their property, even when this involves deliberating on, and possibly taking part in, large-scale development projects. After all, it is the feeling that owners' and their communities' feeling that they are being treated unfairly that tend to lie at the root of controversies surrounding takings for economic development.\footnote{For a similar perspective, see \cite{underkuffler06}.}

This points towards another perspective on legitimacy, whereby focus shifts towards the institutional setting where the relevant decisions are made. Importantly, cases such as {\it Kelo} suggest that this needs to involve more than administrative law and ideas about procedural due process. Specifically, institutional legitimacy appears to have an important substantive component whereby a decision is legitimate only in so far as it results from democratically legitimate decision-making within an administrative framework that is generally conducive to fair and proportional outcomes. Arguably, recent developments at the ECtHR point towards a perspective on legitimacy that emphasises this interconnectedness between substantive and procedural aspects of fairness at the institutional level.

\section{The ECtHR: Legitimacy as Institutional Fairness}\label{sec:3:4}

It is often said that the P1(1) of the ECHR consists of three rules.\indexonly{echr} The first rule guarantees a right to `peaceful enjoyment of possessions', the second rule regulates the legitimacy of `deprivation' and the third rule regulates how the states can legitimately `control the use of property'.\footnote{See \cite[61]{sporrong82}.}

When dealing with complaints pertaining to P1(1), the Court in Strasbourg will typically first consider which of these three rules it should apply.\footnote{See \cite[102-104]{allen05}.} However, as noted by Allen, it is not clear that this choice has any great significance for the outcome.\footnote{See \cite[104-105]{allen05}.} In practice, the evaluation proceeds in much the same way regardless of which rule is used, with an emphasis on the requirement that a {\it fair balance} must be struck between the opposing interests in cases when states interfere with private property rights.\footnote{See \cite[103]{allen05}. It is also typically assumed that an interference is only legitimate when it takes place for an appropriate purpose, but here the ECtHR has consistently maintained a deferential stance, pointing to the `wide margin of appreciation' that the member states enjoy in this regard. See, e.g., \cite[54]{james86}.} The Court has gradually adopted a more active role in assessing whether or not this requirement is met. As argued by Allen, this has caused P1(1) to attain a wider scope than what was originally intended by the signatories.\footcite[1055]{allen10}\indexonly{echr}

In the early case law behind this development, the focus was predominantly on the issue of compensation, with the Court gradually developing the principle that while P1(1) does not entitle owners to full market value in all cases of interference, the fair balance will typically be upset when less than market value is paid, especially if the reduction is significant or inadequately justified.\footnote{See \cite[103]{scordino06}.}

However, the fair balance test encompasses more than the issue of compensation. In particular, the hunting cases discussed in chapter \ref{chap:2} show that the Court in Strasbourg is willing to reflect broadly on the context and purpose of interference, to critically assess the social function of takings.\footnote{See chapter \ref{chap:2}, section \ref{sec:2:4:1}.} Moreover, institutional aspects of fairness have come to play an important role in the Court's reasoning in some other recent cases involving property.\footnote{See \cite{hutten06,lindheim12}.} This is particularly clearly demonstrated by the case of {\it Hutten-Czapska v Poland}.\footnote{See \cite{hutten06}.}

\subsection{Legitimacy {\it Erga Omnes}}\label{sec:3:4:1}

The striking conclusion in {\it Hutten-Czapska v Poland}, underscoring the institutional turn at the ECtHR, was that the case demonstrated ``systemic violation of the right of property''.\footcite[239]{hutten06} The case concerned a house that had been confiscated during the Second World War. After the war, the property was transferred back to the owners, but in the meantime, the ground floor had been assigned to an employee of the local city council.\footcite[20-31]{hutten06} The state implemented strict housing regulations during this time, which eventually led to the applicant's house being placed under direct state management.\footcite[20-31]{hutten06} Following the end of communist rule in 1990, the owners were given back the right to manage their property, but it was still subject to strict regulation that protected the rights of the tenants.\footcite[31-53]{hutten06} In addition to rent control, rules were in place that made it hard to terminate the rental contracts.\footcite[20-53]{hutten06}\indexonly{echr}

After an in-depth assessment of the relevant parts of Polish law and administrative practice, the Grand Chamber of the ECtHR concluded that there had been a violation of P1(1). Importantly, they did not reach this conclusion by focusing on the owners and the interference that had taken place with respect to their individual entitlements. Rather, they focused on the overall character of the Polish system for rent control and housing regulation, as exemplified by the applicant's situation.

Specifically, the consequences for the owners were considered not in isolation, but in order to shed light on a broader question of sustainability.\footcite[60-61]{hutten06} The Court was particularly concerned with the fact that the total rent that could be charged for the house in question was not sufficient to cover the running maintenance costs.\footcite[224]{hutten06} In particular, it was noted that the consequence of this would be ``inevitable deterioration of the property for lack of adequate investment and modernisation''.\footnote{\cite[224]{hutten06}.}

In the end, the Court highlighted how three factors combined to bring both owners and their properties  to a precarious position. First, the rigid rent control system made it hard to sustainably manage rental property. Second, tenancy regulation made it hard for owners to terminate tenancy agreements. Third, the Court noted that the state itself had set up many tenancy agreements during the days of direct state management, shedding doubt on the fairness of the obligations that these contracts imposed on owners.\footcite[224-225]{hutten06}\indexonly{echr}

The Court's reasoning in {\it Hutten-Czapska} is also interesting because of how the `social rights' of the tenants is placed on an equal footing to the property rights of the owners.\footcite[225]{hutten06} Arguably, property rights and social rights are not considered merely as separate sets of entitlements, locked in opposition to one another. In the reasoning of the Court they also appear as mutually dependent social functions, both hampered by an unsustainable approach to property and housing during the communist era and beyond.\footnote{Specifically, the Court attached great significance to the finding that rents were too low to cover maintenance costs, see \cite[224]{hutten06}. A lack of incentives for maintenance is clearly a threat to tenants as much as to owners, illustrating the interdependence between the two groups. Despite this, when summing up their reasoning in broad strokes, the Court itself reverts back to a traditional narrative when it speaks about the ``conflicting interests of landlords and tenants''. See \cite[225]{hutten06}.}

In this regard, the Court places considerable weight on the precarious situation of the owners. Specifically, the Court notes the ``absence of any legal ways and means making it possible for them either to offset or mitigate the losses incurred in connection with the maintenance of property or to have the necessary repairs subsidised by the State in justified cases''.\footnote{See \cite[224]{hutten06}.} Moreover, the Court comments that the ``burden cannot, as in the present case, be placed on one particular social group, however important the interests of the other group or the community as a whole''.\footnote{See \cite[225]{hutten06}.} Importantly, however, the Court does not censor the political reasoning that motivated the rent control scheme, but rather focuses on the fact that it had not been implemented properly.\footnote{See \cite[224]{hutten06}.}\indexonly{echr}

On this basis, the Court concludes that there had been a systemic violation of P1(1), and orders Poland to take measures to rectify the ``malfunctioning of Polish housing regulation''.\footnote{See \cite[237]{hutten06}. The basis originally relied on for formulating such an order was Article 41 in the ECHR, first used in this way in the case of \cite{broniowski05}.} Hence, the lack of legitimacy was pronounced with a kind of {\it erga omnes} (towards all) effect, establishing an obligation for Poland directed at all its citizens in equal measure, not merely the applicant.\footnote{There was some dissent as to whether or not this was an appropriate response, with Judge Zagrebelsky in particular arguing against it on the grounds that it would see the Court ``entering territory belonging specifically to the realm of politics.''. See \cite{hutten06}.} Judgements of this kind, known as `pilot judgements', have now gained formal recognition as a distinct procedural form that the ECtHR can use to address systemic problems.\footnote{In 2011, the pilot procedure was explicitly incorporated and regulated in the Rules of Court, see \cite[87]{leach11}. For pilot judgements generally, see \cite{leach10}.}

The institutional approach conditioned by the introduction of pilot judgements might point to the core function that the ECtHR is likely to serve in the future.\footnote{See, e.g., \cite{greer12} (arguing that a ``constitutional pluralism'' approach to adjudication -- better filtered, more principled, yet still context sensitive --  is the way ahead for the ECtHR).} Indeed, the ECtHR will hardly be able to protect human rights in Europe on a case-by-case basis. Nor would it seem appropriate for it to do so, given its remoteness to local conditions and its relative lack of democratic accountability. However, when the Court is able to identify systemic failures that look set to systematically give rise to imbalances and unfairness, it seems appropriate that it should take action.\indexonly{echr}

This is particularly clear when, as in the case of {\it Hutten-Czapska}, the Court notes that the applicants have insufficient options available for achieving a fair balance by appealing to institutions within the domestic legal order. In such cases, it seems appropriate for the Court to demand a change at the level of the state's own institutions, giving rise to a broad duty for the state to improve those institutions. Moreover, by scrutinizing the procedures and principles that the states apply when fulfilling this duty, it is likely that the Court will still be able to steer and unify the development of the case law on human rights, at least to the extent that this is required to meet minimum standards.

Against the deferential implications of this shift of attention, it could be argued that the judicial or administrative bodies of the signatory states can easily circumvent their obligations by providing superficial reforms or biased assessments of the facts in human rights cases, to avoid embarrassment for the state's political or bureaucratic elites. However, this might then be raised as a more procedurally oriented complaint before the ECtHR, perhaps also against Articles 6 (fair trial) and 13 (effective remedy).\footnote{I note that this also fits with recent developments at the ECtHR, toward somewhat broader scrutiny under Article 6, see \cite{khamidov07}.}\indexonly{echr}  

In this way, the Court can streamline its functions, by always aiming to direct attention at issues that arise at a higher level of abstraction.\footnote{A similar argument was given by Judge Zupan\u{c}i\u{c} in \cite{hutten06} (``Is it better for Poland to be condemned in this Court 80,000 times and to pay all the costs and expenses incurred in 80,000 cases, or is it better to say to the country concerned: `Look, you have a serious problem on your hands and we would prefer you to resolve it at home...! If it helps, these are what we think you should take into account as the minimum standards in resolving this problem...'? Which one of the two solutions is more respectful of national sovereignty?'').} This, in my view, seems highly desirable. The ECtHR should not aim to micromanage the signatory states, particularly not in relation to a norm such a P1(1), which the Court itself regards as highly dependent on context. By shifting attention towards institutional fairness, the Court can avoid getting stuck in deference to the states without overstepping its bounds with regards to the democratic process.\footnote{Getting this balancing act right is a key challenge for any system of judicial review, especially in cases with socio-economic and political overtones. I will not provide a  discussion of previous work on this issue, but mention specifically that the Constitutional Court in South Africa has moved in an interesting direction that also appears conducive to an institutional fairness perspective, particularly through its recent emphasis on ``meaningful engagement'' in cases where social and economic rights are at stake, see \cite{pillay12}.}

Indeed, the case of {\it Hutten-Czapska} is highly suggestive of the merits of such a perspective, not only because of the special measures ordered, but also because the Court reasoned on the basis of institutional information to identify systemic weaknesses of Polish housing regulation.\footnote{Specifically, it seems that the shift signalled by recent cases on property at the ECtHR does not end with a new take on remedies, but also signals some changes in the way the Court approaches the fair balance determination under P1(1). This seems natural; if the Court looks for systemic violations, not (only) individual transgressions, its substantive assessments of fairness will naturally be influenced.} Another example is the recent case of {\it Lindheim and others v Norway}.\footnote{See \cite{lindheim12}.} Here the applicants complained that their rights had been violated by a Norwegian Act that gave lessees the right to demand indefinite extensions of ground leases on pre-existing conditions.\footnote{See \cite[119]{lindheim12} (the Act in question was the \cite{gla96}).}\indexonly{echr}

The Court agreed that this was a breach of P1(1). Moreover, it engaged in the same form of assessment as it had adopted in {\it Hutten-Czapska}. Specifically, it concluded that the \cite{gla96} as such was the underlying source of the violation. The problem was not merely that this Act had been applied in a way that offended the rights of the applicants. In light of this, the Court did not only award compensation, it also ordered that general measures had to be taken by the Norwegian state to address the structural shortcomings that had been identified.\footnote{See \cite{lindheim12}.}

The Court also commented that its decision should be regarded in light of ``jurisprudential developments in the direction of a stronger protection under Article 1 of Protocol No. 1''.\footcite[135]{lindheim12} However, in light of the change in perspective that accompanies this development, it is interesting to ask in what sense exactly the protection is stronger. In particular, it is not {\it prima facie} clear that the Court's remark should be read as a statement expressing a change in its understanding of the content of individual rights under P1(1). Rather, it may be read as a statement to the effect that the Court has assumed greater authority to address structural problems under that provision.

If this is true, it could make a big difference in cases involving takings for economic development. As illustrated by Justice O'Connor's dissent in {\it Kelo}, a main concern is that such takings are likely to have ``perverse'' consequences at the structural level, because they lack democratic merit. \footnote{To quote Justice O'Connor's dissent in {\it Kelo}, see \cite{kelo05}.} In light of cases such as {\it Hutten-Czapska} and {\it Lindheim}, I think the ECtHR would have been likely to approach {\it Kelo} in a manner consistent with Justice O'Connor's approach.\indexonly{echr}

Whether they would reach the same conclusion seems more uncertain, particularly since confidence in the states' ability and willingness to regulate private-public partnerships might be higher in Europe than in the US.\footnote{For a discussion from the point of view of English law, arguing that the prevailing regulatory regime limits the risk of eminent domain abuse largely through regulation of the takings power rather than strict property protection, see \cite{allen08}.} However, it seems unlikely that the ECtHR would follow the majority in {\it Kelo}, by simply deferring to the determinations made by the granting authority. Rather, Justice O'Connor's predictions about the fallout of the {\it Kelo} decision would likely have been of significant interest also to the justices at the Court in Strasbourg.

To conclude, I think notions of institutional fairness can help us locate a welcome middle ground between largely procedural notions of justiciable legitimacy, such as those found in England and Wales, and substantive notions, such as those found in the US. The question remains how Courts adopting such a middle ground should proceed when presented with a concrete case of alleged eminent domain abuse. In the next section, I present a possible heuristic.

\section{The Extended Gray Test}\label{sec:3:5}

Pointing to early US case law on public use as a ``laboratory of elementary proprietary ideas'', Kevin Gray builds on the evidence found there to provide a set of conditions for recognising what he calls ``predatory takings''. \footnote{See \cite[28-30]{gray11}.} His work is clearly relevant to any theory of economic development takings inspired by the notion of human flourishing. Below, I present his main contribution, a collection of abuse indicators that I will refer to as the {\it Gray test}. I then present three addenda inspired by the discussions presented in this and the previous chapter. This gives rise to the {\it extended Gray test}, my proposed heuristic for assessing the legitimacy of takings, especially suited to situations when there are strong commercial interests present on the side of the taker.

Several combinations of conditions might be sufficient to justify designating a taking as eminent domain abuse. The purpose of the extended Gray test is not to produce a definite set of such conditions that provide a final answer in all cases. Rather, the aim is to provide a heuristic to facilitate concrete assessment against the social, economic and political circumstances surrounding the taking in question. If an economic development taking represents an abuse of power, one would expect it to run afoul with regard to some, and probably several, of the criteria set out in the following points.

\subsubsection*{Balance of Power among the Parties}

In a typical case of eminent domain abuse, the parties that stand to benefit will be more economically and politically powerful than those from whom property is taken.\footnote{See \cite[30-31]{gray11}. Gray himself omits any explicit mention of political power, but it is present in Justice O'Connor's dissent in {\it Kelo}, and in my view clearly belongs here.} For example, this can be reflected in the takers' ability to solicit legal assistance and other services to defend the taking, as well as in the  owners' inability to launch a coordinated defence.\footnote{See \cite[30-31]{gray11}.} If there is an imbalance of power, this is particularly likely to be noticeable early on, during the planning stages, before the decision to condemn has actually been made. 

After the decision has been made, the procedural position of the owners might improve. However, this can be insufficient to restore an appropriate balance between the parties; when special procedural protections kick in, it will often be too late for the owners to launch an effective defence against the taking. For instance, strict rules concerning cost reimbursement for costs incurred {\it after} the decision to take has already been made, is not a sufficient response to an imbalance of power, especially not in legal systems that offer limited opportunity for judicial review of takings purposes.

More generally, a possible imbalance of power should be assessed against the decision-making process as a whole, going back to the first initiative made for taking the property in question. A critical  assessment of what role the owners have played in the decision-making process is a good way to uncover more information about imbalances of power, and whether or not such imbalances could have unduly influenced the outcome.

\subsubsection*{The Net Effect on the Parties}

As Gray notes, a hallmark of eminent domain abuse is that the net effect of the taking is a ``significant transfer of valued resource from one set of owners to another''\footnote{See \cite[31]{gray11}.} In itself, this is not a conclusive sign of abuse, but it directs us to ask two important questions. First, we should inquire critically into the main purpose of the taking. Is the transfer of resources between the parties an acknowledged motive or an unacknowledged side-effect of some ostensibly distinct public purpose?

In the latter case, it might be that the public purpose is only a pretext for benefiting the taker, in which case it counts as clear evidence of abuse. In less obvious cases, if the transfer of resources arising from the fulfilment of the public purpose was not properly discussed and critically examined by the decision-maker, this too can point towards predation.

The assessment will be different if redistribution of (control over) resources is openly acknowledged as part of the rationale justifying eminent domain. In such cases, it is pertinent to ask further  questions about the economic and social status of the parties, and the structure of the decision-making process, to shed light on whether the redistributive motive itself appears democratically legitimate. If there is eminent domain abuse, one would expect the taking to fail to stand up to scrutiny in this regard.

In some cases, it might be debatable whether a taking passes the net effect test. However, the scrutiny is still significant, since it helps bring the crucial questions into the open, thereby ensuring higher quality of the decision-making regarding the taking. Indeed, if the extended Gray test is applied at an early stage of the proceedings, this in itself might help increase acceptance of the decisions reached. Making room for more extensive legitimacy tests in takings law might well end up bolstering the government's power to take property, as long as the power is used faithfully.

\subsubsection*{Initiative}

In many suspicious economic development takings, the party benefiting commercially from the taking is the party that initially made the suggestion for using eminent domain.\footnote{See \cite[32]{gray11}.} In uncontroversial cases, on the other hand, the initiative tends to come from some government body that seeks to pursue a specific policy goal, e.g., to provide a public service or bestow a benefit on a particular group that is found to be in need of support. The contrast between this and cases when the initiative lies with the commercial beneficiaries themselves point to a disturbance of the decision-making underlying the decision to use eminent domain. As such, it is an important hallmark of abuse.

To investigate further under this point, one should take into account the wider social and political context of the taking, particularly the position of the parties involved. If the beneficiary is both more powerful and privileged than the owners {\it and} takes the initiative for the taking, this is clearly a sign pointing towards predation. On the other hand, if the beneficiaries are marginalised groups who could only expect any consideration if they were to take the initiative themselves, the situation might have to be viewed differently. %In these cases, the fact that the system leaves room for marginalised non-owners to acquire property interests might have to be considered a strength rather than a weakness. Still, as discussed in later points, the appropriateness of using eminent domain for redistributive purposes can be questioned, even if the redistributive goal itself appears democratically legitimate. In such cases, however, the question of legitimacy is unlikely to turn on the initiative test.

\subsubsection*{Location}

The location, in a broad sense of the word, of the property that is taken, can be a strong indicator that eminent domain is inappropriate.\footnote{See \cite[33-34]{gray11}.} For instance, cases involving the taking of dwellings are naturally more suspect than cases involving the taking of barren or unused plots of land. Similarly, the taking of property that is important to the subsistence of the current owner should raise the bar for when a taking may be considered legitimate. Moreover, if the taker's choice of location appears to be one of convenience rather than necessity, this points towards predation. It is particularly telling if alternative locations would be less intrusive, or obviate the need for using eminent domain altogether.

%The location of the property can also attain relevance independently of the current owner. For instance,

On the other hand, the location of the property can sometimes point towards {\it increased} legitimacy of a taking that would otherwise appear suspect. This might be the case, for instance, if the property that is taken has special value to the taker or the community specifically because of its strategic importance with respect to the taker's own property or the rights of non-owners.\footnote{For instance, if riparian owners cannot make rational use of the water flowing over their land without intruding on the land of their neighbours, using eminent domain to resolve this might be considerably less suspect than other kinds of economic development takings. This particular scenario was much discussed in the US during the 19th century, in relation to mill Acts which authorised neighbour-to-neighbour takings of limited property rights needed for development. See the discussion in section \ref{sec:3:3}.} The proper balance of burdens and benefits might still be upset, but a taking that fits smoothly into a `special value' narrative will be less suspect than one that does not.

\subsubsection*{Social Merit}

As Gray notes, a taking that is hard to justify on the basis of its social merits is more likely to be predatory.\footnote{See \cite[34]{gray11}. Gray writes of lap-dancing clubs and cigarette factories as examples of purposes that are suspect. Importantly, such purposes might well fulfil a public interest requirement via the economic development narrative, yet still fail a social merit test that focuses rather on the social dimensions of the use to which the property will be put.} This asks for closer scrutiny of the kinds of purposes that can be used to justify a taking. If the justification narrative surrounding a taking revolves solely around `trickle-down' effects and the successful business ventures that the taking will facilitate, there is reason to be suspicious. Specifically, if the taking cannot sustain a social merit narrative, whereby attention is shifted away from purely economic considerations, this is a strong independent indication that the taking might count as predation.

The point here is not that the language of social merit should replace the language of public use or public interest as some kind of conclusive test of legitimacy. Rather, the point is that one should always be encouraged to analyse takings specifically in terms of non-economic, social, effects. This is particularly important in difficult cases, because it can help us to arrive at a better understanding of where exactly the taking sits on the gray scale between admissible governance and predatory exploitation. 

%If a taking appears to stand up to scrutiny only when embedded in a purely economic narrative, this in itself suggests a lack of legitimacy. Indeed, even if one concedes that incidental economic effects are relevant, it seems clear that however one circumscribes a notion such as public use, this notion certainly encompass {\it more} than merely those incidental economic effects that tend to occupy center stage in legitimacy disputes.\footnote{Indeed, it bears emphasising that those arguing against economic development takings might achieve more by emphasising non-economic aspects, compared to arguing that incidental economic benefits should not at all be considered relevant as a justification for eminent domain.}

\subsubsection*{Environmental Impact}

According to Gray, a typical feature of eminent domain abuse is that it has an adverse environmental impact. Moreover, a typical feature of eminent domain abusers is that they show disregard for such adverse affects.\footnote{See \cite[34]{gray11} (``predatory takers tend to be relatively unperturbed if they lay waste to the earth.'').} This is an additional element that pertains specifically to the status of the taker, asking us to consider whether it is appropriate to grant their activities public interest status. It is not primarily a question of how the development stands with regard to environmental regulation. Rather, what is at stake is whether or not the characteristics of the taker and the development plans make it appropriate to use the power of eminent domain. 

It might be appropriate to use environmental law as a starting point, but the relevant environmental standard with regard to the legitimacy question should be drawn up more strictly than the standards generally applied to the type of development in question. Indeed, one should be entitled to expect heightened environmental awareness from a developer and a development plan that benefits from the power of eminent domain.

Arguably, the mere fact that takers engage in active lobbying for leniency in relation to environmental standards can be enough to shed doubt on the claim that they act in the public interest. What might otherwise be considered natural and admissible behaviour for a common commercial company can be improper or inadmissible behaviour for one that benefits from eminent domain powers. I note that this particular observation has general import, pertaining to a potentially wider set of obligations that takers may be expected to take on, not only environmental ones. This brings me to the first addendum  that I propose to add to Gray's original evaluation points.

\subsubsection*{Addendum 1: Regulatory Effects}

As discussed in chapter 2, property has an important regulatory effect, also outside the realm of positive law. This effect typically changes following a taking, sometimes quite dramatically.  For instance, if locally owned property is taken by external commercial actors for high-intensity commercial use, the post-taking regulatory status of the property will most likely be completely different to its status prior to the interference. Moreover, the changed status might have as much to do with informal social functions as it has to do with positive regulation.

It might be, for instance, that the property in question is found in a jurisdiction that emphasises  the freedom of owners to do as they please without state interference. In this case, the fallout of allowing external commercial actors to take locally owned property can be particularly severe, as the new owner is likely to be unconstrained by locally grounded systems for sustainable resource management. In these cases, there is a risk that there will be a `tragedy of the taken', arising from how the taking undermines an important building block of sustainability. 

\noo{Indeed, a society based on egalitarianism and strict limits on state interference might find it especially difficult to appropriately restrain the actions of actors who use the eminent domain power to accumulate property for high-intensity use.\footnote{This problem can of course arise independently of the use of eminent domain, e.g., in the context of land grabs arising from voluntary or semi-voluntary transactions. However, the situation appears particularly problematic if the state itself is complicit in bringing about the problem, by undermining property's social function through the use of the takings power.} If this is resolved by increasing the state's power to interfere with private property through regulation, the effect can be a further undermining of local management frameworks, increased subsequent use of eminent domain, and a general spreading and amplification of the democratic deficit already inherent in the original act of taking.}

A different regulatory concern is that the legal status of the property can change, for instance because the development in question brings it under the scope of different rules. A concrete example of this mechanism will be encountered in Part II of the thesis, when we consider expropriation of water rights for hydropower development in Norway. As mentioned in the introduction of the thesis, water rights in Norway are typically held in common by local smallholders, arising from their co-ownership of the surrounding outfields. Water rights are then typically classified as agricultural property, meaning that special rules apply, including rules that protect the local community and identifies the municipalities and other democratic institutions at the local level as the primary regulatory bodies. As soon as water rights are expropriated, the connection with land rights and local agriculture is severed, resulting in an almost complete transfer of regulatory authority from democratic institutions at the local level to the national-level water directorate. This serves as an example of how the regulatory framework can change dramatically when property rights are expropriated. When assessing the legitimacy of a taking, it will then be especially relevant to consider whether the new framework offers better or weaker protection for the local community, the environment, or the general public. If the effects appear to be largely negative in these respects, it will reflect badly on the decision to use eminent domain.%\footnote{The case study of Norwegian waterfall expropriation will offer an example of this mechanism, c.f., Chapter 5, Section \ref{sec:x}.}

\subsubsection*{Addendum 2: Impact on Non-Owners}

Following up on the theoretical arguments made in chapter \ref{chap:2}, it is appropriate to direct special attention at the status of non-owners directly affected by economic development takings. It is of particular importance to ascertain whether or not the interests of such non-owners were given due consideration prior to the decision to use eminent domain. If their interests appear to have been neglected, or have not been considered at all, there is additional reason to be sceptical of the justification provided for the taking. Indeed, just as disregard for the environment is a typical sign of predation, a general disregard for local non-owners is also an indicator of abuse.

\noo{To shed further light on this, one might first ask what role non-owners played in the decision-making process. If the non-owners directly affected by the taking were allowed to express their opinion, and enjoyed some measure of influence, this can enhance legitimacy. If, on the other hand, the most immediately affected members of the public were not consulted, or not given a proper voice in the proceedings, it indicates abuse.

There is also an important substantive aspect to consider: how is the taking going to affect property dependants without recognised ownership rights? If it is clear that they}

Moreover, if directly affected non-owners enjoyed little or no influence over the decision-making or if they will be made to suffer severe adverse effects, this should be counted as an indication of predation irrespective of mitigating procedural arrangements, e.g., measures to ensure `consultation' or the like. If these measures have little or no material significance, they cannot hope to restore legitimacy. Importantly, awarding compensation is also not sufficient to excuse shortcomings in this regard. If people are displaced, for instance, the fact that new opportunities are provided for them somewhere else should not be allowed to detract from the fact that a community has been destroyed.\footnote{See, e.g., \cite{cullet01}.} It is possible that the needs of the public necessitate such a drastic interference with property's proper function, but this should then at once give rise to a more in-depth scrutiny of legitimacy. Moreover, the bar to pass the legitimacy test should be raised considerably in such cases.

It should also be noted that the position of non-owners is largely determined by the regulatory framework surrounding the property in question. Hence, the evaluation of legitimacy of takings with respect to non-owners is closely related to the evaluation with respect to regulatory effects, considered in the previous point. For instance, if property taken under eminent domain is simultaneously removed from the ambit of democratic decision-making at the local level, this can weaken the position of non-owners. Moreover, if the property is re-classified and brought under the scope of different rules, for instance because its status changes from agricultural to industrial property, this can leave the non-owners with weaker substantive rights. With a social function perspective on property, such effects should be looked at more closely when considering the legitimacy of takings.

\subsubsection*{Addendum 3: Democratic Merit}

Perhaps the most important characteristic to consider when assessing the legitimacy of a taking is its democratic merit. In an important sense, putting a taking to the test against this measure serves to encapsulate all the other points raised above. Specifically, it asks us to consider the totality of these factors in order to judge whether good governance standards have been observed within a system based on democratic decision-making. The inquiry made in this regard should not be focused on second-guessing government policies, but should compel us to take seriously the idea that a commitment to democracy places real constraints on the exercise of government power.\footnote{See the discussion in chapter \ref{chap:2}, sections \ref{sec:2:4} and \ref{sec:2:5}.} In this way, an overarching focus on democratic merit can render the principles of scrutiny expressed by the extended Gray test as a possible template for courts across different jurisdictions, with respect to both constitutional and human rights provisions.

The main question that arises with respect to democratic merit is whether the taking in question 
can be said to arise from a legitimate process of decision-making, in the pursuit of a fair and equitable outcome. It bears emphasising that the relevant assessment under this point involves both procedural and substantive elements. Fairness in itself is a constraint on the democratic process, particularly when fundamental economic and social rights are involved. At the same time, the notion of democratic merit rightly brings procedural questions to the foreground. Indeed, it might be a weakness of Gray's original proposal that it does not single out procedural issues for special consideration.

\noo{On the one hand, it is inappropriate to reduce the takings question to a matter of administrative law. But on the other hand, the way in which the taking decision was made can often tell us much about its legitimacy, including how it stands with regard to broader notions of fairness. It is particularly important, in this regard, to inquire into the position of local owners and communities during the planning process leading up to the decision to use eminent domain. In the context of property as a human right, moreover, a stricter standard might be appropriate here, compared to that which would otherwise follow from administrative law.}

A commitment to property as a social institution requires us to take into account that the owners generally make up the group of people who will be most directly affected by any decision involving the future of their property. As such, they should normally be granted a decisive voice in decision-making processes leading up to economic development. At the same time, the social function account leaves room for recognising that this presumption in favour of emphasising the rights of owners can be defeated by the context. It is clear, for instance, that the substantive interests of absentee landlords might be limited compared to the substantive interests of local non-owners who depend more directly on the property for their livelihoods. In these cases, the social function approach allows us to recognise that a taking might have significant democratic merit, even if it is based on a form of decision-making that prioritises the interests and participation rights of non-owners.

Nevertheless, within a system based on private property rights it will always be appropriate to show caution in this regard. The presumption should always be, within such a system, that the owners are the primary stakeholders in decision-making processes involving their property. Moreover, if this presumption is defeated, it would usually point to a structural weakness of property's function within society, a weakness that should arguably be addressed by more general reforms of property, not by inflating the state's power to undermine it. If caution is not observed here, property can soon become a less secure basis on which to support local communities, including those marginalised groups that are most in need of protection from predators.

By itself, however, a legitimacy test cannot make property a more secure basis for promoting good outcomes in cases when the public desires economic development. What the extended Gray test provides is a list of possible symptoms to look our for when attempting to diagnose a suspected case of eminent domain abuse. Hopefully, this can help flag problems and limit the prevalence of abuse, but it can not be regarded as a solution, especially not in cases when the public's apparent desire for economic development is a genuine reflection of a democratic commitment.

In short, after diagnosing a lack of legitimacy, the question becomes how to find a cure. In the next section, I consider this challenge in more depth, premised on the idea that there is a need for alternatives to eminent domain in cases when the collective wishes to take decisive steps to promote economic development on privately owned land.\footnote{If this premise is rejected as too radical, an institutional fairness perspective on legitimacy can still inspire other kinds of reform proposals, e.g., in the context of good governance. For an example, consider the \cite{guide12}. These guidelines contain a section on expropriation which elaborates greatly on typical property clauses by going into more detail about the meaning of fairness, especially in relation to the expropriation procedure as such. Hence, the guidelines appear to embody a perspective on legitimacy that is quite close to the one I have put forth in this thesis. However, the question is whether a ``soft'' law mechanism, such as that of the Guidelines, will be strong enough to keep powerful commercial interests in check. For a more in-depth assessment of the expropriation provisions in the guidelines, see \cite{hoops15}.}

\section{Alternatives to Takings for Economic Development}\label{sec:3:6}

As mentioned briefly in chapter 2, the work of Ostrom and others on common pool resources suggests that sustainable resource management can often be better achieved through local self-governance than through markets or states.\footnote{See generally \cite{ostrom90}. For a recent exposition of the main ideas, placing the work in a broader academic context, see the revised version of Ostrom's Nobel Lecture, \cite{ostrom10}.} The connection between this work and property theory is highly interesting, and has been explored in some recent work, particularly by US legal scholars.\footnote{See generally \cite{rose11,fennel11}.} As these scholars have observed, the connection can be made at a very high level of generality. Indeed, in a democracy, property as such has a kind of (partial) commons structure, since property as an institution depends on the collective choices we make regarding the legal order.\footnote{For similar observations, see \cite[51]{rose90}; \cite[577]{heller01}.} In cases when property is made subject to eminent domain, this perspective becomes particularly salient, since then the collective explicitly withdraws its backing for the rights of the owner, in favour of collective decision-making about the future of the property in question.\footnote{A common pool resource is typically identified by the fact that exclusion is difficult or costly, while use can cause depletion (and hence should be limited), see, e.g., \cite[57]{ostrom10b}. Hence, the mere fact that some property is apt to be regulated, or taken, by the collective, demonstrates that property has common pool characteristics (although these might be imposed by the polity, rather than arising from the nature of the underlying good).} Moreover, in case of an economic development taking, the property in question typically pertains to land or some other natural resource, which invariably form part of a larger resource system with some common pool characteristics.\footnote{See, e.g., \cite[16]{fennel11} (``we are {\it always} operating at least partially within a commons of some sort.''). I also mention Smith's notion of a ``semicommons'', used to describe settings where common pool arrangements for resource management interact with individual property rights, see generally \cite{smith00,smith02}.}

Importantly, to designate something as a common pool resource does not in any way imply that the resource in question is open-access or that it is held as a form of common property, a public trust, or under some other legal construction moving away from the sphere of private property.\footnote{See, e.g., \cite[58]{ostrom10b}.} Perhaps more controversially, designating something as a common pool resource does not in any way imply that the resource {\it should} be removed from this sphere.\footnote{See \cite[58]{ostrom10b} (``there is no automatic association of common-pool resources with common-property regimes -- or, with any other particular type of property regime.'').} According to Ostrom and Hess, the appropriate property regime for a given common pool resource is a pragmatic question that depends on the circumstances.\footnote{See \cite[58]{ostrom10b}.}

It should be noted, however, that this neutral position on the relationship between property and common pool resources is premised on a bundle of rights understanding.\footnote{See \cite[59]{ostrom10b}.} Potentially, a more ambitious theory of property could suggest a different perspective. Specifically, the question arises as to how theories of common pool resource management relates to the social function account. This is a particularly interesting avenue for future work, as it could shed light on the normative stance that private property can be a good basis for sustainable self-governance, at least when backed up by a human flourishing account of what private property should be.

In this thesis, I limit myself to noting that the link between property and theories of commons governance provides a possible route towards an institutional perspective on how to solve legitimacy problems associated with economic development takings. To make progress in this regard, it will be useful to first briefly consider one of the most important theoretical legacies of Ostrom's work, namely a list of eight design principles that she formulated on the basis of empirical studies.\footnote{See \cite[90]{ostrom90}.} These principles were formulated because they seemed to be particularly crucial in ensuring good governance at the local level, and have since been supported by a growing body of empirical evidence.\footnote{See \cite{cox10} (the authors also suggest splitting some of the original principles in two parts, resulting in a slightly more fine-grained list, not needed in this thesis). } In brief, the so-called CPR (common pool resources) design principles are the following:

\begin{enumerate}
\item {\bf Well-defined boundaries:} There should be a clearly defined boundary around the resource in question, and a clear distinction should exist between members of the user community, who are entitled to access the resource, and non-members, who may be excluded. This will internalise the costs of resource exploitation and other externalities, ensuring that proper incentives for sustainable management arise within the community of resource users.\footnote{Importantly, the possibility of excluding non-members marks a distinction between open-access resources and common pool resources, where the latter appears much less susceptible to a commons tragedy than the former, because externalities are internalised to a clearly defined community. See \cite[91-92]{ostrom90}.}
\item {\bf Congruence between appropriation and provision rules and local conditions:} Management principles should be flexible and responsive to changing local conditions. Moreover, management practices should be anchored in the economic, social, and cultural practices prevalent at the local level. In addition, the individual benefits should generally exceed the individual costs associated with membership in the community of users, and collectively managed benefits should be distributed fairly among community members.\footnote{See \cite[92]{ostrom90}.}
\item {\bf Collective-choice arrangements:} The individual members of the user community should have an opportunity to participate in decision-making processes regarding the rules that govern the user community and the resource management. In addition to securing fairness and legitimacy, this will enhance the quality of the decision-making, as the users themselves have first-hand knowledge and low-cost access to information about their situation and the state of the resource in question.\footnote{See \cite[93]{ostrom90}.}
\item {\bf Monitoring:} There should be mechanisms in place to ensure that the behaviour of users is monitored for violations of management rules. To increase efficiency, monitoring should be locally organised. Moreover, to ensure local responsiveness and legitimacy, individuals acting as monitors should themselves be members of the user community or in some way answerable to this community.\footnote{See \cite[94-100]{ostrom90}.}
\item {\bf Graduated sanctions:} There should be an effective system in place for penalising violations of user community rules. These penalties should be graduated so that more severe or repeated violations are sanctioned more severely than minor or one-time transgressions.\footnote{See \cite[94-100]{ostrom90}.}
\item {\bf Conflict-resolution mechanisms:} The user community should be endowed with low-cost procedures for conflict resolution. These procedures should be sensitive to local conditions, to ensure local legitimacy.\footnote{See \cite[100-101]{ostrom90}.}
\item {\bf Minimum recognition of rights:} The user community should be protected from interference by external actors, including government agencies. As a minimum, the existence of local institutions and the right to self-governance should be recognised and respected by external government authorities.\footnote{See \cite[101]{ostrom90}.}
\item {\bf Nested enterprises:} There should be vertical integration between local, small-scale, management institutions and larger institutions aimed at protecting and furthering non-local interests. This integration should be based on the minimum recognition of rights mentioned in the previous point. Furthermore, it should provide a template for integrated decision-making about larger scale issues, where local competences are employed incrementally in more general settings, involving also institutions working on behalf of municipalities, regions, states and the international community. Local institutions for resource management should not only be respected by such larger scale structures, they should also feed into larger scale decision-making and be called to respond to greater community needs.\footnote{See \cite[101-102]{ostrom90}.}
\end{enumerate}

There are at least two interesting connections between the CPR principles and the issue of economic development takings. First, one may observe that when economic development takings appear to lack legitimacy with respect to social functions, this is typically also an indication that the surrounding framework for resource management is not well-designed. In particular, it appears that the extended Gray test closely tracks many of the design principles proposed by Ostrom.

For instance, consider the balance of power between the owners and beneficiaries of a taking, the first point to consider according to the Gray test. When a taking fails on this point, doubts naturally arise also with regard to the underlying framework for resource management, particularly aspects pertaining to the recognition of local rights, the adequacy of collective-choice arrangements, and the congruence between appropriation, provision and local conditions. If property is taken by powerful actors, chances are that these actors are not representative of community interests. Moreover, takings characterised by an imbalance of power typically indicate that the government is unwilling to respect the rights of local people, even when these rights are formally recognised as property rights.

By contrast, the situation might be different if it involves a taking that is not suspect according to the extended Gray test. For instance, if property is taken from absentee landlords and given to local land users in order to facilitate development, this might be an honest attempt at setting up a management framework that complies with CPR principles. In such a case, one would also not expect the balance of power between owners and takers to point towards abuse.

The second link between CPR design and economic development takings is arguably even more relevant. This link becomes apparent as soon as we shift attention away from diagnosing a lack of legitimacy towards coming up with alternative management principles that can restore it. Specifically, work done on local governance of common pool resources point to an {\it alternative} way of approaching the goal of economic development in cases that might otherwise result in the use of eminent domain. Specifically, one could instead try to design institutions for self-governance that can promote economic development in justified cases, without dispossessing local owners and their communities.

When thinking about how to design such institutions, it is important to notice an implicit tension in the CPR principles, between the right to self-governance and the need to integrate local decision-making into non-local institutional structures. As pointed out by much CPR research, finding the right balance can be a challenge, especially when building institutions for local decision-making in communities where such institutions have not formed naturally.\footnote{See \cite{saunders14} (reviewing empirical work on CPR design and criticising some policy makers and scholars for underestimating the risk of failure when trying to impose institutions for self-governance from the outside).} Finding the right balance between local autonomy and integration into the greater institutional order is important not only to protect the interests of the public, but also to protect minorities and weaker parties within the local community. A major concern associated with self-governance, especially if it is imposed by design, is that elites will capture the decision-making at the local level.

It is well-documented that elite capture can occur at {\it all} levels of institutional decision-making, from the local to the international stage.\footnote{See, e.g., \cite{bardhan00,platteau05,cullet13,levien13,mehta14}.} There is also ample evidence that self-governing communities {\it can} succeed in creating and sustaining healthy environments for collective decision-making; there is no reason to think that self-governance will necessarily lead to abuse by local elites.\footnote{See \cite[72-75]{andersson08} (with many further references to empirical studies).} Still, there is no automatic guarantee that decentralisation leads to better governance; decentralisation policies themselves can be captured. This points to the importance of setting up a framework for appropriate nesting of institutions, where legal guarantees serve as safeguards both for weaker groups and the greater public.

The question of how to design such a framework for institutions that can replace eminent domain for economic development in western democracies has not received much attention. One notable exception, discussed in depth in the following subsection, is the work of Heller, Hills, and Dagan.\footnote{The work of Lehavi and Licht also deserves a brief mention, even though it focuses on compensation rather than alternatives to eminent domain. The reason is that this work relies on proposing a novel institution that also touches on issues related to self-governance. In particular, Lehavi and Licht propose that collective price bargaining should be carried out on behalf of owners by a Special Purpose Development Company, in an effort to give them a chance to get a share of the commercial benefit arising from development, see \cite{lehavi07}.} Looking at their work will serve to make the abstract discussion above more concrete, and will set the stage for a comparison between their proposal and solutions that can be facilitated by the system of land consolidation presented in chapter \ref{chap:6}.

\subsection{Land Assembly Districts}\label{sec:3:6:1}

In an article from 2001, Heller and Dagan considered the connection between CPR design and overarching (liberal) property values.\footnote{See \cite{heller01}.} From this, they arrived at a proposal for what they call a ``liberal commons'', which adds some design constraints rooted in a desire to protect individual autonomy and minority rights. In particular, they emphasise the value of exit, the opportunity for rights holders to alienate their share in the commons resource (conceived of as a property right).\footnote{See \cite[567-572]{heller01}.} The right of owners to leave the collective is thought of as a safety mechanism, to prevent failing institutions from trapping its members in a state of oppression. This is argued to be an important overarching design constraint, described as a ``liberal'' idea, that should complement the other design principles for local management of common pool resources.\footnote{Despite their commitment to protect the right of exit, Heller and Dagan are also aware of the destabilising effect exit can have on an otherwise well-functioning institution. To address this, they discuss additional mechanisms, such as rights of first refusal, that can ensure that exit does not prove too disruptive to the local collective, as long as a sufficient number of members choose to remain. See \cite[596-702]{heller01}.}

In a later article, responding to the {\it Kelo} controversy, Heller and Hills build on the idea of the liberal commons by proposing a novel approach to the takings issue, consisting of a new institutional framework to facilitate land assembly for economic development. The key innovation is that of the {\it Land Assembly District} (LAD), an institution that is meant to be set up on demand, whenever the property owners in a specific area need to make a collective decision about whether or not to sell their land to a developer or a municipality.\footcite[1469-1470]{heller08} The idea is that while anyone will be able to propose and promote the formation of a LAD, the planning authorities and the owners themselves must consent before it is formed.\footcite[1488-1489]{heller08} Clearly, some kind of collective action mechanism is required to allow the owners to make such a decision. 

Heller and Hills suggest that voting under the majority rule will be adequate in this regard, at least in most cases.\footnote{See \cite[1496]{heller08}. However, when many of the owners are non-residents who only see their land as an investment, Heller and Hills note that it might be necessary to consider more complicated voting procedures, for instance by requiring separate majorities from different groups of owners, see \cite[1523-1524]{heller08}. For a criticism of the LAD proposal focusing on the shortcomings of majority voting, see \cite{kelly09}.} How to allocate voting rights in the LAD is given careful consideration, with Heller and Hills opting for the proposal that they should in principle be given to owners in proportion to their share in the land belonging to the LAD.\footnote{See \cite[1492]{heller08}. For a discussion of the constitutional one-person-one-vote principle and a more detailed argument in \isr{favour} of the property-based proposal, see \cite[1503-1507]{heller08}.} Owners can opt out of the LAD, but in this case, eminent domain can be used to transfer the land to the LAD using a conventional eminent domain procedure.\footcite[1496]{heller08}

Heller and Hills envision an important role for governmental planning agencies in approving, overseeing and facilitating the LAD process. Their role will be most important early on, in approving and spelling out the parameters within which the LAD is called to function.\footcite[1489-1491]{heller08} While it is not discussed at any length, the assumption appears to be that the planning authorities will define the scope of the LAD by specifying the nature of the development it can pursue in quite some depth. Hence, the powers of the planning authority appear likely to remain quite extensive.

If the owners do not agree to forming a LAD, or if they refuse to sell to any developer, Heller and Hills suggest that the government should be precluded from using eminent domain to assemble the land.\footcite[1491]{heller08} This is a crucial aspect of their proposal that sets the suggestion apart from other proposals for institutional reform that have appeared after {\it Kelo}. A LAD will not only ensure that the owners get to bargain with the developers over compensation, it will also give them an opportunity to refuse any development to go ahead. Hence, the proposal shifts the balance of power in economic development cases, giving owners a greater role also in preparing the decision whether or not to develop, and on what terms. Hence, the LAD proposal promises to address the democratic deficit of economic development takings, without failing to \isr{recognise} that the danger of holdouts is real and that institutions are needed to avoid it.

There are some problems with the model, however. For one, planning authorities might have an incentive to refuse granting approval for LAD formation. After all, doing so entails that they give up the power of eminent domain for the land in question. For this reason, Heller and Hills propose that a procedure of judicial review should exist whereby a decision to deny approval for LAD formation can be scrutinized.\footcite[1490]{heller08} However, the question then arises as to how deferential  courts should be in this regard, echoing the conundrum that engulfs the safeguard intended by the public use restriction. Presumably, one would want the courts to strictly scrutinise LAD rejections, to instil that LADs should normally be promoted. However, would the courts be comfortable providing such scrutiny, also against a government body claiming that the ``public interest'' speaks against LAD formation? This would likely depend on the exact formulation and spirit of the LAD-enabling legislation. To work as intended, some sort of presumption in favour of LAD approval appears to be in order, but this in turn can have the effect of making it easier for powerful landowners to abuse the LAD system, e.g., by pushing through LADs that enable them to impose their will on other community members.

This worry is related to a second possible objection against the LAD proposal, concerning the practicalities of the process leading up to the LAD's decision on whether or not to accept a given offer. Is it possible to organise such a process in a manner that is at once efficient, inclusive and informative, without making it too costly and time consuming? Here Heller and Hills envision a system of public hearings, possibly \isr{organised} by the planning authorities, where potential developers meet with owners and other interested parties to discuss plans for development.\footnote{See \cite[1490-1491]{heller08}. It might also be necessary for the planning authorities or other government agencies to take on some responsibilities with respect to providing guidance and assistance to less resourceful members among the owners.} The process envisioned here would resemble existing planning procedures to such an extent that additional costs could hopefully be kept at a minimum. 

The significant difference would concern the relative influence of the different actors, with the owners as a group receiving a considerable boost as a result of the LAD. Rather than being sidelined by a narrative that sees the use of eminent domain as the culmination of planning, the owners are now likely to occupy center stage throughout, as they now will have the final say on whether or not the development will go ahead.

This raises the question of how the interests of other locals, without property rights, will be protected. Heller and Hills assume that local non-owners will also be represented during the stages leading up to the LAD's final decision, but their role in the process is not clarified in any detail.\footcite[1490-1491]{heller08} This raises the worry that LADs might undermine local democracy by giving property owners a privileged position with respect to policy questions that should be decided jointly by all members of the community. 

If property rights are distributed evenly among community members, the risk of abuse in this regard might be limited. Moreover, the local anchoring that LADs provide should also benefit non-owners, by bringing the decision-making process closer to the people most directly affected, including non-owners. If some members of the local community remain marginalised, this should arguably be regarded as a regulatory failure or a reflection of underlying inequality in society, not a shortcoming of the LAD proposal. In these cases, a reasonable approach might even be to {\it expand} the function of LADs, by granting voting rights to a larger class of local property dependants, not only formally titled owners.%\footnote{The important invariant to maintain, I believe, is that the locally anchored institution should be the active, invested, agent, while more centralised and/or expert-dominated government bodies should act as passive, impartial, regulators. In the processes leading to economic development takings, this equation is typically reversed, with government bodies and commercial companies being the active agents, while the owners and the local community are the passive agents whose property rights and dependencies place some nominal limits on the authority of other parties (limits which, due to the weakness of owners as a group, tend to be easily disregarded).}

%However, the LAD proposal raises some problematic issues pertaining to the proposed mechanism of collective decision-making. As Kelly points out in a commentary, the idea of majority voting might be inherently flawed for decision-making about land assembly.\footcite{kelly09} For instance, if different owners value their property differently, majority voting will tend to \isr{disfavour} those with the strongest views, either in \isr{favour} of, or against, assembly. If these viewpoints are assumed to be non-strategic and genuine reflections of the welfare associated with the land, this can result in inefficiencies. In short, the problem is that a majority can often be found that does not take due account of minority interests. This is worrying, since it can undermine property's function as a means for minorities to protect and assert themselves on the basis of merit rather than voting power.

%For instance, if a minority of owners are planning development on their own land, and this conflicts with some LAD proposal targeting a larger area, the minority might find it difficult to defend themselves against the force of the LAD. Indeed, such a minority might effectively loose the battle for their property as soon as a LAD is formed, if the development description underlying LAD formation is incompatible with the kind of development they wish to pursue. For such owners, a presumption in favour of LAD formation might prove highly disadvantageous.\footnote{Of course, one might imagine these landowners opting out of the LAD, or pursuing their own interests independently of it. However, they are then unlikely to be better off than they would be in a no-LAD regime. In fact, it is easy to imagine that they could come to be further \isr{marginalised}, since the existence of the LAD, acting `on behalf of the owners', might detract from any dissenting voices on the owner-side.}
 
%Indeed, developers might come to rely on LADs to push through {\it de facto} condemnations of property, through a procedure that leaves minorities less protected than the traditional takings process. Indeed, it would be theoretically possible for any landowner to use a LAD to condemn any neighbouring property smaller than their own. Eventually, a whole community might be taken over by one or a few powerful landowners, through a sequence of appropriately designed LADs and development projects. %The government should prevent this, of course, but experiences with eminent domain for economic development illustrate that they might well fail in this regard.

The ideal of the LAD proposal is clearly stated and highly attractive. LADs should help to establish self-governance for land assembly and economic development. In particular, Heller and Hills argue that LADs should have ``broad discretion to choose any proposal to redevelop the \isr{neighbourhood} -- or reject all such proposals''.\footcite[See][1496]{heller08} As they put it, two of the main goals of LAD formation is to ensure ``preservation of the sense of individual autonomy implicit in the right of private property and preservation of the larger community's right to self-government''.\footcite[See][1498]{heller08} The problem is that these ideals turn out to be at odds with some of the concrete rules that Heller and Hills propose, particularly those aiming to ensure good governance of the LAD itself.

In relation to the governance issue, Heller and Hills emphasise, in direct contrast to their comments about ``broad discretion'' and ``self-governance'', that ``LADs exist for a single narrow purpose -- to consider whether to sell a neighborhood''.\footcite[See][1500]{heller08} This is a good thing, according to Heller and Hills, since it provides a safeguard against mismanagement, serving to prevent LADs from becoming battle grounds where different groups attempt to co-opt the community voice to further their own interests. As Heller and Hills put it, the narrow scope of LADs will ensure that ``all differences of interest based on the constituents' different activities and investments, therefore, merge into the single question: is the price offered by the assembler sufficient to induce the constituents to sell?''.\footcite[1500]{heller08}

This means that there is a significant internal tension in the LAD proposal, between the broad goal of self-governance on the one hand and the fear of \isr{neighbourhood} bickering, or even majority tyranny, on the other. Indeed, it is hard to see how LADs can at once have both a ``narrow purpose'' as well as enjoy ``broad discretion'' to choose between competing proposals for development. \noo{If broad discretion is granted to LADs, what prevents special interest groups among the landowners from promoting development projects that will be particularly \isr{favourable} to them, rather than to the landowners as a group? What is to prevent landowners from making behind-the-scene deals with \isr{favoured} developers at the expense of their \isr{neighbours}? It might be difficult to come up with general rules that prevent mechanisms of this kind, without also making substantive self-governance an impossibility.

Moreover, as Kelly points out in a commentary, the idea of majority voting might be inherently vulnerable to abuse when applied to  decision-making about land assembly.\footcite{kelly09} The problem is that if different owners value their property differently, majority voting will tend to \isr{disfavour} those with the strongest views, either in \isr{favour} of, or against, assembly. If these viewpoints are assumed to be non-strategic and genuine reflections of the welfare associated with the land, this can result in inefficiencies and abuse. In short, the problem is that a majority can often be found that does not take due account of minority interests. This is worrying, since it can undermine property's function as a means for minorities to protect and assert themselves on the basis of right rather than voting power.

This weakening of property can be enough to allow powerful groups to pursue nefarious objectives. Indeed, it is conceivable that developers might come to rely on LADs to push through {\it de facto} condemnations of property, through a procedure that leaves minorities less strongly protected than the traditional takings process. For instance, it would be theoretically possible for any landowner to use a LAD to cheaply condemn any neighbouring property that is smaller or less valuable (giving rise to fewer votes) than their own.\footnote{If the owners opt out, as they will be able to under the LAD proposal, it will be possible for the dominating owner to use ordinary eminent domain to complete the transaction, which might now have a veil of legitimacy over it due to the existence of a LAD.} Eventually, a whole community might be taken over by one or a few powerful landowners, through a sequence of appropriately designed LADs and development projects. %The government should prevent this, of course, but experiences with eminent domain for economic development illustrate that they might well fail in this regard.} 
} If a LAD is tightly regulated, required to offer the land on an open auction, and obliged to only look at the price, this will limit the risk of abuse. But it will not give owners broad discretion to consider the social functions of property when choosing among development \isr{proposals}. In my view, therefore, it is undesirable to restrict the operations of LADs in this way. It is easy to imagine cases where competing proposals, perhaps emerging from within the community of owners themselves, will be made in response to the formation of a LAD. Such proposals may involve novel solutions that are superior to the original development plans, in which case it is hard to see why they should be disregarded simply because they are less commercially attractive, or because the developer interested in pursuing such a proposal cannot offer the highest payment to the owners. In the end, the decision that the LAD makes concerns the future of the community as a whole. This is not an exercise in profit-maximization, and there are good reasons to believe that LAD regulation should encourage a broad perspective, not enforce a narrow one.

\noo{By contrast, Heller and Hills are quite determined that the degree of self-governance needs to be limited in favour of strict regulation to reduce the risk of LAD abuse. In particular, they argue that ``LAD-enabling legislation should require especially stringent disclosure requirements and bar any landowner from voting in a LAD if that landowner has any affiliation with the assembler''.\footcite{heller08} Here the notion of self-governance is made very thin indeed, as owners will effectively be barred from using LADs as a template for gaining the right to participate in development projects themselves.

Moreover, new questions arise. For one, what is meant by ``affiliation''? Say that a landowner happens to own shares in some of the companies proposing development. Should they then be barred from voting? If so, should they be barred from voting on all proposals, or just those involving companies in which they are a shareholder? If the answer is yes, how can this be justified? Would it not be easy to construe such a rule as discrimination against landowners who happen to own shares in development companies? On the other hand, if the landowner in question is allowed to vote on all other proposals, would it not be natural to suspect that their vote is biased against assembly that would benefit a competing company? Or what about the case when some of the landowners are employed by some of the development companies? Should such owners be barred from voting on proposals that could benefit their employers? This seems quite unfair as a general rule. But in some cases, employment relations could play a decisive factor in determining the outcome of a vote. This might happen, for instance, if a company proposes development in a \isr{neighbourhood} where it has a large number of employees. Heller and Hills give no clear answer to the questions arising in this regard, and at this point, their proposal has in a sense come full circle. Indeed, similarly to how courts today struggle with the ``public use'' requirement, it seems that the proposed ``affiliation'' criterion for depriving someone of their voting rights would provide a very shaky basis for judicial review.
}
How to best organise a LAD seems to remain an open problem. The challenge is to ensure that LADs deliver a real possibility of self-determination, while also ensuring good governance and protection against abuse. Hiller and Hills themselves point out that further work is needed in this regard, and that the proposal should be fine-tuned based on empirical work.\footnote{See \cite[1498]{heller08}.} Later in the thesis, I will address this challenge when I consider the Norwegian framework for land consolidation. This framework provides a sophisticated institutional embedding of many of the central ideas of LADs. In particular, I will discuss how Norwegian land consolidation can be employed in cases of economic development, and how it is increasingly used as an alternative to expropriation in cases of hydropower development. This will allow me to shed further light on the issues that are left open by Heller and Hills' important article.

\section{Conclusion}\label{sec:3:7}

The legitimacy issue is at the heart of this thesis. There are many ways of approaching it, catering to different ideas about the appropriate role that the courts should play in safeguarding private property. This chapter has tried to distil an approach that is particularly suited in cases when property is taken for economic development. 

This led to a proposal for an institutional fairness approach that combines procedural and substantive standards, to arrive at a template for assessing the democratic quality of the decision-making as such, not merely the outcome. This is appropriate because it helps address a key worry associated with an economic development taking: that the decision to take represents an abuse of power, reflecting badly on the institutions that gave rise to it.
%I argued that such an institutional approach to fairness has started developing at the ECtHR, as a consequence of its development of the pilot judgement framework for assessing cases that might indicate systemic problems at the state institutional level.

On this basis, the chapter went on to provide a possible heuristic for assessing the legitimacy of economic development takings. This heuristic was based on six legitimacy indicators provided by Gray, with three new ones added, based on the work done in this and the previous chapter. The resulting heuristic, the extended Gray test, should be able to identify cases of eminent domain abuse, particularly those that offend against social functions of property at the institutional level, rather than merely the financial entitlements of owners.

Testing for failure is only the first step towards increased legitimacy, to be followed up by proposals for structural improvements. The question is how to respect property and its social functions without giving up on the idea that the collective has an overarching responsibility to regulate property and its uses, in keeping with the principle of democracy. This chapter proposed looking to the work done by Elinor Ostrom and others on common pool resources. Specifically, some key design principles for local self-governance was presented, along with an argument that these could be used as a starting point for coming up with institutions to replace eminent domain for economic development.

Heller and Hills' proposal for Land Assembly Districts represents a first pass at such a solution to the legitimacy problem in the US. However, I argued that the proposal is marked by a severe tension between the overarching goal of self-governance and the need to prevent abuses of power at the local level. In the end, the proposal did not appear to deliver on the initial promise of self-governance, because there were simply too many limits placed on the authority of the local decision-makers.

Arguably, this points to the need for adapting a less abstract perspective on legitimacy, to encourage flexible mechanisms that can be adapted to the circumstances. Limitations and safeguards against abuse that might be reasonable in an inner city neighbourhood with many poor tenants might be entirely misplaced in a village of equally positioned home-owners. This insight also echoes one of the key design principles formulated by Ostrom, concerning the need to maintain congruence with local conditions. According to the empirical evidence available regarding common pool resource management, a one-size-fits-all approach to legitimacy at the local level is not going to work.\footnote{See \cite{cox10}.} In light of this, I believe the critical examination of Land Assembly Districts marked a natural end to this chapter, as well as to the theoretical part of this thesis as a whole.

%This observation marks the end of the first part of the thesis. In the next part, I will adopt a more concrete perspective, by considering takings of water and land rights for Norwegian hydropower. This will lead to an analysis of legitimacy of takings for this purpose along the lines of the Gray test, as well as a case study of Norwegian land consolidation as an alternative to eminent domain. In this way, the second part will aim to shed light on key aspects of the theory developed in the first part, while exploring further the idea that social functions run as a common thread through individual property rights.