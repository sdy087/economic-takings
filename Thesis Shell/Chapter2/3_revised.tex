\part{A Case Study of Norwegian Waterfalls}

\chapter{Norwegian Waterfalls and Hydropower}\label{chap:4}

\section{Introduction}\label{sec:4:1}

Norway is country of many mountains, fjords and rivers, where around 95 \% of the annual domestic electricity supply comes from hydropower.\footnote{See \cite{statistikk13}.} The right to harness energy from rivers, streams and waterfalls generally belongs to local landowners under a riparian system whereby many water rights are derived from ownership of a riverbed.\footnote{This arrangement is rooted in the first known legal sources in Norway, the so-called ``Gulating'' laws, thought to have been in force well before AD 1000. See \cite[111-112,120]{robberstad81}.}  Historically, waterfalls were very important to local communities, particularly as a source of power for grist and saw mills.\footnote{See \cite[121]{tvedt13}.} %%Indeed, the fact that peasants in Norway controlled local water resources can help explain why they were relatively free, both economically and socially, compared to many other places in Europe.\footnote{See \cite[121]{tvedt13}.}

Following the industrial revolution, local ownership and management came under increasing pressure. At the beginning, this pressure was exerted by private commercial interests, often foreign investors, who saw the industrial potential in hydropower and started speculating in Norwegian water resources.\footnote{See \cite[30-31]{nou04}.} Later, the pressure on local self-governance was exerted mainly by the government, following the introduction of new legislation to regulate the development of hydroelectric power.\footnote{See \cite[41-57]{thue96} (describing the  regulatory system set up during this time).} This legislation set up a system that gave highly preferential treatment to public utilities over private actors, including local owners.\footnote{See \cite[46]{thue96} (describing legislation introduced to promote public utilities, including new expropriation authorities directed at local owners of waterfall).} At first, the motivation behind this reform was to facilitate a decentralised form of government control, led by public utilities controlled by the municipality governments.\footnote{See \cite[44-47]{thue96}.} However, the hydroelectric sector underwent gradual centralisation, a process that gained momentum after the Second World War when the state itself assumed a leading role.\footnote{See \cite[59-85]{thue96}. For the history of the state's involvement with hydropower generally, see \cite{thue06, skjold06,thue06b}.} After this, local communities and local riparian owners became increasingly marginalised. In particular, local communities were systematically pressured into shutting down their hydroelectric plants, often as a condition for being granted access to electricity through the national, monopolised, electricity grid.\footnote{See \cite[p.111]{hindrum94}.}

Then, in the early 1990s, the electricity sector was liberalised, largely inspired by the market-orientation and privatisation of the public sector in the UK under Thatcher.\footnote{See generally \cite{midttun98}.} The production sector was decoupled from the grid sector, while public utilities were reorganised as commercial companies.\footnote{See \cite[86]{efta07} (describing how Norwegian electricity companies, most of which are still (partly) publicly owned, now operate as for-profit, limited liability companies).} At the same time, the regulatory system was decoupled from political decision-making processes, to become more expert-based.\footnote{See \cite[26-27]{brekke12}.} Moreover, the sector underwent additional centralisation, as a result of mergers and acquisitions among former public utilities.\footnote{See \cite[583]{bibow03}. I mention that despite significant continuous centralisation from the Second World War to this day, the Norwegian hydroelectric sector is still relatively decentralised compared to other countries, e.g., the UK, see \cite[181]{midttun98}. Arguably, this is a lasting influence of a tradition based on local, egalitarian, ownership of water resources.}

Following the reform, access rights to the national grid are meant to be granted equally to all potential actors on the energy market, including private companies.\footnote{See generally \cite{hammer96}. For an interesting presentation and analysis of grid-based markets in general, see \cite{falch04}.} After the passage of the \cite{ea90}, the energy companies that operate the national grid (the grid is divided into regions) are no longer authorised to shut out competitors.\footnote{See the \indexonly{ea90}\dni\cite[3-4]{ea90}.} A side-effect of this is that it has become possible for local landowners to undertake their own hydropower projects. Local owners can now access the grid to sell the electricity they produce on Nord Pool, the largest electrical energy market in Europe.\footnote{See generally \cite{larsen06,larsen08,larsen12}.} This has led to increased tension between local interests and established hydropower companies. The following fundamental question has arisen: who is entitled to benefit from rivers and waterfalls, and who is entitled to a say in decision-making processes concerning their use?

This chapter sets the stage for discussing this question in more depth, by detailing how the hydropower sector is organised. It looks both to the law and to commercial and administrative practices. Special attention is directed at those aspects that have changed following liberalisation, and which have resulted in conflicts involving property. The main goal is to show that the tension between large-scale development companies and local owners can only be understood on the basis of a social function perspective on riparian ownership. To set the stage for making this point, the chapter first provides a brief overview of the legal system, emphasising the role that private property has played in the development of Norwegian democracy.\footnote{The classic reference on Norwegian constitutional law is \cite{andenes06}.}

\section{Norway in a Nutshell}\label{sec:4:2}

\noo{ %\footnote{It should be noted that the executive branch also enjoys considerable legislative power under Norwegian law. Both informally, because it prepares new legislation, and also formally, because it has wide delegated powers to issue so-called {\it directives} (forskrifter). Indeed, it is typical for acts of parliament to include a general delegation rule which permits the executive to legislate further on the matters dealt with in the act, by clarifying and filling in the gaps left open by it.}

Norway is a constitutional monarchy, based on a representative system of government.\footnote{For Norwegian constitutional law generally, see \cite{andenes06}.} The executive branch is led by the King in Council, the Cabinet, headed by the Prime Minister. Legislative power is vested in the Storting, the Norwegian parliament, elected by popular vote in a multi-party setting. In 1884, the parliamentary system first triumphed in Norway, as the cabinet was forced to resign after it lost the confidence of parliament. The principle has since obtained the status of a constitutional custom. In particular, the cabinet can not continue to sit if parliament expresses mistrust against it. However, an express vote of confidence is not required. In practice, due to the multi-party nature of Norwegian politics, minority cabinets are quite common. These can sustain themselves by making long-term deals with supporting parties, or by looking for a majority on a case-by-case basis.

The judiciary is organised in three levels, with 70 district courts, 6 courts of appeal, and the Supreme Court. The district courts have general jurisdiction over most legal matters; there is no division between constitutional, administrative, civil, criminal courts. \footnote{However, there are distinct procedural rules for civil and criminal cases and a special court for land consolidation, see the \cite{lca79}. Moreover, both the district courts and the courts of appeal follow special procedural rules in appraisal disputes, for instance when compensation is awarded following expropriation, see the \cite{aa17}.} The courts of appeal have a similarly broad scope. Moreover, the right to appeal is ensured in most cases.\footnote{The right to an appeal is not absolute. In civil cases, it is generally required that the stakes are above a certain lower threshold, measured in terms of the appellants' financial interest in the outcome. See \indexonly{cda05}\dni\cite[29-13]{cda05}.} The Supreme Court, on the other hand, operates a very strict restriction on the appeals it will allow.\footnote{See the \indexonly{cda05}\dni\cite[30-4]{cda05}.} It typically only hears cases if a matter of principle is at stake, or if the law is thought to be in need of clarification.\footnote{See, generally, \cite{skoghoy08}.}
}
The Norwegian legal system is often said to be based on a special ``Scandinavian'' variety of civil law, which includes strong common law elements: legislation is not as detailed as elsewhere in continental Europe, some legal areas lack a firm legislative basis, it is generally accepted that courts develop the law, and the opinions of the Supreme Court are often of crucial importance when the lower courts interpret and apply legislation.\footnote{See, generally, \cite{bernitz07}.} In this regard, it should be noted that the Supreme Court operates a very strict restriction on the leave to appeal.\footnote{See the \indexonly{cda05}\dni\cite[30-4]{cda05}.} It typically only hears cases if a matter of principle is at stake, or if the law is thought to be in need of clarification.\footnote{See, generally, \cite{skoghoy08}.} Moreover, legislation remains the primary source used to resolve most legal disputes. When applying it, the courts tend to place great weight on preparatory documents procured by the executive branch. These documents are widely regarded as expressions of legislative intent, even though Parliament is not usually involved in their preparation.

The Constitution of Norway dates back to 1814 and was heavily influenced by then recent political movements, particularly in the US and France.\footnote{See generally \cite{mestad14}.} Moreover, it was influenced by a desire for self-determination, as Norway was at that time a part of Denmark-Norway, largely controlled by the Danish elite. Following the Napoleonic wars, Norwegian politicians sought to take advantage of Denmark's weakened position to gain independence and they drafted the Constitution with this objective in mind. In the end, Norway was forced to enter into a union with Sweden (who was backed by the winning side of the Napoleonic wars), but the Constitution remained in place. Moreover, after the triumph of the parliamentary system in 1884, Norway would also eventually gain independence, in 1905, following a peaceful and democratic transition process.\footnote{See generally \cite{sejersted15}.}

During the 19th century, farmers and smallholders emerged as a powerful group in Norwegian politics. This was in large part due to the fact that they were also landowners, whose rights and contributions were not limited to traditional farming.\footnote{See generally \cite{hommerstad14}. The ``classic'' presentation of the political influence of farmers in Norway is \cite{koht26}.} This had not always been the case; during the middle ages, the Norwegian farmer had usually been a tenant.\footnote{See generally \cite{myking05}.} However, tenant farmers always enjoyed a significant degree of control over the management of the land and its natural resources.\footnote{See \cite[59-60]{pryser99}; \cite[226-238]{myking05}.} Moreover, between the 17th and the end of the 18th century, almost all Norwegian tenant farmers bought their land from their landlords.\footnote{See, e.g., \cite[108]{nordtveit15}.} As a result, the distribution of land ownership in Norway had  become highly egalitarian at the time of the Constitution.

In the years that followed, the landed nobility in Norway was further marginalised. The Constitution itself prohibited the establishment of new noble titles and estates.\indexonly{grunnloven14}\dni\footcite[23|118]{grunnloven14} Then, in 1821, all hereditary titles were abolished (although existing nobles kept their titles for their lifetimes).\footnote{See \cite{adel09}.} By the middle of the 19th century, farmers and smallholders had gained significant political influence. In fact, they emerged as the leading political class, alongside the city bureaucrats.\footnote{See generally \cite{hommerstad14}.} During this time, Norway also introduced a system of powerful local municipalities. These were organised as representative democracies, becoming miniature versions of the cherished, as of yet unfulfilled, nation state (Norway was still in a union with Sweden at this time). Even today, municipalities retain a great deal of power in Norway, particular in relation to land use planning.\footnote{They are the primary decision-makers for spatial planning, as pursuant to \cite{pb08}.} They represent a highly decentralised political structure, with a total of 428 municipalities as of 01 January 2013.\footnote{This is down from the all-time high of 747 in 1930. There have long been proposals to reduce the number of municipalities further, but so far the political resistance against this has prevented major reforms. See \cite{kommuner14} (report to the Ministry from an expert committee on municipality reform, 2014).} %Enjoying private ownership in common is not unusual in Norway, particularly in rural areas, and the law of property in Norway reflects this in various ways.

There can be no doubt that the egalitarian distribution of property rights found in Norway was crucial to the development of a democratic economic and political order, especially in rural parts of the country.\footnote{For a comparative discussion, focusing on how egalitarianism influenced the industrialisation process in Norway, setting it apart from the industrialisation process in the UK, see \cite{brox13}.} Moreover, landownership itself was never understood in purely individualistic terms, but rather as an important building block of local communities. A reflection of this is the fact that outfields in Norway tend to be held under a form of common ownership, whereby each smallholding in the local community owns a share in the surrounding land and its resources.

This type of co-ownership has no exact common law equivalent, but is most similar to the tenancy in common.\footnote{However, there is no requirement that the co-ownership takes place behind a trust -- all individual shareholders are formally registered as owners of their share of the land and there is a presumption in favour of continued co-ownership accompanied by collective productive use of the land, not alienation or individuation.} In the \cite{coa65}, further rules are given to regulate the relationship between the co-owners and their use of the property they share. The main principle is that each owner has a right to the ``normal'' enjoyment of the property, determined by looking at the natural conditions, the local customs, and the original purpose of the co-ownership arrangement (if it is known).\footnote{See \cite[4]{coa65}.} Moreover, an individual owner's use must not exceed what corresponds to their share of the property and must not be unduly burdensome to the other owners. If damage occurs, compensation must be paid.

To some extent, the majority shareholders can enforce a specific use of the property against the will of a minority.\footnote{See \cite[4]{coa65}.} This includes new forms of commercial activity, including activities requiring additional investments in the property. If such activities are organised against the will of a minority, the minority will still be entitled to take part in the enterprise.\footnote{See \cite[5]{coa65}.} There are limits to what the majority can do; they can only order uses for which the property is deemed ``suitable'', and they cannot do anything to dramatically change the character of the property, sell it, or use it as security for debt.\footnote{See \cite[4]{coa65} paras 1 and 3.} Moreover, if the majority does something to interfere with the use rights of a minority shareholder, compensation must be paid.\footnote{See \cite[4]{coa65} para 4.} Because of these restrictions, gridlock can often result if the owners disagree fundamentally about how to manage their land.

For real property, particularly in rural areas, the standard way of resolving conflicts among co-owners is to bring a case before the Norwegian land consolidation courts. These courts are empowered to dissolve systems of co-ownership (if certain conditions are met), but they can also be used to organise joint use of the land, to avoid dissolution. The prevalence of common ownership over outfields means that land consolidation courts are important in rural Norway, and it also explains why these courts have been granted such wide powers to help organise the use of privately owned land. I return to the details of this in chapter \ref{chap:6}, as part of a broader discussion on how the institute of land consolidation can be used as an alternative to eminent domain in economic development situations.

In addition to the form of co-ownership regulated in the \cite{coa65}, there are two other special forms of ownership of land found in Norway that should be briefly mentioned. Both pertain to land over which a large group of people enjoy extensive rights of use that have been recognised as so-called ``almenningsretter'' (common rights) under Norwegian land law. Land to which common rights attach will also have an {\it in rem} owner in the private law sense, but special rules are in place to protect the group of people who enjoy common rights. These rules presuppose that the land is owned either by the state or a council of the local community.\footnote{Historically, the state was not considered the owner of the commons in the private law sense of the word, but rather as a custodian with special regulatory powers. However, perceptions of this changed over time, with the Supreme Court finally declaring with full generality that the state was to be considered the owner of the state commons in the full private law sense of ownership, see \cite{vinstra63}.} If the owner of common land is the state, the primary legislation that protects the rights of the local people is the \cite{ma75}. If the land is owned by a local community, the relevant legislation is the \cite{vca92}. The details differ, but the main purpose of both Acts is to offer special protection for rights in common, especially with regard to traditional land uses that local farmers depend on for their livelihoods.\footnote{For further details on the commons in Norwegian law, see \cite{stenseth05}.} There is no extant concept of a commons in Norway that attaches to land owned by private individuals.\footnote{Traditionally, there was also a concept of a ``private commons'', but this has all but disappeared from the law since the land in question has typically been transformed into co-owned private land or a village commons. The courts might still occasionally derive usage rights over private land from earlier rights in common over that land, but these use rights would be unlikely to receive legal recognition as specially protected commons rights.} This can partly be explained by the fact that private landlords and tenant farming is absent from the structure of rural landownership and has been for quite some time.

Local control over water resources, ensured through property rights, has always been very important to farmers and rural communities in Norway. According to Terje Tvedt, 10 000-30 000 mills were in operation in Norway in the 1830s.\footnote{See \cite[121]{tvedt13}.} As Tvedt argues, the fact that these mills were under local control was particularly important because it helped ensure self-sufficiency. In addition, saw mills became an important source of extra income for Norwegian farming communities. In the introduction, I already highlighted the concept of a waterfall right, used to refer to the right to harness power from a river, an historically important stick in the property bundle associated with landownership in Norway. Moreover, I mentioned briefly that waterfall rights are usually held in common by members of the local population. In some cases, this is because a river suitable for hydropower development runs across many distinct private properties. Hence, the relevant waterfall rights are held in common in the narrow sense that an assembly of private rights is required in order for development to take place. However, in most cases, waterfalls suitable for hydropower development will be owned in common in the stronger sense of arising from co-ownership of the surrounding land. %In these cases, it is also typically appropriate to identify the group of owners with the local community, which tend to be farming communities where each smallholding will tend to have a proprietary stake in the local river.

%Indeed, outfields in Norway are often held under a specific form of co-ownership, 
After the industrial revolution, there was some doubt as to whether rights in common over land could serve to give non-owners a claim to waterfall rights, or whether waterfall rights were held exclusively by the landowners. This question was particularly important for land owned by the state, since common rights would be the only potential route for local community members to claim a proprietary stake in local hydropower resources.\footnote{In village commons, by contrast, the owners themselves are also local community members, although the group of owners is typically smaller than the group of people who have use rights in common.} The question was settled by the Supreme Court in the case of {\it Vinstra} in 1963.\footnote{See \cite{vinstra63}.} Here the Court held that no rights to waterfalls in state commons could be derived from communal use rights over that land. It follows that the takings issue does not arise with respect to hydropower development on commons land, at least not with respect to the waterfall rights as such. For this reason, the Norwegian framework for regulating common rights will not be considered further in this thesis.%Questions that arise specifically with respect to common rights are not practicallywill not be dealt with in at any length in this thesis.%\footnote{When we consider the case of {\it Alta} in Chapter \ref{chap:5}, we will encounter state-owned land where the aboriginal Sami population has claimed to enjoy rights in property similar to common rights. In recent years, this claim has met with some recognition within the Norwegian legal order, giving rise to yet another form of property in Norway. For further details, see the discussion in Chapter \ref{chap:5} section \ref{chap:5:x}.} In most cases, conflicts between local communities and energy companies take place in a setting where the local population have full property rights, as co-owners, over the natural resource in question. An important exception to this is the {\it Alta} case, pertaining to a hydropower project in the north of Norway, with a significant population of Sami people, an indigenous minor was decided, members of the Sami population were considered as rights holders in the traditional private law sense of the word, no different from non-aboriginal holders of property and use rights elsewhere in Norway.\footnote{See the...} 

In chapter \ref{chap:5}, I will discuss the {\it Alta} case, which involved the special property regime found in the north of Norway.\footnote{\cite{alta82}.} The case arose in the 1970s after the national government had decided to build a hydropower project in Finnmark, a part of Norway where the state is traditionally regarded as the owner of all outfields. The state's ownership of land in this region tends to be at odds with the interests of the Sami people, an indigenous group from northern Scandinavia. Traditionally, the state's ownership was consider to be standard private law ownership, more or less entirely unencumbered by indigenous interests, except when Sami rights had been explicitly recognised by the state. In particular, the use rights of the Sami people did not enjoy the protected status given to rights in common over state land elsewhere in Norway.%\footnote{Some scholars disputed this, by arguing that Sami rights should be viewed as common rights by analogy with the legislation in place for state commons.}

Hence, in the {\it Alta} case, the dispute revolved around expressly recognised use and property rights that would be negatively affected by the development. Communal claims based on indigenous rights were summarily rejected, and the case did not involve takings of waterfalls (since the state already held these rights). Still, {\it Alta} has later been considered an important precedent for disputes surrounding expropriation of waterfalls, since it dealt with many aspects of administrative law pertaining to the licensing procedure surrounding hydropower development. As discussed in chapter 5, the case also marked a watershed moment in the legal history of the Sami people, whose rights over land in Finnmark have since received greater recognition within the Norwegian legal order. Today, in the special context of Sami land, the law appears to be moving towards a framework where the Norwegian state is increasingly seen as a custodian of Sami lands, rather than an owner in the standard private law sense.\footnote{See generally \cite{ravna05,bull07,ravna12s}.}

In other parts of Norway, a similar perspective has yet to develop. Natural resources owned by the state, or taken by it under eminent domain, has the same legal status as private property. There is no legal basis for regarding the state as a custodian, and there is no legally enforcible sense in which land owned by the state is held in trust on behalf of the people. However, in a recent revision of the Constitution, a new section 112 was introduced that compels the government to preserve the environment and promote sustainability. The exact wording is as follows:

\begin{quote}
Every person has the right to an environment that is conducive to
health and to a natural environment whose productivity and diversity
are maintained. Natural resources shall be managed on the basis of
comprehensive long-term considerations which will safeguard this
right for future generations as well. \\ \\

In order to safeguard their right in accordance with the foregoing
paragraph, citizens are entitled to information on the state of the
natural environment and on the effects of any encroachment on nature
that is planned or carried out. The authorities of the state shall take measures for the
implementation of these principles.
\end{quote}\footnote{See \cite[112]{grunnloven14}.}

This provision replaces a sustainability clause that was first introduced in the Constitution in 1992, following the influential Rio summit at the United Nations. This clause left little or no impact on Norwegian law, with no consequences discernible at all within the law of property.\footnote{See generally \cite{fauchald07}.} No one, to my knowledge, has proposed to read section 112 as having any direct bearing on the status of the state as a landowner; the standard private law notion of ownership still applies. However, section 112 is meant to give rise to a general obligation to promote sustainability through regulation, meaning that regulatory failures could conceivably be challenged under the provision. So far, however, few challenges of this kind have appeared and none have been successful. %Indeed, it has been argued that the constitutional sustainability provision as such has been a failure.

After the new formulation was introduced in 2014, there have been some indications that the legal status of the sustainability provision might be about to change. Indeed, the legislator itself expressed a desire to make the provision more easily justiciable.\footnote{See \cite[246]{dok16}. As an example of where this might lead in the future, I mention that a group of Norwegian environmental lawyers are presently considering the possibility of class action against the Norwegian state for not doing enough to fight climate change, using section 112 as a legal basis. See \cite{gjengedal15}.} However, there have been no indication so far that section 112 will become relevant in the law of hydropower. Moreover, it seems highly unlikely that the section will attain relevance with respect to the issue of expropriation. For this reason, section 112 will not be examined further in this thesis.\footnote{Of course, a normative argument could well be made that the provision {\it should} entail greater regard for the interests and property rights of local people. Such an argument might perhaps also be backed up by considerations based on international environmental law. Further exploration of economic development takings from this angle will be left for future work.}

%For conservation issues pertaining specifically to water resources, the situation with respect to section 112 is slightly different. Here it seems that the sustainability clause could eventually become a justiciable standard that will restrict the authority of the water authorities to permit development. 
Although the sustainability clause in the Constitution is of marginal importance in the law of hydropower, environmental interests are quite well protected during the licensing procedure in hydropower cases. The rules and practices that apply in this regard are presented in more depth in section \ref{sec:4:3} below. Before looking at the details, it should be mentioned that Norway has implemented the Water Framework Directive of the European Union.\footnote{See \cite{water00}. It has been implemented in Norwegian law as the Directive Regarding Frameworks for Water Management, FOR-2006-12-15-1446.} This directive attempts to ensure that water resources are managed according to an ecosystem perspective where the regulator takes an holistic approach to planning in water systems. Such a perspective is at odds with the sector-based approach that still dominates in Norwegian water law. Some have argued that the Norwegian implementation of the directive has not sufficiently recognized the need for structural reforms.\footnote{See \cite{hanssen14}.} However, the holistic perspective on water resources and sustainability now appears to be gaining ground, especially at the political level. As we will see towards the end of this chapter, the effect of this on local communities and owners have been mixed. In some cases, an increased emphasis on conservation and planning has had a negative effect on communities since it further inflates the power of those that have enough political and financial capital to exert their influence on centralised planning processes. %This includes commercial development interests, who now often appear to be cooperating with environmental interests in an effort to gain control over Norwegian water resources.

To understand how water law works in Norway, it is important to keep in mind that the importance of water does not primarily arise from the fact that water is scarce, but mainly from the fact that it is so plentiful. Not only is water power the main source of domestic energy, it also occupies a special place in Norwegian culture. It is important to the identity of many communities, particularly in the western part of the country, where majestic waterfalls are considered symbols both of the hardship of the natural conditions and the sturdiness of local people. The implications for the tourism industry are also significant, as the natural environment attracts visitors and economic activity to regions of Norway that are otherwise threatened by stagnation and depopulation.

In light of this, it is not surprising that there is a tradition in Norway for local resistance against development projects that are considered damaging to the environment. In the 1960s and 70s, when the state embarked on their most ambitious projects, local environmental movements became nationally significant, as symbols of resistance against centralisation, exploitation of weaker groups, and environmental destruction.\footnote{See \cite{nilsen08}.}

This illustrates that water resources are embedded in the social fabric in such a way that a purely entitlements-based approach to property rights in these resources would be largely inappropriate. Rather, the case of Norwegian water seems to be well-suited for an investigation based on a social function view on property. As I show in this and the following two chapters, rivers and waterfalls serve to bring out tensions between rights and obligations in property, while also shedding light on the question of how to organise decision-making processes regarding economic development.

In the next section, I argue that the present law on hydropower in Norway tends to recognise only a small part of the relevant picture. On the one hand, it recognises the financial entitlements of individual owners, which it tries to balance against the regulatory needs of the state. But it largely fails to take into account that owners have broader interests, even obligations, relating to the sustainable management of their streams and their waterfalls. Moreover, the law appears largely unable to prevent commercial companies and special interest groups from exerting a strong pull on various state bodies, particularly those that are only weakly grounded in processes of democratic decision-making.

\section{Hydropower in the Law}\label{sec:4:3}

As mentioned in chapter \ref{chap:1}, the right to harness power from a river is regarded as a separate unit of private property in Norway, referred to as a waterfall.\footnote{Historically, the law emphasised ownership of traditional agrarian water resources, such as fishing rights. However, new sticks were added to the waterfall bundle over the years, including the right to develop hydropower, see \cite[14-32]{vislie44}. See also \cite[108]{nordtveit15}. For a detailed presentation of the history of water law in pre-industrial times, I refer to \cite{motzfeld08}.} The system is riparian, so by default, a waterfall belongs to the owner of the land over which the water flows.\footnote{See the \indexonly{wra00}\dni\cite[13]{wra00}.} The landowners do not own the water as such -- freely running water is not subject to ownership -- and the riparian owners' right to withhold or divert water is limited.\footnote{See the \indexonly{wra00}\dni\cite[8|15]{wra00}.} 

%However, as mentioned in chapter \ref{chap:1}, the right to the hydropower in a river is considered to be a separate property right in the bundle of rights typically held by riparian owners, referred to as the right to the waterfall.%, a terminology I will also adopt-\footnote{The Norwegian term `fall' has a somewhat broader meaning than its English counterpart, `waterfall'. The word `fall' is used to describe a continuous section of any stream or river, typically identified by giving the total difference in altitude over the relevant stretch of riverbed. Furthermore, the Norwegian term `falleier' refers to a legal person who possesses the rights to the hydropower over such a section. In this thesis, I will typically refer to the owners of waterfalls, streams and rivers with the intended reading being the same as the Norwegian notion of a `falleier'. If special qualification is needed, for instance to distinguish between different classes of riparian owners, I will make a note of this explicitly.}

However, the waterfall owners have the exclusive right to harness the potential energy in the water over the stretch of riverbed belonging to them. This right can be partitioned off from rights in the surrounding land, and large-scale hydropower schemes typically involve such a separation of water rights from land rights; the energy company acquires the right to harness the energy, while the local landowners retain ownership of the surrounding land.

Norwegian rivers, and especially rivers suitable for hydropower schemes, tend to run across outfields that are owned jointly by local farmers. Hence, rights to streams and waterfalls are typically held among several members of the rural community.\footnote{Rivers tend to run through land that has not to been enclosed. Moreover, in places where there has been a land enclosure, water rights are often explicitly left out, such that they are still considered jointly owned rights belonging to the community of local farmers. For more details on (forms of) joint ownership among Norwegian farmers, see, e.g., \cite[570]{stenseth07a}.} Local owners might not be willing to give up their ownership to facilitate development, especially not on terms proposed by external developers. Hence, the authority to expropriate has been an important legal instrument for state-backed hydropower companies. It has also been used extensively, particularly after the state made hydropower development a priority following the Second World War.

This has resulted in a tension where, on the one hand, rights to harness hydropower from streams and waterfalls are considered private property, while on the other hand, it has become common to speak of hydropower as a resource belonging to the public. Since the \cite{ica17} was amended in 2008, this ambivalence in the discourse surrounding hydropower has also been part of the statutory provisions regulating hydropower development. I quote the two relevant sections side by side below:%\footnote{The first quote is taken from the general water law, with roots going back a thousand years to the so-called ``Gulating'' laws mentioned in the introduction. The second quote is taken from a law directed specifically at large-scale hydropower, introduced during the early days of the hydropower industry.}

{\begin{minipage}[t]{16em}
 \begin{aquote}{\tiny \indexonly{wra00}\dni\cite[13]{wra00}} \footnotesize A river system belongs to the owner of the land it covers, unless otherwise dictated by special legal status. [...]

The owners on each side of a river system have equal rights in exploiting its hydropower.
\end{aquote}  
\end{minipage}}
{\begin{minipage}[t]{22em}
\begin{aquote}{\tiny \indexonly{ica17}\dni\cite[1]{ica17} (after amendment in 2008)} \footnotesize Norwegian water resources belong to the general public and are to be managed in their interest. This is to be ensured by public ownership.
\end{aquote}
\end{minipage}} \\

The intended reading of section 1 of the \cite{ica17}, quoted on the right above, is that it provides a ``general starting point''.\footnote{See \cite[72]{otprp61}.} According to the Ministry, it expresses what has always been the purpose of the legislation used to regulate large-scale hydropower.\footnote{See \cite[72]{otprp61}.} 

This should not be understood as an explicit attack on the principle of private ownership expressed in section 13 of the \cite{wra00}, quoted on the left above.\footnote{There are no indications in the preparatory materials that the Ministry sought to confront the principles of ownership encoded in the \cite{wra00}.} However, the Ministry's comment underscores the extent to which the government regards it as natural to interfere with private rights to waterfalls, to pursue policies that it regards to be in the interest of the public. Taken in this light, section 1 of the \cite{ica17} reflects the prevailing opinion that there are few, if any, recognised limits on the state's power to manage privately owned water resources.\footnote{For a reflection of the same attitude, citing the state's broad regulatory competence as the main reason not to nationalise Norwegian water power rights, I refer to the preparatory documents underlying the \cite{wra00}. See \cite[152-153]{nou94}.}

This aspect of the Norwegian system has become particularly significant following the liberalisation of the electricity sector in the early 1990s.\footnote{See, e.g., \cite{larsen06}.} Since then, there have been an increasing number of cases where owners who are interested in undertaking their own development schemes attempt to fend off commercial energy companies wishing to expropriate.\footnote{See, e.g., \cite{sofienlund07}.} Importantly, the state has tended to side with the commercial companies in these cases, granting them the authority to expropriate for economic development. This has resulted in several Supreme Court decisions on hydropower and expropriation in the past few years, all of which have been in favour of the energy companies.\footnote{See  \cite{uleberg08,jorpeland11,klovtveit11,otra13}.} Before discussing these cases in more detail in the next chapter, I provide an in-depth analysis of hydropower in the law and in practice, to shed further light on the underlying conflict that has led to the recent surge in cases on expropriation. First, I briefly present the key legislation regulating the hydropower sector.

\subsection{The Water Resources Act}\label{sec:wra00}

The \cite{wra00} contains the basic rules regarding water management in Norway. This Act is not only concerned with hydropower, but regulates the use of river systems and groundwater generally.\footnote{See the \indexonly{wra00}\dni\cite[1]{wra00}. A river system is defined as ``all stagnant or flowing surface water with a perennial flow, with appurtenant bottom and banks up to the highest ordinary floodwater level'', see the \indexonly{wra00}\dni\cite[2]{wra00}. Artificial watercourses with a perennial flow are also covered (excluding pipelines and tunnels), along with artificial reservoirs, in so far as they are directly connected to groundwater or a river system, see the \indexonly{wra00}\dni\cite[2a, 2b]{wra00}.} In section 8, the Act sets out the basic license requirement for anyone wishing to undertake measures in a river system.\footnote{Measures in a river system are defined as interventions that ``by their nature are apt to affect the rate of flow, water level, the bed of a river or direction or speed of the current or the physical or chemical water quality in a manner other than by pollution'', see the \indexonly{wra00}\dni\cite[3a]{wra00}.} The main rule is that if such measures may be of ``appreciable harm or nuisance'', then a license is required.\footnote{See the \indexonly{wra00}\dni\cite[8]{wra00}. There are two exceptions, concerning measures to restore the course or depth of a river, and concerning the landowner's reasonable use of water for his permanent household or domestic animals, see the \indexonly{wra00}\dni\cite[12|15]{wra00}.} The water authorities themselves decide if this condition is met.\footnote{See the \indexonly{wra00}\dni\cite[18]{wra00}.} In relation to hydropower development, it is established practice that most hydropower projects over 1000 KW will be deemed to require a license.\footnote{See, e.g., \cite{nve09}. Exceptions are possible, for instance projects that upgrade existing plants, or which utilise water flowing between artificial reservoirs.}

The basic assessment criterion is that a license ``may be granted only if the benefits of the measure outweigh the harm and nuisances to public and private interests affected in the river system or catchment area''.\footnote{See the \indexonly{wra00}\dni\cite[25]{wra00}.} Hence, the water authorities are empowered to decide whether a licence {\it should} be granted, if they find that the benefits outweigh the harms. The courts are very reluctant to censor the discretion of the administrative decision-makers on this point.\footnote{This is an expression of the principle of ``freedom of discretion'' ({\it det frie forvaltningsskjønn}) for the administrative branch, a fundamental tenet of Norwegian administrative law. See generally \cite[71-74]{eckhoff14}.}

The Ministry of Petroleum and Energy maintains indirect control over the assessment process by issuing directives regarding the administrative procedure in licensing cases.\footnote{See section 65 of the \cite{wra00}.} In addition, the procedure is determined in large part by administrative practices developed by the water authorities themselves. A further discussion of the licensing procedure is given in section \ref{sec:4:3:1}. 

\noo{In principle, many of the rules in the \cite{paa67} apply at the administrative stage. However, these rules are of limited practical relevance compared to sector-specific practices. This has raised controversy in recent years, particularly in cases involving expropriation, as discussed in chapter \ref{chap:5}, section \ref{sec:5:6}. 

A few basic procedural rules are encoded directly in the \cite{wra00}. This includes rules to ensure that the application is sufficiently documented, so that the authorities have enough information to assess its merits.\footnote{See the \indexonly{wra00}\dni\cite[23]{wra00}.} Moreover, a basic publication requirement is expressed, stating that applications are public documents and that the applicant is responsible for giving public notice. The intention is that interested parties should be given an opportunity to comment on the plans.\footnote{See the \indexonly{wra00}\dni\cite[24]{wra00}. There are some exceptions to the requirement to give public notice, however. It may be dropped in case it appears superfluous, or if the application must be rejected or postponed, see the \indexonly{wra00}\dni\cite{wra00} ss 24a--24c.} More detailed rules for public notice of applications are given in section 27-1 of the \cite{pb08}, which also applies to licensing applications under section 8 of the \cite{wra00}.\footnote{In addition, I mention section 22, which regulates the relationship between licensing and planning in relation to water resources. In essence, the section stipulates that the water authorities may prioritise planning over assessment of individual licensing cases, e.g., by refusing to take applications under consideration if they interfere with ongoing planning procedures. However, the section leaves significant room for discretion in this regard. It also bears noting that watercourse planning is placed under centralised government control. This contrasts with land use planning in general, which is mainly the responsibility of the local municipality governments. See generally \cite{sp}.} }

\noo{Furthermore, an important rule of principle is given in section 22, regarding the relationship between applications for licenses and governmental ``master plans'' for the use or protection of river systems in a greater area. These plans have no clear legislative basis, but were introduced through parliamentary action in the 1980s, when the parliament decided to initiate such planning in an effort to introduce a more holistic basis for assessment of licensing applications.\footnote{Today, the planning authority is delegated to the Directorate of Natural Preservation and the NVE. See \cite{sp}.}
 
According to section 22 of the \cite{wra00}, if a river system falls within the scope of a master plan that is under preparation, an application to undertake measures in this river system may be delayed or rejected without further consideration.\footnote{See the \indexonly{wra00}\dni\cite[22]{wra00}, para 1.} Moreover, a license may only be granted if the measure is without appreciable importance to the plan.\footnote{See the \indexonly{wra00}\dni\cite[22]{wra00}, para 1.} In addition, once a plan has been completed, the processing of applications is to be based on it, meaning that an application which is at odds with some master plan may be rejected without further consideration.\footnote{See the \indexonly{wra00}\dni\cite[22]{wra00}, para 2.} It is still possible to obtain a license for such a project, but if it harnesses less hydropower than the project indicated by the plan, section 22 states that only the Ministry may grant it.\footnote{See the \indexonly{wra00}\dni\cite[22]{wra00}, para 2.}
}

The rules in the \cite{wra00} apply to any measures in river systems, not only hydropower projects. However, special procedures that apply to hydropower cases are described in other statutes, the most important being the \cite{wra17}. \noo{ which is specifically aimed at a certain subgroup of hydropower schemes, namely those that involve regulation of the flow of water in a river system.\footnote{See section \ref{sec:wra17} below.} However, according to section 19 of the \cite{wra00}, many provisions from the \cite{wra17} also apply to unregulated, run-of-river, schemes, if they generate more than 40 GWh per annum.\footnote{See the \indexonly{wra00}\dni\cite[19]{wra00}.} }

\subsection{The Watercourse Regulation Act}\label{sec:wra17}

In order to maximise the output of a hydropower scheme, the flow of water may be regulated using dams or diversions. Regulation was particularly important in the early days of hydropower, before the national electricity grid was developed.\footnote{See \cite[83]{uleberg08}.} Indeed, in the early days, it was common for electricity producers to get paid based on the stable effect they were able to deliver, rather than the total amount of energy they harnessed.\footnote{See \cite{sofienlund07}.}

Today, this has changed, as producers get paid based on the total amount of electricity they deliver,  measured in kilowatt hours (KWh). The price fluctuates over the year, and the supply-side is still influenced by instability in the waterflow in Norwegian rivers. However, the smoothing effect of the national grid means that run-of-river schemes can be carried out profitably, even if most of the electricity from the plant is produced during peak periods.

Despite the growing importance of run-of-river schemes, many key rules regarding hydropower development are still found in the \cite{wra17}. This Act defines regulations as ``installations or other measures for regulating a watercourse's rate of flow''. It also explicitly states that this covers installations that ``increase the rate of flow by diverting water''.\footnote{See the \indexonly{wra17}\dni\cite[1]{wra17}.} The core rule of the Act is that watercourse regulations that affect the rate of flow of water above a certain threshold are subject to a special licensing requirement.\footnote{See the \indexonly{wra17}\dni\cite[2]{wra17}.}

The threshold is defined in terms of the notion of a ``natural horsepower'', such that a license is required if the regulation yields an increase of at least 400 natural horsepower in the river. Natural horsepower is a gross estimate of the power that can be harnessed from a river continuously for at least 350 days a year.\footnote{See the \indexonly{wra17}\dni\cite[2]{wra17}.} The definition is a simple mathematical expression, given below:

$$
nat.hp(Q,H) = 13.33 \times H \times Q
$$

This formula states that the natural horsepower of a regulation project ($nat.hp(Q,H)$) is a function of two variables, $H$ and $Q$. The constant factor $13.33$ is the force of gravity of Earth exerted on a mass of 1 kg (or, approximately, 1 litre of water). The variable $H$ is the difference in altitude (measured in metre) from the intake dam to the power generator. The variable $Q$ is the amount of water (measured in litre) continuously available per second of the day, for at least 350 days per year. The result is then a gross estimate (assuming no energy loss) of the stable horsepower output of a hydroelectric plant that harnesses the power of $Q$ litres of water per second over a difference in altitude of $H$ metres.

Section 2 of the \cite{wra17} asks for the {\it increase} of this figure after regulation. To arrive at this number, one first uses the formula with $Q$ taken to be $Q_1$, the stable water flow prior to regulation, before calculating it with $Q$ taken to be $Q_2$, the stable water flow after regulation. The difference between the second and the first figure ($nat.hp(Q_2,H) - nat.hp(Q_1,H)$) is the increase of natural horsepower resulting from regulation.

Effectively, at a time when electricity had to be produced at a stable effect, from a stable source of power, this increase in natural horsepower was a gross estimate of the value added to the river by regulation. In the present context, it suffices to say that if a hydropower project involves regulation at all (i.e., if it is not a run-of-river scheme), it will indeed yield 400 natural horsepower or more. Hence, a special license will be required pursuant to section 2 of the \cite{wra17}. 

The criteria for granting a regulation license are similar to those for granting a license pursuant to the \cite{wra00}. In particular, section 8 of the \cite{wra17} states that a license should ordinarily be issued only if the benefits of the regulation are deemed to outweigh the harm or inconvenience to public or private interests.\footnote{See the \indexonly{wra17}\dni\cite[8]{wra17}.} In addition, it is made clear that other deleterious or beneficial effects of importance to society should be taken into account.\footnote{See the \indexonly{wra17}\dni\cite[8]{wra17}.} Finally, if an application is rejected, the applicant can demand that the decision is submitted for review by Parliament.\footnote{See the \indexonly{wra17}\dni\cite[8]{wra17}.} However, the \cite{wra17} contains more detailed rules regarding the procedure for dealing with license applications, c.f., section \ref{sec:4:3:1} below. 

\noo{The most practically important is that the applicant is obliged to carry out an impact assessment pursuant to the \cite{pb08}. This means that the applicant must organise a hearing and submit a detailed report on positive and negative effects of the development, prior to submitting a formal application for a licence. Effectively, at least {\it two} detailed rounds of assessment are therefore required before a license is granted.

In addition to prescribing impact assessments, the \cite{wra17} contains more specific rules concerning the second public hearing that should take place, when the application as such is processed. First, the applicant should make sure that the application is submitted to the affected municipalities and other interested government bodies.\indexonly{wra17}\dni\footcite[6]{wra17} Second, the applicant should send the application to organisations, associations and the like whose interests are ``particularly affected''.\indexonly{wra17}\dni\footcite[6]{wra17} Along with the application, these interested parties should be given notification of the deadline for submitting comments, which should not be less than three months.\footnote{See the \indexonly{wra17}\dni\cite[6]{wra17}.} The applicant is also obliged to announce the plans, along with information about the deadline for comments, in at least one commonly read newspaper, as well as the Norwegian Official Journal.\footnote{\indexonly{wra17}\dni\cite[6]{wra17}. The Norwegian Official Journal is the state's own announcement periodical.}
}
%A license pursuant to the \cite{wra17} might be cumbersome to obtain, but a successful application also results in a significant benefit. Most importantly, the license holder then automatically has a right to expropriate the necessary rights needed to undertake the project, including the right to inconvenience other owners.\footnote{See \cite[16]{wra17}.} Hence, expropriation is a side-effect of a regulation license. Even so, the issue of expropriation rarely receives any special consideration in regulation cases. In particular, the assessment undertaken by the water authorities is focused on the licensing issue, which does not compel them to direct any special attention towards owners' interests.\footnote{I demonstrate this, and discuss it in much more depth, in Chapter \ref{chap:4}, Section \ref{sec:jorpeland}.}

In general, the issue of who owns and controls the water resources in question receives little attention in relation to licensing applications, both pursuant to the \cite{wra17} and the \cite{wra00}. Instead, the focus is on weighing environmental interests against the interest of increasing the electricity supply and facilitating economic development. The issue of resource ownership is more prominent in relation to a third important statute, namely the \cite{ica17}.

\subsection{The Industrial Licensing Act}\label{sec:ica17}

In the early 20th century, industrial advances meant that Norwegian waterfalls became increasingly interesting as objects of foreign investment. To maintain national control of water resources, Parliament passed an Act in 1909 that made it impossible to purchase valuable waterfalls without a special license.\footnote{See \cite[59]{falkanger87}.} The follow-up to this Act is the \cite{ica17}, which is still in force. It applies to potential purchasers and leaseholders of rivers that may be exploited so that they yield more than 4000 natural horsepower.\footnote{Unlike section 2 of the \cite{wra17}, this asks only for the number of horsepower in the river (after regulation), not the {\it increase} of this number.}

To reach this number requires a substantial regulation, so the Act does not apply to many run-of-river hydropower schemes, even large-scale projects. Originally, the main rule in the \cite{ica17} stated that all licenses granted to private parties were time-limited, and that the waterfalls would become state property without compensation when they expired, after at most 60 years.\footnote{See the previous \indexonly{ica17}\dni\cite[2]{ica17}, in force before the amendment on 26 September 2008.} This was known as the rule of {\it reversion} in Norwegian law.\footnote{This is a misnomer, however, in light of how most rivers and waterfalls were originally owned by local smallholders, not the state.}

In a famous Supreme Court case from 1918, the rule was upheld after having been challenged by owners on constitutional grounds.\footnote{See \cite{johansen18}.} This was based on the finding that reversion represented a form of regulation of property, not expropriation. Hence, it could not be challenged on the basis of section 105 of the Constitution, even though the owners were not awarded any compensation. 

While the rule of reversion withstood internal challenges, it was eventually struck down by the EFTA Court in 2007, as a breach of the EEA agreement.\footnote{See \cite{efta07}. The EEA (European Economic Area) agreement sets up a framework for the free movement of goods, persons, services and capital between Norway, Iceland, Lichtenstein and the European Union. The EFTA (European Free Trade Association) oversees the implementation of the EEA for those members of EFTA that are also members of the EEA (all except Switzerland). For further details, see generally \cite{bull94,magnussen02,fredriksen09}.} This conclusion was based on the fact that reversion only applied to privately owned companies, which the Court regarded as an illegitimate form of discrimination. After this ruling, the \cite{ica17} was amended. Today, only companies where the state controls more than 2/3 of the shares may purchase waterfalls or rivers to which the Act applies.\footnote{See the \indexonly{ica17}\dni\cite[2]{ica17}.}

This means that such rivers and waterfalls can only be bought, leased or expropriated by companies in which the state is a majority shareholder. In practice, however, landowners are still able to sell the land from which the right to a waterfall originates, even if this also means transferring the waterfall to a new owner. The rule is typically only enforced when riparian rights as such are transferred, specifically for the purpose of large-scale hydropower development. In particular, small-scale development and large run-of-river schemes can still usually be carried out by local owners. The policy justification for the (amended) \cite{ica17} is based on the idea that giving preference to state-owned actors will protect the public. However, this perspective clashes with the fact that the electricity sector itself has been liberalised. The state may be a majority shareholder in the most powerful companies, but these companies are now run according to commercial principles, with little or no direct political involvement.\footnote{See \cite[86]{efta07}.}

Hence, as the EFTA court highlights in its judgement on reversion, there appears to be a lack of convincing policy reasons why state-owned companies should be given preferential treatment.\footnote{See \cite[84-87]{efta07}.} In light of this, Norway's response to the Court's decision is a curious one: instead of creating a level playing field, the preference given to state-owned commercial companies is made even more marked, as privately owned companies are now excluded from one segment of the hydropower market altogether.

%Of course, the public benefits indirectly from the fact that public bodies, as shareholders, are entitled to dividends. But it is not clear why this benefit should be considered in a different light than other indirect financial benefits which might as well be extracted from private companies, e.g., through taxation. Moreover, public-private partnerships are still permitted, as private actors may own up to two-thirds of ``state-owned'' companies. What this means is that the preferential treatment given to state actors is in fact also extended to those private actors that the state happen to prefer. Interestingly, this style of regulation contrasts quite sharply with some of the key ideas behind the basic building block of the liberalised electricity market, namely the \cite{ea90}.

\subsection{The Energy Act}\label{sec:ea}

Before 1990, the Norwegian electricity sector was tightly regulated by the government.\footnote{See generally \cite{bye05,skjold07}.} The responsibility for the national grid was divided between various public utilities that would also typically engage in electricity production, wielding monopoly power within their districts. The most powerful utilities were controlled by the state, which also developed large-scale hydropower to supply the metallurgical industry with cheap electricity.\footnote{See \cite[67-71]{thue96}.} However, the county councils and the municipalities maintained a significant stake in the hydroelectric sector, as they often controlled the utilities responsible for the electricity supply in their own local area.\footnote{See \cite[85]{thue96}.} 
Prior to 1990, there was no real competition on the electricity market, and the local monopolists could deny other energy producers access to their segment of the distribution grid.\footnote{See \cite[83-84]{uleberg08}.}

This system was abandoned following the passage of the \cite{ea90}.\footnote{See generally \cite{bibow11}.} This Act set up a new regulatory framework, where management of the grid was decoupled from the hydropower production sector.\footnote{See generally \cite{bye05}.} In particular, the Act established a system whereby consumers could choose their electricity supplier freely. At the same time, the Act aimed to ensure that producers were granted non-discriminatory access to the electricity grid. This laid the groundwork for what has today become an international market for the sale of electricity, namely the Nord Pool.\footnote{See generally \cite{galtung07}.}

In response to this, monopoly companies were reorganised, becoming commercial companies that were meant to compete against each other, and against new actors that entered the market.\footnote{See \cite{claes11}.} In addition to commercialisation, the market-orientation of the sector has also lead to centralisation, as many of the locally grounded municipality companies have disappeared as a result of mergers and acquisitions.\footnote{Today, the 15 largest companies, largely controlled by the state and some prosperous city municipalities, own roughly 80\% of Norwegian hydropower, measured in terms of annual output. See \cite[28]{otprp61}. The process causing this concentration started long before the market-oriented reform of the sector. In particular, after the Second World War, there was a significant push by the state towards increased centralisation, see \cite{skjold06,thue06b}.} As a result, the local and political grounding of the electricity sector, which used to be ensured through decentralised municipal ownership, has been significantly weakened.

At the same time, the fact that any developer of hydropower is now entitled to connect to the national grid gives private actors a possibility of entering the Norwegian electricity market. They may do so not merely as (minority) shareholders in former utilities, but also as {\it competitors}, as long as they stick to run-of-river or small-scale hydropower.\footnote{See generally \cite{larsen06,larsen08,larsen12}.} In the next section, I give a step-by-step presentation of the licensing procedure for hydropower, which serves to summarise the legislative framework and provide information about the institutional framework within which it is called to function.

\subsection{The Licensing Procedure}\label{sec:4:3:1}

The water authorities in Norway are centrally organised. The most important body is the Norwegian Water Resources and Energy Directorate (NVE), based in Oslo. In many cases, the NVE have been delegated authority to grant development licenses themselves, but in case of large-scale development, they only prepare the case, then hand it over to the Ministry of Petroleum and Energy.\footnote{See Delegation of 19 December 2000, from the Ministry of Petroleum and Energy (FOR-2000-12-19-1705) and Directive of 15 December 2000, from the King in Council (FOR-2000-12-15-1270), pursuant to the \indexonly{wra00}\dni\cite[64]{wra00}.} The Ministry, in turn, gives its recommendation to the King in Council, who makes the final decision.\footnote{See Directive of 15 December 2000, from the King in Council (FOR-2000-12-15-1270).} Parliament must also be consulted for regulations that will yield more than 20 000 natural horsepower.\footnote{See the \indexonly{wra17}\dni\cite[2]{wra17}.}

As indicated by the survey of relevant legislation given in previous sections, there are many categories of hydropower projects. Moreover, different categories call for different licenses. Hence, the first step in the application process is for the developer to determine exactly what kind of license they require. This is further complicated by the fact that some categories overlap, since they are based on different measuring sticks for assessing the scale of an hydropower project. 

One important parameter is the power of the hydropower generator, measured in MW (Megawatts). There are four categories of hydropower formulated on this basis: the micro plants (less than $0.1$ MW), the mini plants (less than $1$ MW), the small-scale plants (less than $10$ MW), and the large-scale plants (more than $10$ MW). In practice, one tends to use small-scale hydropower more loosely, to refer to all projects less than 10 MW. Still, a further qualification is sometimes required. For example, the authority to grant a license for a micro or mini plant has been delegated to the regional county councils since 2010, in an effort to reduce the queue of small-scale applications at the NVE.\footnote{See Delegation letter from the Ministry of Petroleum and Energy, dated 07 December 2009, available at \url{http://www.nve.no} (accessed 24 August 2014). The county council is an elected regional government institution situated between the municipalities and the central government. There are 19 county councils in Norway as of 01 January 2015. They are comparatively less important than both the municipalities and the central government, but have several  responsibilities, particularly in relation to infrastructure, education and resource management. See generally \cite{berg15}.} The council's decision is based on a (simplified) assessment made by the regional office of the NVE. In addition, licenses for micro and mini plants may be granted even in watercourses that have protected status pursuant to environmental law.\footnote{See Decision no 240, Stortinget (2004-2005), St.prp.nr.75 (2003-2004) and Innst.S.nr.116 (2004-2005).}

For small-scale plants proper, the authority to grant a license is delegated to the NVE, with the Ministry serving as the instance of appeal.\footnote{See Delegation of 19 December 2000, from the Ministry of Petroleum and Energy (FOR-2000-12-19-1705).} For large-scale plants, the granting authority is the King in Council, based on a recommendation from the Ministry.\footnote{See Directive of 15 December 2000, from the King in Council (FOR-2000-12-15-1270).} However, in practice, the decision is usually closely based on assessments and recommendations provided by the NVE.\footnote{For a detailed guide to the administrative process for large-scale applications, published by the NVE, see \cite{stokker10}.}

While the relevant licensing authority depends on the effect of the plant, the kind of license required depends on a different categorisation, relating to the level of planned water regulation, measured in natural horsepower. Here, there are three categories: run-of-river schemes  (less than $500$ natural horsepower), non-industrial regulations ($500 - 4000$ natural horsepower), and industrial regulations (more than $4000$ natural horsepower).\footnote{See the \indexonly{wra17}\dni\cite[2]{wra17} and the \indexonly{wra17}\dni\cite[1,2]{ica17}.}

Almost all hydropower schemes require a license pursuant to section 8 of the \cite{wra00}.\footnote{As mentioned in section \ref{sec:wra00}, the exceptions are very small schemes (usually mini or micro) that are deemed to be relatively uncontroversial. Such schemes only require a license pursuant to the \cite{pb08}.} For run-of-river schemes, no further licenses are required for the development itself, although an operating license pursuant to the \cite{ea90} is typically required for the electrical installations.\footnote{See the \indexonly{ea90}\dni\cite[3-1]{ea90}.} For schemes involving a non-industrial regulation, an additional license pursuant to section 8 of the \cite{wra17} is required. Industrial regulation schemes require yet another license, pursuant to section 2 of the \cite{ica17}.

As is to be expected, the complexity of the licensing procedure tends to increase with the number of different licenses required. However, the licensing applications tend to be dealt with in parallel, so that all licenses are granted at the same time, following a unified assessment. In practice, when the \cite{wra17} applies, it structures the procedure as a whole, also those aspects that pertain to other licenses. 

In addition, yet another categorisation of hydropower schemes is used to determine the relevant application procedure. This categorisation is based on the annual production of the proposed plant, measured in GWh/year. There are three categories: simple schemes (less than $30$ GWh/year), intermediate schemes ($30 - 40$ GWh/year), and complicated schemes (more than $40$ GWh/year). As mentioned in section \ref{sec:wra17}, the most important rules in the \cite{wra17} apply to complicated schemes, regardless of whether or not the scheme involves a regulation.\footnote{See the \indexonly{wra00}\dni\cite[19]{wra00}.} In addition, applications for such schemes must be accompanied by an impact assessment pursuant to section 14-6 of the \cite{pb08}.

This means that the applicant is required to organise a public hearing prior to submitting their formal application, to collect opinions on the project and provide an overview of benefits and negative effects of the plans, particularly as they relate to environmental concerns.\footnote{See Directive of 19 December 2014 (FOR-2014-12-19-1758), pursuant to the \indexonly{pb08}\dni\cite[1-2,14-6]{pb08}.} In practice, if an impact assessment is required this significantly increases the scope and complexity of the application process.

For intermediate schemes that do not involve regulation, the rules in the \cite{wra17} do not apply. However, impact assessments {\it may} still be required.\footnote{See \cite[20]{stokker10}.} Here the threshold of 30 GWh/year has been set as an additional threshold by the NVE, who have been delegated authority to require impact assessments for hydropower projects even when these yield less than 40 GWh/year.\footnote{See Directive of 19 December 2014 (FOR-2014-12-19-1758).} For the intermediate schemes, NVE decides whether an impact assessment is required on a case-by-case basis. For simple schemes, on the other hand, impact assessments will not be required. Such schemes make up the core of what is described as small-scale hydropower in daily language.

The time from application to decision can vary widely, depending on the complexity of the case, the level of controversy it raises, and the priority it receives by the licensing authority. Usually, the assessment stage itself will last 1-3 years, sometimes longer.\footnote{See \cite[84-85]{nou129}.} While large-scale schemes involve more complicated procedures, they are also typically given higher priority than small-scale schemes. In recent years, following the surge of interest in small-scale development, a processing queue has formed at the NVE.\footnote{See \cite[84]{nou129}.} This means that small-scale applications typically have to wait a long time, sometimes several years, before the NVE begins processing them.\footnote{See \cite[84]{nou129}.}

%As I will discuss in more depth in the next chapter, the issue of expropriation is rarely given special attention during the application assessment. This is so even in cases when an application to expropriate waterfalls is submitted alongside the licensing applications. The issue of expropriation is rarely singled out for special treatment, at least not in cases of large-scale development. %Moreover, as mentioned in Section \ref{sec:hl}, an automatic right to expropriate follows from section 16 of the \cite{wra17}.

The applicant is expected to submit application notices for publication in local newspapers, and for larger projects there will typically also be an information meeting arranged in the local area, where the applicant and the authorities appear side by side, presenting the plans and the licensing procedure respectively.\footnote{See \cite[23]{stokker10}.} For large-scale projects, it is also common for the applicant to distribute brochures widely in the local area. These procedural arrangements arguably reflect some concern for the interests of local populations. However, the procedure is organised in a way that can also create the impression that the applicant enjoys significant state-backing from the start.

Indeed, applicants not only communicate with locals in place of the authorities, they are also given responsibility for many material aspects of the assessment process, including the often crucial assessment of possible alternatives.\footnote{See \cite[24]{stokker10}.} This would seem to raise competency questions, particularly in cases where the owners themselves propose alternatives that the applicant hoping to expropriate will then assess on behalf of the government. However, the Supreme Court has not found any fault with this remarkable form of administrative subcontracting.\footnote{See \cite[51-55]{jorpeland11}.}

More generally, as shown in chapter \ref{chap:5}, the protection offered to waterfall owners is very limited. For instance, the government does not recognise a duty to notify these owners individually, to ensure that they are informed of what is at stake for them as owners of a very valuable resource. Rather, a generic letter is typically sent by the applicant to all affected private parties. The statement that private property ``will be expropriated'' unless a settlement is reached has also been observed.\footnote{In the case of \cite{sauda09}.}

After the hearing stage, the NVE will usually compile a final report along with a recommendation and send it to the interested parties for comments.\footnote{See the \indexonly{wra17}\dni\cite[6]{wra17}.} It is established practice that local owners do {\it not} count as interested parties in this regard.\footnote{See \cite[46]{jorpeland11}.} Hence, while the municipalities and various environmental interest groups are informed of how the case progresses and asked to comment prior to the final decision, the owners must inquire on their own accord if they wish to be kept up to date on the application process.\footnote{In a written statement to the Supreme Court in the case of \cite{jorpeland11}, the director of the hydropower division of the Ministry pointed out that the documents would be made available on the web page of the NVE and that local owners had to ``look after their own interests''.}

In summary, the procedural framework surrounding licensing of hydropower development leaves local owners in a precarious position, especially when the applicants wish to expropriate their waterfalls. At the same time, the liberalisation of the electricity sector means that owners are in a far better position than before when it comes to developing hydropower themselves. This is discussed in more depth in the next section.
%Given that expropriation is often an automatic side-effect of a development license, this already suggests that legitimacy issues are likely to likely to arise when waterfalls are taken for hydropower. I return to this issue specifically in the next chapter. First, I will discuss market practices in more depth, focusing on the changes that resulted from the liberalisation reform of the early 1990s. 

\section{Hydropower in Practice}\label{sec:4:4}

The history of hydropower in Norway can be roughly divided into four stages. The first stage was the development that took place prior to 1909. During this time, private actors dominated, with public ownership playing a minor role.\footnote{See \cite{otprp61}.} Moreover, there were many private interests speculating in acquiring Norwegian waterfalls, anticipating the value that these would have for industrial development.\footnote{See \cite[30-31]{nou04}.}

After 1909, the introduction of licensing obligations and the rule of reversion made it much harder for private companies to acquire waterfalls that were suitable large-scale industrial development. At the same time, local municipalities began to invest in hydropower to provide electricity to their citizens, a service they were increasingly being obliged to provide.\footnote{See \cite{otprp61}.} This marked the start of the second stage of hydropower development, which saw the development of a more strictly regulated sector. However, this sector was also highly decentralised, for a large part dominated by local actors.

In fact, throughout the first half of the 20th century, most hydroelectric plants were small-scale plants that supplied local communities with electricity.\footnote{See \cite[11]{utbygd46}. This is a report from the water directorate published in 1946, showing that as of 31 December 1943, $97.8 \%$ of all hydroelectric plants in Norway were small-scale plants. However, these plants contributed only $28 \%$ of the total hydroelectric power installed at that time.} Moreover, as late as in 1943, $89 \%$ of all hydroelectric power stations in Norway were still private, many of which were mini and micro plants that were owned and operated by the local community.\footnote{See \cite[6]{utbygd46}. See also \cite[111]{hindrum94}.} However, many bigger plants were also under private ownership, and $57 \%$ of the total hydroelectric power available at this time was supplied by the private sector. 
%This clearly illustrates the importance of smaller, local initiatives, in the process of providing Norway with electricity, particularly in rural areas. Interestingly, while the micro and mini plants accounted for $72.9 \%$ of the total number of plants, they only accounted for $1.6 \%$ of the total electricity supply.\footnote{See \cite[7]{utbygd46}.}

By the end of 1943, $80 \%$ of the Norwegian population had access to electricity at home.\footnote{In rural areas, the corresponding figure was $70 \%$, see \cite[7]{utbygd46}.} Hence, the decentralised approach to hydropower development, based on private ownership and local control, had not been an impediment to the supply of electricity to most of the country's population.

However, the regulatory regime was soon to undergo a significant change, designed to facilitate industrial development and increased state control. This change came quite rapidly after the Second World War, when the central government began to invest heavily in hydropower, often to ensure economic development by subsidising the metallurgical industry.\footnote{See \cite[59-65]{thue96}.} This period saw increased marginalisation of small private electricity companies, as well as local owners.\footnote{At the same time, powerful (private) metallurgical interests benefited greatly, sometimes also at the expense of the general supply of electricity. See \cite[65-71]{thue96}.} Indeed, it was often demanded, as a condition for allowing local communities access to the national electricity grid, that local hydroelectric plants had to be shut down.\footnote{See \cite[111]{hindrum94}.} During this time, the development of hydropower was seen as an important aspect of rebuilding the nation. However, the goal was not primarily to supply the public with electricity, but rather to facilitate a specific kind of economic development that the central government regarded as desirable.\footnote{See \cite[59]{thue96}.}

The state-dominated system set up on this basis remained in place until the 1970s, when increasingly vocal opposition from environmental groups and local populations led to some reforms.\footnote{See \cite[71-75]{thue96}.} As the scale of typical development projects had increased significantly compared to earlier times, new projects would tend to meet with broader and better organised forms of resistance. In many cases, municipal and regional government institutions would join in opposition against large-scale development.\footnote{See \cite[71-72]{thue96}.} The typical response from the state was to introduce measures that sought to pacify the regional and municipal government opposition, which was considered more serious than opposition from local people and environmental groups. The standard approach was to grant an increased share of the financial benefit to local and regional institutions of government, to instil support for state-led development plans.\footnote{See \cite[73-76]{nilsen08}.} The centralisation process in the hydroelectric sector slowed down somewhat during this time.\footnote{See \cite[85]{thue96}.} However, despite limiting the discontent among local power groups, high-profile controversies continued to arise, most notably the {\it Alta} case discussed in the next chapter.

The fourth stage of hydropower development began in 1990 after the passage of the \cite{ea90}. The liberalisation that followed saw the transformation of the hydropower sector into a commercial market, based on profit-maximising and competition. As a result, the structure of decentralised management withered away further, as many municipality companies were either bought up by more commercially aggressive actors or forced to merge and change their business practices in order to remain competitive.\footnote{See \cite[583]{bibow03} (commenting on the increased concentration of power on the electricity market, following acquisitions and mergers after 1990).} At the same time, a new decentralised force emerged in the sector, in the form of local owner-led projects.\footnote{See section \ref{sec:4:4} below.}

The core idea behind the \cite{ea90} was that the electricity sector should be restructured in such a way that production and sale of electricity, activities deemed suitable for market regulation, would be kept organisationally separate from electricity distribution over the national grid, a natural monopoly. However, the Act itself does not explain in any depth how this is to be achieved. In practice, the divide has not been strictly implemented. Most of the large energy companies in Norway continue to maintain interests in both distribution, production and sale of electricity, a phenomenon known as ``vertical integration''.\footnote{See \cite[580-583]{bibow03}.} In fact, the degree of vertical integration in the electricity sector initially increased after the passage of the \cite{ea90}.\footnote{See \cite[583]{bibow03}.}

\noo{ To some extent, the water authorities have responded to this by making use of their authority to give organisational directives when they grant distribution licenses.\footnote{See the \indexonly{ea90}\dni\cite[4-1]{ea90}, para 2, no 1.} For instance, electricity companies are now required to keep separate accounts for production, distribution and sale of electricity.\footnote{See Directive of 11 March 1999 (FOR-1999-03-11-302), s 4-4 a and s 2-6, issued by the NVE pursuant to Directive of 7 December 1990 (FOR-1990-12-07-959), s 9-1, pursuant to the \indexonly{ea90}\dni\cite[10-6]{ea90}.} It is also required that transactions across these functional divides are clearly marked, and that they are based on market prices.\footnote{See Directive of 11 March 1999 (FOR-1999-03-11-302), s 2-8.}}

The water authorities responded to this by accepting increased concentration of ownership, while also ordering distribution activities to be kept organisationally separate from other activities, for instance through the establishment of a special subsidiary company.\footnote{See \cite[581-582]{bibow03}.} Typically, a conglomerate structure is used, with a single parent company that controls both the distribution company, the production company and the sales company. Indeed, this model has now been implemented by most of the large energy companies in Norway.\footcite[582]{bibow03}

\noo{It seems unclear whether this approach really achieves the stated objective. By adopting the conglomerate model of organisation, the major players on the market have successfully gained control over a larger share of both the production and distribution facilities for electricity. Hence, these actors effectively control the core infrastructure that makes up the backbone of the Norwegian electricity sector. The {\it intention} is that monopoly power should only be exercised with respect to the distribution grid on non-discriminatory terms. But is this realistic when the conglomerate controlling the grid operator has significant stakes also in production and the trade of electricity?

This question calls for a separate study, and }
The extent to which this is an adequate response to increased concentration of power in the electricity sector will not be addressed in any depth here. However, I will direct attention at one aspect that arises with particular urgency for small-scale development of hydropower, concerning access to the grid. It is quite common, in particular, that small-scale projects remain unrealised because the grid is regarded to lack sufficient capacity to accommodate new electricity.\footnote{See, e.g., \cite[84,161-162]{nou129}.}

Following an amendment of the Energy Act in 2009, grid companies are now obliged to facilitate access for producers, even when this necessitates new investments.\footnote{See Act no 105 of 19 June 2009 regarding changes in the \cite{ea90}.} However, the energy producer seeking access is typically required to reimburse the grid company for the cost of new investments, as determined in the first instance by the grid company itself (the NVE serves a supervisory function).\footnote{See Directive of 7 December 1990 (FOR-1990-12-07-959), s 3-4.} In addition, grid companies may still deny access in cases when the needed investments are not ``socio-economically rational''.\footnote{See the \indexonly{ea90}\dni\cite[3-4]{ea90}. The authority to decide whether this requirement is fulfilled is vested with the Ministry.}

Often, the relevant grid company will be a sister company of an energy producer operating in direct competition with the company seeking access. This can raise questions about the impartiality of the assessments carried out by the grid company. In expropriation cases, this becomes an issue particularly in relation to the assessment of the cost of undertaking an alternative development scheme.\footnote{This assessment is often crucial, because it provides information about the value of the development potential that the owners stand to loose.} Riparian owners are rarely pleased when they realise that the expropriating party is part of the same conglomerate as the grid company that estimates the grid connection costs associated with owner-led development.\footnote{See, e.g., \cite{smibelg15}.}

%It has been pointed out, in particular, that the practical consequence of liberalization has been that the local accountability of the electricity sector has been lost, both organizationally and politically.\footnote{See \cite{agnell11}. 
Meanwhile, the market-orientation of the electricity sector has reduced the level of political control and accountability. Today, a management model based on economic rationality and expert-rule has become dominant. According to Brekke and Sataøen, this serves to set the reform that took place in Norway apart from similar energy reforms in Sweden and the UK.\footnote{See \cite{brekke12}.} Moreover, Brekke and Sataøen argue that this has resulted in a lack of legitimacy that has been a significant contributory cause of recent national-scale controversies, particularly with regards to the development of the national grid.\footnote{The most serious case so far is that of {\it Sima - Samnanger}, concerning a new distribution line that will cut through the area known as {\it Hardanger}, a scenic part of south-western Norway. The plans met with significant resistance at both the national and the local level, but the government pushed ahead, leading to confrontations that also involved some acts of civil disobedience. See \cite[22-23]{brekke12}.}

At the same time, the growth of the small-scale hydropower sector gives local communities a new voice, as market participants, thereby acting as a counterweight to centralisation and expert-rule. Since the mid- to late 1990s, the small-scale sector has grown significantly. In a recent report, the potential for profitable small-scale hydropower projects was estimated to be around 20 TWh per year.\footnote{See \cite{aanesland09}. For comparison, suggesting the scale of this potential, I mention that the total consumption of electricity in Norway in 2013 amounted to about 120 TWh, see \cite{statistikk13}. According to the government, about one third of the remaining potential for hydropower in Norway, measured in annual energy output, will come from small-scale projects. See \cite[231]{nou129}.} On this basis, the authors of the report estimate that the total present-day value of all waterfalls suitable for small-scale hydropower is about 35 billion Norwegian kroner, i.e., about 3.5 billion pounds.\footnote{See \cite[1]{aanesland09}.} This calculation is based on a model where the waterfalls are exploited in cooperation with an external commercial company, inspired by existing agreements between owners and the limited company {\it Småkraft AS}. Hence, the calculation might be an underestimate of what small-scale hydropower could represent for local communities if they remain in charge of development themselves.

Small-scale hydropower has become socially and political significant in Norway. In the report mentioned above, it is estimated that the value of rivers and waterfalls amount to just under 50 \% of the total equity in Norwegian agriculture.\footcite[1]{aanesland09} Moreover, hydropower is increasingly seen as a possibility for declining regions to counter depopulation and poverty. In some communities, small-scale hydropower is the only growth industry. For these communities, pursuing hydropower development at the local level also provides a way to regain some autonomy with respect to how local natural resources should be managed. Hence, small-scale hydropower takes on great political and social importance, not just for the owners of waterfalls, but for the community as a whole.

For an example of a community where small-scale hydropower has played such a role, I point to Gloppen, a municipality in the county of Sogn og Fjordane, in the western part of Norway. Here, 19 hydropower plants have been built in recent years, all except one by local owners themselves, amounting to a total production of over 250 GWh per year. This prompted the mayor to comment that ``small scale hydro-power is in our blood''.\footnote{See \cite{starheim12}.} When interviewed, he also directed attention at the fact that hydropower had many positive ripple effects, since it significantly increased local investment in other industries, particularly agriculture, which had been severely on the decline.

To achieve such effects, it is important to organise development in an appropriate manner. Moreover, to explain how waterfalls came to be as valuable as they are today, it is crucial to direct attention to the way in which waterfall owners initially asserted themselves on the market. In the following, I do this by giving an in-depth presentation of an early model for local involvement in hydropower development, presented at a seminar in 1996.\footnote{See \cite{dyrkolbotn96}.} This model contains an early expression of several ideas that would prove influential to the development of the small-scale hydropower sector.

%However, certain other aspects of the model have not been widely adopted. These are aspects that pertain to the balance of power between owners and cooperating developers, as well as the relationship that should be established with larger communities of non-owners, including environmental groups and other water users. Hence, considering the model in some depth, and assessing its impact, will allow me to shed light on desirable social functions of waterfall ownership, and the extent to which such functions are fulfilled on the market today.

\section{{\it Nordhordlandsmodellen}}\label{sec:4:5}

In five brief points, the {\it Nordhordlandsmodellen} sets out a framework for cooperation between waterfall owners, professional hydroelectricity companies, local communities, and society as a whole.\footnote{See \cite{dyrkolbotn96}. The model was the result of a collaboration between Otto Dyrkolbotn, a farmer and a lawyer, and Arne Steen, the director of {\it Nordhordland Kraftlag}, a municipality-owned energy company.} 

The first point makes clear that the aim of cooperation should be to ensure local ownership and control: external interests should never be allowed to hold more than 50 \% of the shares in the development company. If the company is organised as a limited liability enterprise, then the model stipulates that local residents -- not necessarily owners -- are to be given a right of preemption in the event that shares come up for sale. %The possibility of organising the development company as a local cooperative is also mentioned.\footnote{References needed.}

The second point of the model sets out a method for valuing the riparian rights prior to development. It stipulates that the appraisal should reflect the real value of such rights, normally estimated on the basis of lease capitalisation. More concretely, the valuation should be based on the premise that the riparian owners will be entitled to rent based on the level of annual production in the planned hydropower project. Then, for the purpose of appraisal, the expected rent per annum is capitalised to find the present value of the riparian rights, relative to the development project in question.\footnote{This approach stands in stark contrast to the earlier valuation method, discussed in chapter \ref{chap:5}, section \ref{sec:5:4:1}.}

After such a value has been calculated, the model stipulates that owners are to be given a choice of either leasing out their water rights to receive rent, or to use the capitalised value of (part of) this rent as equity to acquire shares in the development company. The third point in the model then offers a clarification, by stating that the development company should not in any event acquire ownership of riparian rights, but only a time-limited right of use. After 25-35 years, this usufruct should fall away and the waterfall should revert back to the owners of the surrounding land, free of charge. This is the proposed rule even in cases when the landowners themselves initially control the majority of the shares in the development company. Hence, the rule places a limit on alienation; no separation of water rights from land rights is allowed to last for more than 35 years. 

The {\it Nordhordland} model demonstrated the commercial viability of this organisational model by pointing to a concrete municipality-owned energy company that had stated its willingness to cooperate with owners on such terms, to help with financing and share the risk.\footnote{The company in question is Nordhordland Kraftlag, where one of the authors of {\it Nordhordlandsmodellen}, Arne Steen, was a director.} 

Following up on this organisational blueprint, the fourth and fifth points of the model describe the intended role of the local development company in society, by stressing the relationship between hydropower and other interests and potential uses of the affected river. Importantly, the model stipulates that potential developers should be willing to take on formal obligations towards other user groups. Moreover, obligations should not only be negatively defined, as duties to minimise or avoid harms. Positive obligations should also be introduced, such as duties to improve other qualities of the river system, and to engage in active cooperation with other users.

It is made clear that the overall aim is to ensure sustainable management of the river system as a whole. Interestingly, the model predicts that active local ownership will achieve more in this regard than what can be achieved through governmental regulation alone. This claim is illustrated by a concrete example of a case in which the local owners decided to pursue a scheme that was less environmentally invasive than the project endorsed by the water authorities.\footnote{Today, this project has become Svartdalen Kraftverk, finalised in 2006. It produces 30 GWh annually, enough electricity for about 1500 households.}

The model goes on to emphasise the need for integrated processes of resource planning and decision-making, to ensure that hydropower development is not approached as an isolated economic and environmental concern, but looked at in a broader social and political context. To achieve this, it is argued that local communities need to play an important role in the management of water resources. Another concrete example follows, regarding {\it Romarheimsvassdraget}, a river system in the municipality of Lindås, in the county of Hordaland.

This river system was originally intended for large-scale development undertaken by BKK AS, without the participation of local owners.\footnote{BKK AS is one of the 15 biggest hydropower companies in Norway, and would later also purchase Nordhordland Kraftlag.} The project would involve a total of three river systems, such that the water from {\it Romarheimselva} and another river would be diverted to a neighbouring municipality for hydropower development there. The local owners argued against these plans by proposing a number of smaller development schemes. Eventually, they were successful, as the NVE agreed to endorse an alternative consisting of 7 distinct run-of-river projects to be undertaken by local owners.\footnote{See \cite{vann25}.}

It is important to note that when {\it Nordhordlandsmodellen} was formulated, owner-led development of hydropower was still a recent phenomenon, driven forward by individual owners and local groups that saw the potential and had enough know-how to get organised. Later, commercial companies have emerged that specialises in cooperating with local owners.\footnote{See, e.g., \cite{larsen06}.} Today, this has made it relatively easy for owners to initiate small-scale hydropower development. Moreover, owners are often approached by interested commercial actors who wish to cooperate with them. Most of them rely on cooperation on terms that reflect the main ideas expressed in the first three points of {\it Nordhordlandsmodellen}.

However, several adjustments have become standard, and these systematically benefit the external partner: the requirement that locals should at all times control a majority of the shares is dropped, the period of usufruct is typically longer than 35 years, the reversion to the landowners after this time is made conditional on payment for machines and installations, and no preemption rights are granted to local residents.\footnote{See generally \cite{hauge15}.} However, the core idea that riparian rights are to be valued based on a capitalisation of future rent is accepted. This means, in turn, that local owners rarely need to raise any additional capital to acquire shares in the development company. Moreover, the rent itself can become a significant source of income.

There are two main approaches to calculating this rent. The first approach, introduced already in {\it Nordhordlandsmodellen}, specifies the rent as a percentage of the gross revenue from sale of electricity, today often around 10-20 \%.\footnote{Source: contracts presented to the court in \cite{sauda09} (available from the author upon request). See also \cite[55-57]{hauge15}.} In this way, passive owners need not take on any risk related to the performance of the hydropower company. The second approach has been developed by the company Småkraft AS, which is now the leading market actor specialising in cooperation with local owners.\footnote{It is owned by several large-scale actors on the energy market, see \url{www.smaakraft.no}.} According to their model, riparian owners are paid a share of the annual {\it surplus} from hydropower generation.\footnote{See \cite[57-60]{hauge15} (also discussing variants of this contractual idea, based on how the surplus is actually defined in the contract).}

This share is usually higher than the rent payable based on the net revenue; often, the owners are entitled to $50 \%$ of the profit.\footnote{Source: contracts presented to the court in \cite{sauda09} (available from the author upon request). See also \cite[58]{hauge15}.} Hence, if the project is a success, the riparian owners might be better compensated. However, the owners have to accept some risks as though they were shareholders, and they do so even though they might not have much of a say in how the company is run.\footnote{To limit the risk for owners, companies such as Småkraft AS also operates a system of ``guaranteed'' rent, but this rent is usually quite a lot less than what the owners could expect from an agreement based solely on rent based on gross income. Source: contracts presented to the court in \cite{sauda09} (available from the author upon request).}

To illustrate the financial scale of the rent agreements that have now become standard, let us consider a typical small-scale hydropower plant that produces 10 GWh annually. With an electricity price of NOK 0.3 per KWh, this gives the hydropower plant an annual gross income of NOK 3 million. If the rent payable is 20 \%, the waterfall owners will receive NOK 600 000 annually, approximately GBP 60 000. This is many times more than what the owners could hope to receive according to the traditional method for calculating compensation following expropriation.\footnote{For an example based on comparing two concrete cases, see chapter \ref{chap:5}, section \ref{sec:5:4:1}. See also \cite[283-289]{hauge15}.}

%By contrast, if the rights were expropriated, the traditional method of calculating compensation would be unlikely to result in more than NOK 600 000 as a {\it one time payment} for a waterfall that yields 10 GWh per annum.\footnote{For further details on the compensation issue, see \cite{dyrkolbotn14,dyrkolbotn15,dyrkolbotn15a}. Sometimes, the difference in valuation would be even greater, since the natural horsepower of a development project is highly sensitive to the level of regulation of the waterfall, much more so than the value of the development. For an demonstration of how this affected compensation according to the natural horsepower method, one may consider the case \cite{hellandsfoss97}, which went to the Supreme Court. Here the owners were paid just over NOK 1 million for a waterfall that would yield 100 GWh per annum.}

All in all, the financial consequences of the ideas expressed in {\it Norhordlandsmodellen} have been dramatic. However, the latter two points of the model, addressing the importance of holistic and inclusive management of river systems, have not had the same degree of influence. In the next section, I address what appears to be a negative consequence of this for the small-scale industry, threatening to undermine its status as a sustainable alternative to large-scale exploitation.

\section{The Future of Hydropower}\label{sec:4:6}

In recent years, there has been a growing tension between the small-scale hydropower sector and environmental groups. There is talk of a brewing ``hydropower battle'', as environmentalists grow increasingly critical of what they regard as predatory practices.\footnote{See \cite{haltbrekken12}.}
Reports on small-scale producers who are alleged to have violated environmental regulations help fuel the negative impression of the industry.\footnote{In 2010, the NVE conducted randomised inspections and announced that 4 out of 5 mini and micro plants operated in violation of regulations pertaining to the amount of water they may use at any given time. See \cite{ulovlig10}. In the largest newspaper in Norway, this was reported under the heading that four out of five small-scale plants break the law, see \cite{ulovlig10b}. This is misleading, since mini and micro plants are distinct from small-scale plants proper. Most importantly, the former kinds of plants do not usually require a sector-specific development license. Because of this, it also seems plausible that the reported violations might in large part be due to a lack of knowledge and professionalism, not predation. I remark that questions later emerged regarding the accuracy of the report itself. Apparently, one of the plants that was reported to have violated regulations did not even exist, see \cite{tvilsom10}.}

On the regulatory side, the water authorities have now adopted much stricter procedures to assess licenses for small-scale hydropower.\footnote{See \cite{lie12}.} In addition, different planning routines have been adopted to ensure that small-scale schemes are no longer considered individually, but in so-called ``packages'', collecting together applications from the same area. As a consequence of these changes, the number of rejected small-scale applications have increased dramatically in recent years.\footnote{In 2013, the number of rejections tripled compared to previous years, while the number of accepted applications remained stable. See \cite{sunde14b}.}

At the same time, powerful market actors who favour a traditional mode of exploitation have seized the opportunity to lobby more aggressively against small-scale hydropower, in favour of large-scale projects.\footnote{See, e.g., \cite{alexandersen14}.} Such projects, they argue, are preferable also from an environmental point of view. In recent years, this argument has proven influential in many quarters, particularly among state agencies, such as the NVE and the Norwegian Environmental Agency.\footnote{See \cite{nilsen11}.} It has also been claimed that this perspective is backed up by research done on environmental effects of small-scale and large-scale projects.\footnote{See generally \cite{bakken12,bakken14}.}

The core environmental argument against small-scale solutions has a very simply structure: small-scale plants indirectly affect a greater total area of land per electricity unit produced, therefore they are considered more intrusive than large-scale schemes.\footcite[96-99]{bakken14} The stated premise of this reasoning is no doubt correct; several small-scale plants, at many different locations, are required to match the energy produced by a single larger plant, hence a greater land area will be affected. However, this quantitative observation has no bearing on the issue of how small-scale plants qualitatively affect the surrounding environment, compared to large-scale projects. In particular, the parameters used to compare small-scale and large-scale developments tend to be defined in terms of generic buffer zones that do not take into account differences in the severity of different kinds of environmental intrusions. For instance, as long as both installations are observable by passers by, a small cabin with a turbine inside is considered to have the same ``scenic impact'' as an imposing concrete dam that stretches out for 100 meters and significantly distorts the water level in a lake.\footnote{See \cite[95]{bakken14}.}

\noo{The only buffer zone that is not defined in this way is the {\it scenic} buffer, the area from which some installation can be seen. Here the model takes into account that a large installation should be assessed using a larger buffer zone than a small one, since the former is visible over a greater area. But even for this parameter, no distinction is made based on the actual visual impression; a large dam that dries up a river and makes it possible to regulate the water level in a lake by several meters counts the same as a small cabin with a generator inside, as long as both can be seen.\footnote{See \cite[95]{bakken14}.} For the other parameters, the data analysis is even more dubious, since the buffers are set uniformly according to general rules of thumb.\footnote{See \cite[95]{bakken14}.} For instance, a conflict with a threatened species is assumed to arise whenever a technical installation occurs within a certain distance from its natural habitat.\footnote{See \cite[95]{bakken14}.} Importantly, nothing is said about the severity of conflict, and no distinction is made between a minor installation and a massive disturbance.}

Despite the apparent lack of qualitative arguments, the idea that large-scale development is better for the environment now appears to be gaining ground in Norway. This represents a complete reversal compared to the political narrative that has dominated for the last 15-20 years. Indeed, the merits of small-scale development was strongly emphasised by political leaders around the turn of the century. In his New Year's speech 01 January 2001, the Prime Minister went as far as to declare that the age of large-scale development was over.\footnote{See, e.g., \cite[34]{haltbrekken12}.} The same phrase was then repeated in the policy platforms of two successive national governments, in 2005 and 2009 respectively.\footnote{See the ``Soria Moria'' declaration from 2005, p 57, and ``Soria Moria II'', from 2009, p 52 (available at \url{www.regjeringen.no}).}

However, as administrative practices and case law on hydropower shows, the end of large-scale exploitation has proved impossible to implement. Despite being official policy at the highest level of government for almost 15 years, large-scale development interests continue to dominate in the hydropower sector.\footnote{I believe the material presented in this thesis warrants making this claim. Moreover, it is underscored by the two recent Supreme Court decisions in \cite{jorpeland11} and \cite{otra13}.} Interestingly, the leading national politicians are now changing their position as well.\footnote{See \cite{liemin14} (reporting on recent public statements made by the Minister of Petroleum and Energy in support of large-scale development).} Arguably, this demonstrates how the politicians have responded to pressure from high-ranking bureaucrats and large energy companies.\footnote{In addition to their environmental arguments, these actors also rely on more familiar arguments in favour of large-scale development, especially the idea that large-scale development is more efficient. See \cite{lie12}. For a contrasting view on efficiency, emphasising the efficiency benefits associated with small-scale development and a decentralised approach to hydropower, see \cite[5]{inn101}.}

The political shift observed at present is likely to result in a further weakening of property and the rights of local communities. For example, it provides indirect political legitimacy to the NVE, which pursues an explicit policy of prioritising applications for large-scale projects when these come into conflict with small-scale schemes in the same rivers.\footnote{See \cite[3]{nve12}. See also \cite{lie12}.} Hence, the NVE is likely to refuse to consider applications from owners as long as there are applications pending that might result in the expropriation of their property.\footnote{For a concrete example of this, see \cite{smibelg15}.}

At the same time, the small-scale industry itself has occasionally sought to undermine property rights, 
possibly in an effort to mimic the successes of their large-scale competitors. The industry has argued, in particular, that expropriation should be made more easily available as a tool for small-scale developers and owners who wish to take property from reluctant neighbours.\footnote{See \cite{brekken07,brekken08}. The articles are written by a leading Norwegian energy lawyer, apparently in his capacity as legal representative of ``Småkraftforeningen'', an interest organisation for small-scale hydropower.} This argument rests on a peculiar form of anti-discrimination reasoning; as long as large-scale developers are allowed to take property by force, small-scale developers should be allowed to do the same. In a world where takings are endemic, this might make some sense. However, it is hardly an attitude that helps the small-scale industry preserve its image as the more sustainable hydropower option.

At the same time, the industry is beginning to struggle financially because the price of electricity has been much lower in recent years than what had previously been forecast.\footnote{See \cite{sunde14}.} Moreover, it has become clear that some of the investors on the market have engaged in speculative practices, by aggressively entering into agreements with local owners, without carrying out much hydropower development.\footnote{See \cite{endresen14}.}

These critical remarks should not detract from the fact that the growth in small-scale hydropower has led to dramatically increased benefit sharing with many local owners of rivers and waterfalls. However, recent events indicate that it is inappropriate to look at this development in isolation from other concerns. When assessing the future of small-scale hydropower and local property rights to waterfalls, it seems important to also take into account the broader societal consequences of new commercial practices. If one fails in this regard, the pernicious image of owners as socially passive ``profit-maximisers'' gains a firmer hold both on the political and the legal narrative. The negative consequences this can have for property as an institution are already apparent in Norway, as I will argue in the next chapter. 

More generally, recent developments in the hydropower industry illustrate that an entitlements-based perspective on waterfalls is inappropriate, since local ownership is meant to facilitate sustainable management first, and profit-seeking only second. This insight is also strongly implicit in {\it Nordhordlandsmodellen}. However, as the current debate is evolving, it seems to be at risk of disappearing from view.

\noo{To counter this, I believe the social function view of property must be developed further, so that concrete policy recommendations can be formulated on its basis. The aim, I believe, should be to arrive at frameworks for participatory decision-making regarding hydropower that allows local owners and communities to contribute constructively when society desires commercial development based on  their water rights.

I return to this issue in chapter \ref{chap:6}, where I argue that the Norwegian institution of land consolidation can be used to achieve this. First, I will zoom in on the issue of expropriation, where the mechanisms identified in this section often lead to concrete legal disputes. This will bring into focus important issues surrounding the status of economic development takings under Norwegian law.
}

\section{Conclusion}\label{sec:4:7}

Water resources have been, and still are, very important to Norway as a nation. Not only does the energy of streaming water provide electricity to people and industries, it also provides a source of profit, prestige and power to those who harness it. Historically, many rural communities in Norway benefited greatly from this, as it was they who managed local water resources.

Plainly, they did rather well. By the end of 1943, at a time when small-scale plants still outnumbered large-scale plants 45 to 1, around $80 \%$ of the population had access to electricity at home. The government, especially local governments, also felt responsible for the supply of electricity to the public, but they generally assumed this responsibility without encroaching on local populations who wished to manage their own resources.

As discussed in this chapter, the situation changed dramatically after the Second World War, when the government, especially the central government, assumed more direct control over the nation's water resources. This led to a situation where local owners became increasingly marginalised. In recent years, there has been a partial reversal of this trend, as the liberalisation of the electricity market has enabled local owners and communities to take part in hydropower development once again.

The result has been a growing tension between large-scale and small-scale development, which in turn corresponds to a tension between the owners of waterfalls and the energy companies that wish to expropriate. In the next chapter I will explore this tension in more depth, as I investigate the rules and practices relating to expropriation for hydropower development.