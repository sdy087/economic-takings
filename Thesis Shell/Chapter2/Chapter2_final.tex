\chapter{Norwegian waterfalls and hydropower}\label{chap:3}

\section{Introduction}\label{sec:into3}

Norway is country of mountains, fjords and rivers, and about 95 \% of the annual domestic electricity supply comes from hydro-power.\footnote{See Statistics Norway, data from the year 2011, http://www.ssb.no/en/elektrisitetaar/.} The right to harness rivers for hydro-power is held by local landowners, but historically, this right has not been of much use to them, since the Norwegian electricity sector has been organized as a regulated monopoly, with most hydro-power schemes carried out by non-commercial companies controlled by the state or local governmental bodies. In the early 1990's, however, the sector was liberalized, and it has become increasingly common for local landowners to undertake their own hydro-power projects. This has led to increased tension between local interests and established energy companies. Following liberalization, these companies are now organized for profit. This raises the question: Who is entitled to benefit from Norwegian hydropower?

In this Chapter I present this question as it arises in the context of Norwegian law and business practice. Increasingly, it is becoming clear that this is not merely a question of the opposing commercial interests of individuals and companies, but also crucially involves the local communities directly affected by development. The original owners of hydro-power tend to be farmers residing in the local communities where the resources are found, and therefore, the question of who should benefit also encompasses the question of who should be allowed to participate in decision-making process, and what degree of autonomy local communities are to be granted in this regard. Are local owners and their communities entitled to a say in how the hydro-resource is to be exploited, or must they accept to remain passive, as they were rendered by the monopoly which used to be in place?

In the following, I begin by first presenting a basic overview of the Norwegian political system, focusing on the role that property, and particularly rights to water, has played in Norwegian democracy. I then move on to describe the law relating to hydropower development, by presenting the most important statutes relating to management of water resources. I follow this up by considering the practices followed in hydropower development. After giving a general overview of this, I zoom in on projects based on cooperation with local owners, a mode of development that was non-existent prior to liberalization, but which now occupies an important place on the market. I finish by a case study of a particular model for local involvement in hydropower, dating from before the current business practices for small scale development evolved. The Chapter concludes with a discussion on the extent to which the ideals expressed in this model are embodied in the form of owner-involvement currently seen in the hydropower sector.

\section{Norway in a nutshell}\label{sec:nutshell}

Norway is a constitutional monarchy, based on a representative system of government. The executive branch is led by the King in Council, the Cabinet, headed by the Prime Minister of Norway. Legislative power is vested in the Storting, the Norwegian parliament, elected by popular vote in a multi-party setting.\footnote{It should be noted that the executive branch also enjoys considerable legislative power under Norwegian law, both informally because it prepares new legislation, as well as formally, through delegation of the power to issue so-called {\it directives} (forskrifter).} In 1884, the principle of parliamentarism gained hold in Norway, and it has since obtained the status of a constitutional custom. In particular, the cabinet can not continue to sit if the Storting expresses mistrust against it, although an express vote of confidence is not required. In practice, due to the multi-party nature of Norwegain politics, minority cabinets are quite common. These typically depend either on deals with enough additional parties to secure majority support, or else they operate by looking for a majority in the Storting on a case-by-case basis.

The judiciary is organized in three levels, with 70 district courts, 6 courts of appeal, and the Supreme Court. The district courts have general jurisdiction over most legal matters; there is no division between constitutional, administrative, civil, criminal courts. \footnote{However, a special court exists for {\it land consolidation}, and both the district courts and the courts of appeal follow special procedural rules in {\it appraisement disputes}. I will return to these judicial constructions in Chapters \ref{chap:6} and \ref{chap:5} respectively.} The courts of appeal have a similarly broad scope, and the right to appeal is considered basic, although not absolute.\footnote{In civil cases, for instance, it is generally required that the stakes are above a certain lower threshold, measured in terms of the parties financial interests in the outcome..}  The Supreme Court, on the other hand, operates a very strict restriction on the appeals it will allow. It will typically only hear cases when a matter of principle is at stake, or when the law appears to be in need of clarification.

The Norwegian legal system is generally considered based on a special ``Scandinavian'' variety of civil law, which includes strong common law elements. Legislation is not typically as detailed as elsewhere in continental Europe, some fields of the law lack legislative basis, it is generally accepted that courts develop the law, and the decisions of the Supreme Court are considered crucial for the legislative interpretation of the lower courts. However, at the same time, legislation remains the primary source that is used to resolve most legal disputes, and when applying the law, the courts emphasize the importance of preparatory documents showing the legislator's intent.

The Constitution of Norway dates back to 1814 and was heavily influenced by contemporaneous political movements in the US and France. Moreover, it was influenced by a desire for self-determination, as Noway was at that time a part of Dennmark. Following the Napoleonic wars, the Norwegian political elite fought to secure independence, but in the end, Norway was forced to enter into a union with Sweden. However, the Constitution remained in place, and after the triumph of parliamentarism in 1884, Norway would also eventually gain independence, in 1905. 

During the 19th century, the Norwegian farmers emerged as a powerful group in Norwegian politics. This, in particular, was because they were also landowners, whose rights and contributions were not limited to traditional farming. In addition, Norwegian farmers had a significant voice in the management of land rights and resources more generally. Indeed, the feudal tradition was never as strong in Norway as elsewhere in Europe. 

While the majority of Norwegian farmers were tenants in the 17th century, they generally enjoyed better protection against abuse. For one, the remoteness of the non-coastal countryside, marked by a challenging natural environment, meant that large feudal estates could hardly operate effectively without granting much autonomy to local farmers. In addition, the black death had severely taken its toll on the population in Norway, wiping out entire communities, including the feudal elites and the social and physical infrastructure that sustained them.

Later, when the Danish-Norwegian nobility fell into a fiscal crisis in the 18th century, farmers in Norway began to buy the land from their landlords. As a result, Norwegian farmers were no longer tenants, not  even in name, and the distribution of land ownership in Norway became quite egalitarian. It is worth noting that farmers would typically purchase only parts of larger estates, so that they would acquire sole ownership over their own house and cultivated ground, while becoming co-owners of the surrounding land. Hence, many resources attached to land came to be owned jointly by communities of farmers. In addition, farmers would also often partition their land further, for instance to make room for younger sons to set up their own farms. As a result, Norway became a society were land ownership was not a privilege for the few, but held by the many, at least compared to feudal Europe.

By 1814, the landed nobility was practically eradicated in Norway. Indeed, the Constitution itself provided the death blow, by formally abolishing all noble titles. By the mid 19th century, farmers had gained even greater influence, and they emerged as the leading political class, alongside city bureaucrats and merchants. During this time, Norway also introduced a system of powerful local municipalities, organized as representative democracies. To this day, the municipalities retain a great deal of power in Norway, particular in relation to planning of land use. 

However, in relation to management of water resources, the system is now centralized, and the municipalities occupy a minor role. This is a significant change compared to the early days of Norwegian democracy, when control over water was one of the main assets and sources of power for local farming communities. Indeed, according to Terje Tvedt, as many as ....

In the next section, I will present the basic legal framework surrounding hydropower exploitation today.

\section{Hydropower in the law}\label{sec:hl}

Under Norwegian law, waterfalls are regarded as private property. The system is riparian, so that by default, a waterfall belongs to the owner of the land over which the water flows.\footnote{See Section 13 of \cite{wra00}.} This does not mean that the landowner owns the water as such -- freely running water is not subject to ownership -- but it entitles the owner of the waterfall to harness the potential energy in the water over the stretch of riverbed belonging to him. This right can be partitioned off from any rights in the surrounding land, and large scale hydro-power schemes typically involve such a separation of water-rights from land-rights, giving the energy company the right to harness the energy, while the local landowner retains the rights in the surrounding land.

Norwegian rivers, and especially rivers suitable for hydro-power schemes, tend to run across grazing land owned jointly by farmers, so rights to waterfalls are typically held among several members of the local, rural community.\footnote{The land in question tend not to be enclosed, in particular, and in cases where there has been a land enclosure, water-rights have often explicitly been left out, such that they are still considered common rights, belonging to the community of local farmers.} They might not always be willing to give them up, especially not on the terms proposed by the developer, so the use of expropriation has played an important role in the history of Norwegian hydro-power. This has meant that the terms governing separation of water-rights from land-rights, including the level of compensation paid to landowners, and the influence they are granted in the decision-making process, has been determined by the law. 

Following legislation in the early 20th century, a regulatory system was put in place that centralized the management of Norwegian water resources. It clearly favored exploitation by the state or by companies owned by local governmental bodies, and the local landowners were severely marginalized. In most cases, they would have to accept the terms presented to them by the developer, or else argue the matter in Court, after the developer had already been granted a license to expropriate. 
Landowners were not in a good position to negotiate the terms of the development, and their property rights appeared increasingly nominal, the prevailing political attitude being that waterfalls formed part of the common heritage of the Norwegian people, and should be managed in their interest.\footnote{While some of the claims made here will be further qualified by what is to follow, the general picture we paint here is communicated also by the standard work on Norwegian water law \cite{falk}.}

This created a legal tension where, on the one hand, waterfalls were still considered private property under land law, yet, on the other hand, were considered as belonging to the public as far as large scale hydro-power development was concerned. The following two quotes, the first from the general water law, with roots going back at least to the 19th Century, and the second  from a law directed specifically at large scale hydro-power, illustrates this ambivalence.

{\begin{minipage}[t]{16em}
 \begin{aquote}{\tiny Section 13, Water Resources Act 2000} \footnotesize A river system belongs to the owner of the land it covers, unless otherwise dictated by special legal status. [...]

The owners on each side of a river system have equal rights in exploiting its hydro-power...
\end{aquote}  
\end{minipage}}
{\begin{minipage}[t]{22em}
\begin{aquote}{\tiny Section 1, Industrial Concession Act 1917 (amended 2008)} \footnotesize Norwegian water resources belong to the general public and are to be managed in their interest. This is to be ensured by public ownership...
\end{aquote}
\end{minipage}} \\

Following liberalization of the Norwegian energy sector in the early 1990's, this legal tension in statute has increasingly also become a tension in politics, where the interests of local communities and landowners stand in opposition to the interests of large energy companies, often owned by the State, seeking to harness locally owned resources for commercial gain. The question of how the Norwegian legal framework is actually applied in this regard is therefore a matter that has come under increased scrutiny. Before we delve into the details, we will elaborate a bit further on the context in which the law was called upon to function in this case. Importantly, the economic, social and political context of expropriation has changed rather dramatically in recent years.

There are two developments that have been particularly important. First, there has been a general shift from viewing electricity production as a public service to viewing it as a commercial enterprise. This has made the legitimacy of expropriation appear more controversial, and the argument is often voiced that expropriation does not happen in the interest of the public at all, but \emph{solely} in order to benefit the commercial interests of particular companies.\footnote{This has been a recurring theme in articles appearing in "Småkraftnytt", the newsletter for "Småkraftforeninga", an interest organization for owners of small-scale hydro-power, which currently have 236 associated small scale hydro-power plants, see http://kraftverk.net/ (in Norwegian). In addition to the case of Måland, the question has also been brought before the (lower) national courts in some other cases, such as \emph{Sauda}, LG-2007-176723 (Gulating Lagmannsrett, regional high court), and \emph{Durmålskraft}, see http://www.ranablad.no/nyheter/article5583405.ece (decision from the district court, as reported in a Norwegian newspaper). In both cases, the outcome was generally more favorable to the expropriating party than the local owners, and the reasoning adopted by the courts appears similar to that of \emph{Måland}.}

In this way, expropriation of Norwegian waterfalls raises issues that have become increasingly important also in a global setting, and which seem to arise naturally in systems where economic activities are organized based on public-private partnerships. In such systems, it seems practically inevitable that cases of expropriation -- undertaken to benefit the public -- will also often come to benefit developers that are motivated by purely commercial interests. While this in itself might not be problematic, it will easily lead to the concern that the commercial interests of powerful companies is the main reason why expropriation is permitted, and that expropriation is being used as a commercial tool for powerful market forces, to the detriment of less powerful actors.

For the case of Norwegian waterfalls, however, liberalization of the energy sector has also had a positive effect for local communities, in that it has served to make local owners more active. It has become increasingly common that they exploit their hydro-power resources themselves, often in small scale projects, and often in cooperation with companies that specialize in such development.\footnote{In 2012, the NVE granted 125 new licenses for small scale hydro-power, and at the end of the year they had 859 applications still under consideration. Source: report made by the NVE, available at http://www.nve.no/Global/Energi/Q412\_ny\_energi\_tillatelser\_og\_utbygging.pdf (in Norwegian). } This, of course, only adds to the controversy surrounding expropriation of waterfalls, especially when local owners are deprived of the opportunity for small scale development.

\subsection{Water Resources Act}

The act which sets out the basic rules regarding water management in Norway is the \cite{wra00}.\footnote{Act relating to river systems and groundwater of 24 November 2000 No. 82 (unofficial translation provided by the University of Oslo, \url{http://www.ub.uio.no/ujur/ulovdata/lov-20001124-082-eng.pdf}).} This act is not only concerned with hydropower, but regulates the use of river systems and groundwater generally.\footnote{See \cite[1]{wra00}. A river system is defined as ``all stagnant or flowing surface water with a perennial flow, with appurtenant bottom and banks up to the highest ordinary floodwater level'', see \cite[2]{wra00}. Artificial watercourses with a perennial flow are also covered (excluding pipelines and tunnels), along with artificial reservoirs, in so far as they are directly connected to groundwater or a river system, see \cite[2a-2b]{wra00}.} 

In section 8, it sets out the basic license requirement for anyone wishing to undertake measures in a river system.\footnote{Measures in a river system are defined as interventions that ``by their nature are apt to affect the rate of flow, water level, the bed of a river or direction or speed of the current or the physical or chemical water quality in a manner other than by pollution'', see \cite[3a]{wra00}.} The main rule is that if such measures may be of ``appreciable harm or nuisance''  to public interests, then a license is required.\footnote{See \cite[8]{wra00}. There are two exceptions, concerning measures to restore the course or depth of a river, and the landowner's reasonable use of water for his permanent household or domestic animals, see \cite[12|15]{wra00}.} Further rules regarding the licensing procedure is set out in Chapter 3 of the Act. 

In section 18, it is made clear that the water authorities are competent to decide whether a given measure requires a license pursuant to section 8.\footnote{See \cite[18]{wra00}.}  They are obliged to issue such a decision if the developer, an affected authority, or others with a legal interest request it. In addition, they may prohibit implementation before the decision is reached. In relation to hydropower development, it is established practice that most hydropower projects over 1000 kw will be deemed to require a license.\footnote{See, e.g., \url{http://www.nve.no/no/Konsesjoner/Vannkraft/Konsesjonspliktvurdering/} (accessed 16 August 2014). Exceptions are possible, for instance projects that upgrade existing plants, or which utilize water between artificial reservoirs.}

The basic criteria for granting a license is given in section 25, which states that a license ``may be granted only if the benefits of the measure outweigh the harm and nuisances to public and private interests affected in the river system of catchment area''.\footnote{See \cite[25]{wra00}.} The procedure that the water authorities follow in licensing cases is largely determined by regulation passed by the Ministry of Petroleum and Energy pursuant to section 65, and by administrative practice developed by the  NVE itself. The rules regarding decisions in the \cite{ac67} also apply, but play a relatively minor role compared to the special procedures developed by the water authorities. While most of these practices are not encoded in the \cite{wra00}, some rules are given there. 

This includes rules to ensure that the application is sufficiently documented and that the authorities have sufficient information to assess its merits.\footnote{See \cite[23]{wra00}.} Moreover, a basic publication requirement is enforced, which states that applications are public documents and that the applicant is responsible for giving public notice, so that interested parties may comment on the plans.\footnote{See \cite[24]{wra00}. There are some exceptions to the requirement to give public notice, in case it is superfluous, or the application must be rejected or postponed, see \cite[24a-24c]{wra00}.} More detailed rules for public notice of applications are given in section 27-1 of the Planning and Building Act, which also applies here

In addition, an important rule is given in section 22, regarding the relationship between applications for licenses and governmental ``master plans'' for the use or protection of river systems in a larger area. This rule states, in particular, that if a river system falls within the scope of a plan that is currently being prepared, an application to undertake measures in this river system may be delayed or rejected without further consideration.\footnote{See \cite[22]{wra00}, first paragraph.} Also, a license may only be granted if the measure is without appreciable importance to the plan.\footnote{See \cite[22]{wra00}, first paragraph.} Furthermore, once a plan has been completed, the processing of applications is to be based on it, so that applications that are at odds with the plan may be rejected without further consideration.\footnote{See \cite[22]{wra00}, second paragraph.} It is possible to obtain a license for such a project, but if it results in less hydropower than the use indicated by the plan, only the Ministry may grant it.\footnote{See \cite[22]{wra00}, second paragraph.}

The rules so far all apply to any measures in river systems, not only hydropower projects. Many special procedures exist for this category of cases, and while most of them are administratively sanctioned, some are also provided for in other acts. In section 19, the relationship between the \cite{wra00} and the \cite{wra00} is described. The latter act contains many special rules that apply when a hydropower project involves regulation of a watercourse (see Section \ref{sec:wra17} below). However, many of these rules also apply for run-of-river hydropower schemes, as long as it will result n mean annual generation above 40 GWh.\footnote{See \cite[19]{wra00}}. In the next section, I will present these rules in more detail. 

\subsection{Watercourse Regulation Act}

Hydropower schemes often involve special measures undertaken to regulate the water flow used to produce electricity. Regulation was particularly important in the early days of hydropower, before the national electricity grid was developed. Consumers did not want to pay for more energy than they needed during rainy days, and they wanted to avoid power cuts in periods of drought. Since a few local hydropower plants were typically the only sources of electricity in an area, this meant that regulation of the waterflow was needed to even out the level of output. Indeed, in the early days, it was common for electricity producers to get paid based on the stable effect they were able to deliver, rather than the total amount of energy they harnessed. 

This changed with the development of a large-scale electricity grid, which allowed electricity to be imported and exported between different geographical areas depending on the levels of energy output from those areas. Today, electricity producers get paid based on the total amount of electricity produced, measured in kilowatt hours (KWh). The price fluctuates over the year, and the supply-side is still influenced by instability in the water flow in Norwegian rivers. However, prices do not fluctuate so much as to make increased production during peak periods unprofitable. In addition, technological advances have ensured that the generators needed to exploit fluctuating levels of water have become much cheaper. Hence, unregulated hydropower projects can be highly profitable. They are also often seen as less environmentally controversial.

However, due to the historical context of the present legal system for hydropower exploitation, many key rules regarding hydropower are found in an act that deals specifically with measures undertaken to regulate the waterflow in a river system, namely the \cite{wra17}.\footnote{Act relating to the regulation of watercourses of 14 December 1917 No. 17.} The Act defines regulations as ``installations or other measures for regulating a watercourse's rate of flow'', explicitly including installations that ``increase the rate of flow by diverting water''.\footnote{See \cite[1]{wra17}.} The core rule of the Act is that regulations that affect the rate of flow of water above a certain threshold is subject to a special licensing requirement.\footnote{See \cite[2]{wra17}.} The threshold is defined in terms of the notion of a ``natural horsepower''- We will return to this notion and the threshold in more depth in Chapter \ref{chap:5}, but for now it suffices to say that most regulations undertaken in the context of a hydropower scheme require a special license. In addition, as we mentioned above, the \cite{wra00} contains an important rule that gives effect to several of the most important rules in the \cite{wra17} for all projects that will generate more than 40 GWh annually.\footnote{See \cite[19]{wra00}.}

The criteria for granting a regulation license mirror those for granting licenses pursuant to the \cite{wra00}. In particular, section 8 of the \cite{wra17} states that a license should ordinarly be issued only if the benefits of the regulation are deemed to outweigh the harm or inconvenience to public or private interests.\footnote{See \cite[8]{wra17}.} In addition, it is made clear that other deleterious or beneficial effects of importance to society should be taken into account.\footnote{See \cite[8]{wra17}.} Finally, if an application is rejected, the applicant can demand that the decision is submitted for review by the Storting.\footnote{See \cite[8]{wra17}.}

The \cite{wra17} contains more detailed rules regarding the procedure for dealing with license applications. The most practically important is that the applicant is obliged to organize an environmental impact assessment pursuant to the \cite{pba..}. This means ....

Also, the \cite{wra17} contains more specific rules about the public hearing that should take place before the application is decided. The applicant should make sure that the application is submitted to the affected municipalities and other interested government bodies.\footcite[6]{wra17} In addition, organizations, associations and the like whose interests are ``particularly affected'' should be sent a copy of the application.\footcite[6]{wra17} Along with the application, they should be given notification of the deadline for submitting comments, which may not be less than three months.\footnote{See \cite[6]{wra17}.} The applicant is also obliged to announce the plans, along with information about the deadline for comments, in at least one commonly read newspaper, as well as the Norwegian Official Journal.\footnote{\cite[6]{wra17}. The Norwegian Official Journal is .....}

The applicant is also, to the extent that the water authorities find reasonable, responsible for compensating landowners and other interested parties for expenses accrued in relation to legal and expert assistance sought in relation to the application.\footcite[6]{wra17} 

While a license to regulate might be hard to obtain, and requires the applicant to undertake more elaborate preparatory steps, it also comes with a significant benefit. If a license is granted, in particular, the license holder automatically has a right to expropriate the necessary land and other rights needed to undertake the regulation, including the right to inconvenience other owners.\footnote{See \cite[16]{wra17}.} This rule has raised particular controversy with respect to local landowner's rights, who have claimed that making expropriation a side-effect of a regulation license is an illegitimate way of interfering in property rights. In practice, moreover, it is clear that the issue of expropriation rarely receives separate treatment in regulation cases. The assessment undertaken by the water authorities is completely centered on the licensing issue, which does not compel them to direct any special attention towards owners' interests. 

In general, the issue of who own and controls the resources in question receives little attention in relation to both the \cite{wwra17} and the \cite{wra00}. The focus is almost entirely on the weighing of environmental interests against the interest of economic development. However, a third act deals specifically with the issue of control and transfer of control over waterfalls. This is the \cite{ica17}, to which I now turn. 

\subsection{Industrial Concession Act}

In the early 20th century, industrial advances meant that Norwegian waterfalls became increasingly interesting as objects of foreign investment. To maintain national control of water resources, the Storting passed an act that made purchase of particularly valuable waterfalls pursuant to a special license. The act which regulates this is the \cite{ica17}.\footnote{Act relating to acquisition of waterfalls, mines, etc. of 14 December 1917 No. 16.} It applies to the purchase or lease of waterfalls that may be exploited in such a way as to yield more than 4000 ``natural horsepowers''. I return to this criterion in Chapter \ref{chap:5}. Here I note that many run-of-river hydropower projects, even large-scale ones, fall outside the provision. Only those that produce more than approximately 100 GWh annually are likely to be affected. Even projects that involve some significant degree of regulation may not be covered, but in this case the threshold is much lower. 

Originally, the main rule in the \cite{ica17} stated that all non-State licenses were time-limited and that the hydropower station, with all associated rights and constructions, would be transferred to the state without compensation after a given period of time, at most 60 years.\footnote{See \cite[2]{ica17} (before amendment on ....).} This was known as the rule of {\it reversion} in Norwegian law. In ....., it was deemed to be in breach of the EFTA agreement, since it only applied to private, not state-owned, companies.  After this ruling, the Act was amended so that today {\it only} companies where the state controls more than 2/3 of the shares may acquire waterfalls subject to the Act.

This means that such waterfalls can only be bought, leased or expropriated by state-controlled companies. However, in practice, landowners are still able to sell the land from which the right to the waterfall originates, even if this also means transferring the waterfall to a new owner.  Moreover, local owners may in theory still develop hydropower in these waterfalls, since they already own them. However, this would be difficult in practice due to the fact that they would not be able to partition off the waterfalls, to make them available as stand-alone security for debt commitments. In effect, local owners would have a hard time acquiring financing for projects in such waterfalls, particularly if they were unwilling to put their farms at risk. 

In practice, a development license might also be hard to obtain in such a case, since the Norwegian government appears to take the view that hydropower projects in waterfalls falling under the \cite{ica17} should only be undertaken by companies in which the state has at least 2/3 of the shares. So far, I am not aware of any case that raises this issue in the context of a conflict between local owners and such a company. It is likely that the authorities could use section 2 of the \cite{ica17} to effectively deprive the owners of any possibility for making use of their waterfall, without this being regarded as expropriation under Norwegian law. However, it is not clear if the government is also {\it obliged} to do this, as the wording in section 2 might be taken to suggest. Moreover, it is not clear what the consequences will be for the level of compensation payable to the original owners, in case the waterfall is subsequently transferred to a state-controlled company wishing to develop it. 

I return to this issue in Chapter \ref{chap:5}. In the next section, I look at the basic legislation that set up a liberalized system for production and distribution of electricity in Norway.

\subsection{Energy Act}\label{sec:ea}

Before 1990, the Norwegian energy sector was organized as a state monopoly, with several local monopolists, owned by the state or the municipalities, being responsible for energy supply and distribution in their region. There was no real competition on the market, neither on the supply side nor on the demand side. The local monopolist could deny other energy producers access to the distribution grid, and consumers were not in a position to freely choose from whom they wanted to purchase electricity.

This changed with the passing of the \cite{ea90}.\footnote{Act relating to the generation, conversion, transmission, trading, distribution and use of energy etc. of 29 June 1990 No. 50.} This act set up a new regulatory framework, where management of the grid was decoupled from hydropower production. In particular, the act served to set up a system whereby consumers could choose their electricity supplier freely, while it also ensured that producers were granted non-discriminatory access to the electricity grid. It laid the groundwork for what has today become an international market for the sale of electricity, namely the NordPool. 

The Energy Act introduced the principle that energy consumers and producers should have non-discriminatory access to the national electricity grid, thereby creating a market where any actor, privately owned or otherwise, could supply electricity to the grid, and profit commercially from hydro-power. In the same period of time, monopoly companies were reorganized, becoming commercial companies that were meant to compete against each other, and against new commercial actors that entered the market.\footnote{For a short English summary of how the system is administered, see for instance \cite[p.29-30]{ar2010}, and for more detail, we point to \cite{Hammer2}.}

The Norwegian State retained a significant stake as shareholders in energy companies, however, now often alongside private investors. Moreover, many rules in Norwegian law favor companies where a majority of the shares are held by the State, and to this day the largest and most influential Norwegian energy companies remain under public ownership.\footnote{The fact that publicly owned companies are favored in this way is often seen as a questionable practice with regards to competition law, see for instance the recent EFTA Court case, Case E-2/06, \emph{EFTA Surveillance Authority v. The Kingdom of Norway}, EFTA Court Report 2007, p.164. Here, the Court considered the old Norwegian rule of \emph{reversion}, whereby a license to undertake certain large scale hydro-power schemes (strictly speaking, a license to acquire the waterfalls needed to undertake it) came with a special clause that the private developer had to give up ownership to the State after a fixed period of time. This clause was held to be in breach of the EEA agreement since it only applied to private companies. We remark that the Norwegian government responded to this with an amendment after which reversion no longer applies, but which stated that a license to acquire waterfalls for the purpose of such large scale schemes can not be given at all to any company in which private parties own more than 1/3 of the shares.}

It seems, in particular, that the aim of liberalization in Norway has never been to minimize State control over hydro-power, but rather to give consumers greater freedom in choosing their energy-supplier, and to enhance efficiency in the sector by introducing competition.\footnote{See for instance \cite{liberal}, which offers a comparative study of the liberalization of the energy sectors in Norway and the UK.} Still, the fact that any developer of hydro-power is now legally entitled to connect to the national grid has proved important in giving actors that are not owned by the State a fighting chance on the Norwegian energy market. It has been especially important for local owners of waterfalls, since it means that if they undertake hydro-power projects themselves, they can no longer be refused access to the grid, but will be in a position to benefit commercially.

It should be noted that the Norwegian grid is operated by regional companies, responsible for the supply and distribution of electricity in their region. These will typically also be energy producers themselves, and historically, they would prevent other hydro-power initiatives by refusing them access to the grid. In fact, in the early days on Norwegian hydro-power, in the first half of the 20th century, there were quite a few locally owned and operated power plants, often providing local communities with electricity. When the national grid was established, most of them were closed down, often as a result of an explicit policy on part of the authorities. To increase the cost-effectiveness of the companies responsible for providing the national service, these companies were often allowed to demand, as a condition for allowing local communities access to the grid, that local hydro-power plants had to be shut down.\footnote{See \cite[p.111]{Hindrum} (in Norwegian).}

The \cite{ea90} also contains provisions that lay down basic rules regarding various kinds of licenses needed to construct and operate energy installations, including those needed for the development of hydropower. Usually, however, these licenses are not as controversial or hard to obtain as those pertaining directly to the hydropower plant itself. By far the most important aspect of the act for hydropower development, particularly in cases when different parties are interested in carrying out development, concern the right to connect to the grid. However, this right can become illusory due to the fact that the local grid company -- a monopolist -- has the right to demand contributions from hydropower developers, in so far as the grid needs to be improved in order to handle the output from their hydropower schemes. 

The size of the contribution required, and the technical basis for calculating it, is largely determined by the grid company itself. In many cases, the costs can become so great as to prohibit  profitable development. This can result in conflict, particularly in cases when the grid company is affiliated with a competing hydropower developer who has a commercial interest in preventing other developers from connecting to the local grid. This issue arises in many cases involving expropriation, as the beneficiary is often also the local grid monopolist. In effect, the taker of the waterfall is tasked with calculating the cost of connecting alternative, owner-led, projects to the grid, a crucial factor in determining the  compensation payable. I return to this issue in Chapter \ref{chap:5}. 

In the next section, I consider a recent piece of legislation that creates an additional financial incentive for carrying out hydropower development, by introducing a certificate system to subsidize 
environmentally friendly energy.

%The most significant step towards liberalization of the Norwegian energy sector was made in 1990 when the Energy Act was passed, an important new piece of statute reorganizing the system for the distribution of electricity.\footnote{Act nr. 50 of 29 of June 1990 relating to the generation, conversion, transmission, trading, distribution and use of electricity.} 

\subsection{Electricity Certificate Act}\label{eca11}

In the 1960s and 70s, hydropower projects in Norway often sparked great controversy, with environmental groups in particular protesting what they saw as unjustifiable destruction of nature in the interest of economic development. Today, however, the environmental interests in hydropower are more divided. On the one hand, many still regard hydropower skeptically as destruction of nature. On the other hand, the increased focus on global warming has led many environmentalists to embrace hydropower as a renewable energy source.

In 2011, the \cite{eca11} was passed, to set up a market for trade in so-called ``green energy'', which would effectively subsidize the further development of renewable energy, including hydropower.\footnote{Act relating to electricity certificates of
24 June 2011 No. 39.} The basic building block of the new market is the electricity certificate, which will be issued to all renewable energy projects completed before 2020. The demand for such certificates is then created artificially, as the Act stipulates that energy suppliers and certain categories of end-users are required to purchase certificates based on their electricity consumption. 

The Act sets up an incentive to develop hydropower, and it contributes significantly to the profitability of hydropower development.

\subsection{A step-by-step presentation of an application to develop hydropower}

In this section, I will present the steps typically involved when a developer wishes to undertake a hydropower scheme. I will focus on the licenses that must be obtained, and the presentation is tailored towards presenting the main statues regarding hydropower development in Norway. 

The management of water resources in Norway is centralized, and at the lowest level of authority we find the Norwegian Water Resources and Energy Directorate (NVE), which is a national body, based in Oslo. In some cases they have been delegated authority to grant development licenses themselves, but in most cases of large scale development, they only prepare the case, then hand it over to the Ministry of Petroleum and Energy which then gives its recommendation to the King in Council, who makes the final decision. 

The local municipalities, while generally quite powerful under Norwegian law, are completely sidelined when it comes to management of water resources, and their role is mostly limited to commenting on the plans, alongside other stakeholders.\footnote{Although there seems to be a good case to be made that they could exert greater influence over the process, based both on general planning law, which empowers them a great deal, or on special rules set out in agricultural law, which requires them to approve, on a case by case basis, any shifts in the property structure of agricultural land. In practice, however, they almost never exercise any of these powers with respect to water resources. If the developer has a license to undertake the scheme itself granted by the King, then it seems that municipalities most often take it to mean that they are obliged to follow suit, by granting the (relatively speaking) minor licenses that might be required with respect to general planning law and agricultural law.}

The first step in the hydropower application process is for the developer to determine exactly what kind of licence he requires. Roughly speaking, the Norwegian system operates with five different categories of hydropower development, for which different kinds of licences are required.

\begin{itemize}
\item Concession free schemes: Projects that the water authorities regard as relatively uncontroversial from a water management perspective. If a developer obtains permission from the NVE to develop a concession free scheme, he does not require any licenses from the NVE. A regular planning permission from the local municipality will suffice.
\item Small-scale schemes: Projects that require a special development licencse from the water authorities, pursuant to the \cite{wra00}, but such that section ... of that Act does no apply. This means, in particular, that the licence can be given without regard to the rules in the \cite{wra17}. In addition, the NVE have been delegated power to make the final decision in such ases, and they usually do so.
\item Large-scale schemes: Projects that fall under the \cite{wra00}, but in such a way that section .... applies. This means that several sections of the \cite{wra17} applies, even though the scheme does not need a seperate reguation licence. It is more uncommon for the NVE to make the final decision in these cases. They are usually looked at by the Department of Oil and Enery, and concession is granted by the King in Council. 
\item Regulation schemes: Projects that require a regulation licence pursuant to the \cite{wra17}. These projects also require a license pursuant to the \cite{wra00}, but the regulation license is usually considered the main issue in such cases. The rules of {\it reversion} apply to these cases, meaning that (some of) the hydropower installations fall to public ownership without compensation after a fixed number of years, usually 60.
\item Industrial concession schemes: These projects are almost always large-scale and often also regulation schemes. In addition, they are deemed to be so important that they fall under the \cite{ica17}. This means that a special licence is required already for the development compant to acquire the necessary rights in the waterfall, either by purchase, a lease, or expropriation. A license will not be granted unless the public owns at least 2/3 of the shares in the development company.
\end{itemize}

watercourse regulation, as provided for in the Watercourse Regulation Act of 1917, Section 8.\footnote{Act No. 17 of 14 December 1917 relating to Regulations of Watercourses.} As is customary, the application also included an application for a license to acquire waterfalls, as set out in the Industrial Concession Act, and a right to expropriate necessary rights from local owners, as provided for in the Water Resources Act, Section 51 and the Expropriation Act, Section 2 nr. 51.\footnote{Act No. 16 of 14 December 1917 relating to Acquisition of Waterfalls, Mines and other Real Property, Act No. 82 of 24 November 2000 relating to River Systems and Groundwater and Act No. 3 of 23 November 1959 relating to Expropriation.} In practice, it has not been common to consider such applications separately, but to consider the project as a whole, and to raise issues with respect to special provisions, and particular licenses, only in so far as they arise in connection with assessing the application for a development license, which is considered the main issue.


For the local owners of waterfalls, the situation is worse, since they are not identified as stakeholders in large scale projects. They are not, in particular, mentioned in the Watercourse Regulation Act, Section 6, which regulates the steps that must be taken when preparing such cases.\footnote{Nor do the seem to be mentioned  in any of the documents setting out how the authorities deal with such cases in practice. See, for instance, the guide published by NVE \cite{rettleiar} (in Norwegian), directed at applicants, and setting out how NVE deals with cases involving large scale hydro-power.} Consequently, it is hardly surprising that in administrative practice, it has been uncommon to devote particular attention to local owners. Rather, the focus has typically been on environmental issues and the opinions of various interest groups, such as hunter or fishermen's associations.\footnote{For a more in depth account of the process, we point to the standard legal reference on Norwegian water law \cite{falk}(in Norwegian).}

\section{Hydropower in practice}

Before the introduction of a centralized system of management of water resources, local farmers made extensive use of the power inherent in water. According to Terje Tvedt, there were 20 -- 30 000 watermills in Norway by the 1830s.\footnote{References} These were not used to produce electricity, but as grist mills, to create flour. This pattern of use rendered waterfalls of limited interest to outside investors or the state, so no significant pressures were placed on the local communities and their exercise of self-governance. On the other hand, Tvedt also stresses the importance of water, including its political implications, and how they gave local communities a better basis on which to claim a right to self-governance also in other matters. Indeed, the control over water, emerging as a side-effect of the control over land, played an important role in empowering Norwegian farmers and their communities in the 19th century. 

Towards the end of the 19th century, local self-governance of water came under increasing pressure from outside investors. This happened in response to increased interest in hydropower as a potential source of energy for industrial exploits. Indeed, more and more waterfalls were purchased as mere objects of speculation, often by foreigners. As a result, the Norwegian government felt compelled to introduce regulation to protect national interests. This led to the passing of the predecessors of the \cite{ica17} and \cite{wra17}, referred to in Norway as the ``panic acts'' of 1909. They set up the basic license requirement for purchase of waterfalls and development of large-scale hydropower involving regulation, while also creating a space for the state as a marker player, by introducing the rule of reversion. 

Still, however, development of hydropower was mostly undertaken by private companies, either large industrial companies like Norsk Hydro AS, or else community-owned companies that wanted to supply electricity locally. It was not until after WW2 that the state assumed the role as the leading hydropower developer in Norway. This was also when the national electricity grid was established and put under direct state control. Previously, the grid had been operated by a mix of state and private actors, who had organized themselves in a joint umbrella organization. Now, however, hydropower development for electricity production was linked with management of the grid, by the establishment of municipal and state-controlled electricity providers, organized as public bodies rather than commercial companies. 

This area saw the extensive use of expropriation to facilitate hydropower. In1940, an act was passed which provided the necessary authority to compulsorily acquire waterfalls for hydropower. This act explicitly stipulated that expropriation should only take place in favor of projects that would serve the electricity supply in the local area. Hence, at this time, the use of expropriation to further the development of hydropower was specifically linked to the idea that the electricity supply should be organized as a non-commercial public service.

During this time, many local hydropower plants that had previously supplied electricity locally were shut down. In particular, they were not allowed to connect to the national grid. Moreover, the grid authorities would often explicitly require these plants to shut down as a precondition to allow the local community to connect to the national grid. The waterfall owners were left marginalized during this time, as they had no real opportunity to develop hydropower themselves, or sell their rights to anyone else than the local monopolist. 

However, as I will discuss in more detail in Chapter \ref{chap:5}, a system was developed -- not through legislation, but by the appraisal courts -- to ensure that the owners would receive at least some compensation for their waterfalls. This was despite the fact that under the prevailing regulatory regime, waterfalls were practically worthless to local owners. Hence, the compensation measures adopted reflected the tradition for local control over waterfalls. If such measures had not been introduced, in particular, local owners would hardly be entitled to any compensation at all under ordinary Norwegian expropriation law.

Following liberalization of the energy sector, the basic premise for this way of organizing hydropower development was lost. In particular, hydropower development was now to be regarded as a commercial enterprise, within a market-based system. This removed the conceptual legitimacy of a system that marginalized local owners. Moreover, the most important practical impediment to owner-led development was also removed, as non-discriminatory access to the grid was provided for in statute. 

As a result, there has been a surge of interest in the exploitation of waterfalls in owner-led hydropower schemes. In addition, the established energy companies have, to some extent, embraced this new situation by competing also in attempting to strike deals with the local owners. In this way, they recognize the right of self-governance and owner-management of hydropower, thereby helping to further undermine the rationale behind using expropriation. 

But not all companies have responded in this way. Many of the largest and most influential companies ahve refused to alter their practices, and expect that they will still be able to enjoy the use of expropriation. These actors, which are often partly state-owned, refuse to recognize the right of waterfall owners to assume a decisive role in the mangement of their resource. Moreover, the are unwilling to enter the market for waterfalls in competition with actors that wish to collaborate with local owners. This has created considerable tension in recent years. 

%As we have mentioned, the typical owners of Norwegian waterfalls are communities of farmers and smallholders. Historically, the right to land-based resources, especially in the mountainous areas of the west and the north, where most valuable waterfalls are located, was held by local people, the same people who made use of it on a day to day basis. The main reason for this, which by European standards stands out as quite unusual, was that Norway never really had a separate class of landed nobility. Consequently, the Norwegian farmer occupied a position of relative autonomy and freedom, even to the point of exercising significant political influence, especially in the early days of Norwegian democracy.\footnote{During the 19th century the two dominant group in Norwegian politics were the farmers and the civil servants, and the former group exercised great influence in the Norwegian parliament, with the 1833 election leading to what became known as the farmers parliament. The "classic" academic treatment of farmers' influence over 19th century Norwegian politics is \cite{Koht} (in Norwegian). More on the author and a summary of his views can be found here, http://en.wikipedia.org/wiki/Halvdan\_Koht.} Following industrialization, however, their role became much more marginal, and farming has steadily become more and more unprofitable, with many farming communities having already disappeared, and many others threatened by depopulation. In light of this, the possibility of undertaking small scale hydro-power is often seen as being important to the survival of rural communities themselves, not just as a means for individual members of such communities to make a profit.

Despite resistance from powerful market actors, the influence of local owners on the hydropower sector has been significant in recent years. This has been helped also by technological advances that entail that today, even relatively  small-scale hydropower can become highly profitable. Moreover, such projects are often regarded as less environmentally controversial, with the Prime Minister himself going as far as to declare that the ``days of large-scale hydropower are over''.  Clearly, small-scale projects will be easier to organize and carry out by local owners, and will also help create a situation where the market can accommodate many players, helping to ensure that competition can become a reality rather than just an ideal. 

In a recent report, it was estimated that there is a potential for profitable small scale hydro-power of about 20 TWh/year \cite{Aanesland}, with a total value, before investment, of about 70 billion Norwegian kroner, i.e., about 8 billion pounds.\footnote{For comparison, suggesting the scale of this potential, we mention that the total consumption of electricity in Norway in 2011 amounted to 114 TWh, see http://www.ssb.no/en/energi-og-industri/statistikker/elektrisitetaar.}  This report was based on a particular model of cooperation with a commercial company, Småkraft AS, and might be an underestimate of what small scale hydro-power could represent for local communities if they take a more independent role in developing the resource. 

Thus, small scale development of hydro-power has become socially and political significant, and it is increasingly seen as a possibility for these regions to counter depopulation and poverty, while also regaining some of their autonomy and influence with respect to how the natural resources found locally are to be managed. In many cases, small-scale hydro-power appears to be the only growth industry, and takes on great political and social importance for the community as a whole, not just the owners of waterfalls.\footnote{For an example of a community where small scale hydro-power has played such a role, we can point to Gloppen, a municipality in the county of Sogn og Fjordane, in the western part of Norway. 19 schemes have already having been successfully carried out, all except one by local owners themselves, amounting to a total production of over 250 GWh/year. This prompted the mayor to comment that "small scale hydro-power is in our blood", see \cite{Gloppen}. When interviewed, he also directed attention at the fact that hydro-power had many positive ripple effects, since it significantly increased local investment in other industries, particularly agriculture, which had been severely on the decline.}

In the next section, I consider an early model for local involvement in hydropower development. It sets out a conceptual basis on which the cooperation between local owners, communities and professional developers can take place. It is interesting because it is an early expression, from 1996, of some of the key principles that have later come to characterize the organization of small scale hydropower projects. I will also highlight how certain aspects of the model have not been widely adopted in practice. This concerns mainly those aspects that pertain to the balance of power between the owner and the professional developer, as well as the involvement of the larger community of non-owners, including environmental groups and interests. 

\section{{\it Nordhordlandsmodellen}}

Norhordlandsmodellen was presented at a seminar in the fall of 1996, alongside the ``little cookbook'' for small-scale hydropower development.  It was authored by Otto Dyrkolbotn and  Arne Steen, a director at Nordhordland Kraftlag, a municipality-owned energy company. In five points, it sets out a framework for cooperation between waterfall owners, professional energy companies, local communities, and greater society.

The first point makes clear that the main premise for cooperation is local ownership and control, with outside interests never taking more than 50 \% of the shares in the development company. If this company is organized as a limited liability firm, then the plan stipulates that local residents -- not necessarily owners -- are to be given a right of preemption in the event that shares come up for sale.
The possibility of organizing the development company as a local cooperative is also mentioned. 

The second point of the model sets out the principle for valuating the waterfalls as natural resources, before any development has taken place. It stipulates that the appraisement should reflect the real value of the resource, and goes on to propose that this should usually be done on the basis of a lease model, where the owner of the waterfall is entitled to rent based on the level of annual production. For the purpose of appraisement, the expected rent can then be capitalized to find the present value of the waterfall. The model goes on to explain that owners are to be given a choice of either leasing out their waterfalls to receive the rent, or to use the capitalized value of (part of ) this rent as investment capital to acquire shares in the development company.

The third point 

e model remains neutral as to the form of organization, mentioning the possibility both of 

model is stated aim is to ensure that ``.....''


\section{Conclusion}

Old stuff:

\section{Introduction}\label{intro}

Norway is country of mountains, fjords and rivers, and about 95 \% of the annual domestic electricity supply comes from hydro-power.\footnote{See Statistics Norway, data from the year 2011, http://www.ssb.no/en/elektrisitetaar/.} The right to harness rivers for hydro-power is held by local landowners, but historically, this right has not been of much use to them, since the Norwegian electricity sector has been organized as a regulated monopoly, with most hydro-power schemes carried out by non-commercial companies controlled by the State, or local governmental bodies. In the early 1990's, however, the sector was liberalized, and it has become increasingly common for local landowners to undertake their own hydro-power projects. This has led to increased tension between local interests and established energy companies. Following liberalization, these companies are now organized for profit, and this raises the question: who is entitled to benefit from Norwegian hydro-power? 

Increasingly, it is becoming clear that this is not merely a question of the opposing commercial interests of individuals and companies, but also crucially involves the local communities directly affected by development. The original owners of hydro-power tend to be farmers residing in the local communities where the resources are found, and therefore, the question of who should benefit also encompasses the question of who should be allowed to participate in decision-making process, and what degree of autonomy local communities are to be granted in this regard. Are local owners and their communities entitled to a say in how the hydro-resource is to be exploited, or must they accept to remain passive, as they were rendered by the monopoly which used to be in place?

In this chapter, we address some recent demands made by local people, to the effect that they should be allowed to partake more actively in decision-making processes regarding local waterfalls. We focus on the legal status of such demands under Norwegian law, and we do so by considering the recent Supreme Court Case of \emph{Ola Måland and others v. Jørpeland Kraft AS}.\footnote{Ola M{\aa}land and others v. J{\o}rpeland Kraft AS, Rt 2011 s. 1393. I mention that I represented the local owners in this case, as a trainee lawyer in the district and regional courts, and as the responsible lawyer before the Supreme Court.} In this case, the local owners protested the legality of a license that granted the developer, Jørpeland Kraft AS, a right to divert water away from their waterfalls, thereby reducing the potential for local hydro-power. The owners argued that consultation had been insufficient, and they also contended that the assessment of the case made by the water authorities had been inadequate, and that the decision has been based on an erroneous account of the facts. They won the case in the district Court, Stavanger Tingrett, but lost under appeal to the regional Court, Gulating Lagmannsrett. The Norwegian Supreme Court also found in favor of the developer, and the argument they gave to support this conclusion goes far in suggesting that while the commercial interests of local owners need to be compensated, the presence of such interests do not entitle local communities to a greater say in decision-making processes. Importantly, the Court held that the presence of local interests does not necessitate the adoption of different administrative practices, and that the procedures developed during the decades of direct, non-commercial, administration of the energy sector, could still be followed.

The case is significant, since the traditional hydro-electric scheme in Norway typically involves expropriation, often interfering with the property rights of hundreds of local individuals. Traditionally, owners of waterfalls would be compensated according to a standardized mathematical method that was based on the assumption that they had no interest in hydro-power themselves.\footnote{The method consists in calculating the number of \emph{natural horsepowers} in the waterfall, and then multiplying this number by a price pr. natural horsepower, determined by the discretion of the Court, but in practice based almost solely on what has been awarded in previous cases. For a description of the traditional method, we point to \cite{falk} Chapter 7, page 521-522 (in Norwegian).} In a landmark case from 2008, however, the Norwegian Supreme Court commented, in an \emph{obiter dicta}, that the traditional method for calculating compensation for waterfalls was no longer appropriate, at least not in cases when it can be demonstrated that the original owners could have exploited the resource themselves, if the expropriation had not taken place.\footnote{The case of Agder Energi Produksjon AS vs. Magne Møllen, Rt. 2008 s. 81. The local owner lost the case, the reason being that the Supreme Court held that compensation should not be based on the present day value of the waterfall, but the value it had when the original transferral of rights took place, in the 1960's. The \emph{obiter dicta} has been used as an authority for subsequent Supreme Court decisions, however, see, for instance, Rt. 2010 s. 1056 and Rt. 2011 s. 1683. It has received quite a lot of scholarly attention as well, see \cite{Tf1,Tf2,Tf3}.} This decision has had a profound impact on the level of compensation awarded for waterfall rights, leading to payments that can will typically be ten to a hundred times higher than that which would have been awarded according to the traditional method.\footnote{So far, in cases that have come before the court, there has been about a twenty-fold increase in compensation, see \cite{Tf1}, but the new method will, when applied to certain kinds of projects (cheap to build, and involving little or not regulation of the water-flow), result in compensation having to be paid that amounts to at least a hundred times more than what could be expected if the traditional method had been applied.} More generally, it also served to shift the balance of power in favor of local owners and their communities, who increasingly expect to have their voices heard, and to get a more direct say over how their energy resources are managed and exploited.

After \emph{Måland}, however, it has become unclear to what extent the presence of local, commercial interests will continue to influence the Norwegian energy sector, and if we will see more active participation by local people in the future. In fact, recent statements made by the Norwegian water authorities seem to suggest that this is becoming increasingly unlikely, as a shift in policy seems to have taken place, whereby local, small scale projects, are now to be given lower priority than large scale projects undertaken by established energy companies.\footnote{These statements were not linked to \emph{Måland}, but were made in a more general context, ostensibly motivated by the desire to increase the efficiency of the administrative process, see http://www.nve.no/no/Konsesjoner/Vannkraft/Smaakraft/ where the new policy was announced. It also received some attention from the press, see, for instance, http://www.tu.no/energi/2012/01/18/nve-varsler-flere-smakraft-avslag.}

Taking a broader view on Norwegian law, we believe that recent experiences regarding hydro-power provides an interesting case to study, and one that will shed light on how property rights function in a social, economic and political context. It seems, in particular, that the view of property rights to waterfalls adopted by the Supreme Court rests on a narrow interpretation, seeing such rights merely as bestowing financial interests on certain individuals. This was clearly felt in \emph{Måland}, and we think the case also serves to illuminate certain consequences of such a view, suggesting, in particular, that it can have detrimental social and political consequences, and can very easily lead to perceived injustices. We also think it is pertinent to ask if the narrow view of property which seems to have been adopted for waterfalls in Norway is adequate with respect to human rights law, or if the right to property should also be considered a right to participate, and a right to be heard, in decision-making processes.

In the following, we first give the reader some further background on Norwegian hydro-power, and then 
we present \emph{Måland} in some detail, focusing on giving the reader an impression of current administrative practices, by describing how they played out in this particular case, and by detailing how they came to result in a decision that the original owners felt to be fundamentally unjust. We also address the legal arguments given by the opposing sides and the arguments relied upon by the national courts. We conclude by presenting some overreaching issues that we believe the case raises, regarding both the social context of property rights, the content of property as human right, and the question of whether or not the protection awarded under Norwegian law currently meets the standard set by the European Convention of Human Rights, as interpreted by the Court in Strasbourg. 

\section{Background: local owners making their voices heard by suggesting small scale hydro-power}\label{context}

As we mentioned in the Introduction, Norwegian law regards waterfalls as private property, and by default, a waterfall belongs to the owner of the land over which the water flows.\footnote{See Section 13 of Act No. 82 of 24 November 2000 relating to River Systems and Groundwater.} This does not mean that the landowner owns the water as such -- freely running water is not subject to ownership -- but it entitles the owner of the waterfall to harness the potential energy in the water over the stretch of riverbed belonging to him. This right can be partitioned off from any rights in the surrounding land, and large scale hydro-power schemes typically involve such a separation of water-rights from land-rights, giving the energy company the right to harness the energy, while the local landowner retains the rights in the surrounding land.

Norwegian rivers, and especially rivers suitable for hydro-power schemes, tend to run across grazing land owned jointly by farmers, so rights to waterfalls are typically held among several members of the local, rural community.\footnote{The land in question tend not to be enclosed, in particular, and in cases where there has been a land enclosure, water-rights have often explicitly been left out, such that they are still considered common rights, belonging to the community of local farmers.} They might not always be willing to give them up, especially not on the terms proposed by the developer, so the use of expropriation has played an important role in the history of Norwegian hydro-power. This has meant that the terms governing separation of water-rights from land-rights, including the level of compensation paid to landowners, and the influence they are granted in the decision-making process, has been determined by the law. Following legislation in the early 20th century, a regulatory system was put in place that centralized the management of Norwegian water resources. It clearly favored exploitation by the State or by companies owned by local governmental bodies, and the local landowners were severely marginalized. In most cases, they would have to accept the terms presented to them by the developer, or else argue the matter in Court, after the developer had already been granted a license to expropriate. Landowners were not in a good position to negotiate the terms of the development, and their property rights appeared increasingly nominal, the prevailing political attitude being that waterfalls formed part of the common heritage of the Norwegian people, and should be managed in their interest.\footnote{While some of the claims made here will be further qualified by what is to follow, the general picture we paint here is communicated also by the standard work on Norwegian water law \cite{falk}.}

This created a legal tension where, on the one hand, waterfalls were still considered private property under land law, yet, on the other hand, were considered as belonging to the public as far as large scale hydro-power development was concerned. The following two quotes, the first from the general water law, with roots going back at least to the 19th Century, and the second  from a law directed specifically at large scale hydro-power, illustrates this ambivalence.

{\begin{minipage}[t]{16em}
 \begin{aquote}{\tiny Section 13, Water Resources Act 2000} \footnotesize A river system belongs to the owner of the land it covers, unless otherwise dictated by special legal status. [...]

The owners on each side of a river system have equal rights in exploiting its hydro-power...
\end{aquote}  
\end{minipage}}
{\begin{minipage}[t]{22em}
\begin{aquote}{\tiny Section 1, Industrial Concession Act 1917 (amended 2008)} \footnotesize Norwegian water resources belong to the general public and are to be managed in their interest. This is to be ensured by public ownership...
\end{aquote}
\end{minipage}} \\

Following liberalization of the Norwegian energy sector in the early 1990's, this legal tension in statute has increasingly also become a tension in politics, where the interests of local communities and landowners stand in opposition to the interests of large energy companies, often owned by the State, seeking to harness locally owned resources for commercial gain. The question of how the Norwegian legal framework is actually applied in this regard is therefore a matter that has come under increased scrutiny. This was the question that went before the courts in the case of \emph{Måland}, where local owners protested the legality of expropriation on the grounds that they could harness the water in their own small scale hydro-power scheme. Before we delve into the details, we will elaborate a bit further on the context in which the law was called upon to function in this case. Importantly, the economic, social and political context of expropriation has changed rather dramatically in recent years, and we do not think it is possible to understand the case and the issues it raised except in the context of these changes. 

There are two developments that have been particularly important. First, there has been a general shift from viewing electricity production as a public service to viewing it as a commercial enterprise. This has made the legitimacy of expropriation appear more controversial, and the argument is often voiced that expropriation does not happen in the interest of the public at all, but \emph{solely} in order to benefit the commercial interests of particular companies.\footnote{This has been a recurring theme in articles appearing in "Småkraftnytt", the newsletter for "Småkraftforeninga", an interest organization for owners of small-scale hydro-power, which currently have 236 associated small scale hydro-power plants, see http://kraftverk.net/ (in Norwegian). In addition to the case of Måland, the question has also been brought before the (lower) national courts in some other cases, such as \emph{Sauda}, LG-2007-176723 (Gulating Lagmannsrett, regional high court), and \emph{Durmålskraft}, see http://www.ranablad.no/nyheter/article5583405.ece (decision from the district court, as reported in a Norwegian newspaper). In both cases, the outcome was generally more favorable to the expropriating party than the local owners, and the reasoning adopted by the courts appears similar to that of \emph{Måland}.}

In this way, expropriation of Norwegian waterfalls raises issues that have become increasingly important also in a global setting, and which seem to arise naturally in systems where economic activities are organized according to a mix of socialist and free market principles. In such systems, it seems practically inevitable that cases of expropriation -- undertaken to benefit the public -- will also often come to benefit developers that are motivated by purely commercial interests. While this in itself might not be problematic, it will easily lead to the concern that the commercial interests of powerful companies is the \emph{only} reason why expropriation is permitted, and that expropriation is being used as a commercial tool for powerful market forces, to the detriment of less powerful actors. That this can be highly controversial is illustrated in the US case of \emph{Kelo v. City of New London}, which divided the US Supreme Court and has also attracted great attention, both from legal scholars and in the general public, as a political issue.\footnote{\emph{Kelo v. City of New London}, 545 U.S. 469 (2005).}

For the case of Norwegian waterfalls, however, liberalization of the energy sector has also had a positive effect for local communities, in that it has served to make local owners more active. It has become increasingly common that they exploit their hydro-power resources themselves, often in small scale projects, and often in cooperation with companies that specialize in such development.\footnote{In 2012, the NVE granted 125 new licenses for small scale hydro-power, and at the end of the year they had 859 applications still under consideration. Source: report made by the NVE, available at http://www.nve.no/Global/Energi/Q412\_ny\_energi\_tillatelser\_og\_utbygging.pdf (in Norwegian). } This, of course, only adds to the controversy surrounding expropriation of waterfalls, especially when local owners are deprived of the opportunity for small scale development.

The most significant step towards liberalization of the Norwegian energy sector was made in 1990 when the Energy Act was passed, an important new piece of statute reorganizing the system for the distribution of electricity.\footnote{Act nr. 50 of 29 of June 1990 relating to the generation, conversion, transmission, trading, distribution and use of electricity.} The Energy Act introduced the principle that energy consumers and producers should have non-discriminatory access to the national electricity grid, thereby creating a market where any actor, privately owned or otherwise, could supply electricity to the grid, and profit commercially from hydro-power. In the same period of time, monopoly companies were reorganized, becoming commercial companies that were meant to compete against each other, and against new commercial actors that entered the market.\footnote{For a short English summary of how the system is administered, see for instance \cite[p.29-30]{ar2010}, and for more detail, we point to \cite{Hammer2}.}

The Norwegian State retained a significant stake as shareholders in energy companies, however, now often alongside private investors. Moreover, many rules in Norwegian law favor companies where a majority of the shares are held by the State, and to this day the largest and most influential Norwegian energy companies remain under public ownership.\footnote{The fact that publicly owned companies are favored in this way is often seen as a questionable practice with regards to competition law, see for instance the recent EFTA Court case, Case E-2/06, \emph{EFTA Surveillance Authority v. The Kingdom of Norway}, EFTA Court Report 2007, p.164. Here, the Court considered the old Norwegian rule of \emph{reversion}, whereby a license to undertake certain large scale hydro-power schemes (strictly speaking, a license to acquire the waterfalls needed to undertake it) came with a special clause that the private developer had to give up ownership to the State after a fixed period of time. This clause was held to be in breach of the EEA agreement since it only applied to private companies. We remark that the Norwegian government responded to this with an amendment after which reversion no longer applies, but which stated that a license to acquire waterfalls for the purpose of such large scale schemes can not be given at all to any company in which private parties own more than 1/3 of the shares.}

It seems, in particular, that the aim of liberalization in Norway has never been to minimize State control over hydro-power, but rather to give consumers greater freedom in choosing their energy-supplier, and to enhance efficiency in the sector by introducing competition.\footnote{See for instance \cite{liberal}, which offers a comparative study of the liberalization of the energy sectors in Norway and the UK.} Still, the fact that any developer of hydro-power is now legally entitled to connect to the national grid has proved important in giving actors that are not owned by the State a fighting chance on the Norwegian energy market. It has been especially important for local owners of waterfalls, since it means that if they undertake hydro-power projects themselves, they can no longer be refused access to the grid, but will be in a position to benefit commercially.

It should be noted that the Norwegian grid is operated by regional companies, responsible for the supply and distribution of electricity in their region. These will typically also be energy producers themselves, and historically, they would prevent other hydro-power initiatives by refusing them access to the grid. In fact, in the early days on Norwegian hydro-power, in the first half of the 20th century, there were quite a few locally owned and operated power plants, often providing local communities with electricity. When the national grid was established, most of them were closed down, often as a result of an explicit policy on part of the authorities. To increase the cost-effectiveness of the companies responsible for providing the national service, these companies were often allowed to demand, as a condition for allowing local communities access to the grid, that local hydro-power plants had to be shut down.\footnote{See \cite[p.111]{Hindrum} (in Norwegian).} 

Following legislation whereby access to the grid is provided for in statute, we have seen a surge of interest in the exploitation of waterfalls in small scale hydro-electric schemes, and these schemes are often initiated by local owners. As we have mentioned, the typical owners of Norwegian waterfalls are communities of farmers and smallholders. Historically, the right to land-based resources, especially in the mountainous areas of the west and the north, where most valuable waterfalls are located, was held by local people, the same people who made use of it on a day to day basis. The main reason for this, which by European standards stands out as quite unusual, was that Norway never really had a separate class of landed nobility. Consequently, the Norwegian farmer occupied a position of relative autonomy and freedom, even to the point of exercising significant political influence, especially in the early days of Norwegian democracy.\footnote{During the 19th century the two dominant group in Norwegian politics were the farmers and the civil servants, and the former group exercised great influence in the Norwegian parliament, with the 1833 election leading to what became known as the farmers parliament. The "classic" academic treatment of farmers' influence over 19th century Norwegian politics is \cite{Koht} (in Norwegian). More on the author and a summary of his views can be found here, http://en.wikipedia.org/wiki/Halvdan\_Koht.} Following industrialization, however, their role became much more marginal, and farming has steadily become more and more unprofitable, with many farming communities having already disappeared, and many others threatened by depopulation. In light of this, the possibility of undertaking small scale hydro-power is often seen as being important to the survival of rural communities themselves, not just as a means for individual members of such communities to make a profit.

As local owners started to harness their waterfalls themselves, commercial companies also emerged, specializing in cooperating with them. In a recent report, it was estimated that there is a potential for profitable small scale hydro-power of about 20 TWh/year \cite{Aanesland}, with a total value, before investment, of about 70 billion Norwegian kroner, i.e., about 8 billion pounds.\footnote{For comparison, suggesting the scale of this potential, we mention that the total consumption of electricity in Norway in 2011 amounted to 114 TWh, see http://www.ssb.no/en/energi-og-industri/statistikker/elektrisitetaar.}  This report was based on a particular model of cooperation with a commercial company, Småkraft AS, and might be an underestimate of what small scale hydro-power could represent for local communities if they take a more independent role in developing the resource. Thus, small scale development of hydro-power has become socially and political significant, and it is increasingly seen as a possibility for these regions to counter depopulation and poverty, while also regaining some of their autonomy and influence with respect to how the natural resources found locally are to be managed. In many cases, small-scale hydro-power appears to be the only growth industry, and takes on great political and social importance for the community as a whole, not just the owners of waterfalls.\footnote{For an example of a community where small scale hydro-power has played such a role, we can point to Gloppen, a municipality in the county of Sogn og Fjordane, in the western part of Norway. 19 schemes have already having been successfully carried out, all except one by local owners themselves, amounting to a total production of over 250 GWh/year. This prompted the mayor to comment that "small scale hydro-power is in our blood", see \cite{Gloppen}. When interviewed, he also directed attention at the fact that hydro-power had many positive ripple effects, since it significantly increased local investment in other industries, particularly agriculture, which had been severely on the decline.}

Summing up, we can conclude that expropriation of waterfalls has become more politically and social controversial, and we believe that the case of \emph{Måland}, to which we now turn, must be understood in this context. The case did not attract the same public attention as the cases relating to the revision of the traditional method for awarding compensation, and has, as far as we are aware, not previously received any scholarly attention either. It seems important, however, in that it clarifies the stance that Norwegian Courts take with respect to the question of the extent to which local owners and communities are entitled to take part in the decision-making processes concerning commercial development of the waterfalls they own. Moreover, while local interests have claimed a significant victory with respect to compensation, \emph{Måland} limits its impact since it suggest that there is still very limited legal protection of local owners' right to have their voices heard regarding how Norwegian power is to be managed.

In the following, we give a presentation of the case. We start by presenting the facts, and we do so going back to original sources, not merely looking to the brief presentation provided by the courts in their judgments, but to the preparatory documents assembled by the water authorities, taking special note of both the arguments presented by the expropriating party, and the objections raised by original owners and their representatives. Building on this, we the present the legal arguments that were raised by both sides, aiming to provide a more in depth presentation than the review given by the courts. We then present and compare the various arguments relied upon by the courts, and we offer our own analysis of how to understand the outcome of the case in the context of Norwegian law. We continue by addressing what the decision tells us about the Norwegian legal framework for hydro-power exploitation more broadly, and the questions it raises with respect to the social implications of expropriation of waterfalls, and with respect to human rights law.

\section{The facts of the case}\label{sum}


The applicant in Måland, Jørpeland Kraft AS, is a company jointly owned by Scana Steel Stavanger AS, who own 1/3 of the shares, and Lyse Kraft AS, who is the majority shareholder holding the remaining shares. The former is a steelworks company located in the small town of Jørpeland in Rogaland county, southwestern Norway. Historically, this company has been a major employer in Jørpeland, which is located by the sea, next to a mountainous area. The main source of energy for the steel industry in Norway has been hydro-power, and Scana Steel Stavanger AS is no exception. The company uses energy harnessed from the rivers in the area, and while the primary river runs through the town of Jørpeland itself, it is supplemented by water from other rivers in the area that are diverted so that they can be exploited more efficiently along with the water from the Jørpeland river.

Recently, Norwegian steel companies have become less profitable, due in great part to increased foreign competition and a significant increase in cost of operation associated with this type of industry in Norway, particularly salary costs.\footnote{For a reference on this, see \emph{Information Booklet about Norwegian Trade and Industry}, published by the Ministry of Trade and Industry in 2005.} This has led to many such companies shifting their attention away from labor-intensive steel production, and focusing instead on producing electricity, selling it directly on the national grid. Jørpeland Kraft AS was established as part of such a move being made with regards to the energy resources in Jørpeland, and the role played by Lyse Kraft AS is an important one. As we mentioned, Norwegian law favors companies where the majority of the shares are held by public bodies, and Lyse Kraft AS, being publicly owned, with the city of Stavanger as the main shareholder, is therefore a valuable partner. Moreover, Lyse Kraft AS, while being a commercial company, is also responsible for the electricity grid in the region. It was established as a merger between several local monopoly companies in the Stavanger region which were reorganized following liberalizaion of the sector in the early 1990's. As discussed in Section \ref{context}, there is little doubt that old monopolists still enjoy considerable power and influence.\footnote{In fact, Lyse Kraft AS is good example suggesting that their power might in some cases have \emph{increased}. Since liberalization, the restraints imposed both by the non-commercial nature of former monopolists, and the local, political, anchoring of such companies, have disappeared.} This is another reason why they can serve as valuable partners for private companies wishing to make a profit from Norwegian hydro-power.

With attention shifting from harnessing rivers for the purpose of industrial production to the purpose of producing electricity to sell on the national grid, the main variables that determines the profitability of the undertaking also changes. On the cost side, what matters becomes only the cost of producing the electricity itself, and this is typically determined, for the most part, by the investments required for the original construction works.\footnote{For an overview of the considerations made when assessing the commercial value of small scale hydro-power, we point to \cite{kartlegging}. In fact, due to the importance that small scale hydro-power has assumed in recent years, investigating models for investing in such projects has become an active field of research in Norway, see for instance \cite{investment}.} Running and maintaining a hydro-power station tends to be comparatively inexpensive. On the income side, what matters is the price of energy on the electricity market, a market that is no longer anchored in the local conditions of supply and demand.

Importantly, as long as energy production is the sole focus, the business no longer depends in any significant way on the local labor force, and as a result, it is typical that large scale exploitation becomes much more profitable, compared to the medium or small scale power plants typically needed to facilitate local industrial exploits. Hence, it was in keeping with a general trend in Norway when Jørpeland Kraft AS, following their shift in commercial strategy, proposed to undertake measures to increase their energy output. This could be achieved relatively cheaply, by further constructions aimed at channeling water from nearby waterfalls into dams that were already built to collect the water from the Jørpeland river.

One relatively small waterfall from which Jørpeland Kraft AS suggested to extract water was owned by Ola Måland and five other local farmers. This waterfall is not located in Jørpeland kommune, and does not reach the sea at Jørpeland, but runs through the neighboring municipality of Hjelmeland, on the other side of a mountain range, until it eventually reaches the sea at Tau, another neighboring municipality. The plans to divert this water would deprive original owners of water along some 15 km of riverbed, all the way from the mountains on the border between Hjelmeland and Jørpeland, to the sea at Tau. Far from all the water would be removed, but the water-flow would be greatly reduced in the upper part of the river known as "Sagåna", the rights to which is held jointly by Ola Måland and five other local farmers from Hjelmeland. 

The water in question stems from the \emph{Brokavatn}, located 646 meters above sea level, where altitude soon drops rapidly so that hydro-power is a particularly well-suited form of exploitation for this water. Plans were already in place for making such use of it, from about the altitude of Brokavatn, to the valley in which the original owners' farms are located, at about 80 meters above sea level. In fact, a rough estimate of the potential was originally made by the NVE, and estimated to yield gross annual production of 7.49 GWh pr. annum, about five times more than the water from Brokavatn would contribute to the project proposed by Jørpeland Kraft AS. This estimate was not made in relation to the case, but as part of a national project to survey the remaining energy potential in Norwegian rivers.\footnote{The survey was carried out in 2004, and its results are summarized in \cite{kartlegging}.} \noo{More recent calculations, made by several different experts, acting both on behalf of Jørpeland Kraft AS and original owners, suggests that the water which would be lost would in fact be crucial to the commercial potential of hydro-power for the original owners. Having the water available would take such a project from being somewhat marginal to being a highly profitable endeavor. The owners were not aware of this at the time when the case was being prepared by the water authorities, nor where they informed of this as part of the process.} 

Despite holding the relevant property rights, and despite having considerable commercial interests that would be effected, original owners were not identified as significant stakeholders in the project. Rather, the approach to the case was the traditional one, with focus being directed at the environmental impact, with relevant interests groups being called upon to comment on consequences in this regard, and quite some public debate arising with respect to the balancing of commercial interests and the desire to preserve wildlife and nature.

Nevertheless, one of the owners, Arne Ritland, commented on the proposed project, in an informal letter sent directly to Scana Steel Stavanger AS. In this letter he inquired for further information, and he protested the transferral of water from Brokavatn. He also mentioned the possibility that an alternative hydro-power project could be undertaken by original owners, but he did not go into any details regarding this, stating only that such a locally owned hydro-power plant had previously been in operation in the area. The plant he was referring to dates back to the time before we had a national grid, and was only directed at local supply of electricity. It has since been shut down.

Arne Ritland received a reply stating that more information on the project and its consequences would soon be provided, and he did not pursue the matter further at this time. Meanwhile, Scana Steel Stavanger AS submitted his letter to the water authorities, who in turn presented it to the NVE as a formal comment directed at the application. This prompted Jørpeland Kraft AS to undertake their own survey of alternative hydro-power in Sagåna, and the conclusion, but not the report itself, was sent to the water authorities. The original owners were not informed, and they were not asked to comment on it, or even told that such an investigation of the commercial potential in their waterfalls was being considered by the expropriating party, as a response to Ritland's letter.

Despite being presented with the issue, the water authorities did not take steps to investigate the commercial potential of local hydro power on their own accord. Moreover, the conclusion presented by Jørpeland Kraft AS did not go into details, but merely stated that if the local owners decided to build two hydro-power plants in Sagåna, then one of them, in the upper part of the river, close to Brokavatn, would not be profitable, neither with nor without the water in question. The other project, on the other hand, in the lower part, could still be carried out profitably even after the transferral. No mention was made as to what the original owners actually stood to loose, nor was there any argument given as to why it made sense to build two separate small-scale power plants in Sagåna. In their final report, the NVE handed these findings over to the Ministry, but did not inform the original owners. 

In addition to the report made by Jørpeland Kraft AS themselves, Hjelmeland kommune, the local municipality government, also commented on the possibility of local hydro-power. In their statement to the NVE, they directed attention to the data in the NVE's own national survey, which suggested that a single hydro-power plant in Sagåna would be a highly profitable undertaking. On this basis, they protested the transferral, arguing that original owners should be given the possibility of undertaking such a project. This statement was not communicated to the original owners, and in their final report it was dismissed by the NVE, who stated that the most energy efficient use of the water would be to transfer it and harness it at Jørpeland.

In addition to the statement made by Ritland, one other property owner, Ola Måland, commented on transferral. He did so without having any knowledge of the commercial potential the water held for him and his co-owners, and without having been informed of the statement made by Hjelmeland Kommune. On this basis, he expressed his support for the transferral, citing that the risk of flooding in Sagåna would be reduced. He also phrased his letter in such a way as to suggest he was speaking on behalf of other owners, but he was the only person to sign it. In the final report to the Ministry, the NVE, in their own conclusion, use this as an argument in favor of transferral, stating that the original owners were in favor of it, and that the opinion of Hjelmeland Kommune should therefore not be given any weight. They neglect to mention Arne Ritland's statement in this regard, and earlier in the report, where his statement is referred to along with many others, Ritland is referred to as a private individual, while Ola Måland is referred to as a property owner, and taken to speak on behalf of the others. The report made by the NVE, while it was not communicated to the affected local owners, it was sent to many other stakeholders, including Hjelmeland Kommune. In light of NVE's conclusions, they changed their original position, informing the Ministry that they would not press any further for local hydro-power, since this was not what the original owners wanted themselves. 

While the case was being prepared by the water authorities, the original owners had begun to consider the potential for hydro-power on their own accord, and in late 2006, when the case reached the Ministry, they where not aware that a decision was imminent. Rather, they were under the impression that they would receive further information before the case went further. Still, as they came to realize the commercial value of the water from Brokavatn in their own project, they approached the NVE, inquiring about the status of the plans proposed by Jørpeland Kraft AS. They were subsequently informed that an opinion in support of transferral had already been offered to the Ministry, and that a final decision would soon be made. This communication took place in late November 2006, summarized in minutes from meetings between local owners, dated 21 and 29 of November. On 15 of December 2006, the King in Council granted a concession for Jørpeland Kraft AS to transfer the water from Brokavatn to Jørpeland.

At this point, it was becoming increasingly clear to the original owners that the water from Brokavatn would be crucial to the commercial potential of their own project, and they also retrieved expert opinions suggesting that the NVE was wrong in concluding that transferral would be the most efficient use of the water. In light of this, they decided to question the legality of the transferral, arguing that the decision was invalid.

The license given to Jørpeland Kraft AS was challenged by the original owners on the grounds that the expropriation was materially unjustified, and that the administrative process leading up to the permission to expropriate did not fulfill procedural requirements. The local court, Stavanger Tingrett, held that the original owners were right in protesting the transfer, with the court emphasizing that the preparatory steps taken in cases such as these needed to provide adequate guarantee that the authorities had also considered the fact that the waterfalls could have been exploited commercially by the original owners themselves.\footnote{Stavanger Tingrett 20.05.2009, case nr. 07-185495SKJ-STAV.}

This view was rejected by the regional court, Gulating Lagmannsrett, which held that sufficient steps had been taken to clarify the commercial interests of the owners, and, moreover, that established practice regarding the preparation and evaluation of such cases -- dating from a time when it was not feasible for original owners to undertake hydro-power schemes -- still provided adequate protection.\footnote{Gulating Lagmannsrett 10.01.2011, case nr. 09-138108ASD-GULA/AVD2.} The Supreme Court also held in favor of Jørpeland Kraft AS, and they went even further in stating that established practice was beyond reproach.

In the following section, we present the main legal arguments relied on by the parties, as well as a summary of how the three national courts approached the case, and how they argued for their respective decisions.

\section{The legal arguments, and the view taken by the national courts}\label{view}

The original owners had several arguments in support of their claim that the concession was invalid. Firstly, they argued that procedural mistakes had been made in preparing the case; secondly, they argued that according to Norwegian expropriation law, it was not permissible to expropriate in a situation such as this, when the loss of energy and commercial potential would outweigh the gain to those same interests, which, ostensibly, were the only interests identified in favor of transferral. It seemed to the original owners that expropriation in this case would only serve to benefit the commercial interests of Jørpeland Kraft AS, and that it would do so to the detriment of both local and public interests. For this reason, the owners held that the concession should be regarded as an abuse of power, a manifestly ill-founded decision which could not be upheld.\footnote{There are at least two different ways in which to argue such a point under Norwegian law. One is with respect to water law and general administrative law, whereby clearly ill-founded decisions can be overturned by the courts, even when they involve discretion on part of the executive, which is otherwise not subject to review by the courts. Secondly, an argument can be made with respect to the Norwegian Constitution, Section 105, which gives property a protected status. The former is usually more effective, but in both cases, quite a severe transgression will have to be established before courts consider it within their competence to overturn discretionary decisions. A scholarly examination of these two sets of provisions are given in \cite{Efvl} and \cite{flei} respectively (both in Norwegian).} The owners argued, moreover, that the government had not fulfilled its duty to consider the case with due care, and that the assessment made with respect to the interests of the local community at Hjelmeland, and the local owners residing there, was not adequate. Particular attention was directed at the fact that local owners had not been informed about the progress of the case, and had not been told of, or asked to comment on, those preparatory steps that were being made explicitly with regards to assessing their interests. 

In addition, owners also argued that irrespectively of how the matter stood with respect to national law, the expropriation was unlawful because it would be in breach of the provisions in the ECHR TP1-1 regarding the protection of property.\footnote{European Convention of Human Rights Article 1 of Protocol 1.}\noo{An argument was also made to the effect that expropriation would be in breach of provisions in the EEA agreement regarding unlawful state support for the commercial interests of specific companies.}

Jørpeland Kraft AS protested all these objections to the expropriation, arguing that it was the responsibility of the owners themselves to provide information about possible objections against the project, and that the process had therefore been in accordance with the law. Unfortunate misunderstandings, if any, should be attributed to the fact that original owners had neglected their responsibilities in this regard. Moreover, Jørpeland Kraft AS argued that it was not for the courts to subject the assessment of public and private interests to any further scrutiny, since this was a matter for the government to decide. 

Indeed, according to Norwegian national law, it is traditionally held that unless the exercise of power it clearly unjustified, the courts do not have the authority to overturn decisions based on discretion, unless it can be demonstrated that the government has made procedural mistakes. While this view has become somewhat more relaxed in recent years, with a standard of \emph{reasonableness} increasingly being imposed by courts in similar cases, the inadmissibility of court interference in administrative discretionary decisions is still very much a part of Norwegian national law.\footnote{See \cite{Efvl}, in particular, chapters 24 and 29.}

Finally, Jørpeland Kraft AS argued that there was no issue of human rights at stake in the case. While they argued for this by stating that as the procedural rules had been followed and that the material decision was beyond reproach, they also went far in suggesting that as the owners would be compensated financially by the courts for whatever loss they would incur, no human rights issues could possibly arise in the case. \noo{ They also rejected the view that the case could be seen as an instance of illegitimate state support for Jørpeland Kraft, but failed to provide specific arguments in this regard.}

The matter went before Stavanger Tingrett who gave their judgment on 20 May 2009. In the following, we offer a presentation of the reasons given by this court, leading to the conclusion that the expropriation was unlawful and that the transferral could not be carried out. 

Stavanger Tingrett agreed with the original owners that the decision to grant concession was based on an erroneous account of the relevant facts, and they concluded that it was evident, from the NVE's own figures, that allowing the applicants to use the water from Brokavatn in their own hydro-electric scheme would be the most efficient way of harnessing the potential for hydroelectric production, directly contradicting what the NVE stated in their report. Moreover, they noted that these were the same estimates as those referred to by  Hjelmeland Kommune in their initial objection, and found it to be in breach of procedural rules that this was not considered further by the authorities.

The Court substantiated their decision by giving direct quotes from the report made by the NVE. For instance, in the report, on p. 199, it says, as quoted by Stavanger Tingrett (my translation):
%\begin{quote}Hjelmeland kommune ser helst at kraftressursene i vassdraget blir utnyttet av lokale %grunneiere. 
%Dette står i kontrast til uttalelsen fra grunneierne selv som ønsker at overføring blir gjennomført, 
%slik at flom og erosjonsskader kan bli noe redusert. NVE mener at den beste utnyttelsen med tanke 
%på kraftproduksjon vil være å tillate overføringen da en slik løsning vil innebære at vannet utnittes i 
%størst fallhøyde. Når dette samtidig er grunneiernes eget ønske har vi ikke tillagt Hjelmeland 
%kommunes synspunkt på dette noen vekt
%\end{quote}
%Our own translation follows below: 
\begin{quote}
Hjelmeland kommune would like the hydro-electric potential in the waterfall to be exploited by 
local property owners. This stands in contrast to the statement given by the property owners 
themselves, who wish that the transfer of water takes place, so that damage due to flooding can be 
somewhat reduced. NVE thinks that the best use of the water with respect to hydro-electric 
production is to allow a transfer, since this means that the water can be exploited over the greatest
distance in elevation. When this is also the property owners' own wish, we will not attribute any 
weight to the views of Hjelmeland kommune.
\end{quote}

Stavanger Tingrett concluded that as this was a factually erroneous account of the situation, the decision made to allow transferral of the water could not be upheld. Summing up, the Court offered the following assessment of the case (my translation):

\begin{quote}
It is the opinion of the court, having considered how the case was prepared by the authorities, that the factual basis for the decision made by the government suffers from several significant mistakes and is also incomplete.
\end{quote}

In light of this, Stavanger Tingrett concluded that the decision to grant concession for transfer of water was invalid. As to the legal basis of this, the court relied on the recognized principle of Norwegian public law that while the exercise of discretionary powers is usually not subject to review by court, a decision based on factual mistakes is nevertheless invalid if it can be shown that the mistakes in question were such that they could have affected the outcome. This is not provided for explicitly in statue, but it is one of the core unwritten legal principles of Norwegian public law.\footnote{See \cite{Efvl}}

Concerning the second requirement, that the factual mistakes could have affected the outcome, Stavanger Tingerett found that it was clearly fulfilled in this case since, in fact, the hydro-power suggested by original owners was, based on data available to the government at the time of decision, an objectively speaking \emph{better} use of the resource, even with respect to public interest. In any event, the requirement with regards to factual and procedural mistakes is only that the mistakes \emph{could} have affected the outcome; in the presence of mistakes, the burden of proof is shifted over to the party seeking to defend the decision.

Since Stavanger Tingrett agreed with the original owners that the decision was invalid due to being based on incorrect facts, there was no need to consider further the claims regarding the legitimacy of the decision with respect to human rights law. Stavanger Tingrett did conclude, however, making a more overreaching assessment of the case, that the procedure followed in preparing the case had not taken sufficient regard of owners' interests, and that this was the likely cause of the mistakes that had been made. The Court also argued that the standard of protection for interest of original owners had to interpreted as being more strict now that local hydro-power was an option available to original owners. 

\noo{In this regard, t also seems that Stavanger Tingett found some additional support in its interpretation of Norwegian law that was based on human rights concerns, especially the fact that expropriation, in circumstances such as those of this case, appeared to be a major interference in the rights of owners, and that established practice developed under a different regulatory regime was therefore no longer able to provide adequate protection.}

Jøpeland Kraft AS appealed the decision, and the case then went before the regional court, Gulating Lagmannsrett. They overruled the decision made by Stavanger Tingrett. In their argument, they do not rely on direct assessment of the report made by NVE, nor do they mention the expert statements retrieved by the opposing sides. Instead, they base their decision on general considerations concerning the need for efficient procedures in cases such as these. Such reasoning provides the apparent grounds for making the following rather crucial observation concerning the facts:

\begin{quote}... It was not a mistake to take Ola Måland's statement into consideration, as he was, and still is, a significant property owner. NVE's statement to the effect that granting the concession will facilitate 
a more effective use of the water seems appropriate, as it refers to a current hydro-electric plant that 
exploits a waterfall of 13.5 meters.
\end{quote}

Nowhere in their decision do they mention the statement made by Hjelmeland kommune, nor do they comment on the fact that alternative hydro-power, as suggested by the NVE itself, and pointed to in this statement, amounts to exploiting the waterfall over a difference in altitude of some 550 meters. In fact, the hydroelectric plant that they do mention has nothing to do with Ola Måland and the other owners, but exploits the same water further downstream. It was brought up in the testimony made by a representative from NVE, who, when pressed on the matter, claimed that the reasonable way to interpret the paragraph that Stavanger Tingrett quoted, and to which Gulating Lagmannsrett implicitly refer, was to see it as a statement regarding this hydro- electric plant. In light of the statement provided by Hjelmeland kommune, to which the report explicitly refers, this appears to be a manifestly ill-founded interpretation. But the regional court adopted it, without further comment.

As far as the legal basis of their decision is concerned, it seems that Gulating Lagmannsrett holds, quite generally, that the practice adopted by the water authorities in cases like these still provide adequate protection for original owners, and that it is not for the courts to subject it to critical review. As mentioned, they seem to base their stance in this regard on an overreaching appeal to the need for efficient procedures to deal with cases such as these.

The decision was appealed by Ola Måland and other, and the Norwegian Supreme Court decided to consider the juridical aspects of the case. The appeal concerning the assessment of the facts made by Gulating Lagmannsrett would not be considered, but was to be taken as correct. Since Gulating Lagmannsrett decided to regard as inessential several facts that were seemingly apparent, even from the report made by NVE itself, the appellants presented these facts to the Supreme Court and argued that Stavanger Tingrett was right regarding their consequences. \noo{In addition to this, written statements were retrieved from the Øystein Grundt, the public officer from the NVE that had been responsible for the preparation of the case, and Harald Sollie, }

The Supreme Court ruled in favor of Jørpeland Kraft AS. They comment on the relevant facts on 
p. 9 of their decision. There, they mention that Jørpeland Kraft AS had considered the possibility that a hydro-electric scheme could be undertaken by local property owners. As we mentioned in Section \ref{sum}, a statement was provided to the NVE by Jørpeland Kraft AS themselves -- the parties who stood to benefit from the transferral -- addressing one possible project that was deemed not to be commercially viable. Recall that in the same statement another project was also identified -- in the same river, using the same water -- that they claimed was such a good project that it could be carried out even after the transferral. As we mentioned, the statement does not say anything about what the property owners stand to loose when the water from Brokavatn disappears, and the Supreme Court is also silent on this. Nor do they mention that the statement was never handed over to the applicants, and that the details of the calculations were never handed over to, or considered by, the NVE. In fact, the full report first appeared during the hearing at Gulating Lagmannsrett, but this fact was not considered relevant by the Supreme Court.

Moreover, the Supreme Court remains silent on the fact that the conclusion concerning efficiency of exploitation contradicts both the NVE's own assessment, the statement made by Hjelmeland Kommune, and also all subsequent assessments made both on behalf of the applicants and on behalf of Jørpeland Kraft AS. We mention that all of the above were presented to all national courts, including the Supreme Court.

As to the legal questions raised by the case, the Supreme Court makes a more detailed argument than the regional court, culminating in the conclusion that established practice still provides adequate protection. Interestingly, the Supreme Court base their arguments in this regard on the premise that the case does \emph{not} involve expropriation of waterfalls. A similar sentiment is expressed by Gulating Lagmannsrett, and it was also argued for by Jørpeland Kraft AS, but the true force of this point of view did not become apparent until the case reached the Supreme Court. 

The Court first concludes that a legal basis for the concession to transfer the water is to be found in the Watercourse Regulation Act, Section 16. Moreover, they conclude that while this provision alone does not provide a right to expropriate the waterfall, it does give the applicant a right to divert the water away from it. While the Supreme Court notes that this amounts to an interference in property rights, they take it as an argument in favor of regarding the rules in the Watercourse Regulation Act as the primary source of guidance concerning what should be considered when preparing such cases. The hold, in particular, that the provisions in the Expropriation Act applies only so far as they supplement, and are not in conflict with, the rules of the Watercourse Regulation Act and established practice with respect to the provisions in this Act. Moreover, the main reason they give for this is that the diversion of water is \emph{not} to be considered as an expropriation of a waterfall.

There is, as we mentioned, no rule in the Watercourse Regulation Act which states that the authorities are required to consider specifically the question of how the regulation affects the interests of property owners. Such a rule is found in the Expropriation Act, Section 2, but according to the Supreme Court, it does not apply in cases where water is being diverted away from a river. This is so, according to the Supreme Court, because transferral of water is not regarded as a case of expropriation of a right to the waterfall, but merely an expropriation of a right to deprive the waterfall of water.

This is significant in two ways. First, it is important with respect to the legal status of owners who are affected by projects involving transferral of water. In Norwegian law after Måland, it seems that established practice with respect to the assessment of such cases, focusing on environmental aspects and the positions taken by various interest groups, is beyond reproach already because such cases do not involve expropriation of waterfalls. However, considering that the Norwegian water authorities seem to follow these practices generally, and not just in cases where water is transferred, it remains to be seen if this is a practically significant difference in the level of protection. Is the conclusion regarding the admissibility of current administrative practices supposed to apply only to those cases when water is subject to transferral? If it is, then it leads to the peculiar situation that the level of protection for owners depend solely on the way in which the developer propose to gain control over the water. The difference appears completely arbitrary, however, at least from the point of view of owners. But of course, it will soon cease to be arbitrary for developers, who must be expected to favor gutter projects, collecting water from many small rivers and diverting it, since this mode of exploitation makes it easier to acquire necessary rights. On the other hand, if the Supreme Court is to be understood as saying that traditional practices are adequate in general, the consequences of the decision seem fairly dramatic for local owners. It appears that it is not possible, in cases involving expropriation of waterfalls, to solicit any kind of judicial review, not even in circumstances when the factual basis of the decision is manifestly erroneous, and not even if this appears to be the consequence of the authorities neglecting to keep local owners informed about the assessments made regarding their interests.

To illustrate that a lack of consultation is a general problem, and not confined to the particular case of \emph{Måland}, we will conclude by offering a quote from Harald Solli, director of the Section for Concessions at the Ministry of Petroleum and Energy, who submitted written evidence to the Supreme Court regarding the practices followed in cases involving expropriation of waterfalls. Below, we give one of several exchanges that seem to indicate that under current practices, local owners are left in a rather precarious position (my translation).

\begin{quote}
Q: In cases such as this, should owners affected by a loss of small scale hydro-power potential be kept informed about the factual basis on which the authorities plan to base their decision? I am thinking especially about those cases in which the authorities make an assessment regarding the potential for small scale hydro-power on affected properties. \\
A: Affected owners must look after their own interests. The assessments made by the NVE in their report is a public document, and it can be accessed online through the home page of the NVE.
\end{quote}

By their reasoning in \emph{Måland}, it appears that the Supreme Court gave this dismissive attitude towards local owners a stamp of approval. In light of this, we believe the study of the law in a socio-legal setting becomes all the more relevant. For while this attitude might be a reflection of correct national law, as decided in the final instance by the Supreme Court, it seems pertinent to ask if it is \emph{reasonable} law. Also, it seems that one must ask if a case can not be made with respect to human rights, by arguing that the protection awarded is insufficient in this regard. This point, while it was raised by the original owners in \emph{Måland}, did not receive any separate treatment in the Supreme Court. In the following section, we briefly describe some more questions we think the case raises and which we will address further in subsequent chapters.

\section{Consequences of the case and the questions it raises}\label{cons}

Following \emph{Måland}, it seems we must conclude that the development which has taken place in the energy sector, and has lead to small scale hydro-power becoming profitable and possible for local owners to carry out themselves, does not imply that original owners are entitled to increased participation in decision-making processes under national law. Even if this is the view held by the Norwegian judiciary, we should of course not overlook the possibility that the water authorities themselves will eventually adopt new practices regarding the assessment of such cases. So far, however, it seems that they stick quite closely to the established routine. 

Since the outcome in Norwegian Courts was that established practices were not found to be in breach of principles of Norwegian expropriation law, it seems reasonable to ask instead about the sustainability of these practices. In fact, the case of \emph{Måland} seems to illustrate precisely why the current system is inadequate, and how it can lead to decisions that appear ill-founded and leave the affected communities feeling marginalized. The likelihood of \emph{factual mistakes}, in particular, seems to increase greatly when the involvement of the local population is not ensured in the preparatory stages.

More importantly, it seems that decisions reached following a traditional process can easily lead to takings for which it is difficult to see any legitimate reason why the project proposed by the developer would be a better form of exploitation than allowing the local owners to carry out their own projects. Indeed, in the case of \emph{Måland}, it seemed that small-scale hydro-power would be a better way of harnessing the water in question, even in the sense that it would be more efficient, and would provide the public with more electricity at a lower cost. More generally, unless the issue of alternative exploitation in small scale hydro-power is considered during the assessment made by the water authorities, one risks making decisions that are not in the public interest at all. 

Even worse, it can send out the signal that expropriation of owners' rights is undertaken solely in order to benefit the commercial interests of the energy company applying for a development license. We mentioned in Section \ref{context} that this mechanism, whereby expropriation appears to benefit commercial interests rather than the public, is becoming increasingly important in the international context as well. It is particularly in this regard that we think the case of Norwegian waterfalls warrants attention from the perspective of human rights. At this point, it seems appropriate to recall some concerns expressed by US Justice O'Connor, taken from her dissenting opinion in \emph{Kelo}.

\begin{quote}
Any property may now be taken for the benefit of another private party, but the fallout from this decision will not be random. The beneficiaries are likely to be those citizens with disproportionate influence and power in the political process, including large corporations and development firms. As for the victims, the government now has license to transfer property from those with fewer resources to those with more. The Founders cannot have intended this perverse result.
\end{quote}

In \emph{Kelo}, it seemed that a major point of contention was whether or not these grim predictions did indeed reflect a realistic analysis of the fallout of the decision. Surely, anyone who agrees with Justice O'Connor in her prediction, would also agree with here conclusion that it is perverse. However, whether her pessimism is warranted by empirical fact seems less clear. In this context, we believe the case of Norwegian waterfalls can serve an important broader purpose, as a means towards shedding more light on the hypothesis that a loose interpretation of the public interest requirement will indeed lead to a transfer of property from those with fewer resources to those with more. The \emph{Måland} case, and the current tensions regarding expropriation for the benefit of Norwegian hydro-power, seems to suggest that her concern should indeed be taken seriously. Also, the Norwegian experience seems to show that we need to be clear about the fact that property has a social and political function that goes beyond the financial interests of individuals. For the Norwegian case at least, it seems particularly relevant to ask if local people, by virtue of their right to property and their original attachment to the land, have a legitimate expectation \emph{both} that their commercial interests should be protected, \emph{and} that they should be granted a say in decision-making processes. Financial protection does not necessarily imply social protection, and the right to participate and be heard might be both more significant, and harder won, than the right to be compensated according to whatever the powers that be come to regard as the market value of the property in question.

Another perspective, which we will also pursue further in subsequent chapter, is the question of how property rights relates to the overreaching goal of sustainable development of natural resources. Rather than seeing property rights as a means towards securing sustainable development, it seems more common to see it as an impediment. This, indeed, has shaped much of the Norwegian discourse regarding environmental law and policy, including that which relates to waterfalls.\footnote{For example, such a skeptical view of property rights appear to provide an overriding perspective in \cite{backer1} (in Norwegian), which is a widely used textbook on environmental law in Norway.} Moreover, a typical justification given for interference in property is that an equitable and responsible management of natural resources requires it. It seems to us, however, that an egalitarian system of private ownership of resources -- as we find in Norway for the case of waterfalls -- could itself serve as a sustainable basis for management of these resources. It seems plausible for us to suggest that private property rights is one of the most robust ways in which local communities can be given a degree of self-determination concerning how to manage local resources. This is typically considered desirable also from the point of view of sustainability, but perhaps even more importantly, when property is in the hands of the many rather than the few, is it not also reasonable to expect that the state will be able to more effectively and rationally exercise its regulatory powers? Otherwise, the danger is that the government is being intimidated by large commercial enterprises, perhaps partly owned by the State itself, that command political influence and might not take lightly to what they perceive as undue political interference in their business practices. Such a position might be tenable if you are one of the worlds leading energy companies, but hardly if you are a farmer. 

We think the case of \emph{Måland} suggests that we should investigate these questions in more depth. It seems, in particular, that we must ask about the extent to which commercial companies have succeeded in usurping the notions of sustainable development and public interest, putting the power of these ideas to use in order to secure control over resources and to enlist governmental support, and favorable treatment, for their own commercial undertakings. The extent to which such a mechanism influences the Norwegian energy sector, and the possible implications this might have, both legally and socially, remains to be worked out.

In subsequent chapters, two questions arising from this will receive particular focus. First, we will aim to clarify the importance of the conflict between large scale hydro-power and small scale development by surveying recent and current hydro-power projects in Norway, not in any depth, but by taking note of whether the issue arose. Secondly, we will aim to shed light on the importance of small scale hydro-power to the communities in which local owners reside. As we mentioned, they are usually farmers, and most often in areas were farming is becoming increasingly unprofitable. From the socio-legal point of view it seems highly relevant to ask who the people who loose their resources are, and in what social context we find them. Moreover, while it is clear that hydro-power has become an important source of income in many small and relatively impoverished farming communities, the exact implications of this development, financially and socially, remains to be mapped out.

Following this, it seems natural to return to the legal question of the legitimacy of interference, not from the point of view of national law, but from the point of view of property as a human right. Importantly, it seems to us that property has a clear social dimension, and that mapping out the socio-legal function of specific property rights should inform the judgment we make regarding the level of protection to which owners are entitled. Also, while property is an individual right, it can also be a communal one, and, as such, it can serve to empower local communities that would otherwise be marginalized. The protection of an egalitarian structure of ownership, then, does not appear to be subsumed by, or even conceptually the same as, protecting against individual transgressions. We believe that the case of Norwegian waterfalls demonstrates that this should be kept in mind when analyzing the legitimacy of interference in property for the benefit of commercial undertakings.

\noo{current ownership structure of waterfalls is therefore not simply a question of protecting the commercial interests of individuals who happen to own valuable resources, but also a question of protecting the local communities where these resources are found, giving them the possibility of influencing the way in which the resources are to be harnessed. It seems, however, that local people are often in danger of being seen as an hindrance, both to sustainable development and economic growth, because the commercial companies, along with the environmental interests groups, have claimed this stage as their own. Such, it seems, is the case for Norwegian waterfall. Despite an explosion of interest in small scale hydro-power in recent years, there still seems to be little room left for local communities in the Norwegian discourse concerning hydro-power. It will be an important aim of our work in following chapters to map our in more detail how this influences the law and the administrative policies that are adopted.
}

\section{Conclusion}\label{conc}

As we have shown, \emph{Måland} serves to illustrate many of the current tensions and issues surrounding expropriation of waterfalls in Norway. It also serves to clarify the extent to which local owners are 
marginalized under the regulatory practices currently in place, and shows that the regulatory system does not clearly separate the question of how to judge an application to undertake development from the question of whether or not expropriation should take place. Moreover, the case seems to suggest that this will tend to lead to the emphasis being on issues that have to do with development, while issues relating to expropriation, and owners' interests, will be overlooked. Summing up, the case seems to show that the current regulatory system in Norway functions in such a way that it is bound to give rise to conflicts between local interests and the interests of commercial companies and the State.

The case also sheds new light on the legitimacy of using expropriation in order to benefit commercial interests. In this way, it takes on broader significance, by lending empirical support to the prediction offered by Justice O'Connor with respect to \emph{Kelo}, regarding the fallout of a loose interpretation of the public interest requirement for expropriation.

In our opinion, this contributes to making Norwegian waterfalls an interesting case study on expropriation,  and one that warrants further consideration with respect to human rights. In subsequent chapters, we will offer such an analysis, by addressing the question of whether or not local owners and communities can claim that they are entitled to greater protection than that which is currently provided under Norwegian law.

