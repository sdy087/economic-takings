%\newcommand{\isr}[1]{{#1}}

\chapter{Taking Property for Profit}\label{chap:2}

\section{Introduction}\label{sec:intro}

In the previous chapter, I argued that economic development takings should be considered a separate category of interference in private property. I also placed it in the theoretical landscape, by relating it to the social function theory of property. In particular, I argued that economic development takings raise questions that require us to depart from the individualistic, entitlements-based narrative that otherwise dominates in property law.

So far, I have argued for a certain way of reasoning about economic development takings, but I have not addressed in depth what the law has to say about them. In this chapter, I consider this question, by giving an overview of how economic development cases are dealt with in some representative jurisdictions. In addition, I consider some recent proposals for reform.

First, in Section \ref{sec:lgppp}, I will comment briefly on the importance of economic development takings on the global stage. I note that the core issues I address are relevant also in the context of developing economies, even when property rights as such remain a less stable basis on which to reason about the rights and obligations of individuals and communities. This argument rests on the social function approach to property, which suggests that formal recognition of title is not a necessary precondition for legally recognising community interests inhering in the institution of property.  

In Section \ref{sec:contrast}, I move on to contrast the English and the German approach to the legitimacy of takings, particularly in the context of economic development. This serves to introduce the topic of my thesis from the point of view of European law, where the issue of economic development takings has attracted far less attention than in the US. It appears to be gaining importance, however, as the increasing influence of public-private partnerships means that takings for development are increasingly becoming takings for profit also in Europe.

In Section \ref{sec:echr}, I elaborate on a practically significant pan-European property clause, namely Article 1 of Protocol 1 (P1(1)) of the European Convention of Human Rights (ECHR). I argue that this clause provides an interesting perspective on the legitimacy issue, asking us to focus on the proportionality of the interference, judged relatively to its social and political context. I also consider some possible objections against the human rights approach, including the worry that the court in Strasbourg is not well-placed to be the arbiter of social and individual justice throughout Europe. At the same time, I point to some recent decisions at the Court that I believe signal hope that the case law on property is moving away from ill-conceived ``micro-management'', towards a more open-ended jurisprudence that seeks to force member states to address systemic problems that they might otherwise be reluctant or incapable of raising to the national agenda. Here the involvement of a (hopefully) politically neutral institution like the ECtHR can serve an important purpose, particularly if it manages to tailor its own case law in such a way as to leave room for local institutions of the member states to work out for themselves how to concretely resolve human rights issues flagged by the Court in Strasbourg.

In Section \ref{sec:us}, I return to the US setting, by presenting in detail how the perspective on economic development takings, mediated through case law on the public use restriction, has evolved since the 19th century until today. I structure the presentation as a story in two parts, describing the situation before and after the {\it Kelo} case. For the pre-{\it Kelo} presentation, I begin by pointing out that the case law on the public use restriction was initially developed by state courts, who would adjudicate legitimacy cases against the respective state constitutions (which typically also involve some sort of public use restriction on the takings power). 

I go on to observe how the Supreme Court adopted deference to state {\it courts} initially, before changing their perspective by adopting a policy of deference directed rather at the state {\it legislature} (in practice also the administrative branch).\footnote{I argue that this shift in Supreme Court jurisprudence can be pin-pointed rather precisely to the case of {\it Berman}, see \cite{berman54}.}

I go on to argue that this shift in case law at the federal level had the effect of destabilizing the established state approach to economic development takings, resulting in increased tension and controversy, paving the way to {\it Kelo}. In essence, my argument is that the Supreme Court was right in taking a deferential stance with respect to local institutions, but wrong in stripping the public use restriction of content, a move that would undermine the authority of state courts. In effect, the federal takings jurisprudence weakened the legal authority of a very sensible {\it local} judicial constraint on executive power, a constraint that was also important to the proper division of power at the state level.

In Section \ref{sec:postkelo}, I follow this up by a discussion of developments after {\it Kelo}, which has seen a resurgence in state court scrutiny of the public use requirement, often backed up by state legislation that explicitly seeks to limit the scope of takings for economic development. According to some, such state reforms have been largely ineffective. Ilya Somin, one of the most prolific writers on economic development takings in the US, has argued that this is partly due to so-called ``rational ignorance'' of political decision-makers and voters regarding the subtleties of the public use issue. The idea is that the distance between policy makers and communities affected by economic development takings is too great, so that policy makers have no incentive to consider the finer details of the takings equation. In principle, the US public is almost unanimously on the side of the local communities in cases like {\it Kelo}, but in practice, the great distance between political cause and effect means that effective reform policies are hard to formulate, since they tend to rely on oversimplified narratives tailored to centralised processes of decision-making.

In Section \ref{sec:ir}, I go on to consider a few recent suggestions that I regard as possible answers to this concern. These suggestions, in particular, focus on the need for new frameworks for collective action, institutions that can replace the top-down dynamics of eminent domain in cases of economic development. The idea here is to ensure a greater level of self-governance for the communities directly affected by the development, the individual members of which have a rational incentive to invest time and effort in reaching sophisticated compromises that can replace the use of black-white solutions (be it in the form of an economic development taking or a politically sanctioned top-down {\it ban} on such takings).

I argue that this idea embodies both a natural and necessary counterpart to increased judicial scrutiny of the public use restriction. In particular, I argue that the two ideas are mutually conducive to each other, properly conceived. This argument will set the stage for the case study in the second part of the thesis, where I explore the tension between takings and self-governance in the context of hydropower development in Norway. In Section \ref{sec:conc2}, I offer a conclusion.

\section{The ``Underscrutinised'' Language of Economic Development}\label{sec:lgppp}

Public-private partnerships are becoming increasingly important to the world economic order.\footnote{See generally \cite{saussier13}.} To some, they are the illegitimate children of privatisation and deregulation, while others see them as efforts to make the public sector more efficient and accountable. Either way, their numbers are growing, and they appear to be here to stay.\footnote{Although their potentially pernicious effects on stability and accountability has also been noted. See, e.g., \cite{baker03} (arguing that ``the Enron scandal can be better understood as an American form of public private partnership rather than just another example of capitalism run amok'').} In this situation, it is inevitable that when eminent domain is used to acquire property for economic development, those who directly benefit will often be commercial companies rather than public bodies. In the previous chapter, I pointed out how indirect public benefits are typically used to justify such takings. Standard legitimizing reasons include the prospect of new jobs, increased tax revenues, and various other economic and social ripple effects. However, as I have indicated, economic development takings have a tendency to result in controversy.

In the US after {\it Kelo}, they have also been at the forefront of the constitutional property debate. In the rest of the world, a similar shift in academic outlook has yet to take place, but expropriation-for-profit situations are increasingly coming into focus also on the global stage.\footnote{See, e.g., \cite{gray11,waring13,verstappen14}.} If we lift our perspective slightly, to consider commercially motivated interference in property more generally, it even seems appropriate to speak of a crisis of confidence in property law, particularly in relation to land rights. This is most clearly felt in the developing world, where egalitarian systems of property use and ownership are coming under increasing pressure. It has been noted, in particular, that large-scale commercial actors are assuming control over an increasing share of the world's land rights, a phenomenon known as {\it land grabbing}.\footnote{See generally \cite{borras11}.} 

So far, most research on land grabbing has looked at how commercial interests, often cooperating with nation states, exploit weaknesses of local property institutions, to acquire land voluntarily, or from those who lack formal title. However, the danger of {\it Kelo}-type reasoning has also been \isr{recognised}. In particular, it has been noted how the purported public interest in economic development can be used to justify land grabs that would otherwise appear unjustifiable. In a recent article, Smita Narula cites {\it Kelo} directly and warns that procedural safeguards alone might not provide sufficient protection against abuse. She writes:
\begin{quote}
Procedural safeguards, however, can all too easily be co-opted by a state because its claims about what constitutes a public purpose may not be easy to contest. Particularly within the context of land investments, states could use the very general and under-scrutinized language of ``economic development'' to justify takings in the public interest.\footcite[157]{narula13}
\end{quote}

This underscores the broader relevance of the study of economic development takings. In addition, it reminds us that the question of what can be justified in the name of ``economic development'' is a general one, not confined to particular systems for organizing property rights. To address this, and to restore confidence in the institution of property more generally, many turn towards {\it human rights}. These scholars argue that a human right to land should be \isr{recognised} on the international stage, a right that would apply even when those most affected by a land grab lack formal title.\footnote{See generally \cite{schutter10,schutter11,kunnerman13}.} If successful, this approach promises to deliver basic protection against interference in established patterns of property use independently of how particular jurisdictions approach property.

In Europe, a human rights perspective is already of great practical significance due to the European Convention of Human Rights (ECHR) and the court in Strasbourg (ECtHR). But, of course, in the context of land grabbing, protecting land rights is not primarily a question of protecting the civil law ideal of individual dominion. Rather, it is a question of providing protection against large-scale transactions that \isr{destabilise} or destroy established patterns of land use, to the detriment of local communities. Nevertheless, the questions raised by the public interest  narrative -- and the notion of ``economic development'' in particular -- should arise in much the same way as in cases when formal title is acquired following a state-\isr{authorised} taking.

Hence, it is somewhat surprising that the special category of for-profit takings has not received more attention from the point of view of human rights law. In human rights discourse, the focus tends to be rather on fairness and proportionality as broad benchmarks, in addition to specific values related to food and water security as well as the protection of basic livelihoods, issues that arise with particular urgency in the context of third-world land grabs. However, to achieve effective protection we need firm categories and enforcible legal principles to back up our broad benchmarks and good intentions. In this regard, I think Narula is right to stress that the lack of a convincing approach to the notion of ``economic development'' is a crucial challenge.

On the one hand, economic development is no doubt a sound overreaching goal, particularly for poor nations. But at the same time, the risk of abuse is obvious when such a vague term is used to justify dramatic interferences in property. After all, interferences in property can cause severe disturbances in people's life. This, moreover, is true for middle-class US homeowner in much the same way as it is true for members of self-sustaining agrarian communities in Africa, although the stakes might be very different.

As illustrated by {\it Kelo}, deep conflicts can arise in this regard also in developed democracies with long established and relatively stable systems of private property. In the following, I will attempt to shed further light on the issue as it arises in such legal systems, without considering the additional complications that arise when property itself is a more fragile concept. I note, however, that according to the social function view of property, there is no need to view formally \isr{recognised} property rights as completely distinct from rights arising from property use that is not based on formal title. The two are intertwined and the difference between them is at most a matter of degree.\footnote{Moreover, if the human flourishing account of property values is successfully developed, there should even be hope that a unified normative treatment can be given at some point.}

At the same time, my case study will look to Norway, a prosperous European country with a long and relatively stable tradition of an egalitarian distribution of land rights among the rural population. Hence, it is prudent to narrow down the discussion here by focusing on jurisdictions where property as such is a similarly stable institution.\footnote{The relation with third-world land grabbing is a highly interesting question for future work.} I will do so now, beginning with a brief look at English and German law, to illustrate that there are significant variances in how different European jurisdictions think about property in general and takings in particular. Then I turn my attention to the ECHR and the proportionality test that is now at the core of property adjudication at the ECtHR.

Following this, I move on to consider the US in greater depth, both the historical debate that led to {\it Kelo} and the suggestions for reform that have emerged following its backlash. A closer look is necessary because of the sheer magnitude of writing on this issue in the US. Moreover, while much of it is repetitive and coloured by the tense political climate, I believe some historical points, as well as some recent suggestions for reform, are highly relevant also to the international setting. To single out and analyse those aspects is the main aim of this part of the chapter. Indeed, the current debating climate in the US might be an indication of what is to come also in Europe, if concerns about the legitimacy of economic development takings are not taken seriously.

%I also highlight what I believe to be a connection between the situation in the US leading up to {\it Kelo} and the present situation in Europe, illustrated by the fact that the European Court of Human Rights is now explicitly endorsing ``stronger protection'' of property rights.  I attempt to identify the reasons behind calls for a stricter approach, arguing that it is connected to the fact that interferences in property under modern regulatory regimes is sanctioned in wide a range of different circumstances, serving to undermine their status as a necessary burden imposed on owner's according to the will of the greater public. In some cases, rather, takings appear to both owners and the public as improperly motivated and socially and politically unfair. I note that this happens particularly often in economic development cases, when commercial actors benefit to the detriment of local communities. I go on to list some concrete issues that arise with respect to such takings and that have been flagged as problematic in the literature.
%
%Following up on this, I consider various proposals that have been made to resolve tensions and limit the possibility of abuse in economic development cases. The differences of opinion that have been expressed in this regard have been quite substantial, and proposals have ranged from suggesting an outright ban on economic development takings  (Somin 2007; Cohen 2006) to suggesting that the best way forward is to reassess principles for awarding compensation in such cases (Householder 2007; Lehavi and Licht 2007).

%Much of the current theory focus on assessing traditional judicial safeguards that courts can rely on to prevent abuses, pertaining primarily to the material assessment of proportionality, public purpose, and compensation. 

%In the last part of the chapter, I will focus on a very interesting strand of recent work in the US, which shifts attention towards procedural rules that can help address the worry that economic development takings tend to suffer from a democratic deficit. The core concern is that the manner in which eminent domain decisions are typically made, and the way in which owners are compensated, might be unsuitable for economic development cases. Importantly, the need for special procedures has been noted, to restore legitimacy.\footnote{See generally \cite{lehavi07,heller08}.} This ties the US debate even closer to the European context, where proportionality, not public use, has become the key notion in property protection. Several recent suggestions from the US can be conceptualized as suggestions that aim to secure fairness and proportionality, while paying less attention to the formalistic question of what constitutes a ``public use''.
%
%%Also, it allows us to be very clear about a special concern that arises for economic takings cases: under current regulatory regimes, the government and the developer together often dominate the decision-making process completely, leaving the property owners marginalized. Hence, there is often a {\it democratic deficit} in such cases, resulting in discontent and a feeling that the taking is not in the public interest at all. Importantly, some recent writers hypothesize that if the proper balance can be restored in the decision-making process, so will the decision reached appear more legitimate, also with respect to the public use clause. In my opinion, this idea is crucial, and together with the question of compensation, which raises a similar structural problem, it will guide the rest of the work done in this thesis. 
%

In response to that worry, this chapter aims to  bring into focus the key question of how to ensure meaningful participation for owners and their local communities in decision-making pertaining to economic development on their land. The tentative answers provided in Section \ref{sec:ir} will set the stage for the remainder of the thesis, where these answers will be assessed in depth against the case study of Norwegian hydropower.

%In particular, I will consider two special semi-judicial procedural systems used in such cases in Norway, one targeting compensation following expropriation, and another used as an alternative to expropriation, particularly in cases when development requires cooperation among many owners.

%I conclude by arguing that approaches along procedural lines represent the best way forward in relation to addressing issues associated with economic development takings. This raises the following problem, however: what procedural principles can be used to ensure meaningful participation, without hindering socially and economically desirable development projects? This question sets the stage for the remainder of my thesis, where I conduct a case study of expropriation for the development of hydro-power in Norway. In particular, I will consider two special semi-judicial procedural systems used in such cases in Norway, one targeting compensation following expropriation, and another used as an alternative to expropriation, particularly in cases when development requires cooperation among many owners.

\section{A European Contrast}\label{sec:contrast}

Economic development takings have not become as controversial in Europe as they are in the US, but there have been cases where the issue has come up, in several different jurisdictions.\footnote{For instance, in the UK, Ireland and Germany, as well as in Norway and Sweden. See \cite[466-483]{walt11}; \cite{stenseth10}.} The P1(1) of the ECHR protects property, but the legitimacy of economic development takings has not yet been discussed in case law from the European Court of Human Rights (ECtHR). However, it is interesting to analyse cases like {\it Kelo} against P1(1), particularly since the ECtHR has developed a doctrine that focuses on ``proportionality'' and ``fairness'' rather than the purpose of interference.\footnote{See generally, \cite[Chapter 5]{allen05}. This approach may become even more significant as a source of property protection in the future, as the ECtHR have indicated that there are ``jurisprudential developments in the direction of a stronger protection under Article 1 of Protocol No. 1'', see \cite[135]{lindheim12}.}

In this section, I address economic development takings from the point of view of European sources. I first contrast English and German law, to show that there are significant differences between European jurisdictions in this regard. I then go on to give a more detailed presentation of the unifying property clause in P1(1). The case law from the ECtHR is presented and analysed in some depth, in an effort to assess how the ECtHR would be likely to approach an economic development case such as {\it Kelo}. In particular, I argue that the proportionality doctrine offers an interesting approach to such cases. Importantly, the doctrine stipulates that a ``fair balance'' must be struck  between the interests of the property owner and the public.\footcite[Chapter 5]{allen05} I argue that such a perspective could make it easier to get to the heart of why economic development takings are often seen as problematic, without getting lost in theoretical discussions about the meaning of  terms like ``public use'' or ``public purpose''. However, I also raise the concern that the ECtHR is not the appropriate institution for applying the proportionality test. Indeed, its remoteness to most of Europe suggests that we should look for more locally grounded legitimacy-enhancing institutions. Such institutions will likely be better able to assess the fairness of interference in context.

I go on to discuss whether existing government institutions can serve this purpose, arguing that local courts may well be the best candidates. However, I also note that active application of the proportionality test in property cases might not be found at the local level. In this regard, the ECtHR could play a crucial role, by focusing on the systemic question of what issues local courts need to consider when assessing legitimacy of property interference. 

However, quite apart from this, there is reason to worry that judicial bodies are not ideally suited to carry out the kind of assessment that is required. Hence, new institutional proposals might be in order. I conclude by arguing that once the need for local grounding is recognised and met, the ECtHR has the potential to play an important and constructive role in providing oversight and developing basic principles, also with respect to new institutions that aim to deliver increased legitimacy at the local level.

\subsection{England}\label{sec:england}

In England, the principle of parliamentary supremacy and the lack of a written constitutional property clause has led to expropriation being discussed mostly as a matter of administrative law and property law, not as a constitutional issue.\footcite{taggart98} Moreover, the use of compulsory purchase -- the term most often used to denote takings in the UK -- has not been restricted to particular purposes as a matter of principle. The uses that can warrant compulsory alienation of property are those that parliament regard as worthy of such consideration. However, as private property itself has long been recognised as a fundamental right, the power of compulsory purchase has typically been exercised with caution. 

In his {\it Commentaries on English Law}, William Blackstone famously described property as the ``third absolute right'' that was ``inherent in every Englishman''.\footcite[134-135]{blackstone79}  Moreover, Blackstone expressed a very restrictive view on the possibility of expropriation, arguing that it was only the legislature that could legitimately interfere with property rights. He warned against the dangers of allowing private individuals, or even public tribunals, to be the judge of whether or not the ``common good'' could justify takings. Blackstone went as far as to say that the public good was ``in nothing more invested'' than the protection of private property.\footcite[134-135]{blackstone79}

Historically, Blackstone's description conveys a largely accurate impression of takings practice in England. Indeed, Parliament itself would usually be the granting authority in expropriation cases, through so-called {\it private Acts}. Hence, compulsory purchase would not take place unless it had been discussed at the highest level of government. Moreover, the procedure followed by parliament in such cases strongly resembled a judicial procedure; the interested parties were given an opportunity to present their case to parliament committees that would then decide whether or not compulsion was warranted.\footnote{See \cite[13-16]{allen00}. While this procedure reflected a protective attitude towards private property, recent scholarship has also pointed out that expropriation was in fact used very actively in Britain, particularly following the glorious revolution, see \cite{hoppit11}.}

On the one hand, the direct involvement of parliament in the decision-making process reflected a fundamental respect for property rights. But at the same time, parliamentary supremacy also meant that the question of legitimacy was rendered mute as soon as compulsory purchase powers had been granted. The courts were not in a position to scrutinize takings at all, much less second-guess parliament as to whether or not a taking was for a legitimate purpose.

Eventually, an overworked parliament developed procedures for dealing more expeditiously with takings cases. Moreover, during the 19th Century, as an industrial economy developed, private Acts granting compulsory purchase powers to commercial companies grew massively in scope and importance.\footnote{See \cite[204]{allen00}.} Private railway companies, in particular, regularly benefited from such Acts.\footnote{\cite[204]{allen00}. See generally \cite{kostal97}.} During this time, the expanding scope of private-to-private transfers for economic development led to high-level political debate and controversy. Usually, it would attract particular opposition from the House of Lords. Interestingly, this opposition was not only based on a desire to protect individual property owners. It also often reflected concerns about the cultural and social consequences of changed patterns of land use.\footcite[204]{allen00} 

Hence, the early debate on economic development takings in the UK shows some reflection of a contextual approach to property protection. At the same time, as society changed following increasing industrialisation, an expansive approach to compulsory purchase would eventually emerge as the norm. The idea that economic development could justify takings gradually became less controversial.

Today, the law on compulsory purchase in England is regulated in statute and the role of courts is to a large extent limited to the application and interpretation of statutory rules. Some common law rules still play an important role, such as the {\it Pointe Gourde} rule, which stipulates that changes in value due to the compensation scheme itself should be disregarded when calculating compensation to the owner.\footnote{The rule takes it name the case of \cite{gourde47}. The underlying principle, including also statutory regulations with a similar effect, is referred to as the ``no scheme'' principle, see \cite{lawcom01}. The principle is found in many jurisdictions, see \cite{sluysmans14}. The principle is often quite contentious, and notoriously hard to apply in practice. For a recent attempt at clarifying the principle, see \cite{waters04}. I note that a strict interpretation of the no-scheme principle effectively precludes benefit sharing between takers and owners, a phenomenon that is of particularly relevance in the context of economic development takings. I will not address this particular issue in any depth here -- I choose instead to focus on legitimacy of takings in a broader, non-compensatory sense. However, the compensation aspect of economic development takings is also very interesting (and challenging). For further details, I refer to \cite{dyrkolbotn15}.} With respect to the question of legitimacy, however, the starting point for English courts is that this is a matter of ordinary administrative law.

More recently, the \cite{hra98} adds to this picture, since it incorporates the property clause in P1(1) into English law. Even so, the usual approach in England is to judge objections against compulsory purchase orders on the basis of the statutes that warrant them, rather than constitutional principles or human rights provisions that protect property.\footnote{The important statutes are the \cite{ala81}, the \cite{lca61}, the \cite{tcpa90} and the \cite{pcpa04}. Acquisition of Land Act 1981, the Land Compensation Act 1961, the Town and Country Planning Act 1990 and the Planning and Compulsory Purchase Act 2004.} It is typical for statutory authorities to include standard reservations to the effect that some public benefit must be identified in order to justify a compulsory purchase order, but the scope of what constitutes a legitimate purpose can be very wide. For instance, to warrant a taking under the \cite{tcpa90}, it is enough that it ``facilitates the carrying out of development, redevelopment and improvement on or in relation to the land''.\footcite[226]{tcpa90} 

While various governmental bodies are authorised to issue compulsory purchase orders (CPOs), a CPO typically has to be confirmed by a government minister. The affected owners are given a chance to comment, and if there are objections, a public inquiry is typically held. The inspector responsible for the inquiry then reports to the relevant government minister, who makes the final decision about whether or not it should be granted, and on what terms. The CPO may then be challenged in court, but will usually only be scrutinized on the basis of whether or not it lies within the scope of the statute authorising it. Hence, the discussion and evaluation performed by the court is firmly grounded in statutory rules.

That said, the idea that property may only be compulsorily acquired when the public stands to benefit permeates the system. Indeed, this has also been regarded as a constitutional principle, for instance by Lord Denning in {\it Prest v Secretary of State for Wales}.\footcite{prest82} He said:

\begin{quote}
It is clear that no minister or public authority can acquire any land compulsorily except the power to do so be given by Parliament: and Parliament only grants it, or should only grant it, when it is necessary in the public interest. In any case, therefore, where the scales are evenly balanced – for or against compulsory acquisition – the decision – by whomsoever it is made – should come down against compulsory acquisition. I regard it as a principle of our constitutional law that no citizen is to be deprived of his land by any public authority against his will, unless it is expressly authorised by Parliament and the public interest decisively so demands. If there is any reasonable doubt on the matter, the balance must be resolved in favour of the citizen.\footcite[198]{prest82}
\end{quote}

Lord Denning also supported the doctrine of necessity, as expressed by Forbes J in {\it Brown v Secretary for the Environment}:\footcite{brown78}

\begin{quote}It seems to me that there is a very long and respectable tradition for the view that an authority that seeks to dispossess a citizen of his land must do so by showing that it is necessary, in order to exercise the powers for the purposes of the Act under which the compulsory purchase order is made, that the acquiring authority should have authorisation to acquire the land in question.\footcite[291]{brown78}
\end{quote}

In practice, these principles are mostly implicit in legal reasoning, as a factor that influences the courts when they interpret statutory rules and carry out judicial review of administrative decisions. As Watkins LJ stated in {\it Prest}:

\begin{quote}
The taking of a person's land against his will is a serious invasion of his proprietary rights. The use of statutory authority for the destruction of those rights requires to be most carefully scrutinised. The courts must be vigilant to see to it that that authority is not abused. It must not be used unless it is clear that the Secretary of State has allowed those rights to be violated by a decision based upon the right legal principles, adequate evidence and proper consideration of the factor which sways his mind into confirmation of the order sought.\footcite[211-212]{prest82}
\end{quote}

In {\it R v Secretary of State for Transport, ex p de Rothschild}, Slade LJ referred to {\it Prest} and made clear that he did not regard it as expressing a rule concerning the burden of proof in compulsory purchase cases. Rather, he took it as more general observation on the severity of property interference and the importance of vigilance in such cases.\footcite{rothschild89} He pointed to ``a warning that, in cases where a compulsory purchase order is under challenge, the draconian nature of the order will itself render it more vulnerable to successful challenge''.\footcite[938]{rothschild89}

A nice example of how these sentiments influence the assessment of legitimacy of takings, showing how it is applied in economic development cases, can be found in the recent case of {\it Regina (Sainsbury’s Supermarkets Ltd) v Wolverhampton City Council}.\footcite{sainsbury10} Here a CPO was granted to allow the company Tesco to acquire land from its competitor Sainsbury, in a situation when they were both competing for licenses to undertake commercial development on the same land, owned partly by both. The decisive factor that had led the local authorities to grant the CPO was that Tesco had offered to develop a different property in the same local area, which was currently in need of regeneration. 

Sainsbury protested, arguing that the local council could not strike such a deal on the use of its compulsory purchase power. It was argued, moreover, that taking the land for incidental benefits resulting from development in a different part of town was not legitimate under the Town and Country Planning Act 1990. The UK Supreme Court agreed 4-3, with Lord Walker in particular emphasising the need for heightened judicial scrutiny in cases of private-to-private takings for economic development.\footcite[80-84]{sainsbury10} Lord Walker even cited {\it Kelo}, to further substantiate the need for a stricter standard in such cases.\footcite[81]{sainsbury10} 

However, the main line of reasoning adopted by the majority was based on an interpretation of the Town and Country Planning Act itself. In particular, the majority held that it was improper for the local council to take into consideration the development that Tesco had committed itself to carry out on a different site.\footcite[73-79]{sainsbury10} This, in particular, was not ``improvement on or in relation to the land'', as required by the Act.\footcite[336]{tcpa90} In addition, Lord Collins, who led the majority, said that ``the question of what is a material (or relevant) consideration is a question of law, but the weight to be given to it is a matter for the decision maker''.\footcite[70]{sainsbury10} Hence, the general importance of the decision for economic development cases is unclear.

Still, it is interesting to see how the purpose of the interference featured in the Supreme Court's interpretation and application of the statutory rules. The opinion of Lord Walker is particularly interesting, since he stresses that ``The land is to end up, not in public ownership and used for public purposes, but in private ownership and used for a variety of purposes, mainly retail and residential.''\footcite[81]{sainsbury10} He goes on to state that ``economic regeneration brought about by urban redevelopment is no doubt a public good, but ``private to private'' acquisitions by compulsory purchase may also produce large profits for powerful business interests, and courts rightly regard them as particularly sensitive.``\footcite[81]{sainsbury10}

Lord Walker then makes clear that he does not think it is impermissible, as such, for the local council to take into account positive effects on the local area, even when these do not directly result from the planned use of the land that is being acquired. Instead, he relies explicitly on the for-profit character of the taking, by arguing that ``the exercise of powers of compulsory acquisition, especially in a ``private to private'' acquisition, amounts to a serious invasion of the current owner's proprietary rights. The local authority has a direct financial interest in the matter, and not merely a general interest (as local planning authority) in the betterment and well-being of its area. A stricter approach is therefore called for.''\footcite[84]{sainsbury10} 

Lord Walker's opinion might indicate that the narrative of economic development takings is about to find its way into English case law. Moreover, a more critical approach might be adopted in the future, when compulsory purchase powers are made available to commercial companies wishing to undertake for-profit schemes. However, for schemes where the commercial aspect appears less dominant, English courts still appear very reluctant to quash CPOs, also when the purpose is economic development. This is so even in situations when the owners have requested a stricter standard of review on the basis of human rights law. 

For instance, in the case of {\it Smith \& Others v Secretary of State for Trade and Industry}, a caravan site was compulsorily acquired for development in connection with the London Olympic Games.\footcite{smith08} Some of the owners protested, including Romany Gypsies who used the caravans as their primary residence. A public inquiry was held, after which the inspector recommended that the CPO should not be confirmed until adequate relocation sites had been identified. However, due to the ``urgency, timing and importance'' of the project, the Secretary of State decided to go ahead before a relocation scheme was put in place (although he expressed commitment to ensuring satisfactory relocation).\footcite[10]{smith08} The owners argued that without satisfactory relocation plans, the interference in the property rights was not proportional and had to be struck down on the basis of human rights law, in particular Article 8 in the ECHR regarding respect for the home and private life.\footcite[27-51]{smith08}

The Court of Appeal considered the matter in great depth, applying the doctrine of proportionality developed at the ECtHR. Importantly, this doctrine was understood to go beyond the standard form of judicial review required under English law. However, the Court still concluded that the taking was proportional. This was largely based on the finding that ``the issue of proportionality has to be judged against the background that everyone accepts that an overwhelming case has been made out for compulsory acquisition of the sites for the stated objectives and that compulsory purchase is justified.''\footcite[42]{smith08} 

Justice Williams arrived at this conclusion after noting that the owners' {\it only} substantial objection against the CPO was that it was confirmed before adequate relocation measures had been agreed on.\footcite[42]{smith08} Hence, the question, as he saw it, did not concern the validity of using compulsory purchase powers, but merely the timing with which it had been ordered. On this basis, he framed the question of legitimacy as one relating to the ``necessity'' standard, according to which an infringement of Convention rights is only permissible when the public interest cannot be served in some other way.\footcite[43]{smith08} A strict reading of this standard holds that an interference must be the {\it least intrusive means} of achieving the stated aim.\footnote{Such a standard has been adopted in some Convention cases, for instance in \cite{samaroo01}.}

Justice Williams argued against such a strict reading, subscribing instead to a view expressed as an {\it obiter} in the case of {\it Pascoe v The First Secretary of State}. According to this view, an interference need not be the least intrusive means. Rather, it is sufficient that the measure is ``reasonably necessary'' to achieve that aim.\footnote{See \cite[74-75]{pascoe06} (quoting \cite[25]{clay04}).} However, while noting his agreement with this approach, Justice Williams went on to also apply the stronger necessity test, and found that even if this was applied the CPO in question would still be a proportional interference.\footcite[41-50]{smith08}

It seems clear that while the taking in question was for economic and recreational development purposes, the case was marked by a preliminary finding to the effect that the legitimacy of the aim of interference -- to facilitate the London Olympics -- was beyond reproach. Hence, there was no need for, or even room for, more detailed purposive reasoning of the kind that would later be applied by Lord Walker in {\it Sainsbury}. The fact that the taking was for economic development and recreation, not for a pressing public need, was not considered relevant. Moreover, since the case was construed to be solely about the extent to which the CPO was ``necessary'' to further its stated aim, the proportionality test that was carried out, despite being detailed, was very narrow in scope. It concerned only proportionality of the means, not of the aim itself. The question of how to weigh the public interest in a multi-billion dollar sporting event against the security of someone's home was not considered.

In later cases, a dismissive attitude towards substantive review has been adopted even in situations when the owners have argued against takings by explicitly questioning the proportionality of the interference against the importance of the aim. In the case of {\it Alliance Spring Co Ltd v The First Secretary of State}, a large number of properties were expropriated to build a new football stadium for the football club Arsenal.\footcite{alliance06} Some owners who stood to lose their business premises protested, pointing to the fact that the inspector in charge of the public inquiry had recommended against the takings.\footcite[6-7]{alliance06} As noted by Justice Collins, the main line of argument presented against the taking was that it did not serve a ``proper purpose''.\footcite[19]{alliance06} This argument was dismissed, however, with Justice Collins concluding as follows: 

\begin{quote}
There is nothing in the material put before and accepted by the Inspector which persuades me that that decision was ill founded or was one which the Secretary of State was not entitled to reach. Developments which result in regeneration of an area are often led by private enterprise. Mr Horton perforce accepts that that is so, but submits that this is not the sort of situation where, for example, a private development is the anchor for a particular scheme. I disagree.\footcite[19]{alliance06}
\end{quote}

Hence, unlike the case of {\it Smith}, where the Court did in fact carry out its own assessment of proportionality, the {\it Alliance} Court was content with deferring to the assessment carried out by the executive branch.\footnote{This has been criticized, e.g., by Kevin Grey who describes the reference to Convention Rights in Alliance as ``worryingly brief''. See \cite{gray11}.} As such, the case appears to follow the pattern of judicial review of CPOs established before the Human Rights Act 1998. This means that the decision also contrasts with how English courts have approach the Convention in relation to other  rights, such as those of Article 8 addressed in {\it Smith}.

Whether the approach taken in {\it Alliance} is good law after {\it Sainsbury} is unclear; from Lord Walker's opinion, it seems that a more substantive assessment can be demanded for similar cases in the future. While this might not imply a different outcome for a case like {\it Alliance}, it would mean that courts would have to engage in independent review of the purpose and merits of contested CPOs that benefit commercial actors. In particular, English courts would have to change the way they approach such cases, by being better prepared to assess for themselves whether a fair balance is struck between the interests of the developer and the property owners. Hence, it is not unlikely that the category of economic development takings will become an important point of reference in the future, both for the law and those who study it.

\noo{ \subsection{Germany}\label{sec:germany}

In German law we find an explicit constitutional property clause. In particular, Article 14 of the Basic Law ({\it Grundgesetz}) reads as follows:

\begin{quote}
(1) Property and the right of inheritance shall be guaranteed. Their content and limits shall be defined by the laws. \\
(2) Property entails obligations. Its use shall also serve the public good. \\
(3) Expropriation shall only be permissible for the public good. It may only be ordered by or pursuant to a law that determines the nature and extent of compensation. Such compensation shall be determined by establishing an equitable balance between the public interest and the interests of those affected. In case of dispute concerning the amount of compensation, recourse may be had to the ordinary courts.\footcite[14]{basic49}
\end{quote}

Apart from the fact that the property clause is explicit, I note two further characteristic features of the protection of property in Germany. First, the constitution explicitly stresses that property comes with social obligations as well as rights. The use of property should ``serve the public good''. On the other hand, it is also made clear that expropriation is only permissible when it is ``for the public good''. Hence, it follows immediately that the purpose of expropriation is a relevant factor when determining the legitimacy of a taking, \isr{irrespective} of the specific statute used to authorise it. Importantly, it is clear already from the outset that the question of legitimacy is a \emph{judicial} question, one which the courts can only answer if they form an opinion about that constitutes the ``public good''. 

This means that it is quite natural to approach the question of economic development takings from the point of view of constitutional law. Unlike in England, disputes over the legitimacy of such takings can be comfortably adjudicated directly against a ``public good'' restriction. While this sets Germany apart on the theoretical level, it is unclear how much of an effect it has had in practice. To shed some light on this question, we can look to the two major authorities on the legitimacy of economic development takings, the cases of {\it D\"{u}rkheimer Gondelbahn} and {\it Boxberg}.\footcite{durkheimer81,boxberg86} 

In both cases, the German Constitutional court found that expropriation to the benefit of commercial interests was illegitimate. However, the Court argued for this result on the basis that there was insufficient statutory authority for such takings in the concrete circumstances complained of. That is, the Court did not directly address the question of whether the relevant statutes were compliant with Article 14 of the basic law. Instead, they interpreted statutory authorities on the assumption that they had to be, following a pattern of reasoning that appears to be rather close to the approach followed by English courts in similar cases.\footnote{Although in {\it Dürkheimer Gondelbahn}, Böhmer J gave a separate concurring judgment where he argued for this result on the basis of the public good requirement of the basic law.} It seems, in particular, that even in Germany, the public purpose restriction is primarily relevant as a factor guiding the interpretation of statutory authorities.

That said, the cases of {\it D{\"u}rkheimer Gondelbahn} and {\it Boxberg} show that in situations when the public purpose of a taking is unclear, German courts seem inclined to \isr{favour} a narrow interpretation of the relevant statute. In {\it Bloxberg}, several properties were expropriated \isr{in favour} of the car company Daimler Benz AG, for commercial purposes. The affected local communities suffered from high unemployment rates and a slow economy, so a {\it prima facie} reasonable \isr{case} could be made that allowing Daimler to acquire the land was in the public interest, as it would facilitate economic growth. However, the Federal Constitutional Court agreed with the owners that the expropriation was invalid. This, it held, was because the taking was outside the scope of the relevant statute, which authorised expropriations for ``planning purposes''. The owners had argued extensively using Article 14 of the Basic Law and the constitutional ``public good'' restriction clearly did play a role in the Court's reasoning. But at the same time, the Court stressed that private-to-private transfers that bestow financial benefit on the acquiring party may well satisfy the ``public good'' requirement. The important issue was whether a sufficiently strong public interest could be identified, \isr{irrespective} of any windfall benefits that might fall on private parties.

In light of this, I think it is wrong to exaggerate the importance of the explicit formulation of the public use test offered in the German constitution. Its importance seems to rest mainly in the fact that it provides a particularly authoritative expression guiding the national courts' application of statutory provisions regarding expropriation of property. But developments in common law, where the public use requirement is stressed as a guiding constitutional principle, might well point in the same direction. In principle, both German and English Courts are in a good position to respond to increased tension regarding economic development takings by developing a stricter standard of judicial review in such cases.

A different aspect of German law deserves special attention, however, since it does not appear to have any clear counterpart in the common law tradition. This is the  ``social-obligation'' norm in Article 14 (2), which points to a different \isr{conceptualisation} of property rights as such. As argued by Alexander, the distinguishing feature of the property clause in the German Constitution is that the value of property is thought to relate more strongly to its importance for human dignity and flourishing in a social context, rather than the protection of individual financial entitlements. As Alexander notes regarding the Germans' own \isr{conceptualisation} of their property clause:

\begin{quote}
This theory holds that the core purpose of property is not wealth maximization or the satisfaction of individual preferences, as the American economic theory of property holds, but self-realization, or self-development, in an objective, distinctly moral and civic sense. That is, property is fundamental insofar as it is necessary for individuals to develop fully both
as moral agents and participating members of the broader community.\footcite[745]{alexander03}
\end{quote}

With such a starting point, it is not surprising that in cases such as {\it Boxberg}, resembling {\it Kelo}, German Courts will tend to adopt a strict view on legitimacy. These are cases when the property rights infringed on serve a fundamentally different function for the two opposing private parties. To the owner, the property is a home, an important source of self-identity, autonomy, security and membership in a community. To the taker, it represents an obstacle to commercial development which needs to be removed. In such a situation, it is in keeping with the spirit of the social-obligation norm of property to offer enhanced protection to the homeowner. To this owner, the property serves a purpose which is fundamentally different, and arguably more worthy of protection, then the property's purpose for the developer. A taking in this situation might therefore, because of Article 14, require a particularly clear and strong public interest.

But unless there is an asymmetry between owner and taker, heightened scrutiny does not necessarily follow. Hence, it is interesting to speculate what German courts would have made of a case such as {\it Regina (Sainsbury’s Supermarkets Ltd) v Wolverhampton City Council}. Here, the interests of owner and taker were strictly commercial nature. Both owned part of the contested land and neither one could develop the land according to their plans without buying out the other. The enhanced protection of property offered under German law would probably not have much significance in such a case. 

In fact, it might well be that German courts would be {\it more} likely to accept such a taking. First, their \isr{conceptualisation} of property rights appears to allow greater flexibility to adapt the level of protection to the circumstances and the purposes of the property in question. So even if is correct that private-to-private transfers for commercial projects require a ``stricter approach'' in general, as argued by Lord Walker in \textcite{sainsbury10}, the fact that the interests of the owner were also purely commercial  might make this less relevant. Second, German courts might be more inclined to have regard to socially beneficial additional commitments entered into by the applicant, even if they do not concern the property that is taken. As a tie-breaker, looking to such commitments might be as good an approach as any other.\footnote{This was the view taken by the dissenting minority in \textcite{sainsbury10}.}

Of course, objections could still be raised on the basis of general administrative law. Indeed, some might see the case as an example of government ``auctioning'' off licenses to the highest bidder. This might well be regarded as an affront to good governance. I will not delve into German law to assess the case from this perspective. My point is simply that because of the purposive and contextual nature of Article 14, it seems unlikely that a case like \textcite{sainsbury10} would turn on constitutional property law.

To sum up, German constitutional law serves to create an interesting contrast with English law regarding the question of economic development takings. On the one hand, property appears to be better protected against such takings in Germany, but on the other hand, the extent to which increased protection is offered depends more closely on the social values involved. The German system appears to look more actively at the social function of property for guidance when resolving property disputes, thereby echoing some of the ideas discussed in Chapter \ref{chap:1}. 

In the next section, I will discuss the property clause in the ECHR, which explicitly serves to set up a minimum level of property protection that provides a common standard for all member states, including Germany and the UK.
}
\section{The Property Clause in the European Convention of Human Rights}\label{sec:echr}

The starting point for property adjudication at the ECtHR is that States have a ``wide margin of appreciation'' with regard to the question of whether or not an interference in property rights is in the public interest.\footcite[See][54]{james86} This question is thought to depend on democratically determined policies to such an extent that it is rarely appropriate for the Court to censor the assessments made by member states. At the same time, the Court has gradually adopted a more active role in assessing whether or not particular instances of interference are proportional and able to strike a fair balance between the interests of the public and the property owners.\footnote{See \cite[69]{sporrong82} and \cite[120]{james86}. The standard account of the protection against interference inherent in P1(1) describes it as consisting of three rules. First, there is the rule of {\it legality}, asserting that an interference needs to be authorized by statute. Second, there is the rule of {\it legitimacy}, making clear that interference should only take place in pursuance of a legitimate public purpose. Both of these rules are of little practical significance, however, as the margin of appreciation has been regarded as very wide in regards to both. The third rule is the ``fair balance'' principle, which is applied by the ECtHR in almost all cases when it finds that there has been a violation of P1(1). In the following, I focus only on this rule and on those aspects of it that I think are most relevant to the question of economic development takings. For a more detailed description of P1(1) generally, I refer to \cite{allen05}.} As argued by Tom Allen, this has caused P1(1) to attain a wider scope than what was originally intended by the signatories.\footcite[1055]{allen10}.

In the case law behind this development, the focus has predominantly been on the issue of compensation, with the Court gradually developing the principle that while P1(1) does not entitle owners to full compensation in all cases of interference, the fair balance will likely be upset unless at least some compensation is paid, based on the market value of the property in question.\footnote{See \cite[103]{scordino06}. The case also illustrates that the Court has adopted a fairly strict approach to the question of when it is legitimate to award less than full market value.} The focus on compensation has also been reflected in academic work on P1(1), which tends to address proportionality from a financial perspective, by investigating to what extent owners are entitled to compensation based on the market value of their property. Indeed, when considering case law and literature on the subject, one is left with the impression that ``fair balance'' with regards to P1(1) is crucially linked to financial entitlements, primarily used as a standard that can justify a right to compensation that goes beyond what the wording of P1(1) might initially suggest.

In recent case law, however, it has become clear that the fair balance test encompasses more than this. In particular, it sometimes gives the Court in Strasbourg occasion to reflect on the social context and purpose of interference, in a manner largely consistent with the social function approach to property. 

In {\it Chassagnou and others v France} the situation was that landowners were compelled to permit hunting on their land, following compulsory membership in a hunting association which was set up to manage hunting in the local area.\footcite{chassagnou99} The owners protested this on the grounds that they were ethically opposed to hunting. The Court agreed that there had been a breach of P1(1). 

In the later case of {\it Hermann v Germany}, the circumstances were similar and the Court followed the precedent set in {\it Chassagnou}. In addition, the Court commented that they had ``misgivings of principle'' about the argument that financial compensation could provide adequate protection in such a case.\footcite[See][91]{hermann12}  In this way, the hunting cases illustrate that to the ECtHR, the right to property is more than a  financial entitlement. The fair balance that must be struck could pertain to other aspects, such as the owner's right to make use of his property in accordance with his convictions and to take part in decision-making processes regarding how it should be managed.\footnote{The assessment of proportionality should be concrete and contextual, and it is not based on a narrow or formalistic concept of property as dominion. This is demonstrated, for instance, by \cite{chabauty12}. Here the Court found no violation of P1(1) although the facts seemed close to those of {\it Chassagnou}. The case differed, however, in that the owner himself was not opposed to hunting, but wanted to withdraw his land from the hunters' association to enjoy exclusive hunting rights.}

In addition, the Court has adopted a similarly broad approach in recent cases involving rent control schemes and housing regulation. There are obvious financial interests at stake in such cases, for both landlords and tenants. However, the Court has addressed them by looking to the fairness of the underlying regulation more generally, by taking into account the social, economic and political context. Moreover, the Court has not shied away from using concrete cases as a starting point for providing an assessment of the sustainability of national law as such. 

In {\it Hutten-Czapska v Poland}, for instance, the Court concluded that the case demonstrated ``systemic violation of the right of property''.\footcite[239]{hutten06}
The case concerned a house that had been confiscated during the Second World War. After the war, the property was transferred back to the owners, but in the meantime, the ground floor had been assigned to an employee of the local city council. The state implemented strict housing regulations during this time, which eventually led to the applicant's house being placed under direct state management.\footcite[20-31]{hutten06} Following the end of communist rule in 1990, the owners were given back the right to manage their property, but it was still subject to strict regulation that protected the rights of the tenants.\footcite[31-53]{hutten06} In addition to rent control, rules were in place that made it hard to terminate the rental contracts. Hence, it became impossible for the owners to make use of the house themselves.\footcite[20-53]{hutten06} 

After an in-depth assessment of the relevant parts of Polish law and administrative practice, the Grand Chamber of the ECtHR concluded that there had been a violation of P1(1). Importantly, they did not reach this conclusion by focusing on the house as a source of financial entitlements for the owners. Rather, they focused on the overall character of the Polish system for rent control and housing regulation, as it manifested in the concrete circumstances of the applicant's case. The financial consequences for the owners were considered, as was the financial situation of the tenants.\footcite[60-61]{hutten06} The Court was particularly concerned with the fact that the total rent that could be charged for the house was not sufficient to cover the running maintenance costs.\footcite[224]{hutten06} In particular, it was noted that the consequence of this would be ``inevitable deterioration of the property for lack of adequate investment and modernisation''.\footnote{\cite[224]{hutten06}.}

In the end, the Court highlighted how three factors combined to leave owners in a precarious position. First, the rigid rent control system made it hard to sustainably manage rental property. Second, tenancy regulation made it hard for owners to terminate tenancy agreements. Third, the Court noted that the state itself had set up these tenancy agreements during the days of direct state management, shedding doubt on the legitimacy of the commitments that these contracts imposed on owners. In combination, these factors led the Court to conclude that  a fair balance had not been struck.\footcite[224-225]{hutten06} 

The contextual nature of the Court's reasoning in {\it Hutten-Czapska} is evidenced not only by the extent to which the concrete circumstances were assessed against the goal of fairness. It is also illustrated by how the Court explicitly places the ``social rights'' of the tenants on equal footing with the property rights of the owners.\footcite[225]{hutten06} The result, therefore, was not premised on a narrow understanding of property protection as an individual entitlement, but on a broader vision of property as a social institution.

It is also of interest to note how the Court concludes that the root of the problem is found in the Polish legal order as such. In this regard, great weight is placed on the observation that the regulatory system suffers from a lack of adequate safeguards to protect owners against imbalances such as those identified in {\it Hutten-Czapska}. In particular, the Court reflects on the position of owners and comments on ``the absence of any legal ways and means making it possible for them either to offset or mitigate the losses incurred in connection with the maintenance of property or to have the necessary repairs subsidised by the State in justified cases''. Hence, the rent control scheme alone was not the whole problem, the Court also criticised what it saw as a defective way of implementing it.\footcite[224]{hutten06} Moreover, the Court did not censor the political reasoning that motivated Polish housing legislation, but concluded instead that the ``burden cannot, as in the present case, be placed on one particular social group, however important the interests of the other group or the community as a whole''. 

I think this is the most important aspect of the case, pointing to the core function that the ECtHR should embrace more generally. It seems to me, in particular, that objections can be raised against the appropriateness of having the Court in Strasbourg assess concretely what is fair regarding the relationship between owners and tenants in a specific house in Gdynia. Its remoteness to the local conditions, as well as its lack of sensitivity and accountability to local democratic institutions suggests that the Court is not ideally placed to carry out the kind of contextual assessment that it prescribes for such cases. In addition, the amount of resources and time needed to independently scrutinize these aspects  risks undermining its ability to deal expediently with its case load. The ECtHR will hardly be able to protect human rights in Europe on a case-by-case basis.

Instead, the aim should always be to get at the systemic features that cause perceived imbalances. As in \textcite{hutten06}, the Court serves its function best when it is able to identify a sense in which the domestic legal order needs to be improved to better comply with human rights standards. This is particularly true when, as in that case, the Court notes that the applicants have insufficient options available for achieving a fair balance by appealing to institutions within the domestic legal order. By demanding {\it institutional} changes, the Court effectively delegates responsibility for ensuring the kind of fair balance that is required under the ECHR. Moreover, by scrutinizing the procedures and principles that the states apply when fulfilling this duty, it is likely that the Court will still be able to steer and unify the development of the case law. 

Importantly, they would then be able to do so without having to engage extensively in concrete assessments of fairness. Against this, one may argue that the judicial or administrative bodies of the signatory states can easily circumvent their obligations by giving a superficial or biased assessment of the facts in human rights cases, to avoid embarrassment for the state's political or bureaucratic elite. However, this might then be raised as a procedural complaint before the ECtHR, resulting in cases revolving around Articles 6 (fair trial) and 13 (effective remedy).\footnote{I note that this also fits with recent developments at the ECtHR, toward somewhat broader scrutiny under Article 6, see \cite{khamidov07}.}  In this way, the Court can streamline its functions, by always aiming to direct attention at issues that arise at a higher level of abstraction. This, in my view, is desirable. The ECtHR should not aim to micromanage the signatory states, particularly not in relation to a norm such a P1(1), which the Court itself regards as highly dependent on context.

However, the question arises as to what kind of institutions the Court should focus on in its effort to ensure fairness in relation to Convention rights such as property. It is not given, in particular, that directing attention towards domestic judicial bodies is the most appropriate approach. Rather, it is logical to assume that those institutions most in need of reform will be exactly those that are most often responsible for violations. A possible lack of an effective complaints procedure would be worrying, but not as problematic as systemic weaknesses of those institutions that act in ways that give rise to complaints in the first place. 

By shifting attention towards the institutional context of the primary decision-maker, the Court can also avoid getting stuck in deference to domestic judicial bodies. This can then be accomplished alongside a shift of attention away from concrete assessment of alleged violations. The Court can achieve this by concretely and critically assessing those rules and procedures that are identified as causally significant to individual complaints, at the administrative rather than the judicial level.\footnote{In the future, one might even encounter cases when the Court prefers to remain agnostic about whether a substantive violation occurred, focusing instead on the possible violation inherent in excessive systemic risks and a shortage of adequate safeguards.}

Indeed, the case of \textcite{hutten06} is suggestive of a move towards such a perspective. While the Court went into great detail about the facts of the case, it {\it also} looked at the case from an alternative perspective, more in line with the suggestion sketched above. In fact, I think it is likely that the Court will eventually veer even more towards such an approach, while deferring to national judicial bodies when it comes to concrete factual assessments. If not as a result of policy, I imagine this will happen from necessity, due to the limited capacity of the Court to hear the merits of individual cases.

The proportionality doctrine could still be applied, but approached in more abstract terms as the question of what kinds of rules, and what kinds of institutions, member states need to put in place to ensure fairness. In \textcite{hutten06}, the Court moved in this direction, especially when it explained the basic principle as follows:

\begin{quote}
In assessing compliance with Article 1 of Protocol No. 1, the Court must make an overall examination of the various interests in issue, bearing in mind that the Convention is intended to safeguard rights that are “practical and effective”. It must look behind appearances and investigate the realities of the situation complained of. In cases concerning the operation of wide-ranging housing legislation, that assessment may involve not only the conditions for reducing the rent received by individual landlords and the extent of the State’s interference with freedom of contract and contractual relations in the lease market, but also the existence of procedural and other safeguards ensuring that the operation of the system and its impact on a landlord’s property rights are neither arbitrary nor unforeseeable. Uncertainty – be it legislative, administrative or arising from practices applied by the authorities – is a factor to be taken into account in assessing the State’s conduct. Indeed, where an issue in the general interest is at stake, it is incumbent on the public authorities to act in good time, in an appropriate and consistent manner.\footcite[151]{hutten06} 
\end{quote}

I note how the Court builds on the earlier precedent set by cases such as \textcite{sporrong82} and \textcite{james86}. The first half of the quote, therefore, stresses that the Court itself must ``look to the realities of the situation''. However, in clarifying what is meant by this, the Court goes on to emphasise procedural aspects. In particular, it is made clear that the Court regards such aspects as an integral part of those ``realities'' that need to be assessed. Indeed, the Court even makes specific reference to the importance of several values that arise in the context of administrative law, such as predictability and effectiveness.

The passage above was subsequently quoted in {\it Lindheim and others v Norway}. In this case, the applicants complained that their rights had been violated by a recent Norwegian act that gave lessees the right to demand indefinite extensions of ground leases on pre-existing conditions.\footcite[119]{lindheim12} In the end, the Court concluded that there had indeed been a breach of P1(1). Interestingly, they engaged in the same form of assessment that they had adopted in \textcite{hutten06}. They held, in particular, that the Ground Lease Act itself was the underlying source of the violation. The problem was not merely that it had been applied in a way that offended the rights of the applicants. Hence, the Court did not only award compensation, it also ordered that general measures had to be taken by the Norwegian state to address the structural shortcomings that had been identified.

The Court also commented that its decision should be regarded in light of ``jurisprudential developments in the direction of a stronger protection under Article 1 of Protocol No. 1''.\footcite[135]{lindheim12} However, in light of the change in perspective that accompanies this development, it is interesting to ask in what sense the protection is stronger. In particular, it is not {\it prima facie} clear that the Court's remark should be read as a statement expressing a change in its understanding of the content of individual rights under P1(1). Rather, it may be read  as a statement to the effect that the Court now assumes it has greater authority to address structural problems under that provision. This authority, in particular, extends to the fair balance requirement, not only the (much more narrowly drawn) legality and legitimacy rules. In effect, this would allow the Court to conclude that a violation has occurred due to structural unfairness, even when it is not possible to trace this back to any flawed decision that specifically targets the applicants.

Is this relevant to the issue of economic development takings? I believe so. Indeed, I am struck by how the reasoning of the ECtHR in recent cases on hunting and rent control mirrors the kind of reasoning that Justice O'Connor engaged in when considering {\it Kelo}. The emphasis is on structural aspects and fairness, grounded on the facts of the concrete case and what they reveal about the rules and procedures involved. In this way, the contextual approach to property gains focus without losing its bite. The crux of arguments used to conclude violation is the observation that the established system can offend against the role that owners {\it should} occupy in order to be able to meet those obligations and exercise those freedoms that are attached to the property that they posses.

On this narrative, interference becomes illegitimate when it demonstrates a failure of governance. In the case of \textcite{hutten06}, this boiled down to the observation that it was illegitimate to address problems in the Polish housing sector by placing the burden ``on one particular social group'', namely the owners.\footcite[225]{hutten06} This conclusion was backed up by the concrete observation that the rules and procedures in place meant that owners who were obliged to maintain their properties in good condition for their tenants were in fact prevented from doing so because they were not permitted to charge rents that would cover the costs.

In the case of {\it Kelo}, Justice O'Connor argued in a similar fashion when she concluded that the system which had led to the decision to condemn Suzanne Kelo's house was likely to function so as to systematically ``transfer property from those with fewer resources to those with more''. To Justice O'Connor, there was little doubt that this could become a general pattern, if safeguards were not put in place. Indeed, it must be presumed that a multi-million dollar company is always in a better position than a homeowner when arguing that  ``economic development'' will result from their ownership. \noo{More subtly, her opinion also hinted at the inconsistency involved in asserting abstractly that economic development would benefit the community indirectly, all the while the development would \isr{in} fact require razing it.}

To conclude, I think the ECtHR would have been likely to approach a case like {\it Kelo} in a manner consistent with Justice O'Connor's approach. Whether they would reach the same result seems more uncertain, particularly since confidence in the nation states' ability and willingness to regulate private-public partnerships might be higher in Europe. However, it seems unlikely that the ECtHR would follow the majority in {\it Kelo}, by simply deferring to the determinations made by the granting authority. Moreover, with the recent change in perspective towards a more structural assessment of property institutions at the ECtHR, Justice O'Connor's predictions about the ``fallout'' of the {\it Kelo} decision would likely have been of great interest to justices at the Court in Strasbourg.

\section{The US Perspective on Economic Development Takings}\label{sec:us}

In this section, I consider US law in more depth. First, I track the development of the case law on the public use restriction in the Fifth Amendment and in various state constitutions. I consider the jurisprudential development from the early 19th Century up to the present day.\footnote{The public use clause in the US constitution was not held to apply to state takings until the late 19th Century, see \cite{chicago97}.} Many writers assert that the 19th and early 20th Century was \isr{characterised} by a ``narrow'' approach to public use which eventually gave way to a broader conception.\footnote{See, e.g., \cite[483]{walt11}; \cite[203-204]{allen00}. For a more in-depth argument asserting the same, see \cite{nichols40}.} Against this, I argue that it is more appropriate to think of this period as one when courts adopted a broad approach to {\it judicial scrutiny} of the takings purpose at state level. Importantly, I also argue that while different state courts expressed different theoretical views on the meaning of ``public use'', there was a growing consensus that the approach to judicial scrutiny should be contextual, focused on weighing the rationale of the taking against the concrete social, political and economic circumstances of the local area.\footnote{A summary of state case law that supports this view is given in the little discussed Supreme Court case of \cite{hairston08}.}  In particular, I argue that early state courts did not focus as much on the exact wording of the constitutional property clause as some later commentators have suggested.

I go on to show that the doctrine of deference that was developed by the Supreme Court early in the 20th Century was directed primarily at state courts, not state legislatures and administrative bodies.\footnote{See \cite{vester30} (echoing and citing \cite{hairston08}).} I then present the case of {\it Berman}, arguing that it was a significant departure from previous case law.\footcite{berman54} After {\it Berman}, deference was suddenly taken to mean deference to the (state) legislature, meaning that there would be little or no room for judicial review of the takings purpose. I go on to present the subsequent developments at state level, \isr{characterised} by increasing worry that the eminent domain power could be abused by powerful commercial actors. I discuss the case of {\it Poletown}, where a \isr{neighbourhood} of about 1000 homes was razed to provide General Motors with land to assemble a car factory.\footcite{poletown81} I link this to the subsequent controversy that arose over {\it Kelo}, suggesting that it should be seen as the eventual backlash of {\it Berman}, resulting from unease with the idea that the contextual approach to public use should be abandoned in favour of an almost absolute rule of deference.

After the historical overview, I go on to briefly present the vast amount of research that has targeted economic takings in the US after {\it Kelo}. I devote special attention to proposals for new legitimacy-enhancing institutions for facilitating economic development of jointly owned land. I focus on two suggestions in particular, targeting compensation and participation respectively.\footcite{lehavi07,heller08} These proposals will serve as important reference points later on, when I consider the Norwegian appraisal and land consolidation courts in Chapters 4 and 5.

\section{The History of the Public Use Restriction}\label{sec:hop}

Going back to the time when the Fifth Amendment was introduced, there is not much historical evidence explaining why the takings clause was included in the bill of rights. Moreover, there is little in the way of guidance as to how it was originally understood. James Madison, who drafted it, commented that his proposals for constitutional amendments were intended to be uncontroversial to Congress.\footnote{See letters from Madison to Edmund Randolph dated 15 June 1789 and from Madison to Thomas Jefferson dated 20 June 1789, both included in \cite{madison79}.} Hence, it is natural to regard the property clause as a codification of an existing principle, not a novel proposal. Indeed, several state constitutions pre-dating the Bill of Rights also included takings clauses, seemingly based on codifying principles from English Common law.\footcite[See][299]{johnson11}

As I discussed in subsection \ref{sec:england} above, English legal theory from this time tended to hold private property in high regard. With this background it is not surprising that Madison regarded the property clause as an uncontroversial amendment.\footnote{Indeed, early American scholars also \isr{emphasised} the importance of private property. For instance, in his famous {\it Commentaries}, James Kent described the sense of property as ``graciously implanted in the human breast'' and declared that the right of acquisition ``ought to be sacredly protected'', \cite[see][257]{kent27}.} Its importance may in fact have been greater as a legitimising force, increasing confidence in the regulatory power of the newly established state by setting up clear parameters for the exercise of that power.  However, while the principle of the Fifth Amendment might have been theoretically self-evident, it was never clear what it would mean in practice, particularly in cases when takings where challenged on the basis that they were not for a ``public use''.\footcite[See][317]{johnson11} 

There are two points that I would like to record about early US jurisprudence on this point. First, the distinction between public use and public purpose does not appear to have been considered sharp. In his {\it Commentaries}, James Kent first makes clear that the power of eminent domain is for ``public use, and public use only'', but then goes on to qualify this by stating that a taking which served a ``purpose not of a public nature'' would be unconstitutional.\footcite[See][275-276]{kent27}  He does not address this limitation in any detail, however, suggesting that it was not the subject of much debate at this time. To the founders, it seems that the right to compensation was considered more practically important, a sentiment that is also reflected in the {\it Commentaries}.\footnote{James Kent held it to be  ``founded in natural equity'' and described it as an ``acknowledged principle of universal law'', \cite[see][276]{kent27}.} The public use limitation was probably taken for granted as a matter of principle, while it had not yet proved problematic as a matter of practical adjudication. Moreover, it appears to have been accepted that takings which clearly benefited the public would be legitimate regardless of whether or not the property was physically put to use by the public.\footcite{johnson11}

An interesting early illustration of how courts approached takings controversies at this time can be found in {\it Stowell v Flagg}, a Massachusetts case from 1814. In this case, a landowner complained that his land had been flooded by a mill and sought a remedy in common law. The mill owner protested, however, since he was entitled to flood the land according to a special mill act, which allowed him to exercise the power of eminent domain to gain the right to flood his neighbour (provided statutory compensation was paid). The focus in the case was on whether a common law claim for damages could still be made, irrespective of the act's clear intention to deprive the affected neighbours of this opportunity. Hence, the court implicitly dealt with the legitimacy of the mill act itself, and they actively engaged with the public use requirement in the state constitution when making their assessment.\footcite{stowell14} In the end, they found that the act was legitimate, and they highlighted the purpose of the interference, commenting that ``these mills, early in the settlement of this country, were of great public necessity and utility''.\footcite[366]{stowell14} 

At the same time, however, the court had misgivings about how the act had come to be applied and expressed concern that ``the legislature, as well as the courts of law in this state, seem to have been disposed rather to enlarge, than to curtail, the power of mill owners''.\footcite[366]{stowell14} Still, after noting that affected land owners were entitled to compensation under the act, the court concluded that the act had to be observed and that it precluded any claims for damages under common law. Hence, the case is an early example of judicial deference to the legislature in takings cases. More importantly, however, it also illustrates that the public use requirement was beginning to emerge as a potentially problematic issue in its own right. The presiding judge stated that he could not help thinking that the statute was ``incautiously copied from the ancient colonial and provincial acts''. It was not without reservation, therefore, that he held in favour of the mill owner, concluding that ``as the law is, so must we declare it''.\footcite[368]{stowell14}

While judicial deference was recognised as a guiding principle early on in US takings law, it is important to note in this regard that eminent domain was seldom used in a way that would raise serious controversy. English legal practices at this time ensured that the takings power would typically only be used as a last resort. As Meidinger notes, the British were never really charged with abuse of eminent domain, and private property tended to be respected, also in the colonies.\footcite[17]{meidinger80} This undoubtedly influenced early US law. Indeed, the cautious approach to takings was expressed by the Supreme Court early on, as an example of a fundamental legal principle.\footnote{As reflected in {\it de dicta} comments from {\it Calder v Bull} and {\it Vanhorne’s Lessee v Dorrance}, see \cite[388]{calder98}; \cite[310]{vanhorne95}.} Hence, the relative lack of judicial interest in the question of legitimacy does not appear to have been due to a broad view on the scope of eminent domain, but an established practice of narrow use of that power, inherited from the English.

%The Legislature declare and enact, that such are the public exigencies, or necessities of the State, as to authorise them to take the land of A. and give it to B.; the dictates of reason and the eternal principles of justice, as well as the sacred principles of the social contract, and the Constitution, direct, and they accordingly declare and ordain, that A. shall receive compensation for the land. But here the Legislature must stop; they have run the full length of their authority, and can go no further: they cannot constitutionally determine upon the amount of the compensation, or value of the land. Public exigencies do not require, necessity does not demand, that the Legislature should, of themselves, without the participation of the proprietor, or intervention of a jury, assess the value of the thing, or ascertain the amount of the compensation to be paid for it. This can constitutionally be effected only in three ways.
%1. By the parties that is, by stipulation between the Legislature and proprietor of the land.
%2. By commissioners mutually elected by the parties.
%3. By the intervention of a Jury.

The traditional attitude to eminent domain would eventually give way to a more expansive approach, however. This development became particularly marked during the period of great economic expansion and industrialisation in the mid to late 19th century, when eminent domain was increasingly used to benefit (privately operated) railroads, hydroelectric projects, and the mining industry.\footcite[23-33]{meidinger80} During this time, it also became increasingly common for landowners to challenge the legitimacy of takings in court, undoubtedly a consequence of the fact that eminent domain was now used more widely, for new kinds of projects.\footcite[24]{meidinger80} Controversy arose particularly often with respect to mill acts.\footnote{\cite[24]{meidinger80}. See also \cite[306-313]{johnson11} and \cite[251-252]{horwitz73}.} Such acts were found throughout the US, many of them dating  from pre-industrial times when mills were primarily used to serve the farming needs of  agrarian communities.\footnote{A total of 29 states had passed mill acts, with 27 still in force, when a list of such acts was compiled in \cite[17]{head85}. According to Justice Gray, at pages 18-19 in the same, the ``principal objects'' for early mill acts had been grist mills typically serving local agrarian needs at tolls fixed by law, a purpose which was generally accepted to ensure that they were for public use.}  However, following economic and technological advances, acts that were once used to facilitate the construction of grist mills would increasingly also be relied on by developers wishing to harness hydropower for manufacturing, and eventually, for hydroelectric projects.\footnote{See, e.g., \cite[18-21]{head85} and \cite[449-452]{minn06}.}

Many legitimacy cases pertaining to mill acts came before state courts in the late 19th and early 20th century. In the next subsection I present some of these cases, to shed light on how states courts developed their own approach to the question of legitimacy of takings.

\subsection{Legitimacy in State Courts}\label{subsec:state}

In the mill cases, we find the first clear evidence of how the public use requirement was applied to enable state courts to scrutinize the legitimacy of takings. Generally speaking, when a state court upheld a mill act interference, it would typically emphasise the broader purpose, often focusing on economic ripple effects.\footnote{See, e.g., \cite{hazen53,scudder32,boston32}. A more comprehensive list of cases adopting a broad view can be found in \cite[617]{nichols40}.} By contrast, when a court decided that an interference was unconstitutional (with respect to the relevant state constitution), it would often focus on the concrete use made of the mill, pointing out that it did not directly benefit the public in the sense required by the public use restriction.\footnote{See, e.g., \cite{sadler59,ryerson77,gaylord03,minn06}. A more comprehensive list can be found in {\it Public benefit or convenience as distinguished from use by the public as ground for the exercise of the power of eminent domain} 54 ALR 7 (American Law Reports, 1928).} For a time, a doctrine which sought to distinguish between takings for public use and takings for a public purpose, played a significant role in many states. Under this doctrine, only those takings that were deemed to qualify as public use takings under a narrow view of that term would be upheld.\footnote{Professor Nichols goes as far as to conclude that this emerged as the ``majority'' opinion on public use, see \footcite[617-618]{nichols40}. But contrast this with \cite{berger78} and \cite[24]{meidinger80}, who argue that the narrow view was only dominant in a handful of states, led by New York.}

%For instance, in the case of {\it Gaylord v. Sanitary Dist. of Chicago}, the Supreme Court of Illinois held the state Mill Act to be unconstitutional, as it was not limited to traditional flour mills. In doing so, the court observed that public use was ``something more than a mere benefit to the public''.\footcite[524]{gaylord03} Similar sentiments were expressed in other decisions striking down uses of eminent domain for mill construction, for instance in Vermont, Michigan and New York.\footnote{References.}

It is tempting to associate the narrow view on public use with a more restrictive attitude towards the use of eminent domain. Similarly, it is natural to assume that a broad view on public use suggests a more relaxed attitude. To some extent, the primary sources warrant this. Unsurprisingly, those who endorsed a broad view on the public use question also often spoke in favour of judicial deference in legitimacy cases, while those endorsing a narrow view tended to \isr{emphasise} the importance of constitutional safeguards against abuse of eminent domain. However, it seems that both groups were quite heterogeneous and that differences of opinion about the public use requirement did not necessarily reflect any deep ideological divisions.

It is clear, for instance, that many of the courts which favoured a broad interpretation of public use still viewed the constitutional limitation on the takings power as an important safeguard, not only as a guarantee for compensation but also as a restriction on the purpose of takings. Indeed, it seems that most late 19th Century Courts, including those that upheld economic takings, were influenced by the growing body of case law across the US that actively scrutinized takings, sometimes striking them down. In particular, it seems that the strict deferential view was largely abandoned in economic takings cases during this period. Deference to the legislature still played an important role and was typically called on as an important argument in takings cases. However, it became much more common to discuss legitimacy also in terms of substantive arguments, by directly addressing the context and circumstances of the taking complained of. I believe this is an important insight to record about the case law from this period. Despite differences of opinion about the meaning of public use, a consensus appears to have emerged that judicial review of legitimacy was appropriate and important in economic takings cases.\footnote{A similar point is made by Merrill, see \cite{merril86}.}

A good example is the case of {\it Dayton Gold \& Silver Mining Co. v. Seawell}, concerning a Nevada Act which stipulated that mining was a public use for which the power of eminent domain could be exercised to acquire additional rights needed to facilitate extraction.\footcite{seawell76} The Supreme Court of Nevada decided that the Act was constitutional and adopted a broad understanding of the property clause in the Nevada constitution.\footnote{Nev Const Art 8 § 1.} Interestingly, it argued for this interpretation partly on the basis that it would provide {\it better} protection for landowners:

\begin{quote}
If public occupation and enjoyment of the object for which land is to be condemned furnishes the only and true test for the right of eminent domain, then the legislature would certainly have the constitutional authority to condemn the lands of any private citizen for the purpose of building hotels and theaters. [...] Stage coaches and city hacks would also be proper objects for the legislature to make provision for, for these vehicles can, at any time, be used by the public upon paying a stipulated compensation. It is certain that this view, if literally carried out to the utmost extent, would lead to very absurd results, if it did not entirely destroy the security of the private rights of individuals. Now while it may be admitted that hotels, theaters, stage coaches, and city hacks, are a benefit to the public, it does not, by any means, necessarily follow that the right of eminent domain can be exercised in their favor.\footcite[410-411]{seawell76}
\end{quote}

The quote shows that a broad understanding of ``public use'' need not be synonymous with a less cautious attitude to abuse of the takings power. Indeed, while the Court decided to uphold the Act, it did so only after a careful assessment of legal arguments and factual circumstances. In particular, the Court considered the importance of mining, concluding that it was the ``greatest of the industrial pursuits'' in the state, and that all other interests were ``subservient'' to it.\footcite[409]{seawell76} Moreover, the Court commented that the benefits of the mining industry was ``distributed as much, and sometimes more, among the laboring classes than with the owners of the mines and mills''.\footcite[409]{seawell76}

This shows that the Court actively engaged with the purpose of the Act, thoughtfully assessing it against the constitution. Importantly, it did not do so in isolation, as a linguistic exercise or by attempting to recreate its ``original intent''. Rather, the court approached the constitutional safeguard by making detailed references to the prevailing social and economic conditions in the state of Nevada. The Court noted the importance of deference to the legislature on matters of policy, but it did so only after it had satisfied itself that the Act could be ``enforced by the courts so as to prevent its being used as an instrument of oppression to any one''.\footcite[412]{seawell76} More generally, the court commented as follows on the public purpose test that had to be performed in takings cases, elucidating on the principles on which it should be founded:

\begin{quote}
 Each case when presented must stand or fall upon its own merits, or want of merits. But the danger of an improper invasion of private rights is not, in my judgment, as great by following the construction we have given to the constitution as by a strict adherence to the principles contended for by respondent.\footcite[398]{seawell76}
\end{quote}

In light of this, {\it Dayton Gold \& Silver Mining Co. v. Seawell} must be regarded as an early example of a contextual approach to legitimacy. A formalistic approach based on the phrase ``public use'' was abandoned, but not in \isr{favour} of general deference. Rather, a more nuanced view was adopted, to respect the idea that the legislature should have the final say on policy while also recognising that courts should play a crucial role in protecting citizens from abuse of the takings power. 

The case is not unique, but rather exemplifies the type of reasoning that was used to assess economic development takings cases at this time. Interestingly, many common elements exist between courts that upheld and struck down such takings, \isr{irrespective} of whether or not they subscribed to a narrow or broad view on the public use test. One example is {\it Ryerson v. Brown}, a case often cited as an authority for a narrow view of public use.\footcite{ryerson77} Here the Supreme Court of Michigan explicitly qualifies its decision by stating that it is ``not disposed to say that incidental benefit to the public could not under any circumstances justify an exercise of the right of eminent domain''.

The case concerned the constitutionality of a mill act, and while the court argues that public use should be taken to mean ``use in fact'', it is clear that ``use'' is understood rather loosely, not literally as physical use of the property that is taken.\footnote{The court explains its stance on the public use restriction by stating (emphasis added) ``it would be essential that the statute should require the use to be public in fact; in other words, that it should contain provisions entitling the public to {\it accommodations}.'' The court continues with an illustrative example: ``A flouring mill in this state may grind exclusively the wheat of Wisconsin, and sell the product exclusively in Europe; and it is manifest that in such a case the proprietor can have no valid claim to the interposition of the law to compel his \isr{neighbour} to sell a business site to him, any more than could the manufacturer of shoes or the retailer of groceries. Indeed the two last named would have far higher claims, for they would subserve actual needs, while the former would at most only incidentally benefit the locality by furnishing employment and adding to the local trade''. See \cite[336]{ryerson77}.} Moreover, when clarifying its starting point for judicial scrutiny of mill acts, the court explains that ``in considering whether any public policy is to be subserved by such statutes, it is important to consider the subject from the standpoint of each of the parties''. Following up on this, the court found, with respect to the Act in question, that `` the power to make compulsory appropriation, if admitted, might be exercised under circumstances when the general voice of the people immediately concerned would condemn it''. After considering this and other possible consequences, the court eventually declared the Act to be unconstitutional, summing up its assessment as follows: ``What seems conclusive to our minds is the fact that the questions involved are questions not of necessity, but of profit and relative convenience''.\footcite[336]{ryerson77}

Hence, far from nitpicking on the basis of the public use phrase, the court adopts a contextual approach to takings that is rather similar to the approach of {\it Dayton Gold \& Silver Mining Co. v. Seawell}. The outcome \isr{is} different, but it is also based on a different assessment of the context and the consequences of the takings complained about. Importantly, the case does not rest on any {\it a priori} assumption that economic development takings of the kind in question could not meet a public use test -- no general rule is relied on at all. Hence, it is somewhat strange that later commentators have focused on the case for its comments on public use rather than its broad, but restrictive, assessment of legitimacy.

Many of the important cases from the late 19th Century, on both sides of the public use debate, shares many crucial features with the two cases discussed above.\footnote{See, e.g., \cite{scudder32} (Eminent domain power upheld, but said: ``The great principle remains that there must be a public use or benefit. That is indispensable. But what that shall consist of, or how extensive it shall be to authorize an appropriation of private property, is not easily reducible to a general rule. What may be considered a public use may depend somewhat on the situation and wants of the community for the time being.''), \cite{fallsburg03} (Eminent domain struck down, on holding that ``the private benefit too clearly dominates the public interest to find constitutional authority for the exercise of the power of eminent domain''), \cite[538]{board91} (Eminent domain struck down, qualified by ``not only must the purpose be one in which the public has an interest, but the state must have a voice in the manner in which the public may avail itself of that use'').} This points to a unifying perspective on legitimacy adjudication from this time. Some commentators describe the case law as chaotic, characterised by different ideas competing for dominance.\footcite{berger78,meidinger80}. However, it might be more accurate to say that a broad consensus developed during in this period, regarding the need for special judicial scrutiny of economic development cases. Moreover, state courts were clearly conscious of the special challenges that arose at a time when eminent domain was being used to benefit specific commercial actors. Differences of opinion about public use terminology was an important part of this, but it was rarely considered in isolation from other aspects. On a deeper level, the fact that the public use debate was regarded as important in the first place clearly suggests that deference to the legislature was not held to be an exhaustive answer to the question of legitimacy. This,  is an important observation which is also made in the work of Merrill, but which appears to have been somewhat overlooked in the literature following {\it Kelo}.

I believe it is a relevant observation not only in relation to state law, but also when considering the takings doctrine that has later developed at the federal level. While the narrow view of public use was indeed losing ground at the beginning of the 20th century, the doctrine of extreme deference that was about to be adopted by the Supreme Court represented a new development. 

Importantly, the doctrine of deference was not originally directed primarily at the legislature, but rather towards the judiciary at the state level. Moreover, the  balance of power between states and the federal government played an important role in early federal takings jurisprudence, as discussed in the next subsection.

\subsection{Legitimacy as Discussed in the Supreme Court}\label{subsec:US}

Initially, the Supreme Court held that the takings clause in the US Constitution did not apply to state takings at all.\footcite{barron33} Federal takings, on the other hand, were of limited practical significance since the common practice was that the federal government would rely on the states to condemn property on their behalf.\footcite[30]{meidinger80}. This changed towards the end of the 19th Century, particularly following the decision in {\it Trombley v. Humphrey}, where the Supreme Court of Michigan struck down a taking that would benefit the federal government.\cite{trombley71} Not long after, in 1875, the first Supreme Court adjudication of a federal taking occurred, marking the start of the development of the Supreme Court's own doctrine on public use and legitimacy.\footcite{kohl75} Eventually, in 1897, the Court would also hold that state takings could be scrutinized under the takings clause of the constitution.\footcite{chicago97} 

This development can be traced to the passage of the Fourteenth Amendment after the civil war, concerning due process.\footcite{johnson11}. Indeed, some early Supreme Court cases dealing with state takings were adjudicated against the due process clause directly.\footnote{See, e.g., \cite{head85}.}

After the Supreme Court started developing its own case law on the legitimacy issue, the deferential stance soon became entrenched. As argued by Horwitz, the mid- to late 19th Century was the period in US history when control over property was transferred on a massive scale from agrarian communities to various agents of industrial expansion.\footcite{horwitz73} Moreover, it was a period of great optimism about the ability of {\it laissez faire} capitalism to ensure progress and economic growth. 

This was reflected in the case law on eminent domain, particularly as developed by the Supreme Court. A particularly clear expression of this can be found in {\it Mt. Vernon-Woodberry Cotton Duck Co v Alabama Interstate Power Co}.\footcite{vernon16}  This case dealt with the legitimacy of condemnation arising from the construction of a hydropower plant. The Supreme Court held that it was legitimate, with the presiding judge arguing briskly as follows:

\begin{quote}The principal argument presented that is open here, is that the purpose of the condemnation is not a public one. The purpose of the Power Company's incorporation, and that for which it seeks to condemn property of the plaintiff in error, is to manufacture, supply, and sell to the public, power produced by water as a motive force. In the organic relations of modern society it may sometimes be hard to draw the line that is supposed to limit the authority of the legislature to exercise or delegate the power of eminent domain. But to gather the streams from waste and to draw from them energy, labor without brains, and so to save mankind from toil that it can be spared, is to supply what, next to intellect, is the very foundation of all our achievements and all our welfare. If that purpose is not public, we should be at a loss to say what is. The inadequacy of use by the general public as a universal test is established. The respect due to the judgment of the state would have great weight if there were a doubt. But there is none.\footcite[]{vernon16}
\end{quote}

The quote serves as an indication of how deference was fast gaining ground, without yet being established doctrine. On the one hand, the Court notes the importance of deference to the {\it state} judgement (not specifically the judgement of the state legislature). On the other hand, it prefers to conclude on the basis of its own assessment of the purpose of the taking. This assessment, however, is not grounded in the facts of the case or the circumstances in Alabama. Rather, it is based on sweeping assertions about ``all our welfare'' and the desire to ``save mankind from toil that it can be spared''. 

This judgement, from 1916, was given during the so-called {\it Lochner} era of jurisprudence in the US. During this time, the Supreme Court would famously engage in active censorship of regulation that was meant to promote greater social and economic equality.\footcite{cohen08} In particular, much case law from this period witnesses to a general lack of deference towards political decision-makers. Hence, it is surprising to find that deference actually played an increasingly important role in takings cases.\footnote{The {\it Lochner} era in general was \isr{characterised} by courts engaging in censorship of state regulation, but this general tendency is not well reflected in how eminent domain law developed over the same period. This is interesting, as it points to the shortcoming of another commonly held view on property protection, namely that it largely serves the interests of property-owning elites, to the detriment of regulatory efforts to promote social equality. The cases through which {\it Lochner} era courts developed the deferential stance suggest a different interpretation; those who benefited most directly from takings in these cases were commercial interests, not vulnerable groups of society. Moreover, they benefited from acquiring land rights from members of agrarian communities, not from the elites. Hence allowing such takings to go ahead was no affront to the ideology of progress through {\it laissez faire} capitalism, quite the contrary. In particular, if it is true as many have argued, that the {\it Lochner} courts were ideologically committed to the promotion of unrestrained capitalism, there was little reason for them to oppose expansion of eminent domain into the commercial arena: those who would be likely to benefit were market actors who were proposing large scale commercial development projects. Indeed, the case law from this period makes it natural to argue that the deferential stance developed primarily to cater to the needs of the capitalists, under the perceived view that they represented the class which would bring progress and prosperity to the nation as a whole.} As early as { \it United States v. Gettysburg Electric Railway Co.}, a case from 1896, deference was described as a fundamental guiding principle, which should be adhered to except in very special circumstances.\footcite{gettysburg96} In particular, Justice Peckham relied on a deferential stance, expressed as follows:

\begin{quote}
It is stated in the second volume of Judge Dillon's work on Municipal Corporations (4th Ed. § 600) that, when the legislature has declared the use or purpose to be a public one, its judgment will be respected by the courts, unless the use be palpably without reasonable foundation. Many authorities are cited in the note, and, indeed, the rule commends itself as a rational and proper one.\footcite[680]{gettysburg96}
\end{quote}

The case did not turn on the public use issue, however, as the condemned land would be used for battlefield memorials at Gettysburg, Pennsylvania, clearly a public use. In addition, the case concerned a federal taking authorized by Congress. Hence, its weight as a precedent should have been rather limited. 

Indeed, the deferential stance was not adopted in later federal cases originating from the states. As late as in 1930, in {\it Cincinatti v Vester}, the Supreme Court commented that ``it is well established that, in considering the application of the Fourteenth Amendment to cases of expropriation of private property, the question what is a public use is a judicial one''.\footcite[447]{vester30} In this judgement, Chief Justice Hughes also describes how the judicial assessment of the public use question should be carried out:

\begin{quote}
In deciding such a question, the Court has appropriate regard to the diversity of local conditions and considers with great respect legislative declarations and in particular the judgments of state courts as to the uses considered to be public in the light of local exigencies. But the question remains a judicial one which this Court must decide in performing its duty of enforcing the provisions of the Federal Constitution.\footcite[447]{vester30}
\end{quote}

Notice how this echoes the contextual approach developed at the state level, while explicitly prescribing particular deference to state {\it courts}. In {\it Hairston v. Danville \& W. R. Co.}, the same idea was expressed even more clearly by Justice Moody, who surveyed the state case law and declared that ``the one and only principle in which all courts seem to agree is that the nature of the uses, whether public or private, is ultimately a judicial question.''\footcite[606]{hairston08} Justice Moody continued by describing in more depth the typical approach of the state courts in determining public use cases:

\begin{quote}
The determination of this question by the courts has been influenced in the different states by considerations touching the resources, the capacity of the soil, the relative importance of industries to the general public welfare, and the long-established methods and habits of the people. In all these respects conditions vary so much in the states and territories of the Union that different results might well be expected.\footcite[606]{hairston08}
\end{quote}

Justice Moody goes on to give a long list of cases illustrating this aspect of state case law, showing how assessments of the public use issue is inherently contextual.\footcite[607]{hairston08} He then cites three further Supreme Court cases, pointing out that all of them express similar sentiments of support for state case law on this issue.\footnote{{\it Falbrook, Clark} and {\it Strickley}.} Following up on this, he points out that ``no case is recalled'' in which the Supreme Court overturned ``a taking upheld by the state {\it court} as a taking for public uses in conformity with its laws'' (my emphasis). After making clear that situations might still arise where the Supreme Court would not follow state courts on the public use issue, Justice Moody goes on to conclude that the cases cited ``show how greatly we have deferred to the opinions of the state courts on this subject, which so closely concerns the welfare of their people''.\footcite[606]{hairston08}

I believe {\it Hairston} is a crucially important case for two reasons. First, it makes clear that initially, the deferential stance in cases dealing with state takings was primarily directed at state courts rather than legislatures and administrative bodies. Second, it demonstrates federal recognition of the fact that a consensus had emerged in the states, whereby scrutiny of the public use determination was consistently regarded as a judicial task.\footnote{Indeed, {\it Hariston} provides the authority for {\it Vester} on this point. See \cite[606]{vester30}.} Moreover, the Court clearly looked favourably on the contextual approach whereby state courts would look to the concrete circumstances of the individual takings complained of. The Court's approval of this tradition is explicitly given as the reason for adopting a deferential stance. Put simply, the judicial test provided at state level was held to be of such high quality that there was little use for further scrutiny; a deferential stance was assumed, but made contingent on the fact that state courts would provide the required judicial scrutiny.

Despite this, {\it Hairston} would later be cited as an early authority in favour of almost unconditional deference in {\it US ex rel Tenn Valley Authority v Welch}.\footcite[552]{welch46} This case concerned a federal taking and it cited {\it US v Gettysburg Electric R Co} as an authority in favour of strong deference with regards to the public use limitation.\footcite{gettysburg96} However, the Court also paused to note that the later case of {\it City of Cincinnati v Vester} expressed the opposite view, namely that the public use test was a judicial responsibility.\footcite{vester30} In a very selective citation, the Court then purports to resolve this tension by quoting {\it Hairston} and the observation made there that the Supreme Court had never overruled the state courts in takings cases. Effectively, the importance of judicial scrutiny is thereby downplayed, although as we saw, the rationale behind {\it Hairston} was that state courts already offered high-quality judicial scrutiny of the public purpose. 

{\it Welch} is particularly important because it is used as an authority in the later case of {\it Berman v Parker}, which endorses almost complete deference to the legislature regarding the public use issue.\footcite[32]{berman54} This case concerned condemnation for redevelopment of a partly blighted residential area in the District of Colombia, which would also condemn a non-blighted department store. In a key passage, the Court states that the role of the judiciary in scrutinizing the public purpose of a taking is ``extremely narrow''.\footcite[32]{berman54} The Court provides only two citations for this claim, one of them being {\it Welch}. The other case, {\it Old Dominion Land Co v US}, concerned a federal taking of land on which the military had already invested large sums in buildings.\footnote{The Court commented on the public use test by saying that ``there is nothing shown in the intentions or transactions of subordinates that is sufficient to overcome the declaration by Congress of what it had in mind. Its decision is entitled to deference until it is shown to involve an impossibility. But the military purposes mentioned at least may have been entertained and they clearly were for a public use''. See \cite[66]{dominion25} Hence, the Court took the view that courts should be cautious in second-guessing the intentions of Congress on the basis of what its subordinates had subsequently done and said. This is far from a general deferential stance on public use. Moreover, no cases are cited at all on this point, suggesting further that the Court did not think its remarks would be of general significance. Still, a partial quote, used to substantiate broad deference to the legislature except when it involves an ``impossibility'', has become commonplace. In particular, such a quote was used in the much discussed \cite[240]{midkiff84}.}
In my view, both cases are weak authorities for prescribing general deference regarding public use. Moreover, both cases are concerned with federal takings only, while in {\it Berman} the Court explicitly says that deference is due in equal measure to the state legislature.\footcite[32]{berman54} 

It is possible to regard this merely as a {\it dictum}, since the District of Columbia is governed directly by Congress. At the same time, this passage has had a great impact on future cases. In effect, {\it Berman} caused a departure from a significant and consistent body of case law on judicial scrutiny at state level without engaging with it at all.

In {\it Hawaii Housing Authority v Midkiff}, the Supreme Court further entrenched the principle expressed in {\it Berman}, in a case where the state of Hawaii had made used of the takings power to break up an oligopoly in the housing sector.\footcite{midkiff84} However, Justice Sandra Day O'Connor, joined by a unanimous Supreme Court, also expressed general disapproval of private takings. In particular, Justice O' Connor appears to have felt the need to provide further qualification for the deferential view, which she did in part by observing that ``judicial deference is required because, in our system of government, legislatures are better able to assess what public purposes should be advanced by an exercise of eminent domain''. Hence, judicial deference was not regarded as an absolute and systemic imperative, as in Berman, but made contingent on the fact that legislatures are ``better able'' than courts at conducting public purpose tests. Hence, some of the contextual ideas from earlier case law is echoed in the decision. Importantly, the attention is not directed at the state legislature rather than the courts. It should be noted that {\it Midkiff} follows {\it Berman} also in the authorities consulted. The case does not consider precedents for the importance of judicial scrutiny at state level.

The purpose of interference in {\it Midkiff} was to break up an oligopoly to the benefit of tenants, not to further economic development by allowing commercial interests to take land. Hence, the rationale behind the interference is likely to have struck the Supreme Court as sound and just. Moreover, it seems that such an interference would be easy to uphold also under the doctrine of contextual judicial scrutiny of  public use. Indeed, Justice O'Connor partly relies on an assessment of the merits of the taking, when she points out that  ``regulating oligopoly and the evils associated with it is a classic exercise of a State's police powers''. In conclusion, the ``extremely narrow'' room for judicial review set up by {\it Berman} seems to have been replaced by a slightly more nuanced formulation, which nevertheless made clear that a legal precedent of deference had now become entrenched. 

Indeed, {\it Midkiff} reaffirmed the main new principle, namely that the meaning of public use is a matter for legislatures and that the room for judicial review is narrow.
So far, I have only commented on how the Supreme Court developed its own doctrine on the public use restriction in the early 20th Century. What was the effect of this doctrine at the state level?

As noted by Merrill, it seems clear that {\it Berman} had a significant effect, causing a shift of outlook that would eventually result in the public use clause being seen as a ``dead letter''.\footcite{merrill86}. At the same time, however, eminent domain also became more controversial, as it was put to use more aggressively by some states.

While the takings power had traditionally been used mostly to condemn agrarian land rights, it was now regularly used to condemn middle class homes. The controversy surrounding the case of {\it Poletown Neighborhood Council v City of Detroit}  illustrates this.\footcite[See][380-381]{sandefur05} In {\it Poletown}, the Michigan Supreme Court held that it was not in violation of the public use requirement to allow General Motors to displace some 3500 people for the construction of a car assembly factory. The majority 5-2 cites {\it Berman}, commenting that its own room for review of the public use requirement is limited.\footcite[632-633]{poletown81}

The {\it Poletown} decision was controversial, and the minority, especially Justice Ryan, was highly critical of it. He objects both to the deferential stance in general and to the majority reading of {\it Berman} in particular, pointing out that the Supreme Court's doctrine of deference was in large part directed at the state courts.\footcite[668]{poletown81} Hence, he concludes, the majority's reliance on {\it Berman} is ``particularly disingenuous''.\footcite[668]{poletown81} 

Justice Ryan was not alone in his disapproval of {\it Poletown} and the case is widely regarded as the prelude to an era of increased tensions over economic development takings in the US. This would culminate with {\it Kelo} which, despite upholding an economic development taking, also signalled a move towards more active judicial review of the public use requirement. This effect of {\it Kelo} has become clearer over time, primarily due to state legislative responses caused by disapproval of the outcome. However, it has also been remarked that both the majority and minority opinions in {\it Kelo} indicate that the Supreme Court itself may not be entirely at ease with the doctrine of strict deference that developed after {\it Berman}. In the next subsection, I will give an overview of recent developments, particularly from the secondary literature.

\section{Economic Development Takings after {\it Kelo}}\label{sec:postkelo}

The fact that {\it Kelo} was decided against the homeowner met with wide disapproval among the public. In addition, many scholars expressed concern at what they saw as an ill advised ``abdication'' of the judiciary in takings cases.\footnote{???} The minority opinions given in {\it Kelo}, particularly the opinion of Justice O'Connor, also proved influential, causing further attention to be directed at the perceived dangers of eminent domain abuse. A massive amount of literature has since appeared devoted to studying  economic development takings. 

Moreover, many states have introduced reforms aimed at curbing the use of eminent domain for economic development.\footnote{For an overview and critical examination of the myriad of state reforms that have followed {\it Kelo}, I point to \cite{eagle08}. See also \cite{somin09}.}

As of 2014, 44 states have passed post-{\it Kelo} legislation to curb the use of eminent domain for economic development.\footnote{According to the Castle Coalition, a property activist project associated with the Institute of Justice. See \url{http://www.castlecoalition.org/} for an up-to-date survey of state legislation on eminent domain.} Various legislative techniques have been adopted by the states to achieve this. Some states, including Alabama, Colorado, Michigan, enacted explicit bans on economic development takings and takings that would benefit private parties.\footcite[See][107-108]{eagle08} In South Dakota, the legislature went even further, banning the use of eminent domain ``(1) For transfer to any private person, nongovernmental entity, or other public-private business entity; or (2) Primarily for enhancement of tax revenue''.\footnote{South Dakota Codified Laws § 11-7-22-1, amended by House Bill 1080, 2006 Leg, Reg Ses (2006).}

In other states, more indirect measures were also taken, such as in Florida, where the legislature enacted a rule whereby property taken by the government could not be transferred to a private party until 10 years after the date it was condemned.\footcite[809]{eagle08} Many states also offer inclusive, often lengthy, lists of uses that should count as public, allowing the states to restrict the eminent domain power while also allowing condemnations that are regarded as particularly important to the state.\footcite[804]{eagle08}

Somin points to another interesting trend, namely that state reforms enacted by the public through referendums tend to be far more restrictive and effective in preventing economic and private-to-private takings than reforms passed through the state legislature.\footcite[2143]{somin09}

This reflects how strongly the US public opposed the decision in {\it Kelo}. Surveys show that as many as 80-90 \% believe that it was wrongly decided, an opinion widely shared also among the political elite.\footcite[2109]{somin09} Indeed, {\it Kelo} has had a great effect on the discourse of eminent domain in the US, and this effect is perhaps of greater importance than the various state reforms that have been enacted. According to Somin, most of the reforms have in fact been ineffective, despite the overwhelming popular and political opposition against economic development takings.\footcite[2170-2171]{somin09} 

Somin is not alone in feeling that eminent domain reform in the US promised more than it delivered. A similar sentiment is expressed both by supporters and critics of {\it Kelo}.\footnote{???} On the other hand, while practitioners have noted that it is largely business-as-usual in eminent domain law, they also report a greater feeling of unease regarding the public use requirement, expressing hope that the Supreme Court will soon revisit the issue.\footnote{See \cite{murakami13} (``Until the Supreme Court revisits the issue, we predict that this question will continue to plague the lower courts, property owners, and condemning authorities'').} In this way, the public backlash against {\it Kelo} has served as an influential reminder that the rationale behind eminent domain for economic development is largely out of sync with the sense of fairness and justice endorsed by most non-experts. 

But why have attempts at legislative reform been so inefficient? The underlying cause, according to Somin, can be traced to the fact that people are ``rationally ignorant'' about the economic takings issue. For most people, it is unlikely that eminent domain will come to concern them personally or that they will be able to influence policy in this area. Hence, it makes little sense for them to devote much time to learn more about it. This, in turn, helps create a situation where experts can develop and sustain a system based on practices that are in fact opposed by a large majority of citizens.\footcite[2163-2171]{somin09} Indeed, Somin argues that surveys show how people tend to overestimate the effectiveness of eminent domain reform, possibly due to the fact that symbolic legislative measures are mistaken for materially significant changes in the law.\footcite{somin09}

I think Somin's analysis is on an interesting track. However, it should be noted that the notion of rational ignorance is a double-edged sword in this regard. In particular, it cannot be ruled out {\it a priori} that the critical attitude towards economic development takings is itself an instance of such ignorance. Perhaps people would change their opinion on economic development takings if they were better educated on the issue?

This possibility does nothing to detract from the main message, which is that the {\it Kelo} backlash have caused greater insecurity about what the law is, without significantly curbing those uses of eminent domain that are regarded by many as problematic. Arguably, this shows that the static legislative approach to eminent domain reform, which has dominated the scene in the US so far, needs to be supplemented by more dynamic proposals. In particular, it seems important to target the decision-making processes surrounding land use planning and eminent domain, to look for ways to imbue this process with legitimacy.

In a country where the population expresses antagonism towards eminent domain for economic development, a more inclusive process will likely cause such takings to become more uncommon. On the other hand, if principles of good governance are put in place, it might also restore confidence in eminent domain as a procedure by which to implement democratically legitimate decisions about how to weigh the interests of landowners against the interests of the public. In the next subsection, I will consider two proposals for principles of this kind. The first specifically targets the question of how compensation is determined in economic development cases, a crucial aspect of legitimacy. The second proposal targets the decision-making process more broadly, by proposing a framework for land assembly that is meant to replace the use of eminent domain in certain circumstances.

%\noo{But it is not the general public that are the major stakeholders in such disputes, but rather the communities that are directly affected, including both the private property owners who will be burdened and those community members who stand to benefit. A good framework for balancing their interests relies on finding appropriate principles of good governance, so that governments can play an empowering role when such decisions are made. This is crucial for legitimacy of land use planning generally, but especially for eminent domain, where the gravity of the interference means that legitimacy is unlikely to arise unless the decision to condemn is firmly rooted in the interests of the main stakeholders. To the greatest possible extent, it also seems crucial to emphasize local conditions and ensure that the decision enjoys broad local support. 
%
%Shortly after {\it Poletown} was overturned, the case of Kelo saw the legitimacy of economic takings brought before the Supreme Court once again. This time there was real doubt and disagreement among the justices regarding the scope of the public use limitation. The case revolved around the legitimacy of condemning a home in favour of a research facility for the drug company Pfizer, which was part of a development plan for the City of New London.  The owner, Suzanne Kelo, argued that the condemnation of her home was in breach of the constitution, since it was a private-to-private taking ostensibly to the benefit of Pfizer rather than any clearly defined public use or interest.
%
%In Kelo, Justice Thomas adopted the strictest view on the public use test. He entirely disregarded  the precedent set by Berman and Midkiff in favour of constitutional originalism, the doctrine which asserts that direct assessment of the wording in the Constitution, and the intentions of the founding fathers, is the approach that should be used to decide constitutional cases. Following up on this he held that actual right of use for the public was the test that had to be applied in takings cases. The hundred years of precedent preceding Kelo was described as “wholly divorced from the text, history, and structure of our founding document", and thus Justice Thomas concluded that it had to be abandoned. 
%
%Justice O'Connor, in an expression of dissent joined by Chief Justice Rehnquist and Justices Scalia
%and Thomas, argued against legitimacy on less theoretical grounds, based on the facts of the case and the precedent that would be set for similar cases in the future. Her main legal argument was that while public use should be interpreted broadly, the possibility of positive ripple effects was not enough to justify private-to-private takings. In particular, Justice O'Connor took a very bleak view on the practical consequences that would arise from allowing economic takings that could be justified only by pointing only to indirect positive consequences for the public. She commented on the majority decision to uphold the taking as follows: 
%
%Any property may now be taken for the benefit of another private party, but the fallout from this decision will not be random. The beneficiaries are likely to be those citizens with disproportionate influence and power in the political process, including large corporations and development firms. As for the victims, the government now has license to transfer property from those with fewer resources to those with more. The Founders cannot have intended this perverse result.
%
%It seems that a major point of contention among the judges in the Supreme Court was whether or not these grim predictions was a realistic assessment of what the consequences of the decision would be. Surely, anyone who agrees with Justice O'Connor in her prediction of the fallout would also agree with here conclusion that it is perverse. But the majority in Kelo, in an opinion written by Justice Stevens, disagreed with her assessment, observing instead that a more restrictive view on economic takings would make it more difficult to cater to the "diverse and always evolving needs of society". 
%
%But the majority opinion also stressed that purely private takings where not permissible, and they attached great significance to the substantive assessment that the actual taking of Suzanne Kelo's home formed part of a comprehensive development plan that would not bestow special benefit on any particular group of individuals. Moreover, Justice Kennedy, in his concurring opinion, emphasised that states should not use public purpose as a pretext for interfering in property rights to the benefit of commercial actors.
%Hence the overall impression one is left with when considering Kelo in its historical and legal context is that it reflects an increasingly cautious attitude to economic takings. The precedent of virtually unlimited deference that was set in case law from the mid-to-late 19th Century was eschewed in favour of a more contextual approach where the merits and deeper purpose of the plans underlying a taking is not axiomatically beyond the scrutiny of the courts.
%
%From considering the reception of the case by the general public, we see even more clearly how Kelo in effect marks a change in the US towards greater scrutiny. 
%
%Indeed, the voices that have dominated in the aftermath of Kelo were critical of the decision and criticized the court for not offering better protection to property owners. The case also led to an a surge of academic interest in the pubic use restriction, with many arguing for further restrictions on the scope of the takings power. 
%Hence it seems that Justice O'Connor's opinion largely reflects contemporary worries about takings in the US, worries that are now also becoming increasingly relevant to how the law develops and is understood. Many states have changed their own eminent domain codes  following Kelo, to make it harder to undertake economic takings. Moreover, the federal government also banned such takings from taking place on the basis of federal takings powers.
%It will lead us astray to delve deeply into the question of what caused this change in perspective on economic takings in the US, but we can offer a few hypothesis. First, it seems that cases such as Poletown illustrates the potential danger inherent in making the power of eminent domain available to market players. In particular, the main worry that has been raised is that the pretext of public purpose may be in the process of becoming a powerful instrument for influential market actors to gain access to regulatory powers of government. As these powers has massively expanded in the post-WW2 period, so has the potential for abuse. In addition, it seems that while those who were adversely affected by eminent domain tended to be less privileged and resourceful groups of society, the takings power is now increasingly brought to bear also against members of the middle class, who are in a better position to fight it, both legally and on the political scene.
%
%While opinions differ greatly both regarding the extent of the problem and the causes of recent controversy, there is something near consensus in the US after Kelo that economic development takings raise special problems under the current system of eminent domain, and that these need to be addressed with a view to reducing tensions and restoring faith in the system. Indeed, even the majority in Kelo hint strongly at this when they say that  
%Some have argued forcefully that a strict reading of the public use requirement is the way forward, if not by strict interpretation then by an explicit ban on economic development takings.  However, it is tempting here to echo the worries expressed in Seawell, that a strict formalistic approach to legitimacy runs the risk not only of being inflexible, but also, eventually, of offering less  protection to property owners. How, then, should we reduce the risk of abuses?
%While many have focused on the question of banning economic taking, or reconsidering the public use clause, some have addressed this question from such a broader angle. In my opinion, this is the way forward. It seems, in particular, that a complete ban on economic development takings will leave a vacuum in the current economic system, which presupposes a great deal of cooperation between commercial and public interest. Particularly when it comes to economic development, the private-public partnership model has gained influence to the point that a ban on economic development takings would likely prove impossible to implement in a satisfactory manner. 
%More generally, it seems hard to address the problem of economic takings without considering the role they play in the larger economic context within which current rules and practices have developed. Based on such considerations, I believe the procedural approach to economic takings is the appropriate one. This perspective asks us to take a closer look at judicial safeguards for protecting the role of property owners in the decision-making processes that lead up to the use of eminent domain. To some extent one might approach this on the basis of existing legal principles, asking for better scrutiny of procedural aspects, or by making it easier to bring pretext claims before the courts. However, it might also require new ideas, and, in particular, the introduction of new institutions for decision-making and administration of the eminent domain process.
%
%In the next section, I will look at two concrete proposals in more detail, one concerning the decision-making step and the other concerning the calculation of compensation. 
%They will be important because they serve as starting points for the case study that is to follow, addressing mechanisms that we will return to in Chapters x and y when we look more closely at two Norwegian legal institutions that share many features with the theoretical roposals discussed in the next section.
%}

\section{Institutional Proposals for Increased Legitimacy}\label{sec:ir}

In this subsection, I first present the Special Purpose Development Companies proposed by Lehavi and Licht.\footcite{lehavi07} I relate this proposals to theoretical approaches to the issue of compensation, before I go on to note some shortcomings and open questions that I will later address in my case study. I then go on to consider the Land Assembly Districts proposed by Heller and Hills.\footcite{heller08} I consider this proposal in light of the stated motivation, which is to design an effective mechanism of self-governance that can replace eminent domain in economic development cases. I present some unresolved questions and argue that there is a tension in the proposal between its narrow scope, imposed to prevent majority tyranny and other forms of abuse, and its broad goal of empowering local communities. 

\subsection{Special Purpose Development Companies}

An important distinguishing feature of economic development takings is that they give the taker an opportunity to profit commercially from the development. This may even be the primary aim of the project, with the public benefitting only indirectly through potential economic and social ripple effects. Property owners facing condemnation in such circumstances might expect to take a share in the profit resulting from the use of their land. However, in many jurisdictions, including the US, the rules used to calculate compensation prevents owners from getting any share in the commercial surplus resulting from development.\footnote{See, e.g., \cite[965-966]{fennell04}.} In particular, various {\it elimination rules} are typically in place to ensure that compensation is based entirely on the pre-project value of the land that is being taken.\footcite[See][81]{freilich06} The policy reasons for such rules is that they ensure that the public does not have to pay extra due to their own special want of the property. After all, this is one of the main purposes of using eminent domain in the first place, to ensure that the public does not have to pay extortionate prices for land needed for important projects. However, when the purpose of the project is itself commercial in nature, there appears to be a shortage of good policy reasons for excluding this value from consideration when compensation is calculated. This is especially true when, as in the US, compensation tends to be based on the market value of the land taken. Why should a commercial condemner's prospect of carrying out economic development with a profit be disregarded when assessing the market value? In any fair and friendly transaction among rational agents, one would expect benefit sharing in a case like this. Yet for economic development backed up by eminent domain, the application of elimination rules ensures that all the profit goes to the developer. 

Some authors have argued that failures of compensation is at the heart of the economic takings issue and that worry over the public use restriction is in large part only a response to deeper concerns about the ``uncompensated increment'' of such takings.\footcite[See][962]{fennell04} In addition to the lack of benefit sharing, previous work has identified two further problems of compensation that also tend to become exasperated in economic development cases. First, the problem of ``subjective premium'' has been raised, pointing to the fact that property owners often value their own land higher than the market value, for personal reasons.\footcite[963]{fennell04} For instance, if a home is condemned, the homeowner will typically suffer costs not covered by market value, such as the cost of moving, including both the immediate ``objective'' logistic costs as well as more subtle costs, such as having to familiarize oneself with a new local community. Second, the problem of ``autonomy'' has been discussed, arising from the fact that an exercise of eminent domain deprives the landowner of \isr{their right to decide how to manage their} property.\footnote{Discussed in \cite[966-967]{fennell04}. For a general personhood building theory of property law, see \cite{radin93}. For a general economic theory of the subjective value of independence, see \cite{benz08}.}

In \footcite{lehavi07}, the authors propose a novel approach for addressing the ``uncompensated increment'' in economic takings cases. Their proposal is based on a new kind of structure that they dub a {\it Special Purpose Development Corporation} (SPDC). The idea is that owners affected by eminent domain will be given a choice between standard pre-project market value and shares in a special company. This company will exist only to implement a specific step in the implementation of the development project: the transaction of the land-rights. The SPDC may choose either to offer their rights on an auction or else negotiate a deal with a designated developer.\footcite[1735]{lehavi07} Hence, the idea is to ensure that the owners are paid a value that reflects the post-project value of the land, but in such a way that the holdout problem is avoided. In particular, the SPDC will have a single task: to sell the land for the highest possible price within a given time frame.\footcite[1741]{lehavi07} After the sale is completed, the SPDC will divide the proceeds as dividends and be wound up.\footcite[1741]{lehavi07}

Other suggestions have taken a more static approach to compensation reform, such as proposing to give owners a fixed premium in cases of economic development, or developing mechanisms of self-assessment to ensure that compensation is based on the true value the owner attributes to his own land.\footnote{A range of static proposals have been proposed in the literature: Merrill proposes 150 \% of market value for takings that are deemed to be ``suspect'', including takings for which the nature of the public use is unclear, see \cite[90-93]{merrill86}. Krier and Serkin propose a system that provide compensation for a property's special suitability to its owner, or a system where compensation is based on the court's assessment of post-project value, see \cite[865-873]{krier04}. Fennell proposes a system of self-evaluation of property for takings purposes with tax-breaks given to those who value their property close to market value (to avoid overestimation), see \cite[995-996]{fennell04}. Bell and Parchomovsky also propose self-evaluation, but rely on a different mechanism to prevent overestimation; tax liability is based on the self-reported value and no property can be sold by its owner for less than his reported value, see \cite[890-900]{bell07}.} Compared to such proposals, the idea of SPDCs is more sophisticated and should be looked at in more depth. 

The conceptual premise for the proposal is that takings for economic development can be seen as compulsory incorporation, a pooling of resources useful in overcoming market failures.\footcite[1732-1733]{lehavi07} Just as the corporation is formed to consolidate assets in order to facilitate effective management, so is eminent domain used to assemble property rights in order to facilitate efficient organization of development. According to Lehavi and Licht, this also provides a viable approach to problems of ``opportunistic behavior''; hierarchical governance after assembly ensures that order and unity can be regained even if interests in the land are distributed among a large and heterogeneous group of potentially mischievous shareholders.\footcite[1733]{lehavi07} In the words of Lehavi and Licht:

\begin{quote}
The exercise of eminent domain powers thus resembles an incorporation by the government of all landowners with a view to \isr{bringing} all the critical assets under hierarchical governance. Establishing a corporation for this purpose and transferring land parcels to it thus would be merely a procedural manifestation of the substantive economic reality that already takes place in eminent domain cases.
\end{quote}

As soon as we look at the rationale behind economic development takings in this way, any remnant of good policy reasons for ensuring that the developer gets all the profit seems to disappear. Rather, we are led to consider compensation as an issue entirely separate from the exercise of the takings power. After the land has been \isr{reorganised} by eminent domain and an SPDC has been formed, the land rights might as well be sold {\it freely} to a developer. In this way, the land will be sold for a price that is closer to an actual market value, on the market where the land is destined for development.\footcite[1735-1736]{lehavi07} More generally, the SPDC becomes an aid that the government can use to create more \isr{favourable} market conditions for transferring land that has commercial potential in its public use. Due to the compulsory pooling of resources, no owner can exercise monopoly power by holding out, but due to decoupling of compensation from assembly, the owners can now negotiate with potential developers for a share of the resulting profit. Moreover, the fact that the SPDC offers its rights on an actual market can also help ensure that more information \isr{becomes} available regarding the true economic value of the development, something that may in turn help ensure that only the good projects will be successful in acquiring land. Hence, according to Lehavi and Licht, an additional positive effect of SPDCs is that developers and governments will \isr{shy }away from using the eminent domain power to benefit projects that are not truly welfare-enhancing.\footcite[1735-1736]{lehavi07}

In addition to these substantive consequences, the SPDC-proposal also stands out because it has a significant institutional component, pointing to its potential for restoring procedural legitimacy as well as substantive fairness. Lehavi and Licht discuss corporate governance issues at some length, but without committing themselves to definite answers about how the operations of the SPDC should be \isr{organised}.\footcite[1040-1048]{lehavi07} Indeed, while their proposal is perhaps most interesting because of its procedural aspects, it also appears to be rather preliminary in this regard. The main idea is to let the SPDC structure piggyback on existing corporative structures, particularly those developed for \isr{securitisation} of assets.\footnote{See generally \cite{schwarcz94}. For an up-to-date overview, targeting special challenges that became apparent during the 2008 financial crisis, see \cite{schwarcz13}.} The basic idea is that the corporate structure should be insulated from the original landowners to the greatest possible extent; it should have a narrow scope, it should be managed by neutral administrators, and it should entrust a third party with its voting rights.\footcite[1742]{lehavi07} This is meant to prevent failures of governance within the SPDC itself, making it harder for majority shareholders and self-interested managers to co-opt the process. For instance, if a possible developer already holds a majority of the shares in an SPDC, this structure would prevent him from using this position to acquire the remaining land on \isr{favourable} terms. 

Lehavi and Licht observe that under US law, the government would often be required to make shares in an SPDC available to the landowners as a public offering.\footcite[1745]{lehavi07} Lehavi and Licht deem this to be desirable, arguing that full disclosure will provide owners with a better basis on which to decide whether or not to accept SPDC shares in place of pre-project market value. It will also facilitate trading in such shares, so that they will become more liquid and therefore, presumably, more valuable.\footcite[1746]{lehavi07} 

Lehavi and Licht's proposal is interesting, but I think a fundamental objection can be raised against it. In particular, it seems that their governance model more or less completely alienate property owners from the decision-making process after SPDC formation. Limiting the participation of owners is to a large extent an explicit aim, since governance by experts is held to increase the chances of ensuring good governance. But is expert rule really the answer?

It seems that from the owners' point of view, Lehavi and Licht's proposals for governance reduces the SPDC to a mechanism whereby they can acquire certain financial entitlements. These may exceed those that would follow from standard compensation rules, but they do not directly empower owners vis-{\'a}-vis developers and the government. Instead, a largely independent structure will be introduced. It is this new \isr{organisational} structure, rather than the owners, that will now become an important actor in the eminent domain process. In principle, it is meant to represent owners, but to what extent can it do so effectively? After all, it is specifically intended to operate as neutral player, charged with \isr{maximising} the price, nothing more. Hence, it appears that the SPDC will not be able to give owners an arena to negotiate on the basis of the personal and social importance they attribute to their land rights. How the problem of ``autonomy'' is addressed by the proposal is therefore hard to see and the ``subjective premium'' also appears to be in danger, unless it can be objectively quantified and covered by the surplus from a voluntary sale. But if such quantification is possible, then why not simply tell the appraiser to award some premium under standard compensation rules?

More generally, it seems to me that while all three categories of ``uncompensated increments'' are interesting to study from a financial viewpoint, severe doubts can be raised regarding the feasibility of addressing the subjective aspects of this as a question of compensation. It may be that issues related to ``subjective premium'' and ``autonomy'' are seen as public use issues for good reason; they are hard to quantify otherwise. Moreover, attempting to do so might do more harm than good. On the one hand, it might skew the political process, since owners that have been ``bought off'' don't object to ill-advised development projects, as long as they generate financial revenue. But what about projects that are undesirable for other reasons, for instance because they completely change the character of a \isr{neighbourhood}, or because they are harmful to the environment? On the other hand, the very idea that money can compensate for the subjective importance of property and autonomy can itself prove offensive. At least it seems likely that it would often come to be seen as inadequate and inefficient.\footnote{For more detailed criticism of the compensation approach to the public use issue, see \cite{garnett06}.} Moreover, an owner that is compelled to give up his home after an inclusive process where the public interest has been debated and clearly communicated is likely to feel like he incurs less costs related both to his subjective premium and his autonomy. Hence, the lack of participation in the decision-making process can in itself increase the uncompensated loss. Clearly, no externally managed ``bargain-oriented'' SPDC will be able to resolve this problem. Of course, some ``objective'' elements of, such as relocation costs or cost for juridical assistance, can still be addressed under the banner of compensation. But in most jurisdictions, they already are.\footnote{See, e.g., \cite[121-126]{garnett06}.} For more subtle aspects, the aftermath of {\it Kelo} itself can serve as an illustration of how a compensatory approach is unsatisfactory:

After the case, Suzanne Kelo remained defiant, until she eventually decided to settle in 2006, for an offer of \$ 442 155, more than \$ 319 000 above the appraised value.\footcite[1709]{lehavi07} Apparently, the other owners affected by the same taking were not particularly pleased, arguing that recalcitrant owners were actually rewarded for holding out.\footcite[1709]{lehavi07} On the other hand, there is no indication that Suzanne Kelo was not genuine in her opposition to the taking. Indeed, after the long struggle she had taken part in, it is easy to imagine that financial compensation, if it was to be an effective remedy at all, would have to be very high. Even after she had settled, Kelo apparently toured the country speaking out against economic takings. This, too, is a statement to the inadequacy of a purely financial approach to legitimacy. 

I conclude that SPDCs have serious shortcoming with regards to the subjective aspects of undercompensation, aspects that can only be addressed if the focus turns towards participation. However, SPDCs do seem promising when it comes to profit-sharing. This, after all, is what the structure is specifically aiming to achieve. In addition, I agree that SPDCs will likely have a positive effect on the other actors in the eminent domain process. In particular, I agree with Lehavi and Licht that greater openness is likely to result, revealing the true merits of development projects, at least in so far as these are translatable into financial terms. The fact that developers must negotiate with an SPDC who can threaten to make the land available an an open auction will likely deter developers and government from pursuing fiscally inefficient projects. Hence, the risk that governments will \isr{subsidised} such projects by giving them cheap access to land will also be reduced. In addition, the presence of a third voice, speaking on behalf of owners, is likely to help achieve a better balance of power in development takings. 

Even if the individual landowners do not have a voice in this process, the fact that the landowners are better represented as a group is then still likely to have a positive effect on legitimacy. On the other hand, as long as the power of the SPDC is limited to choosing the best offer and negotiating over price, it seems that SPDCs will easily end up being dominated by developers and government. This is a particular concern in cases when competition fails to arise after SPDC formation. To ensure that there are other interested parties, in particular, sems like an important precondition for the proposal to work in practice. In this regard, it is important to \isr{realise} that a lack of interest from other developers may not be due to the superiority of the original developer's plans. It might rather be due to the fact that the scope of the assembly giving rise to the SPDC is so defined as to make alternatives unfeasible. The danger of abuse in this regard seems significant, particularly when developers themselves participate in coming up with the plans that give rise to SPDC formation. 

Moreover, as long as owners remain marginalized in the planning phase, it is easy to imagine situations where the plan itself will be formulated in such a way that only one developer is in a position to successfully implement it. A simple example would be if a prospective developer already owns some of the land that is critical to the plan, and is able to ensure that this land is kept out of the scope of the SPDC. Clearly, if SPDCs are to operate effectively, such instances of manipulation need to be avoided, suggesting that the proposal as it stands needs to be fleshed out in greater detail.

The problems addressed here both seem to point to the fact that the SPDCs, while more flexible than other suggestions, are still too static to achieve many of their objectives. In particular, to arrive at genuine market conditions for assessing post-project value, there is still a need for changes in the dynamics of the planning process underlying the taking. Moreover, to ensure legitimacy, there is a need for a mechanism that goes beyond expert bargaining and provides owners with better access to the \isr{decision making} process. In the next subsection, I will consider a proposal that aims to address this, by proposing a framework for self-governance. 

\subsection{Land Assembly Districts}

In a recent article, Heller and Hills propose a new institutional framework for carrying out land assembly for economic development. Interestingly, it is meant to replace eminent domain altogether. The goal is to ensure democratic legitimacy while also creating a template for collective decision-making that will prevent inefficient gridlock and holdouts. 

The core idea is to introduce {\it Land Assembly Districts} (LADs), institutions that will enable property owners in a specific area to make a collective decision about whether or not to sell the land to a developer or a municipality.\footcite[1469-1470]{heller08} The idea is that while anyone will be able to propose and promote the formation of a LAD, the official planning authorities and the owners themselves must consent before it is formed.\footcite[1488-1489]{heller08} Clearly, some kind of collective action mechanism is required to allow the owners to make such a decision. Hiller and Hill suggest that voting under the majority rule will be adequate in this regard, at least in most cases.\footnote{See \cite[1496]{heller08}. However, when many of the owners are non-residents who only see their land as an investment, Heller and Hills note that it might be necessary to consider more complicated voting procedures, for instance by requiring separate majorities from different groups of owners. See \cite[1523-1524]{heller08}.} How to allocate voting rights in the LAD is given careful consideration, with Heller and Hills opting for the proposal that they should in principle be given to owners in proportion to their share in the land belonging to the LAD.\footnote{See \cite[1492]{heller08}. For a discussion of the constitutional one-person-one-vote principle and a more detailed argument in \isr{favour} of the property-based proposal, see \cite[1503-1507]{heller08}.} Owners can opt out of the LAD, but in this case eminent domain can be used to transfer the land to the LAD using a conventional eminent domain procedure.\footcite[1496]{heller08}

Heller and Hills envision an important role for governmental planning agencies in approving, overseeing and facilitating the LAD process. Their role will be most important early on, in approving and spelling out the parameters within which the LAD is called to function.\footcite[1489-1491]{heller08} Hence, it appears to be assumed that the planning authorities will define the scope of the LAD by specifying the nature of the development it can pursue. A possible challenge that arises, not discussed by Heller and Hills at any length, is that the scope of the LAD needs to be broad enough to allow for meaningful competition and negotiation after LAD formation. But there will probably be a push, both by governments and initiating developers, to ensure that the scope is defined narrowly enough to give confidence that zoning permissions will not be denied at a later stage. Hence, the LAD proposal needs to ensure a balanced approach to the issue of how the initial development plan should be defined, and to what extent it should limit the authority of the LAD.

If the owners do not agree to forming a LAD, or if they refuse to sell to any developer, the government will be precluded from using eminent domain against them to assemble the land.\footcite[1491]{heller08} This is the crucial novel idea that sets the suggestion apart from other proposals for institutional reform that have appeared after {\it Kelo}. LADs will not only ensure that the owners get to bargain with the developers over compensation, it will also give them an opportunity to refuse any development to go ahead. Hence, the proposal shifts the balance of power in economic development cases, giving owners a greater role also in preparing the decision whether or not to develop, and on what terms. This makes the proposal stand out in the recent literature on economic takings. It is the first concrete suggestion that addresses the democratic deficit in a dynamic, procedural manner, without failing to \isr{recognise} that the danger of holdouts is real and that institutions are needed to avoid it.

There are some problems with the model, however. First, observe that planning authorities might have an incentive to refuse granting approval for LAD formation. After all, doing so entails that they give up the power of eminent domain for the land in question. For this reason, Heller and Hills propose that a procedure of judicial review should exist whereby a decision to deny approval for LAD formation can be scrutinized.\footcite[1490]{heller08} 

After the formation of the LAD, the government will not be able to use eminent domain against it, but the planning authorities will still occupy an important role. Heller and Hills envision a system of public hearings, possibly \isr{organised} by the planning authorities, where potential developers meet with owners and other interested parties to discuss plans for development.\footnote{See \cite[1490-1491]{heller08}.} In this process, it is assumed that other interests will also be represented, such as owners of adjoining land, who might want to raise objections against the project. However, their role in the process is not clarified in any detail, raising worries about the extent to which the LAD will undermine local democracy by giving property owners a privileged position with respect to policy questions that should be decided jointly by all members of the local community.

The LAD proposal also raises issues pertaining to the proposed mechanism of collective decision-making among owners. As Kelly points out, the basic mechanism of majority voting is imperfect.\footcite{kelly09} He argues, in particular, that if different owners value their property differently, majority voting will tend to \isr{disfavour} those with the most extreme viewpoints, either in \isr{favour} of, or against, assembly. If these viewpoints are assumed to be non-strategic and genuine reflections of the welfare associated with the land, the result can be inefficiency. In short, the problem is that a majority can often be found that does not take due account of minority interests. 

For instance, if a minority of owners are planning alternative development, conflicting with the LAD proposal, they might simply be ignored. Indeed, they might have to be ignored, since the formation of the LAD itself precludes the kind of development they wish to pursue. This could become particularly inefficient in cases when this development would also be more socially desirable than the development that will benefit from assembly. The role of the LAD in such cases will not improve the quality of the decision to develop, since it pushes the decision-making process into a track where those interests that {\it should} prevail are voiced only by a marginalised minority inside the new institution.\footnote{Of course, one might imagine these landowners opting out of the LAD, or pursuing their own interests independently of it. However, they are then unlikely to be better off than they would be in a no-LAD regime. In fact, it is easy to imagine that they could come to be further \isr{marginalised}, since the existence of the LAD, acting ``on behalf of the owners'', might detract from any dissenting voices on the owner-side.}

More generally, the lack of clarity regarding the role of LADs in the planning process is a problem. As it stands, the proposal leaves it uncertain how LADs will affect the decision-making process regarding development. 

However, the ideal is clearly stated. LADs should help to establish self-governance for land assembly. In particular, Heller and Hills argue that LADs should have ``broad discretion to choose any proposal to redevelop the \isr{neighbourhood} -- or reject all such proposals''.\footcite[See][1496]{heller08} As they put it, two of the main goals of LAD formation is to ensure ``preservation of the sense of individual autonomy implicit in the right of private property and preservation of the larger community's right to self-government''.\footcite[See][1498]{heller08} Unfortunately, these ideals are somewhat at odds with the concrete rules that Heller and Hills propose, particularly those aiming to ensure good governance of the LAD itself. 

In relation to the governance issue, Heller and Hills echo many of the ``corporate governance''-ideas that also feature heavily in Lehavi and Licht's proposal. Indeed, in direct contrast to their comments about ``broad discretion'' and ``self-governance'', Hiller and Hills also state that ``LADs exist for a single narrow purpose -- to consider whether to sell a neighborhood''.\footcite[See][1500]{heller08} This is a good thing, according to Heller and Hills, since it provides a safe-guard against mismanagement, serving to prevent LADs from becoming battle grounds where different groups attempt to co-opt the community voice to further their own interests. As Heller and Hills puts it, the narrow scope of LADs will ensure that ``all differences of interest based on the constituents' different activities and investments, therefore, merge into the single question: is the price offered by the assembler sufficient to induce the constituents to sell?''.\footcite[1500]{heller08}

But this means that there is an internal tension in the LAD proposal, between the broad goal of self-governance on the one hand and the fear of \isr{neighbourhood} bickering and majority tyranny on the other. Moreover, it is hard to see how LADs can at once have both a ``narrow purpose'' as well as enjoy ``broad discretion'' to choose between competing proposals for development. If such discretion is granted to LADs, what prevents special interest groups among the landowners from promoting development projects that will be particularly \isr{favourable} to them, rather than to the landowners as a group? What is to prevent landowners from making behind-the-scene deals with \isr{favoured} developers at the expense of their \isr{neighbours}? It might be difficult to come up with rules that prevent mechanisms of this kind, without also making meaningful ``self-governance'' an impossibility. 

If a LAD is obliged to only look at the price, this might prevent abuse. But it will not give owners broad discretion to choose among development \isr{proposals}. Effectively, it will render LADs as little more than a variant of SPDCs, where the owners are awarded an extra bargaining-chip, namely the option to refuse all offers. 

In my view, such a restriction on the operations of LADs is not desirable. It is easy to imagine cases where competing proposals, perhaps emerging from within the community of owners themselves, will emerge in response to the formation of a LAD. Such proposals may involve novel solutions that are superior to the original development plans, in which case it is hard to see any good reason why they should not be taken into account, even if they are proposed by a minority. Moreover, it is hard to see why they should be disregarded simply because they are less commercially attractive, even if the price offered is not competitive. In particular, the formation of a LAD and the competition for development that ensues creates an opportunity for tapping into a greater pool of ideas for redevelopment, ideas which may then also be rooted more firmly in the local community. Surely, getting such proposals to the table would be desirable. Moreover, it would take us to the heart of self-governance. At the same time, it is easy to acknowledge that problematic situations may arise, for instance if a majority forms in \isr{favour} of a scheme that involves razing only the homes of the minority, maybe on the rationale that these are the most blighted properties. That would likely give rise to accusations of unfair play, which may or may not be warranted. But \isr{irrespective} of this, an alternative project of this kind might well be a better use of the land in question, also from the point of view of the public. Hence, it would seem that the planning authorities would be obliged to give it some serious consideration. Then, however, the LAD has truly become an arena for a new kind of power play among different \isr{interests}, and a potential vehicle of force for whomever secures support from a majority of owners within the district.

In their proposal, Heller and Hills are aware of this potential problem, which they propose to resolve by strict regulation. In particular, they argue that ``LAD-enabling legislation should require especially stringent disclosure requirements and bar any landowner from voting in a LAD if that landowner has any affiliation with the assembler''.\footcite{heller08} But this raises further questions. For one, what is meant by ``affiliation'' here? Say that a landowner happens to own shares in some of the companies proposing development. Should he then be barred from voting? If so, should he be barred from voting on all proposals, or just those involving companies in which he is a shareholder? If the answer is yes, how would this be justified? Would it not be easy to construe such a rule as  discrimination against landowners who happen to own shares in development companies? On the other hand, if the landowner in question is allowed to vote on all other proposals, would it not be natural to suspect that his vote is biased against assembly that would benefit a competing company? Or what about the case when some of the landowners are employed by some of the development companies? Should such owners be barred from voting on proposals that could benefit their employers? This seems quite unfair as a general rule, especially if a low-level employment relationship has such a dramatic effect. But in some cases even low-level ties could play a decisive factor. This might happen, for instance, if an important local employer proposes development in a \isr{neighbourhood} where it has a large number of employees.

Of course, the most pressing issue that arises is the following: who exactly should be empowered to make the determination of when an affiliation is such that an owner should be deprived of his voting rights? Heller and Hills give no answer, but it is easy to imagine that whoever is given this task in the first instance, the courts would soon enough be asked to consider the question. At this point, the circle has in some sense closed in on the proposal. In particular, one might ask: why is it easier to determine if someone can be deprived of his voting rights due to an ``affiliation'', than it is to determine if someone can be deprived of his land due to some planned ``public use''?

In any event, to come up with a set of rules ensuring that LADs can deliver both self-governance and good governance largely remains an open problem. This is acknowledged by Hiller and Hills themselves, who point out that further work is needed and that only a limited assessment of their proposal can be made in the absence of empirical data. Later in the thesis, I will shed light on this challenge when I consider the Norwegian rules relating to land consolidation, showing how these can be looked at as a highly developed institutional embedding of many of the central ideas of LADs. The assessment of how they function in cases of economic development, and how they are increasingly used as an alternative to expropriation in cases of hydropower development, will allow me to shed further light on the issues that are left open by Heller and Hills' important article.

\section{Conclusion}\label{sec:conc2}

In this chapter, I have given a more in-depth presentation of economic development takings. I began by noting that the issue is particularly pressing for land users that are not regarded as bringing about economic growth. Hence, I argued that the issue is closely related to that of land grabbing, which is currently receiving much attention, both academic and political. Under the social function understanding of property there is in principle no difference between protecting property rights arising from formal title and property rights arising from use. That said, special issues arise in the latter case, not least because it is unclear how the law should deal with rights resulting from cultural practices that western property regimes are not designed to handle. In addition, I noted that special issues related to poverty and basic necessities such as food and water arise with particular urgency in relation to land grabbing.

The nature of my case study makes it natural for me to focus on traditional western systems of property law. Hence, I went on to discuss how economic development takings are dealt with in such legal systems, focusing on Europe and the US respectively. For the case of Europe, this assessment was made more difficult by the fact that the category is not an established part of legal discourse. However, by looking to England for concrete examples, I noted that such cases do arise and that they are increasingly seen as controversial. \noo{ I also noted that there is a contrast between how England and Germany approach such cases, as well as how they approach property more generally. Germany, in particular, goes further in explicitly \isr{recognising} the social functions of property, by actively looking to social and political values when assessing whether interferences are legitimate. In England, similar reasoning is at most applied indirectly, as takings are approached almost entirely as an issue of administrative law. }

I then went on to consider the property protection offered by P1(1) of the ECHR, and how it is applied by the Court in Strasbourg. I zoomed in on those \isr{aspects} that I believe to be the most relevant for economic development takings. While I noted that this category has yet to be discussed by the ECtHR, I argued that a recent shift in the Court's property adjudication is suggestive of the fact that it would likely approach such cases similarly to how Justice O'Connor approached {\it Kelo}. In particular, I noted how the Court has recently adopted a stricter standard of assessment. This standard, I argued, is \isr{characterised} primarily by increased sensitivity to systemic imbalances causing alleged P1(1) violations. Hence, to regard economic development takings as a special category appears to fit well with recent jurisprudential developments at the Court in Strasbourg.

I went on to consider US sources on economic development takings, noting that the issue has receive an extraordinary amount of attention in recent years. I adopted an historical approach to the material, by tracing the case law surrounding the public use restriction in the fifth amendment to the US constitution, which was much debated even before the specific issue of economic development takings rose to prominence. I focused particularly on case law developed by state courts, and I argued that it shows great sensitivity to the need for contextual assessment. Indeed, it seems that many state courts originally adopted an implicit social function view of property when assessing such cases.

I then looked at the history of Supreme Court adjudication of public use cases. I noted that the doctrine of deference was developed early on, but that it was initially directed mainly at state courts. In fact, I showed that the Supreme Court itself explicitly approved the contextual and in-depth approach these courts relied on when dealing with the legitimacy issue.

The shift, I argued, came with {\it Berman}, in which the Supreme Court adopted a deferential doctrine that was directed specifically at the state legislature.\footcite{berman54} This was quite a dramatic departure from the Court's previous attitude towards state takings. Moreover, it was almost entirely backed up by precedent set in cases when {\it federal} takings had been ordered by Congress.

I went on to consider the fallout of {\it Berman} at state level, which culminated with the infamous {\it Poletown} case. This case prompted wide-spread accusations of eminent domain abuse and thus set the stage for {\it Kelo}.

After completing the historical overview, I went on to consider the literature after {\it Kelo}. I expressed particular support for those responses that focus on the need for {\it institutional} reform, to address  dangers that Justice O'Connor pointed to in her minority opinion. As a shorthand, I proposed referring to the mechanisms she identified as the {\it democratic deficit} of economic development takings. 
% I zoomed in on two of those in particular, the Special Purpose Development Companies proposed by Lehavi and Licht, and the Land Assembly Districts suggested by Heller and Hills. I gave an in-depth presentation of these two proposals, pointing out strengths and weaknesses. 
%%In coming chapters, I will refer back to this as I consider similar institutions and mechanisms that are currently operating in Norwegian law relating to hydropower development.

I then gave a thorough presentation of two recent reform suggestions that might help address this deficit. Both are institutional in nature, based on setting up formally recognised coalitions of land owners that can act as a counterweight to the disproportional power of commercial beneficiaries. The first suggestion, by Lehavi and Licht, is limited to dealing with the issue of compensation, \isr{recognising} the need for a system whereby the land owners are compensated based on post-project value. However, this idea alone represents a fairly dramatic break with the currently dominant doctrine in takings law, where compensation is almost always based on the pre-project value of the land {\it to the owner}.\footnote{This is a reflection of the no-scheme principle, mentioned briefly in Section \ref{sec:england} above. For further details, I refer to \cite{dyrkolbotn15}.}

In Chapter \ref{chap:4}, I will briefly discuss how this principle was abandoned in Norway, for some case types involving hydropower development. However, the broader point that will interest me in thesis concerns the conceptual premise of Lehavi and Licht's proposal. By suggesting that economic development takings can be viewed as a form of compulsory incorporation of private rights, they effectively undermine the justification for disallowing the original owners to take up a corresponding share in the resulting enterprise. This, in my view, is a powerful idea that has implications that go well beyond the issue of compensation. In particular, it points to the possibility of avoiding eminent domain altogether, by proposing a suitable framework for collective action regarding economic development.

%I relate this to the special role played by the appraisal courts in Norway. The local grounding of these courts, involving lay people sitting as court appointed appraisers, allows the law to be applied in a way that adapts to the concrete circumstance in a way that may enhance the perceived fairness and legitimacy of the taking. At the same time, however, the judicial procedure, with a (limited) possibility for appeal, puts in place safeguards against abuse.

The second suggestion I looked at in depth, proposed by Heller and Hills, is based on a similar idea. However, it does not go as far as to explicitly suggest that owners themselves should be granted shares in the development enterprise. Instead, the focus is on organising a process for selling the properties required, without the use of compulsion. According to this proposal, local communities should be entitled to greater self-governance in economic development scenarios. At the same time, the proposal \isr{recognises} the need for a mechanism to avoid inefficient and socially harmful gridlock due to holdouts among unwilling owners. Instead of eminent domain, however, a different mechanism is proposed, namely that of a majority decision made by a land assembly district.

This is also a new type of institution, and I pointed out some problems and seeming inconsistencies in the proposal. I highlighted the lack of clarity regarding the exact role LADs are supposed to play during the planning process. I argued that while the risk of abuse and failure increases with the level of participation, so does the overall potential for achieving a positive effect on legitimacy. I concluded that to reduce the democratic deficit in economic development cases, a wide power of participation must be granted to the land owners and their communities. This is needed, in particular, to restore balance in the relationship between owners and others directly connected with the land, the planning authorities, and the commercial actors interested in development for profit. The question that is as of yet unresolved is how to \isr{organise} such participation in a way that avoids obvious pitfalls, such as administrative inefficiency and tyranny by majorities or elites that gain control of the local agenda.

In Chapter \ref{chap:6}, I will shed light on this question by considering the Norwegian institution of land consolidation, which has a very long tradition behind it. It is a flexible \isr{framework} which includes, among other things, a template for establishing institutions that can function as a LAD. I will focus on how land consolidation functions in cases of economic development that would otherwise likely be pursued by eminent domain. The case study is based on considering hydropower development, but I will also discuss planning law and development more generally, as the Norwegian government is now considering making consolidation, traditionally a rural institution, a primary mechanism for land development even in urban areas.

Before I delve into this, I will present an overview of Norwegian hydropower and the role of waterfalls as private property. This will serve as an introduction to the second part of this thesis, to which I now turn.