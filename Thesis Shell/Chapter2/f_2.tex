%\newcommand{\isr}[1]{{#1}}

\chapter{Taking Property for Profit}\label{chap:2}

\section{Introduction}\label{sec:intro}

In the previous chapter, I argued that economic development takings are a separate category of interference with private property. I also placed such takings in the theoretical landscape, by relating them to the social function theory of property. In particular, I argued that economic development takings raise questions that require us to depart from the individualistic, entitlements-based narrative that otherwise dominates in property theory.

This chapter develops this idea further, by considering how economic development cases are dealt with in England, where such takings have yet to be widely recognised as a separate category, and the US, where they first began to attract special attention. In addition, the chapter considers case law from the ECtHR and asks what it tells us about how to approach economic development takings under European human rights law.\footnote{So far, the issue of economic development takings have been brought into focus at the Court in Strasbourg.}
Finally, the chapter considers recent proposals for reform that focus on how to increase legitimacy by developing new institutions for self-governance to replace the traditional takings procedure in economic development cases.

I begin in Section \ref{sec:lgppp} by commenting briefly on the importance of economic development takings on the global stage. Specifically, I note that the core issues raised by such takings appear relevant also in the context of developing economies, even when property rights as such are an unstable basis on which to reason about the rights and obligations of individuals and communities. Specifically, I propose that the social function theory might offer a conceptual bridge between the study of economic development takings and the study of {\it land grabbings}, large-scale land acquisitions in the developing world. In both cases, the worry is often that local communities, who might lack formal title to the land, suffer as a result of a dramatic change in property's social function.

In Section \ref{sec:contrast}, I move on to consider the status of economic development takings in  English law. This also serves to introduce the topic of my thesis from the point of view of an important jurisdiction in Europe, where the issue of economic development takings has attracted far less attention than in the US. It appears to be gaining importance, however, as public-private partnerships and a market-oriented approach to public services has become influential in many jurisdictions, including in England.

In Section \ref{sec:echr}, I elaborate on a practically significant pan-European property clause, namely Article 1 of Protocol 1 (P1(1)) of the European Convention of Human Rights (ECHR). I argue that this clause provides an interesting perspective on the legitimacy issue, asking us to focus on the proportionality of the interference, judged relatively to its social and political context. I also consider some possible objections against the human rights approach, including the worry that the court in Strasbourg is not well-placed to be the arbiter of social and individual justice throughout Europe. At the same time, I point to some recent decisions at the Court that I believe signal hope that the case law on property is moving away from ill-conceived ``micro-management'', towards a more open-ended jurisprudence that seeks to force member states to address systemic problems that they might otherwise be reluctant or incapable of raising to the national agenda. Here the involvement of a (hopefully) politically neutral institution like the ECtHR can serve an important purpose, particularly if it manages to tailor its own case law in such a way as to leave room for local institutions of the member states to work out for themselves how to concretely resolve human rights issues flagged by the Court in Strasbourg.

In Section \ref{sec:us}, I return to the US setting, by presenting in detail how the perspective on economic development takings, mediated through case law on the public use restriction, has evolved since the 19th century until today. I structure the presentation as a story in two parts, describing the situation before and after the {\it Kelo} case. For the pre-{\it Kelo} presentation, I begin by pointing out that the case law on the public use restriction was initially developed by state courts, who would adjudicate legitimacy cases against the respective state constitutions (which typically also contain some sort of public use restriction on the takings power). 

The Supreme Court adopted deference to state {\it courts} initially, before changing their perspective by adopting a policy of deference directed rather at the state {\it legislature} (in practice also the administrative branch). I argue that this shift in Supreme Court jurisprudence can be pin-pointed to the case of {\it Berman}.\footnote{See \cite{berman54}.}

I go on to argue that this shift in case law at the federal level had the effect of destabilizing the established state approach to economic development takings, resulting in increased tension and controversy, paving the way to {\it Kelo}. In essence, my argument is that the Supreme Court was right in taking a deferential stance with respect to local institutions, but wrong in stripping the public use restriction of content, a move that threatened to undermine the authority of state courts. In effect, the federal takings jurisprudence threatened to weaken a very sensible {\it local} judicial constraint on executive power, a constraint that was also important to the proper division of power at the state level.

In Section \ref{sec:postkelo}, I follow this up by a discussion of developments after {\it Kelo}, which has seen a resurgence in state court scrutiny of the public use requirement, often backed up by state legislation that explicitly seeks to limit the scope of takings for economic development. According to some, such state reforms have been largely ineffective. %Ilya Somin, one of the most prolific writers on economic development takings in the US, has argued that this is partly due to so-called ``rational ignorance'' of political decision-makers and voters regarding the subtleties of the public use issue. The idea is that the distance between policy makers and communities affected by economic development takings is too great, so that policy makers have no incentive to consider the finer details of the takings equation. 
In principle, the US public is almost unanimously on the side of the local communities in cases like {\it Kelo}, but in practice, the great distance between political cause and effect makes effective reform policies hard to formulate. The danger is that reform proposals come to rely on oversimplified narratives tailored to centralised processes of decision-making.

In Section \ref{sec:ir}, I consider a proposal due to Heller and Hills that serves as a possible answer to this concern.\footnote{See \cite{heller08}.} This proposal focuses on the need for new frameworks for collective action, institutions that can replace the top-down dynamics of eminent domain in cases of economic development. The goal is to ensure a greater level of self-governance for the communities directly affected by the development, the individual members of which have a rational incentive to invest time and effort in reaching sophisticated compromises that can replace the use of black-white solutions (be it in the form of an economic development taking or a politically sanctioned top-down {\it ban} on such takings).

I argue that this idea embodies both a natural and necessary counterpart to increased judicial scrutiny of the public use restriction. In particular, I argue that the two ideas are mutually conducive to each other, when properly conceived. This argument will set the stage for the case study in the second part of the thesis, where I explore the tension between takings and self-governance in the context of hydropower development in Norway.

\section{The ``Underscrutinised'' Language of Economic Development}\label{sec:lgppp}

Economic development takings can be seen as a form of public-private partnership, whereby the state seeks to rely on for-profit takers and the market to fulfil some public purpose. Public-private partnerships are becoming increasingly important to the world economic order.\footnote{See generally \cite{saussier13}.} To some, they are the illegitimate children of privatisation and deregulation, while others see them as efforts to make the public sector more efficient and accountable. Either way, public-private partnerships are becoming more important, and they appear to be here to stay.\footnote{Although their potentially pernicious effects on stability and accountability has also been noted. See, e.g., \cite{baker03} (arguing that ``the Enron scandal can be better understood as an American form of public private partnership rather than just another example of capitalism run amok'').} In this situation, it is inevitable that when eminent domain is used to acquire property for economic development, those who directly benefit will often be commercial companies rather than public bodies. In the previous chapter, I pointed out how indirect public benefits are typically used to justify such takings. Standard legitimizing reasons include the prospect of new jobs, increased tax revenues, and various other economic and social ripple effects. 

Despite more or less convincing evidence of such benefits, economic development takings have a tendency to result in controversy. After {\it Kelo}, economic development takings have also been at the forefront of the constitutional property debate in the US. In the rest of the world, a similar shift in academic outlook has yet to take place, but expropriation-for-profit situations are increasingly coming into focus also on the global stage.\footnote{See, e.g., \cite{gray11,waring13,verstappen14}.} If we broaden our perspective even more, to consider commercially motivated interference in property on the global scale, it even seems appropriate to speak of a crisis of confidence in property law, particularly in relation to land rights. This is most clearly felt in the developing world, where egalitarian systems of property use and ownership are coming under increasing pressure. It has been noted, in particular, that large-scale commercial actors are assuming control over an increasing share of the world's land rights, a phenomenon known as {\it land grabbing}.\footnote{See generally \cite{borras11}.} 

So far, most research on land grabbing has looked at how commercial interests, often cooperating with nation states, exploit weaknesses of local property institutions, to acquire land voluntarily, or from those who lack formal title. However, the similarity between economic development takings and state-aided land grabbings in favour of large commercial companies is striking. 

In some cases, the two notions may coincide entirely. In India, for example, the scope of eminent domain has apparently become so wide that it allows for a ``complete assertion of power'' by the state.\footnote{See \cite[43]{cullet09}.} This state power, moreover, is often used to ``disempower people and redistribute rights and benefits'', often to the benefit of people who are already better-off than those negatively affected.\footnote{See \cite[33]{cullet09}.} Moreover, the language of eminent domain is apparently also invoked to justify controversial plans for the changed use of land that is not privately owned at all, but rather under forms of state ownership/custodianship.\footnote{See \cite[141]{mehta09}.} Embedding controversial policy choices in a takings narrative has become an effective strategy to silence opposition of all kinds, including that which pertains directly to the question of social and economic justice for the poor and the landless.\footnote{See \cite[143-144]{mehta09} (``the power of eminent domain has been interpreted as being close to absolute power of the State over all land and interests in land within its territory. The effect of this has been that those without access to land and rights over land (including the landless, artisans, women as a composite group), those who may have use rights but no titles, communities holding common rights and others with inchoate interests, have had to bear the burden heaved on to them by eminent domain.'')} 

More specifically, it has been noted how the purported public interest in economic development can be used to justify massive land grabs that would otherwise appear unjustifiable. In a recent article, Smita Narula cites {\it Kelo} directly and warns that procedural safeguards alone might not provide sufficient protection against abuse. She writes:

\begin{quote}
Procedural safeguards, however, can all too easily be co-opted by a state because its claims about what constitutes a public purpose may not be easy to contest. Particularly within the context of land investments, states could use the very general and under-scrutinized language of ``economic development'' to justify takings in the public interest.\footcite[157]{narula13}
\end{quote}

This quote underscores the broader relevance of the study of economic development takings. In addition, it asks us to keep in mind that the question of what can be justified in the name of ``economic development'' is a general one, not confined to particular systems for organizing property rights. To address this, and to restore confidence in the institution of property more generally, some academics and policy makers have proposed a novel concept of property as a human right.\footnote{See generally \cite{schutter10,schutter11,kunnerman13}.} It has been argued, in particular, that a human right to land should be \isr{recognised} on the international stage, a right that would apply even when those affected by a land grab lack formal title. If successful, this approach promises to deliver basic protection against interference in established patterns of property use independently of how particular jurisdictions approach property.

In Europe, the human right to property is still usually understood in more conventional terms, as pertaining primarily to the rights of formally titled owners. However, a broad, social-function perspective on this right is influential due to the ECHR and the court in Strasbourg.\footnote{As discussed in Chapter \ref{chap:1}, Section \ref{sec:3}.} The issue of land grabbing highlights the importance of maintaining such a perspective, particularly when attempting to use western legal categories when analysing the developing world. In the context of land grabbing, protecting land rights is not primarily a question of protecting the civil law ideal of individual dominion. Rather, it is a question of providing protection against large-scale transactions that \isr{destabilise} or destroy established patterns of land use, to the detriment of local communities. 

In human rights discourse, particularly relating to the developing world, the focus is often on pressing problems related to food and water security as well as the protection of basic livelihoods, issues that can arise with particular urgency in the context of land grabbing. However, to achieve effective protection we need firm categories and enforcible legal principles to back up our benchmarks and our good intentions. In this regard, I think Narula is right to stress that the lack of a convincing approach to the notion of ``economic development'' is a crucial challenge.

As an overarching goal, economic development is no doubt sound, particularly for poor nations. The problem is that the risk of abuse is great when such a vague term is used to justify dramatic interferences in property. Such interferences typically cause severe disturbances in people's lives. This, moreover, is true for a middle-class US homeowner in much the same way as it is true for a self-sustaining farmer in Africa, or a landless artisan in India, although the stakes might be very different. Hence, there seems to be great potential for exchange of ideas and insight between those working on economic development takings and those studying land grabs in the developing world.

\noo{ As illustrated by {\it Kelo}, deep conflicts can arise in this regard also in developed democracies with long established and relatively stable systems of private property. In the following, I will attempt to shed further light on the issue as it arises in such legal systems, without considering the additional complications that arise when property itself is a more fragile concept. I note, however, that according to the social function view of property, there is no need to view formally \isr{recognised} property rights as completely distinct from rights arising from property use that is not based on formal title. The two are intertwined and the difference between them is at most a matter of degree.\footnote{Moreover, if the human flourishing account of property values is successfully developed, there should even be hope that a unified normative treatment can be given at some point.}}

In this thesis, I focus on legal systems where private property is well-established and relatively stable as a legal category. Moreover, my case study will look to Norway, a prosperous European country with a long tradition of an egalitarian distribution of land rights among the rural population. Hence, I will focus on situations when those affected by takings of land for economic development have a {\it prima facie} cause for objecting on the basis of recognised property rights. Therefore, the complications that occur when those most severely affected do not have formally recognised property rights will not be considered in any depth. However, I believe this a very interesting avenue for future work.

In the following, I will present a comparative background for my case study. I will begin by considering English law, where courts have generally been reluctant to broadly scrutinize the use of economic development as a justification for state interference in property. After this, I turn to the ECHR and the proportionality test that is now at the core of property adjudication at the ECtHR. I note that while states are considered to have a wide margin of appreciation with regards to the legitimacy of the purpose underlying interference, the balancing required under the proportionality test can still become a powerful basis on which to scrutinize the broader negative effects of economic development takings.

Following this, I move on to consider the US in greater depth, both the historical debate that led to {\it Kelo} and the suggestions for reform that have emerged following its backlash. There has been much written about this issue in the US. Moreover, while much of it is repetitive and coloured by the tense political climate, I believe some historical points, as well as some recent suggestions for reform, are highly relevant also to the international setting. To single out and analyse those aspects is the main aim of this part of the chapter. Indeed, the current debating climate in the US might be an indication of what is to come also in Europe, if concerns about the legitimacy of economic development takings are not taken seriously.

%I also highlight what I believe to be a connection between the situation in the US leading up to {\it Kelo} and the present situation in Europe, illustrated by the fact that the European Court of Human Rights is now explicitly endorsing ``stronger protection'' of property rights.  I attempt to identify the reasons behind calls for a stricter approach, arguing that it is connected to the fact that interferences in property under modern regulatory regimes is sanctioned in wide a range of different circumstances, serving to undermine their status as a necessary burden imposed on owner's according to the will of the greater public. In some cases, rather, takings appear to both owners and the public as improperly motivated and socially and politically unfair. I note that this happens particularly often in economic development cases, when commercial actors benefit to the detriment of local communities. I go on to list some concrete issues that arise with respect to such takings and that have been flagged as problematic in the literature.
%
%Following up on this, I consider various proposals that have been made to resolve tensions and limit the possibility of abuse in economic development cases. The differences of opinion that have been expressed in this regard have been quite substantial, and proposals have ranged from suggesting an outright ban on economic development takings  (Somin 2007; Cohen 2006) to suggesting that the best way forward is to reassess principles for awarding compensation in such cases (Householder 2007; Lehavi and Licht 2007).

%Much of the current theory focus on assessing traditional judicial safeguards that courts can rely on to prevent abuses, pertaining primarily to the material assessment of proportionality, public purpose, and compensation. 

%In the last part of the chapter, I will focus on a very interesting strand of recent work in the US, which shifts attention towards procedural rules that can help address the worry that economic development takings tend to suffer from a democratic deficit. The core concern is that the manner in which eminent domain decisions are typically made, and the way in which owners are compensated, might be unsuitable for economic development cases. Importantly, the need for special procedures has been noted, to restore legitimacy.\footnote{See generally \cite{lehavi07,heller08}.} This ties the US debate even closer to the European context, where proportionality, not public use, has become the key notion in property protection. Several recent suggestions from the US can be conceptualized as suggestions that aim to secure fairness and proportionality, while paying less attention to the formalistic question of what constitutes a ``public use''.
%
%%Also, it allows us to be very clear about a special concern that arises for economic takings cases: under current regulatory regimes, the government and the developer together often dominate the decision-making process completely, leaving the property owners marginalized. Hence, there is often a {\it democratic deficit} in such cases, resulting in discontent and a feeling that the taking is not in the public interest at all. Importantly, some recent writers hypothesize that if the proper balance can be restored in the decision-making process, so will the decision reached appear more legitimate, also with respect to the public use clause. In my opinion, this idea is crucial, and together with the question of compensation, which raises a similar structural problem, it will guide the rest of the work done in this thesis. 
%

In response to that worry, this chapter aims to bring into focus the key question of how to ensure meaningful participation for owners and their local communities in decision-making pertaining to economic development on their land. The tentative answers provided in Section \ref{sec:ir} will set the stage for the remainder of the thesis, where these answers will be assessed in depth against the case study of Norwegian hydropower.

%In particular, I will consider two special semi-judicial procedural systems used in such cases in Norway, one targeting compensation following expropriation, and another used as an alternative to expropriation, particularly in cases when development requires cooperation among many owners.

%I conclude by arguing that approaches along procedural lines represent the best way forward in relation to addressing issues associated with economic development takings. This raises the following problem, however: what procedural principles can be used to ensure meaningful participation, without hindering socially and economically desirable development projects? This question sets the stage for the remainder of my thesis, where I conduct a case study of expropriation for the development of hydro-power in Norway. In particular, I will consider two special semi-judicial procedural systems used in such cases in Norway, one targeting compensation following expropriation, and another used as an alternative to expropriation, particularly in cases when development requires cooperation among many owners.

\section{Economic Development Takings in England}\label{sec:contrast}

Economic development takings have not become as controversial in Europe as they are in the US, but there have been cases where the issue has come up, in several different jurisdictions.\footnote{For instance, in the UK, Ireland and Germany, as well as in Norway and Sweden. See \cite[466-483]{walt11}; \cite{stenseth10}.} \noo{ The P1(1) of the ECHR protects property, but the legitimacy of economic development takings has not yet been discussed in case law from the European Court of Human Rights (ECtHR). However, it is interesting to analyse cases like {\it Kelo} against P1(1), particularly since the ECtHR has developed a doctrine that focuses on ``proportionality'' and ``fairness'' rather than the purpose of interference.\footnote{See generally, \cite[Chapter 5]{allen05}. This approach may become even more significant as a source of property protection in the future, as the ECtHR have indicated that there are ``jurisprudential developments in the direction of a stronger protection under Article 1 of Protocol No. 1'', see \cite[135]{lindheim12}.}}

In this section, I address economic development takings from the point of view of English law.\noo{ I then go on to give a more detailed presentation of the unifying property clause in P1(1). The case law from the ECtHR is presented and analysed in some depth, in an effort to assess how the ECtHR would be likely to approach an economic development case such as {\it Kelo}. In particular, I argue that the proportionality doctrine offers an interesting approach to such cases. Importantly, the doctrine stipulates that a ``fair balance'' must be struck  between the interests of the property owner and the public.\footcite[Chapter 5]{allen05} I argue that such a perspective could make it easier to get to the heart of why economic development takings are often seen as problematic, without getting lost in theoretical discussions about the meaning of  terms like ``public use'' or ``public purpose''. However, I also raise the concern that the ECtHR is not the appropriate institution for applying the proportionality test. Indeed, its remoteness to most of Europe suggests that we should look for more locally grounded legitimacy-enhancing institutions. Such institutions will likely be better able to assess the fairness of interference in context.

I go on to discuss whether existing government institutions can serve this purpose, arguing that local courts may well be the best candidates. However, I also note that active application of the proportionality test in property cases might not be found at the local level. In this regard, the ECtHR could play a crucial role, by focusing on the systemic question of what issues local courts need to consider when assessing legitimacy of property interference. 

However, quite apart from this, there is reason to worry that judicial bodies are not ideally suited to carry out the kind of assessment that is required. Hence, new institutional proposals might be in order. I conclude by arguing that once the need for local grounding is recognised and met, the ECtHR has the potential to play an important and constructive role in providing oversight and developing basic principles, also with respect to new institutions that aim to deliver increased legitimacy at the local level.

\subsection{England}\label{sec:england}
}
In England, the principle of parliamentary supremacy and the lack of a written constitutional property clause has led to expropriation being discussed mostly as a matter of administrative law and property law, not as a constitutional issue.\footcite{taggart98} Moreover, the use of compulsory purchase -- the term most often used to denote takings in the UK -- has not been restricted to particular purposes as a matter of principle. The uses that can warrant compulsory alienation of property are those that parliament regard as worthy of such consideration. However, as private property itself has long been recognised as a fundamental right, the power of compulsory purchase has typically been exercised with caution. 

In his {\it Commentaries on English Law}, William Blackstone famously described property as the ``third absolute right'' that was ``inherent in every Englishman''.\footnote{See \cite[134-135]{blackstone79}. The first right, according to Blackstone, is security, while the second is liberty.}  Moreover, Blackstone expressed a very restrictive view on the possibility of expropriation, arguing that it was only the legislature that could legitimately interfere with property rights. He warned against the dangers of allowing private individuals, or even public tribunals, to be the judge of whether or not the ``common good'' could justify takings. Blackstone went as far as to say that the public good was ``in nothing more invested'' than the protection of private property.\footcite[134-135]{blackstone79}

Historically, Blackstone's description conveys a largely accurate impression of takings practice in England. Indeed, Parliament itself would usually be the granting authority in expropriation cases, through so-called {\it private Acts}. Hence, compulsory purchase would not take place unless it had been discussed at the highest level of government. Moreover, the procedure followed by parliament in such cases strongly resembled a judicial procedure; the interested parties were given an opportunity to present their case to parliament committees that would then decide whether or not compulsion was warranted.\footnote{See \cite[13-16]{allen00}. While this procedure reflected a protective attitude towards private property, recent scholarship has also pointed out that expropriation was in fact used very actively in Britain, particularly following the glorious revolution, see \cite{hoppit11}.}

On the one hand, the direct involvement of parliament in the decision-making process reflected a fundamental respect for property rights. But at the same time, parliamentary supremacy also meant that the question of legitimacy was rendered mute as soon as compulsory purchase powers had been granted. The courts were not in a position to scrutinize takings at all, much less second-guess parliament as to whether or not a taking was for a legitimate purpose.

During the 19th Century, as an industrial economy developed, private acts granting compulsory purchase powers to commercial companies grew massively in scope and importance.\footnote{See \cite[204]{allen00}.} Private railway companies, in particular, regularly benefited from such acts.\footnote{\cite[204]{allen00}. See generally \cite{kostal97}.} During this time, the expanding scope of private-to-private transfers for economic development led to high-level political debate and controversy. Usually, it would attract particular opposition from the House of Lords. Interestingly, this opposition was not only based on a desire to protect individual property owners. It also often reflected concerns about the cultural and social consequences of changed patterns of land use.\footcite[204]{allen00} 

Hence, the early {\it political} debate on economic development takings in the UK shows some reflection of a social function approach to property protection. At the same time, as society changed following increasing industrialisation, an expansive approach to compulsory purchase would eventually emerge as the norm.\footnote{Arguably, the social function perspective is the key to understanding why this happened. Indeed, the expanded use of private takings in England during the 19th century, particularly in connection with the railways, might have served a more easily justifiable social function than that commonly associated with economic development takings today. Waring, in particular, notes how railway takings tended to affect aristocratic landowners rather than marginalised groups (``unlike private takings today, the railway legislation was most likely to affect those who could best defend their property rights from attack''), see \cite[111]{waring09}.} The idea that economic development could justify takings gradually became less controversial.

Today, the law on compulsory purchase in England is regulated in statute and the role of courts is to a large extent limited to the application and interpretation of statutory rules. Some common law rules still play an important role, such as the {\it Pointe Gourde} rule, which stipulates that changes in value due to the compensation scheme itself should be disregarded when calculating compensation to the owner.\footnote{The rule takes it name the case of \cite{gourde47}. The underlying principle, including also statutory regulations with a similar effect, is referred to as the ``no scheme'' principle, see \cite{lawcom01}. The principle is found in many jurisdictions, see \cite{sluysmans14}. The principle is often quite contentious, and notoriously hard to apply in practice. For a recent attempt at clarifying the principle, see \cite{waters04}. I note that a strict interpretation of the no-scheme principle effectively precludes benefit sharing between takers and owners, a phenomenon that is of particularly relevance in the context of economic development takings. I will not address this particular issue in any depth here -- I choose instead to focus on legitimacy of takings in a broader, non-compensatory sense. However, the compensation aspect of economic development takings is also very interesting (and challenging). For further details, I refer to \cite{dyrkolbotn15}.} With respect to the question of legitimacy, however, the starting point for English courts is that this is a matter of ordinary administrative law.\footnote{See \cite{taggart98}.}

More recently, the \cite{hra98} adds to this picture, since it incorporates the property clause in P1(1) into English law. Even so, the usual approach in England is to judge objections against compulsory purchase orders on the basis of the statutes that warrant them, rather than constitutional principles or human rights provisions that protect property.\footnote{See \cite[121-132]{waring09}. The important statutes are the \cite{ala81}, the \cite{lca61}, the \cite{cpa65}, the \cite{tcpa90} and the \cite{pcpa04}.} It is typical for statutory authorities to include standard reservations to the effect that some public benefit must be identified in order to justify a compulsory purchase order, but the scope of what constitutes a legitimate purpose can be very wide. For instance, to warrant a taking under the \cite{tcpa90}, it is enough that it will ``facilitate the carrying out of development, redevelopment or improvement on or in relation to the land''.\footcite[226]{tcpa90} 

While various governmental bodies are authorised to issue compulsory purchase orders (CPOs), a CPO typically has to be confirmed by a government minister.\footnote{See \cite[48]{waring09}.} The affected owners are given a chance to comment, and if there are objections, a public inquiry is typically held. The inspector responsible for the inquiry then reports to the relevant government minister, who makes the final decision about whether or not it should be granted, and on what terms. The CPO may then be challenged in court, but will usually only be scrutinized on the basis of whether or not it lies within the scope of the statute authorising it, not on the basis of whether or not the purpose of the taking appears to be legitimate as such.\footnote{See, e.g., \cite[48-49]{waring09}.}

That said, the idea that property may only be compulsorily acquired when the public stands to benefit permeates the system. Indeed, this has also been regarded as a constitutional principle, for instance by Lord Denning in {\it Prest v Secretary of State for Wales}.\footcite{prest82} He said:

\begin{quote}
It is clear that no minister or public authority can acquire any land compulsorily except the power to do so be given by Parliament: and Parliament only grants it, or should only grant it, when it is necessary in the public interest. In any case, therefore, where the scales are evenly balanced – for or against compulsory acquisition – the decision – by whomsoever it is made – should come down against compulsory acquisition. I regard it as a principle of our constitutional law that no citizen is to be deprived of his land by any public authority against his will, unless it is expressly authorised by Parliament and the public interest decisively so demands. If there is any reasonable doubt on the matter, the balance must be resolved in favour of the citizen.\footcite[198]{prest82}
\end{quote}

Lord Denning also supported the doctrine of necessity, as expressed by Forbes J in {\it Brown v Secretary for the Environment}:\footcite{brown78}

\begin{quote}It seems to me that there is a very long and respectable tradition for the view that an authority that seeks to dispossess a citizen of his land must do so by showing that it is necessary, in order to exercise the powers for the purposes of the Act under which the compulsory purchase order is made, that the acquiring authority should have authorisation to acquire the land in question.\footcite[291]{brown78}
\end{quote}

In practice, these principles are mostly implicit in legal reasoning, as a factor that influences the courts when they interpret statutory rules and carry out judicial review of administrative decisions. As Watkins LJ stated in {\it Prest}:

\begin{quote}
The taking of a person's land against his will is a serious invasion of his proprietary rights. The use of statutory authority for the destruction of those rights requires to be most carefully scrutinised. The courts must be vigilant to see to it that that authority is not abused. It must not be used unless it is clear that the Secretary of State has allowed those rights to be violated by a decision based upon the right legal principles, adequate evidence and proper consideration of the factor which sways his mind into confirmation of the order sought.\footcite[211-212]{prest82}
\end{quote}

In {\it R v Secretary of State for Transport, ex p de Rothschild}, Slade LJ referred to {\it Prest} and made clear that he did not regard it as expressing a rule concerning the burden of proof in compulsory purchase cases. Rather, he took it as more general observation on the severity of property interference and the importance of vigilance in such cases.\footcite{rothschild89} He pointed to ``a warning that, in cases where a compulsory purchase order is under challenge, the draconian nature of the order will itself render it more vulnerable to successful challenge''.\footcite[938]{rothschild89}

\subsection{{\it Sainsbury's Supermarkets Ltd v Wolverhampton City Council}}

An illustration of how English courts approach objections to the legitimacy of takings is found in the recent case of {\it Regina (Sainsbury’s Supermarkets Ltd) v Wolverhampton City Council}.\footcite{sainsbury10} Here a CPO was granted to allow the company Tesco to acquire land from its competitor Sainsbury, in a situation when they were both competing for licenses to undertake commercial development on the same land, owned partly by both. 
The decisive factor that had led the local authorities to grant the CPO was that Tesco had offered to develop a different property in the same local area, which was currently in need of regeneration.

Sainsbury protested, arguing that the local council could not strike such a deal on the use of its compulsory purchase power. It was argued, moreover, that taking the land for incidental benefits resulting from development in a different part of town was not legitimate under the Town and Country Planning Act 1990. The UK Supreme Court agreed 4-3, with Lord Walker in particular emphasising the need for heightened judicial scrutiny in cases of private-to-private takings for economic development.\footcite[80-84]{sainsbury10} Lord Walker even cited {\it Kelo}, to further substantiate the need for a stricter standard in such cases.\footcite[81]{sainsbury10} 

However, the main line of reasoning adopted by the majority was based on an interpretation of the Town and Country Planning Act itself. In particular, the majority held that it was improper for the local council to take into consideration the development that Tesco had committed itself to carry out on a different site.\footcite[73-79]{sainsbury10} This, in particular, was not ``improvement on or in relation to the land'', as required by the Act.\footcite[336]{tcpa90} In addition, Lord Collins, who led the majority, said that ``the question of what is a material (or relevant) consideration is a question of law, but the weight to be given to it is a matter for the decision maker''.\footcite[70]{sainsbury10} These comments reflect the traditional approach to judicial review of CPOs under English law, demonstrating how the underlying statutory authority tends to be at the center of attention.

However, it is interesting to see how the purpose of the interference featured in the background of the Supreme Court's interpretation and application of the statutory rule. The opinion of Lord Walker is particularly interesting, since he stresses that ``the land is to end up, not in public ownership and used for public purposes, but in private ownership and used for a variety of purposes, mainly retail and residential.''\footcite[81]{sainsbury10} He goes on to state that ``economic regeneration brought about by urban redevelopment is no doubt a public good, but ``private to private'' acquisitions by compulsory purchase may also produce large profits for powerful business interests, and courts rightly regard them as particularly sensitive.``\footcite[81]{sainsbury10}

Lord Walker then makes clear that he does not think it is impermissible, as such, for the local council to take into account positive effects on the local area, even when these do not directly result from the planned use of the land that is being acquired. Instead, he relies explicitly on the for-profit character of the taking, by arguing that ``the exercise of powers of compulsory acquisition, especially in a ``private to private'' acquisition, amounts to a serious invasion of the current owner's proprietary rights. The local authority has a direct financial interest in the matter, and not merely a general interest (as local planning authority) in the betterment and well-being of its area. A stricter approach is therefore called for.''\footcite[84]{sainsbury10} 

Lord Walker's opinion might indicate that the narrative of economic development takings is about to find its way into English case law. Moreover, a more critical approach might be adopted in the future, when compulsory purchase powers are made available to commercial companies wishing to undertake for-profit schemes. However, for schemes where the commercial aspect appears less dominant, English courts still appear very reluctant to quash CPOs, also when the purpose is economic development. This is so even in situations when the owners have requested a stricter standard of review on the basis of human rights law.\footnote{See \cite{smith08,alliance06}. See also the commentary in \cite[]{gray11}.}

\noo{ \subsection{Some Other Cases from England} %{\it Smith \& Others v Secretary of Stare for Trade and Industry}}

In the case of {\it Smith \& Others v Secretary of State for Trade and Industry}, a caravan site was compulsorily acquired for development in connection with the London Olympic Games.\footcite{smith08} Some of the owners protested, including Romany Gypsies who used the caravans as their primary residence. A public inquiry was held, after which the inspector recommended that the CPO should not be confirmed until adequate relocation sites had been identified. However, due to the ``urgency, timing and importance'' of the project, the Secretary of State decided to go ahead before a relocation scheme was put in place (although he expressed commitment to ensuring satisfactory relocation).\footcite[10]{smith08} The owners argued that without satisfactory relocation plans, the interference in the property rights was not proportional and had to be struck down on the basis of human rights law, in particular Article 8 in the ECHR regarding respect for the home and private life.\footcite[27-51]{smith08}

The Court of Appeal considered the matter in great depth, applying the doctrine of proportionality developed at the ECtHR. Importantly, this doctrine was understood to go beyond the standard form of judicial review required under English law. However, the Court still concluded that the taking was proportional. This was largely based on the finding that ``the issue of proportionality has to be judged against the background that everyone accepts that an overwhelming case has been made out for compulsory acquisition of the sites for the stated objectives and that compulsory purchase is justified.''\footcite[42]{smith08} 

Justice Williams arrived at this conclusion after noting that the owners' {\it only} substantial objection against the CPO was that it was confirmed before adequate relocation measures had been agreed on.\footcite[42]{smith08} Hence, the question, as he saw it, did not concern the validity of using compulsory purchase powers, but merely the timing with which it had been ordered. On this basis, he framed the question of legitimacy as one relating to the ``necessity'' standard, according to which an infringement of Convention rights is only permissible when the public interest cannot be served in some other way.\footcite[43]{smith08} A strict reading of this standard holds that an interference must be the {\it least intrusive means} of achieving the stated aim.\footnote{Such a standard has been adopted in some Convention cases, for instance in \cite{samaroo01}.}

Justice Williams argued against such a strict reading, subscribing instead to a view expressed as an {\it obiter} in the case of {\it Pascoe v The First Secretary of State}. According to this view, an interference need not be the least intrusive means. Rather, it is sufficient that the measure is ``reasonably necessary'' to achieve that aim.\footnote{See \cite[74-75]{pascoe06} (quoting \cite[25]{clay04}).} However, while noting his agreement with this approach, Justice Williams went on to also apply the stronger necessity test, and found that even if this was applied the CPO in question would still be a proportional interference.\footcite[41-50]{smith08}

It seems clear that while the taking in question was for economic and recreational development purposes, the case was marked by a preliminary finding to the effect that the legitimacy of the aim of interference -- to facilitate the London Olympics -- was beyond reproach. Hence, there was no need for, or even room for, more detailed purposive reasoning of the kind that would later be applied by Lord Walker in {\it Sainsbury}. The fact that the taking was for economic development and recreation, not for a pressing public need, was not considered relevant. Moreover, since the case was construed to be solely about the extent to which the CPO was ``necessary'' to further its stated aim, the proportionality test that was carried out, despite being detailed, was very narrow in scope. It concerned only proportionality of the means, not of the aim itself. The question of how to weigh the public interest in a multi-billion dollar sporting event against the security of someone's home was not considered.

In later cases, a dismissive attitude towards substantive review has been adopted even in situations when the owners have argued against takings by explicitly questioning the proportionality of the interference against the importance of the aim. 

%\subsection{{\it Alliance Spring Co Ltd v The First Secretary of State}}

In the case of {\it Alliance Spring Co Ltd v The First Secretary of State}, a large number of properties were expropriated to build a new football stadium for the football club Arsenal.\footcite{alliance06} Some owners who stood to lose their business premises protested, pointing to the fact that the inspector in charge of the public inquiry had recommended against the takings.\footcite[6-7]{alliance06} According to Justice Collins, the main argument that the owners relied on when protesting the taking was that it did not serve a ``proper purpose''.\footcite[19]{alliance06} This argument was not held to be valid, however, with Justice Collins concluding as follows: 

\begin{quote}
There is nothing in the material put before and accepted by the Inspector which persuades me that that decision was ill founded or was one which the Secretary of State was not entitled to reach. Developments which result in regeneration of an area are often led by private enterprise. Mr Horton perforce accepts that that is so, but submits that this is not the sort of situation where, for example, a private development is the anchor for a particular scheme. I disagree.\footcite[19]{alliance06}
\end{quote}

Hence, unlike the case of {\it Smith}, where the Court did in fact carry out its own assessment of proportionality, the {\it Alliance} Court was content with deferring to the assessment carried out by the executive branch.\footnote{This has been criticized, e.g., by Kevin Grey who describes the reference to Convention Rights in Alliance as ``worryingly brief''. See \cite{gray11}.} As such, the case appears to follow the pattern of judicial review of CPOs established before the Human Rights Act 1998. This means that the decision also contrasts with how English courts have approach the Convention in relation to other  rights, such as those of Article 8 addressed in {\it Smith}.

Whether the approach taken in {\it Alliance} is good law after {\it Sainsbury} is unclear; judging from Lord Walker's opinion, it seems that a more substantive assessment might be required for similar cases in the future. While this might not imply a different outcome for a case like {\it Alliance}, it would mean that courts would have to engage in independent review of the purpose and merits of contested CPOs that benefit commercial actors. In particular, English courts would have to change the way they approach such cases, by being better prepared to assess for themselves whether a fair balance is struck between the interests of the developer and the property owners. Hence, it is not unlikely that the category of economic development takings will become an important point of reference in the future, both for the law and those who study it. }

\noo{ \subsection{Germany}\label{sec:germany}

In German law we find an explicit constitutional property clause. In particular, Article 14 of the Basic Law ({\it Grundgesetz}) reads as follows:

\begin{quote}
(1) Property and the right of inheritance shall be guaranteed. Their content and limits shall be defined by the laws. \\
(2) Property entails obligations. Its use shall also serve the public good. \\
(3) Expropriation shall only be permissible for the public good. It may only be ordered by or pursuant to a law that determines the nature and extent of compensation. Such compensation shall be determined by establishing an equitable balance between the public interest and the interests of those affected. In case of dispute concerning the amount of compensation, recourse may be had to the ordinary courts.\footcite[14]{basic49}
\end{quote}

Apart from the fact that the property clause is explicit, I note two further characteristic features of the protection of property in Germany. First, the constitution explicitly stresses that property comes with social obligations as well as rights. The use of property should ``serve the public good''. On the other hand, it is also made clear that expropriation is only permissible when it is ``for the public good''. Hence, it follows immediately that the purpose of expropriation is a relevant factor when determining the legitimacy of a taking, \isr{irrespective} of the specific statute used to authorise it. Importantly, it is clear already from the outset that the question of legitimacy is a \emph{judicial} question, one which the courts can only answer if they form an opinion about that constitutes the ``public good''. 

This means that it is quite natural to approach the question of economic development takings from the point of view of constitutional law. Unlike in England, disputes over the legitimacy of such takings can be comfortably adjudicated directly against a ``public good'' restriction. While this sets Germany apart on the theoretical level, it is unclear how much of an effect it has had in practice. To shed some light on this question, we can look to the two major authorities on the legitimacy of economic development takings, the cases of {\it D\"{u}rkheimer Gondelbahn} and {\it Boxberg}.\footcite{durkheimer81,boxberg86} 

In both cases, the German Constitutional court found that expropriation to the benefit of commercial interests was illegitimate. However, the Court argued for this result on the basis that there was insufficient statutory authority for such takings in the concrete circumstances complained of. That is, the Court did not directly address the question of whether the relevant statutes were compliant with Article 14 of the basic law. Instead, they interpreted statutory authorities on the assumption that they had to be, following a pattern of reasoning that appears to be rather close to the approach followed by English courts in similar cases.\footnote{Although in {\it Dürkheimer Gondelbahn}, Böhmer J gave a separate concurring judgment where he argued for this result on the basis of the public good requirement of the basic law.} It seems, in particular, that even in Germany, the public purpose restriction is primarily relevant as a factor guiding the interpretation of statutory authorities.

That said, the cases of {\it D{\"u}rkheimer Gondelbahn} and {\it Boxberg} show that in situations when the public purpose of a taking is unclear, German courts seem inclined to \isr{favour} a narrow interpretation of the relevant statute. In {\it Bloxberg}, several properties were expropriated \isr{in favour} of the car company Daimler Benz AG, for commercial purposes. The affected local communities suffered from high unemployment rates and a slow economy, so a {\it prima facie} reasonable \isr{case} could be made that allowing Daimler to acquire the land was in the public interest, as it would facilitate economic growth. However, the Federal Constitutional Court agreed with the owners that the expropriation was invalid. This, it held, was because the taking was outside the scope of the relevant statute, which authorised expropriations for ``planning purposes''. The owners had argued extensively using Article 14 of the Basic Law and the constitutional ``public good'' restriction clearly did play a role in the Court's reasoning. But at the same time, the Court stressed that private-to-private transfers that bestow financial benefit on the acquiring party may well satisfy the ``public good'' requirement. The important issue was whether a sufficiently strong public interest could be identified, \isr{irrespective} of any windfall benefits that might fall on private parties.

In light of this, I think it is wrong to exaggerate the importance of the explicit formulation of the public use test offered in the German constitution. Its importance seems to rest mainly in the fact that it provides a particularly authoritative expression guiding the national courts' application of statutory provisions regarding expropriation of property. But developments in common law, where the public use requirement is stressed as a guiding constitutional principle, might well point in the same direction. In principle, both German and English Courts are in a good position to respond to increased tension regarding economic development takings by developing a stricter standard of judicial review in such cases.

A different aspect of German law deserves special attention, however, since it does not appear to have any clear counterpart in the common law tradition. This is the  ``social-obligation'' norm in Article 14 (2), which points to a different \isr{conceptualisation} of property rights as such. As argued by Alexander, the distinguishing feature of the property clause in the German Constitution is that the value of property is thought to relate more strongly to its importance for human dignity and flourishing in a social context, rather than the protection of individual financial entitlements. As Alexander notes regarding the Germans' own \isr{conceptualisation} of their property clause:

\begin{quote}
This theory holds that the core purpose of property is not wealth maximization or the satisfaction of individual preferences, as the American economic theory of property holds, but self-realization, or self-development, in an objective, distinctly moral and civic sense. That is, property is fundamental insofar as it is necessary for individuals to develop fully both
as moral agents and participating members of the broader community.\footcite[745]{alexander03}
\end{quote}

With such a starting point, it is not surprising that in cases such as {\it Boxberg}, resembling {\it Kelo}, German Courts will tend to adopt a strict view on legitimacy. These are cases when the property rights infringed on serve a fundamentally different function for the two opposing private parties. To the owner, the property is a home, an important source of self-identity, autonomy, security and membership in a community. To the taker, it represents an obstacle to commercial development which needs to be removed. In such a situation, it is in keeping with the spirit of the social-obligation norm of property to offer enhanced protection to the homeowner. To this owner, the property serves a purpose which is fundamentally different, and arguably more worthy of protection, then the property's purpose for the developer. A taking in this situation might therefore, because of Article 14, require a particularly clear and strong public interest.

But unless there is an asymmetry between owner and taker, heightened scrutiny does not necessarily follow. Hence, it is interesting to speculate what German courts would have made of a case such as {\it Regina (Sainsbury’s Supermarkets Ltd) v Wolverhampton City Council}. Here, the interests of owner and taker were strictly commercial nature. Both owned part of the contested land and neither one could develop the land according to their plans without buying out the other. The enhanced protection of property offered under German law would probably not have much significance in such a case. 

In fact, it might well be that German courts would be {\it more} likely to accept such a taking. First, their \isr{conceptualisation} of property rights appears to allow greater flexibility to adapt the level of protection to the circumstances and the purposes of the property in question. So even if is correct that private-to-private transfers for commercial projects require a ``stricter approach'' in general, as argued by Lord Walker in \textcite{sainsbury10}, the fact that the interests of the owner were also purely commercial  might make this less relevant. Second, German courts might be more inclined to have regard to socially beneficial additional commitments entered into by the applicant, even if they do not concern the property that is taken. As a tie-breaker, looking to such commitments might be as good an approach as any other.\footnote{This was the view taken by the dissenting minority in \textcite{sainsbury10}.}

Of course, objections could still be raised on the basis of general administrative law. Indeed, some might see the case as an example of government ``auctioning'' off licenses to the highest bidder. This might well be regarded as an affront to good governance. I will not delve into German law to assess the case from this perspective. My point is simply that because of the purposive and contextual nature of Article 14, it seems unlikely that a case like \textcite{sainsbury10} would turn on constitutional property law.

To sum up, German constitutional law serves to create an interesting contrast with English law regarding the question of economic development takings. On the one hand, property appears to be better protected against such takings in Germany, but on the other hand, the extent to which increased protection is offered depends more closely on the social values involved. The German system appears to look more actively at the social function of property for guidance when resolving property disputes, thereby echoing some of the ideas discussed in Chapter \ref{chap:1}. 

In the next section, I will discuss the property clause in the ECHR, which explicitly serves to set up a minimum level of property protection that provides a common standard for all member states, including Germany and the UK.
}

\section{The Property Clause in the European Convention of Human Rights}\label{sec:echr}

The standard account of the protection against interference inherent in P1(1) describes it as consisting of three rules.\footnote{For a more detailed description of P1(1) generally, I refer to \cite{allen05}.} First, there is the rule of {\it legality}, asserting that an interference needs to be authorized by statute. Second, there is the rule of {\it legitimacy}, making clear that interference should only take place in pursuance of a legitimate public purpose. The third rule is the ``fair balance'' principle, requiring proportionality between the means and the aims in cases involving property interference.\footnote{See \cite[69]{sporrong82} and \cite[120]{james86}.} %which is applied by the ECtHR in almost all cases when it finds that there has been a violation of P1(1).

The starting point for property adjudication at the ECtHR is that States have a ``wide margin of appreciation'' with regard to the legitimacy question.\footcite[See][54]{james86} This question is thought to depend on democratically determined policies to such an extent that it is rarely appropriate for the Court to censor the assessments made by member states. At the same time, the Court has gradually adopted a more active role in assessing whether or not particular instances of interference are proportional and able to strike a fair balance between the interests of the public and the property owners. As argued by Allen, this has caused P1(1) to attain a wider scope than what was originally intended by the signatories.\footcite[1055]{allen10}

In the early case law behind this development, the focus was predominantly on the issue of compensation, with the Court gradually developing the principle that while P1(1) does not entitle owners to full compensation in all cases of interference, the fair balance will likely be upset unless at least some compensation is paid, based on the market value of the property in question.\footnote{See \cite[103]{scordino06}. The case also illustrates that the Court has adopted a fairly strict approach to the question of when it is legitimate to award less than full market value.} %The focus on compensation has also been reflected in academic work on P1(1), which tends to address proportionality from a financial perspective, by investigating to what extent owners are entitled to compensation based on the market value of their property. Indeed, when considering case law and literature on the subject, one is left with the impression that ``fair balance'' with regards to P1(1) is crucially linked to financial entitlements. It seems that d as a standard that can justify a right to compensation that goes beyond what the wording of P1(1) might initially suggest.

As mentioned in Section \ref{sec:x} of Chapter \ref{chap:1}, it has now become clear that the fair balance test encompasses more than this. In particular, the hunting cases show that the Court in Strasbourg is willing to reflect broadly on the context and purpose of interference, to critically assess the social function of the taking.

\noo{ In {\it Chassagnou and others v France} the situation was that landowners were compelled to permit hunting on their land, following compulsory membership in a hunting association which was set up to manage hunting in the local area.\footcite{chassagnou99} The owners protested this on the grounds that they were ethically opposed to hunting. The Court agreed that there had been a breach of P1(1). 

In the later case of {\it Hermann v Germany}, the circumstances were similar and the Court followed the precedent set in {\it Chassagnou}. In addition, the Court commented that they had ``misgivings of principle'' about the argument that financial compensation could provide adequate protection in such a case.\footcite[See][91]{hermann12}  In this way, the hunting cases illustrate that to the ECtHR, the right to property is more than a  financial entitlement. The fair balance that must be struck could pertain to other aspects, such as the owner's right to make use of his property in accordance with his convictions and to take part in decision-making processes regarding how it should be managed.\footnote{The assessment of proportionality should be concrete and contextual, and it is not based on a narrow or formalistic concept of property as dominion. This is demonstrated, for instance, by \cite{chabauty12}. Here the Court found no violation of P1(1) although the facts seemed close to those of {\it Chassagnou}. The case differed, however, in that the owner himself was not opposed to hunting, but wanted to withdraw his land from the hunters' association to enjoy exclusive hunting rights.}
}
Less obviously, a similar sentiment appears to be behind the Court's reasoning in recent cases involving rent control schemes and housing regulation.\footnote{See \cite{hutten06,lindheim12}.} There are obvious financial interests at stake in such cases, for both landlords and tenants. However, the Court has addressed these cases by looking to the fairness of the underlying regulation more generally, by critically evaluating the social, economic and political context. Moreover, the Court has not shied away from using concrete cases as a starting point for providing an assessment of the sustainability of national provisions as such.

\subsection{{\it Hutten-Czapska v Poland}}

The striking conclusion in {\it Hutten-Czapska v Poland}, which makes it interesting for the questions studied in this thesis, was that it demonstrated ``systemic violation of the right of property''.\footcite[239]{hutten06} The case concerned a house that had been confiscated during the Second World War. After the war, the property was transferred back to the owners, but in the meantime, the ground floor had been assigned to an employee of the local city council. The state implemented strict housing regulations during this time, which eventually led to the applicant's house being placed under direct state management.\footcite[20-31]{hutten06} Following the end of communist rule in 1990, the owners were given back the right to manage their property, but it was still subject to strict regulation that protected the rights of the tenants.\footcite[31-53]{hutten06} In addition to rent control, rules were in place that made it hard to terminate the rental contracts. Hence, it became impossible for the owners to make use of the house themselves.\footcite[20-53]{hutten06} 

After an in-depth assessment of the relevant parts of Polish law and administrative practice, the Grand Chamber of the ECtHR concluded that there had been a violation of P1(1). Importantly, they did not reach this conclusion by focusing on the owners and the interference that had taken place with respect to their individual entitlements. Rather, they focused on the overall character of the Polish system for rent control and housing regulation, as it manifested in the concrete circumstances of the applicant's case.

The financial consequences for the owners were considered to shed light on a broader question of sustainability, as was the financial situation of the tenants.\footcite[60-61]{hutten06} The Court was particularly concerned with the fact that the total rent that could be charged for the house in question was not sufficient to cover the running maintenance costs.\footcite[224]{hutten06} In particular, it was noted that the consequence of this would be ``inevitable deterioration of the property for lack of adequate investment and modernisation''.\footnote{\cite[224]{hutten06}.}

In the end, the Court highlighted how three factors combined to bring both owners and their properties  to a precarious position. First, the rigid rent control system made it hard to sustainably manage rental property. Second, tenancy regulation made it hard for owners to terminate tenancy agreements. Third, the Court noted that the state itself had set up these tenancy agreements during the days of direct state management, shedding doubt on the legitimacy of the commitments that these contracts imposed on owners. In combination, these factors led the Court to conclude that  a fair balance had not been struck.\footcite[224-225]{hutten06} 

The contextual nature of the Court's reasoning in {\it Hutten-Czapska} is evidenced not only by the extent to which the concrete circumstances were assessed against the goal of fairness. It is also illustrated by how the Court explicitly places the ``social rights'' of the tenants on equal footing with the property rights of the owners.\footcite[225]{hutten06} The result, therefore, was not premised on a narrow understanding of property protection as an individual entitlement, but on a broader vision of property as a social institution.

It is also of interest to note how the Court concludes that the root of the problem is with the Polish legal order as such. In this regard, great weight is placed on the observation that the regulatory system suffers from a lack of adequate safeguards to protect owners against imbalances such as those identified in {\it Hutten-Czapska}. In particular, the Court reflects on the position of owners and comments on ``the absence of any legal ways and means making it possible for them either to offset or mitigate the losses incurred in connection with the maintenance of property or to have the necessary repairs subsidised by the State in justified cases''. Hence, the rent control scheme alone was not the whole problem, the Court also criticised what it saw as a defective way of implementing it.\footcite[224]{hutten06} Moreover, the Court did not censor the political reasoning that motivated Polish housing legislation, but concluded instead that the ``burden cannot, as in the present case, be placed on one particular social group, however important the interests of the other group or the community as a whole''. 

I think the structural argument at work here is key to understanding the case, pointing also to the core function that the ECtHR should embrace more generally. It seems to me, in particular, that objections may well be raised against the appropriateness of having the Court in Strasbourg assess concretely what is fair regarding the relationship between owners and tenants in a specific house in Gdynia. The Court's remoteness to the local conditions, as well as its lack of accountability to local democratic institutions suggests that the Court is not ideally placed to carry out the kind of contextual assessment that it itself prescribes for such cases. In addition, the amount of resources and time needed to independently scrutinize these aspects concretely risks undermining the Court's ability to deal expediently with its case load. The ECtHR will hardly be able to protect human rights in Europe on a case-by-case basis.

Instead, the aim should always be to get at the systemic features that cause perceived imbalances. As in \textcite{hutten06}, the Court serves its function best when it is able to use concrete information about a suspect case to identify a sense in which the domestic legal order needs to be improved to better comply with human rights standards. This is particularly true when, as in that case, the Court notes that the applicants have insufficient options available for achieving a fair balance by appealing to institutions within the domestic legal order. By demanding {\it institutional} changes, the Court effectively delegates responsibility for ensuring the kind of fair balance that is required under the ECHR. Moreover, by scrutinizing the procedures and principles that the states apply when fulfilling this duty, it is likely that the Court will still be able to steer and unify the development of the case law. 

Importantly, they would then be able to do so without having to engage extensively in concrete assessments of fairness. Against this, one may argue that the judicial or administrative bodies of the signatory states can easily circumvent their obligations by giving a superficial or biased assessment of the facts in human rights cases, to avoid embarrassment for the state's political or bureaucratic elite. However, this might then be raised as a procedural complaint before the ECtHR, resulting in cases revolving around Articles 6 (fair trial) and 13 (effective remedy).\footnote{I note that this also fits with recent developments at the ECtHR, toward somewhat broader scrutiny under Article 6, see \cite{khamidov07}.}  In this way, the Court can streamline its functions, by always aiming to direct attention at issues that arise at a higher level of abstraction. This, in my view, is desirable. The ECtHR should not aim to micromanage the signatory states, particularly not in relation to a norm such a P1(1), which the Court itself regards as highly dependent on context.

However, the question arises as to what kind of institutions the Court should focus on in its effort to ensure fairness in relation to Convention rights such as property. It is not given, in particular, that directing attention towards domestic judicial bodies is the most appropriate approach. Rather, it is logical to assume that those institutions most in need of reform will be exactly those that are most often responsible for violations. A possible lack of an effective complaints procedure would be worrying, but not as problematic as systemic weaknesses of those institutions that act in ways that give rise to complaints in the first place. 

By shifting attention towards the institutional context of the primary decision-maker, the Court can also avoid getting stuck in deference to domestic judicial bodies. This can then be accomplished alongside a shift of attention away from concrete assessment of alleged violations. The Court can achieve this by concretely and critically assessing those rules and procedures that are identified as causally significant to individual complaints, at the administrative rather than the judicial level.\footnote{In the future, one might even encounter cases when the Court prefers to remain agnostic about whether a substantive violation occurred, focusing instead on the possible violation inherent in excessive systemic risks and a shortage of adequate safeguards.}

Indeed, the case of {\it Hutten-Czapska} appears to be suggestive of a move towards such a perspective. While the Court went into great detail about the facts of the case, it {\it also} looked at the case from an alternative perspective, more in line with the suggestion sketched above. In fact, I think it is likely that the Court will eventually veer even more towards such an approach, while deferring to national judicial bodies when it comes to concrete factual assessments. If not as a result of policy, I imagine this will happen from necessity, due to the limited capacity of the Court to hear the merits of individual cases.

The proportionality doctrine could still be applied, but approached in more abstract terms as the question of what kinds of rules, and what kinds of institutions, member states need to put in place to ensure fairness. \noo{ In \textcite{hutten06}, the Court moved in this direction, especially when it explained the basic principle as follows:

\begin{quote}
In assessing compliance with Article 1 of Protocol No. 1, the Court must make an overall examination of the various interests in issue, bearing in mind that the Convention is intended to safeguard rights that are “practical and effective”. It must look behind appearances and investigate the realities of the situation complained of. In cases concerning the operation of wide-ranging housing legislation, that assessment may involve not only the conditions for reducing the rent received by individual landlords and the extent of the State’s interference with freedom of contract and contractual relations in the lease market, but also the existence of procedural and other safeguards ensuring that the operation of the system and its impact on a landlord’s property rights are neither arbitrary nor unforeseeable. Uncertainty – be it legislative, administrative or arising from practices applied by the authorities – is a factor to be taken into account in assessing the State’s conduct. Indeed, where an issue in the general interest is at stake, it is incumbent on the public authorities to act in good time, in an appropriate and consistent manner.\footcite[151]{hutten06} 
\end{quote}

I note how the Court builds on the earlier precedent set by cases such as \textcite{sporrong82} and \textcite{james86}. The first half of the quote, therefore, stresses that the Court itself must ``look to the realities of the situation''. However, in clarifying what is meant by this, the Court goes on to emphasise procedural aspects. In particular, it is made clear that the Court regards such aspects as an integral part of those ``realities'' that need to be assessed. Indeed, the Court even makes specific reference to the importance of several values that arise in the context of administrative law, such as predictability and effectiveness.
}
This perspective appears to have been adopted in the case of {\it Lindheim and others v Norway}. Here the applicants complained that their rights had been violated by a recent Norwegian act that gave lessees the right to demand indefinite extensions of ground leases on pre-existing conditions.\footcite[119]{lindheim12} In the end, the Court concluded that there had indeed been a breach of P1(1). They engaged in the same form of assessment that they had adopted in \textcite{hutten06}. Moreover, they concluded that the Ground Lease Act 1996 as such  was the underlying source of the violation -- the problem was not merely that this act had been applied in a way that offended the rights of the applicants. In light of this, the Court did not only award compensation, it also ordered that general measures had to be taken by the Norwegian state to address the structural shortcomings that had been identified.

The Court also commented that its decision should be regarded in light of ``jurisprudential developments in the direction of a stronger protection under Article 1 of Protocol No. 1''.\footcite[135]{lindheim12} However, in light of the change in perspective that accompanies this development, it is interesting to ask in what sense the protection is stronger. In particular, it is not {\it prima facie} clear that the Court's remark should be read as a statement expressing a change in its understanding of the content of individual rights under P1(1). Rather, it may be read  as a statement to the effect that the Court now assumes it has greater authority to address structural problems under that provision. In effect, this allows the Court to conclude that a violation has occurred due to structural unfairness, even when it is not possible to trace this back to any flawed decision that specifically targets the applicants.

\subsection{How Would the ECtHR Approach an Economic Development Taking?}

Is the jurisprudential developments illustrated by the rent control cases relevant to the issue of economic development takings? I believe so. Indeed, I am struck by how the reasoning of the ECtHR in recent cases on hunting and rent control mirrors the kind of reasoning that Justice O'Connor engaged in when considering {\it Kelo}.\footnote{See \cite{kelo05}.} The emphasis is on structural aspects and fairness, grounded on the facts of the concrete case, but mainly interested in what these facts reveal about the rules and procedures involved. 

This is a contextual approach that can maintain a broad focus without loosing its bite. The crux of arguments used to conclude violation is the observation that the system currently in place can offend against the role that owners {\it should} occupy in order to be able to meet those obligations and exercise those freedoms that are attached to the properties they posses.

On this narrative, interference becomes illegitimate when it demonstrates a failure of governance. In the case of \textcite{hutten06}, this boiled down to the observation that it was illegitimate to address problems in the Polish housing sector by placing the burden ``on one particular social group'', namely the owners.\footcite[225]{hutten06} This conclusion was backed up by the concrete observation that the rules and procedures in place meant that owners who were obliged to maintain their properties in good condition for their tenants were in fact prevented from doing so because they were not permitted to charge rents that would cover the costs.

In the case of {\it Kelo}, Justice O'Connor argued in a similar fashion when she concluded that the system which had led to the decision to condemn Suzanne Kelo's house was likely to function so as to systematically ``transfer property from those with fewer resources to those with more''. To Justice O'Connor, there was little doubt that this could become a general pattern, if safeguards were not put in place. %Indeed, it would have to be assumed that a multi-million dollar company would always be in a better position than a homeowner when arguing that  ``economic development'' would result from their ownership. \noo{More subtly, her opinion also hinted at the inconsistency involved in asserting abstractly that economic development would benefit the community indirectly, all the while the development would \isr{in} fact require razing it.}

To conclude, I think the ECtHR would have been likely to approach a case like {\it Kelo} in a manner consistent with Justice O'Connor's approach. Whether they would reach the same conclusion seems more uncertain, particularly since confidence in the nation states' ability and willingness to regulate private-public partnerships might be higher in Europe.\footnote{For a discussion from the point of view of English law, arguing that the prevailing regulatory regime limits the risk of eminent domain abuse largely through regulation of the takings power rather than strict property protection, see \cite{allen08}.} However, it seems unlikely that the ECtHR would follow the majority in {\it Kelo}, by simply deferring to the determinations made by the granting authority. Moreover, with the recent change in perspective towards structural assessment of property institutions, Justice O'Connor's predictions about the ``fallout'' of the {\it Kelo} decision would likely have been of significant interest to the justices at the Court in Strasbourg.

\section{The US Perspective on Economic Development Takings}\label{sec:us}

In this section, I consider US law in more depth. First, I track the development of the case law on the public use restriction found in the Fifth Amendment and in various state constitutions. I consider the jurisprudential development from the early 19th century up to the present day.\footnote{The public use clause in the US constitution was not held to apply to state takings until the late 19th century, see \cite{chicago97}.} Many writers assert that case law from the 19th and early 20th century was \isr{characterised} by a tension between `narrow' and `broad' readings of the notion of public use.\footnote{See, e.g., \cite[483]{walt11}; \cite[203-204]{allen00}. For a more in-depth argument asserting the same, see \cite{nichols40}.} Adding to this, I argue that while different state courts expressed different theoretical views on the meaning of ``public use'', there was a growing consensus that the approach to judicial scrutiny should be contextual, focused on weighing the rationale of the taking against the social, political and economic circumstances.\footnote{A summary of state case law that supports this view is given in the little discussed Supreme Court case of \cite{hairston08}.}  In particular, early state courts did not focus unduly on the exact wording of constitutional property clauses.

Following up on this, I argue that the doctrine of deference that was developed by the Supreme Court early in the 20th century was directed primarily at state courts, not state legislatures and administrative bodies.\footnote{See \cite{vester30} (echoing and citing \cite{hairston08}).} I then present the case of {\it Berman}, arguing that it was a significant departure from previous case law.\footcite{berman54} After {\it Berman}, deference was now taken to mean deference to the (state) legislature, meaning that there would be little or no room for judicial review of the takings purpose. 

This paved the way for the infamous case of {\it Poletown}, where a \isr{neighbourhood} of about 1000 homes was razed in order to provide General Motors with land to build a car factory.\footnote{See \cite{poletown81}.} I note how {\it Berman} provided a key authority used by the state court to uphold this taking. {\it Poletown} in turn links up with the even greater controversy surrounding {\it Kelo},  the eventual backlash of the deferential stance introduced in {\it Berman}.

After the historical overview, I go on to briefly present the vast amount of research that has targeted economic takings in the US after {\it Kelo}. I devote special attention to a proposal due to Heller and Hills, who propose a new institution that can replace eminent domain as a mechanism for land assembly in case of economic development.\footcite{heller08} This proposal will serve as important reference point later on, when I consider the Norwegian land consolidation courts in Chapters 6.

\subsection{The History of the Public Use Restriction}\label{sec:hop}

Going back to the time when the Fifth Amendment was introduced, there is not much historical evidence explaining why the takings clause was included in the Bill of Rights.\footnote{See \cite{fifth}.} Moreover, there is little in the way of guidance as to how the takings clause was originally understood. James Madison, who drafted it, commented that his proposals for constitutional amendments were intended to be uncontroversial.\footnote{See letters from Madison to Edmund Randolph dated 15 June 1789 and from Madison to Thomas Jefferson dated 20 June 1789, both included in \cite{madison79}.} Hence, it is natural to regard the property clause as a codification of an existing principle, not a novel proposal. Indeed, several state constitutions pre-dating the Bill of Rights also included takings clauses, seemingly based on codifying principles from English common law.\footcite[See][299]{johnson11} As Meidinger notes, the Americans had never really charged the British with abuse of eminent domain, and private property had tended to be respected, also in the colonies.\footcite[17]{meidinger80} This undoubtedly influenced early US law.

Just like English scholars at the time, early American scholars emphasised the importance of private property. For instance, in his famous {\it Commentaries}, James Kent described the sense of property as ``graciously implanted in the human breast'' and declared that the right of acquisition ``ought to be sacredly protected''.\footnote{See \cite[see][257]{kent27}.} Indeed, the Supreme Court itself expressed similar sentiments early on, when it spoke of the impossibility of passing a law that ``takes property from A and gives it to B''.\footnote{This was a {\it de dicta} in \cite[388]{calder98}. See also \cite[310]{vanhorne95}.}

However, just as would happen in England, this early US attitude would soon change in response to industrial advances and a desire for economic development. As the 19th century progressed, eminent domain was used more frequently, now also to benefit (privately operated) railroad operations, hydroelectric projects, and the mining industry.\footcite[23-33]{meidinger80} During this time, it also became increasingly common for landowners to challenge the legitimacy of takings in court, undoubtedly a consequence of the fact that eminent domain was used more widely, for new kinds of projects.\footcite[24]{meidinger80} Controversy arose particularly often with respect to the so-called mill acts.\footnote{\cite[24]{meidinger80}. See also \cite[306-313]{johnson11} and \cite[251-252]{horwitz73}.} Such acts were found throughout the US, many of them dating from pre-industrial times when mills were primarily used to serve the farming needs of agrarian communities.\footnote{A total of 29 states had passed mill acts, with 27 still in force, when a list of such acts was compiled in \cite[17]{head85}. According to Justice Gray, at pages 18-19 in the same, the ``principal objects'' for early mill acts had been grist mills typically serving local agrarian needs at tolls fixed by law, a purpose which was generally accepted to ensure that they were for public use.} Following economic and technological advances, provisions originally enacted to serve local farming purposes were now being used by developers wishing to harness hydropower for manufacturing and hydroelectric plants.\footnote{See, e.g., \cite[18-21]{head85} and \cite[449-452]{minn06}.}

It is important to note, however, that mill acts could not be used to  authorise large-scale compulsory transfer of natural resources from owners to non-owners. Rather, mill acts provided management tools that could be used to ensure that owners of water resources could make better use of their rights. This would sometimes involve allowing riparian owners to interfere with, or take a necessary part of, the property of their neighbours, e.g., by constructing dams that would flood neighbouring land.\footnote{See \cite{head85} (a mill case adjudicated by the Supreme Court, including a summary of mill acts and case law from various states). See also \cite[265]{staples03}.} However, the primary purpose of most mill acts was to facilitate rational coordination among owners, to the benefit of their community as a whole. This point was frequently made by the courts to justify upholding takings on the basis of mill acts, including takings that would benefit the manufacturing industry.\footnote{See \cite{fiske31}. See also the discussion (including references to other cases) in \cite{head85}.}

As the industrial use of mill acts increased in scope, the original aim of these acts gradually became overshadowed by the strength of the commercial interests involved, leading to public use controversy relating to provisions that had not previously raised any such doubts.\footnote{See \cite{head86}.} This mechanism, deeply dependant on the social and economic context, underscores the appropriateness of adopting a social function perspective on the relevant body of case law. More generally, it seems that most of the early case law on the public use test from US state courts is characterised by a contextual understanding of property protection. In the following, I explore this in some further detail.

\subsection{Legitimacy in State Courts}\label{subsec:state}

When considering objections to the legitimacy of takings, state courts would not look to the federal Takings Clause directly, but rather base their decisions on corresponding property clauses from their own respective state constitutions.\footnote{Not all states had such property clauses, and exact formulations varied, but a public use requirement was typically observed, see \cite[293-296]{johnson11}.} Indeed, it was not until the late 19th century that state takings came to be regularly scrutinized at the federal level.\footnote{At first, the federal scrutiny took place on the basis of the due process clause in the Fourteenth Amendment, see \cite{head75}. The federal takings clause itself was only applied to state takings after 1897, see \cite{chicago97}.}

When a state court upheld an interference that would benefit commercial interests, it would typically emphasise the broader purpose, often focusing on economic ripple effects.\footnote{See, e.g., \cite{hazen53,scudder32,boston32}. A more comprehensive list of cases adopting a broad view can be found in \cite[617]{nichols40}.} By contrast, when a court decided that an interference was unconstitutional, it would often focus on the concrete use made of the property that was taken,  pointing out that it did not directly benefit the public in the sense required by the public use restriction.\footnote{See, e.g., \cite{sadler59,ryerson77,gaylord03,minn06}. A more comprehensive list can be found in \cite{alr28}.} Sometimes, the question of legitimacy would turn on how widely the notion of `use' was understood. Should this notion be interpreted narrowly, as requiring that the property had to be literally used by the public, or could it be understood broadly, as pointing to a public purpose or benefit of some sort?\footnote{According to Nichols, the narrow view emerged as the ``majority'' opinion on public use, see \footcite[617-618]{nichols40}. But contrast this with \cite{berger78} and \cite[24]{meidinger80}, who argue that the narrow view was only dominant in a handful of states, led by New York.}

This tension between broad and narrow readings of the public use clause have received much attention from legal scholarship.\footnote{See \cite{nichols40,berger78,meidinger80,johnson11}.} However, when studying the case law in more depth, a complementary picture emerges, testifying to some cohesion in the states'
jurisprudence. Regardless of their reading of the public use requirement, state courts seem to have agreed that the question of what counted as a public use was a judicial question that should be assessed concretely, not abstractly.

%For instance, in the case of {\it Gaylord v. Sanitary Dist. of Chicago}, the Supreme Court of Illinois held the state Mill Act to be unconstitutional, as it was not limited to traditional flour mills. In doing so, the court observed that public use was ``something more than a mere benefit to the public''.\footcite[524]{gaylord03} Similar sentiments were expressed in other decisions striking down uses of eminent domain for mill construction, for instance in Vermont, Michigan and New York.\footnote{References.}

A good example is the case of {\it Dayton Gold \& Silver Mining Co v Seawell}, concerning an act that gave mineral owners a right to acquire additional rights needed to facilitate extraction.\footcite{seawell76} The Supreme Court of Nevada decided that the act was constitutional on the basis of a highly contextual reading of the public use requirement in the property clause of the Constitution of Nevada. Interestingly, the Court argued against a literal (narrow) reading on the basis that such a reading would ultimately provide {\it weaker} protection of property:

\begin{quote}
If public occupation and enjoyment of the object for which land is to be condemned furnishes the only and true test for the right of eminent domain, then the legislature would certainly have the constitutional authority to condemn the lands of any private citizen for the purpose of building hotels and theaters. [...] Stage coaches and city hacks would also be proper objects for the legislature to make provision for, for these vehicles can, at any time, be used by the public upon paying a stipulated compensation. It is certain that this view, if literally carried out to the utmost extent, would lead to very absurd results, if it did not entirely destroy the security of the private rights of individuals. Now while it may be admitted that hotels, theaters, stage coaches, and city hacks, are a benefit to the public, it does not, by any means, necessarily follow that the right of eminent domain can be exercised in their favor.\footcite[410-411]{seawell76}
\end{quote}

The quote presents an argument in favour of a broad understanding of the public use requirement. However, it also prescribes broad judicial review of takings purposes, including purposes that would appear to pass a `narrow' public use test. In this way, it asks us to  resist the temptation to think that a broad understanding of public use necessarily entails a public use test that can be passed more easily.

The Court follows up on its reading of the public use requirement by giving a highly contextual assessment of the takings purpose. Specifically, it considers the social and economic importance of mining, concluding that it is the ``greatest of the industrial pursuits'' and that all other interests are ``subservient'' to it.\footcite[409]{seawell76} Indeed, the Court goes as far as to conclude that the benefits of the mining industry are ``distributed as much, and sometimes more, among the laboring classes than with the owners of the mines and mills''.\footcite[409]{seawell76} On this basis, the Court upholds the taking.

I am agnostic as to whether or not this decision was based on an accurate description of the mining industry in Nevada in the late 19th century. The importance of the decision and the remarks above does not turn on this factual question. Rather, the importance arises from the fact that the Court felt the need to scrutinize the takings purpose very broadly. The issue of legitimacy was not approached as a linguistic exercise or an attempt at recreating the original intent of the relevant property clause. Instead, the court proceeded on the basis of their assessment of the prevailing social and economic conditions in the state of Nevada. 

The Court noted the importance of deference to the legislature on matters of policy, but qualified this by remarking that any authority to take property had to be ``enforced by the courts so as to prevent its being used as an instrument of oppression to any one''.\footcite[412]{seawell76} 
Furthermore, the Court was convinced that its contextual approach in this regard would generally offer {\it increased} protection of private property compared to more formalistic approaches. The Court summarised its view on this as follows:

\begin{quote}
Each case when presented must stand or fall upon its own merits, or want of merits. But the danger of an improper invasion of private rights is not, in my judgment, as great by following the construction we have given to the constitution as by a strict adherence to the principles contended for by respondent.\footcite[398]{seawell76}
\end{quote}

The {\it Seawell} case is not unique. For another example, I mention {\it Ryerson v Brown}, a case often cited as an authority for a narrow view of public use.\footcite{ryerson77} Here the taking in question was held to be unconstitutional. However, the Supreme Court of Michigan qualified also made clear that it was ``not disposed to say that incidental benefit to the public could not under any circumstances justify an exercise of the right of eminent domain''.\footcite[337]{ryerson77}

The case concerned the constitutionality of a taking under a mill act, and while the court argues that public use should be taken to mean ``use in fact'', it is clear that ``use'' is understood rather loosely, not literally as physical use of the property that is taken.\footnote{The court explains its stance on the public use restriction by stating (emphasis added) ``it would be essential that the statute should require the use to be public in fact; in other words, that it should contain provisions entitling the public to {\it accommodations}.'' The court continues with an illustrative example: ``A flouring mill in this state may grind exclusively the wheat of Wisconsin, and sell the product exclusively in Europe; and it is manifest that in such a case the proprietor can have no valid claim to the interposition of the law to compel his \isr{neighbour} to sell a business site to him, any more than could the manufacturer of shoes or the retailer of groceries. Indeed the two last named would have far higher claims, for they would subserve actual needs, while the former would at most only incidentally benefit the locality by furnishing employment and adding to the local trade''. See \cite[336]{ryerson77}.} Moreover, when clarifying its starting point for judicial scrutiny, the court explains that ``in considering whether any public policy is to be subserved by such statutes, it is important to consider the subject from the standpoint of each of the parties''. Following up on this, the court finds, with respect to the case in question, that ``the power to make compulsory appropriation, if admitted, might be exercised under circumstances when the general voice of the people immediately concerned would condemn it''. On thi basis, the Court strikes down the taking, summing up its factual assessment as follows: ``what seems conclusive to our minds is the fact that the questions involved are questions not of necessity, but of profit and relative convenience''.\footcite[336]{ryerson77}

Hence, far from nitpicking on the basis of the public use phrase, the court adopts a contextual approach to takings that is rather similar to the approach of {\it Dayton Gold \& Silver Mining Co. v. Seawell}. The outcome \isr{is} different, but it is also based on a different assessment of the context and the consequences of the takings complained about. Importantly, the case does not rest on any {\it a priori} assumption that economic development takings of the kind in question could not meet a public use test -- no general rule is relied on at all. Hence, it is somewhat strange that later commentators have focused on the case for its ambiguous comments on public use as ``public in fact'' rather than its broad and well-reasoned assessment of actual legitimacy.\footnote{See, for instance, Justice Thomas' dissent in {\it Kelo}, \cite[513]{kelo05} (using {\it Ryerson} as a reference to support an `actual use' interpretation of the public use requirement in the fifth amendment).}

Many of the important cases from the late 19th century, on both sides of the public use debate, share crucial features with the two cases discussed above.\footnote{See, e.g., \cite{scudder32} (Eminent domain power upheld, but said: ``The great principle remains that there must be a public use or benefit. That is indispensable. But what that shall consist of, or how extensive it shall be to authorize an appropriation of private property, is not easily reducible to a general rule. What may be considered a public use may depend somewhat on the situation and wants of the community for the time being.''), \cite{fallsburg03} (Eminent domain struck down, on holding that ``the private benefit too clearly dominates the public interest to find constitutional authority for the exercise of the power of eminent domain''), \cite[538]{board91} (Eminent domain struck down, qualified by ``not only must the purpose be one in which the public has an interest, but the state must have a voice in the manner in which the public may avail itself of that use'').} Hence, a shared trait appears to have emerged among state courts during this period, namely a willingness to engage in broad judicial scrutiny of the legitimacy of economic development takings. Indeed, state courts appear to have been conscious of the special legitimacy questions that arise when eminent domain is used to facilitate economic development through commercial enterprise. The question of how to understand public use terminology was an important part of this, but it was not considered in isolation from other aspects.

This observation is relevant when considering the takings doctrine that later developed at the federal level. In particular, the broad scrutiny offered by state courts suggests that the doctrine of extreme deference that was about to be adopted by the Supreme Court resulted from a completely new development, not a continuous broadening of the public use requirement.\footnote{This contrasts with the argument given by the majority in {\it Kelo}, see \cite[479-480]{kelo05} (placing the doctrine of deference in a tradition emerging from how the narrow view of some early state courts ``steadily eroded'' because of the ``diverse and always evolving needs of society'').}

\subsection{Legitimacy as Discussed by the Supreme Court}\label{subsec:US}

Initially, the Supreme Court held that the takings clause in the US Constitution did not apply to state takings at all.\footcite{barron33} Federal takings, on the other hand, were of limited practical significance since the common practice was that the federal government would rely on the states to condemn property on its behalf.\footcite[30]{meidinger80}

This changed towards the end of the 19th century, particularly following the decision in {\it Trombley v Humphrey}, where the Supreme Court of Michigan struck down a taking that would benefit the federal government.\footcite{trombley71} Not long after, in 1875, the first Supreme Court adjudication of a federal taking occurred, marking the start of the development of the federal doctrine on public use and legitimacy.\footcite{kohl75} 

At the same time, the Supreme Court began to hear takings cases originating from the states, first on the basis of the due process clause of the fourteenth amendment, introduced after the civil war.\footnote{See, e.g, \cite{head85}.} Later, in 1897, the Supreme Court held that state takings could be scrutinized also against the takings clause of the fifth amendment.\footnote{See \cite{chicago97}.}

The early 20th century was a period of great optimism about the ability of {\it laissez faire} capitalism to ensure progress and economic growth, a sentiment that was reflected in the federal case law on eminent domain. A particularly clear expression of this can be found in {\it Mt Vernon-Woodberry Cotton Duck Co v Alabama Interstate Power Co}.\footcite{vernon16}  This case dealt with the legitimacy of condemnation arising from the construction of a hydropower plant. The Supreme Court held that it was legitimate, with the presiding judge arguing briskly as follows:

\begin{quote}The principal argument presented that is open here, is that the purpose of the condemnation is not a public one. The purpose of the Power Company's incorporation, and that for which it seeks to condemn property of the plaintiff in error, is to manufacture, supply, and sell to the public, power produced by water as a motive force. In the organic relations of modern society it may sometimes be hard to draw the line that is supposed to limit the authority of the legislature to exercise or delegate the power of eminent domain. But to gather the streams from waste and to draw from them energy, labor without brains, and so to save mankind from toil that it can be spared, is to supply what, next to intellect, is the very foundation of all our achievements and all our welfare. If that purpose is not public, we should be at a loss to say what is. The inadequacy of use by the general public as a universal test is established. The respect due to the judgment of the state would have great weight if there were a doubt. But there is none.\footcite[32]{vernon16}
\end{quote}

On the one hand, the Court notes the importance of deference to the {\it state} judgement (not specifically the judgement of the state legislature). On the other hand, it prefers to conclude on the basis of its own assessment of the purpose of the taking. This assessment, however, is not grounded in the facts of the case or the circumstances in Alabama. Rather, it is based on sweeping assertions about ``all our welfare'' and the desire to ``save mankind from toil that it can be spared''. This marks a contrast with the approach of state courts, as discussed in the previous subsection.

The contrast was even greater in cases when the takings in question had been authorised by the federal government itself. In such cases, the Supreme Court showed little willingness to subject takings purposes to public use scrutiny. In {\it United States v Gettysburg Electric Railway Co}, a case from 1896, deference to the legislature in federal takings cases was referred to as a principle that should be observed unless the judgement of the legislature was ``palpably without reasonable foundation''.\footcite[680]{gettysburg96} 

However, such a deferential stance was not adopted in cases originating from the states. In {\it Cincinatti v Vester}, a case from 1930, the Supreme Court commented that ``it is well established that, in considering the application of the Fourteenth Amendment to cases of expropriation of private property, the question what is a public use is a judicial one''.\footcite[447]{vester30} In this judgement, Chief Justice Hughes also describes how the judicial assessment of the public use question should be carried out:

\begin{quote}
In deciding such a question, the Court has appropriate regard to the diversity of local conditions and considers with great respect legislative declarations and in particular the judgments of state courts as to the uses considered to be public in the light of local exigencies. But the question remains a judicial one which this Court must decide in performing its duty of enforcing the provisions of the Federal Constitution.\footcite[447]{vester30}
\end{quote}

Notice how this echoes the contextual approach developed at the state level, while explicitly prescribing deference to state {\it courts}. In the earlier case of {\it Hairston v Danville \& W R Co}, from 1908, the same idea was expressed by Justice Moody, who surveyed the state case law and declared that ``the one and only principle in which all courts seem to agree is that the nature of the uses, whether public or private, is ultimately a judicial question.''\footcite[606]{hairston08} Justice Moody continued by describing in more depth the typical approach of the state courts in determining public use cases:

\begin{quote}
The determination of this question by the courts has been influenced in the different states by considerations touching the resources, the capacity of the soil, the relative importance of industries to the general public welfare, and the long-established methods and habits of the people. In all these respects conditions vary so much in the states and territories of the Union that different results might well be expected.\footcite[606]{hairston08}
\end{quote}

Justice Moody goes on to give a long list of cases illustrating this aspect of state case law, showing how assessments of the public use issue is inherently contextual.\footcite[607]{hairston08} %He then cites three further Supreme Court cases, pointing out that all of them express support for state case law on this issue.\footnote{{\it Falbrook, Clark} and {\it Strickley}.} 
Following up on this, he points out that ``no case is recalled'' in which the Supreme Court overturned ``a taking upheld by the state {\it court} as a taking for public uses in conformity with its laws'' (my emphasis). After making clear that situations might still arise where the Supreme Court would not follow state courts on the public use issue, Justice Moody goes on to conclude that the cases cited ``show how greatly we have deferred to the opinions of the state courts on this subject, which so closely concerns the welfare of their people''.\footcite[606]{hairston08}

{\it Hairston} is important for three reasons. First, it makes clear that initially, the deferential stance in cases dealing with state takings was primarily directed at state courts rather than legislatures and administrative bodies. Second, it demonstrates federal recognition of the fact that a consensus had emerged in the states, whereby scrutiny of the public use determination was consistently regarded as a judicial task.\footnote{Indeed, {\it Hariston} provides the authority for {\it Vester} on this point. See \cite[606]{vester30}.} Thirdly, it provides a valuable summary of the contextual approach to the public use test that had developed at the state level. 

The {\it Hairston} Court clearly looked favourably on the case law from state courts. Indeed, the judicial scrutiny provided by state courts was held to be of such high quality that there was in general little need for federal intervention. Hence, when a deferential stance was adopted in {\it Hairston}, this was contingent on the fact that state courts would continue to administer the required public use test.

Despite this, {\it Hairston} would later be cited as an early authority in favour of almost unconditional deference.\footnote{In fact, it was cited in this way also by the majority in {\it Kelo}, see \cite[482-483]{kelo05}.} This happened in {\it US ex rel Tenn Valley Authority v Welch}, concerning a federal taking.\footcite[552]{welch46} The Court first cited {\it US v Gettysburg Electric R Co} as an authority in favour of deference with regards to the public use limitation.\footcite{gettysburg96} The Court then paused to note that {\it Vester} later relied on the opposite view, namely that the public use test was a judicial responsibility.\footcite{vester30} The Court then attempts to undercut this by setting up a contrast between {\it Vester} and {\it Hairston}, by selectively quoting the observation made in the latter case that the Supreme Court had never overruled the state courts on the public use issue.\footnote{See \cite[552]{welch46}.} Hence, {\it Hairston} is effectively used to argue against judicial scrutiny, in a manner that is quite incommensurate with the full rationale behind the Court's decision in that case.

Later, {\it Welch} was used as an authority in the case of {\it Berman v Parker}.\footcite{berman54} This case concerned condemnation for redevelopment of a partly blighted residential area in the District of Colombia, which would also condemn a non-blighted department store. In a key passage, the Court states that the role of the judiciary in scrutinizing the public purpose of a taking is ``extremely narrow''.\footcite[32]{berman54} The Court provides only two references to previous cases to back up this claim, one of them being {\it Welch}.\footnote{The other case, {\it Old Dominion Land Co v US}, concerned a federal taking of land on which the military had already invested large sums in buildings. The Court commented on the public use test by saying that ``there is nothing shown in the intentions or transactions of subordinates that is sufficient to overcome the declaration by Congress of what it had in mind. Its decision is entitled to deference until it is shown to involve an impossibility. But the military purposes mentioned at least may have been entertained and they clearly were for a public use''. See \cite[66]{dominion25} A misleading and partial quote, to the effect that deference to the legislature is in order except when it involves an ``impossibility'', has since become commonplace. In particular, such a quote was repeated by the Supreme Court itself in the later case of \cite[240]{midkiff84}.}

Moreover, both of the cases cited were concerned with federal takings, while in {\it Berman} the Court explicitly says that deference is due in equal measure to the state legislature.\footcite[32]{berman54} It is possible to regard this merely as a {\it dictum}, since the District of Columbia is governed directly by Congress. However, {\it Berman} was to have a great impact on future cases. In effect, it undermined a large body of case law on judicial scrutiny of taking purposes without engaging with it at all.

In {\it Hawaii Housing Authority v Midkiff}, the Supreme Court further entrenched the principle expressed in {\it Berman}.\footcite{midkiff84} Here the state of Hawaii had made use of eminent domain  to break up an oligopoly in the housing sector. Given the circumstances of the case, it would have been natural to argue in favour of this taking on the basis that it served a proper public purpose.

However, the Court instead decided to rely on the doctrine of deference, shunning away from scrutinizing the takings purpose. Justice O'Connor, in particular, observed that ``judicial deference is required because, in our system of government, legislatures are better able to assess what public purposes should be advanced by an exercise of eminent domain''.\footcite[244]{hawaii84}

The formulation here is slightly less absolute than that given in {\it Berman}. In particular, the deferential stance is not presented as a system imperative, but rather made contingent on the fact that legislatures are ``better able'' to assess what counts as a public purpose. Moreover, Justice O'Connor also actively refers to the merits of the taking, especially when she points out that  ``regulating oligopoly and the evils associated with it is a classic exercise of a State's police powers''.\footcite[242]{hawaii84}

Despite these nuances, {\it Midkiff} reaffirmed the main principle expressed in {\it Berman}, namely that the meaning of public use is a matter for legislatures and that the room for judicial review is narrow. In light of this, it is easy to understand why {\it Kelo} was decided in favour of the taker. It would have been a clear break with earlier precedent on the public use restriction if the Supreme Court had chosen to decide otherwise. 

Formally, the case law on the federal takings clause is not binding on state courts when they assess cases against their own constitutions.\footnote{See \cite[95]{merrill86}.} Moreover, as Merrill notes, state courts have not uniformly responded by embracing deference towards their own legislatures.\footcite[65]{merrill86} Rather, many state courts continued to offer scrutiny of taking purposes, despite the signals coming from the federal level.\footcite[65]{merrill86}

It should be noted, however, that the time after {\it Berman} was also a time when many government bodies throughout the US would actively seek to condemn homes for redevelopment projects, to combat ``blight'', but often also to the benefit of commercial enterprises.\footnote{See generally \cite{pritchett03}.} Hence, continued public use scrutiny at state courts might also reflect an increased threat of eminent domain abuse. Sometimes, moreover, state courts seems to have failed in their duty to offer appropriate protection.

The case of {\it Poletown Neighborhood Council v City of Detroit} is a classic example.\footcite{poletown81} In this case, the Michigan Supreme Court held that it was not in violation of the public use requirement in the Michigan Constitution to allow General Motors to displace some 3500 people for the construction of a car assembly factory. The majority 5-2 cites {\it Berman}, commenting that the state court's room for review of the public use requirement is similarly limited.\footcite[632-633]{poletown81}

The {\it Poletown} decision was controversial, and the minority, especially Justice Ryan, was highly critical of it. He objects both to the deferential stance in general and to the majority reading of {\it Berman} in particular, pointing out that the Supreme Court's doctrine of deference outside the context of federal takings was  directed at the state courts, not state legislatures.\footcite[668]{poletown81} Hence, as he concludes, the majority's reliance on {\it Berman} was ``particularly disingenuous''.\footcite[668]{poletown81} 

Justice Ryan was not alone in his disapproval of {\it Poletown}.\footnote{Indeed, the decision would later be overturned by the Supreme Court of Michigan itself, see generally \cite{sandefur05}.} Moreover, the case is widely regarded as the prelude to an era of increased tensions over economic development takings in the US.\footnote{See \cite[664-668]{sandefur05}.} This would culminate with {\it Kelo} which, despite upholding and strengthening the deferential doctrine, also inadvertently caused a shift towards stricter public use scrutiny at the state level, as discussed in the following subsection.

\subsection{Economic Development Takings after {\it Kelo}}\label{sec:postkelo}

The fact that {\it Kelo} was decided against the homeowner met with wide disapproval among the public.\footnote{See \cite[2109]{somin09}.} In addition, many scholars expressed concern that the deferential approach had been taken too far, and that economic development takings such as {\it Kelo} were in need of more substantive public use scrutiny by courts.\footnote{For a small sample, see \cite{cohen06,underkuffler06,sandefur06,somin07,gisler10}.}
Moreover, following {\it Kelo}, much attention was directed at the perceived dangers of eminent domain abuse in the US.\footnote{See generally \cite{somin09}.} %The minority opinions given in {\it Kelo}, particularly the opinion of Justice O'Connor, proved influential, causing further attention to be directed at the perceived dangers of eminent domain abuse. A massive amount of literature has since appeared devoted to studying  economic development takings. 

Many states responded by introducing reforms aimed at limiting the use of eminent domain for economic development.\footnote{For an overview and critical examination of the myriad of state reforms that have followed {\it Kelo}, I point to \cite{eagle08}. See also \cite{somin09}.} Within two years, 44 states had passed post-{\it Kelo} legislation in an attempt to achieve this.\footnote{See \cite{castle}.} Various legislative techniques were adopted. Some states, including Alabama, Colorado and Michigan, enacted explicit bans on economic development takings and takings that would benefit private parties.\footcite[See][107-108]{eagle08} In South Dakota, the legislature went even further, banning the use of eminent domain ``(1) For transfer to any private person, nongovernmental entity, or other public-private business entity; or (2) Primarily for enhancement of tax revenue''.\footnote{South Dakota Codified Laws § 11-7-22-1, amended by House Bill 1080, 2006 Leg, Reg Ses (2006).}

In other states, more indirect measures were taken, such as in Florida, where the legislature enacted a rule whereby property taken by the government could not be transferred to a private party until 10 years after the date it was condemned.\footcite[809]{eagle08} Many states also offered lengthy lists of uses that were to count as public, designed to restrict the room for administrative discretion while allowing condemnations for purposes that were regarded as particularly important.\footcite[804]{eagle08}

Somin points to an interesting trend, namely that state reforms enacted by the public through referendums tend to be more restrictive than reforms passed through the state legislature.\footcite[2143]{somin09} Many of the more radical reform proposals, moreover, did not emerge from the state government, but were initiated by activist groups as ballot measures. In some US states, initiative processes make it possible for activist groups to put measures on the ballot without prior approval by the state legislature.\footnote{See \cite[2148]{somin09}.} As Somin observes, the reforms taking place via this route would be comparatively strict, testifying to the power of direct democracy.\footnote{See \cite[2143-2149]{somin09}.}

Indeed, the successes of popular anti-takings movements  underscores how strongly the US public opposed the decision in {\it Kelo}. Surveys show that as many as 80-90 \% believe that it was wrongly decided, an opinion widely shared also among the political elite.\footcite[2109]{somin09} 

{\it Kelo} has clearly had a great effect on the discourse of eminent domain in the US. However, the effects of the many state reforms that have been enacted are less clear. According to Somin, most of these reforms have in fact been ineffective, despite the overwhelming popular and political opposition against economic development takings.\footcite[2170-2171]{somin09} At the same time, property lawyers report a greater feeling of unease regarding the correct way to approach the public use requirement, expressing hope that the Supreme Court will soon revisit the issue.\footnote{See \cite{murakami13} (``Until the Supreme Court revisits the issue, we predict that this question will continue to plague the lower courts, property owners, and condemning authorities'').} 

Why have legislative reforms proved inadequate and ineffective? Part of the reason, according to Somin, is that people are ``rationally ignorant'' about the economic takings issue.\footnote{See \cite[2170]{somin09}.} For most people, it is unlikely that eminent domain will come to concern them personally or that they will be able to influence policy in this area. Hence, it makes little sense for them to devote much time to learn more about it. This, in turn, helps create a situation where experts can develop and sustain a system based on practices that a majority of citizens actually oppose.\footcite[2163-2171]{somin09} Indeed, Somin argues that surveys show how people tend to overestimate the effectiveness of eminent domain reform, possibly due to the fact that symbolic legislative measures are mistaken for materially significant changes in the law.\footcite[2155-2157]{somin09}

I think Somin's analysis is on an interesting track. However, it should be noted that the notion of rational ignorance is a double-edged sword with regards to his main argument. In particular, it seems possible, in theory, that the prevailing critical attitude towards economic development takings is itself an instance of such ignorance. Perhaps people would change their opinion on economic development takings if they were better educated on the issue?

However, this possibility does nothing to detract from the main message, which is that the {\it Kelo} backlash have in fact caused greater insecurity about what the law is and what it delivers, often without significantly curbing those uses of eminent domain that are regarded as most problematic. Arguably, this shows that the legislative approach so far, which has focused on introducing more elaborate and detailed versions of the public use restriction, need to be supplemented by different kinds of proposals. 

In this regard, it seems important to also target governmental decision-making processes regarding the use of private land for economic development. These processes, it seems, need to be imbued with greater legitimacy. In particular, it seems crucial that owners themselves should be granted a better chance to participate in the management of their own land, even when this involves deliberating on, and possibly taking part in, large-scale development projects. After all, it is the owners' and their communities' feeling that they are being treated unfairly that tend to lie at the root of controversies surrounding takings for economic development.\footnote{For a similar perspective, see \cite{underkuffler06}.}

If improved principles of governance are put in place, this alone might be enough to restore some confidence in eminent domain as a procedure by which to implement democratically accountable decisions about land use. However, it seems that eminent domain as such might often be an unduly blunt instrument when society desires commercial development on private land. Instead, it might be possible to devise mechanisms for collective action that replaces the use of eminent domain altogether. %At least, it should be possible to devise mechanisms for benefit sharing in these cases, to make the imposition of a development project appear less unfair to local owners.

In the next section, I will consider a proposal for reform of this kind. %The first specifically targets the question of benefit sharing by proposing a special negotiation mechanism for determining the level of compensation after an economic development taking. 
This proposal targets the decision-making process leading to economic development by proposing a framework for land assembly that can sometimes replace the use of eminent domain.

\noo{ \section{Institutional Proposals for Increased Legitimacy}\label{sec:ir}

In this subsection, I first present the Special Purpose Development Companies (SPDCs) proposed by Lehavi and Licht.\footcite{lehavi07} I relate this proposals to theoretical approaches to the issue of compensation, before I go on to note some shortcomings and open questions that I will later address in my case study. I then go on to consider the Land Assembly Districts (LADs) proposed by Heller and Hills.\footcite{heller08} I consider this proposal in light of the stated motivation, which is to design an effective mechanism of self-governance that can replace eminent domain in economic development cases. I present some unresolved questions and argue that there is a tension in the proposal between its narrow scope, imposed to prevent majority tyranny and other forms of abuse, and its broad goal of empowering local communities. 

\subsection{Special Purpose Development Companies}

An important distinguishing feature of economic development takings is that they give the taker an opportunity to profit commercially from the development. This may even be the primary aim of the project, with the public benefiting only indirectly through potential economic and social ripple effects. Property owners facing condemnation in such circumstances might expect to take a share in the profit resulting from the use of their land. However, in many jurisdictions, including the US, the rules used to calculate compensation prevents owners from getting any share in the commercial surplus resulting from development.\footnote{See, e.g., \cite[965-966]{fennell04}.} In particular, various {\it elimination rules} are typically in place to ensure that compensation is based entirely on the pre-project value of the land that is being taken.\footcite[See][81]{freilich06} That is, the value of the development potential itself is not to influence the compensation payment (at least not to a greater extent than it was already reflected in the value of the property prior to the development plans). The policy reasons for such rules is that they ensure that the public does not have to pay extra due to their own special want of the property. After all, this is one of the main purposes of using eminent domain in the first place: to ensure that the public does not have to pay extortionate prices for land needed for important projects. However, when the purpose of the project is itself commercial in nature, there appears to be a shortage of good policy reasons for excluding this value from consideration when compensation is calculated. This is especially true when, as in the US, compensation tends to be based on the market value of the land taken. Why should a commercial condemner's prospect of carrying out economic development with a profit be disregarded when assessing the market value? In any fair and friendly transaction among rational agents, one would expect benefit sharing in a case like this. Yet for economic development backed up by eminent domain, the application of elimination rules ensures that all the profit goes to the developer. 

Some authors have argued that failures of compensation is at the heart of the economic takings issue and that worry over the public use restriction is in large part only a response to concerns about the ``uncompensated increment'' of such takings.\footcite[See][962]{fennell04} In addition to the lack of benefit sharing, previous work has identified two further problems of compensation that also tend to become exasperated in economic development cases. First, the problem of ``subjective premium'' has been raised, pointing to the fact that property owners often value their own land higher than the market value, for personal reasons.\footcite[963]{fennell04} For instance, if a home is condemned, the homeowner will typically suffer costs not covered by market value, such as the cost of moving, including both the immediate ``objective'' logistic costs as well as more subtle costs, such as having to familiarize oneself with a new local community. Second, the problem of ``autonomy'' has been discussed, arising from the fact that an exercise of eminent domain deprives the landowner of \isr{their right to decide how to manage their} property.\footnote{Discussed in \cite[966-967]{fennell04}. For a general personhood building theory of property law, see \cite{radin93}. For a general economic theory of the subjective value of independence, see \cite{benz08}.}

In \footcite{lehavi07}, the authors propose a novel approach for addressing the ``uncompensated increment'' in economic takings cases. Their proposal is based on a new kind of structure that they dub a {\it Special Purpose Development Corporation} (SPDC). The idea is that owners affected by eminent domain will be given a choice between standard pre-project market value and shares in a special company. This company will exist only to implement a specific step in the implementation of the development project: the transaction of the land-rights. The SPDC may choose either to offer their rights on an auction or else negotiate a deal with a designated developer.\footcite[1735]{lehavi07} Hence, the idea is to ensure that the owners are paid a value that reflects the post-project value of the land, but in such a way that the holdout problem is avoided. In particular, the SPDC will have a single task: to sell the land for the highest possible price within a given time frame.\footcite[1741]{lehavi07} After the sale is completed, the SPDC will divide the proceeds as dividends and be wound up.\footcite[1741]{lehavi07}

Other suggestions have taken a more static approach to compensation reform, such as proposing to give owners a fixed premium in cases of economic development, or developing mechanisms of self-assessment to ensure that compensation is based on the true value the owner attributes to his own land.\footnote{A range of static proposals have been proposed in the literature: Merrill proposes 150 \% of market value for takings that are deemed to be ``suspect'', including takings for which the nature of the public use is unclear, see \cite[90-93]{merrill86}. Krier and Serkin propose a system that provide compensation for a property's special suitability to its owner, or a system where compensation is based on the court's assessment of post-project value, see \cite[865-873]{krier04}. Fennell proposes a system of self-evaluation of property for taking purposes with tax-breaks given to those who value their property close to market value (to avoid overestimation), see \cite[995-996]{fennell04}. Bell and Parchomovsky also propose self-evaluation, but rely on a different mechanism to prevent overestimation; tax liability is based on the self-reported value and no property can be sold by its owner for less than his reported value, see \cite[890-900]{bell07}.} Compared to such proposals, the idea of SPDCs is more sophisticated and should be looked at in more depth. 

The conceptual premise for the proposal is that takings for economic development can be seen as compulsory incorporation, a pooling of resources useful in overcoming market failures.\footcite[1732-1733]{lehavi07} Just as the corporation is formed to consolidate assets in order to facilitate effective management, so is eminent domain used to assemble property rights in order to facilitate efficient organization of development. According to Lehavi and Licht, this also provides a viable approach to problems of ``opportunistic behavior''; hierarchical governance after assembly ensures that order and unity can be regained even if interests in the land are distributed among a large and heterogeneous group of potentially mischievous shareholders.\footcite[1733]{lehavi07} In the words of Lehavi and Licht:

\begin{quote}
The exercise of eminent domain powers thus resembles an incorporation by the government of all landowners with a view to \isr{bringing} all the critical assets under hierarchical governance. Establishing a corporation for this purpose and transferring land parcels to it thus would be merely a procedural manifestation of the substantive economic reality that already takes place in eminent domain cases.
\end{quote}

As soon as we look at the rationale behind economic development takings in this way, any remnant of good policy reasons for ensuring that the developer gets all the profit seems to disappear. Rather, we are led to consider compensation as an issue entirely separate from the exercise of the takings power. After the land has been \isr{reorganised} by eminent domain and an SPDC has been formed, the land rights might as well be sold {\it freely} to a developer. In this way, the land will be sold for a price that is closer to an actual market value, on the market where the land is destined for development.\footcite[1735-1736]{lehavi07} More generally, the SPDC becomes an aid that the government can use to create more \isr{favourable} market conditions for transferring land that has commercial potential in its public use. Due to the compulsory pooling of resources, no owner can exercise monopoly power by holding out, but due to decoupling of compensation from assembly, the owners can now negotiate with potential developers for a share of the resulting profit. Moreover, the fact that the SPDC offers its rights on an actual market can also help ensure that more information \isr{becomes} available regarding the true economic value of the development, something that may in turn help ensure that only the good projects will be successful in acquiring land. Hence, according to Lehavi and Licht, an additional positive effect of SPDCs is that developers and governments will \isr{shy }away from using the eminent domain power to benefit projects that are not truly welfare-enhancing.\footcite[1735-1736]{lehavi07}

In addition to these substantive consequences, the SPDC-proposal also stands out because it has a significant institutional component, pointing to its potential for restoring procedural legitimacy as well as substantive fairness. Lehavi and Licht discuss corporate governance issues at some length, but without committing themselves to definite answers about how the operations of the SPDC should be \isr{organised}.\footcite[1040-1048]{lehavi07} Indeed, while their proposal is perhaps most interesting because of its procedural aspects, it also appears to be rather preliminary in this regard. The main idea is to let the SPDC structure piggyback on existing corporative structures, particularly those developed for \isr{securitisation} of assets.\footnote{See generally \cite{schwarcz94}. For an up-to-date overview, targeting special challenges that became apparent during the 2008 financial crisis, see \cite{schwarcz13}.} The basic idea is that the corporate structure should be insulated from the original landowners to the greatest possible extent; it should have a narrow scope, it should be managed by neutral administrators, and it should entrust a third party with its voting rights.\footcite[1742]{lehavi07} This is meant to prevent failures of governance within the SPDC itself, making it harder for majority shareholders and self-interested managers to co-opt the process. For instance, if a possible developer already holds a majority of the shares in an SPDC, this structure would prevent him from using this position to acquire the remaining land on \isr{favourable} terms. 

Lehavi and Licht observe that under US law, the government would often be required to make shares in an SPDC available to the landowners as a public offering.\footcite[1745]{lehavi07} Lehavi and Licht deem this to be desirable, arguing that full disclosure will provide owners with a better basis on which to decide whether or not to accept SPDC shares in place of pre-project market value. It will also facilitate trading in such shares, so that they will become more liquid and therefore, presumably, more valuable.\footcite[1746]{lehavi07} 

Lehavi and Licht's proposal is interesting, but I think a fundamental objection can be raised against it. In particular, it seems that their governance model more or less completely alienate property owners from the decision-making process after SPDC formation. Limiting the participation of owners is to a large extent an explicit aim, since governance by experts is held to increase the chances of ensuring good governance. But is expert rule really the answer?

It seems that from the owners' point of view, Lehavi and Licht's proposals for governance reduces the SPDC to a mechanism whereby they can acquire certain financial entitlements. These may exceed those that would follow from standard compensation rules, but they do not directly empower owners vis-{\'a}-vis developers and the government. Instead, a largely independent structure will be introduced. It is this new \isr{organisational} structure, rather than the owners, that will now become an important actor in the eminent domain process. In principle, it is meant to represent owners, but to what extent can it do so effectively? After all, it is specifically intended to operate as neutral player, charged with \isr{maximising} the price, nothing more. Hence, it appears that the SPDC will not be able to give owners an arena to negotiate on the basis of property's social functions. Indeed, the institutional component of the SPDC proposal specifically targets a narrow, entitlements-oriented, perspective on what it means to be an owner and why property should be protected. 

\noo{
the personal and social importance they attribute to their land rights. How the problem of ``autonomy'' is addressed by the proposal is therefore hard to see and the ``subjective premium'' also appears to be in danger, unless it can be objectively quantified and covered by the surplus from a voluntary sale. But if such quantification is possible, then why not simply tell the appraiser to award some premium under standard compensation rules?

More generally, it seems to me that while all three categories of ``uncompensated increments'' are interesting to study from a financial viewpoint, severe doubts can be raised regarding the feasibility of addressing the subjective aspects of this as a question of compensation. It may be that issues related to ``subjective premium'' and ``autonomy'' are seen as public use issues for good reason; they are hard to quantify otherwise. Moreover, attempting to do so might do more harm than good. On the one hand, it might skew the political process, since owners that have been ``bought off'' don't object to ill-advised development projects, as long as they generate financial revenue. But what about projects that are undesirable for other reasons, for instance because they completely change the character of a \isr{neighbourhood}, or because they are harmful to the environment? On the other hand, }

With regards to individual aspects of property's social function, such as the personal attachments it engenders, or the sense of autonomy it provides, the idea that money can compensate for the owners' loss is not entirely implausible. Some standard examples include compensation for relocation costs and compensation for the cost of juridical assistance. And, indeed, in many jurisdictions, the law already provides for such compensation.\footnote{See, e.g., \cite[121-126]{garnett06}.}

In general, however, may would no doubt object that financial compensation is beside the point with regards to subjective values pertaining to the individual's own unique relationship with their property. The aftermath of {\it Kelo} itself can serve as an illustration of this.

After the case, Suzanne Kelo remained defiant at first, but eventually decided to settle in 2006, for an offer of USD 442 155, more than USD 319 000 above the appraised value.\footcite[1709]{lehavi07} Despite this, there is no indication that Suzanne Kelo changed her view of the taking. Indeed, after the long struggle she had taken part in, it is easy to imagine that financial compensation, if it was to be an effective remedy at all, would have to be very high. Even after she had settled, Kelo apparently toured the country speaking out against economic takings.

Hence, the significant overcompensation she received, compared to standard market value, did not restore legitimacy, not even to her personally. Moreover, to the community as a whole, it apparently did more harm than good. Indeed, the other owners affected by the same development plan were not pleased, arguing that recalcitrant owners had been unjustly rewarded for holding out.\footcite[1709]{lehavi07} 

This is indicative of the fact that when we move to consider the role of the community, including property dependants that have no formal claims as owners, the compensatory perspective falls short of providing a meaningful approach. Indeed, as laid down by Lehavi and Licht, it does not seem like SPDC will do much good for the community. It will not better enable its members to fulfil their obligations and responsibilities with regards to each other, their land, and society's desire for economic development. Rather, it will render them just as passive as under traditional eminent domain proceedings, the only difference being that they might benefit financially if the SPDC administrators do their job well.

\noo{ subjective importance of property and autonomy can itself prove offensive. At least it seems likely that it would often come to be seen as inadequate and inefficient.\footnote{For more detailed criticism of the compensation approach to the public use issue, see \cite{garnett06}.} Moreover, an owner that is compelled to give up his home after an inclusive process where the public interest has been debated and clearly communicated is likely to feel like he incurs less costs related both to his subjective premium and his autonomy. Hence, the lack of participation in the decision-making process can in itself increase the uncompensated loss. Clearly, no externally managed ``bargain-oriented'' SPDC will be able to resolve this problem.

I conclude that SPDCs have serious shortcoming with regards to the subjective aspects of undercompensation, aspects that can only be addressed if the focus turns towards participation. However, SPDCs do seem promising when it comes to profit-sharing. This, after all, is what the structure is specifically aiming to achieve. In addition, I agree that SPDCs will likely have a positive effect on the other actors in the eminent domain process. In particular, I agree with Lehavi and Licht that greater openness is likely to result, revealing the true merits of development projects, at least in so far as these are translatable into financial terms. The fact that developers must negotiate with an SPDC who can threaten to make the land available an an open auction will likely deter developers and government from pursuing fiscally inefficient projects. Hence, the risk that governments will \isr{subsidised} such projects by giving them cheap access to land will also be reduced. In addition, the presence of a third voice, speaking on behalf of owners, is likely to help achieve a better balance of power in development takings. 

This is a particular concern in cases when competition fails to arise after SPDC formation. To ensure that there are other interested parties, in particular, sems like an important precondition for the proposal to work in practice. In this regard, it is important to \isr{realise} that a lack of interest from other developers may not be due to the superiority of the original developer's plans. It might rather be due to the fact that the scope of the assembly giving rise to the SPDC is so defined as to make alternatives unfeasible. The danger of abuse in this regard seems significant, particularly when developers themselves participate in coming up with the plans that give rise to SPDC formation. 


Moreover, as long as owners remain marginalized in the planning phase, it is easy to imagine situations where the plan itself will be formulated in such a way that only one developer is in a position to successfully implement it. A simple example would be if a prospective developer already owns some of the land that is critical to the plan, and is able to ensure that this land is kept out of the scope of the SPDC. Clearly, if SPDCs are to operate effectively, such instances of manipulation need to be avoided, suggesting that the proposal as it stands needs to be fleshed out in greater detail.
}

The problems addressed here both seem to point to the fact that the SPDCs, while more flexible than other suggestions, are still too static to achieve many of their objectives. In particular, to arrive at genuine market conditions for assessing post-project value, there is still a need for changes in the dynamics of the planning process underlying the taking. Moreover,

This only adds to the number of external authorities on whose judgement and discretion the faith of the community will depend. If the SPDC proposal is thought of as representing as a layer {\it between} the owners, the government and interested developers, the result can also be a further marginalisation of individual owners, for whom it is now even harder to gain access to primary decision-makers.

To better fulfil the goal of ensuring increased legitimacy, there is a need for a mechanism that goes beyond expert bargaining and provides owners with better access to the decision-making process. In the next subsection, I will consider a proposal that aims to address this, by proposing a framework for self-governance. 
}

\section{Legitimacy and Institutions for Self-Governance}\label{sec:lad}

As mentioned briefly in Section \ref{sec:x} of Chapter 2, the work of Ostrom and others on common pool resources suggests that sustainable resource management can often be better achieved through local self-governance than by markets or states.\footnote{See generally \cite{....}. For a short and to-the-point exposition of main ideas, see Ostrom's Nobel Lecture, \cite{...}.} The connection between this work and property theory is highly interesting, and has been explored in some recent work, particularly by US legal scholars.\footnote{See generally \cite{rose...}.} It bears noting, in particular, that most kinds of property that are subjected to regulation (or a taking) in the public interest has features that make it possible to view them as pertaining to common pool resources.\footnote{The key characteristic of such resources is that exclusion is difficult or costly, while use can cause depletion (and hence should be limited). As noted by X and Y, this is wide enough to potentially cover most kinds of property in which the public might take an interest.}

Importantly, designating something as a common pool resource does not imply that the resource in question is open-access or that it is held as a form of common property. Nor does it imply that the resource in question should  {\it not} be held as private property.\footnote{See \cite{ostrom07} (``there is no automatic association of common-pool resources with common-property regimes -- or, with any other particular type of property regime'').} According to Ostrom and Hess, the appropriate  property regime for a given common pool resource is a pragmatic question that depends on the concrete circumstances.

%The distinction between common pool resources and other kinds of potentials inhering in land and other typical objects of property is not sharp. The characteristic features of such resources is that (1) excluding others from access to them is costly and (2) their beneficial potential may be depleted by use.\footnote{See, e.g., \cite{ostrom07}.}

%As observed by X and others, this means that almost {\it any} property can conceivably be understood as pertaining to a common pool resource. After all, exclusion is a key aspect of property, which is almost always potential costly.\footnote{If exclusion can be ensured at no cost, the concept of ownership is also typically not taken to apply, as it appears redundant. A typical example would be the feelings and thought of an individual, or the emotional attachment that exists between two people.} Moreover, exclusion as such is hardly required unless the resource itself is susceptible to some kind of depletion resulting from use. Indeed, as noted by Carol Rose and others, it is possible to view virtually any resource that is a candidate for property as a common pool resource.

%The theories of resource management developed by political economists working in the tradition of Ostrom usually do not have much to say about the role of private property. This might be due to the continued influence of a narrow entitlement-based property narrative. On this narrative, property as a legal category can come to appear rather insulated from broader social and political concerns. By contrast, a social function theory of property makes it very natural to consider the role that property can play in the development of local frameworks for sustainable resource governance. %Moreover, while private ownership is sometimes thought to stand in opposition to the commons, it has been noted by several scholars that a property regime is in itself a way to manage common resources, by recognising a special link between individuals and their possessions.

%In the context of taking of property for economic development, the link between property and resource management is particularly clear. Economic development takings reflect a special approach to resource management whereby the state regards it as desirable to remove existing property rights from the equation entirely. When seen in this light, the question of legitimacy that arises also pertains to the institutional consequences of an economic development taking. Is the taking conducive to sustainable resource management? 

It should be noted, however, that this neutral position on the relationship between property and common pool resources is premised on a bundle of rights understanding of the nature of property.\footnote{See \cite[]{ostrom07}.} Potentially, a different understanding of property might suggest a different perspective. Specifically, the question arises as to how theories of common pool resource management relates to the social function theory. Clarifying this is interesting in its own rights, and is also a possible first step towards an institutional perspective on the takings question. Intuitively, such a perspective already suggests itself by the work done in this and the previous chapter, which has emphasised the position of the local community and the function of property as an anchor for democracy. It remains to work out in further detail what exactly such a perspective has to offer in relation to the issue of economic development takings.

To make progress in this regard, it will be useful to first briefly consider one of the most important theoretical legacies of Ostrom's work, namely a list of eight design principles that she formulated on the basis of empirical studies. These principles were formulated because they seemed to be particularly crucial in ensuring good governance at the local level, and have since been supported by a growing body of empirical evidence. In brief, the so-called CPR principles are the following:

\begin{enumerate}
\item {\bf Well-defined boundaries:} There should be a clearly defined boundary around the resource in question, and a clear distinction should exist between members of the user community, who are entitled to access the resource, and non-members, who may be excluded. This will internalise the costs of resource exploitation and other externalities, ensuring that proper incentives for sustainable management arises within the community of resource users.\footnote{Importantly, the possibility of excluding non-members marks a distinction between open-access resources and common pool resources, where the latter appears much less susceptible to a commons tragedy than the former, because externalities are internalised to a clearly defined community.}
\item {\bf Congruence between appropriation and provision rules and local conditions:} Management principles should be flexible and responsive to changing local conditions. Moreover, management practices should be anchored in the economic, social, and cultural practices prevalent at the local level. In addition, the individual benefits should exceed the individual costs associated with membership in the community of users, and collectively managed benefits should be distributed fairly among community members.
\item {\bf Collective-choice arrangements:} The individual members of the user community should have an opportunity to participate in decision-making processes regarding the rules that relate to the user community and the resource management. In addition to securing fairness and legitimacy, this will enhance the quality of the decision-making, as the users themselves have first-hand knowledge and low-cost access to information about their situation and the state of the resource in question.
\item {\bf Monitoring:} There should be mechanisms in place to ensure that the behaviour of users is monitored for violations of management rules. To increase efficiency, monitoring should be locally organised. Moreover, to ensure local responsiveness and legitimacy, individuals acting as monitors should themselves be members of the user community or in some way answerable to this community.
\item {\bf Graduated sanctions:} There should be an effective system in place for penalising violations of user community rules. These penalties should be graduated so that more severe or repeated violations are sanctioned more severely than minor or one-time transgressions.  
\item {\bf Conflict-resolution mechanisms:} The user community should be endowed with low-cost procedures for conflict resolution. Again, it is important that these procedures are rooted in local conditions, to ensure local legitimacy.
\item {\bf Minimum recognition of rights:} The user community should be protected from interference by external actors, including government agencies. As a minimum, the right to self-governance and the existence of local institutions should be recognised and respected by the state.
\item {\bf Nested enterprises:} There should be vertical integration between local, small-scale, management institutions and larger institutions aimed at protecting and furthering non-local interests. This integration should be based on the minimum recognition of rights mentioned in the previous point. Furthermore, it should provide a template for integrated decision-making about larger scale issues, where local competences are employed incrementally in more general settings, with respect to institutions working on behalf of municipalities, regions, states and the international community. Self-governing local institutions for resource management should not only be respected by larger scale structures, but should also feed into larger scale decision-making and be called to respond to greater community needs.
\end{enumerate}

There is an interesting link between the principles described here and the property norms discussed in Chapter 1. There the focus was on norms arising in relation to the social function of property, particularly those related to the idea of human flourishing through membership in a community. The main argument was that property plays an important constitutive role in this regard, as the rights and obligations inhering in private property has an important function as a promoter of collective action and equity at the local level, as well as a protector of local communities against exploitation and undue interference from external actors.

The role that property can play in this regard is linked to the local institutions for resource management that property rights can help sustain. Under individualistic, entitlements-based, theories of property, one might worry that institutions for collective management, even at the local level, will have to be rendered very thin in order not to conflict with the rights of owners. However, on the social function theory, this should no longer be a concern. On this theory, there is no opposition between strong property rights and strong community institutions for collective management of local resources, even if these resources are held as private property. The inclusion of non-owners as full members of the user community is also consistent with the social function theory of property and its emphasis on obligations as well as rights.\footnote{More subtly, limitations on the owners' rights imposed on the basis of property's social function do not necessarily deprive owners of their protected status as property holders. Hence, property does not lose its power to protect the community, merely because the rights of individual owners is limited by community interests. For instance, even if (local) institutional arrangements limits the owners' right to alienate their property, this does not mean that they are no longer owners. Private property with restrictions on alienation can provide the legal basis for local resource management, without removing us from the realm of private property. This is seen, for instance, in Norway, as discussed in Section x of Chapter 2. By contrast, Ostrom and Hess adopt a more narrow view on ownership, where the right to alienation is considered so basic that resource users who lack this right are not considered owners at all, but merely proprietors. See \cite[...]{ostrom07}.}

The design principles listed above further underscores how well the human flourishing theory of property coheres with the theory of common pool resources. Specifically, the property concept endorsed by human flourishing theorists promises to deliver a straightforward and robust implementation of several key design features. Recognising something as property will tend to deliver clearly defined boundaries and a minimal recognition of rights. Moreover, on a human flourishing account, property is also required to deliver congruence between burdens and benefits among property users, particularly at the community level. Finally, property provides a template for a form of organisational nesting where larger scale institutions may be awarded regulatory competences, but must still be prevented from usurping the power to manage local resources as {\it de facto} proprietors. In short, it appears that property alone can take us halfway towards a well-functioning CPR institution.

The final step, pertaining specifically to the institutional framework for collective management, might well place significant limits on the ``despotic dominion'' of individual owners. However, on the social function theory, such limitations might be understood as means to fulfil property's purpose, rather than as an attack on the rights of owners. More generally, the social function understanding of property makes it possible to sustain a property narrative by regarding resource users as owners even if local institutional arrangements mean that they are not in possession of some of the sticks typically found in liberal property bundles. Arguably, local people should still be entitled to recognition as owners of their resources, even if the form of ownership in question does not conform to the standard liberal expectation of what private property looks like. If this is true, it would also appear that abrogation of private property is neither required nor desirable, in order to secure sustainable management of local resources.

%This is likely to offer local people a significantly greater level of protection than they could otherwise expect, also under human rights provisions targeting their social and economic condition.

This observation connects Ostrom's theory of common pool resources to the human flourishing theory of property at a high level of abstraction, suggesting that the two are in fact mutually conducive to one another. There are two different ways in which this also connects concretely to the issue of economic development takings. First, one may observe that when economic development takings appear to lack legitimacy with respect to social functions, this is typically also an indication that the surrounding framework for resource management is not well-designed. In particular, it appears that the Gray test presented at the end of Chapter 1 closely tracks many of the design principles proposed by Ostrom.

%It is clear that organising economic development is a crucial aspect of resource management. Moreover, the use of eminent domain always takes place within a management context that can in turn be assessed against the design principles proposed for common pool resources. Importantly, this links up with the discussion in Chapter 1, where it was argued that the assessment of legitimacy of takings should be broad and contextual. This discussion culminated in the presentation of the Gray test, which focuses on social and economic aspects when guiding our judgement of when a taking is legitimate. Interestingly, it seems that whenever this test leads us to consider a taking as an act of predation, there is good reason to think that the accompanying framework for resource management violates the design principles for common pool resources.

For instance, consider the balance of power between the owners and beneficiaries of a taking, the first point to consider according to the Gray test. There can be little doubt that when a taking fails on this point, there is good reason to critically question the workings of the surrounding framework for resource management. In particular, doubts arise with regard to the recognition of local rights, collective-choice arrangements, and the congruence between appropriation, provision and local conditions. If property is taken by powerful actors, chances are that these actors are not representative of local community interests. Hence, takings of this kind will tend to demonstrate that government is unwilling to recognise the rights of local people, even when these rights are formally recognised as property rights.

%This is illustrated very clearly by the US cases, where economic development takings tend to result in controversy precisely when the taker is an external commercial actor who poses a threat to a local community. In these cases, taking property serves to weaken the link between resource management and local conditions.\footnote{In some situations, such as in {\it Poletown}, the taking can consists in removing whole communities, replacing the properties found within it by commercial property, owned and managed by a single large enterprise.} 
By contrast, the situation might be quite different if it involves a taking that is not suspect according to the Gray test. For instance, if property is taken from absentee landlords and given to local land users in order to facilitate development, this might be an honest attempt at setting up a management framework that complies with CPR principles. In such a case, moreover, one would obviously not expect the balance of power between owners and takers to point towards abuse.

%Indeed, it seems that many, possibly all, the conditions of the Gray test tend to track corresponding design principles for CPRs. For instance, the environmental impact test, whereby an economic development is suspect if it demonstrates disregard for nature, appears closely related to both the issue of monitoring of resource abuse and graduated sanctions for bad behaviour. More generally, the regulatory effect of a suspect economic development taking also closely tracks these points.

The second link between CPR design and economic development takings is arguably even more interesting. It becomes apparent as soon as we shift attention away from diagnosing a lack of legitimacy towards coming up with alternative management principles that can restore it. Specifically, the work on local governance of common pool resources point to an {\it alternative} way of approaching the goal of economic development in cases that might otherwise result in the use of eminent domain. 

The search for viable alternatives to eminent domain as a means to ensure economic development on  private property has not received much attention in the literature so far. One notable exception, discussed in depth in the following subsection, is the work of Heller, Dagan and Hills.\footnote{The work of Lehavi and Licht also deserves a brief mention, even though it focuses on compensation rather than alternatives to eminent domain. The reason is that this work relies on proposing a novel institution that also touches on issues related to self-governance. In particular, Lehavi and Licht propose that post-taking, collective, price bargaining should be carried out on behalf of owners by a Special Purpose Development Company, in an effort to give them a chance to get their share of the commercial benefit arising from development. See \cite{lehavi07}. For a more in-depth discussion of this proposal, and the compensatory approach to economic development takings more generally, see \cite{dyrkolbotn15}.} Looking at their work will serve to make the abstract discussion above more concrete, and will set the stage for a comparison between their proposal and solutions that can be facilitated by the system of land consolidation presented in Chapter 5.

\subsection{Land Assembly Districts}\label{sec:lad}

In an important article from X, Heller and Dagan considered the connection between CPR design principles and overarching (liberal) property values. From this, they arrived at a proposal for what they call a ``liberal commons'', which adds some design constraints rooted in a desire to protect individual autonomy and minority rights. In particular, they emphasise the value of exit, the opportunity for members of the governance structure to alienate their share in the commons resource (conceived of as a property right). However, they propose mechanisms, such as rights of first refusal, meant to ensure that exit does not prove too disruptive to the local collective.

In a later article, Heller and Hills build on the idea of the liberal commons by proposing a novel approach to the takings issue, consisting of a proposal for a new institutional framework that can facilitate land assembly for economic development. Importantly, it is meant to replace eminent domain altogether, in certain kinds of cases. The goal is to ensure democratic legitimacy while also setting up a template for collective decision-making that will prevent inefficient gridlock and holdouts. 

The core idea is to introduce {\it Land Assembly Districts} (LADs), institutions that will enable property owners in a specific area to make a collective decision about whether or not to sell the land to a developer or a municipality.\footcite[1469-1470]{heller08} The idea is that while anyone will be able to propose and promote the formation of a LAD, the official planning authorities and the owners themselves must consent before it is formed.\footcite[1488-1489]{heller08} Clearly, some kind of collective action mechanism is required to allow the owners to make such a decision. 

Hiller and Hill suggest that voting under the majority rule will be adequate in this regard, at least in most cases.\footnote{See \cite[1496]{heller08}. However, when many of the owners are non-residents who only see their land as an investment, Heller and Hills note that it might be necessary to consider more complicated voting procedures, for instance by requiring separate majorities from different groups of owners. See \cite[1523-1524]{heller08}.} 

How to allocate voting rights in the LAD is given careful consideration, with Heller and Hills opting for the proposal that they should in principle be given to owners in proportion to their share in the land belonging to the LAD.\footnote{See \cite[1492]{heller08}. For a discussion of the constitutional one-person-one-vote principle and a more detailed argument in \isr{favour} of the property-based proposal, see \cite[1503-1507]{heller08}.} Owners can opt out of the LAD, but in this case, eminent domain can be used to transfer the land to the LAD using a conventional eminent domain procedure.\footcite[1496]{heller08}

Heller and Hills envision an important role for governmental planning agencies in approving, overseeing and facilitating the LAD process. Their role will be most important early on, in approving and spelling out the parameters within which the LAD is called to function.\footcite[1489-1491]{heller08} Hence, it appears to be assumed that the planning authorities will define the scope of the LAD by specifying the nature of the development it can pursue. 

A possible challenge that arises, not discussed by Heller and Hills at any length, is that the scope of the LAD needs to be broad enough to allow for meaningful competition and negotiation after LAD formation. To achieve this might be difficult, particularly in light of incentives to make the outcome of the LAD process more predictable. Indeed, both governments, initiating developers, and landowners eager for development might want to ensure that the scope of the LAD is defined narrowly enough to give confidence that zoning permissions will not later be denied. In addition, there is the obvious nefarious incentive that some actors might have to ensure that a specific development is chosen. In light of this, LAD regulation is needed to ensure a balanced approach to the issue of how the initial development possibility should be defined, and to what extent this definition should limit the authority of the LAD to choose alternative projects.

If the owners do not agree to forming a LAD, or if they refuse to sell to any developer, Heller and Hills suggest that the government should be precluded from using eminent domain to assemble the land.\footcite[1491]{heller08} This is a crucial aspect of their proposal that sets the suggestion apart from other proposals for institutional reform that have appeared after {\it Kelo}. A LAD will not only ensure that the owners get to bargain with the developers over compensation, it will also give them an opportunity to refuse any development to go ahead. Hence, the proposal shifts the balance of power in economic development cases, giving owners a greater role also in preparing the decision whether or not to develop, and on what terms. Hence, the LAD proposal promises to address the democratic deficit of economic development takings, without failing to \isr{recognise} that the danger of holdouts is real and that institutions are needed to avoid it.

There are some problems with the model, however. First, I observe that planning authorities might have an incentive to refuse granting approval for LAD formation. After all, doing so entails that they give up the power of eminent domain for the land in question. For this reason, Heller and Hills propose that a procedure of judicial review should exist whereby a decision to deny approval for LAD formation can be scrutinized.\footcite[1490]{heller08} However, the question then arises to what extent the courts should adopt a deferential stance in this regard, echoing the conundrum that engulfs the safeguard intended by the public use restriction. Presumably, one would want the courts to strictly scrutinise LAD rejections, to instil in governments that LADs should normally be promoted. However, would the courts be comfortable providing such scrutiny, also against a government body claiming that the ``public interest'' speaks against LAD formation? This would likely depend on the exact formulation and spirit of the LAD-enabling legislation. To work as intended, some sort of presumption in favour of LAD approval appears to be in order, but this in turn can have the effect of making it easier for powerful landowners to abuse the LAD system.

A second possible objection against the LAD proposal concerns the practicalities of the process leading up to the LAD's decision on whether or not to accept a given offer. Is it possible to organise such a process in a manner that is at once efficient, inclusive and informative, without making it too costly and time consuming? Here Heller and Hills envision a system of public hearings, possibly \isr{organised} by the planning authorities, where potential developers meet with owners and other interested parties to discuss plans for development.\footnote{See \cite[1490-1491]{heller08}. It might also be necessary for the planning authorities or other government agencies to take on some responsibilities with respect to providing guidance and assistance to less resourceful members among the owners.} The process envisioned here would resemble existing planning procedures to such an extent that additional costs could hopefully be kept at a minimum. 

The significant difference would concern the relative influence of the different actors, with the owners receiving a considerable boost as a result of the LAD. Rather than being sidelined by a narrative that sees the use of eminent domain as the culmination of planning, the owners are now likely to occupy center stage throughout, as they now will have the final say on whether or not the development will go ahead.

From this, however, arises the question of how the interests of other locals, without property rights, will be protected. Heller and Hills assumes that local non-owners will also be represented during the stages leading up to the LAD's final decision, but their role in the process is not clarified in any detail.\footcite[1490-1491]{heller08} This raises the worry that LADs might undermine local democracy by giving property owners a privileged position with respect to policy questions that should be decided jointly by all members of the local community. The risk in this regard depends heavily on the circumstances. In a context of egalitarian property ownership and sensible government regulation of land uses and LAD operations, the risk should be minimal. In principle, the local anchoring that LADs provide should also benefit non-owners, by brining the decision-making process closer and making it more easily accessible. Moreover, if some members of the local community remain marginalised, this is probably best regarded as a regulatory failure or a reflection of underlying inequality, not a shortcoming of the LAD proposal. In these cases, a reasonable approach might even be to {\it expand} the function of LADs, by granting voting rights to a larger class of local property dependants, not only formally titled owners.\footnote{The important invariant to maintain, I believe, is that the locally anchored institution should be the active, invested, agent, while more centralised and/or expert-dominated government bodies should act as passive, impartial, regulators. In the processes leading to economic development takings, this equation is typically reversed, with government bodies and commercial companies being the active agents, while the owners and the local community are the passive agents whose property rights and dependencies place some nominal limits on the authority of other parties (limits which, due to the weakness of owners as a group, tend to be easily disregarded).}

However, the LAD proposal raises some highly problematic issues pertaining to the proposed mechanism of collective decision-making. As Kelly points out, the basic mechanism of majority voting is deeply flawed.\footcite{kelly09} He argues, in particular, that if different owners value their property differently, majority voting will tend to \isr{disfavour} those with the most extreme viewpoints, either in \isr{favour} of, or against, assembly. If these viewpoints are assumed to be non-strategic and genuine reflections of the welfare associated with the land, the result can be inefficiency. In short, the problem is that a majority can often be found that does not take due account of minority interests. 

For instance, if a minority of owners are planning alternative development, conflicting with the LAD proposal, they might simply be ignored. Indeed, they might {\it have to be} ignored, if the development description underlying LAD formation is incompatible with the kind of development they wish to pursue. This could become particularly inefficient in cases when the alternative development is more socially desirable than the development sketched during LAD formation. In such cases, LAD formation will not improve the quality of the decision to develop, since it pushes the decision-making process into a track where those interests that {\it should} prevail are voiced only by a marginalised minority inside the new institution.\footnote{Of course, one might imagine these landowners opting out of the LAD, or pursuing their own interests independently of it. However, they are then unlikely to be better off than they would be in a no-LAD regime. In fact, it is easy to imagine that they could come to be further \isr{marginalised}, since the existence of the LAD, acting `on behalf of the owners', might detract from any dissenting voices on the owner-side.}

More generally, the lack of clarity regarding the role of LADs in the planning process is a problem. If LAD formation as such can be used to privilege a specific development over alternatives, developers would have a great incentive to form a LAD as early as possible. Moreover, if the presumption is in favour of allowing LAD formation, with limited room for government censorship, developers might rely on LADs to push through a {\it de facto} condemnation of property, through a procedure that might leave the minority less protected than the traditional takings process. Indeed, it would be theoretically possible for any landowner to use a LAD to condemn any neighbouring property smaller than their own. Eventually, a whole community might be taken over by one or a few powerful landowners, through a sequence of cleverly designed LAD processes and development projects. %At the same time, it is easy to acknowledge that problematic situations may arise from wide LAD powers, for instance if a majority forms in \isr{favour} of a scheme that involves razing only the homes of the minority, maybe on the rationale that these are the most blighted properties.

Despite these worries, the ideal of the LAD proposal is clearly stated and highly attractive. LADs should help to establish self-governance for land assembly and economic development. In particular, Heller and Hills argue that LADs should have ``broad discretion to choose any proposal to redevelop the \isr{neighbourhood} -- or reject all such proposals''.\footcite[See][1496]{heller08} As they put it, two of the main goals of LAD formation is to ensure ``preservation of the sense of individual autonomy implicit in the right of private property and preservation of the larger community's right to self-government''.\footcite[See][1498]{heller08} Unfortunately, these ideals turn out to be at odds with some of the concrete rules that Heller and Hills propose, particularly those aiming to ensure good governance of the LAD itself.

%echo many of the ``corporate governance''-ideas that also feature heavily in Lehavi and Licht's proposal. Indeed,

In relation to the governance issue, Heller and Hills emphasise, in direct contrast to their comments about ``broad discretion'' and ``self-governance'', that ``LADs exist for a single narrow purpose -- to consider whether to sell a neighborhood''.\footcite[See][1500]{heller08} This is a good thing, according to Heller and Hills, since it provides a safeguard against mismanagement, serving to prevent LADs from becoming battle grounds where different groups attempt to co-opt the community voice to further their own interests. As Heller and Hills puts it, the narrow scope of LADs will ensure that ``all differences of interest based on the constituents' different activities and investments, therefore, merge into the single question: is the price offered by the assembler sufficient to induce the constituents to sell?''.\footcite[1500]{heller08}

This means that there is a significant internal tension in the LAD proposal, between the broad goal of self-governance on the one hand and the fear of \isr{neighbourhood} bickering and majority tyranny on the other. Indeed, it is hard to see how LADs can at once have both a ``narrow purpose'' as well as enjoy ``broad discretion'' to choose between competing proposals for development. If such discretion is granted to LADs, what prevents special interest groups among the landowners from promoting development projects that will be particularly \isr{favourable} to them, rather than to the landowners as a group? What is to prevent landowners from making behind-the-scene deals with \isr{favoured} developers at the expense of their \isr{neighbours}? It might be difficult to come up with rules that prevent mechanisms of this kind, without also making meaningful ``self-governance'' an impossibility. 

If a LAD is obliged to only look at the price, this might prevent abuse. But it will not give owners broad discretion to consider the social functions of property when choosing among development \isr{proposals}. %Effectively, it will render LADs as little more than a variant of SPDCs, where the owners are awarded an extra bargaining-chip, namely the option to refuse all offers. 
In my view, it is undesirable to restrict the operations of LADs in this way. It is easy to imagine cases where competing proposals, perhaps emerging from within the community of owners themselves, will be made in response to the formation of a LAD. Such proposals may involve novel solutions that are superior to the original development plans, in which case it is hard to see any good reason why they should not be taken into account, even if they are proposed by a minority. Moreover, it is hard to see why they should be disregarded simply because they are less commercially attractive, or because the  developer interested in pursuing such a proposal cannot offer the highest payment to the owners. In the end, the decision that the LAD makes concerns the future of the community as a whole. This is not an exercise in profit-maximization, and there are good reasons to believe that LAD regulation should encourage a broad perspective, not enforce a narrow one.

However, when it comes to the details, Heller and Hills seem to give up on the ideals of self-governance in favour of strict regulation to reduce the risk of LAD abuse. In particular, they argue that ``LAD-enabling legislation should require especially stringent disclosure requirements and bar any landowner from voting in a LAD if that landowner has any affiliation with the assembler''.\footcite{heller08} Hence, the notion of self-governance is made even thinner, as owners will effectively be barred from using LADs as a template for gaining the right to participate in development projects on their own land. We are almost back to square one: the owner must sell or receive nothing.  Moreover, new questions arise. For one, what is meant by ``affiliation''? Say that a landowner happens to own shares in some of the companies proposing development. Should they then be barred from voting? If so, should they be barred from voting on all proposals, or just those involving companies in which they are a shareholder? If the answer is yes, how can this be justified? Would it not be easy to construe such a rule as discrimination against landowners who happen to own shares in development companies? On the other hand, if the landowner in question is allowed to vote on all other proposals, would it not be natural to suspect that their vote is biased against assembly that would benefit a competing company? Or what about the case when some of the landowners are employed by some of the development companies? Should such owners be barred from voting on proposals that could benefit their employers? This seems quite unfair as a general rule. But in some cases, employment relations could play a decisive factor in determining the outcome of a vote. This might happen, for instance, if an important local employer proposes development in a \isr{neighbourhood} where it has a large number of employees.

The fundamental issue that arises is the following: who exactly should be empowered to make the determination of when an affiliation is such that an owner should be deprived of their voting rights? Heller and Hills give no answer, but it is easy to imagine that whoever is given this task in the first instance, the courts must be prepared to deal with complaints. At this point, the circle has in some sense closed in on the proposal. In particular, one might ask: why is it less objectionable to deprive someone of their LAD voting rights whenever they enter into an ``affiliation'' with a developer, than it is to deprive someone of their property rights to ensure economic development on their land? In both cases, severe interferences with basic rights are taking place, on the basis of vague proclamations. Moreover, both interferences appear to be based on the premise that owners are passive agents who can at best hope to be remunerated if they step aside, but who are not themselves meant to be active contributors to economic development.

I conclude that how to best organise a LAD remains an open problem. The challenge is to ensure that LADs deliver a real possibility of self-governance, while also ensuring good governance and protection against abuse. That it remains unclear how to do this is acknowledged by Hiller and Hills themselves, who point out that further work is needed and that only a limited assessment of their proposal can be made in the absence of empirical data. Later in the thesis, I will shed light on this challenge when I consider the Norwegian framework for land consolidation. This framework can be looked at as a sophisticated institutional embedding of many of the central ideas of LADs.

I will discuss how Norwegian land consolidation can be employed in cases of economic development, and how it is increasingly used as an alternative to expropriation in cases of hydropower development. This will allow me to shed further light on the issues that are left open by Heller and Hills' important article.

\section{Conclusion}\label{sec:conc2}

In this chapter, I have given a more in-depth presentation of economic development takings. I began by noting that the issue is particularly pressing for land users that are not regarded as bringing about economic growth. Hence, I argued that the issue is closely related to that of land grabbing, which is currently receiving much attention, both academic and political. Under the social function understanding of property there is in principle no difference between protecting property rights arising from formal title and property rights arising from use. That said, special issues arise in the latter case, not least because it is unclear how the law should deal with rights resulting from cultural practices that western property regimes are not designed to handle. In addition, I noted that special issues related to poverty and basic necessities such as food and water arise with particular urgency in relation to land grabbing.

The nature of my case study makes it natural for me to focus on traditional western systems of property law. Hence, I went on to discuss how economic development takings are dealt with in such legal systems, focusing on Europe and the US respectively. For the case of Europe, this assessment was made more difficult by the fact that the category is not an established part of legal discourse. However, by looking to England for concrete examples, I noted that such cases do arise and that they are increasingly seen as controversial.

I then went on to consider the property protection offered by P1(1) of the ECHR, and how it is applied by the Court in Strasbourg. I zoomed in on those \isr{aspects} that I believe to be the most relevant for economic development takings. While I noted that this category has yet to be discussed by the ECtHR, I argued that a recent shift in the Court's property adjudication is suggestive of the fact that it would likely approach such cases similarly to how Justice O'Connor approached {\it Kelo}. In particular, I noted how the Court has recently adopted a stricter standard of assessment. This standard, I argued, is \isr{characterised} primarily by increased sensitivity to systemic imbalances causing alleged P1(1) violations. Hence, to regard economic development takings as a special category appears to fit well with recent jurisprudential developments at the Court in Strasbourg.

I went on to consider US sources on economic development takings, noting that the issue has receive an extraordinary amount of attention in recent years. I adopted an historical approach to the material, by tracing the case law surrounding the public use restriction in the fifth amendment to the US constitution, which was much debated even before the specific issue of economic development takings rose to prominence. I focused particularly on case law developed by state courts, and I argued that it shows great sensitivity to the need for contextual assessment. Indeed, it seems that many state courts originally adopted an implicit social function view of property when assessing such cases.

I then looked at the history of Supreme Court adjudication of public use cases. I noted that the doctrine of deference was developed early on, but that it was initially directed mainly at state courts. In fact, I showed that the Supreme Court itself explicitly approved the contextual and in-depth approach these courts relied on when dealing with the legitimacy issue.

The shift, I argued, came with {\it Berman}, in which the Supreme Court adopted a deferential doctrine that was directed specifically at the state legislature.\footcite{berman54} This was quite a dramatic departure from the Court's previous attitude towards state takings. Moreover, it was almost entirely backed up by precedent set in cases when {\it federal} takings had been ordered by Congress.

I went on to consider the fallout of {\it Berman} at state level, which culminated with the infamous {\it Poletown} case. This case prompted wide-spread accusations of eminent domain abuse and thus set the stage for {\it Kelo}.

After completing the historical overview, I went on to consider the literature after {\it Kelo}. I expressed particular support for those responses that focus on the need for {\it institutional} reform, to address  dangers that Justice O'Connor pointed to in her minority opinion. As a shorthand, I proposed referring to the mechanisms she identified as the {\it democratic deficit} of economic development takings. 

I then gave a thorough presentation of a recent reform suggestion that might help address this deficit, namely the proposal for Land Assembly Districts (LADs) made by Heller and Hills. According to the motivation behind this proposal, local communities should be entitled to greater self-governance in economic development scenarios, organised through LADs. Importantly, this proposal \isr{recognises} the need for a mechanism to avoid inefficient and socially harmful gridlock due to holdouts among unwilling owners. Instead of eminent domain, however, a different mechanism is proposed, namely that of a majority decision made by a land assembly district.

I pointed out some problems and seeming inconsistencies in the proposal, particularly regarding the lack of clarity regarding the exact role LADs are supposed to play during the planning process. I argued that while the risk of abuse and failure increases with the level of participation, so does the overall potential for achieving a positive effect on legitimacy. I concluded that to reduce the democratic deficit in economic development cases, a wide power of participation must be granted to the land owners and their communities. This is needed, in particular, to restore balance in the relationship between owners and others directly connected with the land, the planning authorities, and the commercial actors interested in development for profit. The question that is as of yet unresolved is how to \isr{organise} such participation in a way that avoids obvious pitfalls, such as administrative inefficiency and tyranny by majorities or elites that assume control of the local agenda.

In Chapter \ref{chap:6}, I will shed light on this question by considering the Norwegian institution of land consolidation, which has a very long tradition behind it. It is a flexible \isr{framework} which includes, among other things, a template for establishing institutions that can function as a LAD. I will focus on how land consolidation functions in cases of economic development that would otherwise likely be pursued by eminent domain. The case study is based on considering hydropower development, but I will also discuss planning law and development more generally, as the Norwegian government is now considering making consolidation, traditionally a rural institution, a primary mechanism for land development even in urban areas.

Before I delve into this, I will present an overview of Norwegian hydropower and the role of waterfalls  as private property. This will serve as an introduction to the second part of this thesis, to which I now turn.