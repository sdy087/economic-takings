%\documentclass[12pt,a4paper]{memoir} % for a long document
\documentclass[12pt,a4paper]{article} % for a short document

\usepackage[utf8]{inputenc} % set input encoding to utf8
\usepackage[style = oscola]{biblatex}
\usepackage{setspace}
\doublespacing

% Don't forget to read the Memoir manual: memman.pdf

\newcommand{\noo}[1]{}

\addbibresource{thesis.bib}

%%% BEGIN DOCUMENT
\begin{document}

\title{The {\it Pointe Gourde} principle and its effect on economic development takings}

\maketitle

\begin{abstract}In expropriation cases, the compensation question often occupies centre stage. Moreover, the way it is resolved largely influences the perceived legitimacy of the interference. When the expropriation order affects the value of the taken property, the question of compensation becomes particularly tricky. Many jurisdictions employ so-called ``elimination rules’’ in such cases, to ensure that changes in value due to the expropriation scheme are disregarded. In this paper, I consider elimination rules in UK and Norwegian law, and I focus particularly on situations when expropriation takes place to further economic development. The policy reasons for elimination rules become less clear in such situations, and it has been argued that mechanisms for benefit sharing should be used instead. For a concrete example of such a mechanism, I look to recent case law on expropriation for commercial hydropower in Norway, developed by the district `appraisal courts’, special judicial bodies that rely largely on the discretion of lay people. I discuss how the Norwegian Supreme Court has partly confirmed and partly rejected the new approach. In particular, I note how they have applied an elimination rule similar to what is known as the ``Pointe Gourde’’ principle in common law, to reject benefit sharing for some case types that the appraisal courts have judged differently. I analyse these developments against the debate on the Pointe Gourde principle in the UK, arguing that the rule is often inappropriate when expropriation benefits a commercial scheme.
\end{abstract}

\section{Introduction}

The distinguishing feature of an economic development taking, as that notion has typically been used, is that it gives a third party an opportunity to profit commercially. This may even be the primary aim of the project, with the public benefiting only indirectly through potential economic and social ripple effects. In the absence of compulsion, owners who contribute property to development projects of this kind would naturally expect a share in the profit resulting from profitable use of their land. However, in most jurisdictions, including the US, the UK and Norway, the rules used to calculate compensation following condemnation typically prevents benefit sharing of this kind.\footnote{See, e.g., \cite[965-966]{fennell04}.} 

The most important common law mechanism responsible for this is referred to as the {\it Pointe Gourde} principle, named after a .... case.\footnote{....} In short, it states that any changes in value due to the expropriation `scheme' is to be disregarded when calculating compensation. Depending on whether the scheme increases or decreases the value of the land, this principle will lead to a corresponding increase or decrease in the level of compensation payable to owners. Hence, the principle has both a positive and a negative dimension, the policy reasons for which do not necessarily coincide. 

However, in `classical' cases, when land is taken for non-commercial projects, the core idea behind both dimensions of the principle are easily justified within a narrative of corrective justice. If an expropriation scheme increases the value of land that is taken, the fact that the scheme required the land to be taken in the first place means that the additional value cannot easily be construed as the owner's `loss'. Without the scheme, there would be no additional value. Hence, the negative dimension of the principle appears sound. On the other hand, if the scheme decreases the value of the land that is taken, this is undoubtedly an additional `loss' for the owner, compared to the situation he would have been in had the taking not occurred. Hence, a claim for additional compensation appears well-reasoned.

Complications arise, however, as soon as one asks about the scope of the principle. Consider, for instance, the situation that arises if private land is taken to build a stretch of road, one which represents the last stage of a comprehensive transport plan that has caused a dramatic surge in local property prices over several years. Should the owner be compensated based on what the value of his land would have been if the transport plan as a whole was never implemented? What if the exact location of the final stretch of road was not specified in the transport plan? More generally, is it really so obvious that an owner whose land is taken for an important public project must accept to be worse off than his neighbours, who enjoy the positive effect of the project?

Problems of scope have been flagged as highly problematic by the Supreme Court of England and Wales, with a partial clarification offered in the case of {\it Waters}. In this paper, I will focus instead on the special issue that arises when land is taken for commercial projects. In such cases, I argue, there is an evolving consensus among academics and professionals alike that the negative dimension of the {\it Pointe Gourde} principle is lacking in good policy reasons, at least if it is understood broadly. The question arises how such cases are to be resolved. I look at various proposals that have been made based on a corrective justice 
approach to compensation. I note, in particular, that several recent proposals from this strand correspond to rules that have actually been applied in Norway for more than 100 years, in cases involving expropriation of waterfalls for hydropower production. I go on to identify shortcomings of these approaches, by showing how, in the Norway at least, the corrective justice perspective, and the focus on clear and unambiguous rules, tends to disproportionately benefit the commercial third party that benefits from takings. 

I then argue that the intractable challenges raised by the {\it Pointe Gourde} principle in the context of economic development takings shows, more than anything else, that a corrective justice approach to such takings is inadequate, and should be replaced by an approach that focuses on the owners' {\it right to participate}, {\it qua owner}, in for-profit projects involving their land. I illustrate the possible consequence of recognizing such a right, by considering a recent case where waterfall owners were compensated on the premise that if the expropriation had not taken place, they would have cooperated with the commercial developer in carrying out the project. I then conclude that while a compensatory approach to participation rights can work in individual cases, fair benefit sharing is impossible unless, in the majority of cases, the right to participate is translated into frameworks of {\it actual}, possible {\it compulsory} participation, which can remove the need for expropriation altogether.

\section{Blablbablba}

In the US since the {\it Kelo}, a massive amount of critical attention has been directed at economic development takings. Some have argued that they should be banned altogether, a position that has received considerable political support, resulting in an unprecedented body of state legislation aiming to limit the use of eminent domain to further economic development. However, as noted by Ilya Somin, these reforms are largely ineffectual. This illustrates that the use of compulsion to carry out large-scale development projects is more or less inevitable. Moreover, as long as the regulatory system presupposes that such projects are carried out by commercial bodies, or private-public partnerships, the problem of economic development takings is unlikely to disappear.

Some have argued that the debate centering on the legitiamcy of such takings is unhelpful, and that a perspective based on the right to compensation should be adopted instead. It has been noted, in particular, that compensation is typically felt to be inadequate in economic development cases.

\noo{ particular, various {\it elimination rules} are typically in place to ensure that compensation is based entirely on the pre-project value of the land that is being taken.\footcite[See][81]{ackerman06} The policy reasons for such rules is that they ensure that the public does not have to pay extra due to its own special want of the property. After all, this is one of the main purposes of using eminent domain in the first place; to ensure that the public does not have to pay extortionate prices for land needed for important projects. However, when the purpose of the project is itself commercial in nature, there appears to be a shortage of good policy reasons for excluding this value from consideration when compensation is calculated. 

This is especially true when, as in the US, compensation tends to be based on the market value of the land taken. Why should a commercial condemner's prospect of carrying out economic development with a profit be disregarded from the assessment of market value? In any fair and friendly transaction among rational agents, one would expect benefit sharing in a case like this. Yet for economic development backed up by eminent domain, the application of elimination rules ensures that all the profit goes to the developer. 
}

Some authors have argued that failures of compensation is at the heart of the economic takings issue and that worry over the public use restriction is in large part only a response to deeper concerns about the `uncompensated increment' of such takings.\footcite[See][962]{fennell04} In addition to the lack of benefit sharing, previous work has identified two further problems of compensation that also tend to become exasperated in economic development cases. First, the problem of `subjective premium' has been raised, pointing to the fact that property owners often value their own land higher than the market value, for personal reasons.\footcite[963]{fennell04} For instance, if a home is condemned, the homeowner will typically suffer costs not covered by market value, such as the cost of moving, including both the immediate `objective' logistic costs as well as more subtle costs, such as having to familiarize oneself with a new local community. Second, the problem of `autonomy' has been discussed, arising from the fact that an exercise of eminent domain deprives the landowner of their right to decide how to manage their property.\footnote{Discussed in \cite[966-967]{fennell04}. For a general personhood building theory of property law, see \cite{radin93}. For a general economic theory of the subjective value of independence, see \cite{benz08}.}



\end{document}



For the last part of the Chapter, I consider US law, where economic development takings are already considered as an important special category. I track the development of the case law on the public use restriction in the Fifth Amendment and in various state constitutions, from the early 19th Century up to the present day.\footnote{The public use clause in the US constitution was not held to apply to state takings until the late 19th Century, see \cite{chicago97}.} Many writers assert that the 19th and early 20th Century was characterized by a ``narrow'' approach to public use which eventually gave way to a broader conception.\footnote{See, e.g., \cite[483]{walt11}; \cite[203-204]{allen00}. For a more in-depth argument asserting the same, see \cite{nichols40}.} Against this, I argue that it is more appropriate to think of this period as one when courts adopted a {\it broad} approach to judicial scrutiny of the takings purpose at state level. Importantly, I also argue that while different state courts expressed different theoretical views on the meaning of ``public use'', there was a growing consensus that the approach to judicial scrutiny should be contextual, focused on weighing the rationale of the taking against the concrete social, political and economic circumstances of the local area.\footnote{A summary of state case law that supports this view is given in the little discussed Supreme Court case of \cite{hairston08}.}  In particular, I argue that early state courts did not focus as much on the exact wording of the constitutional property clause as many later commentators have suggested.

I go on to show that the doctrine of deference that was developed by the Supreme Court early in the 20th Century was directed primarily at state courts, not state legislatures and administrative bodies.\footnote{See \cite{vester30} (echoing and citing \cite{hairston08}).} I then present the case of {\it Berman}, arguing that it was a significant departure from previous case law.\footcite{berman54} After {\it Berman}, deference was suddenly taken to mean deference to the (state) legislature, so there would be little or no room for judicial review of the takings purpose. I go on to present the subsequent developments at state level, characterized by increasing worry that the eminent domain power could be abused by powerful commercial actors. I discuss the case of {\it Poletown}, where a neighborhood of about 1000 homes was razed to provide General Motors with land to assemble a car factory.\footcite{poletown81} I link this to the subsequent controversy that arose over {\it Kelo}, suggesting that it should be seen as the eventual backlash of {\it Berman}, a consequence of abandoning the contextual approach to public use in favor of an almost absolute rule of deference.

After the historical overview, I go on to briefly present the vast amount of research that has targeted economic takings in the US after {\it Kelo}. I give special attention to writers that propose new legitimacy-enhancing institutions for facilitating economic development of jointly owned land. I focus on two proposals in particular, targeting compensation and participation respectively.\footcite{lehavi07,heller08} These proposals will serve as important reference points later on, when I consider the Norwegian appraisal  and land consolidation courts in Chapters 4 and 5.

\noo{ In this Section I map the main problems that have been discussed in relation to cases of economic development takings in the US. I note how economic development cases attract much more attention than in Europe, and I present an historical overview aiming to give the reader an idea of how the debate regarding economic takings have developed into its present state. I argue that the recent surge of academic interest in this topic reflects a recent crisis of confidence in the legitimacy of takings in the US, specifically related to cases when the rationale behind the use of compulsion is commercial in nature. Following up on this, I go on to analyse in more detail some recent approaches used to address economic development takings. I argue that recent work from the US provide a useful conceptual framework for addressing such takings as a special category, and that it also serves to illustrate that such takings should indeed be considered a separate issue. %in particular, that the worry in these cases is that there is an imbalance of interest and power that may lead to both real and perceived abuses.

I will pay particularly attention to how constitutional objections to economic development takings, based on strict interpretations of the fifth amendment, appear less important to the debate than it may appear at first sight. It is true that many commentators, including some scholars, have argued for a strict understanding of property protection and they express their concerns more forcefully that what is commonly seen in Europe. However, many scholars in the US also rely on a broad and contextual understanding of property rights, and their work, which has perhaps not received the same level of attention, do in fact closely resemble the approach to property protection adopted by the ECtHR. I make special note of the fact that a great number of voices approach the issue as a question that crucially involves notions of fairness and democratic legitimacy.  I argue that such arguments are often socio-legal in nature, emerging from empirical considerations. Hence, they appear highly relevant also outside the context of the US legal system and the political tradition that accompanies it.

Many academics in the US argue that economic takings are particularly problematic under current practices and that there is need for reform. Moreover, the debate after {\it Kelo} shows that concern about such takings has become more widespread, also among academics that do not necessarily endorse a liberal, individualistic, view on the nature of property. In the last part of this section, I examine a few recent responses that aim to provide a bride across the ideological divide that otherwise dominates the takings debate in the US. I will refer to these as {\it institutional} approaches. They are characterized by a focus on the administrative and judicial procedures that are used in economic takings cases. The idea is that legitimacy of such takings can only be achieved if special procedures are followed, to ensure fairness both in the decision-making process and with regards to the issue of what compensation should be paid following condemnation. Importantly, the need for special procedures is identified with the imbalance of power that tend to exist between developers and property owners in such cases. Hence, the proposals do not argue in favor of limiting the use of eminent domain generally, and should not be read as rhetoric in favor of a more absolutist or liberal view of property rights. As such these proposal become more broadly relevant, also outside the US context. It is also a perspective that directly ties the US debate to the case study presented in subsequent chapters of this thesis. In particular, in Section \ref{sec:ins} I give an in depth presentation of two recent reform proposals, one regarding compensation principles and the other focusing on the decision-making process as such. Both contain (aspects of) the two working institutions I will consider in my case study, in Chapter \ref{chap:4} and Chapter \ref{chap:5} respectively.


The first is a proposal by Professors Lehavi and Licht, which argues for the introduction of special-purpose development corporations to ensure appropriate levels of compensation in economic development cases.\footcite{lehavi07} The crux of their proposal is to award property owners shares in a development company that is set up to bargain with potential developers. Importantly, the eminent domain decision precedes this round of bargaining, so the owners cannot threaten to refuse selling the land in order to get additional compensation based on the public need for development. Bargaining is restricted to the commercial element of the project, reflected in competition that may arise among developers interested in carrying out the project. This proposal will serve as a point of departure when we consider the Norwegian appraisal courts and recent case law on compensation of waterfalls.

The second proposal we will consider in depth is due to Professors Heller and Hills, who argue for the introduction of so-called ``land assembly districts'', institutions for participatory pooling of property for development projects.\footcite{heller08} Land assembly  districts are designed to partially replace the use of eminent domain for economic development, giving property owners both a template on which to bargain with developers and also the final say on whether land assembly should happen at all. The authors argue that with appropriate procedures in place for making collective decisions, such a system will be sufficient to avoid holdouts preventing socially desirable projects. In fact, they argue quite convincingly that even a simple scheme of majority voting will be sufficient in many cases, due to the commercial incentives that are present when land assembly is needed for economic development. Importantly, their proposal is not mean to apply to land assembly in general. It should only be used when reasonable market conditions can be achieved without the use of eminent domain. As we will see in Chapter \ref{sec:5}, this proposal corresponds closely to important aspects of various land consolidation procedures currently in operation under Norwegian law. In particular, I will show how recent case law on land consolidation for hydroelectric development in Norway demonstrates the efficiency and usefulness of a similar procedure, used to assemble water rights under fragmented ownership in Norway.

Before I move on to discuss these recent proposals in more detail, I will give an historical background on economic takings in the US. This will give the reader a better appreciation of the developments that led to the current climate of debate, showing how economic takings became a natural special category in the US. In particular, I will argue that the US system has tended to be more open to the idea of commercial projects benefiting from eminent domain, so that the level of tensions here were naturally higher than those found in Europe. However, recent European trends towards greater levels of public/private commercial partnerships in important public projects may suggest that the US discourse will soon become easier to recognize also in the European setting. The history of the debate in the US shows, in any event, that conflicts over takings law become aggravated whenever the perception takes hold that powerful commercial interests are permitted to usurp the process to their own advantage.
}

\section{From broad judicial review to strict deference: An historical overview}

Going back to the time when the Fifth Amendment was introduced, there is not much historical evidence explaining why the takings clause was included in the bill of rights, and little in the way of guidance as to how it was originally understood. James Madison, who drafted it, commented that his proposals for constitutional amendments were intended to be uncontroversial to Congress.\footnote{See letters from Madison to Edmund Randolph dated 15 June 1789 and from Madison to Thomas Jefferson dated 20 June 1789, both included in \cite{madison79}.}  Hence, it is natural to regard it as a codification of an existing principle, rather than a novel proposal. Indeed, several State constitutions pre-dating the Bill of Rights also included takings clauses, and they all seem to be largely based on a codification of principles from English Common law.\footcite[See][299]{johnson11}

As we discussed in subsection \ref{subsec:1} above, the typical English attitude from this time, which was also reflected in the law, held private property in very high regard. On this background it is not surprising that Madison regarded the property clause as an uncontroversial amendment.\footnote{Indeed, early American scholars also emphasized the importance of private property. For instance, in his famous {\it Commentaries}, James Kent described the sense of property as ``graciously implanted in the human breast'' and declared that the right of acquisition ``ought to be sacredly protected'', \cite[see][257]{kent27}.} Its importance may in fact have been greater as a legitimizing force, increasing confidence in the regulatory power of the newly established state by setting up clear parameters for the exercise of that power.  However, while the principle itself was regarded as self-evident, it was never clear what it would mean in practice, particularly in cases when takings where challenged on the basis that they were not for a ``public use'''.\footcite[See][317]{johnson11} 

There are two points that I would like to record about the early common law in the US  in this regard. First, the distinction between public use and public purpose does not appear to have been considered sharp. In his {\it Commentaries}, James Kent first makes clear that the power of eminent domain is for ``public use, and public use only", but then goes on to qualify this by stating that a taking which served a ``purpose not of a public nature'' would be unconstitutional.\footcite[See][275-276]{kent27}  He does not address this limitation in any detail, however, suggesting that it was not the subject of much debate at this time. To the founders, it seems that the right to compensation was considered the more important principle, something that is also reflected in the {\it Commentaries}.\footnote{James Kent held it to be  ``founded in natural equity'' and described it as an ``acknowledged principle of universal law'', \cite[see][276]{kent27}.} The public use limitation was probably taken for granted as a matter of principle, while it had not yet proved problematic as a matter of practical adjudication. Moreover, it appears to have been accepted that takings which clearly benefited the public would be legitimate regardless of whether or not the property was physically put to use by the public.\footcite{johnson11}

An interesting early illustration of how courts approached takings controversies at this time can be found in {\it Stowell v Flagg}, a Massachusetts case from 1814. In this case, a landowner complained that his land had been flooded by a mill and sought a remedy in common law. The mill owner protested, however, since he was entitled to flood the land according to a special mill act, which allowed him to exercise the power of eminent domain to gain the right to flood his neighbor (provided statutory compensation was paid). The focus in the case was on whether a common law claim for damages could still be made, irrespectively of the act's clear intention to deprive the affected neighbors of this opportunity. Hence, the court implicitly dealt with the legitimacy of the mill act itself, and they actively engaged with the public use requirement in the state constitution when making their assessment.\footcite{stowell14} In the end, they found that the act was legitimate, and they highlighted the purpose of the interference, commenting that ``these mills, early in the settlement of this country, were of great public necessity and utility''.\footcite[366]{stowell14} 

At the same time, however, the court had misgivings about how the act had come to be applied and expressed concern that ``the legislature, as well as the courts of law in this state, seem to have been disposed rather to enlarge, than to curtail, the power of mill owners''.\footcite[366]{stowell14} Still, after noting that affected land owners were entitled to compensation under the act, the court concluded that the act had to be observed and that it precluded any claims for damages under common law. Hence, the case is an early example of judicial deference to the legislature in takings cases, while also illustrating that the public use requirement was beginning to emerge as a potentially problematic issue in its own right. The presiding judge stated that he could not help thinking that the statute was ``incautiously copied from the ancient colonial and provincial acts'', but still held in favor of the mill owner,  concluding that ``as the law is, so must we declare it''.\footcite[368]{stowell14}

While judicial deference was recognized as a guiding principle early on in US takings law, it is important to note in this regard that eminent domain was seldom used in a way that would raise serious controversy. English common law, while lacking clearly defined constitutional safeguards, was, as we have already mentioned, based on a fundamentally cautious attitude, ensuring that the power would typically only be used as a last resort. As Professor Meidinger notes, the British were never really charged with abuse of eminent domain, and private property tended to be respected, also in the colonies.\footcite[17]{meidinger80} This undoubtedly influenced early US law. Indeed, the importance of constitutional limits on the taking power was made clear by the Supreme Court early on, as a matter of principle.\footnote{As reflected in {\it de dicta} comments from {\it Calder v Bull} and {\it Vanhorne’s Lessee v Dorrance}, see \cite[388]{calder98}; \cite[310]{vanhorne95}.} Hence, the relative lack of judicial interest in the question of legitimacy does not appear to have been due to a broad view on the scope of eminent domain, but an established practice of narrow use of that power, inherited from the English.
\noo{
The Legislature declare and enact, that such are the public exigencies, or necessities of the State, as to authorise them to take the land of A. and give it to B.; the dictates of reason and the eternal principles of justice, as well as the sacred principles of the social contract, and the Constitution, direct, and they accordingly declare and ordain, that A. shall receive compensation for the land. But here the Legislature must stop; they have run the full length of their authority, and can go no further: they cannot constitutionally determine upon the amount of the compensation, or value of the land. Public exigencies do not require, necessity does not demand, that the Legislature should, of themselves, without the participation of the proprietor, or intervention of a jury, assess the value of the thing, or ascertain the amount of the compensation to be paid for it. This can constitutionally be effected only in three ways.
1. By the parties that is, by stipulation between the Legislature and proprietor of the land.
2. By commissioners mutually elected by the parties.
3. By the intervention of a Jury.
}
The traditional attitude to eminent domain would eventually give way to a more expansive approach, however. This development became particularly marked during the period of great economic expansion and industrialization in the mid to late 19th century, when eminent domain was increasingly used to benefit (privately operated) railroads, hydroelectric projects, and the mining industry.\footcite[23-33]{meidinger80} During this time, it also became increasingly common for landowners to challenge the legitimacy of takings in court, undoubtedly a consequence of the fact that eminent domain was now used more widely, for new kinds of projects.\footcite[24]{meidinger80} Controversy arose particularly often with respect to mill acts.\footnote{\cite[24]{meidinger80}. See also \cite[306-313]{johnson11} and \cite[251-252]{horwitz73}.} Such acts were found throughout the US, and many of them dated from pre-industrial times when mills were primarily used to serve the needs of self-sufficient agrarian communities.\footnote{A total of 29 states had passed mill acts, with 27 still in force, when a list of such acts was compiled in \cite[17]{head85}. According to Justice Gray, at pages 18-19 in the same, the ``principal objects'' for early mill acts had been grist mills typically serving local agrarian needs at tolls fixed by law, a purpose which was generally accepted to ensure that they were for public use.}  However, following economic and technological advances, acts that were once used to facilitate the construction of grist mills would increasingly also be relied on by developers wishing to harness hydropower for manufacturing, and eventually, for hydroelectric projects.\footnote{See, e.g., \cite[18-21]{head85} and \cite[449-452]{minn06}.}

The mill acts typically contained provisions that enabled the mill developer to condemn both property needed for the construction itself as well as the right to damage surrounding land by flooding or deprivation of water. Such takings became increasingly controversial, however, and many legitimacy cases came before state courts in the late 19th and early 20th century. In the next subsection I present some of these cases, to shed light on how states courts developed their own approach to the question of legitimacy of takings.

\subsection{Legitimacy in state courts}\label{subsec:state}

In the mil cases, we find the first clear evidence of how the public use requirement was put to use to enable state courts to scrutinize the legitimacy of takings. Generally speaking, when a court upheld an interference in private property, it would place decisive weight on the broader purpose of interference, typically by arguing that economic ripple effects ensured that the mill was in the public interest even if the public would not literally make use of it.\footnote{See, e.g., \cite{hazen53,scudder32,boston32}. A more comprehensive list of cases adopting a broad view can be found in \cite[617]{nichols40}.} By contrast, when a court decided that an interference was unconstitutional (with respect to state constitutions), it would often focus on the use made of the mill, arguing that it did not directly benefit the public in the sense required by the public use restriction.\footnote{See, e.g., \cite{sadler59,ryerson77,gaylord03,minn06}. A more comprehensive list can be found in {\it Public benefit or convenience as distinguished from use by the public as ground for the exercise of the power of eminent domain} 54 ALR 7 (American Law Reports, 1928).} For a time, a doctrine which sought to distinguish between takings for public use and takings for a public purpose, played quite a significant role in many states. Under this doctrine, only those takings that were deemed to qualify as public use takings under a narrow view of that term would be upheld.\footnote{Professor Nichols goes as far as to conclude that this emerged as the ``majority'' opinion on public use, see \footcite[617-618]{nichols40}. But contrast this with \cite{berger78} and \cite[24]{meidinger80}, who argue that the narrow view was only dominant in a handful of states, led by New York.}

\noo{ For instance, in the case of {\it Gaylord v. Sanitary Dist. of Chicago}, the Supreme Court of Illinois held the state Mill Act to be unconstitutional, as it was not limited to traditional flour mills. In doing so, the court observed that public use was ``something more than a mere benefit to the public''.\footcite[524]{gaylord03} Similar sentiments were expressed in other decisions striking down uses of eminent domain for mill construction, for instance in Vermont, Michigan and New York.\footnote{References.}}

It is tempting to associate the narrow view on public use with a more restrictive attitude towards the use of eminent domain. Similarly, it is natural to assume that a broad view on public use suggests a more relaxed attitude. To some extent, the primary sources warrant this; unsurprisingly, those who endorsed a broad view on the public use question also often spoke in favor of judicial deference in legitimacy cases, while those endorsing a narrow view tended to emphasize the importance of constitutional safeguards against abuse of eminent domain. However, it seems that both groups were quite heterogeneous and that differences of opinion about the public use requirement did not necessarily reflect any deep ideological divisions.

It is clear, for instance, that many of the courts which favored a broad interpretation of public use still viewed the constitutional limitation on the takings power as an important safeguard, not only as a guarantee for compensation but also as a restriction on the purpose of takings. Indeed, it seems that most late 19th Century Courts, including those that upheld economic takings, were influenced by the growing body of case law across the US that actively scrutinized takings, sometimes striking them down. In particular, it seems that the strict deferential view was largely abandoned in economic takings cases during this period. Deference to the legislature still played an important role and was typically called on as an important argument in takings cases. However, it became much more common to discuss legitimacy also in terms of substantive arguments, by directly addressing the context and circumstances of the taking complained of. I believe this is an important insight to record about the case law from this period; despite differences of opinion about the meaning of public use, a consensus appears to have emerged that judicial review of legitimacy was appropriate and important in economic takings cases.

A good example is the case of {\it Dayton Gold \& Silver Mining Co. v. Seawell}, concerning a Nevada Act which stipulated that mining was a public use for which the power of eminent domain could be exercised to acquire additional rights needed to facilitate extraction.\footcite{seawell76} The Supreme Court of Nevada decided that the Act was constitutional and adopted a broad understanding of the property clause in the Nevada constitution.\footnote{Nev Const Art 8 § 1.} Interestingly, it argued for this interpretation partly on the basis that it would provide {\it better} protection for landowners:\noo{Why not? A hotel is used by the public as much as a railroad. The public have the same right, upon payment of a fixed compensation, to seek rest and refreshment at a public inn as they have to travel upon a railroad. 

One purpose is, so far as the legal rights of the citizen are concerned, as public as the other.}

\begin{quote}
If public occupation and enjoyment of the object for which land is to be condemned furnishes the only and true test for the right of eminent domain, then the legislature would certainly have the constitutional authority to condemn the lands of any private citizen for the purpose of building hotels and theaters. [...] Stage coaches and city hacks would also be proper objects for the legislature to make provision for, for these vehicles can, at any time, be used by the public upon paying a stipulated compensation. It is certain that this view, if literally carried out to the utmost extent, would lead to very absurd results, if it did not entirely destroy the security of the private rights of individuals. Now while it may be admitted that hotels, theaters, stage coaches, and city hacks, are a benefit to the public, it does not, by any means, necessarily follow that the right of eminent domain can be exercised in their favor.\footcite[410-411]{seawell76}
\end{quote}

The quote shows that a broad understanding of ``public use'' need not be synonymous with a less cautious attitude to abuse of the takings power. Indeed, while the Court decided to uphold the Act, it did so only after a very careful assessment of both legal arguments and factual circumstances. In particular, the Court considered the importance of mining, concluding that it was the ``greatest of the industrial pursuits'' in the state, and that all other interests were ``subservient'' to it.\footcite[409]{seawell76} Moreover, the Court commented that the benefits of the mining industry was ``distributed as much, and sometimes more, among the laboring classes than with the owners of the mines and mills''.\footcite[409]{seawell76}

This shows that the Court actively engaged with the purpose of the Act, thoughtfully assessing it against the constitution. Importantly, it did not do so in isolation, as a linguistic exercise or by attempting to recreate its ``original intent''. Rather, the court approached the constitutional safeguard by making detailed references to the prevailing social and economic conditions in the state of Nevada. The Court noted the importance of deference to the legislature on matters of policy, but it did so only after it had satisfied itself that the Act could be ``enforced by the courts so as to prevent its being used as an instrument of oppression to any one''.\footcite[412]{seawell76} More generally, the court commented as follows on the public purpose test that had to be performed in takings cases, elucidating on the principles on which it should be founded:

\begin{quote}
 Each case when presented must stand or fall upon its own merits, or want of merits. But the danger of an improper invasion of private rights is not, in my judgment, as great by following the construction we have given to the constitution as by a strict adherence to the principles contended for by respondent.\footcite[398]{seawell76}
\end{quote}

In light of this, {\it Dayton Gold \& Silver Mining Co. v. Seawell} must be regarded as an early example of a {\it contextual} approach to legitimacy, characterized by the willingness of the Court to engage in a fairly detailed analysis of the concrete circumstances and consequences of takings. A formalistic approach based on the phrase ``public use'' was abandoned, but not in favor of general deference. Rather, a more nuanced view was adopted, to respect the idea that the legislature should have the final say on policy while also recognizing that courts should play a crucial role in protecting citizens from abuse of the takings power. 

The case is not unique, but rather exemplifies the type of reasoning that was used in economic takings cases at this time. Interestingly, many common elements exist between courts that upheld and struck down such takings, irrespectively of whether or not they subscribed to a narrow or broad view on the public use test. One example is {\it Ryerson v. Brown}, a case often cited as an authority in favor of a narrow view.\footcite{ryerson77} Here the Supreme Court of Michigan explicitly qualifies its decision by stating that it is ``not disposed to say that incidental benefit to the public could not under any circumstances justify an exercise of the right of eminent domain'', hardly a clear endorsement of the narrow rule. The case concerned the constitutionality of a mill act, and while the court argues that public use should be taken to mean ``use in fact'', it is clear that ``use'' is understood rather loosely, not literally as physical use of the property that is taken.\footnote{The court explains its stance on the public use restriction by stating (emphasis added) ``it would be essential that the statute should require the use to be public in fact; in other words, that it should contain provisions entitling the public to {\it accommodations}.'' The court continues with an illustrative example: ``A flouring mill in this state may grind exclusively the wheat of Wisconsin, and sell the product exclusively in Europe; and it is manifest that in such a case the proprietor can have no valid claim to the interposition of the law to compel his neighbor to sell a business site to him, any more than could the manufacturer of shoes or the retailer of groceries. Indeed the two last named would have far higher claims, for they would subserve actual needs, while the former would at most only incidentally benefit the locality by furnishing employment and adding to the local trade''. See \cite[336]{ryerson77}.} Moreover, when clarifying its starting point for judicial scrutiny of mill acts, the court explains that ``in considering whether any public policy is to be subserved by such statutes, it is important to consider the subject from the standpoint of each of the parties''. Following up on this with regards to the act in question, the court finds that `` the power to make compulsory appropriation, if admitted, might be exercised under circumstances when the general voice of the people immediately concerned would condemn it''. After considering this and other possible consequences of mill development under the act, the court eventually declares it to be unconstitutional, summing up its assessment as follows: ``What seems conclusive to our minds is the fact that the questions involved are questions not of necessity, but of profit and relative convenience''.\footcite[336]{ryerson77}

Hence, far from nitpicking on the basis of the public use phrase, the court adopts a contextual approach to takings that is in fact rather similar to the approach of {\it Dayton Gold \& Silver Mining Co. v. Seawell}. The outcome it different, but it is also based on a different assessment of the context and the consequences of the takings complained of. Importantly, the case does not rest on any {\it a priori} assumption that economic takings of the kind in question could not meet a public use test -- no general rule is relied on at all. Hence, it is somewhat strange that later commentators have focused on the case for its comments on public use rather than its broad, albeit perhaps somewhat conservative, assessment of legitimacy. 

Many of the important cases from the late 19th Century, on both sides of the public use debate, shares many crucial features with the two cases discussed above.\footnote{See, e.g., \cite{scudder32} (Eminent domain power upheld, but said: ``The great principle remains that there must be a public use or benefit. That is indispensable. But what that shall consist of, or how extensive it shall be to authorize an appropriation of private property, is not easily reducible to a general rule. What may be considered a public use may depend somewhat on the situation and wants of the community for the time being.''), \cite{fallsburg03} (Eminent domain struck down, on holding that ``the private benefit too clearly dominates the public interest to find constitutional authority for the exercise of the power of eminent domain''), \cite[538]{board91} (Eminent domain struck down, qualified by ``not only must the purpose be one in which the public has an interest, but the state must have a voice in the manner in which the public may avail itself of that use'').} In my opinion, this points to an interesting alternative perspective on legitimacy adjudication from this time. Some commentators describe the case law as chaotic, with competing conceptions of constitutional limits competing for dominance.\footcite{berger78,meidinger80}. I think this is more accurate than saying that a narrow interpretation of public use developed as a general rule. However, I also find evidence that there was in fact a broad consensus in this period regarding the need for special judicial scrutiny of economic development cases. State courts widely engaged in contextual assessment of legitimacy, and they were conscious of the special challenges that arose in a time when eminent domain was being used to facilitate economic expansion that would benefit specific commercial actors. Differences of opinion about public use terminology was an important aspect of this, but it was rarely considered in isolation from other aspects. On a deeper lever, the fact that the public use debate was regarded as important in the first place clearly suggests that deference to the legislature was not held to be an exhaustive answer to the question of legitimacy. This, in my opinion, is an important observation which appears to have been somewhat overlooked in the literature. 

It is an observation that I think is relevant not only in relation to state law, but also when considering the takings doctrine that was later developed by the Supreme Court. While the narrow view of public use was indeed losing ground at the beginning of the 20th Century, the doctrine of extreme deference that was about to be adopted at the federal level represents a largely new development. The new deference was not originally directed at the legislature, in particular, but primarily towards the judiciary at the state level. Hence, it represent a development that is in some sense incomparable to the earlier case law from the states. The balance of power between states and the federal government also played an important role, which should not be overlooked.

\subsection{Legitimacy as discussed in the Supreme Court}\label{subsec:US}

Initially, the Supreme Court held that the takings clause in the US Constitution did not apply to state takings at all.\footcite{barron33} Federal takings, on the other hand, were of limited practical significance since the common practice was that the federal government would rely on the states to condemn property on their behalf.\footcite[30]{meidinger80}. This changed towards the end of the 19th Century, particularly following the decision in {\it Trombley v. Humphrey}, where the Supreme Court of Michigan struck down a taking that would benefit the federal government.\cite{trombley71} Not long after, in 1875, the first Supreme Court adjudication of a federal taking case occurred, marking the start of the development of the Supreme Court's own doctrine on public use and legitimacy.\footcite{kohl75} Eventually, in 1897, the Court would also hold that state takings could be scrutinized under the takings clause of the constitution.\footcite{chicago97} This was a development that can be traced to the passage of the Fourteenth Amendment to the Constitution after the civil war, concerning due process.\footcite{johnson11}. Indeed, some early Supreme Court cases dealing with state takings were adjudicated against the due process clause directly.\footnote{See, e.g., \cite{head85}.}

After the Supreme Court started developing its own case law on the legitimacy issue, the deferential stance soon became entrenched. As argued by Professor Horwitz, the mid to late 19th Century was the period in US history when control over property was transferred on a massive scale from agrarian communities to various agents of industrial expansion.\footcite{horwitz73} Moreover, it was a period of great optimism about the ability of {\it laissez faire} capitalism to ensure progress and economic growth. This was also reflected in the case law on eminent domain, particularly as developed by the Supreme Court. A particularly clear expression of this can be found in {\it Mt. Vernon-Woodberry Cotton Duck Co v Alabama Interstate Power Co}.\footcite{vernon16}  This case dealt with the legitimacy of a condemnation arising from the construction of a hydropower plant, which the Alabama Supreme Court had upheld against claims that it was unconstitutional under the constitution of Alabama. The presiding judge held that it was valid using quite brisk language:

\begin{quote}The principal argument presented that is open here, is that the purpose of the condemnation is not a public one. The purpose of the Power Company's incorporation, and that for which it seeks to condemn property of the plaintiff in error, is to manufacture, supply, and sell to the public, power produced by water as a motive force. In the organic relations of modern society it may sometimes be hard to draw the line that is supposed to limit the authority of the legislature to exercise or delegate the power of eminent domain. But to gather the streams from waste and to draw from them energy, labor without brains, and so to save mankind from toil that it can be spared, is to supply what, next to intellect, is the very foundation of all our achievements and all our welfare. If that purpose is not public, we should be at a loss to say what is. The inadequacy of use by the general public as a universal test is established. The respect due to the judgment of the state would have great weight if there were a doubt. But there is none.\footcite[]{vernon16}
\end{quote}

The quote serves as an indication of how deference was fast gaining ground, without yet being established doctrine. On the one hand, the Court stresses that deference to the {\it state} judgment (rather than the judgment of the legislature) should be given great weight in legitimacy cases. On the other hand, it prefers to conclude on the basis of its own assessment of the purpose of the taking. This assessment, however, is not particularly grounded in the circumstances on the ground in Alabama, being based rather on sweeping assertions about the ``organic relations of modern society'' and the desire to ``save mankind from toil that it can be spared''. 

This judgment, from 1916, was given during the so-called {\it Lochner} era of jurisprudence in the US, when the Supreme Court  would famously engage in active censorship of regulation that was meant to promote greater social and economic equality.\footcite{cohen08} In particular, much case law from this period witnesses to a general lack of deference. Hence, it is not unexpected to find that public use cases decided on the basis of substantive arguments. However, it is rather more surprising to find that deference actually played an increasingly important role in takings cases.\footnote{The {\it Lochner} era in general was characterized by courts engaging in censorship of state regulation, but this general tendency is not well reflected in how eminent domain law developed over the same period. This is interesting, as it points to the shortcoming of another commonly held view on property protection, namely that it largely serves the interests of property-owning elites, to the detriment of regulatory efforts to promote social equality. The cases through which {\it Lochner} era courts developed the deferential stance suggest a different interpretation; those who benefited most directly from takings in these cases were commercial interests, not vulnerable groups of society. Moreover, they benefited from acquiring land rights from members of agrarian communities, not from the elites. Hence allowing such takings to go ahead was no affront to the ideology of progress through {\it laissez faire} capitalism, quite the contrary. In particular, if it is true as many have argued, that the {\it Lochner} courts were ideologically committed to the promotion of unrestrained capitalism, there was little reason for them to oppose expansion of eminent domain into the commercial arena: those who would be likely to benefit were market actors who were proposing large scale commercial development projects. Indeed, the case law from this period makes it natural to argue that the deferential stance developed primarily to cater to the needs of the capitalists, under the perceived view that they represented the class which would bring progress and prosperity to the nation as a whole.} As early as { \it United States v. Gettysburg Electric Railway Co.}, a case from 1896, deference was described as a fundamental guiding principle, which should be adhered to except in very special circumstances.\footcite{gettysburg96} In particular, Justice Peckham lended his support to the following deferential stance on the public use test:

\begin{quote}
It is stated in the second volume of Judge Dillon's work on Municipal Corporations (4th Ed. § 600) that, when the legislature has declared the use or purpose to be a public one, its judgment will be respected by the courts, unless the use be palpably without reasonable foundation. Many authorities are cited in the note, and, indeed, the rule commends itself as a rational and proper one.\footcite[680]{gettysburg96}
\end{quote}

The case did not turn on the public use issue, however, as the condemned land would be used for battlefield memorials at Gettysburg, Pennsylvania, clearly a public use. In addition, the case concerned a federal takings, authorized by Congress. In later cases, the deferential stance was not adopted in cases originating from the states. As late as in 1930, in {\it Cincinatti v Vester}, the Supreme Court commented that the ``‘It is well established that, in considering the application of the Fourteenth Amendment to cases of expropriation of private property, the question what is a public use is a judicial one".\footcite[447]{vester30} In this judgment, Chief Justice Hughes also describes in more depth how the judicial assessment of the public use question should be carried out, echoing the contextual approach that had been developed in case law from the states.

\begin{quote}
In deciding such a question, the Court has appropriate regard to the diversity of local conditions and considers with great respect legislative declarations and in particular the judgments of state courts as to the uses considered to be public in the light of local exigencies. But the question remains a judicial one which this Court must decide in performing its duty of enforcing the provisions of the Federal Constitution.\footcite[447]{vester30}
\end{quote}

In {\it Hairston v. Danville \& W. R. Co.}, the same idea was expressed even more clearly by Justice Moody, who surveyed the state case law and declared that ``The one and only principle in which all courts seem to agree is that the nature of the uses, whether public or private, is ultimately a judicial question.''\footcite[606]{hairston08} He continued by describing in more depth the typical approach of the state courts in determining public use cases:

\begin{quote}
The determination of this question by the courts has been influenced in the different states by considerations touching the resources, the capacity of the soil, the relative importance of industries to the general public welfare, and the long-established methods and habits of the people. In all these respects conditions vary so much in the states and territories of the Union that different results might well be expected.\footcite[606]{hairston08}
\end{quote}

Justice Moody goes on to give a long list of cases illustrating this aspect of state case law, showing how assessments of the public use issue is inherently contextual and varies from state to state.\footcite[607]{hairston08} He then cites three further Supreme Court cases, pointing out that all of them express similar sentiments of support for state case law on this issue.\footnote{{\it Falbrook, Clark} and {\it Strickley}} Following up on this, he points out that ``no case is recalled'' in which the Supreme Court overturned ``a taking upheld by the state {\it court} as a taking for public uses in conformity with its laws'' (my emphasis). After making clear that situations might still arise where the Supreme Court would not follow state courts on the public use issue, Justice Moody goes on to conclude that the cases cited `` show how greatly we have deferred to the opinions of the state courts on this subject, which so closely concerns the welfare of their people''.\footcite[606]{hairston08}

I believe {\it Hairston} is an important case for two reasons. First, it makes clear that initially, the deferential stance in cases dealing with state takings was largely directed at the state courts rather than the state legislature. Second, it demonstrates federal recognition of the fact that a consensus had emerged in the states, whereby scrutiny of the public use determination was consistently regarded as a judicial task.\footnote{Indeed, {\it Hariston} provides the authority for {\it Vester} on this point. See \cite[606]{vester30}.} Moreover, the Court clearly looked favorably on the contextual approach adopted in such cases, whereby state courts would look to the concrete circumstances of the individual takings and acts complained of. The Court's approval of this tradition, in particular, is explicitly given as the reason for adopting a deferential stance. Put simply, the judicial test provided at state level was held to be of such high quality that there was little use for further scrutiny; a deferential stance was assumed, but made contingent on the fact that state courts would provide the required judicial scrutiny.

Despite this, {\it Hairston} would later be cited as an early authority in favor of almost unconditional deference in {\it US ex rel Tenn Valley Authority v Welch}.\footcite[552]{welch46} This case concerned a federal taking and it cited {\it US v Gettysburg Electric R Co} as an authority in favor of strong deference with regards to the public use limitation.\footcite{gettysburg96} However, the Court also paused to note that the later case of {\it City of Cincinnati v Vester} expressed the opposite view, that the public use test was a judicial responsibility.\footcite{vester30} In a very selective citation, the Court then purports to resolve this tension by quoting {\it Hairston} and the observation made there that the Supreme Court had never overruled the state courts in takings cases. Effectively, the importance of judicial scrutiny is thereby downplayed, although as we saw, the rationale behind {\it Hairston} was that state courts already offered high-quality judicial scrutiny of the public purpose.

{\it Welch} is particularly important because it is used as an authority in the later case of {\it Berman v Parker}, which endorses almost complete deference to the legislature regarding the public use issue.\footcite[32]{berman54} This case concerned condemnation for redevelopment of a partly blighted residential area in the District of Colombia, which would also condemn a non-blighted department store. In a key passage, the Court states that the role of the judiciary in scrutinizing the public purpose of a taking is ``extremely narrow''.\footcite[32]{berman54} The Court provides only two citations for this claim, one of them being {\it Welch}. The other case, {\it Old Dominion Land Co v US}, concerned a federal taking of land on which the military had already invested large sums in buildings.\footnote{The Court commented on the public use test by saying that ``there is nothing shown in the intentions or transactions of subordinates that is sufficient to overcome the declaration by Congress of what it had in mind. Its decision is entitled to deference until it is shown to involve an impossibility. But the military purposes mentioned at least may have been entertained and they clearly were for a public use''. See \cite[66]{dominion25} Hence, the Court took the view that courts should be cautious in second-guessing the intentions of Congress on the basis of what its subordinates had subsequently done and said. This is far from a general deferential stance on public use, and no cases are cited at all, suggesting further that the Court did not think its remarks would be of general significance. Still, a partial quote, used to substantiate  broad deference to the legislature (not only Congress, but also the states) except when it involves an ``impossibility'', has become commonplace. In particular, such a quote was used in the much discussed \cite[240]{midkiff84}.}
In my view, both cases are weak authorities for prescribing general deference regarding public use. Moreover, both cases are concerned with federal takings only, while in {\it Berman} the Court explicitly says that deference is due in equal measure to the state legislature.\footcite[32]{berman54} It is possible to see this as a {\it dictum}, since the District of Columbia is governed directly by Congress, but it is a passage that has had a great impact on future cases. In effect, {\it Berman} caused departure from a significant and consistent body of case law which recognized the important role of the judiciary, at state level, in assessing the purported public purpose of takings. It did so, moreover, without engaging with any of these cases at all.

In {\it Hawaii Housing Authority v Midkiff}, the Supreme Court further entrenched the principles of {\it Berman}, in a case where the state of Hawaii had made used of the takings power to break up an oligopoly in the housing sector.\footcite{midkiff84}  However, the fact that the case made it to the Supreme Court is perhaps suggestive of an increase in the level of worry and tension associated with eminent domain in the 1980s. Indeed, Justice Sandra Day O'Connor, joined by a unanimous Supreme Court, expressed general disapproval of private takings and she appears to have felt the need to provide further qualification for the deferential view, which she did in part by observing that ``judicial deference is required because, in our system of government, legislatures are better able to assess what public purposes should be advanced by an exercise of eminent domain''. Hence, judicial deference was not regarded as an absolute and systemic imperative, as in Berman, but made contingent on the fact that legislatures are ``better able'' than courts at conducting public purpose tests. Hence, some of the contextual ideas from earlier case law is echoed in the decision, but now with respect to the legislature. It should be noted that {\it Midkiff} follows {\it Berman} also in the authorities consulted, and does not consider the cases which had focused on the importance of judicial scrutiny at state level.

The purpose of interference in {\it Midkiff} was to break up an oligopoly to the benefit of tenants, not to further economic development by allowing commercial interests to take land. Hence, the rationale behind the interference is likely to have struck the Supreme Court as sound and just. Moreover, it seems that such an interference would be easy to uphold also under the doctrine of contextual judicial scrutiny of the public use determination. Indeed, Justice O'Connor partly relies on an assessment of the merits of the taking, pointing out that  ``regulating oligopoly and the evils associated with it is a classic exercise of a State's police powers''. In conclusion, the ``extremely narrow'' room for judicial review set up by {\it Berman} seems to have been replaced by a slightly more nuanced formulation, which nevertheless made clear that a legal precedent of deference had now become entrenched. Fine readings aside, {\it Midkiff} reaffirms the main principle:  the meaning of public use can be broad, and the room for judicial review of governmental assessments in this regard is narrow.

So far we have only commented on how the Supreme Court developed its own doctrine on the public use restriction in the early 20th Century. Given that its role in takings jurisprudence was limited up to this point, it is important to consider also the effect on state case law. In particular, what was the fallout of {\it Berman}, which failed to recognize the importance of the tradition for judicial scrutiny that had developed at the state level? A detailed assessment of this against primary sources will have to be left for future work. However, it seems clear that {\it Berman} had a significant effect, both conceptually and in practice. A clear indication of this can be found in the secondary literature. Indeed, most academics following WW2 seemed to converge towards the view that the public use requirement was of little or no judicial importance. Professor Merrill, in an influential paper from 1986, goes as far as to describe it as a ``dead letter''.\footcite{merrill86}. At the same time, eminent domain became more controversial in this period, as it was also put to use more aggressively by some states.

 Some concrete cases proved particularly controversial, and they were taken to illustrate the dangers of eminent domain, particularly in relation to economic development projects. While the takings power had traditionally been used mostly to condemn agrarian land rights, it was now regularly used to condemn middle class homes. The controversy surrounding the case of Poletown Neighborhood Council v. City of Detroit  illustrates this, and the case marks a watershed moment in the history of  economic development takings in the US.\footcite[See][380-381]{sandefur05} In {\it Poletown}, the Michigan Supreme Court held that it was not in violation of the public use requirement to allow General Motors to displace some 3500 people for the construction of a car assembly factory. The majority 5-2 cites {\it Berman}, commenting that its own room for review of the public use requirement is limited.\footcite[632-633]{poletown81}

The {\it Poletown} decision was controversial, and the minority, especially Justice Ryan, was highly critical of it. He objects both to the deferential stance in general and to the majority reading of {\it Berman} in particular, pointing out that the Supreme Court's doctrine of deference was in large part directed at the state courts.\footcite[668]{poletown81} Hence, he concludes, the majority's reliance on {\it Berman} is ``particularly disingenuous''.\footcite[668]{poletown81} 

Justice Ryan was not alone in his disapproval of {\it Poletown} and the case is widely regarded as the prelude to an era of increased tensions over economic development takings in the US. This would culminate with {\it Kelo} which, despite upholding an economic development taking, also signaled a move towards more active judicial review of the public use requirement. This effect of {\it Kelo} has become more clear over time, primarily due to state responses caused by widespread disapproval with the outcome. However, it has also been remarked that both the majority and minority opinions in {\it Kelo} indicate that the Supreme Court itself may not be entirely at ease with the doctrine of strict deference that developed after {\it Berman}. In the next subsection, I will give an overview of recent developments, particularly from the secondary literature.

\section{Economic development takings after Kelo}

The fact that {\it Kelo} was decided against the homeowner met with wide disapproval by the US public. In addition, many scholars expressed concern at what they saw as an ill advised ``abdication'' of the judiciary in takings cases. The minority opinions given in {\it Kelo}, particularly the opinion of Justice O'Connor, also proved influential, causing further attention to be directed at the perceived dangers of eminent domain abuse. A massive amount of literature has since appeared devoted to studying the ``problem'' of economic takings. Moreover,  many states have seen reforms aimed to curb the use of eminent domain for economic development.\footnote{For an overview and critical examination of the myriad of state reforms that have followed {\it Kelo}, I point to \cite{eagle08}. See also \cite{somin09}.} 

As of 2014, 44 states have passed post-{\it Kelo} legislation to curb the use of eminent domain for economic development.\footnote{According to the Castle Coalition, a property activist project associated with the Institute of Justice. See \url{http://www.castlecoalition.org/} for an up-to-date survey of state legislation on eminent domain.} Various legislative techniques have been adopted by the states to achieve this. Some states, including Alabama, Colorado, Michigan, enacted explicit bans on economic development takings and takings that would benefit private parties.\footcite[See][107-108]{eagle08} In South Dakota, the legislature went even further, banning the use of eminent domain  ``(1) For transfer to any private person, nongovernmental entity, or other public-private business entity; or (2) Primarily for enhancement of tax revenue''.\footnote{South Dakota Codified Laws § 11-7-22-1, amended by House Bill 1080, 2006 Leg, Reg Ses (2006).}

In other states, more indirect measures were also taken, such as in Florida, where the legislature enacted a rule whereby property taken by the government could not be transferred to a private party until 10 years after the date it was condemned.\footcite[809]{eagle08} Many states also offer inclusive, often lengthy, lists of uses that should count as public, allowing the states to restrict the eminent domain power while also allowing condemnations that are regarded as particularly important to the state.\footcite[804]{eagle08}
However, as argued by Somin, many of these legislative reforms are largely ineffective in preventing economic development takings.\footcite[2120]{somin09} Somin also points to another interesting trend, namely that state reforms enacted by the public through referendums tend to be far more restrictive and effective in preventing economic and private-to-private takings than reforms passed through the state legislature.\footcite[2143]{somin09} 

This is a further reflection of the extent to which the US public opposed the decision in {\it Kelo}. Surveys show that as many as 80-90 \% believe that it was wrongly decided, an opinion widely shared also among the political elite.\footcite[2109]{somin09} Indeed, {\it Kelo} has had a great effect on the discourse of eminent domain in the US, and this effect is perhaps of greater importance than the various state reforms that have been enacted. According to Somin, most of the reforms have in fact been ineffective, despite the overwhelming popular and political opposition against economic development takings.\footcite[2170-2171]{somin09} 

Somin is not alone in feeling that eminent domain reform has offered more than it could deliver, this is a sentiment that is expressed both by supporters and critics of {\it Kelo}. On the other hand, while practitioners have noted that it is largely business-as-usual in eminent domain law, they also report a greater feeling of unease regarding the public use requirement, expressing hope that the Supreme Court will soon revisit the issue.\footnote{See \cite{murakami13} (``Until the Supreme Court revisits the issue, we predict that this question will continue to plague the lower courts, property owners, and condemning authorities'').} In this way, the public backlash against {\it Kelo} has served as an influential reminder that the rationale behind eminent domain for economic development is largely out of sync with the sense of fairness and justice endorsed by most non-experts. 

The underlying cause of this, according to Somin, can be traced to the fact that people are ``rationally ignorant'' about the economic takings issue. For most people, it is unlikely that eminent domain will come to concern them personally or that they will be able to influence policy in this area. Hence, it makes little sense for them to devote much time to learn more about it. This, in turn, helps create a situation where experts can develop and sustain a system based on principles that, in fact, are opposed by a large majority of citizens.\footcite[2163-2171]{somin09} Indeed, Somin argues that surveys show how people tend to overestimate the effectiveness of eminent domain reform, possibly due to the fact that symbolic legislative measures are mistaken for materially significant changes in the law.\footcite{somin09}

I think Somin's analysis is on an interesting track, although it seems wrong to assume {\it a priori} that people's critical stance on economic development takings would necessarily remain in place if they educated themselves more on the issue. Rational ignorance, in particular, should be seen as a double-edged sword in disputes of this kind. But this does nothing to detract from the main message, which is that the {\it Kelo} backlash seems to have caused greater insecurity about what the law is, without being able to significantly curb those uses of eminent domain that have been deemed problematic. In my opinion, this shows that the static legislative approach to eminent domain reform, which has dominated the scene in the US so far, needs to be supplemented by more dynamic proposals. In particular, it seems important to target the decision-making processes surrounding planning and eminent domain, to look for principles by which this process can be imbued with legitimacy. 

In a country where the population expresses antagonism towards eminent domain for economic development, a more inclusive process will likely cause such takings to become more uncommon. On the other hand, if principles of good governance are put in place, it might also restore confidence in eminent domain as a procedure by which to implement democratically legitimate decisions about how to weigh the interests of landowners against the interests of the public. In the next subsection, I will consider two proposals for principles of this kind. The first targets specifically the question of how compensation is determined in economic development cases, a crucial aspect of legitimacy. The second proposal targets the decision-making process more broadly, by proposing a framework for land assembly that is meant to replace the use of eminent domain in certain circumstances.

\noo{But it is not the general public that are the major stakeholders in such disputes, but rather the communities that are directly affected, including both the private property owners who will be burdened and those community members who stand to benefit. A good framework for balancing their interests relies on finding appropriate principles of good governance, so that governments can play an empowering role when such decisions are made. This is crucial for legitimacy of land use planning generally, but especially for eminent domain, where the gravity of the interference means that legitimacy is unlikely to arise unless the decision to condemn is firmly rooted in the interests of the main stakeholders. To the greatest possible extent, it also seems crucial to emphasize local conditions and ensure that the decision enjoys broad local support. 

Shortly after {\it Poletown} was overturned, the case of Kelo saw the legitimacy of economic takings brought before the Supreme Court once again. This time there was real doubt and disagreement among the justices regarding the scope of the public use limitation. The case revolved around the legitimacy of condemning a home in favour of a research facility for the drug company Pfizer, which was part of a development plan for the City of New London.  The owner, Suzanne Kelo, argued that the condemnation of her home was in breach of the constitution, since it was a private-to-private taking ostensibly to the benefit of Pfizer rather than any clearly defined public use or interest.

In Kelo, Justice Thomas adopted the strictest view on the public use test. He entirely disregarded  the precedent set by Berman and Midkiff in favour of constitutional originalism, the doctrine which asserts that direct assessment of the wording in the Constitution, and the intentions of the founding fathers, is the approach that should be used to decide constitutional cases. Following up on this he held that actual right of use for the public was the test that had to be applied in takings cases. The hundred years of precedent preceding Kelo was described as “wholly divorced from the text, history, and structure of our founding document", and thus Justice Thomas concluded that it had to be abandoned. 

Justice O'Connor, in an expression of dissent joined by Chief Justice Rehnquist and Justices Scalia
and Thomas, argued against legitimacy on less theoretical grounds, based on the facts of the case and the precedent that would be set for similar cases in the future. Her main legal argument was that while public use should be interpreted broadly, the possibility of positive ripple effects was not enough to justify private-to-private takings. In particular, Justice O'Connor took a very bleak view on the practical consequences that would arise from allowing economic takings that could be justified only by pointing only to indirect positive consequences for the public. She commented on the majority decision to uphold the taking as follows: 

Any property may now be taken for the benefit of another private party, but the fallout from this decision will not be random. The beneficiaries are likely to be those citizens with disproportionate influence and power in the political process, including large corporations and development firms. As for the victims, the government now has license to transfer property from those with fewer resources to those with more. The Founders cannot have intended this perverse result.

It seems that a major point of contention among the judges in the Supreme Court was whether or not these grim predictions was a realistic assessment of what the consequences of the decision would be. Surely, anyone who agrees with Justice O'Connor in her prediction of the fallout would also agree with here conclusion that it is perverse. But the majority in Kelo, in an opinion written by Justice Stevens, disagreed with her assessment, observing instead that a more restrictive view on economic takings would make it more difficult to cater to the "diverse and always evolving needs of society". 

But the majority opinion also stressed that purely private takings where not permissible, and they attached great significance to the substantive assessment that the actual taking of Suzanne Kelo's home formed part of a comprehensive development plan that would not bestow special benefit on any particular group of individuals. Moreover, Justice Kennedy, in his concurring opinion, emphasised that states should not use public purpose as a pretext for interfering in property rights to the benefit of commercial actors.
Hence the overall impression one is left with when considering Kelo in its historical and legal context is that it reflects an increasingly cautious attitude to economic takings. The precedent of virtually unlimited deference that was set in case law from the mid-to-late 19th Century was eschewed in favour of a more contextual approach where the merits and deeper purpose of the plans underlying a taking is not axiomatically beyond the scrutiny of the courts.

From considering the reception of the case by the general public, we see even more clearly how Kelo in effect marks a change in the US towards greater scrutiny. 

Indeed, the voices that have dominated in the aftermath of Kelo were critical of the decision and criticized the court for not offering better protection to property owners. The case also led to an a surge of academic interest in the pubic use restriction, with many arguing for further restrictions on the scope of the takings power. 
Hence it seems that Justice O'Connor's opinion largely reflects contemporary worries about takings in the US, worries that are now also becoming increasingly relevant to how the law develops and is understood. Many states have changed their own eminent domain codes  following Kelo, to make it harder to undertake economic takings. Moreover, the federal government also banned such takings from taking place on the basis of federal takings powers.
It will lead us astray to delve deeply into the question of what caused this change in perspective on economic takings in the US, but we can offer a few hypothesis. First, it seems that cases such as Poletown illustrates the potential danger inherent in making the power of eminent domain available to market players. In particular, the main worry that has been raised is that the pretext of public purpose may be in the process of becoming a powerful instrument for influential market actors to gain access to regulatory powers of government. As these powers has massively expanded in the post-WW2 period, so has the potential for abuse. In addition, it seems that while those who were adversely affected by eminent domain tended to be less privileged and resourceful groups of society, the takings power is now increasingly brought to bear also against members of the middle class, who are in a better position to fight it, both legally and on the political scene.

While opinions differ greatly both regarding the extent of the problem and the causes of recent controversy, there is something near consensus in the US after Kelo that economic development takings raise special problems under the current system of eminent domain, and that these need to be addressed with a view to reducing tensions and restoring faith in the system. Indeed, even the majority in Kelo hint strongly at this when they say that  
Some have argued forcefully that a strict reading of the public use requirement is the way forward, if not by strict interpretation then by an explicit ban on economic development takings.  However, it is tempting here to echo the worries expressed in Seawell, that a strict formalistic approach to legitimacy runs the risk not only of being inflexible, but also, eventually, of offering less  protection to property owners. How, then, should we reduce the risk of abuses?
While many have focused on the question of banning economic taking, or reconsidering the public use clause, some have addressed this question from such a broader angle. In my opinion, this is the way forward. It seems, in particular, that a complete ban on economic development takings will leave a vacuum in the current economic system, which presupposes a great deal of cooperation between commercial and public interest. Particularly when it comes to economic development, the private-public partnership model has gained influence to the point that a ban on economic development takings would likely prove impossible to implement in a satisfactory manner. 
More generally, it seems hard to address the problem of economic takings without considering the role they play in the larger economic context within which current rules and practices have developed. Based on such considerations, I believe the procedural approach to economic takings is the appropriate one. This perspective asks us to take a closer look at judicial safeguards for protecting the role of property owners in the decision-making processes that lead up to the use of eminent domain. To some extent one might approach this on the basis of existing legal principles, asking for better scrutiny of procedural aspects, or by making it easier to bring pretext claims before the courts. However, it might also require new ideas, and, in particular, the introduction of new institutions for decision-making and administration of the eminent domain process.

In the next section, I will look at two concrete proposals in more detail, one concerning the decision-making step and the other concerning the calculation of compensation. 
They will be important because they serve as starting points for the case study that is to follow, addressing mechanisms that we will return to in Chapters x and y when we look more closely at two Norwegian legal institutions that share many features with the theoretical roposals discussed in the next section.
}

\section{Institutional proposals for increased legitimacy}\label{subsec:ins}

In this subsection, I first present the Special Purpose Development Companies proposed by Lehavi and Licht.\footcite{lehavi07} I relate this proposals to theoretical approaches to the issue of compensation, before I go on to note some shortcomings and open questions that I will later address in my case study. I then go on to consider the Land Assembly Districts proposed by Heller and Hills.\footcite{heller08} I consider this proposal in light of the stated motivation, which is to design an effective mechanism of self-governance that can replace eminent domain in economic development cases. I present some unresolved questions and argue that there is a tension in the proposal between its narrow scope, imposed to prevent majority tyranny and other forms of abuse, and its broad goal of empowering local communities. 

\subsection{Special Purpose Development Companies}

The primary distinguishing feature of economic development takings is that they give the taker an opportunity to profit commercially from the development. This may even be the primary aim of the project, with the public benefiting only indirectly through potential economic and social ripple effects. Property owners facing condemnation in such circumstances might expect to take a share in the profit resulting from the use of their land. However, in many jurisdictions, including the US, the rules used to calculate compensation prevents owners from getting any share in the commercial surplus resulting from development.\footnote{See, e.g., \cite[965-966]{fennell04}.} In particular, various {\it elimination rules} are typically in place to ensure that compensation is based entirely on the pre-project value of the land that is being taken.\footcite[See][81]{ackerman06} The policy reasons for such rules is that they ensure that the public does not have to pay extra due to its own special want of the property. After all, this is one of the main purposes of using eminent domain in the first place; to ensure that the public does not have to pay extortionate prices for land needed for important projects. However, when the purpose of the project is itself commercial in nature, there appears to be a shortage of good policy reasons for excluding this value from consideration when compensation is calculated. This is especially true when, as in the US, compensation tends to be based on the market value of the land taken. Why should a commercial condemner's prospect of carrying out economic development with a profit be disregarded from the assessment of market value? In any fair and friendly transaction among rational agents, one would expect benefit sharing in a case like this. Yet for economic development backed up by eminent domain, the application of elimination rules ensures that all the profit goes to the developer. 

Some authors have argued that failures of compensation is at the heart of the economic takings issue and that worry over the public use restriction is in large part only a response to deeper concerns about the ``uncompensated increment'' of such takings.\footcite[See][962]{fennell04} In addition to the lack of benefit sharing, previous work has identified two further problems of compensation that also tend to become exasperated in economic development cases. First, the problem of ``subjective premium'' has been raised, pointing to the fact that property owners often value their own land higher than the market value, for personal reasons.\footcite[963]{fennell04} For instance, if a home is condemned, the homeowner will typically suffer costs not covered by market value, such as the cost of moving, including both the immediate ``objective'' logistic costs as well as more subtle costs, such as having to familiarize oneself with a new local community. Second, the problem of ``autonomy'' has been discussed, arising from the fact that an exercise of eminent domain deprives the landowner of her right to decide how to manage her property.\footnote{Discussed in \cite[966-967]{fennell04}. For a general personhood building theory of property law, see \cite{radin93}. For a general economic theory of the subjective value of independence, see \cite{benz08}.}

In \footcite{lehavi07}, the authors propose a novel approach for addressing the ``uncompensated increment'' in economic takings cases. Their proposal is based on a new kind of structure that they dub a {\it Special Purpose Development Corporation} (SPDC). The idea is that owners affected by eminent domain will be given a choice between standard pre-project market value and shares in a special company. This company will exist only to implement a specific step in the implementation of the development project: the transaction of the land-rights. The SPDC may choose either to offer their rights on an auction or else negotiate a deal with a designated developer.\footcite[1735]{lehavi07} Hence, the idea is to ensure that the owners are paid a value that reflects the post-project value of the land, but in such a way that the holdout problem is avoided. In particular, the SPDC will have a single task: to sell the land for the highest possible price within a given time frame.\footcite[1741]{lehavi07} After the sale is completed, the SPDC will divide the proceeds as dividends and be wound up.\footcite[1741]{lehavi07}

Other suggestions have taken a more static approach to compensation reform, such as proposing to give owners a fixed premium in cases of economic development, or developing mechanisms of self-assessment to ensure that compensation is based on the true value the owner attributes to his own land.\footnote{A range of static proposals have been proposed in the literature: Merrill proposes 150 \% of market value for takings that are deemed to be ``suspect'', including takings for which the nature of the public use is unclear, see \cite[90-93]{merrill86}. Krier and Serkin propose a system that provide compensation for a property's special suitability to its owner, or a system where compensation is based on the court's assessment of post-project value, see \cite[865-873]{krier04}. Fennell proposes a system of self-evaluation of property for takings purposes with tax-breaks given to those who value their property close to market value (to avoid overestimation), see \cite[995-996]{fennell04}. Bell and Parchomovsky also propose self-evaluation, but rely on a different mechanism to prevent overestimation; tax liability is based on the self-reported value and no property can be sold by its owner for less than his reported value, see \cite[890-900]{bell07}.} Compared to such proposals, the idea of SPDCs is more sophisticated and should be looked at in more depth. 

The conceptual premise for the proposal is that takings for economic development can be seen as compulsory incorporation, a pooling of resources useful in overcoming market failures.\footcite[1732-1733]{lehavi07} Just as the corporation is formed to consolidate assets in order to facilitate effective management, so is eminent domain used to assemble property rights in order to facilitate efficient organization of development. According to Lehavi and Licht, this also provides a viable approach to problems of ``opportunistic behavior''; hierarchical governance after assembly ensures that order and unity can be regained even if interests in the land are distributed among a large and heterogeneous group of potentially mischievous shareholders.\footcite[1733]{lehavi07} In the words of Lehavi and Licht:

\begin{quote}
The exercise of eminent domain powers thus resembles an incorporation by the government of all landowners with a view to brining all the critical assets under hierarchical governance. Establishing a corporation for this purpose and transferring land parcels to it thus would be merely a procedural manifestation of the substantive economic reality that already takes place in eminent domain cases.
\end{quote}

As soon as we look at the rationale behind economic development takings in this way, any remnant of good policy reasons for ensuring that the developer gets all the profit seems to disappear. Rather, we are led to consider compensation as an issue entirely separate from the exercise of the takings power. After the land has been reorganized by eminent domain and an SPDC has been formed, the land rights might as well be sold {\it freely} to a developer. In this way, the land will be sold for a price that is closer to an actual market value, on the market where the land is destined for development.\footcite[1735-1736]{lehavi07} More generally, the SPDC becomes an aid that the government can use to create more favorable market conditions for transferring land that has commercial potential in its public use. Due to the compulsory pooling of resources, no owner can exercise monopoly power by holding out, but due to decoupling of compensation from assembly, the owners can now negotiate with potential developers for a share of the resulting profit. Moreover, the fact that the SPDC offers its rights on an actual market can also help ensure that more information become available regarding the true economic value of the development, something that may in turn help ensure that only the good projects will be successful in acquiring land. Hence, according to Lehavi and Licht, an additional positive effect of SPDCs is that developers and governments will shun away from using the eminent domain power to benefit projects that are not truly welfare-enhancing.\footcite[1735-1736]{lehavi07}

In addition to these substantive consequences, the SPDC-proposal also stands out because it has a significant institutional component, pointing to its potential for restoring procedural legitimacy as well as substantive fairness. Lehavi and Licht discuss corporate governance issues at some length, but without committing themselves to definite answers about how the operations of the SPDC should be organized.\footcite[1040-1048]{lehavi07} Indeed, while their proposal is perhaps most interesting because of its procedural aspects, it also appears to be rather preliminary in this regard. The main idea is to let the SPDC structure piggyback on existing corporative structures, particularly those developed for securitization of assets.\footnote{See generally \cite{schwarcz94}. For an up-to-date overview, targeting special challenges that became apparent during the 2008 financial crisis, see \cite{schwarcz13}.} The basic idea is that the corporate structure should be insulated from the original landowners to the greatest possible extent; it should have a narrow scope, it should be managed by neutral administrators, and it should entrust a third party with its voting rights.\footcite[1742]{lehavi07} This is meant to prevent failures of governance within the SPDC itself, making it harder for majority shareholders and self-interested managers to co-opt the process. For instance, if a possible developer already holds a majority of the shares in an SPDC, this structure would prevent him from using this position to acquire the remaining land on favorable terms. 

Lehavi and Licht observe that under US law, the government would often be required to make shares in an SPDC available to the landowners as a public offering.\footcite[1745]{lehavi07} Lehavi and Licht deem this to be desirable, arguing that full disclosure will provide owners with a better basis on which to decide whether or not to accept SPDC shares in place of pre-project market value. It will also facilitate trading in such shares, so that they will become more liquid and therefore, presumably, more valuable.\footcite[1746]{lehavi07} 

Lehavi and Licht's proposal is interesting, but I think a fundamental objection can be raised against it. In particular, it seems that their governance model more or less completely alienate property owners from the decision-making process after SPDC formation. Limiting the participation of owners is to a large extent an explicit aim, since governance by experts is held to increase the chances of ensuring good governance. But is expert rule really the answer?

It seems that from the owners' point of view, Lehavi and Licht's proposals for governance reduces the SPDC to a mechanism whereby they can acquire certain financial entitlements. These may exceed those that would follow from standard compensation rules, but they do not directly empower owners vis-{\'a}-vis developers and the government. Instead, a largely independent structure will be introduced. It is this new organizational structure, rather than the owners, that will now become an important actor in the eminent domain process. In principle, it is meant to represent owners, but to what extent can it do so effectively? After all, it is specifically intended to operate as neutral player, charged with maximizing the price, nothing more. Hence, it appears that the SPDC will not be able to give owners an arena to negotiate on the basis of the personal and social importance they attribute to their land rights. How the problem of ``autonomy'' is addressed by the proposal is therefore hard to see and the ``subjective premium'' also appears to be in danger, unless it can be objectively quantified and covered by the surplus from a voluntary sale. But if such quantification is possible, then why not simply tell the appraiser to award some premium under standard compensation rules?

More generally, it seems to me that while all three categories of ``uncompensated increments'' are interesting to study from a financial viewpoint, severe doubts can be raised regarding the feasibility of addressing the subjective aspects of this as a question of compensation. It may be that issues related to ``subjective premium'' and ``autonomy'' are seen as public use issues for good reason; they are hard to quantify otherwise. Moreover, attempting to do so might do more harm than good. On the one hand, it might skew the political process, since owners that have been ``bought off'' don't object to ill-advised development projects, as long as they generate financial revenue. But what about projects that are undesirable for other reasons, for instance because they completely change the character of a neighborhood, or because they are harmful to the environment? On the other hand, the very idea that money can compensate for the subjective importance of property and autonomy can itself prove offensive. At least it seems likely that it would often come to be seen as inadequate and inefficient.\footnote{For more detailed criticism of the compensation approach to the public use issue, see \cite{garnett06}.} Moreover, an owner that is compelled to give up his home after an inclusive process where the public interest has been debated and clearly communicated is likely to feel like he incurs less costs related both to his subjective premium and his autonomy. Hence, the lack of participation in the decision-making process can in itself increase the uncompensated loss. Clearly, no externally managed ``bargain-oriented'' SPDC will be able to resolve this problem. Of course, some ``objective'' elements of, such as relocation costs or cost for juridical assistance, can still be addressed under the banner of compensation. But in most jurisdictions, they already are.\footnote{See, e.g., \cite[121-126]{garnett06}.} For more subtle aspects, the aftermath of {\it Kelo} itself can serve as an illustration of how a compensatory approach is unsatisfactory:

After the case, Suzanne Kelo remained defiant, until she eventually decided to settle in 2006, for an offer of \$ 442 155, more than \$ 319 000 above the appraised value.\footcite[1709]{lehavi07} Apparently, the other owners affected by the same taking were not particularly pleased, arguing that recalcitrant owners were actually rewarded for holding out.\footcite[1709]{lehavi07} On the other hand, there is no indication that Suzanne Kelo was not genuine in her opposition to the taking. Indeed, after the long struggle she had taken part in, it is easy to imagine that financial compensation, if it was to be an effective remedy at all, would have to be very high. Even after she had settled, Kelo apparently toured the country speaking out against economic takings. This, too, is a statement to the inadequacy of a purely financial approach to legitimacy. 

I conclude that SPDCs have serious shortcoming with regards to the subjective aspects of undercompensation, aspects that can only be addressed if the focus turns towards participation. However, SPDCs do seem promising when it comes to profit-sharing. This, after all, is what the structure is specifically aiming to achieve. In addition, I agree that SPDCs will likely have a positive effect on the other actors in the eminent domain process. In particular, I agree with Lehavi and Licht that greater openness is likely to result, revealing the true merits of development projects, at least in so far as these are translatable into financial terms. The fact that developers must negotiate with an SPDC who can threaten to make the land available an an open auction will likely deter developers and government from pursuing fiscally inefficient projects. Hence, the risk that governments will subsidized such projects by giving them cheap access to land will also be reduced. In addition, the presence of a third voice, speaking on behalf of owners, is likely to help achieve a better balance of power in development takings. 

Even if the individual landowners do not have a voice in this process, the fact that the landowners are better represented as a group is then still likely to have a positive effect on legitimacy. On the other hand, as long as the power of the SPDC is limited to choosing the best offer and negotiating over price, it seems that SPDCs will easily end up being dominated by developers and government. This is a particular concern in cases when competition fails to arise after SPDC formation. To ensure that there are other interested parties, in particular, sems like an important precondition for the proposal to work in practice. In this regard, it is important to realize that a lack of interest from other developers may not be due to the superiority of the original developer's plans. It might rather be due to the fact that the scope of the assembly giving rise to the SPDC is so defined as to make alternatives unfeasible. The danger of abuse in this regard seems significant, particularly when developers themselves participate in coming up with the plans that give rise to SPDC formation. 

Moreover, as long as owners remain marginalized in the planning phase, it is easy to imagine situations where the plan itself will be formulated in such a way that only one developer is in a position to successfully implement it. A simple example would be if a prospective developer already owns some of the land that is critical to the plan, and is able to ensure that this land is kept out of the scope of the SPDC. Clearly, if SPDCs are to operate effectively, such instances of manipulation need to be avoided, suggesting that the proposal as it stands needs to be fleshed out in greater detail.

\noo{
First, since an SPDC does not have the power to stop the development, the system will only work as long as there are several interested parties who are willing to compete for the land rights. In practice, the planning regimes used to facilitate development can make this an unlikely prospect. Eminent domain is often used to implement highly specific projects, where the developer himself has played an important role in the planning process. Hence, while there might be many parties interested in the general kind of development that will be carried out, the concrete project that is being undertaken might not be of interest to anyone other than the developer who initially proposed it. Indeed, this developer might be the only one able to carry it out, say because the project forms part of a greater scheme involving rights that this developer already controls. Therefore, to create a market where the SPDCs can function as intended, it seems that deeper changes in planning practice are required, to avoid natural monopolies from developing in the planning process. This can be challenging in general, and in cases when eminent domain is used only for parts of greater schemes, it seems that it will be practically impossible to make the system work, without also undermining the governments ability to ensure that such schemes are carried out according to plan. In particular, setting up an SPDC alone is not, in general, sufficient to give property owners access to post-project market values. Something more is needed, and it is unclear what, if anything, can ensure that the SPDC gets to operate in healthy market conditions, as intended.

Second, the fact that the proposal is based on the doctrine of market value might in itself be a problem. Some of the clearest voices that have spoken out against economic development takings have done so on the basis of non-commercial objections. For instance, in the case of {\it Kelo}, the high subjective value that Suzanne Kelo attributed to her home was at the heart of the conflict. Indeed, Kelo continued to campaign against the condemnation, as a matter of principle, even after she had accepted a financial settlement whereby she was awarded some four times the estimated market value of her home.\footnote{...} The deeper problem of economic development takings is that they often lack legitimacy due to their partly commercial purpose. To compensate owners on the basis of market value alone does not offer any recognition of the fact that interference tends to appear less legitimate in such cases, compared to cases when the public purpose of the taking is more clearly discernible. It is of course not clear that such a lack of legitimacy can be addressed by compensation at all, but if this is at all going to be possible, then it must involve a compensation method which explicitly recognizes that these cases need to be treated differently than cases where problems of legitimacy do not arise. Moreover, since market value is the standard rule for awarding compensation, applying it in economic development cases will typically fail to provide any recognition of the fact that these cases are special. This in itself can become a perceived injustice; why should someone losing their home to a new school or a hospital be placed in the same category as someone loosing their home to a commercial company? In the latter case, it might be that the loss of a home appears {\it incommensurable} to the societal gains that may result. Hence, any compensation based on market value might lead to a moral deficit, whereby the affected property owners feels that non-commercial interest, including their own, are simply regarded as irrelevant.
}

The problems addressed here both seem to point to the fact that the SPDCs, while more flexible than other suggestions, are still too static to achieve many of their objectives. In particular, to arrive at genuine market conditions for assessing post-project value, there is still a need for changes in the dynamics of the planning process underlying the taking. Moreover, to ensure legitimacy, there is a need for a mechanism that goes beyond expert bargaining and provides owners with better access to the decisionmaking process. In the next subsection, I will consider a proposal that aims to address this, by proposing a framework for self-governance. 

\noo{ In Chapter 4 I will address this challenge by looking at an alternative approach to compensation assessment, whereby legitimacy is sought through the participation of lay people in appraisal process. This system has long traditions in Norwegian law, and it has currently been put to the test in cases of hydro-power development. In response to changes that have rendered such development commercial in nature, the compensation law in Norway has effectively been changed, something that has resulted in greater legitimacy. This process, as we will see, has resulted in large part from the activities of the special appraisal courts. In particular, the lay people who partake in them have played a crucial role. While the procedural context is different, the new compensation method that has been developed does in fact share ,any similarities to the ideas presented in \footcite{lehavi07}. Essentially, they serve as a means to make compensation a function of the post-project value of the land taken. But rather than attempting to set up a well-behaved market for selling the land, the system relies on a special judicial process that relies heavily on the discretion of the lay people, judges and experts that are all involved in deciding the award. Moreover, the owners themselves have a clear voice, as they are recognized as parties in the proceedings which are formally organized as a legal dispute between the owners and the developer. Hence, the system is more flexible than a standard post-taking judicial process. It can be adapted to the special circumstances of individual cases, and the application of the law to the facts is to a large extent influenced by both lay people and experts, rather than legal professionals. As a result, there is ongoing tension between the legal aspects of compensation law, as directed by the Supreme Court, and the administrative practices and local appraisal principles, as developed by the appraisal courts. This tension is the defining feature of the system, and one which we believe show both its strengths and weaknesses. It also highlight its potential as an alternative approach to compensation in a time when takings law is both becoming more practically important and more controversial.

The approach in \footnote{lehavi07} does not provide any guide in resolving controversies that arise in relation to how decision-making processes are organized in economic development cases. In addition to problems associated with compensation, such cases often suffer from a {\it democratic deficit}. While the main beneficiary, the developer, often plays a crucial role in the administrative process leading up to condemnation, property owners are not awarded any special right to participation beyond what follows from general planning and takings law. Just as the law fails to recognize economic takings as a special category for compensation purposes, so does it fail to give special rules to govern the preparation of such cases. This is widely seen as a shortcoming of the law. It creates an imbalance, in particular, between developers and owners, with the former enjoying a far greater ability to influence the decision-making process at the administrative level. In cases when developers are merely in place to execute public plans, and have no agenda on their own, this is no major threat to legitimacy. But in cases when the developer have significant interests of their own, and act as autonomous economic agents, the imbalance in influence becomes problematic.}

\subsection{Land assembly districts}

In a recent article, Heller and Hills propose a new institutional framework for carrying out land assembly for economic development. Interestingly, it is meant to replace eminent domain altogether. The goal is to ensure democratic legitimacy while also creating a template for collective decision-making that will prevent inefficient gridlock and holdouts. The core idea is to introduce {\it Land Assembly Districts} (LADs), institutions that will enable property owners in a specific area to make a collective decision about whether or not to sell the land to a developer or a municipality.\footcite[1469-1470]{heller08} Anyone can propose and promote the formation of a LAD, but both the official planning authorities and the owners themselves must consent before it is formed.\footcite[1488-1489]{heller08} Clearly, some form of collective action mechanism is required to allow the owners to make such a decision. Hiller and Hill suggest that voting under the majority rule will be adequate in this regard, at least in most cases.\footnote{See \cite[1496]{heller08}. However, when many of the owners are non-residents who only see their land as an investment, Heller and Hills note that it might be necessary to consider more complicated voting procedure, for instance by requiring separate majorities from different groups of owners. See \cite[1523-1524]{heller08}.} How to allocate voting rights in the LAD is another issue that require careful consideration, but Heller and Hills land on the proposal that they should in principle be given to owners in proportion to their share in the land belonging to the LAD.\footnote{See \cite[1492]{heller08}. For a discussion of the constitutional one-person-one-vote principle and a more detailed argument in favor of the property-based proposal, see \cite[1503-1507]{heller08}.} Owners can opt out of the LAD, but in this case eminent domain can be used to transfer the land to the LAD using a conventional eminent domain procedure.\footcite[1496]{heller08}

Heller and Hills envision an important role for governmental planning agencies in approving, overseeing and facilitating the LAD process. Their role will be most important early on, in approving and spelling out the parameters within which the LAD is called to function.\footcite[1489-1491]{heller08} Hence, it appears to be assumed that the planning authorities will define the scope of the LAD by specifying the nature of the development it can pursue. A possible challenge that arises, and which Heller and Hills do not address at any length, is that the scope of the LAD needs to be broad enough to allow for meaningful competition and negotiation after LAD formation. At the same time, however, there will probably be a push, both by governments and initiating developers, to ensure that the scope is defined narrowly enough to give confidence that rezoning permissions will not be denied at a later stage. Another potential challenge is that the planning authorities might have an incentive to refuse granting approval for LAD formation, since it effectively entails that they give up the power of eminent domain for the land in question. For this reason, Heller and Hills propose that a procedure of judicial review should exist whereby a decision to deny approval for LAD formation can be scrutinized.\footcite[1490]{heller08} 

After the formation of the LAD, the government will have a less important position, but the planning authorities will still occupy an important facilitating role. Heller and Hills envision a system of public hearings, possibly organized by the planning authorities, where potential developers meet with owners and other interested parties to discuss  plans for development.\footnote{See \cite[1490-1491]{heller08}.} In this process, it is assumed that also other voices will be represented, such as owners of adjoining land, who can use this opportunity to express objections against the project. Their role in the process is not clarified, but presumably the planning authorities would be able to offer this group some protection, if not in relation to the LADs own operations, then later in relation to the decision whether or not to grant the licenses needed for the development project.

Importantly, if the owners do not agree to forming a LAD, or if they refuse to sell to any developer, the government will be precluded from using eminent domain against them to assemble the land.\footcite[1491]{heller08} This is the crucial novel idea that sets the suggestion apart from other proposals for institutional reform that have appeared after {\it Kelo}. LADs will not only ensure that the owners get to bargain with the developers over compensation, it will also give them an opportunity to refuse any development to go ahead, if they should so decide. Hence, the proposal shifts the balance of power in economic development cases, giving owners a greater role also in preparing the decision whether or not to develop, and on what terms. In my opinion, this makes the proposal stand out as particularly interesting in the recent literature on economic takings. It is the first concrete suggestion that addresses the democratic deficit in a dynamic, procedural manner, without failing to recognize that the danger of holdouts is real and that institutions are needed to avoid it, also in economic development cases.

There are some problems with the model, however. Kelly points out that the basic mechanism of majority voting is itself imperfect, and can lead both to overassembly and underassembly, depending on the circumstances.\footcite{kelly09} He points out, in particular, that if different owners value their property differently, majority voting will tend to disfavor those with the most extreme viewpoints, either in favor of, or against, assembly. If these viewpoints are assumed to be non-strategic and genuine reflections of the welfare associated with the land, the result can be inefficiency. In a nutshell, the problem is that a majority can often be found that does not take due account of minority interests. For instance, if some owners are planning alternative development, leading them to attribute a high {\it hope}-value to their land, they can safely be ignored as long as the majority have no such plans. This could become particularly bad in cases when the alternative development itself is more socially desirable than the development that will benefit from assembly. The role of the LAD in such cases will not improve the quality of the decision to develop, since it pushes the decision-making process into a track where those interests that {\it should} prevail are voiced only by a marginalized minority inside the new institution.\footnote{Of course, one might imaging these landowners opting out of the LAD, or pursuing their own interests independently of it. However, they are then unlikely to be better off than they would be in a no-LAD regime. In fact, it is easy to imagine that they could come to be further marginalized, since the existence of the LAD, acting ``on behalf of the owners'', might detract from any dissenting voices on the owner-side.}

More generally, the lack of clarity regarding the role of LADs in the planning process is a problem. As it stands, the proposal leaves it uncertain how LADs will affect the decision-making process regarding development. But the ideal is clearly stated: LADs should help to establish self-governance in land assembly cases. In particular, Heller and Hills argue that LADs should have ``broad discretion to choose any proposal to redevelop the neighborhood -- or reject all such proposals''.\footcite[See][1496]{heller08} As they put it, two of the main goals of LAD formation is to ensure `` preservation of the sense of individual autonomy implicit in the right of private property and preservation of the larger community's right to self-government''.\footcite[See][1498]{heller08} Unfortunately, these ideals are somewhat at odds with the concrete rules that Heller and Hills propose, particularly those aiming to ensure good governance of the LAD itself. 

In relation to the governance issue, Heller and Hills echo many of the ``corporate governance''-ideas that also feature heavily in Lehavi and Licht's proposal. Indeed, in direct contrast to their comments about ``broad discretion'' and ``self-governance'', Hiller and Hills also state that ``LADs exist for a single narrow purpose -- to consider whether to sell a neighborhood''.\footcite[See][1500]{heller08} This is a good thing, according to Heller and Hills, since it provides a safe-guard against mismanagement, serving to prevent LADs from becoming battle grounds where different groups attempt to co-opt the community voice to further their own interests. As Heller and Hills puts it, the narrow scope of LADs will ensure that ``all differences of interest based on the constituents' different activities and investments, therefore, merge into the single question: is the price offered by the assembler sufficient to induce the constituents to sell?''.\footcite[1500]{heller08}

But this means that there is an internal tension in the LAD proposal, between the broad goal of self-governance on the one hand and the fear of neighborhood bickering and majority tyranny on the other. It is hard to see, in particular, how the idea of LADs with a ``narrow purpose'' is compatible with a scenario where the LAD has ``broad discretion'' to choose between competing proposals for development. If such discretion may indeed be exercised, what is to prevent special interest groups among the landowners from promoting development projects that will be particularly favorable to them, rather than to the landowners as a group? And what is to prevent landowners from making behind-the-scene deals with favored developers at the expense of their neighbors? It seems like a great challenge to come up with rules that prevent mechanisms of this kind, without also constraining the landowners so much that meaningful ``self-governance' becomes an impossibility. If a LAD is obliged to only look at the price, this might prevent abuse. But it will not give owners broad discretion to choose among development proposal. Effectively, it will render LADs as little more than a variant of SPDCs, where the owners are awarded an extra bargaining-chip: the option to refuse all offers. 

In my view, such a restriction on the operations of LADs is not desirable. It is easy to imagine cases where competing proposals, perhaps emerging from within the community of owners themselves, will emerge in response to the formation of a LAD. Such proposals may involve novel solutions that are superior to the original development plans, in which case it is hard to see any good reason why they should not be taken into account, even if they are less commercially attractive. In particular, the formation of a LAD and the competition for development that ensues creates an opportunity for tapping into a greater pool of ideas for redevelopment, ideas which may then also be rooted more firmly in the local community. Surely, getting such proposals to the table would be desirable and it would take us to the heart of self-governance. At the same time, it is easy to acknowledge that problematic situations may arise, for instance if a majority forms in favor of a scheme that involves razing only the homes of the minority, maybe on the rationale that these are ``more blighted''. That would likely give rise to accusations of unfair play, which may or may not be warranted. But irrespectively of this, an alternative project of this kind might well be a better use of the land in question, also from the point of view of the public. Hence, it would seem that the planning authorities would be obliged to give it some serious consideration. Then, however, the LAD has truly become an arena for a new kind of power play among different interest, and a potential vehicle of force for whomever secures support from a majority of owners within the district.

In their proposal, Heller and Hills are aware of this potential problem, which they propose to resolve by strict regulation. In particular, they argue that ``LAD-enabling legislation should require especially stringent disclosure requirements and bar any landowner from voting in a LAD if that landowner has any affiliation with the assembler''.\footcite{heller08} But this raises further questions. For one, what is meant by ``affiliation'' here? Say that a land owner happens to own shares in some of the companies proposing development. Should he then be barred from voting? If so, should he be barred from voting on all proposals, or just those involving companies in which he is a shareholder? If the answer is yes, how would this be justified? Would it not be easy to construe such a rule as discrimination against landowners who happen to own shares in development companies? On the other hand, if the landowner in question is allowed to vote on all other proposals, would it not be natural to suspect that his vote is biased against assembly that would benefit a competing company? Or what about the case when some of the land owners are employed by some of the development companies? Should such owners be barred from voting on proposals that could benefit their employers? This seems quite unfair as a general rule, especially if a low-level employment relationship has such a dramatic effect. But in some cases even low-level ties could play a decisive factor. This might happen, for instance, if an important local employer proposes development in a neighborhood where it has a large number of employees.

Of course, the most pressing issue that arises is the following: who exactly should be empowered to make the determination of when an affiliation is such that an owner should be deprived of his voting rights? Heller and Hills give no answer, but it is easy to imagine that whoever is given this task in the first instance, the courts would soon enough be asked to consider the question. At this point, the circle has in some sense closed in on the proposal. In particular, one might ask: why is it easier to determine if someone can be deprived of his voting rights due to an ``affiliation'', than it is to determine if someone can be deprived of his land due to some planned ``public use''?

\noo{ While many aspects of the LAD-proposal is not spelled out in great detail, I think the ``affiliation''-restriction on voting rights is more than just a reflection of unfinished work. Rather, I am inclined to think that this particular aspect of the proposal simply {\it cannot} be spelled out in significantly more detail. Any static rules on this point can hardly be expected to prevent all opportunity for abuse, and at the same time such rules will seriously constrain owners in new and highly problematic ways. Moreover, the very idea of ensuring that LADs face a ``narrow'' agenda, seems unlikely to bear fruits. In practice, would not a LAD invariably take up a strategically important position also in the planning process leading up to development? This, at any rate, seems inevitable if the system is meant to provide landowners with self-governance. This leads us to the second problem with using regulation to narrow the function of LADs. Rather than being ``LAD-enabling'' is it not fairer to say that such regulation, at least if understood strictly, will curtail the power of LADs and limit their relevance to the decision-making process? Indeed, is it even desirable that local owners should not be able to affiliate themselves with developers in an effort to arrive at those schemes that they think will represent the best use of the land? Is this not the fundamental aspect of self-governance, that the local people themselves partake in the process, not just as passive spectators, but also as active agents that consult with planning authorities and developers about the best future use of their land. Hence, it might be that in many cases, those same mechanism that can lead to majority tyranny and bicekring if left unchecked are also crucial to unlocking the true potentail of LADs.}

In any event, to come up with a set of rules ensuring that LADs can deliver both self-governance and good governance largely remains an open problem. This is acknowledged by Hiller and Hills themselves, who point out that further work is needed and that only a limited assessment of their proposal can be made in the absence of empirical data. Later in the thesis, I will shed light on this challenge when I consider the Norwegian rules relating to land consolidation, showing how these can be looked at as a highly developed institutional embedding of many of the central ideas of LADs. The assessment of how they function in cases of economic development, and how they are increasingly used as an alternative to expropriation in cases of hydro-power development, will allow me to shed further light on the issues that are left open by Heller and Hills' important article.

\printbibliography

\end{document}

