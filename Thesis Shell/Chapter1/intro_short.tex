\chapter{Property, privilege and power}\label{chap:1}

\begin{quote}
It's nice to own land.\footnote{Donald Trump}
\end{quote}

\section{Introduction}

\section{Donald Trump in Scotland}

On the 10th of July 2010, the property magnate Donald Trump opened his first golf-course in Scotland, proudly announcing that it would be the ``best golf-course in the world''.\footnote{http://www.golf.com/courses-and-travel/donald-trump-scotland-golf-course-lives-hype (accessed 06 July 2014).} Impressed with the unspoilt and dramatic seaside landscape of Scotland's east coast, the New Yorker, who made his fortune as a real estate entrepreneur, had decided he wanted to develop a golf course in Balmedie, close to Aberdeen.

To realize his plans, he purchased the Menie estate in 2006, with the intention of turning it into a large resort with a five-star hotel, 950 timeshare flats, and two 18-hole golf-courses. The local authorities were not particularly keen on the idea and planning permission was initially denied by Aberdeenshire Council. Particularly worrying to the councilors was the fact that the proposed site for the development was declared to be of special scientific interest under EU conservation legislation. The frailty and richness of the sand dune ecosystem, in particular, suggested that the land should be left unspoilt for future generations.\footnote{See \url{http://en.wikipedia.org/wiki/Donald_Trump#Scottish_golf_course} (accessed 06 July 2014).} But in the end, Trump got his way, as he was able to convince Scottish ministers to give him the go-ahead on the prospect of boosting the economy by creating some 6000 new jobs.\footnote{See \url{http://www.theguardian.com/world/2008/nov/04/donald-trump-scottish-golf-course} (accessed 06 July 2014).}

Activists continued to fight the development, launching the ``Tripping up Trump'' campaign to back up local residents who refused to sell their properties to Trump. One of these, the farmer and quarry worker Michael Forbes, expressed his opposition in particularly clear terms, declaring at one point that Trump could ``shove his money up his arse''. Trump, on his part, had described Forbes as a ``village idiot'' that lived in a ``slum''\footnote{See \url{http://www.bbc.co.uk/news/10205781} (accessed 08 July 2014).} Moreover, he had suggested, through his legal council, that Forbes only refused to sell in an effort to get a better price for his land.  Forbes was offended and he proudly declared that he would never consider selling to Trump, as the issue had now become ``personal''.

At the height of the tensions, Trump considered his legal options, asking the local council to consider issuing compulsory purchase orders (CPOs) that would allow Trump to take property from Forbes and other recalcitrant locals. If carried out, this would have been an iconic example of an economic development taking, where private land is acquired to benefit for-profit development schemes. It would not be the first time that the power of eminent domain had been used to the benefit of Donal Trump's business empire. In the 90s, Trump famously succeeded in convincing Atlantic City to allow him to take the home of one Vera .... , to facilitate further development of Trump's casino facilities. This taking was struck down by state courts, however, a move hailed by many US activists as a milestone in their fight against economic development takings. 

In Scotland, Trump's plans were met with widespread outrage. The media coverage was wide, mostly negative, and an award-winning documentary had been made which painted Trump's activities in Balmedie in a highly negative light. The controversy also made its way into UK property scholarship. Professor Kevin Gray, in particular, a leading expert in property law, expressed his opposition by declaring that the proposed taking would be an act of ``predation''. 

In fact, the case prompted Gray to formulate a number of key features that could be used to identify situations where compulsory purchase would be more likely to represent an abuse of power. He noted, in particular, that Trump's proposed takings would fall in line with a general tendency in the UK towards using compulsory purchase to benefit private enterprise, often in the context of so-called ``public-private'' partnerships that are also meant to serve public purposes.

Hence, the possibility of CPOs being used in Balmedie was very much a real one; if he had put his weight behind it, Trump might well have been able to make a successful case that existing statutory authorities could be used to facilitate a taking of private land for his golf resort. Indirectly, this would undoubtedly benefit the public in terms of job-creation and increased tax revenues. Moreover, Scottish ministers had already gone far in expressing their support for the plans.

But then, in a surprise move, Trump announced he would not seek CPOs after all. Quite possibly, he was discouraged by the negative press and felt that public relations might suffer. More importantly, he had found another strategy, namely that of containment.
\noo{
If Trump had succeeded in getting CPOs to undertake his golf resort, Balmedie would no doubt have become the scene of an interesting legal battle concerning the status of economic development takings under UK law. The importance of such a case would be great and the outcome would be uncertain. But Trump did not pursue this course of action. Instead, he opted for a strategy of containment.
}
\noo{ But it was not to be. outcome would be uncertain. On  the one hand, the right to private property is considered to be at the core of most, if not all, western legal systems. In the UK, it is both a constitutional principle of common law, going back all the way to Magna Carta, as well as a human right enshrined in the Human Rights Act 1998. Recent developments would have been in Trump's favor, though. As one high court judge recently noted, before dismissing CPO objections made by property owners, ``.....''  But with much of the public so clearly on the side of the locals, the outcome would be far from guaranteed. Tampering with property, after all, is still risky business. Far better, then, to turn property into an advantage. }
He erected large fences, planted trees and created artificial sand dunes, all serving to prevent the properties he did not control from becoming a nuisances to his golfing guests. One local owner, Susan Monroe, was fenced in by a wall of sand some 8 meters high. ``I used to be able to see all the way to the other side of Aberdeen'', she said, `` but now I just look into that mound of sand''. She also lamented the lack of support from the Scottish government, expressing surprise that nothing could be done to stop Trump.

But there was little to do. As soon as Trump decided to build around them, the neighboring property owners found themselves completely marginalized. Trump had the backing of the government, having been declared as a job-creator whose activities would boost the economy in the region. Indeed, he had even received an honorary doctorate at the Robert Gordon University, a move that prompted the previous vice-chancellor ... to hand his own honorific back in protest.

In the end, then, it was not by the taking land of others that Trump triumphed in Scotland. Rather, he succeeded by exercising ``despotic dominion' over his own. But the problem was solved nonetheless: after he fenced them in, his neighbors were hard to see and hard to hear. The Balmedie controversy went quiet, the golfers came, Trump got his way. As he declared during the grand opening: ``Nothing will ever be built around this course because I own all the land around it.... It's nice to own land.''\footnote{See \url{http://www.theguardian.com/world/2012/jul/10/donald-trump-100m-golf-course} (accessed 06 July 2014).}

The tale of Trump coming to Scotland serves to illustrate the main issue I will be looking at in this thesis: the legitimacy of economic development takings. In addition, it serves to put the work into perspective, showing that what it means to protect property against undue interference can depend highly on the circumstances. For a while, it looked like Balmedie was about to become a canonical case of an economic development taking. But in the end, it became rather an illustration of something far more subtle, namely that our understanding of property's meaning and value is deeply shaped by social, political and economic structures. It seems clear, in particular, that Donald Trump's ownership of the Menie estate has a vastly different meaning than does Micheal Forbes' ownership of his farm. For many, the one represents some combination of power, privilege and progress, while the other represents a mix of defiance, community and sustenance. Very different and, in this case, diametrically opposed to one another. 

According to Trump and his supporters, protecting property rights against interference in Balmedie should no doubt involve protecting the governmentally sanctioned golf resort plans from interference by ``outsiders''. Perhaps even to the extent that it could justify interference with the property rights of other, less significant, stakeholders. But for Michael Forbes and the other local owners, protecting property rights is likely to have a completely different meaning, focused on defending them and their local community against a disruptive and damaging plan that would seem to turn both them and their land into golfing props. Again, perhaps protection would have to involve limiting what actions Trump might undertake on his own land, to the detriment of his neighbors. Either way, protection implies interference and vice versa. This is a major challenge to a simplistic view whereby protecting property rights is a black-and-white proposition, where governmental power on the one hand stands against personal liberty on the other. 

In reality, the situation is far more subtle and, importantly, how we asses it depends crucially on what we perceive as the ``normal'' state of property, the alignment of rights and responsibilities that we deem to be worthy of protection. This, in turn, depend highly on what values we believe is emodied in property and on what kinds of property rights that serve to enhance those values. For example, one value that is crucial for property activisits iin the US is the value of equality and freedom for the little man. In the US, where proeprty ownership is comparatively egalitarian, this spekasa in favor of limiting govenmental interference in property rights. In Scotalnd, on the other hand, where land owwnership is notoriously unegalitarian, it might as well mean land reform and the implementation of reditributive policies aiming to ensure a more equal distribution of land rights. This is just on eillustration of a braoder theme: the contextual nature of proeprty.

In this thesis, I will aim to take it into due account. Indeed, the main keyword I rely on, that of an economic development taking''  is based on a recogniztion that takings cases need to be looked at in light of the braoder poltical nad social, as well as economic, consequencses of a taking. That is when it becomes relevant to single out economic developmet takings as a specaial caaegor. 
Before goining into details about this, I first present theory. 

sway of looking at property and justice is invariably shaped by our perception of the 
namely that protecting property is not necessarily about giving owner's 
the meaning of property depends on the context. 
the meaning of protecting property against interference can raise quite subtle issues. It 

taking me straight to a core challenge raised by my work. It seems to me, in particular, that property is not an agent-neutral institution. That is, I do not think that property owned by Donald Trump has the same meaning, or even legal status, as property owned by a man like Michael Forbes. This might be a startling claim, but 
ets right to the heart of the issue, by flagging those aspects that are particularly controversial and problematic. 



First, it  serves as an example of the kind of scenario where the use of eminent domain raises special problems of fairness and justice. To take land from private owners to facilitate for-profit development projects is not like taking land for a hospital, a school, or a public road. There is a clear sense in which the taking benefits a privileged group of owners, those holding shares in the development company, and disadvantages another, those who stand to lose their property. 

The perceived unfairness of this is exasperated by the fact that current compensation regimes typically preclude the owner from taking any share of the benefit resulting from development. Hence, the taking of land becomes in effect a mechanism whereby the developer can capture the entire development surplus. On a deeper level, even if compensation mechanisms are put in place to achieve more adequate compensation, the taking still serves to transfer decision making power from landowners to the development company. The owners are marginalized in the decision making process regarding whether or not development should take place, and on what terms, and they are deprived of an opportunity to take part in the project as asset holders.

Hence, economic development takings have a {\it redistribution} function, both with regards to wealth and power. Problematically, the redistribution facilitated by economic takings tends to give {\it more} wealth and power to influential and affluent groups, while marginalizing people that are less privileged to start with. Such redistribution effects are not in themselves desired by government, at least not openly, and they do not serve to legitimize the use of eminent domain. From the government's point of view they are a mere side-effect, an unavoidable consequence of economic growth. But for both the takers and the property owners, the redistribution effect is of crucial importance. 

In this way, economic development projects are typically very different from the building of infrastructure or schools or hospitals or other projects operated by the public to directly further their interests. Moreover, it can often be argued that takings to the benefit of commercial projects are not legitimate, since they do not strike an appropriate balance between the interests of the public and the private owner. Many jurisdictions have constitutional property clauses to ensure that eminent domain only take place in the public interest, and most jurisdictions do not permit the taking of land unless it serves some public purpose. Hence, owners affected by economic development takings can make the case that their property rights are protected against such an interference, since a sufficiently compelling case that the taking is in the public interest can not be made.

But the case of Trump's golf resort in Scotland also illustrates another point, namely that in many cases, financially powerful actors will have no trouble acquiring the land they need from voluntary sellers. In these situations, the most controversial question that arises concerns the use that the developer plans to make of his land; should his plans be approved by the government, will they be in the public interest? This is a seemingly very different question from the question of legitimacy of takings, but as the case of Trump's golf resort shows, there are interesting connections between the two. In particular, while Trump was unable to acquire the land of some recalcitrant locals, he was able to secure enough land rights to enable him to effectively work his way around the opposition from those that refused to sell. His property rights, and the extent to which he could exercise them unhindered by governmental control, was key to his success. Those rights were given priority, with  both neighbors and other members of the public standing  powerless and  unable to prevent Trump from carrying out a project that they felt would be detrimental to the environment and the stability of the local community. They did try, by challenging his right, as a property owner, to make use of his land as he desired. Hence, a contrast emerges against how they would themselves invoke the sanctity of property rights when faced with the prospect of having compulsory purchase orders issued against them. 

Hence, perhaps the most important lesson to be learned from the controversy surrounding Trump's activities in Scotland is that property rights is a double-edged sword, both for owners and the general public, for privileged groups as well as for those that are  marginalized. If fairness is our measuring stick, moreover, it is of crucial importance {\it who} the owner is, and what purposes the land serves, to him and to community at large. Protecting the property rights of Forbes the farmer is largely tantamount to restricting the property rights of Trump the property tycoon, and vice versa. This further suggesting that no black and white perspective is feasible when talking about property as a fundamental human right. 

In light of this insight, I will devote the rest of this Chapter to explore the theoretical background to the question of economic development takings, as I explore this issue in the remainder of the thesis. Importantly, I will stress the contextual and purposive nature of property as a human rights, which should be looked at as an integrated part in a system of fundamental rights that award individuals and communities with a basis upon which they can flourish through self-governance. This view also focuses on the social obligations of property, and leaves great room for recognizing the need for regulation of property use within a framework focused on inclusive and just governance. If this was not so, as illustrated by the case of Trump's golf resort, protection of property would simply not be effective, except for those owners who already wield the financial and political power needed to implement their willed use of their land, possibly to the detriment of other owners. Under a simplistic notion of property rights, where they are conceptualized as a privilege, giving the owner exclusive dominion over that which is his, this creates an apparent paradox of property protection. To protect the property of one, in particular, will too often be tantamount to an assault on the property of another. However, by explicitly recgnizing that property comes with responsibilities as well as privileges, this tension can be resolved. Protection of property, in particular, is not then primarily about protecting the privilege, but protecting an elemental building block of community; the rights and duties bestowed on an individual as an owner, mutually co-dependent on other owners, the public, and various interest groups that interact with him as part of the democratic process.

This vision of property, and of property protection, is particularly helpful when looking at economic development takings, since it allows a more fine-grained analysis of why such cases tend to become controversial, and how it is best to deal with them. Moreover, the relationship between property and the values that it promotes, can suggest enhanced protection of owners in such cases, on the basis of a purposive and contextual reading of constitutional property law. 

The idea of property as a contextual phenomenon that involves both rights and responsibility has received much attention in recent scholarship. It is also an idea that features, at least implicitly, in recent proposals for takings reform in the US. Hence, I will now present the main elements of this theoretical shift in constitutional property law, focusing on how it will inform my analysis of economic development takings and the case study of Norwegian hydro-power development.

\section{Theories of property}

What is property? In common law jurisdictions, the standard answer is that property is a collection of individual rights, or more abstractly, {\it entitlements}.\footnote{The term ``entitlement'' was used to great effect in the seminal article \cite{calabresi72}.} Being an owner, it is often said, amounts to being entitled to one or more among a bundle of ``sticks'', streams of protected benefits associated with, or even serving to define, the property in question.\footcite[357-358]{merrill01} This point of view was first developed by legal realists in response to the natural law tradition, which conceptualized property in terms of the owner's dominion over the owned thing, particularly his right to exclude others from accessing it.\footcite[193-195]{klein11} In civil law jurisdictions, rooted in Roman law, such a perspective is still often taken as the theoretical foundation of property, although it is of course recognized that the owner's dominion is never absolute.\footnote{See, e.g., \cite[?]{foster10}.}

In modern society, the extent to which an owner may freely enjoy his property will invariably be highly sensitive to government's willingness to protect, as well as its desire to regulate. To civil law theorists, this sensitivity is most naturally thought of in terms of restrictions, but for common law theorists, overlooking a legal system with roots in a relatively stable feudal tradition, it is more natural to think of it as {\it constitutive} of property itself.\footcite[7]{chang12} In particular, the bundle of rights theory was not based on entirely novel ideas. While natural law had a significant impact on common law theorizing about property in the 18th and 19th Century, the advent of the modern regulatory state saw the return to a more modest notion.\footcite[195]{klein11} Arguaby, this notion was conceived from the essence of the traditional estates system for real property, which was turned into a theoretical foundation for thinking about property in the abstract.\footnote{See \cite[7]{chang12}(``The ``bundle of rights'' is in a sense the theory implicit in the common law system taken to its extreme, with its inherently analytical tendency, in contrast to the dogged holism of the civil law.'').} 

Property rights under the bundle of rights theory are thought to be directed primarily towards other people, not things.\footnote{See \cite[357-358]{merrill01} (``By and large, this view has become conventional wisdom among legal scholars: Property is a composite of legal relations that holds between persons and only secondarily or incidentally involves a ``thing''.'').} This underscores that the content of rights in property are necessarily relative to the totality of the legal order. Hence, it becomes perfectly natural that a farmer's property rights protects him against trespassing tourists, but not against the neighbor who has an established right of way. 

Under the dominion theory, on the other hand, situations like these must be explained as exceptions to the general rule of ownership. In the case of property, exceptions no doubt make up the norm. But in civil law jurisdictions one lives happily with this. It takes the grandeur away from the dominion concept, but it retains a nice and simple structure to property law. There are many limitations and additional benefits attached to property, but they are all in principle carved out of one initial right, namely that of the owner. In this way, the system becomes more easily navigable. An interested party may ask: ``who is the owner?''. Under the dominion theory, a clear answer is expected and will usually be adequate, even if it does not give a complete picture of all relevant property rights. Under the bundle theory, on the other hand, one may be inclined to respond: ``to which stick are you referring?''. Clearly, this complicates the matter, perhaps unduly so. In recent work, common law scholars have elaborated on this idea and developed a critique of the bundle theory, suggesting that it should at least be complemented by a firm theory of {\it in rem} rights in property.\footnote{\footcite[793]{merrill01b} (``The unique advantage of in rem rights -- the strategy of exclusion -- is that they conserve on information costs relative to in personam rights in situations where the number of potential claimants to resources is large, and the resource in question can be defined at relatively low cost.''); \footcite[389]{merrill01} (``The right to exclude allows the owner to control, plan, and invest, and permits this to happen with a minimum of information costs to others.''). See also \cite{ellickson11} (arguing that Merrill and Smith's analysis nicely complements and improves upon the bundle theory).}

The debate between bundle and dominion theorists is also active along different , and while the bundle of stick theory is dominant in the common law world, it still attracts critical scholarly attention.\footnote{.....} Recently, there has been a surge of writers arguing that an exclusion-based approach is preferable. They point out that the bundle of rights theory does not adequately reflect the sense in which property is a right to a {\it thing}, not merely a set of legal relationships which are set up and regulated by government. Property ownership has a static {\it essence}, it is argued, which is not captured by the metaphor of an ever mutable bundle.

In this thesis, the efficiency and clarity of different property concepts will not be a primary concern. Hence, I will remain largely agnostic about this aspect of the debate between dominion and bundle theorists. In particular, the differences between civil and common law traditions in this regard are not such that they need to be addressed in relation to my analysis of economic development takings. To me, a more important question concerns the {\it values} that various property theories serve to promote, particularly with regards to the question of when interference in property is legitimate under constitutional and human rights law. Are there difference between the two theories in this regard? 

Intuitively, one might think that bundle theorists are likely to endorse greater room for state interference in property rights. Indeed, thinking about property as sticks in a bundle may lead one to think that property rights are intrinsically limited, so that subsequent changes to their content -- carried out by a competent body -- are but reflections of their nature, not a cause for complaint. In particular, the theory serves to enhance the impression that property is malleable. For the theorists that developed the bundle of sticks metaphor in the late 19th and early 20th Century, this aspect was undoubtedly very important. They did not only develop a theory to fit the law as they saw it, they also contributed to change. By providing a highly flexible concept of property, they ensured that the state would gain conceptual authority to control and regulate. This was the clear intention of many early proponents of the bundle theory -- the ``progressives'' of their day.\footcite[195]{klein11}?

In relation to eminent domain, the progressives succeeded in gaining acceptance for the use of eminent domain to benefit a wider range of public purposes than had so far been considered legitimate.\footnote{See generally \cite{yale49}. I return to this development in US law in much more depth in Chapter \ref{chap:2}, Section \ref{sec:?}. } There can be little doubt that this development was helped by, and mutually conducive to, the conceptual reorientation that took place during the same time. In relation to the different, but related, issue of so-called regulatory takings, the bundle theory even  became directly implicated. 

A regulatory taking occurs when governmental control over the use of property becomes so severe that it must be classified as a taking in relation to the law of eminent domain. Particularly in the US, the question of when regulation amounts to a regulatory taking is highly controversial. The stakes are high because takings have to be compensated in accordance with the Fifth Amendment of the US constitution. At the same time, the law is unclear; a lack of statutory rules means that regulatory takings cases are often adjudicated directly against constitutional property clauses (often the relevant state constitution, in the first instance).

If property is thought of as a malleable bundle of entitlements that exist only because it is recognized by the law, it becomes more natural to argue that when government regulates the use of property, it does not deprive anyone of property rights, but merely restructures the bundle. In the case of {\it Andrus v Allard}, the Supreme Court adopted such an argument when it declared that ``where an owner possesses a full ``bundle'' of property rights, the destruction of one ``strand'' of the bundle is not a taking, because the aggregate must be viewed in its entirety''.\footcite[65--66]{andrus79}

Against this, some prominent scholars have argued for an almost entirely opposite view. Professor Epstein, in particular, goes far in suggesting that as every stick in the property bundle represents a property right, government should not be permitted to remove any of them without paying compensation.\footcite[232-233]{epstein11} Moreover, Epstein does not believe that the bundle theory is responsible for the fact that his view of property have not been widely relied on by US courts. Instead, he thinks the real reason behind what he sees as the pernicious influence of progressive thought is related to their ``top-down'' approach to property. That is, their tendency to view property rights as vested in, and arising from, the power of the state, not the possessions of individuals.\footnote{\cite[227-228]{epstein11} (``In my view, the nub of the difficulty with modern property law does not stem from the bundle-of-rights conception, but from the top-down view of property that treats all property as being granted by the state and therefore subject to whatever terms and conditions the state wishes to impose on its grantees''.).} In my view, Epstein is correct in thinking that the bundle theory itself is not likely to serve as a determinate factor with regards to the level of protection private property enjoys in a given legal system. Moreover, he successfully demonstrates that as a rhetorical device, the theory may well be turned on its head. Unsurprisingly, the substance of the law, in the end, turns primarily on the values one adheres to, not the theoretical constructions one relies on when expressing those values.\footnote{To further underscore this point, it may be mentioned that while US courts do in fact recognize that a regulation can amount to a taking, this is practically unheard of in several other common law jurisdictions, including England and Australia. Moreover, while the issue of regulatory takings is considered central to constitutional property law in the US, it is considered a fairly marginal issue in England.\cite{altermann12}}

In the civil law world, the relationship between property theorizing and property values is similarly hard to pin down. To illustrate, I will again point to the question of regulatory takings. In some countries, like Germany and the Netherlands, the right to compensation is quite strong, but in other civil law countries, such as France and Greece, it is very weak. The exclusive dominion understanding of property does not commit us to any particular kind of policy. Indeed, the theory appears to cater comfortably to a range of different politically determined solutions to the problem of striking a balance between the interests of owners and the interests of the state. 

On the one hand, the undeniable fact of modern society is that property rights are both enforced by, and limited by, the power of government. Hanging on to the idea of dominion, then, necessarily forces us to embrace also the idea that dominion is not enjoyed absolutely and that government may interfere in property rights. In this way, the theory may serve as a conceptual basis upon which to argue for a more relaxed approach to protection of property rights. But this story too may be turned on its head: A libertarian may well use the same basis to argue that the notion of property as dominion is under increasing threat from the interfering state. Hence, he may go on, unless we want to completely lose our grasp of what property is, we had better enhance the level of protection we offer owners against state interference. 

To me, the upshot is that the differences between common law and civil law theorizing about property are not significant enough to 
make them crucial to the questions studied in this thesis. In particular, the differences between the bundle theory and the exclusive dominion account of property do not appear to speak decisively in favor of any particular approach to economic development takings.   Property enjoys constitutional protection and is a recognized human right across the divide, but what this means in practice is hard to deduce from both the bundle and the exclusive dominion view of property. In a sense, they are both too extreme in their outlook. They provide a manner of speech, but they do little to enhance our understanding of the reality of property rights in modern society. In particular, they do not provide a functional account of what role property plays in relation to the social, economic and political structures within which it resides.

I will now consider the question of how such an account can be given. The answer will have great significance to the remainder of the thesis. In particular, it will answer a pressing problem related to the core concept of an ``economic development taking''. For it is not {\it prima facie} clear that this concept makes any legal sense at all. It clearly makes intuitive sense; property is taken for economic development. Indeed, in most examples we will consider this is even the explicitly stated aim used to justify eminent domain. But neither the exclusive dominion account nor the bundle theory gives us any reason to think that such takings should be considered as a separate category. It is not clear, moreover, if we may legitimately attach legal significance to their distinguishing features. This, indeed, is a claim that some might be intuitively inclined to reject. Why, after all, is it appropriate to have regard to the {\it purpose} of the taking, when considering its legitimacy with respect to the owner's rights? Would those rights not have been equally interfered with if his property has been taken from some uncontroversially public project, like a new road? 



I will return to this question in Section \ref{sec:x} below, when I set out an argument that economic development takings should be recognized as a special class of takings in relation to which a range of special considerations should typically be made. First, I will expand my theoretical horizon by presenting some theories of property that will aid me in identifying those aspects that make economic development takings special. In addition, I will argue that those aspects are already relevant -- behind the scenes -- when the legitimacy of such takings is assessed. Bringing the hidden theoretical commitments behind this into the open, I argue, is a worthwhile project to pursue. I do so below, by drawing on the so-called {\it social function theory} of property. Unlike many contemporary writers, I present this theory as a descriptive account of property's function in society and as a way to make sense of socio-legal arguments that often play an important, but often unacknowledged role, when courts interpret the fundamental right to property. I then single out for attention some of those values that have been highlighted in relation to property's social function, focusing on the notion of {\it human flourishing}.\footnote{As explored in the context of property law, particularly in the work of Gregory Alexander.}

he dominion and the bundle of rights theory shows that the true conceptual import of the ``bundle of rights'' metaphor is hard to measure along an axis of property protection, understood as the owner's right to retain privileges associated with property ownership. Whether the bundle theory affords the owner more or less protection in this sense depends entirely on the view one takes on the individual sticks of the bundle. If they are thought of as less fundamental and worthy of protection than the bundle as a whole, less protection follows. On the other hand, if such sticks are understood as first-class property rights in themselves, more protection would follow.

My thesis will not actively address this debate, so I will merely note that is is good law in most jurisdiction, including the US, that a governmental action will not be regarded as a taking unless it upsets the structure of the bundle above a certain threshold. In most cases considered in this thesis, identifying this threshold will not be present us with difficulty; economic development takings tend to imply complete deprivation, not a mere reshuffling of the property bundle. Hence, I do not need to venture far into the murky waters of the ``regulatory takings'' issue, which is the name associated to this question in the US.\footnote{I will, however, briefly return to this issue in Chapter 5, when I discuss alternatives to takings in economic development cases. Such alternative, in particular, may involve participatory frameworks where the property bundles are restructured in a less invasive manner, to achieve the desired development without having to resort to outright takings.}

In my opinion, the history of property theorizing shows that the debate between bundle of rights and exclusion theorists does not have a crucial bearing on the question of legitimacy of takings for economic development. On the one hand, under both theories it is usually straightforward to identify typical cases of economic development takings, when both the owner, the taker and the government agree that the interference is such that it is to be classified as an exercise of eminent domain. While it is still of relevance how we think of property, the focus then naturally shifts from the theoretical question of what property is, to the more practical question of when it may legitimately be taken. Moreover, as illustrated by Epstein's work in particular, both theories of property are amenable to interpretations that can be used to argue in favor of a more or less restrictive attitude towards takings. 

My response to this is twofold. First, economic development takings are different because fair compensation is very difficult, if not impossible, to provide in such cases. If an owner gives up his property for a non-commercial project, he is usually entitled to have his economic loss covered by the public, in many jurisdictions based on the {\it market value} of his property, possibly including also some extra compensation for personal losses. When the proposed development is not for commercial profit, this approach, based on compensating the owner's loss compared to the value of his property before the taking, is the commonly accepted approach to compensation determination. It is also usually perceived as fair. 

A possible alternative would be to base the compensation on the amount that the public might be willing to pay, considering the importance of the planned project. But this would effectively allow the owner to capture a profit from a not-for-profit public project, creating a situation where important public projects may end up becoming prohibitively expensive. It would also largely undermine the very idea of eminent domain, which is meant to prevent owners from demanding extortionate prices in voluntary negotiations over the sale of property needed for important public projects. In fact, as long as the not-for-profit nature of the public project is clearly entrenched, an argument may even be made that less then market value is in order, in so far as the regulatory powers of government extends to engaging in the the sort of value-reduction that is entailed by the taking. 

In this regard, economic development takings play out very differently. If compensation is based only on the owner's loss, calculated based on the pre-project value of the land, the taking effectively deprives the owner of the land from his share in the surplus resulting from development of his property. The developer, in particular, does not only profit from the development itself, he also profits directly from the use of eminent domain. Intuitively, it seems perfectly clear that this is a very different situation from the typical cases of not-for-profit takings in the public interest. However, this recognition raises a subtle theoretical point, not resolved neither by the bundle of rights nor the exclusion based theorizing about property. 

First, notice that the intuitive conclusion reached above rests on the conceptual premise that the value released by development is in part inherent to the land as such, not entirely created by the subsequent efforts and investments of the developer. This is probably  uncontroversial, however, as such a perspective lies at the very heart of any market economy based on private property rights. Indeed, after the land is transferred to the developer, the value of the property {\it to him}, will now reflect the part of the development surplus that the market regards as being inherent to the land. If he sold it to a third party, he could expect the price of the property to reflect this fact.

However, there is a second assumption that must be addressed, which is less obvious. What property right, in particular, serves to give the original owner a claim to take as share of the development surplus? That he does have such a claim seems intuitively clear. In particular, any property owner not effected by compulsory acquisition would be able to either take active part in its most profitable permitted use or else bargain for a share of the development surplus when selling to an interested developer. The prospect of this, moreover, is why there exists a market for property in the first place. Intuitively, it seems clear that the surplus stemming from a beneficial, permitted use of property is attached to the property itself and hence belongs to the bundle of rights associated with property ownership, or, if we adopt the exclusion theory, that it falls under the owner's dominion.

However, in some jurisdictions, for instance in England, the idea that all development value belongs to the State has been influential. In this case, while the right to use the property in accordance with governmental regulation is undoubtedly part of the owner's bundle, a {\it change} in the regulatory status of property might have to be seen as an act of government granting a new right in the property. In this case, an economic development taking might be recast as destroying an existing right, belonging to the owner, and creating a new one, belonging to the developer. This perspective, then, fails to provide a theoretical basis on which to rebuke the lack of compensation for the development surplus in economic development takings. The right to the surplus is not taken at all, but {\it created} by the government. Still, it hardly accords with a natural sense of justice when this right is simply bestowed on the commercial development company, with no regard for the original owners. 

In particular, any property owner not effected by compulsory acquisition would be able to either take active part in its most profitable permitted use or else bargain for a share of the development surplus when selling the property. The prospect of this, moreover, is why there exists a market for property in the first place. Hence, we arrive a theoretical deadlock, where the idea that development value belongs to the State gives rise to the unacceptable outcome that the property rights of owners affected by takings are second-class rights, not of the same kind as those rights enjoyed by other owners.

\subsection{The social function of property}

The problem, I believe, is not intrinsically with the idea that development value belongs to the state, but rather that the notion of property has been drawn too narrowly by the dominant theories used in property law. In other fields, such as political science, sociology and human geography, property is not understood as merely a set of individual entitlements. Rather, property is seen as a crucial part of the fabric of society. To be an owner comes with political power, social responsibility, and membership in a community. Those aspects, moreover, are often more important that entitlements explicitly recognized in positive legal terms. Moreover, they are important not only to the individual owners but also to society as a whole. How property rights are distributed among the population, for instance, has obvious political and economic implications, serving as a source of power and privilege to some groups, while marginalizing others. 

What is the relevance of this to property law? Usually, jurists approach property in isolation from these concerns. The political question of what the law should be might require musings about the purpose and social context of property, but in day-to-day workings of the law, it is often assumed, such considerations play a lesser role. In particular, while no functioning theory of property would deny that social aspects and obligations can play a role in relation to property, the classical theories do not make room for the social function as an {\it intrinsic} aspect of property. In the common law world, the bundle of rights metaphor is often thought to provide an adequate theoretical basis for property law, while in the civil law world, the dominion theory tends to serve a similar role. In some jurisdictions, however, it very soon becomes clear that neither of these do justice even to the most basic rules concerning property. In Germany, for instance....

The ECHR....

Even in the US, the idea that jurists need to look more actively to the social function of property is gaining ground, both as a descriptive truth about property law as it is and a normative proposition about property law as it should be. It is fair to say, therefore, that the social function theory of property is gaining ground internationally. In the words of Professor Gray, ``the stuff of modern property theory involves a consonance of entitlement, obligation and mutual respect''.\footnote[37]{gray11}

While the social function theory has important normative implications to which I will return later, I would like to stress that in the first instance it merely recognizes an empirical truth. Property does not arise in a vacuum, but is socially defined. This much is obvious and no one denies it. But what the social function theory asks us to acknowledge is that even after it has been defined, property continues to play an important social role. It continues to influence, and be influenced by, the social fabric within which it is found. Perhaps most importantly, property both reflect and shape relations of power among members of a society. 

It does not act uniformly in this way, however, as the actual effect of property depends on the social context. An owner deep in depth who is subjected to strict state regulation, for instance, might see his ownership becoming a burden rather than a privilege. Simultaneously, a person who has already amassed great power and wealth might purchase it form him, seeing it as an excellent opportunity to consolidate his position, enabling him to entrench his privileges in a more stable legal form. 

Property also shapes and reflects social commitments, but again in a way that is highly contextual. A small business owner, by virtue of being a member of the local community, is discouraged from becoming a nuisance to his neighbors. If he does not conform to social expectations, he will pay a social price. But in addition to this negative constraint, the local connection would also typically serve to make the business owner positively invested in the well-being of the community. If his customers are mainly local, for instance, the prosperity of the community becomes proportional to the profit he makes. Moreover, the fact that his business is embedded in the community in this way will make it hard for him to change his business model if this is perceived to undermine these ties. To become more effective and profitable, economic rationality might suggest that he should expand, by physically acquiring more space and targeting new groups of customers. But social rationality might make this an untenable proposal.

On the other hand, if our local shop owner goes bankrupt, perhaps as a result of new regulations, he might be forced to sell his shop. If we imagine that the buyer is a large commercial actor who is hoping to raze the community in order to build a new shopping center, we are at once reminded of the stark contrasts that exist between various social functions of property. The property rights of a shop owner can be the life nerve of a community, while the exact same rights in the hands of someone else can spell destruction. While this is an undeniable empirical fact of property ownership, it is far from clear what its legal ramifications are. Here, it is tempting to embrace a normative stance, and argue for particular social values that the law {\it should} promote in this regard. However, I would like to hold on to the descriptive mode of analysis a little further. It is perfectly clear, in particular, that the social function of property is often of great practical importance also in relation to the law. 

This is true not only when the law explicitly requires that this function is to be taken into account, such as in relation to the property clause of the basic law of Germany. It also commonly arises as an important source of information guiding the courts in interpreting and applying statutory authorities that seemingly are not concerned with social aspects of property. The classical example from the US is the case of {\it State v Shack}. The case concerned the right of a farmer to deny others access to his land, a basic exercise of the right to exclusion often regarded as fundamental to the very definition of property. The controversy arose after the two defendants, who worked for organizations that provided health-care and legal services to migrant farmworkers, entered the land of a farmer without permission. They were there to provide services to the farmers employees and when the farmer asked them to leave, they refused. 

In the first instance they were convicted of trespassing in keeping with New Jersey state law, but on appeal the Supreme Court of New Jersey overturned the verdict. The court held that the dominion of the land owner did not extend to dominion over people who were rightfully on his land, so that as long as the defendants were there at the request of the workers, the owner had to tolerate this. Importantly, the court argued for this result -- which was not based on any natural reading of New Jersey trespass statutes -- by pointing also to the fact that the community of migrant workers was particularly fragile and in need of protection. Their property right to receive visitors where they work and live, therefore, had to be recognized, in spite of this limiting the exclusion right for the farmer. 

The lesson to take from this is that the social function of property can play a role even when this does not explicitly follow from any property rules. This, in turn, suggests why a shift towards a social function theory is desirable. In so far as the property rules we rely on explicitly directs us to take the social aspect of property into account when applying the law, it might be permissible for the practically minded jurist to conclude that there is little need for theorizing about property's social dimension. But as a matter of fact, cases like {\it State v Shack} show that this dimension can be relevant even when it is not mentioned in any authority, even in relation to clear rules that would otherwise appear to leave little room for statutory interpretation. It arises as relevant, in particular, because the social dimension is intrinsic to property itself. 

This might still be a radical claim, but it is still primarily a descriptive one. Indeed, even if the case of {\it State v Shack} had gone the other way, I would be inclined to take from it the same lesson: if the owner's right to exclusion had received priority over the workers right to receive guests and the owner's obligation to respect this right, that too would be an outcome that would underscore the social function of property, further demonstrating the plight of migrant workers and their fragile communities. But there is nothing in the social function theory, as a descriptive proposal, that would rule out such an outcome. On the other hand, if arguments based on social functions of the exclusion right had been summarily dismissed on the basis that they were irrelevant to the case, the entitlement-based view on property would in effect do unacknowledged normative work, with no basis in anything more authoritative than a palpably oversimplified idea of the meaning of property.

I stress that I take the social function theory to be primarily descriptive, but that is not to say that it does not have normative consequences. It provides a way of talking about property and analysing property disputes that will also invariably influence how we come to judge concrete cases. This is illustrated by the case of {\it State v Shack}. Surely, after the social context and consequences of the right to exclusion contested in that case had been made clear and recognized as legally relevant, any assessment supporting a guilty verdict would have to rely on value judgments that many of us would naturally shun away from. But the crucial aspect of the social function narrative is that it makes this aspect clear, not that it commits us to, or promises to deliver, any morally superior  stance on property that deliver ``correct'' outcomes in cases such as this.

In taking this view, I depart somewhat from many of the contemporary scholars who advocate on behalf of social function theories. Writers like Professor Hanoch Dagan, for instance, explicitly and strongly argue for such theories on the basis that they are morally superior. ``A theory of property that excludes social responsibility is unjust'', he writes, and goes on to argue that ``erasing the social responsibility of ownership would undermine both the freedom-enhancing pluralism and the individuality-enhancing multiplicity that is crucial to the liberal ideal of justice''. If this is true, then it is certainly a persuasive argument for those who believe in a ``liberal idea of justice''. But for those who do not, or believe that the law is -- or should be -- agnostic on this point, a normative justification for the social function theory along these lines can only discourage them from adopting it. Such a reader would be understandably suspicious that the {\it content} of the social function theory itself is biased towards a liberal world view and that it carries with it normative commitments that are meant to promote liberalism. 

Danach is not alone in proposing highly normative social function theories. Value-based argumentation features heavily also in the proposals of ...... While these writers all provide interesting insights into the nature of property, I am struck by a feeling that they overstate the desirable normative implications of adopting the social function view. In addition, they appear to believe that accepting this view on property requires us to embrace certain values and reject others. Moreover, one is left with the impression that the theory has little to offer beyond the values with which it is imbued, which can in turn push the law in some direction that is deemed desirable. 

But I disagree that this is the case, at least for the social function theory as I understand it. Of course, Dagan's theory of property might well be conducive to ``liberal justice'', but if it is, then it entails more than just the proper recognition that property's social function should be considered when applying the law and attempting to adjudicate on the rights and obligations attached to property. It is Dagan's purpose to pursue a theory that promotes specific liberal values. This is quite clear. ``There is room to allow for the virtue of social responsibility and solidarity'', he writes, continuing by suggesting that ``those who endorse these values should seek to incorporate them -- alongside and in perpetual tension with the value of individual liberty -- into our conception of private property''. 

My objection is not that this is necessarily wrong, but that it need not be accepted in order to conclude that the social function of property should be given a more prominent place in property theory. Importantly, I think the focus on normative reasons threatens to overshadow the most straightforward reason for awarding social structures a more prominent place in the analysis, namely that they are almost always crucially important behind the scenes, as a matter of brutal fact. The social function theory, rather than being ``good, period'', as Danach suggests, is nothing more or less than accurate, irrespectively of one's ethical or political inclinations. As such, it provides the foundation for a debate where different values and norms can be presented in a way that is conducive to meaningful debate, on the basis of a minimal number of hidden assumptions and implied commitments. Thus, the first reason to accept the social function theory, for me, is epistemic rather than deontic.

That is not to say that normative theories should not be formulated on the basis of the social function theory, it merely means that I believe it is useful to maintain at least a theoretical division between the descriptive and normative aspects of such theorizing. For this reason, I propose to designate by the term ``social obligation'' theories those aspects of recent sholarship that are most clearly normative in nature. I return to normative aspects in the next section, arguing that the commitment to ``human flourishing'' endorsed by Professor Alexander is a particularly well-argued norm that arises from value-based assessment of the social function of property. This, I argue, is in large part also due to the value pluralism inherent in this idea, suggesting as a positive normative claim that our notions of property {\it should} allow a divergence of opinions and values regarding property to influence the law and its application in this area.

I believe the history of the social function theory lends support to my claim that it is useful to emphasize that the theory is in the first instance descriptive. This theory is highly abstract, in particular, leaving its content hard to pin down. This, indeed, is also recognized by contemporary scholars endorsing a normative view. Moreover, history shows that from social function theorizing alone it is not easy to predict what normative imperatives such theorists might come to endorse regarding the role property {\it should} play in the fabric of society. 

An excellent example of this is provided by Professor di Robilant in a recent paper tracking the history of social function theorizing in Italy during the fascist era. The fascist property scholars, she notes, were happy to embrace the social function theory, since it provided them with a conceptual starting point from which to develop their idea that rights and obligations in property should be made to answer to one core value: the interests of the state.\footnote{See \cite[908-909]{robilant13} (``Fascist property scholars had also appropriated the social function formula. For the Fascists, the social function of property meant the superior interest of the Fascist state.'').} The happiness with which the fascists embraced social function theorizing serves as a reminder that we cannot easily predict what normative values may come to be promoted on its basis. Hence, it is also call for vigilance when it comes to normative assessment and debate. 

At the same time, we are reminded of the danger of attaching too much normative prestige to a theory that is abstract and open to various interpretations. As de Robilant notes, ``earlier writers had been hopelessly evasive about the meaning and content of the social element of property''. The Fascist approach, therefore, must have appeared as an attractive proposal. As De Robilant argues, its supporters made clear that ``social function meant the productive needs of the Fascist nation'', while at the same time denying that there was a ``contradiction between subordinating individual property rights to the larger interest of the Fascist state and the liberal language of autonomy, personhood, and labor''. In this way, fascist liberalism also came to be seen as the true liberalism.

I believe the history of the fascist property scholars points to the desirability of maintaining a descriptive perspective on the social function theory. This theory points to a key aspect of property, often downplayed in legal reasoning, but recognition of this fact is no guarantee for ``better'' law. In particular, it seems to me that failure to recognize the descriptive nature of the core idea can lead to unrealistic expectations of what the social function theory actually provides. In addition, it will make it harder for the theory to gain acceptance as a conceptual common ground from which to depart when taking part in normative debate. Indeed, if no division is recognized between normative and descriptive aspects, the historical record would allow detractors to make a {\it prima facie} plausible attack on the social function theory by arguing that it is fascism in disguise, or that fascism, rather than liberal justice, is where we end up in practice should we chose to adopt it. 

In response, one might retort that this is cherry picking the historical facts, or that the fascists misunderstood or perverted the theory. That is certainly plausible, but the point I am trying to make here is that this kind of debate is in itself unhelpful. Unless the social function theory is rendered neutral enough to be acceptable as the conceptual premise of debate, it is likely going to fail -- in a purely epistemic sense -- as a template for negotiating conflicts about property. Those who oppose the norms associated with the theory will oppose also the core descriptive content, if they feel that the latter commits them to the former. I believe that this, in turn, suggests that those advocating on behalf of the social function theory should take care to avoid rhetorical hubris. The main point to convey, I believe, is that the theory is in fact more accurate, in a purely epistemic sense, than other conceptualizations of property.

To suggest the power of such a perspective, it is illustrative to look at how non-fascist Italian scholars responded to fascist attempts at usurping the theory for their own purposes. Professor di Robilant gives a highly thoughtful and detailed account of their work, serving as an illustration of the kind of spirited public debate about property that a social function theory can promote. 

In response to the normative monism of the fascists, the non-fascists focused on the plurality of values that could potentially inform the social functions of property in different kinds of resources. But in addition, they also noted that property rights were invariably associated with {\it control} over resources. To own property, in particular, provides individuals with a source of privacy, power and freedom that is,  as a matter of fact, highly valued. It is valued, however, for its implications in a social context. Italian scholars adopted the metaphor of a ``tree'' , describing the core social function of property as the trunk, while referring to the various resource-specific values attached to property as branches. According to di Robilant, this approach was largely a response to the fascists:

\begin{quote}
The rise of Fascism, they realized, was the
consequence of the crisis of liberalism. It was the consequence of liberals' insensibility to new ideas about the proper balance between individual rights and the interest of the collectivity.\footcite[907]{robilant13}
\end{quote}

In light of this, the tree-theorists concluded that continued insistence on the protection of the autonomy of owners was not a viable response. Instead, they adopted a theory that `` acknowledges and foregrounds the social dimension of property'', but without committing themselves to fascist ideas about the supreme moral authority of the state. The value of autonomy was in turn recast in terms of property's social function. Arguably, this served to make the case far more compelling. Protecting autonomy could be seen as an aspect of protecting property's freedom-enhancing function, both at the individual level and as a way of ensuring a right to self-governance for families and local communities. This, moreover, could not easily be derided as tantamount to protecting unfair privilege and entitlement. It became more natural to see it as an effort to protect democracy itself.

This shows how the social function theory can serve an important function by allowing us to recast whatever values we wish to promote, providing qualifications for them in terms of their social function. The fascists also tried this, of course, and in the end one could do little more than hope that their vision of an ``ethical state'' that ``every man holds in his heart'' would eventually prove less attractive then the promise of self-governing communities bustling with diversity in life and character. The upshot was, however, that the discussion could go on, without the breakdown of communication that would result if Italian anti-fascists had refused to give up old dogmas. 

The normative reshaping that the social function theory can facilitate is closely related to how it allows us to recognize more subtle distinctions between different kinds of property and different kinds of circumstances. A staunchly entitlements-based approach to autonomy will leave us with little room to differentiate between the protection of investment property and the protection of a home, unless such a distinction is explicitly provided for in the law. But a social function approach compels us to notice the difference and to acknowledge that it may be legally, as well as ethically, relevant. Hence, if we seek to argue for protection of investment property, we must in principle be prepared to face counter-arguments that revolve around particulars of the investor's role in society and his relationship to the community of people that are affected by how he manages his property. Similarly, if someone argues against protecting home ownership, we can respond by drawing on additional arguments based on the importance of the home both to its owner, her family and her friends. Under the social function theory, it becomes generally relevant to address how a home creates a sense of belonging and provides a basis for membership in a community.

This kind of reasoning can also be performed more abstractly, to help us identify certain kinds of situations. In this way, we can provide an analysis of legal problems that it might not even be possible to recognize as such were it not for the conceptual reconfiguration. In my opinion, the problem of economic development takings is an example of this. It is not {\it prima facie} clear in particular, that this category of takings should be addressed as a special category at all. Intuitively it makes sense to do so, but legally, I believe doing so involves at least some commitment to a social function understanding of property. More so than what is typically available under classical theories of property. It is telling, in particular, that economic development takings only really came to prominence as a special category following the public outcry after the case of {\it Kelo}. In many jurisdictions, moreover, it has still to be recognized as a working category of legal (or political for that matter) reasoning. 

I return to economic development takings in Section \ref{}, when I discuss its defining features in terms of the social function understanding of property. First, however, I would like to address the normative foundation of my work. It is not my intention, in particular, to remain entirely descriptive for the remainder of this thesis. Moreover, while I think the social function theory itself can be thought of in this way, there are several closely related, distinctly normative, developments that have emerged alongside it. I will be drawing actively on these in my normative assessment of economic development takings and the data I present on Norwegian waterfalls and hydro-power. Hence, in the coming Section I present some key ideas that have inspired me in my normative analysis. 

While I believe that normative assessment should aim to be as concrete as possible, I still think it is worthwhile to provide more abstract forms of expression for core values, to clarify the ethical premises that provide the basis for concrete value-based conclusions. However, I would also like to stress that I believe abstract ethical assertions are necessarily imprecise. Hence, the most accurate information regarding the values I rely on when assessing cases will be conveyed by my assessment of the cases themselves. On a deeper level, I believe value-systems are unique to individuals, so that ethical theories are helpful primarily in that they provide an introduction to keywords and important lines of argument that will recur in different forms. As such, they enhance understanding, making it easier to communicate ideas and opinions in such a way that potential respondents will at least be provided with a less inaccurate impression of what they are responding to. 

An ethical theory, first and foremost, make values communicable, allowing new ideas to be created in the minds of individuals. It should come as no surprise to the reader, then, that I believe in ethical men and women, but not in ``ethical Man'' or -- God forbid --  the ``ethical state''. Luckily, I find some support for this world view in recent theories that have been proposed as normative extensions of the social function theory of property. These are the subject of my next section.



of intention   pointing in the right direction. 


or the opinions of others is in any event of limited importance when discussing 


avoid misunderstandings or



case study. 





























unhelpful 

 how the basic idea can be molded to support extreme views of any 


this would likely spell failure for any attempts to utilize it




The problem with any theoretical construction is how far it goes in providing a foundation on which to make claims about moral superiority. 


it might well be 


and
This raises a further, more acute question: do the social function theory s




t lead them to believe that they are ``off the hook'', free to discard the theory as just another liberal fancy.

 the social function theory, it would seem, is only a corollary of ``liberal idea of justice''. In addition, it see


Another example is Professor Gregory Alexander, who also conceives of the social function theory in strongly normative terms, by declaring that 



 judgments required to 

A theory of property that excludes
social responsibility
is unjust.





In particular, the argument that 


The issue, in particular, is not primarily about what outcome is normatively correct, but how one should think about cases of this kind. 

property

\begin{quote}Property owners are rights-holders first and foremost; obligations are, with some few exceptions, assigned to non-owners. Social obligation theorists do not reverse this equation so much as they balance it. Of course property owners are rights-holders, but they are also
duty-holders,
and often more than minimally so.\footcite[1023]{alexander11}
\end{quote}

role in social lif


 continues to 

One of the earliest proponents of the social function theory is Denuit.... 

Problem of being indeterminate and abstract, a beautiful philosophical and political formula that offers little guidance about how to address concrete legal problems and solve disputes. The content highly contested, particularly regarding duties it imposed on owners.\footcite[908]{robilant13}

\begin{quote}
The Fascist property theorists had been more specific about the content of the social function of property. For them social function meant the productive needs of the Fascist nation.\footcite[909]{robilant13}
\end{quote}

\begin{quote}
In a Europe threatened by totalitarian
rule, this resource-specific approach helped liberal jurists achieve two
important goals. First, it emphasized the value of pluralism in property law. In
times where property debates were becoming increasingly
focused on the productive efficiency of the Fascist nation, the
theorists of the tree concept of property believed in the
value of pluralism. In their discussion of the different branches of the property tree,
they focused on individual owners' privacy and freedom of action, equality in access to productive resources, and cooperative management of resources. Second, the focus on resources allowed our jurists to deal with
the fundamental problem of the value of pluralism in property law. The plural values and interests property law should promote are often in conflict with each other, and lawmakers will be called upon to make
difficult choices. In Fascist times, liberal property law scholars worried about the arbitrariness of these choices that may potentially
lead to a virtual abrogation of individual property rights. By grounding values and interests in the context of specific resources, they sought to guide and constrain lawmakers' normative reasoning.\footcite[910-911]{robilant13}
\end{quote}

\begin{quote}
For the tree concept's liberal advocates, analysis of the concrete characteristics of resources,
and fidelity to the historical and present legal framework for specific resources,
was the way to reduce the arbitrary nature of normative reasoning in property law and to stem the Fascist regime's potential erosion of property rights in the name of a generic and
unspecified interest of the Fascist state.\footcite[911-912]{robilant13}
\end{quote}

\begin{quote}
The debate between the liberal theorists of the tree concept and Fascist property
scholars suggests that the challenge for progressives is to rethink and
thicken or expand the notion of autonomy rather than drop it.\footcite[928]{robilant13}
\end{quote}

\begin{quote}
The tree concept views property as a tree with a trunk -- representing
the core entitlement that distinguishes property from other rights -- and many branches -- representing
many resource-specific bundles of entitlements. The trunk of the tree is the owner's entitlement to control the use of a resource,
mindful of property's ``social function.'' For the theorists of the tree model, the social function of property evokes a plurality of values: equitable distribution of resources, participatory management of resources, and productive efficiency.\footnote{\cite[872]{robilant13}.}
\end{quote}

\begin{quote}
Fascist property scholars had also appropriated the social function formula. For the Fascists,
the social function of property meant the superior interest of the Fascist state.\footcite[908-909]{robilant13}
\end{quote}

\begin{quote}
The theorists of the tree concept realized that,
to provide a good alternative to Fascist property, protecting the owner's sphere of
autonomous control was not enough. A modern liberal concept of
property is one that acknowledges and
foregrounds the social dimension of property. The rise of Fascism, they realized, was the
consequence of the crisis of liberalism. It was the consequence of liberals' insensibility to new ideas about the proper balance between individual rights and the interest of the collectivity.\footcite[907]{robilant13}
\end{quote}

It is important to keep in mind that the social function of property is descriptive in the first instance?

In dealing with economic development takings, I argue, this is not so. Without looking to the inter-personal and social purposes of property, it is hard to even recognize why these takings are a distinct category.

This is generally true, but becomes particularly clear when we work under the assumption that the development potential of property belongs to the State. It seems, in particular, that the only way to resolve the paradox identified above is to shift attention from the relationship between the owner and the property, to the property's function in regulating the relationship between the owner and the developer. It seems, in particular, that quite apart from any right to undertake development, which may well be a prerogative of the State, the property serves a crucial social function in giving the owner a platform from which he can engage meaningfully with other legal persons, including commercial companies interested in pursuing commercial development on his land. The fact that owners can normally bargain for the development surplus, take part in development themselves, or deny it altogether, need not be look at as a right to development, but as a right to participation in decision making and a minimum of autonomy in dealings with other interested parties. This, undoubtedly, is a {\it function} of property in a social context. My theoretical contention, therefore, is that this is also an aspect of the {\it right} to property, which should be regarded as protected against interference. 

Behind this idea lies a more general perspective which sees property itself as embedded in a social context, thereby committing us to a form of theorizing about property that goes beyond the classicaly dominantl theories relied on in constitutional property law. But there are well developed theories of property that will aid us in this, as they incorporate a social view on the function and purpose of ownership. Such a view has old roots in Europe, and is currently gaining ground also in contemporary US scholarship. 

The first modern expression of the social function theory of property is often attribued to Le{\'o}n Deguit, a French jurist. In a series of lectures he gave in Buenos Aires in 1911, Deguit challenged the classic liberal idea of property rights by pointing to the context-dependence of such rights. Denuit argued that depending on the social circumstances of the owner, property ownership entails obligations towards the owner's community just as much as it entails entitlements and dominion. This, he argued, was not only the reality of property ownership in practical life, it was also a normatively sound arrangement, conducive to justice, and more so than the traditionally liberal idea.

Similar thoughts have been influential in Europe, particularly in the post-WW2 rebuildment period. For instance, the constitution of Germany -- her {\it Basic Law} -- contains a property clause that explicitly includes a provision stating that property entials obligations as well as rights. As argued by Professor Gregory Alexander, this has had a significant effect on German property jurisprudence, creating a clear and interesting contrast with US law. The social function of property was also stressed in relation to the property clause in the First Protocol of the European Convention of Human Rights,.... Later, however, the liberal conception of property has been gaining ground in Europe, causing jurisprudential developments that have been particularly clear in the case law from the European Court of Human Rights (ECtHR).

A common assumption is that a social function view of property invariably leads to less constitutional property protection and greater State control over property. Indeed, this is often seen as the aim of conceptual reconfiguration; the social function view of property tends to be associated with a social democratic, or even socialist, political project, by which the notion of property is recast to justify greater interference in established rights. It is important to note, however, that while social democratic policies may be easier to justify by emphasising the social function of property, the mere recognition that property has an important social dimension does not in itself offer any justification for policies of this kind. For one, policy reasons must be tied to the prevailing social and economic circumstances, they will not automatically succeed merely by virtue of a conceptual shift. In addition, it seems that the most crucial premises used in arguments for greater State control and State-led redistribution projects concern the nature of the State, not the nature of property.

In particular, why should we believe that the State is the ultimate social institution to which property {\it should} answer? Is it not, for instance, equally possible to contend that property should continue to answer to those social structures that it is currently embedded in, by virtue of the owner's membership in a local community? Indeed, more than anything else it seems that embracing State control entails commitment to the the idea that more low-level social structures fail to function properly. This might be possible to argue sometimes, for instance if it can be shown that property owners insulate themselves from, and engage in exploitative practices towards, other communities. But such claims are by nature political and contextual, far removed from the theoretical discussion about the nature of property itself. Indeed, from the recognition that property structures are social in nature it does not follow that {\it any} institution should actively seek to change these structures. The Humean position, that the existing distribution of property rights represent a socially emergent equilibrium, remains plausible. Moreover, the further normative stance that this equilibrium is also {\it good} remains as contentious -- and as arguable -- as ever.

Put shortly, I argue that property's social function is nothing more or less than a descriptive truth. It allows us to rebut simplistic arguments that interference with property is contrary to any ``natural law of man'' or the like. But to go further, to present a positive normative argument that property should be tightly controlled or actively interfered with by some institution or other, requires a whole different line of argument. The importance of the conceptual shift, however, is that arguments regarding property are no longer constrained by conceptual individualism. The premise can no longer be that an owner is merely invested in her property because it gives her access to various streams of benefit which she can then use to maximize her utility against some given preferences. 

\section{Social obligation and human flourishing}

Taking the social function theory seriously forces us to recognize that a person's relation to property can partly be constitutive of that person's social and personal identity, including both its political and economic components. Hence, property influences what the owner's preferences are as well as what paths lie open to her when she considers her life choices. This effect is not limited to the owner, however, it comes into play for anyone who is socially connected to property in some way. The life-significance of property might be clearly felt by a potentially large group of non-owners as well. Its importance abates it we move away from the owner and the property in terms of social or economic distance, but as sociologists take pride in pointing out, social connections are ubiquitous  and the world is often smaller than it seems. 

Hence, there is certainly potential for making wide-reaching normative claims on the basis of this perspective on the meaning and content of property. But which such claims {\it should} we be making? According to some, we should adjust our moral compass by looking to the overriding norm of {\it human flourishing} as the fundamental guiding principle of property law. Professor Crawford explicitly argues that the social function theory of property should ``secure the goal of human flourishing for all citizens within any state''.\footcite[10089]{crawford12} In a recent article, Professor Alexander goes even further, declaring that human flourishing is the ``moral foundation of private property''.\footcite[1261]{alexander14} 

As I have already explained, I believe -- in contrast to both Crawford and Alexander -- that it is useful to decouple such normative claims from the descriptive core of the social function theory. I therefore prefer to speak also of human flourishing theories, to refer to the distinctly normative aspects of these authors' work.\footnote{Crawford comments that the social function theory on its own  ``is not self-defining and invites many interpretations''. The normative theory he proposes is clearly aimed to fill this perceived gap, by pinning down normative commitments that Crawford believes are intrinsic to the theory. However, as I have already argued, I reject this approach, since it unwisely assumes that the social function theory can not meaningfully serve as a common ground among commentators with widely divergent normative views. Indeed, Crawford himself refers unfavorably to a writer who addresses the social function theory, but who, according to Crawford, proposes that ``property's social function is best served by
focusing on overall economic production and efficiency in a given society, allowing the market's invisible hand to work its magic''. Against Crawford, I would argue that it is better to counter such a claim by arguing why it is normatively wrong than by suggesting that people with such values should be discouraged from attempting to argue for them on the basis of a social function understanding of property. Instead, by encouraging such an arguments it should become easier to make the case why the values promoted are ultimately undesirable. This, at least, should follow if Crawford is otherwise largely correct (as I think he might be).}
Human flourishing has a good ring to it, but what does it mean? According to Professor Alexander, several values are implicated, both public and private. Importantly, Alexander stresses that human flourishing is ``value-pluralistic''. There is not one core value that always guarantees a rewarding life. To flourish means to negotiate a range of different impulses, both internal and external. Importantly, these all act in a social context which influences their meaning and impact. 

Still, Alexander maintains that human flourishing provides an ``objective'' standard, and he ``rejects the view that
what is good or valuable for a person is determined entirely by that person's own evaluation of the matter''.\footcite[1263]{alexander14} Some things are good for people, Alexander argues, irrespectively of whether or not they know so themselves. Still, the exact content of goodness in this sense is ``necessarily contestable''. It consists of a list of different values which are all open to dispute, both as to their relevance and their precise meaning.\footcite[1263]{alexander14} Alexander goes on to list some key values that he believes are central, but the list is not meant to be exhaustive.

Among the key values that Alexander discusses, we find many core private values that are commonly seen as important goals for the institution of property. This includes values such as autonomy and self-determination, both of which will feature heavily later in this thesis. However, Alexander also considers several public values, such as equality, inclusiveness and community. These too will be important later, as I will draw on them in my own normative analysis of economic development takings. I will be particularly concerned with the value of {\it participation}, understood, following Alexander, in terms of its broad social function.

In my view, this value is closely related to the value of democracy. Participation in decision-making processes locally, I argue, is the root which enables democracy to come to fruition at the regional and national level. Moreover, participation is a value that will give me occasion to make particular policy suggestions regarding the correct way to approach the issues addressed later in the thesis. Devoting some time to discussing this value in the abstract will therefore be helpful.

Alexander traces the value of participation back to Aristotle and the republican tradition. He notes, however, that this tradition involves a notion of participation that is somewhat narrowly drawn. For thinkers in the republican tradition, participation tends to mean public participation, referring to public engagement with the formal affairs of the polity.\footcite[1275]{alexander14} For Alexander, participation means something more, involving also the value of being part of a community. He writes:

\begin{quote}
We can understand participation more broadly as an aspect of inclusion. In this sense participation means belonging or membership, in a robust respect. Whether or not one actively participates in the formal affairs of the polity, one nevertheless participates in the life of the community if one experiences a sense of belonging as a member of that community.\footcite[1275]{alexander14}
\end{quote}

Importantly, participation in a community can have a crucial influence also on people's preferences and desires. In this way, it will in fact be highly relevant -- behind the scenes -- to any assessment of property that focuses on welfare, utility or public participation in the classical sense. As Alexander puts it, drawing on the work of Sen and others,
\begin{quote}
The communities in which we find ourselves play crucial roles in the formation of our preferences, the extent of our expectations and the scope of our aspirations.\footcite[140]{alexander09}
\end{quote}

Therefore, for anyone adhering to welfarism, rational choice theory, utilitarianism or the like, neglecting the importance of community is not only normatively undesirable, it is also unjustified in an epistemic sense. In particular, it should be recognized as a descriptive fact that community is highly relevant to {\it any} normative theory that attempts to take into account the preferences and desires of individuals.\footnote{Again, I think Alexander and other theorists attempting to incorporate such ideas in property law could benefit from making this descriptive point separately, so as to enable it to be considered in isolation from the more contentious normative arguments they construct on the basis of it.}
But Alexander goes further, by arguing that participation in a community should also be seen as an independent, irreducibly social, value, not merely as a determinant of individual preferences and a precondition for rational choice. He writes,

\begin{quote}
Beyond nurturing the individual capabilities necessary for flourishing, communities of all varieties serve another, equally important function. Community is necessary to create and foster a certain sort of society, one that is characterized above all by just social relations within it. By "just social relations," we mean a society in which individuals can interact with each other in a manner consistent with norms of equality, dignity, respect, and justice as well as freedom and autonomy. Communities foster just relations with societies by shaping social norms, not simply individual interests.
\end{quote}

This, I think, is a crucial aspect of participation. Moreover, it is one that it is hard, if at all possible, to incorporate in theories that take  preferences and other attributes of individuals as the basis upon which to reason about property. For instance, if people in a community comes under pressure to sell their homes to a large commercial company that wishes to raze them in order to construct a shopping mall, it may be appropriate to consider this as an unjustifiable attack on their property rights. Importantly, this may be so {\it irrespectively} of what the individual owners themselves think they should do. If they are offered generous financial compensation for their home, while threatened by the specter of eminent domain, economic incentives might trump the value of social inclusion and participation for all or a majority of these owners. As a consequence, the community might decide to sell.  

Even so, in light of the value of community, it would be in order for planning authorities, maybe even the judiciary, to view such an  agreement as an {\it attack on their property}. It is clear, in particular, that by the sale of the land, the ``just social relations'' inhering in the community will be destroyed. The members of the community -- including all the non-owners -- will lose their ability to participate in those relations. More concretely, the nature of the property rights that once contributed to sustaining ``just relations'' will now be transformed into property rights that serve different purposes. This includes aiding the concentration of power and wealth in the hands of commercially powerful actors. Such a change in the social function of property might have to be regarded -- objectively speaking -- as a threat to participation, community and democracy. Hence, on the human flourishing theory, it is also a threat to property. Our property institutions, therefore, should protect against it.

To demonstrate the general significance of such a line of normative reasoning, it is illustrative to mention a scenario -- not directly implicating property -- that is currently beginning to attract much attention in legal scholarship. This scenario arises in relation to the right to {\it privacy}. This right, of course, is increasingly perceived to be coming under threat in the information age. Crucially, it is beginning to become clear to legal theorists that viewing privacy merely as a private right is not going to provide a sustainable template for dealing with this challenge. It seems, in particular, that people are simply too willing to give it up. This, in turn, contributes to the formation of potentially harmful social structures on the web. In particular, the lack of privacy becomes an impediment to dignity, freedom and respect in web societies. In this way, both individuals and society as a whole will eventually suffer, although this truth is not reflected in our individual preferences. Hence, it has been proposed that privacy should be considered also as a {\it common good}, so that protecting the privacy of individuals, in some cases, is an imperative that should be observed irrespectively of what these individuals themselves desire and prefer. Privacy, in this way, becomes also an obligation, mirroring the similar phenomenon that we have observed with respect to the right to property.

There is a subtle issue that arises on the basis of thins kind of normative reasoning about rights such as property and privacy. Is it appropriate, in particular, to still think of such reasoning -- and the obligations they give rise to -- as an aspect of protecting individuals? Is it not more accurate to say that this is an {\it interference} with individual rights, undertaken to further the public interest? Indeed, when the individual himself does not want his property or privacy to be ``protected'', is it not somewhat perverse to insists that this is what is happening? 

I am inclined to answer in the negative. In my opinion, we are still talking about protecting individual rights, even when this means imposing protections on people that they themselves do not want. Undoubtedly, this is {\it also} an interference in their rights, but just as different rights of different people can sometimes come into conflict, the same right, for the same person, can come into conflict with itself. This happens, in particular, when it is not possible to achieve all those goals that this right seeks to promote. If someone protests a taking on environmental grounds say, and rejects financial compensation as immoral, the courts should still award just compensation for his land, if they find that the taking is valid. If the owner wishes, he can purge himself by making a donation to charity. Similarly, if someone attempts to commit suicide, the health services are still obliged to help, even if suicide is no longer considered a criminal offense as per the public interest. 

Protecting individuals against their will is condescending, no doubt, but it is still different, and often preferable, from subordinating their interests to that of the general public. If the justification for an act of interference is a vague proclamation of the ``public interest'', the individual is marginalized from the very start. A balancing act might be required, but this renders the individual relevant only to one side of the equation. On the other hand, if the act of interference is simultaneously rendered as either as protection, enforcement of an obligation, or a measure to enable participation, the individual occupies center stage. In so far as the public interests triumphs, it is not because the individual loses, but because the public is deemed to know best how to secure the goal of human flourishing, both for the individual herself and other members of the social structures that surrounds her.

For instance, external interests of both a private and a public nature can dictate that owners should avoid becoming a nuisance to their neighbors. But under a human flourishing theory, we are also able to portray this as a case of protecting the individual's membership in the. The public does not ``side with the neighbors'', but undertakes measures to protect the relationship between the owner and his community. I my opinion, a conceptual approach to property law that can make this portrayal plausible is highly desirable. Similarly, environmental concerns can dictate restrictions on what an owner is permitted to do with his land. This too can be rendered as an act of protecting property. But doing so requires the regulatory body to relate the interference positively to the individual's interests and obligations, to ensure that they avoid adopting a narrative where the regulation is rendered an act of enforcing the will of unnamed others against the will of owners. In this way, public values and the public interest can be given considerable weight, but will have to be rendered less abstract. In particular, they must be related concretely to the social functions of the rights of the individuals interfered with. The baseline remains actual persons and implementation of the collective will becomes a guide to human flourishing, not a goal in itself.

An individual might well be offended if the state adopts this narrative and implements behavioral restrictions by declaring ``it's for your own good''. But, I argue, that is exactly as it should be. Any restriction of individual freedom is an offense, but one that is sometimes in order. If this is conveyed to people with a marginalizing ``your interests are not as important as ours'', the response might well be silence. But beneath the silence we may find disinterested apathy, or worse, contempt and despair. The interference is no longer an insult, not necessarily because it succeeds in convincing the individual that interference is ``necessary for the greater good'', but because it fails to properly engage the individual at all. The role of the person interfered with becomes passive -- she becomes an obstacle that should be removed, unless, of course, some old-fashioned concepts of ``liberal rights'' make this impossible. If such a dynamic of governance develops, the individual will learn that she is unimportant in the scheme of things, that her interests are subordinate, that her voice is limited. 

This is normatively undesirable. It represents a situation when the social effect of interference becomes detrimental to society, particularly to the institution of democracy. It damages its roots, namely the ``just social structures'' that Alexander identifies as being at the core of the human flourishing theory. A better alternative, then, is to interfere in a way that constructively targets the individual, aiming to protect her by enabling her -- and compelling her -- to protect others and partake in society. This can then become interference aimed at bringing the individual into the fold, making her play her part, raising her to fruitful citizenship. Such a paternal (or maternal) state is one that cares, but one that may also be overprotective, unfair, or plain stupid. Hence, it becomes natural to resist and to revolt, but not without also carrying forward care and love for the social and political structures within which this agency is (hopefully) permitted to take place.

The upshot, I believe, is that condescension is good in this area of the law. It may be offensive, but it renders interference more meaningful to the individual. It gives her both motivation and a possible cause to resist. Importantly, it does not force the conclusion that the public resides behind closed doors, disinterested in what the individual has to offer. Instead, it is an approach that encourages resistance, or at least response, by focusing always on the person interfered with, whenever interference is deemed necessary. This is a vision of a bottom-up, rather than a top-down, approach to imposing the collective will on individuals. I believe it has considerable merit.

While the human flourishing theory has merit as a framework for justification of interference in property, it also allows us to take a fresh look at the question of {\it legitimacy} of interference. In particular, in cases where the interference is potentially harmful to the values associated with human flourishing, it provides us with new template for assessment. This may even be relevant when adjudicating legitimacy cases, to the extent that property's social function warrants extending the range of values drawn on when interpreting and applying the law. In any event, we are provided with a fresh approach when addressing the normative question of what the law {\it should} be. In the next section I consider the implications of the human flourishing theory for economic development takings. This category, arguably, can not even be properly defined without drawing on values discussed by the human flourishing theory. In particular, the human flourishing theory allows us to argue that it can be circumscribed in a way that is legally relevant, both in regards to administrative law and constitutional property clauses.



the right to property also entails obligations, it is to be expected that such conflicts will occur: an owner has no right to adand


 when it is impossible to perfectly serve with internal tensions that require balancin


 I believe it is fair to say that the same right, for the same individual, can be conflicted. 







. The members of the community, therefore, will no longer be able to participate in 


. In so far as these are worthy of protection, t will completely change the character of the properties ther

a large share of the profits from the development

 to sell who wishes to raze the community, the 



circumscribe the issues 

se preferences are of course essential both to the 




partionsummed up by five further keywords: {\it equality}, {\it inclusiveness}, {\it community}, {\it participation} and {\it self-constitution}. 




clarifying also the ``true'' meaning of the social function theory. This, as I have already argued, is 





For the family of a property owner the significance of the property tends to be great; a home is a home for any non-owner living there, just as much as it is a home for the owner.  This, in turn, creates both commitments and opportunities for the owner, which may or may not find recognition in the law and our legal reasoning. If the property is rented out as a home to someone else, the importance of this ownership may be {\it greater} to a non-owner, in this case the tenant. Indeed, assuming a society where tenancy is a well-functioning social institution, the continuation of the established property pattern might well be of greater importance to him than it is to the owner.



Importantly, the effect on non-owners can also restrict them in socially important ways. If an apartment has an owner, it discourages squatters, for instance. Moreover, this effect clearly depends also on {\it who} the owner is and the choices he makes in managing his property. If the owner lives in the apartment, squatting is of course not going to be a problem. But even the owner of an unoccupied apartment can discourage squatting by managing his property well. However, if owners mismanage their apartments, for instance because they seek to get demolition licenses, squatters can take opportunity of this. The risk, of course, increases if housing cannot be afforded by a large number of society's members. In this case, it is natural to argue that something is amiss with the prevailing property structures. 

Now the social function theory of property can also come into play, since it allows us to attach significance to this also when discussing the property rights of individual owners.  In particular, we are not compelled to pretend as though possible failures of property as a social institution are irrelevant when considering the rights and responsibilities of individual owners. As a matter of fact, they are not; actual squatters clearly affect the owner, influencing both the meaning and the value of his property, both to him, potential buyers, the local government, the state, and other interested parties. Even the mere {\it risk} of squatting can play such a role. But a property theory which does not recognize the social function of property might not allow us to recognize this {\it fact} when discussing the rights and responsibilities of the individual. As long as the standard expectation of an owner is to be able to enjoy his apartment free of squatters, an entitlements-based view on property would force us to into denial regarding any actual (risk of) squatters.

The upshot is that squatting must be regarded as an interference with the owner's rights which the State can not, on pain of disrespecting property, recognize as a legitimate response to mismanagement and imbalances in the property structure. The fact of life, that squatting often happens due to badly managed property, is overlooked because our conceptual glasses block it out. Then, the almost unavoidable consequence is that the State also recognizes a {\it positive} obligation to forbid squatting, and to forcibly remove squatters on behalf of owner's. Under the classical entitlements-based conception, this is the natural outcome, and must be classified as an act of protecting private property. Here, however, the social function theory permits us to take a highly divergent view. 

In particular, if squatting is seen to create interests and obligations attaching to the property, it may now be argued that the use of state powers to forbid it and to forcibly evict people is an act of interference. Indeed, not only interference in whatever housing rights the squatters may have, but in fact also as an interference in {\it property}. Hence, such State action might itself be held in need of justification. In South Africa, which has a constitutional property clause that recognizes property's social dimension, a somewhat similar line of reasoning has been applied by the Supreme Court of Appeal and the Constitutional Court. The case of .... concerned squatting on a massive scale: some 400 people had taken up residence on land owned by Modderklip Farm, apparently under the belief that it belonged to the city of Johannesburg. The owner attempted to have them evicted and obtained an eviction 
order, but the local authorities refused to implement it. Eventually, the settlement grew to 40 000 people and Modderklip Farm complained that its constitutional property rights had not been respected.

The Supreme Court of Appeal concluded that Modderklip's property rights had indeed been violated, but noted that so had the rights of the squatters, since the State had failed to provide them with adequate housing. However, they upheld the eviction order and granted Modderklip Farm compensation for the State's failure to implement it. The Constitutional Court, on the other hand, while agreeing that the eviction order was valid, concluded that as long as the State failed in its obligations towards the squatters, the order should not be implemented. The eviction of the squatters, in particular, was made contingent upon an adequate plan for relocation. In the meantime, Modderklip would receive monetary compensation from the State.

In this way, the Court recognized the social function of property; they refused to give full effect to Modderklip's property rights as long as that meant putting other rights in jeopardy. The fact that the squatters had no place to go, in particular, was allowed to influence the content of Modderklip's right, making it impermissible to implement a standing eviction order. 

Of course, it is possible to cast this as an interference in property rights that is nevertheless acceptable in the public interest. However, the reconceptualization in terms of property itself having a social function appears highly attractive. As argued by Professors Gregory Alexander and ...., it allows us to remove the State as a intermediary between the owner and the other interested parties, in this case the squatters. Instead of thinking about the State as intervening on behalf of squatters by interfering with Modderklip's rights, it becomes possible to think of the Court as adjudging based on Modderklip's own responsibility, as an owner, towards other members of the community that have an interest in his property. 

On this basis, it becomes easier to conclude that it was permissible for Courts to take the social context into account even in the absence of any State action or legislation to indicate that this should be done, or that the public interest was at stake. Indeed, one of the problems in Modderklip was that the State had failed also in its responsibility towards the squatters. Moreover, while the local sheriff had refused to implement it, an eviction order had in fact been granted. Hence, thinking of the case as one where interference in the public interest was sanctioned by the Courts becomes strenuous at best.

More importantly, by taking into account the social function of property, it becomes possible to argue for the outcome in Modderklip positively on the basis of property law, as being a correct way of enforcing the relevant property rights and obligations. In this way, property is no longer seen to stand in the way of justice, requiring us to ``interfere'' with rights to secure an appropriate outcome. As Professor Alexander puts it: 
\begin{quote} The values that are
part of property's public dimension in many instances are necessary
to support, facilitate, and enable property's private ends.
Hence, any account of public and private values that depicts them as categorically
separate is grossly misleading. One important consequence of this
insight is that many legal disputes that appear to pose a conflict between
the private and public spheres or that seemingly
require the involvement of public law can and
should, in fact, be resolved on the basis of private law -- the law
of property alone.\footcite[1295-1296]{alexander14} \end{quote}
Protection of property, when property is understood in this way, becomes a potential source of justice, also for squatters. This, in turn, will potentially strengthen the institution of property itself, while also decreasing the compulsiveness of the idea that the ultimate expression of the public interest is found in the actions taken by the State. The public interest, rather, manifests wherever the public will may reside, including in property.

It is important to stress once again that the social function theory, as I am using it, is intended to be descriptive. Hence, it does not in itself dictate any particular stance on cases such as Modderklip. It dictates only a way of looking at them, allowing us to debate whether we should conclude that Modderklip, {\it qua owner}, has an obligation to take into account the squatters' need for housing and their expectation of not being evicted from their homes. With a classical liberal understanding of property, focusing merely on entitlement, such a debate becomes impossible, at least in the absence of any explicit basis in law for concluding that the owner has such an obligation. Hence, the classical discourse forces us to leave property behind altogether if we want to argue in favor of the squatters. In this way, the discourse easily becomes one where those arguing for social justice are led to take a negative stance on the institution of property itself, while those arguing in favor of this institution are unfairly pegged as advocating on behalf of privileged elites.

Arguably, however, cases such as {\it Modderklip} might be taken to suggest that the social function theory, as soon as it is applied for the purposes of normative assessment, will systematically guide us to conclude that owners are not entitled to as many benefits as would otherwise follow from their property rights. It is fortunate, therefore, that the entire remainder of the thesis will focus on economic development takings, where it will typically appear more natural to conclude the opposite. In these cases, on a common- sense understanding of justice, applying the social function theory will allow us to recognize a sense in which owners should receive {\it increased} protection and more benefits, as a consequence of how such interferences typically prove particularly damaging, both to the owner and to the social fabric of democracy. 

\section{The social effect of economic development takings}

Many constitutional property clauses include a more or less clear indication that eminent domain can only be used to take property if it is for ``public use', in the ``public interest'', or for a ``public purpose''. This is so also in the US, where the Fifth Amendment reads 
...``nor can private property be taken for public use unless just compensation is paid''.  Linguistically, this is not formulated as a restriction on when the state may take property, but merely a rule indicating that when they do so for a public use, they have to pay compensation. Nevertheless, the formulation can be read as presupposing that the state only takes private property for public use. Of course, in some sense this is not true, for instance, when they enforce agreements between private parties, inheritance rules, or allocate property following divorce. However, it seems quite unlikely -- if we assume that there is no presupposition of public use expressed in the formulation -- that the founders should intend the state to be able to take property at will for non-public uses, without paying compensation. Rather, it seems more likely that the assumption is that in cases when the state exercises the power of eminent domain, as distinguished from private law cases involving ``taking'' of property on some other basis, the founders assumed that the property would be designated for some public use.

In any event, the US state courts soon interpreted similarly phrased property clauses in state constitutions as involving a restriction on permissible purposes of eminent domain. During the 19th Century, much attention was devoted to the question of when takings served a legitimate purpose, and by the early 20th Century there was no longer any doubt that also the Fifth Amendment served to restrict the purposes of takings. However, the question of when a taking satisfies the public use requirement remained controversial. Eventually, a deferential doctrine developed whereby courts, particularly the Supreme Court, would typically respect the judgment of the legislature regarding what counts as a public purpose. However, cases raising this issue would still come up in the Supreme Court. The most recent such case, attracting much attention from the public, was the case of Kelo.

The case involved the drug company Pfizer....


The decision in Kelo caused a public outcry never before seen in a US takings case. Some commentators argue that it might in fact be the most controversial and disliked decision that the Court has ever made. Why did the decision prove so controversial? In my opinion, the answer can be found by looking to the social function theory of property. This theory, in particular, lets us point to several factors that serve as plausible reasons for why the decision attracted so much negative attention and was so badly received across the political spectrum. It seems, therefore, that the case also illustrates the need to consider the social functions of property more attentively in cases like Kelo. Indeed, I argue that the minority, and to some extent the minority, particularly the separate concurring opinion of Judge Kennedy (which secured the majority), can be read as indicating that such a change in perspective is also at least partly supported by a majority of the Supreme Court. 

The first thing to note is that many commentators conceptualized the Kelo case by consistently talking about it as an example of an {\it economic development taking}, a {\it taking for profit}, or, more bluntly, a case of {\it Robin Hood in reverse}. These categories find no clear basis in the relevant property rules, as these were typically understood before Kelo. In terms of established legal doctrine, the case revolved around the notion of ``public use''. However, it was already long established in US law that the public use restriction did not preclude takings in the ``public interest'', where the public benefits from the taking only indirectly, and the property rights are as a matter of fact transferred to a private party. It was understood, furthermore, that the courts would defer to the legislature's assessment of public use, and that only the ``naked private-to-private'' transfers and takings that were ``manifestly without reasonable foundation'' would be prohibited. 

However, in terms of the social function theory, it is easy to see why the case was natural to classify along an additional dimension, not concerned directly with the public interest, or lack thereof. Indeed, the feeling of unfairness in Kelo has as much to do with a combination of substantive and procedural shortcomings leading up to the decision to condemn, such as the balance of power between taker and owner, the incommensurable nature of the opposing interest, the lack of due regard for the owner by the local decision making authorities, the close relationship between taker and government, and the feeling that the public benefit, while perhaps sufficient, was to a significant extent {\it also} used as a pretext to confer a commercial benefit on the taker. The commercial interest of the taker, in particular, imbued the case with a character that made it suspicious. In terms of the human flourishing theory, it made it appear as a potentially harmful interference in the ``just social structures'' surrounding the property and its owner. These structures, it seemed, were simply not that just, and would be even less so if the taking went ahead. 

Importantly, neither of the individual concerns seemed strong enough to launch an attack under specific rules that target them, for instance by making procedural complaints or pretext claims. Instead, it was the overall character of the taking that came into focus, with the perceived lack of a clearly identifiable and direct public benefit being only one aspect that raised doubts about its legitimacy. 
Justice O'Connor, who voted with the minority, reasoned as follows:

\begin{quote}

\end{quote}

I note that the focus is on the institutional, social and political aspects of the case, not the economic implications for owners, or even the personal importance of homeownership. It is clear, in particular, that Justice O'Connor is more concerned with what she sees as a structural problem: economic development takings represent a form of governmental interference in property that is likely to systematically favor the rich and powerful to the detriment of the less resourceful. By placing crucial weight on this aspect, Justice O'Connor is in effect approaching the case on the basis of a social function understanding of property. Moreover, the values she looks to for guidance are closely related to the idea of human flourishing presented by Alexander and others. The danger of powerful groups gaining control of the power of eminent domain for their own ends is one that does not directly affect the individual entitlements of individual owners. In this regard, an economic development taking is just like any other: the immediate loss to the owner is the same as in any other case where government takes people's homes. 

But in economic development takings, the dynamic of the social interactions leading up to the decision to condemn are quite different. According to Justice O'Connor, the dynamic is also unjust, since it shifts the balance of power by giving a third party, with a strong commercial interest, considerable opportunity to wield the public interest, and the power of eminent domain, in its own interest. The story does not end there, however. Justice O'Connor does not only focus on the imbalance of the decision-making process, but also the imbalance of the outcome.  The effect of an economic development taking is almost always that property rights are taken from the many and given to the few, taken from ordinary people and given to the powerful. Hence, these cases represent a possibly pernicious redistribution of property, not necessarily in financial terms -- depending on the level of compensation -- but surely in terms of property's social function. The structural imbalances of the condemnation process find permanent expression in the new structure of property ownership and its new functions. Property owned by a community of homeowners is replaced by property owned by a commercial company. The just and inclusive social structures of a living community are dismantled in favor of a social structure that revolves around the commercial interest of a multi-billion dollar drug company. The political and social power of the community is diminished, probably lost in its entirety, while the political and social power of the drug company increases correspondingly.

 It seems clear that to Justice O'Connor, this too is a negative consequence of the taking. Again, we notice that recognizing this effect requires a social function approach to property. There is no clearly quantifiable individual loss -- no one particular ``stick'' in the property bundle that is removed. Rather, it is the community itself that is lost, a community that was not directly implicated in any ``right'', but which played a crucial role in providing meaning to the totality of the bundle enjoyed by the owner. Even if we extend our perspective to account for indirect individual losses, we are not doing justice to the loss in this regard. The owner might relocate, acquire new property with a similar meaning in a new community somewhere else. But that does not make up for the fact that {\it this} community is lost forever, as {\it this} property takes on new meanings and functions. The loss of Kelo, in particular, might eventually be a loss to the City of New London. But even so, as ownership comes with an obligation to sustain the community within which one resides, so does loss of community become an important reason why ownership should be protected. Taken to its logical extreme, this would be so even if such protection yields no personal gain for the owner whatsoever.

Of course, the economic and social gains of development might outweigh such negative effects on community. But, arguably, the balancing of interests required in this regard can only be carried out by an institution that sufficiently recognizes the owners' and their community's right to participation and self-governance. The presence of a commercial third party, in particular, means that public participation in the standard sense might be insufficient. The commercial party, in particular, appears in the process alongside the government, as an integrated part of the structure making the decision to condemn. But the owners, on the other hand, remain excluded, in the sense that their interest is only negative. They are adversely effected and may object, but under standard administrative regimes they play no constructive role in the process. They are not called on to participate, in particular, in the development itself. This, I argue, is in fact the main problem with economic development takings. I will argue for this in more depth later, but I remark here that an important reason to focus on this procedural aspect is that it involves precisely those values that economic development takings are most likely to offend against. This includes substantive values not related to the decision-making process. 

In particular, if the loss of community outweighs the positive effect of economic development, this is unlikely to be recognized by a process that relies only on the positive contribution of the developer and the expert planners. The objections made by owner, moreover, may not only be given too little weight given the imbalance of power between taker and developer. As long as they focus only on the individual loss, under a classical bundle of rights understanding of property, they may not even target those issues that are the most important, for property's social function. To theoretically proclaim that these aspects should now be considered will not solve this problem. To solve the problem, institutional changes must be made, to give those functions a voice in the decision-making process. This, more than anything else, speaks in favor of greater involvement by the community of owners (including, quite possibly, even non-owners) in the decision-making process relating to development. Not only by asking them individually if they have objections, but by first empowering them and then compelling them to assume a constructive role in relation to the proposed development. They should engage directly with both government and potential developers, consider alternative schemes, and be encouraged to make their own proposals. In short, they should {\it participate}, as a community. According to the human flourishing theory, as I understand it, this is not only a right, but also an obligation. It gives a plausible basis on which to strike down economic development takings, and to do so without giving up the value of judicial deference. In addition, it is a call for institutional reform, to search for new governance frameworks that will empower owners and their communities and thereby enable genuine participation. 

\begin{quote}
There is room to allow for the virtue of social responsibility and solidarity and for the ideal of avoiding any structural privileges that favor the better-off. Those who endorse these values should seek to
incorporate them -- alongside and in perpetual tension with the value of
individual liberty -- into our conception of private property and into the legal norms governing
public actions that necessitate some injuries to individual landowner.\footnote{\cite[802]{dagan99} (citations omitted).}
\end{quote}

\section{Conclusion}


done where property dictated a different outcome than normal, because 

Instead of property standing in opposition to a just outcome for the squatters, property can be recast as an argument in favor of such an outcome.

\noo{ A similar line of reasoning, going even further in protecting squatters, was adopted by the Supreme Court of }




g that an otherwise appropriate eviction order should not be implemented due to other parties interest in the land. 


Consitutional Court of South Africa expliclty


her case was therefore one where two fundamental rights tNow, importantly, the Constitutional Court 



that no other course of. Further to this, one is likely forced to conclude that the State must be called on to force out the squatters, as a matter of ``natural'' or otherwise  that nor can our laws. The use of force, then, becomes the only possible outcome: the State must be mobilized to throw the squatters out. 





learly, it is not. The fact that the apartments are occupied is certainly of interest to the owner by others will cert

in such a situation is clearly 


when judging 

 Indeed, we are entitled to think that the {\it content} of 

 latter might indeed be more likely than if he is a private individual.


 {\it quality} 

If a large number of council flats are unoccupied 




 , since  entertain expectations in the property, directly linked to how they live their lives. For instance, they might know they can come to stay over whenever th



 Property empowers people,  but also makes them responsible; to interfere with property is to interfere with the social structure in which it is embedded.

a collection of preferences, seeking to maximize the utility of property use. Similarly, the public is no longer measurable along a metric of welfarism, seeking 


Instead of arguing about property based on utilitarian or welfare-maximizing 






 It is not correct to assume, in particular, that an emphasis on property's social dimension necessarily leads to less protection for property owners and increased State control over property. 

To justify such a policy, it seems that the way one conceives of the State may be far more important than how one coneives of property. Why, in particular, would the effect of State control enhance the 

 in which currently existing property rights function. It is no 

it is still a matter of political choice, in need of further reasons and closely dependent on the circumstances 




 However, this is by no means a necessary consequence of the conceptual premise of property's social aspects. To arrive at the conclusion that protection against State action can be rolled back relies on the further assumption that the State itself poses no threat to the social functions of property. Moreover, to arrive at the even stronger conclusion that State control and regulation is in order, relies additionaly on the premise that State actions are not only no threat, but generally conducive to furthering and improving upon those funcitons. This, however, might be an overly naive assessment of the function of the State in modern market economies.

 


  and that Stcan be extended relies on the further premise that the State is {\it better} able than individuals and local communities 

Professor Hanoch Dagan and Professor Gregory Alexander are important pioneers of this perspective on property in the US, and they both argue for a social turn in the theory of property law, although they do so in slightly different ways.

According to Professor Dagan, the ....

Gray:

\begin{quote}
Property -- if it exists -- is intrinsically about the ranking of moral and social (rather than economic) priorities. If property were purely about the endorsement of rationally calculated commercial interests, we would still support the institutional of slavery and its implicit affirmation of the economic value of coerced human labour-power. What we {\it can} is that proprietary entitlement revolves around some specifically enforceable expectation of autonomous control over valued resources or opportunities. The categories of thing that are ring-fenced in this way are defied by collective perceptions of moral or social worth. These perceptions are never absolute in quantity, requiring instead a constant arbitration between the various goals that we wish to achieve. The individual's expectation of specific performance -- that is, of effectual decisional control over a resource or opportunity -- is generally, but not always, realised. The expectation is just that -- a relatively fortified hope or {\it spes} which is so commonly fulfilled by default that we rarely recognise its inbuilt limitations...
\end{quote}

\begin{quote}
For those who think they own land, the clear message is that money is the sole asset to which their claim of ``property'' can ultimately refer. Reality is always monetisable at the command of the state: money is fast becoming the measure of all value. Nowadays it often seems that the idea of property -- in the shape of an indefeasible entitlement of control -- is actualised only in the context of ideas themselves.
\end{quote}

\begin{quote}
The modern super-rich can easily harness powers of compulsory purchase with the connivance of weak, subservient, short-sighted or corruptible agencies of government. Those embraced within the euphemistic contemporary designation of ``high net worth'' can then do exactly as they please under some pretense of conferring public benefit or creating a ``trickle-down effect'', but remaining all the while beyond the reach of any genuine political control or scrutiny. This is a new species of predator of whom we should all be extremely vary.
\end{quote}

\begin{quote}
As is increasingly affirmed by contemporary property theorists, the stuff of modern property involves a consonance of entitlement, obligation and mutual respect -- in other words, a phenomenon characterised by accommodation and reciprocity rather than by predatory acts of self-interested taking.
\end{quote}

Underkuffler: ``common property'' (a conception in which collective power to alter privileges and powers is assumed to be part of the idea of property, itself'', p 59, p 63:  ``Recognition of an alternative conception of property will help us to remember that individual autonomy and social context are -- in fact, and unalterably -- deeply intertwined.``)  and ``operative property''. 

J.W. Harris:  ``trespassory rules'' and ``ownership spectrum''.

Dagan: ``When incorporating social responsibility into our understanding
of property, the challenge is to show that the concept of property can
encompass social responsibility without destabilizing the effects
of ownership in protecting individuals, particularly politically
weak individuals, from the power of government.'' p. 1262-1263 (soc-prop alex)

\begin{quote}``we must concede that it is far more likely to be sustained at the microlevel of our local communities, where our status as landowners also defines our membership''. Thus, a distinction should be drawn between imposing constraints on private property to benefit the community to which the property owner belongs and prescribing injurious regulations to benefit the public at large. The more the constraint resembles the former type of cases, the higher the threshold of social responsibility that should be implemented (thus legitimizing the imposition of constraints  or uncompensated harms as part of the meaning of ownership) and vice-versa.
\end{quote}p. 1266-1267 (soc-prop alex)

\begin{quote}
For all of these reasons, proponents of including a social-responsibility norm in the meaning
of ownership should support clear and simple rules rather than vague standards.
\end{quote}p. 1268-1269

\begin{quote}
A rule-based regime that draws careful distinctions within types of injured properties and types
of benefited groups is much more capable of successfully integrating
social responsibility into takings doctrine.
\end{quote}

\begin{quote}
Defining property as a set of legal rights or entitlements protected by legal and 
political institutions for the purpose of facilitating wealth acquisition and production
is far too limited to facilitate sustainable relationships between people and their
environments and among people.  The ‘bundle-of-rights’ concept treats both the
resources that are the object of private property rights and the rights-holders as
disconnected from the ecological and social environments in which both exist.
\end{quote}




The social aspect of property has recently gained recognition among legal scholars, not only as an

viz a vis other persons, is one that is deserving of protection.




 when it comes to issues such as regulation and 

parties, including the 

This observation bring us to the second sense in which economic development takings are a unique category. 


I feel that there is no further need to argue theoretically for the assumption that part of the surplus stemming from a beneficial, permitted use of property is attached to the property itself and hence belongs to the bundle of rights associated with property ownership, or, if we adopt the exclusion theory, that it falls under his dominion.




development is still attached to the property, the 


such systems, it is contentious whether or not the owner is entitled to compensation for a {\it loss} of development value when his value is taken. 


might still be of relevance what exactly

constitutes an interference with the rights of the owner that entitles him to protection, {\it qua} owner, under the relevant provisions of property and constitutional law. 

of economic development takings constitute 


, I believe both theories come out lacking because they focus on individual entitlements associated with property

 history of the ``bundle of rights'' theory shows that the debate between t


. Whether it is a point of view that weakens or 


This shows that 

The theory has now emerged as the dominant one


This conception, of course, also served to {\it justify}  rights 

Both 


On a superficial level, the property rights at play in this situation . Under the idea that 

 places the property 

 use they make of their property can still have undesirable redistribution effect, 


\footnote{See \url{http://www.theguardian.com/world/2012/jul/10/donald-trump-100m-golf-course} (accessed 06 July 2014).}

The theoretical conception of property 