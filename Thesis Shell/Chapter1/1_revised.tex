\newcommand{\isr}[1]{{#1}}

\part{Towards a Theory of Economic Development Takings}

\chapter{Property, Protection and Privilege}\label{chap:2}

\noo{ \begin{quote}
It's nice to own land.\footnote{Donald Trump, as quoted in \cite{booth12}.}
\end{quote}

\begin{quote}
A human being needs only a small plot of ground on which to be happy, and even less to lie beneath.\footnote{Johan Wolfgang von Goethe, {\it The sorrows of young Werther and selected writings}.}
\end{quote}
}
\section{Introduction}\label{sec:2:1}

\noo{ This chapter presents a template for analysing economic development takings, based on legal theory.\footnote{I will not provide an extensive presentation of concepts or theoretical approaches developed in other fields, such as political science, sociology, economy, or psychology. However, all these fields engage in interesting ways with the notion of takings and property, and I will quote some sources from these fields as appropriate for my argument based on legal theory. See generally \cite{miceli11,nadler08,katz97,carruthers04}.} It will be argued that the category of economic development takings is relevant to legal reasoning about certain situations when private property is taken by the state. }

The category of economic development takings makes intuitive sense; it targets situations when property is literally taken for economic development. In most cases considered in this thesis, economic development is even the explicitly stated aim used to justify the exercise of eminent domain. However, as mentioned in the introduction, the legal relevance of the category cannot be taken for granted. Indeed, a superficial look at typical approaches to takings in the law would seem to indicate that the nature of the project benefiting from a taking is not usually a major issue when assessing the legitimacy of interference.\footnote{For instance, in Europe, the property jurisprudence at the European Court of Human Rights (ECtHR) deals almost exclusively with other aspects of legitimacy. The Court typically stresses that interference must be in the public interest, but then leaves this aspect of legitimacy behind after making clear that the member states enjoy a wide margin of appreciation in relation to the public interest requirement. See, e.g., \cite{james86,lindheim12} (however, the form and strength of the public interest is potentially relevant to the Court's fair balance assessment, as discussed in chapter \ref{chap:3}). Similarly, in the US in the 1980s, Merrill claimed that most observers thought of the public use clause in the Fifth Amendment of the US Constitution as nothing more than a ``dead letter'', see \cite[61]{merrill86}.} 

This chapter challenges that idea, by offering an argument as to why the purpose and context of a taking matters, not only as a question of public policy but also with respect to property as a fundamental right. From the point of view of US law, providing such an argument might not be strictly required, since economic development takings have already gained recognition as an important category of legal reasoning.\footnote{See generally \cite{cohen06,somin07,malloy08}.} However, a conceptual discussion of what exactly the category represents appears to be missing in the literature so far. In Europe, moreover, the category has so far not received much recognition as a legally relevant way to address the legitimacy of expropriation. Hence, in a comparative study, a conceptual investigation into the very idea of an economic development taking is a necessary first step.

This chapter argues that in order to make this step, we should broaden our theoretical outlook compared to traditional forms of legal reasoning about property. Moreover, it will be argued that a suitable conceptual reconfiguration is already implicit in recent strands of property theory, particularly those that focus on the {\it social function} of property.\footnote{See generally \cite{alexander09a,foster11,singer00,underkuffler03,alexander06,alexander10,dagan11}.} Indeed, the crux of the main argument presented in this chapter is that the social function view compels us to pay attention to the special dynamics of power that tend to manifest in cases when private property is taken by the state for economic development, especially in the context of commercial exploitation.

To make clear why such takings are special, this chapter abandons the traditional entitlements-based perspective on property in favour of a perspective that emphasises the ideal function of property as a guarantor of social justice and a building block of democracy and participatory decision-making, particularly at the local level. This allows us to shift attention away from the effect that a taking has on individuals one-by-one, towards the question of whether the purpose of the taking, and its broader societal effect, merits interfering with private property. When this question is recognised as falling within the sphere of property law, it also provides a potential basis for judicial review rooted in property protection.

This chapter will also argue that private property is important because it gives owners a right to take part in decision-making processes concerning economic development, a right that also typically gives owners a duty to participate, not only on their own behalf, but also on behalf of broader community interests. In my view, this highlights how property rights can empower local communities in their interactions with powerful commercial and central government interests. Clearly, the use of eminent domain can undermine this function of property, thereby threatening the democratic legitimacy of the decision-making process, by depriving local communities of a potentially robust source of participatory competence. Moreover, when property interests are transferred away from the local community on a permanent basis, this threatens to leave a lasting democratic deficit in the wake of economic development. Arguably, this is the key reason why we should recognise economic development takings as a separate conceptual category.

To motivate the theoretical work, I start by considering the Balmedie controversy, pertaining to Donald Trump's plans for a golf resort in Scotland. I use this concrete example to highlight tensions between property's different functions in the context of economic development, to motivate the theoretical arguments that follow.
\noo{
Then, in section \ref{sec:top}, I go on to discuss theories of property, to locate a suitable starting point for further analysis. I argue that neither of the two dominant property theories of the last century, the bundle theory and the dominion theory respectively, provide such a starting point. In section \ref{sec:socfunc}, I move on to consider the social function theory in more depth, to arrive at a more useful theoretical template. Moreover, I argue that the descriptive part of this theory can provide a valuable conceptual tool even if one does not agree with the normative assertions that are typically associated with it. In particular, I argue that normative considerations should be addressed separately from conceptual foundations.

I do so in section \ref{sec:hf}, by building on the human flourishing account of the purpose of property. I argue that the human flourishing theory provides us with a possible path towards answers to the normative questions that arise from the social function perspective. In section \ref{sec:edt}, I apply the theoretical framework developed in preceding sections to a preliminary investigation of economic development takings, to bring out the overarching question of legitimacy, which will occupy a central place in this thesis.}

\section{Donald Trump in Scotland}\label{sec:2:2}

On the 10th of July 2010, the property magnate Donald Trump opened his first golf-course in Scotland, proudly announcing that it would be the ``best golf-course in the world''.\footnote{See \cite{passow12}.} Impressed with the unspoilt and dramatic seaside landscape of Scotland's east coast, the New Yorker, who made his fortune as a real estate entrepreneur, had decided he wanted to develop a golf course in the village of Balmedie, close to Aberdeen.

To realise his plans, Trump purchased the Menie estate in 2006, with the intention of turning it into a large resort with a five-star hotel, 950 timeshare flats, and two 18-hole golf-courses.\footnote{See \cite{siddique08}.} The local authorities were divided on the issue of whether to grant planning permission, which was first denied by Aberdeenshire Council.\footnote{See, e.g., \cite{bbc07}.} One of the reasons for rejecting the plans was that the proposed site for the development had previously been declared to be of special scientific interest under conservation legislation.\footnote{See \cite{bbc07b}.} The frailty and richness of the sand dune ecosystem, it was argued, suggested that the land should be left unspoilt for future generations. Several members of the local population actively campaigned against the plans, with some also refusing to sell property that Trump wanted to include in his development project.\footnote{See \cite{scotsman10}.}

Trump was not deterred, and in the end he was able to convince Scottish ministers that he should be given the go-ahead on the prospect of boosting the economy by creating some 6000 new jobs.\footnote{See \cite{carrell08}. Trump's plans attracted significant public attention, and his interaction with Scottish decision-makers came under critical scrutiny by commentators, see, e.g., \cite{jenkins08}. For a more general assessment from the point of view of conservation interests in the UK, see \cite{koen13}.} Activists continued to fight the development, launching the ``Tripping up Trump'' campaign to back up local residents who refused to sell their properties.\footnote{See \cite{tripping15}.} One of these, the farmer and quarry worker Michael Forbes, expressed his opposition in particularly clear terms, declaring at one point that Trump could ``shove his money up his arse''.\footnote{See \cite{scotsman10}.} Trump, on his part, had described Forbes as a ``village idiot'' that lived in a ``slum''.\footnote{See \cite{bbc10}.} Moreover, he had suggested that Forbes was keeping his property in a state of disrepair on purpose, to coerce Trump to pay more for the land, to remove the blight.\footnote{See \cite{cnn07}.} Forbes was offended and he proudly declared that he would never consider selling, as the issue had become personal.\footnote{See \cite{ferguson12}.}

At the height of the tensions, Trump asked the local council to consider issuing compulsory purchase orders (CPOs) that would allow him to take property from Forbes and other recalcitrant locals against their will.\footnote{See \cite{macaskill09}. It would not have been the first time Donald Trump benefited from eminent domain. In the 1990s, he famously succeeded in convincing Atlantic City to allow him to take the home of Vera Coking, to facilitate further development of his casino facilities. But in this instance, Trump did not get his way. Indeed, the taking of Vera's home was eventually struck down by the New Jersey Superior Court, an influential result that was hailed as a milestone in the fight against ``eminent domain abuse'' in the US. See \cite[297-301]{jones00}. See also \cite{gillespie08}. For the decision itself, consult \cite{banin98}.} These plans met with widespread outrage. The media coverage was wide, mostly negative, and an award-winning documentary was made which painted Trump's activities in Balmedie in a highly negative light.\footnote{See \cite{baxter11}.} The controversy also found its way into UK property scholarship. Kevin Gray, in particular, a leading expert in property law, expressed his opposition by making clear that he thought the proposed taking would be an act of ``predation''.\footcite{gray11}

In fact, the case prompted Gray to formulate a number of key features that could be used to identify situations where compulsory purchase would be likely to represent an abuse of power. Gray noted, moreover, that Trump's proposed takings would fall in line with a general tendency in the UK towards using compulsory purchase to benefit private enterprise, even in the absence of a clear and direct benefit to the public. In light of this, it seemed realistic that CPOs might be used in Balmedie.\footnote{Moreover, a statutory authority is found in section 189 of the \cite{tcpsa97}, stating that local authorities have a general power to acquire land compulsorily in order to ``secure the carrying out of development, redevelopment or improvement''.} It would not be hard to argue that the public would benefit indirectly in terms of job-creation and increased tax revenues. Moreover, Scottish ministers had already gone far in expressing their support for the plans.

But then, in a surprise move, Trump announced he would not seek CPOs.\footnote{See \cite{scotsman11}.} Instead, he decided to pursue a different strategy, namely that of containment. He erected large fences, planted trees and created artificial sand dunes, all serving to prevent the properties he did not control from becoming a nuisance to his golfing guests. One local owner, Susan Monroe, was fenced in by a wall of sand some 8 meters high. ``I used to be able to see all the way to the other side of Aberdeen'', she said, ``but now I just look into that mound of sand''.\footnote{See \cite{booth12}.} She also lamented the lack of support from the Scottish government, expressing surprise that nothing could be done to stop Trump.

There was little left to do. As soon as the decision was made to build around them, the neighbouring property owners found themselves marginalised. Trump, on his part, was declared a valuable job-creator whose activities would boost the economy in the region. He even received an honorary doctorate at Robert Gordon University, a move that prompted the previous vice-chancellor, Dr David Kennedy, to hand his own honorific back in protest.\footnote{See \cite{bbc10b}.}

In the end, then, it was not by taking the land of others that Trump triumphed in Scotland. Rather, he succeeded by exercising ``despotic dominion'' over his own.\footnote{To quote Blackstone, see \cite[2]{blackstone79b}.} This proved highly effective. After he fenced them in, his neighbours were hard to see and hard to hear. The Balmedie controversy went quiet, the golfers came, Trump got his way. As he declared during the grand opening: ``Nothing will ever be built around this course because I own all the land around it. [...] It's nice to own land.''\footnote{See \cite{booth12}.}

\subsubsection*{\ldots}

The tale of Trump coming to Scotland serves to illustrate the kind of scenario that I will be looking at in this thesis. In addition, it puts my work into perspective. For a while, it looked like Balmedie was about to become a canonical case of an economic development taking. But in the end, it became an illustration of something more subtle, namely that what it means to protect property depends on value judgements regarding opposing property interests. In particular, while Trump achieved his ends in Scotland by relying on his own property rights, he did so by undermining the property rights of others, even if he did not formally condemn those rights.

This was made possible by an exercise of regulatory and financial power. Hence, we are reminded that the function of property as such is deeply shaped by social, political and economic structures. For the powerful owner, property can be used offensively to oppress weaker parties. For the marginalised, it might well be the last line of defence against oppression. Indeed, Donald Trump's ownership of the Menie estate has a vastly different meaning than does Michael Forbes' ownership of his small farm. To many observers, the former kind of ownership will represent some combination of power, privilege and profit, while the latter will be regarded as imbued with a mix of defiance, community and sustenance. Different values are inherent in these two forms of ownership, and when Trump came to Balmedie, they clashed in a way that required the legal order to prioritise between them.

In Trump's narrative, upholding the sanctity of property in Balmedie entails allowing him to protect his golf resort plans from what he regards as backwards locals who attempt to fight progress. If this is one's starting point, property protection might even come to involve the use of compulsory purchase of rights that are seen as a hindrance to the full enjoyment of property by a more resourceful owner. 

For Michael Forbes and the other local owners, protecting property has a completely different meaning. To them, it was paramount to protect the local community against what they saw as a disruptive and damaging plan, one that threatened to turn them and their properties into mere golfing props. Again, adequate protection might require an interference in property, to prevent Trump from using his land according to his own wishes, because this causes damage to his neighbours. 

In the case of Balmedie, we are forced to recognise that protection implies interference and vice versa. Moreover, we see how both sides of that equation can involve the interests, ambitions, fears and aspirations of private individuals. This shows the conceptual inadequacy of the idea that property protection is all about weighing private against public interests, to strike a balance between the state's power to do good and owners' right to do as they please. In reality, matters are often more subtle, involving a number of additional dimensions. Importantly, how we assess concrete situations where property is under threat depends crucially on what we perceive as the ``normal'' state of property, the alignment of rights and responsibilities that we deem  worthy of protection. Our stance in this regard clearly depends on our values. But values themselves are in turn influenced by the context of assessment within which they arise. An additional challenge is that our assessments are often influenced by our \emph{perception} of the relevant context, rather than by facts.

For example, property activists in the US tend to regard the value of autonomy as a fundamental aspect of property. But this must be understood in light of the idea that US society is founded on an egalitarian distribution of property, where ownership is meant to empower ordinary people by facilitating self-sufficiency and self-governance.\footnote{See, e.g., \cite[173]{ely07}; \cite{rose96}.} Hence, the autonomy inherent in property ownership is not thought of as being bestowed on the few, but on the many. Protecting autonomy of owners against state interference is not about protecting the privileges of the rich and powerful, but is embraced as a way to protect {\it against} abuse by the privileged classes.\footnote{This narrative is enthusiastically embraced by US activists who fight economic development takings, see, e.g., \cite{castle15}.} 

This, however, is only an {\it idea} of property protection. It might not correspond to the reality surrounding the rules that have been \isr{moulded} in its image. Indeed, it has been noted that despite the great pathos of the egalitarian property idea, egalitarianism has actually played a marginal role to the development of US property law.\footnote{\cite[361]{williams98} (``Why does the egalitarian strain of republicanism have such a substantial presence in American property rhetoric outside the law but so little influence within it?'')} More worryingly still, research indicates that land ownership in the US, which is hard to track due to the idiosyncrasies of the land registration system, is not actually all that egalitarian.\footcite[246-247]{jacobs98} In this way, we are confronted with the danger of a dissociation of values, reality and the law.

In Scotland, a similar story unfolds. Here, the traditional concern is that land rights are mainly held by the elites.\footnote{See generally \cite{wightman96,wightman13}.} As a result, Scottish property activists tend to focus on values such as equality and fairness, calling also on the state to regulate and implement measures to achieve more egalitarian control over the land. Indeed, reforms have been passed that sanction interference in established property rights on behalf of local communities.\footnote{See generally \cite{lovett11,hoffman13}.} At the same time, cases like Balmedie illustrate that the Scottish government, now with enhanced powers of land administration, may well choose to align themselves with the large landowners. Moreover, research indicates that recent reforms in Scottish planning law, which serve to enhance the power of the central government, have the effect of undermining local communities and their capacity for self-governance.\footnote{See generally \cite{pacione13,pacione14}.} Again, the danger of a disconnect between influential property narratives and reality is brought into focus.

On the other hand, it seems that \isr{grass roots} property activists in the US and Scotland may well be closer in spirit than they seem. Although their perception of the role of the state is very different, they appear to share many of the same concerns and aspirations. Arguably, differences arise mainly from the fact that they operate in different contexts and engage with different discourses of property. The challenge is to find categories of understanding that allow us to make sense of both their commonalities and their differences.

I think the example of Balmedie suggests a possible first step. It illustrates, in particular, the need for a framework that will allow us to recognise that opposing the use of compulsory purchase for economic development is perfectly consistent with supporting strict property regulation to prevent the establishment of golf resorts in fragile coastal communities. Both of these positions, moreover, should be viewed as efforts to protect property. To the classical debate about the limits of the state's authority over property, such a dual position can be hard to make sense of. But in my opinion, this only points to the vacuity of the conventional narrative.

In general, I think it is hard to make sense of many contemporary disputes over property if we do not have the conceptual tools to distinguish between (1) egalitarian property held under a stewardship obligation by members of a local community, and (2) neo-liberal property held by large enterprises for investment. Moreover, there is no contradiction between promoting the value of autonomy for one of these, while \isr{emphasising} the need for state control and redistribution when it comes to the other. The broader theme is the contextual nature of property and its implications for protection of property rights. In the coming sections, I will propose a theoretical basis that integrates this viewpoint into legal reasoning about interference in property rights.

\section{Theories of Property}\label{sec:2:3}

What is property? In common law jurisdictions, the standard answer is that property is a collection of individual rights, or more abstractly, a means of protecting {\it entitlements}.\footnote{The idea that property rules are a form of entitlement protection was developed to great effect in the seminal article \cite{calabresi72}.} Being an owner, it is often said, amounts to being entitled to one or more among a bundle of sticks, floating on streams of protected benefits associated with, and thereby serving to legally define, the property in question.\footnote{See \cite[357-358]{merrill01}. The ``classical'' references on the bundle of rights theory in the US and the UK respectively are \cite{hohfeld17,honore61}.} This point of view was first developed by legal realists in response to the natural law tradition, which \isr{conceptualised} property in terms of the owner's dominion over the owned thing, particularly his right to exclude others from accessing it.\footcite[193-195]{klein11} In civil law jurisdictions, rooted in Roman law, a dominion perspective is still often taken as the theoretical foundation of property, although it is of course \isr{recognised} that the owner's dominion is never absolute in practice.\footnote{For a comparison between civil and common law understanding of property, see generally \cite{chang12}.}

In modern society, the extent to which an owner may freely enjoy his property is highly sensitive to government's willingness to protect, as well as its desire to regulate. To dominion theorists, this sensitivity is typically thought of as giving rise to various restrictions on property, but for bundle theorists it is often thought of as {\it constitutive} of property itself.\footcite[7]{chang12} 

The bundle of rights theory has long historical roots in common law. Arguably, it was distilled from the traditional estates system for real property, which was turned into a theoretical foundation for thinking about property in the abstract.\footnote{See \cite[7]{chang12} (``The ``bundle of rights'' is in a sense the theory implicit in the common law system taken to its extreme, with its inherently analytical tendency, in contrast to the dogged holism of the civil law.'').} However, during the 18th and 19th century, natural law and dominion theorising was also influential in common law. This is evidenced, for instance, by the works of William Blackstone and James Kent.\footnote{See generally \cite{blackstone79b,kent27}.} Towards the end of the 19th century, it became increasingly hard to reconcile such an approach to property with the reality of increasing state regulation. Hence, the bundle metaphor that gained prominence in the early 1900s can be seen as a return to a more modest perspective.\footnote{See \cite[195]{klein11}.}

On the bundle account, property rights are thought to be directed primarily towards other people, not things.\footnote{See \cite[357-358]{merrill01} (``By and large, this view has become conventional wisdom among legal scholars: Property is a composite of legal relations that holds between persons and only secondarily or incidentally involves a ``thing''.'').} This underscores an important point about property in the real world, namely that the content of rights in property are necessarily relative to a social context as well as the totality of the legal order. For instance, when relying on a bundle metaphor it becomes easy to explain that a farmer's property rights protects him against trespassing tourists, but not against the \isr{neighbour} who has an established right of way.\footnote{It has been argued that this way of thinking about property, as a web of (legal and social) normative relations between persons, does not entail the bundle of sticks idea, see \cite[23-25]{dorfman10}. I agree, and I also believe that endorsing the property-as-relations perspective is largely appropriate, even if one does not otherwise agree with the bundle perspective. Historically, however, the two ideas have in fact been closely associated with one another, so presenting them together seems appropriate. Moreover, I will not actively enter into the theoretical debate on this point, since I believe that the {\it social function} account of property, discussed in more detail in section \ref{sec:2:4}, takes us further than both bundle and dominion perspectives. However, as will hopefully become clear, the social function theory itself may be seen as a continuation of the property-as-relations idea, catering also to a more holistic perspective on social structures (although it otherwise manages to remain largely neutral on the bundle v dominion issue).}

By contrast, the dominion theory suggests viewing such situations as exceptions to the general rule of ownership, which implies a right to exclusion at its core. In the case of property, exceptions no doubt make up the norm. But in civil law jurisdictions one lives happily with this. It takes the grandeur away from the dominion concept, but it retains a nice and simple structure to property law. In the civil law world, it is common to say that what the owner holds is the {\it remainder}, namely what is left after deducting all positive rights that restrict his dominion.\footcite[25]{chang12} Moreover, while there may be many limitations and additional benefits attached to property, they are all in principle carved out of one initial right, namely that of the owner. In this way, the civil law system can be more easy to navigate.

Some common law scholars have recently elaborated on this to develop a critique of the bundle theory, by suggesting that it should at least be complemented by a firm theory of {\it in rem} rights in property. This, they argue, would allow the law to operate more effectively, by relying on a simple and clear rule that, although defeasible, would generally suffice to inform people about their relevant rights and duties in relation to property.\footnote{\cite[793]{merrill01b} (``The unique advantage of in rem rights -- the strategy of exclusion -- is that they conserve on information costs relative to in personam rights in situations where the number of potential claimants to resources is large, and the resource in question can be defined at relatively low cost.''); \cite[389]{merrill01} (``The right to exclude allows the owner to control, plan, and invest, and permits this to happen with a minimum of information costs to others.''). See also \cite{ellickson11} (arguing that Merrill and Smith's analysis nicely complements and improves upon the bundle theory).} 

In addition, some scholars point out that the bundle theory does not adequately reflect the sense in which property is a right to a {\it thing}, serving to create an attachment that is not easily reducible to a set of interpersonal legal relationships.\footnote{\cite[1862]{merrill07}. For a slightly different take on attachment, highlighting how the `thingness' of property marks its conditional nature and transferability, see \cite[799-818]{penner96}.} In the US, where the bundle theory has traditionally been dominant, this critique seems to be gaining ground.\footnote{See generally \cite{klein11}.}

In this thesis, the efficiency and clarity of different property concepts will not be a primary concern, nor will personal attachments to things in themselves play a particularly important role.\footnote{I mention, however, that the \isr{personhood aspects} of property that are sometimes highlighted in this regard might be relevant to an analysis of economic development takings. The personhood account does not seem to depend on a rejection of the bundle theory, but might carry with it some implicit criticism of it, see, e.g., \cite[127-130]{radin93}.} Hence, I will remain largely agnostic about this aspect of the debate between dominion and bundle theorists. In particular, the differences between civil and common law traditions in this regard do not cause special problems for my analysis of economic development takings. For the purposes of this thesis, it is more important how different ways of looking at property can influence how we assess when interference is legitimate under constitutional and human rights law. Hence, I now turn to the question of whether or not there are any significant differences between dominion and bundle theories when it comes to the question of legitimacy of takings.

\subsection{Takings under Bundle and Dominion Accounts of Property}\label{sec:2:3:1}

Bundle theorists might be expected to have a comparatively relaxed attitude towards state interference in property rights. Indeed, thinking about property as sticks in a bundle may lead one to think that property rights are intrinsically limited, so that subsequent changes to their content, made by a competent body, are reflections of their nature, not a cause for complaint. In particular, the theory conveys the impression that property is highly malleable. 

For the theorists that developed the bundle of sticks metaphor in the late 19th and early 20th century, this aspect was undoubtedly very important. By providing a highly flexible concept of property, they helped the state gain conceptual authority to control and regulate.\footcite[195]{klein11} The early bundle theorists not only developed a theory to fit the law as they saw it, they also contributed to change.

In takings law, the bundle narrative has been particularly important in relation to the contentious issue of so-called regulatory takings.\footnote{See generally \cite{alterman10} (a comparative study of   regulatory takings in thirteen countries).} Such takings occur when government regulates the use of property so severely that it may be classified as a taking in relation to the law of eminent domain.\footnote{See \cite[1]{fischel95}. Regulatory takings proper arise when the (contested) right to compensation is inferred from a takings clause, such as in the US. However, the overarching question is how to deal with changes in property values caused by regulation, a question of universal importance across jurisdictions that may also be addressed by the legislator as a separate issue, see \cite[3-10]{alterman10}.} In the US, the question of when regulation amounts to a regulatory taking is highly controversial.\footnote{See generally \cite{fischel95}.} The stakes are high because takings have to be compensated in accordance with the Fifth Amendment of the US Constitution.\footnote{See \cite{fifth}.} At the same time, the law is unclear, and the limited amount of statutory law on the issue means that cases tend to be adjudicated against the Constitution, often the relevant state Constitution in the first instance.\footnote{See \cite[31-32]{alterman10}. Indeed, the federal doctrine on regulatory takings appears to be marked by deference to state courts. See \cite[66]{fischel95}.}

If property is thought of as a malleable bundle of entitlements that exists only because it is recognised by the law, it becomes natural to argue that when government regulates the use of property, it does not deprive anyone of property rights. It merely restructures the bundle. In the case of {\it Andrus v Allard}, the Supreme Court adopted such an argument when it declared that ``where an owner possesses a full ``bundle'' of property rights, the destruction of one ``strand'' of the bundle is not a taking, because the aggregate must be viewed in its entirety''.\footcite[65--66]{andrus79}

Hence, with regards to the issue of regulatory takings, the bundle theory was actively used by those who favour a less restrictive approach to interference with private property rights. However, it is wrong to conclude that the bundle theory {\it necessarily} implies a less restrictive stance on takings. Epstein, for instance, argues that as every stick in the property bundle represents a property right, government should not be permitted to remove any of them without paying compensation.\footcite[232-233]{epstein11} 

More generally, Epstein does not believe that the bundle theory is responsible for what he regards as a weakening of property rights in the US during the 20th century. Instead, he thinks this weakening resulted from a tendency among modern property scholars to adopt a ``top-down'' approach to property. According to Epstein, too many scholars view property rights as vested in, and arising from, the power of the state, not the possessions of individuals.\footnote{\cite[227-228]{epstein11} (``In my view, the nub of the difficulty with modern property law does not stem from the bundle-of-rights conception, but from the top-down view of property that treats all property as being granted by the state and therefore subject to whatever terms and conditions the state wishes to impose on its grantees.'').}

Epstein successfully shows that as a rhetorical device, the bundle of rights theory may be turned on its head compared to how it was used in {\it Andrus v Allard}. Moreover, his arguments illustrate that the bundle theory itself does not appear to dictate any particular position on the degree of protection that private property should enjoy against state interference.\footnote{To further underscore this point, it may be mentioned that while US courts \isr{recognise} that a regulation can amount to a taking, this is practically unheard of in several other common law jurisdictions, including England and Australia. This is despite the fact that these countries all paint property in a similar conceptual light. Moreover, while the issue of regulatory takings is considered central to constitutional property law in the US, it is considered a fairly marginal issue in England, see \cite{purdue10}.}

In the civil law world, the relationship between property theorising and property protection is similarly hard to pin down at the conceptual level. Again, the issue of regulatory takings illustrates this. In some civil law countries, like Germany and the Netherlands, the owner's right to compensation for burdensome land use regulation is strong, while in other civil law countries, such as France and Greece, it is very weak.\footnote{See generally \cite{alterman10}.} In particular, the exclusive dominion understanding of property does not appear to commit one to any particular kind of policy on this point. 

On the one hand, it cannot be denied that property rights are enforced, and limited, by the power of government. Hanging on to the idea of dominion, then, necessarily forces us to embrace also the idea that dominion is never absolute. In this way, the theory may serve as a conceptual basis for arguing in favour of a relaxed approach to state interference. If property rights are not absolute to start with, why worry about interfering in them for the common good? But, of course, this story too may be turned on its head. Indeed, a libertarian can use the image of limited dominion to argue that property is being ripped apart at its seams. If we want to maintain our grasp of what property is, such a person might argue, we better enhance the level of protection offered to property owners, to restore true dominion.

The upshot is that the differences between common law and civil law \isr{theorising} about property do not appear to be very relevant to the question of legitimacy in the context of state interference. In particular, the differences between the bundle theory and the dominion idea do not appear to speak decisively in \isr{favour} of any particular approach to economic development takings.

In terms of descriptive content, both theories appear oversimplified. They provide a manner of speech, but they do not really get us very far towards uncovering the reality of property rights in modern society. In particular, they do not provide a functional account of what role property plays in relation to the social, economic and political structures within which it resides.\footnote{A similar point is made in \cite[2-6]{alexander12}.}

In terms of normative content, on the other hand, both the bundle theory and the dominion theory appear rather bland. They simply do not offer much clear guidance as to what norms and values the institution of property is meant to promote. They give neat ways of presenting what property can look like, but do not tell us {\it why} it should be protected.\footnote{For a similar criticism, see \cite[535-536]{bell05} (proposing an instrumental theory of property, with both descriptive and normative implications, based on the idea that property exists to protect the value of stable ownership).}

\subsection{Broader Theories}\label{sec:2:3:2}

Based on the discussion so far, it seems that in order to make progress towards a theory of economic development takings we need to start from a property theory with a wider scope than both the bundle account and the dominion theory. There are many candidates that could be considered. In a recent monograph on property, Alexander and Pe\~{n}alver present five key theoretical branches:\footnote{See chapters 1 to 5 of \cite{alexander12}.}
\begin{itemize}
\item {\it Utilitarian} theories, focusing on property's role in helping to maximize utility or welfare with respect to individual preferences and desires. 
\item {\it Libertarian} theories, focusing on property's role in furthering individual autonomy and liberty, as well as the importance of protecting property against state interference, particularly attempts at redistribution. 
\item {\it Hegelian} theories, focusing on the importance of property to the development of personhood and \isr{self-realisation}, particularly the expression and embodiment of free will through control and attachment to one's possessions.
\item {\it Kantian} theories, focusing on how property arises to protect freedom and autonomy in a coordinated fashion so that {\it everyone} may potentially enjoy it, through the development of the state.
\item {\it  Human flourishing} theories, focusing on property's role in facilitating participation in a community, particularly as a template allowing the individual to develop as a moral agent in a world of normative plurality.
\end{itemize}

It it beyond the scope of this thesis to give a detailed presentation and assessment of all these theoretical branches. Suffice it to say that the utilitarian approach has been by far the most influential.\footnote{See \cite[11]{alexander12} (noting that there are many varieties of utilitarianism, including some law-and-economics theories for which the appropriateness of that label is contentious).} The basic tenet of this paradigm is that means-end analysis on the basis of exogenous preferences and utility measures provide a sound foundation on which to reason about law and policy.

In this thesis, I will depart from this form of analysis, by regarding property instead as an integral part of social structures. On this view, property can no longer be seen neither as an end in itself nor as a means to maximise some utility measure. Instead, property is understood in light of how it functionally relates to other building blocks of life, such as sustenance, economic activity, social interaction, interpersonal responsibility, preference change, deliberation, and democratic decision-making.

With such a starting point, I believe the human flourishing theory has more to offer than any of the other theoretical branches mentioned above. In section \ref{sec:2:5} below, I will emphasise how this theory suggests a range of new policy recommendations regarding how the law {\it should} approach the question of economic development takings.

Before I get to this, I will explore descriptive aspects of property theory in some more depth. Indeed, a potential objection against all the theories summarised above is that they are overly normative; they are largely used to argue for particular values associated with property, not to clarify the descriptive core of the notion. This is a challenge, since one of my main aims in this thesis is to argue for a descriptive proposition, namely that economic development takings make sense as a conceptual category for legal reasoning. Hence, before I move on to consider normative aspects, I first need a theoretical framework that allows me to pinpoint what makes economic development takings unique. I would like to do so, moreover, without thereby committing myself to any particular stance on how to normatively assess such takings.

To arrive at a suitable foundation in this regard, I will rely on the so-called {\it social function theory} of property.\footnote{See generally \cite{foster11,mirow10,alexander09a}. Be aware that some authors, particularly in the US, also speak of the {\it social obligation} theory, using it more or less as a synonym for the social function theory.} This theory is often thought of as a normative theory as well, in some sense a precursor to more overtly normative theories such as the human flourishing theory. However, I will argue that the social function theory has a descriptive core that can serve as a common ground for debate among scholars that do not necessarily share the same normative outlook. Crucially, the descriptive core of the social function theory also points towards a descriptive argument in favour of studying economic development takings.

\section{The Social Function of Property}\label{sec:2:4}

As an empirical observation, the fact that property has social functions is beyond doubt. For instance, it is clear that ownership of property gives rise to social obligations, not just rights. Hardly anyone would protest that in practical life, what an owner will do with their property is as much constrained by the expectations of others as it is by law. Moreover, the law of nuisance and rules relating to adverse possession both serve as simple examples that such expectations can also have a bearing on the legal status of property and its owners.\footnote{See \cite[314]{waldron85} (invoking social obligations as well as the notion of nuisance to explain what ownership is, at the conceptual level); \cite[197-198]{gerhard13} (proposing an abstract distinction between trespass and nuisance in US law based on the concept of duty, which, it is argued, is always symmetric in nuisance cases but not in case of trespass); \cite[1169-1172]{penalver07} (analysing adverse possession on the basis of a social functions understanding of property). See also \cite{pye07} (the ECtHR deciding to regard a UK case of adverse possession in bad faith as legitimate under the ECHR).}

Still, many property scholars have surprisingly little regard for social functions when they theorise about ownership. According to Alexander, the classical theories of property convey the impression that ``property owners are rights-holders first and foremost; obligations are, with some few exceptions, assigned to non-owners''.\footcite[1023]{alexander11} Theorists who emphasise property's social function attempt to redress this conceptual imbalance. As Alexander explains, ``social obligation theorists do not reverse this equation so much as they balance it. Of course property owners are rights-holders, but they are also duty-holders, and often more than minimally so''.\footnote{\cite[1023]{alexander11}.}

I remark that what Alexander and others sometimes refer to as the social obligation theory of property is covered by the social function theory as I understand it. However, the social function theory is broader in that it asks us to consider the legal relevance of social dependencies rooted in property more generally, not just obligations. Specifically, the social function theory as it is used in this thesis should be understood as a cross-jurisdictional reference point for a set of abstract ideas about property that includes the ideas of theorists such as Alexander, who also argue specifically for a social obligation norm in US property law.\footnote{For a similar understanding of the social function theory, see \cite{foster11}.} As I discuss in the next subsection, the idea that property serves important social functions is not new. Moreover, it often plays an important implicit role in shaping how property is understood in the law, also in Europe.

\subsection{Historical Roots and European Influence}\label{sec:2:4:1}

The first expression of the social function theory has been attributed to Le{\'o}n Duguit, a French jurist active early in the 20th century.\footnote{See generally \cite{foster11}.} In a series of lectures he gave in Buenos Aires in 1911, Duguit challenged the classic liberal idea of property rights by pointing to their context dependence, adopting a line of argument strikingly similar to how recent scholars have criticized utilitarian discourses about property.\footnote{See \cite[1004-1008]{foster11}. For more details about Duguit's work and the contemporaries that inspired him, see generally \cite{mirow10}.} In particular, Duguit also pointed to the notion of obligation, stressing the fact that individual autonomy only makes sense in a social context where people are dependent on each other as members of  communities. Hence, depending on the social circumstances of the owner, their property could entail as many obligations as entitlements. This, according to Duguit, was not only the inescapable reality of property ownership, it was also a normatively sound arrangement that should inspire the law, more so than individualistic, `liberal', visions of property as entitlement protection.\footnote{See \cite[1005]{foster11} (``The idea of the social function of property is based on a description of social reality that recognizes solidarity as one of its primary foundations'', discussing Duguit's work). It should also be noted that Duguit was particularly concerned with owners' obligations to make productive use of their property, to benefit society as a whole. This raises the question of who exactly should be granted the power to determine what counts as ``productive use''. In this way, Duguit's work also serves to underscore one of the main challenges of regulatory frameworks that seek to incorporate and draw on property's social dimension: how should decisions be made in such regimes?} The social function perspective has since become widespread in Latin America, as reflected for instance in the property clause in Article 21 of the American Convention of Human Rights.\footnote{\cite{achr}. See \cite[61-62]{banning02}. See also \cite{cunha11,mirow11,bonilla11}.}

Related ideas have also been influential in Europe, particularly during the rebuilding period after the Second World War. For instance, the constitution of Germany -- her {\it Basic Law} -- contains a property clause stating explicitly that property entails obligations as well as rights.\footnote{See \cite[14]{basic49}.} As argued by Alexander, this has had a significant effect on German property jurisprudence, creating a clear and interesting contrast with US law.\footnote{See \cite[738]{alexander03} (``The German Constitutional Court has adopted an approach that is both purposive and contextual, while the U.S. Supreme Court has not.''). See also \cite[476-483]{walt11} (noting that the German property clause also imposes a strict public purpose requirement which has been used by the Constitutional Court to strike down illegitimate takings that benefit commercial enterprises).}

A social perspective on property was also influential during the debate among the European states that first drafted the property clause in Article 1 of the First Protocol to the European Convention of Human Rights (P1(1) of the ECHR).\footnote{See \cite[1063-1065]{allen10}. Allen argues that the liberal conception of property has since gained ground in Europe, as reflected in jurisprudential developments at the ECtHR.} The article was eventually formulated as follows:\indexonly{echr}

\begin{quote} Every natural or legal person is entitled to the peaceful enjoyment of his possessions. No one shall be deprived of his possessions except in the public interest and subject to the conditions provided for by law and by the general principles of international law.
\end{quote}
\begin{quote}
The preceding provisions shall not, however, in any way impair the right of a state to enforce such laws as it deems necessary to control the use of property in accordance with the general interest or to secure the payment of taxes or other contributions or penalties.
\end{quote}

I will return to this clause in more depth in section \ref{sec:3:4} of chapter \ref{chap:3}. Here I note how it emphasises both the private right to peaceful enjoyment of possessions and the state's right to interfere with property in the general/public interest. Moreover, it does not explicitly introduce an absolute compensation requirement in case of expropriation by the state, setting it apart from many other property clauses, including that contained in the Fifth Amendment of the US Constitution. Arguably, this reflects a recognition of the social aspects of property.\footnote{See generally \cite{allen10}. For completeness, I mention that a human right to property is also included in Article 17 of the \cite{udhr}. However, its inclusion was controversial and no corresponding right was included in either of the two subsequent covenants containing binding provisions, see \cite{fne,fnp}. In light of this, the property clause in the UDHR appears to have limited practical significance. Similarly, a property clause is included in Article 14 of the \cite{banjul}. However, this property clause has apparently not been given much attention from the African Court, see \cite[83]{walt11}. However, it has been proposed that an internationally binding right to property should be introduced through a new international Covenant, a proposal that appears to be firmly based on a social function understanding of property, see generally \cite{hassmann13} (including a draft formulation of a property clause that states, among other things, that the human right to property does not apply to corporations).}

However, it also fits within a traditional narrative of private property, where social responsibilities attaching to property are regarded as arising from state objectives and policies, not ownership as such. Indeed, the chosen formulation in P1(1) appears to suggest that social aspects are external to private property, vested in the regulatory power of the state.\indexonly{echr}

This marks a possible tension with the social function theory, which asks us to recognise that social obligations are inherent in private property, attaching to owners directly. The importance of this in the present context is that a social function perspective can occasionally suggest stricter limits on state interference, not out of greater concern for individual entitlements, but out of concern for property's proper role as a building block of social and political life.

Despite the conventional formulation used in P1(1), such a perspective does in fact appear to play a role at the ECtHR.\noo{At least, the case law from the Court shows that social and political considerations are not only invoked in the context of adjudicating tensions between private entitlements and the perceived necessity of state interference in the public interest.

The ECtHR emphasises {\it proportionality} and {\it fairness} when adjudicating cases involving interference in property.\footnote{See generally \cite[chapter 5]{allen05}.} Importantly, these broad notions are assessed concretely against the context of interference, also to give appropriate weight to the social and political function of the property interfered with. As a result, specific social functions of property can justify greater protection against interference.} A series of cases involving hunting rights provide an example of this.\footnote{See \cite{chassagnou99,hermann12,chabauty12}.} In these cases, the Court in Strasbourg has explicitly granted stronger property protection to owners who oppose hunting on ethical grounds, compared to owners who want to retain exclusive hunting rights for themselves.

For the former group of owners, it has been held that the state may not compulsorily transfer hunting rights to hunting associations for collective management.\footnote{See \cite{chassagnou99, hermann12}.} For the latter group of owners, by contrast, the Court held in {\it Chabauty v France} that such transfers must be tolerated.\footnote{See \cite{chabauty12}.}\indexonly{echr}

For owners opposing hunting on ethical grounds, an interference with their hunting right is an interference with their moral duty to act in accordance with their beliefs. The belief that hunting is unethical gives the owners a personal obligation to prevent their hunting rights from being used. If owners are deprived of their opportunity to fulfil this obligation, it changes the social function of their property because it severs the link between the owners' value system and the use that is made of their property.

In {\it Chassagnou and others v France}, the Court regarded this as a particularly severe interference in property, which could not be upheld despite the fact that it had been carried out in the public interest to secure sustainable management of hunting rights. The Court concluded that ``compelling small landowners to transfer hunting rights over their land so that others can make use of them in a way which is totally incompatible with their beliefs imposes a disproportionate burden which is not justified under the second paragraph of Article 1 of Protocol No. 1''.\footnote{See \cite[85]{chassagnou99}.}\indexonly{echr}

Clearly, the Court is not expressing an opinion on the ethical status of hunting. However, it is recognised that owners are entitled to have unconventional personal convictions in this regard. Moreover, managing one's property in accordance with one's convictions is recognised as part of what it means to be an owner. Protecting this aspect of ownership appears to be more important to the Court in Strasbourg than protecting the right of exclusion for owners who wish to keep the fruits of the land to themselves, as demonstrated by the ruling in {\it Chabauty}. 

The hunting cases also demonstrate that even when the legal system does not explicitly recognise the value of a social function inherent in property, such a function can still come to play a role when the Court assesses the legitimacy of interference against P1(1). This is reassuring since, as I argue in the next section, the law of property invariably involves prioritising between different social functions, also in situations when this is not openly acknowledged by policy makers and judges.\indexonly{echr}

\subsection{The Impossibility of a Socially Neutral Property Regime}\label{sec:2:4:2}

Property both reflects and shapes relations of power among members of a society.\footnote{This aspect of property's social function was stressed in a recent ``statement of purpose'' made by leading property scholars in support of the social function theory, see \cite{alexander09a}. For a sociological perspective on this, see \cite[23]{carruthers04} (``The right to control, govern, and exploit things entails the power to influence, govern, and exploit people.'').} Moreover, it does not act uniformly in this way -- the effect depends on the circumstances. \noo{ An indebted farmer who is prevented by state regulation from making profitable use of their land might come to find that their property has become a burden rather than a privilege. As a consequence, someone who has already amassed power and wealth elsewhere might be able to purchase the land from the farmer cheaply. By acquiring a farm and transforming it to recreational property, the outsider will symbolically and practically assert their dominance and power, while also reaping a potential financial benefit from investing in a more modern function of property.

In some cases, this dynamic can become endemic in an area, resulting in a complete reshaping of the social fabric surrounding property.} Consider, for instance, a fairly typical scenario leading to depopulation of rural areas: first, impoverished farmers and other locals sell homes to holiday dwellers, causing house prices to soar. As a result, local people with agrarian-related incomes \isr{cannot} afford local homes, causing even more people to sell their land to the urban middle class. In this way, a causal cycle is established, the social consequences of which can be vicious, particularly to the low-income people who are displaced.\footnote{The general mechanism described here is well-documented and known as {\it gentrification} in human geography (often qualified as rural gentrification when it happens outside urban areas). See generally \cite{weesep94,phillips93,slater06}. For a case study demonstrating the role that state regulation can play (perhaps inadvertently) in causing rural gentrification, see \cite[1027-1030]{darling05}.} This gives rise to the following  theoretical contention: setting out to regulate property in a situation like this -- when property rights pull in different directions depending on your vantage point -- requires a principled stance on whose property, and which of property's functions, one is aiming to prioritise. Should the law emphasise the property rights of local people who face displacement, or should it protect the property rights of outsiders wishing to invest in holiday homes?

Some may \isr{shy} away from this way of posing the question,  arguing instead that it would be better to rely on neutral rules that treat all owners the same way. In a gentrification scenario, for instance, such an appeal to neutrality could be the first step in an argument against regulating the property market to prevent the displacement of local people. But would that truly be a socially neutral approach to residential property? Presumably, it would threaten the property interests of local owners, particularly those not wishing to sell their properties.\footnote{The threat might be more or less direct. For instance, a weakened community could reduce the values that make the properties attractive to their current owners, while rising property prices could give rise to rising property taxes that render this ownership unaffordable as well.} Hence, if {\it their} property rights are to be protected, regulation should be put in place.

Importantly, both sides of a conflict like this are in a position to adopt a property narrative to argue for their interests. Hence, it is also inappropriate to think that the law of property can remain neutral. Moreover, a traditional narrative of property might fail to make sense of the ensuing tension. Consider, for instance, the conflict between Donald Trump and the Balmedie locals, as discussed in section \ref{sec:2:2}.

As long as Trump threatened to use compulsory purchase, the local people could adopt a traditional ``pro-property'' stance against Trump. But as soon as Trump decided to fence them in by relying on his own property rights, they had to adopt a seemingly contradictory view on property, whereby Trump's property rights should be limited out of concern for the community. A traditionally minded observer might use this as an opportunity to accuse the locals of having an unprincipled attitude towards private property. By contrast, the social function approach suggests a very different picture.

The locals sought to protect property, but not just any property. The property they wanted to protect was the property which served the social function of sustaining the existing community. The property they wanted to protect was the property that {\it meant} something to them.\footnote{This is more than merely observing that they wanted to protect {\it their} property. In their desire to regulate the use of Trump's property, the locals also wanted to protect certain social functions inherent in that property, against Trump's own actions.} 

Trump and his supporters might well have entertained similar feelings about their property rights, and the development they wanted to carry out. Hence, in conflicts such as these the law will invariably have to take a stand regarding which property interests it wishes to promote. The social function theory asks us to be upfront about this, so that policy making and adjudication in hard cases can proceed on the basis of substantive arguments about social functions rather than unconvincing appeals to neutrality and deference.

\noo{
In all likelihood, such a stand must also sometimes be taken by whoever {\it interprets} the law, since it is exceedingly unlikely that the legislature will ever be able to provide deterministic rules for resolving all conflicts of this kind. Lastly and most controversially, the courts may find occasion to curtail the power of government -- perhaps even the legislature -- if such power is usurped by powerful actors wishing to undermine property's proper functions to further their own interests. 
This, in particular, raises the question of constitutional and human rights limits to interference in property, relative to those functions that are to be protected.}
 
While the law is forced to prioritise in case of conflict, social functions can also work together in a way that promotes certain property uses and decision-making structures for property management. This can even alleviate the pressure for top-down government regulation, with desirable consequences for both owners and the public interest.
%, as discussed in the next subsection.

%\noo{ \subsection{The Regulating Effect of Property}

%Property shapes and reflects societies, but it also shapes and reflects social commitments and dependencies within those societies.\footnote{See generally \cite{alexander09}.} 

Again, this function of property is highly dependent on context. Small business owners, by virtue of being members of the local community, might be socially discouraged from becoming a nuisance to their \isr{neighbours}, with no need for state interventions through detailed legislation or planning.\footnote{See, e.g., \cite[282-283]{ellickson91}.} However, if local owners go out of business and a non-local commercial owner replaces them, the regulatory effect of property can change dramatically. 

Indeed, if we imagine that the new owner hopes to raze the local community in order to build a new shopping center, we are at once reminded of the stark contrasts that can arise between various social functions of property. The property rights of small shop owners can be the lifeblood of a community, while the exact same rights in the hands of a large enterprise can give rise to its destruction.

Mechanisms like these can have important ramifications, not only for property, but also for the regulatory regime surrounding its use. For instance, it seems clear that if a new and more commercially aggressive owner is to be deterred from becoming a nuisance to neighbours, stricter forms of regulation might have to be put in place.\footnote{It has been argued that such a mechanism explains the advent of zoning and land use planning in the US, particularly the Supreme Court's willingness to accept it even though it limited the freedom of property owners. See \cite[99-100]{shoked11}.} The social responsibility that was previously anchored in the community must now be protected more forcefully by the state. In turn, this can cause the institution of property to weaken further, as the government assumes greater power to interfere.\footnote{For an early criticism of zoning in the US, pointing to the merits of a more flexible and highly decentralised approach based in large part on the law of nuisance, see \cite{ellickson73}.} A feedback effect might result, as increased regulation in turn threatens to make property ownership too burdensome or expensive for low-income, or even average-income, community members.\footnote{In the US, the term ``exclusionary zoning'' is used to describe such a mechanism when zoning contributes to pushing low-income people out of suburban, and increasingly also urban, areas. Since the zoning framework in the US is quite decentralised, the process is often pushed forward by locally based affluent home-owners who capture the zoning power of local governments to enhance their property values, e.g., by preventing intrusive development projects that could otherwise attract low-income people by increasing the demand for cheap labour and the supply of cheap housing. Due to this dynamic, the feedback mechanism is amplified: the standard proposals for reform to deal with exclusionary zoning involve further inflating the power of the government or the markets to impose large-scale development projects against the will of local communities. See, e.g., \cite[117-120]{mangin14} (referring favourably to the standard reform suggestion, but noting that it seems politically unrealistic to implement in the US). For a more subtle analysis, resulting in the proposal that home-value insurance should be introduced to make home-owners less likely to pursue exclusionary zoning, see \cite{fischel04}.} Hence, the most resourceful actors, those who are able to meet or influence the government's demands and/or protect themselves against interference, gain more  property, while the government gains more regulatory power.

The social function theory tells us that mechanisms of this kind need to be taken seriously by legal theorists and practitioners. The broader point at stake here can also be brought out in relation to the famous ``tragedy of the commons''.\footnote{See \cite{hardin68}.} In his seminal article, Hardin describes how individually rational users of a commons can eventually cause the depletion of that resource. The problem arises, according to Hardin, because individuals have no proper incentive to refrain from over-exploitation; the damage will be distributed among all resource users, so it will not outweigh the benefit of individual over-use in the short term.

%\footnote{See \cite{demsetz67} (according to Demsetz's theory, subdividing a commons and introducing individual exclusion rights characterises the introduction of a private property regime, the purpose of which is to internalise many externalities while making negotiations about any remaining externalities, e.g., non-localised pollution, more effective).}

In response, it has been typical to regard either state management or stronger individual exclusion rights as the answer.\footnote{See \cite[8-13]{ostrom90}.} State management is supposed to prevent over-exploitation through regulation, while individual exclusion rights are supposed to make it more difficult for resource users to shift the cost of over-exploitation onto other individuals.

Both of these strategies can potentially result in the sort of feedback effect discussed above, where the inadequacies of states and markets combine to make it necessary for both of them to inflate their power at the expense of local communities. However, a cycle of dispossession is not the inevitable outcome of the tragedy of the commons. Indeed, as Elinor Ostrom and others have shown, the traditional narrative overlooks the fact that commons tend to come with community structures that provide appropriate checks and balances through locally grounded institutions or social arrangements.\footnote{See generally \cite{ostrom90}.} As long as external forces do not threaten them, such arrangements can be more robust than either 
individual exclusion regimes or state control. Moreover, they can be anchored in the law of property.\footnote{For the connection between property law and governance theories for common pool resources generally, see \cite{rose11,fennel11}. It is worth emphasising that this link is not only relevant in relation to {\it natural} resources. Indeed, common pool theories have been much discussed also in the context of intellectual property, see \cite[38-43]{rose11}. That being said, in this thesis the focus will be on property forms that are moulded out of land and related resources. The special questions that might arise when the theoretical framework is applied to other forms of property will not be addressed.} I will return to this point in the next chapter, when I discuss possible alternatives to eminent domain in economic development situations.\footnote{See chapter \ref{chap:3}, section \ref{sec:3:6}.}

\noo{ The ideas of Ostrom on common pool management focus on local institutions for collective decision-making, not property rights. As a complementary viewpoint, the idea that local institutions for resource management can be anchored in private property represents a potentially attractive way of thinking.\footnote{For a survey of how US property scholars have been influenced by Ostrom's ideas, including a discussion on the relationship between (private) property and local institutions for self-governance, see \cite{rose11}.} The social function theory of property already suggests pursuing this idea. Based on the social function approach, the descriptive fact that property structures shape decision-making processes at the local level is enough to conclude that local institutions for resource management should not be looked at in isolation from the law of property.\footnote{It is also worth emphasising that there is nothing in this way of thinking that limits us to considering property in {\it natural} resources. Indeed, common pool theories have been much discussed also in the context of intellectual property, see \cite[38-43]{rose11}. That being said, in this thesis the focus will be on property forms that are moulded out of land and related resources. The special questions that might arise when the theoretical framework is applied to other forms of property will not be addressed.}

Moreover, by recognising property as an anchor for equity and decision-making at the local level, social obligations that inhere in private property may be recognised as existing independently of specific institutional arrangements. Hence, if local institutions are marred by corruption and malpractice, a social function theorist can take the normative stance that property ownership still carries with it duties to care for other property dependants in the community. This duty, moreover, would exist independently of the extent to which it is presently fulfilled through local practices and institutional arrangements. 
}
%section \ref{sec:2:5}, when I discuss the human flourishing theory of property and its promise of internalising economic and social rights for non-owners into the structure of property itself. First, I will argue that it is useful to distil a descriptive core from the social function theory, so that it may serve as a common ground for debate, allowing the interchange of ideas across normative divisions.

\subsection{The Descriptive Core of the Social Function Theory}\label{sec:2:4:3}

Social function theorists have been criticised for making too far-reaching normative claims. Eric Claeys, in particular, argues forcefully against normative fundamentalism and what he regards as normative naivety among social function theorists.\footnote{\cite[945]{claeys09}. (``Judges might think they are doing what is equitable and prudent. In reality, however, maybe they are appealing to a perfectionist theory of politics to restructure the law, to redistribute property, and ultimately to dispense justice in a manner encouraging all parties to become dependent on them.'')} Indeed, some social function theorists have gone very far in presenting the social function account of property as a normative theory, attaching specific political commitments to it along the way.

Hanoch Dagan, for instance, is a self-confessed liberal who argues for a social function understanding on the basis that it is morally superior. ``A theory of property that excludes social responsibility is unjust'', he writes, and goes on to argue that ``erasing the social responsibility of ownership would undermine both the freedom-enhancing pluralism and the individuality-enhancing multiplicity that is crucial to the liberal ideal of justice''.\footcite[1259]{dagan07}

If this is true, then it is certainly a persuasive argument for those who believe in a ``liberal idea of justice''. But for those who do not, or believe that property law is -- or should be -- as neutral as possible on this point, a normative argument along these lines can only discourage them from adopting a social function approach. Such a reader would be understandably suspicious that the {\it content} of the social function theory -- as Dagan understands it -- is biased towards a liberal world view. Such a reader might agree that property continuously interacts with social structures, but reject the theory on the basis that it seems to carry with it a normative commitment to promote liberalism.

Dagan is not alone in proposing highly normative social function theories. Indeed, most contemporary scholars endorsing a social function view on property base themselves on highly value-laden assessments of property institutions.\footnote{See, e.g.,  \cite{alexander09,crawford11,davidson11,singer09,penalver09}.} By contrast, the discussion in this chapter so far has aimed to demonstrate that the theory has significant merit already as a {\it descriptive} theory. In my opinion, this is also demonstrated by much of the normatively oriented work that has been done in the social function tradition. When this work focuses on making abstract normative assertions it threatens to overshadow what is arguably the most important insight, namely that considerations related to social functions {\it are} already important in many areas of property law, in many different jurisdictions.\footnote{See, e.g., \cite{gray94,mirow11,cunha11,bonilla11}.} Moreover, the social functions of property, and normative assertions about them, often play a role behind the scenes, where they do unacknowledged work among policy makers and judges alike. As Laura Underkuffler puts it:

\begin{quote}
Property rules, as they now exist, are contingent rules, complex rules, and normatively charged rules. They are crafted and applied in response to the politics of power, security, stability, greed, and a myriad of other aspects of human life.\footnote{\cite[376]{underkuffler10}.}
\end{quote}

Because it embraces this crucial insight, the core of the social function theory, rather than being ``good, period'' as Dagan suggests, is simply more accurate than other proposals, irrespective of one's ethical or political inclinations.\footcite[1259]{dagan07} The theory provides the foundation for a discussion where different values and norms can be presented in a way that is conducive to meaningful debate, on the basis of a minimal number of hidden assumptions and implied commitments. Thus, the first reason to accept the social function theory is epistemic, not deontic.

That is not to say that theories can ever be entirely value-neutral, nor that this should be a goal in itself. However, a good theory is one that can at least serve as a common ground for further discussion based on disagreement about values and priorities. \noo{According to Kevin Gray, ``the stuff of modern property theory involves a consonance of entitlement, obligation and mutual respect''.\footcite[37]{gray11} This is a rather loose way of putting it, but I believe it also points to a measured perspective that is ultimately highly appropriate.} Making room for normative divergences, moreover, can hopefully diminish the worry that a broader theoretical outlook is the first step towards unchecked state power and rule by ``judicial philosopher-kings'', as Claeys puts it.\footnote{See \cite[944]{claeys09}. There is further evidence to suggest that this is a real worry. Specifically, in a recent article, Anna di Robilant discusses how the Italian fascists were happy to embrace a social function perspective on property, because it helped them make the case that property should be made to answer to one core collective value: the interests of the state. As a form of resistance, many Italian property scholars agreed that the social function view was in order, but emphasised the plurality of values associated with this vision, values that have little or nothing to do with the interests of a Fascist state. See \cite{robilant13}.} At the same time, the new descriptive dimensions uncovered by the social function view can also inspire novel normative perspectives, as explored in the next section.

\section{Human Flourishing}\label{sec:2:5}

Taking the social function theory seriously forces us to \isr{recognise} that a person's relation to property can be partly constitutive of that person's social and personal capabilities, both in a political and an economic context.\footnote{I will explore some specific capabilities in more depth later on, when discussing economic and social rights, and the value of participation in democracy. For the notion of a capability more generally, proposed as a foundational concept for economic theory, see \cite{sen85}. For a discussion on the import of this work to property theory, see \cite[105]{alexander09}.} Moreover, property influences people's preferences, as well as what paths lie open to them when they consider their life choices.\footnote{See generally \cite{alexander09}.} This effect is not limited to the owner, it comes into play for anyone who is socially or economically connected to property in some way, including a potentially large group of non-owners.\footcite[128-129]{alexander09d}

Hence, there is great potential for making wide-reaching socio-normative claims on the basis of the social function perspective on the meaning and content of property. But which such claims {\it should} we be making? According to some, we should adjust our moral compass by looking to the overriding norm of {\it human flourishing} as a guiding principle of property law. In a recent article, Alexander goes as far as to declare that human flourishing is the ``moral foundation of private property''.\footcite[1261]{alexander14} 

Human flourishing has a good ring to it, but what does it mean? According to Alexander, several values are implicated, both public and private.\footnote{See generally \cite{alexander14,alexander11}.} Importantly, Alexander stresses that human flourishing is {\it value pluralistic}.\footnote{\cite[750-751]{alexander09}.} There is not one core value that always guarantees a rewarding life. To flourish means to negotiate a range of different impulses, both internal and external. Importantly, these act together in a social context that influences their meaning and impact.\footcite[1035-1052]{alexander11}

In the following, I consider some values that I regard as particularly important for the study of economic development takings. I start by the values enshrined in economic and social rights, which should arguably also inform our understanding and application of property law.

\subsection{Property as an Anchor for Economic and Social Rights}\label{sec:2:5:1}

The so-called ``second generation'' of human rights consists of basic economic, social and cultural  rights that complement the better known civil and political rights.\footnote{See generally \cite[1-14]{baderin07} (arguing that the ``second-generation'' terminology is unfortunate since it can give rise to the misconception that ESC rights are second-class).} This includes rights such as the right to housing, the right to food, and the right to work.\footnote{See \cite[6|7|11]{fnp}.} Economic and social rights of this kind often involve property. Specifically, they often involve interests in property that are not recognised as ownership, e.g., housing rights for squatters or rights to food and work for landless rural people.

If the notion of property is conceptualised in the traditional way, as an arrangement to protect individual entitlements, the relationship between private property and economic and social rights appears to be one filled with tension.\footnote{See, e.g., \cite[138]{garcia14}.} In particular, if economic and social rights require owners to give up some property entitlements, it becomes natural to portray property protection as standing in the way of social justice.

However, the human flourishing theory can be used to tell a very different story, namely one where economic and social rights are anchored in the notion of property itself. Importantly, the human flourishing theory compels us to take into account the interests and needs of property dependants other than owners. As Colin Crawford puts it, the purpose of property should be to ``secure the goal of human flourishing for all citizens within any state''.\footcite[1089]{crawford11} Consider, for instance, the right to housing. If the interests of a property owner come into conflict with the housing rights of a property dependant, the human flourishing theory encourages us to approach this as a tension {\it within} property, between different property functions.

With such a starting point, we should also acknowledge that the appropriate way to approach the rights of non-owners in relation to property might well depend on who the owner is and the choices they make in managing their property.\footnote{See, e.g., \cite[43]{walt11} (commenting on the principle of ``scaling'' of social obligations in German property law, whereby what can be demanded of owners depends on the context).} For instance, if owners live on their land and do not own much more than they need themselves, it becomes hard to maintain the criticism that their private property is somehow an affront to the housing rights of the landless. Moreover, from a practical point of view, squatting is unlikely to occur in these settings unless accompanied by severe trespass or dispossession. Similarly,  the owner of a commercial building can discourage squatting by managing the property well. Arguably, this too can undercut potential criticism on the basis of housing rights, especially if the owner uses the building to engage in a commercial activity that contributes to sustaining the local community.

On the other hand, if owners mismanage their properties, for instance because they seek to obtain demolition licenses or simply wish to await an expected rise in land values, squatters might take opportunity of this and feel encouraged to occupy the property. If private property is thought of merely as entitlement-protection, a property-protecting state might feel obliged to respond in a way that offends against the social and economic rights of the squatters. If this is considered an undesirable outcome, the government or its critics might in turn come to regard strong property protection as an affront to housing rights, even though the real problem is that property does not function as it should within society. Hence, the result can be that property structures are damaged further, as the state pursues policies of interference and centralised management, without addressing how private property as such can promote human flourishing.

By contrast, the human flourishing narrative suggests that both owners and non-owners might appropriately be viewed as victims if the state fails to protect property's proper functions. This perspective might even suggest itself when owners and non-owners would otherwise appear to be adversaries. For a concrete example, I mention the South African case of {\it Modderklip East Squatters v Modderklip Boerdery (Pty) Ltd}.\footnote{See \cite{modderklip05}. For two commentaries focusing specifically on its implications for property law and theory, see \cite{alexander09d,walt05}.}

The case dealt with squatting on a massive scale: some 400 people had initially taken up residence on land owned by Modderklip Farm, apparently under the belief that it belonged to the city of Johannesburg.\footnote{See \cite[4]{modderklip05}.} The owner attempted to have them evicted and obtained an eviction order, but the local authorities refused to implement it. Eventually, the settlement grew to 40 000 people and Modderklip Farm complained that its constitutional property rights had not been respected.\footnote{See \cite[8]{modderklip05}.}

The Supreme Court of Appeal concluded that Modderklip's property rights had indeed been violated, but noted that the housing rights of the squatters were also (in danger of) being violated; the squatters needed a place to go before they could be evicted.\footnote{See \cite[21-26]{modderklip04}.} Hence, the appropriate response was not to evict the squatters, but to pay compensation to Modderklip, for an ongoing violation of its property rights, caused by the state's failure to protect housing rights.

This outcome is consistent with a social function understanding of property, but the rationale behind it builds on a narrative that takes the perceived tension between property and housing rights as its point of departure.\footnote{See \cite[152-156]{walt05}. However, as van der Walt notes, the Supreme Court of Appeal took the traditional narrative in an interesting direction when it held that the failure of the state to protect the housing rights of the squatters was the cause of its failure to protect property. As van der Walt notes, the conflict between rights then became less important than the observation that the state had failed in its duty to protect {\it both} rights. This is an improvement on a narrative focused on the question of which right to prioritise, but still arguably over-emphasises the role of the state.} The Constitutional Court, by contrast, chose to remain agnostic on the issue of the state's duties with respect to both property and housing rights, as well as the relation between them.\footnote{See \cite[25]{modderklip05}.} Rather, the Constitutional Court focused on the state's failure to ``assist Modderklip'' in dealing with the ``burden imposed on it to provide accommodation to such a large number of occupiers''.\footnote{See \cite[49]{modderklip05}.} This was a failure of governance, and the state was ordered to pay compensation, not for a violation of property, but as an appropriate form of assistance to Modderklip.\footnote{For a more detailed presentation of the Court's decision, including references to the relevant governance provision of the South African constitution (guaranteeing access to court with suitable and efficient enforcement procedures), see \cite[156-158]{walt05}.} 

This shift of perspective can arguably be understood as a reflection of the Court's willingness to regard the needs of the squatters as giving rise to a social obligation for Modderklip {\it qua} owner. Effectively, the circumstances of the case meant that Modderklip's ownership entailed an obligation to respect the housing needs of a community of 40 000 people. Moreover, the primary duty-bearer had become the owner, not the state.

With such an approach, private property can be a potential source of justice for anyone, including squatters. The role of the state, meanwhile, becomes that of assisting those who are directly responsible for delivering justice on the ground, including owners such as Modderklip. In a detailed analysis of the case, Alexander and Pe\~{n}alver also argue in this direction. They suggest, in particular, that {\it Modderklip} serves as an illustration of how property owners themselves can have responsibilities towards property dependants, obligations that endure as long as private property remains in place.\footnote{\cite[157]{alexander09d} (``The courts' unwillingness to ratify Modderklip's desire to remove the squatters from its land illustrates the courts' willingness to take seriously the obligations of owners, not only as they concern owners' direct relationship with the state but also in relation to the needs of other citizens.''). It should be noted, moreover, that Modderklip was eager to sell the land to the government. See \cite[61]{modderklip05}.} 

This normative turn makes property owners addressees of obligations arising from the economic, social and cultural rights of non-owners, not by direct horizontal application of these rights, but through the law of property.\footnote{In this way, it arguably strengthens such rights, while potentially circumventing problems and objections associated with the idea of making such rights directly justiciable in disputes among private parties. For the question of horizontal application more generally, see \cite{manisuli07}. As the case of {\it Modderklip} demonstrates, embedding the duty of non-state actors in property will serve to ensure that the state is responsible for providing assistance in cases when the burdens of ownership become disproportionate. The duty of the state in these circumstances will be formally directed towards the owners, even if the beneficiaries of state action will be those property dependants with respect to whom the owners have obligations. This might prove conducive to more effective action also at the state level; presumably, the owners as a group would make a valuably ally for the landless squatters, in cases of failed governance.} In this way, the human flourishing theory points towards a novel way to address the rights of marginalised groups, for whom it often appears that neither neo-liberal property nor state management is capable of delivering basic justice.\footnote{For instance, it has been noted that in India, the human right to water has at times been simultaneously frustrated both by a non-egalitarian distribution of riparian rights as well as a regulatory framework that grants the state almost limitless proprietary power over water resources. See \cite[186]{cullet09}. To address shortcomings of the current system for water management in India, Cullet recommends a conceptual approach that ``leaves aside property rights altogether'' (both private and public) and instead emphasises that water is a common heritage of humankind. A possible alternative, highlighted by the social function theory, is to embrace human flourishing as the core value of property, possibly in a manner that is mutually conducive to recognising water as a common heritage. On such an account, property might regulate some uses of water, such as economic development, without thereby resulting in anyone being granted ownership of water as a substance or a right to charge people for access to drinking water. As discussed in Part II of this thesis, the traditional property regime for water resources in Norway arguably reflects a framework along these lines, applied in a context marked by egalitarian property with no scarcity of water for basic human needs. The appropriateness of proposing similarly spirited (but adjusted) property-based frameworks for different contexts is a matter for future work, but in my view, the human flourishing theory at least suggests that this might be worth exploring.}

Moreover, it strengthens the institution of property, highlighting why it might be appropriate to 
grant it extensive protection against interference. In particular, a human flourishing approach might serve as a bulwark against the idea that the ultimate expression of public interests can be found in the actions taken by the state. Instead, the theory directs attention at how public interests are expressed at their point of origin; values often associated with the public sphere, such as those pertaining to the economic and social rights of marginalised groups, are in fact legally relevant already at the level of private law.\footnote{See also \cite[1295-1296]{alexander14}.} As a consequence, the human flourishing account bolsters the view that public interests and obligations can acquire some justiciable relevance even in the absence of explicit international treaties, legislation or equitable decision-making within (inter)national institutions. 

Perhaps the most important structural aspect of this insight concerns the mechanisms used to resolve tensions between different property values. Importantly, it might not be necessary to introduce intermediaries between owners and other rights holders and property dependants. To introduce such intermediaries, whether they are state bodies, international institutions, NGOs, or commercial enterprises, carries with it the risk that the decision-making process can be captured by forces that either have ulterior motives or are simply too far removed from local conditions to deliver results on basic rights.\footnote{See, e.g., \cite{cullet13} (describing how non-state actors and developed countries increasingly seem to capture the agenda in international water policy, to the detriment of people and states in the developing part of the world); \cite{levien13} (analysing state-led processes of rural dispossession in India, arguing that states now often act as land brokers for private enterprises); \cite{mehta14} (two case studies, from India and Bolivia respectively, demonstrating that elite bias and other democratic deficits at the state level have frustrated efforts to deliver on water rights for marginalised groups in peri-urban areas).} It might be better, therefore, if basic rights are (also) anchored and implemented using private law solutions that target the local level, for instance by relying on the law of property.

\noo{ At least, it should be possible to pursue key economic and social values without massively increasing the power of non-local actors and weakening the institution of property. Moreover, it should be possible to more effectively enforce social obligations on private property owners, particularly when they are locally based. Achieving this in practice requires mechanisms that enable negotiations between competing private property interests, to facilitate a balancing of those interests through participatory decision-making rather than top-down state management. This highlights the importance of another property value that the human flourishing theory emphasises, namely that of participation, discussed in the following section.}

\subsection{Property as an Anchor for Democracy}\label{sec:2:5:2}

It is often argued that property is a crucial building block of democracy, as it both empowers and encourages owners to participate in the political process.\footnote{See generally \cite{rose96} (critically examining common arguments to support the claim that property is the most fundamental right, including the argument that it gives rise to, facilitates, and protects democracy). For an exposition of the converse link, explaining how property law is constrained and determined by the values and principles associated with democracy, see \cite{singer14}.} However, the notion of participation at work here often seems to be drawn up rather narrowly, as pertaining primarily to individual owners, and only to their engagement with the formal affairs of the polity.\footcite[1275]{alexander14} By contrast, the human flourishing theory gives participation a broader meaning, involving also the value of being included in a community. Alexander writes:

\begin{quote}
We can understand participation more broadly as an aspect of inclusion. In this sense participation means belonging or membership, in a robust respect. Whether or not one actively participates in the formal affairs of the polity, one nevertheless participates in the life of the community if one experiences a sense of belonging as a member of that community.\footcite[1275]{alexander14}
\end{quote}

Participating in a community can have a crucial influence also on people's preferences and desires.\footnote{For a more in depth discussion of this, see \cite[140]{alexander09}. Here, Alexander and Pe\~{n}alver draw on the work of Amartya Sen and Martha Nussbaum, see generally \cite{sen84,sen85,sen99,nussbaum00,nussbaum02}.} Therefore, for anyone adhering to welfarism, rational choice theory, or some other utilitarian dogma, neglecting the importance of communities is not only normatively undesirable, it is also unjustified in an epistemic sense. In particular, it should be \isr{recognised} as a descriptive fact that the idea of community is highly relevant to {\it any} normative theory that attempts to take into account the preferences and desires of individuals. But Alexander and Pe\~{n}alver go further, by arguing that participation in a community should also be seen as an irreducibly social value, not merely as a determinant of individual preferences and a precondition for rational choice. They write:

\begin{quote}
Beyond nurturing the individual capabilities necessary for flourishing, communities of all varieties serve another, equally important function. Community is necessary to create and foster a certain sort of society, one that is characterized above all by just social relations within it. By ``just social relations'', we mean a society in which individuals can interact with each other in a manner consistent with norms of equality, dignity, respect, and justice as well as freedom and autonomy. Communities foster just relations with societies by shaping social norms, not simply individual interests.\footcite[140]{alexander09}
\end{quote}

This, I believe, is a crucial aspect of participation. Moreover, it is a notion that invariably leads us to recognise that other property dependants should also have a voice, as they form part of the ``just social relations'' within the community to which the owners belong. In addition, this is a notion of participation that it is hard, if at all possible, to incorporate in theories that take preferences and other attributes of individuals as the basis upon which to reason about their legal status. Instead, the human flourishing perspective asks us to consider how property serves as an anchor for participation that shapes and influences community norms and preferences.

Protecting the function that property plays in this regard can at times require protecting it also against the actions of owners and their communities. This can happen, for example, in a community where people have come under pressure to sell their homes and their land to make way for large enterprises. If owners are offered generous financial compensations, or if they are threatened by eminent domain, economic incentives might trump the value of social inclusion and participation. As a consequence, the community might decide to sell.

Even so, in light of the value of community, it would be in order for planning authorities, maybe even the judiciary, to view such an agreement as an {\it attack on their property}. It is clear that by the sale of the land, the ``just social relations'' inhering in the community will come under pressure. Property rights that once contributed to sustaining these relations will be transformed into property rights that serve a very different purpose, namely that of aiding the concentration of power and wealth in the hands of the commercially powerful. Such a change in the social function of property might have to be regarded -- objectively speaking -- as a threat to participation, community and democracy. Therefore, it is arguable that our property institutions should protect against it, even if this implies limiting the freedom of owners and communities to do as they please.

In Norway, a range of such rules are in place to protect agricultural property, by limiting the owners' right to sell parcels of their land without local government consent, as well as by compelling them to reside on their property and to make use of it for agricultural production.\footnote{See \indexonly{la95}\dni\cite[8|12]{la95} and \indexonly{lca03}\dni\cite[4|5]{lca03}.} In addition, there are rules in place that guarantees certain principles of non-exclusion for outfield land; the owner of such land cannot prevent anyone from travelling over it, camping on it, and must even tolerate that visitors pick berries and roots for their own consumption.\footnote{See generally \cite{backer07}.} This is the so-called ``allemannsretten'' -- the right to roam -- which has been recognised in Norway since ancient times.

When the law actively promotes egalitarian and equitable property using arrangements such as these, the natural counterpart is to limit direct state interference. The danger otherwise is that the limited power of each individual property owner -- appropriate in a community of property owners -- is exploited by the state or other powerful stakeholders who might wish to usurp control over local resources and impose their will on local populations.

The broader issue at stake is highlighted by recent developments in South Africa, where rules resembling many of those in place for agricultural property in Norway have been proposed in a recent Act on land reform.\footnote{See \cite{steyn15}.} In South Africa, however, these rules have been proposed alongside a new framework of state ``custodianship'' of agricultural land, corresponding to a formulation recently introduced in the mineral and petroleum legislation.\footnote{See \cite{agri13} (holding that introducing custodianship over mineral and petroleum is not expropriation, meaning that no compensation is payable to owners, not even when the rights taken may subsequently be transferred --  in a different wrapping, under a new regulatory regime -- to third parties). The decision was made under dissent on the basis of the conclusion that {\it acquisition} by the state -- apparently not implied by ``custodianship'' --- is required for a deprivation of property to count as expropriation. On an uncharitable reading, the decision therefore appears to open the floodgates for economic development takings, which can now even go uncompensated as long as they are embedded in a regulatory narrative based on custodianship. The academic community in South Africa appears to be divided on the issues raised by the case, see \cite[428-451]{walt11}; \cite{marais15a,marais15b}.} If the proposal passes, the proper functioning of agricultural property in South Africa would seem to depend quite strongly on the benevolence and capability of the state, which will significantly increase its own power to interfere with private property.\footnote{Hence, the proposed land reform arguably demonstrates the appropriateness of Justice Froneman's dissent in {\it Agri}, see \cite[79-110]{agri13} (warning against the precedent set by the majority, holding that state custodianship of the type in question amounted to expropriation, but finding that compensation was not required in the circumstances of the case, looking also to the value of social justice and the history of apartheid in South Africa).} Importantly, the human flourishing perspective suggests that even when provisions to promote egalitarian ownership and community commitment are appropriate, provisions that inflate the state's authority might not be. The case study from Norway will illustrate that strict property rules to protect and promote self-governing agrarian communities can work well, but only as long as they are applied consistently and coupled with strong institutions of local democracy and strict limits on state power.\footnote{I discuss the role of agrarian property to the development of Norwegian democracy in more depth in chapter 4.}

This raises the question of what kind of institutions we need to enable local communities and owners to  flourish and make democratically legitimate decisions about how to use their properties. Arguably, there is no appropriate theoretical answer to this question, since institutions for participatory decision-making are successful only when they match local conditions.\footnote{See also the discussion in chapter \ref{chap:2}, section \ref{sec:2:6} (discussing the design principles for self-governance first presented by \cite{ostrom90}).} In the final chapter of the thesis, I return to discuss this concretely by looking to the Norwegian institution of land consolidation.

In the next section, I apply the theory developed so far to economic development takings. Specifically, I introduce this category of takings in more depth and present {\it Kelo} in further detail, drawing on the social function theory to carry out an assessment of the controversy that resulted.\noo{ v City of New London\footcite{kelo05}, which brought this category to prominence in the US discourse on property law. Then I will assess the unique aspects of such takings against the social function theory, to provide an argument that the category has significance for legal reasoning in takings law, as well as with respect to property as a constitutionally protected human right. Finally, I will provide an abstract presentation of the values that I believe are important when normatively assessing the law in this area. In doing so, I will draw on the human flourishing theory, setting out the main values that will inform the concrete policy assessments I provide later.}

\section{Economic Development Takings}\label{sec:2:6}

The notion of an economic development taking is in some sense self-explanatory: it targets situations when property is taken for economic development. However, the obvious follow-up question is a difficult one: what is meant by ``economic development''? In the literature on economic development takings, no clear answer has been provided.\footnote{See, e.g., \cite[558-567]{cohen06} (Cohen proposes a ban on economic development takings, comments on the difficulty of defining the notion precisely, before proceeding to pursue his stated aim indirectly, through a ban on takings benefiting private parties, with exceptions for certain non-profit undertakings).} Rather, one tends to rely on an intuitive understanding to classify takings as being for economic development, where typical cases are those where the decision-makers themselves emphasise the value of economic progress as a reason for authorising eminent domain.\footnote{See, e.g, \cite{somin07,ely09} (both authors also discuss how economic development, or even commercial profit, can be an unacknowledged motive, for instance when property is taken on the pretext of combating ``blight'').} The lack of clarity about what the category covers might seem like a theoretical challenge, possibly even a weakness. However, it can also be argued that the ambiguity of the notion of economic development forms part of the reason why economic development takings merit special attention in the first place.\footnote{See \cite{somin07} (arguing for a complete ban on the ``economic development rationale'', citing its vagueness as a reason why it should {\it never} be used to justify a taking).} 

Some scholars still prefer not to use the notion, choosing instead to speak of ``private takings'' when they discuss the legitimacy issues that arise in cases such as {\it Kelo}.\footnote{See, e.g., \cite{bell09}.} The notion of a private taking is very easy to define: such a taking occurs when the legal person taking title to the property is a non-governmental actor.\footnote{See \cite[519]{bell09}.} Arguably, however, this categorisation is quite unhelpful when the aim is to get at the legitimacy issues that arise specifically in economic development situations. For instance, it might well be that a private organisation, say a tightly regulated charity, functionally mimics a quintessential ``public'' taker. A public body, on the other hand, can well be functionally equivalent to a private enterprise, particularly if there is a lack of political oversight and democratic accountability. Imagine, for instance, a case involving a publicly owned limited liability company. According to the simple definition of a private taking, a taking by such a company would not meet the definition. This would be the conclusion even if the company's interests are completely or predominantly commercial, directed at maximising profit for the shareholders, not at providing a public service.\footnote{Some might argue that the distinction between private and public ownership is still significant. However, such an argument seems difficult to make independently of the social context. If a public company operates for profit and is insulated from political decision-making and principles of administrative law, it is hard to see why takings to benefit such a company should be regarded as {\it a priori} different from other kinds of economic development takings. In particular, it is hard to see why it should matter in such cases whether the associated public benefit is ensured through the payment of dividends, taxes, or some other mechanism. In any event, the public benefit will be indirect in these cases, arising from ordinary commercial activity.}

By contrast, the notion of an economic development taking points to the purpose of the taking, not the outward legal appearance of the taker. As such, it provides a less sharp distinction between different kinds of takings, but also seems more relevant to the question of legitimacy. Specifically, the category performs an important function in that it directs attention at the fact that there might be  inappropriate motives influencing the decision to take private property. Moreover, the main reason for paying particular attention to economic development takings is clear enough: the presence of strong economic incentives, often of a commercial nature, appears to increase the risk of eminent domain abuse.

The benefit of using a comparatively neutral and open-ended designation seems especially clear in mixed economies, where the influence of public-private partnerships can cause a general blurring of lines between private and public sectors.\footnote{For the growing importance of public-private partnerships to the world economic order, see generally \cite{saussier13}.} In such contexts, it seems particularly appropriate to devote special attention to cases where commercial interests stand to benefit -- directly or indirectly -- from a taking of private property. The presence of commercial incentives among the beneficiaries or their partners might contrast with the public spirited rationale provided to legitimise the taking. An important advantage of a categorisation based on the notion of economic development is that it can be used to flag cases where this contrast is present, suggesting that we should further scrutinize the legitimacy of the undertaking as a whole.

If we broaden our perspective even further and consider commercially motivated changes in property structures on the global stage, this perspective suggests itself with even greater force. In fact, it seems appropriate to speak of a crisis of confidence in property, particularly in relation to land rights, arising from how powerful commercial interests usurp proprietary power over an increasingly large share of the world's resources. This is the phenomenon known as {\it land grabbing}, which has received much critical attention in recent years.\footnote{See generally \cite{borras11}.}

So far, most research on land grabbing has looked at how commercial interests, often cooperating with nation states, exploit weaknesses of local property institutions, to acquire land voluntarily, or from those who lack formal title. However, the similarity between economic development takings and state-aided land grabbings in favour of large commercial companies is striking. Specifically, it has been noted that the purported public interest in economic development can be used to justify massive land grabs that would otherwise appear unjustifiable. In a recent article, Smita Narula cites {\it Kelo} directly and warns that procedural safeguards alone might not provide sufficient protection against abuse. She writes:

\begin{quote}
Procedural safeguards, however, can all too easily be co-opted by a state because its claims about what constitutes a public purpose may not be easy to contest. Particularly within the context of land investments, states could use the very general and under-scrutinized language of ``economic development'' to justify takings in the public interest.\footcite[157]{narula13}
\end{quote}

This underscores the broader relevance of the study of economic development takings. In addition, it asks us to keep in mind that the question of what can be justified in the name of ``economic development'' is a general one, not confined to particular systems for organizing property rights.\footnote{To address this, and to restore confidence in the institution of property more generally, some academics and policy makers have proposed a novel concept of property as a human right. It has been argued, in particular, that a human right to land should be \isr{recognised} on the international stage, a right that would apply even when those affected by a land grab lack formal title. If successful, this approach promises to deliver basic protection against interference in established patterns of property use independently of how particular jurisdictions approach property. Specifically, it would establish an important link needed to make the kinds of property protections discussed in this thesis justiciable in the context of land grabbing when those adversely affected lack formal title. See generally \cite{schutter10,schutter11,kunnerman13}.}

In India, for example, people have been displaced and dispossessed on a massive scale in the name of economic development.\footnote{See generally \cite{levien13}.} The idea that the state enjoys eminent domain has been used to enable such processes. In fact, the language of eminent domain has been used to justify controversial policy decisions also with respect to land that is not privately owned, but already under forms of state ownership/custodianship.\footnote{See \cite[141]{usha09}.} The notion of eminent domain is apparently so powerful that it can be used to silence opposition of all kinds, including that arising with respect to social and economic rights for the poor and the landless when they face displacement or loss of livelihoods.\footnote{See \cite[143-144]{usha09} (``the power of eminent domain has been interpreted as being close to absolute power of the State over all land and interests in land within its territory. The effect of this has been that those without access to land and rights over land (including the landless, artisans, women as a composite group), those who may have use rights but no titles, communities holding common rights and others with inchoate interests, have had to bear the burden heaved on to them by eminent domain.'').} Indeed, it has been argued that the scope of eminent domain in India has become so wide that it allows for a ``complete assertion of power'' by the state.\footnote{See \cite[43]{cullet09}. For a concrete example of a case involving displacement on a very large scale, see \cite{cullet01}.} This power, moreover, is apparently often used to ``disempower people and redistribute rights and benefits'', in a way that favours people who are already better-off than those negatively affected.\footnote{See \cite[33]{cullet09}.} Paradoxically, the power of eminent domain has expanded alongside a judicial re-creation of property as a fundamental right (the right to property was removed from the Constitution in 1978).\footnote{See generally \cite{allen15}.} However, as argued by Allen, ``the Supreme Court’s emphasis on liberal entitlement, rather than solidarity or social obligation, is likely to deny the new right to property of relevance in cases where social justice is paramount.''.\footnote{See \cite[30]{allen15}.}

This underscores the broader relevance of the questions studied in this thesis. Moreover, while I generally assume that those adversely affected by the use of eminent domain do have recognised property interests, the social function perspective makes it natural to emphasise the wider societal effects of takings, including effects on non-owners.

In the next section, I consider {\it Kelo} in more depth, to argue that strict judicial deference to legislative and executive decision-makers is inappropriate in this regard. I focus especially on Justice O'Connor's dissent, which I believe suggests a stricter standard of judicial review for economic development takings, without unduly undermining the value of deference to political decision-makers and the executive branch.

\subsection{{\it Kelo}: Casting Doubts on the Narrow Approach to Judicial Review}\label{sec:2:6:1}

In many jurisdictions, constitutional property rules indicate, with varying degrees of clarity, that eminent domain should only be used to take property either for ``public use'', in the ``public interest'', or for a ``public purpose''.\footnote{For instance, a rather unclear ``public use'' formulation is used in the takings clause of the US constitution, as well as in the Norwegian constitution, while both the ``public interest'' and the ``public purpose'' formulations (but not ``public use'') are used in the South African Constitution. See \cite[462]{walt11} (arguing that while ``public purpose'' would traditionally have been understood more narrowly, there is no generally observed difference between the two notions as they are now understood in South African law).} Such a restriction can be regarded as an unwritten rule of constitutional law, as in the UK, or it can be explicitly stated, as in the basic law of Germany.\footnote{See \cite[3-4]{sluysmans15}.} In some jurisdictions, for instance in the US and in Norway, explicit takings clauses exist, but do not provide much information about the intended scope of protection.\footnote{See chapter \ref{chap:3}, section \ref{sec:3:3} and chapter \ref{chap:5}, section \ref{sec:5:2}.}

The question arises to what extent these clauses give the judiciary a duty and a right to restrict the state's power to take property. In the US, most scholars agree that some judicial review based on the public use requirement is warranted, but there is great disagreement about its extent.\footcite[205]{berger78} In Norway, on the other hand, a consensus has developed whereby the notion of public use is interpreted so widely that it hardly amounts to a justiciable restriction at all.\footnote{See, e.g., \cite[368]{aall10}.} Indeed, the courts defer almost completely to the assessments made by the executive branch regarding the purposes that may be used to justify a taking.\footcite[368]{aall10}

Some US scholars adopt a similar stance, but others argue that the public use restriction should be read as a stricter requirement, forbidding the use of eminent domain unless the public will make actual use of the property that is taken.\footnote{Compare \cite{bell06,bell09,claeys04,sandefur06}.} Most scholars fall in between these two extremes. They regard the public use restriction as an important limitation, but they also \isr{emphasise} that the courts should normally defer to the legislature's assessment of what counts as a public use.\footnote{See, e.g., \cite{merrill86,alexander05}.}

As I discuss in more depth in chapter \ref{chap:3}, section \ref{sec:3:3}, the debate in the US has its roots in case law developed by state courts -- the federal property clause was for a long time not applied to state takings. This has changed, and today the Supreme Court has a leading role in this area of US law. It has developed a largely deferential doctrine, resembling the understanding of the public use limitation under Norwegian law.\footnote{See \cite{berman54,midkiff84,kelo05}.} The difference is that in the US, cases raising the issue still regularly arise and prove controversial. As mentioned in the introduction to this thesis, the most important such case in recent times was {\it Kelo}, decided by the Supreme Court in 2005.\footcite{kelo05} This case saw the public use question reach new heights of controversy in the US.\footnote{See, e.g., \cite{somin09}.}

As mentioned in the introduction, {\it Kelo} centred on the legitimacy of taking property to implement a redevelopment plan that involved the construction of research facilities for the drug company Pfizer. The homes of Suzanne Kelo and eight other home-owners stood in the way of this plan and the city decided to use the power of eminent domain to condemn them. Kelo and the other owners protested, arguing that making room for a private research facility was not a permissible ``public use''. The owners were represented by the libertarian legal firm {\it Institute for Justice}, which had previously succeeded in overturning similar instances of eminent domain at the state level.\footnote{See \cite{justice15}.} Kelo and the other owners lost the case before the state courts, but the Supreme Court decided to hear it and assessed its merits in great detail.

The precedent set by earlier federal cases such as {\it Berman} and {\it Midkiff} was clear: as long as the decision to condemn was ``rationally related to a conceivable public purpose'', it was to be regarded as consistent with the public use restriction.\footnote{See \cite[241]{midkiff84}; \cite{berman54}.} Moreover, the role of the judiciary in determining whether a taking was for a public purpose was regarded as ``extremely narrow''.\footcite[32]{berman54} It had even been held that deference to the legislature's public use determination was required ``unless the use be palpably without reasonable foundation'' or involved an ``impossibility''.\footnote{See \cite[66]{dominion25}; \cite[680]{gettysburg96}.}

Despite this, in the case of {\it Kelo}, the court hesitated. Part of the reason was no doubt that takings similar to {\it Kelo} had been heavily criticised at state level, with an impression taking hold across the US that eminent domain abuse was becoming a real problem.\footnote{See, e.g., \cite[667-669]{sandefur05}.} A symbolic case that had contributed to this worry was the infamous case of {\it Poletown}.\footcite{poletown81} In this case, General Motors had been allowed to raze a town to build a car factory, a decision that provoked outrage across the political spectrum.\footnote{See generally \cite{sandefur05}.} The case was similar to {\it Kelo} in that the taker was a powerful commercial actor who wanted to take homes. This, in particular, served to set the case apart from {\it Midkiff}, which involved a taking in \isr{favour} of tenants, and to some extent also {\it Berman}, which involved a taking of businesses (and homes) in the interest of removing blight.\footnote{\cite{berman54,midkiff84}.} Moreover, the Michigan Supreme Court had recently decided to overturn {\it Poletown} in the case of {\it Hatchcock}.\footcite{wayne04} Hence, it seemed that the time had come for the Supreme Court to re-examine the public use question.\footnote{See, e.g., \cite{sandefur05,claeys04}.}

Eventually, in a 5-4 vote, the court decided to apply existing precedent, leading it to uphold the taking of Kelo's home. The majority also made clear that economic development takings were indeed permitted under the public use restriction, also when the public benefit was indirect and a private company would benefit commercially.\footcite[469-470]{kelo05} This resulted in great political controversy in the US. According to Ilya Somin, the {\it Kelo} case ranks among the most disliked decisions in the history of the Supreme Court.\footcite[2]{somin11} 

Importantly, many commentators emphasised that {\it Kelo} was an economic development taking.\footnote{See, e.g., \cite{somin07,cohen06}.} This category had no clear basis in the property discourse before {\it Kelo}. Indeed, in terms of established legal doctrine, it would be more appropriate to say that the case revolved entirely around the notion of ``public use''. However, when considering the most common reasons given for condemning the outcome in {\it Kelo}, it becomes clear why many felt it was natural to classify the case along additional dimensions. A survey of the literature shows that many made use of a combination of substantive and procedural arguments to paint a bleak picture of the {\it context} surrounding the decision to take Kelo's home. Important aspects of this include the imbalance of power between the commercial company and the owners, the incommensurable nature of the opposing interests, the close relationship between the company and the government, and the feeling that the public benefit -- while perhaps not insignificant -- was made conditional on, and rendered subservient to, the commercial benefit that would be bestowed on a commercial beneficiary.\footnote{See, for instance, \cite{underkuffler06,somin07,sandefur06,cohen06,hafetz09,hudson10}.} Plainly, the decision to condemn in {\it Kelo} appeared to suffer from what I will refer to here as a {\it democratic deficit}.

The social function theory of property makes it natural to emphasise the worry that economic development takings can lack democratic merit. Moreover, the theory inspires reasoning that can justify a departure from the established doctrine of extreme deference, in favour of more substantial judicial review. It seems to me that such a perspective was indeed adopted by the minority of the Supreme Court in {\it Kelo}, particularly Justice O'Connor.\footnote{\cite[494-505]{kelo05}.} She wrote a strongly worded dissent, characterising the majority's decision as follows:

\begin{quote}
Any property may now be taken for the benefit of another private party, but the fallout from this decision will not be random. The beneficiaries are likely to be those citizens with disproportionate influence and power in the political process, including large corporations and development firms. As for the victims, the government now has license to transfer property from those with fewer resources to those with more. The Founders cannot have intended this perverse result.\footcite[505]{kelo05}
\end{quote}

As demonstrated by this quote, the overarching concern raised by Justice O'Connor was that allowing takings such as {\it Kelo} could legitimise a form of governmental interference in property that would systematically \isr{favour} the rich and powerful to the detriment of the less resourceful. In this way, the power of eminent domain could become a tool for establishing and sustaining patterns of inequality, under the pretence of providing an economic benefit. Hardly anyone would openly regard this as desirable. Indeed, one of the justices who voted with the majority, Justice Kennedy, formulated a separate concurring opinion to emphasise that detailed, albeit deferential, judicial scrutiny is appropriate in cases like {\it Kelo}. According to Justice Kennedy, this is needed to rule out that the public interest in economic development is used merely as a pretext to bestow benefits on private companies.\footnote{See \cite[490-493]{kelo05}. The form of judicial review Justice Kennedy proposes is rather weak, admitting only that the courts have a role to play in cases when the government decision does not appear to be rationally related to a legitimate purpose (known as rational basis review in US constitutional law).}

The main difference between his opinion and that of Justice O'Connor does not appear to hinge on how they interpret the meaning of public use. Rather, the crucial difference seems to arise from the fact that Justice O'Connor is more willing to address injustices associated with economic development takings at the systemic level. Her perspective is clearly a powerful one, at least partially responsible also for the wide disapproval of {\it Kelo} among the public. Indeed, if Justice O'Connor's predictions about the systemic fallout of {\it Kelo} are correct, most would probably agree that the result would be ``perverse''.

The crucial question therefore becomes whether her predictions are warranted. In fact, the main importance of her dissent might be that it flags this issue as a crucial one in relation to very typical uses of eminent domain in the modern world. In light of its high level of generality, Justice O'Connor's dissent becomes a call for empirical work, to shed light on how economic development takings actually come about, and how they affect political, social and bureaucratic processes. In addition, her dissent raises the question of how to {\it avoid} negative effects, that is, how to design rules and procedures that can help bring about desirable economic development without creating a democratic deficit. These will be the main themes that I discuss in the remainder of this thesis.

\section{Conclusion}\label{sec:2:7}

In this chapter, I have proposed a theoretical foundation for approaching the question of economic development takings. Specifically, I suggested that a social function perspective on property is an appropriate starting point for an analysis of such takings. Furthermore, I argued that the notion of human flourishing provides a good template for carrying out a normative analysis of when economic development takings are legitimate. This approach, in turn, led to an argument in favour of a broader style of judicial review of such takings, namely one that embraces considerations based on social justice and the ideal of democracy.

\noo{ To illustrate the subtleties that this approach helps shed light on, I considered a concrete example of a commercial scheme that looked like it might well result in compulsory acquisition of land, namely Donald Trump's controversial plans to develop a golf course on a site of special scientific interest close to Aberdeen, Scotland. In the end, the plans did {\it not} require takings, as Trump was able to make creative use of property rights he acquired voluntarily, against the complaints of his \isr{neighbours}.

This turn of events did not make the example less relevant to this thesis. Rather, it served to highlight that the question of economic development takings is not a black-and-white balancing act between property privileges on one side and the good intentions of the regulatory state on the other. Specifically, the example of Trump coming to Scotland allowed me to \isr{emphasise} the importance of context when assessing both the nature of property, the many ways of taking, and the meaning of protecting owners against predation.

The protection sought by those who opposed Trump's golf course did not target their entitlements as individuals. Rather, it targeted the community, as the owners felt it would be detrimental to the community, and to their lives, if Trump was allowed to redefine the social functions of local property. After Trump decided not to pursue compulsory purchase, protecting the property of these members of the community became a question of {\it restricting} the degree of dominion that Trump could exercise over his own property. Hence, under a conventional and overly simplistic way of looking at these matters, protecting property became tantamount to restricting its use, a seeming paradox.

To resolve this paradox, and to arrive at a better conceptual understanding of economic development takings, I looked to various theories of property. I noted that there are differences between civil law and common law theorising about property, but I concluded that these differences are not particularly relevant to the questions studied in this thesis. In particular, I observed that neither the bundle theory, dominant in the common law world, nor the dominion theory, taught to many civil law jurists, helped me clarify economic development takings as a category of legal thought.

I then went on to consider more sophisticated accounts of property, focusing on the social function theory, which emphasises how property structures, and is structured by, social and political relations within a society. 

I went on to argue that in the first instance, the social function theory should be understood as giving us {\it descriptive} insights into the workings of property and its role in the legal order. In this regard, I advanced a different stance than many property scholars, by arguing that we should aim to decouple descriptive insights from normative aspects of the theory, to allow the social function theory to serve as a common ground for further value-driven debate.

I then went on to clarify my own starting point for engaging in such debate, by expressing support for the human flourishing theory proposed by Alexander and Pe\~{n}alver. This theory is based on the premise that property {\it should} enable -- and even compel -- individuals and their communities to  participate in social and political processes. I argued that property's purpose in this regard is  fundamental to its proper role in a democratic society, as an anchor for participatory decision-making.  

}

\noo{Moreover, I noted that the human flourishing theory contains a further important insight, concerning the scope of the state's power to protect. In particular, the theory asks us to recognise that protecting property against interference that is harmful to human flourishing is a responsibility that the state has even in cases when the individual owners themselves neglect to defend their property, for instance as a result of financial incentives to remain idle. In other words, some functions of property are such that owners have an obligation to preserve them, while the state has a duty to protect them, potentially even against the will of the owners.}

On this basis, I went on to give a preliminary analysis of economic development takings. To make the discussion concrete, I considered the case of {\it Kelo}, which propelled the notion of an economic development taking to the front of the takings debate in the US. I focused particularly on the dissenting opinion of Justice O'Connor, and I argued that she approached the issue in a way that is consistent with the theoretical basis proposed in this chapter.

In the next chapter, I will continue my analysis of economic development takings, by considering the legitimacy question in more depth. Specifically, I ask what role the law can and should play in ensuring that the state's power to take property is not used improperly in the context of economic development. This will lead to two sets of broad policy recommendations for dealing with the ills of economic development takings, targeting the diagnosis and the cure respectively. %This, in turn, will set the stage for the second part of this thesis, where the recommendations provided in the next chapter will be tested and made more concrete through a case study of takings for Norwegian hydropower development.

\noo{
how such takings are dealt with in Europe and the US respectively. I note that the category has yet to receive much attention in Europe, so the discussion focuses on the US. Here this issue has received a staggering level of attention after {\it Kelo}. To get a broader basis upon which to \isr{assess} all the various arguments that have been presented, I consider the historical background to the issue as it is discussed in the US. This involves giving a detailed presentation of the public use restriction, as it was developed in case law from the states during in the 19th and early 20th century. I then connect this discussion with recent proposals to deal with economic development takings, responding to the backlash of {\it Kelo} by aiming to address the democratic deficit of such takings.

Later, when I begin to consider the law relating to Norwegian hydropower, I will look back at the theoretical basis provided in the present chapter to guide the analysis. In particular, I focus on certain decision-making mechanisms that have developed on the ground in Norway, as a practical response to the increased tendency for local owners to engage in hydropower development. I will argue that this shows the conceptual strength of the idea that property is irreducibly embedded in community, continuously evolving alongside institutions of participatory decision-making. }