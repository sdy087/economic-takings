\chapter{Conclusion}

In this final section of my thesis, I would like to take a step back to briefly follow two broader threads that I believe run through my thesis. The first concerns the many senses of taking that have been brought into focus throughout the analysis, while the second concerns ways in which the law can help to give back some of the legitimacy that is typically lost when eminent domain is used to facilitate economic development.

\section{Property Lost -- Takings and Legitimacy}

Property can be an elusive concept, especially to property lawyers. Indeed, in the law of property, the word itself typically only functions as a metaphor -- an imprecise shorthand that refers to a complex and diverse web of doctrines, rules, and practices, each pertaining to different ``sticks'' in a ``bundle'' of rights. Indeed, this bundle perspective dominates legal scholarship, especially in the common law world. Some even go as far as to argue that words such as ``property'' and ``ownership'' should be removed from the legal vocabulary altogether. 

So is property as a unifying concept lost to the law? It certainly seems hard to pin it down. In the words of Kevin Gray, when a close scrutiny of property law gets under way, property itself seems like it ``vanishes into thin air''.\footnote{See \cite[306-307]{gray91}.} %Indeed, one may argue that different ideas of property, practical and theoretical, are behind most, if not all, the major conflicts and confrontations that have shaped the society in which we live.
%Responding to this, some prominent philosophers have taken the view that property is not a concept suitable for philosophical study at all.
%According to some philosophers, property as a concept is a lost cause, not suitable for conceptual analysis at all. Instead, these scholars have suggested that property should be taken as a pragmatic and contingent derivative of other notions, such as the social order, or, on a normative account, by regarding it as an expedient of {\it justice}.
%Moreover, legal scholars are usually content with theories of property that remain largely descriptive, settling for the more modest aim of exploring how best to think of property given the prevailing legal order, rather than trying to come up with theories to explicate its nature as a pre-legal concept. Indeed, even legal philosophers are sometimes found doubting that there even is such a thing as property.
Arguably, however, property never truly disappears. Indeed, there is empirical evidence to suggest that humans come equipped with a {\it primitive} concept of property, one which pre-exists any particular arrangements used to distribute it or mould it as a legal category.\footnote{See\cite{stake06}.} Perhaps most notably, humans, along with a seemingly select group of other animals, appear to have an innate ability to recognise {\it thievery}, the taking of property (not necessarily one's own) by someone who is not entitled to do so.\footnote{See \cite[11-13]{brosnan11}.}

Taken in this light, Proudhon's famous dictum ``property is theft'', might be more than a seemingly contradictory comment on the origins of inequality. It might point to a deeply rooted aspect of property itself, namely its role as an anchor for the distinction between legitimate and illegitimate acts of taking.

But what is a taking, and when is it legitimate? In this thesis, I will aim to make a contribution to this question. I will study takings of a special kind, namely those that are implemented, or at least formally sanctioned, by a government. In legal language, especially in the US, such acts of government takings are often referred to as takings {\it simpliciter}, while talk of other kinds of ``takings'' require further qualification, e.g., in case of ``takings'' based on contract, tax or occupation. 

The US terminology brings the issue of legitimacy to the forefront in an illustrative manner. We are reminded, in particular, that under the rule of law, taking is not the same as theft. Rather, the default assumption is that the takings that take place under the rule of law are legitimate. If they are not, we may call them by a different name, but not before. At the same time, it falls to the legal order to spell out in further detail what restrictions may be placed on the power to take. 

Indeed, restrictions appear implicit in the very notion of taking. The idea that someone might find occasion to resist an act of taking, and may or may not have good grounds for doing so, appears fundamental to our pre-legal intuitions. But how should we approach the question of legitimacy of takings from the point of view of legal reasoning, and what conceptual categories can we benefit from when doing so? This is the key question that is addressed in this thesis.

\section{Property Regained -- Givings and Participation}

\noo{
\section{Many Aspects of Taking}

The most obvious way to describe a taking is to say that it involves the transfer of property from one legal person to another. However, as I noted in the first chapter of this thesis, property itself is highly multifaceted, serving a range of social and individual functions. Hence, when we begin to unpack the property bundle, we are confronted with a multitude of different senses in which a taking impacts on owners, their communities, and society as a whole.

The economic consequences of a taking might be the most easily recognisable, particularly in the economic development cases. But as my work in this thesis has shown, other consequences can be just as important, particularly those pertaining to property as an anchor for local democracy. If jointly owned property is taken from a community, with full compensation paid to all individual owners, the community suffers a distinct uncompensated loss, namely the loss of future self-governance opportunities.

In the traditional narrative on takings, social and political effects are typically only recognised on one side of the takings equation, namely the side of the taker, particularly the public interest. This has also influenced the debate on economic development takings. In order to make sense of the broader sense of unfairness often associated with such takings, critics tend to focus on the taker rather than the owner, by questioning the legitimacy of the motives behind the taking.

However, this might be tantamount to shifting a variable to the wrong side of the takings equation. In particular, the feeling of unfairness associated with economic development takings clearly arise from a sense in which the owners are victims of an abuse of power. So why shift attention to the taker? 

Perhaps it is tempting to do so simply because the sense of unfairness at work here pertains to a broader notion of justice than that normally associated with property interests. If so, the entire narrative points to a shortcoming of the liberal idea of property. If even property's staunchest defenders must turn to notions of ``public interest'' (and the lack thereof), then why do we need property as a concept at all? Why not simply say that a licence to undertake economic development should not be granted unless all affected parties agree, or the public interest is sufficiently strong to go ahead against some of their wishes? What makes property special in this picture, if all that is at stake is the strength of the public interest used to justify imposing the state's will on private individuals?

Clearly, the gaping hole in the opposition to economic development takings in the US has been a {\it positive} account of why property is worthy of protection in the first place, in cases where economic rationality dictates that it should be put to more profitable use. If the public interest is regarded as insufficient, it must be because there is something valuable inherent in property that raises the threshold for taking property above a certain level.

Such is the conventional narrative, that the owner as an individual suffers a loss in order for the taker, society as a whole, to achieve democratically determined political goals. But in economic development cases, the picture is quite different. In these cases, it is often the case that local communities are deprived of political capital in order for specific commercial interests to make a profit. 

In such cases, it might well be that the balancing of different reasons for and against the taking has taken place prior to the decision to interfere with property. The plans for development themselves may well precede any specific property-oriented implementation steps, such as the use of eminent domain. It might even be that democratically accountable bodies responsible for land use planning have already concluded that some local community interests must give way to other interests.

In these cases, it might be tempting to argue that a narrow takings narrative is appropriate because it pertains only to the final implementation step, which is the only one that involves property rights. But this argument, I believe, rests on a flawed perception of what property is, and should be, in a democratic society. Invariably, property has to do with decision-making and power. If the decision-making process does not grant significant self-determination rights to affected property owners, a taking is already in progress. It might be justified, but it is still a taking. 

More worryingly, it is clear that this kind of taking carries with it a great potential for differential treatment, discrimination, and corruption. The traditional takings narrative does a good job of setting up a framework that makes it difficult to simply pay higher compensation to certain kinds of people, without offering any justification. But with respect to the aspects of taking not recognised, e.g., pertaining to what role the owner has during the planning stages, differences in treatment will not even be notices. But if property is owned by the right sorts of people, then invariably it {\it will} come with considerable decision-making power. 

If property is owned by the marginalised, on the other hand, the most severe act of taking will sometimes have taken place even before the land use planning begins, by the fact that the owners are placed entirely on the sidelines. This, I argued in Part II of this Chapter, is how the Norwegian system for management of hydropower approach riparian owners works. 

At the very outset of planning, often decades before any formal decision to expropriate has been made, a considerable portion of the substance of property is taken from the owners, who are completely excluded from the rest of the decision-making process. In Norway, such takings processes, that clearly transcend the traditional financial narrative, have progressed to the point that even the law today provides an ambiguous account of Norwegian water resources as private property belonging to the general public.

Perhaps, then, the nature of property itself has changed, so that there is nothing left except those financial entitlements that Norwegian expropriation law recognised. If so, the change has not come about by any legislative move, nor has it been preceded by any kind of debate. It has simply emerged, gradually and unplanned, as a result of sector-based regulation and administrative practices. The process, therefore, meets neither the requirements of land reform or expropriation. It is an unacknowledged process about which the law in Norway has had nothing much to say at all, for which silence still persists. 

This is unfortunate. Even if riparian rights should be stripped of all content except a financial entitlement, this should happen on the basis of debate and democratic decision, not because the law fails to cater to a descriptively accurate notion of property.


not how a taking should be carried out, nor does it meet the standard 




 the decision-making process {\it will} normally reflect the power of property, if other contextual factors are not 

although to different degrees depending on other contextual factors, most notably the social status of the owner. Hence, property plays a constuti


particularly 


Property is not merely a placeholder for transient entitlements. It also both help make up and is shaped by the social and political context. 


 This was the key point that I argued for in Chapter 1 of the thesis, by looking to the social function theory of property and the notion of human flourishing. 


which focuses specifically on property rights only {\it after} the public interest has been mapped out and formulated by the decision-makers? 



 individual financial entitlements 

However, the property perspective, which is imposed onl

As I have demonstrated in this thesis, the economic development takings turn this narrative completely on  xplored in depth in this thesis, the con

Rather, the typical narrative places such aspects

After all, on the taker side, many of the primary concepts used to conceptualise typical takings are neither economic nor individualistic.

 on the owner side, do not 

The consequence of this can be that former owners and their communities will be marginalised more generally, as their position within society weakens. In turn, it will become easier to take more property from them, under increasingly weaker arguments of public interest. In the end, when egalitarian property rights no longer provide a foundation for decision-making about land use, the risk is high that a corresponding inequality in decision-making power will follow quite generally. Democracy as such might be at risk.

In any event, the land-less will not have a voice unless they can find different means of asserting themselves. The possibility of achieving participatory equality without egalitarian property should not be overlooked, of course. However, it seems safe to say that the track record of alternative ideas, whereby equality is pursued through institutional arrangements alone, is unimpressive. 

In almost all countries that score well on parameters such as democracy, living standard, transparency, and the rule of law, we find private property rights as a core legal principles. Moreover, while property might be unequally or unfairly distributed among the population, property rights are typically distributed widely enough to give rise to a natural division of power and a plurality of perspectives. Indeed, even the land-less may sometimes attain a voice, albeit a very limited one, if they are still in possession of their own labour.

The negation of property rights

This, in turn, is the very foundation of both democracy and the rule of law.



 as an underlying source of division of power.


 
\section{Some Ways of Giving Back}

\subsection{Locating Primary Stakeholders; The importance of Communities}

\subsection{Making Influence Proportional to Stakes; the Closeness-to-Consequences Test}

\subsection{Robust and Flexible Institutions for Collective Action; the Possibility of a Judicial Approach}

\subsection{Beware of Big Units; the Fine Line between Representation and Usurpation}

\subsection{The Importance of Redundancy; Property Regained}

It seems that property dislikes being concentrated in the hands of the few. }