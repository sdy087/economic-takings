\chapter{Conclusion}

%\begin{quote}
%Gode bønder nå til motstand reises. Enhver som må gi fra seg farsarvi til kongens grever mener det er det rene ran. \\
%Good farmers prepare to resist. Those who must give up their father's inheritance to the lords of the king consider it plain robbery.
%\end{quote}

%There is reason to think that fraudulent behaviour requires some, but not too much, in the way of intellectual refinement. Arguably, therefore, the case can be made that 

%The concept of property has received its share of criticism, but it remains omnipresent.

%
\noo{ t. Further to this, legitimacy-enhancing alternative to expropriation were considered. Specifically, the thesis argued that ideas taken from the research on common pool resources could inspire novel solutions for local self-governance that could obviate the need for using eminent domain to ensure economic development. 



%A concrete proposal along this line, due to Heller and Hills, was considered in some depth. 

%An important tension was identified within this proposal, between the ideal of self-governance and the danger of abuse at the local level. This, in turn, reinforced one of the key design principles of common pool resource management, namely that institutional arrangements need to be sensitive to the local context. This set the stage for the next part of the thesis, where the issue of economic development takings was looked at concretely, from the point of view of waterfall expropriation foor hydropower development in Norway. 

%Hence, to formulate a one-size-fits-all institution to replace eminent domain for economic development is unlikely to work. At least, such a solution would have to be far more flexible than the proposal made by Heller and Hills, which in the end offers owners very little in the way of participation in decision-making regarding economic development on their properties.



In such cases, it might well be that the balancing of different reasons for and against the taking has taken place prior to the decision to interfere with property. The plans for development themselves may well precede any specific property-oriented implementation steps, such as the use of eminent domain. It might even be that democratically accountable bodies responsible for land use planning have already concluded that some local community interests must give way to other interests.

In these cases, it might be tempting to argue that a narrow takings narrative is appropriate because it pertains only to the final implementation step, which is the only one that involves property rights. But this argument, I believe, rests on a flawed perception of what property is, and should be, in a democratic society. Invariably, property has to do with decision-making and power. If the decision-making process does not grant significant self-determination rights to affected property owners, a taking is already in progress. It might be justified, but it is still a taking. 

More worryingly, it is clear that this kind of taking carries with it a great potential for differential treatment, discrimination, and corruption. The traditional takings narrative does a good job of setting up a framework that makes it difficult to simply pay higher compensation to certain kinds of people, without offering any justification. But with respect to the aspects of taking not recognised, e.g., pertaining to what role the owner has during the planning stages, differences in treatment will not even be notices. But if property is owned by the right sorts of people, then invariably it {\it will} come with considerable decision-making power. 
}


\noo{
This has been a thesis in two parts that has addressed economic development takings in two very different ways. In the first part, a theoretical discussion was provided, which started from the notion of property itself and gradually made its way towards the question of legitimacy of takings by exploring a broader meaning of property than that typically embraced by the law. The aim was to argue {\it why} property should be protected, while also providing a template for recognising and discussing special issues that arise for economic development takings. 

Following up on this, the thesis explored different approaches to legitimacy, culminating in a concrete proposal for a legitimacy test, inspired by the work of Kevin Gray. Furthermore, the thesis presented a template for formulating alternatives to outright expropriation, inspired by the work of Elinor Ostrom and others on local self-governance of common pool resources. The aim here was to suggest possible ways to restore legitimacy in cases when an either-or approach to legitimacy of property interference is not appropriate. Better than that it should be interference bottom-up than interference top-down, provided adequate safeguards are put in place to protect local minorities from abuse.

The discussion remained quite abstract in the first part of the thesis. By contrast, the second part of the thesis approached the question from the opposite angle, through an in-depth case study of takings for hydropower development in Norway. The thesis first presented and discussed the social, economic, and legal context of such takings, before proceeding to study practices and rules relating to expropriation of waterfalls in greater depth. This also provided an opportunity to apply the legitimacy test proposed in the first part of the thesis. The conclusion was that current expropriation practices in Norway fail in this regard, as the property rights of local people are very weakly protected in situations when large commercial companies wish to undertake hydropwoer development.

However, the thesis went on to observe that the land consolidation procedure represents a possible alternative to expropriation, one that is already being used extensively in Norway to facilitate owner-led hydropower development. Here the emphasis is on organising joint use of property, possible involving compulsion whereby owners must partake in development against their will. The procedure is judicial in nature, moreover, so also provides safeguards against abuse by local elites. Moreover, it is primarily a service to the owners and their properties, but is required to actively promote solutions that are in the public interest. After recent changes in the law, the scope of obligations that can be imposed on owners for the common good is also likely to increase, making the land consolidation alternative appear like a realistic option even in large-scale development situations that will necessarily also involve non-local actors. The thesis argued that land consolidation is a good example of the kind of institution that can function as an alternative to expropriation in hard cases. It has already proven itself in a setting of egalitarian property, in communities where property rights are held by local people. The possibility of employing it in situations where this is not the case remains more uncertain, but is an interesting prospect. Arguably, this would require including non-owners in the process as well, on the basis of connections to property that might not otherwise be recognised as property interests in the law. Specifically, it seems that one might want to rely on broader social function ideas of property in relation to consolidation, even if a more traditional account is maintained in other areas of the law. In relation to financial law or tax law, one might well wish to entertain an artificially narrow concept of property for efficiency reasons, even if a much broader concept is required in the context of compelled economic development. Moreover, the land consolidation procedure appears attractive because it fills a gap between planning and property, a gap that otherwise appears susceptible to infiltration by powerful commercial interests.

%The thesis made the case that these issues are important enough to suggest that economic development takings should be approached as a special category, also in the law.

%From this, the theoretical discussion continued by an exploration of different ways to approach the legitimacy question for such takings. 
}

%In this final section of my thesis, I would like to take a step back to briefly follow two broader threads that I believe run through my thesis. 
This has been a thesis in two parts, each of which have approach the issue of economic development takings, but which have done so in two very different ways. The first part took a theoretical approach, starting from the notion of property itself, to answer the question of {\it why} it should be protected. This, in turn, gave rise to a framework for assessing the legitimacy of economic development takings, and for formulating alternatives to it that could obviate the need to dispossessing current owners.

The second part of the thesis approached the issue of legitimacy concretely, by giving a case study of takings of waterfalls for hydropower development in Norway. The political, social and economic context was also analysed, leading to an application of the Gray test formulated in the first part of the thesis. Moreover, the case study went on to consider the possibility of alternatives to expropriation, by assessing the Norwegian institution of land consolidation, which is now used extensively by local owners who wish to take hydropower development themselves.

% The current approach to takings for waterfalls in Norway were found wanting, with the current takings practices appearing to fail several, if not all, the points set out by the Gray test. However, the final chapter of the thesis studied a possible alternative to expropriation, which paradoxically is also actively used in the context of hydropower development in Norway. This framework, however, is so far used only when some of the local owners themselves wish to carry out development, and need to sort out their internal disagreements, possible even by compelling unwilling neighbours to join them in their endeavours. 

%The thesis argued that this alternative, although not necessarily applicable in other contexts, provides an interest

%More generally, and especially in its proposal for a possible solution, I hope the thesis has made a valuable contribution to the study of economic development takings. 

To conclude the thesis, I will now take a step back to consider two broader threads that I believe runs through my work, pertaining to the nature of property and how to maintain it as force for good in the world. The discussion is preliminary, but after the work done in this thesis, the reader will hopefully agree that it is more than mere speculation.

%The first concerns the many senses of taking that have been brought into focus throughout the analysis, while the second concerns ways in which the law can help to give back some of the legitimacy that is typically lost when eminent domain is used to facilitate economic development.

\section{Property Lost -- Takings and Legitimacy}

The law does not like it when things get too theoretical. After all, the law is not in the business of settling philosophical debates. Rather, its main responsibility is to deliver effective management of  disputes, involving concrete legal persons and/or governments. Arguably, this has played a significant role in shaping the traditional approach to the legitimacy issue in the law of takings.

It is clear that by focusing on the individual losses of owners following a taking, the law makes things easier for itself. Moreover, by assuming that all losses can be quantified in financial terms, the law can automate its approach to takings. The potentially broad question of legitimacy has been reduced to the question of how to award compensation, for which the ostensibly neutral standard of ``market value'' appear to provide a natural starting point. A large chunk of the remaining work, in turn, can be delegated to the appraisers, allowing the courts to get on with other business. In addition to being effective, this also comes with the added bonus of allowing the courts to distance themselves very clearly from the political undertones of the legitimacy question.

This simplified approach to legitimacy is prevalent, but I believe it needs to be rejected. The reason, as argued in the first part of this thesis, is that property itself cannot be drawn up as narrowly as this traditional approach to legitimacy presupposes. Property interference cannot be understood in mechanical terms, as a matter for the appraises, regardless of how much weight we want to place on the value of efficiency and the ideal of deference to political decision-making. Specifically, unless the issue of legitimacy is recognised as being far more complex than pertaining solely to the question of calculating market values, there will be a significant mismatch between what property is and what the law pretends it to be.

In a setting were takings are rare, taking place only in extraordinary situations, such a mismatch might be tolerable. Arguably, the strong commitment to the sanctity of property by early writers such as Blackstone, apparently conflicting with historical records about takings practice in their day, could be sustained precisely because takings were the exceptions that made up the norm.  Blackstone's claim to accuracy could be simply this: that expropriation took place so seldom that  certain concessions to expediency could be made when it did happen, without this detracting from the greater picture of property as a sacred right.

Such a narrative is no longer plausible in a world where the state has expanded its activities so much that interference in private property, rather than being the exception, has become the norm. Hence, it now seems necessary to also let the complexities of property as a social phenomenon enter the narrative of property in the law, especially in the law of takings. This is particularly important if the idea of property as a fundamental right is to have a future. If the law continues to insist that property is nothing more than a form of entitlement protection, there is perhaps even a case to be made that the notion should just be done away with in its entirety.

In the first part of this thesis, I presented a theory of property which I think suggests that this would be a tragedy, particularly for marginalised groups who are in need of protection against economic and social elites. Importantly, while international law and human rights conventions offer important clarification and protection at the level of principles, what property provides is an imperfect, yet very powerful, framework for implementing such principles at the local level. This is property's promise, which it can only keep if it is recognised as having social functions going well beyond the protection of individual entitlements.

Arguably, principles of human rights should even be recognised as inhering in property as such, not only as mediated by the power of states. If owners fail, it should not be an excuse for the state. However, the opposite is equally true: if states fail, then owners still have obligations. If there is no distinction between owners and states, by contrast, failure in one is also failure in the other. This is dangerous, particularly if proprietary power is exercised only by a small group of people, be they commercial leaders or powerful government officials. Arguably, property should therefore be widely distributed among the population, as a guardian of basic rights and an anchor of democracy.
 
The first chapter of this thesis explored this idea in depth and argued that a broader notion of property needs to be acknowledged also by the law, particularly in the law of takings.\footnote{This is so not only because it can block takings that are offensive to social functions. It is also true because it can help increase the legitimacy of takings and other property interferences that serve such functions because they are grounded in fundamental norms of property rather than the politics of power groups.} It is worth pausing to recognise that the bundle of rights theory did us a favour in this regard, in that it directed our attention at the multifaceted nature of property. However, to make progress, it was necessary to further unpack the property bundle, to get at the substantive content of property in life: social functions as opposed to legal abstractions.

\noo{ Moreover, I argued that while private property might often be found wanting, it remains a potentially powerful force for good in the world. As discussed in Chapter 1, its roles as a building block of democracy and a protector of communities is particularly important in this regard. In addition, I discussed the importance of social obligations arising from property, and how they can potentially function as a guarantee that the basis rights of non-owners will be delivered at the local level.

If these aspects are recognised and embraced as a crucial part of the concept of property in the law, it should hopefully go some way towards restoring confidence that property is neither theft nor fraud, but a promise to work hard for a better future for all. By contrast, the idea of property as a financial entitlement do not appear to offer any such relief. If anything, dismissive attitudes to property will take their fuel from the idea of property as entitlement; entitlements, after all, can often appear undeserved, especially when they are not checked by corresponding obligations. 
}
%In this way, a threat emerges to the stability of the concept of property itself, as a legitimate part of the social and political order.
%In reality, of course, owning property has nothing to do with what one deserves, but rather what task one has been allotted on this earth, to pursue in keeping with one's beliefs and convictions.

%In cases involving regulation of property use, it might still be possible to keep this aspect away from undermining the concept of property as such. At least, it might ensure that only those well versed in the technical details of the law are able to recognise property for the ``phantom'' that it appears to be.

%However, when property is taken outright, even this containment strategy is bound to falter. This is particularly clear if taking become increasingly prevalent not only in situations of pressing public need, but also as a means for companies to turn a profit. In such a setting, even the most naive observer would surely be tempted to think that property as a concept must be altogether rather vacuous.  


%The impression that private property, in the end, is nothing but a shorthand to describe a special class of liability rules, leaves property open to further attack. Indeed, if property and ownership has only such a thin content, why worry about interfering with it in the public interest? At this point, however, it seems prudent to take a step back, to reconsider the origin of the feeling that property is no more than theft, or no better than fraud. Specifically, it might be appropriate to note that unlike property as a concept, the act of taking it without its owner's consent is quite likely to involve both theft or fraud as a matter of fact, not merely a manner of speech.

%In the second part of this thesis, I have explored this in further depth, by analysing the law and practices relating to the expropriation of waterfalls in Norway.

%However, as I noted in the first chapter of this thesis, property itself is highly multifaceted, serving a range of social and individual functions. 

The first chapter set out to do this, in order to get at the multitude of different ways in which a taking can impact on owners, their communities, and society as a whole. The economic consequences of a taking might be the most easily recognisable, particularly in the economic development cases. But as I have argued in this thesis, other consequences can be just as important, particularly those pertaining to property as a building block of communities. If jointly owned property is taken from a community, with full compensation paid to all individual owners, the community suffers a distinct uncompensated loss, namely the loss of future self-governance opportunities.

%In the traditional narrative on takings, social and political effects are typically only recognised on one side of the takings equation, namely the side of the taker, particularly the public interest. This has also influenced the debate on economic development takings. In order to make sense of the broader sense of unfairness often associated with such takings, critics tend to focus on the taker rather than the owner, by questioning the legitimacy of the motives behind the taking.

%However, this might be tantamount to shifting a variable to the wrong side of the takings equation. In particular, the feeling of unfairness associated with economic development takings clearly arise from a sense in which the owners are victims of an abuse of power. So why shift attention to the taker? 

%Perhaps it is tempting to do so simply because the sense of unfairness at work here pertains to a broader notion of justice than that normally associated with property interests. If so, the entire narrative points to a shortcoming of the liberal idea of property. If even property's staunchest defenders must turn to notions of ``public interest'' (and the lack thereof), then why do we need property as a concept at all? Why not simply say that a licence to undertake economic development should not be granted unless all affected parties agree, or the public interest is sufficiently strong to go ahead against some of their wishes? What makes property special in this picture, if all that is at stake is the strength of the public interest used to justify imposing the state's will on private individuals?

%Clearly, the gaping hole in the opposition to economic development takings in the US has been a {\it positive} account of 

%This part of the thesis focused on coming up with an answer as to why property is worthy of protection in the first place, in cases where economic rationality appears to dictate that it should be put to more profitable uses. If there is a reason to resist this, it must be because there is something valuable in property that the law should protect, irrespective of the current owner's financial entitlements.

Moreover, the thesis argued that the dynamics of power in takings cases need to receive more attention: the practice of economic development takings can result in local communities being deprived of highly valuable political capital in order for politically powerful commercial interests to make a profit. I noted that this was also the main concern raised by Justice O'Connor in her {\it Kelo} dissent. In this way, the social function view is arguably also implicit in one of the most forceful voices that have spoken out against economic development takings on the basis of constitutional property law.

However, the thesis also noted a weakness with the typical approach to legitimacy in the US, via the public use requirement. Specifically, this requirement does not appear to get us very far towards a justiciable restriction on the takings power along the lines of reasoning adopted by Justice O'Connor. Unlike the majority in {\it Kelo}, building on recent precedent, and Justice Thomas, building on the original meaning of the public use restriction, Justice O'Connor's more institutionally oriented reasons for rejecting the taking appeared to lack a firm basis in law.

To address this, I pointed to recent developments at the ECtHR, where an institutional perspective on fairness appears to be developing, which might be more likely to embrace the social function perspective on property as a basis that can support a justiciable restriction on the states' takings power. At the same time, the position of the Court in Strasbourg might be conducive to an approach that can adopt broad scrutiny in controversial cases without becoming too entangled with the politics of those cases at the state level. Specifically, the value of deference could be given a firmer expression as a norm that compels recognition of diversity and local democracy, not a norm that calls for passivity or loyalty to governments in politically sensitive situations.

Following up on this, the thesis went on to formulate a legitimacy test based on a set of conditions formulated by Kevin Gray. Three additional points were added to this list, emphasising the regulatory context, the position of non-owners, and the broader issue of democratic merit. With this approach to property and legitimacy in mind, the thesis investigated waterfall expropriation for hydropower development in Norway. The thesis looked not only to the law, but also to administrative practices, as well as the social, economic, and political context surrounding hydropower development in Norway.

Importantly, waterfall rights are privately owned in Norway, typically by members of the rural communities in which the water resources are found. This group of owners is not politically or economically powerful in Norway. Quite the contrary, it is arguably the most marginalised group in Norwegian society, and many rural communities are threatened by depopulation. Locally organised hydropower development is often the only growth industry in these communities.

In keeping with the theoretical basis formulated in Part I of this thesis, this setting needs to be taken into account when addressing the legitimacy of waterfall takings in favour of for-profit energy companies. Specifically, the context of these takings in Norway raises the bar that needs to be passed in order for a taking to count as legitimate. Despite this, the thesis showed how expropriation of waterfalls was so seamlessly integrated into the resource management framework that it hardly receives any attention at all. Rather, expropriation of waterfalls is typically an automatic consequence of a large-scale development license. There is no requirement, in particular, that the applicant needs to have any private rights prior to applying to develop hydropower.

Moreover, the procedural rules and practices governing the licensing process ensure that the owners are placed entirely on the sideline during the planning and application assessment stages. Indeed, the water authorities do not even recognise a duty to inform owners that an application for a development license is under assessment, not even when a license will include an authority to expropriate waterfalls. At the very outset of planning, often decades before any formal decision to expropriate has been made, a considerable portion of the substance of property is in fact already taken from the owners. This is because the water authorities observe the policy that they refuse to process applications from local owners as long as there are plans for large-scale development by a larger energy company.\footnote{Arguably, depending on the formal status of the planning involved, this policy has a legal basis in section 22 of the \cite{wra00}.} 

In Norway, therefore, waterfall takings clearly transcend the traditional financial narrative. In fact, takings practices have become so broad and intrusive that the law itself is ambiguous about whether private ownership of water resources have any real content at all. This might be a rather extreme case, where expropriation orders are so easily granted as to shed doubt on the very nature of property rights in the underlying resource. It is important to note, however, that there is nothing inevitable about this state of the law. Indeed, the historical context shows that in Norway, where water is anything but scarce and most rivers are non-navigable, property rights in waterfalls (as opposed to water as a substance), was long recognised as on par with property rights in land. Expropriation to pursue hydropower development was not permitted under any circumstance until the early 20th century. 

Moreover, despite property being undermined, it is not as if proprietary power is no longer exercised over water resources. Quite the contrary. The commercial companies that acquire control over waterfalls through licenses turn them into commodities whose primary purpose is to turn a profit. However, waterfalls are now encapsulated in development licenses, which have now become the crucial assets in the hydropower sector. The result is that the notion of property backing up this regime is now so thin that it arguably cannot be distinguished at all from the raw political and economic power that backs it up and redistributes it at will. As such, it is also no wonder that property narratives give way to narratives based on thinking about water resources as a common resource. This remains the abstract idea, moreover, while the practice is to let these resources be managed according to the commercial logic of the markets.

Due to this dynamic, the case of Norwegian waterfalls is perhaps an example of how illegitimate takings can do more harm than simply deprive some owners of valued resources. If legitimacy is not ensured and takings become systematic to the point of automation, not only will property be taken, it will eventually also be lost, stripped of all social functions except those pertaining to the commercial interests of its powerful owners. Such a system might well leave us with the impression that property is indeed little more than theft, maintained in the law only as a fraud.

%The best one can do within such a system might be to call the bluff of property, as born of theft and %raised by fraud. 
% leaving us with the impression that it is indeed little more than theft, maintained in the law only as %a fraud.

\noo{ However, given the historical context Perhaps, then, the nature of property itself has changed, so that there is nothing left except those financial entitlements that Norwegian expropriation law recognised. If so, the change has not come about by any legislative move, nor has it been preceded by any kind of debate. It has simply emerged, gradually and unplanned, as a result of sector-based regulation and administrative practices. The process, therefore, meets neither the requirements of land reform or expropriation. It is an unacknowledged process about which the law in Norway has had nothing much to say at all, for which silence still persists. 

%Property can be an elusive concept, especially to property lawyers. Indeed, in the law of property, the word itself typically only functions as a metaphor -- an imprecise shorthand that refers to a complex and diverse web of doctrines, rules, and practices, each pertaining to different ``sticks'' in a ``bundle'' of rights. Indeed, this bundle perspective dominates legal scholarship, especially in the common law world. Some even go as far as to argue that words such as ``property'' and ``ownership'' should be removed from the legal vocabulary altogether. 

%So is property as a unifying concept lost to the law? It certainly seems hard to pin it down. In the words of Kevin Gray, when a close scrutiny of property law gets under way, property itself seems like it ``vanishes into thin air''.\footnote{See \cite[306-307]{gray91}.} %Indeed, one may argue that different ideas of property, practical and theoretical, are behind most, if not all, the major conflicts and confrontations that have shaped the society in which we live.
%Responding to this, some prominent philosophers have taken the view that property is not a concept suitable for philosophical study at all.
%According to some philosophers, property as a concept is a lost cause, not suitable for conceptual analysis at all. Instead, these scholars have suggested that property should be taken as a pragmatic and contingent derivative of other notions, such as the social order, or, on a normative account, by regarding it as an expedient of {\it justice}.
%Moreover, legal scholars are usually content with theories of property that remain largely descriptive, settling for the more modest aim of exploring how best to think of property given the prevailing legal order, rather than trying to come up with theories to explicate its nature as a pre-legal concept. Indeed, even legal philosophers are sometimes found doubting that there even is such a thing as property.
%Arguably, however, property never truly disappears. Indeed, there is empirical evidence to suggest that humans come equipped with a {\it primitive} concept of property, one which pre-exists any particular arrangements used to distribute it or mould it as a legal category.\footnote{See\cite{stake06}.} Perhaps most notably, humans, along with a seemingly select group of other animals, appear to have an innate ability to recognise {\it thievery}, the taking of property (not necessarily one's own) by someone who is not entitled to do so.\footnote{See \cite[11-13]{brosnan11}.}

%Taken in this light, Proudhon's famous dictum ``property is theft'', might be more than a seemingly contradictory comment on the origins of inequality. It might point to a deeply rooted aspect of property itself, namely its role as an anchor for the distinction between legitimate and illegitimate acts of taking.

%But what is a taking, and when is it legitimate? In this thesis, I will aim to make a contribution to this question. I will study takings of a special kind, namely those that are implemented, or at least formally sanctioned, by a government. In legal language, especially in the US, such acts of government takings are often referred to as takings {\it simpliciter}, while talk of other kinds of ``takings'' require further qualification, e.g., in case of ``takings'' based on contract, tax or occupation. 

%The US terminology brings the issue of legitimacy to the forefront in an illustrative manner. We are reminded, in particular, that under the rule of law, taking is not the same as theft. Rather, the default assumption is that the takings that take place under the rule of law are legitimate. If they are not, we may call them by a different name, but not before. At the same time, it falls to the legal order to spell out in further detail what restrictions may be placed on the power to take. 

%Indeed, restrictions appear implicit in the very notion of taking. The idea that someone might find occasion to resist an act of taking, and may or may not have good grounds for doing so, appears fundamental to our pre-legal intuitions. But how should we approach the question of legitimacy of takings from the point of view of legal reasoning, and what conceptual categories can we benefit from when doing so? This is the key question that is addressed in this thesis.
}

\section{Property Regained -- Givings and Participation}

The converse of a taking is a {\it giving}. In the US, this term is used to refer to situations when private property owners benefit from state actions involving property. For instance, it might be characterised as a giving if the state allows someone to purchase property cheaply, or if regulation makes some property appreciate in value. Arguably, and analogously to the case of takings, there is a case to be made that private owners should normally be obliged to pay for givings from the state.

The issue of when this is appropriate, if at all, will not be discussed here. Instead, I wish to direct attention at the terminology itself, and its subtle conceptual commitment to a top-down way of thinking about both takings and givings. Indeed, consider what happens if we turn the terminology on its head. This is not an implausible shift of attention. After all, if the state takes property, the current owners will have to give it up.

The owners' act of giving, however, is rarely if every given any recognition or attention in the way we approach takings. Plainly, the owners' active participation as a giver -- not merely an injured party -- is considered irrelevant. Why is that? The obvious answer is that since the giving takes place under compulsion, it does not express any intention to give. However, there are many situations in life where actions are compelled, but where the person taking that action still gets some credit for it.\footnote{Paying taxes, for instance, is typically associated with being a good citizen.} Moreover, the owners can clearly decide to be more or less cooperative when faced with the government's wishes for their property. 

By shifting attention towards the choices that the owners have in this regard, we can arguably also find a path towards legitimacy that does not involve giving up all the state's power to compel owners in the public interest. Indeed, even a purely symbolic recognition of the owners' role and the importance of their choices in dealing with a takings request, can serve to enhance subjective legitimacy. However, quite apart from recognising more fully the constructive role that owners can already play, this way of thinking also raises the possibility that they should be granted more choices and asked to take part more actively in the proceedings.

%The owners make an active contribution to the public purpose, instead of being regarded as obstacles to it. This, in turn, can become a starting point for coming up with arrangements where the owners are permitted to take up a more lasting interest also in the new use of the property that the public desires. 

In cases of economic development, this way of thinking seems particularly appropriate. As they contribute to the project by giving their property, it seems only fair that the owners should have a stake also the planning and the continued use of their property as a commercial asset. For instance, it might be appropriate to offer owners shares in the development company, or even to make sure that the property is taken under a leasehold rather than by a full transfer of title. Better yet, why not allow the owners themselves to deliberate on how they wish to honour their commitment to the public, to formulate their own plans and implement their own solutions, based on continued interaction both among themselves and with representatives from the collective.

This is a vision of the role that owners can play that also arises quite naturally from the idea that contributing is an obligation, and that responding to the public interest by committing one's property to socially and economically productive uses should be recognised as a giving rather than a taking. Interestingly, as this thesis has shown, such a highly abstract and idealistic idea is present in the system of land consolidation that we find implemented on the ground in Norway.

The case study conducted on this point was also linked to the institutional alternatives to expropriation considered in the first part of the thesis. There it was argued that local self-governance is a good way to obviate the need for expropriation altogether. This is also usually the premise in the context of land consolidation. However, the land consolidation proposal also leaves greater room for the public to {\it compel} owners to come together and participate in specific endeavours. If a proposed development is judged as being beneficial to their properties, the owners may be stripped of their holdout power, not only as individuals but even as a group.

In light of this, land consolidation can indeed replace expropriation, provided our understanding of what property is, and should be, is broad enough to say that the purpose is also in the interest of the properties involved. This limitation is interesting at a conceptual level, because it makes the conceptual content of property directly relevant to determining the extent of the government's power to interfere with it. However, in the context of economic development, this restriction is rarely going to raise many questions. If development is both economically beneficial and represents a sustainable use of resources, the land consolidation courts should have no difficulty justifying consolidation measures on the basis that development is also in the interest of the properties involved.

At the same time, the context of land consolidation, with its emphasis on problem solving and owner participation, means that process will be far more inclusive towards owners than traditional expropriation proceedings. Under consolidation, it is truly more appropriate to speak of givings than to speak of takings. Moreover, what is given in these cases is not normally the property as such, but the right to determine how it is to be used, with the public having to turn to the current owners for help in realising the public purpose. It would still be possible to expropriate, or to rely on a mix between expropriation and consolidation. In Norway, the relationship between the two is not completely clarified in the law. Hence, a general rule of consolidation first, expropriation second, 
should arguably be introduced in order to better promote the consolidation alternative.

This thesis went on to explore how this alternative works in practice for hydropower development, noting that the current system works best when the owners are in reasonable agreement with each other and society about how development should proceed. Hence, the broader applicability of the idea can not be taken for granted. However, it seems to have great potential for being fine-tuned, to make it more effective in situations involving deeper disagreements and conflicting interests. The there is a danger here, however, namely that increasing the power of the institution will also undermine its role as a democracy-on-demand for owners and the community. However, if safeguards can be put in place, the land consolidation model might well be a highly attractive alternative to expropriation, also in cases involving deeper conflicts about property uses.

If property rights are held only by the few, while many property dependants are without property rights, the consolidation model might not be appropriate. However, in these cases, it might be possible to adapt it by allowing a larger group of local people to partake in the proceedings. This, of course, raises the issue of whether or not the administrative costs will make the consolidation alternative worth pursuing. However, as the complexity of planning and administration is already high in many contexts, the necessary resources might already be available to make consolidation feasible, if other branches of government can reduce their costs correspondingly. 

A final question in more complex settings is whether or not the stakeholders -- owners and other property dependants, in particular -- will be able to participate effectively in the process. In general, many people might not be able to. However, the consolidation will at least give them an opportunity to do so. In this way, the system can also serve an empowering and educational role for marginalised groups. Local elites might dominate the process, but the fact that the procedure is judicial in nature means that abuses can be curbed. Access to the decision-making process will be ensured for those who wish to challenge the local leadership. I

Indeed, as an alternative to more static forms of government at the very local level, the flexibility and transient nature of a consolidation court might have advantaged. Specifically, fixed patterns of power are unlikely to take hold from case to case, as the group of people involved, the area covered, and the agenda to be deliberated on, will change from case to case. This, in itself, is likely to serve as a diffusion of power, minimizing the risks of majoritarian or elite tyranny.

Further exploration of the transferability of the consolidation framework to other contexts will have to be left for future work. What this thesis has hopefully shown is that institutional alternatives to expropriation for economic development exist and can work in practice. This can hopefully also inspire more work in this direction at the theoretical level, picking up the thread of Heller and Hills' work. In my opinion, the ideas at work here should be developed further, both in the context of US law and in other jurisdictions that rely on expropriation as a tool to facilitate economic development. In these cases, more than any other, is would be a great victory for both property and equity if public interests could be communicated to owners so that they may give rise to givings in the future, clearly distinguished from the takings of the past.



\noo{ unequally by members of society, o


\noo{
This latter vision appea

consistent with the social function view of property. As discussed in Part I of this thesis, owners also have obligations attached to their property. If they are compelled to give it up and this is legitimate, than arguably it is because owners are obliged to 


Indeed, if the rich pay their taxes for the common good, why should not property owners give away their properties for the public interest? 




My reason for being interested in givings is different. What interests me is the choice of terminology. Why is it that when the state receives property from an owner under threat of compulsion, this is


In principle, the private recipient of property taken by eminent domain will be the indirect beneficiary of a taking but the direct beneficiary of a giving. 

\section{Many Aspects of Taking}

The most obvious way to describe a taking is to say that it involves the transfer of property from one legal person to another. However, as I noted in the first chapter of this thesis, property itself is highly multifaceted, serving a range of social and individual functions. Hence, when we begin to unpack the property bundle, we are confronted with a multitude of different senses in which a taking impacts on owners, their communities, and society as a whole.

The economic consequences of a taking might be the most easily recognisable, particularly in the economic development cases. But as my work in this thesis has shown, other consequences can be just as important, particularly those pertaining to property as an anchor for local democracy. If jointly owned property is taken from a community, with full compensation paid to all individual owners, the community suffers a distinct uncompensated loss, namely the loss of future self-governance opportunities.

In the traditional narrative on takings, social and political effects are typically only recognised on one side of the takings equation, namely the side of the taker, particularly the public interest. This has also influenced the debate on economic development takings. In order to make sense of the broader sense of unfairness often associated with such takings, critics tend to focus on the taker rather than the owner, by questioning the legitimacy of the motives behind the taking.

However, this might be tantamount to shifting a variable to the wrong side of the takings equation. In particular, the feeling of unfairness associated with economic development takings clearly arise from a sense in which the owners are victims of an abuse of power. So why shift attention to the taker? 

Perhaps it is tempting to do so simply because the sense of unfairness at work here pertains to a broader notion of justice than that normally associated with property interests. If so, the entire narrative points to a shortcoming of the liberal idea of property. If even property's staunchest defenders must turn to notions of ``public interest'' (and the lack thereof), then why do we need property as a concept at all? Why not simply say that a licence to undertake economic development should not be granted unless all affected parties agree, or the public interest is sufficiently strong to go ahead against some of their wishes? What makes property special in this picture, if all that is at stake is the strength of the public interest used to justify imposing the state's will on private individuals?

Clearly, the gaping hole in the opposition to economic development takings in the US has been a {\it positive} account of why property is worthy of protection in the first place, in cases where economic rationality dictates that it should be put to more profitable use. If the public interest is regarded as insufficient, it must be because there is something valuable inherent in property that raises the threshold for taking property above a certain level.

Such is the conventional narrative, that the owner as an individual suffers a loss in order for the taker, society as a whole, to achieve democratically determined political goals. But in economic development cases, the picture is quite different. In these cases, it is often the case that local communities are deprived of political capital in order for specific commercial interests to make a profit. 

In such cases, it might well be that the balancing of different reasons for and against the taking has taken place prior to the decision to interfere with property. The plans for development themselves may well precede any specific property-oriented implementation steps, such as the use of eminent domain. It might even be that democratically accountable bodies responsible for land use planning have already concluded that some local community interests must give way to other interests.

In these cases, it might be tempting to argue that a narrow takings narrative is appropriate because it pertains only to the final implementation step, which is the only one that involves property rights. But this argument, I believe, rests on a flawed perception of what property is, and should be, in a democratic society. Invariably, property has to do with decision-making and power. If the decision-making process does not grant significant self-determination rights to affected property owners, a taking is already in progress. It might be justified, but it is still a taking. 

More worryingly, it is clear that this kind of taking carries with it a great potential for differential treatment, discrimination, and corruption. The traditional takings narrative does a good job of setting up a framework that makes it difficult to simply pay higher compensation to certain kinds of people, without offering any justification. But with respect to the aspects of taking not recognised, e.g., pertaining to what role the owner has during the planning stages, differences in treatment will not even be notices. But if property is owned by the right sorts of people, then invariably it {\it will} come with considerable decision-making power. 

If property is owned by the marginalised, on the other hand, the most severe act of taking will sometimes have taken place even before the land use planning begins, by the fact that the owners are placed entirely on the sidelines. This, I argued in Part II of this Chapter, is how the Norwegian system for management of hydropower approach riparian owners works. 

At the very outset of planning, often decades before any formal decision to expropriate has been made, a considerable portion of the substance of property is taken from the owners, who are completely excluded from the rest of the decision-making process. In Norway, such takings processes, that clearly transcend the traditional financial narrative, have progressed to the point that even the law today provides an ambiguous account of Norwegian water resources as private property belonging to the general public.

Perhaps, then, the nature of property itself has changed, so that there is nothing left except those financial entitlements that Norwegian expropriation law recognised. If so, the change has not come about by any legislative move, nor has it been preceded by any kind of debate. It has simply emerged, gradually and unplanned, as a result of sector-based regulation and administrative practices. The process, therefore, meets neither the requirements of land reform or expropriation. It is an unacknowledged process about which the law in Norway has had nothing much to say at all, for which silence still persists. 

This is unfortunate. Even if riparian rights should be stripped of all content except a financial entitlement, this should happen on the basis of debate and democratic decision, not because the law fails to cater to a descriptively accurate notion of property.


not how a taking should be carried out, nor does it meet the standard 




 the decision-making process {\it will} normally reflect the power of property, if other contextual factors are not 

although to different degrees depending on other contextual factors, most notably the social status of the owner. Hence, property plays a constuti


particularly 


Property is not merely a placeholder for transient entitlements. It also both help make up and is shaped by the social and political context. 


 This was the key point that I argued for in Chapter 1 of the thesis, by looking to the social function theory of property and the notion of human flourishing. 


which focuses specifically on property rights only {\it after} the public interest has been mapped out and formulated by the decision-makers? 



 individual financial entitlements 

However, the property perspective, which is imposed onl

As I have demonstrated in this thesis, the economic development takings turn this narrative completely on  xplored in depth in this thesis, the con

Rather, the typical narrative places such aspects

After all, on the taker side, many of the primary concepts used to conceptualise typical takings are neither economic nor individualistic.

 on the owner side, do not 

The consequence of this can be that former owners and their communities will be marginalised more generally, as their position within society weakens. In turn, it will become easier to take more property from them, under increasingly weaker arguments of public interest. In the end, when egalitarian property rights no longer provide a foundation for decision-making about land use, the risk is high that a corresponding inequality in decision-making power will follow quite generally. Democracy as such might be at risk.

In any event, the land-less will not have a voice unless they can find different means of asserting themselves. The possibility of achieving participatory equality without egalitarian property should not be overlooked, of course. However, it seems safe to say that the track record of alternative ideas, whereby equality is pursued through institutional arrangements alone, is unimpressive. 

In almost all countries that score well on parameters such as democracy, living standard, transparency, and the rule of law, we find private property rights as a core legal principles. Moreover, while property might be unequally or unfairly distributed among the population, property rights are typically distributed widely enough to give rise to a natural division of power and a plurality of perspectives. Indeed, even the land-less may sometimes attain a voice, albeit a very limited one, if they are still in possession of their own labour.

The negation of property rights

This, in turn, is the very foundation of both democracy and the rule of law.



 as an underlying source of division of power.


 
\section{Some Ways of Giving Back}

\subsection{Locating Primary Stakeholders; The importance of Communities}

\subsection{Making Influence Proportional to Stakes; the Closeness-to-Consequences Test}

\subsection{Robust and Flexible Institutions for Collective Action; the Possibility of a Judicial Approach}

\subsection{Beware of Big Units; the Fine Line between Representation and Usurpation}

\subsection{The Importance of Redundancy; Property Regained}

It seems that property dislikes being concentrated in the hands of the few. } }