\chapter{Introduction and Summary of Main Themes}\label{chap:intro}

\begin{quote}
Thieves respect property; they merely wish the property to become their property that they may more perfectly respect it.\footnote{G.K. Chesterton, {\it The man who was Thursday: A nightmare}.}
\end{quote}
\begin{quote}
A takings power, then, may not be viewed as an act that wrenches away property rights and places an asset outside the world of property protection. Rather, it may be seen as an act within the larger super-structure of property.\footnote{Abraham Bell, Private Takings, p. 583.}
\end{quote}
%
%A human being needs only a small plot of ground on which to be happy, and even less to lie beneath. %\footnote{Johan Wolfgang von Goethe, {\it The sorrows of young Werther and selected writings}.}
%\end{quote}
%“That's what makes it ours - being born on it, working on it, dying on it. That makes ownership, not a %paper with numbers on it.”
%― John Steinbeck, The Grapes of Wrath
%
%
%
This thesis addresses so-called economic development takings, which occur when government sanctions the taking of property to stimulate economic growth. \noo{My primary interest lies in the legal questions that arise, not the overarching philosophical reflections that these might give rise to. However, I think it is worth reflecting for a minute on how  many different, ideologically tinged, perspectives on property may come together when focus is shifted away from the thing itself towards the legitimacy of the act of taking it. This is clearly demonstrated by recent case law from the US.} The canonical example is {\it Kelo v City of New London}, which brought the category of economic development takings into focus in the US, resulting in great controversy and a surge of academic work on legitimacy of takings.\footcite{kelo05}. The {\it Kelo} case concerned a house that was taken by the government in order to accommodate private enterprise, namely the construction of new research facilities for Pfizer, the multi-national pharmaceutical company.

The home-owner, Suzanne Kelo, protested the taking on the basis that it served no public use and was therefore illegitimate under the Fifth Amendment of the US Constitution. The Supreme Court eventually rejected her arguments, but this decision created a backlash that appears to be unique in the history of US jurisprudence. In their mutual condemnation of the {\it Kelo} decision, commentators from very different ideological backgrounds came together in a shared scepticism towards the legitimacy of economic development takings.

Interestingly, their scepticism lacked a clear foundation in US law at the time, as the {\it Kelo} decision itself did not appear particularly controversial in light of established eminent domain doctrines in the US. Hence, when the response was overwhelmingly negative, from both sides of the political spectrum, it seems that people were responding to a deeper notion of what counts as a legitimate act of taking.

Indeed, the critical response to {\it Kelo} appears to have been a reflection of widely shared sentiments. As such, it also arguably involved pre-legal notions pertaining to legitimacy. Simply stated, people from across the political spectrum simply found the outcome {\it unfair}.

If the law is about delivering justice to the people, this phenomenon deserves attention from legal scholars. In the US, it has received plenty of it. In the context of US law, it is now hard to deny that cases such as {\it Kelo} belong to a separate category of takings that raises special legal questions. \noo{ Moreover, after {\it Kelo}, most US states have passed some sort of legislation to limit economic development takings, in a direct response to the controversy following the {\it Kelo} case. 

There are significant differences between takings law and practice in the US compared to other jurisdictions, e.g., in Europe. However, the backlash of {\it Kelo}, and especially the clear divergence between public opinion on the one hand and established case law on the other, suggests the transformational potential inherent in the category of economic development takings itself.}

It is worth emphasising that this upheaval of US takings law was largely the result of a popular movement. In particular, this suggests the relevance of economic development takings as a legal category more generally, also outside of the US. It seems plausible to assume that as soon as cases such as {\it Kelo} are portrayed as being primarily about bestowing a benefit on powerful commercial interests, people will have a tendency to judge the issue of fairness similarly, irrespectively of differences in the surrounding legal framework. Indeed, it is quite natural that when a taking is characterised as being primarily for profit, this can lead to a perception that it lacks legitimacy.

The question becomes to what extent it is appropriate to deride economic development takings in this way. Here I believe the first important step is to acknowledge that there is at least a {\it risk} that takings for economic development can be improperly influenced by commercial interests. The risk of such capture, moreover, is clearly higher in economic development situations than in cases when takings take place to benefit a concretely identified public interest, such as the building of a new school or a public road. Hence, economic development cases do at least deserve to be singled out as a special category, for which additional scrutiny is appropriate.

This claim is not self-evident. For instance, it seems that many European jurisdictions implicitly reject such a perspective, because the state is regarded as having a ``wide margin of appreciation'' when it comes to deciding on legitimate takings purposes.\footnote{....} This points to the first main theme of this thesis: an analysis of economic development takings as a conceptual category for legal reasoning, leading to an argument that it should be introduced also in other jurisdictions than the US.

%\section{Economic Development Takings as a Conceptual Category}
\noo{
The category of economic development takings is not well established outside of the US, but the influence of the US debate is beginning to show, including in Europe.\footnote{See, e.g., \cite{verstappen14}.} It is a problem, however, that the exact meaning of the category may differ depending on who you ask. It is quite common, for instance, to speak of ``private'' takings more or less as a synonym to economic development takings. But there are key differences here that should be kept in mind.

First, speaking of a private taking already carries with it an implicit pointer to a lack of legitimacy, at least in jurisdictions that explicitly single out {\it public} interests as the only permissible justification for a taking. By contrast, the category of economic development takings does not carry with it any such implicit critique. If economic development takings are in need of special scrutiny, the reason cannot be simply that they can also involve private interests. After all, economic development is typically perceived to be in the public interest, regardless of whether a public or private body is tasked with carrying it out.

A second difference between private takings and economic development takings is that the former notion is easier to define. In fact, I think it is {\it too} easy. It is very tempting, in particular, to simply say that a private taking occurs whenever the legal person taking title to the property in question is a private company or individual. But this gives rise to a perspective that is overly simplistic. It might well be that a private organisation, say a tightly regulated charity, functionally mimics a quintessential ``public'' taker. A public body, on the other hand, can well be functionally equivalent to a private enterprise, particularly if there is a lack of political oversight and democratic accountability. Moreover, imagine a case involving a publicly owned limited liability company. According to the simple definition of a private taking, a taking by such a company would not meet the definition. This would be the conclusion even if the company's interests are completely or predominantly of a private-law nature, directed at maximising profit for the shareholders, not at providing a public service.\footnote{Some might argue that the distinction between private and public ownership is still significant. However, such an argument seems difficult to make convincingly, particularly if the company operates for profit and is  insulated both from political decision-making and principles of administrative law. For such a company, it is hard to see why takings to benefit the company should be regarded as {\it a priori} different from other kinds of economic development takings merely because of public ownership. In particular, it is hard to see why it should matter in such cases whether the associated public benefit is ensured through the payment of dividends, taxes, or some other mechanism. In any event, the public benefit will be indirect in these cases, arising from ordinary commercial activity.}

On the other hand, the notion of economic development takings raise a different problem, namely that a 
clear definition appears to be missing from the literature. Rather, scholarship on these kinds of takings rests on an intuitive understanding of the term, firmly based on the US jurisprudence from which it first arose. At its core, however, we may safely say that the reason for paying particular attention to the cases classified as economic development takings has something to do with the strong economic, often commercial, incentives that persist on the taker side. Such incentives, more specifically, can serve to sow the seeds of doubt as to whether or not due regard has been had to the interests of owners and directly affected local communities.

This concern might be relevant also when the economic incentives in question are of a non-commercial nature. However, cases when the taker acts as a profit-maximiser on a competitive market are certainly likely to be of special concern. Hence, I will argue for an additional qualification that designated {\it takings for profit} as a special sub-class of economic development takings that should receive particular scrutiny. Making this qualification should prove particularly useful in economic systems such as those seen in the west, where the steadily increasing influence of public-private partnerships cause a generally blurring of lines between private and public sectors.

In such societies, economic development takings will often, but not always, be characterised by a strong commercial incentive. It seems appropriate, therefore, to devote special attention to cases when a commercial interested party, often private, stands to gain a significant financial benefit from a taking. The financial motivation for the taker might contrast with the public spirited motivation of the executive or legislative body that grants permission to use compulsion; the (stated) intention of economic development takings is typically to promote public interests, not to bestow commercial benefits on particular parties. The importance of economic development as a category of takings is that it helps us flag those cases when this contrast is so strong as to suggest that we should further scrutinize the legitimacy of the undertaking as a whole.

If the decision-maker fails to identify concrete public interests and relies instead on a vague notion such as economic development, this, in particular, should be cause for increased scrutiny. This seems to be an observation of generally validity, also outside the context of US law. At the very least, The tension between public interests and commercial gains in property interference appears naturally in any system of government that combines a market-based economy with wide state powers over the use and distribution of property. The question becomes how one should reason about this tension in a meaningful way, to analyse interference in property motivated by economic development in a manner that is suited to yielding legally relevant insights.

%This thesis investigates the category of property interferences that are known as {\it economic development takings} in the US. This category came to prominence only quite recently, following the influential {\it Kelo} case.\footnote{See \cite{kelo05}.} 
This thesis studies this question in depth for the most severe among all property interference: the taking of property from its owner against their will.}

\section{Taking Property for Economic Development}

This thesis will argue that the category of economic development takings arises naturally already at the theoretical level. However, this claim will be made relative to a theory of property that is broader than typical approaches to property in legal scholarship. Specifically, the theory of property that will form the backbone of this thesis will encompass more than just the financial entitlements of owners. 

The theoretical framework is discussed in more depth in Chapter 1 of this thesis. There it will be argued that a social function understanding of property should be adopted, with an emphasis on {\it human flourishing} as a normative foundation for property. In short, property should be protected because it can help people flourish. Moreover, property is meant to serve this function not only for the owners themselves, but also for the other members of their communities.

This ambitious take on property and why it should be protected must necessarily also give rise to a broader assessment of legitimacy when the state interferes. This, in turn, is what inspires my initial discussion on economic development takings in the first chapter of this thesis. There I will present the basic definition of the notion and discuss the {\it Kelo} case in some more detail. Specifically, I will argue that Justice O'Connor's strongly worded dissent -- arguing that the taking should be struck down -- embodies in it a social function perspective that is sensitive to the property values discussed in the first part of the chapter.

In Chapter 2, the thesis will follow up on this by studying the legitimacy of economic development takings in more depth. I will consider several approaches to this issue, culminating in a recommendation for a perspective based on institutional fairness that I trace to recent developments at the ECtHR. 

%First, England and Wales will be discussed. Here the notion of parliamentary sovereignty remains an important anchor for reasoning about legitimacy of takings in the law. The emphasis on procedure and administrative that this entails is not by itself an indication that property is weakly protected, but I point to some weaknesses that suggest the need for substantive safeguards. Specifically, I will argue that the idea of parliamentary sovereignty is in danger of translating into deference to the executive. This occurs naturally because of the growing state and -- specifically -- the increasing prevalence of takings for economic development.

%The search for a substantive standard gives rise to a more in-depth study of US law, where the public use restriction in the fifth amendment provides a justiciable substantive standard of legitimacy. The thesis tracks the history of review under this standard back to a time when the courts were more willing to use it. The contextual nature of review is highlighted specifically as a sensible approach to this kind of review. However, I note that substantive standards also have their weaknesses, especially with respect to the broader systemic consequences of economic development takings. It seems, specifically, that Justice O'Connors dissent in {\it Kelo} is not a perfect fit with the review that could be expected under stricter public use clauses.

%This, in turn, leads me to consider recent developments at the ECtHR, where there has been a shift towards an institutional perspective on fairness. 

Specifically, the Court in Strasbourg has begun to look more actively at the systemic reasons why violations of human rights occur, in order to address structural weaknesses at the institutional level in the signatory states. At the theoretical level, this kind of perspective is arguably the one that fits best with the sort of analysis carried out by Justice O'Connor in {\it Kelo}. Importantly, the institutional perspective appears to be a sensible middle ground between the procedural and substantive approaches to legitimacy, which focuses on decision-making processes and structural aspects without giving up on substantive assessment. To strike a fair balance, in particular, is not just about reaching an appropriate outcome, but also about how that outcome came about, and how often dubious outcomes are likely to result from the way the system is organised.

%Arguably, the emphasis on proper procedure and parliamentary authority as the source of legitimacy has proved quite effective in minimising tension and conflicts over the use of eminent domain in England and Wales. However, I also point to a weakness of this approach, whereby the protection offered is arguably too dependent on continued observance of an {\it idea} of property as being a sacred rights. Moreover, the protection is not very targeted, but can strike down government interferences rather randomly, according to the technicalities of administrative law rather than by any substantive standards of fairness.

%s offered weaker protection.



%distil some general lessons from the US debate and its history, to get at the heart of what makes economic development takings special. In addition, I briefly assess the status of economic development takings in Europe, where takings that benefit commercial interests are often allowed to pass without raising special questions, and where the legal relevance of the category of economic development takings may still be called into doubt.

%In fact, I argue that this is a shortcoming of the narrative of property protection in Europe, and I also suggest that the concept of an economic development taking would in fact fit well with jurisprudential developments at the ECtHR, which stresses both the need for contextual assessment and attention to possible systemic imbalances in the expropriation practices of member states.

%Similar requirements, interestingly, are found in many other jurisdictions, and is also found in the property clause in the European Convention of Human Rights (ECHR). But in Europe, it is often understood very loosely, as a clause that places little or no practical limit on the state's taking power. The contrast with the suggestions that are now being considered by US scholars is very great.

%In the US, most work on economic development takings has been anchored in the so-called ``public use'' requirement of the Fifth Amendment. In fact, some US scholars argue that economic development takings are impermissible already because taking property for development cannot ever be said to constitute a ``public use'' of the property. Moreover, even scholars who reject this view tend to agree that the public use of a taking is less obvious, and should be subjected to more intense judicial scrutiny, in economic development cases.

%Interestingly, requirements similar to the public use test are found in many jurisdiction, in various guises, e.g., in rules referring to the need for a {\it public interest} or a {\it public purpose} for takings. On this basis, interesting comparative work has been carried out on the basis of the idea that such a requirement is at the core of the legitimacy issue that arises for economic development takings.

%In this thesis, I challenge this perspective. I do so by first reconsidering the history of the public use debate itself, as documented by case law in the US. I argue, in particular, that more attention should be paid to the fact that the state courts that originally set out to develop public use tests in the 19th century adopted a highly contextualised approach. Importantly, these courts where largely not bothered by the fact that they could not pin down any definite and consistent meaning of ``public use'' as a general concept. 

%Rather, the public use test was simply used as an expedient way of subjecting various acts of taking to a concrete fairness assessment, in the hope that local courts might help deliver corrective justice in cases when the takings power appeared to have been used in an objectionable manner. In this way, the original purpose of the public use test was tailored towards setting up a framework for judicial review that appears quite similar to how the European Court of Human Rights (ECtHR) currently choose to approach cases dealing with property.

%The jurisprudence at the ECtHR typically directs focus away from the question of whether the aim of a taking is legitimate in itself towards the more contextualised question of whether or not the interference is {\it proportional} given the circumstances. This, I argue, is also how the public use test was also originally used by state courts in the US, before the issue of legitimacy turned federal and became subject to a more abstract form of assessment, leading eventually to a tradition for passive deference that gave rise to {\it Kelo}.

%In fact, as soon as the issue of proportionality has been flagged as the primary question, it is not clear that the words ``public use'' are of much interest at all. Hence, my conceptual assessment can be summarised by the following two propositions. First, that the notion of an economic development taking, as developed in the US, is a useful addition for thinking about the legitimacy of takings, in any jurisdiction that aims to place meaningful restrictions on the takings power. Second, that the current focus on the notion of a ``public use'', which is supposed to provide the desired protection against transgressions, is largely misguided. At the very least, I believe alternatives should also be considered. 

This way of thinking about legitimacy brings me to the second focus point of the thesis.

\noo{

So far, the study of such takings has mainly been carried out by US lawyers, who asses it against the Fifth Amendment of the US constitution. This work has attracted some attention elsewhere, but the category of an economic development taking is by no means a universal category of legal analysis.

Perhaps it should be? This is the first main question that I address in this thesis. But has so far not made much of an impact in other jurisdictions, 

Hence, the somewhat counter-intuitive  Hence, one of the ultimate 
not, in particular, a straightforward 

Clearly, the most crucial question that faces any act of taking, government sanctioned or otherwise, is how we should approach objections against it.


To bring about economic development. 
Indeed, the notion of an illegitimate taking, regardless of whether it is seen as an affront to property or at it's core, certainly seems to have stirred the imagination of most property theorists.

More concretely, 


Perhaps a possible route to a better understanding of property goes by way of the study of takings, and th

re, incidentally, is also where we find an apparent sense of commonality between radically different accounts of property and its nature. It is worth noting, in particular, that 

More generally, it seems that people regularly engage in reasoning that seeks to determine what counts as legitimate and illegitimate ways of acting on objects in the material world. Perhaps it is in the midst of such judgements, is also where property as a concept plays a world.

Hence, to understand property as a concept, perhaps a viable route towards progress is to examine its negation, aiming, if nothing else, for a negative definition in terms of legitimate and illegitimate acts involving objects in the material world. In fact, such a more modest approach forces us immediately to recognise certain 

Luckily, as lawyers, we rarely have to worry about the {\it nature} of property, at least not in the philosophical sense of the word. Instead, we can focus on its {\it function}, in the legal system within which we operate. Still, the ``moderate'' and pragmatic view of property that lawyers tend to adhere to might not be very satisfying, particularly not when we are moving to the margins of the legal order, by considering hard cases that raise questions of policy and require novel normative assessments. In such circumstances, the pragmatic stance on property - as taught in law school and applied in courts - can sometimes appear bland, even vacuous, unable to accommodate solutions to genuinely difficult legal problems. These are the cases when received wisdom breaks down, the cases when logic -- or, more generally, {\it reason} -- must be called on to fill the gap left by experience, to paraphrase the famous words of Holmes.\footnote{Holmes quote and critique.}

Of course property is a social construction, a construction of law, but even so, we continue to question its {\it nature}, based on the implicit understanding that there is something more to property than the legal fictions that are used to package it in our legal order.
}

\section{A Democratic Deficit in Takings Law?}

%which link the question of their legitimacy to the public use test prescribed in the Fifth Amendment of the US Constitution.

On the basis of the work done in Chapter 1 and the first part of Chapter 2, the thesis formulates a concrete proposal for a legitimacy test that can be applied to economic development takings. This test consists of a list of indicators that can suggest eminent domain abuse. The first seven points are due to Kevin Gray, while the final three are additions I propose on the basis of the work done in this thesis. I call the resulting list the Gray test, a heuristic for inquiring into the legitimacy of an economic development taking.

Arguably, the most important indicator is also the least precise, namely the one pertaining to the {\it democratic merit} of the taking (one of my additions). By itself, such a broad indicator might not offer much guidance. However, the idea is that when taken together with the other points, this indicator will be a suggestion for an overall assessment of those points against the decision-making process as a whole, not only the final outcome. Hence, this addition is specifically aimed at emphasising the institutional fairness perspective.

Admittedly, asking courts to test for legitimacy, for instance by using the Gray test, is only an incomplete response to the worry that economic development takings represent an abuse of power. Indeed, some US scholars have argued that increased judicial scrutiny is neither a necessary nor a sufficient response to concerns about the legitimacy of commercially motivated takings such as {\it Kelo}. Instead, these authors point out that the takings procedure as such does not seem suited for bringing about this kind of economic development.

This observation has been accompanied by proposals for structural takings law reform, most notably Heller and Hills' article on Land Assembly Districts. This work proposes a novel institution for collective action and self-governance, to replace the traditional takings procedure, especially in cases where property rights are fragmented and the takers have commercial incentives. The basic idea is that the owners themselves should be allowed to decide whether or not development takes place, by some sort of collective choice mechanism (possibly as simple as a majority vote). In this way, the holdout problem can be solved (individual owners cannot threaten to block development to inflate the value of their properties). At the same time, however, the local community's right to self-governance is recognised and respected.
 
The proposal for Land Assembly Districts is closely linked to the theory of sustainable common pool resource management, developed by Elinor Ostrom and others. At the end of Chapter 2, I argue that looking for alternatives to expropriation that is based on this way of thinking is the way forward in the legitimacy debate. At the same time, the context-dependence of solutions along these lines make sweeping reform proposals unlikely to succeed. Rather, it is important that the institutions that are used are appropriately matched to local conditions. 

For instance, a setting where property is evenly distributed among members of the local community might suggest a very different type of institution compared to a setting where the relevant property rights are all in the hands of a small number of absentee landlords. In short, the idea of using self-governance structures in place of eminent domain necessitates a more concrete approach, a move away from property theory towards property practice. This sets the stage for the second part of the thesis, consisting of a case study of takings for Norwegian hydropower development. 

This first key objective of this case study is to test the theoretical assertions made about how to approach legitimacy. Specifically, the question to be addressed is to what extent the traditional narrative of takings is capable of doing justice to the property conflicts that have arisen regarding the development of hydropower in Norway.

\noo{ I arrive at several objections against the details of the particular institutional arrangements proposed, particularly with regards to their likely effectiveness. It seems, in particular, that both proposals fail to recognise the full extent to which prevailing regulatory frameworks concerning land use and planning would have to be reformed in order to make their proposals work.

At the same time, I argue that these novel institutional proposals are extremely useful in that they point towards a novel way to frame the issue of legitimacy in takings law. In particular, I explore the hypothesis that traditional procedural arrangements surrounding takings suffer from a democratic deficit, a particularly powerful source of discontent in economic development cases.

This idea is the second key focus point of my thesis. First, I approach it from a theoretical point of view, by exploring the notion of {\it participation} and its importance to the issue of legitimacy, particularly in the context of economic development. It seems, in particular, that {\it exclusion} could be a particular worrying consequence of certain kinds of economic development takings, namely those that lack democratic legitimacy in the local community where the direct effects of the taking are most clearly felt.

I believe this to be a promising hypothesis, and I back it up by considering the social function theory of property and the notion of human flourishing which has recently been proposed as a normative guide for reasoning about property interests. I pay particular attention to the importance of communities that has been highlighted in recent work, as a way to bridge the gap between individualistic and collectivist ideas about fairness in relation to property.

I take this a step further, by arguing that a focus on communities naturally should bring institutions of local democracy to the forefront of our attention. The role that property plays in facilitating democracy has been emphasised before by other scholars, and I think it has considerable merit. However, I also argue that it is important to resist the temptation of viewing its role in this regard through an individualistic prism. It is especially important to take into account additional structural dimensions that may supervene on both property and democracy, such as tensions between the periphery and the centre, the privileged and the marginalised, as well as between urban and rural communities.

It is especially important, I think, to appreciate the effect takings can have on local democracy. For one, excessive taking of property from certain communities might be a symptom of failures of democracy as well as structural imbalances between different groups and interest. But even more worrying are cases when the takings themselves, brought on by a commercially motivated rationale, appears to undermine the authority of local arrangements for collective decision-making and self-governance. This dimension of legitimacy, in particular, is one that I devote special attention to throughout this thesis.

I also believe, however, that it is hard to get very far with this sub-theme through theoretical arguments alone. Hence, to explore it in more depth, I go on to assess it from an empirical angle, by offering a detailed case study of takings of Norwegian waterfalls for the purpose of hydropower development. This case study, in turn, will allow me to cast light on two further key themes, that I now introduce. %This brings me to the second part of my thesis, which in turn consists of two main themes, where the latter aims to bring me back towards a more general setting, by delivering some recommendations for how best to deal with economic development takings.
}
\noo{
analysis, which must by necessity 

link the idea of the democratic deficit with theoretical work. on the category of economic development takings. In particular, I explore the notion of participation, to  and arguing that it is key to understanding common discontents that arise in for-profit taking situations. 

, by arguing that the recognition of this as a special category is closely related to a shift of focus towards procedural legitimacy. 

Building on this perspective, 

  that legal scholars and policy makers should address more actively.




I believe this suggestion is 


I go on to consider the hypothesis that economic development takings demonstrate that takings law suffer from a {\it democratic deficit}.}

\section{Putting The Traditional Narrative to the Test}

In Norway, the traditional way of thinking about legitimacy of takings is grounded in the notion that owners are entitled to monetary compensation. The law of expropriation clearly reflects the importance attributed to this idea; the constitution itself stipulates that owners have a right to be paid in ``full'' for the loss they suffer as a result of giving up their property. Consequently, the right to compensation in Norway is generally regarded as stronger than in many other jurisdictions, including those that adhere to the minimal standard imposed by the ECHR.

At the same time, the story of legitimacy more or less begins and ends with the issue of compensation. Hence, if an owner has grievances that are directed at the act of taking as such, not the amount of money they receive, takings law has very little to offer. In fact, it does not appear to offer anything that does not already follow from general administrative law. The owner can argue that the decision to take was in breach of procedural rules, or grossly unreasonably, but the chance of succeeding by making such arguments are slim, arguably no higher than in administrative cases that do not involve interference with property rights.

%This narrative of legitimacy is not unique to Norway. It seems that in Europe, unlike in the US, the issue of legitimacy is often seen as predominantly concerned with the issue of compensation. In particular, the jurisprudence at the ECtHR is typically focused on compensatory issues. Moreover, while many constitutions of Europe, including the Norwegian, include public interest clauses, the courts make little or no use of these when adjudicating takings complaints. In the words of the ECtHR, the member states enjoy a ``wide margin of appreciation'' when it comes to determining what counts as a public interest.

In relation to the specific case of takings for hydropower development, the position of the owners is even weaker than in general expropriation cases. This is despite the fact that these are clearly economic development takings, with few if any direct public interests to motivate them. Indeed, since the early 1990s, the hydropower sector in Norway has been liberalised. This means that the hydropower companies that expropriate are now commercial enterprises, not public utilities. Moreover, the property that they seek to take is not merely some ancillary rights that they need to develop the country's resources. In Norway, the right to harness the power of water is a private right, typically owned by members of the rural community in which the resource is found.

Hence, the hydroelectric companies in Norway are traditionally dependent on taking natural resources from local communities in order to exist and make money for their shareholders. Since deregulation, however, local owners have begun to resist such takings. This has been motivated by the fact that owners can now undertake their own hydropower projects as a commercial pursuit; unlike the situation before liberalisation, owner-led development projects can now demand access to the electricity grid as producers on equal terms with established energy companies.

As a result, local owners now regularly protest expropriation of their rights on the grounds that they wish to {\it participate} in economic development, by carrying out alternative development projects, or by cooperating with the energy companies who wish to take their water rights. Hence, while liberalisation has rendered takings for hydropower as takings for profit, it has also empowered local owners and communities to propose alternatives. Unsurprisingly, this has led to tensions that Norwegian courts have had to grapple with in an increasing number of cases.

In their approach to these cases, the courts rely heavily on the traditional narrative, by considering how compensation should be calculated when water rights are taken for hydropower. Compensation practices have already been subject to much attention, with an initially dramatic increase in compensation levels now apparently being reversed by a recent ruling by the Supreme Court. In addition to compensation cases, there have also been cases where the local owners have protested the taking as such, claiming that they should be given the opportunity to develop their own resources. These protests have been entirely unsuccessful, as the courts in Norway adopts a stance on legitimacy that is extremely deferential to the executive.

In this thesis, I nevertheless focus on legitimacy in a broad sense of the word, not the compensation issue. Chapter 4 presents the legal framework for hydropower development and provides a discussion on  current practices, both at the administrative level and in the hydropower sector. This empirical approach will allow me to explore the practical consequences of the current narrative about hydropower in the law, while also allowing me to bring out how decision-making process surrounding development work in practice. 

My main finding is that local owners are being marginalised by the current regulatory framework. Specifically, despite political support for small-scale development by local owners, the large energy companies have continued to enjoy a privileged position in their dealings with the water authorities. Lately, the political narrative also appears to be changing, with large-scale development becoming the preferred mode of exploitation also among politicians.

Following up on this observation, Chapter 5 discusses expropriation of waterfalls in more depth. Specifically, I track the position of owners under the law and administrative practice. The key finding is that expropriation is usually an automatic consequence of a large-scale development license. That is, the licensing system trumps private property, so that property will automatically yield whenever the water authorities grant a development license to some commercial company. There is no need for this company to acquire any property at all before submitting a licensing application to the authorities.

Moreover, the owners' position during the assessment stages is extraordinarily weak. The fact that expropriation tends to follow automatically from a license to develop has led the water authorities to regard the licensing question and the licensing assessment procedures as exhaustive in all cases. No distinction is made between cases involving expropriation and cases that do not, and the owners' interests receive little or no consideration during the assessment stage. According to written testimony during a recent Supreme Court case on legitimacy, the water authorities do not even recognise a duty to inform local owners of pending applications that will involve the expropriation of local property rights. The owners, according to the water authorities, must instead ``look after their own interests''.

In light of this and other data discussed in Chapter 5, my conclusion is that waterfall takings for hydropower in Norway demonstrate the shortcomings of the traditional narrative of legitimacy. Moreover, these takings do not appear to pass the Gray test formulated in Chapter 3. 

Interestingly, and in stark contrast to the expropriation regime, Norwegian law also offers a possible path towards a restoration of legitimacy. Specifically, the institution of {\it land consolidation}, as understood in Norway, can serve such a function. Moreover, it already does so in the context of hydropower development, in cases where local owners aim to undertake hydropower development themselves. If there are local disagreement, or some of the neighbours object to development, it is practically unheard of for local owners to make use of expropriation. Rather, they rely on the use of land consolidation, which can also be used to {\it compel} other owners to partake in development against their will. This brings me to the fourth key theme of this thesis.

\section{A Judicial Framework for Compulsory Participation}

In Norway, the distribution of property rights across the rural population is traditionally highly egalitarian. This has had many consequences for Norwegian society. For one, it meant that the farmers in Norway soon became an active political force, particularly as representative democracy started to gain ground as a form of government in the 19th century. As early as in 1837, the Norwegian parliament was so dominated by farmers that it came to be described as the ``farmer's parliament''.

The Norwegian farmers were often little more than small-holders, and had few privileges to protect. Hence, they became liberals of sorts (although also known for their fiscal conservatism). The farmers as a class were responsible for pushing through important early reforms, such as abolition of noble titles and the establishment of democratically elected municipality governments.

The municipality governments were not the first example of local decision-making institutions in Norway. Indeed, the highly fragmented ownership of land meant that institutions for collective decision making had to be introduced early on in Norwegian history. The most important one, which exists to this day, is the land consolidation court. The final focus point of my thesis consists in an assessment of this institution and its potential as a possible procedural alternative to takings when compulsion appears to be needed in order to ensure economic development.

Importantly, the land consolidation procedure in Norway is a judicial process that warrants the imposition of {\it compulsory participation} by primary stakeholders in decision-making processes to which they are deemed to owe a contribution. One typical situation when the institution will be invoked involves the management of jointly owned land. Here the land consolidation procedure is used to ensure that local owners reach a joint decision on how to regulate the use of their land, if necessary one that is imposed on them by the land consolidation judges.

The judges' power is limited in that they may only impose a measure if the gains are deemed to outweigh the loss for all properties involved. That is, the court must look to the resource packages themselve, not the interests of the current owner. The point is to preserve the property as a productive unit for sustainable resource use in a community. In practice, however, land consolidation judges rarely disregard current owners. Rather, they often act as mediators, to facilitate a collective decision made by the owners themselves. Moreover, one of the most common acts of a land consolidation judge is to set up owner's associations, in a manner that institutionally regulates the continued interaction and decision-making among the stakeholders after the formal consolidation process has concluded.

Chapter 6 explores this framework in some depth, focusing on its uses as an alternative to expropriation. This is especially interesting since land consolidation is presently being put to use in order to organise hydropower development. Hence, my case study provides an excellent opportunity for comparing the land consolidation and the takings process, with respect to the overall aim of ensuring development of hydropower on equitable terms. 

Here, I argue, the land consolidation route is highly preferable, as it ensures legitimacy through participation. At the same time, the procedure remains effective in this context, since participation is compulsory and the judge may intervene to settle conflicts. I discuss possible objections to the procedure in some depth, but conclude that the continued development of the land consolidation institution provides the best way forward for addressing problems associated with economic development takings in Norway.

Finally, I assess the institution of land consolidation against Land Assembly Districts, and -- more generally -- against the idea of self-governance frameworks for managing common pool resources. I argue that it compares favourably, both because it comes equipped with in-built judicial safeguards, but also because it has such a broad scope. I note, however, that its use as a better alternative to economic development takings is dependent on both political will and an ability to retain key feature even in the presence of new and powerful stakeholders in the consolidation process.

If the integrity of the procedure can be secured, adopting it to organise larger scale development involving external actors seems like a very promising approach to legitimacy. Moreover, while the system is designed to work in a setting of egalitarian property rights, it is interesting to also consider the possibility key features of the procedure might also inspire solutions to the takings problem in other jurisdictions. Specifically, the fact that the procedure focuses on benefiting properties rather than owners means that a broader understanding of property can itself suggest a broader range of possible applications. 

It might well be, for instance, that a land consolidation approach coupled with a human flourishing understanding of property can be a good way of including also the non-owners into the process. Possibly, a modification of the framework in setting where property dependants do not have property rights, is to include the set of legal persons with legal standing before the court. This might give rise to increased complexity of the procedure, and new risks of abuse by local elites, but it seems interesting as an idea to explore further. In short, the consolidation alternative seems like a good starting point for escaping the traditional narrative about takings, in order to reassess the notions we rely on when we think about property and how to best go about ensuring that it truly serves the public interest.

%the core features of land consolidation for economic development can be preserved and developed further, 
\noo{ In the second part of the thesis, I put the theoretical framework to the test by applying it to a concrete case study, namely that of Norwegian hydropower. Following liberalisation of the energy sector in the early 1990s, hydropower is now a commercial pursuit in Norway. Moreover, there is a long tradition for granting energy producers the power to acquire property compulsorily, including the necessary rights to exploit the energy of water, rights that are subject to private property under Norwegian law. This has resulted in tension and controversy, however, as the original owners of these rights, typically local farmers and small-holders, see the commercial potential of hydropower being transferred to other commercial interests, to the detriment of their own, and their communities', interest in self-governance and economic benefit.}

\noo{ \section{Structure of the Thesis}

My thesis is divided into two parts. The first is devoted to setting up a conceptual framework and a knowledge base for analysing the legitimacy of economic development takings. I start in Chapter \ref{chap:1}, by examining theories of property as a legal concept, particularly the so-called social function theory. This theory is distinguished by the fact that it highlights the fact that property serves as an anchor of responsibilities as well as rights, thereby helping shape and regulate social systems that are important to society, not just the owners as individuals.

The proponents of the social function theory often make strongly normative claims about property, but here I argue that it is fruitful to take a step back and examine the descriptive content of the theory separately. This, I believe, can help us locate a theoretical template that is less conceptually impoverished than many other descriptive theories of property as a legal concept. This, in turn, can hopefully render the theory suitable as a common ground that can facilitate meaningful discussion among theorists with very different normative ideas and commitments.

Importantly, I go on to argue that the social function theory suggests that economic development takings should, already for the sake of descriptive accuracy, be treated as a separate category when reasoning about legitimacy. This insight, I note, does not arise in the same way from the two main traditional strands of theorising about property in law, based on the {\it dominion} concept and the {\it bundle of rights} metaphor. On these accounts, property is understood in individualistic terms that make it hard to justify why the purpose of the taking should be of any concern at all to the affected owner, as long as due process has been observed.

After establishing economic development takings as a category of descriptive analysis, I set out to provide a theoretical template from which to embark on normative assessment. Here I turn to the notion of {\it human flourishing} which has been proposed as a key concept when reasoning about the {\it purpose} and {\it values} of property. Importantly, the accompanying theory endorses value pluralism, while also flagging the importance of property to the well-being of communities. This latter theme, in particular, will be important throughout the remainder of the thesis. 

In the final part of Chapter \ref{chap:1}, I make a first pass at substantive assessment, by applying the theoretical framework I develop to provide a brief analysis of the {\it Kelo} case and related academic work in the US. In Chapter \ref{chap:2}, I go on to present a much more detailed, comparative, account of how legitimacy of economic development takings are dealt with in the US compared to in Europe. I focus particularly on the history of the public use debate in the US, to argue that there are important commonalities between how the public use restriction was originally applied (at state level) and how the ECtHR current adjudicates property cases, by assessing the {\it proportionality} of the interference.

I follow this up by considering the role of courts as arbiters in relation to proportionality. I argue, in particular, that recent developments at the ECtHR might indicate not only increased level of scrutiny, but also a shift of attention towards examining systemic imbalances. Moreover, I follow those who argue that courts are not well placed to actually {\it ensure} proportionality, and that they should not be called on to micro-manage the takings process through a myriad of rules that seek to explicate what counts as legitimate in any given scenario. Rather, I locate an institutional gap for hard cases, where the traditional takings procedures entrenched in administrative law simply appear to be inadequate.

In the final part of Chapter \ref{chap:2}, I build on this by considering in depth some proposals for institutional reform that have emerged in the literature from the US. Here the focus is on designing mechanisms for collective action and self-governance that can replace takings in the traditional sense, in cases when there are strong economic incentives for development. This promises to ensure forms of benefit sharing with owners and their communities that are unavailable when a traditional compensatory approach is adopted. In addition, proposals for institutional reform can enhance democracy and human flourishing by giving a more prominent place to owners and local communities in those critical decision-making processes that may lead to development and reconfiguration of established property patterns. I pinpoint some shortcomings of existing proposals, raise some questions, and argue that the institutional route is the best way forward for addressing the legitimacy issue in further depth.

This preliminary conclusion leads to the second part of my thesis, where I apply the insights gained from the first half to analyse and distil lessons from the Norwegian framework for expropriation in the context of hydropower development. In Chapter \ref{chap:3}, I begin by offering a brief introduction to the Norwegian legal system, before presenting in more detail the rules regulating the right to harness the power of water. I follow up on this by presenting empirical data on the hydropower sector, aiming to shed light on how the regulatory frameworks work in practice. I emphasise the current tension between hydropower projects controlled by local owners and their communities and competing projects controlled by large commercial, partly state-owned, companies, that rely on expropriation. I emphasise the positive effect local hydropower initiatives can have on the communities in which development takes place, and I present an early vision of the social function of such development in some depth. I conclude by a cautionary assessment of current developments in the owner-led industry itself, where commercial forces appear to be gaining ground at the expense of more rounded perspectives on the purpose of development.

In Chapter \ref{chap:4}, I go on to specifically consider rules and practices relating to the expropriation of the right to harness water power, I give a general introduction to Norwegian expropriation law, while focusing on special rules that apply to hydropower development. I show, moreover, how the current regulatory regime is strongly influenced by the fact that the hydropower sector used to be organised as a state monopoly, under decentralised political control. This democratic and public anchor was largely removed, however, as the sector was deregulated in the early 1990s. As a result, current practices in Norway render takings for hydropower as pure takings for profit, something giving rise to an increasing number of cases where local owners challenge the legitimacy of established practices. I go on to study one such case in great depth, to bring out how the current framework can leave local owners marginalised and excluded from the key decision-making processes that eventually lead to the taking of their property by commercial companies. At the same time, I note how the compensation procedure has been reformed, offering a financial windfall to some owners individually. However, I also note how these reforms have been actively opposed by the hydropower industry which currently appears to be gaining ground in their efforts to reverse recent changes in compensation practice. I conclude by arguing that the traditional narrative of legitimacy has proven inadequate in Norway, as it unduly focuses on compensation without tackling underlying imbalances in the division of decision-making power among owners, local communities, regulators and commercial companies. In this way, my analysis of the Norwegian case culminates in the same conclusion as my theoretical work, pointing to the need for institutional reforms that can give local owners and their communities a stronger voice in decision-making processes concerning their natural resources.

In Chapter \ref{chap:5}, I go on to analyse the Norwegian land consolidation courts as potential answers to this challenge. I begin by presenting the legal framework surrounding this institutional arrangement which has long traditions in Norway. Moreover, I note that it appears to be growing in importance, and that the land consolidation court is widely authorised to order collective action among property owners as well as to set up more permanent institutions for self-governance, with clearly defined rules and purposes. Hence, consolidation courts can in fact be used to order economic development and set up framework that compel owners to participate in it. 

After briefly presenting the procedure itself, I focus on important property-based protections against abuse, such as the no-loss guarantee which states that no consolidation measure can be ordered unless the gains match the loss for all the properties involved. I note, in particular, how the focus here is on the affected properties as such, not on the individuals who happen to have rights to them. After presenting the land consolidation courts, I go on to study how they are used in practice in cases of hydropower development. Interestingly, while the established commercial companies continue to rely on expropriation of water rights to facilitate development, local communities that face interal disagreements about development are far more likely to turn to the land consolidation court. Hence, there is now empirical evidence available on consolidation as an alternative to expropriation in these cases. While the development projects facilitated by consolidation are still typically small-scale compared to those carried out with the help of expropriation, I still believe this material is highly interesting. In Chapter \ref{chap:5}, I analyse several concrete cases of consolidation for hydropower development to bring out how this works in practice. I conclude with an assessment of the consolidation alternative, proposing that it looks very promising and should be explored further. Moreover, I note the striking similarities between the consolidation framework and the institutional proposals that have emerged in the US, which I presented at the end of the first part of the thesis. I note, moreover, that the there are some key differences that I believe speak in favour of Norwegian consolidation courts. In particular, the judicial framework, the flexible authority and scope of the procedure, the possibility of compulsion by a neutral party, as well as the built in procedural safeguards, all seem to be strengths of the Norwegian system. At the same time, however, I note some possible weaknesses, particularly the worry that the land consolidation institutions themselves risk being captured by powerful actors. In addition, worries related to the cost of the procedure, as well as its effectiveness in case of large-scale development, are addressed. Overall, however, my conclusion is that the core ideas inherent in this institutions are sound and can serve as a template for creating legitimacy enhancing institutions for compulsory economic development elsewhere. This observation concludes the material work of this thesis, and in Chapter \ref{chap:conc} I offer my final conclusions.

\noo{ In Chapter \ref{chap:1}, I start by giving a concrete exFirst, I argue that the category makes sense outside the context of US law and that it is worthy of comparative study, even if it has not yet come to prominence outside the US. I develop this argument theoretically by anchoring it in the social function theory of property. 

This theory, which I argue for as a general template for thinking about takings and property, makes it very natural to operate with a special category for cases when there are significant commercial interests on the taker side. Moreover, I believe it highlights the importance of considering the matter of legitimacy contextually and, at it's ore, as an issue of fairness and proportionality. This appears to be at odds with the standard narrative surrounding economic development takings in the US, which tends to focus on the public use requirement.

The second aim of the first part of my thesis is to explore this tension. I do so by exploring the history of the public use debate in the US, by arguing that contextual assessment was originally at the core of the notion, when legitimacy was adjudicated by state courts. Subsequent developments at the federal level, I argue, has served to change the impression of the test, creating the erroneous impression that the key question is whether or not a taking is for a ``public use'' in some abstract sense that should be pegged down by the law. Rather, the basic question is, and remains, as it has always been, whether or not the action of government is able to strike a fair balance among the interests of the affected shareholders.

The third aim of the first part of my thesis is to propose an alternative way of thinking about legitimacy, that brings out the key issue without getting sidetracked by the notion of a public use. Here, I rely on recent institutional proposals for reform of the takings procedure itself, that are meant to empower owners and local communities by providing a novel template for collective action. I analyse these proposals in some depth, raising some objections and proposing some research questions.

This then brings me to the second part of my thesis, where I study the questions that I distilled in the first part by looking at the case of Norwegian hydropower. Here, my aims are again two-fold. First, I present an analysis of the framework of takings law in Norway, and the narrative of judicial reasoning that surrounds it, exemplified by the case of takings of water rights for hydropower development. I argue that tht the traditional narrative does indeed fail, in much the same way as predicted by the theoretical considerations of the first part of the thesis. 

My second aim is to present first steps towards a possible solution, again based on a concrete assessment of Norwegian law. Here, I present the institution of land consolidation courts, as a possible framework for compulsory participation in land development that can at once provide the participation and the compulsion that appears to be necessary to move towards the goal of economic development without offending against the rights of owners and local communities.

Finally, I offer a conclusion that aims to present in short form the main problems investigated, and the solution concepts offered, through the course of this thesis.

}
}