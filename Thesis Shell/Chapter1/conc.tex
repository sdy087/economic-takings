\chapter{Conclusions}

In this final section of my thesis, I would like to take a step back to briefly follow two broader threads that I believe run through my thesis. The first concerns the many senses of taking that have been brought into focus throughout the analysis.

\section{Many Aspects of Taking}

The most obvious way to describe a taking is to say that it involves the transfer of property from one subject to another. However, as I noted at the very beginning of this thesis, the notion of property is itself far from obvious. As soon as we begin to unpack it, we are confronted with a multitude of different senses in which a taking impacts on the owner and his community.

The most straightforward consequence of a taking is the economic consequence, the transfer of the property value to the taker from the owner. This is also the only loss for which the law regularly stipulates that compensation should be paid. But as I believe this thesis shows, different aspects may be equally relevant.

\section{Some Ways of Giving Back}

\subsection{Locating Primary Stakeholders; The importance of Communities}

\subsection{Granting Power Proportional to Stakes; the Closeness-to-Consequences Test}

\subsection{Robust and Flexible Institutions for Collective Action; the Possibility of a Judicial Approach}

\subsection{Beware of Big Units; the Fine Line between Representation and Usurpation}

\subsection{Pluralitas politica; Property Regained}
