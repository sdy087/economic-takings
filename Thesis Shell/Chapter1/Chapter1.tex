\chapter{Economic development takings and the public purpose: Tensions and theory}

\section{Introduction}

In recent decades the use of expropriation for economic development has become increasingly widespread in many jurisdictions, also in cases when those who directly benefit are commercial companies rather than public bodies. The justification usually offered for such takings is that they indirectly benefit society and the greater community, through increased tax revenues, new jobs, and various other economic and social ripple effects. Economic takings often prove controversial, however, and have also provoked much academic debate, particularly in the US.\footnote{References}

Tensions reached new heights following the Supreme Court case of {\it Kelo v City of New London}.\footcite{kelo05}  The company Pfizer was allowed to expropriate homes for the construction of new research facilities, and the questions arose as to whether or not this constituted "public use" in the sense of the property clause in the Fifth Amendment to the US Constitution.  The majority 5-4 found that the expropriation was constitutional, but the decision was controversial. Arguably, the attitude following {\it Kelo} has shifted towards a greater feeling of unease regarding economic takings, both among legal scholars and members of the general public.

The case has also had a great impact on academic writing on takings law in the US, where economic development cases are now usually viewed as a distinct sub-class of takings which merit particular attention. Moreover, a consensus seems to be emerging that there is a need for novel approaches to deal with such takings, possibly even new legal frameworks to resolve the tensions that typically arise. Some authors argue for a simple solution: an outright ban on economic development takings. However, the majority of scholars take a more measured approach, recognizing the need for legal frameworks that can be used to promote development projects without making illegitimate use of compulsion against land owners.

In Europe, there has not yet been been any comparable shift in academic outlook, but perceived expropriation-for-profit situations are increasingly coming into critical focus here as well. In a global context, this seems to form part of a general crisis of confidence regarding protection of private property, related to how  egalitarian systems of property ownership are coming under increasing pressure from large-scale commercial actors who assume control over an increasing share of the world's land rights. The phenomenon known as {\it land grabbing} is widely discusses and is apparently becoming endemic across the third world.

 One approach to property protection that is becoming increasingly relevant on the global stage is to see it as a {\it human rights} issue, offering basic protection to individuals independently of the particular jurisdiction they find themselves in. This perspective is already practically important in Europe, due to the European Convention of Human Rights (ECHR) and the court in Strasbourg (ECtHR). It is also promoted by many academics, international agencies and NGOs as the future of land rights protection on the global stage, as large-scale cross-national, and even international, transactions of such rights are becoming increasingly common. So far, the special category of for-profit takings has not been singled out for special consideration. Rather, the focus tends to be on {\it fairness} and {\it proportionality} as broader benchmarks that needs to be upheld. But, of course, how to achieve this depends on the circumstances, and cases when takings appear to be largely motivated by the commercial interests of private parties must be expected to be particularly problematic. Hence, it seems reasonable to single out economic takings as a special category also in relation to human rights law. This, however, appears to be a largely novel perspective; there is little or  no trace of such categorization in the literature so far, possibly because the very idea of a property rights regime entrenched in human rights law has been seen as controversial. In Europe, however, is is now the reality, with the ECtHR adopting a fairly strict standard in assessing cases of property interference from the signatory states.

% voice and some argue that we are currently witnessing a global pooling of resources and wealth that is unprecedented in modern times.\footnote{references}. In the 21st Century, these processes are driven forward by political expediency just as much as capital, orchestrated by public/private partnerships that sees modern regulatory frameworks employed to benefit those actors on the financial stage that are in the best position to wield the power of compulsion over land and people. This power is on offer through a range of  institutions that were developed following the industrial revolution, giving governments and international institutions and unprecedented level of control over the basic building blocks of the economic system. Increasingly, we are coming to realize that this power will not necessarily be used for good, but is characterized rather by the fact that it will be used extensively, to further the interests of whoever gains control of it. In particular, a range of the institutions of modern government appears to be suffering from an increasing {\it democratic deficit}, due to representative forms of democratic control having proven unable to legitimize the activities of these institutions.

In the following, I will present European human rights law and the takings debate in the US. I will highlight those aspects and issues that my subsequent case study aims to shed light on. For Europe, this means that I will present the basic proportionality test that is now at the core of property adjudication at the ECtHR. I will pay particular attention to the fact that the ECtHR have gradually widened the scope of the property clause in the ECHR. The court is now, in their own words, offering ``stronger protection'' of property rights. I discuss this development, arguing that this is indicative of a general crisis of confidence in Europe, whereby interference in property is increasingly seen as illegitimate. I further suggest that this might be due to a growing feeling of discontent with the rationale behind many instances of interference under modern regulatory regimes. The traditional ``public purpose''-scenario, where private rights must give way to the public will, is becoming increasingly hard to spot under planning regimes that involve extensive public/private partnerships and significant commercial interests.

For the US, I will focus specifically on economic takings, both the historical development that has led to this being seen as a a separate category and the debate following {\it Kelo}. Much of US literature on economic takings must be read in light of the tense political climate surrounding takings in the US. However, I believe that some of it is highly relevant to the international debate and to European human rights law. The post-{\it Kelo} literature in the US is quite different from that which dominated the scene previously and the decision marks a development towards increased critical scrutiny of economic development takings, not a further relaxation of constitutional safeguards. Hence, the development in the US might be an indication of what is to come also in Europe, if concerns about legitimacy of economic takings are not taken seriously. In any event, I believe it suggests strongly that operating with a special category of economic takings is helpful, at least as an abstraction, also in the European context.

%I also highlight what I believe to be a connection between the situation in the US leading up to {\it Kelo} and the present situation in Europe, illustrated by the fact that the European Court of Human Rights is now explicitly endorsing ``stronger protection'' of property rights.  I attempt to identify the reasons behind calls for a stricter approach, arguing that it is connected to the fact that interferences in property under modern regulatory regimes is sanctioned in wide a range of different circumstances, serving to undermine their status as a necessary burden imposed on owner's according to the will of the greater public. In some cases, rather, takings appear to both owners and the public as improperly motivated and socially and politically unfair. I note that this happens particularly often in economic development cases, when commercial actors benefit to the detriment of local communities. I go on to list some concrete issues that arise with respect to such takings and that have been flagged as problematic in the literature.
%
%Following up on this, I consider various proposals that have been made to resolve tensions and limit the possibility of abuse in economic development cases. The differences of opinion that have been expressed in this regard have been quite substantial, and proposals have ranged from suggesting an outright ban on economic development takings  (Somin 2007; Cohen 2006) to suggesting that the best way forward is to reassess principles for awarding compensation in such cases (Householder 2007; Lehavi and Licht 2007).

%Much of the current theory focus on assessing traditional judicial safeguards that courts can rely on to prevent abuses, pertaining primarily to the material assessment of proportionality, public purpose, and compensation. 

In the last part of the chapter I will focus on a very interesting strand of recent work in the US, which shifts attention away from the public use test towards procedural safe-guards. The core idea is that the manner in which eminent domain decisions are typically made, and the way in which owners are compensated, might be unsuitable for economic development cases. Importantly, the need for special procedures has been noted, to restore legitimacy for these case types. This ties the US debate yet closer to the European context, where proportionality, not public use, has become the key notion in property protection. Also, it allows us to be very clear about a special concern that arises for economic takings cases: under current regulatory regimes, the government and the developer together often dominate the decision-making process completely, leaving the property owners marginalized. Hence, there is often a {\it democratic deficit} in such cases, resulting in discontent and a feeling that the taking is not in the public interest at all. Importantly, some recent writers hypothesize that if the proper balance can be restored in the decision-making process, so will the decision reached appear more legitimate, also with respect to the public use clause. In my opinion, this idea is crucial, and together with the question of compensation, which raises a similar structural problem, it will guide the rest of the work done in this thesis. 

Most importantly, the work done in this chapter will bring into focus the following key question: What principles can be used to ensure meaningful participation and just compensation in economic takings cases, without hindering socially and economically desirable development projects?

%This question sets the stage for the remainder of my thesis, where I conduct a case study of expropriation for the development of hydro-power in Norway. In particular, I will consider two special semi-judicial procedural systems used in such cases in Norway, one targeting compensation following expropriation, and another used as an alternative to expropriation, particularly in cases when development requires cooperation among many owners.

%I conclude by arguing that approaches along procedural lines represent the best way forward in relation to addressing issues associated with economic development takings. This raises the following problem, however: what procedural principles can be used to ensure meaningful participation, without hindering socially and economically desirable development projects? This question sets the stage for the remainder of my thesis, where I conduct a case study of expropriation for the development of hydro-power in Norway. In particular, I will consider two special semi-judicial procedural systems used in such cases in Norway, one targeting compensation following expropriation, and another used as an alternative to expropriation, particularly in cases when development requires cooperation among many owners.

\section{The fair balance: Property protection under the ECHR}

The starting point for property adjudication at the ECtHR is that States have a "wide margin of appreciation" with regards to the question of whether or not an interference in property rights is to be considered legitimate in pursuance of the public interest.\footcite[See][54]{james86} This question is regarded as depending on public policy and concrete circumstances to such an extent that it is rarely appropriate for the Court to censor the assessments made by member States. At the same time, however, the Court has gradually showed an increased tendency towards actively assessing whether or not particular instances of interference are ``proportional'' and able to strike a ``fair balance'' between the interests of the public and the interests of the individual property owner.\footnote{See \cite[69]{sporrong82} and \cite[120]{james86}.}  As argued by Professor Allen, this has caused TP1-1 to attain a much wider scope than what was originally intended by the signatories.\footcite[1055]{allen10}.

In the case law behind this development the focus has predominantly been on the issue of compensation, with the Court gradually developing the principle that while TP1-1 does not entitle owners to full compensation in all cases of interference, the fair balance will likely be upset unless at least some compensation is paid, based on the market value of the property in question.\footnote{See \cite[103]{scordino06}. The case also illustrates that the Court has come to adopt a fairly strict approach to the question of when it is legitimate to award less than full market value.} The focus on compensation has also been reflected in academic work on TP1-1, which tends to address proportionality from a financial perspective, focusing on the extent to which owners are entitled to compensation based on the market value of their property. Indeed, when considering the best known case law and literature on the subject, one is left with the impression that "fair balance" with regards to TP1-1 is crucially linked to financial entitlements, primarily used as a standard that can justify a right to compensation that goes beyond what the wording of TP1-1 might initially suggest.

In recent case law, however, it has become clear that the fair balance test is meant to be more than a yardstick for assessing whether adequate compensation has been paid to owners affected by an interference. In {\it Chassagnou and others v France} the situation was that property owners were compelled to permit hunting on their land, following compulsory membership in a hunting association which was set up to manage hunting in the local area.\footcite{chassagnou99} They protested this on the grounds that they were ethically opposed to hunting, and the Court agreed that there had been a breach of TP1-1.  In the later case of {\it Hermann v Germany} the circumstances were similar, and the Court followed the precedent set in {\it Chassagnou}, commenting also that they had ``misgivings of principle'' about the argument that financial compensation could provide adequate protection in such a case.\footcite[See][91]{hermann12}  In this way, the hunting cases illustrate that the right to property is not a mere financial entitlement, and that the fair balance that must be struck could pertain to other aspects, such as the owner's right to make use of his property in accordance with his convictions and to take part in decision-making processes regarding how it should be managed.\footnote{That the assessment should be concrete and contextual was made clear in \cite{chabauty12}. In this case, the Court found no violation of TP1-1 although the facts seemed close to those of {\it Chassagnou}. The case differed, however, in that the owner himself was not opposed to hunting, but wanted to withdraw his land from the hunters' association to enjoy exclusive hunting rights.}

In a different but related development, the Court has also adopted a distinctly broad view in recent cases involving rent control schemes and housing regulation. While there are obvious financial interests at stake, both for landlords and tenants, the Court has not shunned away from using concrete cases as a starting point for providing an assessment of the fairness of national law more generally. In {\it Hutten-Czapska v Poland} the Court concluded that the facts of the case demonstrated ``the existence of an underlying systemic problem, which is connected with a serious shortcoming in the domestic legal order'', and they called for general measures to be put in place to remedy the situation.\footcite[191]{hutten06}

Interestingly, the Court relied on a highly contextual understanding of the fair balance test to reach this result, looking to the practical consequences of current legislation and administrative practices, as evidenced by the circumstance of the case. The Court reasons as follows regarding their understanding of the fair balance test:

\begin{quote}
In assessing compliance with Article 1 of Protocol No. 1, the Court must make an overall examination of the various interests in issue, bearing in mind that the Convention is 	intended to safeguard rights that are ``practical and effective''. It must look behind 	appearances and investigate the realities of the situation complained of. [...] Uncertainty -- be it legislative, administrative or arising from practices applied by the authorities -- is a factor to be taken into account in assessing the State’s conduct.
\end{quote}\footcite[151]{hutten06}

This passage was subsequently quoted in the recent case of {\it Lindheim and others v Norway}, where the applicants complained that they were in need of protection from a recent Norwegian Act of Parliament that gave lessees the right to demand indefinite extensions of ground leases on pre-existing conditions.\footcite[119]{lindheim12}  In the end, the Court concluded that TP1-1 had been violated and that the Act itself was the underlying source of the violation. Consequently, the Court ordered that general measures had to be taken by the Norwegian State to remedy the situation. The Court considered counterarguments based on earlier case law, but commented that their decision should be regarded in light of ``jurisprudential developments in the direction of a stronger protection under Article 1 of Protocol No. 1''.\footcite[135]{lindheim12}

While the problems associated with economic development takings have not, as far as I am aware, been considered by the ECtHR, it seems that the recent developments in the direction of a more contextual and strict approach to the fair balance principle, is worth noting. It seems to suggest that a crisis of confidence in the legitimacy of expropriation might be under way in Europe, similar to that seen in the US.

\section{The US perspective on economic takings}

In this Section I map the main problems that have been discussed in relation to cases of economic development takings in the US. I note how conomic development cases attract much more attention now than 20 years ago, and I present an historical overview aiming to give the reader an idea of how the debate regarding economic takings have developed into its present state. I argue that the recent surge of academic interest in this topic reflects a recent crisis of confidence in the legitimacy of takings in the US, specifically related to cases when the rationale behind the use of compulsion is commercial in nature. Following up on this, I go on to analyse in more detail some recent approaches used to address economic development takings. I argue that recent work from the US provide a useful conceptual framework for addressing such takings as a special category, and that it also serves to illustrate that such takings should indeed be considered a separate issue. %in particular, that the worry in these cases is that there is an imbalance of interest and power that may lead to both real and perceived abuses.

I will pay particularly attention to how constitutional objections to economic development takings, based on strict interpretations of the fifth amendment, appear less important to the debate than it may appear at first sight. It is true that many commentators, including some scholars, have argued for a strict understanding of property protection and they express their concerns more forcefully that what is commonly seen in Europe. However, many scholars in the US also rely on a broad and contextual understanding of property rights, and their work, which has perhaps not received the same level of attention, do in fact closely resemble the approach to property protection adopted by the ECtHR. I make special note of the fact that a great number of voices approach the issue as a question that crucially involves notions of fairness and democratic legitimacy.  I argue that such arguments are often socio-legal in nature, emerging from empirical considerations. Hence, they appear highly relevant also outside the context of the US legal system and the political tradition that accompanies it.

Many academics in the US argue that economic takings are particularly problematic under current practices and that there is need for reform. Moreover, the debate after {\it Kelo} shows that concern about such takings has become more widespread, also among academics that do not necessarily endorse a liberal, individualistic, view on the nature of property. In the last part of this section, I examine a few recent responses that aim to provide a bride across the ideological divide that otherwise dominates the takings debate in the US. I will refer to these as {\it institutional} approaches. They are characterized by a focus on the administrative and judicial procedures that are used in economic takings cases. The idea is that legitimacy of such takings can only be achieved if special procedures are followed, to ensure fairness both in the decision-making process and with regards to the issue of what compensation should be paid following condemnation. Importantly, the need for special procedures is identified with the imbalance of power that tend to exist between developers and property owners in such cases. Hence, the proposals do not argue in favor of limiting the use of eminent domain generally, and should not be read as rhetoric in favor of a more absolutist or liberal view of property rights. As such these proposal become more broadly relevant, also outside the US context. It is also a perspective that directly ties the US debate to the case study presented in subsequent chapters of this thesis. In particular, in Section \ref{sec:ins} I give an in depth presentation of two recent reform proposals, one regarding compensation principles and the other focusing on the decision-making process as such. Both contain (aspects of) the two working institutions I will consider in my case study, in Chapter \ref{chap:4} and Chapter \ref{chap:5} respectively.

The first is a proposal by Professors Lehavi and Licht, which argues for the introduction of special-purpose development corporations to ensure appropriate levels of compensation in economic development cases.\footcite{lehavi07} The crux of their proposal is to award property owners shares in a development company that is set up to bargain with potential developers. Importantly, the eminent domain decision precedes this round of bargaining, so the owners cannot threaten to refuse selling the land in order to get additional compensation based on the public need for development. Bargaining is restricted to the commercial element of the project, reflected in competition that may arise among developers interested in carrying out the project. This proposal will serve as a point of departure when we consider the Norwegian appraisal courts and recent case law on compensation of waterfalls.

The second proposal we will consider in depth is due to Professors Heller and Hills, who argue for the introduction of so-called ``land assembly districts'', institutions for participatory pooling of property for development projects.\footcite{heller08} Land assembly  districts are designed to partially replace the use of eminent domain for economic development, giving property owners both a template on which to bargain with developers and also the final say on whether land assembly should happen at all. The authors argue that with appropriate procedures in place for making collective decisions, such a system will be sufficient to avoid holdouts preventing socially desirable projects. In fact, they argue quite convincingly that even a simple scheme of majority voting will be sufficient in many cases, due to the commercial incentives that are present when land assembly is needed for economic development. Importantly, their proposal is not mean to apply to land assembly in general. It should only be used when reasonable market conditions can be achieved without the use of eminent domain. As we will see in Chapter \ref{sec:5}, this proposal corresponds closely to important aspects of various land consolidation procedures currently in operation under Norwegian law. In particular, I will show how recent case law on land consolidation for hydroelectric development in Norway demonstrates the efficiency and usefulness of a similar procedure, used to assemble water rights under fragmented ownership in Norway.

Before I move on to discuss these recent proposals in more detail, I will give an historical background on economic takings in the US. This will give the reader a better appreciation of the developments that led to the current climate of debate, showing how economic takings became a natural special category in the US. In particular, I will argue that the US system has tended to be more open to the idea of commercial projects benefiting from eminent domain, so that the level of tensions here were naturally higher than those found in Europe. However, recent European trends towards greater levels of public/private commercial partnerships in important public projects may suggest that the US discourse will soon become easier to recognize also in the European setting. The history of the debate in the US shows, in any event, that conflicts over takings law become aggravated whenever the perception takes hold that powerful commercial interests are permitted to usurp the process to their own advantage.

\subsection{The ``dead letter'' raised: A short history of economic takings in the US}

Going back to the time when the Fifth Amendment was introduced, there is not much historical evidence explaining why the takings clause was included in the bill of rights, and little in the way of guidance as to how it was originally understood. James Madison, who drafted it, commented that his proposals for constitutional amendments were intended to be uncontroversial to Congress.\footnote{See letters from Madison to Edmund Randolph dated 15 June 1789 and from Madison to Thomas Jefferson dated 20 June 1789, both included in \cite{madison79}.}  Hence, it is natural to regard it more as a codification of an existing principle, rather than a novel proposal. Indeed, several State constitutions pre-dating the Bill of Rights also included takings clauses, and they all seem to be largely based on a codification of principles from English Common law.\footcite[See][299]{johnson11}

In common law, the right to property has a long tradition behind it, dating back to the Magna Carta. In his {\it Commentaries on English Law}, William Blackstone famously described it as the ``third absolute right'' that was ``inherent in every Englishman''.\footcite[134-135]{blackstone79}.  Moreover, Blackstone expressed a very restrictive view on the possibility of expropriation, arguing that it was only for the legislature to interfere with property rights, warning against the dangers of allowing private individuals, or even public tribunals, to be the judge of whether or not the ``common good'' could justify it. Blackstone went as far as to say that the public good was ``in nothing more invested'' than the protection of private property.\footcite[134-135]{blackstone79}

On this background it is not surprising that Madison regarded the property clause as an uncontroversial amendment.\footnote{Indeed, early American scholars also emphasized the importance of private property. For instance, in his famous {\it Commentaries}, James Kent described the sense of property as ``graciously implanted in the human breast'' and declared that the right of acquisition ``ought to be sacredly protected'', \cite[see][257]{kent27}.} Its importance may in fact have been greater as a legitimizing force, increasing confidence in the regulatory power of the newly established state by setting up clear parameters for the exercise of that power.\footnote{references.}  However, while the principle was regarded as theoretically self-evident, it never seems to have been entirely clear what it meant in practice, particularly for takings of property when it was unclear to what extent it could be said that it was put to ``public use''.\footcite[See][317]{johnson11} 

There are two points that I would like to record about the early common law in the US  in this regard. First, the distinction between public use and public purpose does not appear to have been considered sharp. For instance, in his {\it Commentaries}, James Kent first makes clear that the power of eminent domain is for ``public use, and public use only", but then goes on to qualify this by stating that a taking which served a ``purpose not of a public nature'' would be unconstitutional.\footcite[See][275-276]{kent27}  Moreover, it seems to have been accepted that takings which clearly benefited the public would be acceptable, regardless of whether or not the property was physically put to use by the public.\footnote{References.} The crucial principle encoded in the Fifth Amendment was the right to compensation, and this right was considered fundamental.\footnote{James Kent held it to be  ``founded in natural equity'' and described it as an ``acknowledged principle of universal law'', \cite[see][276]{kent27}.}

An interesting early illustration of how the public use clause was understood can be found in {\it Stowell v Flagg}, a Massachusetts case from 1814, where an act that allowed mill owners to cause damage to adjacent land by flooding was held to satisfy the public use requirement.\footcite{stowell14} The court highlighted the purpose of the interference, commenting that ``these mills, early in the settlement of this country, were of great public necessity and utility''.\footcite[366]{stowell14} At the same time, the court had misgivings about how the act had come to be applied and expressed concern that ``the legislature, as well as the courts of law in this state, seem to have been disposed rather to enlarge, than to curtail, the power of mill owners''.\footcite[366]{stowell14} Still, after noting that  affected land owners were entitled to compensation under the act, the court concluded that the act had to be observed and that it precluded any claims for damages under common law.\footnote{The land owner had claimed that a common law claim for damages could be made, irrespectively of the mill act.} Hence, the case is an early example of judicial deference to the legislature in takings cases. Despite declaring that he could not help thinking that the statute was ``incautiously copied from the ancient colonial and provincial acts'', the presiding judge held in favor of the mill owner,  concluding that ``as the law is, so must we declare it''.\footcite[368]{stowell14}

While judicial deference was recognized as a guiding principle early on in US takings law, it is important to note in this regard that eminent domain was seldom used in a way that would raise serious controversy. English common law, while lacking clearly defined constitutional safeguards, was based on a fundamentally cautious attitude, ensuring that the power would typically only be used as a last resort. As Professor Meidinger notes, the British were never really charged with abuse of eminent domain, and private property tended to be respected, also in the colonies.\footcite[17]{meidinger80} This undoubtedly influenced early US law. Indeed, the importance of property protection was considered fundamental early on, as reflected in {\it de dicta} comments made by judges in the early cases of {\it Calder v Bull} and {\it Vanhorne’s Lessee v Dorrance}.\footnote{\cite[388]{calder98} and \cite[310]{vanhorne95}.} This shows that the constitutional limit of the takings power was clearly recognized and emphasized as a matter of principle. Hence, the relative lack of judicial interest in the question of legitimacy does not appear to have been due to a broad view on the scope of eminent domain. It seems more natural to attribute it to absence of controversy, resulting from an established practice of narrow use, inherited from the English.
\noo{
The Legislature declare and enact, that such are the public exigencies, or necessities of the State, as to authorise them to take the land of A. and give it to B.; the dictates of reason and the eternal principles of justice, as well as the sacred principles of the social contract, and the Constitution, direct, and they accordingly declare and ordain, that A. shall receive compensation for the land. But here the Legislature must stop; they have run the full length of their authority, and can go no further: they cannot constitutionally determine upon the amount of the compensation, or value of the land. Public exigencies do not require, necessity does not demand, that the Legislature should, of themselves, without the participation of the proprietor, or intervention of a jury, assess the value of the thing, or ascertain the amount of the compensation to be paid for it. This can constitutionally be effected only in three ways.
1. By the parties that is, by stipulation between the Legislature and proprietor of the land.
2. By commissioners mutually elected by the parties.
3. By the intervention of a Jury.
}
However, the traditional attitude to eminent domain would eventually give way to a more expansive sentiment. This development became particularly marked during the period of great economic expansion and industrialization in the mid to late 19th century, when eminent domain was increasingly used to benefit (privately operated) railroads, hydroelectric projects, and the mining industry.\footcite[23-33]{meidinger80} During this time, it also became increasingly common for landowners to challenge the legitimacy of takings in court, undoubtedly a consequence of the fact that eminent domain was now used more widely.\footcite[24]{meidinger80} Controversy arose particularly often with respect to mill acts.\footnote{\cite[24]{meidinger80}. See also \cite[306-313]{johnson11} and \cite[251-252]{horwitz73}.} Such acts were found throughout the US, and many of them dated from pre-industrial times when mills were primarily used to serve the needs of self-sufficient agrarian communities.\footnote{A total of 29 states had passed mill acts, with 27 still in force, when a list of such acts was compiled in \cite[17]{head85}. According to Justice Gray, at pages 18--19 in the same, the ``principal objects'' for early mill acts had been grist mills typically serving local agrarian needs at tolls fixed by law, a purpose which was generally accepted to ensure that they were for public use.}  However, following economic and technological advances, acts that were once used to facilitate the construction of grist mills would increasingly also be relied on by developers wishing to harness hydropower for manufacturing, and eventually, for hydroelectric projects.\footnote{See, e.g., \cite[18-21]{head85} and \cite[449-452]{minn06}.}

The mill acts typically contained provisions that enabled the mill developer to condemn property needed for the construction, as well as the right to damage surrounding land by flooding or deprivation of water. Such takings became increasingly controversial, however, and many legitimacy cases came before state courts in the late 19th and early 20th century. In these cases, we find the first clear evidence of how different interpretations of the public use requirement began to develop and diverge. Generally speaking, when a court upheld an interference in private property, it would place decisive weight on the broader purpose of interference, typically by arguing that economic ripple effects ensured that the mill was in the public interest even if the public would not literally make use of it.\footnote{See, e.g., \cite{hazen53,scudder32,boston32}. A more comprehensive list of cases adopting a broad view can be found in \cite[617]{nichols40}.} By contrast, when a court decided that an interference was unconstitutional (with respect to state constitutions), it would tend to focus on the actual use made of the mill, arguing that it did not directly benefit the public in the sense required by the public use restriction.\footnote{See, e.g., \cite{sadler59,ryerson77,gaylord03,minn06}. A more comprehensive list can be found in {\it Public benefit or convenience as distinguished from use by the public as ground for the exercise of the power of eminent domain} 54 ALR 7 (American Law Reports, 1928).} For a time, a doctrine which sought to distinguish sharply between public use and public purpose, striking down the former kind as unconstitutional, played quite a significant role in many states.\footnote{Professor Nichols goes as far as to conclude that it emerged as the ``majority'' opinion on public use, see \footcite[617-618]{nichols40}. But contrast this with \cite{berger78} and \cite[24]{meidinger80}, who argue that the narrow view was only dominant in a handful of states, led by New York.}

\noo{ For instance, in the case of {\it Gaylord v. Sanitary Dist. of Chicago}, the Supreme Court of Illinois held the state Mill Act to be unconstitutional, as it was not limited to traditional flour mills. In doing so, the court observed that public use was ``something more than a mere benefit to the public''.\footcite[524]{gaylord03} Similar sentiments were expressed in other decisions striking down uses of eminent domain for mill construction, for instance in Vermont, Michigan and New York.\footnote{References.}}

It is tempting to associate the narrow view on public use with a more restrictive attitude towards eminent domain generally. Similarly, it is natural to assume that adoption of a broad view suggests a more relaxed attitude. To some extent, the primary sources seems to warrant this; unsurprisingly, those who endorsed a broad view on the public use question also often spoke in favor of judicial deference in legitimacy cases, while those endorsing a narrow view tended to emphasize the importance of constitutional safeguards against abuse of eminent domain. However, it is important to keep in mind that both groups were quite heterogeneous and did not necessarily subscribe to the same reasons for their respective views. Indeed, both the narrow and broad interpretation had many supporters who relied on subtle arguments that defy categorization along any obvious axis of ``property friendliness'' or the like.

It is clear, for instance, that many of the courts which favored a broad interpretation of public use still viewed the constitutional limitation on the takings power as an important safeguard, not only as a guarantee for compensation but also as a restriction on the purpose of takings. Indeed, it seems that most late 19th Century Courts, including those that upheld economic takings, were influenced by the growing body of case law across the US that had struck down such takings as unconstitutional. In particular, it seems that the strict deferential view was largely abandoned in economic takings cases. Deference to the legislature still played an important role, of course, but it became much more common to argue for legitimacy primarily in terms of substantive arguments, by directly addressing the context and circumstances of the taking or act complained of. I believe this is an important insight to record about the case law from this period; despite differences of opinion about the meaning of public use, a consensus appears to have emerged that judicial review of legitimacy was appropriate and important in economic takings cases.

A good example is the case of {\it Dayton Gold \& Silver Mining Co. v. Seawell}, concerning a Nevada Act which stipulated that mining was a public use for which the power of eminent domain could be exercised to acquire additional rights needed to facilitate extraction.\footcite{seawell76} The Supreme Court of Nevada decided that the Act was constitutional and adopted a broad understanding of the property clause in the Nevada constitution.\footnote{Nev Const Art 8 § 1.} Interestingly, it argued for this interpretation partly on the basis that it would provide {\it better} protection for landowners.\noo{Why not? A hotel is used by the public as much as a railroad. The public have the same right, upon payment of a fixed compensation, to seek rest and refreshment at a public inn as they have to travel upon a railroad. 

One purpose is, so far as the legal rights of the citizen are concerned, as public as the other.}

\begin{quote}
If public occupation and enjoyment of the object for which land is to be condemned furnishes the only and true test for the right of eminent domain, then the legislature would certainly have the constitutional authority to condemn the lands of any private citizen for the purpose of building hotels and theaters. [...] Stage coaches and city hacks would also be proper objects for the legislature to make provision for, for these vehicles can, at any time, be used by the public upon paying a stipulated compensation. It is certain that this view, if literally carried out to the utmost extent, would lead to very absurd results, if it did not entirely destroy the security of the private rights of individuals. Now while it may be admitted that hotels, theaters, stage coaches, and city hacks, are a benefit to the public, it does not, by any means, necessarily follow that the right of eminent domain can be exercised in their favor.\footcite[410-411]{seawell76}
\end{quote}

The quote shows that a broad understanding of ``public use'' need not be synonymous with a less cautious attitude to abuse of the takings power. Indeed, while the Court decided to uphold the Act, it did so only after a very careful assessment of both legal arguments and factual circumstances. In particular, the Court considered the importance of mining, concluding that it was the ``greatest of the industrial pursuits'' in the state, and that all other interests were ``subservient'' to it.\footcite[409]{seawell76} Moreover, the Court commented that the benefits of the mining industry was ``distributed as much, and sometimes more, among the laboring classes than with the owners of the mines and mills''.\footnote[409]{seawell76}

This shows that the Court actively engaged with the purpose of the Act, thoughtfully assessing it against the constitution. Importantly, it did not do so in isolation, as a linguistic exercise or by attempting to recreate its ``original intent''. Rather, the court approached the constitutional safeguard by making detailed references to the prevailing social and economic conditions in the state of Nevada. The Court also duly noted the importance of deference to the legislature on matters of policy, but it did so only after it had satisfied itself that the Act could be ``enforced by the courts so as to prevent its being used as an instrument of oppression to any one''.\footcite[412]{seawell76} More generally, the court commented as follows on the public purpose test that had to be performed in takings cases, elucidating on the principles on which it should be founded:

\begin{quote}
 Each case when presented must stand or fall upon its own merits, or want of merits. But the danger of an improper invasion of private rights is not, in my judgment, as great by following the construction we have given to the constitution as by a strict adherence to the principles contended for by respondent.\footcite[398]{seawell76}
\end{quote}

In light of this, {\it Dayton Gold \& Silver Mining Co. v. Seawell} must be regarded as an early example of a {\it contextual} approach to legitimacy, characterized by the willingness of the Court to engage in a fairly detailed analysis of the concrete circumstances and consequences of takings. A formalistic approach based on the phrase ``public use'' was abandoned, but not in favor of general deference. Rather, a more nuanced view was adopted, to respect the idea that the legislature should have the final say on policy while also recognizing that courts play a crucial role in protecting citizens from abuse of the takings power. The case is not unique, but rather exemplifies the type of reasoning that was used in economic takings cases at this time. Interestingly, many common elements exist between courts that upheld and struck down such takings, irrespectively of whether or not they subscribed to a narrow or broad view on the public use test. 

%In fact, upon closer inspection of the primary sources, it becomes clear that the common element goes back all the way to the time %when the two public use doctrine first began to diverge. 

One example is {\it Ryerson v. Brown}, a case often cited as an authority in favor of a narrow view.\footnote{...} Here the Supreme Court of Michigan explicitly qualifies its decision by stating that it is ``not disposed to say that incidental benefit to the public could not under any circumstances justify an exercise of the right of eminent domain''. The case concerned the constitutionality of a mill act, and while the court argues that public use should be taken to mean ``use in fact'', it is clear that ``use'' is understood rather loosely, not literally as physical use of the property that is taken.\footnote{The court explains its stance on the public use restriction by stating (emphasis added) ``it would be essential that the statute should require the use to be public in fact; in other words, that it should contain provisions entitling the public to {\it accommodations}.'' The court continues with an illustrative example: ``A flouring mill in this state may grind exclusively the wheat of Wisconsin, and sell the product exclusively in Europe; and it is manifest that in such a case the proprietor can have no valid claim to the interposition of the law to compel his neighbor to sell a business site to him, any more than could the manufacturer of shoes or the retailer of groceries. Indeed the two last named would have far higher claims, for they would subserve actual needs, while the former would at most only incidentally benefit the locality by furnishing employment and adding to the local trade''. See \cite[336]{ryerson77}.} Moreover, when clarifying its starting point for judicial scrutiny of mill acts, the court explains that ``in considering whether any public policy is to be subserved by such statutes, it is important to consider the subject from the standpoint of each of the parties''. Following up on this with regards to the act in question, the court finds that `` the power to make compulsory appropriation, if admitted, might be exercised under circumstances when the general voice of the people immediately concerned would condemn it''. After considering this and other possible consequences of mill development under the act, the court eventually declares it to be unconstitutional, summing up its assessment as follows: ``What seems conclusive to our minds is the fact that the questions involved are questions not of necessity, but of profit and relative convenience''.

Hence, far from nitpicking on the basis of the public use phrase, the court adopts a contextual approach to takings that is in fact rather similar to the approach of {\it Dayton Gold \& Silver Mining Co. v. Seawell}. The outcome it different, but it is also based on a perceived difference in the context and consequences of the takings complained of. Importantly, it does not rest on any {\it a priori} assumption that economic takings of the kind in question could not possibly meet a public use test. It is somewhat curious that later commentators have focused so exclusively on the case for its comments on public use rather than its broad, albeit perhaps conservative, assessment of legitimacy. However, the case is not unique. It seems that many of the cases from the late 19th Century, on both sides of the public use debate, shares many of its features.\footnote{See, e.g., \cite{scudder32} (Eminent domain power upheld, but said: ``The great principle remains that there must be a public use or benefit. That is indispensable. But what that shall consist of, or how extensive it shall be to authorize an appropriation of private property, is not easily reducible to a general rule. What may be considered a public use may depend somewhat on the situation and wants of the community for the time being.''), \cite{fallsburg03} (Eminent domain struck down, on holding that ``the private benefit too clearly dominates the public interest to find constitutional authority for the exercise of the power of eminent domain''), \cite[538]{board91} (Eminent domain struck down, qualified by ``not only must the purpose be one in which the public has an interest, but the state must have a voice in the manner in which the public may avail itself of that use'').}

In my opinion, this points to an interesting alternative perspective on legitimacy adjudication from this time. Later commentators tend to describe the case law as chaotic, with competing conceptions of constitutional limits competing for dominance.\footcite{nichols40,berger78,meidinger80}. To some extent I agree, but I also find evidence that there was in fact a broad consensus in this period regarding the need for special judicial scrutiny of economic development cases. State courts widely engaged in contextual assessment of legitimacy, and they were conscious of the special challenges that arose in a time when eminent domain was being used extensively to benefit commercial actors as instruments of massive economic expansion. Differences of opinion about public use terminology was an important aspect of this, but it was rarely considered in isolation from other aspects. On a deeper lever, the fact that the public use debate was regarded as important in the first place clearly suggests that deference to the legislature was not held to be an exhaustive answer to the question of legitimacy of private-to-private takings. This, in my opinion, is an important observation which appears to have been somewhat overlooked in the literature. 

It is also relevant when considering later developments in the case law, particularly after the federal courts began to develop their own takings doctrine. While the narrow view of public use was indeed losing ground at the beginning of the 20th Century, the doctrine of extreme deference that was about to be adopted by the Supreme Court appears to represent a largely new development. This new kind of deference was not only directed towards the legislature, but also towards the judiciary at the state level. Hence, it represent a development that is in some sense incomparable to the earlier case law from the states. The balance of power between states and the federal government also played an important role, which should not be overlooked. 

Initially, the Supreme Court held that the takings clause in the US Constitution did not apply to state takings at all.\footcite{barron33} Federal takings, on the other hand, were of limited practical significance since the common practice was that the federal government would rely on the states to condemn property on their behalf.\footcite[30]{meidinger80}. This changed towards the end of the 19th Century, however, particularly following the decision in {\it Trombley v. Humphrey}, where the Supreme Court of Michigan struck down a taking that was going to benefit the federal government.\cite{trombley71} Not long after, in 1875, the first Supreme Court adjudication of a federal taking case occurred, marking the start of the development of the Supreme Court's own doctrine on public use and legitimacy.\footcite{kohl75} Eventually, in 1897, the Court would also hold that state takings could be scrutinized under the takings clause of the constitution.\footcite{chicago97} This was a development that can be traced to the passage of the Fourteenth Amendment to the Constitution after the civil war, concerning due process.\footcite{johnson11}. Indeed, some early Supreme Court cases dealing with state takings were adjudicated against this provision rather than the takings clause.\footnote{See, e.g., \cite{head85}.}

After the Supreme Court started developing its own case law on the legitimacy issue, the deferential stance soon became entrenched. Initially, it seems that deference was directed just as much at the state courts, however, as towards the legislature. Even so, the Supreme Court showed a distinct unwillingness to strike down takings on constitutional grounds, and this probably influenced the further development of state law as well. As argued by Professor Horwitz, the mid to late 19th Century was the period in US history when control over property was transferred on a massive scale from agrarian communities to various agents of industrial expansion.\footcite{horwitz73} Moreover, it was a period of great optimism about the ability of {\it laissez faire} capitalism to ensure progress and economic growth, and this was also reflected in the case law on eminent domain, particularly as developed by the Supreme Court.

A particularly clear expression of this can be found in {\it Mt. Vernon-Woodberry Cotton Duck Co v Alabama Interstate Power Co}.\footcite{vernon16}  This case dealt with the legitimacy of a condemnation arising from the construction of a hydropower plant, which the Alabama Supreme Court had upheld against claims that it was unconstitutional under the constitution of Alabama. The presiding judge held that it was valid using quite brisk language:

\begin{quote}The principal argument presented that is open here, is that the purpose of the condemnation is not a public one. The purpose of the Power Company's incorporation, and that for which it seeks to condemn property of the plaintiff in error, is to manufacture, supply, and sell to the public, power produced by water as a motive force. In the organic relations of modern society it may sometimes be hard to draw the line that is supposed to limit the authority of the legislature to exercise or delegate the power of eminent domain. But to gather the streams from waste and to draw from them energy, labor without brains, and so to save mankind from toil that it can be spared, is to supply what, next to intellect, is the very foundation of all our achievements and all our welfare. If that purpose is not public, we should be at a loss to say what is. The inadequacy of use by the general public as a universal test is established. The respect due to the judgment of the state would have great weight if there were a doubt. But there is none.\footcite[]{vernon16}
\end{quote}

The quote serves as an indication of how deference was fast gaining ground, without yet being established doctrine. On the one hand, the Court stresses that deference to the {\it state} judgment (rather than the judgment of the legislature) should be given great weight in legitimacy cases. On the other hand, it prefers to conclude on the basis of its own assessment of the purpose of the taking. This assessment, however, is not particularly grounded in the circumstances on the ground in Alabama, being based rather on sweeping assertions about the ``organic relations of modern society'' and the desire to ``save mankind from toil that it can be spared''. 

This judgment, from 1916, was given during the so-called {\it Lochner} era of jurisprudence in the US, when the Supreme Court  would famously engage in active censorship of regulation that was meant to promote greater social and economic equality.\footcite{cohen08} In particular, much case law from this period witnesses to a general lack of deference. Hence, it is not unexpected to find that public use cases decided on the basis of substantive arguments. However, it is rather more surprising to find that deference actually played an increasingly important role in takings cases. As early as { \it United States v. Gettysburg Electric Railway Co.}, in 1896, deference was described as a fundamental guiding principle, which should be adhered to except in very special circumstances.\footcite{gettysburg96} In particular, Justice Peckham lended his support to the following deferential stance on the public use test:

\begin{quote}
It is stated in the second volume of Judge Dillon's work on Municipal Corporations (4th Ed. § 600) that, when the legislature has declared the use or purpose to be a public one, its judgment will be respected by the courts, unless the use be palpably without reasonable foundation. Many authorities are cited in the note, and, indeed, the rule commends itself as a rational and proper one.\footcite[680]{gettysburg96}
\end{quote}

The case did not turn on the public use issue, however, as the condemned land would be used for battlefield memorials at Gettysburg, Pennsylvania, clearly a public use. Moreover, in later cases the point of view espoused was not universally adopted. As late as in 1930, the Supreme Court commented that the ``‘It is well established that, in considering the application of the Fourteenth Amendment to cases of expropriation of private property, the question what is a public use is a judicial one".\footcite[447]{vester30} 
In this judgment, Chief Justice Hughes also describes in more depth how the judicial assessment of the public use question should be carried out, echoing the contextual approach that had been developed in case law from the states.

\begin{quote}
In deciding such a question, the Court has appropriate regard to the diversity of local conditions and considers with great respect legislative declarations and in particular the judgments of state courts as to the uses considered to be public in the light of local exigencies. But the question remains a judicial one which this Court must decide in performing its duty of enforcing the provisions of the Federal Constitution.\footcite[447]{vester30}
\end{quote}

In {\it Hairston v. Danville \& W. R. Co.} this was expressed even more clearly, by Justice Moody, who surveyed the state case law and declared that ``The one and only principle in which all courts seem to agree is that the nature of the uses, whether public or private, is ultimately a judicial question.''\footnote[606]{hairston08} He continued by describing in more depth the typical approach of the state courts in determining public use cases:

\begin{quote}
The determination of this question by the courts has been influenced in the different states by considerations touching the resources, the capacity of the soil, the relative importance of industries to the general public welfare, and the long-established methods and habits of the people. In all these respects conditions vary so much in the states and territories of the Union that different results might well be expected.
\end{quote}

Justice Moody goes on to give a long list of cases illustrating this aspect of state case law, showing how assessments of the public use issue is invariably contextual and varies from state to state.\footcite[607]{hairston08} He then cites {\it Falbrook, Clark} and {\it Strickley}, all of which express similar sentiments of support for state case law. Following up on this, he points out that ``no case is recalled'' in which the Supreme Court overturned ``a taking upheld by the state {\it court} as a taking for public uses in conformity with its laws'' (my emphasis). After making clear that situations might still arise where the Supreme Court would not follow state courts on the public issue, Justice Moody goes on to conclude that the cases cited `` show how greatly we have deferred to the opinions of the state courts on this subject, which so closely concerns the welfare of their people''. 

I believe {\it Hairston} is an important case for two reasons. First, it makes clear that initially, the deferential stance in cases dealing with state takings was largely directed at the state courts rather than the state legislature. Second, it demonstrates federal recognition of the fact that a consensus had emerged in state case law, whereby public use scrutiny was consistently regarded as a judicial task.\footnote{Indeed, it provides the authority for {\it Cincinatti} and predates it.} Moreover, the Court clearly looked favorably on the contextual approach typically adopted in such cases, whereby state courts would look to the concrete circumstances of the individual takings and acts complained of. The Court's approval of this tradition, in particular, is explicitly given as the reason for adopting a deferential stance. Put simply, the judicial test provided at state level was held to be of such high quality that there was little use for further federal scrutiny. Deference, in particular, was made contingent on the fact that state courts would provide the required judicial scrutiny of the public use requirement.

Despite this, {\it Hairston} would later be cited as an early authority in favor of a more general deferential stance in {\it U. S. ex rel. Tenn. Valley Authority v. Welch}.\footcite[552]{welch46} This case concerned a federal taking and it cited {\it U.S. v. Gettysburg Electric R. Co.} as an authority in favor of deference with regards to the public use limitation.\footcite{gettysburg96}  But the Court also noted that {\it City of Cincinnati v. Vester} declared that the public use test was a judicial responsibility.\footcite{vester30} In a very selective citation, the Court then purports to resolve this tension by quoting {\it Hairston} and the observation that the Supreme Court had never overruled the state courts in takings cases. Effectively, the importance of judicial scrutiny is downplayed, although as we saw, the rationale behind {\it Hairston} was that state courts already offered high-quality scrutiny of the purpose in takings cases.

The case is important because it is used as an authority in the later case of {\it Berman v. Parker}, which endorses almost complete deference to the legislature regarding the public use issue.\footcite[32]{berman54} This case concerned condemnation for redevelopment of a partly blighted residential area in the District of Colombia. In a key passage the Court states that the role of the judiciary in scrutinizing the public purpose of a taking is ``extremely narrow''.\footcite[32]{berman54} The Court provides only two citations for this claim, one of which is  {\it U. S. ex rel. Tenn. Valley Authority v. Welch}. The other case, {\it Old Dominion Land Co. v. U.S.}, concerned a federal taking of land on which the military had already invested large sums in buildings. Hence, neither of the two authorities seem to support the much more general deferential stance adopted by {\it Berman}.



 towards the {\it legislature}, with no mention of the tradition of careful scrutiny by state courts. 

Still, the case was picked up on in later cases in support of a more general deferential stance. and it eventually became established doctrine.\footcite{welch46}

Hence, while the {\it Lochner} era in general was characterized by courts engaging in censorship of state regulation, this general tendency was not reflected in how eminent domain law developed over the same period. I believe this is quite important to note, since it also reflects the shortcoming of another commonly held view on property protection, namely that it largely serves the interests of property-owning elites, to the detriment of regulatory efforts to promote social equality. The cases through which Lochner era courts developed the deferential stance suggest a different interpretation; those who benefited most directly from takings in these cases were commercial interests, not vulnerable groups of society. Moreover, they benefited from acquiring land rights from members of agrarian communities, not from the elites. Hence allowing such takings to go ahead was no affront to the ideology of progress through laissez faire capitalism, quite the contrary.

In particular, if it is true as many have argued, that the Lochner courts were ideologically committed to the promotion of unrestrained capitalism, there was little reason for them to oppose expansion of eminent domain into the commercial arena: those who would be likely to benefit were market actors who were proposing large scale commercial development projects. Indeed, the case law from this period makes it natural to argue that the deferential stance developed primarily to cater to the needs of the capitalists, under the perceived view that they represented the class which would bring progress and prosperity to the nation as a whole.

In the post-Lochner period, when courts became less willing to engage in activism on behalf of this world view,  little changed in relation to eminent domain. Rather, the deferential stance was entrenched further and made more explicit. In the Supreme Court case of Berman v Parker, an important precedent for Kelo, Justice Douglas codified this attitude when he stated that the room for judicial oversight regarding the public use test was "extremely narrow". 

This was now the dominant view in the US, and it marked the victory not only for the broad interpretation of public use, but also for the general deferential stance on the issue of legitimacy of purpose. This view was dominant also 30 years later when it was upheld in Hawaii Housing Authority v. Midkiff.  However, the fact that the case made it to the Supreme Court is suggestive of an increase in the level of worry and tension associated with eminent domain in the 1980s. Indeed, Justice Sandra Day O'Connor, joined by a unanimous Supreme Court, expressed general disapproval of private takings and she appears to have felt the need to provide further qualification for the deferential view, which she did in part by making the following observation:

[...]judicial deference is required because, in our system of government, legislatures are better able to assess what public purposes should be advanced by an exercise of eminent domain.

Hence judicial deference was not regarded as an absolute and systemic imperative, as in Berman, but made contingent on the fact that legislatures are "better able" than courts at conducting public purpose tests. It should be noted that in the case of Midkiff the purpose of interference was to break up a property oligarchy to the benefit of tenants, not to further economic development by allowing commercial interests to benefit from the takings power. Hence the rationale behind the interference is likely to have struck the Supreme Court as sound and just. Implicitly, Justice O'Connor herself engage in an assessment of its merits when she points out that "regulating oligopoly and the evils associated with it is a classic exercise of a State's police powers".

In conclusion, the "extremely narrow" room for judicial review set up in Berman was replaced by a  slightly more nuanced formulation, which nevertheless made clear that a legal precedent of deference had developed in practice, and that the Supreme Court had no tradition for  adjudicating eminent domain cases on the basis of the public use clause:

where the exercise of eminent domain power is rationally related to a conceivable public purpose, the Court has never held a compensated taking to be proscribed by the Public Use Clause

So while Midkiff might reveal some implied reservations about the deferential stance entrenched in Berman, it clearly reaffirms the main principle at work:  the meaning of public use can be broad, and the room for judicial review of governmental assessments in this regard is narrow.

This view also appears to have been endorsed by most academics following WW2, causing one author to remark that the public use clause was a "dead letter" (Merrill 1986). In light of this, the increased tension associated with takings  in the 1980s can hardly be explained by pointing to legal uncertainty. However, there was a change in how eminent domain was perceived in this period, towards greater scepticism. In part it may have been caused by a general resurgence in liberal political ideology. But in addition, some concrete cases proved particularly controversial, and they were taken to illustrate the dangers of eminent domain, particularly in relation to economic development projects. Now, in particular, it was not only natural resources and land that was subject to eminent domain; the takings power was used more aggressively, to condemn middle class homes.
The controversy surrounding the case of Poletown Neighborhood Council v. City of Detroit  illustrates this, and the case marks a watershed moment in the history of  economic development takings in the US, see e.g., (Underkuffler 2006, 380–381). In Poletown, the Michigan Supreme Court held that it was not in violation of the public use requirement to allow General Motors to displace some 3500 people for the construction of a car assembly factory.  This decision proved highly controversial, however, and it was later overturned in the Michigan Supreme Court, in County of Wayne v. Hathcock, a move widely seen as a response to the increased critical attention directed at these kinds of takings. 

While it never reached the Supreme Court, the case of Poletown seems to represent the prelude to the subsequent controversy that arose regarding Kelo, causing the surge of interest in eminent domain we have seen in recent years. In the next section, we turn to this period in more detail.

\subsection{The "grasping hand": Legitimacy of economic takings in the US after Kelo}

Shortly after Poletown was overturned, the case of Kelo saw the legitimacy of economic takings brought before the Supreme Court once again. This time there was real doubt and disagreement among the justices regarding the scope of the public use limitation. The case revolved around the legitimacy of condemning a home in favour of a research facility for the drug company Pfizer, which was part of a development plan for the City of New London.  The owner, Suzanne Kelo, argued that the condemnation of her home was in breach of the constitution, since it was a private-to-private taking ostensibly to the benefit of Pfizer rather than any clearly defined public use or interest.

In Kelo, Justice Thomas adopted the strictest view on the public use test. He entirely disregarded  the precedent set by Berman and Midkiff in favour of constitutional originalism, the doctrine which asserts that direct assessment of the wording in the Constitution, and the intentions of the founding fathers, is the approach that should be used to decide constitutional cases. Following up on this he held that actual right of use for the public was the test that had to be applied in takings cases. The hundred years of precedent preceding Kelo was described as “wholly divorced from the text, history, and structure of our founding document", and thus Justice Thomas concluded that it had to be abandoned. 

Justice O'Connor, in an expression of dissent joined by Chief Justice Rehnquist and Justices Scalia
and Thomas, argued against legitimacy on less theoretical grounds, based on the facts of the case and the precedent that would be set for similar cases in the future. Her main legal argument was that while public use should be interpreted broadly, the possibility of positive ripple effects was not enough to justify private-to-private takings. In particular, Justice O'Connor took a very bleak view on the practical consequences that would arise from allowing economic takings that could be justified only by pointing only to indirect positive consequences for the public. She commented on the majority decision to uphold the taking as follows: 

Any property may now be taken for the benefit of another private party, but the fallout from this decision will not be random. The beneficiaries are likely to be those citizens with disproportionate influence and power in the political process, including large corporations and development firms. As for the victims, the government now has license to transfer property from those with fewer resources to those with more. The Founders cannot have intended this perverse result.

It seems that a major point of contention among the judges in the Supreme Court was whether or not these grim predictions was a realistic assessment of what the consequences of the decision would be. Surely, anyone who agrees with Justice O'Connor in her prediction of the fallout would also agree with here conclusion that it is perverse. But the majority in Kelo, in an opinion written by Justice Stevens, disagreed with her assessment, observing instead that a more restrictive view on economic takings would make it more difficult to cater to the "diverse and always evolving needs of society". 
But the majority opinion also stressed that purely private takings where not permissible, and they attached great significance to the substantive assessment that the actual taking of Suzanne Kelo's home formed part of a comprehensive development plan that would not bestow special benefit on any particular group of individuals. Moreover, Justice Kennedy, in his concurring opinion, emphasised that states should not use public purpose as a pretext for interfering in property rights to the benefit of commercial actors.
Hence the overall impression one is left with when considering Kelo in its historical and legal context is that it reflects an increasingly cautious attitude to economic takings. The precedent of virtually unlimited deference that was set in case law from the mid-to-late 19th Century was eschewed in favour of a more contextual approach where the merits and deeper purpose of the plans underlying a taking is not axiomatically beyond the scrutiny of the courts.
From considering the reception of the case by the general public, we see even more clearly how Kelo in effect marks a change in the US towards greater scrutiny. Indeed, the voices that have dominated in the aftermath of Kelo were critical of the decision and criticized the court for not offering better protection to property owners. The case also led to an a surge of academic interest in the pubic use restriction, with many arguing for further restrictions on the scope of the takings power. 
Hence it seems that Justice O'Connor's opinion largely reflects contemporary worries about takings in the US, worries that are now also becoming increasingly relevant to how the law develops and is understood. Many states have changed their own eminent domain codes  following Kelo, to make it harder to undertake economic takings. Moreover, the federal government also banned such takings from taking place on the basis of federal takings powers.
It will lead us astray to delve deeply into the question of what caused this change in perspective on economic takings in the US, but we can offer a few hypothesis. First, it seems that cases such as Poletown illustrates the potential danger inherent in making the power of eminent domain available to market players. In particular, the main worry that has been raised is that the pretext of public purpose may be in the process of becoming a powerful instrument for influential market actors to gain access to regulatory powers of government. As these powers has massively expanded in the post-WW2 period, so has the potential for abuse. In addition, it seems that while those who were adversely affected by eminent domain tended to be less privileged and resourceful groups of society, the takings power is now increasingly brought to bear also against members of the middle class, who are in a better position to fight it, both legally and on the political scene.
While opinions differ greatly both regarding the extent of the problem and the causes of recent controversy, there is something near consensus in the US after Kelo that economic development takings raise special problems under the current system of eminent domain, and that these need to be addressed with a view to reducing tensions and restoring faith in the system. Indeed, even the majority in Kelo hint strongly at this when they say that  
Some have argued forcefully that a strict reading of the public use requirement is the way forward, if not by strict interpretation then by an explicit ban on economic development takings.  However, it is tempting here to echo the worries expressed in Seawell, that a strict formalistic approach to legitimacy runs the risk not only of being inflexible, but also, eventually, of offering less  protection to property owners. How, then, should we reduce the risk of abuses?
While many have focused on the question of banning economic taking, or reconsidering the public use clause, some have addressed this question from such a broader angle. In my opinion, this is the way forward. It seems, in particular, that a complete ban on economic development takings will leave a vacuum in the current economic system, which presupposes a great deal of cooperation between commercial and public interest. Particularly when it comes to economic development, the private-public partnership model has gained influence to the point that a ban on economic development takings would likely prove impossible to implement in a satisfactory manner. 
More generally, it seems hard to address the problem of economic takings without considering the role they play in the larger economic context within which current rules and practices have developed. Based on such considerations, I believe the procedural approach to economic takings is the appropriate one. This perspective asks us to take a closer look at judicial safeguards for protecting the role of property owners in the decision-making processes that lead up to the use of eminent domain. To some extent one might approach this on the basis of existing legal principles, asking for better scrutiny of procedural aspects, or by making it easier to bring pretext claims before the courts. However, it might also require new ideas, and, in particular, the introduction of new institutions for decision-making and administration of the eminent domain process. In the next section, I will look at two concrete proposals in more detail, one concerning the decision-making step and the other concerning the calculation of compensation. 
They will be important because they serve as starting points for the case study that is to follow, addressing mechanisms that we will return to in Chapters x and y when we look more closely at two Norwegian legal institutions that share many features with the theoretical proposals discussed in the next section.

\subsection{The institutional approach to economic development takings}

The primary distinguishing feature of economic development takings is that they give the taker an opportunity to profit commercially from the development. This may even be the primary aim of the project, with the public benefiting only indirectly through potential economic and social ripple effects. Property owners facing condemnation in such circumstances might expect to take a share in the commercial profit resulting from the use of their land. However, in many jurisdictions, including the US, the rules used to calculate compensation prevents owners from getting any share in the commercial surplus resulting from development. Indeed, various {\it elimination rules}, or {\it no scheme} rules, are typically in place to ensure that compensation is based entirely on the pre-project value of the land that is being taken.\footnote{References.} The policy reasons for such rules is that they ensure that the public does not have to pay extra due to its own special want of the property; indeed, one of the main purposes of eminent domain is to ensure that the public does not have to pay extortionate prices for land needed for important projects. However, when the purpose of the project is itself commercial in nature, there appears to be a shortage of good policy reasons for excluding this value from consideration when compensation is calculated. This is especially true when, as in the US, compensation tends to be based on the market value of the land taken. Why, when the project attracts commercial interest, should the buyer's prospect of carrying it out with a profit be disregarded from the assessment of market value? Surely, this would have a crucial bearing on the outcome of any friendly transaction among willing and rational economic agents? 

Many US writers have commented on this shortcoming of current compensation rules in economic takings cases, and most commentators appear to agree that special compensation rules are needed for this case type. In \footcite{eminc07}, the authors propose a novel approach to this challenge, based on a new kind of institution which they dub a {\it Special Purpose Development Corporation} (SPDC). The idea is that owners will be given a choice between standard pre-project market value and shares in a special company. This company will exist only to implement a specific step in the implementation of the development project: the transaction of the necessary land rights to the developer, for a negotiated price where the SPDC may also offer the land rights to other potential developers. Hence, the idea is to ensure that the owners are paid a value that reflects the post-project value of the land, but in such a way that the holdout problem is avoided. In particular, the SPDC will not be granted power to refuse to sell the land; as long as an offer is given that meets some minimum level (fixed by the legislature), it is compelled to sell to the highest bidder.  After having sold the land, moreover, the SPDC will cease to exist. 

Unlike other proposals made in the literature, this suggestion has a significant procedural and institutional component. The goal is to create more favorable market conditions for transferring land designated for development, in such a way that no owner can exercise monopoly power by holding out, but without handing over all the development value to the buyer. Other suggestions have focused on more static solutions, such as given owners a fixed premium in cases of economic development, or developing mechanisms of self-assessment to ensure that compensation is based on the true, non-bargaining, value the owner attributes to his own land.\footnote{Reference and explain.} Compared to such proposals, the idea of SPDCs seem more sophisticated, but there are still problems with it. In particular, it seems that the market created by the takings procedure envisioned in \footcite{eminc07} have some less desirable features.

First, since an SPDC does not have the power to stop the development, the system will only work as long as there are several interested parties who are willing to compete for the land rights. In practice, the planning regimes used to facilitate development can make this an unlikely prospect. Eminent domain is often used to implement highly specific projects, where the developer himself has played an important role in the planning process. Hence, while there might be many parties interested in the general kind of development that will be carried out, the concrete project that is being undertaken might not be of interest to anyone other than the developer who initially proposed it. Indeed, this developer might be the only one able to carry it out, say because the project forms part of a greater scheme involving rights that this developer already controls. Therefore, to create a market where the SPDCs can function as intended, it seems that deeper changes in planning practice are required, to avoid natural monopolies from developing in the planning process. This can be challenging in general, and in cases when eminent domain is used only for parts of greater schemes, it seems that it will be practically impossible to make the system work, without also undermining the governments ability to ensure that such schemes are carried out according to plan. In particular, setting up an SPDC alone is not, in general, sufficient to give property owners access to post-project market values. Something more is needed, and it is unclear what, if anything, can ensure that the SPDC gets to operate in healthy market conditions, as intended.

Second, the fact that the proposal is based on the doctrine of market value might in itself be a problem. Some of the clearest voices that have spoken out against economic development takings have done so on the basis of non-commercial objections. For instance, in the case of {\it Kelo}, the high subjective value that Suzanne Kelo attributed to her home was at the heart of the conflict. Indeed, Kelo continued to campaign against the condemnation, as a matter of principle, even after she had accepted a financial settlement whereby she was awarded some four times the estimated market value of her home.\footnote{...} The deeper problem of economic development takings is that they often lack legitimacy due to their partly commercial purpose. To compensate owners on the basis of market value alone does not offer any recognition of the fact that interference tends to appear less legitimate in such cases, compared to cases when the public purpose of the taking is more clearly discernible. It is of course not clear that such a lack of legitimacy can be addressed by compensation at all, but if this is at all going to be possible, then it must involve a compensation method which explicitly recognizes that these cases need to be treated differently than cases where problems of legitimacy do not arise. Moreover, since market value is the standard rule for awarding compensation, applying it in economic development cases will typically fail to provide any recognition of the fact that these cases are special. This in itself can become a perceived injustice; why should someone losing their home to a new school or a hospital be placed in the same category as someone loosing their home to a commercial company? In the latter case, it might be that the loss of a home appears {\it incommensurable} to the societal gains that may result. Hence, any compensation based on market value might lead to a moral deficit, whereby the affected property owners feels that non-commercial interest, including their own, are simply regarded as irrelevant.

The two problems addressed here both seem to point to the fact that the SPDCs, while more flexible than other suggestions, are still too static to achieve many of their objectives. In particular, to arrive at genuine market conditions for assessing post-project value, there is still a need for changes in the dynamic of the planning process underlying the taking, while to ensure legitimacy with respect to non-commercial aspects, there is a need for a mechanism that goes beyond commercial bargaining as such. In Chapter \ref{sec:app} we look at an alternative approach to compensation assessment, whereby legitimacy is sought through the participation of lay people in making the award. This system has long traditions in Norwegian law, and has currently been put to the test in cases of hydro-power development. In response to changes that have rendered such development commercial in nature, the compensation law in Norway has effectively been changed, something that has resulted in greater legitimacy. This process, as we will see, has resulted in large part from the activities of the special appraisal courts, and the lay people who partake in them. Also, the new compensation method that has been developed has many similarities to the ideas presented in \footcite{eminc07}, essentially serving as a means to make compensation a function of the post-project value of the land taken. However, the system relies heavily on the discretion of the lay people, judges and experts involved in deciding the award. Hence, the system is more flexible, and can be adapted to special circumstances of individual cases. There is also an ongoing tension between the legal side of compensation, as determined by the Supreme Court, and the administrative practices, as developed by the local appraisal courts. This tension is the defining feature of the system, and one which we believe show both its strengths and weaknesses, highlight its potential as an alternative approach to compensation in a time when takings law is both becoming more practically important and more controversial.

The approach in \footnote{eminc07} does not provide any guide in resolving controversies that arise in relation to how decision-making processes are organized in economic development cases. In addition to problems associated with compensation, such cases often suffer from a {\it democratic deficit}. While the main beneficiary, the developer, often plays a crucial role in the administrative process leading up to condemnation, property owners are not awarded any special right to participation beyond what follows from general planning and takings law. Just as the law fails to recognize economic takings as a special category for compensation purposes, so does it fail to give special rules to govern the preparation of such cases. This is widely seen as a shortcoming of the law. It creates an imbalance, in particular, between developers and owners, with the former enjoying a far greater ability to influence the decision-making process at the administrative level. In cases when developers are merely in place to execute public plans, and have no agenda on their own, this is no major threat to legitimacy. But in cases when the developer have significant interests of their own, and act as autonomous economic agents, the imbalance in influence becomes problematic.

This challenge is addressed in \footcite{lad08}, where the authors propose a new institutional framework for carrying out land assembly for economic development. The framework they propose is partly an extension of the eminent domain institute and partly presents an alternative to it. In particular, the proposal seeks to address the problem of democratic legitimacy while ensuring that structures remain in place to prevent inefficient gridlock and holdouts that would otherwise make large scale economic development projects hard to implement. The core idea is to introduce {\it Land Assembly Districts} (LAD), an institution that represents property owners in a specific area and has the power to, following a majority vote, to sell the land to a developer or a municipality. Voting rights in the LAD will be allocated in proportion to each owners share in the land belonging to the district. Owners can opt out of the LAD, but in this case eminent domain may be used to give the LAD control of their land. 

If a majority fails to form in favor of a sale, eminent domain can not be used against the land owners in the district. This is the crucial novel idea that sets the suggestion apart from other suggestions that have appeared after {\it Kelo}. LADs will not only ensure that the owners get to bargain with the developers over compensation, it will also give them an opportunity to refuse the development to go ahead, if they should so decide. Hence, the proposal shifts the balance of power in economic development cases, giving owners a greater role also in preparing the decision whether or not to develop, and on what terms. In my opinion, this makes the proposal stand out as particularly interesting in the recent literature on economic takings. It is the first concrete suggestion that addresses the democratic deficit in a dynamic, procedural manner, without failing to recognize that the danger of holdouts is real and that institutions are needed to avoid it, also in economic development cases.

There are some problems with the model proposed, however. In \footcite{ladres09}, the author points out that the basic mechanism of majority voting is itself imperfect, and can lead both to overassembly and underassembly, depending on the circumstances. It is pointed out, in particular, that if different owners value their property differently, majority voting will tend to disfavor those with the most extreme viewpoints, either in favor of, or against, assembly. If these viewpoints are assumed to be non-strategic and genuine reflections of the welfare associated with the land, the result can be inefficiency, since a majority can often be found that does not take due account of minority interests. For instance, if some owners are planning alternative development, leading them to attribute a high {\it hope}-value to their land, they can safely be ignored as long as the majority have no such plans. This could become particularly bad in cases when the alternative development itself is more socially desirable than the development that will benefit from assembly. The role of the LAD in such cases will not improve the quality of the decision to develop, since it pushes the decision-making process into a track where those interests that {\it should} prevail are voiced only by a marginalized minority inside the new institution.\footnote{Of course, one might imaging these land owners opting out of the LAD, or pursuing their own interests independently of it. However, they are then unlikely to be better of than they would be in a no-LAD regime. In fact, it is easy to imagine that they could come to be further marginalized, since the existence of the LAD, acting ``on behalf of the owners'', might detract from dissenting voices on the owner side.}

More generally, it remains unclear what role LADs are envisioned to play in the planning process and how they will affect the decision-making process as a whole. The idea is clear enough: LADs will help to establish self-governance in land assembly cases. More concretely, Heller and Hills argue that LADs should have ``broad discretion to choose any proposal to redevelop the neighborhood -- or reject all such proposals''.\footnote{p. 1496.} As they put it, two of the main goals of LAD formation is to ensure `` preservation of the sense of individual autonomy implicit in the right of private property and preservation of the larger community’s right to self-government''.\footnote{p. 1498.} However, at the same time they also stress that ``LADs exist for a single narrow purpose -- to consider whether to sell a neighborhood''.\footnote{1500} This is taken to be a safe-guard against conflict and abuse, serving to prevent LADs from becoming battle grounds where different parties attempt to co-opt the community voice to further their own interests. As Heller and Hills puts it, the narrow scope of LADs will ensure that ``all differences of interest based on the constituents' different activities and investments, therefore, merge into the single question: is the price offered by the assembler sufficient to induce the constituents to sell?''.

This points to a clear tension in the LAD proposal, between the broad goal of self-governance on the one hand, and the fear of neighborhood bickering and majority tyranny on the other. Moreover, the idea of LADs with a ``narrow purpose'' is hardly compatible with a scenario where the LAD has ``broad discretion'' to choose between competing proposals for development. If such discretion may indeed be exercised, what is to prevent special interest groups among the land owners from promoting those development projects that they happen to find most favorable? On the other hand, if the idea is that a LAD should not be free to assess the quality of development projects but must always sell to the highest bidder, it hardly seems fair to say that it enjoys ``broad discretion'' regarding which development proposal to accept. Moreover, would such a limit in the power of the LAD even be desirable? 

It is easy to imagine cases where competing proposals, perhaps emerging from within the community of owners themselves, will emerge in response to the formation of a LAD, suggesting less invasive forms of redevelopment. Perhaps such a proposal will even permit the majority of owners to either keep or reacquire homes in the area, by restricting demolition to the most blighted areas. What role should the original LAD play in a case like this? On the one hand, the alternative project might easily be a better use of the land in question, also from the point of view of the public. But on the other hand, it seems that the LAD might then become an arena for a new kind of power play among different interest, and a vehicle for oppression for whomever secures support from a majority of the owners within the district. 

 In their proposal, Heller and Hills are aware of this potential problem, which they propose to resolve by strict regulation. For instance, they argue that ``LAD-enabling legislation should require especially stringent disclosure requirements and bar any landowner from voting in a LAD if that landowner has any affiliation with the assembler''. Two questions arise, neither of which are resolved in their proposal. First, it seems unclear what is meant by ``affiliation'' here. For instance, what if a land owner happens to own shares in some of the companies proposing development? --Should he be barred from voting? And if so, should he be barred from voting on all proposals, or just those involving companies in which he is a shareholder?  Or what about the case when one of the land owners is an employee of one of the development companies? Should this render his vote null and void for the purpose of land assembly that might benefit his employer? It seems quite unfair if a previous affiliation should have this effect, but in some cases even such ties might play an important factor that could come to influence the outcome of the vote. This could be, for instance, if an important local employer proposes development in a neighborhood where it has a large number of employees. 

This illustrates how the regulation that Heller and Hills propose will raise a range of issues about how they are to be spelled out in further detail. The very idea of a ``narrow'' decision, in particular, seems somewhat dubious. In practice, would not a LAD invariably take up a strategically important position also in the planning process leading up to development? This, at any rate, seems inevitable if the system is really able to offer the form of self-governance that represent the overall goal. This leads us to the second problem with using regulation to narrow the function of LADs. Rather than being ``LAD-enabling'' is it not fairer to say that such regulation, at least if understood strictly, will curtail the power of LADs and limit their relevance to the decisonmaking process? Indeed, is it even desirable that local owners should not be able to affiliate themselves with developers in an effort to arrive at those schemes that they think will represent the best use of the land? Is this not the fundamental aspect of self-governance, that the local people themselves partake in the process, not just as passive spectators, but also as active agents that consult with planning authorities and developers about the best future use of their land. Hence, it might be that in many cases, those same mechanism that can lead to majority tyranny and bicekring if left unchecked are also crucial to unlocking the true potentail of LADs.

To find the right balance here, and to arrive at concrete rules that successfully regulate LADs in accordance with the overall aim, is very much an open problem. This is acknowledges by Hiller and Hills themselves, who point out that further work is needed, and that only a limited assessment of their proposal can be made in the absence of empirical data. In this thesis I will shed light on this challenge when I consider the Norwegian rules relating to land consolidation, showing how these can be looked at as a highly developed institutional embedding of many of the central ideas of the LADs. The assessment of how they function in cases of economic development, and how they are increasingly used as an alternative to expropriation in cases of hydro-power development, will allow me to shed further light on the issues that are left open by Hillen and Hills' important article, and to do so with an underpinning of empirical data from a jurdisicition that already has a similar framework in place.

\section{Conclusion}

In this Chapter I have presented the key issue that will remain in focus for the remainder of this thesis: legitimacy of economic development takings. I will now move on to consider Norwegian hydro-power as a case study of such takings. This case study is well suited to shed light on many of the issues that have been flagged as central, particularly in the US, and particualrly in relation to {\it institutional} approaches to the problem of economic takings. As we have seen, this approach asks us to focus on the legitimacy of the procedure, and the relationship between procedure and outcome in such cases. Importantly, it asks us to recognize that rather than trying to adjudicate about the meaning of ``public use'', a flexible framework with special safeguards is needed to ensure that the takings procedure remains legitimate in cases when commercial interest play an important role. Importantly, in such cases, it is important to avoid the democratic deficit that occurs when the taker has a disproportionate advantage in furthering his own non-public interests, compared to the land owners. 

I focused on two suggestions that have been made to ensure that this deficit is corrected. Both are institutional in nature, and involve setting up new formally recognized coalitions of land owners that will act as a counterweight to the power of the taker. The first suggestion limited its attention to compensation, recognizing the need for a system whereby the land owners are renumerated based on post-projcet value in such cases. This idea in itself represents a failraly dramtic break with the currently dominatnt doctrine in takings law, where compensation is almost always, and in almost all jurisdictons, based on the pre-project value of the land, or the {\it value to the owner}. In Chapter \ref{sec:noscheme} I will return to this principle in more depth, looking at how it developed interantially, and in Norway. Then, in Chapter \ref{sec:appraise} I will look at how it has now been abandoned in Norwegian law, for some case tpyes involving hydro-power development, and as a result of the special system of appraiseal courts in place in NOrway. The flexibility of these courts, involving lay peoplw, allows the law to be applied in a way that adapts to the concrete circumstance of the case, and the perceived farines and legitmiatmcy of the taking. Hnce, the NORwegian case might ofer interesting lessons about how to ensure fleibility in the compensation law. Moreover, I will look at how exactly the adaptation to economic takings cases is done, focusing on relating the current practices under Norwegin law to the general rules of comepnsation and the proposals made in the litarerature elsehwen, particularly the SPDC suggestion made by .... and.... 

The second suggestion I looked at in depth focused not only on the compensation, bute also ont he decision-making proess leading up to economic development. It recognixed the need for a mehcnaimsm that give local communities greater self-govername in cases of ecnonooimc development. At the same time, it recongized the conitnued tneed for  a mechanaism to acoid inefficint and socially harmful gridlock due to holdsouts among unwilling owners. Instead of emientent domain, however, a different mechanism should be used in eeconomic devellopment cases: a land assembly distirict. This is also a new class of instiuttions, and I pointed out some major problems associated with the extent to which they will be an active pticipcant int he palnning process, not just the transaction of land rights from orignal oners to devleopers. I argued that while the risk of abuse and failure increase with the level of participation, so des the pstive effect of this novel insitutions. To reduce the demoratic deficit in economic devleopment cases, it seems to me that a fairly wide power of participation must be granted to the land owwners, to restore valance and fairness between them and the developer. Hence, the question arises how to oraganixe this in a way that avoids sthe pitfalls associated witth strong and autnomous local government. 

In Chapet \ref{sec:landcons} I will shed light on this qeuston by considering the NOrwegian istntiutsion of land consoildation, which has very long traditions. It is a wery wide and flexible framworks, but includes, among other things, a frameworks for establisheing instutiontions resmbling that of LADs. I will foucs on how land consolidtion functions in cases of economigc developent thta would otherwise likely be spursued by ecminetnt domain. Agains, the case study will be Nroweigan hydro-power devleopmetn, but I will also discuss planning law and declopment more generally, as the Norweigna government is now considering amikng consolidation, traditioanlly a rural intusticment, a primary role in land devlopment also in rural areas, and in rlation to general development procects following land assembly.

Before I delve into these hapters, which all seek to directly dshed light on the issues raised in this chapeer, I will present the basic facutal and legal basis of my cases: the management of hydor.poer devleopment under NOWerrfgian law. Tis will be the subject of the next chapter, written in sucha  aways as to highlight that how it instantiates the same issues that have been discussed iabstractly here.

Blackstone, William. 1979. Commentaries on the Laws of England: A Facsimile of the First Edition of 1765--1769. University of Chicago Press.
Cohen, Charles E. 2006. “Eminent Domain After Kelo v. City of New London: An Argument for Banning Economic Development Takings.” Harvard Journal of Law and Public Policy 29: 491.
———. 2008. “The Abstruse Science: Kelo, Lochner, and Representation Reinforcement in the Public Use Debate.” Duquesne Law Review 46: 375–419.
Heller, Michael, and Rick Hills. 2008. “Land Assembly Districts.” Harvard Law Review 121 (6): pp. 1465–1527.
Horwitz, Morton J. 1973. “The Transformation in the Conception of Property in American Law, 1780-1860.” University of Chicago Law Review 40: 248–90.
Householder, Benjamin A. 2007. “Kelo Compensation: The Future of Economic Development Takings.” Chicago-Kent Law Review 82: 1029–61.
Hoyman, Michele M., and Jamie R. McCall. 2010. “Not Imminent in My Domain! County Leaders’ Attitudes toward Eminent Domain Decisions.” Public Administration Review 70 (6): 885–93. doi:10.1111/j.1540-6210.2010.02220.x.
Johnson, Emily A. 2011. “Reconciling Originalism and the History of the Public Use Clause” 79: 265–319.
Lehavi, Amnon, and Amir N. Licht. 2007. “Eminent Domain, Inc.” Columbia Law Review 107 (7): pp. 1704–1748.
Meidinger, Errol. 1980. “The ‘Public Uses’ of Eminent Domain: History and Policy.” Environmental Law 11: 1–66.
Merrill, Thomas W. 1986. “The Economics of Public Use.” Cornell Law Review 72: 61–116.
Somin, Ilya. 2007. “Controlling the Grasping Hand: Economic Development Takings after Kelo.” Supreme Court Economic Review 15 (1): pp. 183–271.
Underkuffler, Laura S. 2006. “Kelo’s Moral Failure.” William \& Mary Bill of Rights Journal 15 (2): 377–88.
Walsh, Rachael. 2010. “‘The Principles of Social Justice’ and the Compulsory Acquisition of Private Property for Redevelopment in the United States and Ireland.” Dublin University Law Journal 32: 1 – 24.
Waring, Emma J L. 2009. “Aspects of Property: The Impact of Private Takings.”
 


