\chapter{Introduction and Summary of Main Themes}\label{chap:intro}

\begin{quote}
Thieves respect property; they merely wish the property to become their property that they may more perfectly respect it.\footnote{G.K. Chesterton, {\it The man who was Thursday: A nightmare}.}
\end{quote}
\begin{quote}
[Granting] a takings power, then, may not be viewed as an act that wrenches away property rights and places an asset outside the world of property protection. Rather, it may be seen as an act within the larger super-structure of property.\footnote{Abraham Bell, Private Takings, p. 583.}
\end{quote}
%
%A human being needs only a small plot of ground on which to be happy, and even less to lie beneath. %\footnote{Johan Wolfgang von Goethe, {\it The sorrows of young Werther and selected writings}.}
%\end{quote}
%“That's what makes it ours - being born on it, working on it, dying on it. That makes ownership, not a %paper with numbers on it.”
%― John Steinbeck, The Grapes of Wrath
%
%
%
This thesis addresses so-called economic development takings, which occur when government sanctions the taking of property to stimulate economic growth. The canonical example is {\it Kelo v City of New London}, which brought the category of economic development takings into focus in the US, resulting in great controversy and a surge of academic work on legitimacy of takings.\footcite{kelo05}. The {\it Kelo} case concerned a house that was taken by the government in order to accommodate private enterprise, namely the construction of new research facilities for Pfizer, the multi-national pharmaceutical company.

The home-owner, Suzanne Kelo, protested the taking on the basis that it served no public use and was therefore illegitimate under the Fifth Amendment of the US Constitution. The Supreme Court eventually rejected her arguments, but this decision created a backlash that appears to be unique in the history of US jurisprudence. In their mutual condemnation of the {\it Kelo} decision, commentators from very different ideological backgrounds came together in a shared scepticism towards the legitimacy of economic development takings.\footnote{See generally \cite{somin08}.}

Interestingly, their scepticism lacked a clear foundation in US law at the time, as the {\it Kelo} decision itself did not appear particularly controversial in light of established eminent domain doctrines in the US. Hence, when the response was overwhelmingly negative, from both sides of the political spectrum, it seems that people were responding to a deeper notion of what counts as a legitimate act of taking.

%Indeed, the critical response to {\it Kelo} appears to have been a reflection of widely shared sentiments. As such, it also arguably involved pre-legal notions pertaining to legitimacy. Simply stated, people from across the political spectrum simply found the outcome {\it unfair}.

If the law is about delivering justice to the people, widely shared intuitions about legitimacy deserve attention from legal scholars. In the US, legitimacy intuitions pertaining to economic development takings have received plenty of it after {\it Kelo}. In the context of US law, it is now hard to deny that {\it Kelo} belongs to a separate category of takings that raises special legal questions.\footnote{See, e.g., \cite{cohen06,somin07}.} Because this change in the narrative was largely the result of a popular movement, there is reason to think that economic development takings is a powerful conceptual category, also outside of the US. 

Moreover, as soon cases like {\it Kelo} are portrayed as being primarily about bestowing a benefit on powerful commercial interests, it seems natural to expect that people will have a tendency to judge the issue of fairness similarly, irrespectively of differences in the surrounding legal framework. Two questions arise. First, when is it appropriate to deride economic development takings in this way? Second, if it is appropriate, should the law recognise a justiciable basis for the courts to intervene, to strike down illegitimate takings?

Both of these questions will be addressed in this thesis. To address them effectively, it should be acknowledged from the start that there is at least a {\it risk} that takings for economic development can be improperly influenced by commercial interests. The risk of such capture, moreover, is clearly higher in economic development situations than in cases when takings take place to benefit a concretely identified public interest, such as the building of a new school or a public road. Hence, it seems intuitively reasonable to single out economic development takings for special attention at the political and normative level. However, should the categorisation also be recognised as a basis for justiciable restrictions on the use of eminent domain?

This is not obvious. For instance, it seems that many European jurisdictions implicitly reject such a perspective, because the state is regarded as having a ``wide margin of appreciation'' when it comes to deciding on legitimate takings purposes.\footnote{See \cite{james86}.} This points to the first main theme of this thesis: an analysis of economic development takings as a conceptual category for legal reasoning.

\section{Economic Development Takings as a Conceptual Category}\label{sec:1:1}

This thesis will argue that the category of economic development takings should be recognised already at the theoretical level. However, this claim will be made relative to a theory of property that is broader than typical approaches to property in legal scholarship. Specifically, the theory of property that will form the backbone of this thesis will encompass more than just the entitlements of owners. 

The theoretical framework is discussed in more depth in Chapter \ref{chap:2} of this thesis. There it will be argued that a social function understanding of property should be adopted, with an emphasis on {\it human flourishing} as a normative foundation for property. In short, property should be protected because it can help people flourish. Moreover, property is meant to serve this function not only for the owners themselves, but also for the other members of their communities.

This ambitious take on property must necessarily also give rise to a broader assessment of legitimacy when the state interferes. This, in turn, is what inspires my initial discussion on economic development takings in the first chapter. There I will present the basic definition of the notion and discuss the {\it Kelo} case in some more detail. Specifically, I will argue that Justice O'Connor's strongly worded dissent -- finding that the taking should be struck down -- embodies a social function perspective on property.

Chapter \ref{chap:3} follows up on this by studying the legitimacy of economic development takings in more depth. Several approaches to this issue are considered, culminating in a recommendation for a perspective based on institutional fairness that I trace to recent developments at the ECtHR. Specifically, the Court in Strasbourg has begun to look more actively at the systemic reasons why violations of human rights occur, in order to address structural weaknesses at the institutional level in the signatory states. This approach is arguably the one that fits best with the sort of analysis carried out by Justice O'Connor in {\it Kelo}, more so than the approach usually induced by the public use restriction in the US Constitution.

Importantly, the institutional perspective appears to be a sensible middle ground between procedural and substantive approaches to legitimacy, directing us to focus on decision-making processes and structural aspects without giving up on substantive fairness assessments. To strike a fair balance, in particular, is not just about reaching an appropriate outcome, but also about how that outcome came about, and how often dubious outcomes are likely to result from the way the system is organised. This way of thinking about legitimacy brings me to the second focus point of the thesis.

\section{A Democratic Deficit in Takings Law?}\label{sec:1:2}

To make the theoretical work on legitimacy more concrete, Chapter \ref{chap:3} provides a proposal for a legitimacy test that can be applied to economic development takings. This test consists of a list of indicators that can suggest eminent domain abuse. The first six points are due to Kevin Gray, while the final three are additions I propose on the basis of the work done in this thesis.\footnote{For Gray's original points see \cite{gray11}.} I call the resulting list the Gray test, a heuristic for inquiring into the legitimacy of an economic development taking.

Arguably, the most important indicator is also the least precise, namely the one pertaining to the {\it democratic merit} of the taking (one of my additions). By itself, such a broad indicator might not offer much guidance. However, the idea is that when taken together with the other points, this indicator will induce an overall assessment of the other points against the decision-making process as a whole, not only the final outcome. Hence, this addition is specifically aimed at emphasising the institutional fairness perspective. If a taking fails the legitimacy test on this point, moreover, it might indicate an existing weakness or a pernicious deterioration of the decision-making framework more generally.

Admittedly, asking courts to test for legitimacy is an incomplete response to the worry that illegitimate practices surrounding economic development takings signify a democratic deficit in takings law. Moreover, some US scholars have argued that increased judicial scrutiny is neither a necessary nor a sufficient response to concerns about the institutional legitimacy of takings such as {\it Kelo}.\footnote{See generally \cite{lehavi07,heller08}.} Instead, these authors point out that the traditional takings procedure does not in any case seem particularly suited for bringing about this kind of economic development.

This observation has been accompanied by proposals for structural takings law reform, most notably the work of Heller and Hills.\footnote{See \cite{heller08}.} This work proposes that a new type of institution, a so-called Land Assembly District, can replace the traditional takings procedure in cases where property rights are fragmented and the potential takers have commercial incentives. The basic idea is that the owners themselves should be allowed to decide whether or not development takes place, by some sort of collective choice mechanism (possibly as simple as a majority vote). In this way, the holdout problem can be solved (individual owners cannot threaten to block development to inflate the value of their properties). At the same time, however, the local community's right to self-governance is recognised and respected.
 
The proposal for Land Assembly Districts is linked to more general ideas about self-governance and sustainable resource management, particularly the theories developed by Elinor Ostrom and others.\footnote{See \cite{ostrom90}.} On the basis of a large body of empirical work, these scholars have formulated and refined a range of design principles for institutions that can promote good self-governance at the local level.\footnote{See \cite{cox10}.} 

At the end of Chapter \ref{chap:3}, I argue that this work can be used to address the legitimacy of takings in a principled way, to arrive at refinements or alternatives to the proposal made by Heller and Hills. Specifically, it seems that alternatives to expropriation based on self-governance can be a powerful way to address the worry that economic development takings might otherwise be associated with a democratic deficit. At the same time, the context-dependence of solutions along these lines make sweeping reform proposals unlikely to succeed. Rather, it is important that the institutions that are used are appropriately matched to local conditions.\footnote{See \cite[92]{ostrom90}.}

For instance, a setting where property is evenly distributed among members of the local community might suggest a very different type of institution compared to a setting where the relevant property rights are all in the hands of a small number of absentee landlords. In short, the idea of using self-governance structures in place of eminent domain necessitates a more concrete approach, a move away from property theory towards property practice. This sets the stage for the second part of the thesis, consisting of a case study of takings for Norwegian hydropower development. 

This first key objective of this case study is to apply the theory developed in the first part to analyse the legitimacy of takings for hydropower. The second objective is to study a concrete institutional alternative to expropriation in more depth, namely the system of {\it land consolidation courts}. In Norway, these courts are empowered to set up self-governance organisations for local resource management and economic development, if necessary against the will of individual owners.

%In light of this, the case study will shed light on both of the two key conclusions drawn in the theoretical part of the thesis.

%Alternative test the theoretical assertions made about how to approach legitimacy. Specifically, the question to be addressed is to what extent the traditional narrative of takings is capable of doing justice to the property conflicts that have arisen regarding the development of hydropower in Norway.

\noo{ I arrive at several objections against the details of the particular institutional arrangements proposed, particularly with regards to their likely effectiveness. It seems, in particular, that both proposals fail to recognise the full extent to which prevailing regulatory frameworks concerning land use and planning would have to be reformed in order to make their proposals work.

At the same time, I argue that these novel institutional proposals are extremely useful in that they point towards a novel way to frame the issue of legitimacy in takings law. In particular, I explore the hypothesis that traditional procedural arrangements surrounding takings suffer from a democratic deficit, a particularly powerful source of discontent in economic development cases.

This idea is the second key focus point of my thesis. First, I approach it from a theoretical point of view, by exploring the notion of {\it participation} and its importance to the issue of legitimacy, particularly in the context of economic development. It seems, in particular, that {\it exclusion} could be a particular worrying consequence of certain kinds of economic development takings, namely those that lack democratic legitimacy in the local community where the direct effects of the taking are most clearly felt.

I believe this to be a promising hypothesis, and I back it up by considering the social function theory of property and the notion of human flourishing which has recently been proposed as a normative guide for reasoning about property interests. I pay particular attention to the importance of communities that has been highlighted in recent work, as a way to bridge the gap between individualistic and collectivist ideas about fairness in relation to property.

I take this a step further, by arguing that a focus on communities naturally should bring institutions of local democracy to the forefront of our attention. The role that property plays in facilitating democracy has been emphasised before by other scholars, and I think it has considerable merit. However, I also argue that it is important to resist the temptation of viewing its role in this regard through an individualistic prism. It is especially important to take into account additional structural dimensions that may supervene on both property and democracy, such as tensions between the periphery and the centre, the privileged and the marginalised, as well as between urban and rural communities.

It is especially important, I think, to appreciate the effect takings can have on local democracy. For one, excessive taking of property from certain communities might be a symptom of failures of democracy as well as structural imbalances between different groups and interest. But even more worrying are cases when the takings themselves, brought on by a commercially motivated rationale, appears to undermine the authority of local arrangements for collective decision-making and self-governance. This dimension of legitimacy, in particular, is one that I devote special attention to throughout this thesis.

I also believe, however, that it is hard to get very far with this sub-theme through theoretical arguments alone. Hence, to explore it in more depth, I go on to assess it from an empirical angle, by offering a detailed case study of takings of Norwegian waterfalls for the purpose of hydropower development. This case study, in turn, will allow me to cast light on two further key themes, that I now introduce. %This brings me to the second part of my thesis, which in turn consists of two main themes, where the latter aims to bring me back towards a more general setting, by delivering some recommendations for how best to deal with economic development takings.
}
%I go on to consider the hypothesis that economic development takings demonstrate that takings law suffer from a {\it democratic deficit}.

\section{Putting The Theory to the Test}\label{sec:1:3}

In Norwegian law, the story of legitimacy more or less begins and ends with the issue of compensation.\footnote{See generally \cite{dyrkolbotn15}.} If an owner has grievances about the act of taking as such, not the amount of money they receive, takings law has very little to offer. In fact, it does not appear to offer anything that does not already follow from general administrative law. The owner can argue that the decision to take was in breach of procedural rules, or grossly unreasonably, but the chance of succeeding is slim.\footnote{See \cite[384-386]{dyrkolbotn15b}.}

%This narrative of legitimacy is not unique to Norway. It seems that in Europe, unlike in the US, the issue of legitimacy is often seen as predominantly concerned with the issue of compensation. In particular, the jurisprudence at the ECtHR is typically focused on compensatory issues. Moreover, while many constitutions of Europe, including the Norwegian, include public interest clauses, the courts make little or no use of these when adjudicating takings complaints. In the words of the ECtHR, the member states enjoy a ``wide margin of appreciation'' when it comes to determining what counts as a public interest.

In cases involving hydropower development, the position of local owners is also affected by a number of additional variables that pertain specifically to the licensing framework in place to ensure government control over the use of water resources. Chapter 4 presents this framework in some detail, before discussing administrative practices and commercial practices characterising the hydropower sector today. A first important observation is that the hydropower sector in Norway was liberalised in the early 1990s.\footnote{The crucial legislative reform was the \cite{ea90}.} This means that the hydropower companies that expropriate are now commercial enterprises, not public utilities.

A second important observation is that the property rights that the energy companies tend to take from local owners are not merely ancillary rights that are needed in order to develop water resources. Rather, the relevant water resources themselves are also taken. In Norway, the right to harness the power of water is a property right, typically owned by members of the rural community in which the resource is found.\footnote{See \cite[13]{wra00}.}

Hence, the hydroelectric companies in Norway are traditionally dependent on taking natural resources from local communities in order to exist and make money for their shareholders. Since deregulation, however, local owners have begun to resist such takings. This has been motivated by the fact that owners can now undertake their own hydropower projects as a commercial pursuit; unlike the situation before liberalisation, owner-led development projects can now demand access to the electricity grid as producers on equal terms with established energy companies.\footnote{See, e.g., \cite{uleberg08}.} Unsurprisingly, this has led to heightened tensions between takers and owners, tensions that the water authorities are now forced to grapple with on a regular basis.

%As a result, local owners now regularly protest expropriation of their rights on the grounds that they wish to {\it participate} in economic development, by carrying out alternative development projects, or by cooperating with the energy companies who wish to take their water rights. Hence, while liberalisation has rendered takings for hydropower as takings for profit, it has also empowered local owners and communities to propose alternatives. Unsurprisingly, this has led to tensions that the water authorities are now forced to grapple with on a regular basis.\footnote{See Chapter \ref{chap:4}, Section \ref{sec:4:4}.}

Chapter \ref{chap:4} sets the stage for studying these tensions in more depth. The chapter argues that despite their improved position following liberalisation, local owners remain marginalised under the regulatory framework. Specifically, despite political support for locally organised small-scale development, the large energy companies have continued to enjoy a privileged position in their dealings with the water authorities. Lately, the political narrative appears to be changing, with large-scale development becoming the preferred mode of exploitation also among politicians.

Building on these observations, Chapter \ref{chap:5} goes on to discuss expropriation of waterfalls in more depth. Specifically, the chapter tracks the position of owners under the law and administrative practice. The key finding is that expropriation is usually an {\it automatic consequence} of a large-scale development license.\footnote{In some cases, this follows explicitly from the water resource legislation, while in other cases it follows from administrative practice. For further details, see below in Chapter \ref{chap:5}, Section \ref{sec:5:3}.} That is, commercial companies that succeed in obtaining large-scale development licenses will almost always be granted the right to expropriate. This rights will be granted, moreover, without any assessment taking place as to the appropriateness of depriving local owners of their resources.

Moreover, the owners' position during the licensing assessment stage is extraordinarily weak.\footnote{See especially the discussion in Chapter \ref{chap:5}, Sections \ref{sec:5:6} and \ref{sec:5:7}.} The fact that expropriation tends to follow automatically from a license to develop has led the water authorities to regard the licensing question and the associated procedures as exhaustive in all cases. No distinction is made between cases involving expropriation and cases that do not. This has a dramatic effect on the level of protection available to local owners. According to written testimony during a recent Supreme Court case on legitimacy, the water authorities do not even recognise a duty to inform local owners of pending applications that will involve expropriation.\footnote{See Chapter \ref{chap:5}, Section \ref{sec:5:7}. The case in question was \cite{jorpeland11}.}

In relation to the compensation issue, the owners' position initially improved after liberalisation, as the lower courts began to compensate local owners on the basis of what they lost from being unable to carry out their own development project.\footnote{See \cite{uleberg08}.} This led to a dramatic increase in compensation payments compared to earlier practice.\footnote{See especially the discussion in Chapter \ref{chap:5}, Section \ref{sec:5:5:1}.} However, a recent decision from the Supreme Court appears to largely reverse this development, since a large-scale license may now itself be considered proof that alternative development by owners was always unforeseeable.\footnote{See \cite{otra13}.}

In light of this and other data discussed in Chapter 5, my conclusion is that today's typical takings for hydropower do not appear to pass the Gray test. However, Norwegian law also offers a promising institutional path towards the restoration of legitimacy in economic development contexts. Specifically, the institution of land consolidation, as understood in Norway, could serve such a function. Moreover, it already does so in the context of hydropower development, when local owners wish to undertake development themselves but disagree about how it should be done. This brings me to the fourth key theme of this thesis.

\section{A Judicial Framework for Compulsory Participation}\label{sec:4}

\noo{ In Norway, the distribution of property rights across the rural population is traditionally highly egalitarian.\footnote{This is discussed in more depth in Chapter \ref{chap:4}, Section \ref{sec:4:2}.} This meant that the farmers in Norway soon became an active political force, particularly as representative democracy started to gain ground as a form of government in the 19th century.\footnote{As early as in 1837, the Norwegian parliament was so dominated by farmers that it came to be described as the ``farmer's parliament''. See \cite{hommerstad14}}

%The Norwegian farmers were often little more than small-holders, and had few privileges to protect. Hence, they became liberals of sorts (although also known for their fiscal conservatism). The farmers as a class were responsible for pushing through important early reforms, such as the abolition of noble titles and the establishment of democratically elected municipality governments.

%However, the municipality governments were not the first example of local decision-making institutions in Norway.
The highly fragmented ownership of land meant that institutions for collective decision making had to be introduced early on in Norwegian history; some even argue that the first realisation of a truly direct democracy can be traced to Norway in the Viking age.\footnote{See \cite[23]{titlestad14}.} One of the ancient institutions for collective action is the land consolidation court. 
}

The final key theme of this thesis, presented in Chapter \ref{chap:6}, consists of an assessment of land consolidation and its potential function as an alternative to takings when compulsion appears warranted to ensure economic development. This is an especially fruitful topic because land consolidation is presently being used in this way to facilitate hydropower development. The large energy companies never use it, but local owners often do.\footnote{In 2009, land consolidation had facilitated a total of 164 small-scale hydropower projects with a total annual energy output of about 2 TWh per year (enough electricity to supply a city of about 250 000 people), see \cite{gevinst09}.} In these cases, the consolidation courts have proved themselves highly effective in making self-governance work. 

\noo {The typical scenario is that the owners disagree about who owns what and cannot agree on how to organise development. In other cases, some of the owners, or even a majority of them, do not wish any development at all. In these cases, it is possible for the courts to {\it compel} them to participate. 

In these situations, it is less clear how well consolidation works in practice. Plainly, there has not been enough cases of this sort to draw a clear conclusion, especially not in situations when those who favour development are a minority among the owners. However, the consolidation alternative still appears highly preferable to the expropriation alternative, especially in terms of legitimacy. Specifically, the owners who are compelled to participate do not loose their property and are not excluded from the decision-making process.}

The land consolidation alternative can make a great difference, because it strives to ensure legitimacy through participation. The potential democratic deficit associated with compelled economic development is dealt with by efforts to raise owners to take active part in the management of their property in the public interest. At the same time, the procedure can be reasonably effective, since participation is compulsory and the judge may intervene to settle conflicts. In Chapter 6, I discuss possible objections to the procedure, but conclude that the continued development of the land consolidation institution provides the best way forward for addressing problems associated with economic development takings in Norway.

%Finally, the institution of land consolidation is assessed against Land Assembly Districts, and -- more generally -- against the idea of self-governance frameworks for managing common pool resources. I argue that it compares favourably, both because it comes equipped with in-built judicial safeguards, but also because it has such a broad scope. I note, however, that its successful use is dependent on both political will and an ability to retain key feature even in the presence of new and powerful stakeholders in the consolidation process.

If the integrity of the procedure can be secured, adopting it to organise larger scale development involving external actors seems like a very promising approach to legitimacy more generally. Moreover, while the system is designed to work in a setting of egalitarian property rights, it is interesting to also consider the possibility key features of the procedure might also inspire solutions to the takings problem in other jurisdictions. %Specifically, the fact that the procedure focuses on benefiting properties rather than owners means that a broader understanding of property can itself suggest a broader range of possible applications. 

It might well be, for instance, that a land consolidation approach coupled with a human flourishing understanding of property can be a good way of including non-owners in the process. Possibly, a modification of the framework in settings where many property dependants do not have property rights, is to enlarge the set of legal persons with legal standing before the court. This might give rise to increased complexity of the procedure and new risks of abuse by local elites, but it seems like an interesting idea to explore in future work. In short, the consolidation alternative seems like a good starting point for an approach to legitimacy that truly takes into account a wider notion of what property is, and what it can and should be in a democracy where everyone is equal before the law.

%the core features of land consolidation for economic development can be preserved and developed further, 
\noo{ In the second part of the thesis, I put the theoretical framework to the test by applying it to a concrete case study, namely that of Norwegian hydropower. Following liberalisation of the energy sector in the early 1990s, hydropower is now a commercial pursuit in Norway. Moreover, there is a long tradition for granting energy producers the power to acquire property compulsorily, including the necessary rights to exploit the energy of water, rights that are subject to private property under Norwegian law. This has resulted in tension and controversy, however, as the original owners of these rights, typically local farmers and small-holders, see the commercial potential of hydropower being transferred to other commercial interests, to the detriment of their own, and their communities', interest in self-governance and economic benefit.}

\noo {\section{Structure of the Thesis}\label{sec:1:5}

My thesis is divided into two parts. Part I sets up a theoretical framework for reasoning about property and proceeds to study the legitimacy of economic development takings in more depth. Part II consists of a case study of takings for hydropower, focusing on how expropriation and alternatives to it work on the ground in Norway. In brief, the structure of the chapters are as follows.

Chapter 2 introduces the topic of this thesis and presents the social function theory of property. The chapter argues that the descriptive core of this theory should be accepted irrespective of one's normative inclinations; the social function approach is simply more accurate than other theories. From this descriptive assertion, the category of economic development takings arises naturally. To address it normatively, the chapter argues that the notion of human flourishing provides the appropriate starting point. On this basis, the chapter discusses economic development takings and {\it Kelo} in more depth, to introduce the key question of legitimacy.

Chapter 3 proceeds to address the legitimacy question in more depth. The chapter starts from considering the procedural approach to legitimacy, illustrated by the law of England and Wales. Following up on this, the substantive approach is considered, illustrated by the law of the US. Finally, the chapter argues for a middle ground between the two, an institutional fairness perspective that is also linked to recent developments at the ECtHR. Following up on this, the chapter presents the Gray test; a set of indicators of eminent domain abuse suitable for an institutional fairness approach. The chapter concludes by discussing the possibility of providing institutional alternatives to expropriation for economic development, taking inspiration from the theory of self-governance for common pool resources.

Chapter 4 introduces the case study of takings for hydropower in Norway. The chapter briefly presents hydropower in the law, focusing on the licensing legislation. Then the chapter investigates hydropower in practice, noting that the liberalisation of the electricity market in the early 1990s has had a dramatic effect. Specifically, the chapter emphasises how local owners of water resources are now in a better position to develop these themselves, since they can access the electricity grid as producers on equal terms as larger companies. The chapter goes on to study the tension that has resulted between large-scale development facilitated by expropriation and small-scale development facilitated by local property rights. Despite early signs that small-scale solutions enjoyed political support, the large energy companies now appear to be reasserting their control over the hydropower sector, to the detriment of owners and their local communities.

Chapter 5 discusses expropriation of hydropower in more depth. The chapter starts by giving a brief overview of Norwegian expropriation law, before noting that expropriation for hydropower often takes place on the basis of special rules that leave owners with less protection. The history of the law is discussed in quite some detail, to show how the law has gradually developed to undermine local property rights over water resources. Following up on this, the chapter discusses case law on the expropriation and licensing, focusing on the legitimacy question (which is addressed in Norway almost solely on the basis of procedural standards). The chapter studies the recent Supreme Court case of {\it Jørpeland} in depth, to shed light on how current administrative practices impact on owners and their communities. The conclusion is that current takings practices do not appear legitimate.

Chapter 6 discusses land consolidation as an alternative to expropriation. The chapter starts by clarifying the notion of land consolidation and how the Norwegian understanding of that terms is much wider than that found in other jurisdictions. Following up on this, the chapter discusses consolidation as an alternative to expropriation, by focusing on those tools that the consolidation courts have at their disposal in this regard. Then the chapter gives a more in-depth presentation of some cases when consolidation was used to organise small-scale hydropower development. Finally, a discussion is provided on the prospect of using consolidation to replace expropriation more generally, in Norway and possibly also in other jurisdictions.

Chapter 7 contains my conclusions, formulated as an attempt at connecting the concrete and abstract aspects of this work around two threads, tracking property's relationship with excluding and taking on the one hand and its relationship with giving and participation on the other. My final conclusion is that the latter two notions characterise true property, and that property as such is worth defending.

}