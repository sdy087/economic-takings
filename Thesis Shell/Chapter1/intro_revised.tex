\chapter{Introduction and Summary of Main Themes}\label{chap:1}

\begin{quote} \small
Thieves respect property. They merely wish the property to become their property that they may more perfectly respect it.\footnote{\cite[58]{chesterton08}.}
\end{quote}
\begin{quote} \small
[Granting] a takings power, then, may not be viewed as an act that wrenches away property rights and places an asset outside the world of property protection. Rather, it may be seen as an act within the larger super-structure of property.\footnote{\cite[583]{bell09}.}
\end{quote}
%
%A human being needs only a small plot of ground on which to be happy, and even less to lie beneath. %\footnote{Johan Wolfgang von Goethe, {\it The sorrows of young Werther and selected writings}.}
%\end{quote}
%“That's what makes it ours - being born on it, working on it, dying on it. That makes ownership, not a %paper with numbers on it.”
%― John Steinbeck, The Grapes of Wrath
%
%\cite{waring11} (``It is a testament to the elasticity of the concept of
%property that it is able to represent all things to all people, and to accommodate so many conflicting %calls''); 
%

Property can be an elusive concept, especially to property lawyers.\footnote{See, e.g., \cite[225-226]{waring09} (``It is a testament to the elasticity of the concept of 
property that it is able to represent all things to all people, and to accommodate so many conflicting calls.''); \cite[252]{gray91} (``But then, just as the desired object comes finally within reach, just as the notion of property seems reassuringly three-dimensional, the phantom figure dances away through our fingers and dissolves into a formless void.'').} Indeed, in legal language, the word itself often only functions as a metaphor -- an imprecise shorthand that refers to a complex and diverse web of doctrines, rules, and practices, each pertaining to different ``sticks'' in a bundle of rights.\footnote{See generally \cite{grey80,klein11}.} Should we conclude that property as a unifying term is lost to the law? It certainly seems hard to pin it down. In the words of Kevin Gray, when a close scrutiny of property law gets under way, property itself seems like it ``vanishes into thin air''.\footnote{See \cite[306-307]{gray91}. See also \cite[81]{grey80} (arguing that the eventual consequence of the bundle view is that property will cease to be an important category for legal and political reasoning).}

Arguably, however, property never truly disappears.\footnote{See \cite[159]{gray94} (``We are continually prompted by stringent, albeit intuitive, perceptions of 'belonging'.'').} Indeed, there is empirical evidence to suggest that humans come to the world with an innate concept of property, one which pre-exists any particular arrangements used to distribute it or mould it as a legal category.\footnote{See \cite{stake06}.} Specifically, humans and a seemingly select group of other animals appear to have an intuitive ability to recognise {\it thievery}, the taking of property (not necessarily one's own) by someone who is not entitled to do so.\footnote{See \cite[11-13]{brosnan11}; \cite[159]{gray94}.}

%\footnote{See \cite[...]{gray94} (``In this context we are still not far removed from the
%primitive, instinctive cries of identification which resound in the
%playgroup or playground:
%'That's not yours; it's mine.''').}

Taken in this light, Proudhon's famous dictum, ``property is theft'', might be more than a seemingly contradictory comment on the origins of inequality.\footnote{For Proudhon's theory of property generally, distinguishing between {\it de facto} possession and {\it de jure} property (regarded as theft), see \cite{strong14}.} It might point to a deeply rooted aspect of property itself, namely its role as an anchor for the distinction between legitimate and illegitimate acts of taking.

%Therefore, it also seems that the following becomes a key question in property theory: how should we think about takings, and when are they legitimate?

In this thesis, I will study takings of a special kind, namely those that are sanctioned by a government in the pursuit of some public use or interest. Specifically, the word {\it taking} will be used to refer to an exercise of the government's power of eminent domain.\footnote{See generally \cite{stoebuck72} (clarifying the status of eminent domain from a US perspective, tracing its roots back to early civil law writers such as Grotius and Bynkershoek). Takings will also be referred to as expropriations, especially in the context of Norwegian law. In England and Wales, the corresponding notion is that of compulsory purchase.} In legal language, especially in the US, takings by eminent domain are often referred to as takings {\it simpliciter}, while talk of other kinds of ``takings'' require further qualification, e.g., in case of ``takings'' based on contract, taxation, or adverse possession. The US terminology is intuitive and helps bring the issue of legitimacy to the forefront, so I will adopt it throughout this thesis.\footnote{Sometimes it is convenient to draw up the notion of a taking more widely than I do in this thesis, e.g., to include adverse possession, see \cite[19-21]{waring09}. However, such a broader notion will not be used in this thesis. This choice appears natural in light of how the thesis focuses specifically on takings motivated by a government's desire to facilitate concrete economic development projects.}

When guided by eminent domain, the taking of private property without the owners' consent is not theft. But it is not necessarily that far removed from it either; the default assumption is that takings are legitimate, but if they are not, one may well be permitted to call them by a different name.\footnote{See \cite[8-10]{gray11} (discussing case law from the US, with state judges describing illegitimate takings as ``plunder'', ``rapine'', and ``robbery'').}

More generally, the idea that the government's power to take is not unlimited seems fundamental. Indeed, the expectation that an owner might find occasion to resist an act of taking, and may or may not have good grounds for doing so, appears deeply rooted in pre-legal intuitions.\footnote{See, e.g., \cite[159]{gray94}.} This raises the question of how to approach the legitimacy of takings in legal reasoning and what conceptual categories we can benefit from when doing so. This is the key question that is addressed in this thesis, for the special case of so-called {\it economic development takings}.\footnote{For a sample of scholarship based on this term, see \cite{cohen06,somin07,wilt09,yellin11}.}

Such takings occur when a government uses the power of eminent domain to stimulate economic growth, typically by providing property to for-profit companies for use in a concrete development project. The canonical US example is {\it Kelo v City of New London}, which resulted in great controversy and a surge of academic work on the legitimacy of takings.\footcite{kelo05}

The {\it Kelo} case concerned several homes that were taken by the government in order to accommodate private enterprise, namely the construction of new research facilities for Pfizer, the multi-national pharmaceutical company. Several home-owners, among them Suzanne Kelo, protested the taking on the basis that it served no public use and was therefore illegitimate under the Fifth Amendment of the US Constitution. The Supreme Court eventually rejected their arguments, but this decision created a backlash that appears to be unique in the history of US jurisprudence.\footnote{See generally \cite{somin08}.}

In their mutual condemnation of the {\it Kelo} decision, commentators from very different ideological backgrounds came together in a shared scepticism towards the legitimacy of economic development takings.\footnote{See \cite[1413-1415]{bell06} (``Everyone hates {\it Kelo}'', commenting on how criticism was harsh from across the political spectrum).} Interestingly, their scepticism lacked a clear foundation in US law at the time, as the {\it Kelo} decision did not appear particularly controversial in light of established eminent domain doctrines.\footnote{See, e.g., \cite[1418]{bell06} (``The most astounding feature of {\it Kelo}, as even the case's harshest critics agree, is that from a legal standpoint, the ruling broke no new ground.'').} Hence, when the response was overwhelmingly negative, from both sides of the political spectrum, it seems that people were responding to a deeper notion of what counts as legitimate.

%Indeed, the critical response to {\it Kelo} appears to have been a reflection of widely shared sentiments. As such, it also arguably involved pre-legal notions pertaining to legitimacy. Simply stated, people from across the political spectrum simply found the outcome {\it unfair}.

If the law is meant to deliver justice, widely shared intuitions about legitimacy deserve attention from legal scholars. In the US, legitimacy intuitions pertaining to economic development takings have indeed received plenty of attention after {\it Kelo}. Despite the outcome of the case, it is now hard to deny that cases such as {\it Kelo} belong to a separate category of takings that raises special legal questions.\footnote{See, e.g., \cite{cohen06,somin07}.} As this change in the narrative was largely the result of a popular movement, there is reason to think that the category of economic development takings is in itself a powerful addition to the discourse on legitimacy, potentially relevant also outside of the US.

%When cases like {\it Kelo} are portrayed as being primarily about bestowing a benefit on powerful commercial interests, it becomes natural to question their legitimacy, irrespective of details in the surrounding legal framework. But when is it appropriate to deride economic development takings in this way? Moreover, should the law provide a basis for the courts to intervene, to rescind illegitimate takings?%Both of these questions will be addressed in this thesis. To address them effectively, 
To explore this further, it should first be acknowledged that there is a {\it risk} that takings for economic development can be improperly influenced by commercial, rather than public, interests. This risk is clearly higher in economic development situations than in cases when takings take place to benefit a concretely identified public interest, such as the building of a new school or a public road. Hence, it is already intuitively reasonable to single out economic development takings for special attention at the political and normative level. However, should the categorisation also be recognised as a basis for justiciable restrictions on the takings power?

This is not obvious, as it conflicts with the prevalent idea that governments enjoy a ``wide margin of appreciation'' when it comes to their use of eminent domain.\footnote{This expression has been used by the European Court of Human Rights, see \cite[54]{james86}. In the US, the same attitude was clearly a factor motivating the majority in \cite{kelo05}.} However, as the US debate shows, it might be hard to deny judicial review as soon as the special features of economic development takings are brought into focus. This points to the first main theme of this thesis: an analysis of economic development takings as a conceptual category for legal reasoning about property protection.

\section{Property Theory and Economic Development Takings}\label{sec:1:1}

In Part I, this thesis will argue that economic development takings should be recognised as a distinct category of takings at the theoretical level, with respect to fundamental rules that protect private property. This claim will be made on the basis of a theory of property that is broader than typical approaches found in the law and in legal scholarship. Specifically, the thesis rejects the view that private property should be understood as a form of entitlement protection.\footnote{For a famous entitlements-based view of private property, see \cite{calabresi72}.}

Instead, chapter \ref{chap:2} will argue for a social function understanding, with an emphasis on human flourishing as the normative foundation for private property.\footnote{For human flourishing theories of property generally, see \cite[chapter 5]{alexander10}.} In short, property should be protected because it can help people flourish, as members of a democratic society.\footnote{See also \cite[1089]{crawford11}.} Moreover, property is meant to serve this function not only for the owners themselves, but also for the other members of their communities.\footnote{See generally \cite{gray94,alexander09d,alexander14}.}

Such an ambitious take on property must necessarily also give rise to a broader assessment of legitimacy when the state interferes with it.\footnote{See also \cite{underkuffler06}.} This is what inspires my initial discussion on economic development takings in chapter \ref{chap:2}. There I will present the basic definition of the notion and discuss the {\it Kelo} case in some more detail. Specifically, I will argue that Justice O'Connor's strongly worded dissent -- finding that the taking should be rescinded -- is consistent with, and conducive to, a social function perspective on property.\footnote{See \cite[494-505]{kelo05}.}

It should be emphasised that the focus will be on the question of when a taking is legitimate as such, not the question of how much compensation should be paid to the owners. Of course, the two questions are related; the amount of compensation offered can influence the degree of legitimacy of the interference. Some scholars go further and argue that the legitimacy question is primarily about finding the appropriate mechanism for awarding compensation.\footnote{See generally \cite{fennell04,bell07,lehavi07}.} With the theoretical approach to property adopted in this thesis, this view must be rejected; the social functions of property are not reducible to financial entitlements. Moreover, the aim of this thesis is to discuss precisely those aspects of legitimacy that {\it cannot} be addressed through compensation. The link to compensation will be mentioned when it seems relevant, but the compensation issues that arise for economic development takings will not be analysed in any depth.\footnote{For such an analysis, see \cite{dyrkolbotn15a}.}

While I will advocate for a broad approach to the question of legitimacy of economic development takings, this thesis will remain focused on private property. Other potential consequences of economic development, such as effects on the environment and social welfare impacts, will be considered as they arise in disputes about takings, not as issues in their own right. These aspects of economic development will therefore receive less attention here than they would in a thesis focusing specifically on environmental law or social and economic rights.  That said, one of the main arguments made in chapter 2 is that the social function perspective on property implies that we should take broader societal and environmental effects into account also when assessing the legitimacy of interfering with private property. The thesis argues for this coming from property theory, and in so doing will also touch on the conservation and social justice dimensions of economic development. In addition, chapter 2 will argue that if the social functions of private property support egalitarianism and equity at the local level, then private property can serve as an anchor for justice also with respect to the environment and the social and economic rights of non-owners. This point will also be made in Part II of the thesis, when discussing the issue of hydropower development in Norway. In future work, I hope to develop the idea further, to embark on research that will connect the social function account of property even more closely with environmental law and social and economic rights.

Chapter \ref{chap:3} builds on the property theory developed in chapter \ref{chap:2} by giving a more in-depth presentation of the legitimacy question, leading to a proposal for a justiciable legitimacy standard that makes judicial intervention possible. To make the discussion concrete, the chapter first offers a brief review and comparison of jurisprudential developments in the US, the UK, and at the European Court of Human Rights. These jurisdictions are chosen specifically for their many close connections with the Norwegian system, making them a natural reference point also for the case study in Part II of the thesis. Moreover, all the countries discussed are in comparable socio-economic situations, with property having a similar social function across the jurisdictions studied. This permits a comparative discussion that focuses specifically on the issue of takings, reducing the risk that the analysis will be distorted by significant differences in the social and economic context of takings law across different jurisdictions. The broader insights gained from the work done in this thesis might still be highly relevant to jurisdictions that have not been explicitly discussed. But a further treatment of this issue, for instance with respect to jurisdictions from the developing world, raises additional questions that must be left for future work.

Based on the choice of jurisdictions justified above, the thesis reviews several approaches to judicial review, culminating in a recommendation for a perspective based on institutional fairness that I trace to recent developments at the European Court of Human Rights. Specifically, the Court in Strasbourg has begun to look more actively at the systemic reasons why violations of human rights occur, in order to address structural weaknesses at the institutional level in the signatory states.\footnote{See generally \cite{leach10}.} This approach is arguably one that fits very well with the sort of analysis carried out by Justice O'Connor in {\it Kelo}, perhaps more so than the approach induced by the Fifth Amendment.

Importantly, the institutional perspective appears to be a sensible middle ground between procedural and substantive approaches to legitimacy, directing us to focus on decision-making processes without giving up on substantive fairness assessments. To ensure fairness, in particular, is not just about making good decisions, but also about how those decisions came about, and how likely it is that the system will have disproportionate effects at the societal level. This way of thinking about legitimacy brings me to the second main theme of this thesis, concerning the question of {\it democratic merit}.

\section{A Democratic Deficit in Takings Law?}\label{sec:1:2}

As discussed in chapter \ref{chap:2}, two of the key social functions of property is to promote social justice and to facilitate democratic decision-making.\footnote{See also, with further references, \cite{rose96,jackson10}.} In addition, property is meant to serve as a bridge between individual needs, community interests, and policy making at the national and the international stage. Through the law of property, societal priorities can be communicated to owners and communities without depriving them of their right to self-governance.\footnote{This is discussed in depth in chapter \ref{chap:2}, sections \ref{sec:2:4} and \ref{sec:2:5}.}

These functions of property are easily undermined if there is an excessive concentration of power and wealth among the elites of society. As indicated by Justice O'Connor's dissent in {\it Kelo}, this is one of the key reasons why economic development takings should be looked at with suspicion. In short, the concern is that economic development takings can both reflect and exacerbate a {\it democratic deficit}.

%hat can be applied to economic development takings. This test consists of a list of indicators that can suggest eminent domain abuse. The role of property in this regard is particularly clearly felt at the personal and local level, as a fair distribution of property is a highly effective safeguard for the basic rights of individuals and communities. In addition, property is meant to serve a key function as a bridge between the local level and policy making at the national and the international stage. Through the law of property, many societal priorities that might otherwise necessitate direct governmental control and interference can be effectively communicated to communities without depriving them of their right to self-governance.

There are many symptoms that can suggest a lack of democratic legitimacy, and to make the discussion more concrete, chapter \ref{chap:3} proposes a legitimacy test consisting of nine indicators of eminent domain abuse. The first six points are due to Kevin Gray, while the final three are additions I propose on the basis of the work done in this thesis.\footnote{For Gray's original points see \cite{gray11}.} I call the resulting list the extended Gray test, a heuristic for inquiring into the legitimacy of an economic development taking. 

Arguably, the most important indicator is the one pertaining to the overall democratic merit of the taking (one of my additions). When taken together with the other points, this indicator should induce an assessment of legitimacy against the decision-making process as a whole. Hence, it emphasises the institutional fairness perspective. If a taking fails the legitimacy test on this point, it might indicate an existing weakness of the system or a trend towards deterioration of the institutional framework surrounding eminent domain. This problem, moreover, might not be noticeable unless one considers an aggregated view of all the indicators of the extended Gray test, to shed light on what they tell us about the democratic legitimacy of existing practices.

Admittedly, asking courts to test for legitimacy is an incomplete response to the worry that economic development takings might result from, and give rise to, a democratic deficit at the societal level. This point has been argued by some US scholars, who claim that increased judicial scrutiny is neither a necessary nor a sufficient response to concerns about the institutional legitimacy of takings such as {\it Kelo}.\footnote{See generally \cite{lehavi07,heller08}.} Instead, these scholars try to come up with institutional innovations that can restore legitimacy in cases when the government wishes to ensure economic development on private property.

The most notable work in this direction so far is that of Heller and Hills, proposing what they call Land Assembly Districts (LADs) as possible alternatives to the use of eminent domain.\footnote{See \cite{heller08}.} The idea is that LADs will be set up to replace the traditional takings procedure in cases where property rights are fragmented and the potential takers have commercial incentives. The basic mechanism is one of self-governance; the owners themselves should be allowed to decide whether or not development takes place, by some sort of collective choice mechanism (possibly as simple as a majority vote). In this way, the holdout problem can be solved (individual owners cannot threaten to block development to inflate the value of their properties). At the same time, the local community's right to manage its own property is recognised and respected.
 
The LAD proposal is closely linked to more general ideas about self-governance and sustainable resource management, particularly the theories developed by Elinor Ostrom and others.\footnote{See \cite{ostrom90}. For the connection with property theory generally, see \cite{ostrom10b,rose11,fennel11}.} On the basis of a large body of empirical work, these scholars have formulated and refined a range of design principles for institutions that can promote good self-governance at the local level.\footnote{For a more recent empirical assessment (and refinement), see \cite{cox10}.}

At the end of chapter \ref{chap:3}, I argue that this work can be used to address the legitimacy of takings in a principled way, to arrive at refinements or alternatives to the proposal made by Heller and Hills. Specifically, it seems that alternatives to expropriation based on self-governance can be a powerful way to address the worry that economic development takings might otherwise be associated with a democratic deficit. At the same time, the context-dependence of solutions along these lines make sweeping reform proposals unlikely to succeed. Rather, it is important that the institutions that are used are appropriately matched to local conditions.\footnote{See \cite[92]{ostrom90}.} This sets the stage for the second part of the thesis, consisting of a case study of takings for Norwegian hydropower development.

%For instance, a setting where property is evenly distributed among members of the local community might suggest a very different type of institution compared to a setting where the relevant property rights are all in the hands of a small number of absentee landlords. In short, the idea of using self-governance structures in place of eminent domain necessitates a more concrete approach, a move away from property theory towards propty practice. 

%The first key objective of this case study is to apply the theory developed in the first part to analyse the legitimacy of takings for hydropower. The second objective is to study a concrete institutional alternative to expropriation in more depth, namely the system of {\it land consolidation courts}. In Norway, these courts are empowered to set up self-governance organisations for local resource management and economic development, if necessary against the will of individual owners.

%In light of this, the case study will shed light on both of the two key conclusions drawn in the theoretical part of the thesis.

%Alternative test the theoretical assertions made about how to approach legitimacy. Specifically, the question to be addressed is to what extent the traditional narrative of takings is capable of doing justice to the property conflicts that have arisen regarding the development of hydropower in Norway.

\noo{ I arrive at several objections against the details of the particular institutional arrangements proposed, particularly with regards to their likely effectiveness. It seems, in particular, that both proposals fail to recognise the full extent to which prevailing regulatory frameworks concerning land use and planning would have to be reformed in order to make their proposals work.

At the same time, I argue that these novel institutional proposals are extremely useful in that they point towards a novel way to frame the issue of legitimacy in takings law. In particular, I explore the hypothesis that traditional procedural arrangements surrounding takings suffer from a democratic deficit, a particularly powerful source of discontent in economic development cases.

This idea is the second key focus point of my thesis. First, I approach it from a theoretical point of view, by exploring the notion of {\it participation} and its importance to the issue of legitimacy, particularly in the context of economic development. It seems, in particular, that {\it exclusion} could be a particular worrying consequence of certain kinds of economic development takings, namely those that lack democratic legitimacy in the local community where the direct effects of the taking are most clearly felt.

I believe this to be a promising hypothesis, and I back it up by considering the social function theory of property and the notion of human flourishing which has recently been proposed as a normative guide for reasoning about property interests. I pay particular attention to the importance of communities that has been highlighted in recent work, as a way to bridge the gap between individualistic and collectivist ideas about fairness in relation to property.

I take this a step further, by arguing that a focus on communities naturally should bring institutions of local democracy to the forefront of our attention. The role that property plays in facilitating democracy has been emphasised before by other scholars, and I think it has considerable merit. However, I also argue that it is important to resist the temptation of viewing its role in this regard through an individualistic prism. It is especially important to take into account additional structural dimensions that may supervene on both property and democracy, such as tensions between the periphery and the centre, the privileged and the marginalised, as well as between urban and rural communities.

It is especially important, I think, to appreciate the effect takings can have on local democracy. For one, excessive taking of property from certain communities might be a symptom of failures of democracy as well as structural imbalances between different groups and interest. But even more worrying are cases when the takings themselves, brought on by a commercially motivated rationale, appears to undermine the authority of local arrangements for collective decision-making and self-governance. This dimension of legitimacy, in particular, is one that I devote special attention to throughout this thesis.

I also believe, however, that it is hard to get very far with this sub-theme through theoretical arguments alone. Hence, to explore it in more depth, I go on to assess it from an empirical angle, by offering a detailed case study of takings of Norwegian waterfalls for the purpose of hydropower development. This case study, in turn, will allow me to cast light on two further key themes, that I now introduce. %This brings me to the second part of my thesis, which in turn consists of two main themes, where the latter aims to bring me back towards a more general setting, by delivering some recommendations for how best to deal with economic development takings.
}
%I go on to consider the hypothesis that economic development takings demonstrate that takings law suffer from a {\it democratic deficit}.

\section{Putting The Theory to the Test}\label{sec:1:3}

In Norwegian law, the legitimacy questions that arise with respect to takings usually begin and end with the issue of compensation.\footnote{See generally \cite{dyrkolbotn15,dyrkolbotn15a}.} If an owner has grievances about the act of taking as such, rather than the amount of money they receive, takings law has very little to offer. In fact, it does not appear to offer anything that does not already follow from general administrative law. The owner can argue that the taking decision was in breach of procedural rules, or grossly unreasonably, but the chance of succeeding is slim.\footnote{See \cite[384-386]{dyrkolbotn15b}.}

%This narrative of legitimacy is not unique to Norway. It seems that in Europe, unlike in the US, the issue of legitimacy is often seen as predominantly concerned with the issue of compensation. In particular, the jurisprudence at the ECtHR is typically focused on compensatory issues. Moreover, while many constitutions of Europe, including the Norwegian, include public interest clauses, the courts make little or no use of these when adjudicating takings complaints. In the words of the ECtHR, the member states enjoy a ``wide margin of appreciation'' when it comes to determining what counts as a public interest.

In cases involving hydropower development, the position of property owners is also strongly influenced by sector-specific legislation, as well as special administrative and market practices. Chapter \ref{chap:4} begins the case study by discussing this in more depth, setting the stage for the discussion on expropriation that follows in chapter \ref{chap:5}. A first important observation is that the hydropower sector in Norway was liberalised in the early 1990s.\footnote{The crucial legislative reform was the \cite{ea90}.} This means that the energy companies benefiting from eminent domain are now commercial enterprises, not public utilities.

A second important observation is that the right to harness the power of water is considered private property in Norway, typically owned in common by members of nearby rural communities.\footnote{See \indexonly{wra00}\dni\cite[13]{wra00}. This arrangement has long historical roots and makes intuitive sense in a mountainous country with a very vast number of small and medium sized rivers coming down from steep outfield mountains. For the historical development of the law on this point, see \cite[109-116]{nordtveit15}.} This does not mean that freely running water, as a substance, is subject to private property. What it means is that riparian owners have an additional stick in the bundle of rights that the law associates with being the owner of land over which water flows. A useful comparison can be made with fishing rights; the right to the hydropower in a river arises from landownership, but it is conceived of as a separate, transferable, right in property.\footnote{Apart from this explicit recognition of water power as a separate right in property, the Norwegian system of riparian rights appears to be historically quite similar to the riparian common law, see generally \cite{howarth15}.} It is referred to in Norwegian as a ``fallrett'', which can be translated as a {\it waterfall right}. This thesis will therefore often refer to waterfalls and owners of waterfalls when discussing the right to harness the power of water in a river.\footnote{In some cases, especially historically, a waterfall right would be formally registered as a separate unit of real property to facilitate transfer to someone other than the owner of the surrounding agricultural land. However, waterfall rights can also be formally registered as rights of use attaching to the real properties from which they arise. In relation to Norwegian expropriation law, and for the purposes of this thesis, the distinction between these two ways of registering waterfall rights will not play an important role and will not be discussed further.}

As discussed in Part II of this thesis, hydroelectric companies in Norway have traditionally had easy access to privately owned waterfalls, made possible through the government's power of eminent domain. However, since deregulation, local owners have begun to resist such takings. This has been motivated by the fact that owners can now undertake their own hydropower projects as a commercial pursuit; unlike the situation before liberalisation, owner-led development projects can demand access to the electricity grid as producers.\footnote{See, e.g., \cite{uleberg08}.} This has led to heightened tensions between takers and owners, tensions that the water authorities are now forced to grapple with on a regular basis.

%As a result, local owners now regularly protest expropriation of their rights on the grounds that they wish to {\it participate} in economic development, by carrying out alternative development projects, or by cooperating with the energy companies who wish to take their water rights. Hence, while liberalisation has rendered takings for hydropower as takings for profit, it has also empowered local owners and communities to propose alternatives. Unsurprisingly, this has led to tensions that the water authorities are now forced to grapple with on a regular basis.\footnote{See Chapter \ref{chap:4}, Section \ref{sec:4:4}.}

Chapter \ref{chap:4} argues that despite their improved position following liberalisation, local owners remain marginalised under the regulatory framework. Specifically, despite political support for locally organised small-scale development, the large energy companies have continued to enjoy a privileged position in their dealings with the water authorities. Building on this observation, chapter \ref{chap:5} goes on to discuss eminent domain in more depth. The chapter tracks the position of owners under the law and administrative practices that relate to takings of waterfalls. The key finding is that expropriation is usually an {\it automatic consequence} of a large-scale development license.\footnote{In some cases, this follows explicitly from the water resource legislation, while in other cases it follows from administrative practice. For further details, see below in chapter \ref{chap:5}, section \ref{sec:5:3}.} That is, commercial companies that succeed in obtaining large-scale development licenses will almost always be granted the right to expropriate. This right will be granted, moreover, with little or no prior assessment as to the appropriateness of depriving local communities of their resources. Indeed, the fact that expropriation tends to follow automatically from a license to develop has led the water authorities to focus their attention on the licensing question and associated procedures. No distinction appears to be made between cases involving expropriation and cases that do not. This has a significant effect on the level of procedural protection offered to local owners. For instance, according to written testimony during a recent Supreme Court case on legitimacy, the water authorities do not recognise any duty to give individual notice to local owners before processing applications that involve expropriation of their waterfalls.\footnote{The case in question was \cite{jorpeland11}.}

While the appropriateness of taking property from local people is rarely discussed, the issue of how hydropower affects the environment has received increased attention in recent decades. Sometimes, environmental impact assessments will uncover negative effects and the water authorities will reject development applications, also when the applicant is a large energy company. In general, the framework for management of hydropower in Norway has an important conservation dimension that is clearly recognised by the government. \noo{However, conservation issues are often orthogonal to the property question; in some cases, local owners of waterfalls will oppose large-scale development projects that damage the environment, while in other cases, environmental interests will block small-scale projects that the owners themselves would like to carry out. This reflects the importance of environmental issues.} However, as argued in this thesis, the effect on local owners and communities still receives little or no attention.\footnote{See especially the discussion in chapter \ref{chap:5}, sections \ref{sec:5:6} and \ref{sec:5:7}.} In order to explore this phenomenon and investigate its consequences, the thesis focuses specifically on the taking of private property, while conservation issues remain in the background.\footnote{That said, chapters 4 and 5 will touch on two key debates regarding environmental law in Norway in recent years. The first pertains to the sector-based approach to natural resource regulation, which some argue is at odds with the holistic approach encouraged by international law instruments. The second debate, which is closely related to the first, concerns the extent of the government's duty to assess alternative resource uses and development schemes when considering license applications for concrete projects. See generally \cite{winge13,backer12,backer10}.}

%This means that conservation issues remains in the background.  discuss the special issues that arise when the property rights of local owners are restricted to conserve the local environment, a relatively common example of property interference in water law, but one that does not qualify as an economic development taking. 

\noo{Although conservation is not dealt with in any depth, environmental issues will be discussed when they have a bearing on the legitimacy questions that arise when energy companies expropriate waterfalls. As I will demonstrate in chapter \ref{chap:5}, environmental organisations and energy companies both enjoy a strong positions within the regulatory framework. Recently, there has been a tendency for these two power groups to reach compromises, such that large-scale development projects are allowed to go ahead while small-scale projects are stopped because of their environmental effects. Indeed, environmental groups and commercial companies now appear to be reaching a form of mutual understanding that large-scale development is {\it better} for the environment than small-scale projects. As discussed in chapter \ref{chap:5}, this conclusion rests on what appears to be a very narrow and arguably misguided understanding of what sustainability and conservation should entail in the context of hydropower development. %Moreover, when environmental groups and large energy companies unite in this way, it raises the worry that local owners and their communities will be marginalised further.chapter \ref{chap:5} will demonstrate that the owners' position during the licensing assessment stage is highly precarious, contrasting both with the strong position of the development companies and the similarly influential role played by conservation interests.
}
In relation to the compensation issues that arise following expropriation, the owners' legal position initially grew stronger after liberalisation. Specifically, the lower courts started to compensate local owners for the lost opportunity to profit from hydropower.\footnote{See \cite{uleberg08} (specifically, it was observed that waterfalls now had a market value, due to the increasing prevalence of owner-led hydropower).} This led to a dramatic increase in compensation payments compared to earlier practice.\footnote{See especially the discussion in chapter \ref{chap:5}, section \ref{sec:5:5:1}.} However, a recent decision from the Supreme Court appears to largely reverse this development, since a large-scale license may now be considered proof that alternative development by owners is unforeseeable and therefore not compensable.\footnote{See \cite{otra13}.} 

In light of this and other data discussed in chapter \ref{chap:5}, my conclusion is that recent takings for hydropower do not in fact pass the extended Gray test. The current practices appear illegitimate with respect to the theory of property developed in Part I of the thesis. At the same time, Norwegian law offers a promising institutional path towards the restoration of legitimacy in economic development contexts. Specifically, the unique framework for land consolidation found in Norway can serve such a function. This has already been demonstrated in the context of hydropower development, where land consolidation courts have been able to successfully organise development projects on behalf of owners who wish to undertake development but disagree about how it should be done. This brings me to the fourth key theme of this thesis.

\section{A Judicial Framework for Compulsory Participation}\label{sec:4}

\noo{ In Norway, the distribution of property rights across the rural population is traditionally highly egalitarian.\footnote{This is discussed in more depth in chapter \ref{chap:4}, Section \ref{sec:4:2}.} This meant that the farmers in Norway soon became an active political force, particularly as representative democracy started to gain ground as a form of government in the 19th century.\footnote{As early as in 1837, the Norwegian parliament was so dominated by farmers that it came to be described as the ``farmer's parliament''. See \cite{hommerstad14}.}

%The Norwegian farmers were often little more than small-holders, and had few privileges to protect. Hence, they became liberals of sorts (although also known for their fiscal conservatism). The farmers as a class were responsible for pushing through important early reforms, such as the abolition of noble titles and the establishment of democratically elected municipality governments.

%However, the municipality governments were not the first example of local decision-making institutions in Norway.
The highly fragmented ownership of land meant that institutions for collective decision making had to be introduced early on in Norwegian history; some even argue that the first realisation of a truly direct democracy can be traced to Norway in the Viking age.\footnote{See \cite[23]{titlestad14}.} One of the ancient institutions for collective action is the land consolidation court. 
}
The fourth and final key theme, presented in chapter \ref{chap:6}, consists of an assessment of the Norwegian land consolidation courts. These courts have the power to order owners to undertake or allow development projects (without depriving them of their property), as an alternative to expropriation. Moreover, they are presently used in this way in the context of hydropower development. The large energy companies almost never use consolidation, but local communities often do.\footnote{According to the Court Administration, as of 2009, land consolidation proceedings had facilitated a total of 164 small-scale hydropower projects with a total annual energy output of about 2 TWh per year (enough electricity to supply a city of about 250 000 people), see \cite{dom09}.} In these cases, the land consolidation courts have proved themselves effective in making self-governance work, also in cases when some of the owners do not with to undertake development.

\noo {The typical scenario is that the owners disagree about who owns what and cannot agree on how to organise development. In other cases, some of the owners, or even a majority of them, do not wish any development at all. In these cases, it is possible for the courts to {\it compel} them to participate. 

In these situations, it is less clear how well consolidation works in practice. Plainly, there has not been enough cases of this sort to draw a clear conclusion, especially not in situations when those who favour development are a minority among the owners. However, the consolidation alternative still appears highly preferable to the expropriation alternative, especially in terms of legitimacy. Specifically, the owners who are compelled to participate do not loose their property and are not excluded from the decision-making process.}

The land consolidation alternative can make a great difference, especially since it strives to ensure legitimacy through participation. The potential democratic deficit associated with economic development takings  is dealt with by mechanisms that seek to enable owners to take active part in the management of their property in the public interest. At the same time, the procedure can be quite effective, since participation is compulsory and the consolidation judge may intervene to settle conflicts and establish organisational order. Chapter \ref{chap:6} also addresses possible objections to the procedure, but concludes that the continued development of the land consolidation institution provides the best way forward for addressing problems associated with economic development takings in Norway.

%Finally, the institution of land consolidation is assessed against Land Assembly Districts, and -- more generally -- against the idea of self-governance frameworks for managing common pool resources. I argue that it compares favourably, both because it comes equipped with in-built judicial safeguards, but also because it has such a broad scope. I note, however, that its successful use is dependent on both political will and an ability to retain key feature even in the presence of new and powerful stakeholders in the consolidation process.

If the integrity and efficiency of the procedure can be preserved, it appears to have great potential as an alternative to eminent domain in general, also in cases involving large-scale development and cooperation with external commercial actors. Moreover, while the system is designed to work in a setting of egalitarian property rights, it is interesting to consider whether key features of the procedure could inspire solutions to the takings problem in other jurisdictions. %Specifically, the fact that the procedure focuses on benefiting properties rather than owners means that a broader understanding of property can itself suggest a broader range of possible applications. 

It might well be, for instance, that a land consolidation approach coupled with a human flourishing understanding of property can be a good way of including non-owners in the process, in jurisdictions where property rights are not distributed as widely among the population as in Norway. This might make the procedure more complex and give rise to new risks of abuse by local elites, but it seems like an interesting idea to explore in future work. 

In short, the consolidation alternative provides a starting point for an approach to legitimacy that takes a wider view of what property is, and what role it can and should play in a democratic society. In this way, the chapter on consolidation also returns to the conceptual premise discussed in the first chapter of this thesis, whereby the purpose of property is to promote human flourishing.

%the core features of land consolidation for economic development can be preserved and developed further, 
\noo{ In the second part of the thesis, I put the theoretical framework to the test by applying it to a concrete case study, namely that of Norwegian hydropower. Following liberalisation of the energy sector in the early 1990s, hydropower is now a commercial pursuit in Norway. Moreover, there is a long tradition for granting energy producers the power to acquire property compulsorily, including the necessary rights to exploit the energy of water, rights that are subject to private property under Norwegian law. This has resulted in tension and controversy, however, as the original owners of these rights, typically local farmers and small-holders, see the commercial potential of hydropower being transferred to other commercial interests, to the detriment of their own, and their communities', interest in self-governance and economic benefit.}

\section{Some terminological and conceptual clarifications}

{\it Property} is a key notion in this thesis. As mentioned already, it is an elusive legal term, with different decomposable meanings depending on the context of use and the jurisdiction within which we find ourselves. In the first part of this thesis, the notion is explored conceptually, to develop a theory of property's role and purpose within law and society. The details of how the notion is defined in a given jurisdiction will not be our concern. In general, there is quite some variation in this regard, among the different jurisdictions considered in this thesis. With respect to the European Convention of Human Rights, for instance, we encounter a notion of property (or ``possession'') that is very wide. The property clause in the Convention relies on a concept of property that covers a range of social welfare entitlements and immaterial benefits, including future pension payments and goodwill acquired by holding a professional title.\footnote{See \cite[73-77]{allen05}.} This is a broader concept of property than that usually encountered in private law, also in jurisdictions that incorporate the Convention into their national law. The issues that can arise form this, when several distinct notions of property co-exist in the law, will not be considered in this thesis; rather, the thesis will remain focused on ``classical'' instances of property, typically property in land and related resources.

That said, the theoretical argument made in chapter 2 might well be relevant for property lawyers working with disputed definitions of private property within a specific legal framework. Indeed, the theory presented in this thesis can be used to argue normatively that a given jurisdiction relies on a notion of property that is either too wide or too narrow to cater to important social functions. For instance, it would be interesting to consider the implications that a social function understanding can have in the context of intellectual property, specifically to shed light on the normative question of what kinds of incorporeal property the law {\it should} recognise. However, this line of research must be left for future work.

There is one special type of property encountered in this thesis that deserves a special mention. This is the notion of {\it common property} in land and natural resources. This notion is notoriously ambiguous, used to refer to at least three different kinds of legal arrangements.\footnote{\cite[714-715]{bishop75}. See also \cite[12-13]{fennel11}.} First there are open-access resources, which are sometimes (erroneously) referred to as common property. These resources are characterised by the fact that everyone is in principle entitled to make use of them. Hence, it is more accurate to say that they are resources that have no owner. The use of such resources is typically managed by the government through regulation, sometimes under a public trust doctrine. The questions of sustainable resource management and governance that arise in this regard are interesting in their own right, but are not considered in any depth in this thesis.

The second type of legal arrangement often referred to as common property is the collection of rights and responsibilities attaching to common land. This is land over which a specific group of people enjoy use rights and where special rules are in place to regulate the exercise of these rights and the management of the underlying resources. A typical example is found in the law of the commons in England and Wales, as regulated today by the Commons Act 2006.\footnote{See generally \cite{rodgers10}.} Use rights on the commons can be thought of as property rights, but under individualistic accounts of what property is, it might not be appropriate to do so. The distinguishing feature of rights in common is that they provide an anchor for a special legal framework, a set of rules, institutions and customs that pertain specially to the communal character of such rights. This function of rights in common can be distorted if those rights are fitted into an entitlements-based framework for maintaining rights in property.\footnote{For a concrete example, see \cite[469--471]{rodgers10} (analysing the effect of the Commons Registration Act 1965 (England and Wales)).}

Under a social function theory of property, by contrast, it becomes much more appropriate (at the conceptual level, at least) to think of rights in common as private property rights. Moreover, as discussed further in chapter 2, the social function theory can support arguments to the effect that {\it all} instances of property, even traditional forms of private property, can be viewed as being part of a commons in an abstract sense of the word.\footnote{See also \cite[16-18]{fennel11} (``Property, as experienced on the ground, is never wholly individual nor wholly 
held in common, but instead always represents a mix of ownership types.'').} This point will also be made in chapter 6, when I discuss how land consolidation can be used to {\it set up} institutions for collective management of private property rights within a local community. This form of property intervention can be understood as an effort to bring key ideas behind the commons to bear on private property rights. In this regard, the connection with the commons is made at the theoretical level; the thesis will not address concrete regulatory frameworks that regulate the commons as such. The special questions that can arise when property is taken for economic development within a commons, will be left for future work.

\noo{Moreover, according to wider, more functional, definitions of what private property is, rights in common may well be covered. There can be little doubt, for example, that the use rights of individuals having rights in common over some resource {\it do} constitute property rights (``possessions'') within the meaning of Article 1 of Protocol 1 of the European Convention of Human Rights.

At the same time, the distinguishing feature of rights in common is that they are surrounded by a special legal framework, a set of rules, institutions and customs that pertain specially to the communal character of such rights. This thesis will not investigate concrete examples of such legal frameworks, except briefly in chapter 4 when I present different property regimes found in Norway.}

%That said, commons also tend to come with many specific regulatory provisions and institutional arrangements, none of which 

%positive law regulations which might not bthis thesis will not explore in any depth those special rules and arrangements that are in place to regulate the commons in any specific It should be noted, however, that the special questions that might arise when economic development takings take place in the commons will not be addressed in this thesis. The theory developed herein should be applicable, but further exploration of this must be left for future work.

The third legal arrangement that can be referred to as common property is encountered when a property is owned, in the standard private law sense of the word, by a group of owners. Shared forms of private ownership are supported by most jurisdictions, including those considered in this thesis. Shared private ownership is particularly important in Part II of the thesis, since the takings discussed there will typically involve rights to water that are owned by several private parties in common under a legal framework that most closely resembles the common law concept of a tenancy in common. A brief presentation of this form of private ownership in Norway is provided in chapter 5, along with a discussion of the importance of egalitarianism in Norway.

%affect local communities as a whole, not just individuals. However, the cases considered will all be cases where individuals have recognised rights in property, meaning that the cases fall uncontroversially within the ambit of takings law. The margins of takings law, encountered for instance if groups of non-owners make proprietary claims based on customary use rights or the like, will not be considered in any depth. However, the theory of property developed in this thesis might suggest making normative claims to the effect that legal standing in takings proceedings should be extended to cover a larger group of legal persons than those presently recognised. A closer examination of this is left for future work. 

%In the second part of the thesis, when dealing specifically with Norwegian law, we will encounter a few specific forms of private property that deserve a special mention. 
%In Norway, we find a uniquely egalitarian distribution of land ownership, where land and the resources found on it are typically owned by groups of local small-holders, not landlords or public bodies. This form of shared ownership is considered a conventional form of private ownership under Norwegian law. There are also some large commons in Norway, but they are of lesser practice importance due to the prevalence of shared private ownership over outfields. Further details on property arrangements found in Norway are provided in chapter \ref{chap:3}.

{\it Legitimacy} is a second key notion used in this thesis. The notion is consistently used in a normative sense, to describe that an interference in private property appears morally justified.\footnote{For a concise presentation of moral legitimacy, see \cite[438-441]{thomas14}. See also \cite{michelman04,priel11}. Moral legitimacy becomes particularly important under natural law theories, since such theories make the moral status of a rule directly relevant to the question of its legal validity. However, moral legitimacy is also relevant on a positivist understanding of law; it is a descriptive fact that moral considerations shape the law, not only through explicit law-making, but also because judges are unable to completely separate formal reasoning about legal content and validity from moral reasoning about legitimacy. For a longer argument to this effect, coming from a self-described positivist, see \cite[1801-1802]{fallon05}.} It is not used as a term with a specific descriptive meaning within a given jurisdiction. A key aim of this thesis is to address the question of when an interference in private property {\it should} be regarded as legitimate. All the jurisdictions I consider have their own specific rules in place that are meant to ensure legitimacy in takings law. The most common legal terms that are used in this context are the notions of {\it public use}, {\it public purpose}, and {\it public interest}. Specifically, a typical takings provision states that the public must benefit, directly or indirectly, in order for an interference in private property to count as legitimate. Such provisions or their near equivalents can be found in a range of different jurisdictions, including all those studied in this thesis.

The meanings of the terms used are similar across different contexts and legal systems. Still, since these are formal legal terms, it is worth keeping in mind that their meaning is relative to the jurisdiction under consideration. For instance, while most of the jurisdictions considered in this thesis do not recognise any substantial difference between public use, public interest and public purpose, some jurisdictions maintain such distinctions and attach important legal consequences to them. In some cases, the correct way to make these distinctions is a highly controversial question. Most famously, the position that public use literally means use by the public, and is therefore quite distinct from public interest and public purpose, is forcefully advocated by several US legal scholars, including at least one member of the Supreme Court.\footnote{As demonstrated by \cite{kelo05}.}

When terms such as public use, public interest or public purpose are used in this thesis, their exact meaning corresponds to the meaning given to them by the jurisdiction under consideration. If the terms occur in theoretical discussions, their meaning should be understood according to a natural language interpretation that points to the general idea behind using terms like these in the law of takings. The reason why notions such as public interest and public use are important is that they can help enforce the natural idea that interferences in private property should only occur for the good of the people. This much is common to all jurisdictions considered in this thesis. %However, how the general idea is implemented varies quite considerably. To account for this, the thesis will briefly clarify the meaning of the terms whenever they appear in the context of a concrete jurisdiction.

At a more general level, the work done in this thesis suggests that it might be inappropriate to rely on formal terms such as public use or public interest when attempting to ensure legitimacy in takings law. Specifically, this thesis will argue that legitimacy requires decision-making to take place in an equitable and inclusive manner, such that the owners and those who depend most on the properties in question have a say that is commensurate with what is at stake for them. This perspective, combining procedural and substantive ideals of fairness, will not rely on finer distinctions between notions such as public use, public purpose and public interest. In my opinion, this is a strength of the theory developed in this thesis, an escape from what Gregory Alexander calls the ``formalist trap'', characterised by an exaggerated focus on constitutional property clauses and how they are formulated.\footnote{See \cite[Chapter 1]{alexander06}.} As Alexander argues, excessive formalism can cloud the issue of legitimacy because it blocks from view those important institutional and political processes that determine the actual level of protection given to property and its owners within a given jurisdiction. Building on this, my thesis will develop an integrated approach that looks at the institutional context and the substantive fairness of economic development takings, to arrive at a theory of legitimacy that focuses on the social functions of private ownership.

% will be used throughout this thesis, and it will be used in a normative sense.


%These are presented in  section \ref{sec:x} of chapter \ref{ In general, we find a uniquely egalitarian distribution of land ownership in Norway, where undeveloped land and the resources found on it are typically owned by groups of local small-holders, not landlords or public bodies. In section \ref{sec:1:5}, we already mentioned the concept of a waterfall right, used to refer to the right to harness power from a river, an historically important stick in the property bundle associated with landownership in Norway. Moreover, we mentioned briefly that waterfall rights are usually held in common by members of the local population. %Enjoying private ownership in common is not unusual in Norway, particularly in rural areas, and the law of property in Norway reflects this in various ways.

%In some cases, this is because a river suitable for hydropower development will run across many distinct private properties. Hence, the relevant waterfall rights are held in common in the narrow sense that an assembly of private rights is required in order for development to take place. However, in most cases, waterfalls suitable for hydropower development will be owned in common in a somewhat stronger sense. Indeed, outfields in Norway are often held under a specific form of co-ownership, such that each small-holding in the local community owns a share in the land surrounding their local community. This form of co-ownership has no exact common law equivalent, but is most similar to the tenancy in common. However, there is no requirement that the co-ownership takes place behind a trust -- all individual shareholders are formally registered as owners of their share of the land and their is a presumption in favour of continued co-ownership accompanied by productive use of co-owned land, not a presumption in favour of sale and individuation as seen in UK law.

%In the Common Ownership Act 1965, further rules are given to regulate the use of land under co-ownership. The main principle is that each owner has a right to the ``normal'' enjoyment of the property, taken in light of the local conditions, customs, and the original purpose of the co-ownership arrangement (if it is known). Moreover, an individual owner's use must not exceed what corresponds to his share of the property and must not be unduly burdensome to the other owners. If damage occurs, moreover, compensation must be paid. To some extent, the majority shareholders can enforce a specific use of the property which would also be binding on the minority. This includes new forms of commercial activity on the property. If such activity is organised against the will of a minority, the minority will still be entitled to take part in the enterprise. 

%There are limits to what the majority can do. Importantly, they cannot do anything that will limit the ``normal'' use of the property by any owner. Also, they cannot do anything to dramatically change the character of the property, sell it, or use it as security for debt. Because of this, gridlock can often result if the owners disagree fundamentally about how to manage their land. For real property, particularly in rural areas, the standard way of resolving such situations is to bring a case before the Norwegian land consolidation courts. These courts are empowered to either dissolve the system of co-ownership or else to organise joint use of the land. Indeed, the prevalence of common ownership over outfields is one of the reasons why land consolidation courts are so important in Norway, and it also explains why they have been granted wide powers to help organise the use of privately owned land. I return to the details of this in chapter \ref{chap:6}, as part of a broader discussion on how the institute of land consolidation can be used as an alternative to eminent domain in economic development situations.

%In addition to the form of co-ownership regulated in the Common Ownership Act 1965, there are two other special forms of ownership of land found in Norway that should be briefly mentioned. Both pertain to land over which a large group of people enjoy extensive rights of use that have been recognised as so-called common rights under Norwegian land law. There is always an owner of the land in the normal private law sense of the word, but special rules are in place to protect the group of people who enjoy use rights. These rules presuppose that the land is owned either by the state or a council of the local community (which might not include everyone who enjoys use rights over the land). There is no concept of a commons in Norway that attaches to land owned by private individuals, which is quite natural given that private landlords and tenant farming is  absent from the structure of rural landownership in Norway.

%If the owner of common land is the state, the relevant legislation that protects the rights of the local people is the State Commons Act 19... If the land is owned by a local community, the relevant legislation is the Village Commons Act 19... The details differ, but the main principle of both acts is that they offer special protection to use rights holders, especially with regard to traditional land uses that local farmers depend on for their livelihoods.

%After the industrial revolution, there was some doubt as to whether common rights gave non-owners a claim to waterfall rights, or whether waterfall rights were held exclusively by the landowners. This was particularly important for land owned by the state, since common rights was the only potential route for local community members to claim a proprietary stake in local hydropower resources (in village commons, by contrast, the owners would typically themselves be local community members). The question was settled by the Supreme Court in the case of {\it Vinstra} in 196.. Here it was held that no rights to waterfalls on state-owned lands could be derived from rights in common over that land. For this reason, the takings issue does not arise with respect to hydropower development on such land, at least not with respect to the waterfall rights as such. 

%Of course, questions still arise regarding the fate of local communities when development takes place on state-owned land where local people enjoy rights in common. However, questions that arise specifically with respect to Norwegian commons law will not be dealt with in this thesis.\footnote{When we consider the case of {\it Alta} in Chapter \ref{chap:5}, we will encounter state-owned land where the aboriginal Sami population has claimed to enjoy rights in property similar to common rights. In recent years, this claim has met with some recognition within the Norwegian legal order, giving rise to yet another form of property in Norway. For further details, see the discussion in Chapter \ref{chap:5} section \ref{chap:5:x}.}
%However, when the {\it Alta} case was decided, members of the Sami population were considered as rights holders in the traditional private law sense of the word, no different from non-aboriginal holders of property and use rights elsewhere in Norway. See the

%In Chapter \ref{chap:5}, we will discuss the {\it Alta} case in some depth. This case was a takings case arising from hydropower development in Finnmark, a part of Norway where the state is traditionally regarded as the owner of all outfields. The state's ownership of land in this region tends to be at odds with the aboriginal interests of the Sami people. Traditionally, the state's ownership was consider to be entirely unencumbered by aboriginal entitlements except where Sami use rights had been explicitly recognised. Moreover, the rights of the Sami people did not have the protected status granted to rights in common over state land.\footnote{Some scholars disputed this, by arguing that Sami rights should be viewed as common rights by analogy with the legislation in place for state commons.} 

%Hence, in the {\it Alta} case, the formal standing of the Sami people was derived from expressly recognised use and property rights that would be lost or depreciate in value following the development. Specifically, the Supreme Court rejected claims based on aboriginal rights, and the case did not involve takings of waterfalls as such. Still, the case has been considered an important precedent for disputes surrounding expropriation of waterfalls, since it dealt with many aspects of administrative law pertaining to the licensing procedure surrounding hydropower development. In addition, as I discuss briefly in chapter \ref{chap:5}, the case marked a watershed moment in the legal history of the Sami people, whose rights over land in Finnmark have since received greater recognition within the Norwegian legal order. Today, in the special context of Sami land, the law appears to be moving towards a framework where the Norwegian state is increasingly seen as a custodian of Sami lands, rather than an owner in the standard private law sense.

%In other parts of Norway, a similar perspective has not developed. Natural resources owned by the state, or taken under eminent domain, has the same legal status as private property, except that the owner happens to be the state. There is no recognised legal sense in which the state is held to be a custodian of land, and there is no legal doctrine according to which state-owned lands are supposed to be held in trust on behalf of the people. However, in a recent revision of the Constitution, a new section was introduced that compels the government to preserve the environment and promote sustainability. The exact wording is as follows:


%This provision replaces a similarly broad sustainability provision that was first introduced in the Constitution in 1992. In practice, the sustainability requirement has left little impact in Norwegian law, with no consequences discernible at all within the law of property. No one, to my knowledge, has proposed to read section 112 as having any direct bearing on the state's rights and responsibilities as the owner of land. Rather, the section is typically understood to give rise to a general obligation to promote sustainability through regulation, meaning that regulatory failures could conceivably be challenged under the provision. So far, however, few challenges of this kind have appeared and none have been successful. . Indeed, it has been argued that the constitutional sustainability provision as such has been something of a failure.

%After the new formulation was introduced in 2014, there have been some indications that the legal status of the provision might be about to change, in the direction of becoming more easily justiciable. In fact, a group of Norwegian environmental lawyers are presently preparing a case where they will challenge the Norwegian state with not doing enough to fight climate change, a legal action that will be brought under section 112 of the Constitution. In the law of property, however, there is no indication that the provision will become important any time soon. Similarly, in the law of hydropower, there have been no indications to suggest that the provision will be considered relevant to disputes between developers and local people, especially not when such disputes arise with respect to the issue of expropriation. For this reason, the sustainability provision in the Norwegian constitution will not be examined further in this thesis. Of course, a normative argument could well be made that the provision {\it should} entail greater regard for the interests and property rights of local people. Such an argument might perhaps also be backed up by considerations based on sustainability research and international environmental law. Further exploration of economic development takings from this angle will be left for future work.

%First, we will encounter ownership of waterfalls, a concept that appears to be unique to Norwegian law. As mentioned in section \ref{sec:1:5}, above, the right to harness power from a waterfall in Norway has a recognised status as an incident of private landownership. Moreover, it can be transferred separately from the surrounding land, voluntarily or through expropriation. 

%This is the Norwegian property type referred to as a ``vannfall'', literally translated: a waterfall. This is a legal term with a specific (although disputed) meaning in Norwegian law. In the second part of the thesis the word waterfall will be used in the specific sense that the word ``vannfall'' is used under Norwegian law. The intuitive understanding of the term is suggestive but incomplete, so a clarification is in order: a waterfall is used in the law to refer to a power, namely that of water flowing along a given stream or river. That is, the waterfall is a term the law uses to refer to the energy that can be harnessed from a river from a point A to a point B, where A and B are typically given as the altitude where a given waterfall begins to where it ends. 

%What this means is that a waterfall owner, under Norwegian law, has a right to the hydropower of a river. This right usually emerges from ownership of land over which the water flows, but it is considered a distinct ``stick'' in the bundles of riparian owners. It is also (on some conditions) separately alienable. This is important because it means that in Norway, developing a hydropower plant requires ownership of the waterfall, nor merely access to suitable sites for building the dam and the station. In Norwegian law, one does not take the view that the building of a hydropower plant creates the hydropower. Rather, the power of the water already exists and already has owners, usually the local landowners. To some extent, waterfall rights under Norwegian law can be compared to fishing rights in English law.

\noo {\section{Structure of the Thesis}\label{sec:1:5}

My thesis is divided into two parts. Part I sets up a theoretical framework for reasoning about property and proceeds to study the legitimacy of economic development takings in more depth. Part II consists of a case study of takings for hydropower, focusing on how expropriation and alternatives to it work on the ground in Norway. In brief, the structure of the chapters are as follows.

Chapter 2 introduces the topic of this thesis and presents the social function theory of property. The chapter argues that the descriptive core of this theory should be accepted irrespective of one's normative inclinations; the social function approach is simply more accurate than other theories. From this descriptive assertion, the category of economic development takings arises naturally. To address it normatively, the chapter argues that the notion of human flourishing provides the appropriate starting point. On this basis, the chapter discusses economic development takings and {\it Kelo} in more depth, to introduce the key question of legitimacy.

Chapter 3 proceeds to address the legitimacy question in more depth. The chapter starts from considering the procedural approach to legitimacy, illustrated by the law of England and Wales. Following up on this, the substantive approach is considered, illustrated by the law of the US. Finally, the chapter argues for a middle ground between the two, an institutional fairness perspective that is also linked to recent developments at the ECtHR. Following up on this, the chapter presents the extended Gray test; a set of indicators of eminent domain abuse suitable for an institutional fairness approach. The chapter concludes by discussing the possibility of providing institutional alternatives to expropriation for economic development, taking inspiration from the theory of self-governance for common pool resources.

Chapter 4 introduces the case study of takings for hydropower in Norway. The chapter briefly presents hydropower in the law, focusing on the licensing legislation. Then the chapter investigates hydropower in practice, noting that the liberalisation of the electricity market in the early 1990s has had a dramatic effect. Specifically, the chapter emphasises how local owners of water resources are now in a better position to develop these themselves, since they can access the electricity grid as producers on equal terms as larger companies. The chapter goes on to study the tension that has resulted between large-scale development facilitated by expropriation and small-scale development facilitated by local property rights. Despite early signs that small-scale solutions enjoyed political support, the large energy companies now appear to be reasserting their control over the hydropower sector, to the detriment of owners and their local communities.

Chapter 5 discusses expropriation of hydropower in more depth. The chapter starts by giving a brief overview of Norwegian expropriation law, before noting that expropriation for hydropower often takes place on the basis of special rules that leave owners with less protection. The history of the law is discussed in quite some detail, to show how the law has gradually developed to undermine local property rights over water resources. Following up on this, the chapter discusses case law on the expropriation and licensing, focusing on the legitimacy question (which is addressed in Norway almost solely on the basis of procedural standards). The chapter studies the recent Supreme Court case of {\it Jørpeland} in depth, to shed light on how current administrative practices impact on owners and their communities. The conclusion is that current takings practices do not appear legitimate.

Chapter 6 discusses land consolidation as an alternative to expropriation. The chapter starts by clarifying the notion of land consolidation and how the Norwegian understanding of that terms is much wider than that found in other jurisdictions. Following up on this, the chapter discusses consolidation as an alternative to expropriation, by focusing on those tools that the consolidation courts have at their disposal in this regard. Then the chapter gives a more in-depth presentation of some cases when consolidation was used to organise small-scale hydropower development. Finally, a discussion is provided on the prospect of using consolidation to replace expropriation more generally, in Norway and possibly also in other jurisdictions.

Chapter 7 contains my conclusions, formulated as an attempt at connecting the concrete and abstract aspects of this work around two threads, tracking property's relationship with excluding and taking on the one hand and its relationship with giving and participation on the other. My final conclusion is that the latter two notions characterise true property, and that property as such is worth defending.

} 
