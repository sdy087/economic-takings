\chapter{Property, privilege and power}\label{chap:1}

\begin{quote}
It's nice to own land.\footnote{Donald Trump}
\end{quote}

\section{Introduction}

In this chapter, I provide a bird's eye view on my topic, placing it in the theoretical landscape. My aim is to explain the key concepts that I will rely on to make sense of the empirical data considered in subsequent chapters, and to present the main values that I will look to when I give normative assessments. In addition, I will relate my theoretical approach to current strands in property theory, focusing on those aspects of property theorizing that I regard as particularly relevant to the work done in this thesis.

I will strive to show that my approach to the empirical data I have gathered is sound and informative, by focusing on principles of {\it legal} reasoning. I will not provide an extensive presentation of concepts or theoretical approaches developed in other fields, such as political science, sociology, economy, or psychology. However, I note that all these fields also engage in interesting ways with the notion of property, and I think a multi-disciplinary approach can be illuminating, also to the topic studied in this thesis. Hence, while I focus on legal and --  to some extent -- philosophical theories of property, I will try to make a note of those specific questions I consider that are also analysed in related fields.

The crucial argument made in this chapter is that the category of {\it economic development takings} is relevant to legal reasoning about certain kinds of situations when private property is taken by the state. This is not {\it prima facie} clear. In fact, I am prepared to face critics who will argue that the category makes no legal sense at all. Fortunately, it makes perfect intuitive sense; it targets situations when property is, quite literally, taken for economic development. In most cases considered in this thesis, this is even the explicitly stated aim used to justify eminent domain. Hence, the factual basis for our categorization can not be questioned. 

However, the theoretical basis is not at all clear. Indeed, a superficial look at dominant legal theories of property might indicate that under most property rules, the nature of the project benefiting from a taking should not be in focus. Rules aiming to protect property invariably focus on the rights of the affected owner, making clear that he enjoys some degree of protection against uncompensated state interference. But how, then, do we justify having regard to the {\it purpose} of the taking, when considering its legitimacy with respect to the owner's rights? Are those rights not equally interfered with when  property is taken for some uncontroversially public project, like a new public road? Is it not in fact a little small-minded, even short-sighted, to worry about the taker's gain, instead of concentrating on what the owner, if anything, stands to lose?

Much of the work in this chapter, albeit highly abstract, is aimed at countering this very concrete objection. I believe it is an important one, since it threatens to undermine the conceptual basis for the kind of study that I present in this thesis. Moreover, it is an objection that I think it is inappropriate to dismiss without further comment. In the context of US law, this would be possible, since economic development takings have, as a matter of fact, gained recognition as an important category of legal reasoning. In Europe, however, this has not yet happened, at least not to the same extent.

The reason for this difference is not that US law contains special rules that emphasize the importance of the distinguishing features of economic development takings.\footnote{In fact, many state laws now {\it do} contain such rules, following the backlash of the controversial decision in {\it Kelo v. City of New London}. However, such rules were introduced only after the category of economic development takings first came to prominence in legal discourse.} Rather, the difference must largely be attributed to the fact that economic development takings have resulted in great political controversy in the US, a controversy that has influenced both the law and legal scholars. Hence, in the absence of a similar political climate in Europe, a conceptual discussion on the basis of the notion of an economic development taking seems warranted. 

As I argue in this chapter, providing an adequate account of them forces us to broaden our theoretical outlook compared to most traditional forms of legal reasoning about property. However, I find considerable support for the necessary conceptual reconfiguration when I consider recent trends in property theory, particularly those that focus on the {\it social function} of property.
Indeed, the crux of the main argument presented in this chapter is that this function not only allows us, but also compels us, to pay attention to the special dynamic that arises in cases when private property is taken by the state for someone else's profit.

Perhaps the most straightforward way of describing this is to say simply that it is {\it unfair}. Indeed, this has some merit also as a conceptual position, since it will probably explain quite effectively why takings for profit have become so unpopular in the US. However, a more thoughtful assessment of economic development takings reveal that matters are not so simple. Indeed, it seems that economic development takings are an almost unavoidable consequence of any system that emphasize both state control over property and public-private partnerships in the economic sector. To condemn this political model of government is a possible response, but not one I will pursue in this thesis. Rather, I will focus on getting to the heart of what characterizes economic development takings, so that I may address also the question of how to deal with them, to ensure that they become less offending.\footnote{Some propose an outright ban, but even those who support such a measure, should be interested in demarcating the category more closely, to arrive at a better understanding of what exactly will be banned.}

Therefore, the stark contrast between the intuitive response that a taking for profit is unfair, and the legal assessment that what matters is only the loss to the owner, needs to be considered further. Indeed, a somewhat cheeky approach might be to say that the feeling of unfairness is in itself a loss that the owner incurs, so that the law had better take it into account. But, of course, not any subjective feeling should be given weight in a legal context. The question, therefore, is what exactly the feeling of unfairness can help us uncover about the nature of economic development takings. Does it uncover something legally significant?

In my view, the answer is yes, and in the following I will argue for this position in depth. To motivate this largely abstract analysis, I will begin by considering a concrete scenario which illustrates the need for contextual assessment and more fine-grained conceptual categories for reasoning about cases when demands for economic development come to pose a threat to established property rights.

\section{Donald Trump in Scotland}

On the 10th of July 2010, the property magnate Donald Trump opened his first golf-course in Scotland, proudly announcing that it would be the ``best golf-course in the world''.\footnote{http://www.golf.com/courses-and-travel/donald-trump-scotland-golf-course-lives-hype (accessed 06 July 2014).} Impressed with the unspoilt and dramatic seaside landscape of Scotland's east coast, the New Yorker, who made his fortune as a real estate entrepreneur, had decided he wanted to develop a golf course in Balmedie, close to Aberdeen.

To realize his plans, he purchased the Menie estate in 2006, with the intention of turning it into a large resort with a five-star hotel, 950 timeshare flats, and two 18-hole golf-courses. The local authorities were not particularly keen on the idea and planning permission was initially denied by Aberdeenshire Council. Particularly worrying to the councilors was the fact that the proposed site for the development was declared to be of special scientific interest under EU conservation legislation. The frailty and richness of the sand dune ecosystem, in particular, suggested that the land should be left unspoilt for future generations.\footnote{See \url{http://en.wikipedia.org/wiki/Donald_Trump#Scottish_golf_course} (accessed 06 July 2014).} But in the end, Trump got his way, as he was able to convince Scottish ministers to give him the go-ahead on the prospect of boosting the economy by creating some 6000 new jobs.\footnote{See \url{http://www.theguardian.com/world/2008/nov/04/donald-trump-scottish-golf-course} (accessed 06 July 2014).}

Activists continued to fight the development, launching the ``Tripping up Trump'' campaign to back up local residents who refused to sell their properties to Trump. One of these, the farmer and quarry worker Michael Forbes, expressed his opposition in particularly clear terms, declaring at one point that Trump could ``shove his money up his arse''. Trump, on his part, had described Forbes as a ``village idiot'' that lived in a ``slum''\footnote{See \url{http://www.bbc.co.uk/news/10205781} (accessed 08 July 2014).} Moreover, he had suggested, through his legal council, that Forbes only refused to sell in an effort to get a better price for his land.  Forbes was offended and he proudly declared that he would never consider selling to Trump, as the issue had now become ``personal''.

At the height of the tensions, Trump considered his legal options, asking the local council to consider issuing compulsory purchase orders (CPOs) that would allow Trump to take property from Forbes and other recalcitrant locals. If carried out, this would have been an iconic example of an economic development taking, where private land is acquired to benefit for-profit development schemes. It would not be the first time that the power of eminent domain had been used to the benefit of Donal Trump's business empire. In the 90s, Trump famously succeeded in convincing Atlantic City to allow him to take the home of one Vera .... , to facilitate further development of Trump's casino facilities. This taking was struck down by state courts, however, a move hailed by many US activists as a milestone in their fight against economic development takings. 

In Scotland, Trump's plans were met with widespread outrage. The media coverage was wide, mostly negative, and an award-winning documentary had been made which painted Trump's activities in Balmedie in a highly negative light. The controversy also made its way into UK property scholarship. Professor Kevin Gray, in particular, a leading expert in property law, expressed his opposition by declaring that the proposed taking would be an act of ``predation''. 

In fact, the case prompted Gray to formulate a number of key features that could be used to identify situations where compulsory purchase would be more likely to represent an abuse of power. He noted, in particular, that Trump's proposed takings would fall in line with a general tendency in the UK towards using compulsory purchase to benefit private enterprise, often in the context of so-called ``public-private'' partnerships that are also meant to serve public purposes.

Hence, the possibility of CPOs being used in Balmedie was very much a real one; if he had put his weight behind it, Trump might well have been able to make a successful case that existing statutory authorities could be used to facilitate a taking of private land for his golf resort. Indirectly, this would undoubtedly benefit the public in terms of job-creation and increased tax revenues. Moreover, Scottish ministers had already gone far in expressing their support for the plans.

But then, in a surprise move, Trump announced he would not seek CPOs after all. Quite possibly, he was discouraged by the negative press and felt that public relations might suffer. More importantly, he had found another strategy, namely that of containment.
\noo{
If Trump had succeeded in getting CPOs to undertake his golf resort, Balmedie would no doubt have become the scene of an interesting legal battle concerning the status of economic development takings under UK law. The importance of such a case would be great and the outcome would be uncertain. But Trump did not pursue this course of action. Instead, he opted for a strategy of containment.
}
\noo{ But it was not to be. outcome would be uncertain. On  the one hand, the right to private property is considered to be at the core of most, if not all, western legal systems. In the UK, it is both a constitutional principle of common law, going back all the way to Magna Carta, as well as a human right enshrined in the Human Rights Act 1998. Recent developments would have been in Trump's favor, though. As one high court judge recently noted, before dismissing CPO objections made by property owners, ``.....''  But with much of the public so clearly on the side of the locals, the outcome would be far from guaranteed. Tampering with property, after all, is still risky business. Far better, then, to turn property into an advantage. }
He erected large fences, planted trees and created artificial sand dunes, all serving to prevent the properties he did not control from becoming a nuisances to his golfing guests. One local owner, Susan Monroe, was fenced in by a wall of sand some 8 meters high. ``I used to be able to see all the way to the other side of Aberdeen'', she said, `` but now I just look into that mound of sand''. She also lamented the lack of support from the Scottish government, expressing surprise that nothing could be done to stop Trump.

But there was little to do. As soon as Trump decided to build around them, the neighboring property owners found themselves completely marginalized. Trump had the backing of the government, having been declared as a job-creator whose activities would boost the economy in the region. Indeed, he had even received an honorary doctorate at the Robert Gordon University, a move that prompted the previous vice-chancellor ... to hand his own honorific back in protest.

In the end, then, it was not by the taking land of others that Trump triumphed in Scotland. Rather, he succeeded by exercising ``despotic dominion' over his own. But the problem was solved nonetheless: after he fenced them in, his neighbors were hard to see and hard to hear. The Balmedie controversy went quiet, the golfers came, Trump got his way. As he declared during the grand opening: ``Nothing will ever be built around this course because I own all the land around it.... It's nice to own land.''\footnote{See \url{http://www.theguardian.com/world/2012/jul/10/donald-trump-100m-golf-course} (accessed 06 July 2014).}

The tale of Trump coming to Scotland serves to illustrate the main issue I will be looking at in this thesis: the legitimacy of economic development takings. In addition, it serves to put the work into perspective, showing that what it means to protect property against undue interference can depend highly on the circumstances. For a while, it looked like Balmedie was about to become a canonical case of an economic development taking. But in the end, it became rather an illustration of something far more subtle, namely that our understanding of property's meaning and value is deeply shaped by social, political and economic structures. It seems clear, in particular, that Donald Trump's ownership of the Menie estate has a vastly different meaning than does Micheal Forbes' ownership of his farm. For many, the one represents some combination of power, privilege and progress, while the other represents a mix of defiance, community and sustenance. Very different and, in this case, diametrically opposed to one another. 

According to Trump and his supporters, protecting property rights against interference in Balmedie should no doubt involve protecting the governmentally sanctioned golf resort plans from interference by ``outsiders''. Perhaps even to the extent that it could justify interference with the property rights of other, less significant, stakeholders. But for Michael Forbes and the other local owners, protecting property rights is likely to have a completely different meaning, focused on defending them and their local community against a disruptive and damaging plan that would seem to turn both them and their land into golfing props. Again, perhaps protection would have to involve limiting what actions Trump might undertake on his own land, to the detriment of his neighbors. Either way, protection implies interference and vice versa. This is a major challenge to a simplistic view whereby protecting property rights is a black-and-white proposition, where governmental power on the one hand stands against personal liberty on the other. 

In reality, the situation is far more subtle and, importantly, how we asses it depends crucially on what we perceive as the ``normal'' state of property, the alignment of rights and responsibilities that we deem to be worthy of protection. This, in turn, depend highly on what values we believe is embodied in property and on what kinds of property rights that serve to enhance those values. For example, one value that is crucial for property activists in the US is the value of equality and freedom. When property ownership is comparatively egalitarian, this speaks in favor of limiting governmental interference in property rights. In Scotland, however, where land ownership is notoriously unegalitarian, it might as well be a call for land reform and redistribution, aiming to ensure a more equal structure of land ownership. This is just one illustration of a broader theme: the contextual nature of property.

In this thesis, I will aim to take it into due account when I assess concrete cases. First, I will present the theory behind this perspective on property.

\section{Theories of property}

What is property? In common law jurisdictions, the standard answer is that property is a collection of individual rights, or more abstractly, {\it entitlements}.\footnote{The term ``entitlement'' was used to great effect in the seminal article \cite{calabresi72}.} Being an owner, it is often said, amounts to being entitled to one or more among a bundle of ``sticks'', streams of protected benefits associated with, or even serving to define, the property in question.\footcite[357-358]{merrill01} This point of view was first developed by legal realists in response to the natural law tradition, which conceptualized property in terms of the owner's dominion over the owned thing, particularly his right to exclude others from accessing it.\footcite[193-195]{klein11} In civil law jurisdictions, rooted in Roman law, a dominion perspective is still often taken as the theoretical foundation of property, although it is of course recognized that the owner's dominion is never absolute.\footnote{See, e.g., \cite[?]{foster10}.}

In modern society, the extent to which an owner may freely enjoy his property is highly sensitive to government's willingness to protect, as well as its desire to regulate. To civil law theorists, this sensitivity was most naturally thought of as giving rise to varying degrees of restrictions in property rights, but for common law theorists, overlooking a legal system with roots in a relatively stable feudal tradition, it was more natural to think of it as {\it constitutive} of property itself.\footcite[7]{chang12} Indeed, the bundle of rights theory was not based on entirely novel ideas. As natural law thinking about property became increasingly hard to reconcile with the reality of increasing state regulation, common law theorists returned to a more modest perspective on property.\footcite[195]{klein11} Arguably, the bundle metaphor itself was distilled from the traditional estates system for real property, which was turned into a theoretical foundation for thinking about property in the abstract.\footnote{See \cite[7]{chang12}   
(``The ``bundle of rights'' is in a sense the theory implicit in the common law system taken to its extreme, with its inherently analytical tendency, in contrast to the dogged holism of the civil law.'').} 

Property rights under the bundle of rights theory are thought to be directed primarily towards other people, not things.\footnote{See \cite[357-358]{merrill01} (``By and large, this view has become conventional wisdom among legal scholars: Property is a composite of legal relations that holds between persons and only secondarily or incidentally involves a ``thing''.'').} This underscores a second important point about property in the real world, namely that the content of rights in property are necessarily relative to the totality of the legal order. For instance, relying on a bundle metaphor, it becomes perfectly natural that a farmer's property rights protects him against trespassing tourists, but not against the neighbor who has an established right of way. 

By contrast, the dominion theory suggests viewing such situations as exceptions to the general rule of ownership, which implies a right to exclusion. In the case of property, exceptions no doubt make up the norm. But in civil law jurisdictions one lives happily with this. It takes the grandeur away from the dominion concept, but it retains a nice and simple structure to property law. There are many limitations and additional benefits attached to property, but they are all in principle carved out of one initial right, namely that of the owner. In this way, the system becomes more easily navigable.

An interested party may ask, ``who is the owner?'' Then, under the dominion theory, a clear answer is expected and will usually be adequate, even if it does not give a complete picture of all relevant property rights. Under the bundle theory, on the other hand, one may be inclined to respond, ``to which stick are you referring?'' Clearly, this narrative is more complex, and perhaps unduly so. 

Indeed, some common law scholars have recently elaborated on this to develop a critique of the bundle theory, suggesting that it should at least be complemented by a firm theory of {\it in rem} rights in property. This, they argue, allows the law to operate more effectively, by relying on a simple and clear rule that, although being defeasible, will generally suffice to inform people about their relevant rights and duties in relation to property.\footnote{\footcite[793]{merrill01b} (``The unique advantage of in rem rights -- the strategy of exclusion -- is that they conserve on information costs relative to in personam rights in situations where the number of potential claimants to resources is large, and the resource in question can be defined at relatively low cost.''); \footcite[389]{merrill01} (``The right to exclude allows the owner to control, plan, and invest, and permits this to happen with a minimum of information costs to others.''). See also \cite{ellickson11} (arguing that Merrill and Smith's analysis nicely complements and improves upon the bundle theory).} 

There are also other, less pragmatic reasons, why a dominion approach may be preferable, even if the bundle metaphor is arguably more accurate with respect to the practical reality of property. In particular, some scholars point out that the bundle theory does not adequately reflect the sense in which property is a right to a {\it thing}, serving to create personal attachments that are not easily reducible to a set of interpersonal legal relationships.\footnote{References.} In the US, where the bundle theory still dominates, thoughts like these seem to be gaining ground, and the debate about the true nature of property is still very much alive, also among legal scholars.\footnote{See, e.g., .....}

In this thesis, the efficiency and clarity of different property concepts will not be a primary concern, nor will personal attachments to things in themselves play a particularly important role.\footnote{Although the personhood-aspects of property that are often highlighted in this regard will also be relevant to my analysis of economic development takings. However, this is not something that I think warrants extensive engagement with the bundle/dominion debate.} Hence, I will remain largely agnostic about this aspect of the debate between dominion and bundle theorists. In particular, the differences between civil and common law traditions in this regard do not cause special problems for my analysis of economic development takings. To me, a more important question concerns the theoretical justification for considering this as a special category of takings. To what extent does such a classification have legal implications? To what extent {\it should} it have such implications? In addition, I am interested in the {\it values} that various property theories serve to promote, particularly with regards to the question of when interference in property is legitimate under constitutional and human rights law. Hence, a first question to ask is whether there are any differences between dominion and bundle theories in this regard.

Intuitively, one might think that bundle theorists are likely to endorse greater room for state interference in property rights. Indeed, thinking about property as sticks in a bundle may lead one to think that property rights are intrinsically limited, so that subsequent changes to their content -- carried out by a competent body -- are mere reflections of their nature, not a cause for complaint. In particular, the theory conveys the impression that property is highly malleable. For the theorists that developed the bundle of sticks metaphor in the late 19th and early 20th century, this aspect was undoubtedly very important. By providing a highly flexible concept of property, they helped the state gain conceptual authority to control and regulate. Indeed, this was the clear intention of many early proponents of the bundle theory -- the ``progressives'' of their day.\footcite[195]{klein11} The early bundle theorists not only developed a theory to fit the law as they saw it, they also contributed to change.

In relation to takings law, the progressives succeeded in gaining acceptance for the use of eminent domain to benefit a wider range of public purposes than had so far been considered legitimate.\footnote{See generally \cite{yale49}.} In particular, they argued successfully that the so-called ``public use'' restriction, which had previously been enforced quite strictly, particularly by state courts, should be greatly relaxed. This change was important in creating the situation which led to economic development takings becoming a contentious issue in the US, and so provides important background to the main topic of my thesis.  I return to the public use debate in the US in much more depth later, in Chapter \ref{chap:2}, Section \ref{sec:?}. Here I would like to stress that I think there can be little doubt that the increased scope given to eminent domain in the early 20th century was mutually conducive to the conceptual reorientation that took place during the same time.

In relation to the different, but related, issue of so-called regulatory takings, the bundle theory even  became directly implicated. A regulatory taking occurs when governmental control over the use of property becomes so severe that it must be classified as a taking in relation to the law of eminent domain. Particularly in the US, the question of when regulation amounts to a regulatory taking is highly controversial. The stakes are high because takings have to be compensated in accordance with the Fifth Amendment of the US constitution. At the same time, the law is unclear; a lack of statutory rules means that regulatory takings cases are often adjudicated directly against constitutional property clauses (often the relevant state constitution, in the first instance).

If property is thought of as a malleable bundle of entitlements that exists only because it is recognized by the law, it becomes more natural to argue that when government regulates the use of property, it does not deprive anyone of property rights, but merely restructures the bundle. In the case of {\it Andrus v Allard}, the Supreme Court adopted such an argument when it declared that ``where an owner possesses a full ``bundle'' of property rights, the destruction of one ``strand'' of the bundle is not a taking, because the aggregate must be viewed in its entirety''.\footcite[65--66]{andrus79}

Historically, therefore, it seems that bundle theorists have been largely aligned with those that favor a less restrictive approach to eminent domain. But I think it is wrong to conclude that the bundle theory {\it necessarily} implies such a stance on takings. Indeed, some prominent scholars have argued for an almost entirely opposite view. Professor Epstein, in particular, goes far in suggesting that as every stick in the property bundle represents a property right, government should not be permitted to remove any of them without paying compensation.\footcite[232-233]{epstein11} Moreover, Epstein does not believe that the bundle theory is responsible for the fact that his view of property has not been widely endorsed by US courts. Instead, he thinks the real reason behind what he sees as the pernicious influence of progressive thought is related to their ``top-down'' approach to property. That is, their tendency to view property rights as vested in, and arising from, the power of the state, not the possessions of individuals.\footnote{\cite[227-228]{epstein11} (``In my view, the nub of the difficulty with modern property law does not stem from the bundle-of-rights conception, but from the top-down view of property that treats all property as being granted by the state and therefore subject to whatever terms and conditions the state wishes to impose on its grantees''.).} 

In my opinion, Epstein's argument shows that adoption of the bundle theory can hardly be considered a determinate factor which will necessarily influence the level of protection private property enjoys in a given legal system. Moreover, Epstein successfully demonstrates that as a rhetorical device, the theory may well be turned on its head. Unsurprisingly, the substance of the law, in the end, turns primarily on the values one adheres to, not the theoretical constructions one relies on when expressing those values.\footnote{To further underscore this point, it may be mentioned that while US courts do in fact recognize that a regulation can amount to a taking, this is practically unheard of in several other common law jurisdictions, including England and Australia, which nevertheless paint property in a similar conceptual light. Moreover, while the issue of regulatory takings is considered central to constitutional property law in the US, it is considered a fairly marginal issue in England.\cite{altermann12}}

In the civil law world, the relationship between property theorizing and property values is similarly hard to pin down at the logical level. To illustrate, I will again point to the question of regulatory takings. In some countries, like Germany and the Netherlands, the right to compensation is quite strong, but in other civil law countries, such as France and Greece, it is very weak. The exclusive dominion understanding of property does not commit us to any particular kind of policy on this point. Indeed, the theory appears to cater comfortably to a range of different politically determined solutions to the problem of striking a balance between the interests of owners and the interests of the state. 

On the one hand, the undeniable fact of modern society is that property rights are enforced, and limited, by the power of government. Hanging on to the idea of dominion, then, necessarily forces us to embrace also the idea that dominion is not enjoyed absolutely and that government may interfere in property rights. In this way, the theory may serve as a conceptual basis upon which to argue for a more relaxed approach to protection of property rights. These rights are not absolute anyway, so why worry about interfering in them for the common good? But this story too may be turned on its head: A libertarian may well use a similar story to warn us that property is under increasing threat from an interfering state. Hence, he may argue, unless we want to completely lose our grasp of what property is, we had better enhance the level of protection offered to property owners.

To me, the upshot is that the differences between common law and civil law theorizing about property are not significant enough to 
make them crucial to the questions studied in this thesis. In particular, the differences between the bundle theory and the exclusive dominion account of property do not appear to speak decisively in favor of any particular approach to economic development takings, nor does it provide any clear justification for regarding such takings as special in the first place. Property enjoys constitutional protection and is a recognized human right across the divide, but what this means in practice is hard to deduce from both the bundle and the exclusive dominion view of property. 

In terms of descriptive content, both theories are too bold and oversimplified. They provide a manner of speech, but they do little to enhance our understanding of the reality of property rights in modern society. In particular, they do not provide a functional account of what role property plays in relation to the social, economic and political structures within which it resides. In terms of normative content, on the other hand, they are both too bland and imprecise. They simply do not offer much clear guidance as to what norms and values the institution of property is meant to serve. They give neat explanations of what property is, but do not tell us {\it why} it should be protected. 

In the following, I will address both these shortcomings by considering property theories with a wider scope. There are many candidates that could be considered. In a recent book, Alexander and Pe\~{n}alver present five key theoretical branches: 
\begin{itemize}
\item {\it Utilitarian} theories, focusing on property's role in helping to maximize utility or welfare with respect to individual preferences and desires. 
\item {\it Libertarian} theories, focusing on property's role in furthering individual autonomy and liberty and the importance of protecting property against state interference, particularly attempts at redistribution. 
\item {\it Hegelian} theories, focusing on the importance of property to the development of personhood and self-realization, particularly the expression and embodiment of free will through control and attachment to one's possessions. 
\item {\it Kantian} theories, focusing on how property arises to protect freedom and autonomy in a coordinated fashion so that {\it everyone} may potentially enjoy it, through the development of the state. 
\item {\it  Human flourishing} theories, focusing on property's role in facilitating participation in a community, particularly as a template allowing the individual to develop as a moral agent in a world of normative plurality.
\end{itemize}

It it beyond the scope of this thesis to give a detailed presentation and assessment of all these theoretical branches and the various ideas that have been discussed within each of them. However, in Section \ref{sec:x} below, I will present the human flourishing theory in more detail. This is because I believe that if it is adopted, it suggests making a range of new policy recommendations regarding how the law {\it should} approach the question of economic development takings. First, however, I must note that all the theories mentioned above are highly normative, used actively to promote the adoption of particular values associated with property. While I am not unwilling to take a stand in this debate, my main objective is to study economic development takings descriptively, by giving a case study of Norwegian waterfalls and discussing its significance in terms of comparative law and with respect to the general academic debate on economic development takings. In this regard, I first need a theoretical framework that allows me to meaningfully discuss those aspects that make economic development takings unique, without thereby committing myself to any particular stance on how to normatively assess those aspects. 

To arrive at such a foundation, I will rely on the descriptive aspects of the so-called {\it social function theory} of property.\footnote{See generally \cite{foster11,mirow10,alexander09a}. Be aware that some authors, particularly in the US, also speak of the {\it social obligation} theory, using it more or less as a synonym for the social function theory.} While this theory is often implicated in normative theories, including the human flourishing theory, I argue that it has a descriptive core which is also of great significance. Its importance to my work in this thesis is underscored by the fact that I will draw on the social function theory to answer the pressing problem of what makes ``economic development takings'' a legitimate and useful category of legal reasoning. 

Let me first note that it is not {\it prima facie} clear that the category makes any legal sense at all. It clearly makes intuitive sense, however; property is taken for economic development. In most examples I will consider, this is even the explicitly stated aim used to justify eminent domain. So the factual basis for our categorization can not be questioned. However, neither the exclusive dominion account nor the bundle theory gives us any clear reason to think that the categorization is legally relevant. It is not clear, in particular, if we should attach legal significance to the distinguishing features of economic development takings. This, indeed, is a claim that some might be intuitively inclined to reject. Why, after all, is it appropriate to have regard to the {\it purpose} of the taking, when considering its legitimacy with respect to the owner's rights? Would those rights not have been equally interfered with if the property had been taken for some uncontroversially public project, like a new public hospital? In fact, is it not a little small-minded and short-sighted for the law to worry about what the taker stands to gain, instead of concentrating on what the owner, if anything, stands to lose?

To respond successfully to this potential objection, I believe it is necessary to look at property's social function. Property scholars are becoming increasingly aware of the need to do this in general, as they realize that the social function features implicitly in many areas of the law which is ostensibly concerned with individual entitlements. Many still reject that this necessitates conceptual reconfiguration, but the social function of property is steadily gaining ground, also as an important aspect that must be considered already to understand how the law works. I believe its key descriptive insights provide the most appropriate way to argue that it is theoretically desirable to regard economic development takings as a special issue in property law, and I will argue for this below. 

However, before making my specific point about takings, I will present the social function theory of property more generally. I will focus on showing that it captures aspects that are already highly relevant -- behind the scenes -- to how property rules are understood and applied in concrete situations. It seems, in particular, that socio-legal arguments play an important, yet often unacknowledged, role when courts interpret fundamental rules that are meant to protect private property. Bringing those aspects into the open is in itself a worthwhile project to pursue, irrespectively of any further normative stances that the social function theory might give the theorist occasion to adopt.

\section{The social function of property}

There is a growing feeling among property scholars that the notion of property has been drawn too narrowly by many of the traditionally dominant theories of property. Some have even gone as far as to challenge the idea that property is a meaningful and well-defined concept at all. These scholars point out that what counts as property in a given legal system, and what property entails in that system, depends largely on its social and political context, tradition, and even chance.\footnote{For a particularly clear and inspiring exposition of property's elusive nature, see \cite{gray91}.} In the US, a utilitarian law-and-economics approach -- which largely takes the social and political underpinnings of property for granted -- has long been regard as standard, but even there the tide is turning. While most US scholars still regard property as a robust and meaningful category of legal thought, many are increasingly turning away from assessing property rules narrowly against their effectiveness in maximizing individual utility and social welfare. Instead, these scholars adopt a holistic approach, which allows property's social function to come into focus. One of the main proponents of this conceptual shift is Gregory S. Alexander, professor at Cornell University. In a recent article, he writes:

\begin{quote} Welfarism is no longer the only game in the town of property theory. In the last several years a number of property scholars have begun developing various versions of a general vision of property and ownership that, although consistent with welfarism in some respects, purports to provide an alternative to the still-dominant welfarist account.[...] These scholars emphasize the social obligations that are inherent in ownership, and they seek to develop a non-welfarist theory grounding those inherent social obligations.\footcite[1017]{alexander11}
\end{quote}

To scholars coming from political science, sociology or human geography, this trend will not raise many eyebrows, except perhaps for the fact that it is a recent one. After all, in fields such as these, property has never been understood merely as a set of individual entitlements. Rather, property is seen as a crucial part of the fabric of society, one that entrenches privileges and bestows power. Even scholars who believe that the institution of property is a force for good, recognize that being an owner carries with it political capital, social responsibility, and membership in a community. Those aspects, moreover, are often regarded as more important than entitlements explicitly recognized in positive legal terms. Crucially, they are important not only to the individual owners but also to society as a whole. How property rights are distributed among the population, for instance, has obvious political and economic implications, serving as a source of power and privilege to some groups, while marginalizing others. \footnote{References needed.}

But what is the relevance of this to property law? Usually, jurists approach property in isolation from such concerns, and often they do so because of practical necessity. The political question of what the law should be might require musings about the purpose and social context of property, but in the day-to-day workings of the law, such considerations play a lesser role, with the importance of clear and simple rules outweighing the possible benefit that would result from contextual and holistic assessment. But no functioning theory of  property would deny that social aspects such as expectations and obligations play a role in relation to property {\it in life}. The problem, rather, is that classical theories of property may be accused of taking the pragmatic view too far, by failing to recognize that many social functions are {\it intrinsic} to property, so that they may sometimes be impossible to disregard, also when the law is applied in practice.

This accusation can be raised against both bundle and dominion theorists. They both leave little conceptual room for considering property as a social phenomena. It is recognized, of course, that rights in property -- bundled or otherwise -- serve to regulate social relations. But this effect is typically regarded as belonging to the periphery of property as a legal category, more relevant to sociologists than to property scholars. In addition, it is uncommon to observe that the causal relation between property rights and society is bidirectional, since the meaning and content of property itself is partly determined by the very same social structures that property helps establish and sustain. When this aspect of property is not recognized, the risk is that subtle dependencies between property and the social order are not brought into focus, even when they play an important role in practice.

This is particularly clear when it comes to socially defined obligations attached to property. Hardly anyone would protest that in practical life, what an owner will do with his property is as much constrained by the expectations of others as it is by law. But in addition to influencing the owner subjectively, expectations can take on an objective character by being embedded strongly in the social fabric. This, in turn, may give rise to a norm, or even a custom, which may be legally relevant, either because the law gives direct effect to it, or because it influences how we interpret rules relating to the use of property.\footnote{See generally \cite{penalver09,alexander09}.}

This seems hard to dispute as a descriptive assertion, but traditional property theorists had little regard for social obligations attached to property, focusing instead almost solely on the complimentary notion of right. According to Alexander, the classical theories of property convey the impression that ``property owners are rights-holders first and foremost; obligations are, with some few exceptions, assigned to non-owners''.\footcite[1023]{alexander11} The social function theorists attempt to redress this imbalance by developing theories that can naturally accommodate an account of social obligations that attaches greater weight to them as objects of property. As Alexander explains, ``social obligation theorists do not reverse this equation so much as they balance it. Of course property owners are rights-holders, but they are also duty-holders, and often more than minimally so.''\footcite[1023]{alexander11} 

It should be noted that while it has been dormant for some time, particularly in the US, this idea is by no means new. Its first modern expression is often attributed to Le{\'o}n Duguit, a French jurist active early in the 20th century. In a series of lectures he gave in Buenos Aires in 1911, Duguit challenged the classic liberal idea of property rights by pointing to their context-dependence, adopting a line of argument strikingly similar to how recent scholars have criticized the law-and-economics discourse of modern times.\footnote{See \cite[1004-1008]{forster11}. For more details about Duguit's work and the contemporaries that inspired him, see generally \cite{mirow10}.} In particular, Duguit also pointed to the notion of obligation, stressing the fact that individual autonomy only makes sense in a social context, wherein people are also dependent on each other and related through membership in communities. Hence, depending on the social circumstances of the owner, his property could entail as many obligations as it would entail entitlements and dominion. This, according to Duguit, was not only the reality of property ownership in life, it was also a normatively sound arrangement that should inspire the law, more so than the unrealistic visions of property evoked by the liberal tradition.\footnote{See \cite[1005]{forster11} (``The idea of the social function of property is based on a description of social reality that recognizes solidarity as one of its primary foundations'', discussing Duguit's work). It should also be noted that Duguit was particularly concerned with owners' obligations to make productive use of their property, to benefit society as a whole. This raises the question, however, to which we shall return in more depth later, who exactly should be granted the power to determine what counts as ``productive use''. In this way, Duguit's work also serves to underscore one of the main challenges of regulatory frameworks that seek to incorporate and draw on property's social dimension. How should decisions be made in such regimes?} 

Similar thoughts have been influential in Europe, particularly in the post-WW2 rebuilding period. For instance, the constitution of Germany -- her {\it Basic Law} -- contains a property clause that explicitly includes a provision stating that property entails obligations as well as rights. As argued by Alexander, this has had a significant effect on German property jurisprudence, creating a clear and interesting contrast with US law.\footnote{See \cite[338]{alexander03} (``The German Constitutional
Court has adopted an approach that is both purposive and contextual, while the U.S. Supreme Court has not'').} A similar perspective was also influential in the debate among the European states that first drafted the property clause in the First Protocol of the European Convention of Human Rights.\footnote{See \cite[1063-1065]{allen10}.} Later, however, the liberal conception of property gained ground also in Europe, causing jurisprudential developments that have been particularly clear in the case law from the European Court of Human Rights (ECtHR).\footnote{See generally \cite{allen10}.} Even so, property theorizing in Europe is still influenced by a social function view on property, more so than in the US. The ECtHR, for instance, stresses the importance of {\it proportionality} when adjudicating property cases, suggesting the importance of a contextual approach to the balancing of the many private and public interests that may attach to property. 

I will return to possible normative implications of the social function theory later, but here I would like to stress that in the first instance it merely asks us to recognize an empirical truth. Property does not arise in a vacuum, but from within a society. As a philosophical proposition, this is obvious and hardly anyone denies it. But the social function theory asks us to consider something more, namely that property {\it law} continues to influence, and be influenced by, the social structures that surround it. Perhaps most importantly, property both reflects and shapes relations of power among members of a society.\footnote{This aspect of property's social function was stressed in a recent ``statement of purpose'' made by leading property scholars in support of the social function theory, see \cite{alexandder09a}.} Moreover, it does not act uniformly in this way -- the actual effect of property on power depends on the circumstances.

An indebted farmer who is prevented by state regulation from making profitable use of his land might come to find that his property has become a burden rather than a privilege. As a consequence, someone who has already amassed power and wealth elsewhere might be able to purchase it from him cheaply. Indeed, this might provide an excellent opportunity for the outsider to consolidate his position. He can ensure that his privileges become further entrenched, both socially, politically and economically. By acquiring a farm and transforming it to recreational property, he symbolically and practically asserts his dominance and power, while also reaping a potential financial benefit resulting from his investments to further an altered pattern of use. In some cases, this dynamic can even become endemic in an area, resulting in a complete reshaping of the social function of property.

The story might go like this: First, impoverished farmers and other locals sell homes to holiday dwellers. Then house prices soar. As a result, local people with agrarian-related incomes can't afford local homes, causing even more people to sell their land to the urban middle class. In this way, a causal cycle is established, the social consequences of which can be vicious, particularly to the low-income people who are displaced.\footnote{The general mechanism is well-documented and known as {\it gentrification} in human geography (often qualified as rural gentrification when it happens outside urban areas). See generally \cite{weesep94,phillips93,slater06}. For a case study demonstrating the role that state regulation can play (perhaps inadvertently) in causing rural gentrification, see \cite[1027-1030]{darling05}.} My theoretical contention is the following: Setting out to protect property in a situation like this -- when property rights pull in different directions -- requires taking some stance on whose property, and which of property's functions, one is aiming to protect. In particular, should the law protect the property rights of local people who face displacement, or should it protect the property rights of outsiders wishing to invest?

Some may shun away from this way of posing the question, by arguing that it would be better to rely on clear rules that can deliver justice to owners with a minimum level of dependence on the particular social processes that property is involved in at any given time. I am inclined to disagree with such a stance already at the conceptual level, since justice itself is a notion that largely seems to depend on social conditions. However, my main point here is that the prospect of such ``socially neutral'' rules is simply illusory when both sides of a conflict are in a position to adopt a property narrative to argue for their interests. For an excellent example of such a situation, it is enough to return to our original story of Donald Trump coming to Scotland. 

As long as Trump threatened to use compulsory purchase, the local people could adopt a traditional ``pro-property'' stance against Trump. But as soon as Trump decided to fence them in by relying on his own property rights, they had to adopt a seemingly contradictory view, {\it against} property, on the basis that Trump's rights should be limited out of concern for the community. So how do we classify the anti-Trump stance with regards to property? The answer is unclear under classical theories, but under the social function stance, it becomes easy to resolve. The locals sought to protect property, but not just any property. The property they wanted to protect was the property which served the social function of sustaining the existing community. The property they wanted to protect was the property that {\it meant} something to them.

Undoubtedly, this was also the sentiment of Trump and his supporters, who could also make a case based on property. Hence, in conflicts such as these the law will invariably have to take a stand on which properties it wishes to protect, which social functions it wishes to promote. In all likelihood, such a stand must also sometimes be taken by whoever {\it interprets} the law, since it is exceedingly unlikely that the legislature will ever be able to provide clear rules for resolving conflicts of this. Lastly, and most controversially, the courts may find occasion to curtail the power of government -- perhaps even the legislature -- if their power is usurped by powerful actors who wish to change property's functions -- to further {\it their} own interests -- beyond constitutional or human rights limits.

Property shapes and reflects societies, but it also shapes and reflects social commitments and dependencies within those societies.\footnote{See generally \cite{alexander09}.} Again, this function of property is highly dependent on context. A small business owner, by virtue of being a member of the local community, is discouraged from becoming a nuisance to his neighbors. Everything from erecting bright neon signs to proposing condemnation of neighboring properties are actions that he will be socially deterred from taking. If the local shop owner does not conform to social expectations, he will pay a social price. Indeed, most likely even an economic price, especially if his customer-base is local. At the same time, the local connection would serve to make the business owner positively invested in the well-being of the community. This would encourage everything from sponsoring local events to hiring local youths as part-time helpers.

But the business owner might also be discouraged from changing his business model to become more competitive, particularly if this is perceived to undermine the community. Economic rationality might suggest that he should expand, by physically acquiring more space and targeting new groups of customers, but social rationality might make this an untenable proposal. This, however, might render the business economically unsustainable, particularly if it is facing fierce competition from businesses that are not similarly constrained by community ties. Moreover, even if the business is in fact viable as long as the community remains in place to support it, the perception that there is room for improvement might increase external pressures both on the business and the community. Importantly, in the age of regulation for commercial facilitation, the state itself may exert pressure of this kind.

Then, if our local shop owner goes out of business, for whatever reason, the new owner might fail to become integrated in the community in the same way. Indeed, if we imagine that the new owner is a large commercial actor who is hoping to raze the community in order to build a new shopping center, we are at once reminded of the stark contrasts that exist between various social functions of property. The property rights of a shop owner can be the life nerve of a community, while the exact same rights in the hands of someone else can spell destruction. While this is an undeniable empirical fact of property ownership, it is far from clear what its legal ramifications are. Here, it is tempting to embrace a normative stance, and argue for particular social values that the law {\it should} promote. However, I would like to hold on to the descriptive mode of analysis a little further. For it is perfectly clear that regardless of whose interests win out in the end, assessments of the social function of property will have played an important role in brining about that outcome.

This is true not only when the law explicitly requires that this function is to be taken into account, such as in relation to the property clause of the basic law of Germany. It also commonly arises as an important source of information guiding the courts in interpreting and applying statutory authorities that seemingly are not concerned with social aspects of property. The classical example from the US is the case of {\it State v Shack}.\footcite{shack71} The case concerned the right of a farmer to deny others access to his land, a basic exercise of the right to exclusion often regarded as fundamental to the very definition of property. The controversy arose after the two defendants, who worked for organizations that provided health-care and legal services to migrant farmworkers, entered the land of a farmer without permission. They were there to provide services to the farmers employees and when the farmer asked them to leave, they refused.

In the first instance they were convicted of trespassing in keeping with New Jersey state law, but on appeal the Supreme Court of New Jersey overturned the verdict. The court held that the dominion of the land owner did not extend to dominion over people who were rightfully on his land, so that as long as the defendants were there at the request of the workers, the owner had to tolerate this. Importantly, the court argued for this result -- which was not based on any natural reading of New Jersey trespass statutes -- by pointing also to the fact that the community of migrant workers was particularly fragile and in need of protection. Their property right to receive visitors where they work and live, therefore, had to be recognized, in spite of this limiting the exclusion right for the farmer. 

The lesson to take from this is that the social function of property can play a role even when this does not explicitly follow from any property rules. This, in turn, has been used to argue convincingly that a shift towards a social function theory is desirable. In so far as the property rules we rely on explicitly directs us to take the social aspect of property into account when applying the law, it might be permissible for the practically minded jurist to conclude that there is little need for theorizing about property's social dimension. But as a matter of fact, cases like {\it State v Shack} show that this dimension can be relevant even when it is not mentioned in any authority, even in relation to clear rules that would otherwise appear to leave little room for statutory interpretation. It arises as relevant, in particular, because the social dimension is intrinsic to property itself. 

This might still be a radical claim, but it is primarily a descriptive one. Indeed, even if the case of {\it State v Shack} had gone the other way, I would be inclined to take from it the same lesson: If the owner's right to exclusion had received priority over the workers right to receive guests and the owner's obligation to respect this right, that too would be an outcome that would likely underscore the social function of property. To illustrate this, it is helpful to look to an article written by Eric Claeys, where he is critical both of the social function theory in general and the decision in {\it State v Shack} in particular. Given the basis on which that decision was made, he is forced to argue, however, by pointing to those aspects of the social context that speak in favor of the farmer. Hence, implicitly he adopts a form of reasoning that goes far in conforming to the social function approach. He does not simply declare that the trespass rules are absolute and that the social circumstances are irrelevant. Rather, he argues that by considering the circumstances in {\it more} depth, a different outcome suggests itself as correct. But this is no argument against the descriptive content of the social function theory, merely an argument against those who think that particular values need to be endorsed by anyone willing to look to the social context of a property dispute. In this regard, it is not hard to agree with Claeys that normative fundamentalism is wrong. Indeed, he might even have a point in criticizing some social function theorists for normative naivety. That, however, is hardly an argument to be dismissive of the {\it style} of argument that the theory promotes, and that Claeys himself skillfully engages in. Indeed, a much greater degree of normative naivety would be on display if arguments based on social functions of the exclusion right had been summarily dismissed on the basis that they were irrelevant to the case. For in that case, the entitlement-based view on property would in effect do unacknowledged normative work, with no basis in anything more authoritative than a palpably oversimplified idea of the meaning of property.

That is not to say that the social function theory does not have normative consequences. It provides a way of talking about property and analysing conflicts that will also invariably influence how we come to judge concrete cases. This is also illustrated by {\it State v Shack}. Surely, after the social context and consequences of the right to exclusion contested in that case had been made clear and recognized as legally relevant, any assessment supporting a guilty verdict would have to rely on value judgments that many would  shun away from. Despite Claeys skillful advocacy, many would no doubt fail to be convinced of the social merit of recognizing a right to exclusion in a case like {\it State v Shack}. But the crucial aspect of the social function narrative is that it makes this aspect clear, not that it commits us to, or promises to deliver, any morally superior stance on property that can deliver ``correct'' outcomes in cases such as this.

This challenges a common assumption, among both detractors and supporters of the social function theory, who argue that the theory commits us to heightened normative scrutiny of property law, in pursuit of the ``good''. Some even argue that property law should be studied from the point of view of virtue ethics.\footnote{See generally \cite{penalver09}.} Unsurprisingly, critics such as Claeys use this to launch attacks on the social function theory and its supporters, by arguing that it represents a way of thinking that will invariably lead to lessened constitutional property protection and greater risk of abuse of state powers.\footnote{See \cite{claeys09} (``The more ``virtue'' is a dominant theme in property regulation, the less effective ``property'' is in politics, as a liberal metaphor steering religious, ethnic, or ideological extremism out of the public square'').} Indeed, increasing the room for state interference is often seen as the aim of conceptual reconfiguration; the social function view of property tends to be associated with social democratic and/or redistributive political projects, by which the notion of property is recast to justify greater interference in established rights.\footnote{This motivation is clearly felt in the work of Gregory Alexander, who argues, for instance, that the social obligations inherent in property imply that the ``state should be empowered and may even be obligated to compel the wealthy to share their surplus with the poor'', see \cite[746]{alexander09}. For an assessment linking similar views on property in Europe to the dominance of social democratic thought in the post-WW2 period, see \cite{allen10}.}

It is important to note, however, that while social democratic policies may be easier to justify by emphasizing the social function of property, the mere recognition that property has an important social dimension does not in itself offer any justification for policies of this kind. For one, policy reasons must be tied to the prevailing social and economic circumstances, they will not automatically succeed merely by virtue of a conceptual shift. In addition, it seems to me that the most crucial premises used in arguments for greater state control and state-led redistribution projects concern the nature of the state, not the functions of property.

In particular, why should we believe that the state is the ultimate social institution to which property {\it should} answer? Is it not, for instance, equally possible to contend that property should continue to answer to less formal social structures that it is already embedded in by virtue of owners' membership in local communities? On this proposal, some might want to limit the state's role to ensuring fair play among individuals and communities, but others, like myself, would recognize the state's role in assessing what kinds of communities should be promoted in accordance with political goals. Embracing more direct state control, on the other hand, entails rather a commitment to the the idea that more low-level social structures fail to function properly and, crucially, that state control is {\it better}. In my opinion, this requires a separate argument. Hence, to move uncritically between talk of the ``community'' and talk of the state, as writers like Pe\~{n}alver and Alexander sometimes do, is often inappropriate.

In fact, I am inclined to believe that it is only appropriate to equate community with the state in highly special situations, for instance if it can be shown that property owners insulate themselves from, and engage in exploitative practices towards, other communities. Importantly, to argue that such a situation obtains requires a case to be made that is compelling both empirically and politically. In this regard, I believe theory alone has little to offer. This is a reason to conclude that the social function view of property in fact tells us very little about how widely the state should intervene in property in a given society. It allows us to recognize the {\it possibility} that the state may have to intervene on behalf of property values that aim to protect communities. But this is no argument in favour of the position that the state should intervene more or less often than it currently does. Even so, the job it does in allowing us to reason more clearly about {\it when} it is appropriate for the state to intervene, is significant. For instance, the theory will allow us to single out economic development takings for special consideration, but it will not commit us to a particular normative stance on such takings.

More generally, it does not follow from the recognition that property structures are social in nature that {\it any} institution should actively seek to neither change nor protect those structures. The Humean position, namely that the existing distribution of property rights represents a socially emergent equilibrium, remains plausible. Moreover, the normative stance that this equilibrium is a {\it good} one (or at least as good as it gets) remains as contentious -- and as arguable -- as ever. For this reason, I believe it is appropriate to approach the social function theory as a descriptive theory in the first instance.

It is worth emphasizing that in taking this view I depart from the stance taken by many of the contemporary scholars who advocate on behalf of social function theories, also some that reject social democratic ideals. Professor Hanoch Dagan, for instance, is a self-confessed liberal, but still explicitly and strongly argues for a social function understanding on the basis that it is morally superior. ``A theory of property that excludes social responsibility is unjust'', he writes, and goes on to argue that ``erasing the social responsibility of ownership would undermine both the freedom-enhancing pluralism and the individuality-enhancing multiplicity that is crucial to the liberal ideal of justice''.

If this is true, then it is certainly a persuasive argument for those who believe in a ``liberal idea of justice''. But for those who do not, or believe that property law is -- or should be -- largely agnostic on this point, a normative justification for the social function theory along these lines can only discourage them from adopting it. Such a reader would be understandably suspicious that the {\it content} of the social function theory -- as Dagan understands it -- is biased towards a liberal world view. He might agree that property continuously interacts with social structures, but reject the theory on the basis that it carries with it a normative commitment to promote liberalism.

Danach stands out somewhat in the literature by focusing on {\it liberal} values, but as I have already indicated, he is not alone in proposing highly normative social function theories. Indeed, most contemporary scholars endorsing a social function view on property base themselves on highly value-laden assessments of property institutions.\footnote{See, e.g.,  \cite{alexander09,crawford11,davidson11,singer09,penalver09}.} While they provide interesting insights into the nature of property, I am struck by a feeling that these writers all tend to overstate the desirable normative implications of adopting a social function view. In addition, they appear to believe that accepting this view on property requires us to embrace certain values and reject others. Moreover, one is left with the impression that the social function theory has little to offer beyond the values with which it is imbued, which can in turn push the law in the direction that these writers deem desirable. 

I disagree that this is the case, at least for the social function theory as I understand it. Dagan's theory of property might be conducive to ``liberal justice'', but this is because it involves far more than what follows analytically from the proper recognition that property's social function should be considered relevant when adjudicating on the rights and obligations attached to property. Indeed, it is Dagan's clearly stated aim to propose a theory that promotes specific liberal values. ``There is room to allow for the virtue of social responsibility and solidarity'', he writes, continuing by suggesting that ``those who endorse these values should seek to incorporate them -- alongside and in perpetual tension with the value of individual liberty -- into our conception of private property''. This view is reflected further in the concrete policy recommendations he makes, for instance in relation to the question of when it is appropriate to award less than ``full'' (market value) compensation for property following a taking.

My objection is not that his proposals are necessarily wrong, but that they need not be accepted in order to conclude that the social function of property should be given a more prominent place in property theory. Importantly, I think the focus on normative reasons threatens to overshadow the most straightforward reason for awarding social structures a more prominent place in the analysis, namely that they are almost always crucially important behind the scenes, even if they are unacknowledged. The social function theory, rather than being ``good, period'', as Danach suggests, is nothing more or less than accurate, irrespectively of one's ethical or political inclinations. As such, it provides the foundation for a debate where different values and norms can be presented in a way that is conducive to meaningful debate, on the basis of a minimal number of hidden assumptions and implied commitments. Thus, the first reason to accept the social function theory, for me, is epistemic rather than deontic.

That is not to say that normative theories should not be formulated on the basis of the social function theory, it merely means that I believe it is useful to maintain at least a theoretical division between the descriptive and normative aspects of such theorizing. I return to normative aspects in the next section, arguing that the commitment to ``human flourishing'' endorsed by Professor Alexander is a particularly well-argued norm that arises from value-based assessment of the social function of property. This, I argue, is in large part also due to the value-pluralism inherent in this idea, suggesting as a positive normative claim that our notions of property {\it should} allow a divergence of opinions and values to influence the law and its application in this area.

However, I believe the history of the social function theory lends support to my claim that it is useful to emphasize that the theory gives us important descriptive insights that carry no normative commitments. This is particularly important, I believe, in a time when property scholars are showing greater willingness to explore new theoretical ground. Theories can hardly be entirely value-neutral, nor is this a goal in itself. But in my opinion, a good theory is one that can serve as a common ground for further discussion based on disagreement about values and priorities. According to Kevin Gray, ``the stuff of modern property theory involves a consonance of entitlement, obligation and mutual respect''.\footcite[37]{gray11} It is important, I think, that the same measured perspective is reflexively applied towards theory itself, to diminish the worry that a broader theoretical outlook is the first step towards unchecked state power and rule by ``judicial philosopher-kings''.\footcite[944]{claeys09}

In the next subsection, I will argue in some more detail why such a cautious perspective is warranted, by considering how the Italian fascists appropriated the social function theory in 1930s. Building on the highly inspiring work of di Robilant, I will also briefly track how non-fascist property scholars opposed this development by focusing on value-pluralism, local self-governance and freedom.\footcite{robilant13} Importantly, these scholars embraced the social function theory as a common ground from which to launch a meaningful attack on more radical ideas, without alienating those with divergent views. Instead of clinging to the old-style liberal discourse that the fascists had either flatly rejected or completely subverted, many Italian non-fascists were willing to engage in a discourse revolving around property's social function, by spelling out a more measured set of ideas based on this premise. Crucially, this set the stage for a form of intellectual resistance that did not reject those aspects of fascism that had great appeal to the public and which arguably also reflected true insights into the unfairness and lack of sustainability of the established legal order.

\subsection{Fascist appropriation and resistance: {T}he tree of property}

While the social function theory makes intuitive sense, it is also highly abstract. Therefore, its exact content has been notoriously hard to pin down. This is recognized by contemporary scholars endorsing a normative view, who attempt to address this by proposing lists of values that should be taken into account while giving examples of how they should be used to inform the law in concrete areas or cases.\footnote{See, e.g., \cite{alexander14,alexander11,dagan07}.} Unsurprisingly, however, views soon diverge regarding the concrete import of a social function view on legal reasoning. Even so, at least the contemporary debate appears to be based on a common ground that is quite stable, also with respect to the overall notion of what good the theory can do. But as history shows, this state of affairs is by no means guaranteed. 

In a recent article, Professor di Robilant illustrates this point exceptionally well by tracking the history of social function theorizing in Italy during the fascist era. The fascist property scholars, she notes, were happy to embrace the social function theory, since it provided them with a conceptual starting point from which to develop their idea that rights and obligations in property should be made to answer to one core value: the interests of the state.\footnote{See \cite[908-909]{robilant13} (``Fascist property scholars had also appropriated the social function formula. For the Fascists, the social function of property meant the superior interest of the Fascist state.'').} This stance was as effective as it was oversimplified. As de Robilant notes, ``earlier writers had been hopelessly evasive about the meaning and content of the social element of property''. Hence, the fascist approach filled a need for clarity about the implications of the main idea, which was by now attracting increasing support both from the public and the academic community. Established property doctrine, it was widely felt, was both ineffective and unfair to ordinary people. Rather than securing productivity and a livelihood for all, property was used mainly as an instrument for maintaining the privileged position of the elites. By promising to change this state of affairs, the fascists attracted many to their cause.

As De Robilant argues, supporters of the fascist idea of property made clear that ``social function meant the productive needs of the Fascist nation''. But at the same time, they cleverly denied that there was a ``contradiction between subordinating individual property rights to the larger interest of the Fascist state and the liberal language of autonomy, personhood, and labor''. In this way, fascist scholars could claim that fascist liberalism was true liberalism, thereby subverting the conceptual basis for the traditional idea of liberal justice. In this situation, there was reason to suspect that clinging to liberal dogma would be a largely ineffective response. Moreover, it seemed undeniable that the fascism's appeal was rooted in real concerns about the fairness and effectiveness of the liberal legal order. 

Hence, many non-fascists shunned away from uncritical defense of traditional liberalism. Instead, they agreed that property's social function should come into focus, but focused on the plurality of values that could potentially inform this function, not the interests of the state. In addition, they also noted that property rights were invariably associated with {\it control} over resources, and that the social functions of property depended on the resources in question. To own property, they argued, provides individuals with a source of privacy, power and freedom that is, as a matter of fact, highly valued. It is valued, moreover, for its implications in a social context. Italian scholars adopted the metaphor of a ``tree'' , describing the core social function of property as the trunk, while referring to the various resource-specific values attached to property as branches. As di Robilant notes regarding these theorists:

\begin{quote}
The rise of Fascism, they realized, was the
consequence of the crisis of liberalism. It was the consequence of liberals' insensibility to new ideas about the proper balance between individual rights and the interest of the collectivity.\footcite[907]{robilant13}
\end{quote}

In light of this, the tree-theorists concluded that continued insistence on the protection of the autonomy of owners was not a viable response. Instead, they adopted a theory that `` acknowledges and foregrounds the social dimension of property'', but without committing themselves to fascist ideas about the supreme moral authority of the state. The value of autonomy was in turn recast in terms of property's social function. Arguably, this served to make the case far more compelling. Protecting autonomy could be seen as an aspect of protecting property's freedom-enhancing function, both at the individual level and as a way of ensuring a right to self-governance and sustenance for families and local communities. This, moreover, could not easily be derided as tantamount to protecting unfair privilege and entitlement. In fact, the suggestion was made in an effort to protect democracy itself.

I believe the story of fascist appropriation of the social function theory provides further weight to my claim that it is sensible to  maintain a descriptive perspective on its core features. Indeed, the readiness with which the fascists embraced social function theorizing serves as a reminder that we cannot easily predict what normative values may come to be promoted on its basis. Hence, it is also call for continuous vigilance when it comes to normative assessment and debate. At the same time, we are reminded of the danger of attaching too much normative prestige to a theory that is abstract and open to various interpretations.

In particular, it seems to me that failure to recognize the descriptive nature of the core idea can lead to unrealistic expectations of what the social function theory actually provides. In addition, it will make it harder for the theory to gain acceptance as a conceptual common ground from which to depart when engaging in debate. Indeed, if no division is recognized between normative and descriptive aspects, the historical record would allow detractors to make a {\it prima facie} plausible attack on the social function theory by arguing that it is fascism in disguise, or that fascism, rather than liberal justice, is where we end up in practice should we chose to adopt it.\footnote{This would echo the claim already made by Claeys, that the theory (when coupled with virtue ethics) might be a slippery slope towards the kind of extremism and revolt against oppression that gave rise to the Rwanda genocide in the early 90ties \cite{.....}.}

In response, one might retort that this is cherry picking the historical facts, or that the fascists misunderstood or perverted the theory. That is certainly plausible, but the point I am trying to make here is that this kind of debate is in itself unhelpful. Unless the social function theory is rendered neutral enough to be acceptable as the conceptual premise of debate, it is likely going to fail -- in a purely epistemic sense -- as a template for negotiating conflicts about property. Those who oppose the norms associated with the theory will oppose also the core descriptive content, if they feel that the latter commits them to the former. I believe that this, in turn, suggests that those advocating on behalf of the social function theory should take care to avoid rhetorical hubris. The main point to convey, I believe, is that the theory is in fact more accurate, in a purely epistemic sense, than other conceptualizations of property.

The story of the fascist appropriation of the social function theory also points to the danger that often attach to abstract theories with normative implications: That they allow us to recast whatever values we wish to promote, providing qualifications for them in abstract terms that are hard to refute. The fascists did this, and the non-fascists responded. Hence, in the end one could do little more than hope that the fascists' vision of their state as an ``ethical state'' that ``every man holds in his heart'' would eventually prove less attractive then the promise of self-governing communities bustling with diversity in life and character.

\subsection{Towards a normative stance}

The social function theory can facilitate a new kind of normative reasoning, arising from how the theory allows us to recognize more subtle distinctions between different kinds of property and different kinds of circumstances. A staunch entitlements-based approach to autonomy will leave us with little room to differentiate between the protection of investment property and the protection of a home, for instance, unless such a distinction is explicitly provided for in the law. But a social function approach compels us to notice the difference and to acknowledge that it might be legally, as well as ethically, relevant. Hence, if we seek to argue for protection of investment property, we must in principle be prepared to face counter-arguments that revolve around particulars of the investor's role in society and his relationship to the community of people that are affected by how he manages his property. Similarly, if someone argues against protecting home ownership, we can respond by drawing on additional arguments based on the importance of the home both to the owner, her family and friends, and her community. Under the social function theory, it becomes generally relevant to address how a home creates a sense of belonging and provides a basis for membership in social structures.

I believe normative assessments should aim to be as concrete as possible. That said, I still think it is worthwhile to provide more abstract forms of expression for core values, to clarify the ethical premises that provide the basis for concrete value-based conclusions. To me, therefore, normative theories should aim to be meta-ethical, not just ethical. They should provide a vocabulary and a conceptual framework tailored to advancing one's values. However, they should recognize that the ultimate expression of those values is given in relation to concrete facts. This, I believe, is prudent in light of how abstract ethical assertions are necessarily imprecise, and run the risk of being distorted or exaggerated as they gain influence in society.

Invariably, the most accurate information regarding the values I rely on when assessing cases will be conveyed by my assessment of the cases, not by by theorizing. On a deeper level, I am inclined to believe that value-systems are unique to individuals, so that ethical theories are helpful primarily in that they provide an introduction to keywords and important lines of argument that will recur in different forms. As such, they enhance understanding, making it easier to communicate ideas and opinions in such a way that potential respondents are likely to enjoy a somewhat less inaccurate impression of what they are responding to. 

In short, I believe that ethics make moral judgments communicable, allowing new ideas to be created in the minds of individuals. It should come as no surprise to the reader, therefore, that I believe in ethical men and women, but not in ``ethical Man'' or -- God forbid --  the ``ethical State''. Luckily, I find some support for this world view in recent theories that have been proposed as normative extensions of the social function theory of property. These are the subject of my next section.

\section{Human flourishing}

Taking the social function theory seriously forces us to recognize that a person's relation to property can partly be constitutive of that person's social and personal identity, including both its political and economic components. Hence, property influences what the owner's preferences are as well as what paths lie open to her when she considers her life choices. This effect is not limited to the owner, however, it comes into play for anyone who is socially connected to property in some way. The life-significance of property might be clearly felt by a potentially large group of non-owners as well. Its importance abates it we move away from the owner and his property in terms of social or economic distance, but as sociologists take pride in pointing out, social connections are ubiquitous  and the world is often smaller than it seems. 

Hence, there is certainly potential for making wide-reaching normative claims on the basis of this perspective on the meaning and content of property. But which such claims {\it should} we be making? According to some, we should adjust our moral compass by looking to the overriding norm of {\it human flourishing} as the fundamental guiding principle of property law. Professor Crawford explicitly argues that the social function theory of property should ``secure the goal of human flourishing for all citizens within any state''.\footcite[1089]{crawford12} In a recent article, Professor Alexander goes even further, declaring that human flourishing is the ``moral foundation of private property''.\footcite[1261]{alexander14} 

As I have already explained, I believe -- in contrast to both Crawford and Alexander -- that it is useful to decouple such normative claims from the descriptive core of the social function theory.\footnote{Crawford comments that the social function theory on its own  ``is not self-defining and invites many interpretations'', see \cite[1089]{crawford12}. The normative theory he proposes is clearly aimed at filing this perceived gap, by pinning down normative commitments that Crawford believes are intrinsic to the theory. However, as I have already argued, I reject this approach, since it unwisely downplays the fact that the social function theory can serve as a common ground among commentators with widely divergent normative views. Indeed, Crawford himself refers unfavorably to a writer who addresses the social function theory, but who, according to Crawford, proposes that ``property's social function is best served by focusing on overall economic production and efficiency in a given society, allowing the market's invisible hand to work its magic'', \cite[see][1089]{crawford12}. Against Crawford, I would argue that it is better to counter such a claim by arguing why it is normatively wrong than by suggesting that people with such values should be discouraged from attempting to argue for them on the basis of a social function understanding of property. Rather, by encouraging such an argument it should become easier to make the case why the values promoted are ultimately undesirable. This, at least, should follow if Crawford is otherwise largely correct (as I think he might be).} I therefore follow Alexander in referring to the more distinctly normative aspects of their work as human flourishing theorizing. 

Human flourishing has a good ring to it, but what does it mean? According to Professor Alexander, several values are implicated, both public and private. Importantly, Alexander stresses that human flourishing is ``value-pluralistic''. There is not one core value that always guarantees a rewarding life. To flourish means to negotiate a range of different impulses, both internal and external. Importantly, these all act in a social context which influences their meaning and impact.

For the family of a homeowner, the value of the ownership tends to be great; a home is a home for any non-owner living there, just as much as it is a home for the owner. This, in turn, creates both commitments and opportunities for the owner, which may or may not find recognition in the law and our legal reasoning. Regardless of this, it certainly carries significant importance both to her life and those that depend on her. If the property is rented out as a home to someone else, the importance of ownership may be {\it greater} to a non-owner. Indeed, assuming a society where tenancy is a well-functioning social institution, the continuation of the established property pattern might well be of greater importance to the tenant than it is to the owner.

The effect on non-owners can also be restrictive in socially valuable ways. If an apartment has an owner, it discourages squatters, for instance. Moreover, this effect clearly depends also on {\it who} the owner is and the choices she makes in managing her property. If the owner lives in the apartment, squatting is of course not likely to be a problem. But even the owner of an unoccupied apartment can discourage squatting by managing her property well. However, if owners mismanage their apartments, for instance because they seek to obtain demolition licenses, squatters can take opportunity of this. The risk, of course, increases if housing cannot be afforded by a large number of society's members. In this case, it is natural to argue that something is amiss with the prevailing property structures.

Now, the social function theory of property can also come into play, since it allows us to attach significance to this also when discussing the property rights of individual owners.  In particular, we are not compelled to pretend as though possible failures of property as a social institution are irrelevant when considering rights and responsibilities attached to it. As a matter of fact, they are not; actual squatters clearly affect the owner, influencing both the meaning and the value of her property, both to her, potential buyers, the local government, the state, and other interested parties. Even the mere {\it risk} of squatting can play such a role. But a property theory which does not recognize the social function of property might not allow us to recognize this when discussing the rights and responsibilities of the individual. As long as the standard expectation of an owner is to be able to enjoy her apartment free of squatters, an entitlements-based view on property could easily force us to into denial regarding any actual (risk of) squatters.

In particular, we would be led to consider squatting as an interference with the owner's rights which the state can not, on pain of disrespecting property, recognize as a legitimate response to mismanagement and imbalances in the property structure. The normative significance of real life -- where squatting often happens due to badly managed property -- is discounted  because our conceptual glasses block it out. Then, the almost unavoidable consequence is that the state also recognizes a {\it positive} obligation to forbid squatting, and to forcibly remove squatters on behalf of owners. Under the classical entitlements-based conception, this is the natural outcome, and must be classified as an act of protecting private property. Hence, under classical liberal values, it also becomes {\it good}. Here, however, the social function theory permits us to take a highly divergent view, to carry forward different value-judgments.

In particular, if squatting is recognized as creating new interests and obligations attaching to the property, it may now be argued that  it is the use of state power to evict that is the most severe act of interference. Not only interference in whatever housing rights the squatters may have, but in fact also as an interference in {\it property}. Hence, such state action might itself be morally suspect and held to be in need of further justification. In the Netherlands, the Supreme Court adopted a line reasoning reflecting these insights, when it held that the right not to be disturbed in one's home life also applied to squatters. Hence, the property owner could not forcibly evict people who had taken up residence in her property.\footnote{See NJ 1971/38. The court held that the lower court had erred in taking it proven that the ``house in the original charge was ``in use'' by the owner of this house'', as required by the statute under which the squatters were tried. Instead, the Supreme Court held that ``art.138, in so far as it mentions houses, is specifically aimed at protecting home rights, in connection with which the words ``in use'' (differently than the court judged) can only be understood as ``actually in use as a house'' , as in accordance with ordinary use of language''. The upshot was that it was the squatters, not the owner, who enjoyed protection under the statute. In terms of the bundle theory, a right thought to be in the owner's bundle was deemed to actually belong to the bundle of the squatters, as this corresponded better to the circumstances of the case and the purposes meant to be served by the statute in question.}

In South Africa, a somewhat similar line of reasoning was adopted in the recent case of {\it Modderklip}, analysed in depth by Alexander.\footcite{....} The case dealt with squatting on a massive scale: Some 400 people had taken up residence on land owned by Modderklip Farm, apparently under the belief that it belonged to the city of Johannesburg. The owner attempted to have them evicted and obtained an eviction order, but the local authorities refused to implement it. Eventually, the settlement grew to 40 000 people and Modderklip Farm complained that its constitutional property rights had not been respected.

The Supreme Court of Appeal concluded that Modderklip's property rights had indeed been violated, but noted that so had the rights of the squatters, since the State had failed to provide them with adequate housing. However, they upheld the eviction order and granted Modderklip Farm compensation for the State's failure to implement it. The Constitutional Court, on the other hand, while agreeing that the eviction order was valid, concluded that as long as the state failed in its obligations towards the squatters, the order should not be implemented. The eviction of the squatters, in particular, was made contingent upon an adequate plan for relocation. In the meantime, Modderklip would receive monetary compensation from the State. In this way, the Court recognized the social function of property; they refused to give full effect to Modderklip's property rights as long as that meant putting other rights in jeopardy. The fact that the squatters had no place to go, in particular, was allowed to influence the content of Modderklip's right, making it impermissible to implement a standing eviction order. 

It is possible to cast the result as an interference in property rights that is nevertheless acceptable in the public interest. However, the reconceptualization in terms of property itself having a social function appears highly attractive. Moreover, it is also consistent with the South African constitution, which explicitly mentions property's social dimension. As argued by Professors Gregory Alexander and ...., by thinking about cases such as these in terms of property law allows us to remove the state as an intermediary between the owner and the other interested parties, in this case the squatters. In particular, it becomes possible to think of the Court as adjudging based on Modderklip's own responsibility, as an owner, towards other members of the community that have an interest in his property. 

On this basis, it becomes easier to conclude that it was permissible for Courts to take the social context into account even in the absence of any State action or legislation to indicate that this should be done, or that the public interest was at stake. Indeed, one of the problems in Modderklip was that the State had failed also in its responsibility towards the squatters. Moreover, while the local sheriff had refused to implement it, an eviction order had in fact been granted. Hence, thinking of the case as one where interference in the public interest was sanctioned by the Courts becomes strenuous at best.

More importantly, by taking into account the social function of property, it becomes possible to argue for the outcome in Modderklip positively on the basis of property values. In this way, property is no longer seen to stand in the way of justice in cases such as this. We need not ``interfere'' with rights to secure an appropriate outcome, we only need to apply property law. As Professor Alexander puts it in another recent article: 

\begin{quote} The values that are
part of property's public dimension in many instances are necessary
to support, facilitate, and enable property's private ends.
Hence, any account of public and private values that depicts them as categorically
separate is grossly misleading. One important consequence of this
insight is that many legal disputes that appear to pose a conflict between
the private and public spheres or that seemingly
require the involvement of public law can and
should, in fact, be resolved on the basis of private law -- the law
of property alone.\footcite[1295-1296]{alexander14} \end{quote}

Protection of property, when property is understood in this way, becomes a potential source of justice, also for squatters. The basic values attached to property -- freedom, liberty, autonomy -- have not really changed, but we have widened their scope. They no longer only apply to the owner's interest in property, but also to that of other individuals closely connected to it. This normative turn, I argue, will potentially strengthen the institution of property itself, while also decreasing the compulsiveness of the idea that the ultimate expression of the public interest is found in the actions taken by the state. It suggests rather the view that the public interest manifests wherever the public may reside, including in property. This conclusion requires taking a normative stance, but a minimal one; we merely extend the scope of values traditionally attached to property.\footnote{Arguably, cases such as {\it Modderklip} might be taken to suggest that the social function theory, as soon as it is applied for the purposes of normative assessment, will systematically guide us to conclude that owners are not entitled to as many benefits as would otherwise follow from their property rights. It is fortunate, therefore, that the entire remainder of the thesis will focus on economic development takings, where it will typically appear more natural to conclude the opposite. In these cases, on a common- sense understanding of justice, applying the social function theory will allow us to recognize a sense in which owners should receive {\it increased} protection and more benefits, as a consequence of how such interferences can prove particularly damaging, both to the owner and to the social fabric of democracy.} 

That said, in the case of {\it Modderklip} the court was clearly faced with a value conflict that it is hard to resolve by looking to traditional liberal values. If these apply equally to the squatters, we are left with deadlock rather than resolution. Indeed, this was also reflected in the outcome of the case, which did not resolve the matter, but merely concluded that the state had failed in its obligations towards both of the parties. What should the solution be in the end? Should the squatters be allowed to stay, following condemnation of Modderklip's land, or should alternative housing be provided so that the eviction order could be carried out? The answer requires us to resolve a normative conflict, and how to do so may not be obvious. Moreover, value-pluralism suggests that we must be prepared to engage with multiple ways of looking at the matter. In the interest of stability of property as an institution, allowing the squatters to succeed in establishing lasting title to the land might be considered unwise. Against this pragmatic and largely technical value, one would have to consider the values of community and belonging that now attach the squatters to their new homes. These two values are largely incommensurable, and it is not clear how to choose between them.

Still, Alexander maintains that human flourishing provides an ``objective'' standard on which to approach dilemmas such as these. Moreover, he ``rejects the view that what is good or valuable for a person is determined entirely by that person's own evaluation of the matter''.\footcite[1263]{alexander14} Some things are good for people, Alexander argues, irrespectively of whether or not people know so themselves. Hence, it may perhaps be argued that what is truly good for Modderklip is to come to an arrangement with the squatters and the state, to resolve the problem amicably. Moreover, failure to do so may entitle the state to take action that would otherwise seem to undermine the stability of property. This, then, would be partly due to this being conducive also to the flourishing of the people behind Modderklip, not only the squatters.

That, clearly, might be derided as an overly intrusive and moralistic way to approach property law. More generally, as Alexander notes, the exact content of goodness is ``necessarily contestable''. It consists of a list of different values which are all open to dispute, both as to their relevance and their precise meaning.\footcite[1263]{alexander14} Alexander goes on to list some key values that he believes are central, but the list is not meant to be exhaustive. 

Among the key values that Alexander discusses, we find many core private values that are commonly seen as important goals for the institution of property. This includes values such as autonomy and self-determination, both of which will feature heavily later in this thesis. However, Alexander also considers several public values, such as equality, inclusiveness and community. These too will be important later, as I will draw on them in my own normative analysis of economic development takings. I will be particularly concerned with the value of {\it participation}, understood, following Alexander, in terms of its broad social function.

In my view, this value is closely related to the value of democracy. Participation in decision-making processes locally, I argue, is the root which enables democracy to come to fruition at the regional and national level. Moreover, participation is a value that will give me occasion to make particular policy suggestions regarding the correct way to approach the issues addressed later in the thesis. Devoting some time to discussing this value in the abstract will therefore be helpful.

Alexander traces the value of participation back to Aristotle and the republican tradition. He notes, however, that this tradition involves a notion of participation that is somewhat narrowly drawn. For thinkers in the republican tradition, participation tends to mean public participation, referring to public engagement with the formal affairs of the polity.\footcite[1275]{alexander14} For Alexander, participation means something more, involving also the value of being part of a community. He writes:

\begin{quote}
We can understand participation more broadly as an aspect of inclusion. In this sense participation means belonging or membership, in a robust respect. Whether or not one actively participates in the formal affairs of the polity, one nevertheless participates in the life of the community if one experiences a sense of belonging as a member of that community.\footcite[1275]{alexander14}
\end{quote}

Importantly, participation in a community can have a crucial influence also on people's preferences and desires. In this way, it will in fact be highly relevant -- behind the scenes -- to any assessment of property that focuses on welfare, utility or public participation in the classical sense. As Alexander puts it, drawing on the work of Sen and others:
\begin{quote}
The communities in which we find ourselves play crucial roles in the formation of our preferences, the extent of our expectations and the scope of our aspirations.\footcite[140]{alexander09}
\end{quote}
Therefore, for anyone adhering to welfarism, rational choice theory, utilitarianism or the like, neglecting the importance of community is not only normatively undesirable, it is also unjustified in an epistemic sense. In particular, it should be recognized as a descriptive fact that community is highly relevant to {\it any} normative theory that attempts to take into account the preferences and desires of individuals.\footnote{Again, I think Alexander and other theorists attempting to incorporate such ideas in property law could benefit from making this descriptive point separately, so as to enable it to be considered in isolation from the more contentious normative arguments they construct on the basis of it.} But Alexander goes further, by arguing that participation in a community should also be seen as an independent, irreducibly social, value, not merely as a determinant of individual preferences and a precondition for rational choice. He writes:

\begin{quote}
Beyond nurturing the individual capabilities necessary for flourishing, communities of all varieties serve another, equally important function. Community is necessary to create and foster a certain sort of society, one that is characterized above all by just social relations within it. By ``just social relations'', we mean a society in which individuals can interact with each other in a manner consistent with norms of equality, dignity, respect, and justice as well as freedom and autonomy. Communities foster just relations with societies by shaping social norms, not simply individual interests.
\end{quote}

This, I think, is a crucial aspect of participation. Moreover, it is one that it is hard, if at all possible, to incorporate in theories that take  preferences and other attributes of individuals as the basis upon which to reason about property. For instance, if people in a community comes under pressure to sell their homes to a large commercial company that wishes to raze them in order to construct a shopping mall, it may be appropriate to consider this as an unjustifiable attack on their property rights. Importantly, this may be so {\it irrespectively} of what the individual owners themselves think they should do. If they are offered generous financial compensation for their home, or are threatened by the specter of eminent domain, economic incentives might trump the value of social inclusion and participation for all or a majority of these owners. As a consequence, the community might decide to sell.  

Even so, in light of the value of community, it would be in order for planning authorities, maybe even the judiciary, to view such an  agreement as an {\it attack on their property}. It is clear, in particular, that by the sale of the land, the ``just social relations'' inhering in the community will be destroyed. The members of the community -- including all the non-owners -- will lose their ability to participate in those relations. More concretely, the nature of the property rights that once contributed to sustaining ``just relations'' will now be transformed into property rights that serve different purposes. This includes aiding the concentration of power and wealth in the hands of commercially powerful actors. Such a change in the social function of property might have to be regarded -- objectively speaking -- as a threat to participation, community and democracy. Hence, on the human flourishing theory, it is also a threat to property. Our property institutions, therefore, should protect against it.

To demonstrate the general significance of such a line of normative reasoning, it is illustrative to mention a scenario -- not directly implicating property -- that is currently beginning to attract much attention in legal scholarship. This scenario arises in relation to the right to {\it privacy}. This right, of course, is increasingly perceived to be coming under threat in the information age. Crucially, it is beginning to become clear to legal theorists that viewing privacy merely as a private right is not going to provide a sustainable template for dealing with this challenge. It seems, in particular, that people are simply too willing to give it up. This, in turn, contributes to the formation of potentially harmful social structures on the web. In particular, the lack of privacy becomes an impediment to dignity, freedom and respect in web societies. In this way, both individuals and society as a whole will eventually suffer, although this truth is not reflected in our individual preferences. Hence, it has been proposed that privacy should be considered also as a {\it common good}, so that protecting the privacy of individuals, in some cases, is an imperative irrespectively of what these individuals themselves desire and prefer. Privacy, in this way, becomes also an obligation, mirroring the similar phenomenon that we have observed with respect to the right to property.

There is a subtle issue that arises on the basis of thins kind of normative reasoning about rights such as property and privacy. Is it appropriate, in particular, to still think of such reasoning -- and the obligations they give rise to -- as an aspect of protecting individuals? Is it not more accurate to say that this is an {\it interference} with individual rights, undertaken to further the public interest? Indeed, when the individual himself does not want his property or privacy to be ``protected'', is it not somewhat perverse to insists that this is what is happening? 

I am inclined to answer in the negative. In my opinion, we are still talking about protecting individual rights, even when this means imposing protections on people that they themselves do not want. Undoubtedly, this is {\it also} an interference in their rights, but just as different rights of different people can sometimes come into conflict, the same right, for the same person, can come into conflict with itself. This happens, in particular, when it is not possible to achieve all those goals that this right seeks to promote. For instance, if someone protests a taking on environmental grounds and also rejects financial compensation as immoral, the courts should still award just compensation for the land, if they find that the taking is valid. If the owner wishes, he can purge himself by making a donation to charity. Similarly, if someone attempts to commit suicide, the health services are still obliged to help, even against the patients wishes, and even now that suicide is no longer considered a criminal offense in the public interest. 

Protecting individuals against their will is condescending, no doubt, but it is still different, and often preferable, from subordinating their interests to that of the general public. If the justification for an act of interference is a vague proclamation of the ``public interest'', the individual is marginalized from the very start. A balancing act might be required, but this renders the individual relevant only to one side of the equation. On the other hand, if the act of interference is simultaneously rendered as protection, enforcement of an obligation, or a measure to enable participation, the individual occupies center stage. In so far as the public interests triumphs, it is not because the individual loses, but because the public is deemed to know best how to secure the goal of human flourishing, both for the individual herself and other members of the social structures that surrounds her.

For instance, external interests of both a private and a public nature can dictate that owners should avoid becoming a nuisance to their neighbors. But under a human flourishing theory, we are also able to portray this as a case of protecting the individual's membership in the community. The public does not ``side with the neighbors'', but undertakes measures to protect the relationship between the owner and his fellows. In my opinion, a conceptual approach to property law that makes this portrayal plausible is highly desirable. A second example can be environmental concerns that suggests imposing restrictions on what an owner is permitted to do with his land. This too can be rendered as an act of protecting property. But doing so requires the regulatory body to relate the interference positively to the individual's interests and obligations, to ensure that they avoid adopting a narrative where the regulation is rendered as an act of enforcing the will of unnamed others against the will of specific owners. In this way, public values and the public interest can be given considerable weight, but will have to be rendered less abstract. In particular, these interests must be related concretely to the social functions of the rights protecting the individuals interfered with. The baseline for assessment remains actual persons and their well-being, not some abstract ideal of ``goodness''. Moreover, implementation of the collective will becomes a guide towards human flourishing for individuals, not a goal in itself.

An individual might well be offended if the state adopts this narrative and implements behavioral restrictions by declaring ``it's for your own good''. But, I argue, that is exactly as it should be. Any restriction of individual freedom is an offense, but one that is sometimes appropriate. If this is conveyed to people with a marginalizing ``your interests are not as important as ours'', the response might well be silence. But beneath the silence we may find disinterested apathy, or worse, contempt and despair. The interference is no longer an insult, not because it is any more likely to succeed in convincing the individual that interference is ``necessary for the greater good'', but because it fails to properly engage the individual at all. The role of the person interfered with becomes passive -- she becomes an obstacle that needs to be removed. If such a dynamic of governance develops, the individual might take from this the lesson that she is unimportant in the greater scheme of things, that her interests are subordinate to those of ``the others'', and that her voice is not meant to be heard.

This is normatively undesirable. It represents a situation when the social effect of interference may become detrimental to society, particularly to the institution of democracy. It damages its roots, namely the ``just social structures'' that Alexander identifies as being at the core of the human flourishing theory. A better alternative, then, is to interfere in a way that constructively targets the individual, aiming to protect her by enabling her -- and compelling her -- to protect others and partake in social and political life. This can then become interference aimed at bringing the individual into the fold, making her play her part, raising her to fruitful citizenship. Such a paternal (or maternal) state is one that cares, but one that may also be overprotective, unfair, or plain stupid. Hence, it becomes natural to resist and to revolt, but not without also carrying forward care and love for the social, political and legal structures within which this agency is (hopefully) permitted to take place.

The upshot, I believe, is that condescension can be a good thing in this area of the law. It may be offensive, but it renders interference more meaningful to the individual. It provides both a reason to take a more active stance towards the interfering power, and a possible cause for constructive resistance. Importantly, it does not force the conclusion that the public resides behind closed doors, disinterested in what the individual has to offer. Instead, it is an approach that encourages a response, by focusing always on the person interfered with, whenever interference is deemed necessary. This is the vision of a bottom-up, rather than a top-down, approach to imposing the collective will on individuals. I believe it has merit.

In the next section, I will return to the issue that will remain in focus for the remainder of this thesis. First, I will introduce economic development takings by considering the seminal case of {\it Kelo}, which brought this category to prominence in the US discourse on property law. Then I will assess the unique aspects of such takings against the social function theory, to provide an argument that the category has significance for legal reasoning in takings law, as well as with respect to property as a constitutionally protected human right. Finally, I will provide an abstract presentation of the values that I believe should be considered important when normatively assessing the law in this area. In doing so, I will draw on the human flourishing theory, setting out the main values that will inform the concrete policy assessments I provide later. 

\section{Economic development takings}

Constitutional property rules in many jurisdictions indicate, with varying degrees of clarity, that eminent domain should only be used to take property either for ``public use', in the ``public interest'', or for a ``public purpose''. Such a restriction can be regarded as an unwritten rule of constitutional law, as in the UK, or it can be explicitly stated, as in the basic law of Germany. In some jurisdictions, for instance in the US and in Norway, explicit property clauses exist, but are not formulated clearly.

Both the Norwegian and the US property clauses appear to refer to public use only as a precondition for the duty to pay compensation. However, most scholars agree that this must at least be read as expressing a {\it presupposition} that the power of eminent domain is only to be used used in the public interest. Indeed, in cases when one might say that private property is ``taken'' for a non-public use without compensation, for instance in a divorce settlement, it is not commonly regarded as an exercise of eminent domain. Rather, it is justified by making reference to a different category of rules, meant to ensure enforcement of obligations that already exist between private parties.

This much appears to be agreed upon by most legal scholars. However, differences of opinion soon emerge when we turn to the question of whether the presupposed public use of the property that is taken serves also to restrict the power of eminent domain. Most scholars agree that some restriction is intended, but in many jurisdictions, there is great disagreement about its extent. In Norway, for instance, a consensus has developed that the public use limitation is mainly of theoretical interest.\footnote{...} While the Norwegian Constitution speaks of ``statens tarv'', meaning roughly ``the needs of the state'', this is understood very loosely as public interest or public purpose. Moreover, the courts defer almost completely to the assessments made by the executive branch regarding the purposes that may be used to justify a taking. 

In the US, on the other hand, the understanding of the public use restriction continues to be a much debate and highly controversial issue in constitutional property law. Some scholars adopt a stance similar to that taken in Norway, while others argue that the public use presupposition should be read as a strict requirement, forbidding the use of eminent domain unless the public will make actual  use of the property that is taken. Most scholars fall in between these two extremes. They regard the public use restriction as an important, and practically relevant, limitation, but they also emphasize that courts should normally defer to the legislature's assessment of what counts as a public use.\footnote{The fact that US jurists usually stress deference to the legislature, not the executive branch, should be noted as a further contrast with Norway.}

As I discuss in more depth in Chapter \ref{chap:...} Section \ref{sec:...}, the debate in the US has its roots in case law developed by state courts, as the federal property clause was for a long time not enforced against states. This has changed, however, and today the Supreme Court has a leading role also in this area of US law. It has developed a largely deferential doctrine, resembling somewhat the understanding of the public use limitation under Norwegian law. The difference is that in the US, cases raising the issue  still regularly arise, and still prove controversial. The most important such case in recent times is {\it Kelo v City of New London}, decided by the Supreme Court in 2005. This case saw the public use question reach new heights of controversy in the US.

{\it Kelo} centered around the legitimacy of taking property to implement a redevelopment plan that involved construction of research facilities for the drug company Pfizer. The home of Suzanne Kelo stood in the way of this plan, and the city decided to use to power of eminent domain to condemn it. Kelo protested, arguing that making room for a private research facility was not a permissible  `public use''. She was represented by the libertarian legal firm {\it Institute for Justice}, which had previously succeeded in overturning similar instances of eminent domain at the state level. Kelo lost the case before the state courts, but the Supreme Court decided to take it on, and they looked at it in great detail. It was the first case of its kind since 1984. 

The precedent set by earlier cases was clear: Unless the decision to condemn was ``....'', the court should not second-guess the legislature's determination that the taking was for a public use. Moreover, in the case of {\it Hawaii}, the Supreme Court had upheld a taking that would benefit private parties, with no direct benefit to the public. In {\it Berman}, it had upheld a taking for economic redevelopment of a bligthed area, even though the property taken was not itself blighted. But in the case of {\it Kelo}, the court hesitated.  Eventually, in a 5-4 vote, it decided against Suzanne Kelo, holding that economic development takings were indeed permitted under the public use restriction, also when the public benefit was indirect and a private company would benefit commercially. 

The decision was in keeping with precedent, but it caused a severe public outcry. According Ilya Somin, the case ranks among the most disliked decision that the Court has ever made. Some .... \% of the US public expressed great disapproval, with critical voices coming from across the political spectrum. Why did the case prove so controversial? Somin argues that the discontent with the decision  was fueled by the fact that it was seen as a case of the Court siding with the rich and powerful, against ordinary people. Indeed, the party that appeared to benefit the most from the taking was Pfizer -- a multi-billion dollar company -- while Suzanne Kelo, who stood to lose, was a middle class homeowner. Hence, the adversaries were not exactly evenly matched, particularly not as the City of New London sided with Pfizer, by making its power of eminent domain available on its behalf. 

In addition, it is worth noting that many commentators conceptualized the {\it Kelo} case by thinking of it as belonging to a special category, by describing it as an economic development taking, a {\it taking for profit}, or, more bluntly, a case of {\it Robin Hood in reverse}. Categories such as these had no clear basis in the property discourse before {\it Kelo}. Indeed, in terms of established legal doctrine, it would be more appropriate to say that the case revolved entirely around the notion of ``public use''. But when we consider the most common reasons given for condemning the outcome in {\it Kelo}, we readily grasp why critics felt it was natural to classify the case along an additional dimension. 

A survey of the literature shows that many critical voices make use of a combination of substantive and procedural arguments, to  paint a bleak picture of the {\it context} surrounding the decision to take Kelo's home. In particular, many critics focus on bringing out  what they see as the ``real'' reason why the taking was sanctioned, namely that a powerful commercial interests succeeded in usurping state power. This, in turn, is derided as a perversion of legitimate decision-making, used to argue more  broadly that economic development takings such as {\it Kelo} suffer from what I will refer to here as a {\it democratic deficit}. 

Important concrete factors that critics tend to stress include the imbalance of power between taker and owner, the incommensurable nature of the opposing interests, the lack of regard for the owner displayed by the decision makers, the close relationship between the taker and government, and the feeling that the public benefit -- while perhaps not insignificant -- was made conditional on, and rendered subservient to, the commercial benefit that would be bestowed on the taker. This dynamic, where the public bodies involved can no longer be seen as leading and pushing the process forward, but are also -- to quite some extent -- being led and being pushed, are regarded as particularly suspicious, particularly by critics who are not die hard libertarians.

From a theoretical point of view, I take all of this to suggest that many critics of {\it Kelo} effectively adopted a social function view on property, by paying close attention to the wider social and political context of the taking. Importantly, if we now turn to the social function theory of property, we are placed in a position to engage more actively with this form of reasoning, also as an integrated part of our assessment of the law. This may then in turn give us cues as to how we should reason -- about the law -- to justify a departure from the course laid down by previous cases on the ``public use'' requirement, where such a broad perspective was largely missing. Indeed, it seems to me that this is exactly what the minority of the Supreme Court did, particularly Justice O'Connor, who formulated a strongly worded dissent.\footnote{She was joined by the four other dissenters, but Justice Thomas also formulated his own dissent, taking a more narrow view and arguing for the revival of a strict reading of the public use requirement.}
She writes as follows:

\begin{quote}

\end{quote}

It seems to me that the values Justice O'Connor rely on in her assessment are closely related to the idea of human flourishing presented by Alexander and others, particularly those pertaining to the political function of property as an anchor for community and democracy. Indeed, the danger of powerful groups gaining control of the power of eminent domain does not only affect the individual entitlements of owners. It also affects society, as the economic rationality used to justify interference comes to result in an implicit political statement to the effect that the property of the rich and powerful is better protected, and valued higher by the state, than property owned by regular citizens, residing in ordinary communities.

The effect of a traditional economic development taking is that property rights are transferred from the many to the few, taken from ordinary people and given to the powerful. Hence, these cases represent a possibly pernicious redistribution of property, not necessarily in financial terms -- depending on the level of compensation -- but surely in terms of property's social function. The structural imbalances of the condemnation process itself find permanent expression in the new distribution of property. Property owned by a community of homeowners becomes property owned by a commercial company. Hence, the social structures of a living community are dismantled in favor of a social structure that revolves around the commercial interest of a company. The political and social power of the community is diminished, perhaps lost in its entirety, while the political and social power of the drug company increases.

It seems clear that to Justice O'Connor, this too is a negative consequence of the taking. Again, we notice that recognizing this effect requires a social function approach to property. There is no clearly quantifiable individual loss -- no one particular ``stick'' in the property bundle that is not compensated. Rather, it is the community itself that is lost, a community that was not directly implicated in any ``right'', but which played a crucial role in providing meaning to the totality of the bundle enjoyed by the owner. Even if we extend our perspective to account for indirect individual losses, we are not doing justice to the loss in this regard. The owner might relocate, acquire new property with a similar meaning in a new community somewhere else. But that does not make up for the fact that {\it this} community is lost forever, as {\it this} property takes on new meanings and functions. The loss to Kelo, therefore, might  even be a significant loss to the City of New London.

Of course, the economic and social gains of development might outweigh such negative effects on community. But, arguably, the balancing of interests required in this regard can only be carried out by an institution that sufficiently recognizes the owners' and their community's right to participation and self-governance. The presence of a highly active commercial third party, in particular, means that public participation in the standard sense might be insufficient. In economic development takings, the commercial company typically appears alongside the government, as a more or less integrated part of the institutional structure making the decision to condemn. The owners, however, do not enjoy corresponding level of participation.

In particular, their interests are only negatively defined. They are adversely effected and may object, but under standard administrative regimes they play no constructive role in the process. For instance, they are not called on to take part in the development itself, or to assess its merits more broadly than by being asked to respond based on their own individual entitlements. In fact, I think this is one of the main problems with economic development takings. I will argue for this in more depth later, but I remark here that an important reason to focus on this aspect is that it involves precisely those values that economic development takings are most likely to offend against. In particular, if the loss of community outweighs the positive effect of economic development, this is unlikely to be recognized by a process that relies only on the positive contribution of the developer and the expert planners. 

The objections made by owner, moreover, may not only be given too little weight given the imbalance of power between taker and developer. As long as owners themselves focus only on the individual loss, they may not get to those issues that are the most important for property's social function. To theoretically proclaim that these aspects need to be considered will not solve this problem. To solve the problem, institutional changes must be made, to give those functions a voice in the decision-making process.
This, more than anything else, speaks in favor of greater involvement by the community of owners (including, quite possibly, even non-owners) in the decision-making process relating to development. Not only by asking them individually if they have objections, but by first empowering them and then compelling them to assume a constructive role in relation to the proposed development. They should engage directly with both government and potential developers, consider alternative schemes, and be encouraged to make their own proposals. In short, they should {\it participate}, as a community. According to the human flourishing theory, as I understand it, this is not only a right, but also an obligation. It gives a plausible basis on which to strike down economic development takings, and to do so without giving up the value of judicial deference. In addition, it is a call for institutional reform, to search for new governance frameworks that will empower owners and their communities and thereby enable genuine participation. 

It seems to me that Justice O'Connor's argument reflects some of these ideas. Indeed, she seems to believe strongly that the taking of Kelo's home would be a particularly harmful interference in the ``just social structures'' surrounding it. Importantly, a piece-by-piece entitlement-based approach to {\it Kelo} could hardly justify the degree of disapproval seen in Justice O'Connor's opinion. After all, Kelo had been offered generous compensation, there had been no clear breach of concrete procedural rules, and the claim that the taking was {\it only} a pretext to bestow a benefit on Pfizer did not seem supported by the facts. Rather, it is the overall character of the taking that is used to argue that it is unjust. In this picture, moreover, the perceived lack of a clearly identifiable and direct public benefit becomes only one of several factors.

In addition, the institutional, social and political aspects of the case come into focus. The economic implications are less important, even the importance of homeownership to personhood does not receive the same attention as structural aspects. The problem which overshadows everything else to Justice O'Connor, is the concern that economic development takings represent a form of governmental interference in property that may easily come to systematically favor the rich and powerful to the detriment of the less resourceful. Hence, such takings may help establish and sustain patterns of inequality that are not desirable. Indeed, hardly anyone would openly regard them as such; it is not hard to agree that if Justice O'Connor's predictions about the fallout of {\it Kelo} are correct, then this is indeed be ``perverse''. 

The question, of course, is whether her predictions are warranted. This is a call for empirical and contextual assessment of economic development takings, to help us gain a better understanding of how they actual affect political, social and bauraucratic processes. In addition, it raises the question of how to {\it avoid} negative effects, that is, how to design rules and procedures that can reduce the democratic deficit of economic development takings. As I now move away from theory towards concrete assessment of economic development takings, both these questions will be addressed. 

\section{Conclusion}

In this chapter, I have presented the core question of my thesis, concerning the legitimacy of takings that benefit commercial schemes. I started by considering a concrete example of such a scheme that looked like it might well result in compulsory acquisition of land, namely Donald Trump's controversial plans to develop a golf course on a site of special scientific interest close to Aberdeen, Scotland. In the end, the plans did {\it not} require takings, as Trump was able to build his golf course by making creative use of property rights he acquired voluntarily, against the complaints of some recalcitrant neighbors. 

In fact, this turn of events made the example even more relevant to the points I have been trying to make in this chapter. It served to highlight, in particular, that the question studied in this thesis is not a black-and-white issue that sees privileged property rights enthusiasts on the one hand confronted by the collective will and the regulatory state on the other. Rather, the example of Trump's golf course allowed me to emphasize the importance of context when assessing both the meaning of property rights and the extent to which such rights should be protected. In particular, to protect the property rights of those opposing Trump's golf course was not about protecting just any property, it was about protecting the property of members in a local community that felt it would be detrimental to this community, and to their lives, if Trump was allowed to turn it into a golf resort. 

After Trump decided not to pursue compulsory purchase, protecting the property of these members of the community became a question of {\it restricting} the degree of dominion that Trump could exercise over his own property. Hence, under a conventional and overly simplistic way of looking at these matters, protecting property rights became tantamount to restricting them, a seeming paradox. 

To resolve this paradox, and to arrive at a better conceptual understanding of economic development takings, I looked to various theories of property thought. I noted that there are differences between civil law and common law theorizing about property, but I concluded that these differences are not particularly relevant to the questions studied in this thesis. In particular, I observed that neither the bundle theory, dominant in the common law world, nor the dominion theory, used by civil law jurists, helps to justify economic development takings as a well-defined category of legal thought. 

I then went on to consider more sophisticated accounts of property, noting that a range of different {\it normative} theories have been proposed. These differ with respect to the values that they think the institution of property should promote, and as such they are also relevant to the question of what we should focus on when assessing economic development takings. However, they do not readily allow us to address the initial challenge of justifying that this should be regarded as a separate category.

To do this, I argued that a traditional entitlements-based perspective on property had to be abandoned. Instead, I embraced the social function theory of property, which encourages us to take a more contextual perspective on rights and obligations inherent in property. In particular, the theory compels us to recognize the importance of property in regulating social and political relations. Hence, economic development takings are special because they redefine the meaning of the property that is taken and disturb the relationship between the owner and the taker as members of society. For instance, a home may be turned into an investment, which may also communicate to homeowners and their communities that they are regarded as less productive and valuable assets to society than commercial companies. The social function theory asks us to acknowledge that property rules are hardly ever neutral with regards to such effects, and this is the key observation that allows us to make sense of economic development takings as category of legal reasoning. 

After concluding that the social function theory allows me to formulate a coherent conceptual basis for studying such takings, I went on to argue that in the first instance, the theory should be understood as giving us purely {\it descriptive} insights into the workings of property and its role in the legal order. In this, I advanced a different stance than many property scholars, by arguing that it would be better to decouple the more normative aspects of the theory, to allow it to serve as a common ground for further normative debate. 

I then went on to consider normative aspects, by turning to the human flourishing theory proposed by Alexander and Pe\~{n}alver, which focuses on the importance of property to communities and individual participation in social and political processes. I argued that the theory provides a further important insight, namely that protecting property against interference that is harmful to human flourishing is a responsibility that the state has even in cases when the individual owners themselves neglect to defend their property, for instance because of financial incentives to remain idle.

Finally, I arrived at the core notion itself, by providing a bird's eye view on economic development takings drawing on the theoretical insights collected from preceding sections. To make the discussion concrete, I considered the recent case of {\it Kelo} which propelled the notion of an economic development taking to the front of the takings debate in the US. I focused particularly on the dissenting opinion of Justice O'Connor, arguing that it approached the issue in a way that is consistent with the theoretical basis I have attempted to establish in this chapter. 

In the next chapter I make my analysis of economic development takings more concrete, by considering how such takings are dealt with in Europe and the US respectively. I note that the category has yet to receive much attention in Europe, so the discussion focuses on the US. Here, the attention this issues has received after {\it Kelo} has been staggering. To get a broader basis upon which to asses all the various arguments that have been presented, I consider the historical background to the issue as it is discussed in the US, by looking at the history of the public use restriction, particularly in case law developed by the states in the 19th and early 20th century. I then assess recent proposals to deal with economic development takings, particularly those that aim to address the democratic deficit that often from such takings.

In subsequent chapters, when I relate this discussion to the data from the case study on Norwegian hydropower, I look back at the theoretical basis provided here to guide the analysis. In particular, I stress those aspects of the Norwegian case that illustrate that the solutions that have developed on the ground in Norway, as a practical response to the increased ability of local owners to engage in economic development themselves, deserve academic attention. Indeed, they will point to the conceptual strength of the idea that property is irreducibly embedded in community, and that its meaning and function is not -- and should not -- be ordained from above, but should be allowed to arise from the bottom-up through institutions of participatory democracy.









 the forefront  from the preceding dicussion 


from a bird's eye view
e
 reaching the conclusion that the abstraction whereby property protection is not understood in light of them is unwarranted. 





that would allow me to be more precise about the property tvalues that are at stake in 





they should be protected. Indeed, after Trump decided to turn his attention at how he could make use of is own land, as opposed to how he could acquire the land of others, the 

e special considerations that seem appropriate to make once economic development takings are recognized as a special category. It is not correct to think of the controversy surrounding such to taking as an interference in 



Hence, the reader might wonder if the example is not 


and the key theoretical notions that I rely on in the remainder of the thesis.



\begin{quote}
There is room to allow for the virtue of social responsibility and solidarity and for the ideal of avoiding any structural privileges that favor the better-off. Those who endorse these values should seek to
incorporate them -- alongside and in perpetual tension with the value of
individual liberty -- into our conception of private property and into the legal norms governing
public actions that necessitate some injuries to individual landowner.\footnote{\cite[802]{dagan99} (citations omitted).}
\end{quote}




done where property dictated a different outcome than normal, because 

Instead of property standing in opposition to a just outcome for the squatters, property can be recast as an argument in favor of such an outcome.

\noo{ A similar line of reasoning, going even further in protecting squatters, was adopted by the Supreme Court of }




g that an otherwise appropriate eviction order should not be implemented due to other parties interest in the land. 


Consitutional Court of South Africa expliclty


her case was therefore one where two fundamental rights tNow, importantly, the Constitutional Court 



that no other course of. Further to this, one is likely forced to conclude that the State must be called on to force out the squatters, as a matter of ``natural'' or otherwise  that nor can our laws. The use of force, then, becomes the only possible outcome: the State must be mobilized to throw the squatters out. 





learly, it is not. The fact that the apartments are occupied is certainly of interest to the owner by others will cert

in such a situation is clearly 


when judging 

 Indeed, we are entitled to think that the {\it content} of 

 latter might indeed be more likely than if he is a private individual.


 {\it quality} 

If a large number of council flats are unoccupied 




 , since  entertain expectations in the property, directly linked to how they live their lives. For instance, they might know they can come to stay over whenever th



 Property empowers people,  but also makes them responsible; to interfere with property is to interfere with the social structure in which it is embedded.

a collection of preferences, seeking to maximize the utility of property use. Similarly, the public is no longer measurable along a metric of welfarism, seeking 


Instead of arguing about property based on utilitarian or welfare-maximizing 






 It is not correct to assume, in particular, that an emphasis on property's social dimension necessarily leads to less protection for property owners and increased State control over property. 

To justify such a policy, it seems that the way one conceives of the State may be far more important than how one coneives of property. Why, in particular, would the effect of State control enhance the 

 in which currently existing property rights function. It is no 

it is still a matter of political choice, in need of further reasons and closely dependent on the circumstances 




 However, this is by no means a necessary consequence of the conceptual premise of property's social aspects. To arrive at the conclusion that protection against State action can be rolled back relies on the further assumption that the State itself poses no threat to the social functions of property. Moreover, to arrive at the even stronger conclusion that State control and regulation is in order, relies additionaly on the premise that State actions are not only no threat, but generally conducive to furthering and improving upon those funcitons. This, however, might be an overly naive assessment of the function of the State in modern market economies.

 


  and that Stcan be extended relies on the further premise that the State is {\it better} able than individuals and local communities 

Professor Hanoch Dagan and Professor Gregory Alexander are important pioneers of this perspective on property in the US, and they both argue for a social turn in the theory of property law, although they do so in slightly different ways.

According to Professor Dagan, the ....

Gray:

\begin{quote}
Property -- if it exists -- is intrinsically about the ranking of moral and social (rather than economic) priorities. If property were purely about the endorsement of rationally calculated commercial interests, we would still support the institutional of slavery and its implicit affirmation of the economic value of coerced human labour-power. What we {\it can} is that proprietary entitlement revolves around some specifically enforceable expectation of autonomous control over valued resources or opportunities. The categories of thing that are ring-fenced in this way are defied by collective perceptions of moral or social worth. These perceptions are never absolute in quantity, requiring instead a constant arbitration between the various goals that we wish to achieve. The individual's expectation of specific performance -- that is, of effectual decisional control over a resource or opportunity -- is generally, but not always, realised. The expectation is just that -- a relatively fortified hope or {\it spes} which is so commonly fulfilled by default that we rarely recognise its inbuilt limitations...
\end{quote}

\begin{quote}
For those who think they own land, the clear message is that money is the sole asset to which their claim of ``property'' can ultimately refer. Reality is always monetisable at the command of the state: money is fast becoming the measure of all value. Nowadays it often seems that the idea of property -- in the shape of an indefeasible entitlement of control -- is actualised only in the context of ideas themselves.
\end{quote}

\begin{quote}
The modern super-rich can easily harness powers of compulsory purchase with the connivance of weak, subservient, short-sighted or corruptible agencies of government. Those embraced within the euphemistic contemporary designation of ``high net worth'' can then do exactly as they please under some pretense of conferring public benefit or creating a ``trickle-down effect'', but remaining all the while beyond the reach of any genuine political control or scrutiny. This is a new species of predator of whom we should all be extremely vary.
\end{quote}

\begin{quote}
As is increasingly affirmed by contemporary property theorists, the stuff of modern property involves a consonance of entitlement, obligation and mutual respect -- in other words, a phenomenon characterised by accommodation and reciprocity rather than by predatory acts of self-interested taking.
\end{quote}

Underkuffler: ``common property'' (a conception in which collective power to alter privileges and powers is assumed to be part of the idea of property, itself'', p 59, p 63:  ``Recognition of an alternative conception of property will help us to remember that individual autonomy and social context are -- in fact, and unalterably -- deeply intertwined.``)  and ``operative property''. 

J.W. Harris:  ``trespassory rules'' and ``ownership spectrum''.

Dagan: ``When incorporating social responsibility into our understanding
of property, the challenge is to show that the concept of property can
encompass social responsibility without destabilizing the effects
of ownership in protecting individuals, particularly politically
weak individuals, from the power of government.'' p. 1262-1263 (soc-prop alex)

\begin{quote}``we must concede that it is far more likely to be sustained at the microlevel of our local communities, where our status as landowners also defines our membership''. Thus, a distinction should be drawn between imposing constraints on private property to benefit the community to which the property owner belongs and prescribing injurious regulations to benefit the public at large. The more the constraint resembles the former type of cases, the higher the threshold of social responsibility that should be implemented (thus legitimizing the imposition of constraints  or uncompensated harms as part of the meaning of ownership) and vice-versa.
\end{quote}p. 1266-1267 (soc-prop alex)

\begin{quote}
For all of these reasons, proponents of including a social-responsibility norm in the meaning
of ownership should support clear and simple rules rather than vague standards.
\end{quote}p. 1268-1269

\begin{quote}
A rule-based regime that draws careful distinctions within types of injured properties and types
of benefited groups is much more capable of successfully integrating
social responsibility into takings doctrine.
\end{quote}

\begin{quote}
Defining property as a set of legal rights or entitlements protected by legal and 
political institutions for the purpose of facilitating wealth acquisition and production
is far too limited to facilitate sustainable relationships between people and their
environments and among people.  The ‘bundle-of-rights’ concept treats both the
resources that are the object of private property rights and the rights-holders as
disconnected from the ecological and social environments in which both exist.
\end{quote}




The social aspect of property has recently gained recognition among legal scholars, not only as an

viz a vis other persons, is one that is deserving of protection.




 when it comes to issues such as regulation and 

parties, including the 

This observation bring us to the second sense in which economic development takings are a unique category. 


I feel that there is no further need to argue theoretically for the assumption that part of the surplus stemming from a beneficial, permitted use of property is attached to the property itself and hence belongs to the bundle of rights associated with property ownership, or, if we adopt the exclusion theory, that it falls under his dominion.




development is still attached to the property, the 


such systems, it is contentious whether or not the owner is entitled to compensation for a {\it loss} of development value when his value is taken. 


might still be of relevance what exactly

constitutes an interference with the rights of the owner that entitles him to protection, {\it qua} owner, under the relevant provisions of property and constitutional law. 

of economic development takings constitute 


, I believe both theories come out lacking because they focus on individual entitlements associated with property

 history of the ``bundle of rights'' theory shows that the debate between t


. Whether it is a point of view that weakens or 


This shows that 

The theory has now emerged as the dominant one


This conception, of course, also served to {\it justify}  rights 

Both 


On a superficial level, the property rights at play in this situation . Under the idea that 

 places the property 

 use they make of their property can still have undesirable redistribution effect, 



Problem of being indeterminate and abstract, a beautiful philosophical and political formula that offers little guidance about how to address concrete legal problems and solve disputes. The content highly contested, particularly regarding duties it imposed on owners.\footcite[908]{robilant13}

\begin{quote}
The Fascist property theorists had been more specific about the content of the social function of property. For them social function meant the productive needs of the Fascist nation.\footcite[909]{robilant13}
\end{quote}

\begin{quote}
In a Europe threatened by totalitarian
rule, this resource-specific approach helped liberal jurists achieve two
important goals. First, it emphasized the value of pluralism in property law. In
times where property debates were becoming increasingly
focused on the productive efficiency of the Fascist nation, the
theorists of the tree concept of property believed in the
value of pluralism. In their discussion of the different branches of the property tree,
they focused on individual owners' privacy and freedom of action, equality in access to productive resources, and cooperative management of resources. Second, the focus on resources allowed our jurists to deal with
the fundamental problem of the value of pluralism in property law. The plural values and interests property law should promote are often in conflict with each other, and lawmakers will be called upon to make
difficult choices. In Fascist times, liberal property law scholars worried about the arbitrariness of these choices that may potentially
lead to a virtual abrogation of individual property rights. By grounding values and interests in the context of specific resources, they sought to guide and constrain lawmakers' normative reasoning.\footcite[910-911]{robilant13}
\end{quote}

\begin{quote}
For the tree concept's liberal advocates, analysis of the concrete characteristics of resources,
and fidelity to the historical and present legal framework for specific resources,
was the way to reduce the arbitrary nature of normative reasoning in property law and to stem the Fascist regime's potential erosion of property rights in the name of a generic and
unspecified interest of the Fascist state.\footcite[911-912]{robilant13}
\end{quote}

\begin{quote}
The debate between the liberal theorists of the tree concept and Fascist property
scholars suggests that the challenge for progressives is to rethink and
thicken or expand the notion of autonomy rather than drop it.\footcite[928]{robilant13}
\end{quote}

\begin{quote}
The tree concept views property as a tree with a trunk -- representing
the core entitlement that distinguishes property from other rights -- and many branches -- representing
many resource-specific bundles of entitlements. The trunk of the tree is the owner's entitlement to control the use of a resource,
mindful of property's ``social function.'' For the theorists of the tree model, the social function of property evokes a plurality of values: equitable distribution of resources, participatory management of resources, and productive efficiency.\footnote{\cite[872]{robilant13}.}
\end{quote}

\begin{quote}
Fascist property scholars had also appropriated the social function formula. For the Fascists,
the social function of property meant the superior interest of the Fascist state.\footcite[908-909]{robilant13}
\end{quote}

\begin{quote}
The theorists of the tree concept realized that,
to provide a good alternative to Fascist property, protecting the owner's sphere of
autonomous control was not enough. A modern liberal concept of
property is one that acknowledges and
foregrounds the social dimension of property. The rise of Fascism, they realized, was the
consequence of the crisis of liberalism. It was the consequence of liberals' insensibility to new ideas about the proper balance between individual rights and the interest of the collectivity.\footcite[907]{robilant13}
\end{quote}


I then single out for attention some of those values that have been highlighted in relation to property's social function, focusing on the notion of {\it human flourishing}.\footnote{As explored in the context of property law, particularly in the work of Gregory Alexander.}

he dominion and the bundle of rights theory shows that the true conceptual import of the ``bundle of rights'' metaphor is hard to measure along an axis of property protection, understood as the owner's right to retain privileges associated with property ownership. Whether the bundle theory affords the owner more or less protection in this sense depends entirely on the view one takes on the individual sticks of the bundle. If they are thought of as less fundamental and worthy of protection than the bundle as a whole, less protection follows. On the other hand, if such sticks are understood as first-class property rights in themselves, more protection would follow.

My thesis will not actively address this debate, so I will merely note that is is good law in most jurisdiction, including the US, that a governmental action will not be regarded as a taking unless it upsets the structure of the bundle above a certain threshold. In most cases considered in this thesis, identifying this threshold will not be present us with difficulty; economic development takings tend to imply complete deprivation, not a mere reshuffling of the property bundle. Hence, I do not need to venture far into the murky waters of the ``regulatory takings'' issue, which is the name associated to this question in the US.\footnote{I will, however, briefly return to this issue in Chapter 5, when I discuss alternatives to takings in economic development cases. Such alternative, in particular, may involve participatory frameworks where the property bundles are restructured in a less invasive manner, to achieve the desired development without having to resort to outright takings.}

In my opinion, the history of property theorizing shows that the debate between bundle of rights and exclusion theorists does not have a crucial bearing on the question of legitimacy of takings for economic development. On the one hand, under both theories it is usually straightforward to identify typical cases of economic development takings, when both the owner, the taker and the government agree that the interference is such that it is to be classified as an exercise of eminent domain. While it is still of relevance how we think of property, the focus then naturally shifts from the theoretical question of what property is, to the more practical question of when it may legitimately be taken. Moreover, as illustrated by Epstein's work in particular, both theories of property are amenable to interpretations that can be used to argue in favor of a more or less restrictive attitude towards takings. 

My response to this is twofold. First, economic development takings are different because fair compensation is very difficult, if not impossible, to provide in such cases. If an owner gives up his property for a non-commercial project, he is usually entitled to have his economic loss covered by the public, in many jurisdictions based on the {\it market value} of his property, possibly including also some extra compensation for personal losses. When the proposed development is not for commercial profit, this approach, based on compensating the owner's loss compared to the value of his property before the taking, is the commonly accepted approach to compensation determination. It is also usually perceived as fair. 

A possible alternative would be to base the compensation on the amount that the public might be willing to pay, considering the importance of the planned project. But this would effectively allow the owner to capture a profit from a not-for-profit public project, creating a situation where important public projects may end up becoming prohibitively expensive. It would also largely undermine the very idea of eminent domain, which is meant to prevent owners from demanding extortionate prices in voluntary negotiations over the sale of property needed for important public projects. In fact, as long as the not-for-profit nature of the public project is clearly entrenched, an argument may even be made that less then market value is in order, in so far as the regulatory powers of government extends to engaging in the the sort of value-reduction that is entailed by the taking. 

In this regard, economic development takings play out very differently. If compensation is based only on the owner's loss, calculated based on the pre-project value of the land, the taking effectively deprives the owner of the land from his share in the surplus resulting from development of his property. The developer, in particular, does not only profit from the development itself, he also profits directly from the use of eminent domain. Intuitively, it seems perfectly clear that this is a very different situation from the typical cases of not-for-profit takings in the public interest. However, this recognition raises a subtle theoretical point, not resolved neither by the bundle of rights nor the exclusion based theorizing about property. 

First, notice that the intuitive conclusion reached above rests on the conceptual premise that the value released by development is in part inherent to the land as such, not entirely created by the subsequent efforts and investments of the developer. This is probably  uncontroversial, however, as such a perspective lies at the very heart of any market economy based on private property rights. Indeed, after the land is transferred to the developer, the value of the property {\it to him}, will now reflect the part of the development surplus that the market regards as being inherent to the land. If he sold it to a third party, he could expect the price of the property to reflect this fact.

However, there is a second assumption that must be addressed, which is less obvious. What property right, in particular, serves to give the original owner a claim to take as share of the development surplus? That he does have such a claim seems intuitively clear. In particular, any property owner not effected by compulsory acquisition would be able to either take active part in its most profitable permitted use or else bargain for a share of the development surplus when selling to an interested developer. The prospect of this, moreover, is why there exists a market for property in the first place. Intuitively, it seems clear that the surplus stemming from a beneficial, permitted use of property is attached to the property itself and hence belongs to the bundle of rights associated with property ownership, or, if we adopt the exclusion theory, that it falls under the owner's dominion.

However, in some jurisdictions, for instance in England, the idea that all development value belongs to the State has been influential. In this case, while the right to use the property in accordance with governmental regulation is undoubtedly part of the owner's bundle, a {\it change} in the regulatory status of property might have to be seen as an act of government granting a new right in the property. In this case, an economic development taking might be recast as destroying an existing right, belonging to the owner, and creating a new one, belonging to the developer. This perspective, then, fails to provide a theoretical basis on which to rebuke the lack of compensation for the development surplus in economic development takings. The right to the surplus is not taken at all, but {\it created} by the government. Still, it hardly accords with a natural sense of justice when this right is simply bestowed on the commercial development company, with no regard for the original owners. 

In particular, any property owner not effected by compulsory acquisition would be able to either take active part in its most profitable permitted use or else bargain for a share of the development surplus when selling the property. The prospect of this, moreover, is why there exists a market for property in the first place. Hence, we arrive a theoretical deadlock, where the idea that development value belongs to the State gives rise to the unacceptable outcome that the property rights of owners affected by takings are second-class rights, not of the same kind as those rights enjoyed by other owners.

It is important to keep in mind that the social function of property is descriptive in the first instance?

In dealing with economic development takings, I argue, this is not so. Without looking to the inter-personal and social purposes of property, it is hard to even recognize why these takings are a distinct category.

This is generally true, but becomes particularly clear when we work under the assumption that the development potential of property belongs to the State. It seems, in particular, that the only way to resolve the paradox identified above is to shift attention from the relationship between the owner and the property, to the property's function in regulating the relationship between the owner and the developer. It seems, in particular, that quite apart from any right to undertake development, which may well be a prerogative of the State, the property serves a crucial social function in giving the owner a platform from which he can engage meaningfully with other legal persons, including commercial companies interested in pursuing commercial development on his land. The fact that owners can normally bargain for the development surplus, take part in development themselves, or deny it altogether, need not be look at as a right to development, but as a right to participation in decision making and a minimum of autonomy in dealings with other interested parties. This, undoubtedly, is a {\it function} of property in a social context. My theoretical contention, therefore, is that this is also an aspect of the {\it right} to property, which should be regarded as protected against interference. 

Behind this idea lies a more general perspective which sees property itself as embedded in a social context, thereby committing us to a form of theorizing about property that goes beyond the classicaly dominantl theories relied on in constitutional property law. But there are well developed theories of property that will aid us in this, as they incorporate a social view on the function and purpose of ownership. Such a view has old roots in Europe, and is currently gaining ground also in contemporary US scholarship. 

Hence, it does not in itself dictate any particular stance on cases such as Modderklip. It dictates only a way of looking at them, allowing us to debate whether we should conclude that Modderklip, {\it qua owner}, has an obligation to take into account the squatters' need for housing and their expectation of not being evicted from their homes. With a classical liberal understanding of property, focusing merely on entitlement, such a debate becomes impossible, at least in the absence of any explicit basis in law for concluding that the owner has such an obligation. Hence, the classical discourse forces us to leave property behind altogether if we want to argue in favor of the squatters. In this way, the discourse easily becomes one where those arguing for social justice are led to take a negative stance on the institution of property itself, while those arguing in favor of this institution are unfairly pegged as advocating on behalf of privileged elites.

\footnote{See \url{http://www.theguardian.com/world/2012/jul/10/donald-trump-100m-golf-course} (accessed 06 July 2014).}

The theoretical conception of property 


sway of looking at property and justice is invariably shaped by our perception of the 
namely that protecting property is not necessarily about giving owner's 
the meaning of property depends on the context. 
the meaning of protecting property against interference can raise quite subtle issues. It 

taking me straight to a core challenge raised by my work. It seems to me, in particular, that property is not an agent-neutral institution. That is, I do not think that property owned by Donald Trump has the same meaning, or even legal status, as property owned by a man like Michael Forbes. This might be a startling claim, but 
ets right to the heart of the issue, by flagging those aspects that are particularly controversial and problematic. 



First, it  serves as an example of the kind of scenario where the use of eminent domain raises special problems of fairness and justice. To take land from private owners to facilitate for-profit development projects is not like taking land for a hospital, a school, or a public road. There is a clear sense in which the taking benefits a privileged group of owners, those holding shares in the development company, and disadvantages another, those who stand to lose their property. 

The perceived unfairness of this is exasperated by the fact that current compensation regimes typically preclude the owner from taking any share of the benefit resulting from development. Hence, the taking of land becomes in effect a mechanism whereby the developer can capture the entire development surplus. On a deeper level, even if compensation mechanisms are put in place to achieve more adequate compensation, the taking still serves to transfer decision making power from landowners to the development company. The owners are marginalized in the decision making process regarding whether or not development should take place, and on what terms, and they are deprived of an opportunity to take part in the project as asset holders.

Hence, economic development takings have a {\it redistribution} function, both with regards to wealth and power. Problematically, the redistribution facilitated by economic takings tends to give {\it more} wealth and power to influential and affluent groups, while marginalizing people that are less privileged to start with. Such redistribution effects are not in themselves desired by government, at least not openly, and they do not serve to legitimize the use of eminent domain. From the government's point of view they are a mere side-effect, an unavoidable consequence of economic growth. But for both the takers and the property owners, the redistribution effect is of crucial importance. 

In this way, economic development projects are typically very different from the building of infrastructure or schools or hospitals or other projects operated by the public to directly further their interests. Moreover, it can often be argued that takings to the benefit of commercial projects are not legitimate, since they do not strike an appropriate balance between the interests of the public and the private owner. Many jurisdictions have constitutional property clauses to ensure that eminent domain only take place in the public interest, and most jurisdictions do not permit the taking of land unless it serves some public purpose. Hence, owners affected by economic development takings can make the case that their property rights are protected against such an interference, since a sufficiently compelling case that the taking is in the public interest can not be made.

But the case of Trump's golf resort in Scotland also illustrates another point, namely that in many cases, financially powerful actors will have no trouble acquiring the land they need from voluntary sellers. In these situations, the most controversial question that arises concerns the use that the developer plans to make of his land; should his plans be approved by the government, will they be in the public interest? This is a seemingly very different question from the question of legitimacy of takings, but as the case of Trump's golf resort shows, there are interesting connections between the two. In particular, while Trump was unable to acquire the land of some recalcitrant locals, he was able to secure enough land rights to enable him to effectively work his way around the opposition from those that refused to sell. His property rights, and the extent to which he could exercise them unhindered by governmental control, was key to his success. Those rights were given priority, with  both neighbors and other members of the public standing  powerless and  unable to prevent Trump from carrying out a project that they felt would be detrimental to the environment and the stability of the local community. They did try, by challenging his right, as a property owner, to make use of his land as he desired. Hence, a contrast emerges against how they would themselves invoke the sanctity of property rights when faced with the prospect of having compulsory purchase orders issued against them. 

Hence, perhaps the most important lesson to be learned from the controversy surrounding Trump's activities in Scotland is that property rights is a double-edged sword, both for owners and the general public, for privileged groups as well as for those that are  marginalized. If fairness is our measuring stick, moreover, it is of crucial importance {\it who} the owner is, and what purposes the land serves, to him and to community at large. Protecting the property rights of Forbes the farmer is largely tantamount to restricting the property rights of Trump the property tycoon, and vice versa. This further suggesting that no black and white perspective is feasible when talking about property as a fundamental human right. 

In light of this insight, I will devote the rest of this Chapter to explore the theoretical background to the question of economic development takings, as I explore this issue in the remainder of the thesis. Importantly, I will stress the contextual and purposive nature of property as a human rights, which should be looked at as an integrated part in a system of fundamental rights that award individuals and communities with a basis upon which they can flourish through self-governance. This view also focuses on the social obligations of property, and leaves great room for recognizing the need for regulation of property use within a framework focused on inclusive and just governance. If this was not so, as illustrated by the case of Trump's golf resort, protection of property would simply not be effective, except for those owners who already wield the financial and political power needed to implement their willed use of their land, possibly to the detriment of other owners. Under a simplistic notion of property rights, where they are conceptualized as a privilege, giving the owner exclusive dominion over that which is his, this creates an apparent paradox of property protection. To protect the property of one, in particular, will too often be tantamount to an assault on the property of another. However, by explicitly recgnizing that property comes with responsibilities as well as privileges, this tension can be resolved. Protection of property, in particular, is not then primarily about protecting the privilege, but protecting an elemental building block of community; the rights and duties bestowed on an individual as an owner, mutually co-dependent on other owners, the public, and various interest groups that interact with him as part of the democratic process.

This vision of property, and of property protection, is particularly helpful when looking at economic development takings, since it allows a more fine-grained analysis of why such cases tend to become controversial, and how it is best to deal with them. Moreover, the relationship between property and the values that it promotes, can suggest enhanced protection of owners in such cases, on the basis of a purposive and contextual reading of constitutional property law. 

The idea of property as a contextual phenomenon that involves both rights and responsibility has received much attention in recent scholarship. It is also an idea that features, at least implicitly, in recent proposals for takings reform in the US. Hence, I will now present the main elements of this theoretical shift in constitutional property law, focusing on how it will inform my analysis of economic development takings and the case study of Norwegian hydro-power development.

This kind of reasoning can also be performed more abstractly, to help us identify certain kinds of situations that deserve special treatment. In this way, we can study legal problems that it might not even be possible to properly recognize as such were it not for the conceptual reconfiguration. In my opinion, the problem of economic development takings is an example of this. It is not {\it prima facie} clear in particular, that this category of takings should be addressed as a special category. In fact, I think doing so involves at least a degree of commitment to the core ideas of the social function understanding of property. On the other hand, when this theory is embraced, I believe it becomes very natural -- indeed, necessary -- to study economic development takings as distinct from other kinds of interferences in property. 

It is also telling that economic development takings only really came to prominence as a special category following the public outcry after the case of {\it Kelo}. Arguably, it was the intuition of lay people, not the theorizing of experts, that propelled this category ito the forefront of legal discourse in the US. In many jurisdictions, no corresponding shift in perspective has so far taken place. However, it is my hope that this thesis will contribute to changing this state of affairs.

I return to economic development takings specifically in Section \ref{}, where I discuss its defining features in terms of the social function understanding of property. First, however, I would like to address normative aspects of property. While I think the social function theory itself can be thought of in  descriptive terms, there are several closely related, distinctly normative, theories that have appeared alongside it. I will be drawing actively on these in my normative assessment of economic development takings, to make policy recommendations based on the data I present on Norwegian waterfalls and hydropower. Hence, in the coming Section I will present some key ideas that have inspired me in my normative analysis.

While the human flourishing theory has merit as a framework for justification of interference in property, it also allows us to take a fresh look at the question of {\it legitimacy} of interference. In particular, in cases where the interference is potentially harmful to the values associated with human flourishing, it provides us with a new template for assessment. This is relevant when adjudicating legitimacy cases, to the extent that property's social function warrants extending the range of values drawn on when interpreting and applying the law. In any event, we are provided with a fresh approach when addressing the normative question of what the law {\it should} be. In the next section I consider the implications of the human flourishing theory for economic development takings. This category, arguably, can not even be properly defined without drawing on values discussed by the human flourishing theory. In particular, the human flourishing theory allows us to argue that it can be circumscribed in a way that is legally relevant, both in regards to administrative law and constitutional property clauses. 

This is so also in the US, where the Fifth Amendment says ``private property [shall not] be taken for public use, without just compensation.”.  Linguistically, this is not formulated as a restriction on when the state may take property, but merely a rule indicating that when they do so for a public use, they have to pay compensation. Nevertheless, the formulation can be read as presupposing that the state only takes private property for public use. Of course, in some sense this is not true, for instance, when they enforce agreements between private parties, inheritance rules, or allocate property following divorce. However, it seems quite unlikely -- if we assume that there is no presupposition of public use expressed in the formulation -- that the founders should have intended the state to be able to take property at will for non-public uses, without paying compensation. Rather, it seems more likely that the assumption is that in cases when the state exercises the power of eminent domain, as distinguished from private law cases involving ``taking'' of property on some other basis, the founders assumed that the property would be designated for some public use.

However, it was already long established in US law that the public use restriction did not preclude takings in the ``public interest'', where the public benefits from the taking only indirectly, and the property rights are as a matter of fact transferred to a private party. It was understood, furthermore, that the courts would defer to the legislature's assessment of public use, and that only the ``naked private-to-private'' transfers and takings that were ``manifestly without reasonable foundation'' would be prohibited. 