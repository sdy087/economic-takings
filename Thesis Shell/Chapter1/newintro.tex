\chapter{Introduction and Summary of Main Themes}\label{chap:intro}

\begin{quote}
Thieves respect property; they merely wish the property to become their property that they may more perfectly respect it.\footnote{G.K. Chesterton, {\it The man who was Thursday: A nightmare}.}
\end{quote}
%
%A human being needs only a small plot of ground on which to be happy, and even less to lie beneath. %\footnote{Johan Wolfgang von Goethe, {\it The sorrows of young Werther and selected writings}.}
%\end{quote}
%“That's what makes it ours - being born on it, working on it, dying on it. That makes ownership, not a %paper with numbers on it.”
%― John Steinbeck, The Grapes of Wrath 
%
%
%s
\section{Takings and Legitimacy}

%Judging from academic literature on the subject, property is a difficult and often paradoxical concept,
As a concept, property has been something of a problem child for western philosophy, especially its analytic variety. Property, in particular, appears to have an inherent tendency to stir up divergences that break out of the realm of conceptual analysis to enter the realm of politics, often in a way that makes it difficult to continue a meaningful and inclusive academic debate.

From the extreme antagonism directed at it by radical Marxists to the evangelical praise bestowed on it by libertarians, there is no shortage of politically charged accounts of what actually property is, not to mention what it {\it should} be, assuming, of course, that it is entitled to exist at all. 
Moreover, there appears to be little room for rapprochement between many leading strands of thought. Indeed, one is often left with the impression that different philosophical theories of property tend to diverge largely due to personal conviction, rather than differences based on reasoned argument. %Indeed, one may argue that different ideas of property, practical and theoretical, are behind most, if not all, the major conflicts and confrontations that have shaped the society in which we live.

Responding to this, some prominent philosophers have taken the view that property is not a concept suitable for philosophical study at all. Rather, it has been suggested that property is best taken as a derivative of other notions, such as the social order, or, on a normative account, {\it justice}.
Lawyers, on the other hand, rarely need to consider the philosophical underpinnings of the notion of property that they find entrenched in the law. Instead, hard cases often come to turn on a combination of broader value judgements, politics and expediency. Legal scholars, for their part, are usually content with theories of property that remain largely descriptive, settling for the more modest aim of exploring how best to think of property given the prevailing legal order, rather than trying to come up with theories to explicate its nature as a pre-legal concept.

Still, in certain situations, we can record what seems to be empirical evidence to support the claim that humans reflect a working {\it primitive} notion of property, one which arguably pre-exists any particular social arrangements used to mould property as a socio-political category of law. Most notably, humans, as well as many other species of animal, appear to have an innate ability to recognise {\it thievery}, the taking of property by someone who is not entitled to do so.

Indeed, the famous dictum ``property is theft'', may be more than a flippant and seemingly self-contradictory comment on the origins of inequality. In fact, it might point to a possible alternative direction for investigating the nature of property as such, as a concept that emerges from a more basic distinction between legitimate and illegitimate acts of taking, broadly construed. It seems quite tempting, after all, to describe a person's property as that which they have taken by legitimate means, which may not be taken from them without due process.

Interestingly, while the abstract notion of property has arguably received more than its fair share of attention from disciplines other than law, the common-sense notion of a taking has not received much academic attention outside of the legal community. No great philosophical debates have revolved around this notion, and no chasms has opened as to the correct way to understand it. Moreover, legal scholars rarely attempt to define this notion, preferring instead to regard it as a derivative of the legal order surrounding property. In relation to the notion of a taking, the pragmatic and often jurisdiction-bound perspective of the lawyer appears to reign supreme.

This is problematic, since the notion of property itself is so contested that it might not provide a secure foundation for thinking about takings. But it also suggests an interesting possibility; perhaps studying takings is a path towards a better understanding of property as well? After all, this is where vastly different accounts of property do seem to share at least an important common point of reference.

In this thesis, I will study a certain kind of taking, namely that which is implemented, or at least formally sanctioned, by a government. In legal language, especially as developed in the US, such acts of government are referred to as takings {\it simpliciter}, while talk of other kinds of ``takings'' require further qualification, e.g., in case of contract, theft, tax or occupation. This, in itself, might be cause for reflection as to the ideological commitments inherent in legal language. Moreover, it brings the issue of legitimacy into focus. 

We are reminded, in particular, that under the rule of law, taking is not the same as theft. Rather, the default assumption is that the takings that take place are legitimate. If they are not, we may call them by a different name, but not before. At the same time, it falls to the legal order to spell out in further detail what restrictions may be placed on the power to take. Restrictions, in particular, appear implicit in the very notion of taking something, no less so when the taker is the government or someone that the government endorses. Indeed, in the power to take was unrestricted, how could taking something be any different from using something, for a while, while waiting for the next taker to come along? In particular, the idea that someone might have occasion to resist an act of taking, and may or may not have good grounds for doing so, appears fundamental to our intuitions concerning the notion itself.

But how should we approach the question of legitimacy of takings, and what conceptual categories can we benefit from when doing so? In this thesis, I aim to make a contribution to this question. I will focus on a special case, namely the so-called economic development takings, when government sanctions the taking of property in order to further economic development. 

My primary interest lies in the legal questions that arise, not the overarching philosophical reflections that these might give rise to. However, I believe these introductory remarks fit my chosen topic well. It is especially noteworthy that in the US, the infamous case of {\it Kelo} did in fact shed light on the proposition that vastly different perspectives on property may be brought together by shifting focus to the notion of a taking. In relation to {\it Kelo}, in particular, different commentators came together in a shared scepticism towards the legitimacy of economic development takings.

Moreover, it is worth noting that this scepticism had only a very limited basis in US law, as the {\it Kelo} decision itself did not appear particularly controversial to property lawyers. Hence, when the response was overwhelmingly negative, from both sides of the political spectrum, it seems that people were responding to a deeper notion of what counts as a legitimate act of taking. This response, moreover, does not appear to have been primed solely by the prevailing legal order or political sentiment. It also seemed to involve certain pre-legal notions of legitimacy of takings.

Such notions are surely worthy of consideration, also from legal scholars. Moreover, they point to the possibility that the response to {\it Kelo} and, more generally, the category of an economic development taking, is relevant also outside the context of US law. This claim is by no means self-evident, an observation that brings me to the first key focus point that I will explore in this thesis.

\section{Economic Development Takings as a Conceptual Category}

%This thesis investigates the category of property interferences that are known as {\it economic development takings} in the US. This category came to prominence only quite recently, following the influential {\it Kelo} case.\footnote{See \cite{kelo05}.} 
Economic development takings is a much discussed category in the US, but has so far failed to make a similar impact in other jurisdictions. However, the influence of the US debate is now beginning to show elsewhere, including in Europe.\footnote{See, e.g., \cite{verstappen14}.}

A clear definition of economic development takings is largely missing in the literature. Rather, scholarship on these kinds of takings rests on an intuitive understanding of the term, firmly based on the US jurisprudence from which it first arose. At its core, the economic development taking is characterised by a commercial purpose, meaning that a commercial interested party, often private, stands to gain a significant financial benefit from compulsorily acquiring private property. This financial motivation for the taker contrasts with the public spirited motivation of the executive or legislative body that grants permission to use compulsion; the (stated) intention of economic development takings is to promote public interests, not to bestow commercial benefits on particular parties.

In my opinion, the tension between public interest and commercial gain in property interference is of general interest in any system of government that combines a market-based economy with wide state powers over the use and distribution of property. Hence, I believe the notion of an economic development taking makes sense to study also outside of the US context wherein it arose. In this thesis, I set out to argue for this claim in more depth. 

In the first part of my thesis, I do so from a theoretical point of view, by arguing that the category arises naturally already at the theoretical level, provided one chooses a suitable theoretical framework for reasoning about takings and property. Moreover, I set out to distil some lessons from the US debate and its history. In addition, I briefly assess the status of economic development takings in Europe, where takings that benefit commercial interests are often not recognised as raising any special questions of legal relevance. 

I argue that this is a shortcoming of the narrative of property protection in Europe, and I also suggest that the concept of an economic development taking would in fact fit well with jurisprudential developments at the ECtHR, which stresses both the need for contextual assessment and attention to possible systemic imbalances in the expropriation practices of member states.

At the same time, I note that in the US, most work on economic development takings has been anchored in the so-called ``public use'' requirement of the Fifth Amendment. Indeed, some authors argue that economic development takings are impermissible already because taking property for development cannot ever be said to constitute a ``public use'' of the property. Moreover, even scholars who reject this view tend to agree that the public use of a taking is less obvious, and should be subjected to more intense judicial scrutiny, in economic development cases.

Interestingly, requirements similar to the public use test are found in many jurisdiction, in various guises, e.g., in rules referring to the need for a {\it public interest} or a {\it public purpose} for takings. On this basis, interesting comparative work has been carried out on the basis of the idea that such a requirement is at the core of the legitimacy issue that arises for economic development takings.

In this thesis, I challenge this perspective. In fact, I think the best way to do so is to reconsider the history of the public use debate itself, as documented by case law in the US. In Chapter X, I offer a new perspective on this, by arguing that more attention should be paid to the fact that the state courts which developed the public use test in the 19th century adopted a highly contextualised approach that did not seek to locate any definite meaning of ``public use'' as a general concept. Rather, the public use test was used as an expedient way of subjecting various acts of taking to a concrete fairness assessment, in the hope that courts might help deliver justice in cases when the takings power appeared to have been used in a imbalanced manner. The purpose of the public use test, I argue, has always been to prevent takings that bestow a disproportional disadvantage on certain stakeholders, to the expense of others.

But here, the words ``public use'' are of little interest. What is more important is how to best achieve proportionality. It is by no means clear that continued focus on public use or similar phrases is the best way to go. At the very least, alternatives should also be considered. This brings me to the second focus point of my thesis.

\noo{

So far, the study of such takings has mainly been carried out by US lawyers, who asses it against the Fifth Amendment of the US constitution. This work has attracted some attention elsewhere, but the category of an economic development taking is by no means a universal category of legal analysis.

Perhaps it should be? This is the first main question that I address in this thesis. But has so far not made much of an impact in other jurisdictions, 

Hence, the somewhat counter-intuitive  Hence, one of the ultimate 
not, in particular, a straightforward 

Clearly, the most crucial question that faces any act of taking, government sanctioned or otherwise, is how we should approach objections against it.


To bring about economic development. 
Indeed, the notion of an illegitimate taking, regardless of whether it is seen as an affront to property or at it's core, certainly seems to have stirred the imagination of most property theorists.

More concretely, 


Perhaps a possible route to a better understanding of property goes by way of the study of takings, and th

re, incidentally, is also where we find an apparent sense of commonality between radically different accounts of property and its nature. It is worth noting, in particular, that 

More generally, it seems that people regularly engage in reasoning that seeks to determine what counts as legitimate and illegitimate ways of acting on objects in the material world. Perhaps it is in the midst of such judgements, is also where property as a concept plays a world.

Hence, to understand property as a concept, perhaps a viable route towards progress is to examine its negation, aiming, if nothing else, for a negative definition in terms of legitimate and illegitimate acts involving objects in the material world. In fact, such a more modest approach forces us immediately to recognise certain 

Luckily, as lawyers, we rarely have to worry about the {\it nature} of property, at least not in the philosophical sense of the word. Instead, we can focus on its {\it function}, in the legal system within which we operate. Still, the ``moderate'' and pragmatic view of property that lawyers tend to adhere to might not be very satisfying, particularly not when we are moving to the margins of the legal order, by considering hard cases that raise questions of policy and require novel normative assessments. In such circumstances, the pragmatic stance on property - as taught in law school and applied in courts - can sometimes appear bland, even vacuous, unable to accommodate solutions to genuinely difficult legal problems. These are the cases when received wisdom breaks down, the cases when logic -- or, more generally, {\it reason} -- must be called on to fill the gap left by experience, to paraphrase the famous words of Holmes.\footnote{Holmes quote and critique.}

Of course property is a social construction, a construction of law, but even so, we continue to question its {\it nature}, based on the implicit understanding that there is something more to property than the legal fictions that are used to package it in our legal order.
}

\section{The Democratic Deficit of Takings Law}

%which link the question of their legitimacy to the public use test prescribed in the Fifth Amendment of the US Constitution.
I am not the first to challenge the traditional narrative that surrounds economic development takings. Indeed, some US authors have now begun to argue forcefully that increased judicial scrutiny of the public use requirement is neither a necessary nor a sufficient response to concerns about the legitimacy of commercially motivated takings. Instead, these authors argue that the takings procedure as such is not appropriately designed to deal with commercial incentives on the taker side.

This claim has in turn led to procedural proposals for takings law reform, most notably Professors Heller and Hills' article on Land Assembly Districts and Professor Hellavi and Lehvi's article on Special Purpose Development Companies. Both of these works propose novel institutions for collective action and self-governance, to replace (parts of) the traditional takings procedure, especially in cases where the taker has commercial incentives. 

Towards the end of Chapter x, I examine these proposals in depth. I raise several objections against the details of the institutional arrangements proposed, particularly with regards to their likely effectiveness. It seems, in particular, that both proposals fail to recognise the full extent to which prevailing regulatory frameworks concerning land use and planning would have to be reformed in order to make their proposals work.

At the same time, I argue that these institutional proposals are extremely useful in that they point towards a novel way to frame the issue of legitimacy in takings law. In particular, I explore the hypothesis that traditional procedural arrangements surrounding takings suffer from a democratic deficit, a particularly powerful source of discontent in economic development cases.

This idea is the second key focus point of my thesis. In the first part, I approach it from a theoretical point of view, by exploring the notion of {\it participation} and its importance to the issue of legitimacy, particularly in the context of economic development. It seems, in particular, that {\it exclusion} could be a particular worrying consequence of certain kinds of economic development takings, namely those that lack democratic legitimacy in the local community where the direct effects of the taking are most clearly felt.

I believe this to be a promising hypothesis, but also one that cannot be satisfactorily examined by theoretical arguments alone. Hence, to explore it in more depth, I go on to study it from within the context of a specific jurisdiction, by offering a detailed case study of takings of Norwegian waterfalls for the purpose of hydropower development. This brings me to the second part of my thesis, which in turn consists of two main themes, where the latter aims to bring me back towards a more general setting, by delivering some recommendations for how best to deal with economic development takings.

\noo{
analysis, which must by necessity 

link the idea of the democratic deficit with theoretical work. on the category of economic development takings. In particular, I explore the notion of participation, to  and arguing that it is key to understanding common discontents that arise in for-profit taking situations. 

, by arguing that the recognition of this as a special category is closely related to a shift of focus towards procedural legitimacy. 

Building on this perspective, 

  that legal scholars and policy makers should address more actively.




I believe this suggestion is 


I go on to consider the hypothesis that economic development takings demonstrate that takings law suffer from a {\it democratic deficit}.}


\section{Failures of the Traditional Narrative}

In Norway, the traditional way of thinking about takings law revolves around the issue of compensation. The issue of legitimacy is more or less entirely covered by rules relating to the owners' right to be paid in ``full'' for the loss they suffer as a result of having to give up their property. Indeed, the right to compensation under Norwegian constitutional law is typically regarded as stronger than that which follows from the European Convention of Human Rights. 

In so far as an owner has grievances that are directed at the act of taking as such, not the level of compensation to be paid, takings law has very little to offer. The owner is left with the possibility of arguing his point on the basis of general administrative law, which gives little or no special consideration to property and acts of taking. 

Through my case study, I present a detailed analysis of how this traditional narrative functions in relation to takings for hydropower development, when energy companies seeks to expropriation water rights from local farmers in order to profit commercially. My assessment focuses on cases when the farmers who lose their water rights wish to {\it participate} in economic development, by carrying out their own hydropower projects. This type of farmer-led hydropower development is becoming increasingly common in Norway.

Interestingly, the possibility for this kind of development results from the same legal framework that has now transformed Norwegian energy companies into commercial entities. Both changes are due to liberalisation of the electricity sector in the early 1990s. Hence, while liberalisation empowers local owners to develop hydropower, it also renders taking for hydropower as takings for profit. Unsurprisingly, this has led to tensions that Norwegian courts have had to grapple with in an increasing number of cases.

In their approach to these cases, the courts rely heavily on the traditional narrative, by reconsidering how compensation is calculated when water rights are taken for hydropower. Compensation practices have already changed dramatically. However, there has also been cases when the local owners of these rights have protested the taking as such, claiming that they should be given the opportunity to develop their own resources. These protests have been entirely unsuccessful, as the courts in Norway adopts a stance on legitimacy that is extremely deferential to the executive, provided adequate compensation is paid.

By giving a detailed assessment of a few select cases, I explore the practical consequences of this, while also aiming to bring out how decision-making process surrounding hydropower actually work in Norway. I show, in particular, that local owners risk being completely marginalised, and that new compensation practices have proven inadequate as a means of redressing concerns that arise in this regard. My conclusion is that the case study of Norwegian waterfalls demonstrate clearly the shortcomings of the traditional narrative of legitimacy of takings.

However, the Norwegian system also offers a path towards a possible solution to this problem, one that has also been put to active use in recent years, particularly in cases when farmers themselves aim to undertake hydropower development, but wish to do so against the will of other members of the local community.

\section{A Judicial Framework for Compulsory Participation}

In Norway, the distribution of property rights across the rural population is traditionally highly egalitarian. This has had many consequences for Norwegian society. For one, it meant that the farmers in Norway soon became an active political force, particularly as representative democracy started to gain ground as a form of government in the 19th century. As early as in 1837, the Norwegian parliament was so dominated by farmers that it came to be described as the ``farmer's parliament''.

The Norwegian farmers were often little more than small-holders, and had few priveligues to protect. Hence, they became liberals of sorts, responsible for pushing through important early reforms, such as the abolishment of noble titles and the establishment of semi-autonomous, elected, municipality governments. 

However, the municipality governments were not the first example of local, participatory, decision-making institutions in Norway. Indeed, the highly fragmented ownership of land meant that institutions for land management are among the oldest known in Norway. One of the most important ones exists to this day, namely the {\it land consolidation court}. The final focus point of my thesis consists in an assessment of this institution and its potential as a possible procedural alternative to takings in economic development cases.

Importantly, the land consolidation procedure in Norway is a semi-judicial process that warrants the imposition of {\it compulsory participation} by primary stakeholders in decision-making processes to which they are deemed to owe a contribution. One typical situation when the institution will be invoked involves the management of jointly owned land, where the land consolidation procedure is used to ensure that local owners may reach a joint decision on how to regulate the use of their land, if necessary one that is imposed on them by the land consolidation judges.

However, the judges' power is limited in that they may only impose a measure if the gains are deemed to outweigh the loss for all stakeholders involved. In practice, land consolidation judges often act as mediators, to facilitate a collective decision. Moreover, one of the most common acts of a land consolidation judge is to set up owner's associations, in a manner that institutionally regulates the continued interaction and decision-making among the stakeholders even after the formal consolidation process has concluded.

In Chapter x, I explore this framework in some depth, focusing on its potential as an alternative to exporpriation. This is especially interesting since land consolidation is presently being put to use in order to organise hydropower development. Hence, my case study provides an excellent opportunity for comparing the land consolidation and the takings process, with respect to the overall aim of ensuring development of hydropower on equitable terms. 

Here, I argue, the land consolidation route may be preferable, as it ensured legitimacy through participation. At the same time, the procedure remains effective, since participation is in fact compulsory. I discuss possible objections to the procedure in some depth, but conclude that the continued development of the land consolidation institution provides the best way forward for addressing economic development takings in Norway.

Finally, I compare the institution of land consolidation with the institutional proposals that have been made specifically in the context of the debate on economic development takings. I argue that it compares favourably, both because it comes equipped with in-built judicial safeguards, but also because it has a broader scope. I note, however, that its use as a better alternative to economic development takings is dependent on both political will and an ability to retain key feature even in the presence of new and powerful stakeholders in the consolidation process itself.


\noo{ In the second part of the thesis, I put the theoretical framework to the test by applying it to a concrete case study, namely that of Norwegian hydropower. Following liberalisation of the energy sector in the early 1990s, hydropower is now a commercial pursuit in Norway. Moreover, there is a long tradition for granting energy producers the power to acquire property compulsorily, including the necessary rights to exploit the energy of water, rights that are subject to private property under Norwegian law. This has resulted in tension and controversy, however, as the original owners of these rights, typically local farmers and small-holders, see the commercial potential of hydropower being transferred to other commercial interests, to the detriment of their own, and their communities', interest in self-governance and economic benefit.}

\section{Structure of the Thesis}

My thesis is divided into two parts. In the first, I explore the theoretical basis for the study of economic development takings as a separate category of interference in property rights. My aims in this part are threefold. First, I argue that the category makes sense outside the context of US law and that it is worthy of comparative study, even if it has not yet come to prominence outside the US. I develop this argument theoretically by anchoring it in the social function theory of property. 

This theory, which I argue for as a general template for thinking about takings and property, makes it very natural to operate with a special category for cases when there are significant commercial interests on the taker side. Moreover, I believe it highlights the importance of considering the matter of legitimacy contextually and, at it's ore, as an issue of fairness and proportionality. This appears to be at odds with the standard narrative surrounding economic development takings in the US, which tends to focus on the public use requirement.

The second aim of the first part of my thesis is to explore this tension. I do so by exploring the history of the public use debate in the US, by arguing that contextual assessment was originally at the core of the notion, when legitimacy was adjudicated by state courts. Subsequent developments at the federal level, I argue, has served to change the impression of the test, creating the erroneous impression that the key question is whether or not a taking is for a ``public use'' in some abstract sense that should be pegged down by the law. Rather, the basic question is, and remains, as it has always been, whether or not the action of government is able to strike a fair balance among the interests of the affected shareholders.

The third aim of the first part of my thesis is to propose an alternative way of thinking about legitimacy, that brings out the key issue without getting sidetracked by the notion of a public use. Here, I rely on recent institutional proposals for reform of the takings procedure itself, that are meant to empower owners and local communities by providing a novel template for collective action. I analyse these proposals in some depth, raising some objections and proposing some research questions.

This then brings me to the second part of my thesis, where I study the questions that I distilled in the first part by looking at the case of Norwegian hydropower. Here, my aims are again two-fold. First, I present an analysis of the framework of takings law in Norway, and the narrative of judicial reasoning that surrounds it, exemplified by the case of takings of water rights for hydropower development. I argue that tht the traditional narrative does indeed fail, in much the same way as predicted by the theoretical considerations of the first part of the thesis. 

My second aim is to present first steps towards a possible solution, again based on a concrete assessment of Norwegian law. Here, I present the institution of land consolidation courts, as a possible framework for compulsory participation in land development that can at once provide the participation and the compulsion that appears to be necessary to move towards the goal of economic development without offending against the rights of owners and local communities.

Finally, I offer a conclusion that aims to present in short form the main problems investigated, and the solution concepts offered, through the course of this thesis.


