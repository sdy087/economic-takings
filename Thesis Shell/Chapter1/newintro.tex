\chapter{Introduction and Summary of Main Themes}\label{chap:intro}

\begin{quote}
Thieves respect property; they merely wish the property to become their property that they may more perfectly respect it.\footnote{G.K. Chesterton, {\it The man who was Thursday: A nightmare}.}
\end{quote}
%
%A human being needs only a small plot of ground on which to be happy, and even less to lie beneath. %\footnote{Johan Wolfgang von Goethe, {\it The sorrows of young Werther and selected writings}.}
%\end{quote}
%“That's what makes it ours - being born on it, working on it, dying on it. That makes ownership, not a %paper with numbers on it.”
%― John Steinbeck, The Grapes of Wrath 
%
%
%s
\section{Property Lost; Takings and Legitimacy}

%Judging from academic literature on the subject, property is a difficult and often paradoxical concept,
As a concept, property has been something of a problem child for western philosophy. Time and again it has demonstrated its recalcitrant nature. It has proven, in particular, that it has an inherent tendency to stir up divergences that break out of the realm of conceptual analysis to enter the realm of politics, often in a way that makes it difficult to continue a meaningful and inclusive academic debate.

From the extreme antagonism directed at it by radical Marxists to the evangelical praise bestowed on it by libertarians, there is no shortage of politically charged accounts of what actually property is, not to mention what it {\it should} be, assuming, of course, that it is entitled to exist at all. 
Moreover, there appears to be little room for rapprochement between many leading strands of thought. Indeed, one is sometimes left with the impression that different philosophical theories of property tend to diverge largely due to sympathies, rather than differences based on reasoned argument. %Indeed, one may argue that different ideas of property, practical and theoretical, are behind most, if not all, the major conflicts and confrontations that have shaped the society in which we live.
%Responding to this, some prominent philosophers have taken the view that property is not a concept suitable for philosophical study at all.

Indeed, some philosopher have argued that property is a lost cause, not suitable for conceptual analysis at all. Rather, they have suggested that property is best taken as a pragmatic and contingent derivative of other notions, such as the social order, or, on a normative account, {\it justice}. 

Lawyers might object to such a claim at first sight, but actually, they rarely need to consider the philosophical underpinnings of the notion of property that they find entrenched in the law. Indeed, when grappling with hard cases there is little doubt that legal professionals are soon forced to abandon conceptual clarity in favour of considerations based on value judgements, politics and expediency.

Legal scholars, for their part, are usually content with theories of property that remain largely descriptive, settling for the more modest aim of exploring how best to think of property given the prevailing legal order, rather than trying to come up with theories to explicate its nature as a pre-legal concept.

Hence, even the legal philosopher may well end up loosing sight of property as a concept, coming to doubt that there even is such a thing. As one prominent UK scholar put it, property sometimes appears to be floating in ``thin air''.

Still, in certain situations, we can record what seems to be empirical evidence to support the claim that humans reflect a working {\it primitive} notion of property, one which arguably pre-exists any particular social arrangements used to mould property as a socio-political category of law. Most notably, humans, as well as many other species of animal, appear to have an innate ability to recognise {\it thievery}, the taking of property by someone who is not entitled to do so.

Indeed, the famous dictum ``property is theft'', may be more than a flippant and seemingly self-contradictory comment on the origins of inequality. It might point to a possible alternative direction for investigating the nature of property as such, as a concept that emerges from a more basic distinction between legitimate and illegitimate acts of taking, broadly construed. It seems quite tempting, after all, to describe a person's property as that which has been taken for them by legitimate means, which may not be taken from them without due process.

Interestingly, while the abstract notion of property has arguably received more than its fair share of attention from disciplines other than law, the common-sense notion of a taking has not received much academic attention outside of the legal community. No great philosophical debates have revolved around this notion, and no chasms has opened as to the correct way to understand it. Moreover, legal scholars rarely attempt to define takings as an abstract notion, preferring instead to regard it as a derivative of the legal order surrounding property. Here, the pragmatic and often jurisdiction-bound perspective of the lawyer appears to reign supreme.

This is problematic, since the notion of property itself is so contested that it might not provide a secure foundation for thinking about takings. But it also suggests an interesting possibility; perhaps studying takings is a path towards a better understanding of property as well? After all, this is where vastly different accounts of property do seem to share at least an important common point of reference.
It is my hope that this thesis will shed some light on this idea.

However, I should make clear at the very outset that I will limit myself to the study a certain kind of taking, namely that which is implemented, or at least formally sanctioned, by a government. In legal language, especially as developed in the US, such acts of government are referred to as takings {\it simpliciter}, while talk of other kinds of ``takings'' require further qualification, e.g., in case of contract, theft, tax or occupation. This in itself might be cause for reflection as to the ideological commitments inherent in legal language. Moreover, it brings the issue of legitimacy to the forefront in an instructive way.

We are reminded, in particular, that under the rule of law, taking is not the same as theft. Rather, the default assumption is that the takings that take place are legitimate. If they are not, we may call them by a different name, but not before. At the same time, it falls to the legal order to spell out in further detail what restrictions may be placed on the power to take. 

Indeed, restrictions appear implicit in the very notion of taking something. If the power to take was unrestricted, how could one distinguish the act of taking something from the act of putting something to use, for a while, waiting for the next taker to come along? In particular, the idea that someone might have occasion to resist an act of taking, and may or may not have good grounds for doing so, appears fundamental to our intuitions concerning the notion itself.

But how should we approach the question of legitimacy of takings, and what conceptual categories can we benefit from when doing so? In this thesis, I aim to make a contribution to this question. I will focus on a special case, namely the so-called economic development takings, when government sanctions the taking of property in order to further economic development.

My primary interest lies in the legal questions that arise, not the overarching philosophical reflections that these might give rise to. However, I believe my introductory remarks here serve to at least sketch an interesting broader perspective on my topic, one which I should like to explore in more detail in the future. I have found it striking, in particular, how recent case law from the US shows that vastly different perspectives on property may indeed come together when focus is shifted towards the legitimacy of takings.

The best example is the case of {\it Kelo}, which propelled the category of economic development takings forward, first to the political scene, then by causing a surge of academic work by US scholars. 
The {\it Kelo} case concerned a house that was taken by the government in order to accommodate private enterprise, more specifically the construction of new research facilities for Pfizer, the multi-national pharmaceutical company.

The homeowner, Suzanne Kelo, protested the taking on the basis that it served no public use and was therefore illegitimate. The Supreme Court eventually rejected her arguments, but this decision created a backlash that appears to be unique in the history of US jurisprudence. In their mutual condemnation of the {\it Kelo} decision, commentators from very different ideological backgrounds came together in a shared scepticism towards the legitimacy of economic development takings.

This is particularly noteworthy in light of the fact that their scepticism had a very limited basis in US law, as the {\it Kelo} decision itself did not appear particularly controversial to most property lawyers. Hence, when the response was overwhelmingly negative, from both sides of the political spectrum, it seems that people were responding to a deeper notion of what counts as a legitimate act of taking.

The critical response to {\it Kelo}, specifically, did not appear to have been primed by the prevailing legal order. It may have been a reflection of political sentiment, but as such it arguably also involved pre-legal notions pertaining to legitimacy. Simply stated, people from across the political spectrum simply found the outcome {\it unfair}.

This phenomenon is surely worthy of consideration from legal scholars, and in the US it has received plenty of it. In the US, it is now hard or impossible to deny that cases such as {\it Kelo} do at least belong to a separate category of takings that deserves to be addressed as such. Moreover, after {\it Kelo}, most US states have passed some sort of legislation to limit economic development takings, in a direct response to the controversy following the {\it Kelo} case. 

I believe the fact that this was largely a popular movement should be noted. In particular, I think it suggests the likely relevance of economic development taking as a legal category outside the context of US law. There are certainly significant differences between takings law and practice in the US compared to many other jurisdictions, e.g., in Europe. However, the backlash of {\it Kelo}, particularly the manner in which public opinion diverged so dramatically from the outcome dictated by established case law, suggests the transformational potential inherent in the category of economic development takings itself.

As soon as critical attention is directed at the special issues that arise in cases such as {\it Kelo}, it might well be that people will have a tendency to judge the issue of fairness similarly, irrespectively of divergences in the surrounding legal framework used to deal with such cases. After all, it seems only natural that characterising certain kinds of takings as takings that primarily seeks to bestow profit on a commercial entity will lead to a changed perception of their legitimacy.

The question, then, becomes to what extent one may appropriately speak of economic development takings in this way. Here, I believe the first important step is to acknowledge at least the {\it potential} for characterising cases of taking for economic development in this fashion. The possibly problematic nature of such takings, pertaining to the commercial interests on the side of the taker, should at least be acknowledge as a relevant dimension along which to asses cases.

This claim is by no means self-evident. For instance, it seems that many European jurisdictions implicitly reject such a perspective, already by failing to recognise that the category of economic development takings can be a useful anchor for reasoning about legitimacy. This brings me to the first key contribution of this thesis, which is a detailed argument in favour of economic development takings as a conceptual category for legal reasoning.

\section{Economic Development Takings as a Conceptual Category}

%This thesis investigates the category of property interferences that are known as {\it economic development takings} in the US. This category came to prominence only quite recently, following the influential {\it Kelo} case.\footnote{See \cite{kelo05}.} 
While the category of economic development takings does not yet appear to be well established outside of the US, the influence of the US debate is beginning to show elsewhere, including in Europe.\footnote{See, e.g., \cite{verstappen14}.} It is a problem, however, that the exact meaning of the category may differ depending on who you ask. It is quite common, for instance, to speak of ``private'' takings, at first sight often more or less as a synonym to economic development takings. But there are some differences here that I think one should keep in mind, especially if one is aiming for conceptual precision.

First, a private taking already carries with it an implicit pointer to a lack of legitimacy, at least in jurisdictions that explicitly single out {\it public} interests as those that must be used to justify takings. Speaking of an economic development taking, on the other hand, does not appear to carry with it any such implicit commitment, quite the contrary. Indeed, if economic development takings are in need of special scrutiny, the reason cannot be simply that it involves ``private'' interests. After all, many would agree that economic development itself is usually in the public interest.

A second difference between private takings and economic development takings is that the former notion is far easier to define. In fact, I think it is {\it too} easy. It is very tempting, in particular, to simply say that a private taking occurs whenever the legal person taking title to the property in question is a private company or individual. But this is too simple. It might well be, in particular, that a private organisation, for instance a tightly regulated charity, functionally mimics the quintessential ``public'' takers. On the other hand, it is equally easy to imagine public bodies that are functionally equivalent to private enterprises. Or consider a taking that benefits a publicly owned limited liability company. According to the simple definition of a private taking, this would not qualify, even if the interests involved are completely or predominantly of a private-law nature, directed at maximising profit, rather than providing a public service.

For economic development takings, on the other hand, we face a different problem. Here, a clear definition appears to be entirely missing in the literature. Rather, scholarship on these kinds of takings rests on an intuitive understanding of the term, firmly based on the US jurisprudence from which it first arose. At its core, however, we may safely say that the reason for paying particular attention to the cases classified as economic development takings has something to do with the strong economic, often commercial, incentives that persist on the taker side. Such incentives, more specifically, can serve to sow the seeds of doubt as to whether or not due regard has been had to the interests of owners and directly affected local communities.

In my opinion, this concern is relevant also when the economic incentives in question are of a non-commercial nature. However, cases when the taker acts as a profit-maximiser on a competitive market are certainly likely to be of special concern. This is particularly relevant in economic systems such as those seen in the west, where the steadily increasing influence of public-private partnerships cause a generally blurring of lines between private and public sectors. 

In such societies, economic development takings will almost always be characterised by a commercial incentive, meaning that a commercial interested party, often private, stands to gain a significant financial benefit from compulsorily acquiring private property. This financial motivation for the taker contrasts with the public spirited motivation of the executive or legislative body that grants permission to use compulsion; the (stated) intention of economic development takings is to promote public interests, not to bestow commercial benefits on particular parties. In my opinion, the  importance of economic development as a category of takings is that it helps us flag those cases when this contrast is so strong as to suggest that we further question the legitimacy of the undertaking as a whole.

If the decision-maker fails to identify concrete public interests, but must rely on a vague notion such as economic development, this, in particular, should be cause for increased scrutiny. This, I believe, is an observation of generally validity, also outside the context of US law. At the very least, the tension between public interest and commercial gain in property interference is of general interest in any system of government that combines a market-based economy with wide state powers over the use and distribution of property. The question becomes how one should reason about this tension in a meaningful way, to analyse economic development takings in a manner that is suited to yielding legally relevant insights.

Later in this thesis, I will devote much attention to this question. First I will do so from a theoretical point of view, by first arguing that the category of economic development takings arises naturally already at the theoretical level, provided one chooses a suitable theoretical framework for reasoning about takings and property. Following up on this, I set out to distil some general lessons from the US debate and its history. In addition, I briefly assess the status of economic development takings in Europe, where takings that benefit commercial interests are often allowed to pass without raising special questions, and where the legal relevance of the category of economic development takings may still be called into doubt.

In fact, I argue that this is a shortcoming of the narrative of property protection in Europe, and I also suggest that the concept of an economic development taking would in fact fit well with jurisprudential developments at the ECtHR, which stresses both the need for contextual assessment and attention to possible systemic imbalances in the expropriation practices of member states.

%Similar requirements, interestingly, are found in many other jurisdictions, and is also found in the property clause in the European Convention of Human Rights (ECHR). But in Europe, it is often understood very loosely, as a clause that places little or no practical limit on the state's taking power. The contrast with the suggestions that are now being considered by US scholars is very great.

In the US, most work on economic development takings has been anchored in the so-called ``public use'' requirement of the Fifth Amendment. In fact, some US scholars argue that economic development takings are impermissible already because taking property for development cannot ever be said to constitute a ``public use'' of the property. Moreover, even scholars who reject this view tend to agree that the public use of a taking is less obvious, and should be subjected to more intense judicial scrutiny, in economic development cases.

Interestingly, requirements similar to the public use test are found in many jurisdiction, in various guises, e.g., in rules referring to the need for a {\it public interest} or a {\it public purpose} for takings. On this basis, interesting comparative work has been carried out on the basis of the idea that such a requirement is at the core of the legitimacy issue that arises for economic development takings.

In this thesis, I challenge this perspective. I do so by first reconsidering the history of the public use debate itself, as documented by case law in the US. I argue, in particular, that more attention should be paid to the fact that the state courts that originally set out to develop public use tests in the 19th century adopted a highly contextualised approach. Importantly, these courts where largely not bothered by the fact that they could not pin down any definite and consistent meaning of ``public use'' as a general concept. 

Rather, the public use test was simply used as an expedient way of subjecting various acts of taking to a concrete fairness assessment, in the hope that local courts might help deliver corrective justice in cases when the takings power appeared to have been used in an objectionable manner. In this way, the original purpose of the public use test was tailored towards setting up a framework for judicial review that appears quite similar to how the European Court of Human Rights (ECtHR) currently choose to approach cases dealing with property.

The jurisprudence at the ECtHR typically directs focus away from the question of whether the aim of a taking is legitimate in itself towards the more contextualised question of whether or not the interference is {\it proportional} given the circumstances. This, I argue, is also how the public use test was also originally used by state courts in the US, before the issue of legitimacy turned federal and became subject to a more abstract form of assessment, leading eventually to a tradition for passive deference that gave rise to {\it Kelo}.

In fact, as soon as the issue of proportionality has been flagged as the primary question, it is not clear that the words ``public use'' are of much interest at all. Hence, my conceptual assessment can be summarised by the following two propositions. First, that the notion of an economic development taking, as developed in the US, is a useful addition for thinking about the legitimacy of takings, in any jurisdiction that aims to place meaningful restrictions on the takings power. Second, that the current focus on the notion of a ``public use'', which is supposed to provide the desired protection against transgressions, is largely misguided. At the very least, I believe alternatives should also be considered. This brings me to the second focus point of my thesis.

\noo{

So far, the study of such takings has mainly been carried out by US lawyers, who asses it against the Fifth Amendment of the US constitution. This work has attracted some attention elsewhere, but the category of an economic development taking is by no means a universal category of legal analysis.

Perhaps it should be? This is the first main question that I address in this thesis. But has so far not made much of an impact in other jurisdictions, 

Hence, the somewhat counter-intuitive  Hence, one of the ultimate 
not, in particular, a straightforward 

Clearly, the most crucial question that faces any act of taking, government sanctioned or otherwise, is how we should approach objections against it.


To bring about economic development. 
Indeed, the notion of an illegitimate taking, regardless of whether it is seen as an affront to property or at it's core, certainly seems to have stirred the imagination of most property theorists.

More concretely, 


Perhaps a possible route to a better understanding of property goes by way of the study of takings, and th

re, incidentally, is also where we find an apparent sense of commonality between radically different accounts of property and its nature. It is worth noting, in particular, that 

More generally, it seems that people regularly engage in reasoning that seeks to determine what counts as legitimate and illegitimate ways of acting on objects in the material world. Perhaps it is in the midst of such judgements, is also where property as a concept plays a world.

Hence, to understand property as a concept, perhaps a viable route towards progress is to examine its negation, aiming, if nothing else, for a negative definition in terms of legitimate and illegitimate acts involving objects in the material world. In fact, such a more modest approach forces us immediately to recognise certain 

Luckily, as lawyers, we rarely have to worry about the {\it nature} of property, at least not in the philosophical sense of the word. Instead, we can focus on its {\it function}, in the legal system within which we operate. Still, the ``moderate'' and pragmatic view of property that lawyers tend to adhere to might not be very satisfying, particularly not when we are moving to the margins of the legal order, by considering hard cases that raise questions of policy and require novel normative assessments. In such circumstances, the pragmatic stance on property - as taught in law school and applied in courts - can sometimes appear bland, even vacuous, unable to accommodate solutions to genuinely difficult legal problems. These are the cases when received wisdom breaks down, the cases when logic -- or, more generally, {\it reason} -- must be called on to fill the gap left by experience, to paraphrase the famous words of Holmes.\footnote{Holmes quote and critique.}

Of course property is a social construction, a construction of law, but even so, we continue to question its {\it nature}, based on the implicit understanding that there is something more to property than the legal fictions that are used to package it in our legal order.
}

\section{The Democratic Deficit of Takings Law}

%which link the question of their legitimacy to the public use test prescribed in the Fifth Amendment of the US Constitution.
I am not the first to challenge the traditional narrative that surrounds economic development takings. Indeed, some US scholars have now begun to argue forcefully that increased judicial scrutiny of the public use requirement is neither a necessary nor a sufficient response to concerns about the legitimacy of commercially motivated takings. Instead, these authors point out that the takings procedure as such does not seem able to appropriately deal with commercial incentives on the taker side.

This has been accompanied by procedural proposals for takings law reform, most notably Professors Heller and Hills' article on Land Assembly Districts and Professor Hellavi and Lehvi's article on Special Purpose Development Companies. Both of these works propose novel institutions for collective action and self-governance, to replace (parts of) the traditional takings procedure, especially in cases where the taker has commercial incentives.

By examining these proposals in some depth, I arrive at several objections against the details of the particular institutional arrangements proposed, particularly with regards to their likely effectiveness. It seems, in particular, that both proposals fail to recognise the full extent to which prevailing regulatory frameworks concerning land use and planning would have to be reformed in order to make their proposals work.

At the same time, I argue that these novel institutional proposals are extremely useful in that they point towards a novel way to frame the issue of legitimacy in takings law. In particular, I explore the hypothesis that traditional procedural arrangements surrounding takings suffer from a democratic deficit, a particularly powerful source of discontent in economic development cases.

This idea is the second key focus point of my thesis. First, I approach it from a theoretical point of view, by exploring the notion of {\it participation} and its importance to the issue of legitimacy, particularly in the context of economic development. It seems, in particular, that {\it exclusion} could be a particular worrying consequence of certain kinds of economic development takings, namely those that lack democratic legitimacy in the local community where the direct effects of the taking are most clearly felt.

I believe this to be a promising hypothesis, and I back it up by considering the social function theory of property and the notion of human flourishing which has recently been proposed as a normative guide for reasoning about property interests. I pay particular attention to the importance of communities that has been highlighted in recent work, as a way to bridge the gap between individualistic and collectivist ideas about fairness in relation to property.

I take this a step further, by arguing that a focus on communities naturally should bring institutions of local democracy to the forefront of our attention. The role that property plays in facilitating democracy has been emphasised before by other scholars, and I think it has considerable merit. However, I also argue that it is important to resist the temptation of viewing its role in this regard through an individualistic prism. It is especially important to take into account additional structural dimensions that may supervene on both property and democracy, such as tensions between the periphery and the centre, the privileged and the marginalised, as well as between urban and rural communities.

It is especially important, I think, to appreciate the effect takings can have on local democracy. For one, excessive taking of property from certain communities might be a symptom of failures of democracy as well as structural imbalances between different groups and interest. But even more worrying are cases when the takings themselves, brought on by a commercially motivated rationale, appears to undermine the authority of local arrangements for collective decision-making and self-governance. This dimension of legitimacy, in particular, is one that I devote special attention to throughout this thesis.

I also believe, however, that it is hard to get very far with this sub-theme through theoretical arguments alone. Hence, to explore it in more depth, I go on to assess it from an empirical angle, by offering a detailed case study of takings of Norwegian waterfalls for the purpose of hydropower development. This case study, in turn, will allow me to cast light on two further key themes, that I now introduce. %This brings me to the second part of my thesis, which in turn consists of two main themes, where the latter aims to bring me back towards a more general setting, by delivering some recommendations for how best to deal with economic development takings.

\noo{
analysis, which must by necessity 

link the idea of the democratic deficit with theoretical work. on the category of economic development takings. In particular, I explore the notion of participation, to  and arguing that it is key to understanding common discontents that arise in for-profit taking situations. 

, by arguing that the recognition of this as a special category is closely related to a shift of focus towards procedural legitimacy. 

Building on this perspective, 

  that legal scholars and policy makers should address more actively.




I believe this suggestion is 


I go on to consider the hypothesis that economic development takings demonstrate that takings law suffer from a {\it democratic deficit}.}


\section{Failures of the Traditional Narrative}

In Norway, the traditional way of thinking about legitimacy of takings is grounded in the notion that owners are entitled to monetary compensation. The law of expropriation clearly reflects the importance attributed to this idea; the constitution itself stipulates that owners have a right to be paid in ``full'' for the loss they suffer as a result of giving up their property. Consequently, the right to compensation in Norway is generally regarded as stronger than in many other jurisdictions, including those that adhere to the minimal standard imposed by the ECHR.

On the other hand, the story of legitimacy more or less begins and ends with the issue of compensation. Hence, if an owner has grievances that are directed at the act of taking as such, not the amount of money they receive, takings law has very little to offer. In fact, it does not appear to have anything at all to offer, that does not already follow from general principles of administrative law. The owner can certainly argue that the decision to authorise the taking was in breach of procedural rules, or grossly unreasonably, but the chance of succeeding by making such arguments are slim, arguably no higher than in administrative cases that do not involve interference with property rights.

This narrative of legitimacy is not unique to Norway. It seems that in Europe, unlike in the US, the issue of legitimacy is often seen as predominantly concerned with the issue of compensation. In particular, the jurisprudence at the ECtHR is typically focused on compensatory issues. Moreover, while many constitutions of Europe, including the Norwegian, include public interest clauses, the courts make little or no use of these when adjudicating takings complaints. In the words of the ECtHR, the member states enjoy a ``wide margin of appreciation'' when it comes to determining what counts as a public interest.

Through my case study, I present a detailed analysis of how this traditional narrative actually plays out in Norway, in relation to takings for hydropower development. Such takings form an interesting sub-class because they are clearly economic development takings, in the most interesting sense of the word. Since the early 1990s, the hydropower sector in Norway has been deregulated, so the hydropower companies, to which the government may grant permission to expropriate, are now predominantly commercial entities. Moreover, the property that they seek to takes is not merely some ancillary rights that they need to develop the country's resources. In Norway, the right to harness the power of water is a private right, under a riparian system that is otherwise quite similar to that found in the UK.

Hence, the primary right that the energy companies tend to take is the right to harness the natural resource itself, a right that is typically held jointly by groups of small-holders and local farmers. In effect, the established hydropower sector in Norway is entirely dependent on taking natural resources from local communities by use of compulsion, with the help of government, in order to exist. Since deregulation, however, not only have energy companies been reorganised as limited liability commercial companies, local owners have also begun to make use of their right to harness water power by undertaking their own hydropower projects. 

As a result, local owners now regularly protest expropriation of their rights on the grounds that they wish to {\it participate} in economic development, by carrying out alternative development projects, or even by cooperating with the established energy companies who wish to take their water rights. Hence, while liberalisation empowers local owners to develop hydropower, it also renders taking for hydropower as takings for profit. Unsurprisingly, this has led to tensions that Norwegian courts have had to grapple with in an increasing number of cases.

In their approach to these cases, the courts rely heavily on the traditional narrative, by reconsidering how compensation is calculated when water rights are taken for hydropower. Compensation practices have already changed dramatically. However, there has also been cases when the local owners of these rights have protested the taking as such, claiming that they should be given the opportunity to develop their own resources. These protests have been entirely unsuccessful, as the courts in Norway adopts a stance on legitimacy that is extremely deferential to the executive, provided adequate compensation is paid.

In my case study, I start by presenting the legal framework, including a short excursion into legal history, before I give a detailed assessment of a few select cases. This concrete empirical approach will allow me to explore the practical consequences of the current takings narrative, while also aiming to bring out how decision-making process surrounding hydropower work in practice. My main finding is that local owners risk being marginalised by the current regulatory framework, and that new compensation practices have proven inadequate as a means of redressing concerns that arise in this regard. My conclusion is that the case study of Norwegian waterfalls demonstrate concretely the shortcomings of the traditional narrative of legitimacy of takings.

However, I also believe that Norwegian law may offer a possible path towards a solution to this problem, one that has also been put to active use in recent years, particularly in cases when farmers themselves aim to undertake hydropower development, but wish to do so against the will of other members of the local community. This brings me to the second key theme of my case study.

\section{A Judicial Framework for Compulsory Participation}

In Norway, the distribution of property rights across the rural population is traditionally highly egalitarian. This has had many consequences for Norwegian society. For one, it meant that the farmers in Norway soon became an active political force, particularly as representative democracy started to gain ground as a form of government in the 19th century. As early as in 1837, the Norwegian parliament was so dominated by farmers that it came to be described as the ``farmer's parliament''.

The Norwegian farmers were often little more than small-holders, and had few privileges to protect. Hence, they became liberals of sorts (although also known for their fiscal conservatism). The farmers as a class were responsible for pushing through important early reforms, such as the establishment of semi-autonomous, elected, municipality governments.\footnote{The farmers were also responsible for abolishing noble titles in Norway. Clearly, they owed no allegiance to the established aristocracy of landed nobility in Europe.}

However, the municipality governments were not the first example of local, participatory, decision-making institutions in Norway. Indeed, the highly fragmented ownership of land meant that institutions for land management are among the oldest known in Norway. One of the most important ones exists to this day, namely the {\it land consolidation court}. The final focus point of my thesis consists in an assessment of this institution and its potential as a possible procedural alternative to takings when compulsion appears to be needed in order to ensure economic development.

Importantly, the land consolidation procedure in Norway is a semi-judicial process that warrants the imposition of {\it compulsory participation} by primary stakeholders in decision-making processes to which they are deemed to owe a contribution. One typical situation when the institution will be invoked involves the management of jointly owned land, where the land consolidation procedure is used to ensure that local owners may reach a joint decision on how to regulate the use of their land, if necessary one that is imposed on them by the land consolidation judges.

However, the judges' power is limited in that they may only impose a measure if the gains are deemed to outweigh the loss for all stakeholders involved. In practice, land consolidation judges often act as mediators, to facilitate a collective decision. Moreover, one of the most common acts of a land consolidation judge is to set up owner's associations, in a manner that institutionally regulates the continued interaction and decision-making among the stakeholders even after the formal consolidation process has concluded.

I explore this framework in some depth, focusing on its potential as an alternative to exporpriation. This is especially interesting since land consolidation is presently being put to use in order to organise hydropower development. Hence, my case study provides an excellent opportunity for comparing the land consolidation and the takings process, with respect to the overall aim of ensuring development of hydropower on equitable terms. 

Here, I argue, the land consolidation route may be preferable, as it ensures legitimacy through participation. At the same time, the procedure remains effective, since participation is in fact compulsory. I discuss possible objections to the procedure in some depth, but conclude that the continued development of the land consolidation institution provides the best way forward for addressing economic development takings in Norway.

Finally, I compare the institution of land consolidation with the institutional proposals that have been made specifically in the context of the debate on economic development takings. I argue that it compares favourably, both because it comes equipped with in-built judicial safeguards, but also because it has a broader scope. I note, however, that its use as a better alternative to economic development takings is dependent on both political will and an ability to retain key feature even in the presence of new and powerful stakeholders in the consolidation process itself.

\noo{ In the second part of the thesis, I put the theoretical framework to the test by applying it to a concrete case study, namely that of Norwegian hydropower. Following liberalisation of the energy sector in the early 1990s, hydropower is now a commercial pursuit in Norway. Moreover, there is a long tradition for granting energy producers the power to acquire property compulsorily, including the necessary rights to exploit the energy of water, rights that are subject to private property under Norwegian law. This has resulted in tension and controversy, however, as the original owners of these rights, typically local farmers and small-holders, see the commercial potential of hydropower being transferred to other commercial interests, to the detriment of their own, and their communities', interest in self-governance and economic benefit.}

\section{Structure of the Thesis}

My thesis is divided into two parts. In the first, I collect my theoretical analysis of economic development takings as a separate category of interference in property rights. My aims in this part are threefold. First, I argue that the category makes sense outside the context of US law and that it is worthy of comparative study, even if it has not yet come to prominence outside the US. I develop this argument theoretically by anchoring it in the social function theory of property. 

This theory, which I argue for as a general template for thinking about takings and property, makes it very natural to operate with a special category for cases when there are significant commercial interests on the taker side. Moreover, I believe it highlights the importance of considering the matter of legitimacy contextually and, at it's ore, as an issue of fairness and proportionality. This appears to be at odds with the standard narrative surrounding economic development takings in the US, which tends to focus on the public use requirement.

The second aim of the first part of my thesis is to explore this tension. I do so by exploring the history of the public use debate in the US, by arguing that contextual assessment was originally at the core of the notion, when legitimacy was adjudicated by state courts. Subsequent developments at the federal level, I argue, has served to change the impression of the test, creating the erroneous impression that the key question is whether or not a taking is for a ``public use'' in some abstract sense that should be pegged down by the law. Rather, the basic question is, and remains, as it has always been, whether or not the action of government is able to strike a fair balance among the interests of the affected shareholders.

The third aim of the first part of my thesis is to propose an alternative way of thinking about legitimacy, that brings out the key issue without getting sidetracked by the notion of a public use. Here, I rely on recent institutional proposals for reform of the takings procedure itself, that are meant to empower owners and local communities by providing a novel template for collective action. I analyse these proposals in some depth, raising some objections and proposing some research questions.

This then brings me to the second part of my thesis, where I study the questions that I distilled in the first part by looking at the case of Norwegian hydropower. Here, my aims are again two-fold. First, I present an analysis of the framework of takings law in Norway, and the narrative of judicial reasoning that surrounds it, exemplified by the case of takings of water rights for hydropower development. I argue that tht the traditional narrative does indeed fail, in much the same way as predicted by the theoretical considerations of the first part of the thesis. 

My second aim is to present first steps towards a possible solution, again based on a concrete assessment of Norwegian law. Here, I present the institution of land consolidation courts, as a possible framework for compulsory participation in land development that can at once provide the participation and the compulsion that appears to be necessary to move towards the goal of economic development without offending against the rights of owners and local communities.

Finally, I offer a conclusion that aims to present in short form the main problems investigated, and the solution concepts offered, through the course of this thesis.

