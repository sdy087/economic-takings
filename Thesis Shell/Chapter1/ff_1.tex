\newcommand{\isr}[1]{{#1}}

\part{Towards a Theory of Economic Development Takings}

\chapter{Property, Protection and Privilege}\label{chap:1}

\begin{quote}
It's nice to own land.\footnote{Donald Trump, as quoted in \cite{booth12}.}
\end{quote}

\begin{quote}
A human being needs only a small plot of ground on which to be happy, and even less to lie beneath.\footnote{Johan Wolfgang von Goethe, {\it The sorrows of young Werther and selected writings}.}
\end{quote}

\section{Introduction}

This chapter will present a template for analysing economic development takings, based on legal theory.\footnote{I will not provide an extensive presentation of concepts or theoretical approaches developed in other fields, such as political science, sociology, economy, or psychology. However, all these fields engage in interesting ways with the notion of takings and property. Hence, while I focus on legal and --  to some extent -- philosophical theories, I will make a special note of relevant research questions that are also analysed in other academic disciplines. For some examples of relevant work from economics, psychology and political science respectively, see, generally, \cite{miceli11,nadler08,katz97,carruthers04}.} It will be argued that the category of economic development takings is relevant to legal reasoning about certain situations when private property is taken by the state. Clearly, this category makes intuitive sense; it targets situations when property is, quite literally, taken for economic development. In most cases considered in this thesis, economic development is even the explicitly stated aim used to justify the exercise of eminent domain. Hence, the factual basis for the categorization is beyond doubt.

The juridical basis, on the other hand, cannot be taken for granted. Indeed, a superficial look at dominant legal approaches to property would seem to indicate that in many property regimes, the nature of the project benefiting from a taking is not a major issue when assessing the legitimacy of interference.\footnote{For instance, in Europe, the property jurisprudence at the ECtHR deals almost exclusively with other aspects of legitimacy. The Court typically stresses that interference must be in the public interest, but then leave this aspect of legitimacy behind after making clear that the member states enjoy a wide margin of appreciation in relation to the public interest requirement. See, e.g., \cite{james86,lindheim12}. Similarly, in the US in the 1980s, Merrill claimed that most observers thought of the public use clause in the fifth amendment of the US constitution as nothing more than a ``dead letter'', see \cite[61]{merrill86}.} 

This chapter aims to clarify why the purpose and context of a taking matters, not only as a question of public policy but also with respect to property protection and the rights of owners and their communities. I believe it is important to do so thoroughly, to establish a secure conceptual basis for the rest of the thesis. From the point of view of US law, this is not strictly necessary, since economic development takings have already gained recognition as an important category of legal reasoning.\footnote{See generally \cite{cohen06,somin07,malloy08}.} In Europe, however, this has not yet happened, at least not to the same extent.

The reason for this difference is not that US law contains special rules that directly point to distinguishing features of economic development takings.\footnote{In fact, many state laws now {\it do} contain such rules, following the backlash of the controversial decision in \cite{kelo05}. However, such rules were introduced only after the category of economic development takings first came to prominence in legal discourse. See generally \cite{eagle08,somin09,jacobs11}.} Rather, the difference is largely due to the fact that economic development takings have resulted in political controversy in the US, a controversy that has influenced both the law and legal scholars.\footnote{See, e.g., \cite[1190-1192]{somin08}.} Hence, in the absence of a similar political climate in Europe, a conceptual investigation into the very idea of an economic development taking is warranted.

This chapter argues that in order to make progress in this regard, we must broaden our theoretical outlook compared to traditional forms of legal reasoning about property. Interestingly, a suitable conceptual reconfiguration appears to be implicit in recent strands of property theory, particularly those that focus on the {\it social function} of property.\footnote{See generally \cite{alexander09a,foster11,singer00,underkuffler03,alexander06,alexander10,dagan11}.} Indeed, the crux of the main argument presented in this chapter is that the social function view compels us to pay attention to the special dynamics of power that tend to manifest in cases when private property is taken by the state for economic development, especially in the context of commercial exploitation.

To make clear why such takings are special, this chapter abandons the traditional entitlements-based perspective on property in favour of a perspective that emphasises the function of property as a building block of democracy and participatory decision-making, particularly at the local level. This will allow us to shift attention away from the individual effect on owners, towards the question of whether the purpose of the taking, and its broader societal effect, merits interfering with private property. The social function theory strongly encourages such a change of perspective, by compelling us to recognise the importance of property in regulating social and political relations. Moreover, the social function theory emphasises the social {\it obligations} attached to property, particularly with respect to communities of property dependants. 

On this basis, I will argue that private property is important because it gives owners a right to take part in decision-making processes concerning economic development, a right that also typically gives owners a duty to participate, not only on their own behalf, but also on behalf of local community interests. This highlights that property rights can empower local communities in their interactions with powerful commercial and central government interests. Moreover, the use of eminent domain can undermine this crucial function of property, thereby threatening the democratic legitimacy of the decision-making process, by depriving local communities of a potentially robust source of participatory competence. Indeed, when property interests are transferred away from the local community on a permanent basis, this threatens to leave a lasting democratic deficit in the wake of economic development. This, I argue, is the key reason why we need to recognise economic development takings as a separate conceptual category.

To motivate the theoretical work, I will begin in Section \ref{sec:dts} by considering the Balmedie controversy, pertaining to Donald Trump's plans for a golf resort in Balmedie, a village on the east coast of Scotland. I use this concrete example to highlight tensions between property's different functions in the context of economic development. Then, in Section \ref{sec:top}, I go on to discuss theories of property, to locate a suitable starting point for further analysis. I argue that neither of the two dominant property theories of the last century, the bundle theory and the dominion theory respectively, provide such a starting point. In Section \ref{sec:socfunc}, I move on to consider the social function theory in more depth, to arrive at a more useful theoretical template. Moreover, I argue that the descriptive part of this theory can provide a valuable conceptual tool even if one does not agree with the normative assertions that are typically associated with it. In particular, I argue that normative considerations should be addressed separately from conceptual foundations.

I do so in Section \ref{sec:hf}, by building on the human flourishing account of the purpose of property. I argue that the human flourishing theory provides us with a possible path towards answers to the normative questions that arise from the social function perspective on property. In Section \ref{sec:edt}, I make the discussion more concrete by applying the social function theory to a preliminary investigation of economic development takings. The human flourishing theory is then used to formulate some overriding normative constraints that will rely on for the concrete policy assessments I offer in this thesis. In Section \ref{sec:conc1}, I offer a conclusion.

\section{Donald Trump in Scotland}\label{sec:dts}

On the 10th of July 2010, the property magnate Donald Trump opened his first golf-course in Scotland, proudly announcing that it would be the ``best golf-course in the world''.\footnote{See \cite{passow12}.} Impressed with the unspoilt and dramatic seaside landscape of Scotland's east coast, the New Yorker, who made his fortune as a real estate entrepreneur, had decided he wanted to develop a golf course in the village of Balmedie, close to Aberdeen.

To realise his plans, Trump purchased the Menie estate in 2006, with the intention of turning it into a large resort with a five-star hotel, 950 timeshare flats, and two 18-hole golf-courses. The local authorities were divided on the issue of whether to grant planning permission, which was first denied by Aberdeenshire Council.\footnote{See, e.g., \cite{bbc07}.} Critical attention was directed at the fact that the proposed site for the development had previously been declared to be of special scientific interest under conservation legislation.\footnote{See \cite{bbc07b}.} The frailty and richness of the sand dune ecosystem, many argued, suggested that the land should be left unspoilt for future generations. 

Trump was not deterred, and in the end he was able to convince Scottish ministers that he should be given the go-ahead on the prospect of boosting the economy by creating some 6000 new jobs.\footnote{See \cite{carrell08}. Trump's plans attracted significant public attention, and his interaction with Scottish decision-makers came under critical scrutiny by commentators, see, e.g., \cite{jenkins08}. For a more general assessment from the point of view of conservation interests in the UK, see \cite{koen13}.} Activists continued to fight the development, launching the ``Tripping up Trump'' campaign to back up local residents who refused to sell their properties.\footnote{See \cite{tripping15}.} One of these, the farmer and quarry worker Michael Forbes, expressed his opposition in particularly clear terms, declaring at one point that Trump could ``shove his money up his arse''.\footnote{See \cite{scotsman10}.} Trump, on his part, had described Forbes as a ``village idiot'' that lived in a ``slum''.\footnote{See \cite{bbc10}.} Moreover, he had suggested that Forbes was keeping his property in a state of disrepair on purpose, to coerce Trump to pay more for the land, to remove the blight.\footnote{See \cite{cnn07}.} Forbes was offended. He proudly declared that he would never consider selling, as the issue had become personal.\footnote{See \cite{ferguson12}.}

At the height of the tensions, Trump asked the local council to consider issuing compulsory purchase orders (CPOs) that would allow him to take property from Forbes and other recalcitrant locals against their will.\footnote{See \cite{macaskill09}. It would not have been the first time Donald Trump benefited from eminent domain. In the 1990s, he famously succeeded in convincing Atlantic City to allow him to take the home of Vera Coking, to facilitate further development of his casino facilities. But in this instance, Trump has been unsuccessful. Indeed, the taking of Vera's home was eventually struck down by the New Jersey Superior Court, an influential result that was hailed as a milestone in the fight against ``eminent domain abuse'' in the US. See \cite[297-301]{jones00}. See also \cite{gillespie08}. For the decision itself, consult \cite{banin98}.} These plans were met with widespread outrage. The media coverage was wide, mostly negative, and an award-winning documentary was made which painted Trump's activities in Balmedie in a highly negative light.\footnote{See \cite{baxter11}.} The controversy also found its way into UK property scholarship. Kevin Gray, in particular, a leading expert in property law, expressed his opposition by making clear that he thought the proposed taking would be an act of ``predation''.\footcite{gray11}

In fact, the case prompted Gray to formulate a number of key features that could be used to identify situations where compulsory purchase would be likely to represent an abuse of power. Gray noted, moreover, that Trump's proposed takings would fall in line with a general tendency in the UK towards using compulsory purchase to benefit private enterprise, even in the absence of a clear and direct benefit to the public. In light of this, it seemed realistic that CPOs might be used in Balmedie.\footnote{Moreover, a statutory authority is found in section 189 of the \cite{tcpsa97}, stating that local authorities have a general power to acquire land compulsorily in order to ``secure the carrying out of development, redevelopment or improvement''.} It would not be hard to argue that the public would benefit indirectly in terms of job-creation and increased tax revenues. Moreover, Scottish ministers had already gone far in expressing their support for the plans.

But then, in a surprise move, Trump announced he would not seek CPOs, claiming also, to the consternation of local residents, that it had never been his intention to do so.\footnote{See \cite{scotsman11}.} Instead, Trump decided to pursue a different strategy, namely that of containment. He erected large fences, planted trees and created artificial sand dunes, all serving to prevent the properties he did not control from becoming a nuisance to his golfing guests. One local owner, Susan Monroe, was fenced in by a wall of sand some 8 meters high. ``I used to be able to see all the way to the other side of Aberdeen'', she said, ``but now I just look into that mound of sand''.\footnote{See \cite{booth12}.} She also lamented the lack of support from the Scottish government, expressing surprise that nothing could be done to stop Trump.

There was little left to do. As soon as the decision was made to build around them, the neighbouring property owners found themselves marginalized. Trump, on his part, was declared a valuable job-creator whose activities would boost the economy in the region. He even received an honorary doctorate at Robert Gordon University, a move that prompted the previous vice-chancellor, Dr David Kennedy, to hand his own honorific back in protest.\footnote{See \cite{bbc10b}.}

In the end, then, it was not by taking the land of others that Trump triumphed in Scotland. Rather, he succeeded by exercising ``despotic dominion'' over his own.\footnote{To quote William Blackstone, \cite[2]{blackstone79b}.} This proved highly effective. After he fenced them in, his neighbours were hard to see and hard to hear. The Balmedie controversy went quiet, the golfers came, Trump got his way. As he declared during the grand opening: ``Nothing will ever be built around this course because I own all the land around it. [...] It's nice to own land.''\footnote{See \cite{booth12}.}

\subsubsection*{\ldots}

The tale of Trump coming to Scotland serves to illustrate the kind of scenario that I will be looking at in this thesis. In addition, it puts my work into perspective. For a while, it looked like Balmedie was about to become a canonical case of an economic development taking. But in the end, it became an illustration of something more subtle, namely that what it means to protect property depends on value judgements regarding opposing property interests. In particular, while Trump achieved his ends in Scotland by relying on his own property rights, he did so by undermining the property rights of others, even if he did not formally condemn those rights.

This was made possible by an exercise of regulatory and financial power. Hence, we are reminded that the function of property as such is deeply shaped by social, political and economic structures. For the powerful owner, property can be used offensively to oppress weaker parties. For the marginalised, it might well be the last line of defence against oppression. Indeed, Donald Trump's ownership of the Menie estate has a vastly different meaning than does Michael Forbes' ownership of his small farm. To many observers, the former kind of ownership will represent some combination of power, privilege and profit, while the latter will be regarded as imbued with a mix of defiance, community and sustenance. Very different values are inherent in these two forms of ownership, and after Trump came to Balmedie, they clashed in a way that required the legal order to prioritise between them.

In Trump's narrative, upholding the sanctity of property in Balmedie entails allowing him to protect his golf resort plans from what he regards as backwards locals who attempt to fight progress. If this is one's starting point, property protection might even come to involve the use of compulsory purchase of rights that are seen as a hindrance to the full enjoyment of property by a more resourceful owner. 

For Michael Forbes and the other local owners, protecting property has a completely different meaning. To them, it was paramount to protect the local community against what they saw as a disruptive and damaging plan, one that threatened to turn them and their properties into mere golfing props. Again, adequate protection might require an interference in property, to prevent Trump from using his land according to his own wishes, because this causes damage to his neighbours. 

Regardless of who we support, in the case of Balmedie, we are forced to recognise that protection implies interference and vice versa. This shows the conceptual inadequacy of the idea that property protection is all about weighing private and public interests against each other, to strike a balance between the state's power to do good and owners' right to do as they please. In reality, matters can be more subtle, involving a number of additional dimensions. Importantly, how we assess concrete situations where property is under threat depends crucially on what we perceive as the ``normal'' state of property, the alignment of rights and responsibilities that we deem  worthy of protection. Our stance in this regard clearly depends on our values. But values themselves are in turn influenced by the context of assessment within which they arise. An additional challenge is that our assessments are often influenced by our \emph{perception} of the relevant context, rather than by facts.

For example, property activists in the US tend to regard the value of autonomy as a fundamental aspect of property. But this must be understood in light of the idea that US society is founded on an egalitarian distribution of property, where ownership is meant to empower ordinary people by facilitating self-sufficiency and self-governance.\footnote{See, e.g., \cite[173]{ely07}.} Hence, the autonomy inherent in property ownership is not thought of as being bestowed on the few, but on the many. Protecting autonomy of owners against state interference is not about protecting the privileges of the rich and powerful, but is embraced as a way to protect {\it against} abuse by the privileged classes.\footnote{This narrative is enthusiastically embraced by US activists who fight economic development takings, see, e.g., \cite{castle15}.} 

This, however, is only an {\it idea} of property protection. It might not correspond to the reality surrounding the rules that have been \isr{moulded} in its image. Indeed, it has been noted that despite the great pathos of the egalitarian property idea, egalitarianism has actually played a marginal role to the development of US property law.\footnote{\cite[361]{williams98} (``Why does the egalitarian strain of republicanism have such a substantial presence in American property rhetoric outside the law but so little influence within it?'')} More worryingly still, research indicates that land ownership in the US, which is hard to track due to the idiosyncrasies of the land registration system, is not actually all that egalitarian.\footcite[246-247]{jacobs98} In this way, we are confronted with the danger of a dissociation of values, reality and the law.

In Scotland, a similar story unfolds. Here, the traditional concern is that land rights are mainly held by the elites.\footnote{See generally \cite{wightman96,wightman13}.} As a result, Scottish property activists tend to focus on values such as equality and fairness, calling also on the state to regulate and implement measures to achieve more egalitarian control over the land. Indeed, reforms have been passed that sanction interference in established property rights on behalf of local communities.\footnote{See generally \cite{lovett11,hoffman13}.} At the same time, cases like Balmedie illustrate that the Scottish government, now with enhanced powers of land administration, may well choose to align themselves with the large landowners. Moreover, research indicates that recent reforms in Scottish planning law, which serve to enhance the power of the central government, have the effect of undermining local communities and their capacity for self-governance.\footnote{See generally \cite{pacione13,pacione14}.} Again, the danger of a disconnect between influential property narratives and reality is brought into focus.

On the other hand, it seems that \isr{grass roots} property activists in the US and Scotland may well be closer in spirit than they seem. Although their perception of the role of the state is very different, they appear to share many of the same concerns and aspirations. Arguably, differences arise mainly from the fact that they operate in different contexts and engage with different discourses of property. The challenge is to find categories of understanding that allow us to make sense of both their commonalities and their differences.

I think the example of Balmedie suggests a possible first step. It illustrates, in particular, the need for a framework that will allow us to recognise that opposing the use of compulsory purchase for economic development is perfectly consistent with supporting strict property regulation to prevent the establishment of golf resorts in fragile coastal communities. Both of these positions, moreover, should be viewed as efforts to protect property. To the classical debate about the limits of the state's authority over property, such a dual position can be hard to make sense of. But in my opinion, this only points to the vacuity of the conventional narrative.

In general, I think it is hard to make sense of many contemporary disputes over property if we do not have the conceptual acumen to distinguish between (1) egalitarian property held under a stewardship obligation by members of a local community, and (2) ``feudal'' property held by large enterprises for investment. Moreover, there is no contradiction between promoting the value of autonomy for one of these, while \isr{emphasising} the need for state control and redistribution when it comes to the other. The broader theme is the contextual nature of property and its implications for protection of property rights. In the coming sections, I will propose a theoretical basis that integrates this viewpoint into legal reasoning about interference in property rights.

\section{Theories of Property}\label{sec:top}

What is property? In common law jurisdictions, the standard answer is that property is a collection of individual rights, or more abstractly, a means of protecting {\it entitlements}.\footnote{The idea that property rules are a form of entitlement protection was developed to great effect in the seminal article \cite{calabresi72}.} Being an owner, it is often said, amounts to being entitled to one or more among a bundle of ``sticks'', streams of protected benefits associated with, and thereby serving to legally define, the property in question.\footnote{See \cite[357-358]{merrill01}. The ``classical'' references on the bundle of rights theory in the US and the UK respectively are \cite{hohfeld17,honore61}.} This point of view was first developed by legal realists in response to the natural law tradition, which \isr{conceptualised} property in terms of the owner's dominion over the owned thing, particularly his right to exclude others from accessing it.\footcite[193-195]{klein11} In civil law jurisdictions, rooted in Roman law, a dominion perspective is still often taken as the theoretical foundation of property, although it is of course \isr{recognised} that the owner's dominion is never absolute in practice.\footnote{For a comparison between civil and common law understanding of property, see generally \cite{chang12}.}

In modern society, the extent to which an owner may freely enjoy his property is highly sensitive to government's willingness to protect, as well as its desire to regulate. To dominion theorists, this sensitivity is typically thought of as giving rise to various restrictions on property, but for bundle theorists it is often thought of as {\it constitutive} of property itself.\footcite[7]{chang12} 

The bundle of rights theory has long historical roots in common law. Arguably, it was distilled from the traditional estates system for real property, which was turned into a theoretical foundation for thinking about property in the abstract.\footnote{See \cite[7]{chang12}(``The ``bundle of rights'' is in a sense the theory implicit in the common law system taken to its extreme, with its inherently analytical tendency, in contrast to the dogged holism of the civil law.'').} However, during the 18th and 19th century, natural law and dominion theorising was also influential in common law. This is evidenced, for instance, by the works of William Blackstone and James Kent.\footnote{See generally \cite{blackstone79b,kent27}.} Towards the end of the 19th century, it became increasingly hard to reconcile such an approach to property with the reality of increasing state regulation. Hence, the bundle metaphor that gained prominence in the early 1900s can be seen as a return to a more modest perspective.\footnote{See \cite[195]{klein11}.}

On the bundle account, property rights are thought to be directed primarily towards other people, not things.\footnote{See \cite[357-358]{merrill01} (``By and large, this view has become conventional wisdom among legal scholars: Property is a composite of legal relations that holds between persons and only secondarily or incidentally involves a ``thing''.'').} This underscores an important point about property in the real world, namely that the content of rights in property are necessarily relative to a social context as well as the totality of the legal order. For instance, when relying on a bundle metaphor it becomes easy to explain that a farmer's property rights protects him against trespassing tourists, but not against the \isr{neighbour} who has an established right of way.\footnote{It has been argued that this way of thinking about property, as a web of (legal and social) normative relations between persons, does not entail the bundle of sticks idea, see \cite[23-25]{dorfman10}. I agree, and I also believe that endorsing the property-as-relations perspective is largely appropriate, even if one does not otherwise agree with the bundle perspective. Historically, however, the two ideas have in fact been closely associated with one another, so presenting them together seems appropriate. Moreover, I will not actively enter into the theoretical debate on this point, since I believe that the {\it social function} account of property, discussed in more detail in Section \ref{sec:socfunc}, takes us further than both bundle and dominion perspectives. However, as will hopefully become clear, the social function theory itself may be seen as a continuation of the property-as-relations idea, catering also to a more holistic perspective on social structures (although it otherwise manages to remain largely neutral on the bundle v dominion issue).}

By contrast, the dominion theory suggests viewing such situations as exceptions to the general rule of ownership, which implies a right to exclusion at its core. In the case of property, exceptions no doubt make up the norm. But in civil law jurisdictions one lives happily with this. It takes the grandeur away from the dominion concept, but it retains a nice and simple structure to property law. In the civil law world, it is common to say that what the owner holds is the {\it remainder}, namely what is left after deducting all positive rights that restrict his dominion.\footcite[25]{chang12} Moreover, while there may be many limitations and additional benefits attached to property, they are all in principle carved out of one initial right, namely that of the owner. In this way, the civil law system can be more easy to navigate.

%An interested party may ask, ``who owns this property?'' Then, under the dominion theory, a clear answer is expected and will usually be adequate, even if it does not give a complete picture of all relevant property rights. Under the bundle theory, on the other hand, one might be inclined to respond, ``to which stick are you referring?'' Clearly, this narrative is more complex, perhaps unduly so. 

Some common law scholars have recently elaborated on this to develop a critique of the bundle theory, by suggesting that it should at least be complemented by a firm theory of {\it in rem} rights in property. This, they argue, would allow the law to operate more effectively, by relying on a simple and clear rule that, although defeasible, would generally suffice to inform people about their relevant rights and duties in relation to property.\footnote{\cite[793]{merrill01b} (``The unique advantage of in rem rights -- the strategy of exclusion -- is that they conserve on information costs relative to in personam rights in situations where the number of potential claimants to resources is large, and the resource in question can be defined at relatively low cost.''); \cite[389]{merrill01} (``The right to exclude allows the owner to control, plan, and invest, and permits this to happen with a minimum of information costs to others.''). See also \cite{ellickson11} (arguing that Merrill and Smith's analysis nicely complements and improves upon the bundle theory).} 

In addition, some scholars point out that the bundle theory does not adequately reflect the sense in which property is a right to a {\it thing}, serving to create an attachment that is not easily reducible to a set of interpersonal legal relationships.\footnote{\cite[1862]{merrill07}. For a slightly different take on attachment, highlighting how the `thingness' of property marks its conditional nature and transferability, see \cite[799-818]{penner96}.} In the US, where the bundle theory has traditionally been dominant, this critique seems to be gaining ground.\footnote{See generally \cite{foster10}.}

In this thesis, the efficiency and clarity of different property concepts will not be a primary concern, nor will personal attachments to things in themselves play a particularly important role.\footnote{I mention, however, that the \isr{personhood aspects} of property that are sometimes highlighted in this regard will also be relevant to an analysis of economic development takings. However, this is not something that I think warrants extensive engagement with the bundle v dominion debate. I note, for instance, that in the work of Margaret Jane Radin, one of the main proponents of \isr{personhood} accounts, the bundle theory is not challenged as much as it is readjusted, although in places it also seems to be the object of some implicit criticism, see, e.g., \cite[127-130]{radin93}.}
Hence, I will remain largely agnostic about this aspect of the debate between dominion and bundle theorists. In particular, the differences between civil and common law traditions in this regard do not cause special problems for my analysis of economic development takings. In this regard, it is more important how different ways of looking at property can influence how we assess when interference is legitimate under constitutional and human rights law. Hence, I now turn to the question of whether or not there are any significant differences between dominion and bundle theories in this regard.

\subsection{Takings under Bundle and Dominion Accounts of Property}

Bundle theorists might be expected to have a comparatively relaxed attitude towards state interference in property rights. Indeed, thinking about property as sticks in a bundle may lead one to think that property rights are intrinsically limited, so that subsequent changes to their content, made by a competent body, are reflections of their nature, not a cause for complaint. In particular, the theory conveys the impression that property is highly malleable. 

For the theorists that developed the bundle of sticks metaphor in the late 19th and early 20th century, this aspect was undoubtedly very important. By providing a highly flexible concept of property, they helped the state gain conceptual authority to control and regulate.\footcite[195]{klein11} The early bundle theorists not only developed a theory to fit the law as they saw it, they also contributed to change.

In takings law, the bundle narrative has been particularly important in relation to the contentious issue of so-called regulatory takings. Such takings occur when governmental control over the use of property, limiting the freedom of the owner, becomes so severe that it is classified as a taking in relation to the law of eminent domain. In the US, the question of when regulation amounts to a regulatory taking is highly controversial. The stakes are high because takings have to be compensated in accordance with the Fifth Amendment of the US constitution. At the same time, the law is unclear; the lack of statutory rules means that regulatory takings cases are often adjudicated directly against constitutional property clauses (often the relevant state constitution, in the first instance).

If property is thought of as a malleable bundle of entitlements that exists only because it is recognised by the law, it becomes natural to argue that when government regulates the use of property, it does not deprive anyone of property rights. It merely restructures the bundle. In the case of {\it Andrus v Allard}, the Supreme Court adopted such an argument when it declared that ``where an owner possesses a full ``bundle'' of property rights, the destruction of one ``strand'' of the bundle is not a taking, because the aggregate must be viewed in its entirety''.\footcite[65--66]{andrus79}

Hence, with regards to the issue of regulatory takings, the bundle theory was actively used by those who favour a less restrictive approach to interference with private property rights. However, it is wrong to conclude that the bundle theory {\it necessarily} implies a less restrictive stance on takings. Epstein, for instance, argues that as every stick in the property bundle represents a property right, government should not be permitted to remove any of them without paying compensation.\footcite[232-233]{epstein11} 

More generally, Epstein does not believe that the bundle theory is responsible for what he regards as a weakening of property rights in the US during the 20th century. Instead, he thinks this weakening resulted from a tendency among modern property scholars to adopt a ``top-down'' approach to property. According to Epstein, too many scholars view property rights as vested in, and arising from, the power of the state, not the possessions of individuals.\footnote{\cite[227-228]{epstein11} (``In my view, the nub of the difficulty with modern property law does not stem from the bundle-of-rights conception, but from the top-down view of property that treats all property as being granted by the state and therefore subject to whatever terms and conditions the state wishes to impose on its grantees'').} 

Epstein successfully shows that as a rhetorical device, the bundle of rights theory may be turned on its head compared to how it was used in {\it Andrus v Allard}. Moreover, his arguments illustrate that the bundle theory itself does not appear to dictate any particular position on the degree of protection that private property should enjoy against state interference.\footnote{To further underscore this point, it may be mentioned that while US courts do in fact \isr{recognise} that a regulation can amount to a taking, this is practically unheard of in several other common law jurisdictions, including England and Australia. This is despite the fact that these countries all paint property in a similar conceptual light. Moreover, while the issue of regulatory takings is considered central to constitutional property law in the US, it is considered a fairly marginal issue in England, see \cite{purdue10}.}

In the civil law world, the relationship between property theorising and property values is similarly hard to pin down at the conceptual level. Again, the issue of regulatory takings illustrates this. In some civil law countries, like Germany and the Netherlands, the owner's right to compensation for burdensome land use regulation is strong, while in other civil law countries, such as France and Greece, it is very weak.\footnote{See generally \cite{alterman10}.} In particular, the exclusive dominion understanding of property does not appear to commit one to any particular kind of policy on this point. 

On the one hand, it cannot be denied that property rights are enforced, and limited, by the power of government. Hanging on to the idea of dominion, then, necessarily forces us to embrace also the idea that dominion is never absolute. In this way, the theory may serve as a conceptual basis for arguing in favour of a relaxed approach to state interference. If property rights are not absolute to start with, why worry about interfering in them for the common good? But, of course, this story too may be turned on its head. Indeed, a libertarian can use the image of limited dominion to argue that property is being ripped apart at its seams. If we want to maintain our grasp of what property is, such a person might argue, we had better enhance the level of protection offered to property owners, to restore true dominion.

To me, the upshot is that the differences between common law and civil law \isr{theorising} about property are not very relevant to the question of legitimacy in the context of state interference. In particular, the differences between the bundle theory and the dominion idea do not appear to speak decisively in \isr{favour} of any particular approach to economic development takings.

In terms of descriptive content, both theories appear oversimplified. They provide a manner of speech, but they do not really get us very far towards uncovering the reality of property rights in modern society. In particular, they do not provide a functional account of what role property plays in relation to the social, economic and political structures within which it resides. 

In terms of normative content, on the other hand, both the bundle theory and the dominion theory appear rather bland. They simply do not offer much clear guidance as to what norms and values the institution of property is meant to promote. They give neat ways of presenting what property looks like, but do not tell us {\it why} it should be protected. 

\subsection{Broader Theories}

Based on the discussion so far, it seems that in order to make progress towards a theory of economic development takings we need to start from a property theory with a wider scope than both the bundle account and the dominion theory. There are many candidates that could be considered. In a recent monograph on property, Alexander and Pe\~{n}alver present five key theoretical branches:
\begin{itemize}
\item {\it Utilitarian} theories, focusing on property's role in helping to maximize utility or welfare with respect to individual preferences and desires.\footnote{\cite[Chapter 1]{alexander10}.} 
\item {\it Libertarian} theories, focusing on property's role in furthering individual autonomy and liberty, as well as the importance of protecting property against state interference, particularly attempts at redistribution.\footnote{\cite[Chapter 2]{alexander10}.} 
\item {\it Hegelian} theories, focusing on the importance of property to the development of personhood and \isr{self-realisation}, particularly the expression and embodiment of free will through control and attachment to one's possessions.\footnote{\cite[Chapter 3]{alexander10}.}
\item {\it Kantian} theories, focusing on how property arises to protect freedom and autonomy in a coordinated fashion so that {\it everyone} may potentially enjoy it, through the development of the state.\footnote{\cite[Chapter 4]{alexander10}.}
\item {\it  Human flourishing} theories, focusing on property's role in facilitating participation in a community, particularly as a template allowing the individual to develop as a moral agent in a world of normative plurality.\footnote{\cite[Chapter 5]{alexander10}.}
\end{itemize}

It it beyond the scope of this thesis to give a detailed presentation and assessment of all these theoretical branches. Suffice it to say that the utilitarian approach has been by far the most influential.\footnote{See \cite[11]{alexander12} (noting also that there are many varieties of utilitarianism, including some law-and-economics theories for which the appropriateness of that label is contentious).} The basic tenet of this paradigm is that means-end analysis on the basis of exogenous preferences and utility measures provide a sound foundation on which to reason about law and policy.

In this thesis, I will depart from this form of analysis, by regarding property instead as an integral part of social structures. On this view,  property can no longer be seen neither as an end in itself nor as a means to maximise some utility measure. Instead, property is understood in light of how it functionally relates to other building blocks of life, such as sustenance, economic activity, social interaction, interpersonal responsibility, preference change, deliberation,  and democratic decision-making.

With such a starting point, I believe the human flourishing theory has more to offer than any of the other theoretical branches mentioned above. In Section \ref{sec:hf} below, I will emphasise how this theory suggests making a range of new policy recommendations regarding how the law {\it should} approach the question of economic development takings.

Before I get to this, I will explore descriptive aspects of property theory in some more depth. Indeed, a potential objection against all the theories summarised above is that they are overly normative; they are largely used to argue for particular values associated with property, not to clarify the descriptive core of the notion. This is a challenge, since one of my main aims in this thesis is to argue for a descriptive proposition, namely that economic development takings make sense as a conceptual category for legal reasoning. Hence, before I move on to consider normative aspects, I first need a theoretical framework that allows me to pinpoint what makes economic development takings unique. I would like to do so, moreover, without thereby committing myself to any particular stance on how to normatively assess such takings.

To arrive at a suitable foundation in this regard, I will rely on the so-called {\it social function theory} of property.\footnote{See generally \cite{foster11,mirow10,alexander09a}. Be aware that some authors, particularly in the US, also speak of the {\it social obligation} theory, using it more or less as a synonym for the social function theory.} This theory is often thought of as a normative theory as well, in some sense a precursor to more overtly normative theories such as the human flourishing theory. However, I will argue that the social function theory has a descriptive core that can serve as a common ground for debate among scholars that do not necessarily share the same normative outlook. Crucially, the descriptive core of the social function theory also point towards a descriptive argument in favour of studying economic development takings.
\noo{
Before making my specific point about takings, I will present the social function theory of property in some further detail. I will focus on showing that it captures aspects that are already highly relevant -- behind the scenes -- to how property rules are understood and applied in concrete situations.
}
\section{The Social Function of Property}\label{sec:socfunc}
\noo{
There is a growing feeling among property scholars that the notion of property has been drawn too narrowly by many of the traditionally dominant theories of property. \noo{ Moreover, it has been noted that what counts as property in a given legal system, and what property entails in that system, depends largely on its social and political context, tradition, and sometimes even chance.\footnote{For a clear exposition of property's elusive nature, see \cite{gray91}.} In the US, a utilitarian law-and-economics approach, which tends to take the social and political underpinnings of property for granted, has long been regarded as standard, but the tide is turning.} Moreover, an increasing number of scholars are turning away from assessing property rules against their effectiveness in maximising utility and social welfare.\footnote{For a nice early commentary on the limits of the utility-maximizing perspective in property law, focusing on the importance of changes, choices, and narratives, see \cite{rose90}.} Instead, some scholars adopt a holistic approach, allowing property's social function to come into focus. One of the main proponents of this conceptual shift is Gregory S. Alexander, professor at Cornell University. In a recent article, he writes:

\begin{quote} Welfarism is no longer the only game in the town of property theory. In the last several years a number of property scholars have begun developing various versions of a general vision of property and ownership that, although consistent with welfarism in some respects, purports to provide an alternative to the still-dominant welfarist account.[...] These scholars emphasize the social obligations that are inherent in ownership, and they seek to develop a non-welfarist theory grounding those inherent social obligations.\footcite[1017]{alexander11}
\end{quote}

\noo{ To scholars coming from political science, sociology or human geography, this trend will not raise many eyebrows, except perhaps for the fact that it is a recent one. After all, in fields such as these, property has never been understood merely as a set of individual entitlements that are meant to result in increased welfare. Rather, property is seen as a crucial part of the fabric of society, one that entrenches privileges and bestows power.\footnote{See generally \cite{carruthers04}.} 
Even scholars who believe that the institution of property is a force for good recognise that being an owner carries with it political capital, social responsibility, and membership in a community. Those aspects, moreover, are often regarded as more important than entitlements explicitly \isr{recognised} in positive legal terms. Crucially, they are important not only to the individual owners but also to society as a whole. How property rights are distributed among the population, for instance, has obvious political and economic implications, serving as a source of power and prosperity to some groups, while \isr{marginalising} others.\footnote{See, e.g., \cite[23]{carruthers04}. (``The right to control, govern, and exploit things entails the power to influence, govern, and exploit people'').}

But what is the relevance of this to property law? Usually, jurists approach property in isolation from such concerns, and often they do so because of practical necessity. The political question of what the law should be depends on assessments of the purpose and social context of property, but in the day-to-day workings of the law, the story goes, such considerations play a lesser role, with the importance of clear and simple rules outweighing the possible benefit that would result from contextual and holistic assessment. Classical theories of property can be accused of taking such a pragmatic view too far, by failing to \isr{recognise} that many social functions are {\it intrinsic} to property, so that they may become directly relevant when the law is applied to resolve concrete disputes.

The same accusation can be raised against both bundle and dominion theorists. They both tend to leave little room for considering property as a social phenomena. It is recognised, of course, that rights in property -- bundled or otherwise -- serve to regulate social relations. But this effect is typically regarded as belonging to the periphery of property as a legal category, more relevant to sociologists than to property scholars. In addition, it is uncommon to observe that the causal relation between property rights and society is bidirectional, since the meaning and content of property itself is partly determined by the very same social structures that property helps establish and sustain. When this aspect of property is not recognised, the risk is that subtle dependencies between property and the political order are not brought into focus, even when they play an important role in practice.

This is particularly clear when it comes to socially defined obligations attached to property. Hardly anyone would protest that in practical life, what an owner will do with their property is as much constrained by the expectations of others as it is by law. But in addition to influencing the owner subjectively, expectations can take on an objective character by being embedded strongly in the social fabric. This, in turn, can give rise to a norm, or even a custom, which may be legally relevant, either because the law gives direct effect to it, or because it influences how we interpret rules relating to the use of property.\footnote{See generally \cite{penalver09,alexander09}.}
}
}
As an empirical observation, the fact that property has social functions is beyond doubt. For instance, it is clear that ownership of property gives rise to social obligations, not just rights. Hardly anyone would protest that in practical life, what an owner will do with their property is as much constrained by the expectations of others as it is by law. Moreover, the law of nuisance and rules relating to adverse possession both serve as simple examples that such expectations also feature in the law of property, at the concrete level. \noo{Crucially, the social function theory asks us to focus esHowever, in the space between informal social obligations and formal legal duties, there is an interesting space of interaction that the social function theory asks us to address. 

Indeed, informal expectations can sometimes take on an objective character merely by being embedded strongly in the social fabric. This, in turn, can give rise to a norm, or even a custom, which may be legally relevant, either because the law gives direct effect to it, or because it influences how we interpret rules relating to the use of property.\footnote{See generally \cite{penalver09,alexander09}.}}

Still, many property scholars have surprisingly little regard for social functions when they theorise about ownership. According to Alexander, the classical theories of property convey the impression that ``property owners are rights-holders first and foremost; obligations are, with some few exceptions, assigned to non-owners''.\footcite[1023]{alexander11} The social function theorists attempt to redress this conceptual imbalance. As Alexander explains, ``social obligation theorists do not reverse this equation so much as they balance it. Of course property owners are rights-holders, but they are also duty-holders, and often more than minimally so''.\footnote{\cite[1023]{alexander11} I remark that what Alexander refers to as the social obligation theory of property is more or less the same as what I refer to in this thesis as the social function theory. Although the latter theory might have a somewhat broader scope, no clear distinction between the two have been made in the literature so far, with both names used frequently to denote roughly the same set of ideas. See, e.g., \cite{...}.}

As I discuss in the next subsection, this idea is not new. Moreover, it seems to play an important implicit role in shaping how property is understood, particularly in Europe.

\subsection{Historical Roots and European Influence}

The first expression of the social function theory has been attributed to Le{\'o}n Duguit, a French jurist active early in the 20th century. In a series of lectures he gave in Buenos Aires in 1911, Duguit challenged the classic liberal idea of property rights by pointing to their context dependence, adopting a line of argument strikingly similar to how recent scholars have criticized utilitarian discourses about property.\footnote{See \cite[1004-1008]{foster11}. For more details about Duguit's work and the contemporaries that inspired him, see generally \cite{mirow10}.} In particular, Duguit also pointed to the notion of obligation, stressing the fact that individual autonomy only makes sense in a social context where people are dependent on each other as members of  communities. Hence, depending on the social circumstances of the owner, their property could entail as many obligations as entitlements. This, according to Duguit, was not only the inescapable reality of property ownership, it was also a normatively sound arrangement that should inspire the law, more so than individualistic, `liberal', visions of property as entitlement protection.\footnote{See \cite[1005]{foster11} (``The idea of the social function of property is based on a description of social reality that recognizes solidarity as one of its primary foundations'', discussing Duguit's work). It should also be noted that Duguit was particularly concerned with owners' obligations to make productive use of their property, to benefit society as a whole. This raises the question of who exactly should be granted the power to determine what counts as ``productive use''. In this way, Duguit's work also serves to underscore one of the main challenges of regulatory frameworks that seek to incorporate and draw on property's social dimension. How should decisions be made in such regimes?}

Similar thoughts have been influential in Europe, particularly during the rebuilding period after the Second World War. For instance, the constitution of Germany -- her {\it Basic Law} -- contains a property clause stating explicitly that property entails obligations as well as rights. As argued by Alexander, this has had a significant effect on German property jurisprudence, creating a clear and interesting contrast with US law.\footnote{See \cite[338]{alexander03} (``The German Constitutional Court has adopted an approach that is both purposive and contextual, while the U.S. Supreme Court has not'').}

A social perspective on property was also influential during the debate among the European states that first drafted the property clause in Article 1 of the First Protocol to the European Convention of Human Rights (P1(1) of the ECHR).\footnote{See \cite[1063-1065]{allen10}. As Allen notes, the liberal conception of property has since gained ground in Europe, causing jurisprudential developments that have been particularly clear in the case law from the European Court of Human Rights (ECtHR).} The article was eventually formulated as follows:

\begin{quote} Every natural or legal person is entitled to the peaceful enjoyment of his possessions. No one shall be deprived of his possessions except in the public interest and subject to the conditions provided for by law and by the general principles of international law.
The preceding provisions shall not, however, in any way impair the right of a state to enforce such laws as it deems necessary to control the use of property in accordance with the general interest or to secure the payment of taxes or other contributions or penalties.
\end{quote}

I will return to this clause in more depth in Section \ref{sec:eu} of Chapter \ref{chap:2}. Here I note how it emphasises both the private right to peaceful enjoyment of possessions and the state's right to interfere with property in the general/public interest. Moreover, it does not explicitly introduce an absolute compensation requirement in case of expropriation by the state, setting it apart from many other property clauses, including that contained in the Fifth Amendment of the US constitution. Arguably, this reflects a recognition of the social aspects of property.\footnote{See generally \cite{allen10}.} 

However, this particular aspect of social function reasoning also fits within a traditional narrative of private property rights, whereby social responsibilities attaching to property are regarded as arising from state objectives and policies, not ownership as such. Indeed, the chosen formulation in P1(1) appears to suggest that social aspects are external to private property, vested in the regulatory power of the state. 

This marks a possible tension with the social function theory, which asks us to recognise that social obligations are inherent in private property, attaching to owners directly. The importance of this in the present context is that a social function perspective can occasionally suggest stricter limits on state interference, not out of greater concern for individual entitlements, but out of concern for property's proper functions as a building block of social and political life.

Despite the conventional formulation used in P1(1), such a perspective does in fact play a role to the European Court of Human Rights (ECtHR). \noo{At least, the case law from the Court shows that social and political considerations are not only invoked in the context of adjudicating tensions between private entitlements and the perceived necessity of state interference in the public interest.

The ECtHR emphasises {\it proportionality} and {\it fairness} when adjudicating cases involving interference in property.\footnote{See generally \cite[Chapter 5]{allen05}.} Importantly, these broad notions are assessed concretely against the context of interference, also to give appropriate weight to the social and political function of the property interfered with. As a result, specific social functions of property can justify greater protection against interference.} A series of cases involving hunting rights provide a clear example of this.\footnote{See \cite{chassagnou99,hermann12,chabauty12}.} In these cases, the Court in Strasbourg has explicitly granted stronger property protection to owners who oppose hunting on ethical grounds, compared to owners who want to retain exclusive hunting rights for themselves.

For the former group of owners, it has been held that the state may not compulsorily transfer hunting rights to hunting associations for collective management.\footnote{See \cite{chassagnou99, hermann12}.} For the latter group of owners, by contrast, the Court held in {\it Chabauty v France} that such transfers must be tolerated.\footnote{See \cite{chabauty12}.}

For owners opposing hunting on ethical grounds, an interference with their hunting right is an interference with their moral duty to act in accordance with their beliefs. The belief that hunting is unethical gives the owners a personal obligation to prevent their hunting rights from being used. If owners are deprived of their opportunity to fulfil this obligation, it changes the social function of their property because it severs the link between the owners' value system and the use that is made of their property.

In {\it Chassagnou and others v France}, the Court regarded this as a particularly severe interference in property, which could not be upheld despite the fact that it had been carried out in the public interest to secure sustainable management of hunting rights. The Court concluded that ``compelling small landowners to transfer hunting rights over their land so that others can make use of them in a way which is totally incompatible with their beliefs imposes a disproportionate burden which is not justified under the second paragraph of Article 1 of Protocol No. 1''.\footnote{See \cite[85]{chassagnou99}.}

Clearly, the Court is not expressing an opinion on the ethical status of hunting. However, owners are entitled to have unconventional personal convictions in this regard. Moreover, managing one's property in accordance with one's convictions is part of what it means to be an owner. As demonstrated by the ruling in {\it Chabauty}, protecting this aspect of ownership is more important to the Court in Strasbourg than protecting the right of owners to keep the fruits of the land to themselves. The owners' right -- and social obligation -- to manage their property in accordance with their beliefs is a socially desirable aspect of property ownership, one that is entitled to increased protection under the ECHR. \noo{In the jurisprudence of the ECtHR, this form of interference, targeting a social obligation arising from the conscience of owners, is {\it more severe} than one which directly targets  hunting entitlements.}

\noo{I will return to possible normative implications of the social function theory later. Here I would like to stress that in the first instance it merely asks us to recognise an empirical truth: Property does not arise in a vacuum, but from within a society. As a philosophical proposition, this is obvious and hardly anyone denies it. But the social function theory asks us to consider something more, namely that property {\it law} continues to influence, and be influenced by, surrounding social and political structures.}

The hunting cases also demonstrate that even when the legal system does not explicitly recognise the value of a social function inherent in property, such a function can still come to play a role when the Court assesses the legitimacy of interference against P1(1). In the next section, I make a more general claim, arguing that the law invariably prioritises between different social functions of property, even when this is not explicitly acknowledged.

\subsection{The Impossibility of a Socially Neutral Property Regime}

Property both reflects and shapes relations of power among members of a society.\footnote{This aspect of property's social function was stressed in a recent ``statement of purpose'' made by leading property scholars in support of the social function theory, see \cite{alexander09a}.} Moreover, it does not act uniformly in this way -- the effect depends on the circumstances. \noo{ An indebted farmer who is prevented by state regulation from making profitable use of their land might come to find that their property has become a burden rather than a privilege. As a consequence, someone who has already amassed power and wealth elsewhere might be able to purchase the land from the farmer cheaply. By acquiring a farm and transforming it to recreational property, the outsider will symbolically and practically assert their dominance and power, while also reaping a potential financial benefit from investing in a more modern function of property.

In some cases, this dynamic can become endemic in an area, resulting in a complete reshaping of the social fabric surrounding property.} Consider, for instance, a typical scenario leading to depopulation of rural areas: first, impoverished farmers and other locals sell homes to holiday dwellers, causing house prices to soar. As a result, local people with agrarian-related incomes \isr{cannot} afford local homes, causing even more people to sell their land to the urban middle class. In this way, a causal cycle is established, the social consequences of which can be vicious, particularly to the low-income people who are displaced.\footnote{The general mechanism described here is well-documented and known as {\it gentrification} in human geography (often qualified as rural gentrification when it happens outside urban areas). See generally \cite{weesep94,phillips93,slater06}. For a case study demonstrating the role that state regulation can play (perhaps inadvertently) in causing rural gentrification, see \cite[1027-1030]{darling05}.} My theoretical contention is the following: setting out to regulate property in a situation like this -- when property rights pull in different directions depending on your vantage point -- requires taking some stance on whose property, and which of property's functions, one is aiming to prioritise. Should the law emphasise the property rights of local people who face displacement, or should it protect the property rights of outsiders wishing to invest in holiday homes? 

Some may \isr{shy} away from this way of posing the question, by arguing that it would be better to rely on neutral rules that treat all owners the same way. In the gentrification scenario, such an appeal to neutrality could be the first step in an argument against regulating the property market to prevent the displacement of local people. But would that truly be a socially neutral approach to residential property? Presumably, it would be in the interest of local owners, particularly those not wishing to sell, to introduce tighter regulation. Hence, if {\it their} property rights are to be given priority, regulation should be put in place.

Importantly, both sides of a conflict like this are in a position to adopt a property narrative to argue for their interests. If escalation occurs, it will become practically impossible to insist that  property rules are neural on the issue of property's social functions. Here it is illustrative to briefly revisit the conflict between Donald Trump and Balmedie locals, discussed in Section \ref{sec:dts}.

As long as Trump threatened to use compulsory purchase, the local people could adopt a traditional ``pro-property'' stance against Trump. But as soon as Trump decided to fence them in by relying on his own property rights, they had to adopt a seemingly contradictory view on property, whereby Trump's property rights should be limited out of concern for the community. So how do we classify the anti-Trump stance with regard to property?

The answer is unclear under classical theories; a traditionally minded observer might even come to accuse the locals of having an unprincipled attitude towards private property. But under the social function stance, a completely different picture suggests itself. The locals sought to protect property, but not just any property. The property they wanted to protect was the property which served the social function of sustaining the existing community. The property they wanted to protect was the property that {\it meant} something to them.\footnote{This is more than merely observing that they wanted to protect {\it their} property. In their desire to regulate the use of Trump's property, the locals also wanted to protect certain social functions inherent in that property, against Trump's own actions.}

Trump and his supporters might well have entertained similar feelings about their property rights, and the development they wanted to carry out. Hence, in conflicts such as these the law will invariably have to take a stand regarding which social functions it wishes to promote. The social function theory asks us to be upfront about this, so that property adjudication in hard cases can proceed on the basis of substantive arguments about how to prioritise between different functions of property.

\noo{
In all likelihood, such a stand must also sometimes be taken by whoever {\it interprets} the law, since it is exceedingly unlikely that the legislature will ever be able to provide deterministic rules for resolving all conflicts of this kind. Lastly and most controversially, the courts may find occasion to curtail the power of government -- perhaps even the legislature -- if such power is usurped by powerful actors wishing to undermine property's proper functions to further their own interests. 
This, in particular, raises the question of constitutional and human rights limits to interference in property, relative to those functions that are to be protected.}

The law is sometimes forced to prioritise, when various functions come into conflict with one another. However, social functions can also work together in a way that promotes certain property uses and decision-making structures for property management. This can even alleviate the pressure for top-down government regulation, with desirably consequences for both owners and the public interest.
%, as discussed in the next subsection.

%\noo{ \subsection{The Regulating Effect of Property}

%Property shapes and reflects societies, but it also shapes and reflects social commitments and dependencies within those societies.\footnote{See generally \cite{alexander09}.} 

Again, this function of property is highly dependent on context. Small business owners, by virtue of being members of the local community, might be discouraged from becoming a nuisance to their \isr{neighbours}, regardless of formal rules found in the law of nuisance. \noo{Everything from erecting bright neon signs to proposing condemnation of \isr{neighbouring} properties are actions that an owner will be socially deterred from taking. If the local shop owner does not conform to social expectations, he will pay a social price. Indeed, most likely even an economic price, especially if his customer-base is local. At the same time, the local connection would serve to make the business owner positively invested in the well-being of the community. This would encourage everything from sponsoring local events to hiring local youths as part-time helpers.

But at the same time, the local business owner might be discouraged from changing his business model to become more competitive, if this is perceived as a threat to other members of the community. Economic rationality might suggest that he should expand, say, by physically acquiring more space and targeting new groups of customers, but social rationality might make this an untenable proposal. This, however, might render the business economically unsustainable, particularly if it is facing fierce competition from businesses that are not similarly constrained by community ties. 

Moreover, even if the business is in fact viable as long as the community remains in place to support it, the perception that it is not fulfilling its commercial potential can increase external pressures both on the business and the community. Importantly, in the age of regulation for commercial facilitation, the state itself might exert pressure of this kind, by acting in a way that makes it hard to sustain local businesses that are regarded as failing commercially.}

However, if local shop owners go out of business and a non-local commercial owner replaces them, the  regulatory effect of property can change dramatically. %for higher intensity commercial development, Then, if our local shop owner goes out of business, for whatever reason, the new owner might not become integrated in the community in the same way, with obvious consequences for the property's function in that community. 
Indeed, if we imagine that the new owner hopes to raze the local community in order to build a new shopping center, we are at once reminded of the stark contrasts that can arise between various social functions of property. The property rights of small shop owners can be the lifeblood of a community, while the exact same rights in the hands of a large-scale retailer can give rise to its destruction.

\noo{While this is an undeniable empirical fact of property ownership, it is far from clear what its legal ramifications are. Here it is tempting to embrace a normative stance, and argue for particular social values that the law {\it should} promote. However, I will hold on to the descriptive mode of analysis a little further. 

For it is clear that regardless of whose interests win out in the end, the changed social function of property can in turn cause further changes to occur in the law of property.}

iMechanisms like these can have important ramifications, not only for property, but also for the regulatory regime surrounding its use. For instance, it seems clear that if a non-local, commercially aggressive, owner is to be deterred from becoming a nuisance to neighbours, new and much stricter nuisance laws might have to be put in place. The social responsibility that was previously anchored in the community must now be protected more forcefully by the state. In turn, this can cause the institution of property to weaken further, as the state assumes greater power to interfere. A feedback effect might result, as increased state regulation in turn threatens to make property ownership too burdensome for average community members. Hence, the most resourceful actors, those who are able to meet the state's demands and/or protect themselves against interference, gain more and more property, while the state gains more and more regulatory power.

The social function theory tells us that mechanisms such as this should be taken seriously as potential consequences of changes to the property regime. Moreover, by prioritising between social functions of property, the law indirectly serves a regulatory function, since different property functions correspond to different kinds of concrete property uses. On the one hand, direct regulation of land use can potentially be replaced by a more nuanced approach to property law, e.g., by promoting property ownership for marginalised groups. On the other hand, attempting to serve the public interest through direct state interference can have undermine some of property's social functions, e.g., by leading to concentration of property rights in the hands of the most resourceful, creating a need for yet more forceful mechanisms of state control.

The broader point at stake here can also be brought out in relation to the famous ``tragedy of the commons''.\footnote{hardin68} In his seminal article, Hardin describes how individually rational users of a common resource can eventually cause the depletion of that resource. The problem arises, according to Hardin, because individuals have no proper incentive to refrain from over-exploitation; the damage will be distributed among all resource users, so it will not outweigh the benefit of individual over-use in the short term.

In response, it has been typical to regard either state management or individual private ownership as the answer.\footnote{See \cite[8-13]{ostrom90}.} State management is supposed to provide external incentives not to over-exploit, while private ownership is supposed to make it more difficult for individual resource users to shift the cost of over-exploitation onto the community.

However, as Elinor Ostrom and others have shown, the traditional narrative overlooks the fact that commons tend to come with community structures that provide appropriate incentives through locally grounded institutions or social arrangements.\footnote{See generally \cite{ostrom90}.} Moreover, as long as external forces do not threaten them, such arrangements can be more robust than either individual ownership or state control.

\noo{The former can be illusory, since damaging effects may not actually be limited to one's own land, e.g., in case of environmental harms. The second, on the other hand, can be ineffective due to the remoteness of the decision-makers to the effects of their decisions (causing a higher-order lack of appropriate incentives). In addition, state-led management can also increase the risk that the most important decision-making processes are captured by powerful interests, e.g., large commercial companies.}

The ideas of Ostrom on common pool management focus on local institutions for collective decision-making, not property rights. \noo{ Moreover, the idea of private property is not identified as an important anchor for such institutions, which are rather thought to operate independently of the property regime. By contrast, legal scholars discussing Ostrom's work have sometimes emphasised how common pool governance and private rights can be viewed as two sides of the property coin. This, specifically, is a key insight behind Smith's notion of a ``semicommons'', referring to property arrangements based on a combination of individual property rights and locally grounded institutions for collective action.\footnote{See generally \cite{smith00,smith02}.}

At the same time, some authors have pointed out weaknesses with the typical local institutions that tend to emerge for managing the commons. In particular, Heller and Dagan argue that such institutions often leave inadequate room for ``exit'' -- the possibility of alienating one's rights and obligations. This, they argue, can cause local institutions to become oppressive towards individual members.\footnote{See \cite{heller00}.} } As a complementary viewpoint, the idea that local institutions for resource management can be anchored in private property represents a potentially attractive way of thinking. Importantly, the social function theory of property already suggests pursuing this idea. Based on the social function approach, the descriptive fact that property structures shape decision-making processes at the local level is enough to conclude that local institutions for resource management should not be looked at in isolation from the law of property. %I will build on this perspective in Chapter \ref{chap:3}, when I consider the institution of land consolidation in Norway and the management tools it offers as a possible alternative to expropriation in economic development cases.

\noo{Importantly, it suggests the insight that local resource management institutions can operate on the basis of private property, and that informal management practice are in many cases anchored in such rights. 

Recognising this entails the further realisation that it would be plainly inaccurate to proceed to interfere with property or introduce new property rules on the assumption that the effects pertain only to individual entitlements. In addition, one should recognise institutional and regulatory effects.

It also bears noting that property rights can entail a wider set of values than those associated with specific institutions for local management of resources. For instance, making property a conceptual starting point makes it natural to also recognise that private property can function to protect individuals against abuse by local elites, e.g., by offering opportunities for exit. }

Moreover, by recognising property as an anchor for equity and decision-making at the local level, social obligations that inhere in private property may be recognised as existing independently of specific institutional arrangements. Hence, if local institutions are marred by corruption and malpractice, a social function theorist can take the normative stance that property ownership still carries with it duties to care for other property dependants in the community. This duty, moreover, would exist independently of the extent to which it is presently fulfilled through local practices and institutional arrangements. 

I will return to this point later, when I discuss the human flourishing theory of property and its promise of internalising economic and social rights for non-owners into the structure of property itself. First, I will argue that it is useful to distil a descriptive core from the social function theory, so that it may serve as a common ground for debate, allowing the interchange of ideas between different normative perspectives. \noo{Differences of opinion about what the law of property should be like should not detract from the insight that when discussing the law of property, one  that how one decides in this regard has important consequences for society as a whole.

In the next section, I will first briefly consider the much discussed US case of {\it State v Shack}.\footcite{shack71} The reason for doing this is two/fold. First, the case serves as the standard example of how social function reasoning can come to influence the application of rules that seem to be socially neutral. Second, the case was used by Eric Claeys to launch an attack on social function theorising. I consider his argument, concluding that while it sheds light on the outcome of {\it State v Shack}, it does nothing to detract from the descriptive core of the social function theory, quite the opposite.}

\subsection{The Descriptive Core of the Social Function Theory}

Social function theorists have been criticised for making too many normative claims. Eric Claeys, in particular, argues forcefully against normative fundamentalism and what he regards as normative naivety among social function theorists.\footnote{\cite[945]{claeys09}. (``Judges might think they are doing what is equitable and prudent. In reality, however, maybe they are appealing to a perfectionist theory of politics to restructure the law, to redistribute property, and ultimately to dispense justice in a manner encouraging all parties to become dependent on them.'')} Indeed, some social function theorists have gone very far in presenting the social function account of property as a normative theory, attaching specific political commitments to it on the way.

Hanoch Dagan, for instance, is a self-confessed liberal who argues for a social function understanding on the basis that it is morally superior. ``A theory of property that excludes social responsibility is unjust'', he writes, and goes on to argue that ``erasing the social responsibility of ownership would undermine both the freedom-enhancing pluralism and the individuality-enhancing multiplicity that is crucial to the liberal ideal of justice''.\footcite[1259]{dagan07}

If this is true, then it is certainly a persuasive argument for those who believe in a ``liberal idea of justice''. But for those who do not, or believe that property law is -- or should be -- largely neutral on this point, a normative argument along these lines can only discourage them from adopting a social function approach. Such a reader would be understandably suspicious that the {\it content} of the social function theory -- as Dagan understands it -- is biased towards a liberal world view. Such a reader might agree that property continuously interacts with social structures, but reject the theory on the basis that it seems to carry with it a normative commitment to promote liberalism.

Dagan is not alone in proposing highly normative social function theories. Indeed, most contemporary scholars endorsing a social function view on property base themselves on highly value-laden assessments of property institutions.\footnote{See, e.g.\cite{alexander09,crawford11,davidson11,singer09,penalver09}.} By contrast, the discussion in this chapter so far has aimed to demonstrate that the theory has significant merit already as a {\it descriptive} theory. In my opinion, this is also demonstrated by much of the normatively oriented work that has been done by scholars such as Dagan and Alexander. However, it seems that their focus on abstract normative assertions threaten to overshadow what is arguably the most important insight of their work, namely that considerations related to social functions {\it are} already important in many areas of property law, in many different jurisdictions. Moreover, the social functions of property, and normative assertions about them, often play a role behind the scenes, where they do unacknowledged work among policy makers and judges alike.

Because it embodies this crucial insight, the core of the social function theory, rather than being ``good, period'' as Dagan suggests, is simply accurate, irrespective of one's ethical or political inclinations. The theory provides the foundation for a discussion where different values and norms can be presented in a way that is conducive to meaningful debate, on the basis of a minimal number of hidden assumptions and implied commitments. \noo{
On this basis, Claeys criticism appears to have some merit. However, it seems that while it appropriately targets excessive normative claims that have been made in the name of the social function theory, it does not effectively target the



case of {\it State v Shack} is a standard US example of how social function reasoning can come to influence the application of rules that seem to be neutral on property's social functions.\footcite{shack71} The case concerned the right of a farmer to deny others access to his land, a basic exercise of the right to exclusion. The controversy arose after the two defendants, who worked for organizations that provided health-care and legal services to migrant farmworkers, entered the land of a farmer without permission. They were there to provide services to the farmer's employees, and when the farmer asked them to leave, they refused.

In the first instance, they were convicted of trespassing in keeping with New Jersey state law. However, the Supreme Court of New Jersey overturned the verdict on appeal. The Court held that as long as the defendants were there at the request of the workers, the owner's right to exclude them was more limited. Importantly, the court argued for this result -- which was not based on a natural reading of the New Jersey trespass statute -- by pointing also to the fact that the community of migrant workers was particularly fragile and in need of protection. Their right to receive visitors on the land where they worked and lived, therefore, had to be \isr{recognised}, also in a situation when this would involve a limitation to the farmer's right to exclude.

In so far as the property rules we rely on explicitly directs us to take the social aspect of property into account when applying the law, it might be permissible for the practically minded jurist to conclude that there is little need for general \isr{theorising} about property's social dimension. This dimension, in so far as it is relevant, is primarily a matter for the legislature, not theories that seek to explain property law. However, cases like {\it State v Shack} show that the social dimension can be relevant even when it is not mentioned in any authority, even in relation to clear rules that would otherwise appear to leave little room for statutory interpretation. It arises as relevant, in such cases, because the social dimension is intrinsic to property itself.

This might be a radical claim, but it is primarily a descriptive one. Indeed, even if the case of {\it State v Shack} had gone the other way, the same conceptual conclusion might well have been appropriate. If the right to exclusion had received priority over the workers' right to receive guests and the owner's obligation to respect this, that too would be an outcome underscoring the social function of property. 

A nice demonstration of how neutrality is elusive in this regard can be found in an article by Eric Claeys, where he is critical both of the social function theory generally and {\it State v Shack} in particular.\footcite{claeys09} Importantly, despite his intention to criticise the social function theory, Claeys is led to argue against the ruling of {\it State v Shack} by pointing to those aspects of the social context that spoke in favour of the farmer.\footnote{\cite[941-942]{claeys09}.} Essentially, his argument is that by considering the social circumstances in {\it more} depth, a different outcome suggests itself.\footnote{\cite[941]{claeys09} (``there are good reasons for suspecting that there was more blame to go around in Shack than comes across in the case's statement of facts'').} 

If this is true, it is no argument against the descriptive content of the social function theory. Rather, it becomes a further affirmation of the descriptive adequacy of such an account of property. At the same time, it becomes an argument against those who think that the social function idea dictates the ``correct'' outcome in cases such as {\it State v Shack}. As Claey's advocacy on behalf of the farmer shows, the descriptive part of the social function theory does not necessarily entail specific normative commitments.

Claeys argues forcefully against normative fundamentalism, and he might have a point in criticising some social function theorists for normative naivety.\footnote{\cite[945]{claeys09} (``Judges might think they are doing what is equitable and prudent. In reality, however, maybe they are appealing to a perfectionist theory of politics to restructure the law, to redistribute property, and ultimately to dispense justice in a manner encouraging all parties to become dependent on them.'')} However, I do not follow Claeys when he takes this to be an argument against the form of legal reasoning that social function theories promote and which he himself skilfully engages in.\footnote{In particular, I do not follow the leap Claeys makes when he suggests that it is beneficial to keep ``discretely submerged'' what he describes as ``culture war overtones'' in legal reasoning, see \cite[947]{claeys09}.}

In {\it State v Shack}, such reasoning was clearly in order. To engage in it was far less naive than to dismiss it on the basis that it would be irrelevant to the case. Indeed, if the social function view had been dismissed, the narrow exclusion-based idea of property would in effect do {\it unacknowledged} normative work, by pushing social aspects out of sight and out of mind.

By contrast, the social function narrative pushes us towards a more complete picture of the relevant facts. This is its primary contribution, in my opinion. However, many of its supporters appear to argue that the main significance of the theory is that it delivers an ethically superior approach to property law.\footnote{See, e.g., \cite{penalver09}.} Unsurprisingly, critics such as Claeys use this to launch attacks on the social function theory, by suggesting that it represents a way of thinking that will invariably lead to lessened constitutional property protection and greater risk of abusive state interference.\footnote{See \cite{claeys09} (``The more ``virtue'' is a dominant theme in property regulation, the less effective ``property'' is in politics, as a liberal metaphor steering religious, ethnic, or ideological extremism out of the public square'').} %Indeed, increasing the room for state interference is often seen as the aim of conceptual reconfiguration; the social function view of property tends to be associated with social democratic and/or redistributive political projects, by which the notion of property is recast to justify greater interference in established rights.\footnote{Despite his commitment to ``value-pluralism'', this motivation is also clearly felt in the work of Gregory Alexander. He argues, for instance, that the social obligations inherent in property imply that the ``state should be empowered and may even be obligated to compel the wealthy to share their surplus with the poor'', see \cite[746]{alexander09}. For an assessment linking similar views on property in Europe to the increasing influence of social democratic thought after the Second World War, see \cite{allen10}.}

%It is important to note, however, that while social democratic policies may be easier to justify by \isr{emphasising} the social function of property, the mere recognition that property has an important social dimension does not in itself justify such policies. Moreover, it seems that the most crucial premise used in arguments for greater state control and state-led redistribution projects concern the nature of the state, not the functions of property.

But why should it follow from property's social function that the state is the ultimate social institution to which property {\it should} answer? Why not take the view that property should answer to informal social structures, such as those that it is embedded in by virtue of owners' membership in local communities?

If so, one might as well want to limit the state's role, to allows local institutions to flourish independently. Indeed, from the point of view of social function theories, the appropriateness of direct state control seems to depend on evidence that existing property-based structures fail to function properly and, crucially, that state control is a {\it better} alternative.

%%This requires arguments anchored in specific social and political property contexts. Hence, to move uncritically between talk of the ``community'' and talk of the state, as writers like Pe\~{n}alver and Alexander sometimes appear to do, is inappropriate.

%In my opinion, the social function view of property tells us little about how widely the state should intervene in property in a given society. It allows us to recognize the {\it possibility} that the state may have to intervene on behalf of certain property values, say those that aim to protect communities. 
%But this is no argument in favour of any general position on state interference. 

The social function theory provides us with a conceptual tool for reasoning more clearly about {\it when} it is appropriate for the state to intervene. But the Humean position, namely that the existing property structure represents a socially emergent equilibrium, remains plausible. Moreover, the normative stance that this equilibrium is a {\it good} one (or at least as good as it gets) remains as contentious as ever.

It bears emphasising that by promoting normative neutrality at the conceptual level, I depart from the stance taken by many contemporary scholars who advocate on behalf of social function theories. Hanoch Dagan, for instance, is a self-confessed liberal who argues for a social function understanding on the basis that it is morally superior. ``A theory of property that excludes social responsibility is unjust'', he writes, and goes on to argue that ``erasing the social responsibility of ownership would undermine both the freedom-enhancing pluralism and the individuality-enhancing multiplicity that is crucial to the liberal ideal of justice''.\footcite[1259]{dagan07}

If this is true, then it is certainly a persuasive argument for those who believe in a ``liberal idea of justice''. But for those who do not, or believe that property law is -- or should be -- largely neutral on this point, a normative argument along these lines can only discourage them from adopting a social function approach. Such a reader would be understandably suspicious that the {\it content} of the social function theory -- as Dagan understands it -- is biased towards a liberal world view. Such a reader might agree that property continuously interacts with social structures, but reject the theory on the basis that it seems to carry with it a normative commitment to promote liberalism.

Dagan is not alone in proposing highly normative social function theories. Indeed, most contemporary scholars endorsing a social function view on property base themselves on highly value-laden assessments of property institutions.\footnote{See, e.g.\cite{alexander09,crawford11,davidson11,singer09,penalver09}.} \noo{ These scholars provide interesting insights into the nature of property, but they might overstate the desirable normative implications of adopting a social function view. In addition, they appear to believe we should embrace certain values and reject others. Hence, one is sometimes left with the impression that the social function theory has little to offer beyond the values with which it is imbued, which can in turn push the law in the direction that these writers deem desirable. 

For instance, it is Dagan's stated aim to propose a theory that promotes specific liberal values. ``There is room to allow for the virtue of social responsibility and solidarity'', he writes, continuing by suggesting that ``those who endorse these values should seek to incorporate them -- alongside and in perpetual tension with the value of individual liberty -- into our conception of private property''.\footcite[802]{dagan99} This view is reflected further in the concrete policy recommendations he makes, for instance in relation to the question of when it is appropriate to award less than ``full'' (market value) compensation for property following a taking.\footnote{See generally \cite{dagan14b}.}

Normative assertions like these are not necessarily wrong, but they need not be accepted in order to conclude that the social function of property should be given a more prominent place in the legal theory of property.} However, the focus on abstract normative assertions threatens to overshadow the most straightforward reason for looking to social structures, namely that they are almost always crucially important behind the scenes, even when they go unacknowledged by policy makers and judges.

For this reason, the social function theory, rather than being ``good, period'', as Dagan suggests, is simply more accurate, irrespective of one's ethical or political inclinations. As such, it provides the foundation for a debate where different values and norms can be presented in a way that is conducive to meaningful debate, on the basis of a minimal number of hidden assumptions and implied commitments.} Thus, the first reason to accept the social function theory is epistemic, not deontic.

\noo{ That is not to say that normative theories should not be formulated on the basis of the social function theory. I maintain only that  it is useful to maintain a distinction  between the descriptive and normative aspects of such theorising. I return to normative aspects in the next section, arguing that the commitment to ``human flourishing'' endorsed by Professor Alexander is a particularly well-argued norm that arises from value-based assessment of the social function of property. This, I believe, is in large part also due to the value-pluralism inherent in this idea, suggesting as a positive normative claim that our notions of property {\it should} allow for a divergence of opinions and values to influence the law and its application in this area.}

That is not to say that theories can ever be entirely value-neutral, nor that this should be a goal in itself. However, a good theory is one that can at least serve as a common ground for further discussion based on disagreement about values and priorities. \noo{According to Kevin Gray, ``the stuff of modern property theory involves a consonance of entitlement, obligation and mutual respect''.\footcite[37]{gray11} This is a rather loose way of putting it, but I believe it also points to a measured perspective that is ultimately highly appropriate.} Making room for normative divergences, moreover, can hopefully diminish the worry that a broader theoretical outlook is the first step towards unchecked state power and rule by ``judicial philosopher-kings'', as Claeys puts it.\footcite[944]{claeys09}

To further demonstrate that a cautious perspective is warranted here, it is useful to briefly consider how the Italian fascists appropriated the social function theory in 1930s. \noo{Building on the work of di Robilant, I will also briefly track how non-fascist property scholars opposed this development by focusing on value-pluralism, local self-governance and freedom.\footcite{robilant13} Importantly, these scholars embraced the social function theory as a common ground from which to launch a meaningful attack on more radical ideas, without alienating those with divergent views. Instead of clinging to the old-style liberal discourse that the fascists had either rejected or subverted, many Italian non-fascists were willing to engage in a discourse revolving around property's social function, by spelling out a more measured set of ideas based on this premise.

Crucially, this set the stage for a form of intellectual resistance that did not reject those aspects of fascism that had appeal to the public and which arguably reflected true insight into the unfairness and lack of sustainability of the established legal order.

\subsection{Rooting out Fascism}

\noo{ While the social function theory makes intuitive sense, it is also highly abstract. Therefore, its exact content has been notoriously hard to pin down. This is \isr{recognised} by contemporary scholars endorsing a normative view, who attempt to address this by proposing lists of values that should be taken into account while giving examples of how they should be used to inform the law in concrete areas or cases.\footnote{See, e.g., \cite{alexander14,alexander11,dagan07}.} Unsurprisingly, however, views soon diverge regarding the concrete import of a social function view on legal reasoning. Even so, the contemporary debate appears to be based on a common ground that is quite stable, also with respect to the overall notion of what good the theory can do. However, as history shows, this state of affairs is by no means guaranteed.

In a recent article, Anna di Robilant illustrates this point by tracking the history of social function theorising in Italy during the fascist era.}}

According to Anna di Robilant, the fascists were very happy to embrace a social function theory. This was because it provided them with an excellent conceptual starting point from which to develop their idea that rights and obligations in property should be made to answer to one core value: the interests of the state.\footnote{See \cite[908-909]{robilant13} (``Fascist property scholars had also appropriated the social function formula. For the Fascists, the social function of property meant the superior interest of the Fascist state.'').} This stance was as effective as it was oversimplified. As di Robilant notes, ``earlier writers had been hopelessly evasive about the meaning and content of the social element of property''.\footcite[909]{robilant13} Hence, the fascist approach filled a need for clarity about the implications of the main idea, which was by now attracting increasing support both from the public and the academic community. Established property doctrine, it was widely felt, was both ineffective and unfair to ordinary people. Rather than securing productivity and a livelihood for all, property was used mainly as an instrument for maintaining the privileged position of the elites. By promising to change this state of affairs, the fascists attracted many to their cause.

As di Robilant notes, supporters of the fascist idea of property made clear that ``social function meant the productive needs of the Fascist nation''.\footcite[909]{robilant13} But at the same time, they cleverly denied that there was a ``contradiction between subordinating individual property rights to the larger interest of the Fascist state and the liberal language of autonomy, personhood, and labor''.\footcite[900]{robilant13} In this way, fascist scholars could claim that fascist liberalism was true liberalism, thereby subverting the conceptual basis for the traditional idea of liberal justice.\footcite[900]{robilant13} In this situation, there was reason to suspect that clinging to liberal dogma would be a largely ineffective response. Moreover, it seemed undeniable that fascism's appeal was rooted in real concerns about the fairness and effectiveness of the liberal legal order. 

Hence, many non-fascists shunned away from uncritical defence of traditional liberalism. Instead, they agreed that property's social function should come into focus, but \isr{emphasised} the plurality of values that could potentially inform this function. Importantly, these values might not align with the interests of the state. In addition, non-fascist social function theorists noted that property rights were invariably associated with control over resources, and that the social functions of property depended on the resources in question. 
\noo{To own property, they argued, provides individuals with a highly valued source of privacy, power and freedom that is worthy of protection. }
To summarise this insight, Italian scholars adopted the metaphor of a ``tree'', by describing the core social function of property as the trunk, while referring to the various resource-specific values attached to property as branches.\footcite[894-916]{robilant13} %As di Robilant notes regarding these theorists:

%\begin{quote}
%The rise of Fascism, they realized, was the
%consequence of the crisis of liberalism. It was the consequence of liberals' insensibility to new ideas about the proper balance between individual rights and the interest of the collectivity.\footcite[907]{robilant13}
%\end{quote}

%In light of this, the tree-theorists concluded that continued insistence on the protection of the autonomy of owners was not a viable response. 
These tree-theorists proceeded to formulate a theory that ``acknowledges and foregrounds the social dimension of property'', without committing themselves to fascist ideas about the supreme moral authority of the state.\footcite[907]{robilant13} The value of autonomy was in turn recast in terms of property's social function. Arguably, this served to make the case far more compelling. Protecting autonomy could be seen as an aspect of protecting property's freedom-enhancing function, both at the individual level and as a way of ensuring a right to self-governance and sustenance for families and local communities. This, moreover, could not easily be derided as tantamount to protecting unfair privilege and entitlement. Rather, property became elevated from an individual liberal right to a crucial building block of participatory democracy.

The story of fascist appropriation of the social function theory demonstrates why it is sensible to maintain a descriptive perspective on its core features. Indeed, the readiness with which the fascists embraced social function \isr{theorising} serves as a reminder that we cannot easily predict what normative values may come to be promoted on its basis. At the same time, we are reminded of the danger of attaching too much normative prestige to a theory that is abstract and open to various interpretations.

\noo{In particular, it seems that a failure to \isr{recognise} the descriptive nature of the core idea can lead to unrealistic expectations of what the social function theory provides. In addition, it will make it harder for the theory to gain acceptance as a conceptual common ground from which to depart when engaging in debate. Indeed, if no division is \isr{recognised} between normative and descriptive aspects, the historical record would allow detractors to make a {\it prima facie} plausible attack on the social function theory by arguing that it is fascism in disguise, or that fascism, rather than liberal justice, is where we end up in practice should we \isr{choose} to adopt it.\footnote{This would echo the claim already made by Claeys, that the theory (when coupled with virtue ethics) might become a slippery slope towards the kind of extremism and revolt against oppression that gave rise to the Rwanda genocide in the early 1990s \cite[926-927]{claeys09}.}

In response, one might retort that this is cherry picking the historical facts, or that the fascists misunderstood or perverted the theory. That is certainly plausible, but this kind of debate is in itself rather unhelpful. Unless the social function theory is rendered neutral enough to be acceptable as the conceptual premise of debate, it is likely going to fail as a template for negotiating conflicts about property. Those who oppose the norms associated with the theory will oppose also the core descriptive content, if they feel that the latter commits them to the former. This, in turn, suggests that those advocating on behalf of the social function theory should take care to avoid rhetorical hubris. The main point to convey is that the theory is more accurate, in a purely epistemic sense, than other conceptualisations of property. }

That said, the new descriptive dimensions uncovered by the social function view can also inspire novel normative perspectives, as explored in the next section.

\noo{
The story of the fascist appropriation of the social function theory also points to a danger often attached to abstract theories, namely that they allow us to opportunistically recast whatever values we wish to promote by providing qualifications for them in abstract terms that are hard to refute. The fascists did this, and the non-fascists responded. Hence, in the end one could do little more than hope that the fascists' vision of their state as an ``ethical state'' that ``every man holds in his heart'' would eventually prove less attractive then the promise of self-governing communities characterised by diversity in life and character.

\subsection{Towards a Normative Stance}

The social function theory can facilitate a new kind of normative reasoning, arising from how the theory allows us to \isr{recognise} more subtle distinctions between different kinds of property and different kinds of circumstances. For instance, a staunch entitlements-based approach to autonomy will leave us with little room to differentiate between the protection of investment property and the protection of a home, unless such a distinction is explicitly provided for in the law. But a social function approach compels us to notice the difference and to acknowledge that it might be legally, as well as ethically, relevant. Hence, if we seek to argue for protection of investment property, we must in principle be prepared to face counter-arguments that revolve around particulars of the investor's role in society and their relationship to the community of people that are affected by how they manage their property. Similarly, if someone argues against protecting home ownership, we can respond by drawing on additional arguments based on the importance of the home both to the owner, their family and friends, and their community. Under the social function theory, it becomes relevant to address how a home creates a sense of belonging and provides a basis for membership in social structures.

I believe normative assessments should aim to be as concrete as possible. That said, I think it is worthwhile to provide more abstract forms of expression for core values, to clarify the ethical premises that provide the basis for concrete value-based conclusions. Therefore, normative theories should aim to be meta-ethical, not just ethical. They should provide a vocabulary and a conceptual framework tailored to advancing one's values. However, they should recognise that the ultimate expression of those values is provided only in relation to concrete facts. This, I believe, is prudent in light of how abstract ethical assertions are necessarily imprecise and run the risk of being distorted or exaggerated.

Invariably, the most accurate information regarding the values I rely on when assessing cases will be conveyed by my assessment of the cases, not by my theorising. On a deeper level, I am inclined to believe that value-systems are more or less unique to individuals, so that ethical theories are helpful primarily in that they provide an introduction to keywords and important lines of argument that will recur in different forms. As such, they enhance understanding, making it easier to communicate ideas and opinions in such a way that potential respondents are likely to enjoy a more accurate impression of what they are responding to. 

In short, I believe that ethics make moral judgements communicable, allowing new ideas to be created in the minds of individuals. I believe in ethical men and women, but not in ``ethical Man'' or -- God forbid -- the ``ethical State''. Luckily, I find some support for this view in recent theories that have been proposed as normative extensions of the social function theory of property. These are the subject of my next section.
}

\section{Human Flourishing}\label{sec:hf}

Taking the social function theory seriously forces us to \isr{recognise} that a person's relation to property can be partly constitutive of that person's social and personal capabilities, including both their political and economic components.\footnote{Reference on capabilities...} Hence, property influences people's preferences, as well as what paths lie open to them when they consider their life choices.\footnote{See generally \cite{alexander09}.} This effect is not limited to the owner, it comes into play for anyone who is socially or economically connected to property in some way, including a potentially large group of non-owners.\footcite[128-129]{alexander09d} %The importance of property is obviously reduced if we move away from it in terms of social or economic distance. Hence, in many cases, property will be most important to its owner, simply because they are closest to it. This is not always the case, however, especially not if property rights are unevenly distributed, or in the possession of disinterested or negligent owners. Moreover, as mathematically oriented sociologists like pointing out, social connections are ubiquitous and the world is often smaller than it seems.\footnote{See generally \cite{schnettler09}.}

Hence, there is great potential for making wide-reaching socio-normative claims on the basis of the social function perspective on the meaning and content of property. But which such claims {\it should} we be making? According to some, we should adjust our moral compass by looking to the overriding norm of {\it human flourishing} as a guiding principle of property law. In a recent article, Alexander goes as far as to declare that human flourishing is the ``moral foundation of private property''.\footcite[1261]{alexander14} 

Human flourishing has a good ring to it, but what does it mean? According to Alexander, several values are implicated, both public and private.\footnote{See generally \cite{alexander14,alexander11}.} Importantly, Alexander stresses that human flourishing is {\it value pluralistic}.\footnote{\cite[750-751]{alexander09}.} There is not one core value that always guarantees a rewarding life. To flourish means to negotiate a range of different impulses, both internal and external. Importantly, these act in a social context which influences their meaning and impact\footcite[1035-1052]{alexander11}

In the following, I consider some values that I regard as particularly important for the study of economic development takings. I start by the values enshrined in economic and social rights, which should arguably also inform our understanding of property law.

\subsection{Property as an Anchor for Economic and Social Rights}

The so-called ``second generation'' of human rights consists of basic economic and social rights that complement traditional political rights. This includes rights such as the right to housing, the right to food, and the right to work. Economic and social rights of this kind often involve property. Specifically, they often involve interests in property that are not recognised as ownership, e.g., housing rights for squatters or rights to food and work for landless rural people. 

If the notion of property is conceptualised in the traditional way, as an arrangement to protect individual entitlements, the relationship between private property and economic and social rights appears to be one filled with tension. In particular, if economic and social rights require owners to give up some property entitlements, it becomes natural to portray property protection as standing in the way of social justice. 

However, the human flourishing theory can be used to tell a very different story, namely one where economic and social rights are anchored in the notion of property itself. \noo{As I have already explained, I believe -- in contrast to both Crawford and Alexander -- that it is useful to decouple normative claims from the descriptive core of the social function theory.\footnote{Crawford comments that the social function theory on its own  ``is not self-defining and invites many interpretations'', see \cite[1089]{crawford11}. The normative theory he proposes is clearly aimed at \isr{filling} this perceived gap, by pinning down normative commitments that Crawford believes are intrinsic to the theory. However, as I have already argued, I reject this approach, since it unwisely downplays the fact that the social function theory can serve as a common ground among commentators with widely divergent normative views. Indeed, Crawford himself refers \isr{unfavourably} to a writer who addresses the social function theory, but who, according to Crawford, proposes that ``property's social function is best served by focusing on overall economic production and efficiency in a given society, allowing the market's invisible hand to work its magic'', \cite[see][1089]{crawford11}. Against Crawford, I would argue that it is better to counter such a claim by arguing why it is normatively wrong rather than by suggesting that people with such values should be 
discouraged from attempting to argue for them on the basis of a social function understanding of property. Rather, by encouraging such an argument it should become easier to make the case why the values promoted are ultimately undesirable. This, at least, should follow if Crawford is otherwise largely correct (as I think he might be).} I therefore refer to the more distinctly normative aspects of their work as human flourishing theorising.}

%Indeed, if securing human flourishing is the primary purpose of property, our property rules should protect the economic and social rights of the community of property dependants as a whole, not just the owner.\footcite[1035-1052]{alexander11} Hence, protecting such economic and social rights would not necessarily entail interference with private property. In many situations, rather, the human flourishing view would allow us to add the interests of non-owners to the list of reasons why property interference is best to avoid.
Importantly, the human flourishing theory compels us to take into account the interests and needs of property dependants other than owners. As Colin Crawford puts it, the purpose of property should be to ``secure the goal of human flourishing for all citizens within any state''.\footcite[1089]{crawford11} Consider, for instance, the right to housing. If the interests of a property owner come into conflict with the housing rights of a property dependant, the human flourishing theory encourages us to approach this as a tension {\it within} property, between different property functions. 

%This way of understanding economic and social rights for non-owners can inform new perspectives on how to approach such rights when they involve property. 
\noo{

Indeed, if the owner of a house rent out their property as a home to someone else, the importance of the property might be {\it greater} to a non-owner. Moreover, assuming a society where tenancy is a well-functioning social institution, the continuation of the established property pattern might well be of greater importance to the tenant than it is to the owner. 

State interference in these private property rights might offend against human flourishing not only because it interferes with the landlord's autonomy and financial entitlements, but also because it interferes with the housing interests of the tenant. Both are aspects of property's social function, which the law should protect, also against state interference.

If the interests of the owner come into conflict with the housing rights of a property dependant, the resulting tension can be seen as a tension {\it within} property, between different property functions. For instance, if a piece of land has an owner, it discourages squatting. This is perhaps a good thing, but can also make it harder for the marginalised to find adequate housing.} With such a starting point, we should also acknowledge that the appropriate way to approach the rights of non-owners in relation to property might well depend on who the owner is and the choices they make in managing their property.

For instance, if owners live on their land and don't own much more than they need themselves,  squatting is effectively discouraged. At the same time, if this situation obtains it is harder to maintain the criticism that private property is somehow an affront to the housing rights of the landless. Similarly, but possibly more controversially, the owner of an unoccupied building can discourage squatting by managing the property well. Moreover, this too can undercut potential criticism on the basis of housing rights, especially if the owner uses the building to engage in commercial activity that contributes to sustaining the local community.

On the other hand, if owners mismanage their properties, for instance because they seek to obtain demolition licenses or engage in other forms of speculation, squatters might take opportunity of this and feel encouraged to occupy the property. This risk clearly increases if housing cannot be afforded by a large number of a society's members. 

\noo{ Importantly, the human flourishing account allows us to argue that something is amiss with prevailing property structures in such cases. In particular, the theory allows us to recognise that possible failures of property as a social institution might be relevant when considering rights and responsibilities of private ownership. Moreover, it becomes possible to see the fulfilment of social and economic rights as aspects of property's proper function, which property law should promote.

\noo{ Squatting, for instance, clearly affect the owner, influencing both the meaning and the value of \isr{their} property, both to \isr{them}, potential buyers, the local government, the state, and other interested parties. Even the mere {\it risk} of squatting can play such a role. But a property theory which does not \isr{recognise} the social function of property might not allow us to take this into due regard. As long as the standard expectation of an owner is to be able to enjoy their apartment free of squatters, an entitlements-based view on property could easily force us into denial regarding actual (risk of) squatters.}}

If private property is thought of merely as entitlement-protection, the state might stake from this the lesson that interference in property is required to secure housing rights, even though the real problem is that property itself does not function as it should within society. Hence, the result can be that property structures are damaged further, as the state pursues policies of interference and centralised management, without addressing how private property as such can promote human flourishing.

\noo{  The normative significance of real life -- where squatting often happens due to badly managed property -- is discounted because our conceptual glasses block it out. Then, the unavoidable consequence is that the state also \isr{recognises} a {\it positive} obligation to forbid squatting, and to forcibly remove squatters on behalf of owners. Under a narrow entitlements-based conception, this is the natural outcome, which must be classified as an act of protecting private property. Hence, under classical liberal values, it also becomes {\it good}. 

Here, however, the social function theory permits us to take a highly divergent view, to carry forward different value-judgements. If squatting is \isr{recognised} as creating new interests and obligations attaching to the property, it may now be argued that it is the use of state power to evict that is the most severe act of interference. Moreover, this might be more than merely an interference in whatever housing rights the squatters might have, if any. It can also be seen as an interference in {\it property}, in need of further justification. 

In the Netherlands, the Supreme Court adopted a line of reasoning reflecting a similar insight when it held that the right not to be disturbed in one's home life also applied to squatters. Hence, the property owner could not forcibly evict people who had taken up residence in their property.\footnote{See NJ 1971/38. The court held that the lower court had erred in taking it proven that the ``house in the original charge was \isr{`in use'} by the owner of this house'', as required by the statute under which the squatters were tried. Instead, the Supreme Court held that ``art.138, in so far as it mentions houses, is specifically aimed at protecting home rights, in connection with which the words \isr{`in use'} (differently than the court judged) can only be understood as \isr{`actually in use as a house'}, as in accordance with ordinary use of language''. The upshot was that it was the squatters, not the owner, who enjoyed protection under the statute. In terms of the bundle theory, a right thought to be in the owner's bundle was deemed to actually belong to the bundle of the squatters, as this corresponded better to the circumstances of the case and the purposes meant to be served by the statute in question.}}

By contrast, the human flourishing narrative suggests that both owners and non-owners might appropriately be viewed as victims if the state fails to protect property's proper functions. This perspective might even suggest itself when owners and non-owners would otherwise appear to be adversaries. For a concrete example, I mention the case of {\it Modderklip East Squatters v. Modderklip Boerdery (Pty) Ltd}, analysed in depth by Alexander and Pen\~{n}alver.\footcite[154-160]{alexander11} 

The case dealt with squatting on a massive scale: some 400 people had initially taken up residence on land owned by Modderklip Farm, apparently under the belief that it belonged to the city of Johannesburg. The owner attempted to have them evicted and obtained an eviction order, but the local authorities refused to implement it. Eventually, the settlement grew to 40 000 people and Modderklip Farm complained that its constitutional property rights had not been respected.

The Supreme Court of Appeal concluded that Modderklip's property rights had indeed been violated, but noted that so had the rights of the squatters, since the state had failed to provide them with adequate housing.\footnote{See \cite{modderklip04}.} However, the Court upheld the eviction order and granted Modderklip Farm compensation for the state's failure to implement it. Hence, while the Court recognised that both housing rights and property rights were at stake, it pursued a traditional balancing approach, finding in favour of property.

The Constitutional Court, on the other hand, adopted an agnostic view on the relationship between housing rights and property rights. The Court agreed that the eviction order was valid, but concluded that as long as the state failed in its obligations towards the squatters, the order should not be implemented.\footcite{modderklip05} Hence, the owner's right to have the squatters evicted was made contingent upon an adequate plan for relocation. 

However, the Court also ordered that Modderklip should receive monetary compensation from the state. In this way, the Court implicitly \isr{recognised} the social function of property; they refused to give full effect to Modderklip's property rights as long as that meant putting other rights in jeopardy. The fact that the squatters had no place to go influenced the content of Modderklip's right, making it impermissible to implement a standing eviction order. Importantly, however, this was simultaneously a failure to appropriately protect the owners, on the basis of which compensation should be paid.

The failure to protect property that the Court recognised in {\it Modderklip} can be understood broadly, as encompassing also the failure to provide adequate housing to a large group of property dependants. Indeed, this perspective is suggested by how the housing rights of the squatters influenced the content of the property rights and obligations of Modderklip. Taking this one step further means recognising that protection of property can be a potential source of justice for anyone, including squatters.

In a detailed analysis of {\it Modderklip}, Alexander and Pe\~{n}alver go quite far in this direction. They argue that the case highlights how property owners themselves can have responsibilities towards property dependants, obligations that endure as long as private property is protected.\footnote{\cite[157]{alexander11} (``The courts' unwillingness to ratify Modderklip's desire to remove the squatters from its land illustrates the courts' willingness to take seriously the obligations of owners, not only as they concern owners' direct relationship with the state but also in relation to the needs of other citizens'').} This normative turn makes property owners addressees of obligations arising from the economic and social rights of non-owners. In this way, it strengthens such rights. 

However, it also strengthens the institution of property, highlighting why it might be appropriate to grant it strong protection against interference by external forces. In particular, a human flourishing approach might serve as a bulwark against the idea that the ultimate expression of the public interest can be found in the actions taken by the state. In this way, the human flourishing theory also points towards a novel way to address the economic and social rights of marginalised groups, for whom it often appears that neither private property nor state management is capable of delivering basic justice.\footnote{For instance, it has been noted that in India, the human right to water has at times been simultaneously frustrated both by a non-egalitarian distribution of riparian rights as well as a regulatory framework that grants the state almost limitless proprietary power over water resources. See \cite[186]{cullett12}.
%Cullet recommends addressing this by a system that ``leaves aside property rights altogether According to Cullet, both these features of the Indian system are problematic, because they can give rise to management decisions regarding water that improperly prioritise the interests of more prosperous members of society at the expense of weaker groups. Se
To address shortcomings of the current system for water management in India, Cullet recommends a conceptual approach that ``leaves aside property rights altogether'' (both private and public) and instead emphasises that water is a common heritage of humankind. Leaving aside the special characteristics (and great importance) of water as a physical substance, the human flourishing theory points to an alternative, whereby fairness is to be achieved by embracing a progressive notion of property, rather than by abrogating it altogether.}
\noo{
\footnote{See \cite[186]{cullet12}.} This, Cullet argues, is a good way to give legal effect to the idea that sustainable management and equitable access to water is a common task and responsibility.\footnote{See \cite[188]{cullet12}.} Moreover, Cullet argues that a shift away from proprietary narratives will increase the level of democratic accountability on part of the state, particularly towards the landless and other groups that would otherwise easily come to find themselves marginalised.\footnote{See \cite[188]{cullet12}.}

It seems unclear, however, whether it is in fact possible to remove all proprietary characteristics of water resource management, to arrive at a truly equitable framework of democratic decision-making that involves no concept of ownership of resources. Indeed, that this can be hard to achieve appears to be illustrated already by Cullet's observations on the shortcomings of public trust and public good doctrines in the context of Indian water management. The tendency, it appears, is that states and other powerful stakeholders invariably tend to exercise power that has a proprietary character, particularly in the sense that their decisions tend to reflect the relative political and economic strength of different groups. The poor, in particular, might remain marginalised because they remain powerless in practice, despite being nominally protected by a range of human rights provisions in theory.

As long as the regulatory framework does in fact leave in place a proprietary form of resource management, with unequal distribution of power among people and institutions, one may argue that principles of property should still be considered relevant. Specifically, it would seem appropriate to hold on to the idea that proprietary interests comes with inherent social obligations, not contingent on positive regulation, as stressed by human flourishing theorists. Moreover, one might be in a position to argue that some forms of property are more conducive to social justice than others. For instance, in a context where land is distributed in an egalitarian fashion, or constrained by appropriate (informal) governance structures, it might be that decoupling land rights from water rights by turning water rights into a separately traded commodity is a much less attractive option than continuing to tie some water rights to land ownership. This, moreover, can be particularly attractive for certain types of property rights involving water, such as the right to make commercial use of water not needed to satisfy basic human needs.\footnote{Indeed, it would appear easier to implement basic access rights to water in a system that relies on state management of this particular aspect as a separate concern, while vesting commercial interests in water in some other parties, e.g., by allowing it to remain attached to the surrounding land. If a decision-maker, private or governmental, is ever awarded a choice between prioritising drinking water to the poor and commercial profit for the powerful, it seems highly likely that social justice will suffer. By contrast, if water as sustenance  is owned by no one and managed by the state, no such choice will ever need to be made, as long as water as a commercial asset is managed by someone else, such as the land owners.}}

%In particular, the human flourishing theory appears to offer an alternative route to economic and social equity, through an embrace of private property. 
Specifically, the human flourishing account suggests the view that public interests and obligations can acquire legal relevance even in the absence of international treaties or equitable decision-making within (inter)national institutions. This is so because the values typically associated with the public sphere, such as those pertaining to economic and social rights, are in fact legally relevant already at the level of interaction between private subjects.\footnote{See \cite[1295-1296]{alexander14}.} \noo{ , including in relation to private property. As Alexander puts it in a recent article:

\begin{quote} The values that are
part of property's public dimension in many instances are necessary
to support, facilitate, and enable property's private ends.
Hence, any account of public and private values that depicts them as categorically
separate is grossly misleading. One important consequence of this
insight is that many legal disputes that appear to pose a conflict between
the private and public spheres or that seemingly
require the involvement of public law can and
should, in fact, be resolved on the basis of private law -- the law
of property alone.\footcite[1295-1296]{alexander14} \end{quote} } Perhaps the most important structural aspect of this insight concerns the mechanisms used to resolve tensions between different property values. Importantly, it might not be necessary to introduce intermediaries between owners and other primary stakeholders. To introduce such intermediaries, whether they are state bodies, international institutions, NGOs, or commercial enterprises, carries with it the risk that the decision-making process can be captured by external forces. It might be better, therefore, if the necessary balancing of interests occurs at the level of property law.\footnote{This can also involve institutions as long as they are directly based on property, set up to facilitate participatory decision-making about property among the class of property dependants most directly affected. For private owners of parts of jointly owner property, the land consolidation courts discussed in Chapter \ref{chap:4} are an example of such a mechanism. However, as I discuss more in that chapter, exporting this institution to a setting of non-egalitarian property ownership might require granting legal standing to a larger group of property dependants.}

\noo{ This would also allow economic and social rights to be promoted to the private law sphere, even in the absence of specific state action or legislation. Effectively, it would become necessary for private parties to recognise that human rights give rise to obligations that apply to them directly, as property owners. If successful, this idea can obviate the need for direct state interference to secure desirable social and economic objectives. 

At the same time, a human flourishing account could be used to argue against the legitimacy of interference on the basis of broader concerns rooted in communitarian economic and social values. If such values inform property's proper function, it underscores that the state has an obligation to not undermine property. Instead, the state's duty becomes that of continuously facilitating improved legal frameworks for private property.

A clear commitment to property as an institution is needed in order for this aspect of the human flourishing idea to come to fruition. Moreover, such a commitment can not be taken for granted. A good example is again found by looking to South Africa, specifically the fallout of the Mineral and Petroleum Resources Development Act 2002.\footnote{See \cite{mrdpa02}.} This act introduced state ``custodianship'' over mineral and petroleum rights that had previously been privately owned.\footnote{See \cite[3]{mrdpa02}.} Importantly, in {\it Agri South Africa v Minister for Minerals and Energy}, the Constitutional Court ruled that this did not amount to expropriation, only deprivation.\footnote{See \cite{agri13}.} 

The property clause in section 25 of the South African Constitution gives effect to an important distinction between expropriation and deprivation, as it only demands compensation in case of expropriation.\footnote{See \cite[18-19]{walt11}.} Hence, the decision in {\it Agri} implied that privately held mineral and petroleum rights in South Africa could be brought under state custodianship without any payment of compensation.\footnote{For a commentary on the decision, see \cite{marais15a,marais15b}.}

The state custodianship introduced by the act in question would undoubtedly deprive the current owner of their mineral rights. Moreover, the state would subsequently be empowered to grant the minerals to a third party, e.g., a competing commercial company. In my terminology, therefore, the introduction of state custodianship amounted to a sweeping authorisation for economic development takings affecting all mineral and petroleum resources found in South Africa.

The crucial finding of the majority in {\it Agri} was that this did not involve any transfer of mineral rights to the state. On plain reading, this seems quite absurd. In effect, the rights would be taken from the owner by the state, which would then be free to pass those rights onto someone else. However, the majority did not engage in any plain reading. Rather, its decision was motivated by what it thought would be appropriate given the history of South Africa and the prevailing social context of property, shaped by a past of racial discrimination.

Importantly, the majority did not use this past as an argument to inform the assessment of the owners' compensation rights in the concrete case that was before the Court. Instead, it relied on the social context as an argument when interpreting the word ``expropriation'' in full generality. This may have been a grave mistake. As the minority pointed out, the understanding of expropriation established by the majority ``in effect immunises, by definition, any legislative transfer of property from existing property holders to others if it is done by the state as custodian of the country's resources, from being recognised as expropriation''.

This is not a precedent that strengthens the institution of property. Moreover, it reverts back to the traditional narrative that curing social ills requires negating private property or at least placing it under direct state management. It even seems to indirectly diminish the social obligations of property owners. Indeed, to the majority, the uncompensated taking of mineral rights was justified under reference to the ``obligation imposed by section 25 not to over-emphasise private property rights at the expense of the state’s social responsibilities''.

Notice how social obligations are talked about as though they only target states. On the Court's narrative, honouring ``social responsibilities'' can only mean enhancing state power at the expense of private property. Hence, the implication appears to be that in the absence of active state interference, private property is a privilege unbridled by social obligations. However, in the future of mineral and petroleum exploitation in South Africa, this will be a privilege enjoyed only by those chosen by the state.

The principle established in {\it Agri} could prove very important to the future of property in South Africa. Because {\it Agri} was decided on the basis of an interpretation of the word ``expropriation'' as such, there is no need in the future to continue pointing to the social context of past injustice when arguing that the state should be allowed to carry out economic development takings. As a custodian, the state is by definition entitled to do so, without paying any compensation at all to the original owners, regardless of the social merits of the taking. This opens the floodgates to predatory practices, creating very strong incentives for the government to abuse its power.

It is interesting to note that on this point of principle, {\it Agri} could easily have gone differently. This would have been achieved by a ruling consistent with the minority opinion, which held (1) that compensation should be paid and (2) that sufficient compensation {\it had} in fact been paid, as the mineral and petroleum act provided for a limited form of compensation in kind whereby the previous mineral owners were given a chance of maintaining substantially the same mineral interests provided they could meet certain conditions within a deadline (the mineral owner in {\it Agri} had been unable to do so due to insolvency).

This did not amount to anything resembling market value compensation, but the minority invoked the social context concretely to argue why it was nevertheless sufficient. Hence, by following the opinion of the minority, the outcome would have been the same, but the apparent implications for the institution of property would have been very different.

This shows the importance of one's conceptual understanding, since the main point of difference between the minority and the majority arose with regard to the role that the social context of property should play in legal reasoning. The minority invoked such concerns concretely, to determine the extent of a concrete right to compensation. By contrast, the majority invoked this reasoning at the highest possible level of abstraction, by holding that owners generally have no compensation rights following property deprivation in favour of state custodianship (even when this entails a right for the state to give the property to third parties).

The human flourishing theory clearly speaks in favour of the minority view. However, it also asks us to consider the possibility of going a step further, to challenge the very idea that state custodianship is the best way to address past injustices or social ills more generally. Indeed, it would perhaps have been more appropriate to look for a solution consistent with the protection of private property, for instance by building on {\it Modderklip} to improve the legal position of non-owning property dependants and communities, who may well be entitled to a share of the benefit arising from privately owned mineral and petroleum resources.
}
Ideally, it should be possible to pursue key economic and social values without massively increasing the power of the state and weakening the institution of property. At the same time, it should be possible to more effectively enforce social obligations on private property owners.\footnote{Indeed, enforcing such obligations will become much harder after a new class of owners are in place, chosen and approved of by an increasingly powerful state. } Achieving this in practice requires mechanisms that enable negotiations between competing private property interests, to facilitate a balancing of those interests through participatory decision-making rather than top-down state management. This highlight the importance of another property value that the human flourishing theory emphasises, namely that of participation, discussed in the following section.

\noo{ he Constitutional Court of South Africa essentially held that economic development takings are not takings at all, so that no compensation needs to be paid. 


It is interesting to note that this argument is based on a rather narrow understanding of what property is, contrasting with the broader view of property implicit in the South African constitution. 


The enemies of property, it seems, are first in line when it comes to adopting a traditional narrative that only recognises a narrow role for property institutions. In this, they are closely followed by neo-liberals who are similarly interested in a narrow notion of property that frees them from obligations and allows them to engage in exploitative practices. It is not a suprise therefore that the phenomenon of land grabbing is a thouroughly collaborative effort between statist national governments in Africa and 


 narrowing the understanding of property in South African law. 


suggestion that property should be strengthened to facilitate a reduction in the level of state interference might be hard to sell in practice. 


is too narrowly understood can be should be looked at more broadly can fuel those who wish to undermine property as an institution by stripping it of content.




can be weakened 



recast in terms that suit the powers that be, no


y in which property

 emphasised by the human flourishing idea, but it is also a potential weakness. 

There is a danger that this particular aspect of the human flourishing approach will not be successful, as property is 

 in practice, as it requires powerful state actors to g

given that the Constitutional Court upheld both the property rights of {\it Modderklip} and the housing rights of the squatters. Under a traditional property narrative, by contrast, one would expect the Court to perform a balancing test to give priority to one over the other. But the Court did not do so in {\it Modderklip}, preferring instead to focus on the fact that the state had provided inadequate access to an effective remedy for both the owners and the squatters. 

However, in the more recent case of {\it Agri South Africa v Minister for Minerals and Energy}, the Constitutional Court explicitly refused to regard ``state custodianship'' over mineral rights as expropriation. 

 a property limiting perspective -- whereby private property ireplaced by state control to solve social and economic problems -- appears to be favoured by the political leadership in South Africa. Since 2002, mineral and petroleum rights in South Africa have been put under ``state custodianship'', effectively authorising uncompensated state takings of mineral rights. 


However, the reconceptualisation in terms of property itself having a social function appears highly attractive. Moreover, it is also consistent with the South African constitution, which also focuses on property's social dimension.\footnote{See section 25 of the Constitution of the Republic of South Africa, Act 108 of 1996.}



This conclusion requires taking a normative stance, but a minimal one; we merely extend the scope of values traditionally attached to property.\footnote{Arguably, cases such as {\it Modderklip} might be taken to suggest that the social function theory, as soon as it is applied for the purposes of normative assessment, will systematically guide us to conclude that owners are not entitled to as many benefits as would otherwise follow from their property rights. It is fortunate, therefore, that the entire remainder of the thesis will focus on economic development takings, where it will typically appear more natural to conclude the opposite. In these cases, on a common- sense understanding of justice, applying the social function theory will allow us to \isr{recognise} a sense in which owners should receive {\it increased} protection and more benefits, as a consequence of how such interferences can prove particularly damaging, both to the owner and to the social fabric of democracy.} 

That said, in the case of {\it Modderklip} the court was clearly faced with a value conflict that it is hard to resolve by looking to traditional liberal values. If these apply equally to the squatters, we are left with deadlock rather than resolution. Indeed, this was also reflected in the outcome of the case, which did not resolve the matter, but merely concluded that the state had failed in its obligations towards both of the parties. What should the solution be in the end? Should the squatters be allowed to stay, following condemnation of Modderklip's land, or should alternative housing be provided so that the eviction order can be carried out? The answer depends on how we resolve a normative conflict, and how to do so might not be obvious. Moreover, value pluralism suggests that we must be prepared to engage with multiple ways of looking at the matter. In the interest of stability of property as an institution, allowing the squatters to succeed in establishing lasting title to the land might be considered unwise. Against this pragmatic and largely technical value, one would have to consider the values of community and belonging that now attach the squatters to their new homes. These two values are largely incommensurable, and it is not clear how to choose between them.

Still, Alexander maintains that human flourishing provides an ``objective'' standard on which to approach dilemmas such as these. Moreover, he ``rejects the view that what is good or valuable for a person is determined entirely by that person's own evaluation of the matter''.\footcite[1263]{alexander14} Some things are good for people, Alexander argues, \isr{irrespective} of whether or not people know so themselves. Hence, it may perhaps be argued that what is truly good for Modderklip is to come to an arrangement with the squatters and the state, to resolve the problem amicably. Moreover, failure to do so might entitle the state to take action that would otherwise seem to undermine the stability of property. This, then, would be partly due to this being conducive also to the flourishing of the people behind Modderklip, not only the squatters.

This might be derided as an overly intrusive and moralistic way to approach property law. More generally, as Alexander notes, the exact content of goodness is ``necessarily contestable''. It consists of a list of different values which are all open to dispute, both as to their relevance and their precise meaning.\footcite[1263]{alexander14} Alexander goes on to list some key values that he believes are central, but the list is not meant to be exhaustive.\footcite[1764-1776]{alexander14}




Among the key values that Alexander discusses, we find many core private values that are commonly seen as important goals for the institution of property. This includes values such as autonomy and self-determination, both of which will feature heavily later in this thesis. However, Alexander also considers several public values, such as equality, inclusiveness and community. These too will be important later, as I will draw on them in my own normative analysis of economic development takings. I will be particularly concerned with the value of {\it participation}, understood, following Alexander, in terms of its broad social function.\footcite[1275-1276]{alexander14}
}

\subsection{Property as an Anchor for Democracy}

The value of participation is closely related to the value of democracy. Participation in local decision-making processes is the root which enables democracy to come to fruition at the regional and national level. Of course, the role that property plays in this regard, as it empowers and encourages owners to take active part in the political process, has been noted before.\footnote{For a thorough assessment of the idea that property, for this reason, is the most fundamental right, I refer to \cite{rose96}.} However, as Alexander notes, the value of participation is often drawn up too narrowly, as pertaining only to people's engagement with the formal affairs of the polity.\footcite[1275]{alexander14} For Alexander, participation has a broader meaning, involving also the value of being included in a community. He writes:

\begin{quote}
We can understand participation more broadly as an aspect of inclusion. In this sense participation means belonging or membership, in a robust respect. Whether or not one actively participates in the formal affairs of the polity, one nevertheless participates in the life of the community if one experiences a sense of belonging as a member of that community.\footcite[1275]{alexander14}
\end{quote}

Importantly, participation in a community can have a crucial influence also on people's preferences and desires.\footnote{For a more in depth discussion of this, see \cite[140]{alexander09}. Here, Alexander and Pe\~{n}alver draw on the work of Amartya Sen and Martha Nussbaum, see generally \cite{sen84,sen85,sen99,nussbaum00,nussbaum02}.}
%\begin{quote}
%The communities in which we find ourselves play crucial roles in the formation of our preferences, the %extent of our expectations and the scope of our aspirations.\footcite[140]{alexander09}
%\end{quote}
Therefore, for anyone adhering to welfarism, rational choice theory, utilitarianism or the like, neglecting the importance of community is not only normatively undesirable, it is also unjustified in an epistemic sense. In particular, it should be \isr{recognised} as a descriptive fact that community is highly relevant to {\it any} normative theory that attempts to take into account the preferences and desires of individuals.\footnote{Again, I think Alexander and other theorists attempting to incorporate such ideas in property law could benefit from making this descriptive point separately, so as to enable it to be considered in isolation from the more contentious normative arguments they construct on its basis.} But Alexander and Pe\~{n}alver go further, by arguing that participation in a community should also be seen as an independent, irreducibly social, value, not merely as a determinant of individual preferences and a precondition for rational choice. They write:

\begin{quote}
Beyond nurturing the individual capabilities necessary for flourishing, communities of all varieties serve another, equally important function. Community is necessary to create and foster a certain sort of society, one that is characterized above all by just social relations within it. By ``just social relations'', we mean a society in which individuals can interact with each other in a manner consistent with norms of equality, dignity, respect, and justice as well as freedom and autonomy. Communities foster just relations with societies by shaping social norms, not simply individual interests.\footcite[140]{alexander09}
\end{quote}

This, I believe, is a crucial aspect of participation. Moreover, it is a notion that invariably leads us to recognise that other property dependants should also have a voice, as they form part of the ``just social relations'' within the community to which the owners belong. In addition, this is a notion of participation that it is hard, if at all possible, to incorporate in theories that take preferences and other attributes of individuals as the basis upon which to reason about their legal status. Instead, the human flourishing perspective asks us to consider how property serves as an anchor for participation that shapes and influences community norms and preferences. To protect the function that property plays in this regard is not straightforward under traditional property theories. To see this, notice how the property protection called for here might even require protection against the actions of the community itself.

This can happen, for example, in a community where people have come under pressure to sell their homes to a large commercial company that wishes to construct a shopping mall. These plans, it would appear, might be considered an attack on local property. Importantly, under the human flourishing theory, this may be so \isr{{\it irrespective}} of what the individual owners and other community members think they should do. If owners are offered generous financial compensations for their homes, or if they are threatened by eminent domain, economic incentives might trump the value of social inclusion and participation. As a consequence, the community might decide to sell. 

Even so, in light of the value of community, it would be in order for planning authorities, maybe even the judiciary, to view such an agreement as an {\it attack on their property}. It is clear that by the sale of the land, the ``just social relations'' inhering in the community will be destroyed. The members of the community -- including all the non-owners -- will lose their ability to participate in those relations. The property rights that once contributed to sustaining just relations will now be transformed into property rights that serve different purposes. This includes aiding the concentration of power and wealth in the hands of commercially powerful actors. Such a change in the social function of property might have to be regarded -- objectively speaking -- as a threat to participation, community and democracy. Hence, on the human flourishing theory, it is also a threat to property that our property institutions should protect against.

In Norway, a range of such rules are in place to protect agricultural property, by limiting the owners' right to sell parcels of their land without local government consent, as well as by compelling them to reside on their property and to make use of it for agricultural production.\footnote{See \cite[8|12]{la95} and \cite[4|5]{lca03}.} When the law actively promotes egalitarian property in this way, the natural counterpart is limit direct state interference. The danger otherwise is that the limited economic strength of each individual property owner -- appropriate in a democracy of property owners -- is exploited by the state or other powerful stakeholders who might wish to usurp control over local resources and impose their will on local populations.

The broader issue at stake here is highlighted by recent developments in South Africa, where  rules closely resembling those found for agricultural property in Norway have been proposed in a recent act on land reform. In South Africa, however, these rules have been proposed alongside a new framework of ``state custodianship'' of agricultural land, corresponding to a formulation recently introduced in the mineral and petroleum legislation.\footnote{See \cite{steyn15}.}

If the proposal passes, the proper functioning of agricultural property in South Africa will seem to depend very strongly on the benevolence of the state, which will greatly increase its own power to interfere with private property. This, one worries, contradicts the aim of creating a property regime that is truly democracy-enhancing.

The human flourishing perspective suggests that even when provisions to promote egalitarian ownership and community commitment are appropriate, provisions that inflate the state's authority might not be. As I believe the history of democracy in Norway shows, strict property rules to protect and promote self-governing agrarian communities can work well, as long as they are applied consistently and coupled with strong institutions of local democracy and strict limits on state power.\footnote{I discuss the role of agrarian property to the development of Norwegian democracy in more depth in Chapter \ref{ref:chap3}.}

\noo{ \subsection{The Duty to Contribute}

There is a subtle issue that arises on the basis of normative reasoning about individual rights as a source of human flourishing. Is it appropriate to think of such reasoning -- and the obligations it gives rise to -- as an aspect of protecting individual rights? Is it not more accurate to say that this kind of reasoning asks for interference with individual rights, undertaken to make sure that those rights contribute towards fulfilment of public and community interests? 

Indeed, it might seem peculiar to insist that enforcing stricter social obligations for owners is an aspect of protecting property rights. However, we might still be talking about protecting individual rights, even when this means imposing protections on people that they themselves do not want. Undoubtedly, this is {\it also} an interference in their rights, but just as different rights of different people can sometimes come into conflict, it seems that aspects of the same right, for the same person, can sometimes come into conflict with itself. This happens, in particular, when it is not possible to simultaneously protect all those functions that this right seeks to promote. 

For instance, if someone protests a taking on environmental grounds and also rejects financial compensation as immoral, the courts should still award just compensation for the land, if they find that the taking is valid. If the owner wishes, they can purge themselves by making a donation to charity. For an example from a completely different area of the law, if someone attempts to commit suicide, the health services are still obliged to help, even against the \isr{patient's} wishes. This remains the case, moreover, even though suicide is no longer considered a criminal offence. Moreover, no one would hesitate to consider this as a service to the person who is thereby prevented from taking their own life.

A similar kind of reasoning can be applied to rules stressing the social responsibilities and participatory duties of property owners. This serves owners not only by providing a bulwark against the pernicious tendency to conflate social responsibility with state interference. It also creates a narrative that can empower owners more generally, not as holders of entitlements, but as members of a society.

A proclamation to the effect that the ``public interest'' necessitates negating property rights is clearly very \isr{marginalising} to owners. They are then an obstacle to progress, not a resource that can be used to bring it about. A balancing act might be required to determine if interference is truly justified, but the individual is now relevant only to one side of the equation, the side of private harms. If, on the other hand, the act of interference itself can be rendered as a form of protection, enforcement of an obligation, or a measure to enable participation, the individual occupies center stage. The owner is now also a part of the public good that is to be achieved through measures involving their property.

On this narrative, if public interests lead to an imposition, it is not because individual rights are negated, but because the public is deemed to know best how to secure the goal of human flourishing, for owners and their society. To take an obvious example: both private and a public interests suggest that the law should discourage owners from becoming a nuisance to their \isr{neighbours}. Indeed, most jurisdictions have rules to this effect. Now, under a human flourishing theory, we are entitled to highlight how such rules protect the institution of property, also by protecting the owners' membership in their community. The public does not ``side with the \isr{neighbours}'' when it compels someone to behave with more care. Rather, the public undertakes measures to protect ``just social relations'' among owners and other community members.

For a second example, consider situations when environmental concerns suggest imposing restrictions on what an owner is permitted to do with their land. This too can be rendered as an act of protecting property. But doing so requires the regulatory body to relate the interference positively to the owner's interests and obligations, to emphasise that this in effect extracts from the owner a positive contribution to society.

Such a narrative does not prevent public values and the public interest from being given considerable weight. However, it compels us to be more concrete about how exactly state interference serves to enhance the social functions of the rights interfered with. The baseline for assessment, therefore, become actual persons and their well-being, not some abstract ideal of ``goodness''. Moreover, imposing the collective will on individuals becomes a means to ensure human flourishing for all, not an abstract goal in itself.

The upshot is that on the human flourishing account, if property interference is unavoidable, it is still important to interfere in a way that constructively targets the individual, aiming to protect them by enabling them -- and compelling them -- to protect others and partake in social and political life. This can then become interference aimed at bringing the individual into the fold, \isr{making them play their part, by raising them to fruitful citizenship.} This might be the vision of a paternal state, one that runs the risk of becoming overprotective, unfair, or just plain stupid. However, it is 
not the vision of a cold and alienating state of experts and elites that govern form afar.

Hence, even if the paternal state engenders resistance against interference, those who resist are also likely to carry forward care and affection for the social, political and legal structures within which this agency is (hopefully) permitted to take place. To \isr{conceptualise} an act of restriction as a means to empower the persons restricted is something they might find offensive, but it also renders interference more meaningful. It provides both a reason to take a more active role in relation to the interfering power, and a possible cause for constructive resistance. 

Importantly, it does not force the conclusion that the public resides behind closed doors, disinterested in what the affected individual have to offer. Instead, it is an approach that encourages a response, by focusing always on the persons interfered with, whenever interference is deemed necessary. This is the vision of a bottom-up, rather than a top-down, approach to imposing the collective will on individuals. I believe it has merit, and 
}

This raises the question of what kind of institutions we need to enable local communities and owners to  flourish and make informed decisions about how to use their properties. In the final chapter of the thesis, I discuss this concretely in the context of economic development situations, by looking to the Norwegian institution of land consolidation.

In the next section, I will focus specifically on economic development takings. First, I introduce such takings by considering the seminal case of {\it Kelo v City of New London}\footcite{kelo05}, which brought this category to prominence in the US discourse on property law. Then I will assess the unique aspects of such takings against the social function theory, to provide an argument that the category has significance for legal reasoning in takings law, as well as with respect to property as a constitutionally protected human right. Finally, I will provide an abstract presentation of the values that I believe are important when normatively assessing the law in this area. In doing so, I will draw on the human flourishing theory, setting out the main values that will inform the concrete policy assessments I provide later. 

\section{Economic Development Takings}\label{sec:edt}

Constitutional property rules in many jurisdictions indicate, with varying degrees of clarity, that eminent domain should only be used to take property either for ``public use'', in the ``public interest'', or for a ``public purpose''. Such a restriction can be regarded as an unwritten rule of constitutional law, as in the UK, or it can be explicitly stated, as in the basic law of Germany.\footnote{See Chapter \ref{chap:2}. Section \ref{sec:contrast} below.} In some jurisdictions, for instance in the US and in Norway, explicit property clauses exist, but do not provide much information about the intended scope of protection.\footnote{See Chapter \ref{chap:2}, Section \ref{sec:us} and Chapter \ref{chap:4}, Section \ref{sec:explaw}.}

Both the Norwegian and the US property clauses appear to refer to public use only as a precondition for the duty to pay compensation. However, this is universally understood as expressing the {\it presupposition} that the power of eminent domain is only to be used in the public interest.\footnote{In the literature, it is rare to even note that a different interpretation is linguistically possible. But see \cite[205]{berger78}.} Indeed, in cases when one might say that private property is ``taken'' for a non-public use without compensation, for instance in a divorce settlement, it is not commonly regarded as an exercise of eminent domain. Rather, it is justified by reference to a different category of rules, meant to ensure enforcement of obligations that arise between private parties independently of the state's power to single out and compulsorily acquire specific properties.

The exact boundary between eminent domain and other forms of state interference in property is not always clear, but I will not worry too much about it in this thesis. Indeed, it suffices to note that legal scholars agree that the power of eminent domain is meant to be exercised in the public interest. However, interesting differences of opinion emerge when we turn to the question of whether the presupposed public use or public interest gives the judiciary a right and/or a duty to restrict the state's power to take. In the US, most scholars agree that some such restriction is intended, but there is great disagreement about its extent.\footcite[205]{berger78} In Norway, on the other hand, a consensus has developed whereby the notion of public use is interpreted so widely that it hardly amounts to a practical restriction at all.\footnote{See, e.g., \cite[368]{aall10}.} Moreover, the courts defer almost completely to the assessments made by the executive branch regarding the purposes that may be used to justify a taking.\footcite[368]{aall10}

Some US scholars adopt a similar stance, but others argue that the public use presupposition should be read as a strict requirement, forbidding the use of eminent domain unless the public will make actual use of the property that is taken.\footnote{Compare \cite{bell06,bell09,claeys04,sandefur06}.} Most scholars fall in between these two extremes. They regard the public use restriction as an important limitation, but they also \isr{emphasise} that courts should normally defer to the legislature's assessment of what counts as a public use.\footnote{See, e.g., \cite{merrill86,alexander05}. The fact that US jurists usually stress deference to the legislature, not the executive branch, should be noted as a further contrast with Norway.}

As I discuss in more depth in Chapter \ref{chap:2}, Section \ref{sec:hop}, the debate in the US has its roots in case law developed by state courts -- the federal property clause was for a long time not applied to state takings. This has changed, and today the Supreme Court has a leading role in this area of US law. It has developed a largely deferential doctrine, resembling the understanding of the public use limitation under Norwegian law.\footnote{See \cite{berman54,midkiff84,kelo05}.} The difference is that in the US, cases raising the issue still regularly arise and prove controversial. As mentioned in the introduction to this thesis, the most important such case in recent times was {\it Kelo}, decided by the Supreme Court in 2005.\footcite{kelo05} This case saw the public use question reach new heights of controversy in the US.\footnote{See, e.g., \cite{somin09}.}

{\it Kelo} centred on the legitimacy of taking property to implement a redevelopment plan that involved construction of research facilities for the drug company Pfizer. The home of Suzanne Kelo stood in the way of this plan and the city decided to use the power of eminent domain to condemn it. Kelo protested, arguing that making room for a private research facility was not a permissible ``public use''. She was represented by the libertarian legal firm {\it Institute for Justice}, which had previously succeeded in overturning similar instances of eminent domain at the state level.\footnote{See \url{https://www.ij.org/cases/privateproperty}.} Kelo lost the case before the state courts, but the Supreme Court decided to hear it and assessed its merits in great detail.

The precedent set by earlier federal cases was clear: As long as the decision to condemn was ``rationally related to a conceivable public purpose'', it was to be regarded as consistent with the public use restriction.\footcite[241]{midkiff84} Moreover, the role of the judiciary in determining whether a taking was for a public purpose was regarded as ``extremely narrow''.\footcite[32]{berman54} It had even been held that deference to the legislature's public use determination was required ``unless the use be palpably without reasonable foundation'' or involved an ``impossibility''.\footnote{See \cite[66]{dominion25}; \cite[680]{gettysburg96}.}

This understanding was also reflected in the outcome of related cases. In {\it Midkiff}, the Supreme Court had upheld a taking that would benefit private parties, with no direct benefit to the public.\footnote{\cite{midkiff84}. For a more detailed discussion, see Chapter \ref{chap:2}, Section \ref{sec:hop} below.} In {\it Berman}, it had upheld a taking for economic redevelopment of a blighted area, even though the property taken was not itself blighted.\footnote{\cite{berman54}. For a more detailed discussion, see Chapter \ref{chap:2}, Section \ref{sec:hop}.} But in the case of {\it Kelo}, the court hesitated.

Part of the reason was no doubt that takings similar to {\it Kelo} had been heavily criticised at state level, with an impression taking hold across the US that eminent domain ``abuse'' was becoming a real problem.\footnote{See, e.g., \cite[667-669]{sandefur05}.} A symbolic case that had contributed to this worry was the infamous \textcite{poletown81}. In this case, General Motors had been allowed to raze a town to build a car factory, a decision that provoked outrage across the political spectrum.\footnote{See generally \cite{sandefur05}.} The case was similar to {\it Kelo} in that the taker was a powerful commercial actor who wanted to take homes. This, in particular, served to set the case apart from {\it Midkiff}, which involved a taking in \isr{favour} of tenants, and to some extent also {\it Berman}, which involved a taking of businesses (and homes) in the interest of removing blight. Moreover, the Michigan Supreme Court had recently decided to overturn {\it Poletown} in the case of \textcite{wayne04}. Hence, it seemed that the time had come for the Supreme Court to re-examine the public use questions.\footnote{See, e.g., \cite{sandefur05,claeys04}.}

Eventually, in a 5-4 vote, the court decided to apply existing precedent and upheld the taking in {\it Kelo}. The majority also made clear that economic development takings were indeed permitted under the public use restriction, also when the public benefit was indirect and a private company would benefit commercially.\footcite[469-470]{kelo05} This resulted in great political controversy in the US. According to Ilya Somin, the {\it Kelo} case ranks among the most disliked decisions in the history of the Supreme Court.\footcite[2]{somin11} 

Some 80 - 90 \% of the US public expressed great disapproval, with critical voices coming from across the political spectrum.\footcite[2108-2110]{somin09} Why did the case prove so controversial? No doubt, the discontent with the decision was \isr{fuelled} in large part by the fact that it was seen as a case of the government siding with the rich and powerful, against ordinary people.\footnote{\cite[630-634]{baron07}} Indeed, the party that appeared to benefit the most from the taking was Pfizer -- a multi-billion dollar company -- while Suzanne Kelo, who stood to lose, was a middle class homeowner. In this context, the taking of Kelo's home seemed morally suspect, an act of \isr{favouritism} showing disregard for less influential members of society.\footnote{See, e.g., \cite{underkuffler06}.}

In addition, it is worth noting that many commentators conceptualised the {\it Kelo} case by thinking of it as belonging to a special category, by describing it as an economic development taking, a {\it taking for profit}, or, more bluntly, a case of {\it Robin Hood in reverse}.\footcite{somin05} Categories such as these had no clear basis in the property discourse before {\it Kelo}. Indeed, in terms of established legal doctrine, it would be more appropriate to say that the case revolved entirely around the notion of ``public use''. 

However, when considering the most common reasons given for condemning the outcome in {\it Kelo}, it becomes clear that many critics felt it was natural to classify the case along additional dimensions. A survey of the literature shows that many made use of a combination of substantive and procedural arguments to paint a bleak picture of the {\it context} surrounding the decision to take Kelo's home. Important aspects of this include the imbalance of power between the commercial company and the owner, the incommensurable nature of the opposing interests, the lack of regard for the owner displayed by the decision makers, the close relationship between the company and the government, and the feeling that the public benefit -- while perhaps not insignificant -- was made conditional on, and rendered subservient to, the commercial benefit that would be bestowed on a commercial beneficiary.\footnote{See, for instance, \cite{underkuffler06,somin07,sandefur06,cohen06,hafetz09,hudson10}.}  This dynamic, in which public bodies no longer seem to be leading and pushing the process forward, but are rather being led and being pushed, is regarded as particularly suspicious. This, in turn, is typically derided as a perversion of legitimate decision-making, used to argue that economic development takings such as {\it Kelo} suffer from what I will refer to here as a {\it democratic deficit}.

\noo{ It is noteworthy that in economic development situations, this broader worry comes to occupy center stage in a dispute over the legitimacy of interfering in individual property rights. On the traditional entitlements-based account of property, this is a turn of attention that it can be hard to make sense of. However, it fits nicely with the social function theory of property and the emphasis on participation in decision-making espoused by the human flourishing account of property's ends.}

From a theoretical point of view, I take all of this to suggest that many critics of {\it Kelo} effectively adopted a social function view on property, by paying close attention to the wider social and political context of the taking.\footnote{For a particularly clear example of this, see \cite{underkuffler06}.} Hence, the social function theory of property appears to provide a highly natural starting point for further exploring economic development takings. Specifically, the theory inspires reasoning that can justify a departure from the course laid down by previous cases on the ``public use'' requirement, by encouraging a broader perspective on legitimacy in takings cases. In fact, it seems to me that such a perspective was indeed adopted by the minority of the Supreme Court in {\it Kelo}, particularly Justice O'Connor, who formulated a strongly worded dissent.\footnote{\cite[494-505]{kelo05}.}

\subsection{Justice O'Connor as a Social Function Theorist}

Justice O'Connor was joined by all the four other dissenters, but Justice Thomas also formulated his own dissent where he argued for the revival of a strict literal reading of the public use requirement.\footnote{\cite[505-523]{kelo05}.} As such, Justice Thomas' dissent fits better with a traditional property narrative, while also being less relevant outside the context of US law. Justice O'Connor, by contrast, made a broad assessment of the social and political consequences of allowing takings in cases like {\it Kelo}, an assessment that seems to be of general relevance to any jurisdiction where commercial interests benefit from the power of eminent domain. Her analysis culminates in the following warning:

\begin{quote}
Any property may now be taken for the benefit of another private party, but the fallout from this decision will not be random. The beneficiaries are likely to be those citizens with disproportionate influence and power in the political process, including large corporations and development firms. As for the victims, the government now has license to transfer property from those with fewer resources to those with more. The Founders cannot have intended this perverse result.\footcite[505]{kelo05}
\end{quote}

Interestingly, Justice O'Connor's concern is directed at substantive notions of fairness and democracy, emphasising how unrestrained economic development takings can distort the democratic process and lead to socially unjust results. This perspective provides a plausible basis on which to strike down certain kinds of economic development takings. Moreover, it allows us to do so without giving up the value of judicial deference, since it focuses on the democratic deficit rather than the exact meaning given to the notion of public use. In addition, it is a call for institutional reform, to search for new governance frameworks that will empower owners and their communities.

Moreover, the values Justice O'Connor relies on appear to be closely related to the values associated with the notion of human flourishing, particularly those relating to the political function of property as an anchor for community and democracy. 

\noo{ \subsection{Justice O'Connor as a Social Function Theorist}

Importantly, cases like {\it Kelo} not only appear to threaten individual entitlements of owners. They also appear to threaten equality in civic society, as the economic rationality used to justify interference results in an implicit political statement to the effect that the property of the rich and powerful is better protected, and valued higher by the state, than property owned by regular citizens, who reside in ordinary communities.

The effect of a traditional economic development taking is that property rights are transferred from the many to the few, taken from ordinary people and given to the powerful. Hence, these cases represent a possibly pernicious redistribution of property, particularly in terms of property's social function. The structural imbalances of the condemnation process itself find permanent expression in the new distribution of property. The social structures of a living community are dismantled in favour of a social structure that revolves around the commercial interest of designated companies that enjoy the support of government. The political and social power of the community is diminished, perhaps lost in its entirety, while the political and social power of designated companies increase.

It seems clear that to Justice O'Connor, this too had to be recognised as a negative consequence of the taking in {\it Kelo}. Again, I stress that recognising such effects appear to require a social function approach to property. There is no clearly quantifiable individual loss -- no  particular ``stick'' in the property bundle that is not compensated. Rather, it is the community itself that is lost, a community that was not directly implicated in any formal ``entitlement'', but which played a crucial role in providing meaning to the totality of the bundle of rights and obligations enjoyed by the owners. 

Even if we extend our perspective to account for indirect individual losses, we are not doing justice to such losses. The owner might relocate, acquire new property with a similar meaning in a new community somewhere else. But that does not make up for the fact that {\it this} community is lost forever, as {\it this} property takes on new meanings and functions. The loss to Suzanne Kelo, therefore, might even be a significant loss to the City of New London, whose democracy suffers as a result of the taking.

Of course, the economic and social gains of development might outweigh such negative effects. In any event, it seems that the balancing of interests required in this regard should be carried out by an institution that sufficiently \isr{recognises} the owners' right to participation and self-governance. The presence of a highly active commercial third party, in particular, means that public participation in the standard sense might be insufficient. In economic development takings, the commercial company typically appears alongside the government, as a more or less integrated part of the institutional structure making the decision to condemn. The owners, however, do not enjoy a corresponding level of participation.

Specifically, their interests are only negatively defined. They are adversely effected and may object, but under standard administrative regimes they play no constructive role in the process. For instance, they are not called on to take part in the development itself, or to assess its merits more broadly than by being asked to respond based on their own individual entitlements. This might be one of the main problems with economic development takings, resulting in a democratic deficit. I will argue for this in more depth later, but I remark here that an important reason to focus on this aspect is that it involves precisely those values that economic development takings are most likely to threaten. Moreover, if the loss of community outweighs the positive effect of economic development, this is unlikely to be recognised following a process that relies primarily on the contribution of the developer and the expert planners.\footnote{A similar point is made in \cite{underkuffler06}.} 

The objections made by owners may not only be given too little weight. It is also possible that owners themselves unduly focus only on the individual loss. They might not even consider those issues that are most important for property's social function. To address this concern, I do not think it is sufficient to theoretically proclaim that social function aspects need to be considered. Such aspects are likely to already form part of the package of interests that expert planners are supposed to take into account. However, to address the democratic deficit of economic development takings, it seems that institutional reforms might be in order, to give owners and their community a more significant voice in the decision-making process. This is a call for greater involvement by the local community (including, perhaps, even non-owners) in the decision-making process relating to development. It is not sufficient to merely ``consult'' local communities by asking if they have objections. It is also necessary to include communities in a constructive way, perhaps even be compelled to assume an active role in relation to the proposed project.

This is a proposal that envisages owners engaging directly with both government and potential developers, by considering alternative schemes, and making their own proposals. In short, this asks for a system where owners and communities are co-authors of the government's plans for their land.  According to the human flourishing theory as I understand it, acting as such a co-author is not only an owners' right, but also their obligation. 

It seems to me that Justice O'Connor's argument reflects some of the ideas I have sketched here. 

}

Indeed, Justice O'Connor seems to argue that the taking of Kelo's home would be a particularly harmful interference in exactly those ``just social structures'' that Alexander highlights as the bedrock of a well-functioning property regime.\footnote{See ...} At least, it seems clear that an entitlement-based approach to property cannot explain the degree of disapproval seen in Justice O'Connor's opinion. After all, Kelo had been offered generous compensation, there had been no clear breach of concrete procedural rules, and the claim that the taking was {\it only} a pretext to bestow a benefit on Pfizer did not seem supported by the facts.\footnote{See \cite{bell06}.} Hence, in the absence of support for a literal reading of public use, it had to be the overall character and consequence of the taking that rendered it illegitimate. In this regard, the lack of a clearly identifiable public benefit becomes only one of many observations pointing towards a democratic problem with the use of eminent domain on display in {\it Kelo}.

It also bears noting how structural aspects are crucial in Justice O'Connor's argument. The economic implications for the owner are comparatively unimportant and even the importance of home-ownership to the integrity of the person falls into the background compared to issues relating to democratic legitimacy and good governance. The overarching concern is that economic development takings can come to result in a form of governmental interference in property that systematically \isr{favours} the rich and powerful to the detriment of the less resourceful. In this way, the power of eminent domain can be used to establish and sustain patterns of inequality, under the pretence of providing an economic benefit. Hardly anyone would openly regard this as desirable. Indeed, it is not hard to agree that if Justice O'Connor's predictions about the fallout of {\it Kelo} are correct, then it is indeed ``perverse''. 

The crucial question becomes whether her predictions are warranted. In fact, the main importance of her dissent might be that it flags this issue as a crucial one in relation to very typical uses of eminent domain in the modern world. In light of its high level of generality, Justice O'Connor's dissent becomes a call for empirical and qualitative assessment of economic development takings, a quest for understanding of how they actual affect political, social and bureaucratic processes. In addition, it raises the question of how to {\it avoid} negative effects, that is, how to design rules and procedures that can reduce the democratic deficit of economic development takings. These will be the main two themes that will occupy the remainder of this thesis.

\noo{I will start by recording in some more detail a tentative list of conditions that can be used for identifying those takings that qualify as evidence for eminent domain abuse, in the broader sense of that term found in Justice O'Connor's dissent. This is not a trivial task, but in the following section I present a template that I believe can prove useful, based on a proposal due to Kevin Gray.\footnote{See \cite{gray11}.}
}
%As I now move away from theory towards concrete assessment of economic development takings, both these questions will be in focus.

\noo{ 

arly with the


In this thesis, however, such takings will not be the focus of attention. Rather, attention will be directed to takings where the marginalised groups own the property to start with,

One might still  misgivings about allowing economic development takings to go ahead, because they unduly inflate the power of government. 

Property has merit because it offers such protection, and for this reason it should be protected, as an institution. Property reform should aim to strengthen property, not weaken it, and exercising the power of eminent domain on a case-by-case basis is {\it not} the same as implementing a property reform.

As with the other points, assessing the democratic merit of a taking requires taking into consideration both procedural and substantive aspects. First, we should consider how the decision to use eminent domain was made, and if the owners in particular were granted a say proportional to their stake in the matter. Second, we should consider the effect of the taking on future decision-making processes involve the property rights in question. The mere fact that eminent domain serves to take property from the many and give it to the few, thereby making the property regime less egalitarian, is an argument for increased scrutiny of legitimacy.

}

\noo{
\section{Conclusion}\label{sec:conc1}

In this chapter, I have presented the core notion of my thesis, namely that of an economic development taking. I started by noting that while the notion is straightforward to define factually, it is far from obvious what implications it has for legal reasoning. I illustrated the subtleties involved by considering a concrete example of a commercial scheme that looked like it might well result in compulsory acquisition of land, namely Donald Trump's controversial plans to develop a golf course on a site of special scientific interest close to Aberdeen, Scotland. In the end, the plans did {\it not} require takings, as Trump was able to make creative use of property rights he acquired voluntarily, against the complaints of recalcitrant \isr{neighbours}.

This turn of events made the example even more relevant to the points I have been trying to make in this chapter. It served to highlight, in particular, that the question studied in this thesis is not a black-and-white balancing act between property privileges on one side and the good will of the regulatory state on the other. Rather, the example of Trump's golf course allowed me to \isr{emphasise} the importance of context when assessing both the nature of property rights and the meaning of protecting them. In particular, to protect the property rights of those opposing Trump's golf course was not about protecting just any property, it was about protecting the property of members in a local community that felt it would be detrimental to this community, and to their lives, if Trump was allowed to redefine it. In particular, after Trump decided not to pursue compulsory purchase, protecting the property of these members of the community became a question of {\it restricting} the degree of dominion that Trump could exercise over his own property. Hence, under a conventional and overly simplistic way of looking at these matters, protecting property then became tantamount to restricting its use, a seeming paradox.

To resolve this paradox, and to arrive at a better conceptual understanding of economic development takings, I looked to various theories of property. I noted that there are differences between civil law and common law theorising about property, but I concluded that these differences are not particularly relevant to the questions studied in this thesis. In particular, I observed that neither the bundle theory, dominant in the common law world, nor the dominion theory, taught to many civil law jurists, helped me clarify economic development takings as a category of legal thought.

I then went on to consider more sophisticated accounts of property, focusing on the social function theory, which emphasises how property structures, and is structured by, social and political relations within a society. I went on to argue that in the first instance, the social function theory should be understood as giving us {\it descriptive} insights into the workings of property and its role in the legal order. In this regard, I advanced a different stance than many property scholars, by arguing that we should aim to decouple descriptive insights from normative aspects of the theory, to allow the social function theory to serve as a common ground for further value-driven debate.

I then went on to clarify my own starting point for engaging in such debate, by expressing support for the human flourishing theory proposed by Alexander and Pe\~{n}alver. This theory is based on the premise that property {\it should} enable -- and even compel -- individuals and their communities to  participate in social and political processes. I argued that property's purpose in this regard is  fundamental to its proper role in a democratic society, as an anchor for participatory decision-making.  

Moreover, I noted that the human flourishing theory contains a further important insight, concerning the scope of the state's power to protect. In particular, the theory asks us to recognise that protecting property against interference that is harmful to human flourishing is a responsibility that the state has even in cases when the individual owners themselves neglect to defend their property, for instance as a result of financial incentives to remain idle. In other words, some functions of property are such that owners have an obligation to preserve them, while the state has a duty to protect them, potentially even against the will of the owners.

After this, I went on to consider economic development takings specifically, by drawing on the theoretical insights collected from preceding sections. To make the discussion concrete, I considered the case of {\it Kelo}, which propelled the notion of an economic development taking to the front of the takings debate in the US. I focused particularly on the dissenting opinion of Justice O'Connor, and I argued that she approached the issue in a way that is consistent with the theoretical basis proposed in this chapter.

I will now go on to make my analysis of economic development takings more concrete, by considering how such takings are dealt with in Europe and the US respectively. I note that the category has yet to receive much attention in Europe, so the discussion focuses on the US. Here this issue has received a staggering level of attention after {\it Kelo}. To get a broader basis upon which to \isr{assess} all the various arguments that have been presented, I consider the historical background to the issue as it is discussed in the US. This involves giving a detailed presentation of the public use restriction, as it was developed in case law from the states during in the 19th and early 20th century. I then connect this discussion with recent proposals to deal with economic development takings, responding to the backlash of {\it Kelo} by aiming to address the democratic deficit of such takings.

Later, when I begin to consider the law relating to Norwegian hydropower, I will look back at the theoretical basis provided in the present chapter to guide the analysis. In particular, I focus on certain decision-making mechanisms that have developed on the ground in Norway, as a practical response to the increased tendency for local owners to engage in hydropower development. I will argue that this shows the conceptual strength of the idea that property is irreducibly embedded in community, continuously evolving alongside institutions of participatory decision-making. }