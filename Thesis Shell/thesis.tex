%input macros (i.e. write your own macros file called MacroFile1.tex)
\newcommand{\PdfPsText}[2]{
  \ifpdf
     #1
  \else
     #2
  \fi
}

\newcommand{\IncludeGraphicsH}[3]{
  \PdfPsText{\includegraphics[height=#2]{#1}}{\includegraphics[bb = #3, height=#2]{#1}}
}

\newcommand{\IncludeGraphicsW}[3]{
  \PdfPsText{\includegraphics[width=#2]{#1}}{\includegraphics[bb = #3, width=#2]{#1}}
}

\newcommand{\InsertFig}[3]{
  \begin{figure}[!htbp]
    \begin{center}
      \leavevmode
      #1
      \caption{#2}
      \label{#3}
    \end{center}
  \end{figure}
}


%%% Local Variables: 
%%% mode: latex
%%% TeX-master: "~/Documents/LaTeX/CUEDThesisPSnPDF/thesis"
%%% End: 


%\includeonly{Chapter1/Chapter1}
%NOTE: if you want to work on just one Chapter, you can take out the `%' sign on the previous line and compile the thesis accordingly. The above command, for instance, will give you just the first Chapter. The bonus of doing it this way is that your cross references and page numbers will remain as they are in the full file. 


\documentclass[a4paper,oneside,12pt]{thesisPSnPDF}

\usepackage[utf8]{inputenc}
\usepackage{babel}
%\DeclareUnicodeCharacter{00A0}{ }

\newcommand{\sjur}[1]{SJUR: #1}
\newcommand{\nathp}[1]{NatHp(#1)}

\def\signed #1{{\leavevmode\unskip\nobreak\hfil\penalty50\hskip2em
  \hbox{}\nobreak\hfil(#1)%
  \parfillskip=0pt \finalhyphendemerits=0 \endgraf}}

\newsavebox\mybox
\newenvironment{aquote}[1]
  {\savebox\mybox{#1}\begin{quote}}
  {\signed{\usebox\mybox}\end{quote}}

\newcommand{\noo}[1]{}

\usepackage{titlesec}
\titleformat{\chapter}[hang]
  {\normalfont\huge\bfseries\centering}{\thechapter}{20pt}{\Huge}

\addbibresource{thesis.bib}

% turn of those nasty overfull and underfull hboxes
\hbadness=10000
\hfuzz=50pt

% Put all the style files you want in the directory StyleFiles and usepackage like this:
%\usepackage{StyleFiles/watermark}

%The following indexes are to ensure the table of cases functions properly. You can leave this to one side for now, though it is worth learning early on how to make the table of cases. It is pretty easy; but it'd be a shame if it got to near submission and you couldn't figure out how to do it. 
% NB: I haven't provided for Northern Irish cases here
\makeindex[name=casesgb, title={England and Wales}, columns=1,intoc]
\makeindex[name=casessc, title={Scotland}, columns=1,intoc]
\makeindex[name=casesus, title={The United States}, columns=1,intoc]
\makeindex[name=casesnz, title={New Zealand}, columns=1,intoc]
\makeindex[name=casesau, title={Australia}, columns=1,intoc]
\makeindex[name=casesca, title={Canada}, columns=1,intoc]
\makeindex[name=legis, title={United Kingdom}, columns=2,intoc]
\makeindex[name=casesother, title={Other Jurisdictions}, columns=1,intoc]
\DeclareIndexAssociation{gbcases}{casesgb}% ENGLAND
\DeclareIndexAssociation{sccases}{casessc}% SCOTLAND
\DeclareIndexAssociation{aucases}{casesau}% AUSTRALIA
\DeclareIndexAssociation{cacases}{casesca}% CANADA
\DeclareIndexAssociation{nzcases}{caseszn}% NEW ZEALAND
\DeclareIndexAssociation{uscases}{casesus}% UNITED STATES
\DeclareIndexAssociation{eucases}{casesother}% EU
\DeclareIndexAssociation{echrcases}{casesother}% ECHR
\DeclareIndexAssociation{pilcases}{casesother}%
\DeclareIndexAssociation{othercases}{casesother}% ANYTHING ELSE
%\DeclareIndexAssociation{gbprimleg}{legis}% LEGISLATION
%\DeclareIndexAssociation{gbsecleg}{legis}% LEGISLATION
\DeclareIndexAssociation{enprimleg}{legis}% LEGISLATION


\indexsetup{level=\section*,toclevel=section,noclearpage}

\begin{document}
\renewcommand\baselinestretch{1.5}
\baselineskip=24pt


%\maketitle

\begin{titlepage}

\begin{center}



\vspace*{\fill}
\centering

{\Huge\textsc{On the legitimacy of economic development takings}}\\[3cm]

\large {Thesis submitted to the School of Law at Durham University for the degree of Doctor of Philosophy}\\

by

{Sjur K. Dyrkolbotn}\\

%\emph{{Your College}}\\
\vspace*{\fill}

 

\vfill

{\Large Autumn 2014}\\
{c. 90 000 Words}

\end{center}

\end{titlepage}


%set the number of sectioning levels that get number and appear in the contents
\setcounter{secnumdepth}{4}
\setcounter{tocdepth}{1}

\frontmatter
%
\begin{center}
\vspace{4cm}
I hereby certify that this thesis is the result of my own work except where otherwise indicated and due acknowledgement is given.
\vspace{1cm}

I also certify that this thesis is XXXXX words long excluding the bibliography.\\

\vspace{4cm}


\begin{tabular}{lr}
& DATE OF SUBMISSION \\
& \\
SIGNED & DATE \\
\end{tabular}


\end{center}


% ----------------------------------------------------------------------


%%% Local Variables: 
%%% mode: latex
%%% TeX-master: "../thesis"
%%% End: 

%% Thesis Abstract -----------------------------------------------------

% NOTE: As with acknowledgements, I had to create a new format for this -- I couldn't get the original one to work. As with the acknowledgements, if you are able to fix the code so it's less messy, do pass the fix back to the Law Faculty.
\cleardoublepage
\addcontentsline{toc}{chapter}{Abstract}
%\begin{abstractslong}    %uncommenting this line, gives a different abstract heading
%\begin{abstracts}        %this creates the heading for the abstract page

\begin{quoting}
  \singlespace
    \begin{center}
  {\LARGE \bfseries  On the Legitimacy of Economic Development Takings }\\
  \vspace*{0.5cm}
      {\large Sjur Kristoffer Dyrkolbotn}\\
  %\vspace*{0.1cm}  
   %   {\large \emph{Ustinov College}}\\
  \vspace*{0.2cm}  
    {\normalsize Thesis submitted to Durham Law School at Durham University for the degree of Doctor of Philosophy}

  \vspace*{0.2cm}  
    {\normalsize \today}\\
  \vspace*{0.5cm}  
    {\normalsize \bfseries Abstract}      
  \end{center}
  {\parindent0pt
For most governments, facilitating economic growth is a top priority. Sometimes, in their pursuit of this objective, governments interfere with private property. Often, they do so by indirect means, for instance through their power to regulate permitted land uses or by adjusting the tax code. However, many governments are also prepared to use their power of eminent domain in the pursuit of economic development. That is, they sometimes compel private owners to give up their property to make way for a new owner that is expected to put the property to a more economically profitable use. %This new owner is sometimes the government itself, represented by one of its administrative bodies. But in many cases it will be a private company, operating for profit, possibly in cooperation with government entities through some form of public-private partnership.
}
\vspace{0.7mm}

This thesis asks how the law should respond to government actions of this kind, often referred to as {\it economic development takings}. The thesis makes two main contributions in this regard. First, in Part I, it proposes a theoretical foundation for reasoning about the legitimacy of economic development takings, including an assessment of possible standards for judicial review. Moreover, the thesis links the legitimacy question to the work done by Elinor Ostrom and others on sustainable management of common pool resources. Specifically, it is argued that using institutions for local self-governance to manage development potentials as common pool resources can potentially undercut arguments in favour of using eminent domain for economic development.

Then, in Part II, the thesis puts the theory to the test by considering takings of property for hydropower development in Norway. It is argued that current eminent domain practices appear illegitimate, according to the normative theory developed in Part I. At the same time, the Norwegian system of land consolidation offers an alternative to eminent domain that is already being used extensively to facilitate community-led hydropower projects. The thesis investigates this as an example of how to design self-governance arrangements to increase the democratic legitimacy of decision-making regarding property and economic development.

%This shows how local governance arrangements can work, suggesting that more attention should be devoted to studying the nexus between property, common pool resource management, and eminent domain.

%This theoretical basis is formulated independently of specific jurisdictions, but based on considering existing approaches to the legitimacy question from England and Wales, the United States, and at the European Court of Human Rights. In addition, the thesis draws a link between the legitimacy question and the work done by Elinor Ostrom and others on sustainable management of common pool resources. Specifically, it is argued that institutions for local self-governance that treat development potentials as common pool resources can often undercut arguments in favour of using eminent domain for economic development.

%Such rules are in place in most developed countries, and the fundamental status of property has been expressed explicitly in both the US constitution and the European Convention of Human Rights. The tension between these provisions and the practice of taking property for economic development, in many cases for commercial profit, is clear and worth considering further.

%\vspace{0.7mm}

%The second part of the thesis puts the theory developed in the first part to the test by considering takings for hydropower development in Norway. Under Norwegian law, the right to exploit the hydropower in most streams and rivers belong to the riparian owners. That is, the right to the hydropower belongs to the people who own the land over which the water flows, usually local community members. To acquire these rights, energy companies tend to rely on the government's power of eminent domain. Recently, however, local communities have begun to protest this practice, by arguing that they should be allowed to take a more active role in managing their own resources. This has resulted in tensions in Norway, shedding light on the legitimacy question as it arises in the context of Norwegian expropriation law. In addition, new light has been shed on the role of the so-called land consolidation courts, which are now increasingly asked to deliver alternatives to eminent domain in hydropower cases. The thesis investigates this in depth and argues that the unique system of land consolidation found in Norway demonstrates how to design self-governance arrangements that can increase the democratic legitimacy of decision-making regarding property and economic development.

\end{quoting}


%\end{abstracts}
%\end{abstractslong}


% ----------------------------------------------------------------------


%%% Local Variables: 
%%% mode: latex
%%% TeX-master: "../thesis"
%%% End: 

%% Thesis Acknowledgements ------------------------------------------------

\cleardoublepage
%\begin{acknowledgementslong} %uncommenting this line, gives a different acknowledgements heading
%\begin{acknowledgements}      %this creates the heading for the acknowlegments
\addcontentsline{toc}{chapter}{Acknowledgements}
\begin{quoting}
  \singlespace
    \begin{center}
  {\LARGE \bfseries  Acknowledgements}\\
  \vspace*{0.5cm}
  \end{center}
\noindent

My supervisor, Professor Tom Allen, has been a great support and inspiration ever since we first corresponded about the possibility of me doing a PhD in Durham, in the spring of 2012. His style as a supervisor has been superb: calm and unperturbed, yet always sharp and focused, readily available to offer insightful comments and valuable guidance. Thank you, Tom.

Second, I would like to thank Durham Law School for offering me a place at their department and for treating me well while I have been there. Thanks also to Professor Leigh and Professor Masterman for taking an interest and giving me valuable comments following my first year review.

Third, I would like to thank Professor Jacques Sluysmans, Professor Hanri Mostert, and Professor Leon Verstappen, for welcoming me to their regular colloquia on expropriation law. Attending and speaking at these meetings has been a very valuable experience for me, allowing me to learn from expropriation lawyers and scholars from many different jurisdictions. A special thanks to Professor Sluysmans, Dr Emma Waring and Dr Stijn Verbijst for inviting me to contribute a chapter on Norway in their book on expropriation in Europe. Another special thanks to Dr Waring for sending me a copy of her doctoral thesis on private takings; it has been a great help and inspiration for my own work. Also a special thanks to Dr Ernst Marais and Bj\"{o}rn Hoops for organising an excellent conference in Rome and being very helpful and welcoming to new members of the expropriation research community. Hopefully, this community will stay together and continue to prosper.

Fourth, I would like to thank Ustinov College for welcoming me as a student in Durham and providing a relaxed and friendly atmosphere during my first year as a PhD student. Thanks also to the friends I met there, including Julia, Alan, Noel, Meghan, Lloyd, Peter, and Alma. A special thanks to Isabel Richardson, for being both a highly valued friend and an excellent proofreader.

Fifth, I would like to thank family, friends and colleagues in Norway, especially Ragnhild, Truls and Piotr (who is just on holiday in Germany, I am sure). A special thanks to my father and my brother, for motivation, guidance, and support. A special thanks also to my mother and my sisters, for kindness and inspiration. Thanks to Einar Sofienlund for sharing his insight and providing invaluable information about small-scale hydropower. Thanks also to Johan Fr Remmen, for making clear why this thesis should be written. Hopefully, I have made a good start towards doing justice to the subject.

Lastly, I would like to thank Marijn Visscher. No doubt, coming to Durham was the best decision I ever made.

\end{quoting}


%\end{acknowledgements}
%\end{acknowledgmentslong}

% ------------------------------------------------------------------------

%%% Local Variables: 
%%% mode: latex
%%% TeX-master: "../thesis"
%%% End: 


% A note on the format. I could not get the acknowledgements macro to work properly, despite some effort, and so I redesigned it rather messily, as you will see above. If you enter text, it will work -- but the more technologically competent among you will probably be able to fix it. If and when that is done, if you could send the resultant document back to the Law Faculty to update it, that would be great. 
\tableofcontents


% NOTE:
% To generate the indexes properly you need to run the following commands (in a terminal shell -- you need to navigate to the Thesis file in terminal. There will be guidance online for how to navigate within Terminal. Otherwise, most scientists should be able to help you!):
% splitindex -- thesis -s oscola (THIS IS THE COMMAND WHICH WORKS FOR ME)
% splitindex thesis --s oscola thesis (TRY THIS IF THE FIRST ONE DOESN'T WORK)

% this section includes various indexes/tables of cases and legislation

\chapter*{Cases Cited}
\addcontentsline{toc}{chapter}{Cases Cited}
\markboth{CASES CITED}{CASES CITED}

\printindexearly[casesgb]% ENGLAND & WALES
\printindexearly[casessc]% SCOTLAND (GB too, of course, but ...)
\printindexearly[casesau]% AUSTRALIA
\printindexearly[casesnz]% NEW ZEALAND
\printindexearly[casesca]% CANADA
\printindexearly[casesus]% UNITED STATES
\printindexearly[casesother]% OTHERS

\chapter*{Legislation Cited}
\addcontentsline{toc}{chapter}{Legislation Cited}
\markboth{LEGISLATION CITED}{LEGISLATION CITED}

\printindexearly[legis]% ALL LEGISLATION


%\addcontentsline{toc}{chapter}{Acknowledgements}
%\addcontentsline{toc}{chapter}{Abstract}
%\listoffigures

%\include{LawTable/LawTable}
%\listoftables
%\addcontentsline{toc}{chapter}{List of Tables}
%\printglossary  %% Print the nomenclature
%\addcontentsline{toc}{chapter}{Nomenclature}


\mainmatter
\part{Towards a Theory of Takings for Commercial Gain}

\noo{The gulf between the conscious and the unconscious must be embraced by explanation models. It is the distinguishing human trait, after all, that we grant ourselves ... The unconscious is only looked {\it at} by the conscious. No particular theoretical unit of the sub-conscious change predictably when you think about yourself (or, as some would say, ``form an intention''). What changes is {\it you}. The predictability of the outcome (or otherwise, in people who are unwell, for instance) is largely attributable to the workings of the sub-conscious, i.e., must await explanation in behavioural (physical/chemical) terms. The difference between a person and a dog is that by looking at themselves and the world in a (mental) language, persons are capable -- through a purely behavioural/physical mechanism -- of modifying their instincts. Thought prevails over instinct (sometimes), but the power of thought itself is purely instinctual... The power of language is in the world, not in language. Thank God!}

\chapter{Property, protection and privilege}\label{chap:1}

\begin{quote}
It's nice to own land.\footnote{Donald Trump}
\end{quote}

\section{Introduction}

In this chapter, I provide a bird's eye view on my topic, by placing it in the theoretical landscape. My aim is to explain the key concepts that I will rely on to make sense of the empirical data considered in subsequent chapters. I will also present the main values that I will look to when I give normative assessments. In addition, I will relate my theoretical approach to current strands in property theory, focusing on those aspects of property theorizing that I regard as particularly relevant to the work done in this thesis.

I will strive to show that my approach to the empirical data is sound and informative, while focusing on principles of {\it legal} reasoning. I will not provide an extensive presentation of concepts or theoretical approaches developed in other fields, such as political science, sociology, economy, or psychology. However, I note that all these fields engage in interesting ways with the notion of property, and I think a multi-disciplinary approach can be illuminating.\footnote{For some examples of relevant work from economics, psychology and political science respectively, consider \cite{miceli11,nadler08,katz97,carruthers04}.} Hence, while I focus on legal and --  to some extent -- philosophical theories of property, I will try to make a note of specific questions I consider that are also analysed in related fields.

The crucial argument made in this chapter is that the category of {\it economic development takings} is relevant to legal reasoning about certain kinds of situations when private property is taken by the state. This is not {\it prima facie} clear. In fact, I am prepared to face critics who will argue that the category makes no legal sense at all. Fortunately, it makes perfect intuitive sense; it targets situations when property is, quite literally, taken for economic development. In most cases I will consider, this is even the explicitly stated aim used to justify eminent domain. Hence, the factual basis for our categorization can not be questioned.

The theoretical basis, on the other hand, can not be taken for granted. Indeed, a superficial look at dominant legal approaches to property would seem to indicate that in most property regimes, the nature of the project benefiting from a taking should not be in focus when assessing the legitimacy of interference. Rules aiming to protect property invariably focus on the rights of the affected owner, making clear that she enjoys some degree of protection against uncompensated state interference. But how can we, on this basis, justify having regard to the {\it purpose} of the taking? What bearing does this have for the question of legitimacy with respect to the owner's rights? At first sight, it might seem unwarranted to think that it should matter at all. Are not owners' rights  equally interfered with when property is taken for some uncontroversially public project, like a new public road, compared to the situation when it is taken for economic development? Is it not in fact a little small-minded, even short-sighted, to worry about the taker's gain, instead of concentrating on what the owner, if anything, stands to lose?

Much of the work in this chapter, albeit theoretical, is aimed at countering this very concrete objection. I believe it is important to do so thoroughly, since it is an objection that threatens to undermine the conceptual basis for the kind of study that I present in this thesis. Moreover, it is an objection that I think it is inappropriate to dismiss without further comment. In the context of US law, it might be possible to do so, since economic development takings have, as a matter of fact, gained recognition as an important category of legal reasoning.\footnote{See generally \cite{cohen06,somin07,malloy08}.}  In Europe, however, this has not yet happened, at least not to the same extent.

The reason for this difference is not that US law contains special rules that emphasize the importance of the distinguishing features of economic development takings.\footnote{In fact, many state laws now {\it do} contain such rules, following the backlash of the controversial decision in \cite{kelo05}. However, such rules were introduced only after the category of economic development takings first came to prominence in legal discourse. See generally \cite{eagle08,somin09,jacobs11}.} Rather, the difference must largely be attributed to the fact that economic development takings have resulted in great political controversy in the US, a controversy that has influenced both the law and legal scholars.\footnote{See, e.g., \cite[1190-1192]{somin08}.} Hence, in the absence of a similar political climate in Europe, a conceptual investigation into the very idea of an economic development taking is warranted, if not also required.

As I argue in this chapter, providing an adequate account of such takings forces us to broaden our theoretical outlook compared to most traditional strands of legal reasoning about property. However, I find considerable support for the necessary conceptual reconfiguration when I consider recent trends in property theory, particularly those that focus on the {\it social function} of property.\footnote{See generally \cite{alexander09a,foster11,singer00,underkuffler03,alexander06,alexander10,dagan11}.} Indeed, the crux of the main argument presented in this chapter is that this function allows us, even compels us, to pay attention to the special dynamics of power that tend to manifest in cases when private property is taken by the state for someone else's profit.

Some might argue that the most straightforward way of describing economic development takings is to say simply that they are {\it unfair}. Indeed, this has some merit also as a conceptual position, since it would seem to explain quite effectively why takings for profit have become so unpopular in the US, particularly when people's homes are taken.\footnote{See, e.g., \cite[742-748]{nadler08}.}  However, a more thoughtful assessment reveals that matters are not quite so simple. Indeed, it seems that economic development takings are an almost unavoidable consequence of any system that emphasize both state control over property and public-private partnerships in the economic sector. To condemn this political model of government is a possible response, but not one I will pursue in this thesis. Rather, I will focus on getting to the heart of what characterizes economic development takings, so that I may address also the question of how to deal with them, to ensure that their positive functions can be fulfilled in ways less offending.\footnote{Even those who support an outright ban on economic development takings should be interested in demarcating the category more closely, to arrive at a better understanding of what exactly will be banned.}

Therefore, the stark contrast between the intuitive response that a taking for profit is unfair, and the legal assessment that what matters is only the loss to the owner, needs to be considered further. A tentative first reconciliation can be achieved by arguing that the feeling of unfairness is in itself a loss that the owner incurs, so that the law had better take it into account. But, of course, not any subjective feeling should be given weight in a legal context. The question, therefore, is what exactly the feeling of unfairness can help us uncover about the nature of economic development takings. Does it uncover something legally significant?

In my view, the answer is yes, and in the following I will argue for this position in depth. To motivate this largely abstract analysis, I will begin by considering a concrete scenario which illustrates the need for contextual assessment and more fine-grained conceptual categories for reasoning about cases when demands for economic development come to pose a threat against established patterns of property.

\section{Donald Trump in Scotland}\label{sec:dts}

On the 10th of July 2010, the property magnate Donald Trump opened his first golf-course in Scotland, proudly announcing that it would be the ``best golf-course in the world''.\footnote{http://www.golf.com/courses-and-travel/donald-trump-scotland-golf-course-lives-hype (accessed 06 July 2014).} Impressed with the unspoilt and dramatic seaside landscape of Scotland's east coast, the New Yorker, who made his fortune as a real estate entrepreneur, had decided he wanted to develop a golf course in the village of Balmedie, close to Aberdeen.

To realize his plans, Trump purchased the Menie estate in 2006, with the intention of turning it into a large resort with a five-star hotel, 950 timeshare flats, and two 18-hole golf-courses. The local authorities were not particularly keen on the idea, and planning permission was initially denied by Aberdeenshire Council. Particularly worrying to the councilors was the fact that the proposed site for the development was declared to be of special scientific interest under EU conservation legislation. The frailty and richness of the sand dune ecosystem, in particular, suggested that the land should be left unspoilt for future generations.\footnote{See \url{http://en.wikipedia.org/wiki/Donald_Trump#Scottish_golf_course} (accessed 06 July 2014).} Trump was not deterred, however, and started lobbying Scottish politicians to gain support. In the end, he was able to convince Scottish ministers that he should be given the go-ahead on the prospect of boosting the economy by creating some 6000 new jobs.\footnote{See \url{http://www.theguardian.com/world/2008/nov/04/donald-trump-scottish-golf-course} (accessed 06 July 2014).}

Activists continued to fight the development, launching the ``Tripping up Trump'' campaign to back up local residents who refused to sell their properties.\footnote{See \url{http://www.trippinguptrump.co.uk/} (accessed 03 August 2014).} One of these, the farmer and quarry worker Michael Forbes, expressed his opposition in particularly clear terms, declaring at one point that Trump could ``shove his money up his arse''.\footnote{See \url{http://www.scotsman.com/news/donald-trump-s-plea-to-homeowners-on-the-menie-estate-1-1370270}. (accessed 03 August 2014)} Trump, on his part, had described Forbes as a ``village idiot'' that lived in a ``slum''.\footnote{See \url{http://www.bbc.co.uk/news/10205781} (accessed 08 July 2014).} Moreover, he had suggested that Forbes was keeping his property in a state of disrepair to pressure up the price of the property.\footnote{See \url{http://edition.cnn.com/2007/WORLD/europe/10/10/trump.golf/} (accessed 03 August 2014).} Forbes was offended. He proudly declared that he would never consider selling, as the issue had become personal.\footnote{See \url{http://www.scotsman.com/news/scotland/top-stories/farmer-who-took-on-trump-triumphs-in-spirit-awards-1-2668649} (accessed 03 August 2014).}

At the height of the tensions, Trump considered his legal options, by asking the local council to consider issuing compulsory purchase orders (CPOs) that would allow him to take property from Forbes and other recalcitrant locals against their will.\footnote{See \url{http://www.thesundaytimes.co.uk/sto/news/uk_news/article184090.ece} (accessed 03 August 2014).} If carried out, this would have been an iconic example of an economic development taking. Moreover, it would not be the first time that the power of eminent domain had been used to the benefit of Donal Trump's business empire. In the 1990s, Trump famously succeeded in convincing Atlantic City to allow him to take the home of one Vera Coking, to facilitate further development of his casino facilities.\footcite[297-301]{jones00} This taking was eventually struck down by the New Jersey Superior Court, however, a result that was hailed as a milestone in the fight against ``eminent domain abuse'' in the US.\footnote{See \url{https://www.ij.org/cases/privateproperty} (accessed 12 August 2014). The case also caused a surge of attention directed at the issue, see \url{http://reason.com/archives/2008/03/03/litigating-for-liberty/4} (accessed 12 August 2014). For the decision itself, consult \cite{banin98}.}

In Scotland, Trump's plans were met with widespread outrage. The media coverage was wide, mostly negative, and an award-winning documentary was made which painted Trump's activities in Balmedie in a highly negative light.\footnote{See \url{http://www.youvebeentrumped.com}.} The controversy also made its way into UK property scholarship. Kevin Gray, in particular, a leading expert in property law, expressed his opposition by making clear that he thought the proposed taking would be an act of ``predation''.\footcite{gray11}

In fact, the case prompted Gray to formulate a number of key features that could be used to identify situations where compulsory purchase would be more likely to represent an abuse of power. He noted, in particular, that Trump's proposed takings would fall in line with a general tendency in the UK towards using compulsory purchase to benefit private enterprise, even in the absence of a clear and direct benefit to the public. Indeed, it was not unrealistic to think that CPOs might come to be used in Balmedie; if he had put his weight behind it, Trump might well have been able to make a successful case that existing statutory authorities could be used to justify takings of this kind.\footnote{In particular, the \cite{tcpsa97} contains a wide authority in s. 189, stating that local authorities has a general power to acquire land compulsorily in order to ``secure the carrying out of development, redevelopment or improvement''.} It would not be hard to argue that the public would benefit indirectly in terms of job-creation and increased tax revenues. Moreover, Scottish ministers had already gone far in expressing their support for the plans.

But then, in a surprise move, Trump announced he would not seek CPOs after all.\footnote{See \url{http://news.stv.tv/north/224662-relief-for-residents-trump-lifts-threat-of-compulsory-purchase-orders/} (accessed 03 August 2014).} Quite possibly, he was discouraged by the negative press. But in addition, he had found another strategy, namely that of containment: He erected large fences, planted trees and created artificial sand dunes, all serving to prevent the properties he did not control from becoming a nuisance to his golfing guests. One local owner, Susan Monroe, was fenced in by a wall of sand some 8 meters high. ``I used to be able to see all the way to the other side of Aberdeen'', she said, ``but now I just look into that mound of sand''.\footnote{See \url{http://www.theguardian.com/world/2012/jul/10/donald-trump-100m-golf-course} (accessed 03 August 2014).} She also lamented the lack of support from the Scottish government, expressing surprise that nothing could be done to stop Trump.

But there was little left to do. As soon as Trump decided to build around them, the neighboring property owners found themselves completely marginalized. After all, Trump had the backing of the government, having been declared as a job-creator whose activities would boost the economy in the region. He had even received an honorary doctorate at the Robert Gordon University, a move that prompted the previous vice-chancellor, Dr David Kennedy, to hand his own honorific back in protest.\footnote{See \url{http://www.bbc.co.uk/news/uk-scotland-north-east-orkney-shetland-11421376}.}

But in the end, it was not by taking the land of others that Trump triumphed in Scotland. Rather, he succeeded by exercising ``despotic dominion' over his own.\footnote{To quote William Blackstone, \cite[2]{blackstone79b}.} This proved highly effective;  after he fenced them in, his neighbors were hard to see and hard to hear. The Balmedie controversy went quiet, the golfers came, Trump got his way. As he declared during the grand opening: ``Nothing will ever be built around this course because I own all the land around it. [...] It's nice to own land.''\footnote{See \url{http://www.theguardian.com/world/2012/jul/10/donald-trump-100m-golf-course} (accessed 06 July 2014).}

\subsubsection*{\ldots}

The tale of Trump coming to Scotland not only serves to illustrate the kind of scenario that I will be looking at in this thesis, it also puts the work into perspective. It shows, in particular, that what it means to protect property against undue interference can depend highly on the circumstances. For a while, it looked like Balmedie was about to become a canonical case of an economic development taking. But in the end, it became rather an illustration of something far more subtle, namely that the meaning of protecting property rights depends highly on context, our own perspective, and the values we aim to promote. 

Moreover, we are reminded that the function of property as such is deeply shaped by social, political and economic structures. It seems clear, in particular, that Donald Trump's ownership of the Menie estate has a vastly different meaning than does Michael Forbes' ownership of his small farm. To many observers, the former kind of ownership will represent some combination of power, privilege and profit, while the latter will be regarded as coming imbued with a mix of defiance, community and sustenance. Very different values are inherent in these two forms of ownership, and after Trump came to Balmedie, they clashed in a way that required the legal order to prioritize between them.

According to Trump and his supporters, protecting property rights against interference in Balmedie no doubt involves protecting the governmentally sanctioned golf resort plans from backwards locals who attempt to fight progress. In this narrative, ``protection'' can maybe even be used justify compulsory acquisition of property rights that are regarded as a hindrance to the full enjoyment of property for more resourceful members of the community. But for Michael Forbes and the other local owners, protecting property rights is likely to have a completely different meaning. To them, protecting property means above all else to protect a local community against what they see as a disruptive and damaging plan that will see both them and their properties turned into golfing props. Again, adequate protection might require an interference in property, to prevent Trump from using his land in a way that causes damage to his neighbors. Regardless of who we support, we are forced to recognize that protection implies interference and vice versa. 

This shows the conceptual inadequacy of a simplistic perspective whereby protecting property rights is seen as a black-and-white proposition, a call for limits on the state's power to do good, enforced to protect owners' right to do as they please. In reality, the situation is  more subtle, involving a number of additional dimensions. Importantly, how we assess concrete situations where property is under threat depends crucially on what we perceive as the ``normal'' state of property, the alignment of rights and responsibilities that we deem to be worthy of protection. Our stance in this regard clearly depends on our values. But values themselves are in turn influenced by the context of assessment within which they arise. More problematically still, they may be influenced by our \emph{perception} of this context, rather than by reality.

For example, property activists in the US tend to regard the value of autonomy as a fundamental aspect of property. But this must be understood in light of the idea that US society is founded on an egalitarian distribution of property, where ownership is meant to empower ordinary people by facilitating self-sufficiency and self-governance.\footnote{See, e.g., \cite[173]{ely07}.} Hence, the autonomy inherent in property ownership is not thought of as being bestowed on the few, but on the many. Protecting autonomy of owners against state interference is not about protecting the privileges of the rich and powerful, but is embraced as a way to protect {\it against} abuse by the privileged classes.\footnote{This narrative is enthusiastically embraced by US activists who fight economic development takings, see, e.g., \url{http://www.castlecoalition.org/}.} 

This, however, is only an {\it idea} of property protection. It might not correspond to the reality surrounding the rules that have been molded in its image. Indeed, it has been noted that despite the great pathos of the egalitarian property idea, egalitarianism has actually played a marginal role to the development of US property law.\footnote{\cite[361]{williams98} (``Why does the egalitarian strain of republicanism have such a substantial presence in American property rhetoric outside the law but so little influence within it?'')} More worryingly still, research indicates that land ownership in the US, while hard to track due to the idiosyncrasies of the land registration system, is not actually all that egalitarian.\footcite[246-247]{jacobs98} In this way, we are confronted with the danger of a disconnect between  values, reality and the law.

In Scotland, a similar story unfolds. Here, the traditional concern is that land rights are mainly held by the elites.\footnote{See generally \cite{wightman96,wightman13}.} As a result, Scottish property activists tend to focus on values such as equality and fairness, calling also on the state to regulate and implement measures to achieve more egalitarian control over the land. Indeed, reforms have been passed that sanction interference in established property rights on behalf of local communities.\footnote{See generally \cite{lovett11,hoffman13}.} At the same time, cases like Balmedie illustrate that the Scottish government, now with enhanced powers of land administration, may well choose to align themselves with the large landowners. Moreover, research indicates that recent reforms in Scottish planning law, which serve to enhance the power of the central government, has the effect of undermining local communities and their capacity for self-governance.\footnote{See generally \cite{pacione13,pacione14}.} Again, the danger of a disconnect between influential property narratives and reality is brought into focus.

On the other hand, it seems that grass root property activists in the US and Scotland may well be closer in spirit than they seem. Upon closer examination I cannot help thinking that they share many of the same concerns and aspirations, and that the differences arise mainly from the fact that they operate in different contexts and engage with different discourses of property. The challenge is to find categories of understanding that allow us to make sense of their shared spirit, as well as the spirit they oppose.

I think the example of Balmedie suggests a possible first step, by illustrating the need for a framework that will allow commentators to  deny that there is any inconsistency between opposing compulsory purchase orders while also supporting strict property regulation in the context of fighting a golf resort. Both of these positions, moreover, should be viewed as efforts to protect property. To the classical ``individual rights v state interference'' debate, such a dual position can be hard to make sense of. But in my opinion, this only points to the vacuity of such a conventional narrative.

In general, I think it is hard, close to impossible, to make sense of many contemporary disputes over property if we do not have the conceptual acumen to distinguish between egalitarian property held under a stewardship obligation by members of a local community, and feudal property held by businesses for investment. Moreover, there is no contradiction between promoting the value of autonomy for one of these, while emphasizing state control and redistribution for the other. The broader theme is the contextual nature of property, and its implications for protection of property rights. In the coming sections, I will locate a theoretical basis that will allow me to take advantage of this viewpoint in my legal analysis.

\section{Theories of property}\label{sec:top}

What is property? In common law jurisdictions, the standard answer is that property is a collection of individual rights, or more abstractly, {\it entitlements}.\footnote{The term ``entitlement'' was used to great effect in the seminal article \cite{calabresi72}.} Being an owner, it is often said, amounts to being entitled to one or more among a bundle of ``sticks'', streams of protected benefits associated with, or even serving to define, the property in question.\footcite[357-358]{merrill01} This point of view was first developed by legal realists in response to the natural law tradition, which conceptualized property in terms of the owner's dominion over the owned thing, particularly his right to exclude others from accessing it.\footcite[193-195]{klein11} In civil law jurisdictions, rooted in Roman law, a dominion perspective is still often taken as the theoretical foundation of property, although it is of course recognized that the owner's dominion is never absolute in practice.\footnote{For a comparison between civil and common law understanding of property, see generally \cite{chang12}.}

In modern society, the extent to which an owner may freely enjoy his property is highly sensitive to government's willingness to protect, as well as its desire to regulate. To civil law theorists, this sensitivity has been thought of as giving rise to various restrictions in property rights, but for common law theorists, overlooking a legal system with roots in a relatively stable feudal tradition, it has been thought of as {\it constitutive} of property itself.\footcite[7]{chang12} Indeed, the bundle of rights theory has long historical roots in common law. Arguably, it was distilled from the traditional estates system for real property, which was turned into a theoretical foundation for thinking about property in the abstract.\footnote{See \cite[7]{chang12}   
(``The ``bundle of rights'' is in a sense the theory implicit in the common law system taken to its extreme, with its inherently analytical tendency, in contrast to the dogged holism of the civil law.'').} 

However, during the 18th and 19th century, natural law thinking was also highly influential in common law. This is evidenced, for instance, by the works of William Blackstone and James Kent.\footnote{See generally \cite{blackstone79b,kent27}.} But towards the end of the 19th century, it became increasingly hard to reconcile such an approach to property with the reality of increasing state regulation. Hence, the bundle metaphor that gained prominence in the early 1900s can be seen to indicate a return to a more modest perspective.\footcite[195]{klein11}

Property rights under the bundle theory are thought to be directed primarily towards other people, not things.\footnote{See \cite[357-358]{merrill01} (``By and large, this view has become conventional wisdom among legal scholars: Property is a composite of legal relations that holds between persons and only secondarily or incidentally involves a ``thing''.'').} This underscores a second point about property in the real world, namely that the content of rights in property are necessarily relative to the totality of the legal order. For instance, relying on a bundle metaphor, it becomes perfectly natural that a farmer's property rights protects him against trespassing tourists, but not against the neighbor who has an established right of way. 

By contrast, the dominion theory suggests viewing such situations as exceptions to the general rule of ownership, which implies a right to exclusion at its core. In the case of property, exceptions no doubt make up the norm. But in civil law jurisdictions one lives happily with this. It takes the grandeur away from the dominion concept, but it retains a nice and simple structure to property law. In the civil law world, it is common to say that what the owner holds is the {\it remainder} after all positive rights, serving to restrict his dominion, have been deducted.\footcite[25]{chang12} Moreover, while there may be many limitations and additional benefits attached to property, they are all in principle carved out of one initial right, namely that of the owner. In this way, the system becomes more easily navigable.

An interested party may ask, ``who owns this land?'' Then, under the dominion theory, a clear answer is expected and will usually be adequate, even if it does not give a complete picture of all relevant property rights. Under the bundle theory, on the other hand, one might be inclined to respond, ``to which stick are you referring?'' Clearly, this narrative is more complex, perhaps unduly so. 

Some common law scholars have recently elaborated on this to develop a critique of the bundle theory, by suggesting that it should at least be complemented by a firm theory of {\it in rem} rights in property. This, they argue, would allow the law to operate more effectively, by relying on a simple and clear rule that, although defeasible, will generally suffice to inform people about their relevant rights and duties in relation to property.\footnote{\footcite[793]{merrill01b} (``The unique advantage of in rem rights -- the strategy of exclusion -- is that they conserve on information costs relative to in personam rights in situations where the number of potential claimants to resources is large, and the resource in question can be defined at relatively low cost.''); \footcite[389]{merrill01} (``The right to exclude allows the owner to control, plan, and invest, and permits this to happen with a minimum of information costs to others.''). See also \cite{ellickson11} (arguing that Merrill and Smith's analysis nicely complements and improves upon the bundle theory).} 

There are also other, less pragmatic, reasons why a dominion approach might be preferable, even if the bundle metaphor is arguably more accurate. In particular, some scholars point out that the bundle theory does not adequately reflect the sense in which property is a right to a {\it thing}, serving to create an attachment that is not easily reducible to a set of interpersonal legal relationships.\footnote{\cite[1862]{merrill07}. For a slightly different take on attachment, highlighting how the thingness of property marks its conditional nature and transferability, see \cite[799-818]{penner96}.} In the US, where the bundle theory has traditionally been dominant, critique like this seems to be gaining ground.\footnote{See generally \cite{foster10}.}

But in this thesis, the efficiency and clarity of different property concepts will not be a primary concern, nor will personal attachments to things in themselves play a particularly important role.\footnote{I mention, however, that the personhood-aspects of property that are sometimes highlighted in this regard will also be relevant to my analysis of economic development takings. However, this is not something that I think warrants extensive engagement with the bundle v dominion debate. I note, for instance, that in the work of Margaret Jane Radin, one of the main proponents of persoonhood accounts, the bundle theory is not challenged as much as it is readjusted, although in places it also seems to be the object of some implicit criticism, see, e.g., \cite[127-130]{radin93}.}
Hence, I will remain largely agnostic about this aspect of the debate between dominion and bundle theorists. In particular, the differences between civil and common law traditions in this regard do not cause special problems for my analysis of economic development takings. However, I am also more broadly interested in the values that are promoted by different ways of looking at property, particularly with regards to the question of when interference is legitimate under constitutional and human rights law. Hence, I  now turn to the question of whether or not there are any significant differences between dominion and bundle theories in this regard.

Intuitively, one might think that bundle theorists are likely to endorse greater room for state interference in property rights. Indeed, thinking about property as sticks in a bundle may lead one to think that property rights are intrinsically limited, so that subsequent changes to their content -- carried out by a competent body -- are mere reflections of their nature, not a cause for complaint. In particular, the theory conveys the impression that property is highly malleable. For the theorists that developed the bundle of sticks metaphor in the late 19th and early 20th century, this aspect was undoubtedly very important. By providing a highly flexible concept of property, they helped the state gain conceptual authority to control and regulate. Indeed, this was the clear intention of many early proponents of the bundle theory -- the ``progressives'' of their day.\footcite[195]{klein11} The early bundle theorists not only developed a theory to fit the law as they saw it, they also contributed to change.

In relation to takings law, the progressives succeeded in gaining acceptance for the use of eminent domain to benefit a wider range of public purposes than had so far been considered legitimate.\footnote{See generally \cite{yale49}.} In particular, they argued successfully that the so-called ``public use'' restriction, which had previously been enforced quite strictly, particularly by state courts, should be greatly relaxed. This change was important in creating the situation which led to economic development takings becoming a contentious issue in the US, and so provides important background to the main topic of my thesis.  I return to the public use debate in the US in much more depth later, in Chapter \ref{chap:2}, Section \ref{sec:hop}. Here I would like to stress that I think there can be little doubt that the increased scope given to eminent domain in the early 20th century was mutually conducive to the conceptual reorientation that took place during the same time.

In relation to the different, but related, issue of so-called regulatory takings, the bundle theory even  became directly implicated. A regulatory taking occurs when governmental control over the use of property becomes so severe that it must be classified as a taking in relation to the law of eminent domain. Particularly in the US, the question of when regulation amounts to a regulatory taking is highly controversial. The stakes are high because takings have to be compensated in accordance with the Fifth Amendment of the US constitution. At the same time, the law is unclear; a lack of statutory rules means that regulatory takings cases are often adjudicated directly against constitutional property clauses (often the relevant state constitution, in the first instance).

If property is thought of as a malleable bundle of entitlements that exists only because it is recognized by the law, it becomes more natural to argue that when government regulates the use of property, it does not deprive anyone of property rights, but merely restructures the bundle. In the case of {\it Andrus v Allard}, the Supreme Court adopted such an argument when it declared that ``where an owner possesses a full ``bundle'' of property rights, the destruction of one ``strand'' of the bundle is not a taking, because the aggregate must be viewed in its entirety''.\footcite[65--66]{andrus79}

Historically, therefore, it seems that bundle theorists have been largely aligned with those that favor a less restrictive approach to eminent domain. But I think it is wrong to conclude that the bundle theory {\it necessarily} implies such a stance on takings. Indeed, some prominent scholars have argued for an almost entirely opposite view. Professor Epstein, in particular, goes far in suggesting that as every stick in the property bundle represents a property right, government should not be permitted to remove any of them without paying compensation.\footcite[232-233]{epstein11} Moreover, Epstein does not believe that the bundle theory is responsible for the fact that his view of property has not been widely endorsed by US courts. Instead, he thinks the main (negative) impact of ``progressive'' thinkers stems from their tendency to adopt a ``top-down'' approach to property. That is, Epstein directs attention towards their tendency to view property rights as vested in, and arising from, the power of the state, not the possessions of individuals.\footnote{\cite[227-228]{epstein11} (``In my view, the nub of the difficulty with modern property law does not stem from the bundle-of-rights conception, but from the top-down view of property that treats all property as being granted by the state and therefore subject to whatever terms and conditions the state wishes to impose on its grantees'').} 

In my opinion, Epstein's argument shows that adoption of the bundle theory can hardly be considered a determinate factor for the kind of protection private property enjoys in a given legal system. Moreover, Epstein successfully demonstrates that as a rhetorical device, the theory may well be turned on its head. Unsurprisingly, the substance of the law, in the end, turns primarily on the values one adheres to, not the theoretical constructions one relies on when expressing those values.\footnote{To further underscore this point, it may be mentioned that while US courts do in fact recognize that a regulation can amount to a taking, this is practically unheard of in several other common law jurisdictions, including England and Australia, which nevertheless paint property in a similar conceptual light. Moreover, while the issue of regulatory takings is considered central to constitutional property law in the US, it is considered a fairly marginal issue in England, see \cite{purdue10}.}

In the civil law world, the relationship between property theorizing and property values is similarly hard to pin down at the conceptual level. To illustrate, I will again point to the question of regulatory takings. In some countries, like Germany and the Netherlands, the right to compensation is quite strong, but in other civil law countries, such as France and Greece, it is very weak.\footnote{See generally \cite{alterman10}.} In particular, the exclusive dominion understanding of property does not commit us to any particular kind of policy on this point. Indeed, the theory appears to cater comfortably to a range of different politically determined solutions to the problem of striking a balance between the interests of owners and the interests of the state. 

On the one hand, the undeniable fact of modern society is that property rights are enforced, and limited, by the power of government. Hanging on to the idea of dominion, then, necessarily forces us to embrace also the idea that dominion is not enjoyed absolutely and that government may interfere in property rights. In this way, the theory may serve as a conceptual basis upon which to argue for a more relaxed approach to protection of property rights. These rights are not absolute anyway, so why worry about interfering in them for the common good? But this story too may be turned on its head: A libertarian may well use the same image to tell a tale of property being ripped apart at its seams. Hence, he may argue, unless we want to completely lose our grasp of what property is, we had better enhance the level of protection offered to property owners.

To me, the upshot is that the differences between common law and civil law theorizing about property are not significant enough to 
make them crucial to the questions studied in this thesis. In particular, the differences between the bundle theory and the dominion idea do not appear to speak decisively in favor of any particular approach to economic development takings, nor does it provide any clear justification for regarding such takings as special in the first place. Property enjoys constitutional protection and is a recognized human right across the divide, but what this means in practice is hard to deduce from either account.

In terms of descriptive content, both theories are too bold and oversimplified. They provide a manner of speech, but they do little to enhance our understanding of the reality of property rights in modern society. In particular, they do not provide a functional account of what role property plays in relation to the social, economic and political structures within which it resides. In terms of normative content, on the other hand, they are both too bland and imprecise. They simply do not offer much clear guidance as to what norms and values the institution of property is meant to serve. They give neat explanations of what property is, but do not tell us {\it why} it should be protected. 

In the following, I will address both these shortcomings by considering property theories with a wider scope. There are many candidates that could be considered. In a recent book on property theory, Alexander and Pe\~{n}alver present five key theoretical branches: 
\begin{itemize}
\item {\it Utilitarian} theories, focusing on property's role in helping to maximize utility or welfare with respect to individual preferences and desires.\footnote{\cite[Chapter 1]{alexander10}.} 
\item {\it Libertarian} theories, focusing on property's role in furthering individual autonomy and liberty, as well as the importance of protecting property against state interference, particularly attempts at redistribution.\footnote{\cite[Chapter 2]{alexander10}.} 
\item {\it Hegelian} theories, focusing on the importance of property to the development of personhood and self-realization, particularly the expression and embodiment of free will through control and attachment to one's possessions.\footnote{\cite[Chapter 3]{alexander10}.}
\item {\it Kantian} theories, focusing on how property arises to protect freedom and autonomy in a coordinated fashion so that {\it everyone} may potentially enjoy it, through the development of the state.\footnote{\cite[Chapter 4]{alexander10}.}
\item {\it  Human flourishing} theories, focusing on property's role in facilitating participation in a community, particularly as a template allowing the individual to develop as a moral agent in a world of normative plurality.\footnote{\cite[Chapter 5]{alexander10}.}
\end{itemize}

It it beyond the scope of this thesis to give a detailed presentation and assessment of all these theoretical branches and the various ideas that have been discussed within each of them. However, in Section \ref{sec:hf} below, I will present the human flourishing theory in more detail. This is because I believe that if it is adopted, it suggests making a range of new policy recommendations regarding how the law {\it should} approach the question of economic development takings. 

First, however, I note that all the theories mentioned above are highly normative, used actively to promote the adoption of particular values associated with property. While I am not unwilling to take a stand in this debate, my main objective is to study economic development takings descriptively, by giving a case study of Norwegian waterfalls and discussing its significance in terms of comparative and human rights law. Hence, before I move on to consider other aspects, I first need a theoretical framework that allows me to meaningfully discuss those aspects that make economic development takings unique. I would like to do so, moreover, without thereby committing myself to any particular stance on how to normatively assess those aspects. 

To arrive at such a foundation, I will rely on the descriptive parts of the so-called {\it social function theory} of property.\footnote{See generally \cite{foster11,mirow10,alexander09a}. Be aware that some authors, particularly in the US, also speak of the {\it social obligation} theory, using it more or less as a synonym for the social function theory.} While this theory is often implicated in normative theories, including the human flourishing theory, I argue that it has a descriptive core which is also of great significance. Its importance to my work in this thesis is underscored by the fact that I will draw on the social function theory to answer the pressing problem of what makes economic development takings a legitimate and useful category of legal reasoning. 

Let me first reiterate that it is not {\it prima facie} clear that the category makes any legal sense at all, due to the fact that many jurisdictions lack rules that explicitly make the purpose of interference a relevant measuring stick for assessing legitimacy. To respond successfully to this potential objection, I believe it is necessary to look at property's social functions. In fact, property scholars are becoming increasingly aware of the need to do this in general, as they note that existing theories are overly focused on a narrative that revolves around individual entitlements. Many still reject that this necessitates conceptual reconfiguration, but the social function idea of property appears to be gaining ground, also as an important aid in making sense of how the law actually works. I believe this descriptive aspect of the theory provides the most appropriate way to argue that it is theoretically desirable to regard economic development takings as a special issue in property law, and I will argue for this in Section \ref{sec:edt}.

However, before making my specific point about takings, I will present the social function theory of property more generally. I will focus on showing that it captures aspects that are already highly relevant -- behind the scenes -- to how property rules are understood and applied in concrete situations. It seems, in particular, that socio-legal arguments play an important, yet often unacknowledged, role when courts interpret fundamental rules that are meant to protect private property. Bringing those aspects into the open is in itself a worthwhile project to pursue, irrespectively of any further normative stances that the social function theory might give the theorist occasion to adopt.

\section{The social function of property}

There is a growing feeling among property scholars that the notion of property has been drawn too narrowly by many of the traditionally dominant theories of property. Some have even gone as far as to challenge the idea that property is a meaningful and well-defined concept at all. These scholars point out that what counts as property in a given legal system, and what property entails in that system, depends largely on its social and political context, tradition, and even chance.\footnote{For a particularly inspiring exposition of property's elusive nature, see \cite{gray91}.} In the US, a utilitarian law-and-economics approach -- which largely takes the social and political underpinnings of property for granted -- has long been regard as standard, but even there the tide is turning. While most US scholars still regard property as a robust and meaningful category of legal thought, many are increasingly turning away from assessing property rules narrowly against their effectiveness in maximizing individual utility and social welfare. Instead, these scholars adopt a holistic approach, which allows property's social function to come into focus. One of the main proponents of this conceptual shift is Gregory S. Alexander, professor at Cornell University. In a recent article, he writes:

\begin{quote} Welfarism is no longer the only game in the town of property theory. In the last several years a number of property scholars have begun developing various versions of a general vision of property and ownership that, although consistent with welfarism in some respects, purports to provide an alternative to the still-dominant welfarist account.[...] These scholars emphasize the social obligations that are inherent in ownership, and they seek to develop a non-welfarist theory grounding those inherent social obligations.\footcite[1017]{alexander11}
\end{quote}

To scholars coming from political science, sociology or human geography, this trend will not raise many eyebrows, except perhaps for the fact that it is a recent one. After all, in fields such as these, property has never been understood merely as a set of individual entitlements that are meant to result in increased welfare. Rather, property is seen as a crucial part of the fabric of society, one that entrenches privileges and bestows power.\footnote{See generally \cite{carruthers04}.} Even scholars who believe that the institution of property is a force for good, recognize that being an owner carries with it political capital, social responsibility, and membership in a community. Those aspects, moreover, are often regarded as more important than entitlements explicitly recognized in positive legal terms. Crucially, they are important not only to the individual owners but also to society as a whole. How property rights are distributed among the population, for instance, has obvious political and economic implications, serving as a source of power and prosperity to some groups, while marginalizing others.\footnote{See, e.g., \cite[23]{carruthers04}. (``The right to control, govern, and exploit things entails the power to influence, govern, and exploit people'').}

But what is the relevance of this to property law? Usually, jurists approach property in isolation from such concerns, and often they do so because of practical necessity. The political question of what the law should be might require musings about the purpose and social context of property, but in the day-to-day workings of the law, the story goes, such considerations play a lesser role, with the importance of clear and simple rules outweighing the possible benefit that would result from contextual and holistic assessment. But at the same time, no functioning theory of  property would deny that social aspects such as expectations and obligations play a role in relation to property {\it in life}. The problem, rather, is that classical theories of property may be accused of taking the pragmatic view too far, by failing to recognize that many social functions are {\it intrinsic} to property, so that they may sometimes be impossible to disregard, also when the law is applied to concrete disputes.

This accusation can be raised against both bundle and dominion theorists. They both tend to leave little conceptual room for considering property as a social phenomena. It is recognized, of course, that rights in property -- bundled or otherwise -- serve to regulate social relations. But this effect is typically regarded as belonging to the periphery of property as a legal category, more relevant to sociologists than to property scholars. In addition, it is uncommon to observe that the causal relation between property rights and society is bidirectional, since the meaning and content of property itself is partly determined by the very same social structures that property helps establish and sustain. When this aspect of property is not recognized, the risk is that subtle dependencies between property and the political order are not brought into focus, even when they play an important role in practice.

This is particularly clear when it comes to socially defined obligations attached to property. Hardly anyone would protest that in practical life, what an owner will do with his property is as much constrained by the expectations of others as it is by law. But in addition to influencing the owner subjectively, expectations can take on an objective character by being embedded strongly in the social fabric. This, in turn, may give rise to a norm, or even a custom, which may be legally relevant, either because the law gives direct effect to it, or because it influences how we interpret rules relating to the use of property.\footnote{See generally \cite{penalver09,alexander09}.}

This seems hard to dispute as a descriptive assertion, but traditional property theorists have surprisingly little regard for it. According to Alexander, the classical theories of property convey the impression that ``property owners are rights-holders first and foremost; obligations are, with some few exceptions, assigned to non-owners''.\footcite[1023]{alexander11} The social function theorists attempt to redress this imbalance by developing theories that can naturally accommodate an account of social obligations that attaches greater weight to them as objects of property. As Alexander explains, ``social obligation theorists do not reverse this equation so much as they balance it. Of course property owners are rights-holders, but they are also duty-holders, and often more than minimally so.''\footcite[1023]{alexander11} 

It should be noted that while it lay dormant for some time, particularly in the US, this idea is by no means new. Its first modern expression is often attributed to Le{\'o}n Duguit, a French jurist active early in the 20th century. In a series of lectures he gave in Buenos Aires in 1911, Duguit challenged the classic liberal idea of property rights by pointing to their context-dependence, adopting a line of argument strikingly similar to how recent scholars have criticized the law-and-economics discourse of modern times.\footnote{See \cite[1004-1008]{foster11}. For more details about Duguit's work and the contemporaries that inspired him, see generally \cite{mirow10}.} In particular, Duguit also pointed to the notion of obligation, stressing the fact that individual autonomy only makes sense in a social context, wherein people are also dependent on each other and related through membership in communities. Hence, depending on the social circumstances of the owner, his property could entail as many obligations as it would entail entitlements and dominion. This, according to Duguit, was not only the reality of property ownership in life, it was also a normatively sound arrangement that should inspire the law, more so than the unrealistic visions of property evoked by the liberal tradition.\footnote{See \cite[1005]{foster11} (``The idea of the social function of property is based on a description of social reality that recognizes solidarity as one of its primary foundations'', discussing Duguit's work). It should also be noted that Duguit was particularly concerned with owners' obligations to make productive use of their property, to benefit society as a whole. This raises the question, however, to which we shall return in more depth later, who exactly should be granted the power to determine what counts as ``productive use''. In this way, Duguit's work also serves to underscore one of the main challenges of regulatory frameworks that seek to incorporate and draw on property's social dimension. How should decisions be made in such regimes?} 

Similar thoughts have been influential in Europe, particularly in the post-WW2 rebuilding period. For instance, as I discuss further in Chapter \ref{chap:2}, Section \ref{sec:germany}, the constitution of Germany -- her {\it Basic Law} -- contains a property clause that explicitly includes a provision stating that property entails obligations as well as rights. As argued by Alexander, this has had a significant effect on German property jurisprudence, creating a clear and interesting contrast with US law.\footnote{See \cite[338]{alexander03} (``The German Constitutional Court has adopted an approach that is both purposive and contextual, while the U.S. Supreme Court has not'').} 

A social perspective on property was also influential during the debate among the European states that first drafted the property clause in the First Protocol to the European Convention of Human Rights.\footnote{See \cite[1063-1065]{allen10}.} Later, however, the liberal conception of property gained ground also in Europe, causing jurisprudential developments that have been particularly clear in the case law from the European Court of Human Rights.\footnote{See generally \cite{allen10}.} Even so, property theorizing in Europe is still influenced by a social function view on property, more so than in the US. The European Court of Human Rights, for instance, stresses the importance of {\it proportionality} and {\it fairness} when adjudicating property cases, suggesting the importance of a contextual approach to the balancing of the many private and public interests involved.\footnote{See generally \cite[Chapter 5]{allen05}.}

I will return to possible normative implications of the social function theory later, but here I would like to stress that in the first instance it merely asks us to recognize an empirical truth. Property does not arise in a vacuum, but from within a society. As a philosophical proposition, this is obvious and hardly anyone denies it. But the social function theory asks us to consider something more, namely that property {\it law} continues to influence, and be influenced by, the social structures that surround it. Perhaps most importantly, property both reflects and shapes relations of power among members of a society.\footnote{This aspect of property's social function was stressed in a recent ``statement of purpose'' made by leading property scholars in support of the social function theory, see \cite{alexander09a}.} Moreover, it does not act uniformly in this way -- the actual effect of property on power depends on the circumstances.

An indebted farmer who is prevented by state regulation from making profitable use of his land might come to find that his property has become a burden rather than a privilege. As a consequence, someone who has already amassed power and wealth elsewhere might be able to purchase it from him cheaply. Indeed, this might provide an excellent opportunity for the outsider to consolidate his position. He can ensure that his privileges become further entrenched, both socially, politically and economically. By acquiring a farm and transforming it to recreational property, he symbolically and practically asserts his dominance and power, while also reaping a potential financial benefit resulting from his investments in a more ``modern'' pattern of use. In some cases, this dynamic can even become endemic in an area, resulting in a complete reshaping of the social fabric surrounding property.

The story might go like this: First, impoverished farmers and other locals sell homes to holiday dwellers. Then house prices soar. As a result, local people with agrarian-related incomes can't afford local homes, causing even more people to sell their land to the urban middle class. In this way, a causal cycle is established, the social consequences of which can be vicious, particularly to the low-income people who are displaced.\footnote{The general mechanism is well-documented and known as {\it gentrification} in human geography (often qualified as rural gentrification when it happens outside urban areas). See generally \cite{weesep94,phillips93,slater06}. For a case study demonstrating the role that state regulation can play (perhaps inadvertently) in causing rural gentrification, see \cite[1027-1030]{darling05}.} My theoretical contention is the following: Setting out to protect property in a situation like this -- when property rights pull in different directions -- requires taking some stance on whose property, and which of property's functions, one is aiming to protect. In particular, should the law protect the property rights of local people who face displacement, or should it protect the property rights of outsiders wishing to invest?

Some may shun away from this way of posing the question, by arguing that it would be better to rely on clear rules that can deliver justice to owners with a minimum level of dependence on the particular social processes that property is involved in at any given time. I am inclined to disagree with such a stance from the outset, since justice itself is a notion that largely seems to depend on social conditions. However, my main point here is that the prospect of such ``socially neutral'' rules is simply illusory when both sides of a conflict are in a position to adopt a property narrative to argue for their interests. For an excellent example of such a situation, it is enough to return to the story of Donald Trump coming to Scotland that we told in Section \ref{sec:dts}.

As long as Trump threatened to use compulsory purchase, the local people could adopt a traditional ``pro-property'' stance against Trump. But as soon as Trump decided to fence them in by relying on his own property rights, they had to adopt a seemingly contradictory view, {\it against} property, on the basis that Trump's rights should be limited out of concern for the community. So how do we classify the anti-Trump stance with regards to property? The answer is unclear under classical theories, but under the social function stance, it becomes easy to resolve. The locals sought to protect property, but not just any property. The property they wanted to protect was the property which served the social function of sustaining the existing community. The property they wanted to protect was the property that {\it meant} something to them.

Undoubtedly, this was also the sentiment of Trump and his supporters, who could also make a case based on property. Hence, in conflicts such as these the law will invariably have to take a stand regarding which social functions it wishes to promote. In all likelihood, such a stand must also sometimes be taken by whoever {\it interprets} the law, since it is exceedingly unlikely that the legislature will ever be able to provide clear rules for resolving all conflicts of this kind. Lastly, and most controversially, the courts may find occasion to curtail the power of government -- perhaps even the legislature -- if their power is usurped by powerful actors wishing to undermine property's proper functions to further their own interests. This, in particular, becomes the question of constitutional and human rights limits to interference in property.

Property shapes and reflects societies, but it also shapes and reflects social commitments and dependencies within those societies.\footnote{See generally \cite{alexander09}.} Again, this function of property is highly dependent on context. A small business owner, by virtue of being a member of the local community, is discouraged from becoming a nuisance to his neighbors. Everything from erecting bright neon signs to proposing condemnation of neighboring properties are actions that he will be socially deterred from taking. If the local shop owner does not conform to social expectations, he will pay a social price. Indeed, most likely even an economic price, especially if his customer-base is local. At the same time, the local connection would serve to make the business owner positively invested in the well-being of the community. This would encourage everything from sponsoring local events to hiring local youths as part-time helpers.

But at the same time, the local business owner might be discouraged from changing his business model to become more competitive, if this is perceived as a threat to other members of the community. Economic rationality might suggest that he should expand, say, by physically acquiring more space and targeting new groups of customers, but social rationality might make this an untenable proposal. This, however, might render the business economically unsustainable, particularly if it is facing fierce competition from businesses that are not similarly constrained by community ties. Moreover, even if the business is in fact viable as long as the community remains in place to support it, the perception that there is room for improvement might increase external pressures both on the business and the community. Importantly, in the age of regulation for commercial facilitation, the state itself might exert pressure of this kind.

Then, if our local shop owner goes out of business, for whatever reason, the new owner might fail to become integrated in the community in the same way, with obvious consequences for the property's function in that community. Indeed, if we imagine that the new owner is a large commercial actor who is hoping to raze the community in order to build a new shopping center, we are at once reminded of the stark contrasts that can arise between various social functions of property. The property rights of a shop owner can be the life nerve of a community, while the exact same rights in the hands of someone else can spell destruction. While this is an undeniable empirical fact of property ownership, it is far from clear what its legal ramifications are. Here, it is tempting to embrace a normative stance, and argue for particular social values that the law {\it should} promote. However, I would like to hold on to the descriptive mode of analysis a little further. For it is perfectly clear that regardless of whose interests win out in the end, assessments of the social function of property will have played an important role in brining about that outcome.

This is true not only when the law explicitly requires that this function is to be taken into account, such as in relation to the property clause of the basic law of Germany. It also commonly becomes true, as courts search for information to guide them in their interpretation and application of statutory rules that are seemingly not concerned with social aspects of property. The classical example from the US is the case of {\it State v Shack}.\footcite{shack71} The case concerned the right of a farmer to deny others access to his land, a basic exercise of the right to exclusion often regarded as fundamental to the very definition of property. The controversy arose after the two defendants, who worked for organizations that provided health-care and legal services to migrant farmworkers, entered the land of a farmer without permission. They were there to provide services to the farmers employees, and when the farmer asked them to leave, they refused.

In the first instance. they were convicted of trespassing in keeping with New Jersey state law, but on appeal the Supreme Court of New Jersey overturned the verdict. The court held that the dominion of the land owner did not extend to dominion over people who were rightfully on his land. Hence, as long as the defendants were there at the request of the workers, the owner had to tolerate this. Importantly, the court argued for this result -- which was not based on any natural reading of the New Jersey trespass statute -- by pointing also to the fact that the community of migrant workers was particularly fragile and in need of protection. Their right -- in property -- to receive visitors where they work and live, therefore, had to be recognized, in spite of this limiting the farmer's exclusion right.

The lesson to take from this is that the social function of property can play a role even when this does not explicitly follow from any property rules. This, in turn, may be used to argue that a shift towards a social function theory is desirable. In so far as the property rules we rely on explicitly directs us to take the social aspect of property into account when applying the law, it might be permissible for the practically minded jurist to conclude that there is little need for theorizing about property's social dimension. This dimension, in so far as it is relevant, is quantified inside the law itself, not by theories that encompass it. But as a matter of fact, cases like {\it State v Shack} show that the social dimension can be relevant even when it is not mentioned in any authority, even in relation to clear rules that would otherwise appear to leave little room for statutory interpretation. It arises as relevant, in such cases, because the social dimension is intrinsic to property itself. 

This might still be a radical claim, but it is primarily a descriptive one. Indeed, even if the case of {\it State v Shack} had gone the other way, I would be inclined to take from it the same lesson. If the owner's right to exclusion had received priority over the workers right to receive guests and the owner's obligation to respect this right, that too would be an outcome that would likely underscore the social function of property. To illustrate this, it is helpful to look to an article written by Eric Claeys, where he is critical both of the social function theory in general and {\it State v Shack} in particular.\footcite{claeys09} Given the basis on which that decision was made, he is led to argue, however, by also pointing to those aspects of the social context that speak in favor of the farmer.\footnote{\cite[941-942]{claeys09}.} Indeed, since he aims to engage with the social function theorists, he cannot simply declare  that the trespass rules are absolute and that the social circumstances are irrelevant. 

Instead, he argues that by considering the circumstances in {\it more} depth, a different outcome suggests itself.\footnote{\cite[941]{claeys09} (``there are good reasons for suspecting that there was more blame to go around in Shack than comes across in the case's statement of facts'').} But even if this is true, it is no argument against the descriptive content of the social function theory, merely an argument against those who think that particular values need to be endorsed by anyone willing to look to the social context of a property dispute. In this regard, it is not hard to agree with Claeys that normative fundamentalism is wrong. Indeed, he might even have a point in criticizing some social function theorists for normative naivety.\footnote{\cite[945]{claeys09} (``Judges might think they are doing what is equitable and prudent. In reality, however, maybe
they are appealing to a perfectionist theory of politics to restructure the law, to redistribute property, and ultimately to dispense justice in a manner encouraging all parties to become dependent on them.'')} 

I do not follow Claeys, however, when he takes this to be an argument {\it against} the form of legal reasoning that social function theories promote, and which he himself skillfully engages in.\footnote{In particular, I do not follow the leap Claeys makes when he suggests that it is beneficial to keep ``discretely submerged'' what he describes as ``culture war overtones'' in legal reasoning.\cite[947]{claeys09}.} In {\it State v Shack}, for instance, such reasoning was clearly in order. To engage in it was far {\it less} naive than to dismiss it on the basis that it would be irrelevant to the case. Indeed, if it the social function view had been dismissed, the entitlement-based idea of property would in effect do {\it unacknowledged} normative work, with no basis in anything more authoritative than a palpably oversimplified idea of the meaning of property. 

However, I agree with Claeys that prudence is in order. Moreover, I am not saying that the social function theory does not have normative consequences. It clearly does. Invariably so, simply because it provides a new way of talking about property and analysing conflicts, which will in turn influence our normative assessments. This is also illustrated by {\it State v Shack}. Despite Claeys skillful advocacy, many would no doubt fail to be convinced of the social merit of recognizing a right to exclusion in this case. But the crucial aspect of the social function narrative is that it makes such aspects clear, not that it commits us to, or promises to deliver, any morally superior stance on property that can deliver ``correct'' outcomes in cases such as this.

This challenges a common assumption, among both detractors and supporters of the social function theory, who argue that the theory commits us to a particular form of normative assessment, in pursuit of the ``good''. Some even argue that property law should be studied from the point of view of virtue ethics.\footnote{See generally \cite{penalver09}.} Unsurprisingly, critics such as Claeys use this to launch attacks on the social function theory and its supporters, by arguing that it represents a way of thinking that will invariably lead to lessened constitutional property protection and greater risk of abuse of state powers.\footnote{See \cite{claeys09} (``The more ``virtue'' is a dominant theme in property regulation, the less effective ``property'' is in politics, as a liberal metaphor steering religious, ethnic, or ideological extremism out of the public square'').} Indeed, increasing the room for state interference is often seen as the aim of conceptual reconfiguration; the social function view of property tends to be associated with social democratic and/or redistributive political projects, by which the notion of property is recast to justify greater interference in established rights.\footnote{Despite his commitment to ``value-pluralism'', this motivation is also clearly felt in the work of Gregory Alexander. He argues, for instance, that the social obligations inherent in property imply that the ``state should be empowered and may even be obligated to compel the wealthy to share their surplus with the poor'', see \cite[746]{alexander09}. For an assessment linking similar views on property in Europe to the dominance of social democratic thought in the post-WW2 period, see \cite{allen10}.}

It is important to note, however, that while social democratic policies may be easier to justify by emphasizing the social function of property, the mere recognition that property has an important social dimension does not in itself offer any justification for policies of this kind. For one, policy reasons must be tied to the prevailing social and economic circumstances, they will not automatically succeed merely by virtue of a conceptual shift. In addition, it seems to me that the most crucial premises used in arguments for greater state control and state-led redistribution projects concern the nature of the state, not the functions of property.

In particular, why should we believe that the state is the ultimate social institution to which property {\it should} answer? Is it not, for instance, equally possible to contend that property should continue to answer to less formal social structures that it is already embedded in by virtue of owners' membership in local communities? If so, one might as well want to limit the state's role to that of ensuring fair play among individuals and communities. A contentious question, then, might be to what extent the state should actively promote certain kinds of communities in accordance with political goals. Embracing more direct state control, on the other hand, would no longer seem very natural, at least not as a goal in itself. Indeed, on the social function view, the very idea of direct state control seems to depend on the claim that more low-level social structures fail to function properly and, crucially, that state control is {\it better}. In my opinion, this requires a separate argument. Hence, to move uncritically between talk of the ``community'' and talk of the state, as writers like Pe\~{n}alver and Alexander sometimes do, is in my opinion inappropriate.

In fact, I am inclined to believe that it is only appropriate to equate community with the state in highly special situations, for instance if it can be shown that owners insulate themselves from, and engage in exploitative practices towards, other people and communities. Importantly, to argue that such a situation obtains requires a case to be made that is compelling both empirically and politically. In this regard, I believe theory alone has little to offer. This is a reason to conclude that the social function view of property in fact tells us very little about how widely the state should intervene in property in a given society. It allows us to recognize the {\it possibility} that the state may have to intervene on behalf of certain property values, say those that aim to protect communities. But this is no argument in favour of the position that the state should intervene more or less often than it currently does. Importantly, the theory can still serve a crucial purpose in that it allows us to reason more clearly about {\it when} it is appropriate for the state to intervene. For instance, the social function theory will later be used by me to single out economic development takings for special consideration. But this will not commit me to a particular normative stance on such takings.

More generally, it does not follow from the recognition that property structures are social in nature that {\it any} institution should actively seek to neither change nor protect those structures. The Humean position, namely that the existing distribution of property rights represents a socially emergent equilibrium, remains plausible. Moreover, the normative stance that this equilibrium is a {\it good} one (or at least as good as it gets) remains as contentious -- and as arguable -- as ever. For this reason, I believe it is appropriate to approach the social function theory as a descriptive theory in the first instance.

It is worth emphasizing that in taking this view I depart from the stance taken by many of the contemporary scholars who advocate on behalf of social function theories, including some that reject social democratic ideals. Hanoch Dagan, for instance, is a self-confessed liberal, but still explicitly and strongly argues for a social function understanding on the basis that it is morally superior. ``A theory of property that excludes social responsibility is unjust'', he writes, and goes on to argue that ``erasing the social responsibility of ownership would undermine both the freedom-enhancing pluralism and the individuality-enhancing multiplicity that is crucial to the liberal ideal of justice''.\footcite[1259]{dagan07}

If this is true, then it is certainly a persuasive argument for those who believe in a ``liberal idea of justice''. But for those who do not, or believe that property law is -- or should be -- largely agnostic on this point, a normative justification for the social function theory along these lines can only discourage them from adopting it. Such a reader would be understandably suspicious that the {\it content} of the social function theory -- as Dagan understands it -- is biased towards a liberal world view. The reader might agree that property continuously interacts with social structures, but reject the theory on the basis that it seems to carry with it a normative commitment to promote liberalism.

Danach stands out somewhat in the literature by focusing on {\it liberal} values, but as I have already indicated, he is not alone in proposing highly normative social function theories. Indeed, most contemporary scholars endorsing a social function view on property base themselves on highly value-laden assessments of property institutions.\footnote{See, e.g.,  \cite{alexander09,crawford11,davidson11,singer09,penalver09}.} While they provide interesting insights into the nature of property, I am struck by a feeling that these writers all tend to overstate the desirable normative implications of adopting a social function view. In addition, they appear to believe that accepting this view on property requires us to embrace certain values and reject others. Moreover, one is left with the impression that the social function theory has little to offer beyond the values with which it is imbued, which can in turn push the law in the direction that these writers deem desirable. 

I disagree that this is the case, at least for the social function theory as I understand it. Dagan's theory of property might be conducive to ``liberal justice'', but this is because it involves far more than what follows analytically from the proper recognition that  social functions should be considered relevant when adjudicating on the rights and obligations attached to property. Indeed, it is Dagan's clearly stated aim to propose a theory that promotes specific liberal values. ``There is room to allow for the virtue of social responsibility and solidarity'', he writes, continuing by suggesting that ``those who endorse these values should seek to incorporate them -- alongside and in perpetual tension with the value of individual liberty -- into our conception of private property''.\footcite[802]{dagan99} This view is reflected further in the concrete policy recommendations he makes, for instance in relation to the question of when it is appropriate to award less than ``full'' (market value) compensation for property following a taking.\footnote{See generally \cite{dagan14b}.}

My objection is not that his proposals are necessarily wrong, but that they need not be accepted in order to conclude that the social function of property should be given a more prominent place in property theory. Importantly, I think the focus on normative reasons threatens to overshadow the most straightforward reason for awarding social structures a more prominent place in the analysis, namely that they are almost always crucially important behind the scenes, even if they go unacknowledged. The social function theory, rather than being ``good, period'', as Danach suggests, is nothing more or less than accurate, irrespectively of one's ethical or political inclinations. As such, it provides the foundation for a debate where different values and norms can be presented in a way that is conducive to meaningful debate, on the basis of a minimal number of hidden assumptions and implied commitments. Thus, the first reason to accept the social function theory, for me, is epistemic rather than deontic.

That is not to say that normative theories should not be formulated on the basis of the social function theory, it merely means that I believe it is useful to maintain at least a theoretical division between the descriptive and normative aspects of such theorizing. I return to normative aspects in the next section, arguing that the commitment to ``human flourishing'' endorsed by Professor Alexander is a particularly well-argued norm that arises from value-based assessment of the social function of property. This, I argue, is in large part also due to the value-pluralism inherent in this idea, suggesting as a positive normative claim that our notions of property {\it should} allow a divergence of opinions and values to influence the law and its application in this area.

Moreover, I believe the history of the social function theory lends support to my claim that it is useful to emphasize that the theory gives us important descriptive insights that carry few normative commitments. This is particularly important, I believe, in a time when property scholars are showing greater willingness to explore new theoretical ideas. Theories can hardly be entirely value-neutral, nor is this a goal in itself. But in my opinion, a good theory is one that can serve as a common ground for further discussion based on disagreement about values and priorities. According to Kevin Gray, ``the stuff of modern property theory involves a consonance of entitlement, obligation and mutual respect''.\footcite[37]{gray11} It is important, I think, that the same measured perspective is reflexively applied towards theory itself, to diminish the worry that a broader theoretical outlook is the first step towards unchecked state power and rule by ``judicial philosopher-kings''.\footcite[944]{claeys09}

In the next subsection, I will argue in some more detail why such a cautious perspective is warranted, by considering how the Italian fascists appropriated the social function theory in 1930s. Building on the highly inspiring work of di Robilant, I will also briefly track how non-fascist property scholars opposed this development by focusing on value-pluralism, local self-governance and freedom.\footcite{robilant13} Importantly, these scholars embraced the social function theory as a common ground from which to launch a meaningful attack on more radical ideas, without alienating those with divergent views. Instead of clinging to the old-style liberal discourse that the fascists had either flatly rejected or completely subverted, many Italian non-fascists were willing to engage in a discourse revolving around property's social function, by spelling out a more measured set of ideas based on this premise. Crucially, this set the stage for a form of intellectual resistance that did not reject those aspects of fascism that had great appeal to the public and which arguably also reflected true insights into the unfairness and lack of sustainability of the established legal order.

\subsection{Rooting out fascism: {T}he tree of property}

While the social function theory makes intuitive sense, it is also highly abstract. Therefore, its exact content has been notoriously hard to pin down. This is recognized by contemporary scholars endorsing a normative view, who attempt to address this by proposing lists of values that should be taken into account while giving examples of how they should be used to inform the law in concrete areas or cases.\footnote{See, e.g., \cite{alexander14,alexander11,dagan07}.} Unsurprisingly, however, views soon diverge regarding the concrete import of a social function view on legal reasoning. Even so, the contemporary debate appears to be based on a common ground that is quite stable, also with respect to the overall notion of what good the theory can do. But as history shows, this state of affairs is by no means guaranteed. 

In a recent article, Anna di Robilant illustrates this point exceptionally well by tracking the history of social function theorizing in Italy during the fascist era. The fascist property scholars, she notes, were happy to embrace the social function theory, since it provided them with a conceptual starting point from which to develop their idea that rights and obligations in property should be made to answer to one core value: the interests of the state.\footnote{See \cite[908-909]{robilant13} (``Fascist property scholars had also appropriated the social function formula. For the Fascists, the social function of property meant the superior interest of the Fascist state.'').} This stance was as effective as it was oversimplified. As di Robilant notes, ``earlier writers had been hopelessly evasive about the meaning and content of the social element of property''.\footcite[909]{robilant13} Hence, the fascist approach filled a need for clarity about the implications of the main idea, which was by now attracting increasing support both from the public and the academic community. Established property doctrine, it was widely felt, was both ineffective and unfair to ordinary people. Rather than securing productivity and a livelihood for all, property was used mainly as an instrument for maintaining the privileged position of the elites. By promising to change this state of affairs, the fascists attracted many to their cause.

As di Robilant notes, supporters of the fascist idea of property made clear that ``social function meant the productive needs of the Fascist nation''.\footcite[909]{robilant13} But at the same time, they cleverly denied that there was a ``contradiction between subordinating individual property rights to the larger interest of the Fascist state and the liberal language of autonomy, personhood, and labor''.\footcite[900]{robilant13} In this way, fascist scholars could claim that fascist liberalism was true liberalism, thereby subverting the conceptual basis for the traditional idea of liberal justice.\footcite[900]{robilant13} In this situation, there was reason to suspect that clinging to liberal dogma would be a largely ineffective response. Moreover, it seemed undeniable that fascism's appeal was rooted in real concerns about the fairness and effectiveness of the liberal legal order. 

Hence, many non-fascists shunned away from uncritical defense of traditional liberalism. Instead, they agreed that property's social function should come into focus, but emphasized the plurality of values that could potentially inform this function, not the interests of the state. In addition, they also noted that property rights were invariably associated with {\it control} over resources, and that the social functions of property depended on the resources in question. To own property, they argued, provides individuals with a source of privacy, power and freedom that is, as a matter of fact, highly valued. It is valued, moreover, for its implications in a social context. To capture these insights, Italian scholars adopted the metaphor of a ``tree'', by describing the core social function of property as the trunk, while referring to the various resource-specific values attached to property as branches.\footcite[894-916]{robilant13} As di Robilant notes regarding these theorists:

\begin{quote}
The rise of Fascism, they realized, was the
consequence of the crisis of liberalism. It was the consequence of liberals' insensibility to new ideas about the proper balance between individual rights and the interest of the collectivity.\footcite[907]{robilant13}
\end{quote}

In light of this, the tree-theorists concluded that continued insistence on the protection of the autonomy of owners was not a viable response. Instead, they adopted a theory that ``acknowledges and foregrounds the social dimension of property'', but without committing themselves to fascist ideas about the supreme moral authority of the state.\footcite[907]{robilant13} The value of autonomy was in turn recast in terms of property's social function. Arguably, this served to make the case far more compelling. Protecting autonomy could be seen as an aspect of protecting property's freedom-enhancing function, both at the individual level and as a way of ensuring a right to self-governance and sustenance for families and local communities. This, moreover, could not easily be derided as tantamount to protecting unfair privilege and entitlement. In fact, the suggestion was made in an effort to protect democracy itself.

I believe the story of fascist appropriation of the social function theory provides further weight to my claim that it is sensible to  maintain a descriptive perspective on its core features. Indeed, the readiness with which the fascists embraced social function theorizing serves as a reminder that we cannot easily predict what normative values may come to be promoted on its basis. Hence, it is also call for continuous vigilance when it comes to normative assessment and debate. At the same time, we are reminded of the danger of attaching too much normative prestige to a theory that is abstract and open to various interpretations.

In particular, it seems to me that failure to recognize the descriptive nature of the core idea can lead to unrealistic expectations of what the social function theory actually provides. In addition, it will make it harder for the theory to gain acceptance as a conceptual common ground from which to depart when engaging in debate. Indeed, if no division is recognized between normative and descriptive aspects, the historical record would allow detractors to make a {\it prima facie} plausible attack on the social function theory by arguing that it is fascism in disguise, or that fascism, rather than liberal justice, is where we end up in practice should we chose to adopt it.\footnote{This would echo the claim already made by Claeys, that the theory (when coupled with virtue ethics) might become a slippery slope towards the kind of extremism and revolt against oppression that gave rise to the Rwanda genocide in the early 1990s \cite[926-927]{claeys09}.}

In response, one might retort that this is cherry picking the historical facts, or that the fascists misunderstood or perverted the theory. That is certainly plausible, but the point I am trying to make here is that this kind of debate is in itself unhelpful. Unless the social function theory is rendered neutral enough to be acceptable as the conceptual premise of debate, it is likely going to fail -- in a purely epistemic sense -- as a template for negotiating conflicts about property. Those who oppose the norms associated with the theory will oppose also the core descriptive content, if they feel that the latter commits them to the former. I believe that this, in turn, suggests that those advocating on behalf of the social function theory should take care to avoid rhetorical hubris. The main point to convey, I believe, is that the theory is in fact more accurate, in a purely epistemic sense, than other conceptualizations of property.

The story of the fascist appropriation of the social function theory also points to the danger that often attach to abstract theories with normative implications: That they allow us to opportunistically recast whatever values we wish to promote, by providing qualifications for them in abstract terms that are hard to refute. The fascists did this, and the non-fascists responded. Hence, in the end one could do little more than hope that the fascists' vision of their state as an ``ethical state'' that ``every man holds in his heart'' would eventually prove less attractive then the promise of self-governing communities bustling with diversity in life and character.

\subsection{Towards a normative stance}

The social function theory can facilitate a new kind of normative reasoning, arising from how the theory allows us to recognize more subtle distinctions between different kinds of property and different kinds of circumstances. For instance, staunch entitlements-based approach to autonomy will leave us with little room to differentiate between the protection of investment property and the protection of a home, unless such a distinction is explicitly provided for in the law. But a social function approach compels us to notice the difference and to acknowledge that it might be legally, as well as ethically, relevant. Hence, if we seek to argue for protection of investment property, we must in principle be prepared to face counter-arguments that revolve around particulars of the investor's role in society and his relationship to the community of people that are affected by how he manages his property. Similarly, if someone argues against protecting home ownership, we can respond by drawing on additional arguments based on the importance of the home both to the owner, her family and friends, and her community. Under the social function theory, it becomes generally relevant to address how a home creates a sense of belonging and provides a basis for membership in social structures.

I believe normative assessments should aim to be as concrete as possible. That said, I still think it is worthwhile to provide more abstract forms of expression for core values, to clarify the ethical premises that provide the basis for concrete value-based conclusions. To me, therefore, normative theories should aim to be meta-ethical, not just ethical. They should provide a vocabulary and a conceptual framework tailored to advancing one's values. However, they should recognize that the ultimate expression of those values is given in relation to concrete facts. This, I believe, is prudent in light of how abstract ethical assertions are necessarily imprecise, and run the risk of being distorted or exaggerated, particularly as they gain influence.

Invariably, the most accurate information regarding the values I rely on when assessing cases will be conveyed by my assessment of the cases, not by my theorizing. On a deeper level, I am inclined to believe that value-systems are more or less unique to individuals, so that ethical theories are helpful primarily in that they provide an introduction to keywords and important lines of argument that will recur in different forms. As such, they enhance understanding, making it easier to communicate ideas and opinions in such a way that potential respondents are likely to enjoy a somewhat less inaccurate impression of what they are responding to. 

In short, I believe that ethics make moral judgments communicable, allowing new ideas to be created in the minds of individuals. It should come as no surprise to the reader, therefore, that I believe in ethical men and women, but not in ``ethical Man'' or -- God forbid --  the ``ethical State''. Luckily, I find some support for this world view in recent theories that have been proposed as normative extensions of the social function theory of property. These are the subject of my next section.

\section{Human flourishing}\label{sec:hf}

Taking the social function theory seriously forces us to recognize that a person's relation to property can be partly constitutive of that person's social and personal identity, including both its political and economic components. Hence, property influences people's preferences, as well as what paths lie open to them when they consider their life choices.\footnote{See generally \cite{alexander09}.} This effect is not limited to the owner, it comes into play for anyone who is socially or economically connected to property in some way. The life-significance of property might be clearly felt by a potentially large group of non-owners as well.\footcite[128-129]{alexander09d} The importance of property is obviously reduced if we move away from it in terms of social or economic distance. Hence, in many cases, property will be most important to its owner, simply because she is closest to it. This is not always the case, however, especially not if property rights are unevenly distributed, or in the possession of disinterested or negligent owners. Moreover, as mathematically oriented sociologists take pride in pointing out, social connections are ubiquitous  and the world is often smaller than it seems.\footnote{See generally \cite{schnettler09}.}

Hence, there is certainly potential for making wide-reaching socio-normative claims on the basis of this perspective on the meaning and content of property. But which such claims {\it should} we be making? According to some, we should adjust our moral compass by looking to the overriding norm of {\it human flourishing} as a guiding principle of property law. Colin Crawford, for example, explicitly argues that the social function theory of property should ``secure the goal of human flourishing for all citizens within any state''.\footcite[1089]{crawford11} In a recent article, Alexander goes even further, by declaring that human flourishing is the ``moral foundation of private property''.\footcite[1261]{alexander14} 

As I have already explained, I believe -- in contrast to both Crawford and Alexander -- that it is useful to decouple such normative claims from the descriptive core of the social function theory.\footnote{Crawford comments that the social function theory on its own  ``is not self-defining and invites many interpretations'', see \cite[1089]{crawford11}. The normative theory he proposes is clearly aimed at filing this perceived gap, by pinning down normative commitments that Crawford believes are intrinsic to the theory. However, as I have already argued, I reject this approach, since it unwisely downplays the fact that the social function theory can serve as a common ground among commentators with widely divergent normative views. Indeed, Crawford himself refers unfavorably to a writer who addresses the social function theory, but who, according to Crawford, proposes that ``property's social function is best served by focusing on overall economic production and efficiency in a given society, allowing the market's invisible hand to work its magic'', \cite[see][1089]{crawford11}. Against Crawford, I would argue that it is better to counter such a claim by arguing why it is normatively wrong than by suggesting that people with such values should be discouraged from attempting to argue for them on the basis of a social function understanding of property. Rather, by encouraging such an argument it should become easier to make the case why the values promoted are ultimately undesirable. This, at least, should follow if Crawford is otherwise largely correct (as I think he might be).} I therefore refer to the more distinctly normative aspects of their work as human flourishing theorizing. 

Human flourishing has a good ring to it, but what does it mean? According to Alexander, several values are implicated, both public and private.\footnote{See generally \cite{alexander14,alexander11}.} Importantly, Alexander stresses that human flourishing is {\it value pluralistic}.\footnote{\cite[750-751]{alexander09}.} There is not one core value that always guarantees a rewarding life. To flourish means to negotiate a range of different impulses, both internal and external. Importantly, these all act in a social context which influences their meaning and impact.\footcite[1035-1052]{alexander11}

For the family of a homeowner, the value of the ownership tends to be great; a home is a home for any non-owner living there, just as much as it is a home for the owner. This, in turn, creates both commitments and opportunities for the owner, which may or may not find recognition in the law and our legal reasoning. Regardless of this, it certainly carries significant importance both to her life and the lives of those that depend on her. If property is rented out as a home to someone else, the importance of ownership may be {\it greater} to a non-owner. Indeed, assuming a society where tenancy is a well-functioning social institution, the continuation of the established property pattern might well be of greater importance to the tenant than it is to the owner.

The effect on non-owners can also be restrictive in socially desirable ways. If an apartment has an owner, it discourages squatters, for instance. Moreover, this effect clearly depends also on {\it who} the owner is and the choices she makes in managing her property. If the owner lives in the apartment, squatting is hopefully not going to be an issue. But even the owner of an unoccupied apartment can discourage squatting by managing her property well. However, if owners mismanage their apartments, for instance because they seek to obtain demolition licenses, squatters can take opportunity of this. The risk, of course, increases if housing cannot be afforded by a large number of society's members. In this case, it is natural to argue that something is amiss with the prevailing property structures.

Now, the social function theory of property can also come into play, since it allows us to attach significance to this also when discussing the property rights of individual owners.  In particular, we are not compelled to pretend as though possible failures of property as a social institution are irrelevant when considering rights and responsibilities attached to it. As a matter of fact, they are not; actual squatters clearly affect the owner, influencing both the meaning and the value of her property, both to her, potential buyers, the local government, the state, and other interested parties. Even the mere {\it risk} of squatting can play such a role. But a property theory which does not recognize the social function of property might not allow us to take this into due regard. As long as the standard expectation of an owner is to be able to enjoy her apartment free of squatters, an entitlements-based view on property could easily force us into denial regarding actual (risk of) squatters.

In particular, we would be led to consider squatting as an interference with the owner's rights which the state can not, on pain of disrespecting property, recognize as a legitimate response to mismanagement and imbalances in the property structure. The normative significance of real life -- where squatting often happens due to badly managed property -- is discounted  because our conceptual glasses block it out. Then, the almost unavoidable consequence is that the state also recognizes a {\it positive} obligation to forbid squatting, and to forcibly remove squatters on behalf of owners. Under a narrow entitlements-based conception, this is the natural outcome, and must be classified as an act of protecting private property. Hence, under classical liberal values, it also becomes {\it good}. Here, however, the social function theory permits us to take a highly divergent view, to carry forward different value-judgments.

In particular, if squatting is recognized as creating new interests and obligations attaching to the property, it may now be argued that  it is the use of state power to evict that is the most severe act of interference. Not only interference in whatever housing rights the squatters may have, but in fact also as an interference in {\it property}. Hence, such state action might itself be morally suspect and held to be in need of further justification. In the Netherlands, the Supreme Court adopted a line of reasoning reflecting these insights, when it held that the right not to be disturbed in one's home life also applied to squatters. Hence, the property owner could not forcibly evict people who had taken up residence in her property.\footnote{See NJ 1971/38. The court held that the lower court had erred in taking it proven that the ``house in the original charge was ``in use'' by the owner of this house'', as required by the statute under which the squatters were tried. Instead, the Supreme Court held that ``art.138, in so far as it mentions houses, is specifically aimed at protecting home rights, in connection with which the words ``in use'' (differently than the court judged) can only be understood as ``actually in use as a house'' , as in accordance with ordinary use of language''. The upshot was that it was the squatters, not the owner, who enjoyed protection under the statute. In terms of the bundle theory, a right thought to be in the owner's bundle was deemed to actually belong to the bundle of the squatters, as this corresponded better to the circumstances of the case and the purposes meant to be served by the statute in question.}

In South Africa, a somewhat similar line of reasoning was adopted in the recent case of {\it Modderklip East Squatters v. Modderklip Boerdery (Pty) Ltd}, analysed in depth by Alexander and Pen\~{n}alver.\footcite[154-160]{alexander11} The case dealt with squatting on a massive scale: Some 400 people had taken up residence on land owned by Modderklip Farm, apparently under the belief that it belonged to the city of Johannesburg. The owner attempted to have them evicted and obtained an eviction order, but the local authorities refused to implement it. Eventually, the settlement grew to 40 000 people and Modderklip Farm complained that its constitutional property rights had not been respected.

The Supreme Court of Appeal concluded that Modderklip's property rights had indeed been violated, but noted that so had the rights of the squatters, since the state had failed to provide them with adequate housing.\footnote{See \cite{modderklip04}.} However, they upheld the eviction order and granted Modderklip Farm compensation for the state's failure to implement it. The Constitutional Court, on the other hand, while agreeing that the eviction order was valid, concluded that as long as the state failed in its obligations towards the squatters, the order should not be implemented.\footcite{modderklip05} The eviction of the squatters, in particular, was made contingent upon an adequate plan for relocation. In the meantime, Modderklip would receive monetary compensation, from the state rather than the squatters. In this way, the Court recognized the social function of property; they refused to give full effect to Modderklip's property rights as long as that meant putting other rights in jeopardy. The fact that the squatters had no place to go, in particular, was allowed to influence the content of Modderklip's right, making it impermissible to implement a standing eviction order.

It is possible to cast this outcome as an interference in property rights that was regarded as acceptable in the public interest. However, the reconceptualization in terms of property itself having a social function appears highly attractive. Moreover, it is also consistent with the South African constitution, which also focuses on property's social dimension.\footnote{See section 25 of the Constitution of the Republic of South Africa, Act 108 of 1996.} Thinking about cases like {\it Modderklip} in terms of property's social function allows us to remove the state as an intermediary between the owner and the other interested parties, in this case the squatters. As argued by Alexander and Pe\~{n}alver, it becomes possible to think of the Court as adjudging based on Modderklip's own responsibility, as an owner, towards other members of the community that have an interest in the property.\footnote{\cite[157]{alexander11} (``The courts' unwillingness to ratify Modderklip's desire to remove the squatters from its land illustrates the courts' willingness to take seriously the obligations of owners, not only as they concern owners' direct relationship with the state but also in relation to the needs of other citizens'').}

On this basis, it becomes easier to conclude that it is permissible for courts to take the social context into account even in the absence of any specific state action or legislation to indicate that this should be done, or that the public interest is at stake. Indeed, one of the problems in {\it Modderklip} was that the state had failed also in its responsibility towards the squatters. Moreover, while the local sheriff had refused to implement it, an eviction order had in fact been granted. Hence, thinking of the case as interference in the public interest becomes difficult.

More importantly, by taking into account the social function of property, it becomes possible to argue for the outcome in Modderklip positively on the basis of property values. In this way, property is no longer seen to stand in the way of justice in cases such as this. We need not ``interfere'' with rights to secure an appropriate outcome, we only need to apply property law. As Alexander puts it in another recent article: 

\begin{quote} The values that are
part of property's public dimension in many instances are necessary
to support, facilitate, and enable property's private ends.
Hence, any account of public and private values that depicts them as categorically
separate is grossly misleading. One important consequence of this
insight is that many legal disputes that appear to pose a conflict between
the private and public spheres or that seemingly
require the involvement of public law can and
should, in fact, be resolved on the basis of private law -- the law
of property alone.\footcite[1295-1296]{alexander14} \end{quote}

Protection of property, when property is understood in this way, becomes a potential source of justice, also for squatters. The basic values attached to property -- freedom, liberty, autonomy -- have not really changed, but are applied in a new way. In particular, they no longer only apply to the owner's interest in property, but also to that of other individuals closely connected to it. This normative turn, I argue, will potentially strengthen the institution of property itself, while also serving to loosen the compulsiveness of the idea that the ultimate expression of the public interest is found in the actions taken by the state. It suggests rather the view that the public interest manifests wherever the public may reside, including in property. This conclusion requires taking a normative stance, but a minimal one; we merely extend the scope of values traditionally attached to property.\footnote{Arguably, cases such as {\it Modderklip} might be taken to suggest that the social function theory, as soon as it is applied for the purposes of normative assessment, will systematically guide us to conclude that owners are not entitled to as many benefits as would otherwise follow from their property rights. It is fortunate, therefore, that the entire remainder of the thesis will focus on economic development takings, where it will typically appear more natural to conclude the opposite. In these cases, on a common- sense understanding of justice, applying the social function theory will allow us to recognize a sense in which owners should receive {\it increased} protection and more benefits, as a consequence of how such interferences can prove particularly damaging, both to the owner and to the social fabric of democracy.} 

That said, in the case of {\it Modderklip} the court was clearly faced with a value conflict that it is hard to resolve by looking to traditional liberal values. If these apply equally to the squatters, we are left with deadlock rather than resolution. Indeed, this was also reflected in the outcome of the case, which did not resolve the matter, but merely concluded that the state had failed in its obligations towards both of the parties. What should the solution be in the end? Should the squatters be allowed to stay, following condemnation of Modderklip's land, or should alternative housing be provided so that the eviction order can be carried out? The answer requires us to resolve a normative conflict, and how to do so might not be obvious. Moreover, value pluralism suggests that we must be prepared to engage with multiple ways of looking at the matter. In the interest of stability of property as an institution, allowing the squatters to succeed in establishing lasting title to the land might be considered unwise. Against this pragmatic and largely technical value, one would have to consider the values of community and belonging that now attach the squatters to their new homes. These two values are largely incommensurable, and it is not clear how to choose between them.

Still, Alexander maintains that human flourishing provides an ``objective'' standard on which to approach dilemmas such as these. Moreover, he ``rejects the view that what is good or valuable for a person is determined entirely by that person's own evaluation of the matter''.\footcite[1263]{alexander14} Some things are good for people, Alexander argues, irrespectively of whether or not people know so themselves. Hence, it may perhaps be argued that what is truly good for Modderklip is to come to an arrangement with the squatters and the state, to resolve the problem amicably. Moreover, failure to do so might entitle the state to take action that would otherwise seem to undermine the stability of property. This, then, would be partly due to this being conducive also to the flourishing of the people behind Modderklip, not only the squatters.

That, clearly, might be derided as an overly intrusive and moralistic way to approach property law. More generally, as Alexander notes, the exact content of goodness is ``necessarily contestable''. It consists of a list of different values which are all open to dispute, both as to their relevance and their precise meaning.\footcite[1263]{alexander14} Alexander goes on to list some key values that he believes are central, but the list is not meant to be exhaustive.\footcite[1764-1776]{alexander14}

Among the key values that Alexander discusses, we find many core private values that are commonly seen as important goals for the institution of property. This includes values such as autonomy and self-determination, both of which will feature heavily later in this thesis. However, Alexander also considers several public values, such as equality, inclusiveness and community. These too will be important later, as I will draw on them in my own normative analysis of economic development takings. I will be particularly concerned with the value of {\it participation}, understood, following Alexander, in terms of its broad social function.\footcite[1275-1276]{alexander14}

In my view, this value is closely related to the value of democracy. Participation in local decision-making processes is the root which enables democracy to come to fruition at the regional and national level. Moreover, participation is a value that will give me occasion to make particular policy suggestions regarding the correct way to approach economic development takings. Devoting some time to discussing this value in the abstract will therefore be helpful.

Alexander traces the value of participation back to Aristotle and the republican tradition. He notes, however, that this tradition involves a notion of participation that is somewhat narrowly drawn. For thinkers in the republican tradition, participation tends to mean public participation, meaning people's engagement with the formal affairs of the polity.\footcite[1275]{alexander14} For Alexander, participation has a broader meaning, involving also the value of being included in a community. He writes:

\begin{quote}
We can understand participation more broadly as an aspect of inclusion. In this sense participation means belonging or membership, in a robust respect. Whether or not one actively participates in the formal affairs of the polity, one nevertheless participates in the life of the community if one experiences a sense of belonging as a member of that community.\footcite[1275]{alexander14}
\end{quote}

Importantly, participation in a community can have a crucial influence also on people's preferences and desires. In this way, it is also invariably relevant -- behind the scenes -- to any assessment of property that focuses on welfare, utility or public participation in the classical sense. As Alexander and Pe\~{n}alver put it, drawing on the work of Amartya Sen and Martha Nussbaum:\footcite{sen84,sen85,sen99,nussbaum00,nussbaum02}
\begin{quote}
The communities in which we find ourselves play crucial roles in the formation of our preferences, the extent of our expectations and the scope of our aspirations.\footcite[140]{alexander09}
\end{quote}
Therefore, for anyone adhering to welfarism, rational choice theory, utilitarianism or the like, neglecting the importance of community is not only normatively undesirable, it is also unjustified in an epistemic sense. In particular, it should be recognized as a descriptive fact that community is highly relevant to {\it any} normative theory that attempts to take into account the preferences and desires of individuals.\footnote{Again, I think Alexander and other theorists attempting to incorporate such ideas in property law could benefit from making this descriptive point separately, so as to enable it to be considered in isolation from the more contentious normative arguments they construct on the basis of it.} But Alexander and Pe\~{n}alver go further, by arguing that participation in a community should also be seen as an independent, irreducibly social, value, not merely as a determinant of individual preferences and a precondition for rational choice. They write:

\begin{quote}
Beyond nurturing the individual capabilities necessary for flourishing, communities of all varieties serve another, equally important function. Community is necessary to create and foster a certain sort of society, one that is characterized above all by just social relations within it. By ``just social relations'', we mean a society in which individuals can interact with each other in a manner consistent with norms of equality, dignity, respect, and justice as well as freedom and autonomy. Communities foster just relations with societies by shaping social norms, not simply individual interests.\footcite[140]{alexander09}
\end{quote}

This, I think, is a crucial aspect of participation. Moreover, it is one that it is hard, if at all possible, to incorporate in theories that take  preferences and other attributes of individuals as the basis upon which to reason about property. For instance, if people in a community come under pressure to sell their homes to a large commercial company that wishes to raze them in order to construct a shopping mall, it may be appropriate to consider this as an unjustifiable attack on their property rights. Importantly, this may be so {\it irrespectively} of what the individual owners themselves think they should do. If they are offered generous financial compensations for their homes, or are threatened by eminent domain, economic incentives might trump the value of social inclusion and participation for all or a majority of these owners. As a consequence, the community might decide to sell.  

Even so, in light of the value of community, it would be in order for planning authorities, maybe even the judiciary, to view such an  agreement as an {\it attack on their property}. It is clear, in particular, that by the sale of the land, the ``just social relations'' inhering in the community will be destroyed. The members of the community -- including all the non-owners -- will lose their ability to participate in those relations. More concretely, the nature of the property rights that once contributed to sustaining ``just relations'' will now be transformed into property rights that serve different purposes. This includes aiding the concentration of power and wealth in the hands of commercially powerful actors. Such a change in the social function of property might have to be regarded -- objectively speaking -- as a threat to participation, community and democracy. Hence, on the human flourishing theory, it is also a threat to property. Our property institutions, therefore, should protect against it.

To demonstrate the general significance of such a line of normative reasoning, it is illustrative to mention a scenario -- not directly implicating property -- that is currently beginning to attract much attention in legal scholarship. This scenario arises in relation to the right to {\it privacy}. This right, of course, is increasingly perceived to be coming under threat in the information age. Crucially, it is beginning to become clear to legal theorists that viewing privacy merely as a private right is not going to provide a sustainable template for dealing with this challenge.\footnote{See generally \cite{schafer14}.} It seems, in particular, that people are simply too willing to give it up. This, in turn, contributes to the formation of potentially harmful social structures on the web. In particular, the lack of privacy becomes an impediment to dignity, freedom and respect in web societies. In this way, both individuals and society as a whole will eventually suffer, although this truth is not reflected in our individual preferences. Hence, it has been proposed that privacy should be considered also as a {\it common good}, so that protecting the privacy of individuals, in some cases, is an imperative irrespectively of what these individuals themselves desire and prefer. Privacy, in this way, becomes also an obligation, mirroring the similar phenomenon that we have observed with respect to the right to property.

There is a subtle issue that arises on the basis of this kind of normative reasoning about individual rights. Is it appropriate, in particular, to still think of such reasoning -- and the obligations it gives rise to -- as an aspect of protecting individuals? Is it not more accurate to say that this is an {\it interference} with individual rights, undertaken to further the public interest? Indeed, when the individual himself does not want his property or privacy to be ``protected'', is it not somewhat perverse to insists that this is what is happening? 

I am inclined to answer in the negative. In my opinion, we are still talking about protecting individual rights, even when this means imposing protections on people that they themselves do not want. Undoubtedly, this is {\it also} an interference in their rights, but just as different rights of different people can sometimes come into conflict, I am inclined to think that the same right, for the same person, can sometimes come into conflict with itself. This happens, in particular, when it is not possible to simultaneously protect all those functions that this right seeks to promote. 

For instance, if someone protests a taking on environmental grounds and also rejects financial compensation as immoral, the courts should still award just compensation for the land, if they find that the taking is valid. If the owner wishes, he can purge himself by making a donation to charity. Similarly, if someone attempts to commit suicide, the health services are still obliged to help, even against the patients wishes. This remains the case, moreover, even though suicide is no longer considered a criminal offense in the public interest. 

Protecting individuals against their will is condescending, no doubt, but it is still different, and often preferable, from subordinating their interests to that of the general public. If the justification for an act of interference is a vague proclamation of the ``public interest'', the individual is marginalized from the very start. A balancing act might be required, but this renders the individual relevant only to one side of the equation. On the other hand, if the act of interference is simultaneously rendered as protection, enforcement of an obligation, or a measure to enable participation, the individual occupies center stage. In so far as the public interests triumph, it is not because the individual loses, but because the public is deemed to know best how to secure the goal of human flourishing, both for the individual herself and other members of the social structures that surrounds her.

For instance, external interests of both a private and a public nature can dictate that owners should avoid becoming a nuisance to their neighbors. But under a human flourishing theory, we are also able to portray this as a case of protecting the individual's membership in the community. The public does not ``side with the neighbors'', but undertakes measures to protect the relationship between the owner and his fellows. In my opinion, a conceptual approach to property law that makes this portrayal plausible is highly desirable. 

For a second example, consider situations when environmental concerns suggest imposing restrictions on what an owner is permitted to do with his land. This too can be rendered as an act of protecting property. But doing so requires the regulatory body to relate the interference positively to the individual's interests and obligations, to ensure that they avoid adopting a narrative where the regulation is rendered as an act of enforcing the will of unnamed others against the will of specific owners. In this way, public values and the public interest can be given considerable weight, but will have to be rendered less abstract. In particular, these interests must be related concretely to the social functions of the rights protecting the individuals interfered with. The baseline for assessment remains actual persons and their well-being, not some abstract ideal of ``goodness''. Moreover, implementation of the collective will becomes a guide towards human flourishing for a society of individuals, not a goal in itself.

An individual might well be offended if the state adopts this narrative and implements behavioral restrictions by declaring ``it's for your own good''. But, I would argue, that is exactly as it should be. Any restriction of individual freedom is an offense, but one that is sometimes appropriate. If this is conveyed to people with a marginalizing ``your interests are not as important as ours'', the response might well be silence. But beneath the silence we may find disinterested apathy, or worse: contempt and despair. The interference is no longer an insult, but this is not because it is any more convincing to proclaim that interference is ``necessary for the greater good''. Rather, the interference is no longer an insult because it fails to properly engage the individual at all. The role of the person interfered with becomes passive -- she becomes an obstacle that needs to be removed. If such a dynamic of governance develops, the individual might take from this the lesson that she is unimportant in the greater scheme of things, that her interests are subordinate to those of ``the others'', and that her voice is not meant to be heard.

This is normatively undesirable. It represents a situation when the social effect of interference might become detrimental to society itself, particularly to the institution of democracy. It damages its roots, namely the ``just social structures'' that Alexander identifies as being at the core of the human flourishing theory. A better alternative, then, is to interfere in a way that constructively targets the individual, aiming to protect her by enabling her -- and compelling her -- to protect others and partake in social and political life. This can then become interference aimed at bringing the individual into the fold, making her play her part, by raising her to fruitful citizenship. Such a paternal (or maternal) state is one that cares, but one that may also be overprotective, unfair, or plain stupid. Hence, it becomes natural to resist and to revolt, but not without also carrying forward care and love for the social, political and legal structures within which this agency is (hopefully) permitted to take place.

The upshot, I believe, is that condescension in property law can be a good thing. To conceptualize an act of restriction as a means to empower the persons restricted is something they might well find offensive, but it also renders interference more meaningful to them. It provides both a reason to take a more active role in relation to the interfering power, and a possible cause for constructive resistance. Importantly, it does not force the conclusion that the public resides behind closed doors, disinterested in what the affected individual have to offer. Instead, it is an approach that encourages a response, by focusing always on the persons interfered with, whenever interference is deemed necessary. This is the vision of a bottom-up, rather than a top-down, approach to imposing the collective will on individuals. I believe it has merit. 

It will remain in the background as I now move on to apply the theories discussed in this and preceding sections. In the next section,  I return to the issue that will remain in focus for the remainder of this thesis. First, I will introduce economic development takings by considering the seminal case of {\it Kelo v City of New London}\footcite{kelo05}, which brought this category to prominence in the US discourse on property law. Then I will assess the unique aspects of such takings against the social function theory, to provide an argument that the category has significance for legal reasoning in takings law, as well as with respect to property as a constitutionally protected human right. Finally, I will provide an abstract presentation of the values that I believe should be considered important when normatively assessing the law in this area. In doing so, I will draw on the human flourishing theory, setting out the main values that will inform the concrete policy assessments I provide later. 

\section{Economic development takings}\label{sec:edt}

Constitutional property rules in many jurisdictions indicate, with varying degrees of clarity, that eminent domain should only be used to take property either for ``public use', in the ``public interest'', or for a ``public purpose''. Such a restriction can be regarded as an unwritten rule of constitutional law, as in the UK, or it can be explicitly stated, as in the basic law of Germany.\footnote{See Chapter \ref{chap:2}. Section \ref{sec:contrast} below.} In some jurisdictions, for instance in the US and in Norway, explicit property clauses exist, but are not formulated clearly.\footnote{See Chapter \ref{chap:2}, Section \ref{sec:us} and Chapter 3, Section \ref{sec:norexp} below.}

Both the Norwegian and the US property clauses appear to refer to public use only as a precondition for the duty to pay compensation. However, they are also universally read as expressing the {\it presupposition} that the power of eminent domain is only to be used in the public interest.\footnote{In the literature, it is rare to even note that a different interpretation is linguistically possible. But see \cite[205]{berger78}.} Indeed, in cases when one might say that private property is ``taken'' for a non-public use without compensation, for instance in a divorce settlement, it is not commonly regarded as an exercise of eminent domain. Rather, it is justified by making reference to a different category of rules, meant to ensure enforcement of obligations that arise between private parties independently of the state's power to single out and compulsorily acquire specific properties.

The exact boundary between eminent domain and other forms of state interference in property may not always be clear, but I will not worry too much about it in this thesis. I note, moreover, that most, if not all, legal scholars seem to agree that the power of eminent domain is meant to be exercised in the public interest. However, differences of opinion emerge when we turn to the question of whether the presupposed public use or public interest in the property taken serves also to restrict the power to take. In the US, most scholars agree that some restriction is intended, but there is great disagreement about its extent.\footcite[205]{berger78} In Norway, on the other hand, a consensus has developed that the public use limitation is so wide that it hardly amounts to a restriction at all.\footnote{See, e.g., \cite[368]{aall10}.} Moreover, the courts defer almost completely to the assessments made by the executive branch regarding the purposes that may be used to justify a taking.\footcite[368]{aall10}

Some US scholars adopt a similar stance, but others argue that the public use presupposition should be read as a strict requirement, forbidding the use of eminent domain unless the public will make actual use of the property that is taken.\footnote{Compare \cite{bell06,bell09,claeys04,sandefur06}.} Most scholars fall in between these two extremes. They regard the public use restriction as an important, practically relevant, limitation, but they also emphasize that courts should normally defer to the legislature's assessment of what counts as a public use.\footnote{See, e.g., \cite{merrill86,alexander05}. The fact that US jurists usually stress deference to the legislature, not the executive branch, should be noted as a further contrast with Norway.}

As I discuss in more depth in Chapter \ref{chap:2}, Section \ref{sec:hop}, the debate in the US has its roots in case law developed by state courts -- the federal property clause was for a long time not enforced against states. This has changed, however, and today the Supreme Court has a leading role also in this area of US law. It has developed a largely deferential doctrine, resembling the understanding of the public use limitation under Norwegian law.\footnote{See \cite{berman54,midkiff84,kelo05}.} The difference is that in the US, cases raising the issue  still regularly arise, and still prove controversial. The most important such case in recent times was {\it Kelo}, decided by the Supreme Court in 2005.\footnote{kelo05} This case saw the public use question reach new heights of controversy in the US.\footnote{See, e.g., \cite{somin09}.}

{\it Kelo} centered around the legitimacy of taking property to implement a redevelopment plan that involved construction of research facilities for the drug company Pfizer. The home of Suzanne Kelo stood in the way of this plan, and the city decided to use the power of eminent domain to condemn it. Kelo protested, arguing that making room for a private research facility was not a permissible  `public use''. She was represented by the libertarian legal firm {\it Institute for Justice}, which had previously succeeded in overturning similar instances of eminent domain at the state level.\footnote{See \url{https://www.ij.org/cases/privateproperty}.} Kelo lost the case before the state courts, but the Supreme Court decided to take it on, and they looked at it in great detail.

The precedent set by earlier federal cases was clear: As long as the decision to condemn was ``rationally related to a conceivable public purpose'', it was to be regarded as consistent with the public use restriction.\footcite[241]{midkiff84} Moreover, the role of the judiciary in determining whether a taking was for a public purpose was regarded as ``extremely narrow''.\footcite[32]{berman54} It had even been held that deference to the legislature's public use determination was required ``unless the use be palpably without reasonable foundation'' or involved an ``impossibility''.\footnote{See \cite[66]{dominion25}; \cite[680]{gettysburg96}.}

This understanding had also been reflected in the outcome of concrete cases resembling the situation in {\it Kelo}: In {\it Hawaii}, the Supreme Court had upheld a taking that would benefit private parties, with no direct benefit to the public.\footnote{\cite{midkiff84}. For a more detailed discussion, see Chapter \ref{chap:2}, Section \ref{sec:hop} below.} In {\it Berman}, it had upheld a taking for economic redevelopment of a blighted area, even though the property taken was not itself blighted.\footnote{\cite{berman54}. For a more detailed discussion, see Chapter \ref{chap:2}, Section \ref{sec:hop}.} But in the case of {\it Kelo}, the court hesitated.

Part of the reason was no doubt that takings similar to {\it Kelo} had been heavily criticized at state level, with an impression taking hold across the US that eminent domain ``abuse'' was becoming a real problem.\footnote{See, e.g., \cite[667-669]{sandefur05}.} A symbolic case that had contributed to this worry was the infamous \textcite{poletown81}. In this case, General Motors had been allowed to raze a town to build a car factory, a decision that provoked outrage across the political spectrum.\footnote{See generally \cite{sandefur05}.} The case was similar to {\it Kelo} in that the taker was a powerful commercial actor who wanted to take homes. This, in particular, served to set the case apart from  {\it Hawaii}, which involved a taking in favor of tenants, and to some extent also {\it Berman}, which involved a taking of businesses (and homes) in the interest of combating blight. Moreover, the Michigan Supreme Court had recently decided to overturn {\it Poletown} in the case of \textcite{wayne04}. Hence, it seemed that the time had come for the Supreme Court to reexamine the public use questions.\footnote{See, e.g., \cite{sandefur05,claeys04}.}

Eventually, in a 5-4 vote, the court decided to apply existing precedent and held against Suzanne Kelo. The majority also made clear that economic development takings were indeed permitted under the public use restriction, also when the public benefit was indirect and a private company would benefit commercially.\footcite[469-470]{kelo05} The backlash of this decision was severe. According to Ilya Somin, the case ranks among the most disliked decision that the Court has ever made.\footcite[2]{somin11} Some 80 - 90 \% of the US public expressed great disapproval, with critical voices coming from across the political spectrum\footcite[2108-2110]{somin09} Why did the case prove so controversial? No doubt, the discontent with the decision was fueled in large part by the fact that it was seen as a case of the government siding with the rich and powerful, against ordinary people.\footnote{\cite[630-634]{baron07}} Indeed, the party that appeared to benefit the most from the taking was Pfizer -- a multi-billion dollar company -- while Suzanne Kelo, who stood to lose, was a middle class homeowner. In this context, the taking of Kelo's home seemed morally suspect, an act of favoritism showing disregard for less influential members of society.\footnote{See, e.g., \cite{underkuffler06}.}

In addition, it is worth noting that many commentators conceptualized the {\it Kelo} case by thinking of it as belonging to a special category, by describing it as an economic development taking, a {\it taking for profit}, or, more bluntly, a case of {\it Robin Hood in reverse}.\footcite{somin05} Categories such as these had no clear basis in the property discourse before {\it Kelo}. Indeed, in terms of established legal doctrine, it would be more appropriate to say that the case revolved entirely around the notion of ``public use''. 

However, when we consider the most common reasons given for condemning the outcome in {\it Kelo}, we readily grasp why critics felt it was natural to classify the case along an additional dimension. A survey of the literature shows that many critical voices made use of a combination of substantive and procedural arguments to  paint a bleak picture of the {\it context} surrounding the decision to take Kelo's home. Important concrete factors that critics tend to stress include the imbalance of power between the commercial company and the owner, the incommensurable nature of the opposing interests, the lack of regard for the owner displayed by the decision makers, the close relationship between the company and the government, and the feeling that the public benefit -- while perhaps not insignificant -- was made conditional on, and rendered subservient to, the commercial benefit that would be bestowed on the commercial beneficiary.\footnote{See, for instance, \cite{underkuffler06,somin07,sandefur06,cohen06,hafetz09,hudson10}.}  This dynamic, in which public bodies no longer seem to be leading and pushing the process forward, but are also -- to quite some extent -- being led and being pushed, is regarded as particularly suspicious. This, in turn, is derided as a perversion of legitimate decision-making, used to argue more broadly that economic development takings such as {\it Kelo} suffer from what I will refer to here as a {\it democratic deficit}.

From a theoretical point of view, I take all of this to suggest that many critics of {\it Kelo} effectively adopted a social function view on property, by paying close attention to the wider social and political context of the taking.\footnote{For a particularly clear example of this, see \cite{underkuffler06}.} Importantly, if we now turn to the social function theory of property, we are placed in a position to engage more actively with this form of reasoning, as an integrated part of our assessment of the law. This may then in turn give us cues as to how we should reason -- within the law -- to justify a departure from the course laid down by previous cases on the ``public use'' requirement, where such a perspective was not adopted. Indeed, it seems to me that this is exactly what the minority of the Supreme Court did, particularly Justice O'Connor, who formulated a strongly worded dissent.\footnote{\cite[494-505]{kelo05}. Justice O'Connor was joined by the four other dissenters, but Justice Thomas also formulated his own dissent, taking a more narrow view and arguing for the revival of a strict reading of the public use requirement, see \cite[505-523]{kelo05}.} She writes as follows:

\begin{quote}
Any property may now be taken for the benefit of another private party, but the fallout from this decision will not be random. The beneficiaries are likely to be those citizens with disproportionate influence and power in the political process, including large corporations and development firms. As for the victims, the government now has license to transfer property from those with fewer resources to those with more. The Founders cannot have intended this perverse result.\footcite[505]{kelo05}
\end{quote}

It seems to me that the values Justice O'Connor rely on in her assessment are closely related to the idea of human flourishing presented by Alexander and others, particularly those pertaining to the political function of property as an anchor for community and democracy. Indeed, the danger of powerful groups gaining control of the power of eminent domain does not only affect the individual entitlements of owners. It also affects society, as the economic rationality used to justify interference comes to result in an implicit political statement to the effect that the property of the rich and powerful is better protected, and valued higher by the state, than property owned by regular citizens, who reside in ordinary communities.

The effect of a traditional economic development taking is that property rights are transferred from the many to the few, taken from ordinary people and given to the powerful. Hence, these cases represent a possibly pernicious redistribution of property, not necessarily in financial terms -- depending on the level of compensation -- but surely in terms of property's social function. The structural imbalances of the condemnation process itself find permanent expression in the new distribution of property. The social structures of a living community are dismantled in favor of a social structure that revolves around the commercial interest of a company. The political and social power of the community is diminished, perhaps lost in its entirety, while the political and social power of the company increases.

It seems clear that to Justice O'Connor, this too is a negative consequence of the taking. Again, we notice that recognizing this effect requires a social function approach to property. There is no clearly quantifiable individual loss -- no one particular ``stick'' in the property bundle that is not compensated. Rather, it is the community itself that is lost, a community that was not directly implicated in any ``entitlement'', but which played a crucial role in providing meaning to the totality of the bundle enjoyed by the owner. Even if we extend our perspective to account for indirect individual losses, we are not doing justice to the loss in this regard. The owner might relocate, acquire new property with a similar meaning in a new community somewhere else. But that does not make up for the fact that {\it this} community is lost forever, as {\it this} property takes on new meanings and functions. The loss to Suzanne Kelo, therefore, might  even be a significant loss to the City of New London.

Of course, the economic and social gains of development might outweigh such negative effects on community. But, arguably, the balancing of interests required in this regard can only be carried out by an institution that sufficiently recognizes the owners' and their community's right to participation and self-governance. The presence of a highly active commercial third party, in particular, means that public participation in the standard sense might be insufficient. In economic development takings, the commercial company typically appears alongside the government, as a more or less integrated part of the institutional structure making the decision to condemn. The owners, however, do not enjoy a corresponding level of participation.

In particular, their interests are only negatively defined. They are adversely effected and may object, but under standard administrative regimes they play no constructive role in the process. For instance, they are not called on to take part in the development itself, or to assess its merits more broadly than by being asked to respond based on their own individual entitlements. In fact, I think this is one of the main problems with economic development takings. I will argue for this in more depth later, but I remark here that an important reason to focus on this aspect is that it involves precisely those values that economic development takings are most likely to offend against. In particular, if the loss of community outweighs the positive effect of economic development, this is unlikely to be recognized by a process that relies mainly on the positive contribution of the developer and the expert planners.\footnote{A similar point is made in \cite{underkuffler06}.} 

The objections made by owners, moreover, may not only be given too little weight given the imbalance of power between owners and developers. As long as owners themselves focus only on the individual loss, they may not get to those issues that are the most important for property's social function. However, I do not think it is sufficient to theoretically proclaim that these aspects need to be considered. To address the democratic deficit of economic development takings, it seems likely that institutional changes will have to be made, to give those functions a voice in the decision-making process. This should ensure greater involvement by the local community (including, perhaps, even non-owners) in the decision-making process relating to development. Not only should they be asked if they have objections. They should be be included in a constructive way, perhaps even be compelled to assume an active role in relation to the proposed project.

This is a proposal that envisages owners engaging directly with both government and potential developers, consider alternative schemes, and make their own proposals. In short, this asks for a system where owners participate as a community. According to the human flourishing theory as I understand it, this is not only a right, but also an obligation. It gives a plausible basis on which to strike down economic development takings, and to do so without giving up the value of judicial deference. In addition, it is a call for institutional reform, to search for new governance frameworks that will empower owners and their communities.

It seems to me that Justice O'Connor's argument reflects some of these ideas. Indeed, she seems to believe strongly that the taking of Kelo's home would be a particularly harmful interference in the ``just social structures'' surrounding it. Importantly, a piece-by-piece entitlement-based approach to {\it Kelo} could hardly justify the degree of disapproval seen in Justice O'Connor's opinion. After all, Kelo had been offered generous compensation, there had been no clear breach of concrete procedural rules, and the claim that the taking was {\it only} a pretext to bestow a benefit on Pfizer did not seem supported by the facts.\footnote{See \cite{bell06}.} Rather, it was the overall character of the taking that could be used to argue that it was illegitimate. In this picture, moreover, the perceived lack of a clearly identifiable and direct public benefit becomes only one of several factors.

In addition, the institutional, social and political aspects of the case come into focus. The economic implications are less important to Justice O'Connor. Even the importance of homeownership to personhood does not receive the same attention as structural aspects. The problem which overshadows everything else is the concern that economic development takings represent a form of governmental interference in property that might come to systematically favor the rich and powerful to the detriment of the less resourceful. Hence, such takings may help establish and sustain patterns of inequality. Hardly anyone would openly regard this as desirable; it is not hard to agree that if Justice O'Connor's predictions about the fallout of {\it Kelo} are correct, then this is indeed be ``perverse''. 

The question, of course, is whether her predictions are warranted. This is a call for empirical and contextual assessment of economic development takings, to help us gain a better understanding of how they actual affect political, social and bureaucratic processes. In addition, it raises the question of how to {\it avoid} negative effects, that is, how to design rules and procedures that can reduce the democratic deficit of economic development takings. As I now move away from theory towards concrete assessment of economic development takings, both these questions will be in focus.

\section{Conclusion}

In this chapter, I have presented the core notion of my thesis, that of an economic development taking. I started by noting that while the notion is straightforward enough to define factually, it is far from obvious what implications it has for legal reasoning. I illustrated the subtleties involved by considering a concrete example of a commercial scheme that looked like it might well result in compulsory acquisition of land, namely Donald Trump's controversial plans to develop a golf course on a site of special scientific interest close to Aberdeen, Scotland. In the end, the plans did {\it not} require takings, as Trump was able to make creative use of property rights he acquired voluntarily, against the complaints of recalcitrant neighbors.

This turn of events made the example even more relevant to the points I have been trying to make in this chapter. It served to highlight, in particular, that the question studied in this thesis is not a black-and-white issue that sees privileged property rights enthusiasts on one side of the equation balanced against the good will of the regulatory state on the other. Rather, the example of Trump's golf course allowed me to emphasize the importance of context when assessing both the nature of property rights and the meaning of protecting them. In particular, to protect the property rights of those opposing Trump's golf course was not about protecting just any property, it was about protecting the property of members in a local community that felt it would be detrimental to this community, and to their lives, if Trump was allowed to redefine it. In particular, after Trump decided not to pursue compulsory purchase, protecting the property of these members of the community became a question of {\it restricting} the degree of dominion that Trump could exercise over his own property. Hence, under a conventional and overly simplistic way of looking at these matters, protecting property then became tantamount to restricting its use, a seeming paradox.

To resolve this paradox, and to arrive at a better conceptual understanding of economic development takings, I looked to various theories of property. I noted that there are differences between civil law and common law theorizing about property, but I concluded that these differences are not particularly relevant to the questions studied in this thesis. In particular, I observed that neither the bundle theory, dominant in the common law world, nor the dominion theory, used by civil law jurists, helped me clarify economic development takings as a category of legal thought.

I then went on to consider more sophisticated accounts of property, noting that a range of different {\it normative} theories have been proposed. These differ with respect to the values that they think the institution of property should promote, and as such they were also relevant to the question of assessing economic development takings. However, they do not allow us to zoom in on such takings in a more value-neutral way, to argue that regardless of one's normative persuasions, one should acknowledge that they deserve special attention.

I argued that in order to make this point successfully, the traditional entitlements-based perspective on property had to be abandoned. Instead, I looked to the social function theory of property, which encourages us to take a more contextual perspective on rights and obligations inherent in property. In particular, I noted that the social function theory compels us to recognize the importance of property in regulating social and political relations. Hence, economic development takings are special because they redefine the meaning of the property that is taken and cause a lasting disturbance to the established economic, social and political relationships that exist between owners, communities, state bodies, and commercial actors. The social function theory asks us to acknowledge that property rules are hardly ever neutral with regards to such effects. I identified this as the key observation that allowed me to make sense of economic development takings as category of legal reasoning.

After concluding that the social function theory allowed me to formulate a coherent conceptual basis for studying such takings, I went on to argue that in the first instance, the theory should be understood as giving us purely {\it descriptive} insights into the workings of property and its role in the legal order. In this, I advanced a different stance than many property scholars, by arguing that it would be better to decouple the more normative aspects of the theory, to allow the social function theory to serve as a common ground for further value-based debate.

I then went on to clarify my own starting point for engaging in such debate, by expressing support for the human flourishing theory proposed by Alexander and Pe\~{n}alver. I noted that this theory focuses on how property enables communities and individuals to  participate in social and political processes. I argued that protecting this function of property was good, and that this value should be considered fundamental in property law. Moreover, I noted that the human flourishing theory also contains a further important insight, concerning the scope of the state's power to protect. In particular, the theory asks us to recognize that protecting property against interference that is harmful to human flourishing is a responsibility that the state has even in cases when the individual owners themselves neglect to defend their property, for instance because of financial incentives to remain idle. In other words, some functions of property are such that owners have an obligation to preserve them, while the state has a duty to protect them, potentially even against the will of the owners.

After this, I went on to provide some introductory remarks on economic development takings, drawing on the theoretical insights collected from preceding sections. To make the discussion concrete, I considered the case of {\it Kelo}, which propelled the notion of an economic development taking to the front of the takings debate in the US. I focused particularly on the dissenting opinion of Justice O'Connor, and I argued that she approached the issue in a way that is consistent with the theoretical basis proposed in this chapter.

I will now go on to make my analysis of economic development takings more concrete, by considering how such takings are dealt with in Europe and the US respectively. I note that the category has yet to receive much attention in Europe, so the discussion focuses on the US. Here, the attention this issues has received after {\it Kelo} has been staggering. To get a broader basis upon which to asses all the various arguments that have been presented, I consider the historical background to the issue as it is discussed in the US. This involves giving a detailed presentation of the public use restriction, as it was developed in case law from the states during in the 19th and early 20th century. I then connect this discussion with recent proposals to deal with economic development takings, responding to the backlash of {\it Kelo}, by aiming to address the democratic deficit of such takings.

Later, when I begin to consider the law relating to Norwegian hydropower, I will look back at the theoretical basis provided in the present chapter to guide the analysis. In particular, I focus on certain decision-making mechanisms that have developed on the ground in Norway, as a practical response to the increased tendency for local owners to engage in hydropower development. I will argue that this shows the conceptual strength of the idea that property is irreducibly embedded in community, and that its meaning and function is not -- and should not -- be ordained from above, but should be allowed to arise from its grassroots through continuously evolving institutions of participatory democracy.

%If property rights, particularly rights to land, are distributed fairly in a local community, property is not a privilege. Even if most people do not hold land rights, as long as no one holds excessive amounts, there is no reason why owners and non-owners should not be on equal footing in the local community. They are mutually dependent on one another; non-owners need access to natural resources, while owners need access to services. Moreover, the bonds of community will tend to ensure that owners are deterred from engaging in exploitative practices towards non-owners in much the same way as non-owners are deterred from undermining property rights. 
\chapter{Economic development takings and the public purpose: Tensions and theory}

\section{Introduction}

In recent decades the use of expropriation for economic development has become increasingly widespread in many jurisdictions, also in cases when those who directly benefit are commercial companies rather than public bodies. The justification usually offered for such takings is that they indirectly benefit society and the greater community, through increased tax revenues, new jobs, and various other economic and social ripple effects. Economic takings often prove controversial, however, and have also provoked much academic debate, particularly in the US.\footnote{References}

Tensions reached new heights following the Supreme Court case of {\it Kelo v City of New London}.\footcite{kelo05}  The company Pfizer was allowed to expropriate homes for the construction of new research facilities, and the questions arose as to whether or not this constituted "public use" in the sense of the property clause in the Fifth Amendment to the US Constitution.  The majority 5-4 found that the expropriation was constitutional, but the decision was controversial. Arguably, the attitude following {\it Kelo} has shifted towards a greater feeling of unease regarding economic takings, both among legal scholars and members of the general public.

The case has also had a great impact on academic writing on takings law in the US, where economic development cases are now usually viewed as a distinct sub-class of takings which merit particular attention. Moreover, a consensus seems to be emerging that there is a need for novel approaches to deal with such takings, possibly even new legal frameworks to resolve the tensions that typically arise. Some authors argue for a simple solution: an outright ban on economic development takings. However, the majority of scholars take a more measured approach, recognizing the need for legal frameworks that can be used to promote development projects without making illegitimate use of compulsion against land owners.

In Europe, there has not yet been been any comparable shift in academic outlook, but perceived expropriation-for-profit situations are increasingly coming into critical focus here as well. In a global context, this seems to form part of a general crisis of confidence regarding protection of private property, related to how  egalitarian systems of property ownership are coming under increasing pressure from large-scale commercial actors who assume control over an increasing share of the world's land rights. The phenomenon known as {\it land grabbing} is widely discusses and is apparently becoming endemic across the third world.

 One approach to property protection that is becoming increasingly relevant on the global stage is to see it as a {\it human rights} issue, offering basic protection to individuals independently of the particular jurisdiction they find themselves in. This perspective is already practically important in Europe, due to the European Convention of Human Rights (ECHR) and the court in Strasbourg (ECtHR). It is also promoted by many academics, international agencies and NGOs as the future of land rights protection on the global stage, as large-scale cross-national, and even international, transactions of such rights are becoming increasingly common. So far, the special category of for-profit takings has not been singled out for special consideration. Rather, the focus tends to be on {\it fairness} and {\it proportionality} as broader benchmarks that needs to be upheld. But, of course, how to achieve this depends on the circumstances, and cases when takings appear to be largely motivated by the commercial interests of private parties must be expected to be particularly problematic. Hence, it seems reasonable to single out economic takings as a special category also in relation to human rights law. This, however, appears to be a largely novel perspective; there is little or  no trace of such categorization in the literature so far, possibly because the very idea of a property rights regime entrenched in human rights law has been seen as controversial. In Europe, however, is is now the reality, with the ECtHR adopting a fairly strict standard in assessing cases of property interference from the signatory states.

% voice and some argue that we are currently witnessing a global pooling of resources and wealth that is unprecedented in modern times.\footnote{references}. In the 21st Century, these processes are driven forward by political expediency just as much as capital, orchestrated by public/private partnerships that sees modern regulatory frameworks employed to benefit those actors on the financial stage that are in the best position to wield the power of compulsion over land and people. This power is on offer through a range of  institutions that were developed following the industrial revolution, giving governments and international institutions and unprecedented level of control over the basic building blocks of the economic system. Increasingly, we are coming to realize that this power will not necessarily be used for good, but is characterized rather by the fact that it will be used extensively, to further the interests of whoever gains control of it. In particular, a range of the institutions of modern government appears to be suffering from an increasing {\it democratic deficit}, due to representative forms of democratic control having proven unable to legitimize the activities of these institutions.

In the following, I will present European human rights law and the takings debate in the US. I will highlight those aspects and issues that my subsequent case study aims to shed light on. For Europe, this means that I will present the basic proportionality test that is now at the core of property adjudication at the ECtHR. I will pay particular attention to the fact that the ECtHR have gradually widened the scope of the property clause in the ECHR. The court is now, in their own words, offering ``stronger protection'' of property rights. I discuss this development, arguing that this is indicative of a general crisis of confidence in Europe, whereby interference in property is increasingly seen as illegitimate. I further suggest that this might be due to a growing feeling of discontent with the rationale behind many instances of interference under modern regulatory regimes. The traditional ``public purpose''-scenario, where private rights must give way to the public will, is becoming increasingly hard to spot under planning regimes that involve extensive public/private partnerships and significant commercial interests.

For the US, I will focus specifically on economic takings, both the historical development that has led to this being seen as a a separate category and the debate following {\it Kelo}. Much of US literature on economic takings must be read in light of the tense political climate surrounding takings in the US. However, I believe that some of it is highly relevant to the international debate and to European human rights law. The post-{\it Kelo} literature in the US is quite different from that which dominated the scene previously and the decision marks a development towards increased critical scrutiny of economic development takings, not a further relaxation of constitutional safeguards. Hence, the development in the US might be an indication of what is to come also in Europe, if concerns about legitimacy of economic takings are not taken seriously. In any event, I believe it suggests strongly that operating with a special category of economic takings is helpful, at least as an abstraction, also in the European context.

%I also highlight what I believe to be a connection between the situation in the US leading up to {\it Kelo} and the present situation in Europe, illustrated by the fact that the European Court of Human Rights is now explicitly endorsing ``stronger protection'' of property rights.  I attempt to identify the reasons behind calls for a stricter approach, arguing that it is connected to the fact that interferences in property under modern regulatory regimes is sanctioned in wide a range of different circumstances, serving to undermine their status as a necessary burden imposed on owner's according to the will of the greater public. In some cases, rather, takings appear to both owners and the public as improperly motivated and socially and politically unfair. I note that this happens particularly often in economic development cases, when commercial actors benefit to the detriment of local communities. I go on to list some concrete issues that arise with respect to such takings and that have been flagged as problematic in the literature.
%
%Following up on this, I consider various proposals that have been made to resolve tensions and limit the possibility of abuse in economic development cases. The differences of opinion that have been expressed in this regard have been quite substantial, and proposals have ranged from suggesting an outright ban on economic development takings  (Somin 2007; Cohen 2006) to suggesting that the best way forward is to reassess principles for awarding compensation in such cases (Householder 2007; Lehavi and Licht 2007).

%Much of the current theory focus on assessing traditional judicial safeguards that courts can rely on to prevent abuses, pertaining primarily to the material assessment of proportionality, public purpose, and compensation. 

In the last part of the chapter I will focus on a very interesting strand of recent work in the US, which shifts attention away from the public use test towards procedural safe-guards. The core idea is that the manner in which eminent domain decisions are typically made, and the way in which owners are compensated, might be unsuitable for economic development cases. Importantly, the need for special procedures has been noted, to restore legitimacy for these case types. This ties the US debate yet closer to the European context, where proportionality, not public use, has become the key notion in property protection. Also, it allows us to be very clear about a special concern that arises for economic takings cases: under current regulatory regimes, the government and the developer together often dominate the decision-making process completely, leaving the property owners marginalized. Hence, there is often a {\it democratic deficit} in such cases, resulting in discontent and a feeling that the taking is not in the public interest at all. Importantly, some recent writers hypothesize that if the proper balance can be restored in the decision-making process, so will the decision reached appear more legitimate, also with respect to the public use clause. In my opinion, this idea is crucial, and together with the question of compensation, which raises a similar structural problem, it will guide the rest of the work done in this thesis. 

Most importantly, the work done in this chapter will bring into focus the following key question: What principles can be used to ensure meaningful participation and just compensation in economic takings cases, without hindering socially and economically desirable development projects?

%This question sets the stage for the remainder of my thesis, where I conduct a case study of expropriation for the development of hydro-power in Norway. In particular, I will consider two special semi-judicial procedural systems used in such cases in Norway, one targeting compensation following expropriation, and another used as an alternative to expropriation, particularly in cases when development requires cooperation among many owners.

%I conclude by arguing that approaches along procedural lines represent the best way forward in relation to addressing issues associated with economic development takings. This raises the following problem, however: what procedural principles can be used to ensure meaningful participation, without hindering socially and economically desirable development projects? This question sets the stage for the remainder of my thesis, where I conduct a case study of expropriation for the development of hydro-power in Norway. In particular, I will consider two special semi-judicial procedural systems used in such cases in Norway, one targeting compensation following expropriation, and another used as an alternative to expropriation, particularly in cases when development requires cooperation among many owners.

\section{The fair balance: Property protection under the ECHR}

The starting point for property adjudication at the ECtHR is that States have a "wide margin of appreciation" with regards to the question of whether or not an interference in property rights is to be considered legitimate in pursuance of the public interest.\footcite[See][54]{james86} This question is regarded as depending on public policy and concrete circumstances to such an extent that it is rarely appropriate for the Court to censor the assessments made by member States. At the same time, however, the Court has gradually showed an increased tendency towards actively assessing whether or not particular instances of interference are ``proportional'' and able to strike a ``fair balance'' between the interests of the public and the interests of the individual property owner.\footnote{See \cite[69]{sporrong82} and \cite[120]{james86}.}  As argued by Professor Allen, this has caused TP1-1 to attain a much wider scope than what was originally intended by the signatories.\footcite[1055]{allen10}.

In the case law behind this development the focus has predominantly been on the issue of compensation, with the Court gradually developing the principle that while TP1-1 does not entitle owners to full compensation in all cases of interference, the fair balance will likely be upset unless at least some compensation is paid, based on the market value of the property in question.\footnote{See \cite[103]{scordino06}. The case also illustrates that the Court has come to adopt a fairly strict approach to the question of when it is legitimate to award less than full market value.} The focus on compensation has also been reflected in academic work on TP1-1, which tends to address proportionality from a financial perspective, focusing on the extent to which owners are entitled to compensation based on the market value of their property. Indeed, when considering the best known case law and literature on the subject, one is left with the impression that "fair balance" with regards to TP1-1 is crucially linked to financial entitlements, primarily used as a standard that can justify a right to compensation that goes beyond what the wording of TP1-1 might initially suggest.

In recent case law, however, it has become clear that the fair balance test is meant to be more than a yardstick for assessing whether adequate compensation has been paid to owners affected by an interference. In {\it Chassagnou and others v France} the situation was that property owners were compelled to permit hunting on their land, following compulsory membership in a hunting association which was set up to manage hunting in the local area.\footcite{chassagnou99} They protested this on the grounds that they were ethically opposed to hunting, and the Court agreed that there had been a breach of TP1-1.  In the later case of {\it Hermann v Germany} the circumstances were similar, and the Court followed the precedent set in {\it Chassagnou}, commenting also that they had ``misgivings of principle'' about the argument that financial compensation could provide adequate protection in such a case.\footcite[See][91]{hermann12}  In this way, the hunting cases illustrate that the right to property is not a mere financial entitlement, and that the fair balance that must be struck could pertain to other aspects, such as the owner's right to make use of his property in accordance with his convictions and to take part in decision-making processes regarding how it should be managed.\footnote{That the assessment should be concrete and contextual was made clear in \cite{chabauty12}. In this case, the Court found no violation of TP1-1 although the facts seemed close to those of {\it Chassagnou}. The case differed, however, in that the owner himself was not opposed to hunting, but wanted to withdraw his land from the hunters' association to enjoy exclusive hunting rights.}

In a different but related development, the Court has also adopted a distinctly broad view in recent cases involving rent control schemes and housing regulation. While there are obvious financial interests at stake, both for landlords and tenants, the Court has not shunned away from using concrete cases as a starting point for providing an assessment of the fairness of national law more generally. In {\it Hutten-Czapska v Poland} the Court concluded that the facts of the case demonstrated ``the existence of an underlying systemic problem, which is connected with a serious shortcoming in the domestic legal order'', and they called for general measures to be put in place to remedy the situation.\footcite[191]{hutten06}

Interestingly, the Court relied on a highly contextual understanding of the fair balance test to reach this result, looking to the practical consequences of current legislation and administrative practices, as evidenced by the circumstance of the case. The Court reasons as follows regarding their understanding of the fair balance test:

\begin{quote}
In assessing compliance with Article 1 of Protocol No. 1, the Court must make an overall examination of the various interests in issue, bearing in mind that the Convention is 	intended to safeguard rights that are ``practical and effective''. It must look behind 	appearances and investigate the realities of the situation complained of. [...] Uncertainty -- be it legislative, administrative or arising from practices applied by the authorities -- is a factor to be taken into account in assessing the State’s conduct.
\end{quote}\footcite[151]{hutten06}

This passage was subsequently quoted in the recent case of {\it Lindheim and others v Norway}, where the applicants complained that they were in need of protection from a recent Norwegian Act of Parliament that gave lessees the right to demand indefinite extensions of ground leases on pre-existing conditions.\footcite[119]{lindheim12}  In the end, the Court concluded that TP1-1 had been violated and that the Act itself was the underlying source of the violation. Consequently, the Court ordered that general measures had to be taken by the Norwegian State to remedy the situation. The Court considered counterarguments based on earlier case law, but commented that their decision should be regarded in light of ``jurisprudential developments in the direction of a stronger protection under Article 1 of Protocol No. 1''.\footcite[135]{lindheim12}

While the problems associated with economic development takings have not, as far as I am aware, been considered by the ECtHR, it seems that the recent developments in the direction of a more contextual and strict approach to the fair balance principle, is worth noting. It seems to suggest that a crisis of confidence in the legitimacy of expropriation might be under way in Europe, similar to that seen in the US.

\section{The US perspective on economic takings}

In this Section I map the main problems that have been discussed in relation to cases of economic development takings in the US. I note how conomic development cases attract much more attention now than 20 years ago, and I present an historical overview aiming to give the reader an idea of how the debate regarding economic takings have developed into its present state. I argue that the recent surge of academic interest in this topic reflects a recent crisis of confidence in the legitimacy of takings in the US, specifically related to cases when the rationale behind the use of compulsion is commercial in nature. Following up on this, I go on to analyse in more detail some recent approaches used to address economic development takings. I argue that recent work from the US provide a useful conceptual framework for addressing such takings as a special category, and that it also serves to illustrate that such takings should indeed be considered a separate issue. %in particular, that the worry in these cases is that there is an imbalance of interest and power that may lead to both real and perceived abuses.

I will pay particularly attention to how constitutional objections to economic development takings, based on strict interpretations of the fifth amendment, appear less important to the debate than it may appear at first sight. It is true that many commentators, including some scholars, have argued for a strict understanding of property protection and they express their concerns more forcefully that what is commonly seen in Europe. However, many scholars in the US also rely on a broad and contextual understanding of property rights, and their work, which has perhaps not received the same level of attention, do in fact closely resemble the approach to property protection adopted by the ECtHR. I make special note of the fact that a great number of voices approach the issue as a question that crucially involves notions of fairness and democratic legitimacy.  I argue that such arguments are often socio-legal in nature, emerging from empirical considerations. Hence, they appear highly relevant also outside the context of the US legal system and the political tradition that accompanies it.

Many academics in the US argue that economic takings are particularly problematic under current practices and that there is need for reform. Moreover, the debate after {\it Kelo} shows that concern about such takings has become more widespread, also among academics that do not necessarily endorse a liberal, individualistic, view on the nature of property. In the last part of this section, I examine a few recent responses that aim to provide a bride across the ideological divide that otherwise dominates the takings debate in the US. I will refer to these as {\it institutional} approaches. They are characterized by a focus on the administrative and judicial procedures that are used in economic takings cases. The idea is that legitimacy of such takings can only be achieved if special procedures are followed, to ensure fairness both in the decision-making process and with regards to the issue of what compensation should be paid following condemnation. Importantly, the need for special procedures is identified with the imbalance of power that tend to exist between developers and property owners in such cases. Hence, the proposals do not argue in favor of limiting the use of eminent domain generally, and should not be read as rhetoric in favor of a more absolutist or liberal view of property rights. As such these proposal become more broadly relevant, also outside the US context. It is also a perspective that directly ties the US debate to the case study presented in subsequent chapters of this thesis. In particular, in Section \ref{sec:ins} I give an in depth presentation of two recent reform proposals, one regarding compensation principles and the other focusing on the decision-making process as such. Both contain (aspects of) the two working institutions I will consider in my case study, in Chapter \ref{chap:4} and Chapter \ref{chap:5} respectively.

The first is a proposal by Professors Lehavi and Licht, which argues for the introduction of special-purpose development corporations to ensure appropriate levels of compensation in economic development cases.\footcite{lehavi07} The crux of their proposal is to award property owners shares in a development company that is set up to bargain with potential developers. Importantly, the eminent domain decision precedes this round of bargaining, so the owners cannot threaten to refuse selling the land in order to get additional compensation based on the public need for development. Bargaining is restricted to the commercial element of the project, reflected in competition that may arise among developers interested in carrying out the project. This proposal will serve as a point of departure when we consider the Norwegian appraisal courts and recent case law on compensation of waterfalls.

The second proposal we will consider in depth is due to Professors Heller and Hills, who argue for the introduction of so-called ``land assembly districts'', institutions for participatory pooling of property for development projects.\footcite{heller08} Land assembly  districts are designed to partially replace the use of eminent domain for economic development, giving property owners both a template on which to bargain with developers and also the final say on whether land assembly should happen at all. The authors argue that with appropriate procedures in place for making collective decisions, such a system will be sufficient to avoid holdouts preventing socially desirable projects. In fact, they argue quite convincingly that even a simple scheme of majority voting will be sufficient in many cases, due to the commercial incentives that are present when land assembly is needed for economic development. Importantly, their proposal is not mean to apply to land assembly in general. It should only be used when reasonable market conditions can be achieved without the use of eminent domain. As we will see in Chapter \ref{sec:5}, this proposal corresponds closely to important aspects of various land consolidation procedures currently in operation under Norwegian law. In particular, I will show how recent case law on land consolidation for hydroelectric development in Norway demonstrates the efficiency and usefulness of a similar procedure, used to assemble water rights under fragmented ownership in Norway.

Before I move on to discuss these recent proposals in more detail, I will give an historical background on economic takings in the US. This will give the reader a better appreciation of the developments that led to the current climate of debate, showing how economic takings became a natural special category in the US. In particular, I will argue that the US system has tended to be more open to the idea of commercial projects benefiting from eminent domain, so that the level of tensions here were naturally higher than those found in Europe. However, recent European trends towards greater levels of public/private commercial partnerships in important public projects may suggest that the US discourse will soon become easier to recognize also in the European setting. The history of the debate in the US shows, in any event, that conflicts over takings law become aggravated whenever the perception takes hold that powerful commercial interests are permitted to usurp the process to their own advantage.

\subsection{The ``dead letter'' raised: A short history of economic takings in the US}

Going back to the time when the Fifth Amendment was introduced, there is not much historical evidence explaining why the takings clause was included in the bill of rights, and little in the way of guidance as to how it was originally understood. James Madison, who drafted it, commented that his proposals for constitutional amendments were intended to be uncontroversial to Congress.\footnote{See letters from Madison to Edmund Randolph dated 15 June 1789 and from Madison to Thomas Jefferson dated 20 June 1789, both included in \cite{madison79}.}  Hence, it is natural to regard it more as a codification of an existing principle, rather than a novel proposal. Indeed, several State constitutions pre-dating the Bill of Rights also included takings clauses, and they all seem to be largely based on a codification of principles from English Common law.\footcite[See][299]{johnson11}

In common law, the right to property has a long tradition behind it, dating back to the Magna Carta. In his {\it Commentaries on English Law}, William Blackstone famously described it as the ``third absolute right'' that was ``inherent in every Englishman''.\footcite[134-135]{blackstone79}.  Moreover, Blackstone expressed a very restrictive view on the possibility of expropriation, arguing that it was only for the legislature to interfere with property rights, warning against the dangers of allowing private individuals, or even public tribunals, to be the judge of whether or not the ``common good'' could justify it. Blackstone went as far as to say that the public good was ``in nothing more invested'' than the protection of private property.\footcite[134-135]{blackstone79}

On this background it is not surprising that Madison regarded the property clause as an uncontroversial amendment.\footnote{Indeed, early American scholars also emphasized the importance of private property. For instance, in his famous {\it Commentaries}, James Kent described the sense of property as ``graciously implanted in the human breast'' and declared that the right of acquisition ``ought to be sacredly protected'', \cite[see][257]{kent27}.} Its importance may in fact have been greater as a legitimizing force, increasing confidence in the regulatory power of the newly established state by setting up clear parameters for the exercise of that power.\footnote{references.}  However, while the principle was regarded as theoretically self-evident, it never seems to have been entirely clear what it meant in practice, particularly for takings of property when it was unclear to what extent it could be said that it was put to ``public use''.\footcite[See][317]{johnson11} 

There are two points that I would like to record about the early common law in the US  in this regard. First, the distinction between public use and public purpose does not appear to have been considered sharp. For instance, in his {\it Commentaries}, James Kent first makes clear that the power of eminent domain is for ``public use, and public use only", but then goes on to qualify this by stating that a taking which served a ``purpose not of a public nature'' would be unconstitutional.\footcite[See][275-276]{kent27}  Moreover, it seems to have been accepted that takings which clearly benefited the public would be acceptable, regardless of whether or not the property was physically put to use by the public.\footnote{References.} The crucial principle encoded in the Fifth Amendment was the right to compensation, and this right was considered fundamental.\footnote{James Kent held it to be  ``founded in natural equity'' and described it as an ``acknowledged principle of universal law'', \cite[see][276]{kent27}.}

An interesting early illustration of how the public use clause was understood can be found in {\it Stowell v Flagg}, a Massachusetts case from 1814, where an act that allowed mill owners to cause damage to adjacent land by flooding was held to satisfy the public use requirement.\footcite{stowell14} The court highlighted the purpose of the interference, commenting that ``these mills, early in the settlement of this country, were of great public necessity and utility''.\footcite[366]{stowell14} At the same time, the court had misgivings about how the act had come to be applied and expressed concern that ``the legislature, as well as the courts of law in this state, seem to have been disposed rather to enlarge, than to curtail, the power of mill owners''.\footcite[366]{stowell14} Still, after noting that  affected land owners were entitled to compensation under the act, the court concluded that the act had to be observed and that it precluded any claims for damages under common law.\footnote{The land owner had claimed that a common law claim for damages could be made, irrespectively of the mill act.} Hence, the case is an early example of judicial deference to the legislature in takings cases. Despite declaring that he could not help thinking that the statute was ``incautiously copied from the ancient colonial and provincial acts'', the presiding judge held in favor of the mill owner,  concluding that ``as the law is, so must we declare it''.\footcite[368]{stowell14}

While judicial deference was recognized as a guiding principle early on in US takings law, it is important to note in this regard that eminent domain was seldom used in a way that would raise serious controversy. English common law, while lacking clearly defined constitutional safeguards, was based on a fundamentally cautious attitude, ensuring that the power would typically only be used as a last resort. As Professor Meidinger notes, the British were never really charged with abuse of eminent domain, and private property tended to be respected, also in the colonies.\footcite[17]{meidinger80} This undoubtedly influenced early US law. Indeed, the importance of property protection was considered fundamental early on, as reflected in {\it de dicta} comments made by judges in the early cases of {\it Calder v Bull} and {\it Vanhorne’s Lessee v Dorrance}.\footnote{\cite[388]{calder98} and \cite[310]{vanhorne95}.} This shows that the constitutional limit of the takings power was clearly recognized and emphasized as a matter of principle. Hence, the relative lack of judicial interest in the question of legitimacy does not appear to have been due to a broad view on the scope of eminent domain. It seems more natural to attribute it to absence of controversy, resulting from an established practice of narrow use, inherited from the English.
\noo{
The Legislature declare and enact, that such are the public exigencies, or necessities of the State, as to authorise them to take the land of A. and give it to B.; the dictates of reason and the eternal principles of justice, as well as the sacred principles of the social contract, and the Constitution, direct, and they accordingly declare and ordain, that A. shall receive compensation for the land. But here the Legislature must stop; they have run the full length of their authority, and can go no further: they cannot constitutionally determine upon the amount of the compensation, or value of the land. Public exigencies do not require, necessity does not demand, that the Legislature should, of themselves, without the participation of the proprietor, or intervention of a jury, assess the value of the thing, or ascertain the amount of the compensation to be paid for it. This can constitutionally be effected only in three ways.
1. By the parties that is, by stipulation between the Legislature and proprietor of the land.
2. By commissioners mutually elected by the parties.
3. By the intervention of a Jury.
}
However, the traditional attitude to eminent domain would eventually give way to a more expansive sentiment. This development became particularly marked during the period of great economic expansion and industrialization in the mid to late 19th century, when eminent domain was increasingly used to benefit (privately operated) railroads, hydroelectric projects, and the mining industry.\footcite[23-33]{meidinger80} During this time, it also became increasingly common for landowners to challenge the legitimacy of takings in court, undoubtedly a consequence of the fact that eminent domain was now used more widely.\footcite[24]{meidinger80} Controversy arose particularly often with respect to mill acts.\footnote{\cite[24]{meidinger80}. See also \cite[306-313]{johnson11} and \cite[251-252]{horwitz73}.} Such acts were found throughout the US, and many of them dated from pre-industrial times when mills were primarily used to serve the needs of self-sufficient agrarian communities.\footnote{A total of 29 states had passed mill acts, with 27 still in force, when a list of such acts was compiled in \cite[17]{head85}. According to Justice Gray, at pages 18--19 in the same, the ``principal objects'' for early mill acts had been grist mills typically serving local agrarian needs at tolls fixed by law, a purpose which was generally accepted to ensure that they were for public use.}  However, following economic and technological advances, acts that were once used to facilitate the construction of grist mills would increasingly also be relied on by developers wishing to harness hydropower for manufacturing, and eventually, for hydroelectric projects.\footnote{See, e.g., \cite[18-21]{head85} and \cite[449-452]{minn06}.}

The mill acts typically contained provisions that enabled the mill developer to condemn property needed for the construction, as well as the right to damage surrounding land by flooding or deprivation of water. Such takings became increasingly controversial, however, and many legitimacy cases came before state courts in the late 19th and early 20th century. In these cases, we find the first clear evidence of how different interpretations of the public use requirement began to develop and diverge. Generally speaking, when a court upheld an interference in private property, it would place decisive weight on the broader purpose of interference, typically by arguing that economic ripple effects ensured that the mill was in the public interest even if the public would not literally make use of it.\footnote{See, e.g., \cite{hazen53,scudder32,boston32}. A more comprehensive list of cases adopting a broad view can be found in \cite[617]{nichols40}.} By contrast, when a court decided that an interference was unconstitutional (with respect to state constitutions), it would tend to focus on the actual use made of the mill, arguing that it did not directly benefit the public in the sense required by the public use restriction.\footnote{See, e.g., \cite{sadler59,ryerson77,gaylord03,minn06}. A more comprehensive list can be found in {\it Public benefit or convenience as distinguished from use by the public as ground for the exercise of the power of eminent domain} 54 ALR 7 (American Law Reports, 1928).} For a time, a doctrine which sought to distinguish sharply between public use and public purpose, striking down the former kind as unconstitutional, played quite a significant role in many states.\footnote{Professor Nichols goes as far as to conclude that it emerged as the ``majority'' opinion on public use, see \footcite[617-618]{nichols40}. But contrast this with \cite{berger78} and \cite[24]{meidinger80}, who argue that the narrow view was only dominant in a handful of states, led by New York.}

\noo{ For instance, in the case of {\it Gaylord v. Sanitary Dist. of Chicago}, the Supreme Court of Illinois held the state Mill Act to be unconstitutional, as it was not limited to traditional flour mills. In doing so, the court observed that public use was ``something more than a mere benefit to the public''.\footcite[524]{gaylord03} Similar sentiments were expressed in other decisions striking down uses of eminent domain for mill construction, for instance in Vermont, Michigan and New York.\footnote{References.}}

It is tempting to associate the narrow view on public use with a more restrictive attitude towards eminent domain generally. Similarly, it is natural to assume that adoption of a broad view suggests a more relaxed attitude. To some extent, the primary sources seems to warrant this; unsurprisingly, those who endorsed a broad view on the public use question also often spoke in favor of judicial deference in legitimacy cases, while those endorsing a narrow view tended to emphasize the importance of constitutional safeguards against abuse of eminent domain. However, it is important to keep in mind that both groups were quite heterogeneous and did not necessarily subscribe to the same reasons for their respective views. Indeed, both the narrow and broad interpretation had many supporters who relied on subtle arguments that defy categorization along any obvious axis of ``property friendliness'' or the like.

It is clear, for instance, that many of the courts which favored a broad interpretation of public use still viewed the constitutional limitation on the takings power as an important safeguard, not only as a guarantee for compensation but also as a restriction on the purpose of takings. Indeed, it seems that most late 19th Century Courts, including those that upheld economic takings, were influenced by the growing body of case law across the US that had struck down such takings as unconstitutional. In particular, it seems that the strict deferential view was largely abandoned in economic takings cases. Deference to the legislature still played an important role, of course, but it became much more common to argue for legitimacy primarily in terms of substantive arguments, by directly addressing the context and circumstances of the taking or act complained of. I believe this is an important insight to record about the case law from this period; despite differences of opinion about the meaning of public use, a consensus appears to have emerged that judicial review of legitimacy was appropriate and important in economic takings cases.

A good example is the case of {\it Dayton Gold \& Silver Mining Co. v. Seawell}, concerning a Nevada Act which stipulated that mining was a public use for which the power of eminent domain could be exercised to acquire additional rights needed to facilitate extraction.\footcite{seawell76} The Supreme Court of Nevada decided that the Act was constitutional and adopted a broad understanding of the property clause in the Nevada constitution.\footnote{Nev Const Art 8 § 1.} Interestingly, it argued for this interpretation partly on the basis that it would provide {\it better} protection for landowners.\noo{Why not? A hotel is used by the public as much as a railroad. The public have the same right, upon payment of a fixed compensation, to seek rest and refreshment at a public inn as they have to travel upon a railroad. 

One purpose is, so far as the legal rights of the citizen are concerned, as public as the other.}

\begin{quote}
If public occupation and enjoyment of the object for which land is to be condemned furnishes the only and true test for the right of eminent domain, then the legislature would certainly have the constitutional authority to condemn the lands of any private citizen for the purpose of building hotels and theaters. [...] Stage coaches and city hacks would also be proper objects for the legislature to make provision for, for these vehicles can, at any time, be used by the public upon paying a stipulated compensation. It is certain that this view, if literally carried out to the utmost extent, would lead to very absurd results, if it did not entirely destroy the security of the private rights of individuals. Now while it may be admitted that hotels, theaters, stage coaches, and city hacks, are a benefit to the public, it does not, by any means, necessarily follow that the right of eminent domain can be exercised in their favor.\footcite[410-411]{seawell76}
\end{quote}

The quote shows that a broad understanding of ``public use'' need not be synonymous with a less cautious attitude to abuse of the takings power. Indeed, while the Court decided to uphold the Act, it did so only after a very careful assessment of both legal arguments and factual circumstances. In particular, the Court considered the importance of mining, concluding that it was the ``greatest of the industrial pursuits'' in the state, and that all other interests were ``subservient'' to it.\footcite[409]{seawell76} Moreover, the Court commented that the benefits of the mining industry was ``distributed as much, and sometimes more, among the laboring classes than with the owners of the mines and mills''.\footnote[409]{seawell76}

This shows that the Court actively engaged with the purpose of the Act, thoughtfully assessing it against the constitution. Importantly, it did not do so in isolation, as a linguistic exercise or by attempting to recreate its ``original intent''. Rather, the court approached the constitutional safeguard by making detailed references to the prevailing social and economic conditions in the state of Nevada. The Court also duly noted the importance of deference to the legislature on matters of policy, but it did so only after it had satisfied itself that the Act could be ``enforced by the courts so as to prevent its being used as an instrument of oppression to any one''.\footcite[412]{seawell76} More generally, the court commented as follows on the public purpose test that had to be performed in takings cases, elucidating on the principles on which it should be founded:

\begin{quote}
 Each case when presented must stand or fall upon its own merits, or want of merits. But the danger of an improper invasion of private rights is not, in my judgment, as great by following the construction we have given to the constitution as by a strict adherence to the principles contended for by respondent.\footcite[398]{seawell76}
\end{quote}

In light of this, {\it Dayton Gold \& Silver Mining Co. v. Seawell} must be regarded as an early example of a {\it contextual} approach to legitimacy, characterized by the willingness of the Court to engage in a fairly detailed analysis of the concrete circumstances and consequences of takings. A formalistic approach based on the phrase ``public use'' was abandoned, but not in favor of general deference. Rather, a more nuanced view was adopted, to respect the idea that the legislature should have the final say on policy while also recognizing that courts play a crucial role in protecting citizens from abuse of the takings power. The case is not unique, but rather exemplifies the type of reasoning that was used in economic takings cases at this time. Interestingly, many common elements exist between courts that upheld and struck down such takings, irrespectively of whether or not they subscribed to a narrow or broad view on the public use test. 

%In fact, upon closer inspection of the primary sources, it becomes clear that the common element goes back all the way to the time %when the two public use doctrine first began to diverge. 

One example is {\it Ryerson v. Brown}, a case often cited as an authority in favor of a narrow view.\footnote{...} Here the Supreme Court of Michigan explicitly qualifies its decision by stating that it is ``not disposed to say that incidental benefit to the public could not under any circumstances justify an exercise of the right of eminent domain''. The case concerned the constitutionality of a mill act, and while the court argues that public use should be taken to mean ``use in fact'', it is clear that ``use'' is understood rather loosely, not literally as physical use of the property that is taken.\footnote{The court explains its stance on the public use restriction by stating (emphasis added) ``it would be essential that the statute should require the use to be public in fact; in other words, that it should contain provisions entitling the public to {\it accommodations}.'' The court continues with an illustrative example: ``A flouring mill in this state may grind exclusively the wheat of Wisconsin, and sell the product exclusively in Europe; and it is manifest that in such a case the proprietor can have no valid claim to the interposition of the law to compel his neighbor to sell a business site to him, any more than could the manufacturer of shoes or the retailer of groceries. Indeed the two last named would have far higher claims, for they would subserve actual needs, while the former would at most only incidentally benefit the locality by furnishing employment and adding to the local trade''. See \cite[336]{ryerson77}.} Moreover, when clarifying its starting point for judicial scrutiny of mill acts, the court explains that ``in considering whether any public policy is to be subserved by such statutes, it is important to consider the subject from the standpoint of each of the parties''. Following up on this with regards to the act in question, the court finds that `` the power to make compulsory appropriation, if admitted, might be exercised under circumstances when the general voice of the people immediately concerned would condemn it''. After considering this and other possible consequences of mill development under the act, the court eventually declares it to be unconstitutional, summing up its assessment as follows: ``What seems conclusive to our minds is the fact that the questions involved are questions not of necessity, but of profit and relative convenience''.

Hence, far from nitpicking on the basis of the public use phrase, the court adopts a contextual approach to takings that is in fact rather similar to the approach of {\it Dayton Gold \& Silver Mining Co. v. Seawell}. The outcome it different, but it is also based on a perceived difference in the context and consequences of the takings complained of. Importantly, it does not rest on any {\it a priori} assumption that economic takings of the kind in question could not possibly meet a public use test. It is somewhat curious that later commentators have focused so exclusively on the case for its comments on public use rather than its broad, albeit perhaps conservative, assessment of legitimacy. However, the case is not unique. It seems that many of the cases from the late 19th Century, on both sides of the public use debate, shares many of its features.\footnote{See, e.g., \cite{scudder32} (Eminent domain power upheld, but said: ``The great principle remains that there must be a public use or benefit. That is indispensable. But what that shall consist of, or how extensive it shall be to authorize an appropriation of private property, is not easily reducible to a general rule. What may be considered a public use may depend somewhat on the situation and wants of the community for the time being.''), \cite{fallsburg03} (Eminent domain struck down, on holding that ``the private benefit too clearly dominates the public interest to find constitutional authority for the exercise of the power of eminent domain''), \cite[538]{board91} (Eminent domain struck down, qualified by ``not only must the purpose be one in which the public has an interest, but the state must have a voice in the manner in which the public may avail itself of that use'').}

In my opinion, this points to an interesting alternative perspective on legitimacy adjudication from this time. Later commentators tend to describe the case law as chaotic, with competing conceptions of constitutional limits competing for dominance.\footcite{nichols40,berger78,meidinger80}. To some extent I agree, but I also find evidence that there was in fact a broad consensus in this period regarding the need for special judicial scrutiny of economic development cases. State courts widely engaged in contextual assessment of legitimacy, and they were conscious of the special challenges that arose in a time when eminent domain was being used extensively to benefit commercial actors as instruments of massive economic expansion. Differences of opinion about public use terminology was an important aspect of this, but it was rarely considered in isolation from other aspects. On a deeper lever, the fact that the public use debate was regarded as important in the first place clearly suggests that deference to the legislature was not held to be an exhaustive answer to the question of legitimacy of private-to-private takings. This, in my opinion, is an important observation which appears to have been somewhat overlooked in the literature. 

It is also relevant when considering later developments in the case law, particularly after the federal courts began to develop their own takings doctrine. While the narrow view of public use was indeed losing ground at the beginning of the 20th Century, the doctrine of extreme deference that was about to be adopted by the Supreme Court appears to represent a largely new development. This new kind of deference was not only directed towards the legislature, but also towards the judiciary at the state level. Hence, it represent a development that is in some sense incomparable to the earlier case law from the states. The balance of power between states and the federal government also played an important role, which should not be overlooked. 

Initially, the Supreme Court held that the takings clause in the US Constitution did not apply to state takings at all.\footcite{barron33} Federal takings, on the other hand, were of limited practical significance since the common practice was that the federal government would rely on the states to condemn property on their behalf.\footcite[30]{meidinger80}. This changed towards the end of the 19th Century, however, particularly following the decision in {\it Trombley v. Humphrey}, where the Supreme Court of Michigan struck down a taking that was going to benefit the federal government.\cite{trombley71} Not long after, in 1875, the first Supreme Court adjudication of a federal taking case occurred, marking the start of the development of the Supreme Court's own doctrine on public use and legitimacy.\footcite{kohl75} Eventually, in 1897, the Court would also hold that state takings could be scrutinized under the takings clause of the constitution.\footcite{chicago97} This was a development that can be traced to the passage of the Fourteenth Amendment to the Constitution after the civil war, concerning due process.\footcite{johnson11}. Indeed, some early Supreme Court cases dealing with state takings were adjudicated against this provision rather than the takings clause.\footnote{See, e.g., \cite{head85}.}

After the Supreme Court started developing its own case law on the legitimacy issue, the deferential stance soon became entrenched. Initially, it seems that deference was directed just as much at the state courts, however, as towards the legislature. Even so, the Supreme Court showed a distinct unwillingness to strike down takings on constitutional grounds, and this probably influenced the further development of state law as well. As argued by Professor Horwitz, the mid to late 19th Century was the period in US history when control over property was transferred on a massive scale from agrarian communities to various agents of industrial expansion.\footcite{horwitz73} Moreover, it was a period of great optimism about the ability of {\it laissez faire} capitalism to ensure progress and economic growth, and this was also reflected in the case law on eminent domain, particularly as developed by the Supreme Court.

A particularly clear expression of this can be found in {\it Mt. Vernon-Woodberry Cotton Duck Co v Alabama Interstate Power Co}.\footcite{vernon16}  This case dealt with the legitimacy of a condemnation arising from the construction of a hydropower plant, which the Alabama Supreme Court had upheld against claims that it was unconstitutional under the constitution of Alabama. The presiding judge held that it was valid using quite brisk language:

\begin{quote}The principal argument presented that is open here, is that the purpose of the condemnation is not a public one. The purpose of the Power Company's incorporation, and that for which it seeks to condemn property of the plaintiff in error, is to manufacture, supply, and sell to the public, power produced by water as a motive force. In the organic relations of modern society it may sometimes be hard to draw the line that is supposed to limit the authority of the legislature to exercise or delegate the power of eminent domain. But to gather the streams from waste and to draw from them energy, labor without brains, and so to save mankind from toil that it can be spared, is to supply what, next to intellect, is the very foundation of all our achievements and all our welfare. If that purpose is not public, we should be at a loss to say what is. The inadequacy of use by the general public as a universal test is established. The respect due to the judgment of the state would have great weight if there were a doubt. But there is none.\footcite[]{vernon16}
\end{quote}

The quote serves as an indication of how deference was fast gaining ground, without yet being established doctrine. On the one hand, the Court stresses that deference to the {\it state} judgment (rather than the judgment of the legislature) should be given great weight in legitimacy cases. On the other hand, it prefers to conclude on the basis of its own assessment of the purpose of the taking. This assessment, however, is not particularly grounded in the circumstances on the ground in Alabama, being based rather on sweeping assertions about the ``organic relations of modern society'' and the desire to ``save mankind from toil that it can be spared''. 

This judgment, from 1916, was given during the so-called {\it Lochner} era of jurisprudence in the US, when the Supreme Court  would famously engage in active censorship of regulation that was meant to promote greater social and economic equality.\footcite{cohen08} In particular, much case law from this period witnesses to a general lack of deference. Hence, it is not unexpected to find that public use cases decided on the basis of substantive arguments. However, it is rather more surprising to find that deference actually played an increasingly important role in takings cases. As early as { \it United States v. Gettysburg Electric Railway Co.}, in 1896, deference was described as a fundamental guiding principle, which should be adhered to except in very special circumstances.\footcite{gettysburg96} In particular, Justice Peckham lended his support to the following deferential stance on the public use test:

\begin{quote}
It is stated in the second volume of Judge Dillon's work on Municipal Corporations (4th Ed. § 600) that, when the legislature has declared the use or purpose to be a public one, its judgment will be respected by the courts, unless the use be palpably without reasonable foundation. Many authorities are cited in the note, and, indeed, the rule commends itself as a rational and proper one.\footcite[680]{gettysburg96}
\end{quote}

The case did not turn on the public use issue, however, as the condemned land would be used for battlefield memorials at Gettysburg, Pennsylvania, clearly a public use. Moreover, in later cases the point of view espoused was not universally adopted. As late as in 1930, the Supreme Court commented that the ``‘It is well established that, in considering the application of the Fourteenth Amendment to cases of expropriation of private property, the question what is a public use is a judicial one".\footcite[447]{vester30} 
In this judgment, Chief Justice Hughes also describes in more depth how the judicial assessment of the public use question should be carried out, echoing the contextual approach that had been developed in case law from the states.

\begin{quote}
In deciding such a question, the Court has appropriate regard to the diversity of local conditions and considers with great respect legislative declarations and in particular the judgments of state courts as to the uses considered to be public in the light of local exigencies. But the question remains a judicial one which this Court must decide in performing its duty of enforcing the provisions of the Federal Constitution.\footcite[447]{vester30}
\end{quote}

In {\it Hairston v. Danville \& W. R. Co.} this was expressed even more clearly, by Justice Moody, who surveyed the state case law and declared that ``The one and only principle in which all courts seem to agree is that the nature of the uses, whether public or private, is ultimately a judicial question.''\footnote[606]{hairston08} He continued by describing in more depth the typical approach of the state courts in determining public use cases:

\begin{quote}
The determination of this question by the courts has been influenced in the different states by considerations touching the resources, the capacity of the soil, the relative importance of industries to the general public welfare, and the long-established methods and habits of the people. In all these respects conditions vary so much in the states and territories of the Union that different results might well be expected.
\end{quote}

Justice Moody goes on to give a long list of cases illustrating this aspect of state case law, showing how assessments of the public use issue is invariably contextual and varies from state to state.\footcite[607]{hairston08} He then cites {\it Falbrook, Clark} and {\it Strickley}, all of which express similar sentiments of support for state case law. Following up on this, he points out that ``no case is recalled'' in which the Supreme Court overturned ``a taking upheld by the state {\it court} as a taking for public uses in conformity with its laws'' (my emphasis). After making clear that situations might still arise where the Supreme Court would not follow state courts on the public issue, Justice Moody goes on to conclude that the cases cited `` show how greatly we have deferred to the opinions of the state courts on this subject, which so closely concerns the welfare of their people''. 

I believe {\it Hairston} is an important case for two reasons. First, it makes clear that initially, the deferential stance in cases dealing with state takings was largely directed at the state courts rather than the state legislature. Second, it demonstrates federal recognition of the fact that a consensus had emerged in state case law, whereby public use scrutiny was consistently regarded as a judicial task.\footnote{Indeed, it provides the authority for {\it Cincinatti} and predates it.} Moreover, the Court clearly looked favorably on the contextual approach typically adopted in such cases, whereby state courts would look to the concrete circumstances of the individual takings and acts complained of. The Court's approval of this tradition, in particular, is explicitly given as the reason for adopting a deferential stance. Put simply, the judicial test provided at state level was held to be of such high quality that there was little use for further federal scrutiny. Deference, in particular, was made contingent on the fact that state courts would provide the required judicial scrutiny of the public use requirement.

Despite this, {\it Hairston} would later be cited as an early authority in favor of a more general deferential stance in {\it U. S. ex rel. Tenn. Valley Authority v. Welch}.\footcite[552]{welch46} This case concerned a federal taking and it cited {\it U.S. v. Gettysburg Electric R. Co.} as an authority in favor of deference with regards to the public use limitation.\footcite{gettysburg96}  But the Court also noted that {\it City of Cincinnati v. Vester} declared that the public use test was a judicial responsibility.\footcite{vester30} In a very selective citation, the Court then purports to resolve this tension by quoting {\it Hairston} and the observation that the Supreme Court had never overruled the state courts in takings cases. Effectively, the importance of judicial scrutiny is downplayed, although as we saw, the rationale behind {\it Hairston} was that state courts already offered high-quality scrutiny of the purpose in takings cases.

The case is important because it is used as an authority in the later case of {\it Berman v. Parker}, which endorses almost complete deference to the legislature regarding the public use issue.\footcite[32]{berman54} This case concerned condemnation for redevelopment of a partly blighted residential area in the District of Colombia. In a key passage the Court states that the role of the judiciary in scrutinizing the public purpose of a taking is ``extremely narrow''.\footcite[32]{berman54} The Court provides only two citations for this claim, one of which is  {\it U. S. ex rel. Tenn. Valley Authority v. Welch}. The other case, {\it Old Dominion Land Co. v. U.S.}, concerned a federal taking of land on which the military had already invested large sums in buildings. Hence, neither of the two authorities seem to support the much more general deferential stance adopted by {\it Berman}.



 towards the {\it legislature}, with no mention of the tradition of careful scrutiny by state courts. 

Still, the case was picked up on in later cases in support of a more general deferential stance. and it eventually became established doctrine.\footcite{welch46}

Hence, while the {\it Lochner} era in general was characterized by courts engaging in censorship of state regulation, this general tendency was not reflected in how eminent domain law developed over the same period. I believe this is quite important to note, since it also reflects the shortcoming of another commonly held view on property protection, namely that it largely serves the interests of property-owning elites, to the detriment of regulatory efforts to promote social equality. The cases through which Lochner era courts developed the deferential stance suggest a different interpretation; those who benefited most directly from takings in these cases were commercial interests, not vulnerable groups of society. Moreover, they benefited from acquiring land rights from members of agrarian communities, not from the elites. Hence allowing such takings to go ahead was no affront to the ideology of progress through laissez faire capitalism, quite the contrary.

In particular, if it is true as many have argued, that the Lochner courts were ideologically committed to the promotion of unrestrained capitalism, there was little reason for them to oppose expansion of eminent domain into the commercial arena: those who would be likely to benefit were market actors who were proposing large scale commercial development projects. Indeed, the case law from this period makes it natural to argue that the deferential stance developed primarily to cater to the needs of the capitalists, under the perceived view that they represented the class which would bring progress and prosperity to the nation as a whole.

In the post-Lochner period, when courts became less willing to engage in activism on behalf of this world view,  little changed in relation to eminent domain. Rather, the deferential stance was entrenched further and made more explicit. In the Supreme Court case of Berman v Parker, an important precedent for Kelo, Justice Douglas codified this attitude when he stated that the room for judicial oversight regarding the public use test was "extremely narrow". 

This was now the dominant view in the US, and it marked the victory not only for the broad interpretation of public use, but also for the general deferential stance on the issue of legitimacy of purpose. This view was dominant also 30 years later when it was upheld in Hawaii Housing Authority v. Midkiff.  However, the fact that the case made it to the Supreme Court is suggestive of an increase in the level of worry and tension associated with eminent domain in the 1980s. Indeed, Justice Sandra Day O'Connor, joined by a unanimous Supreme Court, expressed general disapproval of private takings and she appears to have felt the need to provide further qualification for the deferential view, which she did in part by making the following observation:

[...]judicial deference is required because, in our system of government, legislatures are better able to assess what public purposes should be advanced by an exercise of eminent domain.

Hence judicial deference was not regarded as an absolute and systemic imperative, as in Berman, but made contingent on the fact that legislatures are "better able" than courts at conducting public purpose tests. It should be noted that in the case of Midkiff the purpose of interference was to break up a property oligarchy to the benefit of tenants, not to further economic development by allowing commercial interests to benefit from the takings power. Hence the rationale behind the interference is likely to have struck the Supreme Court as sound and just. Implicitly, Justice O'Connor herself engage in an assessment of its merits when she points out that "regulating oligopoly and the evils associated with it is a classic exercise of a State's police powers".

In conclusion, the "extremely narrow" room for judicial review set up in Berman was replaced by a  slightly more nuanced formulation, which nevertheless made clear that a legal precedent of deference had developed in practice, and that the Supreme Court had no tradition for  adjudicating eminent domain cases on the basis of the public use clause:

where the exercise of eminent domain power is rationally related to a conceivable public purpose, the Court has never held a compensated taking to be proscribed by the Public Use Clause

So while Midkiff might reveal some implied reservations about the deferential stance entrenched in Berman, it clearly reaffirms the main principle at work:  the meaning of public use can be broad, and the room for judicial review of governmental assessments in this regard is narrow.

This view also appears to have been endorsed by most academics following WW2, causing one author to remark that the public use clause was a "dead letter" (Merrill 1986). In light of this, the increased tension associated with takings  in the 1980s can hardly be explained by pointing to legal uncertainty. However, there was a change in how eminent domain was perceived in this period, towards greater scepticism. In part it may have been caused by a general resurgence in liberal political ideology. But in addition, some concrete cases proved particularly controversial, and they were taken to illustrate the dangers of eminent domain, particularly in relation to economic development projects. Now, in particular, it was not only natural resources and land that was subject to eminent domain; the takings power was used more aggressively, to condemn middle class homes.
The controversy surrounding the case of Poletown Neighborhood Council v. City of Detroit  illustrates this, and the case marks a watershed moment in the history of  economic development takings in the US, see e.g., (Underkuffler 2006, 380–381). In Poletown, the Michigan Supreme Court held that it was not in violation of the public use requirement to allow General Motors to displace some 3500 people for the construction of a car assembly factory.  This decision proved highly controversial, however, and it was later overturned in the Michigan Supreme Court, in County of Wayne v. Hathcock, a move widely seen as a response to the increased critical attention directed at these kinds of takings. 

While it never reached the Supreme Court, the case of Poletown seems to represent the prelude to the subsequent controversy that arose regarding Kelo, causing the surge of interest in eminent domain we have seen in recent years. In the next section, we turn to this period in more detail.

\subsection{The "grasping hand": Legitimacy of economic takings in the US after Kelo}

Shortly after Poletown was overturned, the case of Kelo saw the legitimacy of economic takings brought before the Supreme Court once again. This time there was real doubt and disagreement among the justices regarding the scope of the public use limitation. The case revolved around the legitimacy of condemning a home in favour of a research facility for the drug company Pfizer, which was part of a development plan for the City of New London.  The owner, Suzanne Kelo, argued that the condemnation of her home was in breach of the constitution, since it was a private-to-private taking ostensibly to the benefit of Pfizer rather than any clearly defined public use or interest.

In Kelo, Justice Thomas adopted the strictest view on the public use test. He entirely disregarded  the precedent set by Berman and Midkiff in favour of constitutional originalism, the doctrine which asserts that direct assessment of the wording in the Constitution, and the intentions of the founding fathers, is the approach that should be used to decide constitutional cases. Following up on this he held that actual right of use for the public was the test that had to be applied in takings cases. The hundred years of precedent preceding Kelo was described as “wholly divorced from the text, history, and structure of our founding document", and thus Justice Thomas concluded that it had to be abandoned. 

Justice O'Connor, in an expression of dissent joined by Chief Justice Rehnquist and Justices Scalia
and Thomas, argued against legitimacy on less theoretical grounds, based on the facts of the case and the precedent that would be set for similar cases in the future. Her main legal argument was that while public use should be interpreted broadly, the possibility of positive ripple effects was not enough to justify private-to-private takings. In particular, Justice O'Connor took a very bleak view on the practical consequences that would arise from allowing economic takings that could be justified only by pointing only to indirect positive consequences for the public. She commented on the majority decision to uphold the taking as follows: 

Any property may now be taken for the benefit of another private party, but the fallout from this decision will not be random. The beneficiaries are likely to be those citizens with disproportionate influence and power in the political process, including large corporations and development firms. As for the victims, the government now has license to transfer property from those with fewer resources to those with more. The Founders cannot have intended this perverse result.

It seems that a major point of contention among the judges in the Supreme Court was whether or not these grim predictions was a realistic assessment of what the consequences of the decision would be. Surely, anyone who agrees with Justice O'Connor in her prediction of the fallout would also agree with here conclusion that it is perverse. But the majority in Kelo, in an opinion written by Justice Stevens, disagreed with her assessment, observing instead that a more restrictive view on economic takings would make it more difficult to cater to the "diverse and always evolving needs of society". 
But the majority opinion also stressed that purely private takings where not permissible, and they attached great significance to the substantive assessment that the actual taking of Suzanne Kelo's home formed part of a comprehensive development plan that would not bestow special benefit on any particular group of individuals. Moreover, Justice Kennedy, in his concurring opinion, emphasised that states should not use public purpose as a pretext for interfering in property rights to the benefit of commercial actors.
Hence the overall impression one is left with when considering Kelo in its historical and legal context is that it reflects an increasingly cautious attitude to economic takings. The precedent of virtually unlimited deference that was set in case law from the mid-to-late 19th Century was eschewed in favour of a more contextual approach where the merits and deeper purpose of the plans underlying a taking is not axiomatically beyond the scrutiny of the courts.
From considering the reception of the case by the general public, we see even more clearly how Kelo in effect marks a change in the US towards greater scrutiny. Indeed, the voices that have dominated in the aftermath of Kelo were critical of the decision and criticized the court for not offering better protection to property owners. The case also led to an a surge of academic interest in the pubic use restriction, with many arguing for further restrictions on the scope of the takings power. 
Hence it seems that Justice O'Connor's opinion largely reflects contemporary worries about takings in the US, worries that are now also becoming increasingly relevant to how the law develops and is understood. Many states have changed their own eminent domain codes  following Kelo, to make it harder to undertake economic takings. Moreover, the federal government also banned such takings from taking place on the basis of federal takings powers.
It will lead us astray to delve deeply into the question of what caused this change in perspective on economic takings in the US, but we can offer a few hypothesis. First, it seems that cases such as Poletown illustrates the potential danger inherent in making the power of eminent domain available to market players. In particular, the main worry that has been raised is that the pretext of public purpose may be in the process of becoming a powerful instrument for influential market actors to gain access to regulatory powers of government. As these powers has massively expanded in the post-WW2 period, so has the potential for abuse. In addition, it seems that while those who were adversely affected by eminent domain tended to be less privileged and resourceful groups of society, the takings power is now increasingly brought to bear also against members of the middle class, who are in a better position to fight it, both legally and on the political scene.
While opinions differ greatly both regarding the extent of the problem and the causes of recent controversy, there is something near consensus in the US after Kelo that economic development takings raise special problems under the current system of eminent domain, and that these need to be addressed with a view to reducing tensions and restoring faith in the system. Indeed, even the majority in Kelo hint strongly at this when they say that  
Some have argued forcefully that a strict reading of the public use requirement is the way forward, if not by strict interpretation then by an explicit ban on economic development takings.  However, it is tempting here to echo the worries expressed in Seawell, that a strict formalistic approach to legitimacy runs the risk not only of being inflexible, but also, eventually, of offering less  protection to property owners. How, then, should we reduce the risk of abuses?
While many have focused on the question of banning economic taking, or reconsidering the public use clause, some have addressed this question from such a broader angle. In my opinion, this is the way forward. It seems, in particular, that a complete ban on economic development takings will leave a vacuum in the current economic system, which presupposes a great deal of cooperation between commercial and public interest. Particularly when it comes to economic development, the private-public partnership model has gained influence to the point that a ban on economic development takings would likely prove impossible to implement in a satisfactory manner. 
More generally, it seems hard to address the problem of economic takings without considering the role they play in the larger economic context within which current rules and practices have developed. Based on such considerations, I believe the procedural approach to economic takings is the appropriate one. This perspective asks us to take a closer look at judicial safeguards for protecting the role of property owners in the decision-making processes that lead up to the use of eminent domain. To some extent one might approach this on the basis of existing legal principles, asking for better scrutiny of procedural aspects, or by making it easier to bring pretext claims before the courts. However, it might also require new ideas, and, in particular, the introduction of new institutions for decision-making and administration of the eminent domain process. In the next section, I will look at two concrete proposals in more detail, one concerning the decision-making step and the other concerning the calculation of compensation. 
They will be important because they serve as starting points for the case study that is to follow, addressing mechanisms that we will return to in Chapters x and y when we look more closely at two Norwegian legal institutions that share many features with the theoretical proposals discussed in the next section.

\subsection{The institutional approach to economic development takings}

The primary distinguishing feature of economic development takings is that they give the taker an opportunity to profit commercially from the development. This may even be the primary aim of the project, with the public benefiting only indirectly through potential economic and social ripple effects. Property owners facing condemnation in such circumstances might expect to take a share in the commercial profit resulting from the use of their land. However, in many jurisdictions, including the US, the rules used to calculate compensation prevents owners from getting any share in the commercial surplus resulting from development. Indeed, various {\it elimination rules}, or {\it no scheme} rules, are typically in place to ensure that compensation is based entirely on the pre-project value of the land that is being taken.\footnote{References.} The policy reasons for such rules is that they ensure that the public does not have to pay extra due to its own special want of the property; indeed, one of the main purposes of eminent domain is to ensure that the public does not have to pay extortionate prices for land needed for important projects. However, when the purpose of the project is itself commercial in nature, there appears to be a shortage of good policy reasons for excluding this value from consideration when compensation is calculated. This is especially true when, as in the US, compensation tends to be based on the market value of the land taken. Why, when the project attracts commercial interest, should the buyer's prospect of carrying it out with a profit be disregarded from the assessment of market value? Surely, this would have a crucial bearing on the outcome of any friendly transaction among willing and rational economic agents? 

Many US writers have commented on this shortcoming of current compensation rules in economic takings cases, and most commentators appear to agree that special compensation rules are needed for this case type. In \footcite{eminc07}, the authors propose a novel approach to this challenge, based on a new kind of institution which they dub a {\it Special Purpose Development Corporation} (SPDC). The idea is that owners will be given a choice between standard pre-project market value and shares in a special company. This company will exist only to implement a specific step in the implementation of the development project: the transaction of the necessary land rights to the developer, for a negotiated price where the SPDC may also offer the land rights to other potential developers. Hence, the idea is to ensure that the owners are paid a value that reflects the post-project value of the land, but in such a way that the holdout problem is avoided. In particular, the SPDC will not be granted power to refuse to sell the land; as long as an offer is given that meets some minimum level (fixed by the legislature), it is compelled to sell to the highest bidder.  After having sold the land, moreover, the SPDC will cease to exist. 

Unlike other proposals made in the literature, this suggestion has a significant procedural and institutional component. The goal is to create more favorable market conditions for transferring land designated for development, in such a way that no owner can exercise monopoly power by holding out, but without handing over all the development value to the buyer. Other suggestions have focused on more static solutions, such as given owners a fixed premium in cases of economic development, or developing mechanisms of self-assessment to ensure that compensation is based on the true, non-bargaining, value the owner attributes to his own land.\footnote{Reference and explain.} Compared to such proposals, the idea of SPDCs seem more sophisticated, but there are still problems with it. In particular, it seems that the market created by the takings procedure envisioned in \footcite{eminc07} have some less desirable features.

First, since an SPDC does not have the power to stop the development, the system will only work as long as there are several interested parties who are willing to compete for the land rights. In practice, the planning regimes used to facilitate development can make this an unlikely prospect. Eminent domain is often used to implement highly specific projects, where the developer himself has played an important role in the planning process. Hence, while there might be many parties interested in the general kind of development that will be carried out, the concrete project that is being undertaken might not be of interest to anyone other than the developer who initially proposed it. Indeed, this developer might be the only one able to carry it out, say because the project forms part of a greater scheme involving rights that this developer already controls. Therefore, to create a market where the SPDCs can function as intended, it seems that deeper changes in planning practice are required, to avoid natural monopolies from developing in the planning process. This can be challenging in general, and in cases when eminent domain is used only for parts of greater schemes, it seems that it will be practically impossible to make the system work, without also undermining the governments ability to ensure that such schemes are carried out according to plan. In particular, setting up an SPDC alone is not, in general, sufficient to give property owners access to post-project market values. Something more is needed, and it is unclear what, if anything, can ensure that the SPDC gets to operate in healthy market conditions, as intended.

Second, the fact that the proposal is based on the doctrine of market value might in itself be a problem. Some of the clearest voices that have spoken out against economic development takings have done so on the basis of non-commercial objections. For instance, in the case of {\it Kelo}, the high subjective value that Suzanne Kelo attributed to her home was at the heart of the conflict. Indeed, Kelo continued to campaign against the condemnation, as a matter of principle, even after she had accepted a financial settlement whereby she was awarded some four times the estimated market value of her home.\footnote{...} The deeper problem of economic development takings is that they often lack legitimacy due to their partly commercial purpose. To compensate owners on the basis of market value alone does not offer any recognition of the fact that interference tends to appear less legitimate in such cases, compared to cases when the public purpose of the taking is more clearly discernible. It is of course not clear that such a lack of legitimacy can be addressed by compensation at all, but if this is at all going to be possible, then it must involve a compensation method which explicitly recognizes that these cases need to be treated differently than cases where problems of legitimacy do not arise. Moreover, since market value is the standard rule for awarding compensation, applying it in economic development cases will typically fail to provide any recognition of the fact that these cases are special. This in itself can become a perceived injustice; why should someone losing their home to a new school or a hospital be placed in the same category as someone loosing their home to a commercial company? In the latter case, it might be that the loss of a home appears {\it incommensurable} to the societal gains that may result. Hence, any compensation based on market value might lead to a moral deficit, whereby the affected property owners feels that non-commercial interest, including their own, are simply regarded as irrelevant.

The two problems addressed here both seem to point to the fact that the SPDCs, while more flexible than other suggestions, are still too static to achieve many of their objectives. In particular, to arrive at genuine market conditions for assessing post-project value, there is still a need for changes in the dynamic of the planning process underlying the taking, while to ensure legitimacy with respect to non-commercial aspects, there is a need for a mechanism that goes beyond commercial bargaining as such. In Chapter \ref{sec:app} we look at an alternative approach to compensation assessment, whereby legitimacy is sought through the participation of lay people in making the award. This system has long traditions in Norwegian law, and has currently been put to the test in cases of hydro-power development. In response to changes that have rendered such development commercial in nature, the compensation law in Norway has effectively been changed, something that has resulted in greater legitimacy. This process, as we will see, has resulted in large part from the activities of the special appraisal courts, and the lay people who partake in them. Also, the new compensation method that has been developed has many similarities to the ideas presented in \footcite{eminc07}, essentially serving as a means to make compensation a function of the post-project value of the land taken. However, the system relies heavily on the discretion of the lay people, judges and experts involved in deciding the award. Hence, the system is more flexible, and can be adapted to special circumstances of individual cases. There is also an ongoing tension between the legal side of compensation, as determined by the Supreme Court, and the administrative practices, as developed by the local appraisal courts. This tension is the defining feature of the system, and one which we believe show both its strengths and weaknesses, highlight its potential as an alternative approach to compensation in a time when takings law is both becoming more practically important and more controversial.

The approach in \footnote{eminc07} does not provide any guide in resolving controversies that arise in relation to how decision-making processes are organized in economic development cases. In addition to problems associated with compensation, such cases often suffer from a {\it democratic deficit}. While the main beneficiary, the developer, often plays a crucial role in the administrative process leading up to condemnation, property owners are not awarded any special right to participation beyond what follows from general planning and takings law. Just as the law fails to recognize economic takings as a special category for compensation purposes, so does it fail to give special rules to govern the preparation of such cases. This is widely seen as a shortcoming of the law. It creates an imbalance, in particular, between developers and owners, with the former enjoying a far greater ability to influence the decision-making process at the administrative level. In cases when developers are merely in place to execute public plans, and have no agenda on their own, this is no major threat to legitimacy. But in cases when the developer have significant interests of their own, and act as autonomous economic agents, the imbalance in influence becomes problematic.

This challenge is addressed in \footcite{lad08}, where the authors propose a new institutional framework for carrying out land assembly for economic development. The framework they propose is partly an extension of the eminent domain institute and partly presents an alternative to it. In particular, the proposal seeks to address the problem of democratic legitimacy while ensuring that structures remain in place to prevent inefficient gridlock and holdouts that would otherwise make large scale economic development projects hard to implement. The core idea is to introduce {\it Land Assembly Districts} (LAD), an institution that represents property owners in a specific area and has the power to, following a majority vote, to sell the land to a developer or a municipality. Voting rights in the LAD will be allocated in proportion to each owners share in the land belonging to the district. Owners can opt out of the LAD, but in this case eminent domain may be used to give the LAD control of their land. 

If a majority fails to form in favor of a sale, eminent domain can not be used against the land owners in the district. This is the crucial novel idea that sets the suggestion apart from other suggestions that have appeared after {\it Kelo}. LADs will not only ensure that the owners get to bargain with the developers over compensation, it will also give them an opportunity to refuse the development to go ahead, if they should so decide. Hence, the proposal shifts the balance of power in economic development cases, giving owners a greater role also in preparing the decision whether or not to develop, and on what terms. In my opinion, this makes the proposal stand out as particularly interesting in the recent literature on economic takings. It is the first concrete suggestion that addresses the democratic deficit in a dynamic, procedural manner, without failing to recognize that the danger of holdouts is real and that institutions are needed to avoid it, also in economic development cases.

There are some problems with the model proposed, however. In \footcite{ladres09}, the author points out that the basic mechanism of majority voting is itself imperfect, and can lead both to overassembly and underassembly, depending on the circumstances. It is pointed out, in particular, that if different owners value their property differently, majority voting will tend to disfavor those with the most extreme viewpoints, either in favor of, or against, assembly. If these viewpoints are assumed to be non-strategic and genuine reflections of the welfare associated with the land, the result can be inefficiency, since a majority can often be found that does not take due account of minority interests. For instance, if some owners are planning alternative development, leading them to attribute a high {\it hope}-value to their land, they can safely be ignored as long as the majority have no such plans. This could become particularly bad in cases when the alternative development itself is more socially desirable than the development that will benefit from assembly. The role of the LAD in such cases will not improve the quality of the decision to develop, since it pushes the decision-making process into a track where those interests that {\it should} prevail are voiced only by a marginalized minority inside the new institution.\footnote{Of course, one might imaging these land owners opting out of the LAD, or pursuing their own interests independently of it. However, they are then unlikely to be better of than they would be in a no-LAD regime. In fact, it is easy to imagine that they could come to be further marginalized, since the existence of the LAD, acting ``on behalf of the owners'', might detract from dissenting voices on the owner side.}

More generally, it remains unclear what role LADs are envisioned to play in the planning process and how they will affect the decision-making process as a whole. The idea is clear enough: LADs will help to establish self-governance in land assembly cases. More concretely, Heller and Hills argue that LADs should have ``broad discretion to choose any proposal to redevelop the neighborhood -- or reject all such proposals''.\footnote{p. 1496.} As they put it, two of the main goals of LAD formation is to ensure `` preservation of the sense of individual autonomy implicit in the right of private property and preservation of the larger community’s right to self-government''.\footnote{p. 1498.} However, at the same time they also stress that ``LADs exist for a single narrow purpose -- to consider whether to sell a neighborhood''.\footnote{1500} This is taken to be a safe-guard against conflict and abuse, serving to prevent LADs from becoming battle grounds where different parties attempt to co-opt the community voice to further their own interests. As Heller and Hills puts it, the narrow scope of LADs will ensure that ``all differences of interest based on the constituents' different activities and investments, therefore, merge into the single question: is the price offered by the assembler sufficient to induce the constituents to sell?''.

This points to a clear tension in the LAD proposal, between the broad goal of self-governance on the one hand, and the fear of neighborhood bickering and majority tyranny on the other. Moreover, the idea of LADs with a ``narrow purpose'' is hardly compatible with a scenario where the LAD has ``broad discretion'' to choose between competing proposals for development. If such discretion may indeed be exercised, what is to prevent special interest groups among the land owners from promoting those development projects that they happen to find most favorable? On the other hand, if the idea is that a LAD should not be free to assess the quality of development projects but must always sell to the highest bidder, it hardly seems fair to say that it enjoys ``broad discretion'' regarding which development proposal to accept. Moreover, would such a limit in the power of the LAD even be desirable? 

It is easy to imagine cases where competing proposals, perhaps emerging from within the community of owners themselves, will emerge in response to the formation of a LAD, suggesting less invasive forms of redevelopment. Perhaps such a proposal will even permit the majority of owners to either keep or reacquire homes in the area, by restricting demolition to the most blighted areas. What role should the original LAD play in a case like this? On the one hand, the alternative project might easily be a better use of the land in question, also from the point of view of the public. But on the other hand, it seems that the LAD might then become an arena for a new kind of power play among different interest, and a vehicle for oppression for whomever secures support from a majority of the owners within the district. 

 In their proposal, Heller and Hills are aware of this potential problem, which they propose to resolve by strict regulation. For instance, they argue that ``LAD-enabling legislation should require especially stringent disclosure requirements and bar any landowner from voting in a LAD if that landowner has any affiliation with the assembler''. Two questions arise, neither of which are resolved in their proposal. First, it seems unclear what is meant by ``affiliation'' here. For instance, what if a land owner happens to own shares in some of the companies proposing development? --Should he be barred from voting? And if so, should he be barred from voting on all proposals, or just those involving companies in which he is a shareholder?  Or what about the case when one of the land owners is an employee of one of the development companies? Should this render his vote null and void for the purpose of land assembly that might benefit his employer? It seems quite unfair if a previous affiliation should have this effect, but in some cases even such ties might play an important factor that could come to influence the outcome of the vote. This could be, for instance, if an important local employer proposes development in a neighborhood where it has a large number of employees. 

This illustrates how the regulation that Heller and Hills propose will raise a range of issues about how they are to be spelled out in further detail. The very idea of a ``narrow'' decision, in particular, seems somewhat dubious. In practice, would not a LAD invariably take up a strategically important position also in the planning process leading up to development? This, at any rate, seems inevitable if the system is really able to offer the form of self-governance that represent the overall goal. This leads us to the second problem with using regulation to narrow the function of LADs. Rather than being ``LAD-enabling'' is it not fairer to say that such regulation, at least if understood strictly, will curtail the power of LADs and limit their relevance to the decisonmaking process? Indeed, is it even desirable that local owners should not be able to affiliate themselves with developers in an effort to arrive at those schemes that they think will represent the best use of the land? Is this not the fundamental aspect of self-governance, that the local people themselves partake in the process, not just as passive spectators, but also as active agents that consult with planning authorities and developers about the best future use of their land. Hence, it might be that in many cases, those same mechanism that can lead to majority tyranny and bicekring if left unchecked are also crucial to unlocking the true potentail of LADs.

To find the right balance here, and to arrive at concrete rules that successfully regulate LADs in accordance with the overall aim, is very much an open problem. This is acknowledges by Hiller and Hills themselves, who point out that further work is needed, and that only a limited assessment of their proposal can be made in the absence of empirical data. In this thesis I will shed light on this challenge when I consider the Norwegian rules relating to land consolidation, showing how these can be looked at as a highly developed institutional embedding of many of the central ideas of the LADs. The assessment of how they function in cases of economic development, and how they are increasingly used as an alternative to expropriation in cases of hydro-power development, will allow me to shed further light on the issues that are left open by Hillen and Hills' important article, and to do so with an underpinning of empirical data from a jurdisicition that already has a similar framework in place.

\section{Conclusion}

In this Chapter I have presented the key issue that will remain in focus for the remainder of this thesis: legitimacy of economic development takings. I will now move on to consider Norwegian hydro-power as a case study of such takings. This case study is well suited to shed light on many of the issues that have been flagged as central, particularly in the US, and particualrly in relation to {\it institutional} approaches to the problem of economic takings. As we have seen, this approach asks us to focus on the legitimacy of the procedure, and the relationship between procedure and outcome in such cases. Importantly, it asks us to recognize that rather than trying to adjudicate about the meaning of ``public use'', a flexible framework with special safeguards is needed to ensure that the takings procedure remains legitimate in cases when commercial interest play an important role. Importantly, in such cases, it is important to avoid the democratic deficit that occurs when the taker has a disproportionate advantage in furthering his own non-public interests, compared to the land owners. 

I focused on two suggestions that have been made to ensure that this deficit is corrected. Both are institutional in nature, and involve setting up new formally recognized coalitions of land owners that will act as a counterweight to the power of the taker. The first suggestion limited its attention to compensation, recognizing the need for a system whereby the land owners are renumerated based on post-projcet value in such cases. This idea in itself represents a failraly dramtic break with the currently dominatnt doctrine in takings law, where compensation is almost always, and in almost all jurisdictons, based on the pre-project value of the land, or the {\it value to the owner}. In Chapter \ref{sec:noscheme} I will return to this principle in more depth, looking at how it developed interantially, and in Norway. Then, in Chapter \ref{sec:appraise} I will look at how it has now been abandoned in Norwegian law, for some case tpyes involving hydro-power development, and as a result of the special system of appraiseal courts in place in NOrway. The flexibility of these courts, involving lay peoplw, allows the law to be applied in a way that adapts to the concrete circumstance of the case, and the perceived farines and legitmiatmcy of the taking. Hnce, the NORwegian case might ofer interesting lessons about how to ensure fleibility in the compensation law. Moreover, I will look at how exactly the adaptation to economic takings cases is done, focusing on relating the current practices under Norwegin law to the general rules of comepnsation and the proposals made in the litarerature elsehwen, particularly the SPDC suggestion made by .... and.... 

The second suggestion I looked at in depth focused not only on the compensation, bute also ont he decision-making proess leading up to economic development. It recognixed the need for a mehcnaimsm that give local communities greater self-govername in cases of ecnonooimc development. At the same time, it recongized the conitnued tneed for  a mechanaism to acoid inefficint and socially harmful gridlock due to holdsouts among unwilling owners. Instead of emientent domain, however, a different mechanism should be used in eeconomic devellopment cases: a land assembly distirict. This is also a new class of instiuttions, and I pointed out some major problems associated with the extent to which they will be an active pticipcant int he palnning process, not just the transaction of land rights from orignal oners to devleopers. I argued that while the risk of abuse and failure increase with the level of participation, so des the pstive effect of this novel insitutions. To reduce the demoratic deficit in economic devleopment cases, it seems to me that a fairly wide power of participation must be granted to the land owwners, to restore valance and fairness between them and the developer. Hence, the question arises how to oraganixe this in a way that avoids sthe pitfalls associated witth strong and autnomous local government. 

In Chapet \ref{sec:landcons} I will shed light on this qeuston by considering the NOrwegian istntiutsion of land consoildation, which has very long traditions. It is a wery wide and flexible framworks, but includes, among other things, a frameworks for establisheing instutiontions resmbling that of LADs. I will foucs on how land consolidtion functions in cases of economigc developent thta would otherwise likely be spursued by ecminetnt domain. Agains, the case study will be Nroweigan hydro-power devleopmetn, but I will also discuss planning law and declopment more generally, as the Norweigna government is now considering amikng consolidation, traditioanlly a rural intusticment, a primary role in land devlopment also in rural areas, and in rlation to general development procects following land assembly.

Before I delve into these hapters, which all seek to directly dshed light on the issues raised in this chapeer, I will present the basic facutal and legal basis of my cases: the management of hydor.poer devleopment under NOWerrfgian law. Tis will be the subject of the next chapter, written in sucha  aways as to highlight that how it instantiates the same issues that have been discussed iabstractly here.

Blackstone, William. 1979. Commentaries on the Laws of England: A Facsimile of the First Edition of 1765--1769. University of Chicago Press.
Cohen, Charles E. 2006. “Eminent Domain After Kelo v. City of New London: An Argument for Banning Economic Development Takings.” Harvard Journal of Law and Public Policy 29: 491.
———. 2008. “The Abstruse Science: Kelo, Lochner, and Representation Reinforcement in the Public Use Debate.” Duquesne Law Review 46: 375–419.
Heller, Michael, and Rick Hills. 2008. “Land Assembly Districts.” Harvard Law Review 121 (6): pp. 1465–1527.
Horwitz, Morton J. 1973. “The Transformation in the Conception of Property in American Law, 1780-1860.” University of Chicago Law Review 40: 248–90.
Householder, Benjamin A. 2007. “Kelo Compensation: The Future of Economic Development Takings.” Chicago-Kent Law Review 82: 1029–61.
Hoyman, Michele M., and Jamie R. McCall. 2010. “Not Imminent in My Domain! County Leaders’ Attitudes toward Eminent Domain Decisions.” Public Administration Review 70 (6): 885–93. doi:10.1111/j.1540-6210.2010.02220.x.
Johnson, Emily A. 2011. “Reconciling Originalism and the History of the Public Use Clause” 79: 265–319.
Lehavi, Amnon, and Amir N. Licht. 2007. “Eminent Domain, Inc.” Columbia Law Review 107 (7): pp. 1704–1748.
Meidinger, Errol. 1980. “The ‘Public Uses’ of Eminent Domain: History and Policy.” Environmental Law 11: 1–66.
Merrill, Thomas W. 1986. “The Economics of Public Use.” Cornell Law Review 72: 61–116.
Somin, Ilya. 2007. “Controlling the Grasping Hand: Economic Development Takings after Kelo.” Supreme Court Economic Review 15 (1): pp. 183–271.
Underkuffler, Laura S. 2006. “Kelo’s Moral Failure.” William \& Mary Bill of Rights Journal 15 (2): 377–88.
Walsh, Rachael. 2010. “‘The Principles of Social Justice’ and the Compulsory Acquisition of Private Property for Redevelopment in the United States and Ireland.” Dublin University Law Journal 32: 1 – 24.
Waring, Emma J L. 2009. “Aspects of Property: The Impact of Private Takings.”
 



\chapter{Norwegian waterfalls and hydropower}\label{chap:3}

\section{Introduction}\label{sec:into3}

\section{Norway in a nutshell}\label{sec:nutshell}

\section{Hydropower in the law}

\section{Hydropower in practice}

\section{{\it Nordhordlandsmodellen}}

\section{Conclusion}

Old stuff:

\section{Introduction}\label{intro}

Norway is country of mountains, fjords and rivers, and about 95 \% of the annual domestic electricity supply comes from hydro-power.\footnote{See Statistics Norway, data from the year 2011, http://www.ssb.no/en/elektrisitetaar/.} The right to harness rivers for hydro-power is held by local landowners, but historically, this right has not been of much use to them, since the Norwegian electricity sector has been organized as a regulated monopoly, with most hydro-power schemes carried out by non-commercial companies controlled by the State, or local governmental bodies. In the early 1990's, however, the sector was liberalized, and it has become increasingly common for local landowners to undertake their own hydro-power projects. This has led to increased tension between local interests and established energy companies. Following liberalization, these companies are now organized for profit, and this raises the question: who is entitled to benefit from Norwegian hydro-power? 

Increasingly, it is becoming clear that this is not merely a question of the opposing commercial interests of individuals and companies, but also crucially involves the local communities directly affected by development. The original owners of hydro-power tend to be farmers residing in the local communities where the resources are found, and therefore, the question of who should benefit also encompasses the question of who should be allowed to participate in decision-making process, and what degree of autonomy local communities are to be granted in this regard. Are local owners and their communities entitled to a say in how the hydro-resource is to be exploited, or must they accept to remain passive, as they were rendered by the monopoly which used to be in place?

In this chapter, we address some recent demands made by local people, to the effect that they should be allowed to partake more actively in decision-making processes regarding local waterfalls. We focus on the legal status of such demands under Norwegian law, and we do so by considering the recent Supreme Court Case of \emph{Ola Måland and others v. Jørpeland Kraft AS}.\footnote{Ola M{\aa}land and others v. J{\o}rpeland Kraft AS, Rt 2011 s. 1393. I mention that I represented the local owners in this case, as a trainee lawyer in the district and regional courts, and as the responsible lawyer before the Supreme Court.} In this case, the local owners protested the legality of a license that granted the developer, Jørpeland Kraft AS, a right to divert water away from their waterfalls, thereby reducing the potential for local hydro-power. The owners argued that consultation had been insufficient, and they also contended that the assessment of the case made by the water authorities had been inadequate, and that the decision has been based on an erroneous account of the facts. They won the case in the district Court, Stavanger Tingrett, but lost under appeal to the regional Court, Gulating Lagmannsrett. The Norwegian Supreme Court also found in favor of the developer, and the argument they gave to support this conclusion goes far in suggesting that while the commercial interests of local owners need to be compensated, the presence of such interests do not entitle local communities to a greater say in decision-making processes. Importantly, the Court held that the presence of local interests does not necessitate the adoption of different administrative practices, and that the procedures developed during the decades of direct, non-commercial, administration of the energy sector, could still be followed.

The case is significant, since the traditional hydro-electric scheme in Norway typically involves expropriation, often interfering with the property rights of hundreds of local individuals. Traditionally, owners of waterfalls would be compensated according to a standardized mathematical method that was based on the assumption that they had no interest in hydro-power themselves.\footnote{The method consists in calculating the number of \emph{natural horsepowers} in the waterfall, and then multiplying this number by a price pr. natural horsepower, determined by the discretion of the Court, but in practice based almost solely on what has been awarded in previous cases. For a description of the traditional method, we point to \cite{falk} Chapter 7, page 521-522 (in Norwegian).} In a landmark case from 2008, however, the Norwegian Supreme Court commented, in an \emph{obiter dicta}, that the traditional method for calculating compensation for waterfalls was no longer appropriate, at least not in cases when it can be demonstrated that the original owners could have exploited the resource themselves, if the expropriation had not taken place.\footnote{The case of Agder Energi Produksjon AS vs. Magne Møllen, Rt. 2008 s. 81. The local owner lost the case, the reason being that the Supreme Court held that compensation should not be based on the present day value of the waterfall, but the value it had when the original transferral of rights took place, in the 1960's. The \emph{obiter dicta} has been used as an authority for subsequent Supreme Court decisions, however, see, for instance, Rt. 2010 s. 1056 and Rt. 2011 s. 1683. It has received quite a lot of scholarly attention as well, see \cite{Tf1,Tf2,Tf3}.} This decision has had a profound impact on the level of compensation awarded for waterfall rights, leading to payments that can will typically be ten to a hundred times higher than that which would have been awarded according to the traditional method.\footnote{So far, in cases that have come before the court, there has been about a twenty-fold increase in compensation, see \cite{Tf1}, but the new method will, when applied to certain kinds of projects (cheap to build, and involving little or not regulation of the water-flow), result in compensation having to be paid that amounts to at least a hundred times more than what could be expected if the traditional method had been applied.} More generally, it also served to shift the balance of power in favor of local owners and their communities, who increasingly expect to have their voices heard, and to get a more direct say over how their energy resources are managed and exploited.

After \emph{Måland}, however, it has become unclear to what extent the presence of local, commercial interests will continue to influence the Norwegian energy sector, and if we will see more active participation by local people in the future. In fact, recent statements made by the Norwegian water authorities seem to suggest that this is becoming increasingly unlikely, as a shift in policy seems to have taken place, whereby local, small scale projects, are now to be given lower priority than large scale projects undertaken by established energy companies.\footnote{These statements were not linked to \emph{Måland}, but were made in a more general context, ostensibly motivated by the desire to increase the efficiency of the administrative process, see http://www.nve.no/no/Konsesjoner/Vannkraft/Smaakraft/ where the new policy was announced. It also received some attention from the press, see, for instance, http://www.tu.no/energi/2012/01/18/nve-varsler-flere-smakraft-avslag.}

Taking a broader view on Norwegian law, we believe that recent experiences regarding hydro-power provides an interesting case to study, and one that will shed light on how property rights function in a social, economic and political context. It seems, in particular, that the view of property rights to waterfalls adopted by the Supreme Court rests on a narrow interpretation, seeing such rights merely as bestowing financial interests on certain individuals. This was clearly felt in \emph{Måland}, and we think the case also serves to illuminate certain consequences of such a view, suggesting, in particular, that it can have detrimental social and political consequences, and can very easily lead to perceived injustices. We also think it is pertinent to ask if the narrow view of property which seems to have been adopted for waterfalls in Norway is adequate with respect to human rights law, or if the right to property should also be considered a right to participate, and a right to be heard, in decision-making processes.

In the following, we first give the reader some further background on Norwegian hydro-power, and then 
we present \emph{Måland} in some detail, focusing on giving the reader an impression of current administrative practices, by describing how they played out in this particular case, and by detailing how they came to result in a decision that the original owners felt to be fundamentally unjust. We also address the legal arguments given by the opposing sides and the arguments relied upon by the national courts. We conclude by presenting some overreaching issues that we believe the case raises, regarding both the social context of property rights, the content of property as human right, and the question of whether or not the protection awarded under Norwegian law currently meets the standard set by the European Convention of Human Rights, as interpreted by the Court in Strasbourg. 

\section{Background: local owners making their voices heard by suggesting small scale hydro-power}\label{context}

As we mentioned in the Introduction, Norwegian law regards waterfalls as private property, and by default, a waterfall belongs to the owner of the land over which the water flows.\footnote{See Section 13 of Act No. 82 of 24 November 2000 relating to River Systems and Groundwater.} This does not mean that the landowner owns the water as such -- freely running water is not subject to ownership -- but it entitles the owner of the waterfall to harness the potential energy in the water over the stretch of riverbed belonging to him. This right can be partitioned off from any rights in the surrounding land, and large scale hydro-power schemes typically involve such a separation of water-rights from land-rights, giving the energy company the right to harness the energy, while the local landowner retains the rights in the surrounding land.

Norwegian rivers, and especially rivers suitable for hydro-power schemes, tend to run across grazing land owned jointly by farmers, so rights to waterfalls are typically held among several members of the local, rural community.\footnote{The land in question tend not to be enclosed, in particular, and in cases where there has been a land enclosure, water-rights have often explicitly been left out, such that they are still considered common rights, belonging to the community of local farmers.} They might not always be willing to give them up, especially not on the terms proposed by the developer, so the use of expropriation has played an important role in the history of Norwegian hydro-power. This has meant that the terms governing separation of water-rights from land-rights, including the level of compensation paid to landowners, and the influence they are granted in the decision-making process, has been determined by the law. Following legislation in the early 20th century, a regulatory system was put in place that centralized the management of Norwegian water resources. It clearly favored exploitation by the State or by companies owned by local governmental bodies, and the local landowners were severely marginalized. In most cases, they would have to accept the terms presented to them by the developer, or else argue the matter in Court, after the developer had already been granted a license to expropriate. Landowners were not in a good position to negotiate the terms of the development, and their property rights appeared increasingly nominal, the prevailing political attitude being that waterfalls formed part of the common heritage of the Norwegian people, and should be managed in their interest.\footnote{While some of the claims made here will be further qualified by what is to follow, the general picture we paint here is communicated also by the standard work on Norwegian water law \cite{falk}.}

This created a legal tension where, on the one hand, waterfalls were still considered private property under land law, yet, on the other hand, were considered as belonging to the public as far as large scale hydro-power development was concerned. The following two quotes, the first from the general water law, with roots going back at least to the 19th Century, and the second  from a law directed specifically at large scale hydro-power, illustrates this ambivalence.

{\begin{minipage}[t]{16em}
 \begin{aquote}{\tiny Section 13, Water Resources Act 2000} \footnotesize A river system belongs to the owner of the land it covers, unless otherwise dictated by special legal status. [...]

The owners on each side of a river system have equal rights in exploiting its hydro-power...
\end{aquote}  
\end{minipage}}
{\begin{minipage}[t]{22em}
\begin{aquote}{\tiny Section 1, Industrial Concession Act 1917 (amended 2008)} \footnotesize Norwegian water resources belong to the general public and are to be managed in their interest. This is to be ensured by public ownership...
\end{aquote}
\end{minipage}} \\

Following liberalization of the Norwegian energy sector in the early 1990's, this legal tension in statute has increasingly also become a tension in politics, where the interests of local communities and landowners stand in opposition to the interests of large energy companies, often owned by the State, seeking to harness locally owned resources for commercial gain. The question of how the Norwegian legal framework is actually applied in this regard is therefore a matter that has come under increased scrutiny. This was the question that went before the courts in the case of \emph{Måland}, where local owners protested the legality of expropriation on the grounds that they could harness the water in their own small scale hydro-power scheme. Before we delve into the details, we will elaborate a bit further on the context in which the law was called upon to function in this case. Importantly, the economic, social and political context of expropriation has changed rather dramatically in recent years, and we do not think it is possible to understand the case and the issues it raised except in the context of these changes. 

There are two developments that have been particularly important. First, there has been a general shift from viewing electricity production as a public service to viewing it as a commercial enterprise. This has made the legitimacy of expropriation appear more controversial, and the argument is often voiced that expropriation does not happen in the interest of the public at all, but \emph{solely} in order to benefit the commercial interests of particular companies.\footnote{This has been a recurring theme in articles appearing in "Småkraftnytt", the newsletter for "Småkraftforeninga", an interest organization for owners of small-scale hydro-power, which currently have 236 associated small scale hydro-power plants, see http://kraftverk.net/ (in Norwegian). In addition to the case of Måland, the question has also been brought before the (lower) national courts in some other cases, such as \emph{Sauda}, LG-2007-176723 (Gulating Lagmannsrett, regional high court), and \emph{Durmålskraft}, see http://www.ranablad.no/nyheter/article5583405.ece (decision from the district court, as reported in a Norwegian newspaper). In both cases, the outcome was generally more favorable to the expropriating party than the local owners, and the reasoning adopted by the courts appears similar to that of \emph{Måland}.}

In this way, expropriation of Norwegian waterfalls raises issues that have become increasingly important also in a global setting, and which seem to arise naturally in systems where economic activities are organized according to a mix of socialist and free market principles. In such systems, it seems practically inevitable that cases of expropriation -- undertaken to benefit the public -- will also often come to benefit developers that are motivated by purely commercial interests. While this in itself might not be problematic, it will easily lead to the concern that the commercial interests of powerful companies is the \emph{only} reason why expropriation is permitted, and that expropriation is being used as a commercial tool for powerful market forces, to the detriment of less powerful actors. That this can be highly controversial is illustrated in the US case of \emph{Kelo v. City of New London}, which divided the US Supreme Court and has also attracted great attention, both from legal scholars and in the general public, as a political issue.\footnote{\emph{Kelo v. City of New London}, 545 U.S. 469 (2005).}

For the case of Norwegian waterfalls, however, liberalization of the energy sector has also had a positive effect for local communities, in that it has served to make local owners more active. It has become increasingly common that they exploit their hydro-power resources themselves, often in small scale projects, and often in cooperation with companies that specialize in such development.\footnote{In 2012, the NVE granted 125 new licenses for small scale hydro-power, and at the end of the year they had 859 applications still under consideration. Source: report made by the NVE, available at http://www.nve.no/Global/Energi/Q412\_ny\_energi\_tillatelser\_og\_utbygging.pdf (in Norwegian). } This, of course, only adds to the controversy surrounding expropriation of waterfalls, especially when local owners are deprived of the opportunity for small scale development.

The most significant step towards liberalization of the Norwegian energy sector was made in 1990 when the Energy Act was passed, an important new piece of statute reorganizing the system for the distribution of electricity.\footnote{Act nr. 50 of 29 of June 1990 relating to the generation, conversion, transmission, trading, distribution and use of electricity.} The Energy Act introduced the principle that energy consumers and producers should have non-discriminatory access to the national electricity grid, thereby creating a market where any actor, privately owned or otherwise, could supply electricity to the grid, and profit commercially from hydro-power. In the same period of time, monopoly companies were reorganized, becoming commercial companies that were meant to compete against each other, and against new commercial actors that entered the market.\footnote{For a short English summary of how the system is administered, see for instance \cite[p.29-30]{ar2010}, and for more detail, we point to \cite{Hammer2}.}

The Norwegian State retained a significant stake as shareholders in energy companies, however, now often alongside private investors. Moreover, many rules in Norwegian law favor companies where a majority of the shares are held by the State, and to this day the largest and most influential Norwegian energy companies remain under public ownership.\footnote{The fact that publicly owned companies are favored in this way is often seen as a questionable practice with regards to competition law, see for instance the recent EFTA Court case, Case E-2/06, \emph{EFTA Surveillance Authority v. The Kingdom of Norway}, EFTA Court Report 2007, p.164. Here, the Court considered the old Norwegian rule of \emph{reversion}, whereby a license to undertake certain large scale hydro-power schemes (strictly speaking, a license to acquire the waterfalls needed to undertake it) came with a special clause that the private developer had to give up ownership to the State after a fixed period of time. This clause was held to be in breach of the EEA agreement since it only applied to private companies. We remark that the Norwegian government responded to this with an amendment after which reversion no longer applies, but which stated that a license to acquire waterfalls for the purpose of such large scale schemes can not be given at all to any company in which private parties own more than 1/3 of the shares.}

It seems, in particular, that the aim of liberalization in Norway has never been to minimize State control over hydro-power, but rather to give consumers greater freedom in choosing their energy-supplier, and to enhance efficiency in the sector by introducing competition.\footnote{See for instance \cite{liberal}, which offers a comparative study of the liberalization of the energy sectors in Norway and the UK.} Still, the fact that any developer of hydro-power is now legally entitled to connect to the national grid has proved important in giving actors that are not owned by the State a fighting chance on the Norwegian energy market. It has been especially important for local owners of waterfalls, since it means that if they undertake hydro-power projects themselves, they can no longer be refused access to the grid, but will be in a position to benefit commercially.

It should be noted that the Norwegian grid is operated by regional companies, responsible for the supply and distribution of electricity in their region. These will typically also be energy producers themselves, and historically, they would prevent other hydro-power initiatives by refusing them access to the grid. In fact, in the early days on Norwegian hydro-power, in the first half of the 20th century, there were quite a few locally owned and operated power plants, often providing local communities with electricity. When the national grid was established, most of them were closed down, often as a result of an explicit policy on part of the authorities. To increase the cost-effectiveness of the companies responsible for providing the national service, these companies were often allowed to demand, as a condition for allowing local communities access to the grid, that local hydro-power plants had to be shut down.\footnote{See \cite[p.111]{Hindrum} (in Norwegian).} 

Following legislation whereby access to the grid is provided for in statute, we have seen a surge of interest in the exploitation of waterfalls in small scale hydro-electric schemes, and these schemes are often initiated by local owners. As we have mentioned, the typical owners of Norwegian waterfalls are communities of farmers and smallholders. Historically, the right to land-based resources, especially in the mountainous areas of the west and the north, where most valuable waterfalls are located, was held by local people, the same people who made use of it on a day to day basis. The main reason for this, which by European standards stands out as quite unusual, was that Norway never really had a separate class of landed nobility. Consequently, the Norwegian farmer occupied a position of relative autonomy and freedom, even to the point of exercising significant political influence, especially in the early days of Norwegian democracy.\footnote{During the 19th century the two dominant group in Norwegian politics were the farmers and the civil servants, and the former group exercised great influence in the Norwegian parliament, with the 1833 election leading to what became known as the farmers parliament. The "classic" academic treatment of farmers' influence over 19th century Norwegian politics is \cite{Koht} (in Norwegian). More on the author and a summary of his views can be found here, http://en.wikipedia.org/wiki/Halvdan\_Koht.} Following industrialization, however, their role became much more marginal, and farming has steadily become more and more unprofitable, with many farming communities having already disappeared, and many others threatened by depopulation. In light of this, the possibility of undertaking small scale hydro-power is often seen as being important to the survival of rural communities themselves, not just as a means for individual members of such communities to make a profit.

As local owners started to harness their waterfalls themselves, commercial companies also emerged, specializing in cooperating with them. In a recent report, it was estimated that there is a potential for profitable small scale hydro-power of about 20 TWh/year \cite{Aanesland}, with a total value, before investment, of about 70 billion Norwegian kroner, i.e., about 8 billion pounds.\footnote{For comparison, suggesting the scale of this potential, we mention that the total consumption of electricity in Norway in 2011 amounted to 114 TWh, see http://www.ssb.no/en/energi-og-industri/statistikker/elektrisitetaar.}  This report was based on a particular model of cooperation with a commercial company, Småkraft AS, and might be an underestimate of what small scale hydro-power could represent for local communities if they take a more independent role in developing the resource. Thus, small scale development of hydro-power has become socially and political significant, and it is increasingly seen as a possibility for these regions to counter depopulation and poverty, while also regaining some of their autonomy and influence with respect to how the natural resources found locally are to be managed. In many cases, small-scale hydro-power appears to be the only growth industry, and takes on great political and social importance for the community as a whole, not just the owners of waterfalls.\footnote{For an example of a community where small scale hydro-power has played such a role, we can point to Gloppen, a municipality in the county of Sogn og Fjordane, in the western part of Norway. 19 schemes have already having been successfully carried out, all except one by local owners themselves, amounting to a total production of over 250 GWh/year. This prompted the mayor to comment that "small scale hydro-power is in our blood", see \cite{Gloppen}. When interviewed, he also directed attention at the fact that hydro-power had many positive ripple effects, since it significantly increased local investment in other industries, particularly agriculture, which had been severely on the decline.}

Summing up, we can conclude that expropriation of waterfalls has become more politically and social controversial, and we believe that the case of \emph{Måland}, to which we now turn, must be understood in this context. The case did not attract the same public attention as the cases relating to the revision of the traditional method for awarding compensation, and has, as far as we are aware, not previously received any scholarly attention either. It seems important, however, in that it clarifies the stance that Norwegian Courts take with respect to the question of the extent to which local owners and communities are entitled to take part in the decision-making processes concerning commercial development of the waterfalls they own. Moreover, while local interests have claimed a significant victory with respect to compensation, \emph{Måland} limits its impact since it suggest that there is still very limited legal protection of local owners' right to have their voices heard regarding how Norwegian power is to be managed.

In the following, we give a presentation of the case. We start by presenting the facts, and we do so going back to original sources, not merely looking to the brief presentation provided by the courts in their judgments, but to the preparatory documents assembled by the water authorities, taking special note of both the arguments presented by the expropriating party, and the objections raised by original owners and their representatives. Building on this, we the present the legal arguments that were raised by both sides, aiming to provide a more in depth presentation than the review given by the courts. We then present and compare the various arguments relied upon by the courts, and we offer our own analysis of how to understand the outcome of the case in the context of Norwegian law. We continue by addressing what the decision tells us about the Norwegian legal framework for hydro-power exploitation more broadly, and the questions it raises with respect to the social implications of expropriation of waterfalls, and with respect to human rights law.

\section{The facts of the case}\label{sum}

The case started 10 September 2004 with Jørpeland Kraft AS submitting an application to undertake a watercourse regulation, as provided for in the Watercourse Regulation Act of 1917, Section 8.\footnote{Act No. 17 of 14 December 1917 relating to Regulations of Watercourses.} As is customary, the application also included an application for a license to acquire waterfalls, as set out in the Industrial Concession Act, and a right to expropriate necessary rights from local owners, as provided for in the Water Resources Act, Section 51 and the Expropriation Act, Section 2 nr. 51.\footnote{Act No. 16 of 14 December 1917 relating to Acquisition of Waterfalls, Mines and other Real Property, Act No. 82 of 24 November 2000 relating to River Systems and Groundwater and Act No. 3 of 23 November 1959 relating to Expropriation.} In practice, it has not been common to consider such applications separately, but to consider the project as a whole, and to raise issues with respect to special provisions, and particular licenses, only in so far as they arise in connection with assessing the application for a development license, which is considered the main issue.

The management of water resources in Norway is centralized, and at the lowest level of authority we find the Norwegian Water Resources and Energy Directorate (NVE), which is a national body, based in Oslo. In some cases they have been delegated authority to grant development licenses themselves, but in most cases of large scale development, they only prepare the case, then hand it over to the Ministry of Petroleum and Energy which then gives its recommendation to the King in Council, who makes the final decision. The local municipalities, while generally quite powerful under Norwegian law, are completely sidelined when it comes to management of water resources, and their role is mostly limited to commenting on the plans, alongside other stakeholders.\footnote{Although there seems to be a good case to be made that they could exert greater influence over the process, based both on general planning law, which empowers them a great deal, or on special rules set out in agricultural law, which requires them to approve, on a case by case basis, any shifts in the property structure of agricultural land. In practice, however, they almost never exercise any of these powers with respect to water resources. If the developer has a license to undertake the scheme itself granted by the King, then it seems that municipalities most often take it to mean that they are obliged to follow suit, by granting the (relatively speaking) minor licenses that might be required with respect to general planning law and agricultural law.}

For the local owners of waterfalls, the situation is worse, since they are not identified as stakeholders in large scale projects. They are not, in particular, mentioned in the Watercourse Regulation Act, Section 6, which regulates the steps that must be taken when preparing such cases.\footnote{Nor do the seem to be mentioned  in any of the documents setting out how the authorities deal with such cases in practice. See, for instance, the guide published by NVE \cite{rettleiar} (in Norwegian), directed at applicants, and setting out how NVE deals with cases involving large scale hydro-power.} Consequently, it is hardly surprising that in administrative practice, it has been uncommon to devote particular attention to local owners. Rather, the focus has typically been on environmental issues and the opinions of various interest groups, such as hunter or fishermen's associations.\footnote{For a more in depth account of the process, we point to the standard legal reference on Norwegian water law \cite{falk}(in Norwegian).}

The applicant in Måland, Jørpeland Kraft AS, is a company jointly owned by Scana Steel Stavanger AS, who own 1/3 of the shares, and Lyse Kraft AS, who is the majority shareholder holding the remaining shares. The former is a steelworks company located in the small town of Jørpeland in Rogaland county, southwestern Norway. Historically, this company has been a major employer in Jørpeland, which is located by the sea, next to a mountainous area. The main source of energy for the steel industry in Norway has been hydro-power, and Scana Steel Stavanger AS is no exception. The company uses energy harnessed from the rivers in the area, and while the primary river runs through the town of Jørpeland itself, it is supplemented by water from other rivers in the area that are diverted so that they can be exploited more efficiently along with the water from the Jørpeland river.

Recently, Norwegian steel companies have become less profitable, due in great part to increased foreign competition and a significant increase in cost of operation associated with this type of industry in Norway, particularly salary costs.\footnote{For a reference on this, see \emph{Information Booklet about Norwegian Trade and Industry}, published by the Ministry of Trade and Industry in 2005.} This has led to many such companies shifting their attention away from labor-intensive steel production, and focusing instead on producing electricity, selling it directly on the national grid. Jørpeland Kraft AS was established as part of such a move being made with regards to the energy resources in Jørpeland, and the role played by Lyse Kraft AS is an important one. As we mentioned, Norwegian law favors companies where the majority of the shares are held by public bodies, and Lyse Kraft AS, being publicly owned, with the city of Stavanger as the main shareholder, is therefore a valuable partner. Moreover, Lyse Kraft AS, while being a commercial company, is also responsible for the electricity grid in the region. It was established as a merger between several local monopoly companies in the Stavanger region which were reorganized following liberalizaion of the sector in the early 1990's. As discussed in Section \ref{context}, there is little doubt that old monopolists still enjoy considerable power and influence.\footnote{In fact, Lyse Kraft AS is good example suggesting that their power might in some cases have \emph{increased}. Since liberalization, the restraints imposed both by the non-commercial nature of former monopolists, and the local, political, anchoring of such companies, have disappeared.} This is another reason why they can serve as valuable partners for private companies wishing to make a profit from Norwegian hydro-power.

With attention shifting from harnessing rivers for the purpose of industrial production to the purpose of producing electricity to sell on the national grid, the main variables that determines the profitability of the undertaking also changes. On the cost side, what matters becomes only the cost of producing the electricity itself, and this is typically determined, for the most part, by the investments required for the original construction works.\footnote{For an overview of the considerations made when assessing the commercial value of small scale hydro-power, we point to \cite{kartlegging}. In fact, due to the importance that small scale hydro-power has assumed in recent years, investigating models for investing in such projects has become an active field of research in Norway, see for instance \cite{investment}.} Running and maintaining a hydro-power station tends to be comparatively inexpensive. On the income side, what matters is the price of energy on the electricity market, a market that is no longer anchored in the local conditions of supply and demand.

Importantly, as long as energy production is the sole focus, the business no longer depends in any significant way on the local labor force, and as a result, it is typical that large scale exploitation becomes much more profitable, compared to the medium or small scale power plants typically needed to facilitate local industrial exploits. Hence, it was in keeping with a general trend in Norway when Jørpeland Kraft AS, following their shift in commercial strategy, proposed to undertake measures to increase their energy output. This could be achieved relatively cheaply, by further constructions aimed at channeling water from nearby waterfalls into dams that were already built to collect the water from the Jørpeland river.

One relatively small waterfall from which Jørpeland Kraft AS suggested to extract water was owned by Ola Måland and five other local farmers. This waterfall is not located in Jørpeland kommune, and does not reach the sea at Jørpeland, but runs through the neighboring municipality of Hjelmeland, on the other side of a mountain range, until it eventually reaches the sea at Tau, another neighboring municipality. The plans to divert this water would deprive original owners of water along some 15 km of riverbed, all the way from the mountains on the border between Hjelmeland and Jørpeland, to the sea at Tau. Far from all the water would be removed, but the water-flow would be greatly reduced in the upper part of the river known as "Sagåna", the rights to which is held jointly by Ola Måland and five other local farmers from Hjelmeland. 

The water in question stems from the \emph{Brokavatn}, located 646 meters above sea level, where altitude soon drops rapidly so that hydro-power is a particularly well-suited form of exploitation for this water. Plans were already in place for making such use of it, from about the altitude of Brokavatn, to the valley in which the original owners' farms are located, at about 80 meters above sea level. In fact, a rough estimate of the potential was originally made by the NVE, and estimated to yield gross annual production of 7.49 GWh pr. annum, about five times more than the water from Brokavatn would contribute to the project proposed by Jørpeland Kraft AS. This estimate was not made in relation to the case, but as part of a national project to survey the remaining energy potential in Norwegian rivers.\footnote{The survey was carried out in 2004, and its results are summarized in \cite{kartlegging}.} \noo{More recent calculations, made by several different experts, acting both on behalf of Jørpeland Kraft AS and original owners, suggests that the water which would be lost would in fact be crucial to the commercial potential of hydro-power for the original owners. Having the water available would take such a project from being somewhat marginal to being a highly profitable endeavor. The owners were not aware of this at the time when the case was being prepared by the water authorities, nor where they informed of this as part of the process.} 

Despite holding the relevant property rights, and despite having considerable commercial interests that would be effected, original owners were not identified as significant stakeholders in the project. Rather, the approach to the case was the traditional one, with focus being directed at the environmental impact, with relevant interests groups being called upon to comment on consequences in this regard, and quite some public debate arising with respect to the balancing of commercial interests and the desire to preserve wildlife and nature.

Nevertheless, one of the owners, Arne Ritland, commented on the proposed project, in an informal letter sent directly to Scana Steel Stavanger AS. In this letter he inquired for further information, and he protested the transferral of water from Brokavatn. He also mentioned the possibility that an alternative hydro-power project could be undertaken by original owners, but he did not go into any details regarding this, stating only that such a locally owned hydro-power plant had previously been in operation in the area. The plant he was referring to dates back to the time before we had a national grid, and was only directed at local supply of electricity. It has since been shut down.

Arne Ritland received a reply stating that more information on the project and its consequences would soon be provided, and he did not pursue the matter further at this time. Meanwhile, Scana Steel Stavanger AS submitted his letter to the water authorities, who in turn presented it to the NVE as a formal comment directed at the application. This prompted Jørpeland Kraft AS to undertake their own survey of alternative hydro-power in Sagåna, and the conclusion, but not the report itself, was sent to the water authorities. The original owners were not informed, and they were not asked to comment on it, or even told that such an investigation of the commercial potential in their waterfalls was being considered by the expropriating party, as a response to Ritland's letter.

Despite being presented with the issue, the water authorities did not take steps to investigate the commercial potential of local hydro power on their own accord. Moreover, the conclusion presented by Jørpeland Kraft AS did not go into details, but merely stated that if the local owners decided to build two hydro-power plants in Sagåna, then one of them, in the upper part of the river, close to Brokavatn, would not be profitable, neither with nor without the water in question. The other project, on the other hand, in the lower part, could still be carried out profitably even after the transferral. No mention was made as to what the original owners actually stood to loose, nor was there any argument given as to why it made sense to build two separate small-scale power plants in Sagåna. In their final report, the NVE handed these findings over to the Ministry, but did not inform the original owners. 

In addition to the report made by Jørpeland Kraft AS themselves, Hjelmeland kommune, the local municipality government, also commented on the possibility of local hydro-power. In their statement to the NVE, they directed attention to the data in the NVE's own national survey, which suggested that a single hydro-power plant in Sagåna would be a highly profitable undertaking. On this basis, they protested the transferral, arguing that original owners should be given the possibility of undertaking such a project. This statement was not communicated to the original owners, and in their final report it was dismissed by the NVE, who stated that the most energy efficient use of the water would be to transfer it and harness it at Jørpeland.

In addition to the statement made by Ritland, one other property owner, Ola Måland, commented on transferral. He did so without having any knowledge of the commercial potential the water held for him and his co-owners, and without having been informed of the statement made by Hjelmeland Kommune. On this basis, he expressed his support for the transferral, citing that the risk of flooding in Sagåna would be reduced. He also phrased his letter in such a way as to suggest he was speaking on behalf of other owners, but he was the only person to sign it. In the final report to the Ministry, the NVE, in their own conclusion, use this as an argument in favor of transferral, stating that the original owners were in favor of it, and that the opinion of Hjelmeland Kommune should therefore not be given any weight. They neglect to mention Arne Ritland's statement in this regard, and earlier in the report, where his statement is referred to along with many others, Ritland is referred to as a private individual, while Ola Måland is referred to as a property owner, and taken to speak on behalf of the others. The report made by the NVE, while it was not communicated to the affected local owners, it was sent to many other stakeholders, including Hjelmeland Kommune. In light of NVE's conclusions, they changed their original position, informing the Ministry that they would not press any further for local hydro-power, since this was not what the original owners wanted themselves. 

While the case was being prepared by the water authorities, the original owners had begun to consider the potential for hydro-power on their own accord, and in late 2006, when the case reached the Ministry, they where not aware that a decision was imminent. Rather, they were under the impression that they would receive further information before the case went further. Still, as they came to realize the commercial value of the water from Brokavatn in their own project, they approached the NVE, inquiring about the status of the plans proposed by Jørpeland Kraft AS. They were subsequently informed that an opinion in support of transferral had already been offered to the Ministry, and that a final decision would soon be made. This communication took place in late November 2006, summarized in minutes from meetings between local owners, dated 21 and 29 of November. On 15 of December 2006, the King in Council granted a concession for Jørpeland Kraft AS to transfer the water from Brokavatn to Jørpeland.

At this point, it was becoming increasingly clear to the original owners that the water from Brokavatn would be crucial to the commercial potential of their own project, and they also retrieved expert opinions suggesting that the NVE was wrong in concluding that transferral would be the most efficient use of the water. In light of this, they decided to question the legality of the transferral, arguing that the decision was invalid.

The license given to Jørpeland Kraft AS was challenged by the original owners on the grounds that the expropriation was materially unjustified, and that the administrative process leading up to the permission to expropriate did not fulfill procedural requirements. The local court, Stavanger Tingrett, held that the original owners were right in protesting the transfer, with the court emphasizing that the preparatory steps taken in cases such as these needed to provide adequate guarantee that the authorities had also considered the fact that the waterfalls could have been exploited commercially by the original owners themselves.\footnote{Stavanger Tingrett 20.05.2009, case nr. 07-185495SKJ-STAV.}

This view was rejected by the regional court, Gulating Lagmannsrett, which held that sufficient steps had been taken to clarify the commercial interests of the owners, and, moreover, that established practice regarding the preparation and evaluation of such cases -- dating from a time when it was not feasible for original owners to undertake hydro-power schemes -- still provided adequate protection.\footnote{Gulating Lagmannsrett 10.01.2011, case nr. 09-138108ASD-GULA/AVD2.} The Supreme Court also held in favor of Jørpeland Kraft AS, and they went even further in stating that established practice was beyond reproach.

In the following section, we present the main legal arguments relied on by the parties, as well as a summary of how the three national courts approached the case, and how they argued for their respective decisions.

\section{The legal arguments, and the view taken by the national courts}\label{view}

The original owners had several arguments in support of their claim that the concession was invalid. Firstly, they argued that procedural mistakes had been made in preparing the case; secondly, they argued that according to Norwegian expropriation law, it was not permissible to expropriate in a situation such as this, when the loss of energy and commercial potential would outweigh the gain to those same interests, which, ostensibly, were the only interests identified in favor of transferral. It seemed to the original owners that expropriation in this case would only serve to benefit the commercial interests of Jørpeland Kraft AS, and that it would do so to the detriment of both local and public interests. For this reason, the owners held that the concession should be regarded as an abuse of power, a manifestly ill-founded decision which could not be upheld.\footnote{There are at least two different ways in which to argue such a point under Norwegian law. One is with respect to water law and general administrative law, whereby clearly ill-founded decisions can be overturned by the courts, even when they involve discretion on part of the executive, which is otherwise not subject to review by the courts. Secondly, an argument can be made with respect to the Norwegian Constitution, Section 105, which gives property a protected status. The former is usually more effective, but in both cases, quite a severe transgression will have to be established before courts consider it within their competence to overturn discretionary decisions. A scholarly examination of these two sets of provisions are given in \cite{Efvl} and \cite{flei} respectively (both in Norwegian).} The owners argued, moreover, that the government had not fulfilled its duty to consider the case with due care, and that the assessment made with respect to the interests of the local community at Hjelmeland, and the local owners residing there, was not adequate. Particular attention was directed at the fact that local owners had not been informed about the progress of the case, and had not been told of, or asked to comment on, those preparatory steps that were being made explicitly with regards to assessing their interests. 

In addition, owners also argued that irrespectively of how the matter stood with respect to national law, the expropriation was unlawful because it would be in breach of the provisions in the ECHR TP1-1 regarding the protection of property.\footnote{European Convention of Human Rights Article 1 of Protocol 1.}\noo{An argument was also made to the effect that expropriation would be in breach of provisions in the EEA agreement regarding unlawful state support for the commercial interests of specific companies.}

Jørpeland Kraft AS protested all these objections to the expropriation, arguing that it was the responsibility of the owners themselves to provide information about possible objections against the project, and that the process had therefore been in accordance with the law. Unfortunate misunderstandings, if any, should be attributed to the fact that original owners had neglected their responsibilities in this regard. Moreover, Jørpeland Kraft AS argued that it was not for the courts to subject the assessment of public and private interests to any further scrutiny, since this was a matter for the government to decide. 

Indeed, according to Norwegian national law, it is traditionally held that unless the exercise of power it clearly unjustified, the courts do not have the authority to overturn decisions based on discretion, unless it can be demonstrated that the government has made procedural mistakes. While this view has become somewhat more relaxed in recent years, with a standard of \emph{reasonableness} increasingly being imposed by courts in similar cases, the inadmissibility of court interference in administrative discretionary decisions is still very much a part of Norwegian national law.\footnote{See \cite{Efvl}, in particular, chapters 24 and 29.}

Finally, Jørpeland Kraft AS argued that there was no issue of human rights at stake in the case. While they argued for this by stating that as the procedural rules had been followed and that the material decision was beyond reproach, they also went far in suggesting that as the owners would be compensated financially by the courts for whatever loss they would incur, no human rights issues could possibly arise in the case. \noo{ They also rejected the view that the case could be seen as an instance of illegitimate state support for Jørpeland Kraft, but failed to provide specific arguments in this regard.}

The matter went before Stavanger Tingrett who gave their judgment on 20 May 2009. In the following, we offer a presentation of the reasons given by this court, leading to the conclusion that the expropriation was unlawful and that the transferral could not be carried out. 

Stavanger Tingrett agreed with the original owners that the decision to grant concession was based on an erroneous account of the relevant facts, and they concluded that it was evident, from the NVE's own figures, that allowing the applicants to use the water from Brokavatn in their own hydro-electric scheme would be the most efficient way of harnessing the potential for hydroelectric production, directly contradicting what the NVE stated in their report. Moreover, they noted that these were the same estimates as those referred to by  Hjelmeland Kommune in their initial objection, and found it to be in breach of procedural rules that this was not considered further by the authorities.

The Court substantiated their decision by giving direct quotes from the report made by the NVE. For instance, in the report, on p. 199, it says, as quoted by Stavanger Tingrett (my translation):
%\begin{quote}Hjelmeland kommune ser helst at kraftressursene i vassdraget blir utnyttet av lokale %grunneiere. 
%Dette står i kontrast til uttalelsen fra grunneierne selv som ønsker at overføring blir gjennomført, 
%slik at flom og erosjonsskader kan bli noe redusert. NVE mener at den beste utnyttelsen med tanke 
%på kraftproduksjon vil være å tillate overføringen da en slik løsning vil innebære at vannet utnittes i 
%størst fallhøyde. Når dette samtidig er grunneiernes eget ønske har vi ikke tillagt Hjelmeland 
%kommunes synspunkt på dette noen vekt
%\end{quote}
%Our own translation follows below: 
\begin{quote}
Hjelmeland kommune would like the hydro-electric potential in the waterfall to be exploited by 
local property owners. This stands in contrast to the statement given by the property owners 
themselves, who wish that the transfer of water takes place, so that damage due to flooding can be 
somewhat reduced. NVE thinks that the best use of the water with respect to hydro-electric 
production is to allow a transfer, since this means that the water can be exploited over the greatest
distance in elevation. When this is also the property owners' own wish, we will not attribute any 
weight to the views of Hjelmeland kommune.
\end{quote}

Stavanger Tingrett concluded that as this was a factually erroneous account of the situation, the decision made to allow transferral of the water could not be upheld. Summing up, the Court offered the following assessment of the case (my translation):

\begin{quote}
It is the opinion of the court, having considered how the case was prepared by the authorities, that the factual basis for the decision made by the government suffers from several significant mistakes and is also incomplete.
\end{quote}

In light of this, Stavanger Tingrett concluded that the decision to grant concession for transfer of water was invalid. As to the legal basis of this, the court relied on the recognized principle of Norwegian public law that while the exercise of discretionary powers is usually not subject to review by court, a decision based on factual mistakes is nevertheless invalid if it can be shown that the mistakes in question were such that they could have affected the outcome. This is not provided for explicitly in statue, but it is one of the core unwritten legal principles of Norwegian public law.\footnote{See \cite{Efvl}}

Concerning the second requirement, that the factual mistakes could have affected the outcome, Stavanger Tingerett found that it was clearly fulfilled in this case since, in fact, the hydro-power suggested by original owners was, based on data available to the government at the time of decision, an objectively speaking \emph{better} use of the resource, even with respect to public interest. In any event, the requirement with regards to factual and procedural mistakes is only that the mistakes \emph{could} have affected the outcome; in the presence of mistakes, the burden of proof is shifted over to the party seeking to defend the decision.

Since Stavanger Tingrett agreed with the original owners that the decision was invalid due to being based on incorrect facts, there was no need to consider further the claims regarding the legitimacy of the decision with respect to human rights law. Stavanger Tingrett did conclude, however, making a more overreaching assessment of the case, that the procedure followed in preparing the case had not taken sufficient regard of owners' interests, and that this was the likely cause of the mistakes that had been made. The Court also argued that the standard of protection for interest of original owners had to interpreted as being more strict now that local hydro-power was an option available to original owners. 

\noo{In this regard, t also seems that Stavanger Tingett found some additional support in its interpretation of Norwegian law that was based on human rights concerns, especially the fact that expropriation, in circumstances such as those of this case, appeared to be a major interference in the rights of owners, and that established practice developed under a different regulatory regime was therefore no longer able to provide adequate protection.}

Jøpeland Kraft AS appealed the decision, and the case then went before the regional court, Gulating Lagmannsrett. They overruled the decision made by Stavanger Tingrett. In their argument, they do not rely on direct assessment of the report made by NVE, nor do they mention the expert statements retrieved by the opposing sides. Instead, they base their decision on general considerations concerning the need for efficient procedures in cases such as these. Such reasoning provides the apparent grounds for making the following rather crucial observation concerning the facts:

\begin{quote}... It was not a mistake to take Ola Måland's statement into consideration, as he was, and still is, a significant property owner. NVE's statement to the effect that granting the concession will facilitate 
a more effective use of the water seems appropriate, as it refers to a current hydro-electric plant that 
exploits a waterfall of 13.5 meters.
\end{quote}

Nowhere in their decision do they mention the statement made by Hjelmeland kommune, nor do they comment on the fact that alternative hydro-power, as suggested by the NVE itself, and pointed to in this statement, amounts to exploiting the waterfall over a difference in altitude of some 550 meters. In fact, the hydroelectric plant that they do mention has nothing to do with Ola Måland and the other owners, but exploits the same water further downstream. It was brought up in the testimony made by a representative from NVE, who, when pressed on the matter, claimed that the reasonable way to interpret the paragraph that Stavanger Tingrett quoted, and to which Gulating Lagmannsrett implicitly refer, was to see it as a statement regarding this hydro- electric plant. In light of the statement provided by Hjelmeland kommune, to which the report explicitly refers, this appears to be a manifestly ill-founded interpretation. But the regional court adopted it, without further comment.

As far as the legal basis of their decision is concerned, it seems that Gulating Lagmannsrett holds, quite generally, that the practice adopted by the water authorities in cases like these still provide adequate protection for original owners, and that it is not for the courts to subject it to critical review. As mentioned, they seem to base their stance in this regard on an overreaching appeal to the need for efficient procedures to deal with cases such as these.

The decision was appealed by Ola Måland and other, and the Norwegian Supreme Court decided to consider the juridical aspects of the case. The appeal concerning the assessment of the facts made by Gulating Lagmannsrett would not be considered, but was to be taken as correct. Since Gulating Lagmannsrett decided to regard as inessential several facts that were seemingly apparent, even from the report made by NVE itself, the appellants presented these facts to the Supreme Court and argued that Stavanger Tingrett was right regarding their consequences. \noo{In addition to this, written statements were retrieved from the Øystein Grundt, the public officer from the NVE that had been responsible for the preparation of the case, and Harald Sollie, }

The Supreme Court ruled in favor of Jørpeland Kraft AS. They comment on the relevant facts on 
p. 9 of their decision. There, they mention that Jørpeland Kraft AS had considered the possibility that a hydro-electric scheme could be undertaken by local property owners. As we mentioned in Section \ref{sum}, a statement was provided to the NVE by Jørpeland Kraft AS themselves -- the parties who stood to benefit from the transferral -- addressing one possible project that was deemed not to be commercially viable. Recall that in the same statement another project was also identified -- in the same river, using the same water -- that they claimed was such a good project that it could be carried out even after the transferral. As we mentioned, the statement does not say anything about what the property owners stand to loose when the water from Brokavatn disappears, and the Supreme Court is also silent on this. Nor do they mention that the statement was never handed over to the applicants, and that the details of the calculations were never handed over to, or considered by, the NVE. In fact, the full report first appeared during the hearing at Gulating Lagmannsrett, but this fact was not considered relevant by the Supreme Court.

Moreover, the Supreme Court remains silent on the fact that the conclusion concerning efficiency of exploitation contradicts both the NVE's own assessment, the statement made by Hjelmeland Kommune, and also all subsequent assessments made both on behalf of the applicants and on behalf of Jørpeland Kraft AS. We mention that all of the above were presented to all national courts, including the Supreme Court.

As to the legal questions raised by the case, the Supreme Court makes a more detailed argument than the regional court, culminating in the conclusion that established practice still provides adequate protection. Interestingly, the Supreme Court base their arguments in this regard on the premise that the case does \emph{not} involve expropriation of waterfalls. A similar sentiment is expressed by Gulating Lagmannsrett, and it was also argued for by Jørpeland Kraft AS, but the true force of this point of view did not become apparent until the case reached the Supreme Court. 

The Court first concludes that a legal basis for the concession to transfer the water is to be found in the Watercourse Regulation Act, Section 16. Moreover, they conclude that while this provision alone does not provide a right to expropriate the waterfall, it does give the applicant a right to divert the water away from it. While the Supreme Court notes that this amounts to an interference in property rights, they take it as an argument in favor of regarding the rules in the Watercourse Regulation Act as the primary source of guidance concerning what should be considered when preparing such cases. The hold, in particular, that the provisions in the Expropriation Act applies only so far as they supplement, and are not in conflict with, the rules of the Watercourse Regulation Act and established practice with respect to the provisions in this Act. Moreover, the main reason they give for this is that the diversion of water is \emph{not} to be considered as an expropriation of a waterfall.

There is, as we mentioned, no rule in the Watercourse Regulation Act which states that the authorities are required to consider specifically the question of how the regulation affects the interests of property owners. Such a rule is found in the Expropriation Act, Section 2, but according to the Supreme Court, it does not apply in cases where water is being diverted away from a river. This is so, according to the Supreme Court, because transferral of water is not regarded as a case of expropriation of a right to the waterfall, but merely an expropriation of a right to deprive the waterfall of water.

This is significant in two ways. First, it is important with respect to the legal status of owners who are affected by projects involving transferral of water. In Norwegian law after Måland, it seems that established practice with respect to the assessment of such cases, focusing on environmental aspects and the positions taken by various interest groups, is beyond reproach already because such cases do not involve expropriation of waterfalls. However, considering that the Norwegian water authorities seem to follow these practices generally, and not just in cases where water is transferred, it remains to be seen if this is a practically significant difference in the level of protection. Is the conclusion regarding the admissibility of current administrative practices supposed to apply only to those cases when water is subject to transferral? If it is, then it leads to the peculiar situation that the level of protection for owners depend solely on the way in which the developer propose to gain control over the water. The difference appears completely arbitrary, however, at least from the point of view of owners. But of course, it will soon cease to be arbitrary for developers, who must be expected to favor gutter projects, collecting water from many small rivers and diverting it, since this mode of exploitation makes it easier to acquire necessary rights. On the other hand, if the Supreme Court is to be understood as saying that traditional practices are adequate in general, the consequences of the decision seem fairly dramatic for local owners. It appears that it is not possible, in cases involving expropriation of waterfalls, to solicit any kind of judicial review, not even in circumstances when the factual basis of the decision is manifestly erroneous, and not even if this appears to be the consequence of the authorities neglecting to keep local owners informed about the assessments made regarding their interests.

To illustrate that a lack of consultation is a general problem, and not confined to the particular case of \emph{Måland}, we will conclude by offering a quote from Harald Solli, director of the Section for Concessions at the Ministry of Petroleum and Energy, who submitted written evidence to the Supreme Court regarding the practices followed in cases involving expropriation of waterfalls. Below, we give one of several exchanges that seem to indicate that under current practices, local owners are left in a rather precarious position (my translation).

\begin{quote}
Q: In cases such as this, should owners affected by a loss of small scale hydro-power potential be kept informed about the factual basis on which the authorities plan to base their decision? I am thinking especially about those cases in which the authorities make an assessment regarding the potential for small scale hydro-power on affected properties. \\
A: Affected owners must look after their own interests. The assessments made by the NVE in their report is a public document, and it can be accessed online through the home page of the NVE.
\end{quote}

By their reasoning in \emph{Måland}, it appears that the Supreme Court gave this dismissive attitude towards local owners a stamp of approval. In light of this, we believe the study of the law in a socio-legal setting becomes all the more relevant. For while this attitude might be a reflection of correct national law, as decided in the final instance by the Supreme Court, it seems pertinent to ask if it is \emph{reasonable} law. Also, it seems that one must ask if a case can not be made with respect to human rights, by arguing that the protection awarded is insufficient in this regard. This point, while it was raised by the original owners in \emph{Måland}, did not receive any separate treatment in the Supreme Court. In the following section, we briefly describe some more questions we think the case raises and which we will address further in subsequent chapters.

\section{Consequences of the case and the questions it raises}\label{cons}

Following \emph{Måland}, it seems we must conclude that the development which has taken place in the energy sector, and has lead to small scale hydro-power becoming profitable and possible for local owners to carry out themselves, does not imply that original owners are entitled to increased participation in decision-making processes under national law. Even if this is the view held by the Norwegian judiciary, we should of course not overlook the possibility that the water authorities themselves will eventually adopt new practices regarding the assessment of such cases. So far, however, it seems that they stick quite closely to the established routine. 

Since the outcome in Norwegian Courts was that established practices were not found to be in breach of principles of Norwegian expropriation law, it seems reasonable to ask instead about the sustainability of these practices. In fact, the case of \emph{Måland} seems to illustrate precisely why the current system is inadequate, and how it can lead to decisions that appear ill-founded and leave the affected communities feeling marginalized. The likelihood of \emph{factual mistakes}, in particular, seems to increase greatly when the involvement of the local population is not ensured in the preparatory stages.

More importantly, it seems that decisions reached following a traditional process can easily lead to takings for which it is difficult to see any legitimate reason why the project proposed by the developer would be a better form of exploitation than allowing the local owners to carry out their own projects. Indeed, in the case of \emph{Måland}, it seemed that small-scale hydro-power would be a better way of harnessing the water in question, even in the sense that it would be more efficient, and would provide the public with more electricity at a lower cost. More generally, unless the issue of alternative exploitation in small scale hydro-power is considered during the assessment made by the water authorities, one risks making decisions that are not in the public interest at all. 

Even worse, it can send out the signal that expropriation of owners' rights is undertaken solely in order to benefit the commercial interests of the energy company applying for a development license. We mentioned in Section \ref{context} that this mechanism, whereby expropriation appears to benefit commercial interests rather than the public, is becoming increasingly important in the international context as well. It is particularly in this regard that we think the case of Norwegian waterfalls warrants attention from the perspective of human rights. At this point, it seems appropriate to recall some concerns expressed by US Justice O'Connor, taken from her dissenting opinion in \emph{Kelo}.

\begin{quote}
Any property may now be taken for the benefit of another private party, but the fallout from this decision will not be random. The beneficiaries are likely to be those citizens with disproportionate influence and power in the political process, including large corporations and development firms. As for the victims, the government now has license to transfer property from those with fewer resources to those with more. The Founders cannot have intended this perverse result.
\end{quote}

In \emph{Kelo}, it seemed that a major point of contention was whether or not these grim predictions did indeed reflect a realistic analysis of the fallout of the decision. Surely, anyone who agrees with Justice O'Connor in her prediction, would also agree with here conclusion that it is perverse. However, whether her pessimism is warranted by empirical fact seems less clear. In this context, we believe the case of Norwegian waterfalls can serve an important broader purpose, as a means towards shedding more light on the hypothesis that a loose interpretation of the public interest requirement will indeed lead to a transfer of property from those with fewer resources to those with more. The \emph{Måland} case, and the current tensions regarding expropriation for the benefit of Norwegian hydro-power, seems to suggest that her concern should indeed be taken seriously. Also, the Norwegian experience seems to show that we need to be clear about the fact that property has a social and political function that goes beyond the financial interests of individuals. For the Norwegian case at least, it seems particularly relevant to ask if local people, by virtue of their right to property and their original attachment to the land, have a legitimate expectation \emph{both} that their commercial interests should be protected, \emph{and} that they should be granted a say in decision-making processes. Financial protection does not necessarily imply social protection, and the right to participate and be heard might be both more significant, and harder won, than the right to be compensated according to whatever the powers that be come to regard as the market value of the property in question.

Another perspective, which we will also pursue further in subsequent chapter, is the question of how property rights relates to the overreaching goal of sustainable development of natural resources. Rather than seeing property rights as a means towards securing sustainable development, it seems more common to see it as an impediment. This, indeed, has shaped much of the Norwegian discourse regarding environmental law and policy, including that which relates to waterfalls.\footnote{For example, such a skeptical view of property rights appear to provide an overriding perspective in \cite{backer1} (in Norwegian), which is a widely used textbook on environmental law in Norway.} Moreover, a typical justification given for interference in property is that an equitable and responsible management of natural resources requires it. It seems to us, however, that an egalitarian system of private ownership of resources -- as we find in Norway for the case of waterfalls -- could itself serve as a sustainable basis for management of these resources. It seems plausible for us to suggest that private property rights is one of the most robust ways in which local communities can be given a degree of self-determination concerning how to manage local resources. This is typically considered desirable also from the point of view of sustainability, but perhaps even more importantly, when property is in the hands of the many rather than the few, is it not also reasonable to expect that the state will be able to more effectively and rationally exercise its regulatory powers? Otherwise, the danger is that the government is being intimidated by large commercial enterprises, perhaps partly owned by the State itself, that command political influence and might not take lightly to what they perceive as undue political interference in their business practices. Such a position might be tenable if you are one of the worlds leading energy companies, but hardly if you are a farmer. 

We think the case of \emph{Måland} suggests that we should investigate these questions in more depth. It seems, in particular, that we must ask about the extent to which commercial companies have succeeded in usurping the notions of sustainable development and public interest, putting the power of these ideas to use in order to secure control over resources and to enlist governmental support, and favorable treatment, for their own commercial undertakings. The extent to which such a mechanism influences the Norwegian energy sector, and the possible implications this might have, both legally and socially, remains to be worked out.

In subsequent chapters, two questions arising from this will receive particular focus. First, we will aim to clarify the importance of the conflict between large scale hydro-power and small scale development by surveying recent and current hydro-power projects in Norway, not in any depth, but by taking note of whether the issue arose. Secondly, we will aim to shed light on the importance of small scale hydro-power to the communities in which local owners reside. As we mentioned, they are usually farmers, and most often in areas were farming is becoming increasingly unprofitable. From the socio-legal point of view it seems highly relevant to ask who the people who loose their resources are, and in what social context we find them. Moreover, while it is clear that hydro-power has become an important source of income in many small and relatively impoverished farming communities, the exact implications of this development, financially and socially, remains to be mapped out.

Following this, it seems natural to return to the legal question of the legitimacy of interference, not from the point of view of national law, but from the point of view of property as a human right. Importantly, it seems to us that property has a clear social dimension, and that mapping out the socio-legal function of specific property rights should inform the judgment we make regarding the level of protection to which owners are entitled. Also, while property is an individual right, it can also be a communal one, and, as such, it can serve to empower local communities that would otherwise be marginalized. The protection of an egalitarian structure of ownership, then, does not appear to be subsumed by, or even conceptually the same as, protecting against individual transgressions. We believe that the case of Norwegian waterfalls demonstrates that this should be kept in mind when analyzing the legitimacy of interference in property for the benefit of commercial undertakings.

\noo{current ownership structure of waterfalls is therefore not simply a question of protecting the commercial interests of individuals who happen to own valuable resources, but also a question of protecting the local communities where these resources are found, giving them the possibility of influencing the way in which the resources are to be harnessed. It seems, however, that local people are often in danger of being seen as an hindrance, both to sustainable development and economic growth, because the commercial companies, along with the environmental interests groups, have claimed this stage as their own. Such, it seems, is the case for Norwegian waterfall. Despite an explosion of interest in small scale hydro-power in recent years, there still seems to be little room left for local communities in the Norwegian discourse concerning hydro-power. It will be an important aim of our work in following chapters to map our in more detail how this influences the law and the administrative policies that are adopted.
}

\section{Conclusion}\label{conc}

As we have shown, \emph{Måland} serves to illustrate many of the current tensions and issues surrounding expropriation of waterfalls in Norway. It also serves to clarify the extent to which local owners are 
marginalized under the regulatory practices currently in place, and shows that the regulatory system does not clearly separate the question of how to judge an application to undertake development from the question of whether or not expropriation should take place. Moreover, the case seems to suggest that this will tend to lead to the emphasis being on issues that have to do with development, while issues relating to expropriation, and owners' interests, will be overlooked. Summing up, the case seems to show that the current regulatory system in Norway functions in such a way that it is bound to give rise to conflicts between local interests and the interests of commercial companies and the State.

The case also sheds new light on the legitimacy of using expropriation in order to benefit commercial interests. In this way, it takes on broader significance, by lending empirical support to the prediction offered by Justice O'Connor with respect to \emph{Kelo}, regarding the fallout of a loose interpretation of the public interest requirement for expropriation.

In our opinion, this contributes to making Norwegian waterfalls an interesting case study on expropriation,  and one that warrants further consideration with respect to human rights. In subsequent chapters, we will offer such an analysis, by addressing the question of whether or not local owners and communities can claim that they are entitled to greater protection than that which is currently provided under Norwegian law.


\chapter{Taking waterfalls}\label{chap:4}

\section{Introduction}\label{sec:intro4}

In this chapter, I address expropriation of waterfalls, starting from a brief overview on the legislation relating to expropriation generally. I then go on to give a chronological presentation of expropriation for hydropower development. The story begins in the late 19th century, when the first statutory authorities for such expropriation began to emerge. Initially, these authorities were very narrow, however, and they did not cover expropriation of waterfalls, only additional land and rights that waterfall owners might need to develop their resource. 

Later, in the early 20th century, the public sector began to expropriate waterfalls for hydropower, but this was only authorised on narrowly defined conditions, to enable the state to provide a new public service: electricity supply. Private companies could not expropriate waterfalls unless they were already majority owners in the local area, a situation that did not change until the passage of the \cite{wra00}. 

Even before this, when hydropower development was still seen as a public service, it was sometimes met with resistance by local people and environmental groups, particularly as the state begun to pursue large-scale projects after World War Two. I discuss the case law that developed in this regard, particularly in relation to procedural rules. I conclude that the public-sector characteristics of hydropower development led the courts to defer very broadly to the discretion of the executive and the legislature, also in relation to the content and scope of provisions in administrative law. 

I then go on to present expropriation of waterfalls under the wide expropriation authority that emerged alongside commercialization of the sector in the early 1990s. For the first time in Norwegian history, waterfalls could now be expropriated for private, commercial, gain. I note that this change in the law was not given much attention by the executive committee that prepared the act. It was described merely as a ``simplification'' of existing rules. Moreover, the legislature did not address the change at all when the Act passed through parliament. In fact, the wording of the \cite{wra00} does not explicitly make clear that private expropriation is now permitted. However, the Act empowers the executive to decide, using directives, what class of legal persons can be given a license to expropriate waterfalls. Such a directive has been issued, with little or no debate, granting the possibility to benefit from an expropriation for hydropower development to ``anyone''.

I present the law relating to expropriation of waterfalls in some depth, before I go on to consider more concretely how the procedure plays out in the context of for-profit takings. Here I anchor the presentation in the recent Supreme Court case of {\it Måland}, where the issue of procedrual legitimacy arose after a commercial company was granted permission to deprive local owners of waterfall rights that the owners wished to make use of in their own hydropower project.\footcite{jorpeland11}

I use {\it Måland} to illustrate that the owners' standing under administrative law is extraordinarily weak, particularly compared to the magnitude of their interests in typical hydropower cases. Moreover, I argue that {\it Måland} shows how the Supmpre Court adheres to a narrow perspective on the meaning of property protection, taking it to be an issue that begins and ends with the question of compensation. This, I argue, fails to do justice to the issue at stake, which, more than anything else, concerns the democratic legitimacy of the Norwegian regulatory system for managing water resources. As this system has now become depoliticised, expert-oriented and market-dominated, the continued annihilation of local property rights threatens to render it entirely subservient to the interests of the most powerful market actors.
%
%
%
%
%g
%
%In this chapter, we address some recent demands made by local people, to the effect that they should be allowed to partake more actively in decision-making processes regarding local waterfalls. We focus on the legal status of such demands under Norwegian law, and we do so by considering the recent Supreme Court Case of \emph{Ola Måland and others v. Jørpeland Kraft AS}.\footnote{Ola M{\aa}land and others v. J{\o}rpeland Kraft AS, Rt 2011 s. 1393. I mention that I represented the local owners in this case, as a trainee lawyer in the district and regional courts, and as the responsible lawyer before the Supreme Court.} In this case, the local owners protested the legality of a license that granted the developer, Jørpeland Kraft AS, a right to divert water away from their waterfalls, thereby reducing the potential for local hydro-power. The owners argued that consultation had been insufficient, and they also contended that the assessment of the case made by the water authorities had been inadequate, and that the decision has been based on an erroneous account of the facts. They won the case in the district Court, Stavanger Tingrett, but lost under appeal to the regional Court, Gulating Lagmannsrett. The Norwegian Supreme Court also found in favor of the developer, and the argument they gave to support this conclusion goes far in suggesting that while the commercial interests of local owners need to be compensated, the presence of such interests do not entitle local communities to a greater say in decision-making processes. Importantly, the Court held that the presence of local interests does not necessitate the adoption of different administrative practices, and that the procedures developed during the decades of direct, non-commercial, administration of the energy sector, could still be followed.
%
%The case is significant, since the traditional hydro-electric scheme in Norway typically involves expropriation, often interfering with the property rights of hundreds of local individuals. Traditionally, owners of waterfalls would be compensated according to a standardized mathematical method that was based on the assumption that they had no interest in hydro-power themselves.\footnote{The method consists in calculating the number of \emph{natural horsepowers} in the waterfall, and then multiplying this number by a price pr. natural horsepower, determined by the discretion of the Court, but in practice based almost solely on what has been awarded in previous cases. For a description of the traditional method, we point to \cite{falk} Chapter 7, page 521-522 (in Norwegian).} In a landmark case from 2008, however, the Norwegian Supreme Court commented, in an \emph{obiter dicta}, that the traditional method for calculating compensation for waterfalls was no longer appropriate, at least not in cases when it can be demonstrated that the original owners could have exploited the resource themselves, if the expropriation had not taken place.\footnote{The case of Agder Energi Produksjon AS vs. Magne Møllen, Rt. 2008 s. 81. The local owner lost the case, the reason being that the Supreme Court held that compensation should not be based on the present day value of the waterfall, but the value it had when the original transferral of rights took place, in the 1960's. The \emph{obiter dicta} has been used as an authority for subsequent Supreme Court decisions, however, see, for instance, Rt. 2010 s. 1056 and Rt. 2011 s. 1683. It has received quite a lot of scholarly attention as well, see \cite{Tf1,Tf2,Tf3}.} This decision has had a profound impact on the level of compensation awarded for waterfall rights, leading to payments that can will typically be ten to a hundred times higher than that which would have been awarded according to the traditional method.\footnote{So far, in cases that have come before the court, there has been about a twenty-fold increase in compensation, see \cite{Tf1}, but the new method will, when applied to certain kinds of projects (cheap to build, and involving little or not regulation of the water-flow), result in compensation having to be paid that amounts to at least a hundred times more than what could be expected if the traditional method had been applied.} More generally, it also served to shift the balance of power in favor of local owners and their communities, who increasingly expect to have their voices heard, and to get a more direct say over how their energy resources are managed and exploited.
%
%After \emph{Måland}, however, it has become unclear to what extent the presence of local, commercial interests will continue to influence the Norwegian energy sector, and if we will see more active participation by local people in the future. In fact, recent statements made by the Norwegian water authorities seem to suggest that this is becoming increasingly unlikely, as a shift in policy seems to have taken place, whereby local, small scale projects, are now to be given lower priority than large scale projects undertaken by established energy companies.\footnote{These statements were not linked to \emph{Måland}, but were made in a more general context, ostensibly motivated by the desire to increase the efficiency of the administrative process, see http://www.nve.no/no/Konsesjoner/Vannkraft/Smaakraft/ where the new policy was announced. It also received some attention from the press, see, for instance, http://www.tu.no/energi/2012/01/18/nve-varsler-flere-smakraft-avslag.}
%
%Taking a broader view on Norwegian law, we believe that recent experiences regarding hydro-power provides an interesting case to study, and one that will shed light on how property rights function in a social, economic and political context. It seems, in particular, that the view of property rights to waterfalls adopted by the Supreme Court rests on a narrow interpretation, seeing such rights merely as bestowing financial interests on certain individuals. This was clearly felt in \emph{Måland}, and we think the case also serves to illuminate certain consequences of such a view, suggesting, in particular, that it can have detrimental social and political consequences, and can very easily lead to perceived injustices. We also think it is pertinent to ask if the narrow view of property which seems to have been adopted for waterfalls in Norway is adequate with respect to human rights law, or if the right to property should also be considered a right to participate, and a right to be heard, in decision-making processes.
%
%In the following, we first give the reader some further background on Norwegian hydro-power, and then 
%we present \emph{Måland} in some detail, focusing on giving the reader an impression of current administrative practices, by describing how they played out in this particular case, and by detailing how they came to result in a decision that the original owners felt to be fundamentally unjust. We also address the legal arguments given by the opposing sides and the arguments relied upon by the national courts. We conclude by presenting some overreaching issues that we believe the case raises, regarding both the social context of property rights, the content of property as human right, and the question of whether or not the protection awarded under Norwegian law currently meets the standard set by the European Convention of Human Rights, as interpreted by the Court in Strasbourg. 

\section{Norwegian Expropriation Law: A Brief Overview}

As mentioned in Chapter \ref{chap:2}, the right to property is entrenched in s 105 of the Norwegian Constitution. There, it is made clear that when property is taken for public use, full compensation is to be paid to the owner. As I have already mentioned, the public use requirement is understood very widely, or not regarded as a requirement at all. However, it is a rule of unwritten constitutional law that administrative decisions which affect the rights of individuals can only be carried out when they are positively authorised by law. Moreover, s 105 is not regarded as an authority to expropriate. Hence, the administrative branch must rely on other provisions, usually provided for in acts of parliament, that authorises compulsory acquisition of property on specific terms.  

Historically, there was no general act relating to expropriation, and a range of different acts provided the necessary authority to expropriate for specific purposes, such as roads, public buildings, and schools. Today, many of these authorities have been collected, broadened, and included in the \cite{ea59}.\footnote{Act no 3 of 23 October 1959 Relating to Expropriation of Real Property.} Still, many specific authorities remain, such as s 16 of the \cite{wra17}, which gives an automatic right to expropriate to the holder of a watercourse regulation license.

Following the \cite{wra00}, the general authority used to expropriate waterfalls has been included in the general act on expropriation.\footcite[2 no 51]{ea59}. Here it is stated that expropriation may take place in order to facilitate ``hydropower production''. In addition, it is made clear that expropriation can only be authorised if the benefits clearly outweigh the harms.  Notice that this sets expropriation licenses apart from the various hydropower licenses discussed in Sections \ref{wra00}-\ref{ea90} of Chapter \ref{chap:3}. In relation to the latter, in particular, the benefit is required to outweigh the harm, but it need not be ascertained that this is {\it clearly} the case. However, the practical significance of this difference is limited.\footnote{...}

The authorising authority is the King in Council. However, the Act also makes clear that this authority can be delegated further, to ministries or other state bodies that the King in Council may instruct.\footnote{See \cite[5]{ea59}.} The compensation to the owner is determined following a judicial procedure administered by the special  appraisement courts.\footnote{\cite[2]{ea59}.} 
The \cite{ea59} states that unless the Kind in Council decides otherwise, expropriation orders may only be granted to state or municipality bodies. This is formulated as a limiting principle, but in effect it serves as a general authorisation for the executive to decide, without parliamentary involvement, what class of legal persons may be granted expropriation licenses. 

For many of purposes, directives have been issued that extend the class of possible beneficiaries to any legal person, including companies operating for profit. Such a directive has been issued, in particular, for the authority to expropriate in favour of hydropower production.\footnote{See Directive no 391 of 06 April 2001.} In addition to providing a general authority for expropriation, the \cite{ea59} also contains several procedural rules. These are collected in Chapter 3 of the Act. Section 11 gives minimal requirements for what an application for an expropriation license must include, stating that it should make clear who will be affected, how the property is to be used, and what the purpose of acquisition is. In addition, the section requires the applicant to specify what land will actually be acquired, and to include information about the type of land in question and the current use that is made of it.

An obligation to give notice to affected owners follows from the second paragraph of section 12. The starting point is that every owner is to be given individual notice, although this obligation falls away when it is ` unreasonable difficult'' to fulfill. In this is found to be the case, it is sufficient that the documents of the case are made available at a suitable place in the local area. A public announcement must also be made in the official notification publication of the government, as well as in two widely read local newspapers.

The first paragraph of section 12 sets out a general obligation on part of the licensing authority to ensure that the facts of the case are clarified to the ``greatest extent possible''.\footnote{The Norwegian expression is ``best råd er'', which literally means ``best possible way''.} This formulation seems to establish a principle that is stronger than the general duty to duly assess cases involving rights and responsibilities of individuals, a duty that follows from general administrative law. However, the exact meaning of the phrase  ``greatest possible extent'' can be hard to pin down. In fact, administrative practice from several fields, including the hydropower sector, suggests that special attention is rarely devoted to the expropriation issue, particularly not when the expropriation license is granted to implement a state-sanctioned plan for the use of the land in question. I return to this issue in more depth in Section \ref{sec:paa67} when I discuss expropriation in light of administrative law, and in Section \ref{sec:ola} when I relate the discussion specifically to the case of waterfall expropriation.

The last rule of section 12, expressed in the third paragraph, states that a decision to grant an expropriation license must be accompanied by reasons that are submitted to the parties, in accordance with general administrative law. This rule is largely superfluous, as the obligation to give reasons would in most cases also follow independently from administrative law, c.f., Section \ref{sec:paa67}.

According to section 15, the costs incurred by owners in relation to a pending application for expropriation against them is to be covered by the applicant. The exact formulation is that the applicant is obliged to cover the costs that ``the rules in this chapter carry with them''. That is, the applicant is obliged to cover the costs that are related to the owners' rights pursuant to Chapter 3 of the \cite{ea59}. What this actually means is unclear, and in practice an applicant will be denied costs if the competent authority takes the view that they are unreasonable or disproportionate to his interests in the case.\footnote{If the case progresses to an appraisement dispute, the competent authority to decide on costs is the appraisement court. Otherwise, the decision is left with the executive.}

Particularly problematic are cases for which there is no clear division between those aspects of the case that relate to expropriation and those that relate to other licenses or land use planning more generally. This is the situation, for instance, in relation to hydropower development. In such cases, it is unusual for local owners to get any significant coverage of costs relating to the application processing. Legal expenses, for instance, are rarely covered unless they are incurred in relation to a subsequent appraisement dispute. This can be a problem for owners that wish to resist expropriation. Obviously, it is crucial for them to voice convincing objections already at the application processing stage.

In addition to the procedural rules in the \cite{ea59}, many rules of administrative law apply in expropriation cases. In the next section, I give an overview of the most important ones.

\subsection{The Public Administration Act}\label{sec:paa67}

After WW2, public administration in Norway underwent a reform whereby administrative bodies came to be placed more directly under centralized political control. At the same time, the established system based on legal expertise and strict adherence to the letter of the law was replaced by a form of management that actively sought to pursue political goals. As a result, the ambit of administrative decision-making power widened significantly. Many new administrative bodies were set up, while many of those already established were empowered greatly by statutory authorities that only specified their purpose and competence in broad strokes. The new style of legislation that developed often left great room for the exercise of administrative discretion, which, the argument went, was reasonable all the while the level of direct control exercised by the central government had been increased.\footnote{See generally \cite{grønli04}.}

However, these reform processes eventually led to concerns about the lack of formal safeguards for those individuals and groups of people that were targeted by administrative decisions. such safeguards would have been largely superfluous so long as the competence of administrative bodies was narrowly drawn up and expressly limited by the authorising statute. But as administrative bodies became increasingly sharpened as instruments of political decision-making, critical voices began to warn against the dangers of an unrestrained public administration free to implement political decisions without much scrutiny and public debate. Particularly worrying was the fact that administrative bodies were often empowered to implement such decisions directly against specific individuals, without having to bring out their general consequences, or justify them as matters of general policy.

In response to these worries, a general statute was proposed that would set out some minimum standards of due process for administrative decision-making process. This proposal eventually became the \cite{paa67}.\footnote{Act no 86 of 10 February 1967 Relating to Procedure in Cases Concerning the Public Administration.} This Act sets out the fundamental procedural principles that the executive is meant to follow when preparing to make administrative decisions. Some rules apply to any such decision, but a particularly important class of rules apply specifically to so-called {\it individual decisions}, which affect the rights and responsibilities of one or more specific persons.\footcite[2]{paa67} These persons are then referred to as parties to the decision. Clearly, owners of property covered by an expropriation license fall into this category, so that owners are parties to the individual decision to grant such a license.

Many of the rules in the \cite{paa67} mirror those of the \cite{ea59}, although they tend to include more detailed, albeit less strict, formulations. Section 16 stipulates that advance notice is required to all those affected by an individual decision. As was the case for expropriation, a possible exception is granted if it is practically unfeasible to reach the parties. However, section 16 also specifies in more depth what the notice must contain. In the second paragraph, it stated that ``the advance notification shall explain the nature of the case, and otherwise contain such information as is considered necessary to enable the party to protect his interests in a proper manner''. Hence, it is not enough simply to inform the party, the Act also explicitly stipulates that the notice has to meet a minimum quality standard. In relation to expropriation of waterfalls this takes on special significance, since, as I discussed in Chapter \ref{chap:3}, it is established practice in such cases fro the expropriating party to send out this notice, with no involvement of the water authorities. 

In section 17, the duty to clarify cases is expressed, mirroring the rules in section 12 of the \cite{ea59}. The formulation is similarly imprecise, as it declare that cases are to be ``clarified as thoroughly as possible'' before a decision is made. Importantly, however, section 17 also includes specific rules that oblige the authorities to inform parties about information they retrieve during their assessment of the case, and to submit such information for comments in so far as the party must be assumed to have an interest in it.\footnote{See the 2 and 3 paragraphs of \cite[17]{paa67}.}

In section 24, the duty to give grounds for the decision is expressed. It applies to most individual decision, with some narrowly defined exceptions concerning cases when no party can be assumed to be dissatisfied, or when giving grounds would involve disclosing information to which the party is not entitled. Moreover, the King is authorised to limit the duty to give grounds when ``special circumstances so require''. All these exceptions are unusual, and hardly ever apply to hydropower cases.

In section 25, requirements are given concerning the content of grounds given for decisions. It is stipulated that the grounds should mention the relevant rules authorising the decision, the factual assessment the underlies it, as well as the main considerations that have been decisive for the use of discretionary power. 

In some cases, the complication of the matter at hand may be such that some parties are ill-equipped to look after their interests, even if the safeguards mentioned above are respected. This can be the case, for instance, in hydropower cases, as many waterfall owners must be expected to not possess the technical, commercial, and legal knowledge necessary to realize the meaning and value of their ownership. In section 11, a more recent amendment of the \cite{paa67}, a general rule of guidance is given, stipulating that the administration is obliged to provide guidance to parties so that they may look after their interests in the ``best possible way''. Again, the formulation is vague, and it is explicitly stated that the level of guidance must be adapted to the circumstances and the capacity that the agency has for offering such assistance. However, in the second paragraph it is stated that the agency must assess, on their own motion, the parties' need for guidance.

As demonstrated in this and the previous section, expropriation law and general administrative law imposes a range of procedural rules that must be followed when deciding on an application for a license to expropriate. In principle these apply also when waterfalls are expropriated, but as there are special rules that regulate the procedure followed in hydropower cases, the question becomes how these rules relate to each other. Also, the practical question is to what extent the water authorities interpret these rules in concrete cases, and whether they actually observe the more general rules regarding expropriation alongside the rules that target the licensing applications under water law. I return to this issue by giving an in-depth study of a concrete dispute in Section \ref{sec:ola}. First, I elaborate a little on the expropriation rules found in the law relating to hydropower, and its relationship, at the theoretical level, with the rules discussed above.

\section{Special Rules for Waterfalls}\label{sec:special}

As I mentioned in Chapter \ref{chap:3}, Section \ref{sec:wra17}, section 16 of the \cite{wra17} established an automatic right to expropriate rights needed to undertake a watercourse regulation. This is not understood to include a right to expropriate waterfalls needed for the hydropower development. However, in a recent decision, it was made clear that it does include a right to transfer water away from a river course for development somewhere else. This is a {\it de facto} license to expropriate a waterfall, as the water disappears form the river in which the owners have rights. It is also recognized as such in terms of compensation, which is paid for the waterfalls in cases like this, as they loose their value. However, it is still not considered as expropriation of the waterfalls themselves, but only of the right to deprive them of value.

Hence, section 16 \cite{wra17} entitles the license holder to a form of regulatory taking of the waterfalls of the owners in the river system where the water disappears. The status of such a takings after \cite{måland11} is in effect half-way between regulation and expropriation. The right to compensation is recognized, but the procedural and substantive rules that otherwise apply to expropriation of waterfalls do not apply. In particular, section 16 alone is sufficient authority for this kind of taking. The question arises about the extent to which the rules in the \cite{ea59} applies in such cases. 

First, this questions arises because the rules there are often regarded as expressing general principles of expropriation law. Second, it arises specifically in relation to section 30, number 2, which states that the rules apply to expropriation pursuant to the \cite{wra17}, in so far as they are ``suitable'' and do not ``contradict'' the special rules given in that Act. As we will see in Section \ref{sec:ola}, the established practice is to regard the procedural rules in the \cite{wra17} as exhaustive, and in keeping with the procedural rules in the \cite{ea59}. In addition, the strengthened assessment requirement in section 2, which stipulates that expropriation must``undoubtedly'' be of more benefit than harm, is not considered to have any independent significance alongside the assessment criterion of section 8 in the \cite{wra17}, which does not include any such formulation.

However, there is no doubt that the rules of the \cite{paa67} apply to takings of water rights pursuant to the \cite{wra17}. Moreover, there is no doubt that when a separate expropriation license is sought for waterfalls, these rules, as well as the rules in \cite{ea59} both apply. In practice, they nevertheless play a minimal role when the water authorities assess cases, as the assessment is unified, and the focus remains on balancing environmental and energy interest.

Hence, the broader question is how the practices adopted by the water authorities hold up against the requirements of administrative and expropriation law. This issue was not specifically addressed in \cite{måland11}, but the Supreme Court made some comments that can be taken to imply that they find no fault with current practices. I will shed more light on that they consist in over the course of the following sections. 

First, it is important to note that the reform of the energy sector means that expropriation of waterfalls takes place in a different context today than it did when many of the current practices developed. The value of precedent set during the time of the energy monopoly may be of limited value. In any event, it needs to be understood as a reflection of its time. This means that it is natural to structuring the presentation of expropriation practices chronologically, by dealing first with the period prior to the reform implemented by the \cite{ea90}. I now pursue this approach.

\section{Taking Waterfalls for Progress}\label{sec:twp}

Historically, Norwegian law admitted no general authority for the state to expropriate waterfalls, neither on its own behalf nor on behalf of private parties. However, there were a range of special provisions that authorized the state to appropriate water for specific purposes, but the criteria were typically quite narrow. For instance, the Water Resources Act 1887 authorized expropriation for the purpose of drinking water, but not for use in industry. Moreover, the purpose of expropriation was largely understood to be binding also on the future use, so that the taker would not gain unrestricted control over the rights he acquired, but were obliged to use them in accordance with the authorised purpose.

While there were no circumstances in which private parties could expropriate waterfalls for industrial development, but private owners of waterfalls could obtain licenses to expropriate surrounding land needed to exploit waterfalls that they already owned. In addition, a new right had been granted through the Water Resources Act 1887, giving the owners of waterfalls a right to engage in various industrial exploits, even if these would damage other landowners, for instance through deprivation of water or flooding. These rules are highly similar to many of the rules found in the so-called mill acts from the US, that I discussed in Chapter \ref{chap:2}, Section \ref{sec:mill}. Some of them could even theoretically have the effect of a {\it de facto} expropriation of waterfalls, but such cases were relatively rare.

An important reason for this was that expropriation law in general was based on the principle that eminent domain should not be exercised when the interests of the expropriating party were of the same kind as the interests of the owner. This applied regardless of whether or not the owner, subjectively speaking, were likely to pursue those interests in an optimal way. The principle was guiding for expropriation law until the early 20th century, and it applied regardless of whether the taker was public or private. It meant, for instance, that expropriation of waterfalls for the purpose of hydropower was ruled out already as a matter of principle. In particular, as the regulatory system of the day made private hydropower development possible, no owner could be deprived of rights to 
a waterfall by any hydropower developer, private or public. 

However, as the industrial advances meant that the interest in hydropower exploded in the late 19th century, the state increasingly came to see it as a political priority to secure that waterfalls were used in the public interest. The most important expression of this came in the form of the licensing acts presented in Chapter \ref{chap:3}, Sections \ref{sec:wra17} and \ref{sec:ica17}. However, during the same time, parliament passed legislation that authorised expropriation of waterfalls to the benefit of public bodies for the purpose of hydropower development.\footnote{Legislation that made it possible to expropriate waterfalls to the benefit of the municipalities was introduced in 1911, and a similar authority that authorised expropriation in favour of the state appeared in 1917, see \cite[29]{amundsen..}}

In 1940, these authorities were consolidated and integrated in the general water resources legislation, through the Water Systems Act 1940 (replaced by the \cite{wra00}). Still, the authority to expropriate waterfalls applied only to the state and to the municipalities, for the latter on the explicit condition that the purpose of expropriation was for ``general electricity supply in the district''. 

Hence, the required public purpose of expropriation was explicitly stipulated in the authority, and expropriation licenses could not be granted to private or commercial entities. Expropriation during this time therefore had a clear public character; in so far as the letter of the law was respected, little doubt could be raised that expropriation was indeed taking place in the public interest. Moreover, the public had to benefit directly, and locally. Economic development in itself was not regarded as a sufficiently public purpose to justify expropriation.

In addition, the fact that the energy sector was organized as a regional monopoly under direct political control meant that it was hard to contend, as a matter of fact, that expropriation was not an expression of the public will. At the same time, however, there were severe political conflicts over hydropower, conflicts that could then be meaningfully addressed within the framework of a politically managed electricity sector. Flaws in this system emerged, however, particularly as the state began to aggressively pursue hydropower development for economic development. 

This still took the form of a public undertaking, but as the scale of development grew massively following WW2, hydropower became increasingly politically sensitive. The democratic legitimacy of development with respect to the local communities that were affected also often seemed very weak. The decision-making authority was completely centralized, the benefit would tend to accrue in urban areas, but the negative effects were almost entirely contained in the local rural communities.

In practice, during this time the limitation to general electricity supply also became less important in practice. In particular, the interpretation of the supply requirement was relaxed significantly over the years, especially following the development of the national electricity grid and the liberalization of the energy sector in the early 1990s. It was no longer obvious, from a technical point of view, when exactly a hydropower development could be said to qualify as making a contribution to the local electricity supply. The electricity was not necessarily used locally, but, indirectly, also the local supply situation might be said to improve.

However, the rule that private parties could not expropriate waterfalls was enforced, and it remained in place until the executive passed the directive mentioned in Section \ref{sec:ea59}, in the year 2000. Only then, some 90 years after the introduction of a general expropriation right for the state and the municipalities, did it become possible for arbitrary commercial entities to acquire waterfalls compulsorily for hydropower.

In light of this, the vast majority of cases dealing with waterfall expropriation under Norwegian law can not be looked at as pure economic development takings. Certainly, the desire for economic development played a crucial part in motivating state and municipality development of hydropower. But their activities in this regard were not themselves commercial in nature. Rather, supplying electricity was regarded as a public service, one that would in turn stimulate commercial activity in other areas of the economy. However, the issue of the extent to which state could legitimately interfere with the rights of waterfall owners still arose. It was often contested, in particular, who the true beneficiaries were, particularly in relation to large-scale developments that would benefit communities far removed from those in which the water resources were found. In addition, particularly in the early 20th century, there was a general feeling of unease about how far the state could go in regulating and monopolizing the hydropower sector without thereby depriving the owners of waterfalls of constitutionally protected rights.

This debate culminated in the conflict surrounding the rule of reversion that was introduced by the licensing acts passed between 1906 and 1917. As mentioned, the rule of reversion meant that in order to sell a waterfall to a private development, the owners and the purchaser had to apply for a license that was only ever granted on the condition that after some number of years, at most 60, the state would acquire the waterfalls without paying compensation. The question that arose was whether this was merely a regulation of the permitted use of waterfalls, or whether it should be regarded as expropriation, so that compensation would have to be paid pursuant to section 105 of the Constitution.

The conflict over this issue became fierce, with many influential conservatives, including legal scholars, attacking the rule of reversion as a ploy by the state to confiscate Norwegian waterfalls without paying compensation to the owners.\footnote{Morgenstiærne.} However, in a 4-3 decision, the Supreme Court eventually held that section 105 did not apply, since reversion was merely a licensing condition, not an act of expropriation. No owner was compelled to hand over his property to the state, or to sell it to a private party so that the state would eventually acquire it.

One of the judges voting with the majority summed up their view by commenting that he would not regard it as expropriation if the state were to forbid sale of waterfalls to private parties altogether. Why then, he asked, should it be regarded as expropriation if such a sale was allowed to take place only on specific conditions? Against this, the minority argued that the licensing requirement as such was so severe that it had to be regarded as a {\it de facto} expropriation that entitled the owners to compensation. Moreover,  as the purpose was clearly to ensure that waterfalls were eventually brought under state ownership, the minority did not think is was appropriate to consider reversion merely as a regulation of use.

After the decision by the Supreme Court in the reversion case, the legal foundation for the hydropower monopoly solidified. The development of this monopoly happened gradually, however, and expropriation on a large scale did not take place until after WW2. At this time, the state became to involve itself greater in hydropower projects, and it typically pursued very large-scale projects. This caused a new period of controversy, mainly motivated by environmental concerns. However, the interest of local people also featured strongly in this debate. Moreover, as the regulatory system was beyond reproach at this point, the local interest were typically aligned with the environmental interests. Large-scale hydropower projects, in particular, would tend to cause nuisance, or even significant loss, to traditional forms of agriculture. Therefore, in a situation when local owners could not themselves benefit significantly from hydropower, their rational response was to oppose it.

The patterns of conflict that emerged during this time converged in the case of \cite{alta8.}. In this case, there was an added complication: The local people all lacked formal title. This was because the development that was being planned would take place in the northernmost part of Norway, in the native land of the Sami people. Norway has a history of discrimination against the Sami, and as their culture is largely nomadic, their land rights were never formalized in the law. As a result, almost the entire northern region of Finnmark is owned by the state. Despite lacking title to the land, the Sami have continued to struggle for their rights to use the land, particularly for their nomadic form of reindeer farming, with an extensive additional reliance on fishing and hunting.

The plans to develop large-scale hydropower in a Sami area therefore raised particularly strong criticism. Particularly significant was the fact that the opposition to the plans brought together environmental groups and groups fighting for aboriginal rights. A broad political mass movement was mobilized in opposition to the plans, eventually resulting in several serious cases of civil disobedience, including what might today well be classified as ``terrorism''.\footnote{In particular, there were several instances when local people blew up equipment that was meant to be used to construct the hydropower plant. In one famous episode, the person behind the bombing miscalculated, resulting in the loss of his own arm. Apart from this, however, the protests were relatively peaceful.}
The case was also dealt with by the court, as the Sami interests claimed, primarily on the basis of administrative law, that the development licenses that had been granted for the development were invalid. 

At first sight, the case is not particularly relevant to the question of expropriation. However, as the Norwegian regulatory system focuses on the development issue, with little or no separate attention paid to the issue of expropriation, the case has in fact been highly significant to the owners of waterfalls. It effectively serves as the primary measuring stick with which the executive and the courts assess the level of substantive and procedural protection that local people are entitled to.

Due to the controversy surrounding the case, it was admitted directly from the District Court to the Supreme Court in plenum. The presiding judge commented that as far as he knew it was the longest and most extensive civil case that the Court had ever heard.\footcite[254]{alta82} In an opinion totalling 138 pages, the Court argues that the decision to grant the license is valid. The opinion deals mostly with procedural rules. The substantive arguments, and arguments relating to international law, were not subjected to much scrutiny, as the Court express strong confidence in the opinion that no objection could be raised against the licenses on such grounds. 

However, in addition to arguing against the decision on this basis, the opponents of the development had pointed out a very wide range of purported shortcomings of the decision-making process. This aspect of the case was considered in great depth by the courts, leading also to a further elucidation of the procedural rules of the \cite{wra17} and the application of general administrative law to hydropower cases.

It was clear that the original licensing application did not meet the requirements stipulated in section 5 of the \cite{wra17}. Essentially, the original application contained little more than the technical details about the planned development, with little or no identification or assessment of deleterious effects on other interests, neither private nor public. This shortcoming, moreover, had been acknowledge by the water authorities themselves, who had nevertheless initiated a public hearing, citing an electricity deficit in the northern part of Norway. 

The Supreme Court concluded that this was ``clearly unfortunate''.\footcite[265]{alta82} However, several reports and assessments had subsequently been provided by the water authorities, to fill the gaps left open by the initial application. The Supreme Court held that this might well serve to make the initial mistakes irrelevant to the validity of the licenses, as it was the licensing process as a whole that should be assessed against the procedural rules. Hence, shortcomings of specific stages in the assessment would not be given weight it they did not imbue the process with a dubious character overall.\footcite[265]{alta82}

The question then turned to the question of whether the process as a whole fulfilled procedural requirements. This turned largely on 
the extent to which the various assessments that had been made in the case adequately served to clarify the case, in accordance with section .... of the \cite{wra17} and section 16 of the \cite{paa67}. 

In this regard, the local people objecting the development pointed to a range of negative effects that they believed had not been considered, or had not been considered in enough depth. In relation to nomadic reindeer interests, for instance, it was argued that the water authorities had failed to adequately consider the indirect consequences of development. These effects were described as ``catastrophic'' by an expert testimony presented to the Court. By contrast, the water authorities had based their decision on assessments that did not place much weight on indirect consequences, citing the difficult involved in attempting to quantify such effects. 

After considering the reports and assessments in some depth, the Supreme Court did not find fault with the procedure in this regard. Importantly, the Court stresses that the water authorities were aware of the possibility of indirect negative consequences, but simply chose, as a matter of expert discretion, not to place much weight on such consequences. This, moreover, was construed as an expression of disagreement with those claiming (later) that the effects would be catastrophic. As a result, the grounds for claiming procedural error disappeared, as the lack of attention directed at indirect consequences was held to reflect administrative discretion that could not be made subject to judicial review.

More generally, the Court's opinion on this point reflects how indeterminate the distinction between administrative discretion and procedure can become. Importantly, the Court makes statements of principle in this regard, that serve to limit the scope of judicial review under procedural rules in hydropower cases. In particular, the Court concludes that many of the relevant procedural rules in such cases by their very nature tend to be largely ``discretionary' '. As the licensing decision itself is a discretionary one, the argument goes, it is appropriate to admit to the executive a wide discretionary authority to decide for themselves also how to interpret many of the admittedly rather vague procedural requirements of administrative law. By contrast, the view taken by the appellants, based on the idea that the content and scope of such rules is a purely judicial question, is described by the Court as ``overly formalistic''.

The Court makes a second statement of principle, namely that the scope of assessment required for the purposes of reaching a licensing decision is not in any event as extensive as the level of assessment that is required in a subsequent appraisement dispute. Hence, the meaning of the obligation to clarify cases to the best possible extent is put into perspective: assessments of deleterious effects may be omitted at the executive's discretion even in circumstances when such assessments are practically relevant to the level of compensation payable and {\it will} in fact be provided at a later stage.

More concretely, in relation to the negative effects on fishing, the {\it Alta} Court conceded that the assessments could have been better, but pointed out that the purpose of assessment was only to answer yes or no to development, not to give a detailed presentation of its effects.\footcite[330]{alta82}. Crucially, the Court goes on to note that in so far as mistakes are uncovered as a result of insufficient assessment, this will influence the compensation payments and can also motivate subsequent regulation.\footcite[330]{alta82} In effect, the risk of error is downplayed by making reference to the compensation right and the regulatory authority of the state. This echoes the dichotomy mentioned in Chapter \ref{chap:3}, whereby there is a tendency in Norwegian law to perceive the interests of affected owners in purely monetary terms, while the state is assumed to be the sole protector of social and environmental values attached to property.

In relation to some negative effects of the {\it Alta} development, it was made apparent that they had not been considered at all. In addition, it was clear that erroneous information had been used in relation to some issues, particularly regarding alternative ways to meet the need for electricity in Finnmark and Norway as a whole, as well as the extent of this need. The Supreme Court agreed that this was a flaw, but held that it did not imply invalidity of the license. In this regard, a third statement of principle was made. The Court held, in particular, that the duty to consider alternatives -- different ways in which the public purpose could be satisfied -- was very limited.

It was regarded as sufficient to mention some alternative courses of action, without making them subject to any detailed assessment. This position of principle, in turn, was used by the Court to argue that the errors in the information provided about alternative were unlikely to have affected the outcome of the case.\footcite[346]{alta82} Apparently, alternatives had in fact been assessed in greater depth than what was necessary and for that reason, factual errors were held to be of little consequence. In fact, erroneous information about alternatives had been handed over to the Storting itself, who had approved the plans on three separate occasions, but always under reference to the precarious electricity situation in Finnmark.

Here the Court established a principle whereby the nature of possible alternatives is considered a marginal issue in relation to assessment of licensing applications, regardless of what the political decision-makers emphasised when {\it they} considered the matter. In {\it Alta}, for instance, the Court's perspective appears to have been at odds with how the Storting approached the case. There is little doubt, in particular, that the favourable political assessment of the plans depended heavily on the perceived electricity crisis in Finnmark and the supply situation in Norway generally, as well as the perceived inadequacies of alternative solutions.

In relation to this question, the legal council acting for the state in {\it Alta} suggested explicitly that as these aspects came into focus only at the political stage of the decision-making, they were largely irrelevant to the legal issues that the case raised.\footcite[341]{alta82} This line of argument strikes me as particularly disturbing, since the decision to grant the license was in fact very much a political one. The information gathered by the water authorities, in particular, was put to use in a largely political decision-making context. In light of this, it seems that the procedural rules were, if anything, {\it more} important to observe in so far as they pertained to the quality of the factual basis that would be used in subsequent political assessments.

The Supreme Court clearly did not approach the matter from this angle, but how exactly it reasoned in this regard is not clear from the opinion. In fact, it is very noticeable how briefly the Court comments on this particular issue compared to other aspects of the case. By contrast, it goes into great detail about purported weaknesses of the licensing procedure that seem relatively minor comparatively speaking. As a result, the Court's assessment that the mistakes were not relevant to the outcome of the case do not come as a surprise. However, in relation to the duty to assess alternatives and necessity criterion, the Court says nothing expect that the duty is very limited. For the details, which demonstrate factual inadequacies in the material given to the political decision-makers, the Court only refers briefly to the state's arguments. These arguments, based on the contention that the inadequacies were not significant, is accepted with no further discussion.\footcite[346]{alta82}

The dismissive attitude towards the duty to correctly assess alternatives is a controversial aspect of the {\it Alta}-decision which has been criticized by legal scholars. Today, maintaining such a dismissive attitude becomes particularly problematic also with respect to owners' rights. Indeed, alternatives are no longer limited to other public projects that can potentially provide the same public service at a smaller environmental and social cost. Instead, the most important alternatives now typically consist in owner-led projects proposed in commercial competition to the applicant's commercial project. In so far as the duty to assess these alternatives is construed as loosely as the duty to assess alternatives was construed in {\it Alta}, it will hardly be reassuring for those owners of waterfalls that oppose commercial development projects based on their own hydropower plans. 

In the next two sections, we will see that so far, no adjustments have been made to the way the water authorities approach this issue. Moreover, the dismissive attitude to this question in {\it Alta} has been upheld in a recent Supreme Court decision involving a concrete owner-led alternative regarding which the NVE had provided manifestly erroneous information to the Ministry.

\section{Taking Waterfalls for Profit}

As I mentioned in the previous section, private companies could not expropriate waterfalls in Norway prior to the passage of the \cite{wra00}. Moreover, the public purpose requirement was enforced strictly by the authorising statute, particularly in cases when the development was undertaken by municipality companies. I also mentioned how the hydropower sector developed after WW2 from a sector dominated by local municipality companies, to a sector dominated by the state. This, in turn, was accompanied by increased conflicts and doubts regarding the legitimacy of the established licensing procedures, particularly the highly centralized nature of the decision-making process. 

Even so, the debate at this time was still very much anchored in a system that presupposed political management of the hydropower sector as a public service provider. Importantly, the conflicts rarely, if ever, involved significant commercial interests on the part of the local waterfall owners. Many critics voiced arguments to the effect that the fiscal interest of the state motivated wanton destruction of nature and local patterns of land use, including commercial uses. But in financial terms, these interest were typically negligible compared to the scale of the hydropower development. 

As a result, controversies relating to the legitimacy of interference involved only the waterfall rights at their periphery. More focused conflicts involving waterfalls specifically arose in relation to the question of compensation, but the issues typically discussed in this regard were also of relatively minor structural importance, although they could of course be important enough for the individuals directly affected.

In Chapter \ref{chap:3}, I presented the reform of the energy sector of the early 1990s, after which hydropower development has been regarded as a commercial pursuit. Following the regulatory reform, a new general statute dealing with water resources was also proposed, eventually leading to the passage of the \cite{wra00}. This Act provided the first every authority for the state to allow developers to take waterfalls compulsorily for profit. Moreover, it made possible the later executive directive by which waterfalls could be expropriated and handed over to {\it any} legal person, including private companies.

The combination of the legal and regulatory reforms mean that today, takings of waterfalls for hydropower are takings for profit. But this change in the function of expropriation received little attention when these reforms were introduced. When the \cite{wra00} was proposed, the increased scope of expropriation was not singled out for political consideration by the MoPE when it handed the case over to the Storting. In the legislative proposal handed over to the Storting, the new expropriation authority for waterfalls is described merely as a ``simplification'' of older law. 

As the discussion above shows, this is a hardly accurate. However, the commission appointed by the Ministry to prepare the Act also adopted a very low-key approach to expropriation. The commission mentioned that its proposals would imply increased scope for expropriation, but it did  not discussed the desirability of this in any depth. In particular, it did not related its proposals in this regard to the recent market -based reform of the energy sector.  The report from the commission, totaling almost 500 pages, devote only three pages to the proposed ``simplification'' of the expropriation authority.\footcite[235-237]{nou94}

First, the commission notes that a range of different authorities for expropriation co-exist in the law, with many of them positing strict and concrete public interest requirements as a precondition for granting a license. This, the commission argues, is not a very ``pedagogical'' way of providing expropriation authorities. Moreover, the commission notes that it runs the risk of omitting important purposes for which expropriation should be possible. Hence, the commission proposes to replace all older authorities by a sweeping authority that makes expropriation possible for any project that involves ``measures in water courses''.  

The commission comments that their formulation might seem wide, but remark that this is not a problem since the executive can simply deny giving an expropriation license in so far as they regard expropriation as undesirable. The commission does not reflect on the  constitutional consequences of such a perspective, neither in relation to property rights nor in relation to the balance of power between the legislature and the executive. The commission does offer a very brief presentation of the rationale behind dropping the local supply restriction for municipal expropriation, by remarking that these rules complicate the law and might make desirable expropriations impossible.\footcite[235]{nou94}  But the commission do not clarify what kind of desirable expropriations it thinks might be left out. In particular, it does not relate its proposals to the recent market-based reform of the energy sector. Hence, the obvious practical consequences of their proposal, namely that expropriation will be made available as a profit-making mechanism for commercial companies, is not discussed or critically assessed.

The issue of {\it who} should be permitted to benefit from an expropriation license is also dealt with very superficially. In this regard, the commission structures their presentation around the so-called redemption rule that was introduced in \cite{wra40}. Recall that this rule made it possible for a majority owner of a waterfall to compulsorily acquire minority rights, if this was necessary to facilitate hydropower development. Hence, it was a rule that provided only a limited opportunity for private takings, restricted to local owners themselves or external developers that had been able to strike a deal with a locally based majority. 

The main justification given by the commission for introducing a general private takings authority is that the special redemption rule had not been much used in practice. Why this is an argument in favour of extending private expropriation rights is not made clear. Indeed, it seems just as natural to regard it as an argument {\it against} doing so. Why extend the possibility for private expropriation if the demand for such expropriation has been limited in past? Presumably, the commission thought there would be a demand for such expropriation in the future, but this is not stated explicitly, nor is the appropriateness of meeting such a demand discussed. As to the requirement that private takers must already control a majority of the waterfall rights in the local area, the commission only remarks that it regards such a restriction as old-fashioned.\footcite[236]{nou94} No discussion is offered regarding the consequences for local waterfall owners at a time when the energy sector was also being reformed according to market principles.

Since the passage of the \cite{wra00}, it has become clear that the new authority for expropriation has been one of the most practically significant, and controversial, aspects of the Act. During the last 14 years, an unprecedented number of cases has raised the issue of legitimacy of expropriation of waterfalls. Today, practically all cases of expropriation imply that local owners are deprived of the development potential in favour of a commercial actor seeking development of the same kind. According to the law before the \cite{wra00}, expropriation of this kind would not be easy to justify against the relevant authorities. In so far as the beneficiary was a private company, it would not be possible at all. 

In \cite{sauda08}, this issue came into focus, as the waterfall owners protested a license that granted a private company the right to expropriate their waterfalls. Here, the owner's principal argument was that the executive directive granting such rights to private parties was in fact invalid, since it had not been sanctioned by the Storting. Formally speaking, this argument was very weak, since the \cite{ea59} had been amended in such a way that the executive was in fact authorized to determine the class of legal persons that could be granted an expropriation license to pursue hydropower. However, the owners argued that the executive had not appropriately informed the Storting that this would be the consequence of the amendment, which had been passed as a formality following the adoption of the \cite{wra00}. 

The owners pointed to interviews with two of the members of the parliamentary committee that had prepared the case for the Storting, noting that neither of them could recollect that they were even aware that a right to expropriate for private developers would result from the Act they had passed. This, as noted earlier, was not conveyed to them by the executive. Moreover, it was not explicitly stated anywhere in the Act that had been passed. Rather, it followed implicitly from three different sections in two separate Acts that the executive would be empowered to issue such a directive. This, the owners argued, meant that the purported authority was not in fact constitutionally valid.

This argument was rejected, but the level of compensation paid for the waterfall rights was dramatically increased compared to earlier practice, c.f., Chapter \ ref{chap:5}. Because of this, the development company appealed the decision to the Supreme Court, with the owners lodging a counter-appeal regarding the question of legitimacy. The Supreme Court decided not to hear the case, however, as it had recently addressed the compensation question from a similar angle in the paradigmatic case of \cite{møllen08}.

In addition to raising the issue of constitutional legitimacy of the new expropriation authority, the owners in \cite{sauda08} also raised several procedural objections against the expropriation license. This line of argument also proved unsuccessful, but it foreshadows the later case of \cite{måland13}, where the owners were initially successful in arguing that the procedures developed in the takings for progress era were no longer appropriate. This decision was overturned on appeal, however, a decision that was in turn upheld by the Supreme Court, in a decision relying on the precedent set by \cite{alta82}. This case is very well suited to bringing out how administrative practices relating to expropriation function in the context of commercial development projects were the owners have competing plans. It also illustrate common grievances raised by local owners, as well as serving to highlight the response to these grievances by the water authorities and the judiciary.

\section{Ola Måland v Jørpeland Kraft AS}

The expropriating party in \cite{måland13} is Jørpeland Kraft AS, a company jointly owned by Scana Steel Stavanger AS, who owns 1/3 of the shares, and Lyse Kraft AS, who is the majority shareholder holding the remaining shares. The former is a steelworks company located in the small town of Jørpeland in Rogaland county, southwestern Norway. Historically, this company has been a major employer in Jørpeland, which is located by the sea, next to a mountainous area. The main source of energy for the steel industry in Norway has been hydro-power, and Scana Steel Stavanger AS is no exception. The company uses energy harnessed from the rivers in the area, and while the primary river runs through the town of Jørpeland itself, it is supplemented by water from other rivers in the area that are diverted so that they can be exploited more efficiently along with the water from the Jørpeland river.

Recently, Norwegian steel companies have become less profitable, due in great part to increased foreign competition and a significant increase in cost of operation associated with this type of industry in Norway, particularly salary costs.\footnote{For a reference on this, see \emph{Information Booklet about Norwegian Trade and Industry}, published by the Ministry of Trade and Industry in 2005.} This has led to many such companies shifting their attention away from labor-intensive steel production, and focusing instead on producing electricity, selling it directly on the national grid. Jørpeland Kraft AS was established as part of such a move being made with regards to the energy resources in Jørpeland, and the role played by Lyse Kraft AS is an important one. As we mentioned, Norwegian law favors companies where the majority of the shares are held by public bodies, and Lyse Kraft AS, being publicly owned, with the city of Stavanger as the main shareholder, is therefore a valuable partner. Moreover, Lyse Kraft AS, while being a commercial company, is also responsible for the electricity grid in the region. It was established as a merger between several local monopoly companies in the Stavanger region which were reorganized following liberalizaion of the sector in the early 1990's. As discussed in Section \ref{context}, there is little doubt that old monopolists still enjoy considerable power and influence.\footnote{In fact, Lyse Kraft AS is good example suggesting that their power might in some cases have \emph{increased}. Since liberalization, the restraints imposed both by the non-commercial nature of former monopolists, and the local, political, anchoring of such companies, have disappeared.} This is another reason why they can serve as valuable partners for private companies wishing to make a profit from Norwegian hydro-power.

With attention shifting from harnessing rivers for the purpose of industrial production to the purpose of producing electricity to sell on the national grid, the main variables that determines the profitability of the undertaking also changes. On the cost side, what matters becomes only the cost of producing the electricity itself, and this is typically determined, for the most part, by the investments required for the original construction works.\footnote{For an overview of the considerations made when assessing the commercial value of small scale hydro-power, we point to \cite{kartlegging}. In fact, due to the importance that small scale hydro-power has assumed in recent years, investigating models for investing in such projects has become an active field of research in Norway, see for instance \cite{investment}.} Running and maintaining a hydro-power station tends to be comparatively inexpensive. On the income side, what matters is the price of energy on the electricity market, a market that is no longer anchored in the local conditions of supply and demand.

Importantly, as long as energy production is the sole focus, the business no longer depends in any significant way on the local labor force, and as a result, it is typical that large scale exploitation becomes much more profitable, compared to the medium or small scale power plants typically needed to facilitate local industrial exploits. Hence, it was in keeping with a general trend in Norway when Jørpeland Kraft AS, following their shift in commercial strategy, proposed to undertake measures to increase their energy output. This could be achieved relatively cheaply, by further constructions aimed at channeling water from nearby waterfalls into dams that were already built to collect the water from the Jørpeland river.

One relatively small waterfall from which Jørpeland Kraft AS suggested to extract water was owned by Ola Måland and five other local farmers. This waterfall is not located in Jørpeland kommune, and does not reach the sea at Jørpeland, but runs through the neighboring municipality of Hjelmeland, on the other side of a mountain range, until it eventually reaches the sea at Tau, another neighboring municipality. The plans to divert this water would deprive original owners of water along some 15 km of riverbed, all the way from the mountains on the border between Hjelmeland and Jørpeland, to the sea at Tau. Far from all the water would be removed, but the water-flow would be greatly reduced in the upper part of the river known as "Sagåna", the rights to which is held jointly by Ola Måland and five other local farmers from Hjelmeland. 

The water in question stems from the \emph{Brokavatn}, located 646 meters above sea level, where altitude soon drops rapidly so that hydro-power is a particularly well-suited form of exploitation for this water. Plans were already in place for making such use of it, from about the altitude of Brokavatn, to the valley in which the original owners' farms are located, at about 80 meters above sea level. In fact, a rough estimate of the potential was originally made by the NVE, and estimated to yield gross annual production of 7.49 GWh per annum, about five times more than the water from Brokavatn would contribute to the project proposed by Jørpeland Kraft AS. This estimate was not made in relation to the case, but as part of a national project to survey the remaining energy potential in Norwegian rivers.\footnote{The survey was carried out in 2004, and its results are summarized in \cite{kartlegging}.} %\noo{More recent calculations, made by several different experts, acting both on behalf of Jørpeland Kraft AS and original owners, suggests that the water which would be lost would in fact be crucial to the commercial potential of hydro-power for the original owners. Having the water available would take such a project from being somewhat marginal to being a highly profitable endeavor. The owners were not aware of this at the time when the case was being prepared by the water authorities, nor where they informed of this as part of the process.} 

Despite holding the relevant property rights, and despite having considerable commercial interests that would be effected, original owners were not identified as significant stakeholders in the project. Rather, the approach to the case was the traditional one, with focus being directed at the environmental impact, with relevant interests groups being called upon to comment on consequences in this regard, and quite some public debate arising with respect to the balancing of commercial interests and the desire to preserve wildlife and nature.

Nevertheless, one of the owners, Arne Ritland, commented on the proposed project, in an informal letter sent directly to Scana Steel Stavanger AS. In this letter he inquired for further information, and he protested the transferral of water from Brokavatn. He also mentioned the possibility that an alternative hydro-power project could be undertaken by original owners, but he did not go into any details regarding this, stating only that such a locally owned hydro-power plant had previously been in operation in the area. The plant he was referring to dates back to the time before we had a national grid, and was only directed at local supply of electricity. It has since been shut down.

Arne Ritland received a reply stating that more information on the project and its consequences would soon be provided, and he did not pursue the matter further at this time. Meanwhile, Scana Steel Stavanger AS submitted his letter to the water authorities, who in turn presented it to the NVE as a formal comment directed at the application. This prompted Jørpeland Kraft AS to undertake their own survey of alternative hydro-power in Sagåna, and the conclusion, but not the report itself, was sent to the water authorities. The original owners were not informed, and they were not asked to comment on it, or even told that such an investigation of the commercial potential in their waterfalls was being conducted by the expropriating party, as a response to Ritland's letter.

Despite being presented with the issue, the water authorities did not take steps to investigate the commercial potential of local hydro power on their own accord. Moreover, the conclusion presented by Jørpeland Kraft AS did not go into details, but merely stated that if the local owners decided to build two hydro-power plants in Sagåna, then one of them, in the upper part of the river, close to Brokavatn, would not be profitable, neither with nor without the water in question. The other project, on the other hand, in the lower part, could still be carried out profitably even after the transferral. No mention was made as to what the original owners actually stood to loose, nor was there any argument given as to why it made sense to build two separate small-scale power plants in Sagåna. In their final report, the NVE handed these findings over to the Ministry, but did not inform the original owners. 

In addition to the report made by Jørpeland Kraft AS themselves, Hjelmeland kommune, the local municipality government, also commented on the possibility of local hydro-power. In their statement to the NVE, they directed attention to the data in the NVE's own national survey, which suggested that a single hydro-power plant in Sagåna would be a highly profitable undertaking. On this basis, they protested the transferral, arguing that original owners should be given the possibility of undertaking such a project. This statement was not communicated to the original owners, and in their final report it was dismissed by the NVE, who stated that the most energy efficient use of the water would be to transfer it and harness it at Jørpeland.

In addition to the statement made by Ritland, one other property owner, Ola Måland, commented on transferral. He did so without having any knowledge of the commercial potential the water held for him and his co-owners, and without having been informed of the statement made by Hjelmeland Kommune. On this basis, he expressed his support for the transferral, citing that the risk of flooding in Sagåna would be reduced. He also phrased his letter in such a way as to suggest he was speaking on behalf of other owners, but he was the only person to sign it. In the final report to the Ministry, the NVE, in their own conclusion, use this as an argument in favor of transferral, stating that the original owners were in favor of it, and that the opinion of Hjelmeland Kommune should therefore not be given any weight. They neglect to mention Arne Ritland's statement in this regard, and earlier in the report, where his statement is referred to along with many others, Ritland is referred to as a private individual, while Ola Måland is referred to as a property owner, and taken to speak on behalf of the others. The report made by the NVE, while it was not communicated to the affected local owners, it was sent to many other stakeholders, including Hjelmeland Kommune. In light of NVE's conclusions, they changed their original position, informing the Ministry that they would not press any further for local hydro-power, since this was not what the original owners wanted themselves. 

While the case was being prepared by the water authorities, the original owners had begun to consider the potential for hydro-power on their own accord, and in late 2006, when the case reached the Ministry, they where not aware that a decision was imminent. Rather, they were under the impression that they would receive further information before the case went further. Still, as they came to realize the commercial value of the water from Brokavatn in their own project, they approached the NVE, inquiring about the status of the plans proposed by Jørpeland Kraft AS. They were subsequently informed that an opinion in support of transferral had already been offered to the Ministry, and that a final decision would soon be made. This communication took place in late November 2006, summarized in minutes from meetings between local owners, dated 21 and 29 of November. On 15 of December 2006, the King in Council granted a concession for Jørpeland Kraft AS to transfer the water from Brokavatn to Jørpeland.

At this point, it was becoming increasingly clear to the original owners that the water from Brokavatn would be crucial to the commercial potential of their own project, and they also retrieved expert opinions suggesting that the NVE was wrong in concluding that transferral would be the most efficient use of the water. In light of this, they decided to question the legality of the transferral, arguing that the decision was invalid.

The license given to Jørpeland Kraft AS was challenged by the original owners on the grounds that the expropriation was materially unjustified, and that the administrative process leading up to the permission to expropriate did not fulfill procedural requirements. The local court, Stavanger Tingrett, held that the original owners were right in protesting the transfer, with the court emphasizing that the preparatory steps taken in cases such as these needed to provide adequate guarantee that the authorities had also considered the fact that the waterfalls could have been exploited commercially by the original owners themselves.\footnote{Stavanger Tingrett 20.05.2009, case nr. 07-185495SKJ-STAV.}

This view was rejected by the regional court, Gulating Lagmannsrett, which held that sufficient steps had been taken to clarify the commercial interests of the owners, and, moreover, that established practice regarding the preparation and evaluation of such cases -- dating from a time when it was not feasible for original owners to undertake hydro-power schemes -- still provided adequate protection.\footnote{Gulating Lagmannsrett 10.01.2011, case nr. 09-138108ASD-GULA/AVD2.} The Supreme Court also held in favor of Jørpeland Kraft AS, and they went even further in stating that established practice was beyond reproach.

In the following section, we present the main legal arguments relied on by the parties, as well as a summary of how the three national courts approached the case, and how they argued for their respective decisions.

\subsection{The legal arguments, and the view taken by the national courts}\label{view}

The original owners had several arguments in support of their claim that the concession was invalid. Firstly, they argued that procedural mistakes had been made in preparing the case; secondly, they argued that according to Norwegian expropriation law, it was not permissible to expropriate in a situation such as this, when the loss of energy and commercial potential would outweigh the gain to those same interests, which, ostensibly, were the only interests identified in favor of transferral. It seemed to the original owners that expropriation in this case would only serve to benefit the commercial interests of Jørpeland Kraft AS, and that it would do so to the detriment of both local and public interests. For this reason, the owners held that the concession should be regarded as an abuse of power, a manifestly ill-founded decision which could not be upheld.\footnote{There are at least two different ways in which to argue such a point under Norwegian law. One is with respect to water law and general administrative law, whereby clearly ill-founded decisions can be overturned by the courts, even when they involve discretion on part of the executive, which is otherwise not subject to review by the courts. Secondly, an argument can be made with respect to the Norwegian Constitution, Section 105, which gives property a protected status. The former is usually more effective, but in both cases, quite a severe transgression will have to be established before courts consider it within their competence to overturn discretionary decisions. A scholarly examination of these two sets of provisions are given in \cite{Efvl} and \cite{flei} respectively (both in Norwegian).} The owners argued, moreover, that the government had not fulfilled its duty to consider the case with due care, and that the assessment made with respect to the interests of the local community at Hjelmeland, and the local owners residing there, was not adequate. Particular attention was directed at the fact that local owners had not been informed about the progress of the case, and had not been told of, or asked to comment on, those preparatory steps that were being made explicitly with regards to assessing their interests. 

In addition, owners also argued that irrespectively of how the matter stood with respect to national law, the expropriation was unlawful because it would be in breach of the provisions in the ECHR TP1-1 regarding the protection of property.\footnote{European Convention of Human Rights Article 1 of Protocol 1.}\noo{An argument was also made to the effect that expropriation would be in breach of provisions in the EEA agreement regarding unlawful state support for the commercial interests of specific companies.}

Jørpeland Kraft AS protested all these objections to the expropriation, arguing that it was the responsibility of the owners themselves to provide information about possible objections against the project, and that the process had therefore been in accordance with the law. Unfortunate misunderstandings, if any, should be attributed to the fact that original owners had neglected their responsibilities in this regard. Moreover, Jørpeland Kraft AS argued that it was not for the courts to subject the assessment of public and private interests to any further scrutiny, since this was a matter for the government to decide. 

Indeed, according to Norwegian national law, it is traditionally held that unless the exercise of power it clearly unjustified, the courts do not have the authority to overturn decisions based on discretion, unless it can be demonstrated that the government has made procedural mistakes. While this view has become somewhat more relaxed in recent years, with a standard of \emph{reasonableness} increasingly being imposed by courts in similar cases, the inadmissibility of court interference in administrative discretionary decisions is still very much a part of Norwegian national law.\footnote{See \cite{Efvl}, in particular, chapters 24 and 29.}

Finally, Jørpeland Kraft AS argued that there was no issue of human rights at stake in the case. While they argued for this by stating that as the procedural rules had been followed and that the material decision was beyond reproach, they also went far in suggesting that as the owners would be compensated financially by the courts for whatever loss they would incur, no human rights issues could possibly arise in the case. \noo{ They also rejected the view that the case could be seen as an instance of illegitimate state support for Jørpeland Kraft, but failed to provide specific arguments in this regard.}

The matter went before Stavanger Tingrett who gave their judgment on 20 May 2009. In the following, we offer a presentation of the reasons given by this court, leading to the conclusion that the expropriation was unlawful and that the transferral could not be carried out. 

Stavanger Tingrett agreed with the original owners that the decision to grant concession was based on an erroneous account of the relevant facts, and they concluded that it was evident, from the NVE's own figures, that allowing the applicants to use the water from Brokavatn in their own hydro-electric scheme would be the most efficient way of harnessing the potential for hydroelectric production, directly contradicting what the NVE stated in their report. Moreover, they noted that these were the same estimates as those referred to by  Hjelmeland Kommune in their initial objection, and found it to be in breach of procedural rules that this was not considered further by the authorities.

The Court substantiated their decision by giving direct quotes from the report made by the NVE. For instance, in the report, on p. 199, it says, as quoted by Stavanger Tingrett (my translation):
%\begin{quote}Hjelmeland kommune ser helst at kraftressursene i vassdraget blir utnyttet av lokale %grunneiere. 
%Dette står i kontrast til uttalelsen fra grunneierne selv som ønsker at overføring blir gjennomført, 
%slik at flom og erosjonsskader kan bli noe redusert. NVE mener at den beste utnyttelsen med tanke 
%på kraftproduksjon vil være å tillate overføringen da en slik løsning vil innebære at vannet utnittes i 
%størst fallhøyde. Når dette samtidig er grunneiernes eget ønske har vi ikke tillagt Hjelmeland 
%kommunes synspunkt på dette noen vekt
%\end{quote}
%Our own translation follows below: 
\begin{quote}
Hjelmeland kommune would like the hydro-electric potential in the waterfall to be exploited by 
local property owners. This stands in contrast to the statement given by the property owners 
themselves, who wish that the transfer of water takes place, so that damage due to flooding can be 
somewhat reduced. NVE thinks that the best use of the water with respect to hydro-electric 
production is to allow a transfer, since this means that the water can be exploited over the greatest
distance in elevation. When this is also the property owners' own wish, we will not attribute any 
weight to the views of Hjelmeland kommune.
\end{quote}

Stavanger Tingrett concluded that as this was a factually erroneous account of the situation, the decision made to allow transferral of the water could not be upheld. Summing up, the Court offered the following assessment of the case (my translation):

\begin{quote}
It is the opinion of the court, having considered how the case was prepared by the authorities, that the factual basis for the decision made by the government suffers from several significant mistakes and is also incomplete.
\end{quote}

In light of this, Stavanger Tingrett concluded that the decision to grant concession for transfer of water was invalid. As to the legal basis of this, the court relied on the recognized principle of Norwegian public law that while the exercise of discretionary powers is usually not subject to review by court, a decision based on factual mistakes is nevertheless invalid if it can be shown that the mistakes in question were such that they could have affected the outcome. This is not provided for explicitly in statue, but it is one of the core unwritten legal principles of Norwegian public law.\footnote{See \cite{Efvl}}

Concerning the second requirement, that the factual mistakes could have affected the outcome, Stavanger Tingerett found that it was clearly fulfilled in this case since, in fact, the hydro-power suggested by original owners was, based on data available to the government at the time of decision, an objectively speaking \emph{better} use of the resource, even with respect to public interest. In any event, the requirement with regards to factual and procedural mistakes is only that the mistakes \emph{could} have affected the outcome; in the presence of mistakes, the burden of proof is shifted over to the party seeking to defend the decision.

Since Stavanger Tingrett agreed with the original owners that the decision was invalid due to being based on incorrect facts, there was no need to consider further the claims regarding the legitimacy of the decision with respect to human rights law. Stavanger Tingrett did conclude, however, making a more overreaching assessment of the case, that the procedure followed in preparing the case had not taken sufficient regard of owners' interests, and that this was the likely cause of the mistakes that had been made. The Court also argued that the standard of protection for interest of original owners had to interpreted as being more strict now that local hydro-power was an option available to original owners. 

\noo{In this regard, t also seems that Stavanger Tingett found some additional support in its interpretation of Norwegian law that was based on human rights concerns, especially the fact that expropriation, in circumstances such as those of this case, appeared to be a major interference in the rights of owners, and that established practice developed under a different regulatory regime was therefore no longer able to provide adequate protection.}

Jøpeland Kraft AS appealed the decision, and the case then went before the regional court, Gulating Lagmannsrett. They overruled the decision made by Stavanger Tingrett. In their argument, they do not rely on direct assessment of the report made by NVE, nor do they mention the expert statements retrieved by the opposing sides. Instead, they base their decision on general considerations concerning the need for efficient procedures in cases such as these. Such reasoning provides the apparent grounds for making the following rather crucial observation concerning the facts:

\begin{quote}... It was not a mistake to take Ola Måland's statement into consideration, as he was, and still is, a significant property owner. NVE's statement to the effect that granting the concession will facilitate 
a more effective use of the water seems appropriate, as it refers to a current hydro-electric plant that 
exploits a waterfall of 13.5 meters.
\end{quote}

Nowhere in their decision do they mention the statement made by Hjelmeland kommune, nor do they comment on the fact that alternative hydro-power, as suggested by the NVE itself, and pointed to in this statement, amounts to exploiting the waterfall over a difference in altitude of some 550 meters. In fact, the hydroelectric plant that they do mention has nothing to do with Ola Måland and the other owners, but exploits the same water further downstream. It was brought up in the testimony made by a representative from NVE, who, when pressed on the matter, claimed that the reasonable way to interpret the paragraph that Stavanger Tingrett quoted, and to which Gulating Lagmannsrett implicitly refer, was to see it as a statement regarding this hydro- electric plant. In light of the statement provided by Hjelmeland kommune, to which the report explicitly refers, this appears to be a manifestly ill-founded interpretation. But the regional court adopted it, without further comment.

As far as the legal basis of their decision is concerned, it seems that Gulating Lagmannsrett holds, quite generally, that the practice adopted by the water authorities in cases like these still provide adequate protection for original owners, and that it is not for the courts to subject it to critical review. As mentioned, they seem to base their stance in this regard on an overreaching appeal to the need for efficient procedures to deal with cases such as these.

The decision was appealed by Ola Måland and other, and the Norwegian Supreme Court decided to consider the juridical aspects of the case. The appeal concerning the assessment of the facts made by Gulating Lagmannsrett would not be considered, but was to be taken as correct. Since Gulating Lagmannsrett decided to regard as inessential several facts that were seemingly apparent, even from the report made by NVE itself, the appellants presented these facts to the Supreme Court and argued that Stavanger Tingrett was right regarding their consequences. \noo{In addition to this, written statements were retrieved from the Øystein Grundt, the public officer from the NVE that had been responsible for the preparation of the case, and Harald Sollie, }

The Supreme Court ruled in favor of Jørpeland Kraft AS. They comment on the relevant facts on 
p. 9 of their decision. There, they mention that Jørpeland Kraft AS had considered the possibility that a hydro-electric scheme could be undertaken by local property owners. As we mentioned in Section \ref{sum}, a statement was provided to the NVE by Jørpeland Kraft AS themselves -- the parties who stood to benefit from the transferral -- addressing one possible project that was deemed not to be commercially viable. Recall that in the same statement another project was also identified -- in the same river, using the same water -- that they claimed was such a good project that it could be carried out even after the transferral. As we mentioned, the statement does not say anything about what the property owners stand to loose when the water from Brokavatn disappears, and the Supreme Court is also silent on this. Nor do they mention that the statement was never handed over to the applicants, and that the details of the calculations were never handed over to, or considered by, the NVE. In fact, the full report first appeared during the hearing at Gulating Lagmannsrett, but this fact was not considered relevant by the Supreme Court.

Moreover, the Supreme Court remains silent on the fact that the conclusion concerning efficiency of exploitation contradicts both the NVE's own assessment, the statement made by Hjelmeland Kommune, and also all subsequent assessments made both on behalf of the applicants and on behalf of Jørpeland Kraft AS. We mention that all of the above were presented to all national courts, including the Supreme Court.

As to the legal questions raised by the case, the Supreme Court makes a more detailed argument than the regional court, culminating in the conclusion that established practice still provides adequate protection. Interestingly, the Supreme Court base their arguments in this regard on the premise that the case does \emph{not} involve expropriation of waterfalls. A similar sentiment is expressed by Gulating Lagmannsrett, and it was also argued for by Jørpeland Kraft AS, but the true force of this point of view did not become apparent until the case reached the Supreme Court. 

The Court first concludes that a legal basis for the concession to transfer the water is to be found in the Watercourse Regulation Act, Section 16. Moreover, they conclude that while this provision alone does not provide a right to expropriate the waterfall, it does give the applicant a right to divert the water away from it. While the Supreme Court notes that this amounts to an interference in property rights, they take it as an argument in favor of regarding the rules in the Watercourse Regulation Act as the primary source of guidance concerning what should be considered when preparing such cases. The hold, in particular, that the provisions in the Expropriation Act applies only so far as they supplement, and are not in conflict with, the rules of the Watercourse Regulation Act and established practice with respect to the provisions in this Act. Moreover, the main reason they give for this is that the diversion of water is \emph{not} to be considered as an expropriation of a waterfall.

There is, as we mentioned, no rule in the Watercourse Regulation Act which states that the authorities are required to consider specifically the question of how the regulation affects the interests of property owners. Such a rule is found in the Expropriation Act, Section 2, but according to the Supreme Court, it does not apply in cases where water is being diverted away from a river. This is so, according to the Supreme Court, because transferral of water is not regarded as a case of expropriation of a right to the waterfall, but merely an expropriation of a right to deprive the waterfall of water.

This is significant in two ways. First, it is important with respect to the legal status of owners who are affected by projects involving transferral of water. In Norwegian law after Måland, it seems that established practice with respect to the assessment of such cases, focusing on environmental aspects and the positions taken by various interest groups, is beyond reproach already because such cases do not involve expropriation of waterfalls. However, considering that the Norwegian water authorities seem to follow these practices generally, and not just in cases where water is transferred, it remains to be seen if this is a practically significant difference in the level of protection. Is the conclusion regarding the admissibility of current administrative practices supposed to apply only to those cases when water is subject to transferral? If it is, then it leads to the peculiar situation that the level of protection for owners depend solely on the way in which the developer propose to gain control over the water. The difference appears completely arbitrary, however, at least from the point of view of owners. But of course, it will soon cease to be arbitrary for developers, who must be expected to favor gutter projects, collecting water from many small rivers and diverting it, since this mode of exploitation makes it easier to acquire necessary rights. On the other hand, if the Supreme Court is to be understood as saying that traditional practices are adequate in general, the consequences of the decision seem fairly dramatic for local owners. It appears that it is not possible, in cases involving expropriation of waterfalls, to solicit any kind of judicial review, not even in circumstances when the factual basis of the decision is manifestly erroneous, and not even if this appears to be the consequence of the authorities neglecting to keep local owners informed about the assessments made regarding their interests.

To illustrate that a lack of consultation is a general problem, and not confined to the particular case of \emph{Måland}, we will conclude by offering a quote from Harald Solli, director of the Section for Concessions at the Ministry of Petroleum and Energy, who submitted written evidence to the Supreme Court regarding the practices followed in cases involving expropriation of waterfalls. Below, we give one of several exchanges that seem to indicate that under current practices, local owners are left in a rather precarious position (my translation).

\begin{quote}
Q: In cases such as this, should owners affected by a loss of small scale hydro-power potential be kept informed about the factual basis on which the authorities plan to base their decision? I am thinking especially about those cases in which the authorities make an assessment regarding the potential for small scale hydro-power on affected properties. \\
A: Affected owners must look after their own interests. The assessments made by the NVE in their report is a public document, and it can be accessed online through the home page of the NVE.
\end{quote}

By their reasoning in \emph{Måland}, it appears that the Supreme Court gave this dismissive attitude towards local owners a stamp of approval. In light of this, we believe the study of the law in a socio-legal setting becomes all the more relevant. For while this attitude might be a reflection of correct national law, as decided in the final instance by the Supreme Court, it seems pertinent to ask if it is \emph{reasonable} law. Also, it seems that one must ask if a case can not be made with respect to human rights, by arguing that the protection awarded is insufficient in this regard. This point, while it was raised by the original owners in \emph{Måland}, did not receive any separate treatment in the Supreme Court. In the following section, we briefly describe some more questions we think the case raises and which we will address further in subsequent chapters.

Following \emph{Måland}, it seems we must conclude that the development which has taken place in the energy sector, and has lead to small scale hydro-power becoming profitable and possible for local owners to carry out themselves, does not imply that original owners are entitled to increased participation in decision-making processes under national law. Even if this is the view held by the Norwegian judiciary, we should of course not overlook the possibility that the water authorities themselves will eventually adopt new practices regarding the assessment of such cases. So far, however, it seems that they stick quite closely to the established routine. 

Since the outcome in Norwegian Courts was that established practices were not found to be in breach of principles of Norwegian expropriation law, it seems reasonable to ask instead about the sustainability of these practices. In fact, the case of \emph{Måland} seems to illustrate precisely why the current system is inadequate, and how it can lead to decisions that appear ill-founded and leave the affected communities feeling marginalized. The likelihood of \emph{factual mistakes}, in particular, seems to increase greatly when the involvement of the local population is not ensured in the preparatory stages.

More importantly, it seems that decisions reached following a traditional process can easily lead to takings for which it is difficult to see any legitimate reason why the project proposed by the developer would be a better form of exploitation than allowing the local owners to carry out their own projects. Indeed, in the case of \emph{Måland}, it seemed that small-scale hydro-power would be a better way of harnessing the water in question, even in the sense that it would be more efficient, and would provide the public with more electricity at a lower cost. More generally, unless the issue of alternative exploitation in small scale hydro-power is considered during the assessment made by the water authorities, one risks making decisions that are not in the public interest at all. 

Even worse, it can send out the signal that expropriation of owners' rights is undertaken solely in order to benefit the commercial interests of the energy company applying for a development license. At this point, it seems appropriate to recall the concerns expressed by US Justice O'Connor in the case of {\it Kelo}.

There, a major point of contention was whether or not her grim predictions about the fallout of the decision did indeed reflect a realistic analysis of the fallout of the decision. Surely, anyone who agrees with Justice O'Connor that the powerful will usurp the power of eminent domain to the detriment of the poor, would also agree with here conclusion that it is perverse. However, whether her pessimism is warranted by empirical fact seems less clear. In this context, we believe the case of Norwegian waterfalls can serve an important broader purpose, as a means towards shedding more light on the hypothesis that a loose interpretation of the public interest requirement will indeed lead to a transfer of property from those with fewer resources to those with more. 

The \emph{Måland} case, and the current tensions regarding expropriation for the benefit of Norwegian hydro-power, seems to suggest that her concern should indeed be taken seriously. Also, the Norwegian experience seems to show that we need to be clear about the fact that property has a social and political function that goes beyond the financial interests of individuals. For the Norwegian case at least, it seems particularly relevant to ask if local people, by virtue of their right to property and their original attachment to the land, have a legitimate expectation \emph{both} that their commercial interests should be protected, \emph{and} that they should be granted a say in decision-making processes. Financial protection does not necessarily imply social protection, and the right to participate and be heard might be both more significant, and harder won, than the right to be compensated according to whatever the powers that be come to regard as the market value of the property in question.

Another perspective, which we will also pursue further in subsequent chapter, is the question of how property rights relates to the overreaching goal of sustainable development of natural resources. Rather than seeing property rights as a means towards securing sustainable development, it seems more common to see it as an impediment. This, indeed, has shaped much of the Norwegian discourse regarding environmental law and policy, including that which relates to waterfalls.\footnote{For example, such a skeptical view of property rights appear to provide an overriding perspective in \cite{backer1} (in Norwegian), which is a widely used textbook on environmental law in Norway.} 

Moreover, a typical justification given for interference in property is that an equitable and responsible management of natural resources requires it. It seems to us, however, that an egalitarian system of private ownership of resources -- as we find in Norway for the case of waterfalls -- could itself serve as a sustainable basis for management of these resources. It seems plausible for us to suggest that private property rights is one of the most robust ways in which local communities can be given a degree of self-determination concerning how to manage local resources. This is typically considered desirable also from the point of view of sustainability, but perhaps even more importantly, when property is in the hands of the many rather than the few, is it not also reasonable to expect that the state will be able to more effectively and rationally exercise its regulatory powers? 

Otherwise, the danger is that the government is being intimidated by large commercial enterprises, perhaps partly owned by the State itself, that command political influence and might not take lightly to what they perceive as undue political interference in their business practices. Such a position might be tenable if you are one of the worlds leading energy companies, but hardly if you are a farmer. 

I think the case of \emph{Måland} suggests that we should investigate these questions in more depth. It seems, in particular, that we must ask about the extent to which commercial companies have succeeded in usurping the notions of sustainable development and public interest, putting the power of these ideas to use in order to secure control over resources and to enlist governmental support, and favorable treatment, for their own commercial undertakings. The extent to which such a mechanism influences the Norwegian energy sector, and the possible implications this might have, both legally and socially, remains to be worked out.

In subsequent chapters, two questions arising from this will receive particular focus. First, we will aim to clarify the importance of the conflict between large scale hydro-power and small scale development by surveying recent and current hydro-power projects in Norway, not in any depth, but by taking note of whether the issue arose. Secondly, we will aim to shed light on the importance of small scale hydro-power to the communities in which local owners reside. As we mentioned, they are usually farmers, and most often in areas were farming is becoming increasingly unprofitable. From the socio-legal point of view it seems highly relevant to ask who the people who loose their resources are, and in what social context we find them. Moreover, while it is clear that hydro-power has become an important source of income in many small and relatively impoverished farming communities, the exact implications of this development, financially and socially, remains to be mapped out.

Following this, it seems natural to return to the legal question of the legitimacy of interference, not from the point of view of national law, but from the point of view of property as a human right. Importantly, it seems to us that property has a clear social dimension, and that mapping out the socio-legal function of specific property rights should inform the judgment we make regarding the level of protection to which owners are entitled. Also, while property is an individual right, it can also be a communal one, and, as such, it can serve to empower local communities that would otherwise be marginalized. The protection of an egalitarian structure of ownership, then, does not appear to be subsumed by, or even conceptually the same as, protecting against individual transgressions. We believe that the case of Norwegian waterfalls demonstrates that this should be kept in mind when analyzing the legitimacy of interference in property for the benefit of commercial undertakings.

\noo{current ownership structure of waterfalls is therefore not simply a question of protecting the commercial interests of individuals who happen to own valuable resources, but also a question of protecting the local communities where these resources are found, giving them the possibility of influencing the way in which the resources are to be harnessed. It seems, however, that local people are often in danger of being seen as an hindrance, both to sustainable development and economic growth, because the commercial companies, along with the environmental interests groups, have claimed this stage as their own. Such, it seems, is the case for Norwegian waterfall. Despite an explosion of interest in small scale hydro-power in recent years, there still seems to be little room left for local communities in the Norwegian discourse concerning hydro-power. It will be an important aim of our work in following chapters to map our in more detail how this influences the law and the administrative policies that are adopted.
}

\section{Conclusion}\label{conc}

In this Chapter, I set out to show that the law relating to expropriation of waterfalls in Norway is based on a tradition that assumes owners to be profit-maximizing while the state is welfare-seeking. Hence, the question of striking a balance between private and public interests is approached under the presumption that private property rights embody mainly private values, while public values are pursued through regulation that ensures public ownership and control. I observed how this perspective shaped the law of expropriation of waterfalls, so that expropriation could only take place for narrowly defined public purposes and only to the benefit of public bodies.

I noted, however, how the increasing centralization of the energy sector and the increasing scale of projects following WW2 led to increased worry about the legitimacy of interference in property and nature on behalf of public hydropower interests. I concluded that the ensuing conflicts, while severe, largely failed to make an impact on the law relating to hydropower, as the public nature of this sector, and the level of political control exercised over it, meant that courts shunned away from adopting a strict view on legitimacy. This did not only apply to the question of authority to expropriate, which was hardly raised at all in the period between the reversion controversy of the early 20th century and the market-reform of the early 1990s.  It also applied to the procedural rules, which the Supreme Court adopting an explicit stance that these rules were themselves largely ``discretionary'' in nature, so that it would fall under the authority of the executive to determine their scope and application in concrete cases.

i noted how this perspective has been maintained by the courts and the executive even after the market-reform meant that expropriation largely lost its public characteristics. I argued that today, expropriation of waterfalls for hydropower development can no longer be looked at as an aspect of providing a public service, but must be regarded as takings for profit, typical economic development takings. I discussed how the law came to be changed on this point, with a dramatically widened expropriation authority introduced in conjunction with the \cite{wra00}. I observed how the issue of expropriation was not considered politically, with the reform in the legislative basis taking place without the active involvement or consideration by the members of the Norwegian Storting.

I concluded with a description of the fallout from this, as expressed concretely in the case of \cite{måland11}. The case serves to illustrate how administrative practices developed and sanctioned during the monopoly days are now applied in a context of competing commercial interests, meaning that expropriation becomes an important tool that the powerful market players now use to gain the upper hand in competition with locally based companies or smaller companies that rely on cooperation with owners. I noted, in particular, that the law is entirely unprepared for dealing with this dynamic. Still, in the case of \cite{måland11}, the Supreme Court explicitly denied that established practices were in need of revision. Moreover, it refused to reconsider the established interpretation of the scope of procedural rules in hydropower cases, rejecting arguments to the effect that these must now be understood to provide protection for waterfall owners that matches the protection offered to other affected parties.

In the next Chapter, I will consider an aspect of the law were the Supreme Court {\it has} taken the view that a revision of established practices is in order, namely in relation to the question of compensation. I note, however, that the Court's emphasis on the compensation issue serves to reinforces the idea that private property rights pertain mainly to financial entitlements. As I have already argued, this perspective hardly does justice to the role of private ownership of waterfalls in Norway. I will return to this in the last Chapter of the thesis, where I consider land consolidation as an alternative to expropriation. However, as demonstrated in the present Chapter, Norwegian courts do not seem to recognize the shortcomings of the current system. Until they do, or are directed to do so by political bodies or international tribunals, it is unlikely that expropriation law will evolve much from its current fixation on the compensation issue. 


\chapter{Just compensation}\label{chap:5}

\section{Introduction}\label{sec:into5}

In this Chapter, I consider the question of compensation following expropriation of waterfalls. The main question that arises is whether or not owners should be compensated for the loss of a commercial hydropower potential. If so, the compensation payments can be very large, so large that expropriation becomes unfeasible in practice. However, traditionally, no such compensation was awarded and the amounts paid to were negligible. In fact, no compensation would have been paid at all, were it not for the fact that a theoretical compensation formula was developed to avoid reaching the conclusion that waterfalls could be expropriated with no compensation payable to owners.

The question of whether or not to base compensation on the loss of a commercial hydropower potential is closely related to the so-called ``no scheme'' principle, by which compensation is to be based on the value of the property taken such as it would have been if the expropriation scheme had not been proposed. If one takes the view that hydropower development is the prerogative of the party that has been granted a development license, it might then follow that no compensation is payable to the owners of the waterfalls, at least not for the hydropower potential. The value of hydropower, in particular, may be regarded as being due to ``the scheme'', not due to the natural resource that the waterfall represents. 

In Section \ref{sec:nsp}, I describe the no-scheme principle in more depth, by comparing its origin and current status in English law with that of Norwegian law. It is of independent interest to note the close correspondence with common law. Moreover, considering the English debate in particular will also serve to bring out some concrete idiosyncrasies regarding how the principle is understood in Norway. The debate about how to apply it in waterfall cases is ongoing in Norway, and the Supreme Court has dealt with the issue in a range of recent cases. 

Traditionally, the principle was {\it not} applied, however, since it would have led to little or no compensation for owners during the monopoly era. Instead, a theoretical method was applied that was meant to give owners a share of the benefit in hydropower development. This method was developed early in the 20th century, however, so it gradually became more and more removed from the realities of hydropower development. I describe the method in some depth in Section \ref{sec:nathp}, culmination in a number of reasons why it no longer succeeds in achieving a meaningful form of benefit sharing.

Then, in Section \ref{sec:fa}, I go on to consider the processes that eventually led to the method being abandoned in cases when alternative owner-led development schemes would have been ``foreseeable'', assuming the expropriation project had not been proposed. I link the reform in this area with the institutional framework used to determine compensation payments under Norwegian law. I note, in particular, that the natural horsepower method was first abandoned by the appraisal courts, where lay people sit as appraisers alongside a regular judge. In the first case, moreover, the lay people eventually adopted the new method against the dissenting opinion of the regular judge. 

The Supreme Court struck down their judgment on a technicality, but refused to reject the principle that lay people were free to adopt a new method in cases when the traditional method would not adequately reflect the value of ``foreseeable'' use. I argue that this shows the strength of a long tradition of respecting the discretion of lay people in appraisement disputes. Many legal scholars, in particular, had previously regarded the natural horsepower method as a {\it rule of law}, fixed by precedent.

The method has not been abandoned as a matter of principle, however. As made clear recently by the Supreme Court, it is still to be applied in cases when a calculation based on ``foreseeable use'' does not lead to higher compensation payments. The crucial question becomes what exactly is meant by this notion. I address this in some depth, by pointing to how Norwegian law in general is  marked by a tendency to disregard any use that is not sanctioned by public plans, including in cases when these plans themselves provide the rationale for expropriation. This appears to be contrary to the no-scheme principle, demonstrating more generally that only ``one half'' of the principle tend to apply in a Norwegian setting. In so far as the principle precludes giving the owner a share of the expropriation surplus, it is applied, but in so far as it entitles him to compensation based on a future use that is rendered unforeseeable by the planning underlying the expropriation license, it is not.

There are some exceptions to this, however, and the Supreme Court has indicated that one of them applies to hydropower cases. At the same time, however, it has been stressed that even if the expropriation plans themselves are not binding for the compensation assessment, the ``public rationale'' underlying these plans must be taken into consideration when awarding compensation. In effect, this means that compensation is not offered for alternative uses in so far as the project proposed by the expropriating party is superior and could not be undertaken by the current owner. In effect, it seems that a partly {\it subjective} standard is introduced into compensation law, whereby local owners are denied compensation for a commercial value that is deemed to be such that it is only realizable by the expropriating party.

The Supreme Court has not been entirely consistent about the scope and exact content of the ``public rationale'' principle, however,   and the issue is still very much contested in Norwegian courts. In Section \ref{sec:ko}, I illustrate the current unclear state of the law by contrasting two recent Supreme Court cases. In the first, the court embraced an objective version of the ``public rationale'' principle by holding that as the expropriating party's project resulted in more public benefits, compensation could be based on the premise that the owners' foreseeable use of the waterfalls was to cooperate with the large energy company in realizing the plans, to take their share of its commercial potential. 

In {\it Otra II} on the other hand, the Court held that this should not be the conclusion in so far as cooperation was deemed to be ``impractical'', following a concrete assessment of the facts. It seems quite clear that the notion of ``impracticality'', as it was used here, serves to introduce a subjective assessment standard, contrary to what otherwise dictated by Norwegian compensation law. 
I go on to consider the merits of {\it Otra II} against human rights law, anticipating also the outcome of the appeal currently lodged with the ECtHR in Strasbourg.

Finally, I conclude that the case law on compensation demonstrates the intrinsic inadequacy of a narrow perspective on takings for profit. It seems clear, in particular, that all of the approaches currently in use to calculate compensation for waterfalls leave great room for bickering, manipulation and long-winded court battles. Moreover, the factual premise for the calculation is typically extremely uncertain, meaning that the whole procedure appears as something of a gamble, for both owners and developers. Hence, the developers favour the use of the natural horsepower method, which is completely removed from the reality of hydropower, but deliver predictably low compensation payments that will not prove too damaging to the profit-margin of the development company. On the other hand, owners have an incentive to push for compensation mechanisms that will allow them to collect the entire financial potential of hydropower development without actually investing any effort in planning or administerting such development, and without subjecting themselves to any of the risks involved. 

The current system means that if they are lucky, or employ skilled arguers, they can collect a very substantial sum of money for little or no efforts and with no social responsibilities attached to it. On the other hand, if they are unlucky, they are forced to give up their most valuable asset for nothing but a symbolic payment. I conclude by arguing that a much better approach would be to try and get owners involved in sustainable hydropower in a way that can remove the need for expropriation altogether. As development is now organized as a commercial pursuit, this should in principle be possible, since the owners {\it do} have an incentive to get involved in  development, also in cases when the public severely restrict the set of possible terms through regulation.  In practice, however, what is needed is a mechanism for organizing such owner-involvement. This mechanism, moreover, will undoubtedly also need to be endowed with powers of coercion if it is to be effective.

This, then, sets the stage for the last chapter, where I return to the question of how to replace expropriation by mechanisms of participatory democracy, referring back also to the discussion in Chapter \ref{chap:1}.

\section{The ``no scheme'' principle}\label{sec:nsp}

In most jurisdictions, a fundamental principle relating to compensation following expropriation is that the owner's loss should be calculated without taking into account changes in the value of their property that are due to the expropriation itself, or the scheme underlying it. In a recent Law Commission consultation paper, this principle is referred to as the \emph{no-scheme} rule, a terminology we will also adopt here, noting that while the exact details of the rule might differ between jurisdictions, the underlying principle appears to play a crucial role both in civil and common law traditions for regulating compensation following expropriation.\footnote{Need a good reference for this...}

While the no-scheme rule is easy enough to comprehend when it is stated in general terms, it raises many difficult questions when it is to be applied in concrete cases. What the rule asks of the valuer, in particular, is quite daunting; he is forced to consider the counterfactual "no-scheme world", and he must calculate the value of the property based on the workings of this imaginary world. One crucial question that arises, and which has traditionally proved to be highly contentious, is the question of what exactly this world looks like.

In the first instance, it might be tempting to take the view that this is a ``question of fact for the arbitrator in each case'', as expressed by the Privy Council in \emph{Fraser}, an influential Canadian case from 1917.\footnote{\emph{Fraser v City of Fraserville}, [1917] AC 187, p. 194} However, as the history of the no-scheme rule has shown, this point of view is not tenable.\footnote{For an history of the rule in UK law, clearly illustrating the difficulty in interpreting it and applying it to concrete cases, we point to Appendix D of Law Commission Report No 286, 2003} The problems associated with the rule were discussed in great detail by Lord Nicholls in the recent case of \emph{Waters}, who summarized the state of the law as follows.\footnote{\emph{Waters and other v Welsh National Assembly} [2004] UKHL 19}

\begin{quote}
Unhappily the law in this country on this important subject is fraught with complexity and obscurity. To understand the present state of the law it is necessary to go back 150 years to the Lands Clauses Consolidation Act 1845. From there a path must be traced, not always easily, through piecemeal development of the law by judicial exposition and statutory provision. Some of the more recent statutory provisions defy ready comprehension. Difficulties and uncertainties abound. One of the most intractable problems concerns the 'Pointe Gourde principle' or, as it is sometimes known, the 'no scheme rule'. On this appeal your Lordships' House has the daunting task of considering the content and application of this principle.
\end{quote}

\noo{
\begin{quote}
The extreme complexity of the issues that I have had to consider, the
uncertainty in the law, the obscurity of the statutory provisions, and
the difficulties of looking back over a long period of time in order to
decide what would have happened in the no-scheme world
demonstrate, in my view, that legislation is badly needed in order to
produce a simpler and clearer compensation regime. I believe that
fairness, both to claimants and to acquiring authorities, requires
this
\end{quote}
}
In the case of \emph{Waters}, the House of Lords seems to have made it an explicit aim to offer a clarification of the no-scheme rule and how to interpret it, and their judgment went into much more detail than warranted by the concrete case at hand, which seems to have been fairly straightforward. Even if it was not needed for the result, the House of Lords addressed many of the issues raised by the Law Commission in their report, focusing particularly on resolving the tension which was identified there between the principle relied on in \emph{Pointe Gourde} and the reasoning adopted in the so-called \emph{Indian} case from 1939.\footnote{\emph{Vyricherla Narayana Gajapatiraju v Revenue Divisional
Officer, Vizagapatam} [1939] AC 302.} In the \emph{Indian} case, the scheme was given a very narrow interpretation, with Lord Romer interpreting the scope as follows.

\begin{quote}
The only difference that the scheme has made is that the acquiring
authority, who before the scheme were possible purchasers only, have
become purchasers who are under a pressing need to acquire the
land; and that is a circumstance that is never allowed to enhance the
value.
\end{quote}

This, however, did not entail that the purchaser's demand for the property was to be disregarded, since, as Lord Romer puts it:

\begin{quote}
[...] The fact is that the only possible purchaser of a potentiality is
usually quite willing to pay for it […]
\end{quote}

In \emph{Pointe Gourde}, a different stance appears to have been adopted. The case concerned a quarry that was expropriated for the construction of a US naval base in Trinidad. The quarry had value to the owner as a business, and the valuer had found that if the quarry had not been forcibly acquired, it could also have supplied the US navel base for a profit. However, the value of this potential fell to be disregarded, with Lord MacDermott describing the no-scheme rule as follows:

\begin{quote}
It is well settled that compensation for the compulsory acquisition of
land cannot include an increase in value, which is entirely due to the
scheme underlying the acquisition
\end{quote}

Seemingly, this is at odds with the position taken by Lord Romer in the {\it Indian} case. In \emph{Waters}, both Lord Nicholls and Lord Scott addressed this tension in great detail, and offered a reconciliatory interpretation, which seems to narrow the no-scheme rule compared to how it has most commonly been understood following \emph{Pointe Gourde}. Moreover, the House of Lords also noted the need for reform and legislation, with Lord Scott going as far as referring to what he described as the present ``highly unsatisfactory state of the law''.

To understand how a seemingly simple principle could become so troubling in practice, it is crucial to keep in mind that after the introduction of extensive planning legislation in the 20th century, commercial development of property now tends to be contingent on governmental licenses and subject to extensive oversight. Moreover, the power to expropriate is often granted as a result of comprehensive regulation of the property-use in an area, often following public plans that are wider and encompass more than the particular project that will benefit from such a power. As a result, it has become increasingly difficult to ascertain what is meant by the scheme. Does it include the whole planning history leading to expropriation, does it only refer to the power to expropriate, or is it something in between?

The policy reasons motivating the no-scheme principle suggest that a fine balancing act must be made when addressing this issue. Under a wide interpretation of ``the scheme'', forcing the valuer to entertain many counterfactual assumptions, the property owner might come to feel that he is not compensated for his true loss, but rather an imaginary one. Indeed, the no-scheme world that the valuer must consider can end up being far removed from the actual one, forcing him to go back many years, perhaps decades, to establish what would have been the status of the property in question if the sequence of planning steps eventually leading to expropriation had not taken place. 

This can leave the property owner in a perilous situation, particularly if it makes the outcome seem arbitrary. Taken to extremes, the no-scheme principle can then also come to run amiss with respect to human rights law and constitutional provisions protecting private property. On the other hand, if the scheme is interpreted too narrowly, one runs the risk of endangering important public schemes by compelling the public to pay extortionate amounts. In many cases, it is undoubtedly also true that the value of property is increased by public investments and plans for the area in which the property is found, investments and plans that would not have materialised unless the power to expropriate had been anticipated.

It is important to keep in mind here, as noted by the Law Commission, that the no-scheme rule serves at two distinct purposes.\footnote{Ibid ....} First, it should be noted that the rule has an important \emph{positive} dimension: Property owners are not only compensated for the direct loss of their property, but also for the possible depreciation of their property's value following the decision to carry out a scheme which requires expropriation. Seemingly, this is easy enough to justify: It seems intuitively unreasonable if the deleterious effects of a threat of compulsion is permitted to result in reduced compensation payments.

However, under the extensive planning regimes common today, it is not clear where to draw the line: When is the regulation leading up to the scheme to be regarded as reflecting general public control over property use, and when is it to be regarded as a measure specifically aimed at compelling private owners to give up their property? As we will see when we consider the role of the no-scheme rule in Norwegian law, this question can easily become highly controversial, especially when it is linked with the more general question of whether or not the state should be liable to pay compensation for regulation that adversely affects the potential for future development. In jurisdictions that do not recognize owners' right to such compensation, like Norway and the UK, it is easily argued that the positive aspect of the no-scheme rule must be limited correspondingly. Why would a depreciation of value following regulation imply compensation when the property is eventually expropriated, but not otherwise?

In addition to its positive dimension, the no-scheme rule also has an important \emph{negative} dimension, expressed in {\it Pointe Gourde} as the principle that an {\it increase} in value should be disregarded when it is ``entirely due to the scheme''. The negative dimension has attracted more interest and controversy than the positive dimension, especially in the UK, and it was also at the center of attention in {\it Waters}. This is understandable all the while the negative aspect of the principle is more likely to result in protests from property owners. However, on a traditional understanding of the public purpose of expropriation, the negative aspect of the rule is also seemingly easy to justify. In \emph{Waters}, Lord Nicholls describes the most important policy reasons as follows:

\begin{quote}
When granting a power to acquire land compulsorily for a particular purpose Parliament cannot have intended thereby to increase the value of the subject land. Parliament cannot have intended that the acquiring authority should pay as compensation a larger amount than the owner could reasonably have obtained for his land in the absence of the power. For the same reason there should also be disregarded the ``special want'' of an acquiring authority for a particular site which arises from the authority having been authorised to acquire it.
\end{quote}

This appears like a reasonable justification. Notice, however, that Lord Nicholls avoids using the word ``scheme''. In particular, he does not identify the scheme's absence as the measuring stick for ascertaining on what basis parliament intends for compensation to be based. Rather, Lord Nicholls speaks of what the owner could reasonably have obtained in \emph{the absence of the power} to acquire the land compulsory. In this way, he seems to prescribe a rather narrow interpretation of the negative dimension of the no-scheme rule.\footnote{I mention that this interpretation of \emph{Waters} is also argued for in \cite{newuk}.} It is the power to expropriate that should not give rise to an increased value -- nothing is said at this stage about the scheme that benefits from it.

It would appear, therefore, that there is nothing in principle that prevents the property from being compensated on the basis of its value in a scheme that differs from the scheme underlying expropriation only in that it does not have such powers. Indeed, this subtle distinction appears to have been rather crucial for the remainder of Lord Nicholls' reasoning, where he attempts to reconcile the principle adopted in the \emph{Indian} case with the \emph{Pointe Gourde} case.

It will lead us to far astray to go into further details about the interpretation of the no-scheme rule in UK law and the possible implications of \emph{Waters}. Rather, we would like to turn our attention to the recent UK Supreme Court case of \emph{Bocardo}.\footnote{\emph{Star Energy Weald Basin Limited and another (Respondents) v Bocardo SA (Appellant) [2010] UKSC 35}} This case was decided under dissent, and it suggests that the clarification offered in \emph{Waters} might not have been as conclusive as one  had hoped. This worry arises, as we will see, particularly in those cases when expropriation benefits commercial schemes.\noo{ and for which the conceptual framework surrounding expropriation is, in our opinion, in need of refinement.}

\emph{Bocardo} was such a case. In short, it concerned a reservoir of petroleum that extended beneath the appellant's estate, and could not be exploited without carrying out works beneath their land. The first question that arose was whether or not extraction of the petroleum amounted to an infringement on property rights. This was answered in the affirmative. The second question that arose was what principle of compensation should be adopted to compensate the owner. The Supreme Court, following some deliberation, found that the general rules applied, and that the case should be decided on the basis of an application of the no-scheme rule. 

However, opinions differed as to the correct interpretation of the law, as well as how the facts should be held against the law. The crucial point of disagreement arose with respect to whether or not the special suitability, or \emph{key value}, of the appellant's land for the purpose of petroleum exploitation was to be regarded as \emph{pre-existing} with respect to the petroleum scheme.

In \emph{Waters}, the House of Lords had cited and expressed support for the following passage, taken from Mann LJ's judgment in \emph{Batchelor}.\footnote{\emph{Batchelor v Kent County Council} 59 P \& CR 357 p. 361}

\begin{quote}
If a premium value is ``entirely due to the scheme underlying the acquisition'' then it must be disregarded. If it was pre-existent to the acquisition it must in my judgment be regarded. To ignore the pre-existent value would be to expropriate it without compensation and would be to contravene the fundamental principle of equivalence (see \emph{Horn v Sunderland Corporation}).
\end{quote}

Relying on this distinction between the potentialities that are ``pre-existing'' and those that are due to the scheme, the minority in \emph{Bocardo}, led by Lord Clarke, made the following observation.

\begin{quote}
Anyone who had obtained a licence to search, bore for and get the petroleum under Bocardo’s
land would have had precisely the same need to obtain a wayleave to obtain access
to it if it was not to commit a trespass. So it was not the respondents' scheme that
gave the relevant strata beneath Bocardo’s land its peculiar and unusual value. It
was the geographical position that its land occupies above the apex of the
reservoir, coupled with the fact that it was only by drilling through Bocardo’s land
that any licence holder could obtain access to that part of the reservoir that gives it
its key value.
\end{quote}

This, however, was rejected by the majority, led by Lord Brown, who interpreted the no-scheme rule quite differently in this respect. He made the following comments regarding the issue of whether or not the value of the appellants land for petroleum extraction existed prior to the scheme.

\begin{quote}To my mind it is impossible to characterise the key value in the ancillary
right being granted here as ``pre-existent'' to the scheme. There is, of course,
always the chance that a statutory body with compulsory purchase powers may
need to acquire land or rights over land to accomplish a statutory purpose for
which these powers have been accorded to them. But that does not mean that upon
the materialisation of such a scheme, the ``key'' value of the land or rights which
now are required is to be regarded as “pre-existent”.
\end{quote}

While the case was resolved in keeping with this view, the dissent suggests that the clarification in \emph{Waters} has not resolved all issues. Moreover, it suggests that special questions arise with respect to the question of what potentials for development should be taken into account when evaluating a property. Crucially, the question raised in \emph{Bocardo} does \emph{not} relate to the scope of the scheme -- it was obvious that the scheme was the entire project aimed at extracting petroleum from the reserve. However, even when the scheme was unambiguously circumscribed, significant questions arose as to what ``value to the owner'' actually meant in this situation. 

In fact, it seems to us that \emph{Bocardo} serves to take the debate regarding compensation and the no-scheme rule one step further, and in a somewhat different direction compared to the debate revolving around the ``classic'' problem of determining the extent of the scheme. In some sense, it seems that the question raised by \emph{Bocardo} goes deeper, to the very core of the idea underlying the negative aspect of the no-scheme rule. When is it appropriate to say that some particular value is \emph{due to} the scheme and therefore not part of the owner's bundle of entitlements?

This asks us to establish a causal link between scheme and value. But as \emph{Bocardo} illustrates, it is by no means obvious what should be taken to constitute evidence for such a link. Moreover, it seems that the answer can depend largely on the point of view with which you \emph{choose} to analyze the matter at hand. For instance, when Lord Clarke went on to point out that the state, as owners of the Petroleum following nationalization in 1937, could have given the right to extract it to \emph{someone else}, he was certainly not incorrect.\footnote{References.} 

Moreover, it seems that this fact does in some sense break the causal link between scheme and value, although weakly so, since the difference between all schemes so conceived would only relate to \emph{who} the developers are, not the nature of the schemes as such. Consider, however, a scheme that is conceived of slightly differently, based on cooperation with the owner rather than expropriation. Would it not follow that this scheme would have \emph{precisely the same need to obtain a wayleave}, as Lord Clarke puts it? Moreover, is it still not also true that those behind it might be \emph{quite willing to pay}, as Lord Romer expressed it in the \emph{Indian} case? If so, the scheme does not cause the value. The fact that the wayleave is acquired compulsorily, in particular, is no precondition for the scheme. The project could equally well have been carried out by a developer who was willing to pay the owner for the special suitability of his land. If this is true, it appears that the justification for disregarding the special suitability of the land disappears.

On the other hand, it is also possible to adopt the point of view adopted by Lord Brown, which I choose to interpret as follows: Since the relevant strata did not have any special value except in relation to a petroleum-scheme requiring access, its value was causally dependent on the existence of \emph{some} such scheme. Hence, since the legislature {\it had} in fact made powers of compulsory acquisition available for developers of such schemes, this value was not, under the current regulatory system, regarded as pre-existent. The special value had, pursuant to political determination, been apportioned wholly to the developers of petroleum schemes. If this is true, then there is little doubt that the special value should be disregarded for the purposes of compensation. It seems clear, in particular, that under English law, no compensation claims for owners arise from such regulation.\footnote{I mention that the dispute in {\it Bocardo} also addressed more generally the question of how to appropriately compensate property that has ``key value'' with respect to the development of other property. It seems, in particular, that the strata, in its absence of any inherent value, more easily fell to be disregarded, and that Lord Brown's arguments in particular relies on establishing such a lack of intrinsic value. However, the question of when a particular aspect of value is to be regarded as pre-existent tend to arise in many other cases as well, and can be expected to arise particularly often with respect to commercial schemes. An extreme case obtains when we consider expropriation of \emph{natural resources}. Surely, if what was subject to expropriation in \emph{Bocardo} had been the petroleum itself, and not a right to access it, then even Lord Brown would have concluded that its value was pre-existent? This seems likely indeed, and then it appears to be good law in the UK after \emph{Waters} that it should also be compensated, irrespectively of whether or not the expropriating party is the only potential buyer.}

I conclude that the underlying question in {\it Bocardo} became the following: Did parliament really intend to let the entire value go to the petroleum-developer in such cases? This, in turn, seems to turn on the question of whether or not the use of compulsion in such cases was meant to be the norm or the exception. If compulsion is used in most or all cases, one can hardly claim that an alternative scheme based on cooperation is a realistic scenario that can provide the basis for calculating compensation. If, on the other hand, the power of compulsion is only made available to prevent holdouts and extortion, the case could still well be made that the value to the owner that would result from an equitable non-compulsion scheme should be compensated.

In this way, the case of \emph{Bocardo} highlights what I believe to be the crucial keyword underlying many of the most intractable problems associated with the negative aspect of the no-scheme principle, namely {\it benefit sharing}. In particular, disagreements about how to apply the principle invariably tends to boil down to the question of whether or not benefit sharing is considered desirable, or at least possible, under the current regulatory regime. In the first instance, this is a political question. But as cases regarding the no-scheme principle shows, it is one that the legislature often fails to address explicitly. 

Hence, it often falls to the courts to make determinations that are politically sensitive. This is in itself problematic. It seems, in particular, that in so far the courts' application of the no-scheme principle implements specific policies on benefit sharing, the lack of democratic accountability is worrying. This worry is exasperated by the fact that the law relating to the principle is notoriously unclear and complex. The obscurity of the law means that what is initially a clear and straightforward political issue runs the risk of becoming an impenetrable subject that only specialist jurists can hope to engage with in a meaningful way. This, in turn, opens up the possibility of abuse, as the courts do important work in shaping the politics of benefit sharing, work that is unacknowledged by many, but not necessarily by the wealthy and powerful.

An additional issue that arises is to what extent constitutional and human rights law makes it possible for owners to {\it demand} benefit sharing. This question arises most clearly in situations when a lack of benefit sharing points to a possible {\it unequal} treatment of opposing commercial interests. In so far as public interests in a specific form of activity dictates general limitations to the possible commercial benefits that can result from this activity, constitutional and human rights issues are unlikely to arise. However, in so far as commercial benefits accrue to {\it some}, but not to the affected owners, it would seem much more likely that fundamental principles could come to be violated.

This issue comes into focus when the scheme underlying expropriation itself has a commercial nature. In several of the ``classical'' cases that are cited as the foundation for the original no-scheme rule, this issue arose with great urgency. I mention, in particular, the cases of \emph{Cedars} (1914) and \emph{Fraser} (1917), two important Canadian compensation disputes regarding expropriation for hydropower, cited both by the Law Commission and the House of Lords in \emph{Waters}.\footnote{\emph{Cedars Rapids Manufacturing and Power Co v Lacoste}, [1914] AC 569 and \emph{Fraser v City of Fraserville} [1917] AC 187.} In \emph{Fraser}, it was the waterfalls themselves that were subject to expropriation, yet the Privy Council still found that the value of the potential for hydropower exploitation of these falls should be disregarded when compensating them.

The reasoning adopted seems to follow a standard ''value to the owner`` approach. However, reflecting back on {\it Bocardo}, it is hard to see how anyone could think that the value of the waterfalls is not ``pre-existent'' to the scheme to develop them. Surely, as a natural resource a waterfall has significant value in and of itself? This, however, was not the view taken by the Privy Council, which found that owners of waterfalls were not in fact capable of developing hydropower themselves. Hence, a subjective standard was in effect employed, whereby the entitlements regarded as arising from the professional company's  ownership of the waterfalls far exceeded the entitlements regarded to arise from the farmers' ownership of the same. This unequal treatment of owners, it seems, is such that is could, in the present day, be attacked from the point of view of human rights and constitutional law. But such an approach might not be required, as the Canadian cases already appear to be at offs with both {\it Waters} and {\it Bocardo}.

Still, they can serve as great examples of the type of situation where the need for a distinction between commercial and non-commercial aspects arise most forcefully. On the one hand, there can be no doubt that the energy inherent in water pre-exists any scheme seeking to harness it. Moreover, it seems clear that energy has great value, meaning that the value of a waterfall pre-exists any scheme for hydro-power exploitation. However, we must also ask: what \emph{kind} of value is it?

In fact, it seems that any value resulting in compensation to the owner must by the nature of things either be \emph{personal}, related to claims for disturbance etc, or else \emph{commercial}, namely such a kind of value that can be realized by a company or an individual operating for profit -- possibly the owner, possibly some buyer of their property. A different kind of value altogether is the \emph{public value}, which can not be realized for profit by \emph{anyone}. 

The distinction between commercial and public value is, obviously, down to a political decision. Moreover, it can hardly be regarded as permanent. In addition, it can often be difficult to assess where the line is to be drawn, especially in cases when public-private partnerships are relied on to provide public services. Nevertheless, it seems perfectly legitimate to make this distinction, and it seems like it can be very helpful in many cases. It seems, in particular, that both the political desirability and the constitutional necessity of benefit sharing is greatest in relation to what one would refer to as commercial value. 

This perspective can hopefully also help clarify the distinction between the concrete judicial application of the no-scheme principle and its political determined scope. In particular, I think that a slight departure from previous case law suggests itself in this regard. For instance, even if the public value of hydro-power pre-exists the hydro-power scheme, this does \emph{not} necessarily mean that there is any pre-existent commercial value in hydro-power. Hence, to structure the discussion around what counts as pre-existent value, as done in {\it Bocardo}, might not lead to a sufficiently fine-grained approach.  What counts as {\it commercial} value, in particular, must first be answered. This, moreover, depends entirely on whether or not the public has settled on a regulatory regime that allows commercial exploitation. In the case of {\it Bocardo}, I think this perspective would have been particularly helpful to Lord Brown, who argued that the value of the strata was not pre-existent. From an intuitive point of view, his argument seems rather strained. After all, it was the physical conditions that gave the land its value, not the abstract fact that a development license had been granted. However, by looking at his argument in more depth, it is tempting to rephrase his conclusion by saying that he regarded the value of the strata as having no commercial value under the prevailing regulatory regime.

Hence, I arrive at a modified version of the ``pre-existence'' test used in \emph{Bocardo}: An owner should always be compensated for the value of any pre-existent \emph{commercial} value that his property has.\footnote{Certainly, a clarification along these line would not resolve all issues. It would not, for instance, offer any conclusive guidance with respect to the specific issues related to "key value" raised in \emph{Bocardo}.} I remark that the question of what commercial value can be said to pre-exist a scheme might turn rather more on facts than on law. Hence, this way of posing the problem is also apt to bring out how the determination is highly sensitive to the political context in which expropriation takes place.

It seems quite clear, in particular, that in order to answer the question of what should be counted as a pre-existing commercial value, one must take a broad look at the prevailing regulatory regime. Moreover, one must expect that the correct assessment of this question will depend on the context of regulation, in particular the extent to which the state \emph{allows} the disputed value to be commercially realized. The law relating to compensation should be such that it can tolerate significant changes in these parameters. It  seems, therefore, that the important legal question in this regard is to provide a sound conceptual foundation that can enable sound and equitable assessment across a range of different scenarios. Moreover, it seems that the courts, in light also of human rights law, has an important supervisory role to play in this regard.

In the next section, I will address Norwegian compensation law. I will observe, in particular, how the system was originally based on an approach to compensation that was not overly constrained by the idea that all disputes had to be resolved uniformly on the basis of clear and precise rules. The no-scheme principle, while clearly relevant, was understood broadly, as a guideline for assessment. But the assessment itself was above all else discretionary. Moreover, legitimacy of the process was ensured by the involvement of lay people sitting as appraisers, alongside a regular judge. 

I note, however, that this system has largely been modified so that, today, the appraisal courts are far more constrained from the top down, by legislation and the Supreme Court. I argue that this has been a source of difficulty for the system, particularly in relation to the no-scheme principle which, as I argued above, necessitates a concrete assessment, and can not -- should not -- be resolved by all-encompassing principles. I note, in particular, how the increasingly constrained room for discretion by lay people means that the distinction between commercial and public value -- which must now be determined centrally -- becomes muddled. In many cases, the idea acting as a premise for the general rules applied simply does not correspond to reality. This often leads to unacknowledged commercial windfalls for takers, arising when owners are denied compensation for commercially valuable rights that the law presupposes to be wholly public, even though they are not. 

\section{Appraisal courts and ``foreseeable alternatives''}

The right to compensation following expropriation of property is enshrined in Section 105 of the Norwegian Constitution of 1814, in very simply terms. The constitution simply demands that \emph{full compensation} is to be paid, in all cases when the public interest warrants the compulsory acquisition of property. For more than 150 years, until the introduction of the Compensation Act 1973, this was the sole legislative basis for compensation rules in Norway. The concrete methods employed to calculate full compensation for different types of property developed through case law. 

According to a long legal tradition in Norway, going back even further than our constitution, the discretionary aspects of property valuation is regulated by special procedure, with a significant reliance on so called \emph{unwilling appraisers}, members of the general public, or, in some cases, technical experts, who have no interests in the case at hand, but who are regarded as being in a better position to judge the value of property than legal professionals.

This special legal procedure has a long history, going back to customary law that predates even the constitution. The rules regulating it were revised and codified in their current form by the Appraisal Act of 1917.\footnote{Act no 1 of 1. June 1917 relating to Appraisal Disputes and Expropriation Cases.} In short, the Norwegian system now organizes these disputes similarly to regular civil disputes, and the procedure is administered by the district courts.\footnote{See Section 5 of the Appraisal Act 1917.} The presence of laymen is the major distinguishing feature: the court is composed of a panel consisting of one judge and normally four appraisers, who do not have any special legal competence. The standard arrangement is that they are chosen from the general public in the district where the property in question is located, but the Act opens up for the possibility that they may also be chosen for their special technical expertise.\footnote{See Sections 11 and 12 of the Appraisal Act 1917.}

Their role in the procedure is on par with the judge, however, and the panel decides both the legal and the technical questions together, usually following technical reports assembled by the acquiring party, which the property owner might then challenge more or less as if it was presented as evidence in a standard legal dispute.\footnote{See particularly Section 27 and Section 22 of the Appraisal Act 1917, with further references to the Dispute Act 2005 (Act No 90 of 17 June 2005 relating to the Mediation and Procedure in Civil Disputes).}

There is a possibility for appeal to the high appraisal court, which is organized alongside the regular regional high courts, and the possibility of getting the appeal heard depends on the importance of the case, following rules that correspond to those in place for regular civil disputes.\footnote{See Section 32 of the Appraisal Act 1917.} The procedure followed is an adaptation of those used for appraisal disputes at the district level, again according to the standard adaptations used for appeal procedures in civil cases.\footnote{See Section 38 of the Appraisal Act 1917.} However, the decision made by the high appraisal court is final as far the appraisal assessment is concerned, an appeal to the Supreme Court can only be accepted on legal grounds.

As a consequence of this system, and the lack of legislation regarding the meaning of "full compensation", the appraisal courts have been very important in interpreting and developing the law relating to compensation in Norway. At the same time, the practical viewpoint and emphasis suggested by the special procedural form led to legal aspects often being situated in the background in such cases, only coming to the forefront if and when the legal aspects of the case reached the Supreme Court. Indeed, the primary criticism voiced against the system, particularly following the Second World War, was that it gave the appraisal courts too much discretionary power and that legislation was needed to make the outcome of appraisal cases more predictable.\footnote{See, for instance, Part 2, Chapter 1 of the \emph{Report Regarding Appraisal Procedures and Compensation following Expropriation}, NUT 1969 nr. 2 (Norwegian governmental reports), handed over to the Department of Justice by the so called Husaas committee, appointed by the King in Council 6. Aug 1965.}

However, while the law regarding compensation was not formalized in written form, and also opened up for considerable discretion on part of the appraisal courts, there were legal scholars who developed theories and aimed to explicate its content based on the body of case-law that was available. Also, the Supreme Court did regularly hear cases concerning legal arguments regarding compensation, and they developed a consistent position on at least some of the more critical and recurring legal issues. The central source of legal reasoning regarding appraisal at this point was still to be found in the constitution itself, and the theories regarding compensation law that were \emph{absolutist} in the sense that they looked directly to wording in Section 105, also when tackling specific problems of interpretation. This general starting point was widely accepted as late in the 1940s, and in \cite[p. 177]{schj} it was summed up as follows.

\begin{quote}
When an owner is entitled to compensation, he is entitled to have his full economic loss covered. He should receive full compensation, see p. 42 ff. This is the great principle that remains absolute and any dispute must be resolved on its basis.
\end{quote}

A typical example of the style of legal reasoning that this view gave rise to can be found in the writings of the prominent legal scholar Frede Castberg. One of the problems he addressed was the extent to which increases in value due to the scheme underlying expropriation was to be taken into account when calculating compensation, and he based his reasoning in this regard directly on a reading of the constitution. His interpretation, moreover, was based on the principle of \emph{equality}, which was considered particularly crucial in understanding constitutional law. He wrote as follows, in \cite[Volume 2, p. 268]{castberg}.

\begin{quote}
The owner is entitled to full compensation. The expropriation should not leave him worse off economically than other owners. Hence if the public has knowledge that an industrial undertaking is being planned, that a railway will be built etc, and this affects the value of property generally in a district, then the increased value of the property that will be expropriated must be taken into account. If not, the owners of such property will be worse off than other owners from the same district. On the other hand, if the expectation of the scheme underlying expropriation leads to a general depreciation of value, then it is this new value -- not the original value -- that is relevant for calculating compensation. The crucial question is what the actual value is, when expropriation takes place.
\end{quote}

We mention that the problem analyzed by Castberg in this passage has been considered in many jurisdiction, and is dealt with in common law by the so called \emph{no-scheme} rule. This is more a principle than a single rule, and it is typically understood as a mechanism that is meant to ensure that changes in value due to the scheme underlying expropriation are disregarded.\footnote{For an history of the rule in common law (primarily the UK), which also illustrates the difficulty in interpreting it and applying it to concrete cases, we point to Appendix D of Law Commission Report No 286, 2003} In comparative terms, Castberg appears to favor a \emph{narrow} interpretation of the principle -- a restrictive view on when additional value due to the scheme should be disregarded -- quite close in spirit to the so called \emph{Indian} case from 1939\footnote{\emph{Vyricherla Narayana Gajapatiraju v Revenue Divisional Officer, Vizagapatam} [1939] AC 302.}, which was been much discussed in common law and was dealt with extensively by the House of Lords as late as in 2004.\footnote{In the case of \emph{Waters and other v Welsh National Assembly} [2004] UKHL 19. 
The primary precedent for a broader interpretation of the non-statutory no-scheme rule, on the other hand, is \emph{Pointe Gourde}, \emph{Pointe Gourde Quarrying and Transport Co v Sub-Intendent of Crown Lands} [1947] AC 565, PC, 572, per Lord MacDermott. This case proved highly influential for the understanding of compensation rules in the post-war period, in many common law jurisdictions, but has recently been challenged by a renewed interest in more narrow viewpoints such as that expressed in the \emph{Indian} case, see  \cite{newuk} and also the case of \emph{Star Energy Weald Basin Limited and another (Respondents) v Bocardo SA (Appellant) [2010] UKSC 35}.}

In the context of Norwegian law, it is of particular interest to note how Castberg's views in this regard is arrived at through considering the constitution itself, founded on the principle of equality.\footnote{In this way, he arrives at a narrow no-scheme rule quite abstractly, and through a different route than the one adopted in the \emph{Indian} case, where the outcome appears to have turned crucially on the particular facts in the case, a close reading of precedent, as well as the perceived fairness of the result.} He does not, therefore, engage in any reasoning based on the extent to which it can be regarded as socially fair for the public to pay compensation for value that encompass the beneficial consequences of the project itself. Crucially, he does not address the concern that this can be seen as a form of double payment. Such pragmatic, utilitarian reasoning was not widely adopted in the legal tradition Castberg was part of, and his theory appears as an example of constitutional absolutism. 

However, against the idea that this style of reasoning is ``owner friendly'', let me point out that his absolutism based on the principle of equality can lead to rigid interpretations that disfavor property owners. For instance, it was regarded as beyond doubt by Castberg that owners of expropriated property could not claim compensation based on the special want of the acquiring party. This, apparently, should also apply quite generally. He continues as follows, immediately after the passage quoted above.

\begin{quote}
The situation is different if the property has increased value due to the expectation that it will be expropriated. The owner can not demand that this increase is compensated since that would be the same as giving him a special advantage compared to those from whom no property is expropriated.
\end{quote}

While Castberg's view appears to have been shared by many academics of his day, and was also, to some extent reflected in case law from the Supreme Court,

The very nature of the system for deciding appraisal disputes gave the local appraisers great freedom in adapting the rules to suit the concrete circumstances of the case. To quite some extent, this would also involve making an assessment of what was regarded as a fair and just outcome, but on a case by case basis, not necessarily leading to special rules for specific types of cases. Indeed, when one looks more closely at case-law from the Supreme Court, one sees that there was  great tolerance for the use of discretion in the appraisal courts, vested within an absolutist theoretical framework.

As long as appraisal courts did not cross the line with regards to the constitution, they were largely allowed to adapt more pragmatic viewpoints. But such viewpoints were \emph{not} extensively codified in terms of special principles used to deal with special case types or issues, which the local courts where then obliged to follow in future cases. Rather, it arose as a logical consequence of the way in which appraisal disputes were organized, giving room for discretion, demanding consultation with laymen from the local communities, also on matters of legal interpretation. Hence, with absolutism as the theoretical underpinning of the system, a pragmatic approach to compensation was largely achieved \emph{indirectly} through a \emph{decentralized} system which gave local courts great freedom when applying the law. 

Again, the way in which the no-scheme rule was applied serves as an excellent illustration. On the one hand, the theoretical views of Castberg were widely accepted, but at the same time they were regarded as general guidelines that would necessarily have to be adapted to the circumstances. Moreover, it was not unheard of for the appraisers to disagree with the judge about how this should be done, and to award compensation according to a different understanding of the law than that favored by the judge. 

This happened, for instance, in the case of \emph{Tuddal}, where land was expropriated for construction of a power grid, and the expropriating party also acquired the right to use a private road.\footnote{Rt. 1956 p. 109}. According to the judge in the high appraisal court, who seems to have followed the teaching of Castberg, compensation should be awarded solely on the basis of what the owners stood to lose, calculated in this case based on the increased cost in maintaining the road resulting from increased use. However, the lay appraisers found this result unreasonable and awarded compensation also for the special value the use of the road would have for the acquiring party. The Supreme Court, although they found fault with the argumentation relied on by the appraisers, agreed that such compensation was possible in principle. The presiding judge offered the following perspective.

\begin{quote}
Since they were the private owners of the road, A/S Tuddal could, before the expropriation, refuse to let the Water Authorities to make use of it. Hence it might be possible for A/S Tuddal, through negotiation and voluntary agreement with the Water Authorities or others with a similar interest, to demand a reasonable fee, and in this way achieve a greater total benefit than full compensation for damages and disadvantages. Following the expropriation, it is no longer possible for A/S Tuddal, in its dealings with the Water Authorities, to economically benefit from their ownership of the road in this way. If the company suffer an economic loss as a result of this, I believe they are entitled to compensation. Whether or not such an opportunity as I have mentioned -- all things considered -- was present at the time of the expropriation, falls to the appraisal court to decide, on the basis of whether or not an economic loss is suffered beyond that which follows from damages and disadvantages. On this basis, I assume that the high appraisal court's decision to awarded compensation for the value of the right of way that is acquired can not -- in and of itself -- be regarded as an erroneous application of the law.
\end{quote}

The Supreme Court's reasoning illustrates two main points. First and most notably, we see how the Supreme Court adopts absolutism in its interpretation of the law, and makes sure, through careful use of wording, that the compensation for the value of the use of the road is not conceptualized as compensation based on the value of the road to the acquiring authority, but rather as compensation for the loss of potential profit following from a voluntary agreement. Hence the appropriateness of this form of compensation follows from the requirement that full compensation should be paid, based on the owners' loss. This particular interpretation of full compensation led to arguments in the post-war period, regarding whether or not owners had a right to compensation based on the loss of profit from hypothetical voluntary agreements with the acquiring party. In the end, a consensus formed that this type of compensation should not in general be awarded.\footnote{NUT 1969 nr. 2, Part 2, Chapter 4, Section 2.E.}

Despite this, we think \emph{Tuddal} is very interesting, also for the law as it stands today. It illustrates a second point, in particular, which also seems more relevant for our paper. We notice, in particular, the clear sense of commitment and loyalty to the procedural system displayed by the Supreme Court in its reasoning. This sentiment might be mostly implicit, but there can be no doubt, especially in light of the dissent from the judge in the high appraisal court and the legal theorizing of the day, that the Supreme Court went far in defending the discretion of the laypeople, as a \emph{systemic} feature. They seem to have actively sought out ways in which to legally justify the decision reached by the laymen, and to test with great caution whether it was truly outside the permissible legal boundary, or simply an exercise of the lay judgment that the system presupposed. 

This impression of the case is accentuated when we consider other cases dealing with the same and similar issues, and where a similar tendency to defend the role of the laypeople in the appraisal process can also be identified. A particularly clear expression of this can be found in \emph{Marmor}, a different case from 1956, where the Supreme Court overturned a decision made by the high appraisal court on the grounds that the court had not engaged in an assessment that had wide enough scope to do justice to the constitutional principle of full compensation, and the principle of evaluation by impartial laymen.\footnote{Rt. 1956 s. 493.} The case involved expropriation of a private railway track, for the construction of a public railway, and it was clear that the track which was being expropriated did not have market value in general. The expropriating party hence argued that the value of these tracks to the public railway should not be taken into account when calculating compensation, and the high appraisal court agreed with this, pointing to the standard teaching of the day. The Supreme Court disagreed, however, and felt that a standardized approach to the case was inappropriate given the circumstances. The first voter, in particular, made the following remarks.

\begin{quote}
In my opinion one can not simply assume that a property does not have market value when it has no value for anyone other than the expropriating party. The question needs to be assessed concretely. I agree with the expropriating party -- as has also been confirmed on several occasions by the Supreme Court -- that in general one should not take into consideration the special value that the purpose of expropriation gives the property. This should not lead to a spike in compensation payments. On the other hand, I can not agree that it is automatically reasonable, or in keeping with Section 105 of the constitution, if the expropriating party in cases like this one could acquire property at a price that is below what it would be natural and commercially appropriate to pay in a voluntary purchase.
\end{quote}

Again we notice that there are two main building blocks used in the argument; firstly, a reference is made to the constitution, reflecting the absolutism of the day, and secondly, a reference is made to the need for \emph{concrete assessment}, reflecting strong confidence in the integrity and autonomy the appraisal procedure. Moreover, we notice how absolutism regarding the constitutional protection of property owners is \emph{not} used to argue for specific rules or principles that should be adopted, but rather to back up the argument that compensation should result from real assessment, and not be overly reliant on such rules, not even when these rules appear sound in general, and have been backed up by a series of Supreme Court decisions.

In addition to making these overreaching remarks, the Supreme Court also gave pointers as to the kinds of facts that should be considered. For instance, they paid particular attention to the wider \emph{context} of expropriation, and the manner in which expropriation was used to benefit certain interests. They also noted how it had come to replace voluntary agreement as the standard means of acquisition for this type of development, therefore effectively preventing a market from developing. In the word of the first voter, below.

\begin{quote}
I also point to the fact that the case concerns an area of activity where the expropriating party has a de facto monopoly which makes it impossible for anyone else to make use of the property for the same purpose. This in itself makes it questionable to simply assume that the lack of financial value for other purchasers provides the appropriate basis for calculating compensation. When considering this question, it is also appropriate to take into account that we have lately seen a great increase in the use of expropriation to undertake projects such as this. Compulsion is becoming the primary mode for acquisition of property -- not voluntary sale following friendly negotiations.
\end{quote} 

In our opinion, the primary historical importance of this decision, which we think makes it highly relevant even today, is not to be found with regards to the particular legal interpretation of the no-scheme rule that the Supreme Court appears to endorse. Indeed, it seems to us that it would be an \emph{erroneous} reading of this judgment to take it as expressing support for a general principle that compensation can always be based on the value of hypothetical agreements that could have been made with the expropriating party. Rather, we believe that the judgment should be read as arguing against the blind obedience to \emph{any} such general rules for calculating compensation. At the very least, it seems clear upon closer inspection of the argument that the main objective of the court was not to express any particular view regarding the content of the no-scheme rule, but to instill to the appraisal courts that they could not use this rule as an excuse not to engage in concrete assessment to ensure a reasonable outcome in keeping with the constitution.

We believe this point is important to stress. It illustrates how absolutism need not, and did not, result in a rigid system with little room for assessment based on justice and fairness, broadly conceived. Quite the contrary, the absolutism endorsed by the Supreme Court, and inherent in the Norwegian system of appraisal courts, was not characterized by blind obedience to specific rules, like those proposed by Castberg. Rather, the system was flexible, and it was explicitly intended to function such that fairness assessments based on concrete circumstances could be accommodated. 

Going back to even older legal scholarship, we see that this view on the meaning of absolutism has a long history in Norway. For instance in the work of the famous 19th Century scholar Aschehough, who stressed the link between the constitution and the appraisal procedure when he considered the (then) hypothetical situation that legislation was introduced with the specific aim of reducing the level of compensation payments following expropriation. We quote from \cite[p.48]{asch} 

\begin{quote}
If it becomes common practice to award compensation payments that are unreasonably high, this would make important public projects more expensive and difficult to carry out, greatly to the detriment of society. In many cases it might not be possible to rely on legislation to prevent such excessive compensation payments, since this would restrict the appraisers too much. To some extent this might be possible, however, and as far as it goes, parliament must be permitted to do so. However, if enacted rules clearly lead to less than full compensation in an individual case, they will be overruled by Section 105 of the constitution, and fall to be disregarded in that particular case.
\end{quote}

This quote is important because it does not rely on any particular interpretation of the constitutional demand for full compensation, but sees this inherently as an issue that needs to be resolved by concrete assessment of individual cases. Absolutism to Aschehough implies freedom and responsibility for the appraisers; freedom to judge individual cases by its merits, and a responsibility to award full compensation, irrespectively of any specific rules that might be in place to curtail excessive payments. The important point is that Aschehough here sees absolutism as a principle that should be applied to cases, not to principles. He does \emph{not} argue that rules introduced to limit compensation payments would be inadmissible merely because they might sometimes suggest less than full compensation. Rather, he takes it for granted that it falls to the appraisal courts to \emph{apply} the rules in a way that would prevent such outcomes. As long as the appraisal courts remain free to apply the rules in such a way that full compensation is awarded, specific rules intending to prevent excessive payments can happily coexist with absolutism.

The subtle view taken by Aschehough was largely overlooked in debates following the introduction of the Compensation Act 1973, however, even though this act introduced radical rules of exactly the kind he had predicted and considered almost 90 years earlier. More generally, and as we will discuss in more detail below, the 60s and 70s appears to be a period when the crucial role of the appraisal procedure was to some extent forgotten, and also undermined, following a heated political and ideological debate regarding the appropriateness and admissibility of introducing rules to ensure that compensation payments were brought down to a lower level. This had deep and lasting effects on Norwegian compensation law, and it is popularly described as a period when the social democrats won recognition for the principle that social fairness suggested the introduction of compensation rules and disregards that were more extensive than what had previously been considered appropriate. 

This was conceived of as a fight for social justice against outdated and conservative ideas of constitutional absolutism. But it seems to us that this view of the history of Norwegian compensation law is erroneous, and largely unhelpful. The approach taken by Aschehough, in particular, placing emphasis on the important role played by the appraisers in achieving fairness and justice in concrete cases, does not appear to contradict social democratic goals at all. In fact, it seems that his approach might be better suited to serve such goals, and to accommodate a variety of different political opinions and ideas, than an approach which is based on attempting to flesh out in painstaking detail how the appraisal courts should go about achieving the balance between social fairness and owners' rights. We will return to this point later, but first we will take a closer look at the history of the radical Compensation Act 1973 and the censorship to which it was subjected by the Supreme Court, leading to the Compensation Act 1984, currently in place.

%%%%%%%%%%%%% New section

Following the Second World War, the social democratic \emph{Labour Party} gained a secure grip on political power in Norway, and many reforms were carried out that would reshape Norwegian society. One of the most important reforms concerned the introduction of extensive planning law to ensure that land use was put under public control, and in this period expropriation was also becoming used more extensively to further public projects, such as the large scale construction of hydro-power to ensure general supply of electricity.\footnote{References.} As a result of these changes, the opinion was soon voiced that there was a need for a more uniform approach to compensation, which collected some basic principles in a common body of written law, and which could serve to bring compensation payments down. This, it was felt, should be done in order to facilitate more efficient implementation of public policies. 

In 1965, as a result of these new ideas, the so called \emph{Husaas committee} was appointed by the King and charged with the task of assessing the compensation rules currently in place.\footnote{Appointed by the King in Council on 6. Aug 1965.} They were also ordered to make a concrete suggestion regarding the need for additional principles of compensation, and if these should be given in the form of a special compensation act. Initially there was some doubt as to the extent to which is was at all permissible to give rules regulating compensation, as the constitution itself addressed the matter. However, the committee noted that existing legal scholarship seemed to suggest that such rules could be given, and that, moreover, specific rules had already been introduced, for instance in relation to expropriation for hydro-power development.\footnote{See Section 16 of the Watercourse Regulation Act 1917 (Act No. 17 of 14 December 1917 relating to Regulations of Watercourses).} Hence the majority of the committee concluded that while the constitution provided an important outer barrier, serving also to protect owners' rights against acts of parliament, it did not prevent legislation providing legally binding guidance as to how the notion of full compensation should be understood and applied by the courts in appraisal disputes.

Moreover, the majority pointed out that a vague general principle such as that provided by the constitution would by necessity have to be interpreted in order to be applied to concrete cases. Hence it was not only permissible, but also desirable, for parliament to give more detailed instructions as to how is should be applied and understood by the courts and the appraisal courts. Leaving it to the judiciary to flesh out the exact meaning of full compensation through case law, it was felt, was not appropriate in a regulatory regime where expropriation had become increasingly important as a means to ensure modernization and development of critical infrastructure. 

%In addition to this, the Supreme Court itself had recently expressed its support for a new view on regulation of property use, supported by contemporary legal scholars and politicians, whereby the State was regarded as having wide discretionary powers to determine how property should be used. This right to regulate, in particular, was increasingly coming to be seen as a right that did not infringe on property rights, so that the State would not have to compensate owners if they exercised it, except in special cases.\footnote{See, in particular, Rt. 1970 p. 67.}.

More generally, constitutional law was increasingly seen from a pragmatic utilitarian point of view, as a collection of foundational principles that aimed to stress the importance of ensuring social fairness just as much as individual justice. This was by no means a consensus view among legal scholars, however, and it was particularly contentious with regards to property. As a result, some disagreed strongly with the very idea of legislation regarding compensation, and tensions arose that have led to much legal controversy and are still important in the law today. 

This problem area was mapped out in some detail by the Husaas committee, who traced the pragmatic view on compensation, identifying it using the following quote by the leading scholar Knoph from \cite[p. 113]{knoph}.

\begin{quote}
Since Section 105 is a rule prescribing practical justice, directed at parliament, and not an ethical postulate of absolute validity, it must be permitted to make technical legal considerations, so that one accepts compensation rules that lead to correct and just results on average, even if it does not grant the owner full individual justice in every case.
\end{quote}

This view was becoming influential in the 60s, see for instance \cite{grunn,opshal}. However, there were many that disagreed vehemently, based on absolutism principles \cite{robb2,schj}. The latter describes Knoph's reading of the law scathingly as follows, on p. 44.

\begin{quote}Luckily it has not had any effect on judicial practice whatsoever. No court of law would accept that compensation should be set according to a norm that may be practical and just in general, but does not grant the owner full compensation in all individual cases.
\end{quote}

When assessing the current state of the law, the Husaas committee encountered many manifestations of the tension between a pragmatic and principled understanding of the protection of property, and in proposing a set of general principles for compensation which are still, in a modified form, with us today, they engaged in a fine balancing act. While they were clearly aiming to move in the pragmatic direction, the were nevertheless cautious, and they refrained from encoding principles that would appear too offensive to the absolutists, even if the pervading political sentiment was that more was needed to ensure a more effective state regulation of property use. 

%%%%%%%%%%% Can go earlier

The basic starting point for compensation that the Husaas committee identified in case-law was that of compensation based on loss of value according to ``foreseeable use'', as this could be reflected either in the owner's own use, or the market value, reflecting the value of such use as an average buyer might make of the property.\footnote{See, for instance, Rt. 1925 s. 47 and Rt. 1926 p. 669.} Moreover, the committee took the view that deviating from this starting point would not be constitutionally admissible, and instead they sought to codify what they saw as the existing interpretation of Section 105 in this regard. They concluded as follows.\footnote{NUT 1969 nr. 2, Part 2, Chapter 5, Section 3.}

\begin{quote}
It is the view of the committee that it is correct to encode in the act the principle that the owner is entitled to compensation based on the value that results from taking into account the foreseeable and natural use of the property, given its location and the surrounding conditions. The exact meaning of ``natural and foreseeable'' use must be decided after a concrete assessment in individual cases. By encoding this general principle, however, it will become clear that compensation should not be based on private or public plans unless these plans coincide with the use of the property that is natural and foreseeable, independently of the scheme underlying expropriation.
\end{quote}

This view, however, was not in keeping with the political motivation for an act regarding compensation, and the Department of Justice deviated from it in their final proposition to parliament. Instead of encoding existing  principles, they sought a pragmatic system whereby compensation would in general be based on the value of the \emph{current use} of the property, thus preventing the public from having to pay a financial premium to owners based on possible future value that would in any event, in most cases, be reliant on public development permissions. %Such permissions, it was argued, could never be foreseeable in circumstances when it was in the public interest that the property should be expropriated, and hence all future development potential should in principle fall to be disregarded.

The Department commented on this as follows.\footnote{Ot.prp.nr.56 (1970-1971), p.19-20}

\begin{quote}
The Department is of the opinion that it is particularly important to arrive at a rule that can bring the assessment of property value down to a realistic level, and believes that the natural starting point for such an assessment must be the current use of the property, especially for expropriation of real property. As mentioned, it is the opinion of the Department that a practice has developed that gives too much weight to more or less uncertain future possibilities for the property, something that has led to a sharp rise in compensation payments.
\end{quote}

After intense debate in parliament, where the minority center-right parties all opposed its introduction, the current use rule was eventually encoded in the Compensation Act of 1973, in Section 4 nr. 1.\footnote{Act No 4 of 26 March 1973 Regarding Compensation following Expropriation of Real Property.} This, moreover, was largely seen as a social democratic victory, and a clear indication that the absolutist view on property protection was increasingly losing ground to a more pragmatic approach. In particular, when clarifying their principled starting point regarding what should count as \emph{realistic}, the Department made the following assertion regarding the scope of the constitutional protection offered in Section 105, showing the ideological underpinnings of the act.

\begin{quote}
However, a right to complete -- or almost complete -- equality can not be derived from the constitution. It must be taken into account that we are here discussing equality with regards to increases in property value that are, in themselves, undeserved. [...]  %  The starting point must be that it is not, in and of itself, contrary to the constitution that one property owner do not benefit from the same increase in value as another, when the increase in value, for both of them, is due to public investment and does not stem from their own efforts. \\ \\
Certainly, it would be best to avoid any kind of inequality, if it was possible. But the examples we have considered illustrate that, today, inequality between property owners is tolerated with regards to public investments and regulation, and that, moreover, practical and economic considerations dictate that we \emph{should} make use of differential treatment in this regard.
\end{quote}

Here we see a clear expression of support for a pragmatic view of property rights, echoing Knoph, but going much further. In particular, the Department explicitly states that differential treatment is appropriate in the context of expropriation, and, by implication, that this should be done precisely to avoid compensation payments that include compensation for "undeserved" increases in value. Also, in proposing that compensation payments should be based on current use, the scope of "undeserved value" was made extremely wide -- in principle it would seem to include \emph{any} value that could be attributed to an as of yet unrealized potential that the property in question might have. The question of whether or not this value was reflected in the market value of the property, in particular, was not regarded as relevant. This was in itself very radical, since market value compensation had been the dominant starting point for reasoning about compensation following expropriation.\footnote{References.}

It seems to us that we should not underestimate the conceptual significance of this change in perspective. Here, the Department stood firmly behind a pragmatic view, where social fairness was the overriding constraint, also with respect to constitutional property protection. However, on taking this view to its logical conclusion, it was recognized that any general compensation rules that might be introduced should themselves be subject to a fairness test, so that, for instance, the current use principle could not itself be absolute or without exception. 

Rather, it could only be applied in so far as it served the overreaching goal of social justice and fairness which was, after all, regarded as the fundamental component of property protection that made such a rule possible. This, in particular, seems like a crucial observation, and one that has in our opinion been overlooked, with unfortunate consequence for the subsequent debate and development of the law. Indeed, it echoes the sentiment expressed by Aschehough that we quoted above -- similarly overlooked -- and thus it points to the existence of possible \emph{common ground} between absolutist and pragmatist views on compensation. 

Sensible voices from both camps, in particular, seem to arrive at the conclusion that in the end, there is no way around a \emph{concrete, contextual} assessment, where the assessment of social fairness and justice is held against the concrete circumstances of individual cases. Approaching such a view, from a pragmatist angle, the Department commented as follows.  

\begin{quote}
One is aware that the principle of current use compensation cannot be without exception. Even though this rule will be fair in general it can, in some cases, disproportionately disadvantage property owners. One has therefore suggested rules that modify the principle to some extent. These are given for somewhat different  reasons. \\ \\

One case addresses the situation when current use compensation means that a property owner will be significantly worse off that other owners of similar property in the same district, according to how these properties are normally used. In these cases, the principle of equality suggest that the owner receives some -- but not necessarily full -- compensation for the inequality that would otherwise arise from the fact that his property was made subject to expropriation. %Etter departementets oppfatning har en ekspropriat etter grunnloven ikke noe krav på å bli satt helt i samme stilling som om ekspropriasjonen ikke var skjedd, en forskjellbehandling innen rimelige grenser må grunnloven tillate når dette tilsies av tungtveiende samfunnsmessige grunner. 
\end{quote}

This principle was eventually encoded in the Compensation Act 1973 Section 5 nr. 1-3, and they would prove highly controversial. In \emph{Kløfta}, in particular, the Supreme Court interpreted additional compensation according to Section 5 nr. 1 as being \emph{obligatory} in a range of cases when the intention had clearly been that the rule should be used sparingly, and only when the courts considered it reasonable to do so. In this way, and possibly inadvertently, the Supreme Court defended owners' interest by \emph{limiting} the power of the appraisal courts. This, however, led to a change of perspective on the law, with the role of direct guidance from the Supreme Court becoming increasingly important, and the role of the appraisal courts in interpreting the law becoming increasingly narrow.

Before moving on to consider this in more detail, we should not forget the second exception to the current use rule that the Department introduced. In some sense, it is the more interesting of the two, even if it has been largely forgotten. In fact, we think directing attention to it, and to the idea that it captures, is highly relevant for one of the most pressing issues regarding compensation today, namely the case of \emph{commercial expropriation}, i.e., when expropriation is used as a tool by commercial actors who with to acquire property and who enjoy a financial benefit from being allowed to employ compulsion when doing so. 

The second exception rule from the Compensation Act 1973 sought to address precisely circumstances such as this, as it addressed the question of the \emph{power balance} between the expropriating party and the owner and the \emph{purpose} of the expropriation. In the words of the Department:

\begin{quote}
The second modification we make has to do with the relationship between the property owner and the expropriating party. If the use of the property that the expropriation presupposes gives the property a value that is significantly higher than the value suggested by current use, this will entail a transfer of value from the property owner to the acquiring party. In some cases this might be unreasonable. As an example of when this can become an issue, we mention an agricultural property that is expropriation for the purposes of industrial production. In such a case it might be natural that the owner receives a certain share in the increased value that the new use of the property will lead to.[...] %This would be different than, say, a situation where an agricultural property is expropriated for constructing a road or for setting up recreational outdoor grounds. In such cases, the expropriation will not lead to any such economically advantageous use of the property that will give the expropriating party an economic advantage. 

To establish a flexible system, the Department has concluded that it is practical that the King gives rules concerning the cases where an enhanced compensation payment, based on these principles, might be appropriate. This should not be decided by individual assessment, but governed by rules for special case types. Hence, the proposed Act states that the King can pass regulation concerning this matter.
\end{quote}

Again, this quote expresses the crucial insight that fairness with regards to compensation following expropriation can not be arrived at without adapting the rules to the circumstances. For any pragmatic approach to compensation, the \emph{context} of expropriation must by necessity come to play a crucial role, especially if the starting point is explicitly taken to be that compensation should only encompass the "deserved" value. What this value should be taken to be, in particular, can hardly be determined once and for all and in general terms, but must rather be subject to continuous revision depending on how expropriation is \emph{actually used} in society, the purpose it is meant to serve, the parties who benefit, and the degree of commercial economic benefit that results for individual parties other than the original owners. Indeed, stipulating that compensation should be "deserved" appears to provide a benchmark that is just as unclear as the stipulation that compensation should be "full". This, however, might be a source of inspiration rather than despair. It seems, in particular, that the inherent ambiguity of these terms allows us to draw two conclusions: first, that they might very well have the same meaning, and second, that they cannot possibly be defined once 
and for all by any act of parliament, or by any decision in the Supreme Court.

This, in turn, suggests that the Norwegian system of appraisal courts, and the presence of laymen in the decision-making processes of these courts, bears crucial influence on how well the Norwegian system is able to meet both the requirement of social fairness and justice for the individual. Unfortunately, however, the procedural, contextual aspect of fairness and justice was not recognized following the passing of the 1973 Act, with attention shifting towards issues of legal interpretation that arose from it. The primary such issue, and the most serious one, concerned the question of whether the law as such was in breach of the constitution. This  was eventually resolved by the Supreme Court in the case of \emph{Kløfta} in 1976.\footnote{Rt. 1976 p. 1}. 

Here the 1973 Act would be significantly reinterpreted to make it appear less offensive to the constitutional standard of full compensation. However, taking a broader perspective, it seems to us that \emph{Kløfta} largely accepted that the intention behind the act should be respected, and that appraisal practice needed to be adjusted accordingly. In this, the Supreme Court signaled loyalty to the political system and the democratic process. However, in implementing this adjustment in practice, they also, possibly inadvertently, set up a system where the role of the local appraisal courts appears to have been undermined, and where the Supreme Court itself assumed greater control over how the compensation law was to be applied in concrete cases. This characterizes the current state of the law, which we describe in more detail in the following section.

\section{Market value, but \emph{we} determine the market! The era of centralization and pragmatism by regulation}\label{sec:regab}

When the constitutionality of the Compensation Act 1973 came before the Supreme Court in \emph{Kløfta}, they chose to sit as a grand chamber and they reached a decision under dissent, being divided into two fractions, consisting of 9 and 8 supreme judges respectively. However, both fractions approached the problem of constitutionality by endorsing an interpretation of Section 5 nr. 1 in the Compensation Act 1973 that gave the exception to the current use much wider scope than what had been intended by parliament. The majority went farthest, and unlike the minority they also regarded the compensation payment in the concrete case to be insufficient. The first voter for the majority commented as follows on the constitutional aspect of the case.

\begin{quote}
[...] But the main question in this case, is whether or not it is in keeping with Section 105 to generally award compensation at a level below the market value that could legally be estimated, and that the owner could actually have achieved, if expropriation had not taken place. In my view, this involves allowing expropriation to transfer a right that the owner had, with a value to which he was entitled. If he is refused compensation for this value, he would, depending on the circumstances, be left significantly worse off than others in a similar position, who owns property that is not expropriated. Such a result I cannot accept. It would be a breach of established customary law and a practice that has been established throughout the years both by the appraisal courts and the Supreme Court. I refer particularly to Rt 1951 s. 87 (particularly p. 89, Opdahl). This practice is in itself a significant contribution to interpreting Section 105 on this point.
\end{quote}

We notice in particular the emphasis placed on \emph{market value} in the majority's reasoning. This may appear to be in keeping with an absolutist doctrine, but as we have mentioned, and will argue in more detail below, it can have unfortunate, possibly unintended, consequences for property owners, especially when combined with a restrictive view on what counts as foreseeable future development. We note, however, a technical point that might be of some significance for the interpretation of \emph{Kløfta}; instead of stating outright that a market value rule follows from the wording of the Constitution as such, the majority takes the view that this interpretation suggests itself based on the compensation practice that had currently been established. This might limit the scope of the majority's remarks in this regard, but it also serves to give further support to the claim that the role of the appraisal courts, and their assessments, still had a strong position in Norwegian compensation law at the time of \emph{Kløfta}. 

We remark that the minority disagreed on the constitutional status of the market value rule. Indeed, it was in this regard that the difference of opinion between the minority and the majority was most clearly felt. The minority, in particular, explicitly rejected the view that this rule could be derived from the constitution itself, and they also disagreed with the understanding that it would have status as a constitutional rule simply because it had been adopted in practice. This bestowed merely the status of ordinary legal precedent. In the words of the first voter for the minority:

\begin{quote}
Case-law in this area cannot be understood as preventing parliament from changing the rules in accordance with what they regard as necessary. That would prevent a reasonable and natural development and would not be in keeping with the consensus view that Section 105 of the constitution is a rule that must be interpreted in light of, and adapted to, how society has developed and how the law is viewed. I believe the practice that have evolved cannot be decisive if a new situation and new needs require a different solution. Whether the Compensation Act is in breach of the right to full compensation enshrined in the constitution, must depend on an interpretation of the wording in the constitution itself.[...] \\ \\
In my opinion, neither the intentions of parliament nor the way they are sought implemented through Sections 4 and 5 are in breach of the equality principle upon which Section 105 of the constitution is based. It does not follow from the constitution that an owner is in all circumstances -- and irrespectively of the economic forces from which the market value results -- entitled to compensation that is at least as great as the greatest legal value that the property could represent on a free market. A different matter is that Section 105 of the constitution could be important to the interpretation and application of the rules.
\end{quote} 

Hence the market value rule was explicitly renounced as a constitutional principle by the minority, who nevertheless conceded that the constitution could be used to interpret Sections 4 and 5 of the Compensation Act 1973. Both the minority and the majority agreed, however, that  it would be wrong to go on to consider Section 4 of the Compensation Act 1973 in isolation. For the majority, this would clearly have led to the Compensation Act 1973 being held to be in breach of the constitution, something that was avoided since the Supreme Court chose to consider the law as a whole, with the majority using the reasoning detailed above to argue for a new interpretation of Section 5, rather than as a means to undermine Section 4. Still, their interpretation of Section 5 went well beyond what parliament had intended, leading some scholars to claim that \emph{Kløfta} should be read as holding that the Compensation Act 1973 was unconstitutional.\footnote{References.} In the words of the majority:

\begin{quote}
The purpose of this rule is to award compensation beyond current use in cases where valuations according to Section 4 could be in breach with Section 105 of the Constitution. As it stands, Section 5 nr. 1 is not sufficiently suited for this purpose. By its wording it gives the appraisal courts an opportunity to assess whether or not it is reasonable to award additional compensation, even when the conditions for this is otherwise met, and even then with the limitation that the compensation would otherwise be significantly unreasonable. Such a free position for the individual appraisal courts -- without possibility of legal appeal -- would not be in keeping neither with the purpose of the rule nor the demand for full compensation set out in the constitution.
\end{quote}

On this basis, the Supreme Court chose to interpret Section 5 nr. 1 in such a way that whenever the conditions were fulfilled, the appraisal courts were \emph{obliged} to award additional compensation, and on this basis they found that the property owners in \emph{Kløfta} was entitled to have their compensation looked at again, in a new round before the appraisal courts. The minority agreed in principle, yet did not go as far as the majority, concluding that based on the particular facts at hand Section 5 had been adequately considered by the appraisal court in this particular case.

The upshot of \emph{Kløfta} was that Section 5 nr. 1 came to be seen as an obligatory rule, leading to compensation having to be enhanced whenever the current use rule led to payments that did not reflect the market value of comparable properties. However, the conditions stated in Section 5 nr. 2 and nr. 3 were still regarded as relevant, and in interpreting these conditions, a body of law developed whereby the market value rule was applied in a way that would come to involve significant reduction in compensation compared to what would result from practice as it had been prior to the Compensation Act 1973. In this way, the pragmatic approach proved triumphant, not because current use value was introduced as the general starting point, on the contrary, but because a range of new disregards were introduced to reduce the level of compensation in a range of different circumstances. After \emph{Kløfta}, in particular, the following rules were all considered legitimate ways to decrease the level of compensation.

In Section 5 nr. 2 and nr. 3, the following three disregard principles are encoded, all of which are, to varying degrees, still important in compensation law today.

\begin{enumerate}
\item Changes in value that are due to the expropriation scheme or investments or other activities should be disregarded, both when these are already carried out as well as when they are planned, c.f., Section 5 nr. 2 of the Compensation Act 1973.
\item To the extent that it is regarded reasonable, \emph{increases} in value that are due to public plans or investments should be disregarded, irrespectively of whether or not they have already been carried out, c.f., Section 5 nr.2 of the Compensation Act 1973.
\item An increased value falls to be disregarded if it results from considering a use of the property which is not in accordance with public plans, c.f., Section 5 nr. 3 of the Compensation Act 1973.
\end{enumerate}

These rules severely limits the level of compensation payments, and in many cases it appears to make the principle of full compensation based on market value rather illusory. Notice, in particular, that on the one hand, disregard rule nr. 2 can be applied to disregard the value arising from any use of the property that is not in keeping with the current public plan, whereas disregard rule nr. 3 can be used to also disregard any value that is due to this plan. While the outcome, logically speaking, should then be that no compensation can be awarded whatsoever, the disregard rule nr. 3 is usually seen to revert back to the current use compensation in such cases. For instance, if agricultural land is expropriated for the purpose of a motorway, and it would otherwise appear foreseeable that it could be used for housing, the compensation will be based on agricultural use because the value for housing is disregarded according to disregard rule nr. 3.

In practice, then, with virtually all novel economic activity making use of land is dependent on acquiring new planning permissions, the current use rule will typically be applied as intended by the Compensation Act 1973, with the only difference being that it is not thought of or described as such.\footnote{A similar point was made in \cite{stor}.} Rather, outcomes that are basically in keeping with current use thinking will be designated as "full compensation based on market value" -- the standard phrase adopted in most appraisal judgments -- and the fact that the outcome is equivalent to current use compensation remains unclear until one considers the range of disregards that have been applied. In this way, the state of law that followed \emph{Kløfta}, and which has largely been upheld and codified in later case-law, is greatly influenced by, and largely in keeping with the intentions behind the Compensation Act 1973. 

The Compensation Act 1984 was eventually introduced to reflect the principles laid down in \emph{Kløfta}, but it did not in any essentially way change or influence the course of the law that had already been set. Its main purpose was to bring the wording of the legislation more into keeping with how the law was interpreted by the Supreme Court. It explicitly returned to the starting point of the Husaas committee, namely that the compensation should be based on the value of the "foreseeable use" that the owner himself, or an average buyer, might make of the property. But it maintained and endorsed disregard rules nr. 1-3, except for restricting disregard nr. 2 to public investments, such that increased value due to public plans currently in place could not be disregarded.\footnote{In this way, the paradox mentioned above, that compensation could become impossible to award because there was no possible basis upon which to calculate it, was avoided.}

Beyond this, it did not give any further guidance as to how the disregard rules should be understood or applied, nor did it consider or resolve the question of when, if ever, they would need to be applied with caution in order not to go against the constitution. However, it was expected that cases where such issues arose would be resolved by strict adherence to firm principles, and that unless these principles could be derived from the Compensation Act 1984 itself, they should be laid down by the Supreme Court. Deciding on the law in such matters should not, in particular, be left to the discretion of the appraisers. The age when the appraisal courts were considered free to assess the cases based on their merits and directly against the overriding goal of achieving justice and fairness grounded in the constitution was over. Rather, an ethos had taken hold where the need to curb their freedom, in the interest of ensuring predictability and centralized control, was considered more important than upholding the system of lay judgment. 

As a result, difficult cases now routinely end up in the Supreme Court, who attempt to stick to established standardized rules as much as possible, but who will formulate new such rules for compensation of specific case types, if this proves unavoidable. As an example of this mechanism, it is enlightening to consider the case-law based on disregard rule nr. 3, which states that public plans currently in place are binding when calculating compensation. This rule cannot apply without exception, as recognized already by the Compensation Act 1973, since it may lead to outcomes that run counter to both the constitution and a common, rudimentary sense of fairness. 

One case which was considered by the Supreme Court in \emph{Østensjø} concerned land that was being expropriated for housing purposes, but such that one unlucky owner would only contribute land used for infrastructure that would serve the larger housing project.\footnote{Rt. 1977 p. 24} In this case, the Supreme Court agreed that he was entitled to compensation based on value of his land for housing purposes, irrespectively of the fact that \emph{his} land could not be used in this way according to the plan. However, in many other cases, the disregard rule is upheld even when it is hard to see it as either fair or just, simply on account of it having status as a general rule.\footnote{For instance in \emph{Malvik}, Rt. 1993 p. 409, where owners of property used for a motorway were only entitled to compensation based on current agricultural use because the regulation for motorway use was assumed binding for the compensation assessment.} One example is found in \emph{Sea Farm} which dealt with the issue of whether or not the owner of a commercial property should be awarded compensation for the value of investments carried out by the previous tenant.\footnote{Rt. 2008 p. 240} There was no doubt that the owner was entitled to these investments, but since the acquiring authority was the only purchaser who was likely to benefit commercially from them, no compensation was awarded for the loss of these investments. This, in particular, followed from a strict reading of the requirement that compensation should be based on the foreseeable use that an "average" buyer could make of the property, encoded in Section 5 of the Compensation Act 1984. Adherence to the wording used in the act seems to have taken priority over an assessment based on the facts of the case. It seems difficult to argue that it would be either unjust or unreasonable, in particular, to compensate the owner for investments that would prove commercially valuable to the acquiring party.\footnote{The decision was sharply criticized by a former supreme judge \cite{skog}.}

In our opinion, this example illustrates how the development of compensation law towards greater reliance on specific rules rather than concrete assessment based on general principles can be harmful, and how it also threatens to undermine the idea behind the special procedure used to decide appraisal disputes, which has a long history in Norwegian law.\footnote{One might ask if it has status of constitutional customary law, especially since it concerns the mechanism by which a constitutional rule is meant to be upheld.} It also seems to severely underestimate the extent to which compensation rules, when applied to concrete cases, must and should be interpreted based on the context of the case. It seems difficult indeed, if not completely impossible, to achieve social fairness and individual justice by a set of specific rules on the basis of which all legal issues can be resolved mechanically by blind application of such rules. %Moreover, it would be wrong to think that Section ... of the Appraisal Act 1917, encoding the principle that laymen should take part in the decision-making both with regards to legal and technical matters that arose in appraisal disputes.

In the following section we will address this issue in more detail, and we will argue for a different conceptual approach to compensation law, grounded both in the procedural tradition of appraisal courts and the more subtle parts of the absolutist and pragmatic theoretical traditions. It seems to use, in particular, that the most striking lesson that should be drawn from considering the history of Norwegian compensation law is that a \emph{contextual} view of compensation has been a common denominator that both the absolutist and pragmatist camps have endorsed. Unfortunately, this common element was overshadowed by political conflict regarding the weighing of different values. However, there can be little doubt that social fairness and individual justice should \emph{both} to be regarded as important objectives for compensation rules. Moreover, while they may sometimes be opposing, they need not be, and their exact relationship depends largely on the circumstances. It seems to us that it is simply inappropriate to let particular political sentiments regarding their relationship and relative importance, sentiments that are usually dependent on the particulars of the prevailing political, social and economic conditions, dictate the development of the legal framework for resolving compensation disputes.

Considering current trends and recent issues in expropriation law, particularly related to commercial expropriation, further suggests that a different perspective is needed on this matter. In particular, we believe it is time to recall the idea of the independent and impartial discretion of the appraisal court, relying on the good common sense of laymen as well as the legal expertise of judges. The appraisal courts should in our opinion be set with the task of more actively evaluating how fairness and justice is best served in individual cases, at least if the overall goal is truly to arrive at a socially fair and individually just compensation system. We discuss this idea in more detail in the final section below.

\section{``Natural horsepowers''}

In the early 1900s, Norwegian hydro-power was not subjected to much regulation, and waterfalls, having recently been discovered as an important supply of cheap electricity for industrial exploits, were rapidly falling into the hand of foreign speculators. In response to this, Norwegian politicians introduced legislation to secure national interests, the main provision being that concession from the state was made a requirement for anyone who wanted to acquire a waterfall.\footnote{References.} As a result, the market for waterfalls in Norway dwindled and the State assumed control of hydro-power exploitation. Unlike private investors, the State would tend to expropriate waterfalls rather than acquire them through voluntary agreements, and the question arose as to how the original owner should be compensated. This question, if resolved by a standard no-scheme approach, could easily prove shockingly unfair to owners of waterfalls. Presumably, since waterfalls could not be exploited for any significant commercial gain except through hydro-power exploitation, disregarding the hydro-power scheme when calculating compensation could lead to nil or close to nil being awarded to the owner. But this was not seen as an acceptable outcome, and instead the Norwegian courts introduced a special method to compensate waterfalls that gave the owner a \emph{share in the value of the hydro-power scheme} for which expropriation was taking place.

Norway did not at this time have any legislation specifically aimed at regulating compensation following expropriation, and when formulating the special rules for compensation of waterfalls, the Norwegian courts seems to have relied on an analogical application of the gross valuation techniques introduced in the Industrial Concession Act 1917 and the Watercourse Regulation Act 1917.\footnote{Act No. 17 of 14 December 1917 relating to Regulations of Watercourses and Act No. 16 of 14 December 1917 relating to Acquisition of Waterfalls, Mines and other Real Property}. Neither of these acts were aimed at compensating owners, but they relied on methods for assessing the potential and significance of hydro-power projects with respect to the question of whether or not a special concession from the State was required.\footnote{To acquire the waterfall and the right to regulate the water-flow respectively.} In effect, by relying on the methods of valuation introduced there, the compensation mechanism that was introduced deviated completely from the "value to the owner" principle. On the other hand, it also closely mimicked the manner in which owners of waterfalls would be compensated on the market in the early days, prior to the introduction of our concession laws, when speculators would pay for waterfalls on the basis of what they assumed to get out of them in subsequent hydro-power projects.\footnote{References.}

In the Supreme Court case of \emph{Hellandsfoss}, some 80 years after it was first introduced, the traditional method for compensation was still in use, and the Court described it as follows, starting from the observation that the general principles that were later encoded in the Compensation Act 1984 were of little use for determining the right level of compensation for waterfalls (my translation).\footnote{Rt. 1997 s. 1594.} 
\begin{quote}
The principle set out in the Compensation Act, Section 5, is that compensation should be determined on the basis of an estimation of what ordinary buyers would pay for the property in a voluntary sale, taking into account such use of the property as could reasonably be anticipated. For waterfalls, however, this often offers little guidance, and the value of waterfall rights have traditionally been determined based on the number of natural horsepowers in the fall, which are then multiplied by a price per unit. The unit price is determined after an overall assessment of the waterfall, including the cost of the scheme, its location, and levels of compensation paid for similar types of waterfalls in the past. The number of natural horsepowers is calculated by the formula "natural horsepower = $13.33 \ \times \ Qreg \ \times \ H$", where $Qreg$ is the regulated water flow and $Hbr$ is the height of the waterfall.
\end{quote}

In this formula, $Qreg$ represents a quantity of water, measured in cubic metres per second (m3/sec), while $H$ is the height of the waterfall measured in meters. Horsepower is an old-fashioned measure of effect, and in the standard account of the traditional method, it is said that the number of natural horsepowers in a waterfall is a measure of gross effect in the waterfall, giving us the amount of “raw” water-power in the waterfall.\footnote{See \cite{Falk}(in Norwegian)} This, however, is flat out false for most waterfalls, and it has always been more accurate to regard the number of natural horsepowers as a measure of the \emph{level of regulation} involved in a given planned hydro-power project. This, indeed, is what the concept was actually introduced to measure, and it is how it is used in the Industrial Concession Act 1917 and the Watercourse Regulation Act 1917. Historically, however, it made  sense to conflate the energy-potential of the waterfall with the level of planned regulation, since regulation of water-flow used to be crucial for the development of efficient hydro-power generation. This has changed, however, and the traditional method, when applied as a tool to assess the energy-potential of a waterfall, is horribly outdated. In the following subsection, we give a detailed presentation showing this.

\subsection{Not so natural: The physics and the law behind the notion of a "natural horsepower"}\label{subsec:notnat}

Let us first remark that horsepower is no longer used as a measure of effect in the energy business. Today, it is general practice to use kilowatts (kW) instead, at least as long as there are not any lawyers present.\footnote{1 Kilowatt(Kw) = 1000 Watt(W)} In the following, we will give the reader quite a detailed presentation of the concept of effect and the link between the two units horsepower and kilowatt. We shall try to give a rudimentary explanation of the physical facts which underlie it, leading to the conclusion that the number of natural horsepowers in a waterfall no longer has any relevance to its value in hydro-power production.

\noo{
8    “Erstatning for erverv av fallretter” (Compensation for Acquisition of Waterfall Rights) by Ulf 
        Larsen Karoline Lund and Stein Erik Stinesen in Tidsskrift for Eiendomsrett (Journal of Property 
         Righs) Nu 4 2006
9      See paragraph 3.5 in Larsen/Lund/Stinson (above)
10    See for instance p 262 in “Vassdrag og Energirett” (Law of Waterfalls and Energy) by
        Falkanger/Haagensen (Ed.) 2002. Referred to as “type” because there are several definitions of Q,
        see paragraph 1.4.
}

The notion of an \emph{Effect} ($E$) is defined in physics by means of the more elementary concepts of \emph{Work} ($Wr$), \emph{Time} ($t$), \emph{Distance} ($d$) and \emph{Force} ($F$). The relationship between them, in particular, satisfies the following equation.

\begin{equation}\label{eq:effect}
E = Wr/t = (F \times d)/t
\end{equation}

The formal notion of Work ($Wr$), in turn, is defined as follows.

\begin{equation}\label{eq:work}
Wr = F \times d
\end{equation}

The last non-trivial concept needed to define effect is the notion of Force ($F$), which is fundamental in physics. It can be explained by what is needed to change the speed of an object, or cause it to move, c.f. the First Law of Newton. The unit most commonly used to denote this is \emph{Newton}. One Newton is defined as the force which is needed to increase the speed of a mass of 1 Kilogram (Kg) by 1 Meter (m) per sec in one second. 

It is the force of gravity which is harnessed to produce an electric effect, and in turn, to harness the power of water. The force of gravity is $9,81$ Newton per Kg mass. That is, if 1 Kg falls freely to the ground it will accelerate by $9,81 \ m$ per sec in a second (in reality somewhat less because of air resistance). The power of gravity works continuously. Therefore, the mass will accelerate by $9,81$ Newton as long as it is falling. This means that for every second the mass is falling the speed will increase by $9,81 \ m/sec$.

Historically, the unit of force was often defined as the force needed to pull 1 Kg to the ground. This force was often called 1 “Kilo” and named “Kilopond” (Kp) to distinguish it from the concept of mass which was also, in everyday usage, often referred to as “Kilo” (but should actually be called "Kilogram”). When Newton is used as a unit for force, the corresponding unit of work becomes Newton $\times$ meter ($Nm$), also called Joule ($J$). Looking to the definition of work in Equation \ref{eq:work}, we can then make the following calculation: When a mass of 1 Kg falls 1 Meter, the work done amounts to $9,81 \ Nm$. Similarly, when Kilopond is used as a unit for force, you get Kilopond $\times$ meter ($Kpm$) as a unit for work. The work which is done when 1 Kg mass is falling 1 meter can then be described by the following formula:

\begin{equation}\label{eq:work}
Wr = 1 \ Kp \times 1 \ M = 1 \ Kpm 
\end{equation}

When we look at formula \ref{eq:effect} for effect, if we use Newton as the unit of force, we get \emph{Watt} ($W$) as the corresponding unit of effect, so that we get an effect of $9,81 \ Nm/Sec$, or $9,81 \ W$, when 1 Kg mass falls $1$ meter in a second. Similarly, if we use $Kp$ as the unit of force, we get an Effect of $1 \ Kpm/Sec$. Somewhat curiously, $1 \ Kpm/Sec$ does not equal $1$ Horsepower ($Hp$). Instead, the choice was made to define horsepower as follows
\begin{equation}\label{eq:hp}
1 \ Hp = 75 \ Kpm/Sec
\end{equation}
Consequently, we get the following relation between $Hp$ and $W$ as a measure of effect:

\begin{equation}\label{eq:hpw}
1 \ Hp = 75 \ Kpm /Sec = 9,81 \times 75 \ W = 735,75 \ W = 0,736 \ kW
\end{equation}

This might seem like an unduly technical exercise for a law paper. However, it is needed to understand that there is nothing magical about a Natural Horsepower ($nat.Hp$), and that horsepower, which is no longer used by the energy business as a measure of effect (or the general public, save for car enthusiasts), can easily be replaced by Watts. This, indeed would make more sense in this day and age, and open the method up to be more readily scrutinized. Indeed, moving from a quantity of $x \ nat.Hp$ to the same amount of effect, measured in nat.kW is easy; the latter is obtained from the former when multiplying by $0.736$, i.e., such that 
\begin{equation}\label{eq:natkw}
x \ nat.HP = 0.736 \times x \ nat.kW
\end{equation}

We are in a position to properly explain the formula for natural horsepowers in a waterfall. It measures the effect in the waterfall in Hp, given certain information and certain assumptions about the features of the waterfall under consideration. When we go through this in detail, however, we come to realize that the horsepowers arrived at are not so natural after all, since they are based on assumptions that are largely irrelevant for modern hydro-electric schemes, and that have been upheld in law seemingly for no other reason than the fact that they have been habitually used by lawyers and judges with no regards to, or understanding of, their underlying \emph{meaning}.

In light of what we have already seen, it is now easy to explain the constant factor of "13.33", used in the formula for natural horsepowers: Since one cubic meter ($m3$) of water ($1000 \ l$) has a mass of $1000 \ kg$, it is pulled towards the center of the earth at a force of $1000 \ Kp$. If we lift, or allow to fall, $1 \ m3$ of water by $1 m$ in one second, we work with an effect, or, in the case of falling, \emph{release} an effect, of $1000 \ Kpm /sec$. From the formulas in Equations (\ref{eq:effect}-\ref{eq:hp}), we then see that we work with, or release, an effect which measured in Hp amounts to the following.

\begin{equation}\label{eq:whp}
1000 \ Kpm /Sec : (75 \ Kpm /Sec) /Hp = 13,33 \ Hp
\end{equation}

That is, $13,33$ is simply the effect, measured in $Hp$, of $1 m3$ of water falling $1 m$ in $1 sec$. So if we have an amount of water measured in $Q \ m3$, we must then multiply $13,33$ by $Q$ to get the effect of this amount of water falling $1 \ m/sec$. Then, if we have a waterfall that is $H$ meters high, we can multiply by $H$ to find the effect of $Q \ m3$ of water falling $H m$ in $1 sec$. If the amount of water flowing through a river is Q \ m3/sec, it means that we have the amount of water $Q$ available every second. This will in turn ensure that as long as the water supply continues to be $Q \ m3/sec$ -- that is, as long as the water-supply is \emph{constant} -- the effect of the work being done by the water will also be constant and it will be given by the formula $13,33 \ x \ Q \ x \ H$.

Energy is defined as the capacity to do work. There are several units for energy, but when energy takes the form of electricity, such as in a hydro-power plant, it is standard to use the unit kWh (kW $\times$ Hour). Notice that this is consistent with the definition of work as $Wr = E \times t$ ("Energy times Time"). If you have a certain effect available over time, the amount of energy you acquire is measured by multiplying the effect by the amount of time that the effect is operative. When mechanical energy is transformed into electrical energy in a power station, the effect of the generator is multiplied by the time the generator is operative with the same effect. To get the result in $kWh$, all you must do is to ensure that you measure your time in hours and your effect in kilowatts, and then multiply the two together. 

However, in practice, the effect harnessed in a hydro-power station changes when there are changed in the water flow, so you get the true amount of energy produced only when you make this calculation sufficiently often, by multiplying a given effect with the number of hours (maybe just minutes) for which the station is operative at this particular effect. The sum of these chunks of energy you get over the year will be the amount of annual production, and this, today, is what the energy business use as a yardstick when \emph{they} assess the value of a waterfall. 

With modern technology, the energy output is registered by fine-tuned electrical equipment, maybe every 15 minutes or so, and hence the annual amount of energy generated can easily be registered and measured, even if there are significant fluctuations in the available water, leading to the generator operating at different levels of effect. This is a significant observation, since it means that the assumption inherent in the natural horsepower formula, namely that the water-flow, and hence the effect, remains constant, is no longer tenable, and gives a completely erroneous account of the energy-potential in a waterfall. 

Certainly, the amount of energy generated in a power plant could be measured in other units than kWh, e.g. in terms of the amount of horsepower-hours per year. However, an energy producer gets paid for the amount of energy he can deliver, \emph{not} the effect he can maintain in his station constantly, and as a result it simply would not correspond to reality if one would attempt to measure the energy by \emph{natural} horsepower-hours. This would only be correct if the owner of the hydro-power plant \emph{chose} to produce electricity at a constant effect all year round, which he would never do.\footnote{In addition, and pulling somewhat in the opposite direction, come the fact that it is not a realizable effect that is derived from the formula of natural horsepower, but only a gross estimate. The effect that we find is calculated based on an assumption of ideal circumstances, i.e. without any loss of energy in the production step. In reality, there will always be some loss of energy both in the pipes and in the turbine/generator. That is the reason why natural horsepower is often described as the “raw” or as the “gross effect”. But a far more important mechanism is the gross simplifications involved in moving from the physical fact that the flow of water in a river varies quite a lot during a year, to one fixed amount, Qreg, assumed to be available constantly.}
This makes it practically meaningless to talk about the number of natural horsepower in a waterfall as a measure of its potential. Effect is not something we have in a waterfall, but something we get when a certain amount of water is falling. 

From this observation also follows that the amount of water, $Q = Qreg$, that is to be put to use in the formula for natural horsepower must be chosen by an application of \emph{law}. Indeed, what to choose for $Qreg$ is a legal question of great significance, and as long as $Qreg$ is conceived of as a constant, as in the traditional method, it measure the degree of \emph{regulation of water-flow}, not the potential for energy generation.

So how is $Qreg$ typically defined? This is actually a very tricky question, although apparently, the amount of water to be used in the calculation is actually prescribed by statute. In practice, however, the statutory definition can lead to such offensive results, when applied in the context of compensation, that the courts, or, as it were, the experts presenting these calculations on behalf of the expropriating party, usually adopts a \emph{different} definition. 

The Watercourse Regulation Act 1917, Section 2 reads as follows.

\begin{quote}
Section 2:  Waterfall regulation for the production of electric energy which increases hydro power:
\begin{itemize}
\item[a)]            by at least 500 natural horsepower in one or several waterfalls which can be developed 
                 collectedly, or

\item [b)]             by at least 3.000 natural horsepowers in the whole watercourse, or

\item [c)]              which alone or together with earlier regulations significantly affects the environmental
                 conditions or other public interests, can only be exploited by the State or  
               a developer who obtains permission from the King.

\end{itemize}

If regulation of a watercourse increases the water-power in the river by at least 20.000 natural horsepowers, or if there are essential conflicting interests, then the case should be submitted to Parliament before license is given, unless the Department finds it unnecessary.

The increase of the hydro-power according to the first and second point is calculated on the basis of the increase of the low water-flow of the watercourse, which the regulation is supposed to cause beyond the water-flow which is considered foreseeable for 350 days a year. When making the calculation it is to be assumed that the regulation is operated in such a way that the water-flow during the low water periods becomes as even and regular as possible.  
\end{quote}

In the third paragraph, the definition of $Qreg$ is provided, when it states that the \emph{increase of the hydro-power}, measured in natural horsepower, is to be calculated based on the water-flow which it is foreseeable that will be present for at least 350 days a year. That is, $Qreg$ is to be taken as the maximum amount of water that one can expect to be present for at least 350 days of the year after regulation minus the water that could be expected for 350 days without regulation, which is the quantity referred to as the \emph{low water-flow}.

Regulation of a watercourse can involve building a reservoir and/or installations that transfer water from one river to another. Then, if there is excess water, for instance due to flooding, water can be stored in the dam for later use, while if there is drought, the stored water can be released. In this way it becomes possible to even out the water-flow over the year. This again means that the water which is guaranteed to be present for at least 350 days a year will typically increase. In light of the definition of $Qreg$, it is clear that the definition of natural horsepower depends crucially on the level of regulation involved in the planned hydro-power project. But the definition only takes into account the gross effect resulting from the \emph{increase} in low water-flow following regulation. It follows that if the planned project does not involve regulation, which is common today, especially for small-scale hydro-power, the number $Qreg$ will by necessity be $0$ and the waterfall will be deemed not to posses any natural horsepowers at all.

In fact, if the traditional method for calculating compensation had remained true to the wording of the Watercourse Regulation Act, things could sometimes have been even worse for the owner. This we notice, in particular, when we  consult Section 10 of the Water Resources Act 2000.\footnote{Act No. 82 of 24 November 2000 relating to River Systems and Groundwater} Here, the NVE is given the power to compel the owner of a hydro-power scheme to ensure that a certain quantity of water is always allowed to pass through the intake of the hydro-power plant. Moreover, there is nothing to prevent the NVE from demanding that this minimum water-flow is set \emph{higher} than the low water-flow, and this, indeed, is often the case, especially in cases when the low water-flow only amounts to a small fraction of the average water-flow, and environmental concerns arise with respect to wildlife and fisheries. Then, indeed, it appears that the minimum water-flow, required to be left untouched, should be subtracted from the regulated water-flow when calculating $Qreg$.\footnote{In fact, this was done in the case of \emph{Sauda}, LG-2007-176723 (Gulating Lagmannsrett, a regional Court of Appeal} Intuitively, this even appear reasonable; The minimum water-flow can not, as a matter of fact, be harnessed for energy production.

However, for hydro-power projects that do not involve regulation, this would then lead to the regulated low water-flow being \emph{less than} the low water-flow. Subtracting then the latter from the first, as required by Section 2 of the Watercourse Regulation Act 2000, would lead to a \emph{negative} number for $Qreg$, and a corresponding \emph{negative} number of natural horsepowers attributed to the waterfall. Logically speaking, then, the traditional method would entail awarding a negative sum as compensation, compelling the owner to pay the expropriating party for taking over his waterfall!

In practice, of course, the traditional method has never been applied in this way, and more generally, the definition in Section 2 of the Watercourse Regulation Act has tended to be completely disregarded by valuers using the traditional method for calculating compensation. Instead, the definition has been changed for this purpose, such that the low water-flow prior to regulation is not deducted from the low water-flow after regulation.

Even after this modification, the number of natural horsepowers give a drastically skewed picture of the potential of the waterfall, especially for projects that do not involve regulation. It is not unusual, in particular, especially not for waterfalls suitable for small-scale hydro-power, that the low water-flow amounts to only about 3-5 \% of the average water-supply. In modern hydro-power projects, one would expect 70-80 \% of this water-flow to be harnessed for energy production even in the absence of any regulation. So in these cases, the traditional method of compensation is effectively based on compensating the owner for only a small fraction of the energy that can actually be extracted from his waterfall.

This observation, which follows from elementary facts about physics and contemporary hydro-power production, was not noted or discussed in connection with the principles used for compensation before the early 2000s, when the issue was raised following the growing interest in small-scale hydro-power. However, the Norwegian government has certainly been aware of these facts, as illustrated for instance by the following passage, taken from a report presented to parliament in 1991-1992.\footnote{Ot.prp. No 50 (1991-1992) p 19, discussing the notion of natural horsepower in connection to the uses made of that term in other parts of the law.}

\begin{quote}
The Department of Oil and Energy have considered moving a proposition for changing the hydrological definitions in the Industrial Concession Act 1917 and the Watercourse Regulation Act 1917. Today the act uses a calculation method based on an increase in regulated water-flow, i.e. that of natural horsepower.[.......] The hydrological definitions of these acts, supposed to indicate how much electricity can be generated, were made on the basis of technical and operative conditions differing very much from contemporary circumstances. In implementing the definitions referred to above one has tried to adapt to the new technological realities of the present day. Therefore, in practice, a calculation based on current production is used instead. From several quarters, particularly the Association of Waterfall Regulators, there has been raised a strong wish to authorize this practice by altering the definitions of the relevant laws. The Department of Oil and Energy agree, but have not as yet made a sufficient elucidation of the issues to be able to move a proposition of alteration of these acts.
\end{quote}

Within the ranks of the water authorities, it has actually been well-known for decades that the notion of a natural horsepower fails to give an adequate picture of the potential that a waterfall has for hydro-power. The development of new technology had made this apparent already in the 1950's, when it was also raised as an issue, specifically with respect to compensation following expropriation, by a director at the NVE, who commented, in 1957, that he failed to see how the traditional method could be an adequate means for valuating waterfalls.\footnote{See \cite{....}. The director even went as far as to illustrate a different method, which would also be outdated given today's regulatory regime, but which would reflect contemporary \emph{actual} valuations, used by the NVE itself.}

Considering the physics behind the traditional method is enough to reveal that it fails to give rise to valuations that reflect the value of waterfalls, under any reasonable set of assumptions about the correct general compensation principles one should adopt. Firstly, the traditional method, by relying on data from the expropriating party's project, deviates from the "value to the owner" principle. Secondly, and even more importantly, it amounts to compensating the owner based only on the level of regulation, which is not only mostly irrelevant to the value, but is also the one aspect of the scheme which can not readily be traced to properties of the waterfall, but depends rather crucial on the investment decisions made by the expropriating party. While a case can be made that any extra power harnessed by regulation should \emph{also} be compensated, for instance if it can be established that the expropriating party is not the only one who could have regulated the waterfall, it seems rather perverse to \emph{only} compensate the owner based on this parameter.

However, while the idea of compensating the owner of waterfalls by a price per natural horsepower is fundamentally flawed already at the level of physics, there are even more serious concerns that arise when one begins to consider the way in which the \emph{unit price} has been determined, and the effects this has had on the level of compensation payments. In case-law based on the traditional method, it is often said that the price set per natural horsepower is set according to "market price" for waterfalls, but for the most part, what this means is that the court looks to prices awarded in earlier compensation cases, not to prices obtained in voluntary sales.

This, in turn, gives rise to a price level that is entirely artificial, reflecting, more than anything else, the power balance between buyer and seller in the courtroom, and not any genuine market value. Indeed, while the prices paid did see some increase during the 80 years that the traditional method was used, this hardly reflected the general increase in value of hydro-power, nor did it reflect the general level of inflation.\footnote{References needed.} Moreover, and particularly worrying, while the price-level was determined by the courts, there were also some cases of voluntary agreements that used the same method, and could thus be used to justify is status as a genuine market-based valuation principle. In fact, as late as in 2002, a waterfall belonging to local landowners in the rural community of Måren, in Western Norway, was sold for the sum of kr 45 000 (roughly £ 5000), based on traditional calculations. The waterfall has now been exploited in a small-scale hydro-power plant belonging to the large energy company BKK, with annual energy output of 21 GWh.\footnote{$http://www.bkk.no/om_oss/anlegg-utbygging/Kraftverk_og_vassdrag/andre-vassdrag/article29899.ece$} For comparison, we mention that in the case of \emph{Sauda}, based on the new method for calculating compensation, the owners received a compensation which totaled about 1 kr/kWh annual production.\footnote{LG-2007-176723 (I acted as council for some of the owners in this case).} Applied to the Måren case, this would take the compensation from kr 45 000 to kr 21 000 000, that is, almost 500 times more.\footnote{In fact, the Måren waterfalls were cheaper to exploit, so in reality, one would expect that the new method applied to Måren would yield even greater compensation per kWh. We also remark that the value awarded in \emph{Sauda} was market-value, not value of use, since it was assumed that the owners would have to cooperate with a so-called "professional" energy company to develop hydro-power. This, in effect, halved the compensation awarded.}

The case of Måren is somewhat extreme, but in no way unique.\footnote{I should assemble a list probably....} Moreover, it illustrates an important point, namely that when the traditional method was used, and described as the "market value" of waterfalls by the courts, this became a self-fulfilling prophecy in many cases. The prices paid in voluntary transactions were influences by the practice adopted by the courts far more than the other way around. This, indeed, appears to be a general danger in cases when expropriation is widely used for the purpose of commercial development. Then, it seems, prices paid can easily be kept artificially low by developers turning to the use of expropriation as soon as they threaten to rise, and relying on the "market value" thus established when arguing in court for the appropriateness of those compensation levels that so benefits them commercially. That this mechanism can be severe is nicely illustrated by the case of Norwegian waterfalls, and how to prevent it is, in our opinion, a main challenge that is likely to arise in any regulatory system that aims to make extensive use of expropriation to further economic development.

\section{{\it Kløvtveit} and {\it Otra Kraft}}

Following the liberalization of the Norwegian energy sector in the 1990s, the traditional method came under increasing pressure. It was argued to be unjust by owners who felt that they were being deprived of a valuable commercial assets, and it was held to be illogical by engineers working on developing small-scale hydro-power.\footnote{References} Eventually, legal professionals followed suit, and came to the realization that established rules based on market value could now be applied. Indeed, a new market for waterfalls had begun to develop at this point, following the increased interest in small-scale hydro-power and the formation of new companies specializing in cooperating with local owners. For transactions of rights to waterfalls taking place in this market, the traditional method of valuation was not used, and waterfalls were rarely sold at all, but rather leased to the development company for an annual fee. Typically, this fee was calculated by fixing a percentage of the energy produced during the year, and compensating the owners of the waterfall by multiplying this with the market price for electricity obtained throughout the year, possibly deducting production specific taxes, but with no deduction of other cost. In effect, owners would get a fee corresponding to a set percentage of annual gross income in the hydro-power plant.\footnote{References}

Usually, the fee entitles the owners to 10-20 \% of the income from sale of electricity, depending on the cost of the project. Moreover, it is common that the owners are entitled to up to 50 \% of the income derived from so-called \emph{green certificates}, a support mechanism for new renewable energy projects, corresponding to the Renewables Obligation in the UK.\footnote{See http://www.ofgem.gov.uk/Sustainability/Environment/RenewablObl/ for further details.} Essentially, and somewhat simplified, the scheme allows the energy producer to collect a premium on his sale of electricity, which, owning to its "green" status, is valued more highly by buyers (usually electricity suppliers), who are required to ensure that a certain proportion of the energy they offer to their customers (usually consumers, like you and me) is considered green. In Norway, such a scheme has been talked about for years, but was only put in force in 2012.\footnote{http://www.regjeringen.no/en/dep/oed/Subject/energy-in-norway/electricity-certificates.html?id=517462} Currently, energy producers can claim a premium of about 2 pp per KWh per year, meaning that about a third of the annual income for new renewable energy projects comes from the sale of green certificates.\footnote{While the premium must be expected to go down somewhat as the certificate market matures and more energy producers acquire "green" status, it will certainly remain an important source of extra income for renewable energy producers also in the future.}

In light of the fact that the agreements on the new market are based on leases that tie compensation to the fate of the particular hydro-power project that is being undertaken, several questions arise when attempting to value waterfalls by looking to this market. If a given project has been identified as providing the basis for valuation, the task is difficult, but mainly a question of factual assessment. The valuer have to determine first what the annual production would be and also determine the costs of carrying out the project. Then, on this basis, he must move on to determine what the annual fee would be, and then, in order to complete the process, he must stipulate what the price of electricity, and of green certificates, is likely to be for the next 20 years or so. On the basis of this information, it becomes possible to determine the annual income to the owner of the waterfall over a period of 20 years, and then one would also have a basis upon which to calculate a reasonable present-day value of the scheme to the owner of the waterfall. 

Indeed, this is the model that has been used in the cases that have been before the courts and where the traditional method has not been adopted. The first such case was \emph{Møllen}, and while the Supreme Court rejected the method as it was applied in this case, because it was found that the date of valuation was to be based on prices obtained for waterfalls in the 1960s, they commented that they supported the adoption of the new method in cases when \emph{alternative} small-scale development was deemed a \emph{foreseeable} use of the waterfall in the absence of the scheme.\footnote{Rt. 2008 s. 82.} 
Since \emph{Møllen} the new method has been used in many cases before the lower courts and the Lands Tribunal.\footnote{See for instance \cite{tf1,tf2,tf3}, a series of academic papers discussing the new method (in Norwegian).}

Unsurprisingly, the new method tends to lead to a rather protracted process of valuation, mostly dominated by experts. Moreover, given all the uncertain elements of the calculation, it is typical that the opposing parties produce expert witnesses that diverge significantly in their valuations. While this is problematic enough, the fundamental \emph{legal} challenge arises with respect to the choice made about what scheme the compensation should be based on. This becomes especially tricky if one attempts to follow a standard no-scheme approach. In the following, we summarize the main issues that arise.

\begin{enumerate}
\item In the absence of the hydro-power scheme benefiting from expropriation, is it foreseeable that the waterfall would nevertheless be used in a hydro-power project?
\item If the answer to Question (1) is yes, what would such a foreseeable project look like?
\item Is it foreseeable that an alternative project would get planning permission?
\item Does the no-scheme rule imply that the project benefiting from expropriation cannot be regarded as the foreseeable use for the purpose of compensation?
\item Can the fact that the scheme underlying expropriation obtained planning permission be taken as evidence to support that alternative uses of the waterfall would not be given planning permission?
\item How should compensation be calculated if it is determined that no alternative project is foreseeable? 
\end{enumerate}

In some cases, for instance when the project benefiting from expropriation is not commercially viable but is carried out for public purposes with the help of special State funding, the answer to Question (1) might be no. However in most cases, the question will be answered in the affirmative, since the scheme benefiting from expropriation already serves as an indication that the waterfall can be harnessed for energy. However, here the no-scheme rule comes into play and creates severe difficulty once we reach Question (2). For what kind of scheme can be assumed foreseeable all the while we are obliged to disregard the scheme underlying expropriation? In most cases that have come before the courts so far, the owners have claimed that alternative development in small-scale hydro-power should serve as the basis for compensation, and such cases the problem of the no-scheme rule appears to have been circumvented. This, however, is not necessarily the case. It appears, in particular, that the answer to Question (3), asking about the likelihood of planning permission, might again depend on how you view the no-scheme rule. It seems, in particular, that anyone who answers Question (5) in the affirmative, looking to the planning permission actually given to the expropriating scheme for evidence, might be inclined to say that the alternative project could not expect to get planning permission, and that this is so \emph{because} planning permission was granted to the expropriating party (or this line of reasoning might be sugarcoated by pointing to whatever underlying reasons the authorities had for considering the scheme underlying expropriation the optimal use of the waterfall).

Then the question arises: Is someone who reasons like this at odds with the no-scheme rule? It would seem so, but remember the earlier discussion on the no-scheme rule in Norwegian law, where we noted that the rule has tended to be applied much more narrowly along its positive dimension. Following up on this, it can be argued, in keeping with the general tendency in how the law is applied in Norway, that while the expropriation scheme is to be disregarded for the purpose of compensation valuation, the regulation underlying the scheme -- or at least the rationale behind this regulation -- is nevertheless to be taken into account. If this point of view is taken, then the conclusion can easily become that alternative development is to be regarded as unforeseeable, and that the reason why this must be the case is precisely the fact that the expropriation scheme received planning permission. Indeed, this line of reasoning was given a stamp of approval in the recent Supreme Court case of \emph{Otra II}. Here, the presiding judge made the following remarks, quoting Gulating Lagmannsrett (the regional Court of Appeal), expressing his support, and adding a few comments of his own.

\begin{quote}
"[....] The Court of Appeal finds it difficult to distinguish this case from other cases when it has been established that alternative development is not foreseeable. It does not seem relevant whether this is the case because the alternative is not commercially viable or because the alternative must yield to a different exploitation of the waterfall" 
I agree with the Court of Appeal, and I would like to add the following: As the survey of the general principles have shown, it is assumed, both in the Expropriation Act, Sections 5 and 6, and in case-law, that only the value of a foreseeable alternative should be compensated. This starting point means that it would be in breach of the general arrangement if a waterfall that can not be used in foreseeable small-scale hydro-power was to be compensated as if it could be put to such use.
\end{quote}

Having used the planning permission granted to the expropriating party as evidence that alternative development was unforeseeable, the Court needed to answer Question (6) by coming up with some alternative way of compensating the owners.  To do so, the Court also had to answer Question (4), however, asking whether or not the scheme underlying expropriation could be taken into account at this stage. Moreover, one would think, given how the expropriation scheme was used as an argument when answering Question (5) in the negative, that it \emph{could} be taken into account. This, however, was answered in the negative by the Supreme Court in \emph{Otra II}, where the presiding judge reasoned as follows.

\begin{quote}
Based on the arguments presented to the Supreme Court, I find it safe to assume that there does not today exist any market for the sale and leasing of waterfalls for which alternative development is not foreseeable, but where the waterfalls can be used in more complex hydro-power schemes. The appellants have not been able to produce documents or prices to document the existence of such a market
\end{quote}

Let us first remark that it is hard to imagine how a market such as that asked for here could ever develop, all the while alternative buyers, by the courts own reasoning, are excluded from being taken into account. In this case, if there was to be a market, it would have to be down either to the regular benevolence of the expropriating parties in cases like these, or to the government compelling them to enter friendly negotiations. In the Norwegian context, both seem unlikely. More important, however, is the implicit assumption that in order to value the waterfall according to its potential for hydro-power production, a market needs to be identified. It is \emph{not} considered sufficient that the scheme for which expropriation takes place is itself a hydro-power project, on the basis of which the value of the waterfall could be assessed following exactly the same steps as in the new method.

In fact, the Supreme Court's reasoning in \emph{Otra II} serve as an excellent example of the type of reasoning that makes the no-scheme rule highly problematic for cases of expropriation that benefit commercial schemes. Indeed, it seems to follow that the scheme itself should not provide a basis for calculating the compensation, but then, on other hand, it also appears that there really is no other way to calculate it, seeing as the State, by giving permission to carry out the scheme, have effectively acted in such a way as to make an alternative use \emph{of the same kind} seem inherently unforeseeable. This is not so much of a problem when the use that the owner could make of the property has a different character than the scheme, since in this case, if we disregard the scheme, this other use might seem foreseeable. However, when the alternative use of the property is of \emph{exactly the same kind} as the use made of it by the scheme, it does seem counterintuitive, as noted by the Court of Appeal in \emph{Otra II}, to regard it as foreseeable, all the while the scheme will tend to appear the more rational form of exploitation.

It seems, however, that when taken to its logical conclusion, this line of reasoning, based on the no-scheme rule, leads to an offensive results; the commercial value of the property is not to be compensated because the optimal commercial use of the property is the use that the expropriating party aims to make of it in the scheme underlying expropriation. Note that the conclusion is not just that this optimal value, inherent in the scheme, should not be compensated. No, the conclusion in \emph{Otra II} was that \emph{no} compensation could be estimated for any use of the same \emph{kind}, since such use was not foreseeable, owing to the absence of a market.

It is certainly possible to argue that this decision represent a misguided application of the no-scheme rule. In effect, the Supreme Court allowed the planning permission given to the expropriating party to act as evidence that alternative development was unforeseeable, while it used the no-scheme rule to argue that the hydro-power scheme for which this planning permission was given could not itself form basis for compensation payment based on market value.  On the other hand, it seems that even if we disregard the scheme completely, it is unnatural to base the compensation payment on the value of a hydro-power scheme that is less beneficial, both commercially and in terms of resource efficiency, than the scheme for which expropriation takes place. Such a scheme would not, one must presume, \emph{actually} have been carried out, regardless of the questions of whether or not it would have been given planning permission. But it does seem particularly difficult, intuitively speaking, to predict what use would have been made of the waterfall if these were the facts: more or less exactly the same scheme as that underlying expropriation would be implemented, but by some from of voluntary agreement with the owners, not by means of expropriation. 

In \emph{Otra II}, however, this line of thought was also rejected, although this was, in part, due to the point not having been argued before the Court of Appeal. But then the question arose as to how exactly compensation should be calculated. The answer, following up on the "value to the owner" principle so forcefully adopted elsewhere in the judgment, would appear to be that no compensation should be paid at all, save perhaps for loss of fishing rights and the like. This, however, was \emph{not} the conclusion reached by the Supreme Court. Instead, the Court states that a return to the traditional method is in order. However, they do not apply it in the traditional way. Rather, they casually sanction the replacement of regulated low water-flow in the definition of $Qreg$ by the average water-flow, thus moving away from compensation based on the level of regulation, to compensation based on average effect in the project. Moreover, they also sanctioned the use of a significantly increased unit price compared to earlier times.

What to make of this? It seems hard indeed to make sense of indeed, since effectively, by relying on the traditional method, the Supreme Court contradicts its own conclusion that compensations should be based on market value. Instead, they rely on a method that, in effect, is based on an attempt to quantify the value of the waterfall as it is being used by the expropriating party in his project. However, by relying on a technical method that has been completely outdated, and have lived its own life in the courts, it becomes difficult to assess the outcome properly, at least for a non-expert. It seems, in particular, that the Supreme Court prefers the obscurity of the traditional method, and its status as an established principle, over the possibly radical conclusion that, in cases such as this, it simply is not tenable to adopt the "value to the owner" principle, as least not as construed in Norwegian law.

Indeed, it is simple enough to be critical of the Supreme Court based on the fact that they regarded alternative development as unforeseeable in this case, when the planning permission granted to the expropriating scheme itself appears to have provided the decisive bit of evidence. Still, it is not possible to escape the fact that this reflects a general tendency in Norwegian law, and so, even if it appears unreasonable, it might very well be a correct application of national law. Moreover, it could very well have been that alternative development was unforeseeable for \emph{some other reason}, for instance because the only commercially viable exploitation was the scheme planned by the expropriating party. In this case, the problem of how to compensate the owners in the absence of an alternative form of exploitation would still arise. It is this question, in particular, which seems entirely unsatisfactorily resolved under an application of a "value to the owner" principle.

This is witnessed by \emph{Otra II}, and, in fact, it appears that the Supreme Courts decision \emph{not} to follow their own reasoning to its logical consequence is the main lesson to be learned from the case. For all intents and purposes, the Supreme Court \emph{rejects} the "value to the owner" principle, but they obscure this by wrapping it up in the traditional method, which is deeply flawed. However, the problem it attempts to solve appears significant, and it pertains directly to the question discussed more generally in Section \ref{sec:noscheme}, namely how to apply the "value to the owner" principle with respect to commercial schemes. It seems that even the fiercest supporters of limiting owners' right to compensation tend to find it too offensive to apply this principle when it leaves the owners with no form of compensation for giving up property to multi-million, purely commercial undertakings. Indeed, such a practice would surely also be in breach of the human rights law. It seems, in particular, that the subjective aspect of the "value to the owner" principle is impossible to maintain. Indeed, if the commercial value falls to be disregarded for no other reason than the fact that the State happens to have granted planning permission to the expropriating party rather than the owner, this is not only dubious with respect to human rights protecting property, but also appears to be a case of \emph{discrimination}, e.g., as prohibited by ECHR Article 14.

The problem does not arise when the buyer sees value in the property that is of a different \emph{kind} than that realizable by \emph{any} private owner. In this case, the rule simply states that the owner should not be able to demand that "public value" is transformed into commercial value just for him. This appears like a reasonable principle. But when there is commercial value already present on the "public" side of the transaction, it seems completely unwarranted that the public should be allowed to transfer this value from the owner to someone else without compensation. Thus, it seems that more accurately and acceptably, the "value to the owner" principle should be thought of as a "commercial value" principle. It seems, in particular, that the principle need to be stripped of any suggestion that a preferential financial position is to be awarded to whoever benefits from expropriation.\footnote{Exceptions might be possible to imagine, but, one would think, only when they can be construed as falling under the "public value" banner in some way.}

It seems unfortunate that this aspect has not been made explicit, and the difficulties that arise in the absence of this nuance seems nicely illustrated by the case of Norwegian waterfalls. Still, as the case of \emph{Otra II} seems to indicate, an interpretation of the "value to the owner" principle along less offensive lines is in reality already in place with regards to Norwegian hydro-power. Here, it seems that "value to the owner" has in fact \emph{never} been applied in the traditional way. Hopefully, rather than obscuring this fact by relying on an unsatisfactory and artificial method for calculating the compensation, the future will see further developments that recognize the need for new principles. It should be recognized, in particular, that as the law has been applied for the last 80 years, despite its grave flaws and injustices, there has always been an implicit recognition in Norwegian law that the owners of waterfalls are \emph{entitled to their share} of the commercial benefits of hydro-power. 

In fact, in the recent Supreme Court case of \emph{Kløvtveit}, a further illustration of this is found. The conclusion here was also that alternative development was not foreseeable. However, unlike in \emph{Otra II}, the Court of Appeal had compensated the owners based on the fact that they regarded it foreseeable that in the absence of the scheme, the waterfalls would have been exploited in exactly the same way, except that it would have happened in the form of \emph{cooperation} between the owners and the expropriating party. By this line of reasoning, the Court effectively seems to have adopted a more rational "commercial value" principle, to replace the traditional method. 

Indeed, is it not always the case, at least under objective standards of assessment, that when alternative development is unforeseeable, then a rational alternative buyer -- assumed to operate in a world where there are no "powers of compulsion", to paraphrase Lord Nicholls in \emph{Waters} -- would look precisely to the likely possibility of cooperating with the expropriating party? This, on the other hand, would \emph{never} be a safe assumption to make for non-commercial aspects which, in the absence of commercial potential would not give an alternative buyer financial incentive to do so.

We mention that \emph{Kløvtveit} was discussed in \emph{Otra II}, and that the presiding judge made some reflections, focusing on what he regarded to be "practical problems" with cooperation. However, this was not crucial to the decision, since the cooperation model was not argued for by council. In light of this, one can only hope that \emph{Kløvtveit}, rather than \emph{Otra II}, will become the influential precedent for future cases.

\section{Conclusion}

We have presented an overview of Norwegian law relating to compensation following expropriation. First, we identified two different strands of thought regarding this matter, which we referred to as absolutist and pragmatist respectively. We noted that the tension between these two perspective became aggravated in the 60s and 70s, when legislation was passed with the explicit intent of bringing compensation payments down and to enforce a more pragmatic approach. The legacy of this era was a lasting pragmatist turn in compensation law, but also a greater centralization of power regarding the assessment of appraisal disputes. In \emph{Kløfta}, the Supreme Court modified some pragmatist rules introduced by parliament, but they also sanctioned a range of disregards that reflected the pragmatic intent behind these rules. Moreover, they assumed a greater role in providing special rules for the appraisal courts to follow in these matters, hence limiting the role of the laypeople in the appraisal process, and thus also changing the character of this process, which has long roots in the Norwegian legal tradition.

We focused particular attention on this latter change in the law, and we argued that it has resulted in an overly simplistic and often unhelpful narrative regarding compensation. Moreover, we argued that it inadvertently went against one crucial principle that more subtle thinkers from both the absolutist and pragmatist camps agreed on: the need for concrete fairness assessment. We went on to suggest that the importance of this principle is further accentuated today, when the context of expropriation is often quite different from the standard assumption of property taken for the public good. Often, the economic system currently in place, and the widespread use of expropriation that has followed the advent of extensive planning law, leads to expropriation appearing primarily as a means for commercial actors to make a profit. It might be hard to directly address the legitimacy of this in legal terms, by demanding that courts take an active role in interpreting the public interest requirement. But then the nature of compensation rules applied to such cases becomes a crucial special question. If dealt with in the right way, compensation can be used to achieve greater fairness in such cases, and also, more importantly, can serve as an effective safeguard against excesses. 

Commercial companies, presumably, only want to use expropriation as long as this is the most \emph{profitable} or \emph{practical} manner in which to acquire property. Moreover, it seems that a system where expropriation regularly comes to be used for this reason would have to be regarded as inherently flawed by both pragmatists and absolutists. Hence there should be cause for reaching common ground on the principle that compensation rules needs to be such that they prevent commercial companies from profiting merely from being able to use compulsion against other members of society. Achieving success in this regard, however, might not be so easy if one is committed to a top-down approach relying on the introduction of yet more special rules. Rather, justice and fairness might be better served by taking note of the potential inherent in the special way that Norway organizes appraisal disputes. By focusing on the need for concrete fairness assessment, and demanding that appraisers look to the power balance between the parties, the purpose of the expropriation, and the possible commercial interests involved, it seems that much can be achieved. The recent developments in case-law regarding waterfalls illustrates this. One can only hope that the powers that be are not too invested in the idea that \emph{they} are the ultimate authority on fairness, to allow these 
encouraging trends to develop further.

\section{The no-scheme rule in Norwegian law}\label{sec:nonor}

Before 1973, the Norwegian law relating to compensation for expropriation of property was based on case law. The courts would interpret Section 105 in the Norwegian constitution which demands that ``full'' compensation is to be paid. The no-scheme rule was typically applied, such that when assessing the value of the property, the element of compulsion was disregarded, and changes in value that could be attributed to the underlying scheme tended not to be taken into account. As in the UK, difficult questions would arise for comprehensive schemes based on public plans for the use of the property, and it proved difficult to identify any clear rule concerning the distinction between the scheme itself and the regulation of property-use that preceded it.\footnote{References.}

Following the Second World War, there was an increasing trend that effects of regulation \emph{would} be taken into account, but only with regards to the \emph{positive} aspect of the no-scheme rule. That is, the scheme was taken into account in so far as it could be used to argue against alternative development, but not in such a way that it could lead to an increase in the value of the property. In some cases, even the regulation directly preceding, and providing the basis for, the use of compulsion, would be taken into account.\footnote{Such as in Rt. 1970 s. 1028, where a property which had been used by a local business owner was expropriated to implement a public plan that regulated the property for use as a public road with parking spaces. The owner was not compensated for the loss of business revenue, since, according to the majority in the Supreme Court, the regulation of the property had to be taken into account (the decision was given under dissent). The case was somewhat special, however, since the business value could be realized by the owner only if he had been given the opportunity to rebuild his store, following a fire. Moreover, he was already ensured compensation based on the value of the property as a plot for housing, for independent reasons.}

The general picture, however, was that a no-scheme rule applied to underlying regulation of property use as long as there was a causal link between this regulation and the subsequent expropriation.\footnote{References. \noo{Husaas-komiteen}} To apply this in concrete cases often proved problematic, however, as illustrated by the many conflicting opinions voiced about the current law during the preparation of the original Compensation Act from 1973.\footnote{See, for instance, the historical overview given in NOU 2003:29}

In the 1973 Act, a radical rule was put in place to resolve all outstanding issues: valuation should be based on the  \emph{existing use} of the property at the time of expropriation.\footnote{So the Norwegian law mirrored the rule introduced in the UK Town and Country Planning Act 1947 which was later replaced by the current Land Compensation Act 1961.} The rule went further than the no-scheme rule in that it prescribed that compensation should disregard \emph{all} kinds of hypothetical development of the property, notwithstanding their status with respect to existing plans and regulations. But it also involved a break with it, since, on the face of it, the implication would be that any kind of regulation predating the scheme would be taken into account when it provided the basis for the "existing use".

However, in Section 4, nr. 3 of the Act, this aspect of the "existing use" principle was limited by a \emph{separate} provision implementing the \emph{negative} aspect of the no-scheme rule; the value of existing use due to public regulation underlying the expropriation should be \emph{deducted} from the compensation payment. As such, the Compensation Act 1973 implemented a system whereby the positive part of the no-scheme rule would be given a very narrow interpretation -- any scheme or regulation limiting current use was to be be taken into account -- while the negative aspect was explicitly provided for in statute -- the value of existing use that could be attributed to public regulation underlying the scheme was to be deducted.

This new rule was quite controversial, and to make the system more flexible, the 1973 Act included a rule that allowed the Lands Tribunal to increase compensation, on a discretionary basis, by taking into account that value of comparable properties in the district where expropriation took place. Still, property owners felt that the new Act went too far in depriving them of the right to compensation, and the matter came before the Supreme Court in \emph{Kløfta}.\footnote{Rt. 1976 s. 1} After deliberating in plenum, the Court presented a revisionary interpretation of the new Act, essentially judging the intention behind the main rule as being incompatible with the protection of property encoded in the Norwegian Constitution, Section 105. The main step taken by the Court was to regard the discretionary increase of compensation as a \emph{mandatory} step, one that had to be carried out whenever certain conditions were fulfilled. What exactly these conditions amounted to, and how they should be interpreted, was not conclusively resolved, however. Indeed, the confusion that arose after \emph{Kløfta} led to a heated academic debate in Norway, and a long line of Supreme Court cases has since attempted to clarify the current state of the law. \footnote{References.}

In 1984, taking into account the ruling in \emph{Kløfta}, a new Compensation Act was passed, which is still in force today.\footnote{Act No. 17 of 06. April 1984 relating to Compensation following Expropriation of Real Property} According to Section 4 of the Compensation Act, compensation is to be calculated as the highest of either the value of the property as it could be put to use by the owner, his \emph{value of use}, or the \emph{market value}, the payment he could expect to receive from a typical willing buyer. In both cases, a no-scheme rule applies: in Section 5, Paragraph 3, it is stated that when calculating the market value, changes in value due to the scheme is to be disregarded, while in Section 6, it is stated that the value of use should be based on \emph{foreseeable} use of the property. In practice, this has been interpreted as referring to such use as it is reasonable to expect would have occurred in the absence of the scheme.\footnote{References.} 

However, the spirit of the 1973 Act is still clearly felt in Norwegian law, and the no-scheme rule is thought of somewhat differently than in the UK. Firstly, it is common to distinguish more sharply between the positive and the negative aspects of the rule, and unlike in the UK, much, if not most, attention has been devoted to the former aspect, when the scheme is used to justify decreased levels of compensation. Secondly, and specifically as it relates to the positive aspect of the rule, the tendency has been to give "the scheme" a narrow interpretation, regarding public plans and regulation as binding for valuation, even when they are intimately related to the undertaking for which a right to expropriate is granted. Simultaneously, if the plan leads to an increased value that \emph{can not be realized by the current owner}, a "value to the owner" principle typically applies directly, such that this value is not compensated, irrespectively of what the scheme is taken to be.\footnote{References.}

As we mentioned, much attention in Norway has been directed at the positive aspect of the no-scheme rule, and there are some important exceptions to the main principle of regarding public plans as binding for the evaluation. The main exception is that a plan tends to be disregarded when it has no other purpose than to facilitate the scheme for which expropriation is needed. The important precedent in this regard is the influential Supreme Court case of \emph{Lena}.\footnote{Rt. 1996 s. 521.} However, the line of reasoning adopted by the Supreme Court in this case has so far been called on almost exclusively in cases when compulsory acquisition takes place to implement plans for public buildings or other kinds of public installations, like playgrounds or parking spaces.

In case of expropriation taking place to implement such plans, if a valuation were to be made on the basis of the use prescribed by the plan itself, one would expect the market value of the property to come out as nil or close to nil, such that one would be forced to return to existing use as the basis for valuation.\footnote{In fact, a logical continuation of the line of reasoning prescribed by the main rule would suggest that even existing use would be inadmissible as a basis for compensation since continuation of this use would not be in accordance with the plan, and hence unforeseeable. However, this has not, as far as we are aware, ever been argued.} It is not hard to understand how this could come to be felt as unfair to property owners. Moreover, it seems that it would easily come to run counter to Section 105 of the Norwegian Constitution. In some sense, it would represent a watering down of this provision, allowing the State to deprive property owners of value by using unfavorable regulation as an explicit means to later acquire properties cheaply by use of expropriation.

On the other hand, it can be argued that the question raised by cases such as these is not really a question of compensation for expropriation, but rather a question of whether or not property owners should have a claim of compensation for losses incurred due to \emph{regulation}. This is not generally granted under Norwegian law, and so the special rules that entitles the owner to such compensation in cases of regulation leading to expropriation can be seen as insufficiently justified within the broader context of Norwegian planning law. Indeed, some scholars have voiced this opinion forcefully, and the current state of the law is unclear at best, with the special rule introduced in \emph{Lena} being hard to apply to other cases, and giving rise to further Supreme Court decisions attempting to map out in more detail when exactly they come into play. Moreover, attempts to reform the law on this point have so far stranded in controversy.\footnote{NOU 2003:29 and further references.}

It seems to us, however, that while this debate is interesting, the truly fundamental questions about the current state of Norwegian law do not arise in this regard, but with regards to the other principle we mentioned, namely that no compensation is offered for value that \emph{the owner can not realize}.\footnote{By "realize" here, we mean realizable either as the owner's "value of use" or else realizable by selling the property at "market value", as prescribed by the Compensation Act.} This rule seemingly applies without reservation, and, as in the UK context, it appears like a logical consequence of the \emph{value to the owner} principle. However, as we argued for in the general discussion on the no-scheme rule in Section \ref{sec:noscheme}, it seems pertinent to distinguish between the \emph{subjective} and \emph{objective} aspect of this principle. In particular, if the owner can not realize the value because the value is not of a \emph{kind} that is available for commercial realization, for instance because it only represents non-commercial value to the public, then the rule appears easy to defend along traditional lines. On the other hand, if the owner can not realize the value because the State desires that \emph{someone else} be allowed to realize it, then the principle appears highly problematic.

While the general debate on Norwegian compensation law has completely neglected to consider this aspect, it features extensively, albeit implicitly, in one particular branch, namely the law relating to compensation for waterfalls. In the remainder of this paper, we turn to this particular area of Norwegian law, offering a detailed analysis of the problems that arise, and how they severely challenge the traditional "value to the owner" reasoning about compensation for commercial undertakings. 


\chapter{Compulsory participation in economic development projects}\label{chap:6}

\section{Introduction}\label{sec:intro6}

In this Chapter, I will consider {\it alternatives} to expropriation in the context of economic development. This is also where I ended my theoretical discussion in Part I, by presenting and analysing the proposals of Heller and Hills in the context of the US debate on economic development takings. Here, I return to this point in the context of my case study, by exploring the Norwegian institution of {\it land consolidation}. 

In recent years, this institution has been used extensively to facilitate hydropower projects. So far, however, it is used almost exclusively in situations when owners themselves organize such projects. In these situations, expropriation is rarely sought and rarely authorized. Instead, various consolidation measures are used, including the practically important measure of a ``use directive'', to set up an organizational framework and a binding plan for development involving jointly owned property,

In addition to the practical use of consolidation in the context of hydropower development, there are recent legislative developments in Norway that sees the consolidation alternative gain importance also in relation to other forms of development. This includes urban and non-agrarian development projects, something that represents a departure from the tradition of land consolidation in Norway. Some argue that these uses of land consolidation leave the owners in a precarious position, and may weaken private property rights. In this Chapter, I argue for the opposite perspective, that the use of consolidation in these new contexts will enhance property as an institution. Moreover, I argue that it can be used to  address the democratic deficit of economic development takings in a very elegant way, provided the land consolidation process itself remains intact, as a service to owners and local communities, and is not usurped by external actors.

I begin in Section \ref{sec:lce}, by presenting the basic idea of using land consolidation as an alternative to expropriation. I discuss broad notions of land consolidation in general, relating them also to the discussing found in Heller and Hills' article. Then I point out some special features of the Norwegian system, which I believe make it particularly natural to consider in the context of economic development cases. 

Then, in Section \ref{sec:lcc}, I present the Norwegian system of land consolidation in more depth, focusing on the procedural rules and the rules in place to protect property against unwarranted interference. I focus particularly on the so-called ``no-loss'' guarantee, which states that no consolidation measure can be implemented unless the benefits make up for the harm, for all affected properties. Hence, land consolidation is quite distinct from expropriation in general. However, in situations when benefit sharing is possible or natural, it becomes possible to fulfil the no-loss criterion, and in doing so land consolidation could become a powerful alternative to expropriation for such cases. 

In Section \ref{sec:lch}, I discuss land consolidation specifically in the context of hydropower development. I consider several cases in detail, based on court documents and recent work done in a master thesis on consolidation, where the author carried out interviews with affected owners. Then, in Section \ref{sec:lca}, I offer a assessment and discuss some challenges. I argue more extensively hat the land consolidation alternative is should be seen as a possible way to strengthen property rights, not as a threat. I also pinpoint what I believe to be the main challenge, namely to ensure that the land consolidation process remains intact as a service to owners and local communities, even after powerful commercial interests enter the scene. In Section \ref{sec:conc}, I conclude.

\section{Land consolidation as an alternative to expropriation}\label{sec:lce}

The notion of land consolidation is somewhat ambiguous. At its core, it refers to  mechanisms whereby boundaries in real property are redrawn to reduce fragmentation, without affecting the relative value of the different owners' holdings. However, it is also common to use consolidation to refer to mechanisms for pooling together small parcels of land to create larger units. There is a tension between these two notions of consolidation, with some claiming that consolidation in the latter sense is sometimes used to surreptitiously bestow benefits on powerful property owners, at the expense of weaker groups.

In light of this, I should stress at the outset that I will use the term land consolidation in a very broad sense in this chapter, much wider than {\it both} of the interpretations mentioned above. Land consolidation, as I use the term, refers to any mechanism by which the state intervenes, at the request of some interested party, to reorganize property rights in a given local area. Hence, a consolidation measure might as well involve {\it increased} fragmentation of property, if this is deemed a rational form of ``consolidation'' of the property values of the affected area. Importantly, I also use land consolidation to refer to efforts directed at organizing the {\it use} of property, not just redrawing boundaries.

Some might argue that this terminology is strained, but I adopt it for a reason. It is motivated by the fact that in Norway, the institution known as ``jordskifte'', officially translated as land consolidation, has a very broad meaning. Norway is not unique in this regard. Land consolidation has a similarly broad scope in many jurisdictions of continental Europe, as well as in Japan and in parts of the developing world.\footnote{See \cite{sky07;viliainen04}.} Moreover, in Heller and Hills' work on land assembly districts, a comparison with land consolidation is presented, based on the same broad notion that I use here.\footcite{heller08} One of my main aims in this chapter is to pick up on this, by offering a more detailed comparison and assessment, specifically anchored in the Norwegian system and its application in the context of hydropower development.

As land consolidation tends to involve interference in property rights, one may ask about the legitimacy of various consolidation measures, held against rules that protect private property owners. In some cases, it can also be argued that land consolidation {\it is } a form of expropriation, even if it is not necessarily recognized as such in the jurisdiction where it takes place. Such legitimacy issues have in fact been raised before the ECtHR on a few occasions, where the Court has found that land consolidation measures have been applied in breach of P1-1. Similarly, in the US, a proposal to introduce a land consolidation regime was struck down in the 1980s, as not in keeping with the property clauses of the US constitution.

\noo{ Move: On the other hand, if land consolidation is used to facilitate or impose specific uses of property, it can also be used as an {\it alternative} to expropriation, a compulsory measure that can obviate the need for depriving owners of their property rights. I think this latter perspective on land consolidation is particularly interesting, and it is the perspective I adopt in this chapter.}

In relation to the legitimacy issue, the Norwegian system stands out in two important regards. First, the consolidation procedure is managed by judicial bodies, namely the {\it land consolidation courts}. Second, land consolidation is largely seen as a service to owners, not a tool for increased state control and top-down management. In particular, a case before the land consolidation courts is almost always initiated by (some of) the affected owners themselves, in an effort to clarify who owns what, readjust boundaries, or organize the use of property in the local area. Moreover, it is a core principle of law that no land consolidation measure may be undertaken unless the benefits make up for the harms, for all the properties involved. This is known as the ``no-loss'' criterion. It is one of the key principles of land consolidation in Norway. The combination of a judicial procedure that places great emphasis on owner-participation and a no-loss criterion means that, arguably, land consolidation in Norway {\it strengthens} property as an institution. 

However, the beneficial effects of land consolidation are not primarily supposed to target individual owners, but rather the properties as such, as productive units  of importance to the community of owners as a whole. Hence, I believe land consolidation can serve as an effective countermeasure against two of the most widely discussed challenges to any property regime. First, consolidation can serve to protect an egalitarian distribution of property rights against the deleterious effect of inefficiency and underdevelopment that might otherwise arise from fragmentation. Importantly, it can do so without disturbing the underlying property structure and without necessarily resulting in disproportionate benefits or harms to owners and other private parties. In particular, land consolidation can obviate the need for handing property rights over to powerful market actors to ensure development. Second, land consolidation can serve to ensure sustainable and rational management of jointly owned land, without necessarily forcing an enclosure process (enclosure {\it can} be the result of land consolidation, but it is only one of many measures in the consolidation toolbox). 

In short, land consolidation can be used to address both commons and anti-commons problems, in a way that protects, and possibly enhances, desirable social functions of property, through a judicial system of participatory/adversarial decision-making. Hence, land consolidation in Norway is based on a conceptual premise that -- potentially -- offers protection to owners and their properties, by recognizing them as members of a community that are mutually dependent on each other. In this way, the form of property protection offered in the context of land consolidation is distinct from the protection offered in the context of expropriation law, in a manner that is in itself interesting, particularly from the perspective of property's social functions, as discussed in Part I.

Importantly, since land consolidation can be used to facilitate or impose specific uses of property, it can also be an {\it effective} alternative to expropriation, a compulsory measure that can obviate the need for depriving owners of their property rights. I think this perspective on land consolidation is particularly interesting, and it is the point of view I adopt in this chapter. Indeed, consolidation as an alternative to expropriation is particularly natural for economic development projects. Importantly, it seems that the no-loss criterion should be possible to fulfil in these cases, through benefit sharing. 

The land consolidation courts can, moreover, {\it impose} benefit sharing on the parties. However, it  is usually {\it not} permitted to address the no-loss criterion by compensatory means, particularly not if those means are monetary. Instead, the general idea is not only that the benefits resulting from the consolidation measure must match the harms, the benefits must also be distributed fairly among the affected properties.\footnote{The latter principle is not as strictly encoded in the law, but finds formal expression in certain special provisions. Hence, the extent to which fair benefit sharing is {\it actually} achieved following consolidation to facilitate large-scale economic development projects, is an interesting question that I return to in ....} So the principle of benefit sharing at work is not one where the owner is a passive recipient of compensation, but rather an active participant in the development itself, possibly against his own will. This, too, I find highly interesting, particularly from the point of view of the human flourishing conceptions of property that I discussed in Part I.

More pragmatically, the emphasis on benefit sharing in land consolidation reveals a concrete potential advantage of land consolidation over expropriation, from the owners' point of view. In particular, while some sort of benefit sharing is typically ensured through land consolidation, it is much less commonly achieved through compensation in the context of expropriation.\footnote{This is largely due to the so-called {\it no scheme} principle, which states that compensation to the owner following expropriation should not reflect changes in value that are due to the expropriation scheme. I am not aware of a single jurisdiction that does not include a variant of this principle. For a detailed investigation into the question of whether or not it stands in the way of benefit sharing in economic development cases, I point to \cite{....}.} This means that the use of land consolidation in place of expropriation has considerable potential also in relation to the problem of the ``uncompensated increment'' in economic development takings, as discussed in Part I.

On the other hand, this also means that commercially motivated developers may have an {\it incentive} to favour expropriation. This, then, raises the question of whether or not calling for the use of land consolidation as an alternative can act as a {\it defence} against expropriation, or, if this is not possible presently, if such a defence {\it should} be open to property owners facing condemnation for commercial purposes. Secondly, the fact that consolidation can act as an alternative to expropriation  also gives developers an incentive to push for changes in land consolidation law itself, so that it will become more profitable for them to make use of it.

Hence, one question that arises in the present context is the following: will land consolidation remain a service to owners, or will it become a service to developers who seeks cheap access to property owned by others? This question is becoming increasingly relevant in Norway, as the scope of land consolidation has been broadened in recent years, so that it {\it can} in fact be applied in many expropriation contexts, also to facilitate commercial development in urban areas.

So far, the Norwegian system is moving along a trajectory where land consolidation as an alternative to expropriation is primarily seen as a service to developers and the public, not as a means for empowering owners. It is noteworthy, in particular, that owners are not normally entitled to demand land consolidation in place of expropriation. Instead, following a change in the law that takes effect in 2016, the {\it developers} will be granted a new right, namely that of bringing a case before the land consolidation courts, to seek help in implementing their project. In fact, developers might well be motivated to do so, because of the potential for reduced administrative costs, a more effective and flexible procedure, and a chance of limiting compensation claims by imposing (cheaper) compensatory consolidation measures (e.g., by providing owners with replacement property).

\noo{However, the issue of benefit sharing is bound to come up, particularly in the context of commercial development. In this regard, the risk for developers is that they will be compelled to share the benefits with the owners. However, in order for this to happen, the Land Consolidation Court must actively take steps to make it happen, by recognizing the owners' right to benefit sharing. Moreover, while benefit sharing is a fundamental principle for land consolidation among owners, it remains to be seen if this way of thinking will be preserved when new and powerful external actors enter the scene.}

However, it seems that in order to be a truly effective alternative to expropriation, not only the takers, but also the owners, should be granted the opportunity to request implementation by consolidation. In addition to the question of whether a land consolidation measure can be requested in an expropriation scenario, one must also ask how exactly it would work, and what policy aims it could help achieve. Here there is already quite some data available, arising from situations when owners themselves are behind economic development, but prefer to make use of consolidation measures, instead of expropriation, in order to deal with their neighbours.

Interestingly, in the context of hydropower development, this use of land consolidation has become very important in recent years. In 2009, the Court Administration estimated that land consolidation had helped realise small-scale hydropower projects with a total annual energy output of about x TWh/year. Moreover, in a recent Supreme Court case, the importance of land consolidation was stressed specifically, as a justification for requiring a commercial taker to pay additional compensation to the owners of waterfalls that were to be used for hydropower generation.

In the next section, I give some more details on the Norwegian system. Then, in Section \ref{sec:x}, I return to the use of land consolidation to facilitate small-scale hydropower, which I approach as a test case for the more general proposition that land consolidation can be a legitimacy-enhancing alternative to expropriation, particularly in the context of economic development.

%%%%%%%%%%%%%%%%%%%%%%%%%%%%%%%%%%%%%

\noo{
In particular, a case can now be brought before the land consolidation court by an external developer who would otherwise need to expropriate land to implement a project. However, this change in the law also contributes to a shift away from seeing land consolidation as a service to owners, towards seeing it also a service to developers who seek control of property they do not own. This shift could in turn change the dynamics of land consolidation in a way that makes it less distinct from expropriation.

Even though this definition is broad, I note that a clear distinction can be drawn between land consolidation and national or regional land {\it policies}, which do not target specific properties. The distinction between land consolidation and land-use planning can be harder to draw, but looking to the theoretical starting points of these two kinds of interventions, suffice to establish sketch.

While state planning is an expression of the state's right to regulate the use of land, a land consolidation measure is a {\it service} provided by the state, to facilitate property uses and structures that are deemed desirable from the point of view of the properties as productive units under private ownership.

The distinction between consolidation and measures of land reform may sometimes also be difficult to draw, particularly with my wide notion of consolidation. However, while land reforms tend to arise from centrally directed measures that apply generally within a jurisdiction and come about as the result of a special political initiative, consolidation usually denotes a more flexible framework where local communities are restructured in a way that aims to bring benefits to all owners and rights holders within that community. As such, consolidation rules may alleviate the need for new land reforms, and they may come to represent a ``bottom up" approach to the restructuring of real property.\footnote{The potential for this has been noted even for the traditional understanding of consolidation, as a reduction in the level of property fragmentation. See, e.g., \cite{oldenburg90}. For a different perspective, arguing that land consolidation is generally not sufficient to achieve the noble ends of land reform, see \cite{lipton74}. For a more recent, comprehensive, assessment of the relationship between land reform and consolidation (in the narrow sense), I refer to \cite[237-244]{lipton09}.}

In the following, I adopt this normative stance on the {\it purpose} of consolidation. Hence, I use land consolidation to refer to a regulated process of land reorganization that come about as a result of a concrete, often local initiative, has a limited geographical scope, relies on the involvement of the local population, and seeks to promote the best interests of all the affected land users. I remark that while Norway has a particularly broad approach, land consolidation more or less in line with my understanding here serves an important function in many jurisdictions.\footnote{For a survey of contemporary land consolidation rules in Europe, reflecting also the need for a wide understanding of the term, I point to \cite{vitikainen2004}.}

One attractive feature of consolidation is that it provides a flexible, dynamic, framework that allows for gradual adaptation of ownership structures, so that they better suit prevailing economic and social conditions. Moreover, land consolidation can become significant in relation to concrete development projects, particularly when such projects necessitate cooperation among several owners. This, in particular, is the use of consolidation that I aim to shed particular light on in this chapter, by giving a case-study of Norwegian law.

Mechanisms for facilitating and organizing concrete development schemes are now integrated into the law relating to consolidation in Norway. These rules do not form part of the historical core of consolidation rules in Norway, however, the focus on land consolidation for development is of a more recent date. It is reflected in a number of new provisions, most recently in the Land Consolidation Act 2013 which will take effect on 1 January 2016.\footnote{Act no 97 of 10 June 2013 relating to the determination and change of structures of ownership- and rights to real property etc.}

When land consolidation is used as a means to organize development projects it also becomes natural to view it as an alternative to expropriation, especially in cases when development has commercial potential and is meant to be carried out by companies operating for profit. Moreover, the controversy that often surrounds such cases further suggests that it should be explored to what extent processes of consolidation can replace expropriation as an implementation mechanism for development of this kind. As we will see, the principle of local participation and benefit sharing is more firmly entrenched in the rules and procedures that govern the consolidation process than in the processes that govern the use of expropriation. 

The contrast between expropriation and consolidation is particularly clear in Norwegian law, where a ``no loss" principle is enforced with regards to the latter, protecting all affected owners and rights holders. It states that the consolidation process should not leave any owner or rights holder worse off after consolidation. The aim of consolidation is to bestow a benefit on \emph{all} interested parties.\footnote{See Section 3 a) of the Land Consolidation Act 1979 (currently in force) and Section 3-18 of the Land Consolidation Act 2013. For a paper discussing the rule in more detail we point to \cite{rygg1998}. Rygg is also critical of what he sees as a development away from a strict interpretation of the no loss rule.} For instance, if ownership is highly fragmented, consolidation mechanisms may be used to exchange property between owners or to introduce joint ownership, but due to the no loss rule it will not be possible to use consolidation in order to deprive some owners of their property to the benefit of others.

In the following, I map the differences between consolidation and expropriation in Norwegian law, starting with an overview of the land consolidation rules, focusing on the development towards giving these rules greater application in connection with concrete development schemes. I then study some cases of locally controlled hydro-power development where land consolidation was used as a means to organize projects involving many different owners and rights holders. We argue that these cases illustrate how the consolidation rules currently in place are well suited to meet local demands for participation and benefit sharing, more so than the existing framework regulating expropriation.

The structure of the remaining part of the chapter is as follows. In Section \ref{sec:2} I briefly present the basic rules regarding land consolidation in Norwegian law, including a presentation of the special consolidation courts used to administer the process. Then in Section \ref{sec:3} I go on to consider in more depth the rules relating to so-called \emph{use directives}, permitting the court to actively pursue development projects on behalf of, and in cooperation, with local owners. Use directives represent a form of compulsory cooperation which I believe deserves further attention in the context of land development, especially as an alternative to expropriation. I follow up with a case-study of hydropower in Section \ref{sec:4}, and in Section \ref{sec:5} I contrast the use of directives with more commonly seen approaches to pooling of resources and commercial land development. In Section \ref{sec:6} I offer a conclusion.
}

\section{The system of land consolidation courts}\label{sec:lcc}

Rules regarding land consolidation have a long history in Norwegian law. The first consolidation rules were included already in King Magnus Lagabøte's \emph{landslov} (law of the land) from 1274, the first piece of written legislation known to have been introduced at the national level in Norway.\footnote{See Chapter 4, Section 2 in \emph{Jordskifterettens stilling og funksjoner}, NOU 2002 no 9, report to the Department of Agriculture from special committee appointed by the King in Council 10 October 2000.} The earliest rules targeted jointly held rights in farming land, giving any owner or tenant farmer on that land an opportunity to demand apportionment that would give him exclusive rights on a parcel of land corresponding to his share of the joint rights.\footnote{The share in joint rights belonging to each individual farm was historically determined based on the amount of rent (``skyld") that each farmer paid to the land owner. However, following the union with Dennmark and especially after the advent of enlightened absolutism, tenant farmers in Norway increasingly bought their land from increasingly marginalized Danish land owners. Indeed, tenant farming became uncommon in Norway after the 18th Century, but the notion of skyld was kept as a measure of the share each farm had in the now jointly owned larger estate, and it is still relied on in various contexts, such as for the purpose of apportionment. References needed.}

Many of the rules currently in place were developed in the 19th Century. At this time, the main use of land consolidation was still to divide jointly owned land into parcels, but the relative importance of such measures increased greatly since it was seen as a necessary adjustment in an age when industrialization introduced a range of new and more efficient farming techniques. In particular, as changes in farming methods resulted in an increased need for capital in agriculture, full ownership came to be regarded as more favourable since it meant that better security could be offered to financial institutions.\footnote{References needed.} 

However, in some cases it was noted that full division of ownership might not be required and that use directives could be employed instead, to provide the individual farmers with clear rights of use over joint land. In addition to providing clarity and security for users, the rules introduced to facilitate this also created a legal framework for implementing development without altering the underlying structure of ownership, even in cases when development would require considerable pooling of resources and decision-making power. The rules were initially targeted at more rational organization of land use in agriculture, enabling rural communities to adapt to changing economic conditions without fundamentally altering them or leading to displacement or depopulation. Hence they were mostly relied on in connection with farming, and not commonly used to facilitate different kinds of development.

In recent years this has changed. Today, use directives are increasingly applied also to organize development projects that are not associated with traditional farming. Moreover, many additional mechanisms of land consolidation have been introduced, all aiming in various ways to ensure better organization of land use and ownership. These mechanisms are administered by the \emph{consolidation court}, a special tribunal which has land consolidation as its sole task.\footnote{It appears to be a unique administrative unit in the European setting, although Austria have land tribunals that resemble it. References.} There are three main categories of consolidation tools that the court may use, and they are summarized in the following.

\begin{itemize}
\item \emph{Apportionment of land}: Rules that empower the court to dissolve systems of joint ownership by apportioning to each estate a parcel corresponding to its share, or by reallocating property through exchange of land. This is the traditional form of land consolidation in Norway, and the main legislative basis for it is provided in Section 2 a-b) of the Land Consolidation Act 1979.
\item \emph{Delimitation of boundaries:} Rules that empower the court to determine, mark and describe boundaries between properties and the content and extent of different rights of use attached to the land. The main legislative basis for this form of consolidation is found in Section 88 of the Land Consolidation Act 1979.
\item \emph{Directives for use}: Rules that empower the court to prescribe rules for the use of jointly held land, and to organize such use, including setting up organizational units for carrying out specific development projects, as described in Section 2 c) and Sections 34-35 of the Land Consolidation Act 1979. 
\end{itemize}

In all cases, the consolidation court can only employ these tools when they are called on to do so by someone who is regarded as having a valid legal interest in the matter.\footnote{See \cite[5]{lca79}.} Traditionally, this meant one of the owners of the land involved, but gradually the system have also come to recognize that others might have a legitimate interest in consolidation. This includes developers who have obtained planning permission for specific projects that require reorganization of property rights. The legal role of such actors is currently changing from passive to more active, and this development raises particular questions that we will return to below. For now, I note that the traditional situation is that a consolidation process is initiated by one of the local owners of the land. It is still true, in particular, that consolidation is primarily a mechanism by which any one among the owners can work out what their legal position is, and, if the court agrees that it would be favourable to the use of the land, can ensure that the rights in the land are restructured.

The condition that restructuring only takes place when it is regarded as favourable is an additional condition that limits the court's authority to take action that involves apportionment and directives for use. To determine whether or not it has been met, the court will look to the current economic and political climate, and so the consolidation rules are \emph{dynamic}, capable of being adapted to the circumstances. In this regard the court is also influenced by what it regards as the prevailing public interests in property use, and recent developments in consolidation law stress the importance of this link, with recent reforms seeking to strengthen it.\footnote{See for instance Prop. 101 L (2012-2013) (report from the Department of Agriculture regarding the new Consolidation Act).}

However, the contextual nature of land consolidation has always been clear. We have already mentioned how the basic building blocks of the current system can be traced back to the influence of technological advances in farming and the modernization processes that Norwegian society underwent in the 19th Century. The law responded to these changes, and consolidation became a vitally important instrument for change and development in this period. It was also at this time that it was decided to establish a tribunal system for administering the process, first in the Land Consolidation Act from 1857 and then revised and developed further in acts from 1882 and 1950.\footnote{An overview of the history of consolidation law is given in Chapter 3 of Prop 101 L (2012-2013).} The procedural rules closely mimics those that pertain to the regular civil courts. This ensures that consolidation tools are only put to use if a court orders it, and only after a public hearing where all involved parties are given an opportunity to present their case, give supporting evidence, and to contradict each others' testimony. For a more detailed description of the consolidation court, I refer the reader to Section \ref{subsec:21} below.

The current system for land consolidation is based on the Land Consolidation Act 1979, but this act will be replaced in 2016 when the new Land Consolidation Act 2013 will take effect. The new act was passed on 10 June 2013, and while it does not introduce any dramatic changes to the law, it further widens the scope of consolidation, particularly with regards to directives for joint use, as discussed in Section \ref{sec:3} below. The new act also contains an explicit description of the purpose of land consolidation, which we now quote.\footnote{Act no 97 of 10 June 2013 relating to the determination and change of structures of ownership- and rights to real property etc. (henceforth the \emph{Land Consolidation Act 2013}). It will take effect on January 1 2016.}

\begin{quote}
Section 1-1 The purpose of the Act

The purpose of the act is to facilitate efficient and rational use of real property in the best interests of the owners, rights holders and society. This objective will be pursued by the land consolidation courts which will implement remedies for unpractical structures concerning ownership and use of property, ascertain and determine property boundaries, as well as decide appraisal disputes and other cases as pursuant by this and other acts.

The act also seeks to facilitate fair, responsible, quick and effective processing of cases in independent and impartial public courts that will operate in such a way as to enhance confidence in the consolidation process.
\end{quote}

This statement of purpose highlights how the new act incorporates and extends the trend towards giving the consolidation process wider scope. I also note how it reiterates and emphasises that the process is to be tribunal in nature. In my opinion, it also suggests that land consolidation is likely to become more important in the future, increasingly also outside the traditional agricultural setting within which the ancient body of law regulating it has hitherto developed.\footnote{For instance, following a change in the Land Consolidation Act 1979 in 2006, Land Consolidation may now also be called on in order to manage restructuring of ownership in urban areas, in connection with specific development schemes. This rule has been extended further in the new Act, and it will be interesting to see how the division of labour will be in the future, between planning authorities, regular courts and the land consolidation courts.}

It is now explicitly stated that the purpose of land consolidation is to make conditions of property use more favourable for all the affected owners and rights holders. Hence the new act accentuates how consolidation represents a form of interference that is fundamentally different from expropriation. As before, the consolidation court is not empowered to take action unless it is called on by one of the stakeholders in the property, see Section 1-5 of the Land Consolidation Act 2013. However, according to the new act, a developer who has obtained permission to expropriate is to be counted as a stakeholder in that land for the purposes of consolidation.\footnote{Previously, a developer was only regarded as a stakeholder in consolidation in some cases of public projects, c.f. Sections 5, 88 and 88 a) of the \cite{lca79}.} This further reflects how it is becoming increasingly natural to see land consolidation as an alternative to expropriation. It also flags how the relationship between expropriation and consolidation is now becoming an important topic in Norwegian land law.

In 2005 the Department of Agriculture made some comments in this regard, in connection with a revision of the current 1979 act that gave consolidation greater applicability in urban areas and with respect to implementing public plans.\footnote{See in particular Section 2 h-i) of the Land Consolidation Act 1979.} Some members of the preparatory committee had raised the concern that giving consolidation extended scope in this way would be problematic since it would encroach on expropriation law and effectively render consolidation a form of expropriation. The Department disagreed, commenting as follows.\footnote{See Chapter 3.3 of Ot.prp. no 78 (2004-2005), report to parliament from the Department of Agriculture regarding changes in the Land Consolidation Act 1979.}

\begin{quote}
The Department would like to point out that one of the main preconditions for consolidation is that a net profit is created for the land in question. This profit is then divided among the parties in an orderly fashion. Individually, the law also guarantees that no one suffers a loss, see Section 3 a). [...] \\ \\ In the Departments opinion, expropriation takes place on a different factual and legal basis. In cases of expropriation the public makes decisions that deprives the parties of economic value. The purpose then becomes to compensated them in accordance with Section 105 of the Constitution, not to increase the value of their land or the annual income they may derive from it.
\end{quote}

When preparing the new act, the Department of Agriculture reiterated this position but they did not reflect further on the question of the exact relationship between consolidation and expropriation. They observed, however, that changing the law so that expropriating parties could appear in consolidation cases was \emph{reasonable} since it would then be left up to the developer whether to make use of his permission to expropriate or to rely on consolidation instead.\footnote{See page 84 of Prop.101 L (2012-2013).} Indeed, it seems that in many cases, the well-organized and tightly regulated process of consolidation might be a more practical alternative for developers than the more fragmented rules and administrative bodies that come into play following traditional expropriation. 

However, it seems clear that the choice made by the expropriating party in this regard will tend to be even more important for the affected owners and rights holders. In particular, as the Department themselves made clear in the passage quoted above, it is an absolute precondition for implementation of any consolidation measures that alter the rights structure that they must serve to make the structure of ownership and use more favourable for \emph{everyone}. This, moreover, refers explicitly to the \emph{area within which consolidation takes place}, as stated in Section 3-3 of the new act. No similar rule is in place to protect the affected local area following expropriation. Moreover, the practices that have developed for dealing with consolidation cases are centred on the interests of the local owners and their land to an extent that is quite different from any procedure that is currently in place to facilitate development by use of expropriation.

For instance, the rule regarding expropriation that corresponds most closely to the no-loss rule  
requires merely that the benefit to private and public interest exceeds the disadvantages \emph{overall}, not locally and certainly not for individual local owners.\footnote{See Section 2 of the Expropriation Act 1959.} However, I also note that the strict consolidation rules do not serve as a restriction on \emph{what} kind of development should be carried out, only on \emph{how} it should be organized. The former question is left to the planning authorities, and the consolidation courts must always base their decisions on existing public regulation of property use.\footnote{In Section 3-17 of the Land Consolidation Act 2013 it is explicitly stated that the consolidation court cannot prescribe solutions that are not in keeping with such regulation. However, it is also made clear that the consolidation court itself can apply for necessary planning permissions on behalf of the owners and the land in question.}

Hence, if the public interest suggests a particular form of land use, the fact that a planning decision detailing development of such use is implemented through consolidation does not entitle the court to review the plans themselves, going against the public interest. But it does introduce an obligation, emerging at the time of implementation, to turn specifically to the interests of original owners and rights holders and to look for solutions that minimize the burden and maximizes the benefit for all the involved parties.

The rules that give the consolidation court authority to give directives of use are particularly relevant in this regard, and we return to them in Section \ref{sec:3} below. First we will present the consolidation process itself. It seems, in particular, that the procedural guarantees resulting from the fact that this process is organized as a tribunal are in themselves an important factor to consider when looking at consolidation as an alternative to expropriation.

\subsection{A Brief Presentation of the Consolidation Process}\label{subsec:21}

A consolidation case is usually initiated by an owner or a permanent rights holder.\footnote{See Section 5, Paragraph 1 of the Land Consolidation Act 1979.} The request for consolidation measures is to be directed at the relevant district consolidation court, one of the 34 district courts for land consolidation that have been set up by the King in accordance with Section 7 of the Land Consolidation Act 1979. The request is meant to include further details about the affected properties, the owners and rights holder involved, as well as the specific issues that consolidation should address. But the requirements in this regard are not usually interpreted very strictly and the district consolidation court will often take on quite some responsibility for further clarifying what the case should encompass, more so than in civil disputes.\footnote{References needed.} Even so, the court is entitled to reject the request due to technical shortcomings, following the same rules as those which applies to civil disputes.\footnote{See Section 12, Paragraph 2 of the Land Consolidation Act 1979, which refers to Section 16-5 of the Civil Dispute Act 2005}

If the court decides that the request is well-formed and that it includes sufficient detail to permit consideration of the substance, they go on to prepare public hearings, following the rules set out in Chapter 3 of the Land Consolidation Act 1979. These rules mirror those that are in place for civil hearings in general, including the duty to inform affected parties (Section 13), the parties' right to present their claims, and their duty and right to give testimony and provide evidence supporting it (Sections 15, 17 a) and 18). As in civil cases, the decision is usually only reached after at least one hearing in which the parties are present and permitted to contradict the evidence provided and the testimony given by other parties. However, unlike in civil cases, the main hearing typically takes place on the disputed land itself, and consists in mapping and clarifying the prevailing conditions aided by visual inspection. Moreover, a consolidation case will usually not take the form of a two-party adversarial process, but rather as a multi-party discussion where the court interacts with a large number of interested parties who may have a range of common as well as conflicting interests. Usually, consolidation cases involve at least 10 or more different parties, and in some cases there can be hundreds. In addition, it is quite common that the parties are not represented by legal council, but rather take an active part in the process themselves.\footnote{References needed.}

The request for consolidation will be the court's point of departure when assessing the case, but the court is not bound by the claims put forth in it, or by the claims put forth by the other parties. This again marks a differences with most cases of civil dispute. With a few exceptions explicitly listed in statute, the consolidation court may decided to use any measure that it deems suitable to ensure a favourable structure of rights and ownership for the future. However, there is some restriction placed on the court in that the measures taken must be regarded as \emph{necessary} in light of considerations based on the original request.\footnote{See Sections 26 and 29 of the Land Consolidation Act 1979.} So while the court should remain focused on the issues raised by the parties, it should be free to address these issues using the tools they deem most suited for the job. The consolidation court, in particular, is meant to be a general ``problem solver", more so than the ordinary civil courts.

When a decision is reached, the rules in Section 17 and 22 of the 1979 act ensure that the parties are notified and that the decision is presented and argued for in keeping with the rules of the Civil Dispute Act 2005. The appropriate form of the decision will depend on its content. A regular civil ruling is the form used for decisions that only involve ascertaining the boundaries between properties, while a special ``consolidation decision" is the form relied on to implement apportionment and directives of use. The difference becomes clear as soon as we consider the appeals procedure; while civil rulings are dealt with by the regular courts of appeal, the consolidation decisions can only be appealed to one of 4 designated consolidation courts of appeal.\footnote{See Section 61 of the Land Consolidation Act 1979.} 

In the latter case, the procedural rules remain largely the same in the consolidation court of appeal, meaning also that there is a new assessment of all aspects of the case.\footnote{See Section 69 of the Land Consolidation Act 1979.} When the case is concluded here, however, it can only be appealed on the grounds that it is based on an incorrect understanding of the law, or that procedural mistakes have been made. In this case, the ordinary appeal courts have authority, with the Supreme Court being the last instance of possible appeal.\footnote{See Section 71 of the Land Consolidation Act 1979.}

From the brief overview of the process given above, we see that consolidation cases are different from other civil cases in that they have fundamentally different scope. A consolidation case is not primarily centred on deciding the merits of individual claims, but rather at introducing structures of ownership and rights that will prove favourable in general. In this respect the process has an administrative character. However, the fact that it is organized more or less like a regular civil dispute means that the protection of each affected party, and the influence of the local rights holders as a group, is much stronger than what would tend to be the case if these decisions were made by regular administrative bodies.

Given this context of arbitration, it is not surprising that the judges appointed to the consolidation courts are required to have a special skill set, different from that of regular civil law judges. In fact, consolidation judges are required to have successfully completed a special masters level degree in consolidation, which is not a law degree at all but a separate form of education.\footnote{See Section 7, Paragraph 5 of the Land Consolidation Act 1979. The degree in question is currently offered only at the Norwegian College of Life Sciences and Agriculture.} 

The consolidation court also relies on the participation of lay judges who sit alongside the specialist judge.\footnote{See Section 8 of the Land Consolidation Act 1979.} These judges are appointed by the specialist judge from a committee of laymen that are elected by the local municipalities in accordance with Section 64 of the Courts Act 1915.\footnote{See Section 8 of the Land Consolidation Act 1979.} In the district courts, the specialist judge usually sits with two appointed lay judges which he chooses himself from among members of the the relevant local committees. In the court of appeal, the specialist sits with 4 laymen, and in complicated cases 4 laymen may also be called on in the district courts, but only if one of the parties requests it.\footnote{See Section 9, Paragraph 2 of the Land Consolidation Act 1979.} To the extent possible, the appointed laymen should have special knowledge of the issues raised by the case, but they are drawn from the general population.\footnote{See Section 9, Paragraph 5.}

Summing up, we observe that the consolidation process has both administrative and adversarial characteristics. While the content and scope of the court's decision will often have an administrative flavour and is not primarily directed at settling any specific dispute, the process is judicial. Hence everyone is entitled, and to some extent even \emph{obliged}, to have his voice heard and to partake in the process. Moreover, while the process is guided and overseen by the court, it is fundamentally based on considerations arising from the interests of the parties. However, this interest is always interpreted in light of prevailing notions of what counts as favourable and rational property use. Importantly, in relation to this latter assessment, the court will look beyond the interests of the individual owners. The court will pose the question with regards to the use of the land as such, drawing on its understanding of the relevant economic, social and political conditions.\footnote{References needed.} But the decisions made are always prepared using information that is retrieved and discussed in public hearings, so the affected parties will take part in discussions that may also address more overreaching concerns about the form of land use that should be regarded as favourable for the area in question.

To flag the dual nature of the consolidation process it is tempting to designate it as a process of judicially structured \emph{deliberation}. The final decision-making authority is granted to the court, but the court is required to act on behalf of the rights holders, in the best interests of their land, and based on the information that they themselves provide. This particular form of decision-making based on multi-party deliberation is interesting in its own right, as it provides a template for management of land that seems capable of catering both to the idea of public oversight and control as well as to the idea of local participation. In addition to this, it seems to be a form of land management that might be especially suitable as a means to implement concrete projects undertaken in the public interest, particularly when these would otherwise appear to adversely affect individual land owners and local communities.

This is of particular interest in mixed economies such as seen in Norway, where decisions regarding development and use of property are typically made by the public but carried out by private property owners. In many cases, implementation of public policy requires some form of reorganization of ownership and rights structures, the most common being a pooling of resources from many different owners. Such processes have tended to be implemented rather crudely, by displacing the original owners in favour of commercial companies who serve as state agents. This relies on the use of expropriation, and it typically completely deprives the original owners of any chance to take part in the future development of the land. The land consolidation rules allows us to consider alternative means of implementation in such cases. 

They allow us to ask whether a more measured approach might be sufficient, allowing the original owners to retain their rights, but restructuring them using consolidation mechanisms. This question is particularly interesting to consider due to the high levels of tension often associated with cases of commercial expropriation, where companies operating for profit benefit from implementing public plans. Critics argue that such uses of expropriation are both unfair in themselves and also destabilizing in that they raise doubts abut the true motives behind specific acts of public planning.\footnote{References needed.} In particular, it seems that in a system of land management where development is organized in this way commercial companies will have much to gain from attempting to exert influence over the planning process, particularly if they can also succeed in being granted permission to expropriate property rights that they would otherwise have to acquire on an open market. However, to counter critics it may appear easy to argue from necessity, by pointing out that the system is the best known alternative for efficient and rational economic development in a system based on public control over planning and private rights to property.

In relation to this debate it seems that the consolidation procedure takes on particular relevance. It may point to an alternative, a system of public-private development where the original owners and local communities are better integrated into the process. Moreover, it allows us to introduce an additional conceptual layer between the planning stage and the implementation step, a layer of management devoted to translating public plans into concrete action by orderly restructuring of existing ownership patterns. This, in particular, might be a layer of administration that deserves more attention and more fine-grained tools than those currently offered in systems relying on expropriation. Clearly identifying such a consolidation layer in property management might also make for a cleaner delineation between commercial implementation on the one hand, governed by the market, and public planning on the other, governed by administrative law and political bodies. 

In the next section, I argue that Norwegian consolidation law already include tools that make it possible to view consolidation in this light. The rules that I believe warrant this conclusion are the rules relating to joint use directives, briefly mentioned above. I present them in more detail below, noting that recent changes in consolidation law give them wider applicability in relation to concrete development projects. 

\subsection{Joint use, joint action and joint investment}\label{sec:3}

In accordance with Section 2 c) of the Land Consolidation Act 1979 the consolidation court can give directives regarding the use of land which involves more then one property. Typically this will target land or land rights that are owned jointly or for which some form of shared use has already been established. However, if the court finds that there are \emph{special reasons} for giving joint use directives, it can do so even if there is no prior connection between the different rights and properties in question.\footnote{See Section 2 c), Paragraph 2 of the Land Consolidation Act 1979.} Traditional examples include directives for the shared use of a private road which crosses several different properties, or regulation of hunting that takes place across property boundaries.

The joint use rules emerged as an alternative to apportionment of jointly owned property, a more subtle and less invasive measure that could often give rise to the same positive effect as a full division of ownership, but without leading to unwanted fragmentation of control and use of property. Hence in the now repealed Land Consolidation Act 1950 it was stated that joint use directives should be the \emph{primary} mechanism of consolidation, and that apportionment should only take place if such directives were deemed insufficient to reach the goal of creating more favourable conditions for the use of the land.\footnote{References needed.} In the 1979 act the two mechanisms were formally put side by side, but in cases that are motivated by a specific planned use of the land in question, directives will still be the main tool relied on by the court.

Moreover, there has been a gradual increase in the willingness of the court to rely on use directives to facilitate \emph{new development} on the land, not just as a means to regulate an existing activity. In parallel with this development, the consolidation court has gradually come to take on cases that pertain to organization of land use that was previously thought to lie outside its area of competence. The more restricted view on use directives and on the function of consolidation in general is reflected in the way the 1979 act lists a range of different concrete circumstances in which such directives might be applied.\footnote{See Section 35 of the Land Consolidation Act 1979.} The list is not understood to be exhaustive however, and the courts have gradually come to feel less deterred by it and more willing to consider new types of cases.\footnote{References needed.}

Hence in the new act of 2013, the list is replaced by an explicit general rule which makes it clear that the  consolidation courts have the authority to give directives whenever they regard this to be favourable to the properties involved.\footnote{See Section 3-8 of the Land Consolidation Act 2013.} In addition to this, the new act also introduces a general rule which gives the court authority to \emph{set up} systems of joint ownership when a joint use directive is deemed insufficient.\footnote{See Section 3-5 of the Land Consolidation Act 2013.} Hence in the new act apportionment and pooling of property is on equal footing, although a priority rule is introduced for the latter; pooling will only be considered if directives of joint use are regarded as an insufficient means to ensure more favourable conditions. Moreover, the new act maintains the principle that directives regarding the joint use of land for which there are no existing joint rights can only be given if there are special reasons.

This requirement is not intended to be very strict, and the Ministry of Agriculture was initially inclined to remove it. However, it was eventually decided that it should be kept in order to flag that there two distinct questions that arise in such cases. The court must first consider the question of whether or not joint use is in itself desirable, before it goes on to consider how to best organize such use.\footnote{For a discussion on this see page 140-141 of Prop. 101 L (2012-2013).}

In addition to giving directives prescribing how joint use is to be organized, the consolidation court may also give rules compelling the owners to take joint action to help facilitate better realization of the potential inherent in the land. Rules to this effect were first introduced in the 1979 act, in Section 2 e) and Sections 42-44. These rules only pertain to joint action by property owners (see Section 34 a), and they have wider scope in relation to specific case types (Sections 43 and 44). Following the new act, however, the consolidation courts will have authority to prescribe joint action also for right holders, and the special rules listing concrete circumstances will be replaced by a general joint action rule.\footnote{See Section 3-9 of the Land Consolidation Act 2013.} This broadens the scope of these rules in accordance with the general spirit of the new act. Indeed, when they commented on this change in the law, the Department noted that the rules in question have been widely used following their introduction in 1979 and that applying them is now one of the core responsibilities of the consolidation court.\footnote{See page 146 of Prop. 101 L (2012-2013).}

I note that joint action directives can include prescriptions for joint investments.\footnote{See Section 3-9 of the Land Consolidation Act 2013.} On the one hand this means that such directives can be used to facilitate capital intensive new development, but it also raises the question of the extent to which it is legitimate to rely on compulsion in this regard, directed towards individual owners of property. The extent of the joint actions and investments required to undertake development projects can easily become quite burdensome for these individuals, and this is especially likely to arise as a concern in cases where the land lends itself well to large- scale commercial development.

The 1979 act attempts to resolve this in Section 34 b) and in Section 42. The former states that if joint actions or investments may come to involve``great risk", the court must set up two \emph{distinct} organizational units to undertake it. First, the rights needed to undertake the scheme will be pooled together and managed by an owners' association, and then, to undertake the scheme itself, a cooperative company structure will be set up on behalf of the owners. Hence the risk is diverted away from the individual owners onto a company controlled by them. This company will be entitled to any potential profit from the scheme, but it will also be required to pay compensation to the owner' association on terms established by the parties themselves, with the help of the court.\footnote{See Section 34 b) of the Land Consolidation Act 1979.} Moreover, the owners are entitled to shares in this company proportional to their share of the relevant rights in the land, as determined by the consolidation court. An owner is not obliged to take part in the undertaking by acquiring such shares, but he will benefit from membership in the owners' association regardless of whether or not he chooses to do so.

After this brief survey of the rules, I conclude that the land consolidation courts in Norway already have all the tools they need to organize development projects on behalf of local owners. Moreover, the process of consolidation means that they must do so in a way that enables the original owners to retain considerable decision-making power as well as the right to any commercial benefit that may result from the development. Hence, the rules currently found in consolidation law adds weight to the claim that and consolidation might point to an alternative and possibly fruitful way of implementing development projects in a system which presupposes that development takes place through commercial initiatives on the basis of public  planning and control. In particular, the system already provide the tools needed to organize large-scale development even when it requires considerable reorganization of land rights and diversification of risk. Consolidation may therefore become an alternative way of pooling together fragmented rights for the purpose of development, without displacing the original owners in favour of commercial companies who have no prior connection to the local community in which development takes place.

In addition to this, I observe that the consolidation rules also point to a form of implementation that will allow the public to exercise \emph{more} extensive oversight and control. Not only is the position of the original owners much better protected under this system, but it also greatly \emph{curbs} the power and influence of commercial forces with no prior connection to the land. Hence, it must be expected that implementation through consolidation is better suited also to serve the social and political aims which originally motivated the underlying planning decision. Indeed, commercial development through consolidation give the public a greater say in the implementation stage; after all, the development is organized as a cooperative, and the company structure is set up and regulated by the courts who is obliged to consider also the public and societal interests in land use.

In addition to this, after the new act takes affects, both planning authorities and commercial developers may take up a role as formally recognized parties in the consolidation process. This seems particularly useful in connection with large scale industrial development, as it might otherwise be hard to implement such projects successfully. In these cases, then, the consolidation system sets up an arena for interaction and deliberation between the three main groups of stakeholders: the public, the local owners and the commercially motivated developers. Such an arena is so far missing at the implementation stage of big development projects, while recent controversies regarding expropriation suggest that it might come to serve an important function.

It remains unclear to what extent the consolidation rules will actually be used in this way. But as we will see in the next section, consolidation is beginning to emerge as an important means for organizing local hydro-power development. On the theoretical side, then, it is also unclear to what extent original owners may \emph{demand} that the rules are put to use. For instance, may an owner request consolidation to prevent a permission to expropriate from being implemented? It will be interesting to see how the Norwegian legal systems will deal with this and related questions, after the new act takes effect in 2016.

I conclude this section by addressing a new special rule that has been included in the new act, and which is specifically targeted at the planning authorities, encouraging them to make use of consolidation to achieve  greater fairness in public planning. The new rules are contained in Chapter 5 of the new act and they target benefit that arises from planning in cases when the benefit appears to fall disproportionately on some owners. Such cases of ``windfall" benefit due to public plans are often flagged as problematic, and they arise with particular frequency in systems based on commercial implementation. For example, if one parcel of land is designated for housing and some neighbouring land is designated as a playground, it might easily come to be seen as unfair that a considerable financial benefit falls to the owner of the land designated for housing, while the playground owner is left with virtually nothing.

Following a change of the Consolidation Act in 2006 which has been further extended in the new act, the law now makes it possible for the planning authorities to decide that apportionment of the \emph{benefit} arising from the plan may be carried out by the consolidation courts.\footnote{See Section 3-30 of the Land Consolidation Act 2013 and Section 12-7 nr. 13 of the Planning and Buildings Act 2008.} When doing so, the consolidation court will follow the same procedure as in other cases, and it will allocate the benefit arising from the plan based on an assessment of the development potential of the different parcels. Importantly, the court will consider this question independently, and the decision will not be based on the particular manner in which the plan dictates that development is to be carried out. For instance, if the land used for the playground could just as well have been used for housing, the court may decide that the rights to housing development is to be shared equally between the two properties. On the other hand, if there is some independent reason why the playground property is not suited for housing, the court will reduce this property's share in the housing development correspondingly.

The court can implement solutions such as this more effectively and rationally by applying the other tools that it has available. For instance, if the the owner of he playground is entitled to an equal share in housing, then apportionment can be used to actually provide him with such a share, trading it for a corresponding share in the playground. However, if such material reallocation of development rights prove unfeasible, the new act also opens up for a solution where the benefit sharing is implemented using financial compensation.\footnote{See Section 3-32 of the Land Consolidation Act 2013.}

To sum up, directives of use rules are highly versatile and may be used to organize extensive projects of land development on behalf of original owners. This form of development makes it possible for original owners to maintain their interest in the land, it can prevent the need for expropriation, and it may give the public a greater opportunity to exert influence and control over how their planning decisions are implemented in practice. In the next section, I consider in depth the particular case of hydropower, where the consolidation courts have recently started to make use of a wide arsenal of its tools to ensure that development can be carried out in this way. I think this case-study sheds further light on consolidation as an alternative to expropriation, and further strengthens the argument that directives of use issued by a consolidation court can in many cases obliterate the need for depriving local people of their resources to implement public development plans.

\section{Compulsory participation in hydropower development}\label{sec:lch}

In this section, I look at four recent cases in detail, all of which involved directives of use for hydropower development by original owners. The waterfalls dealt with in these cases are all located in the county of \emph{Hordaland}, in south-western Norway. Three of the cases involved small-scale hydro-power which some of the owners wanted to develop themselves, while the fourth was a case when the owners were also considering a development plan which would involve cooperation with an external energy company. The cases are particularly interesting because we have access to data on how the process of consolidation, and the outcome, was perceived by the owners themselves. Interviews were conducted and used in a recent master thesis on land consolidation which is devoted to the study of how consolidation measures is now increasingly being used to facilitate hydropower development \cite{master}.

In the following, I first present each case separately, focusing on those issues that were raised regarding how to organize development, the solutions prescribed by the court, and the subsequent reception among the parties. I then assess this from the point of view of developing a better understanding of compulsory cooperation as an alternative to expropriation. I conclude with some unresolved questions, particularly regarding those situations when the court is called on to resolve disagreement regarding how the development itself should be organized. These are the cases when the relationship between consolidation law and other legal frameworks, such as company law, planning law, and water law, becomes pressing, and there are many unresolved questions.

\subsection{\emph{Vika}}

The case was brought before the consolidation court in 2005, by owners who all agreed that hydropower development should be pursued.\footnote{Haugalandet og Sunnhordland jordkifterett, case no 1210-2005-0014.} The owners disagreed on how to organize the owners' association, and on how the shares in this association were to be divided among the different properties involved, 15 in total. The main principle was agreed upon from the start, however, namely that the owners would rent out their waterfall to a separate development company which every owner would have a right (but not a duty) to take part in. 

The parties in \emph{Vika} were highly involved in the consolidation process, and the statutes for the owners' association were based on suggestions made by the owners themselves. The main point of disagreement concerned how the shares in this association should be allotted, a question that was made more difficult by the fact that some owners benefited from old water-mill rights in the river. In the end, the consolidation court landed on the view that these rights were tied to the form of use relevant at the time they were established, and did not regard them as having any financial value. Hence these rights were extinguished without compensation, as provided for in Sections 2, 36 and 38 in the Land Consolidation Act 1979.

There was also some disagreement about whether the number of votes in the owners' association should be tied to the number of shares belonging to each owner, or if the owners should simply be allotted one vote each, irrespectively of their share in the waterfall. The consolidation court went for the first option, but the way in which they allotted shares in the owners' association deserves special mention. In particular, the court decided to take into account that some additional water entered the waterfall from smaller rivers where only a sub-group of the owners had waterfall rights. These owners' share in the association was increased accordingly, and this is surprising in light of Norwegian water law, as water rights are otherwise not tied to where the water comes from, but arises solely from the rights one has in the waterfall itself. 

The statutes of the owners' association also contains a second interesting provision, based on a suggestion made by the owners. It is a rule to the effect that all rights in the association are to be tied to the larger agricultural properties that give rise to them, and that they can not be divided from these properties and transferred to new owners separately. In Norway, such division of agricultural land would in any event require permission from the local municipality.\footnote{See Section 12 of the Land Act 1995.} In recent years, however, this protection of farming communities has grown weaker in practice, and it was the view of the owners in \emph{Vika} that a dissociation of water rights from underlying agricultural land should be forbidden altogether.

According to \cite{master}, interviews conducted with the parties demonstrated that a general consensus had developed whereby the land consolidation procedure was seen as a success. It allowed for an orderly and fair decision-making process regarding the conflicts that had arisen, and it was based on continuous interaction between the owners and the court, where everyone felt they had been given an opportunity to have his voice heard. Initially, tensions among the owners had been high, but the consolidation process had served to alleviate them. Some owners also pointed to the fact that the main hearing had been physically conducted in the local community, in a meeting hall that was familiar to the owners. This also gave them a feeling that they were meant to actively partake in the decision-making process. 

When the interviews were conducted, some 5 years after the case was concluded, the owners also appears to agree that the association was working as intended and that the climate of cooperation among the owners was good. The hydro-power scheme itself had been completed in 2008, yielding an annual production of around 15 GWh per year, providing enough energy for around 700 households. Moreover, following the experience of land consolidation, a culture of deliberation towards consensus had developed among the owners, and great emphasis had subsequently been placed on attempting to find common ground and to reach agreement on important issues. This was reflected, for instance, in the fact that the owner who contributed the land for the power station was given a generous annual fee, in addition to his compensation as a waterfall owner. According to \cite{master}, this fee exceeds what he would likely get if this decision had been left to the discretion of the consolidation court. Hence it reflected a premium that the owners were now willing to pay to ensure agreement and a continued good climate for cooperation.

All in all, we agree with \cite{master} that the case of \emph{Vika} serves as an example of how land consolidation can empower local communities and may enable them to embark on substantial development projects.

\subsection{\emph{Oma}}

The second case we will consider is the case of \emph{Oma}, which was brought before the courts in 2006.\footnote{Nord- og Midthordaland jordskifterett, case no 1200-2006-0015.} In this case there were four involved properties. The owners of three of them, $A,B$ and $C$, wanted to develop hydro-power, while the fourth, owner $D$, was opposed to the development. Rather than attempting to expropriate the necessary rights from owner $D$, owners $A,B$ and $C$ took the case to the consolidation court. They argued that development would benefit all the properties involved, and also pointed out that a more restricted project, which would not make use of owner $D$'s rights, would be less economical. Hence in their view, the consolidation court should compel $D$ to cooperate in a joint scheme. Owner $D$ protested, arguing that the project would not economically benefit him, and that it would also be to the detriment of his plans to build cottages for holiday dwellers in the same area.

The case of \emph{Oma} differs from that of \emph{Vika} in that the question of whether it was appropriate to use compulsion was more prominent. In particular, this aspect came up already in relation to the question of whether or not hydro-power development should be pursued at all. As we discussed in Section \ref{sec:3}, the fact that some owners do not desire development does not prevent the consolidation court from putting directives in place to facilitate it, but the courts often exercise restraint in such cases. In \emph{Oma}, however, the court agreed with the majority of the owners argued that an owners' association with compulsory membership should be set up. In doing so, the court relied on Section 2 c) of the Land Consolidation Act 1979. To justify the use of compulsion against $D$, the court first observed that joint development of hydro-power would benefit all the properties in question, including $D$. Then they commented specifically on owner $D$'s plans for building of cottage homes, noting first that he was unlikely to be given planning permission, and secondly that hydro-power would not in any event adversely affect such plans in any significant way. Moreover, the court noted that while owner $D$'s rights were relatively minor, they were quite crucial for the profitability of the project, particularly because owner $D$ controlled the best location for the construction of a dam to collect the water used in the scheme. Overall, the court's conclusion was that a joint hydro-power scheme would be a better option for everyone than a project that did not include owner $D$'s property.

The question then arose as to how the shares in the owners' association, and the right to rent that would go with it, should be divided among the owners and their land. In regards to this question, the court departed significantly from one of the basic principles that have been entrenched in Norwegian water-law since the early 20th Century. The principle in question states that no right to hydro-power can be derived from being in possession of land suitable for the construction of dams or other facilities necessary to exploit the waterfalls.\footnote{The principle was is reflected in Supreme Court decisions as early as \emph{Herlandsfossen} and \emph{Drammenselven}, Rt. 1922 p. 489 and Rt. 1923 p. 185 respectively, and has been maintained consistently ever since.} But the land consolidation court broke with this principle in the case of \emph{Oma}, deciding instead to set the value of the land designated for construction of a dam and a power station to represent $6 \%$ of the total value of the rights that went into the owners' association. The proportion of financial benefit and decision-making power awarded to the unwilling owner $D$ thus increased accordingly, since these right were all held by him. In fact, his share went from $1.75 \%$ to $7.75 \%$, so the consolidation process itself led to a situation where he would have a far greater incentive for supporting the development. In many ways, the decision in \emph{Oma} was more to the benefit of owner $D$ than any other among the involved parties. If the rights in question had been expropriated, for instance, he would be given next to nothing in compensation and would lose his rights forever. Instead, the solution prescribed by the consolidation court gave him a lasting and substantial interest in local hydro-power.

According to \cite{master}, interviews with the parties shows how the process and outcome of consolidation in \emph{Oma} served to create a much better climate for further cooperation among the parties. Indeed, when the interviews where conducted, 4 years after the courts' decision, owner $D$ had changed his mind and was now in favor of the development. Moreover, he had also decided that he wanted to take part in the development company. He was not obliged to do so, but his right to take part was encoded in the deal with the development company, as detailed in the statutes of the owners' association and in keeping with Section 34 b) no 3 of the Land Consolidation Act 1979.

The owners all reported that the consolidation process had been very successful and that the court had listen to them, allowing everyone to have their voices heard. Moreover, some owners reported that the court had cleverly maintained a ``birds eye view" on the best way to develop the land in question, ensuring both long terms benefit to all involved properties as well as creating an improved climate for cooperation and mutual understanding. The consensus was that making concessions to owner $D$ was appropriate and had been in the interest of all the involved parties. In 2011 the hydro-power project in \emph{Oma} was completed and today its output is roughly 3 GWh per year.

We think the case of \emph{Oma} serves as a good illustration of how consolidation can be an effective instrument for facilitating locally controlled development, also in cases when this requires the use of compulsion against some owners. Interestingly, in this case the successful outcome appears to be partly due to the fact that the consolidation court actively used its discretionary powers when deciding how to organize joint use. This power allowed them to deviate from established rights-based legal doctrine and adopt a more context-dependent approach, pursuing solutions that suited the situation better. Interesting legal questions arise in this regard, particularly regarding the competence that the consolidation court has in such cases, and the extent to which decisions can be made subject to review by the normal courts on the basis that they do not follow established principles and practices. For instance, one may ask what would have happened if the majority owners in \emph{Oma} had appealed the decision to the regular courts on the basis that $D$ was awarded too many shares in the owners' association. Would this be regarded as a question of the court's interpretation of the law regarding the owners' \emph{rights}, or would it be regarded as a discretionary decision regarding the best way to organize development? In the first case, the decision would almost certainly have been overturned on appeal, but in the latter case it would likely be beyond reproach.

A second interesting question that arises is whether or not consolidation can work as well as it did in \emph{Oma} in cases where conflicts run more deeply, or where the parties favoring development are a minority among the owners. The next two cases we consider shed some light on this issue.

\subsection{\emph{Djønno}}

This case was brought before the courts in 2006, by a local owner $A$ who wanted to develop hydro-power in a small river crossing his land, the so called \emph{Kvernhusbekken}.\footnote{Indre Hordaland Jordskifterett, case no 1230-2006-0010.} This owner wanted the court to help him implement a hydro-power project, by compelling the other owners, $B, C$ and $D$, to rent out their share in the necessary rights on terms dictated by the court. The starting point for the other owners was that they did not want any hydro-power development at all, and they were not willing to rent out their rights to owner $A$ or any other developer. There was also a dispute regarding the ownership of the waterfall rights, with $A$ believing initially that he controlled a large majority. It soon became clear that this was not the case, and before the main hearing the parties agreed that the waterfall in question was owned jointly with shares divided according to each owners' share in unconsolidated farming land. Owner $A$'s share in these rights did not amount to more than $5 \%$, so his own financial interest in hydro-power was in fact limited compared to the owners who opposed development.

On the other hand, the rights needed for the necessary physical constructions were predominantly held by owner $A$ alone, and $A$ maintained his position that the court should use compulsion to allow him to go on with his plans. The court agreed that hydro-power would be rational use of the waterfall, and they initially assessed the case against Section 2 e) regarding compulsory joint undertakings. A decision made on the basis of this provision would allow the court to give more concrete directives regarding how the hydro-power development should be carried out, but in the end the court held that this would place too much of a burden on the owners opposing hydro-power. Hence they chose to decide the case on the basis of Section 2 c), as in the other cases we have considered. By doing so they also restricted the scope of their decision to the establishment of an owners' association that would be responsible for renting out the rights. The court would not consider the question of deciding on a concrete scheme.

The model used for the owners' association was similar to the one the court adopted in \emph{Oma}. This included an adjustment of the rights in the owners' association reflecting the special importance of land needed for physical constructions. In total, these rights were estimated at a value corresponding to $6 \%$ of the shares in the association. Since these rights were all held by owner $A$ alone, his share in the association was doubled. In addition to this, owner $A$ purchased the shares from owner $B$, so that his total share ended up amounting to $22 \%$. Still, for the majority of stakeholders, membership in the association was imposed by the consolidation court against their will.

The wording of the statutes for the association apparently attempts to take into account that it would be run by a majority of unwilling shareholders. The wording used is different from that used in the other statutes, and it is stated in very clear terms that the association is going to rent out the rights in the waterfall such that hydro-power can be developed. In \emph{Oma} and \emph{Vika}, on the other hand, the statutes merely state that this is the \emph{purpose} of the association, leaving the shareholders with greater freedom to determine whether or not to go through with development.

More generally, it seems that in the case of \emph{Djønno} the court attempted to facilitate development not so much by trying to make the owners more positive towards development, but rather by giving the proponent more power, providing him with a starting point which would make it easier to later enforce concrete hydro-power plans.

In interviews, those who were compelled to take part in the association against their will expressed dissatisfaction and surprise at the result. Moreover, while the association had ostensibly tried to be loyal to the wording of the statutes, and had looked for partners who might be interested in developing hydro-power, there had been no willingness among the majority to engage actively with this work. No deals had been made, no separate development company had been set up, and the conflict among the owners was ongoing. As of 2011, owner $A$ was still pushing for development on terms that were unacceptable to the other owners. Hence while the case of \emph{Djønno} is an example that consolidation can be used even when it involves compulsion against the majority of owners, it also serves to illustrate that the chance of a successful outcome may then be more limited.

The question arises as to why this is so, and how such cases will be dealt with by courts in the future. According to owner $A$, the problem was that the directives of use were not specific enough and that they should not have been restricted to merely setting up an owners' association for renting out the rights. In this case, more was needed. The court should actively engage also with the question of how the development company should be organized, and at least give guidance as to \emph{who} should be set with the task of carrying it out. Among the majority owners, on the other hand, the feeling appears to have been that the development in question, which they would be required to partake in against their will, was more or less doomed to fail already from the start.

This reflects two interesting viewpoints regarding such cases. Indeed, it seems reasonable to assume that unless one is prepared to see an increase in the use of compulsion, compulsory cooperation will only work when at least a basic agreement that development should take place can be established among the majority of the involved owners.

\subsection{\emph{Tokheim}}

This case was brought before the consolidation court in 2008, by the owners of \emph{Tokheimselva}.\footnote{Indre Hordaland jordskifterett, Case no 1230-2008-0020.} The five involved owners all agreed that development should take place, but they disagreed about how it should be done, and about the proportion of each owners' share in waterfall. Some owners argue that development should be organized by the owners themselves, but other owners thought it would be best to rent out the rights to an external developer. The case was further complicated by the fact that the waterfall in question was so big that it would be possible to develop hydro-power that would require transferral concession pursuant to the Industrial Concession Act 1917, a concession that can only be given when the purchaser is a company where the State controls at least $\frac{2}{3}$ of the shares. 

Like the precious cases we have considered the consolidation court eventually based its decision on Section 2 c) of the Land Consolidation Act 1979, setting up an owners' association such that each owner was allotted a share in accordance to the rights that the court found he had in the waterfall. Unlike the previous cases we have considered, there was no adjustment made for land that would be needed for physical constructions. However, the statutes state that owners will be entitled to a lump sum estimated on the basis of the damages and disadvantages that a concrete hydro-power project will bring. This also marks a departure from established practice in expropriation law, where it has been a long established principle that owners can be compensated on the basis of \emph{either} the value of their waterfalls \emph{or} the damages and disadvantages caused by the project, not both.\footnote{See for instance the case of \emph{Vikfalli}, \cite{vikfalli71}.} 

In other respects, the statutes for the owners' association follow the same model adopted in the previously considered cases. They do not, however, resolve any of the controversial questions regarding how development should be carried out, and the question of the extent to which interested owners should be given the opportunity to develop the resource themselves. This was the issue that the main conflict in the case centred on, and the consolidation court explicitly decides not to implement any solutions in this regard. In particular, the statutes of the owners' association explicitly provides separate rules that cover both the case that a group of owners undertake development themselves, and the case that development is carried out by an external company. 

In interviews, the owners expressed that they were happy with how the case was dealt with by the court. Everyone appears to have been heard, and the owners' association was set up in consultation with the parties. However, the main issues were still unresolved as of 2011, and this was felt by the owners as a major shortcoming of the outcome of consolidation. Some of the owners expressed criticism against the court for not engaging more actively with what appeared to be the most pressing issues.

The case of \emph{Tokheim} serves to illustrate that established practices of consolidation, while being well received and understood by local owners, face some new challenges in relation to hydro-power, challenges that consolidation courts might be reluctant to take on. It seems that the court in \emph{Tokheim} felt that they were not in a position to assess the question of what kind of development would be best, and it also seems that they were wary of expressing any opinion about the legal status of a project led by local owners, in relation to concession law. They did not, in particular, form an opinion about whether it would be possible for local owners to carry out their own large scale development, in a waterfall that might otherwise be subject to the provisions set out in the Industrial Concession Act 1917.

It remains to be seen whether such an agnostic attitude can be maintained by the consolidation courts as local owners increasingly turn to them for help in resolving disputes regarding hydropower. Moreover, it will be interesting to see how the new Land Consolidation Act 2013 will influence case law in this area. It seems that a case like \emph{Tokheim} could benefit from the court taking a broader view, possible even including public bodies as parties in the case, as will become possible when the new Act takes effect. In this way one could perhaps have hoped for a more conclusive outcome, a solution that gave sufficient consideration both to the public interest and the interests of local owners.

\section{Assessment and Future Challenges}\label{sec:lca}

The concrete cases that I discussed in the previous section shows, in my opinion, that the system of land consolidation is well suited as an alternative to expropriation in the context of hydropower development. At the same time, the cases suggest that the land consolidation courts may find it hard to deliver effective directives of use in situations when the different stakeholders disagree fundamentally about how the water resources should be managed. In addition, one may question the effectiveness of land consolidation courts in contexts when rules and regulations from other areas of law come into play. It seems, in particular, that the land consolidation courts may be cautious about implementing solutions that they fear will raise questions in relation to the special legal provisions that regulate the form of economic development that their directives aim to facilitate.

In so far as the sector-specific rules disadvantage owners and benefit external commercial interests, as is the case for hydropower development, one may rightly fear that the land consolidation courts will become impotent in situations when powerful market players enter the scene. It may be considerably easier to strike a fair balance between the interests of local farmers of comparable economic and political standing, then to do the same when one of the stakeholders is a partly state-owned power company that is accustomed to expropriating the property rights that it desires.

Paradoxically, the impotence of the land consolidation courts may be enhanced by the fact that they are not authorized to make use of appropriate forms of compulsion against owners, on pain of interfering too much in property as an individual right. This, in particular, threatens to undermine the effectiveness of the land consolidation court as an alternative to expropriation, making it possible to argue that the public interest in development can not be sufficiently accommodated through the use of consolidation measures. 

In fact, there is some evidence to suggest that land consolidation law might offer \emph{too} much protection to owners in order to circumvent such objections. One example is the Supreme Court case of 
{\it Holen v Holen}, concerning a quarry owned by a local farmer and landowner.\footcite{holen95} In order to continue extracting his minerals, the owner of the quarry would have to interfere with the property of a neighbouring owner, who was using his land for more traditional forms of agriculture. This owner was unwilling to reach an agreement with the quarry owner, so the latter brought a case before the land consolidation court. The court noted that it would be possible to reach an accommodation that would benefit both parties, and issued directives of use that would allow the quarry to continue its operations.

The directives involved giving the agriculturally minded farmer a replacement property, to make up for his loss of the property needed to access the minerals. However, the consolidation court also noted that the quarry would, in the future, also be likely to extract minerals that belonged to this owner. For the minerals as such, awarding replacement property made little sense, so the court decided that the minerals should still belong to the previous surface owner (who had no interest in extracting them). However, a directive of use was issued that gave the quarry owner a right to extract these minerals, provided he paid market value to the owner. 

Hence, not only was the farmer awarded replacement property for agricultural purposes, he was also granted a share of the benefits that would result from the continued operation of his neighbour's quarry. This, it seems, was clearly beneficial to his property, economically speaking. The owner himself, however, objected to the arrangement, since he was opposed to the quarry as such. The Supreme Court found in his favour. Interestingly, this was not because they sanctioned his right to oppose the continued operations of the quarry, or because they thought the replacement property or the payment model was inappropriate. Instead, the Court held that the right to extract the farmer's minerals could not be transferred to someone else, even if the farmer was ensured payment. This, the Court held, was a compulsory measure that fell outside the scope of use directives in land consolidation.

The perspective underlying this decision is interesting, because it underscores a reluctance to use land consolidation in what would otherwise be a fairly typical expropriation scenario. As such, it also raises doubts about the feasibility of proposing land consolidation as a practical alternative for such scenarios. However, {\it Holen v Holen} was decided in 1995, and as I have already mentioned, the legislature has signalled a shift in the law in recent years, by explicitly facilitating the use of land consolidation as an alternative to expropriation in certain circumstances. This has been criticized, however, by scholars arguing that private property rights receive a more adequate form of protection when normal expropriation procedures are observed. In light of earlier case law, this criticism must be taken seriously, also a possible formal objection against awarding the land consolidation courts increased powers of compulsion. Today, the exact relationship between land consolidation and expropriation law, including the constitutional property clause, appears to be an increasingly relevant open question that awaits further clarification in case law. 

I would like to stress, however, that I do not agree with those who argue that land consolidation offers less protection to owners than administrative expropriation. The property protection offered in the context of land consolidation is quite different, but not necessarily weaker. This, moreover, depends on one's vision of property, and what property values one deems to be most in need of protection. An administrative expropriation procedure might offer more {\it formal} safeguards. A range of procedural rules must be observed, pertaining to notification to the owners, impact assessments, a duty to provide guidance and reasons for the decision, and a possibility (sometimes several) for administrative appeal. Then, after an expropriation order has been granted, the owner can challenge its validity before the appraisal court (which also awards compensation), in principle at the expense of the expropriating party. 

In practice, however, the administrative expropriation procedure can easily leave the owner marginalized, as they are overshadowed by other more powerful stakeholders that are not property owners. This is particularly clear in situations when expropriation arises as a result of more comprehensive planning or licensing procedures, that do not focus on the owners' interest. As discussed in Chapter \ref{chap:x}, this as is the case, for instance, in the context of hydropower development. In addition to this, the possibility of raising validity objections before the courts is mostly a theoretical one. It is very unusual for such objections to be made successfully, as the courts typically defer to the discretion of the administrative decision-maker in expropriation cases.

In the context of land consolidation, on the other hand, the interests of the owners are meant to occupy center stage throughout the proceedings. Moreover, the owners have a formal standing in a deliberative and adversarial context, presupposing their active input to a greater extent than in the context of an administrative decision. In addition, the {\it grounds} for imposing compulsory measures that interfere with property rights need to be anchored specifically in the interests of the affected properties themselves. A measure is warranted only when it benefits the properties as such, in addition to whatever broader societal benefits that might arise. Clearly, this latter principle offers substantial protection of a kind that is completely absent in the context of administrative expropriation. In these contexts, rather, the premise is that the the affected properties and their owners will suffer disadvantages and losses that they can be compelled to bear in the public interest.

Such a narrative is often unavoidable for typical public interest takings, but seem misplaced in the context of takings for profit, since these will as a matter of fact increase the value of properties that are taken. Here, the land consolidation approach seems appropriate, also in situations when interference in established property rights appear necessary to facilitate overall benefits to the community of property owners. Of course, there are challenges that must be dealt with, particularly when some property owners appear to benefit more than others, or when the notion of benefit itself is hard to pin down because property owners disagree about the most important property values inherent in their land. However, it seems to me that a procedural framework that focuses on the community of property owners as the primary stakeholder, is well suited to dealing with these challenges. Much better suited, it would seem, than an administrative decision-making process that conflates the taking for profit scenario to a run-of-the-mill expropriation scenario or an instance of spatial or sector-based planning, with expropriation as a mere side-effect.

Hence, I conclude that principled objections against land consolidation in expropraition contexts appear largely misplaced when for the sub-group of takings that realise commercial potentials. However, a second question arises, of a more practical nature. Will the land consolidation process work in practice, if it is applied to organize commercial development. Increased powers of compulsion might be required, and in keeping with my argument above, I believe such powers may well be granted, as long as land consolidation remains directed at improving the situation for existing properties and their owners, rather than bestowing benefits on someone else. A second question, which I think is far more challenging, concerns the future development of the land consolidation procedure itself. Will it remain a service to owners, placing them at the center of attention, even if its scope is broadened and other, more powerful, stakeholders enter the stage?

It is too early to say, since there has not yet, to my knowledge, been any cases where the issue has come into focus. However, as any legal person with a right to expropriate may now act as a party to a consolidation dispute, the question is bound to arise, in various forms. What will the role of the new parties be? Is the land consolidation procedure still going to be a service to owners, providing a forum for equitable interaction with potential developers, or will it become a service to developers, providing a template for cheap and easy access to property? It will be very interesting to follow this development further, to see if the promise of using land consolidation to regain legitimacy for the use of compulsion to facilitate economic development can be fulfilled.

\section{Conclusion}\label{sec:conc}

In this Chapter, I have addressed land consolidation as an alternative to expropriation for economic development, anchored in a case study of hydropwer. I started by presenting the basic idea of using land consolidation in this way, emphasising that the notion of consolidation at work here is a broad notion that includes measures seeking to enforce particular uses of property. I briefly presented a comparative vision of this kind of land consolidation, noting that a broad notion is a work in many jurisdictions. I then focused specifically on the Norwegian context, where the judicial decision-making framework for land consolidation sets the procedure apart from that found in many other jurisdictions. I also noted how the procedure is conceputalized as a service to owners, with a no-loss guarantee in place to ensure that consolidation measures are only implemented when the benfefits make up for the harms for all the involved properties individually.

I then went on to present the Norwegian system in more detail, focusing on procedural aspects, particularly those related to so-called use directives, that empower the consolidation courts to impose and organize joint use of property rights, including economic development projects. I noted how recent changes in the law envisions an extended scope for these rules, including in the context of non-agrarian and urban development. I then went on to consider some concrete examples, from the context of hydropower development, where owner-led projects already tend to rely on land consolidation rather than expropriation, to facilitate development. I concluded that while the land consolidation alterantice works well when there is basic agreement among the owners that development is desirable, it seems somewhat less effective when there is deep disagreement about how, or whether, development should proceed. 

In these contexts, I argued, it might be necessary to enhance the power of the land consolidation court, also in the direction of increasing its power to compel land owners to take part in, or allow the implementation of, development projects that they disagree with. While this is already possible, to some extent, the power of the land consolidation court in this regard appears somewhat limited, particularly in light of earlier case law that has stressed the distinction between consolidation and expropriation. However, recent legislative developments suggest that this is perspective is now changing, with an increased emphasis on land consolidation alternatives even in cases that require quite severe interferences with the interests of individual property owners.

I argued against those that see this as a threat to property, by pointing out that the formal protection awarded to owners through administrative law is hardly as practically relevant as the fact that the land consolidation process, as traditionally administered, is continuously centred on owners and property interests, making sure that external interests, particularly private interests, can not dominate the process. This, however, might be set to change now that such actors are about to receive a new formal standing in consolidation disputes, and will be granted the opportunity to bring cases before these courts themselves, if they favour it over expropriation. On the one hand, his change will enhance the power of the land consolidation court, making it more effective in dealing with cases that involve external parties. On the other hand, there is a possibility that the presence of new and powerful stakeholders will change the nature of the land consolidation process itself, so that it becomes yet another planning instrument that favour powerful developers, not a property-enhancing institution that promotes self-governance. 

I believe, however, that the land consolidation regime in Norway functions in a way that sheds interesting light on collective-action alternatives to expropriation. It is also a dynamic and versatile framework, more so than other suggestions, such as the land assembly districts proposed by Heller and Hills. In general, it seems that decision-making in a context where local interests are largely vested in property rights require special procedures, if one is to prevent the formation of a democratic deficit. It simply appears imbalanced to make use of the standard administrative planning institutions in such cases, particularly when these institutions are dominated by external, commercial, actors. This is particularly clear when the democratic grounding of these institutions is weak, or centralized, as is the case for Norwegian hydropower. In my opinion, the institution of land consolidation can provide a useful kind of democracy-on-demand for such sectors, facilitating a better balance between the interests of local community, the interests of commercial actors, and the interests of society as a whole. 
% \chapter{Sixth Chapter Title}


\section{First Section}
\subsection{First Subsection}
Here is some text. 

\subsection{Second Subsection}

\section{Conclusion}
Here is some more text. 		
% \chapter{Seventh Chapter Title}


\section{First Section}
\subsection{First Subsection}
Here is some text. 

\subsection{Second Subsection}

\section{Conclusion}
Here is some more text. 
% \chapter{Eighth Chapter Title}


\section{First Section}
\subsection{First Subsection}
Here is some text. 

\subsection{Second Subsection}

\section{Conclusion}
Here is some more text.  


%\bibliographystyle{Classes/CUEDbiblio}
%\bibliographystyle{oxford_en}
%\bibliographystyle{Classes/jmb} % bibliography style
%\renewcommand{\bibname}{References} % changes default name Bibliography to References
%\addcontentsline{toc}{chapter}{Bibliography} %adds References to contents page
%\bibliographystyle{Classes/jmb} % bibliography style


%\printindex
\nocite{*}

%If you want to input ship names, put them HERE (after \nocite, before %bibliography.tex

\chapter*{Bibliography}
\addcontentsline{toc}{chapter}{Bibliography}

% This filter is used to identify works which are either of the inbook or incollection type
\defbibfilter{inbookorincoll}{%
  \( \type{inbook} \or \type{incollection} \)}

% Define a bibheading that prints a subheading, with appropriate addition to table of contents, and sets right and left marks accordingly
\defbibheading{mysubbibintoc}{%
  \section*{#1}%
  \addcontentsline{toc}{section}{#1}%
  \markboth{BIBLIOGRAPHY -- \MakeUppercase{#1}}{BIBLIOGRAPHY -- \MakeUppercase{#1}}}

% BOOKS

\printbibliography[title={Books}, type=book, heading=mysubbibintoc, category = cited]

% WORKS IN COLLECTIONS

\printbibliography[title={Contributions to Collections}, filter=inbookorincoll, heading=mysubbibintoc, category = cited]

% ARTICLES IN JOURNALS

\printbibliography[title={Articles}, type=article, heading=mysubbibintoc, category = cited]

% ALL OTHER WORKS INCLUDING UNPUBLISHED MATERIAL

\printbibliography[title={Other Works}, nottype=book, nottype=jurisdiction, nottype=legal, nottype=legislation, nottype=article, nottype=inbook, nottype=incollection, heading=mysubbibintoc, category = cited]
. I have left one example, commented out, which should work (assuming you have the case, etc.).

\index[casesgb]{Achilleas, The@\emph{Achilleas,} The|see{Transfield Shipping Inc v Mercator Shipping Inc}}


%bibliography.tex

\chapter*{Bibliography}
\addcontentsline{toc}{chapter}{Bibliography}

% This filter is used to identify works which are either of the inbook or incollection type
\defbibfilter{inbookorincoll}{%
  \( \type{inbook} \or \type{incollection} \)}

% Define a bibheading that prints a subheading, with appropriate addition to table of contents, and sets right and left marks accordingly
\defbibheading{mysubbibintoc}{%
  \section*{#1}%
  \addcontentsline{toc}{section}{#1}%
  \markboth{BIBLIOGRAPHY -- \MakeUppercase{#1}}{BIBLIOGRAPHY -- \MakeUppercase{#1}}}

% BOOKS

\printbibliography[title={Books}, type=book, heading=mysubbibintoc, category = cited]

% WORKS IN COLLECTIONS

\printbibliography[title={Contributions to Collections}, filter=inbookorincoll, heading=mysubbibintoc, category = cited]

% ARTICLES IN JOURNALS

\printbibliography[title={Articles}, type=article, heading=mysubbibintoc, category = cited]

% ALL OTHER WORKS INCLUDING UNPUBLISHED MATERIAL

\printbibliography[title={Other Works}, nottype=book, nottype=jurisdiction, nottype=legal, nottype=legislation, nottype=article, nottype=inbook, nottype=incollection, heading=mysubbibintoc, category = cited]



\end{document}
