%input macros (i.e. write your own macros file called MacroFile1.tex)
\newcommand{\PdfPsText}[2]{
  \ifpdf
     #1
  \else
     #2
  \fi
}

\newcommand{\IncludeGraphicsH}[3]{
  \PdfPsText{\includegraphics[height=#2]{#1}}{\includegraphics[bb = #3, height=#2]{#1}}
}

\newcommand{\IncludeGraphicsW}[3]{
  \PdfPsText{\includegraphics[width=#2]{#1}}{\includegraphics[bb = #3, width=#2]{#1}}
}

\newcommand{\InsertFig}[3]{
  \begin{figure}[!htbp]
    \begin{center}
      \leavevmode
      #1
      \caption{#2}
      \label{#3}
    \end{center}
  \end{figure}
}


%%% Local Variables: 
%%% mode: latex
%%% TeX-master: "~/Documents/LaTeX/CUEDThesisPSnPDF/thesis"
%%% End: 


\includeonly{Chapter2/final_2}
%NOTE: if you want to work on just one Chapter, you can take out the `%' sign on the previous line and compile the thesis accordingly. The above command, for instance, will give you just the first Chapter. The bonus of doing it this way is that your cross references and page numbers will remain as they are in the full file. 

\documentclass[a4paper,oneside,10pt]{thesisPSnPDF}

\usepackage[utf8]{inputenc}
\usepackage{babel}
%\DeclareUnicodeCharacter{00A0}{ }

\newcommand{\isr}[1]{#1}
\newcommand{\noo}[1]{}
\newcommand{\sjur}[1]{SJUR: #1}
\newcommand{\nathp}[1]{NatHp(#1)}

\def\signed #1{{\leavevmode\unskip\nobreak\hfil\penalty50\hskip2em
  \hbox{}\nobreak\hfil(#1)%
  \parfillskip=0pt \finalhyphendemerits=0 \endgraf}}

\newsavebox\mybox
\newenvironment{aquote}[1]
  {\savebox\mybox{#1}\begin{quote}}
  {\signed{\usebox\mybox}\end{quote}}

\usepackage{titlesec}
\titleformat{\chapter}[hang]
  {\normalfont\huge\bfseries\centering}{\thechapter}{20pt}{\Huge}

\addbibresource{thesis.bib}

% turn of those nasty overfull and underfull hboxes
\hbadness=10000
\hfuzz=50pt

% Put all the style files you want in the directory StyleFiles and usepackage like this:
%\usepackage{StyleFiles/watermark}

%The following indexes are to ensure the table of cases functions properly. You can leave this to one side for now, though it is worth learning early on how to make the table of cases. It is pretty easy; but it'd be a shame if it got to near submission and you couldn't figure out how to do it. 
% NB: I haven't provided for Northern Irish cases here
\makeindex[name=casesgb, title={England and Wales}, columns=1,intoc]
\makeindex[name=casessc, title={Scotland}, columns=1,intoc]
\makeindex[name=casesus, title={The United States}, columns=1,intoc]
\makeindex[name=casesnz, title={New Zealand}, columns=1,intoc]
\makeindex[name=casesau, title={Australia}, columns=1,intoc]
\makeindex[name=casesca, title={Canada}, columns=1,intoc]
\makeindex[name=legis, title={United Kingdom}, columns=2,intoc]
\makeindex[name=casesother, title={Other Jurisdictions}, columns=1,intoc]
\DeclareIndexAssociation{gbcases}{casesgb}% ENGLAND
\DeclareIndexAssociation{sccases}{casessc}% SCOTLAND
\DeclareIndexAssociation{aucases}{casesau}% AUSTRALIA
\DeclareIndexAssociation{cacases}{casesca}% CANADA
\DeclareIndexAssociation{nzcases}{caseszn}% NEW ZEALAND
\DeclareIndexAssociation{uscases}{casesus}% UNITED STATES
\DeclareIndexAssociation{eucases}{casesother}% EU
\DeclareIndexAssociation{echrcases}{casesother}% ECHR
\DeclareIndexAssociation{pilcases}{casesother}%
\DeclareIndexAssociation{othercases}{casesother}% ANYTHING ELSE
%\DeclareIndexAssociation{gbprimleg}{legis}% LEGISLATION
%\DeclareIndexAssociation{gbsecleg}{legis}% LEGISLATION
\DeclareIndexAssociation{enprimleg}{legis}% LEGISLATION

\indexsetup{level=\section*,toclevel=section,noclearpage}

\begin{document}
\renewcommand\baselinestretch{1.5}
\baselineskip=24pt


%\maketitle

\begin{titlepage}

\begin{center}



\vspace*{\fill}
\centering

{\Huge\textsc{On the legitimacy of economic development takings}}\\[3cm]

\large {Thesis submitted to the School of Law at Durham University for the degree of Doctor of Philosophy}\\

by

{Sjur K. Dyrkolbotn}\\

%\emph{{Your College}}\\
\vspace*{\fill}

 

\vfill

{\Large Autumn 2014}\\
{c. 90 000 Words}

\end{center}

\end{titlepage}


%set the number of sectioning levels that get number and appear in the contents
\setcounter{secnumdepth}{4}
\setcounter{tocdepth}{2}

\frontmatter
%
\begin{center}
\vspace{4cm}
I hereby certify that this thesis is the result of my own work except where otherwise indicated and due acknowledgement is given.
\vspace{1cm}

I also certify that this thesis is XXXXX words long excluding the bibliography.\\

\vspace{4cm}


\begin{tabular}{lr}
& DATE OF SUBMISSION \\
& \\
SIGNED & DATE \\
\end{tabular}


\end{center}


% ----------------------------------------------------------------------


%%% Local Variables: 
%%% mode: latex
%%% TeX-master: "../thesis"
%%% End: 

%% Thesis Abstract -----------------------------------------------------

% NOTE: As with acknowledgements, I had to create a new format for this -- I couldn't get the original one to work. As with the acknowledgements, if you are able to fix the code so it's less messy, do pass the fix back to the Law Faculty.
\cleardoublepage
\addcontentsline{toc}{chapter}{Abstract}
%\begin{abstractslong}    %uncommenting this line, gives a different abstract heading
%\begin{abstracts}        %this creates the heading for the abstract page

\begin{quoting}
  \singlespace
    \begin{center}
  {\LARGE \bfseries  On the Legitimacy of Economic Development Takings }\\
  \vspace*{0.5cm}
      {\large Sjur Kristoffer Dyrkolbotn}\\
  %\vspace*{0.1cm}  
   %   {\large \emph{Ustinov College}}\\
  \vspace*{0.2cm}  
    {\normalsize Thesis submitted to Durham Law School at Durham University for the degree of Doctor of Philosophy}

  \vspace*{0.2cm}  
    {\normalsize \today}\\
  \vspace*{0.5cm}  
    {\normalsize \bfseries Abstract}      
  \end{center}
  {\parindent0pt
For most governments, facilitating economic growth is a top priority. Sometimes, in their pursuit of this objective, governments interfere with private property. Often, they do so by indirect means, for instance through their power to regulate permitted land uses or by adjusting the tax code. However, many governments are also prepared to use their power of eminent domain in the pursuit of economic development. That is, they sometimes compel private owners to give up their property to make way for a new owner that is expected to put the property to a more economically profitable use. %This new owner is sometimes the government itself, represented by one of its administrative bodies. But in many cases it will be a private company, operating for profit, possibly in cooperation with government entities through some form of public-private partnership.
}
\vspace{0.7mm}

This thesis asks how the law should respond to government actions of this kind, often referred to as {\it economic development takings}. The thesis makes two main contributions in this regard. First, in Part I, it proposes a theoretical foundation for reasoning about the legitimacy of economic development takings, including an assessment of possible standards for judicial review. Moreover, the thesis links the legitimacy question to the work done by Elinor Ostrom and others on sustainable management of common pool resources. Specifically, it is argued that using institutions for local self-governance to manage development potentials as common pool resources can potentially undercut arguments in favour of using eminent domain for economic development.

Then, in Part II, the thesis puts the theory to the test by considering takings of property for hydropower development in Norway. It is argued that current eminent domain practices appear illegitimate, according to the normative theory developed in Part I. At the same time, the Norwegian system of land consolidation offers an alternative to eminent domain that is already being used extensively to facilitate community-led hydropower projects. The thesis investigates this as an example of how to design self-governance arrangements to increase the democratic legitimacy of decision-making regarding property and economic development.

%This shows how local governance arrangements can work, suggesting that more attention should be devoted to studying the nexus between property, common pool resource management, and eminent domain.

%This theoretical basis is formulated independently of specific jurisdictions, but based on considering existing approaches to the legitimacy question from England and Wales, the United States, and at the European Court of Human Rights. In addition, the thesis draws a link between the legitimacy question and the work done by Elinor Ostrom and others on sustainable management of common pool resources. Specifically, it is argued that institutions for local self-governance that treat development potentials as common pool resources can often undercut arguments in favour of using eminent domain for economic development.

%Such rules are in place in most developed countries, and the fundamental status of property has been expressed explicitly in both the US constitution and the European Convention of Human Rights. The tension between these provisions and the practice of taking property for economic development, in many cases for commercial profit, is clear and worth considering further.

%\vspace{0.7mm}

%The second part of the thesis puts the theory developed in the first part to the test by considering takings for hydropower development in Norway. Under Norwegian law, the right to exploit the hydropower in most streams and rivers belong to the riparian owners. That is, the right to the hydropower belongs to the people who own the land over which the water flows, usually local community members. To acquire these rights, energy companies tend to rely on the government's power of eminent domain. Recently, however, local communities have begun to protest this practice, by arguing that they should be allowed to take a more active role in managing their own resources. This has resulted in tensions in Norway, shedding light on the legitimacy question as it arises in the context of Norwegian expropriation law. In addition, new light has been shed on the role of the so-called land consolidation courts, which are now increasingly asked to deliver alternatives to eminent domain in hydropower cases. The thesis investigates this in depth and argues that the unique system of land consolidation found in Norway demonstrates how to design self-governance arrangements that can increase the democratic legitimacy of decision-making regarding property and economic development.

\end{quoting}


%\end{abstracts}
%\end{abstractslong}


% ----------------------------------------------------------------------


%%% Local Variables: 
%%% mode: latex
%%% TeX-master: "../thesis"
%%% End: 

%% Thesis Acknowledgements ------------------------------------------------

\cleardoublepage
%\begin{acknowledgementslong} %uncommenting this line, gives a different acknowledgements heading
%\begin{acknowledgements}      %this creates the heading for the acknowlegments
\addcontentsline{toc}{chapter}{Acknowledgements}
\begin{quoting}
  \singlespace
    \begin{center}
  {\LARGE \bfseries  Acknowledgements}\\
  \vspace*{0.5cm}
  \end{center}
\noindent

My supervisor, Professor Tom Allen, has been a great support and inspiration ever since we first corresponded about the possibility of me doing a PhD in Durham, in the spring of 2012. His style as a supervisor has been superb: calm and unperturbed, yet always sharp and focused, readily available to offer insightful comments and valuable guidance. Thank you, Tom.

Second, I would like to thank Durham Law School for offering me a place at their department and for treating me well while I have been there. Thanks also to Professor Leigh and Professor Masterman for taking an interest and giving me valuable comments following my first year review.

Third, I would like to thank Professor Jacques Sluysmans, Professor Hanri Mostert, and Professor Leon Verstappen, for welcoming me to their regular colloquia on expropriation law. Attending and speaking at these meetings has been a very valuable experience for me, allowing me to learn from expropriation lawyers and scholars from many different jurisdictions. A special thanks to Professor Sluysmans, Dr Emma Waring and Dr Stijn Verbijst for inviting me to contribute a chapter on Norway in their book on expropriation in Europe. Another special thanks to Dr Waring for sending me a copy of her doctoral thesis on private takings; it has been a great help and inspiration for my own work. Also a special thanks to Dr Ernst Marais and Bj\"{o}rn Hoops for organising an excellent conference in Rome and being very helpful and welcoming to new members of the expropriation research community. Hopefully, this community will stay together and continue to prosper.

Fourth, I would like to thank Ustinov College for welcoming me as a student in Durham and providing a relaxed and friendly atmosphere during my first year as a PhD student. Thanks also to the friends I met there, including Julia, Alan, Noel, Meghan, Lloyd, Peter, and Alma. A special thanks to Isabel Richardson, for being both a highly valued friend and an excellent proofreader.

Fifth, I would like to thank family, friends and colleagues in Norway, especially Ragnhild, Truls and Piotr (who is just on holiday in Germany, I am sure). A special thanks to my father and my brother, for motivation, guidance, and support. A special thanks also to my mother and my sisters, for kindness and inspiration. Thanks to Einar Sofienlund for sharing his insight and providing invaluable information about small-scale hydropower. Thanks also to Johan Fr Remmen, for making clear why this thesis should be written. Hopefully, I have made a good start towards doing justice to the subject.

Lastly, I would like to thank Marijn Visscher. No doubt, coming to Durham was the best decision I ever made.

\end{quoting}


%\end{acknowledgements}
%\end{acknowledgmentslong}

% ------------------------------------------------------------------------

%%% Local Variables: 
%%% mode: latex
%%% TeX-master: "../thesis"
%%% End: 


% A note on the format. I could not get the acknowledgements macro to work properly, despite some effort, and so I redesigned it rather messily, as you will see above. If you enter text, it will work -- but the more technologically competent among you will probably be able to fix it. If and when that is done, if you could send the resultant document back to the Law Faculty to update it, that would be great. 
\tableofcontents


% NOTE:
% To generate the indexes properly you need to run the following commands (in a terminal shell -- you need to navigate to the Thesis file in terminal. There will be guidance online for how to navigate within Terminal. Otherwise, most scientists should be able to help you!):
% splitindex -- thesis -s oscola (THIS IS THE COMMAND WHICH WORKS FOR ME)
% splitindex thesis --s oscola thesis (TRY THIS IF THE FIRST ONE DOESN'T WORK)

%% this section includes various indexes/tables of cases and legislation

\chapter*{Cases Cited}
\addcontentsline{toc}{chapter}{Cases Cited}
\markboth{CASES CITED}{CASES CITED}

\printindexearly[casesgb]% ENGLAND & WALES
\printindexearly[casessc]% SCOTLAND (GB too, of course, but ...)
\printindexearly[casesau]% AUSTRALIA
\printindexearly[casesnz]% NEW ZEALAND
\printindexearly[casesca]% CANADA
\printindexearly[casesus]% UNITED STATES
\printindexearly[casesother]% OTHERS

\chapter*{Legislation Cited}
\addcontentsline{toc}{chapter}{Legislation Cited}
\markboth{LEGISLATION CITED}{LEGISLATION CITED}

\printindexearly[legis]% ALL LEGISLATION
%\addcontentsline{toc}{chapter}{Acknowledgements}


%\addcontentsline{toc}{chapter}{Abstract}
%\listoffigures

%\include{LawTable/LawTable}
%\listoftables
%\addcontentsline{toc}{chapter}{List of Tables}
%\printglossary  %% Print the nomenclature
%\addcontentsline{toc}{chapter}{Nomenclature}

\mainmatter
%\part{Towards a Theory of Takings for Commercial Gain}

\noo{The gulf between the conscious and the unconscious must be embraced by explanation models. It is the distinguishing human trait, after all, that we grant ourselves ... The unconscious is only looked {\it at} by the conscious. No particular theoretical unit of the sub-conscious change predictably when you think about yourself (or, as some would say, ``form an intention''). What changes is {\it you}. The predictability of the outcome (or otherwise, in people who are unwell, for instance) is largely attributable to the workings of the sub-conscious, i.e., must await explanation in behavioural (physical/chemical) terms. The difference between a person and a dog is that by looking at themselves and the world in a (mental) language, persons are capable -- through a purely behavioural/physical mechanism -- of modifying their instincts. Thought prevails over instinct (sometimes), but the power of thought itself is purely instinctual... The power of language is in the world, not in language. Thank God!}

\chapter{Property, protection and privilege}\label{chap:1}

\begin{quote}
It's nice to own land.\footnote{Donald Trump}
\end{quote}

\section{Introduction}

In this chapter, I provide a bird's eye view on my topic, by placing it in the theoretical landscape. My aim is to explain the key concepts that I will rely on to make sense of the empirical data considered in subsequent chapters. I will also present the main values that I will look to when I give normative assessments. In addition, I will relate my theoretical approach to current strands in property theory, focusing on those aspects of property theorizing that I regard as particularly relevant to the work done in this thesis.

I will strive to show that my approach to the empirical data is sound and informative, while focusing on principles of {\it legal} reasoning. I will not provide an extensive presentation of concepts or theoretical approaches developed in other fields, such as political science, sociology, economy, or psychology. However, I note that all these fields engage in interesting ways with the notion of property, and I think a multi-disciplinary approach can be illuminating.\footnote{For some examples of relevant work from economics, psychology and political science respectively, consider \cite{miceli11,nadler08,katz97,carruthers04}.} Hence, while I focus on legal and --  to some extent -- philosophical theories of property, I will try to make a note of specific questions I consider that are also analysed in related fields.

The crucial argument made in this chapter is that the category of {\it economic development takings} is relevant to legal reasoning about certain kinds of situations when private property is taken by the state. This is not {\it prima facie} clear. In fact, I am prepared to face critics who will argue that the category makes no legal sense at all. Fortunately, it makes perfect intuitive sense; it targets situations when property is, quite literally, taken for economic development. In most cases I will consider, this is even the explicitly stated aim used to justify eminent domain. Hence, the factual basis for our categorization can not be questioned.

The theoretical basis, on the other hand, can not be taken for granted. Indeed, a superficial look at dominant legal approaches to property would seem to indicate that in most property regimes, the nature of the project benefiting from a taking should not be in focus when assessing the legitimacy of interference. Rules aiming to protect property invariably focus on the rights of the affected owner, making clear that she enjoys some degree of protection against uncompensated state interference. But how can we, on this basis, justify having regard to the {\it purpose} of the taking? What bearing does this have for the question of legitimacy with respect to the owner's rights? At first sight, it might seem unwarranted to think that it should matter at all. Are not owners' rights  equally interfered with when property is taken for some uncontroversially public project, like a new public road, compared to the situation when it is taken for economic development? Is it not in fact a little small-minded, even short-sighted, to worry about the taker's gain, instead of concentrating on what the owner, if anything, stands to lose?

Much of the work in this chapter, albeit theoretical, is aimed at countering this very concrete objection. I believe it is important to do so thoroughly, since it is an objection that threatens to undermine the conceptual basis for the kind of study that I present in this thesis. Moreover, it is an objection that I think it is inappropriate to dismiss without further comment. In the context of US law, it might be possible to do so, since economic development takings have, as a matter of fact, gained recognition as an important category of legal reasoning.\footnote{See generally \cite{cohen06,somin07,malloy08}.}  In Europe, however, this has not yet happened, at least not to the same extent.

The reason for this difference is not that US law contains special rules that emphasize the importance of the distinguishing features of economic development takings.\footnote{In fact, many state laws now {\it do} contain such rules, following the backlash of the controversial decision in \cite{kelo05}. However, such rules were introduced only after the category of economic development takings first came to prominence in legal discourse. See generally \cite{eagle08,somin09,jacobs11}.} Rather, the difference must largely be attributed to the fact that economic development takings have resulted in great political controversy in the US, a controversy that has influenced both the law and legal scholars.\footnote{See, e.g., \cite[1190-1192]{somin08}.} Hence, in the absence of a similar political climate in Europe, a conceptual investigation into the very idea of an economic development taking is warranted, if not also required.

As I argue in this chapter, providing an adequate account of such takings forces us to broaden our theoretical outlook compared to most traditional strands of legal reasoning about property. However, I find considerable support for the necessary conceptual reconfiguration when I consider recent trends in property theory, particularly those that focus on the {\it social function} of property.\footnote{See generally \cite{alexander09a,foster11,singer00,underkuffler03,alexander06,alexander10,dagan11}.} Indeed, the crux of the main argument presented in this chapter is that this function allows us, even compels us, to pay attention to the special dynamics of power that tend to manifest in cases when private property is taken by the state for someone else's profit.

Some might argue that the most straightforward way of describing economic development takings is to say simply that they are {\it unfair}. Indeed, this has some merit also as a conceptual position, since it would seem to explain quite effectively why takings for profit have become so unpopular in the US, particularly when people's homes are taken.\footnote{See, e.g., \cite[742-748]{nadler08}.}  However, a more thoughtful assessment reveals that matters are not quite so simple. Indeed, it seems that economic development takings are an almost unavoidable consequence of any system that emphasize both state control over property and public-private partnerships in the economic sector. To condemn this political model of government is a possible response, but not one I will pursue in this thesis. Rather, I will focus on getting to the heart of what characterizes economic development takings, so that I may address also the question of how to deal with them, to ensure that their positive functions can be fulfilled in ways less offending.\footnote{Even those who support an outright ban on economic development takings should be interested in demarcating the category more closely, to arrive at a better understanding of what exactly will be banned.}

Therefore, the stark contrast between the intuitive response that a taking for profit is unfair, and the legal assessment that what matters is only the loss to the owner, needs to be considered further. A tentative first reconciliation can be achieved by arguing that the feeling of unfairness is in itself a loss that the owner incurs, so that the law had better take it into account. But, of course, not any subjective feeling should be given weight in a legal context. The question, therefore, is what exactly the feeling of unfairness can help us uncover about the nature of economic development takings. Does it uncover something legally significant?

In my view, the answer is yes, and in the following I will argue for this position in depth. To motivate this largely abstract analysis, I will begin by considering a concrete scenario which illustrates the need for contextual assessment and more fine-grained conceptual categories for reasoning about cases when demands for economic development come to pose a threat against established patterns of property.

\section{Donald Trump in Scotland}\label{sec:dts}

On the 10th of July 2010, the property magnate Donald Trump opened his first golf-course in Scotland, proudly announcing that it would be the ``best golf-course in the world''.\footnote{http://www.golf.com/courses-and-travel/donald-trump-scotland-golf-course-lives-hype (accessed 06 July 2014).} Impressed with the unspoilt and dramatic seaside landscape of Scotland's east coast, the New Yorker, who made his fortune as a real estate entrepreneur, had decided he wanted to develop a golf course in the village of Balmedie, close to Aberdeen.

To realize his plans, Trump purchased the Menie estate in 2006, with the intention of turning it into a large resort with a five-star hotel, 950 timeshare flats, and two 18-hole golf-courses. The local authorities were not particularly keen on the idea, and planning permission was initially denied by Aberdeenshire Council. Particularly worrying to the councilors was the fact that the proposed site for the development was declared to be of special scientific interest under EU conservation legislation. The frailty and richness of the sand dune ecosystem, in particular, suggested that the land should be left unspoilt for future generations.\footnote{See \url{http://en.wikipedia.org/wiki/Donald_Trump#Scottish_golf_course} (accessed 06 July 2014).} Trump was not deterred, however, and started lobbying Scottish politicians to gain support. In the end, he was able to convince Scottish ministers that he should be given the go-ahead on the prospect of boosting the economy by creating some 6000 new jobs.\footnote{See \url{http://www.theguardian.com/world/2008/nov/04/donald-trump-scottish-golf-course} (accessed 06 July 2014).}

Activists continued to fight the development, launching the ``Tripping up Trump'' campaign to back up local residents who refused to sell their properties.\footnote{See \url{http://www.trippinguptrump.co.uk/} (accessed 03 August 2014).} One of these, the farmer and quarry worker Michael Forbes, expressed his opposition in particularly clear terms, declaring at one point that Trump could ``shove his money up his arse''.\footnote{See \url{http://www.scotsman.com/news/donald-trump-s-plea-to-homeowners-on-the-menie-estate-1-1370270}. (accessed 03 August 2014)} Trump, on his part, had described Forbes as a ``village idiot'' that lived in a ``slum''.\footnote{See \url{http://www.bbc.co.uk/news/10205781} (accessed 08 July 2014).} Moreover, he had suggested that Forbes was keeping his property in a state of disrepair to pressure up the price of the property.\footnote{See \url{http://edition.cnn.com/2007/WORLD/europe/10/10/trump.golf/} (accessed 03 August 2014).} Forbes was offended. He proudly declared that he would never consider selling, as the issue had become personal.\footnote{See \url{http://www.scotsman.com/news/scotland/top-stories/farmer-who-took-on-trump-triumphs-in-spirit-awards-1-2668649} (accessed 03 August 2014).}

At the height of the tensions, Trump considered his legal options, by asking the local council to consider issuing compulsory purchase orders (CPOs) that would allow him to take property from Forbes and other recalcitrant locals against their will.\footnote{See \url{http://www.thesundaytimes.co.uk/sto/news/uk_news/article184090.ece} (accessed 03 August 2014).} If carried out, this would have been an iconic example of an economic development taking. Moreover, it would not be the first time that the power of eminent domain had been used to the benefit of Donal Trump's business empire. In the 1990s, Trump famously succeeded in convincing Atlantic City to allow him to take the home of one Vera Coking, to facilitate further development of his casino facilities.\footcite[297-301]{jones00} This taking was eventually struck down by the New Jersey Superior Court, however, a result that was hailed as a milestone in the fight against ``eminent domain abuse'' in the US.\footnote{See \url{https://www.ij.org/cases/privateproperty} (accessed 12 August 2014). The case also caused a surge of attention directed at the issue, see \url{http://reason.com/archives/2008/03/03/litigating-for-liberty/4} (accessed 12 August 2014). For the decision itself, consult \cite{banin98}.}

In Scotland, Trump's plans were met with widespread outrage. The media coverage was wide, mostly negative, and an award-winning documentary was made which painted Trump's activities in Balmedie in a highly negative light.\footnote{See \url{http://www.youvebeentrumped.com}.} The controversy also made its way into UK property scholarship. Kevin Gray, in particular, a leading expert in property law, expressed his opposition by making clear that he thought the proposed taking would be an act of ``predation''.\footcite{gray11}

In fact, the case prompted Gray to formulate a number of key features that could be used to identify situations where compulsory purchase would be more likely to represent an abuse of power. He noted, in particular, that Trump's proposed takings would fall in line with a general tendency in the UK towards using compulsory purchase to benefit private enterprise, even in the absence of a clear and direct benefit to the public. Indeed, it was not unrealistic to think that CPOs might come to be used in Balmedie; if he had put his weight behind it, Trump might well have been able to make a successful case that existing statutory authorities could be used to justify takings of this kind.\footnote{In particular, the \cite{tcpsa97} contains a wide authority in s. 189, stating that local authorities has a general power to acquire land compulsorily in order to ``secure the carrying out of development, redevelopment or improvement''.} It would not be hard to argue that the public would benefit indirectly in terms of job-creation and increased tax revenues. Moreover, Scottish ministers had already gone far in expressing their support for the plans.

But then, in a surprise move, Trump announced he would not seek CPOs after all.\footnote{See \url{http://news.stv.tv/north/224662-relief-for-residents-trump-lifts-threat-of-compulsory-purchase-orders/} (accessed 03 August 2014).} Quite possibly, he was discouraged by the negative press. But in addition, he had found another strategy, namely that of containment: He erected large fences, planted trees and created artificial sand dunes, all serving to prevent the properties he did not control from becoming a nuisance to his golfing guests. One local owner, Susan Monroe, was fenced in by a wall of sand some 8 meters high. ``I used to be able to see all the way to the other side of Aberdeen'', she said, ``but now I just look into that mound of sand''.\footnote{See \url{http://www.theguardian.com/world/2012/jul/10/donald-trump-100m-golf-course} (accessed 03 August 2014).} She also lamented the lack of support from the Scottish government, expressing surprise that nothing could be done to stop Trump.

But there was little left to do. As soon as Trump decided to build around them, the neighboring property owners found themselves completely marginalized. After all, Trump had the backing of the government, having been declared as a job-creator whose activities would boost the economy in the region. He had even received an honorary doctorate at the Robert Gordon University, a move that prompted the previous vice-chancellor, Dr David Kennedy, to hand his own honorific back in protest.\footnote{See \url{http://www.bbc.co.uk/news/uk-scotland-north-east-orkney-shetland-11421376}.}

But in the end, it was not by taking the land of others that Trump triumphed in Scotland. Rather, he succeeded by exercising ``despotic dominion' over his own.\footnote{To quote William Blackstone, \cite[2]{blackstone79b}.} This proved highly effective;  after he fenced them in, his neighbors were hard to see and hard to hear. The Balmedie controversy went quiet, the golfers came, Trump got his way. As he declared during the grand opening: ``Nothing will ever be built around this course because I own all the land around it. [...] It's nice to own land.''\footnote{See \url{http://www.theguardian.com/world/2012/jul/10/donald-trump-100m-golf-course} (accessed 06 July 2014).}

\subsubsection*{\ldots}

The tale of Trump coming to Scotland not only serves to illustrate the kind of scenario that I will be looking at in this thesis, it also puts the work into perspective. It shows, in particular, that what it means to protect property against undue interference can depend highly on the circumstances. For a while, it looked like Balmedie was about to become a canonical case of an economic development taking. But in the end, it became rather an illustration of something far more subtle, namely that the meaning of protecting property rights depends highly on context, our own perspective, and the values we aim to promote. 

Moreover, we are reminded that the function of property as such is deeply shaped by social, political and economic structures. It seems clear, in particular, that Donald Trump's ownership of the Menie estate has a vastly different meaning than does Michael Forbes' ownership of his small farm. To many observers, the former kind of ownership will represent some combination of power, privilege and profit, while the latter will be regarded as coming imbued with a mix of defiance, community and sustenance. Very different values are inherent in these two forms of ownership, and after Trump came to Balmedie, they clashed in a way that required the legal order to prioritize between them.

According to Trump and his supporters, protecting property rights against interference in Balmedie no doubt involves protecting the governmentally sanctioned golf resort plans from backwards locals who attempt to fight progress. In this narrative, ``protection'' can maybe even be used justify compulsory acquisition of property rights that are regarded as a hindrance to the full enjoyment of property for more resourceful members of the community. But for Michael Forbes and the other local owners, protecting property rights is likely to have a completely different meaning. To them, protecting property means above all else to protect a local community against what they see as a disruptive and damaging plan that will see both them and their properties turned into golfing props. Again, adequate protection might require an interference in property, to prevent Trump from using his land in a way that causes damage to his neighbors. Regardless of who we support, we are forced to recognize that protection implies interference and vice versa. 

This shows the conceptual inadequacy of a simplistic perspective whereby protecting property rights is seen as a black-and-white proposition, a call for limits on the state's power to do good, enforced to protect owners' right to do as they please. In reality, the situation is  more subtle, involving a number of additional dimensions. Importantly, how we assess concrete situations where property is under threat depends crucially on what we perceive as the ``normal'' state of property, the alignment of rights and responsibilities that we deem to be worthy of protection. Our stance in this regard clearly depends on our values. But values themselves are in turn influenced by the context of assessment within which they arise. More problematically still, they may be influenced by our \emph{perception} of this context, rather than by reality.

For example, property activists in the US tend to regard the value of autonomy as a fundamental aspect of property. But this must be understood in light of the idea that US society is founded on an egalitarian distribution of property, where ownership is meant to empower ordinary people by facilitating self-sufficiency and self-governance.\footnote{See, e.g., \cite[173]{ely07}.} Hence, the autonomy inherent in property ownership is not thought of as being bestowed on the few, but on the many. Protecting autonomy of owners against state interference is not about protecting the privileges of the rich and powerful, but is embraced as a way to protect {\it against} abuse by the privileged classes.\footnote{This narrative is enthusiastically embraced by US activists who fight economic development takings, see, e.g., \url{http://www.castlecoalition.org/}.} 

This, however, is only an {\it idea} of property protection. It might not correspond to the reality surrounding the rules that have been molded in its image. Indeed, it has been noted that despite the great pathos of the egalitarian property idea, egalitarianism has actually played a marginal role to the development of US property law.\footnote{\cite[361]{williams98} (``Why does the egalitarian strain of republicanism have such a substantial presence in American property rhetoric outside the law but so little influence within it?'')} More worryingly still, research indicates that land ownership in the US, while hard to track due to the idiosyncrasies of the land registration system, is not actually all that egalitarian.\footcite[246-247]{jacobs98} In this way, we are confronted with the danger of a disconnect between  values, reality and the law.

In Scotland, a similar story unfolds. Here, the traditional concern is that land rights are mainly held by the elites.\footnote{See generally \cite{wightman96,wightman13}.} As a result, Scottish property activists tend to focus on values such as equality and fairness, calling also on the state to regulate and implement measures to achieve more egalitarian control over the land. Indeed, reforms have been passed that sanction interference in established property rights on behalf of local communities.\footnote{See generally \cite{lovett11,hoffman13}.} At the same time, cases like Balmedie illustrate that the Scottish government, now with enhanced powers of land administration, may well choose to align themselves with the large landowners. Moreover, research indicates that recent reforms in Scottish planning law, which serve to enhance the power of the central government, has the effect of undermining local communities and their capacity for self-governance.\footnote{See generally \cite{pacione13,pacione14}.} Again, the danger of a disconnect between influential property narratives and reality is brought into focus.

On the other hand, it seems that grass root property activists in the US and Scotland may well be closer in spirit than they seem. Upon closer examination I cannot help thinking that they share many of the same concerns and aspirations, and that the differences arise mainly from the fact that they operate in different contexts and engage with different discourses of property. The challenge is to find categories of understanding that allow us to make sense of their shared spirit, as well as the spirit they oppose.

I think the example of Balmedie suggests a possible first step, by illustrating the need for a framework that will allow commentators to  deny that there is any inconsistency between opposing compulsory purchase orders while also supporting strict property regulation in the context of fighting a golf resort. Both of these positions, moreover, should be viewed as efforts to protect property. To the classical ``individual rights v state interference'' debate, such a dual position can be hard to make sense of. But in my opinion, this only points to the vacuity of such a conventional narrative.

In general, I think it is hard, close to impossible, to make sense of many contemporary disputes over property if we do not have the conceptual acumen to distinguish between egalitarian property held under a stewardship obligation by members of a local community, and feudal property held by businesses for investment. Moreover, there is no contradiction between promoting the value of autonomy for one of these, while emphasizing state control and redistribution for the other. The broader theme is the contextual nature of property, and its implications for protection of property rights. In the coming sections, I will locate a theoretical basis that will allow me to take advantage of this viewpoint in my legal analysis.

\section{Theories of property}\label{sec:top}

What is property? In common law jurisdictions, the standard answer is that property is a collection of individual rights, or more abstractly, {\it entitlements}.\footnote{The term ``entitlement'' was used to great effect in the seminal article \cite{calabresi72}.} Being an owner, it is often said, amounts to being entitled to one or more among a bundle of ``sticks'', streams of protected benefits associated with, or even serving to define, the property in question.\footcite[357-358]{merrill01} This point of view was first developed by legal realists in response to the natural law tradition, which conceptualized property in terms of the owner's dominion over the owned thing, particularly his right to exclude others from accessing it.\footcite[193-195]{klein11} In civil law jurisdictions, rooted in Roman law, a dominion perspective is still often taken as the theoretical foundation of property, although it is of course recognized that the owner's dominion is never absolute in practice.\footnote{For a comparison between civil and common law understanding of property, see generally \cite{chang12}.}

In modern society, the extent to which an owner may freely enjoy his property is highly sensitive to government's willingness to protect, as well as its desire to regulate. To civil law theorists, this sensitivity has been thought of as giving rise to various restrictions in property rights, but for common law theorists, overlooking a legal system with roots in a relatively stable feudal tradition, it has been thought of as {\it constitutive} of property itself.\footcite[7]{chang12} Indeed, the bundle of rights theory has long historical roots in common law. Arguably, it was distilled from the traditional estates system for real property, which was turned into a theoretical foundation for thinking about property in the abstract.\footnote{See \cite[7]{chang12}   
(``The ``bundle of rights'' is in a sense the theory implicit in the common law system taken to its extreme, with its inherently analytical tendency, in contrast to the dogged holism of the civil law.'').} 

However, during the 18th and 19th century, natural law thinking was also highly influential in common law. This is evidenced, for instance, by the works of William Blackstone and James Kent.\footnote{See generally \cite{blackstone79b,kent27}.} But towards the end of the 19th century, it became increasingly hard to reconcile such an approach to property with the reality of increasing state regulation. Hence, the bundle metaphor that gained prominence in the early 1900s can be seen to indicate a return to a more modest perspective.\footcite[195]{klein11}

Property rights under the bundle theory are thought to be directed primarily towards other people, not things.\footnote{See \cite[357-358]{merrill01} (``By and large, this view has become conventional wisdom among legal scholars: Property is a composite of legal relations that holds between persons and only secondarily or incidentally involves a ``thing''.'').} This underscores a second point about property in the real world, namely that the content of rights in property are necessarily relative to the totality of the legal order. For instance, relying on a bundle metaphor, it becomes perfectly natural that a farmer's property rights protects him against trespassing tourists, but not against the neighbor who has an established right of way. 

By contrast, the dominion theory suggests viewing such situations as exceptions to the general rule of ownership, which implies a right to exclusion at its core. In the case of property, exceptions no doubt make up the norm. But in civil law jurisdictions one lives happily with this. It takes the grandeur away from the dominion concept, but it retains a nice and simple structure to property law. In the civil law world, it is common to say that what the owner holds is the {\it remainder} after all positive rights, serving to restrict his dominion, have been deducted.\footcite[25]{chang12} Moreover, while there may be many limitations and additional benefits attached to property, they are all in principle carved out of one initial right, namely that of the owner. In this way, the system becomes more easily navigable.

An interested party may ask, ``who owns this land?'' Then, under the dominion theory, a clear answer is expected and will usually be adequate, even if it does not give a complete picture of all relevant property rights. Under the bundle theory, on the other hand, one might be inclined to respond, ``to which stick are you referring?'' Clearly, this narrative is more complex, perhaps unduly so. 

Some common law scholars have recently elaborated on this to develop a critique of the bundle theory, by suggesting that it should at least be complemented by a firm theory of {\it in rem} rights in property. This, they argue, would allow the law to operate more effectively, by relying on a simple and clear rule that, although defeasible, will generally suffice to inform people about their relevant rights and duties in relation to property.\footnote{\footcite[793]{merrill01b} (``The unique advantage of in rem rights -- the strategy of exclusion -- is that they conserve on information costs relative to in personam rights in situations where the number of potential claimants to resources is large, and the resource in question can be defined at relatively low cost.''); \footcite[389]{merrill01} (``The right to exclude allows the owner to control, plan, and invest, and permits this to happen with a minimum of information costs to others.''). See also \cite{ellickson11} (arguing that Merrill and Smith's analysis nicely complements and improves upon the bundle theory).} 

There are also other, less pragmatic, reasons why a dominion approach might be preferable, even if the bundle metaphor is arguably more accurate. In particular, some scholars point out that the bundle theory does not adequately reflect the sense in which property is a right to a {\it thing}, serving to create an attachment that is not easily reducible to a set of interpersonal legal relationships.\footnote{\cite[1862]{merrill07}. For a slightly different take on attachment, highlighting how the thingness of property marks its conditional nature and transferability, see \cite[799-818]{penner96}.} In the US, where the bundle theory has traditionally been dominant, critique like this seems to be gaining ground.\footnote{See generally \cite{foster10}.}

But in this thesis, the efficiency and clarity of different property concepts will not be a primary concern, nor will personal attachments to things in themselves play a particularly important role.\footnote{I mention, however, that the personhood-aspects of property that are sometimes highlighted in this regard will also be relevant to my analysis of economic development takings. However, this is not something that I think warrants extensive engagement with the bundle v dominion debate. I note, for instance, that in the work of Margaret Jane Radin, one of the main proponents of persoonhood accounts, the bundle theory is not challenged as much as it is readjusted, although in places it also seems to be the object of some implicit criticism, see, e.g., \cite[127-130]{radin93}.}
Hence, I will remain largely agnostic about this aspect of the debate between dominion and bundle theorists. In particular, the differences between civil and common law traditions in this regard do not cause special problems for my analysis of economic development takings. However, I am also more broadly interested in the values that are promoted by different ways of looking at property, particularly with regards to the question of when interference is legitimate under constitutional and human rights law. Hence, I  now turn to the question of whether or not there are any significant differences between dominion and bundle theories in this regard.

Intuitively, one might think that bundle theorists are likely to endorse greater room for state interference in property rights. Indeed, thinking about property as sticks in a bundle may lead one to think that property rights are intrinsically limited, so that subsequent changes to their content -- carried out by a competent body -- are mere reflections of their nature, not a cause for complaint. In particular, the theory conveys the impression that property is highly malleable. For the theorists that developed the bundle of sticks metaphor in the late 19th and early 20th century, this aspect was undoubtedly very important. By providing a highly flexible concept of property, they helped the state gain conceptual authority to control and regulate. Indeed, this was the clear intention of many early proponents of the bundle theory -- the ``progressives'' of their day.\footcite[195]{klein11} The early bundle theorists not only developed a theory to fit the law as they saw it, they also contributed to change.

In relation to takings law, the progressives succeeded in gaining acceptance for the use of eminent domain to benefit a wider range of public purposes than had so far been considered legitimate.\footnote{See generally \cite{yale49}.} In particular, they argued successfully that the so-called ``public use'' restriction, which had previously been enforced quite strictly, particularly by state courts, should be greatly relaxed. This change was important in creating the situation which led to economic development takings becoming a contentious issue in the US, and so provides important background to the main topic of my thesis.  I return to the public use debate in the US in much more depth later, in Chapter \ref{chap:2}, Section \ref{sec:hop}. Here I would like to stress that I think there can be little doubt that the increased scope given to eminent domain in the early 20th century was mutually conducive to the conceptual reorientation that took place during the same time.

In relation to the different, but related, issue of so-called regulatory takings, the bundle theory even  became directly implicated. A regulatory taking occurs when governmental control over the use of property becomes so severe that it must be classified as a taking in relation to the law of eminent domain. Particularly in the US, the question of when regulation amounts to a regulatory taking is highly controversial. The stakes are high because takings have to be compensated in accordance with the Fifth Amendment of the US constitution. At the same time, the law is unclear; a lack of statutory rules means that regulatory takings cases are often adjudicated directly against constitutional property clauses (often the relevant state constitution, in the first instance).

If property is thought of as a malleable bundle of entitlements that exists only because it is recognized by the law, it becomes more natural to argue that when government regulates the use of property, it does not deprive anyone of property rights, but merely restructures the bundle. In the case of {\it Andrus v Allard}, the Supreme Court adopted such an argument when it declared that ``where an owner possesses a full ``bundle'' of property rights, the destruction of one ``strand'' of the bundle is not a taking, because the aggregate must be viewed in its entirety''.\footcite[65--66]{andrus79}

Historically, therefore, it seems that bundle theorists have been largely aligned with those that favor a less restrictive approach to eminent domain. But I think it is wrong to conclude that the bundle theory {\it necessarily} implies such a stance on takings. Indeed, some prominent scholars have argued for an almost entirely opposite view. Professor Epstein, in particular, goes far in suggesting that as every stick in the property bundle represents a property right, government should not be permitted to remove any of them without paying compensation.\footcite[232-233]{epstein11} Moreover, Epstein does not believe that the bundle theory is responsible for the fact that his view of property has not been widely endorsed by US courts. Instead, he thinks the main (negative) impact of ``progressive'' thinkers stems from their tendency to adopt a ``top-down'' approach to property. That is, Epstein directs attention towards their tendency to view property rights as vested in, and arising from, the power of the state, not the possessions of individuals.\footnote{\cite[227-228]{epstein11} (``In my view, the nub of the difficulty with modern property law does not stem from the bundle-of-rights conception, but from the top-down view of property that treats all property as being granted by the state and therefore subject to whatever terms and conditions the state wishes to impose on its grantees'').} 

In my opinion, Epstein's argument shows that adoption of the bundle theory can hardly be considered a determinate factor for the kind of protection private property enjoys in a given legal system. Moreover, Epstein successfully demonstrates that as a rhetorical device, the theory may well be turned on its head. Unsurprisingly, the substance of the law, in the end, turns primarily on the values one adheres to, not the theoretical constructions one relies on when expressing those values.\footnote{To further underscore this point, it may be mentioned that while US courts do in fact recognize that a regulation can amount to a taking, this is practically unheard of in several other common law jurisdictions, including England and Australia, which nevertheless paint property in a similar conceptual light. Moreover, while the issue of regulatory takings is considered central to constitutional property law in the US, it is considered a fairly marginal issue in England, see \cite{purdue10}.}

In the civil law world, the relationship between property theorizing and property values is similarly hard to pin down at the conceptual level. To illustrate, I will again point to the question of regulatory takings. In some countries, like Germany and the Netherlands, the right to compensation is quite strong, but in other civil law countries, such as France and Greece, it is very weak.\footnote{See generally \cite{alterman10}.} In particular, the exclusive dominion understanding of property does not commit us to any particular kind of policy on this point. Indeed, the theory appears to cater comfortably to a range of different politically determined solutions to the problem of striking a balance between the interests of owners and the interests of the state. 

On the one hand, the undeniable fact of modern society is that property rights are enforced, and limited, by the power of government. Hanging on to the idea of dominion, then, necessarily forces us to embrace also the idea that dominion is not enjoyed absolutely and that government may interfere in property rights. In this way, the theory may serve as a conceptual basis upon which to argue for a more relaxed approach to protection of property rights. These rights are not absolute anyway, so why worry about interfering in them for the common good? But this story too may be turned on its head: A libertarian may well use the same image to tell a tale of property being ripped apart at its seams. Hence, he may argue, unless we want to completely lose our grasp of what property is, we had better enhance the level of protection offered to property owners.

To me, the upshot is that the differences between common law and civil law theorizing about property are not significant enough to 
make them crucial to the questions studied in this thesis. In particular, the differences between the bundle theory and the dominion idea do not appear to speak decisively in favor of any particular approach to economic development takings, nor does it provide any clear justification for regarding such takings as special in the first place. Property enjoys constitutional protection and is a recognized human right across the divide, but what this means in practice is hard to deduce from either account.

In terms of descriptive content, both theories are too bold and oversimplified. They provide a manner of speech, but they do little to enhance our understanding of the reality of property rights in modern society. In particular, they do not provide a functional account of what role property plays in relation to the social, economic and political structures within which it resides. In terms of normative content, on the other hand, they are both too bland and imprecise. They simply do not offer much clear guidance as to what norms and values the institution of property is meant to serve. They give neat explanations of what property is, but do not tell us {\it why} it should be protected. 

In the following, I will address both these shortcomings by considering property theories with a wider scope. There are many candidates that could be considered. In a recent book on property theory, Alexander and Pe\~{n}alver present five key theoretical branches: 
\begin{itemize}
\item {\it Utilitarian} theories, focusing on property's role in helping to maximize utility or welfare with respect to individual preferences and desires.\footnote{\cite[Chapter 1]{alexander10}.} 
\item {\it Libertarian} theories, focusing on property's role in furthering individual autonomy and liberty, as well as the importance of protecting property against state interference, particularly attempts at redistribution.\footnote{\cite[Chapter 2]{alexander10}.} 
\item {\it Hegelian} theories, focusing on the importance of property to the development of personhood and self-realization, particularly the expression and embodiment of free will through control and attachment to one's possessions.\footnote{\cite[Chapter 3]{alexander10}.}
\item {\it Kantian} theories, focusing on how property arises to protect freedom and autonomy in a coordinated fashion so that {\it everyone} may potentially enjoy it, through the development of the state.\footnote{\cite[Chapter 4]{alexander10}.}
\item {\it  Human flourishing} theories, focusing on property's role in facilitating participation in a community, particularly as a template allowing the individual to develop as a moral agent in a world of normative plurality.\footnote{\cite[Chapter 5]{alexander10}.}
\end{itemize}

It it beyond the scope of this thesis to give a detailed presentation and assessment of all these theoretical branches and the various ideas that have been discussed within each of them. However, in Section \ref{sec:hf} below, I will present the human flourishing theory in more detail. This is because I believe that if it is adopted, it suggests making a range of new policy recommendations regarding how the law {\it should} approach the question of economic development takings. 

First, however, I note that all the theories mentioned above are highly normative, used actively to promote the adoption of particular values associated with property. While I am not unwilling to take a stand in this debate, my main objective is to study economic development takings descriptively, by giving a case study of Norwegian waterfalls and discussing its significance in terms of comparative and human rights law. Hence, before I move on to consider other aspects, I first need a theoretical framework that allows me to meaningfully discuss those aspects that make economic development takings unique. I would like to do so, moreover, without thereby committing myself to any particular stance on how to normatively assess those aspects. 

To arrive at such a foundation, I will rely on the descriptive parts of the so-called {\it social function theory} of property.\footnote{See generally \cite{foster11,mirow10,alexander09a}. Be aware that some authors, particularly in the US, also speak of the {\it social obligation} theory, using it more or less as a synonym for the social function theory.} While this theory is often implicated in normative theories, including the human flourishing theory, I argue that it has a descriptive core which is also of great significance. Its importance to my work in this thesis is underscored by the fact that I will draw on the social function theory to answer the pressing problem of what makes economic development takings a legitimate and useful category of legal reasoning. 

Let me first reiterate that it is not {\it prima facie} clear that the category makes any legal sense at all, due to the fact that many jurisdictions lack rules that explicitly make the purpose of interference a relevant measuring stick for assessing legitimacy. To respond successfully to this potential objection, I believe it is necessary to look at property's social functions. In fact, property scholars are becoming increasingly aware of the need to do this in general, as they note that existing theories are overly focused on a narrative that revolves around individual entitlements. Many still reject that this necessitates conceptual reconfiguration, but the social function idea of property appears to be gaining ground, also as an important aid in making sense of how the law actually works. I believe this descriptive aspect of the theory provides the most appropriate way to argue that it is theoretically desirable to regard economic development takings as a special issue in property law, and I will argue for this in Section \ref{sec:edt}.

However, before making my specific point about takings, I will present the social function theory of property more generally. I will focus on showing that it captures aspects that are already highly relevant -- behind the scenes -- to how property rules are understood and applied in concrete situations. It seems, in particular, that socio-legal arguments play an important, yet often unacknowledged, role when courts interpret fundamental rules that are meant to protect private property. Bringing those aspects into the open is in itself a worthwhile project to pursue, irrespectively of any further normative stances that the social function theory might give the theorist occasion to adopt.

\section{The social function of property}

There is a growing feeling among property scholars that the notion of property has been drawn too narrowly by many of the traditionally dominant theories of property. Some have even gone as far as to challenge the idea that property is a meaningful and well-defined concept at all. These scholars point out that what counts as property in a given legal system, and what property entails in that system, depends largely on its social and political context, tradition, and even chance.\footnote{For a particularly inspiring exposition of property's elusive nature, see \cite{gray91}.} In the US, a utilitarian law-and-economics approach -- which largely takes the social and political underpinnings of property for granted -- has long been regard as standard, but even there the tide is turning. While most US scholars still regard property as a robust and meaningful category of legal thought, many are increasingly turning away from assessing property rules narrowly against their effectiveness in maximizing individual utility and social welfare. Instead, these scholars adopt a holistic approach, which allows property's social function to come into focus. One of the main proponents of this conceptual shift is Gregory S. Alexander, professor at Cornell University. In a recent article, he writes:

\begin{quote} Welfarism is no longer the only game in the town of property theory. In the last several years a number of property scholars have begun developing various versions of a general vision of property and ownership that, although consistent with welfarism in some respects, purports to provide an alternative to the still-dominant welfarist account.[...] These scholars emphasize the social obligations that are inherent in ownership, and they seek to develop a non-welfarist theory grounding those inherent social obligations.\footcite[1017]{alexander11}
\end{quote}

To scholars coming from political science, sociology or human geography, this trend will not raise many eyebrows, except perhaps for the fact that it is a recent one. After all, in fields such as these, property has never been understood merely as a set of individual entitlements that are meant to result in increased welfare. Rather, property is seen as a crucial part of the fabric of society, one that entrenches privileges and bestows power.\footnote{See generally \cite{carruthers04}.} Even scholars who believe that the institution of property is a force for good, recognize that being an owner carries with it political capital, social responsibility, and membership in a community. Those aspects, moreover, are often regarded as more important than entitlements explicitly recognized in positive legal terms. Crucially, they are important not only to the individual owners but also to society as a whole. How property rights are distributed among the population, for instance, has obvious political and economic implications, serving as a source of power and prosperity to some groups, while marginalizing others.\footnote{See, e.g., \cite[23]{carruthers04}. (``The right to control, govern, and exploit things entails the power to influence, govern, and exploit people'').}

But what is the relevance of this to property law? Usually, jurists approach property in isolation from such concerns, and often they do so because of practical necessity. The political question of what the law should be might require musings about the purpose and social context of property, but in the day-to-day workings of the law, the story goes, such considerations play a lesser role, with the importance of clear and simple rules outweighing the possible benefit that would result from contextual and holistic assessment. But at the same time, no functioning theory of  property would deny that social aspects such as expectations and obligations play a role in relation to property {\it in life}. The problem, rather, is that classical theories of property may be accused of taking the pragmatic view too far, by failing to recognize that many social functions are {\it intrinsic} to property, so that they may sometimes be impossible to disregard, also when the law is applied to concrete disputes.

This accusation can be raised against both bundle and dominion theorists. They both tend to leave little conceptual room for considering property as a social phenomena. It is recognized, of course, that rights in property -- bundled or otherwise -- serve to regulate social relations. But this effect is typically regarded as belonging to the periphery of property as a legal category, more relevant to sociologists than to property scholars. In addition, it is uncommon to observe that the causal relation between property rights and society is bidirectional, since the meaning and content of property itself is partly determined by the very same social structures that property helps establish and sustain. When this aspect of property is not recognized, the risk is that subtle dependencies between property and the political order are not brought into focus, even when they play an important role in practice.

This is particularly clear when it comes to socially defined obligations attached to property. Hardly anyone would protest that in practical life, what an owner will do with his property is as much constrained by the expectations of others as it is by law. But in addition to influencing the owner subjectively, expectations can take on an objective character by being embedded strongly in the social fabric. This, in turn, may give rise to a norm, or even a custom, which may be legally relevant, either because the law gives direct effect to it, or because it influences how we interpret rules relating to the use of property.\footnote{See generally \cite{penalver09,alexander09}.}

This seems hard to dispute as a descriptive assertion, but traditional property theorists have surprisingly little regard for it. According to Alexander, the classical theories of property convey the impression that ``property owners are rights-holders first and foremost; obligations are, with some few exceptions, assigned to non-owners''.\footcite[1023]{alexander11} The social function theorists attempt to redress this imbalance by developing theories that can naturally accommodate an account of social obligations that attaches greater weight to them as objects of property. As Alexander explains, ``social obligation theorists do not reverse this equation so much as they balance it. Of course property owners are rights-holders, but they are also duty-holders, and often more than minimally so.''\footcite[1023]{alexander11} 

It should be noted that while it lay dormant for some time, particularly in the US, this idea is by no means new. Its first modern expression is often attributed to Le{\'o}n Duguit, a French jurist active early in the 20th century. In a series of lectures he gave in Buenos Aires in 1911, Duguit challenged the classic liberal idea of property rights by pointing to their context-dependence, adopting a line of argument strikingly similar to how recent scholars have criticized the law-and-economics discourse of modern times.\footnote{See \cite[1004-1008]{foster11}. For more details about Duguit's work and the contemporaries that inspired him, see generally \cite{mirow10}.} In particular, Duguit also pointed to the notion of obligation, stressing the fact that individual autonomy only makes sense in a social context, wherein people are also dependent on each other and related through membership in communities. Hence, depending on the social circumstances of the owner, his property could entail as many obligations as it would entail entitlements and dominion. This, according to Duguit, was not only the reality of property ownership in life, it was also a normatively sound arrangement that should inspire the law, more so than the unrealistic visions of property evoked by the liberal tradition.\footnote{See \cite[1005]{foster11} (``The idea of the social function of property is based on a description of social reality that recognizes solidarity as one of its primary foundations'', discussing Duguit's work). It should also be noted that Duguit was particularly concerned with owners' obligations to make productive use of their property, to benefit society as a whole. This raises the question, however, to which we shall return in more depth later, who exactly should be granted the power to determine what counts as ``productive use''. In this way, Duguit's work also serves to underscore one of the main challenges of regulatory frameworks that seek to incorporate and draw on property's social dimension. How should decisions be made in such regimes?} 

Similar thoughts have been influential in Europe, particularly in the post-WW2 rebuilding period. For instance, as I discuss further in Chapter \ref{chap:2}, Section \ref{sec:germany}, the constitution of Germany -- her {\it Basic Law} -- contains a property clause that explicitly includes a provision stating that property entails obligations as well as rights. As argued by Alexander, this has had a significant effect on German property jurisprudence, creating a clear and interesting contrast with US law.\footnote{See \cite[338]{alexander03} (``The German Constitutional Court has adopted an approach that is both purposive and contextual, while the U.S. Supreme Court has not'').} 

A social perspective on property was also influential during the debate among the European states that first drafted the property clause in the First Protocol to the European Convention of Human Rights.\footnote{See \cite[1063-1065]{allen10}.} Later, however, the liberal conception of property gained ground also in Europe, causing jurisprudential developments that have been particularly clear in the case law from the European Court of Human Rights.\footnote{See generally \cite{allen10}.} Even so, property theorizing in Europe is still influenced by a social function view on property, more so than in the US. The European Court of Human Rights, for instance, stresses the importance of {\it proportionality} and {\it fairness} when adjudicating property cases, suggesting the importance of a contextual approach to the balancing of the many private and public interests involved.\footnote{See generally \cite[Chapter 5]{allen05}.}

I will return to possible normative implications of the social function theory later, but here I would like to stress that in the first instance it merely asks us to recognize an empirical truth. Property does not arise in a vacuum, but from within a society. As a philosophical proposition, this is obvious and hardly anyone denies it. But the social function theory asks us to consider something more, namely that property {\it law} continues to influence, and be influenced by, the social structures that surround it. Perhaps most importantly, property both reflects and shapes relations of power among members of a society.\footnote{This aspect of property's social function was stressed in a recent ``statement of purpose'' made by leading property scholars in support of the social function theory, see \cite{alexander09a}.} Moreover, it does not act uniformly in this way -- the actual effect of property on power depends on the circumstances.

An indebted farmer who is prevented by state regulation from making profitable use of his land might come to find that his property has become a burden rather than a privilege. As a consequence, someone who has already amassed power and wealth elsewhere might be able to purchase it from him cheaply. Indeed, this might provide an excellent opportunity for the outsider to consolidate his position. He can ensure that his privileges become further entrenched, both socially, politically and economically. By acquiring a farm and transforming it to recreational property, he symbolically and practically asserts his dominance and power, while also reaping a potential financial benefit resulting from his investments in a more ``modern'' pattern of use. In some cases, this dynamic can even become endemic in an area, resulting in a complete reshaping of the social fabric surrounding property.

The story might go like this: First, impoverished farmers and other locals sell homes to holiday dwellers. Then house prices soar. As a result, local people with agrarian-related incomes can't afford local homes, causing even more people to sell their land to the urban middle class. In this way, a causal cycle is established, the social consequences of which can be vicious, particularly to the low-income people who are displaced.\footnote{The general mechanism is well-documented and known as {\it gentrification} in human geography (often qualified as rural gentrification when it happens outside urban areas). See generally \cite{weesep94,phillips93,slater06}. For a case study demonstrating the role that state regulation can play (perhaps inadvertently) in causing rural gentrification, see \cite[1027-1030]{darling05}.} My theoretical contention is the following: Setting out to protect property in a situation like this -- when property rights pull in different directions -- requires taking some stance on whose property, and which of property's functions, one is aiming to protect. In particular, should the law protect the property rights of local people who face displacement, or should it protect the property rights of outsiders wishing to invest?

Some may shun away from this way of posing the question, by arguing that it would be better to rely on clear rules that can deliver justice to owners with a minimum level of dependence on the particular social processes that property is involved in at any given time. I am inclined to disagree with such a stance from the outset, since justice itself is a notion that largely seems to depend on social conditions. However, my main point here is that the prospect of such ``socially neutral'' rules is simply illusory when both sides of a conflict are in a position to adopt a property narrative to argue for their interests. For an excellent example of such a situation, it is enough to return to the story of Donald Trump coming to Scotland that we told in Section \ref{sec:dts}.

As long as Trump threatened to use compulsory purchase, the local people could adopt a traditional ``pro-property'' stance against Trump. But as soon as Trump decided to fence them in by relying on his own property rights, they had to adopt a seemingly contradictory view, {\it against} property, on the basis that Trump's rights should be limited out of concern for the community. So how do we classify the anti-Trump stance with regards to property? The answer is unclear under classical theories, but under the social function stance, it becomes easy to resolve. The locals sought to protect property, but not just any property. The property they wanted to protect was the property which served the social function of sustaining the existing community. The property they wanted to protect was the property that {\it meant} something to them.

Undoubtedly, this was also the sentiment of Trump and his supporters, who could also make a case based on property. Hence, in conflicts such as these the law will invariably have to take a stand regarding which social functions it wishes to promote. In all likelihood, such a stand must also sometimes be taken by whoever {\it interprets} the law, since it is exceedingly unlikely that the legislature will ever be able to provide clear rules for resolving all conflicts of this kind. Lastly, and most controversially, the courts may find occasion to curtail the power of government -- perhaps even the legislature -- if their power is usurped by powerful actors wishing to undermine property's proper functions to further their own interests. This, in particular, becomes the question of constitutional and human rights limits to interference in property.

Property shapes and reflects societies, but it also shapes and reflects social commitments and dependencies within those societies.\footnote{See generally \cite{alexander09}.} Again, this function of property is highly dependent on context. A small business owner, by virtue of being a member of the local community, is discouraged from becoming a nuisance to his neighbors. Everything from erecting bright neon signs to proposing condemnation of neighboring properties are actions that he will be socially deterred from taking. If the local shop owner does not conform to social expectations, he will pay a social price. Indeed, most likely even an economic price, especially if his customer-base is local. At the same time, the local connection would serve to make the business owner positively invested in the well-being of the community. This would encourage everything from sponsoring local events to hiring local youths as part-time helpers.

But at the same time, the local business owner might be discouraged from changing his business model to become more competitive, if this is perceived as a threat to other members of the community. Economic rationality might suggest that he should expand, say, by physically acquiring more space and targeting new groups of customers, but social rationality might make this an untenable proposal. This, however, might render the business economically unsustainable, particularly if it is facing fierce competition from businesses that are not similarly constrained by community ties. Moreover, even if the business is in fact viable as long as the community remains in place to support it, the perception that there is room for improvement might increase external pressures both on the business and the community. Importantly, in the age of regulation for commercial facilitation, the state itself might exert pressure of this kind.

Then, if our local shop owner goes out of business, for whatever reason, the new owner might fail to become integrated in the community in the same way, with obvious consequences for the property's function in that community. Indeed, if we imagine that the new owner is a large commercial actor who is hoping to raze the community in order to build a new shopping center, we are at once reminded of the stark contrasts that can arise between various social functions of property. The property rights of a shop owner can be the life nerve of a community, while the exact same rights in the hands of someone else can spell destruction. While this is an undeniable empirical fact of property ownership, it is far from clear what its legal ramifications are. Here, it is tempting to embrace a normative stance, and argue for particular social values that the law {\it should} promote. However, I would like to hold on to the descriptive mode of analysis a little further. For it is perfectly clear that regardless of whose interests win out in the end, assessments of the social function of property will have played an important role in brining about that outcome.

This is true not only when the law explicitly requires that this function is to be taken into account, such as in relation to the property clause of the basic law of Germany. It also commonly becomes true, as courts search for information to guide them in their interpretation and application of statutory rules that are seemingly not concerned with social aspects of property. The classical example from the US is the case of {\it State v Shack}.\footcite{shack71} The case concerned the right of a farmer to deny others access to his land, a basic exercise of the right to exclusion often regarded as fundamental to the very definition of property. The controversy arose after the two defendants, who worked for organizations that provided health-care and legal services to migrant farmworkers, entered the land of a farmer without permission. They were there to provide services to the farmers employees, and when the farmer asked them to leave, they refused.

In the first instance. they were convicted of trespassing in keeping with New Jersey state law, but on appeal the Supreme Court of New Jersey overturned the verdict. The court held that the dominion of the land owner did not extend to dominion over people who were rightfully on his land. Hence, as long as the defendants were there at the request of the workers, the owner had to tolerate this. Importantly, the court argued for this result -- which was not based on any natural reading of the New Jersey trespass statute -- by pointing also to the fact that the community of migrant workers was particularly fragile and in need of protection. Their right -- in property -- to receive visitors where they work and live, therefore, had to be recognized, in spite of this limiting the farmer's exclusion right.

The lesson to take from this is that the social function of property can play a role even when this does not explicitly follow from any property rules. This, in turn, may be used to argue that a shift towards a social function theory is desirable. In so far as the property rules we rely on explicitly directs us to take the social aspect of property into account when applying the law, it might be permissible for the practically minded jurist to conclude that there is little need for theorizing about property's social dimension. This dimension, in so far as it is relevant, is quantified inside the law itself, not by theories that encompass it. But as a matter of fact, cases like {\it State v Shack} show that the social dimension can be relevant even when it is not mentioned in any authority, even in relation to clear rules that would otherwise appear to leave little room for statutory interpretation. It arises as relevant, in such cases, because the social dimension is intrinsic to property itself. 

This might still be a radical claim, but it is primarily a descriptive one. Indeed, even if the case of {\it State v Shack} had gone the other way, I would be inclined to take from it the same lesson. If the owner's right to exclusion had received priority over the workers right to receive guests and the owner's obligation to respect this right, that too would be an outcome that would likely underscore the social function of property. To illustrate this, it is helpful to look to an article written by Eric Claeys, where he is critical both of the social function theory in general and {\it State v Shack} in particular.\footcite{claeys09} Given the basis on which that decision was made, he is led to argue, however, by also pointing to those aspects of the social context that speak in favor of the farmer.\footnote{\cite[941-942]{claeys09}.} Indeed, since he aims to engage with the social function theorists, he cannot simply declare  that the trespass rules are absolute and that the social circumstances are irrelevant. 

Instead, he argues that by considering the circumstances in {\it more} depth, a different outcome suggests itself.\footnote{\cite[941]{claeys09} (``there are good reasons for suspecting that there was more blame to go around in Shack than comes across in the case's statement of facts'').} But even if this is true, it is no argument against the descriptive content of the social function theory, merely an argument against those who think that particular values need to be endorsed by anyone willing to look to the social context of a property dispute. In this regard, it is not hard to agree with Claeys that normative fundamentalism is wrong. Indeed, he might even have a point in criticizing some social function theorists for normative naivety.\footnote{\cite[945]{claeys09} (``Judges might think they are doing what is equitable and prudent. In reality, however, maybe
they are appealing to a perfectionist theory of politics to restructure the law, to redistribute property, and ultimately to dispense justice in a manner encouraging all parties to become dependent on them.'')} 

I do not follow Claeys, however, when he takes this to be an argument {\it against} the form of legal reasoning that social function theories promote, and which he himself skillfully engages in.\footnote{In particular, I do not follow the leap Claeys makes when he suggests that it is beneficial to keep ``discretely submerged'' what he describes as ``culture war overtones'' in legal reasoning.\cite[947]{claeys09}.} In {\it State v Shack}, for instance, such reasoning was clearly in order. To engage in it was far {\it less} naive than to dismiss it on the basis that it would be irrelevant to the case. Indeed, if it the social function view had been dismissed, the entitlement-based idea of property would in effect do {\it unacknowledged} normative work, with no basis in anything more authoritative than a palpably oversimplified idea of the meaning of property. 

However, I agree with Claeys that prudence is in order. Moreover, I am not saying that the social function theory does not have normative consequences. It clearly does. Invariably so, simply because it provides a new way of talking about property and analysing conflicts, which will in turn influence our normative assessments. This is also illustrated by {\it State v Shack}. Despite Claeys skillful advocacy, many would no doubt fail to be convinced of the social merit of recognizing a right to exclusion in this case. But the crucial aspect of the social function narrative is that it makes such aspects clear, not that it commits us to, or promises to deliver, any morally superior stance on property that can deliver ``correct'' outcomes in cases such as this.

This challenges a common assumption, among both detractors and supporters of the social function theory, who argue that the theory commits us to a particular form of normative assessment, in pursuit of the ``good''. Some even argue that property law should be studied from the point of view of virtue ethics.\footnote{See generally \cite{penalver09}.} Unsurprisingly, critics such as Claeys use this to launch attacks on the social function theory and its supporters, by arguing that it represents a way of thinking that will invariably lead to lessened constitutional property protection and greater risk of abuse of state powers.\footnote{See \cite{claeys09} (``The more ``virtue'' is a dominant theme in property regulation, the less effective ``property'' is in politics, as a liberal metaphor steering religious, ethnic, or ideological extremism out of the public square'').} Indeed, increasing the room for state interference is often seen as the aim of conceptual reconfiguration; the social function view of property tends to be associated with social democratic and/or redistributive political projects, by which the notion of property is recast to justify greater interference in established rights.\footnote{Despite his commitment to ``value-pluralism'', this motivation is also clearly felt in the work of Gregory Alexander. He argues, for instance, that the social obligations inherent in property imply that the ``state should be empowered and may even be obligated to compel the wealthy to share their surplus with the poor'', see \cite[746]{alexander09}. For an assessment linking similar views on property in Europe to the dominance of social democratic thought in the post-WW2 period, see \cite{allen10}.}

It is important to note, however, that while social democratic policies may be easier to justify by emphasizing the social function of property, the mere recognition that property has an important social dimension does not in itself offer any justification for policies of this kind. For one, policy reasons must be tied to the prevailing social and economic circumstances, they will not automatically succeed merely by virtue of a conceptual shift. In addition, it seems to me that the most crucial premises used in arguments for greater state control and state-led redistribution projects concern the nature of the state, not the functions of property.

In particular, why should we believe that the state is the ultimate social institution to which property {\it should} answer? Is it not, for instance, equally possible to contend that property should continue to answer to less formal social structures that it is already embedded in by virtue of owners' membership in local communities? If so, one might as well want to limit the state's role to that of ensuring fair play among individuals and communities. A contentious question, then, might be to what extent the state should actively promote certain kinds of communities in accordance with political goals. Embracing more direct state control, on the other hand, would no longer seem very natural, at least not as a goal in itself. Indeed, on the social function view, the very idea of direct state control seems to depend on the claim that more low-level social structures fail to function properly and, crucially, that state control is {\it better}. In my opinion, this requires a separate argument. Hence, to move uncritically between talk of the ``community'' and talk of the state, as writers like Pe\~{n}alver and Alexander sometimes do, is in my opinion inappropriate.

In fact, I am inclined to believe that it is only appropriate to equate community with the state in highly special situations, for instance if it can be shown that owners insulate themselves from, and engage in exploitative practices towards, other people and communities. Importantly, to argue that such a situation obtains requires a case to be made that is compelling both empirically and politically. In this regard, I believe theory alone has little to offer. This is a reason to conclude that the social function view of property in fact tells us very little about how widely the state should intervene in property in a given society. It allows us to recognize the {\it possibility} that the state may have to intervene on behalf of certain property values, say those that aim to protect communities. But this is no argument in favour of the position that the state should intervene more or less often than it currently does. Importantly, the theory can still serve a crucial purpose in that it allows us to reason more clearly about {\it when} it is appropriate for the state to intervene. For instance, the social function theory will later be used by me to single out economic development takings for special consideration. But this will not commit me to a particular normative stance on such takings.

More generally, it does not follow from the recognition that property structures are social in nature that {\it any} institution should actively seek to neither change nor protect those structures. The Humean position, namely that the existing distribution of property rights represents a socially emergent equilibrium, remains plausible. Moreover, the normative stance that this equilibrium is a {\it good} one (or at least as good as it gets) remains as contentious -- and as arguable -- as ever. For this reason, I believe it is appropriate to approach the social function theory as a descriptive theory in the first instance.

It is worth emphasizing that in taking this view I depart from the stance taken by many of the contemporary scholars who advocate on behalf of social function theories, including some that reject social democratic ideals. Hanoch Dagan, for instance, is a self-confessed liberal, but still explicitly and strongly argues for a social function understanding on the basis that it is morally superior. ``A theory of property that excludes social responsibility is unjust'', he writes, and goes on to argue that ``erasing the social responsibility of ownership would undermine both the freedom-enhancing pluralism and the individuality-enhancing multiplicity that is crucial to the liberal ideal of justice''.\footcite[1259]{dagan07}

If this is true, then it is certainly a persuasive argument for those who believe in a ``liberal idea of justice''. But for those who do not, or believe that property law is -- or should be -- largely agnostic on this point, a normative justification for the social function theory along these lines can only discourage them from adopting it. Such a reader would be understandably suspicious that the {\it content} of the social function theory -- as Dagan understands it -- is biased towards a liberal world view. The reader might agree that property continuously interacts with social structures, but reject the theory on the basis that it seems to carry with it a normative commitment to promote liberalism.

Danach stands out somewhat in the literature by focusing on {\it liberal} values, but as I have already indicated, he is not alone in proposing highly normative social function theories. Indeed, most contemporary scholars endorsing a social function view on property base themselves on highly value-laden assessments of property institutions.\footnote{See, e.g.,  \cite{alexander09,crawford11,davidson11,singer09,penalver09}.} While they provide interesting insights into the nature of property, I am struck by a feeling that these writers all tend to overstate the desirable normative implications of adopting a social function view. In addition, they appear to believe that accepting this view on property requires us to embrace certain values and reject others. Moreover, one is left with the impression that the social function theory has little to offer beyond the values with which it is imbued, which can in turn push the law in the direction that these writers deem desirable. 

I disagree that this is the case, at least for the social function theory as I understand it. Dagan's theory of property might be conducive to ``liberal justice'', but this is because it involves far more than what follows analytically from the proper recognition that  social functions should be considered relevant when adjudicating on the rights and obligations attached to property. Indeed, it is Dagan's clearly stated aim to propose a theory that promotes specific liberal values. ``There is room to allow for the virtue of social responsibility and solidarity'', he writes, continuing by suggesting that ``those who endorse these values should seek to incorporate them -- alongside and in perpetual tension with the value of individual liberty -- into our conception of private property''.\footcite[802]{dagan99} This view is reflected further in the concrete policy recommendations he makes, for instance in relation to the question of when it is appropriate to award less than ``full'' (market value) compensation for property following a taking.\footnote{See generally \cite{dagan14b}.}

My objection is not that his proposals are necessarily wrong, but that they need not be accepted in order to conclude that the social function of property should be given a more prominent place in property theory. Importantly, I think the focus on normative reasons threatens to overshadow the most straightforward reason for awarding social structures a more prominent place in the analysis, namely that they are almost always crucially important behind the scenes, even if they go unacknowledged. The social function theory, rather than being ``good, period'', as Danach suggests, is nothing more or less than accurate, irrespectively of one's ethical or political inclinations. As such, it provides the foundation for a debate where different values and norms can be presented in a way that is conducive to meaningful debate, on the basis of a minimal number of hidden assumptions and implied commitments. Thus, the first reason to accept the social function theory, for me, is epistemic rather than deontic.

That is not to say that normative theories should not be formulated on the basis of the social function theory, it merely means that I believe it is useful to maintain at least a theoretical division between the descriptive and normative aspects of such theorizing. I return to normative aspects in the next section, arguing that the commitment to ``human flourishing'' endorsed by Professor Alexander is a particularly well-argued norm that arises from value-based assessment of the social function of property. This, I argue, is in large part also due to the value-pluralism inherent in this idea, suggesting as a positive normative claim that our notions of property {\it should} allow a divergence of opinions and values to influence the law and its application in this area.

Moreover, I believe the history of the social function theory lends support to my claim that it is useful to emphasize that the theory gives us important descriptive insights that carry few normative commitments. This is particularly important, I believe, in a time when property scholars are showing greater willingness to explore new theoretical ideas. Theories can hardly be entirely value-neutral, nor is this a goal in itself. But in my opinion, a good theory is one that can serve as a common ground for further discussion based on disagreement about values and priorities. According to Kevin Gray, ``the stuff of modern property theory involves a consonance of entitlement, obligation and mutual respect''.\footcite[37]{gray11} It is important, I think, that the same measured perspective is reflexively applied towards theory itself, to diminish the worry that a broader theoretical outlook is the first step towards unchecked state power and rule by ``judicial philosopher-kings''.\footcite[944]{claeys09}

In the next subsection, I will argue in some more detail why such a cautious perspective is warranted, by considering how the Italian fascists appropriated the social function theory in 1930s. Building on the highly inspiring work of di Robilant, I will also briefly track how non-fascist property scholars opposed this development by focusing on value-pluralism, local self-governance and freedom.\footcite{robilant13} Importantly, these scholars embraced the social function theory as a common ground from which to launch a meaningful attack on more radical ideas, without alienating those with divergent views. Instead of clinging to the old-style liberal discourse that the fascists had either flatly rejected or completely subverted, many Italian non-fascists were willing to engage in a discourse revolving around property's social function, by spelling out a more measured set of ideas based on this premise. Crucially, this set the stage for a form of intellectual resistance that did not reject those aspects of fascism that had great appeal to the public and which arguably also reflected true insights into the unfairness and lack of sustainability of the established legal order.

\subsection{Rooting out fascism: {T}he tree of property}

While the social function theory makes intuitive sense, it is also highly abstract. Therefore, its exact content has been notoriously hard to pin down. This is recognized by contemporary scholars endorsing a normative view, who attempt to address this by proposing lists of values that should be taken into account while giving examples of how they should be used to inform the law in concrete areas or cases.\footnote{See, e.g., \cite{alexander14,alexander11,dagan07}.} Unsurprisingly, however, views soon diverge regarding the concrete import of a social function view on legal reasoning. Even so, the contemporary debate appears to be based on a common ground that is quite stable, also with respect to the overall notion of what good the theory can do. But as history shows, this state of affairs is by no means guaranteed. 

In a recent article, Anna di Robilant illustrates this point exceptionally well by tracking the history of social function theorizing in Italy during the fascist era. The fascist property scholars, she notes, were happy to embrace the social function theory, since it provided them with a conceptual starting point from which to develop their idea that rights and obligations in property should be made to answer to one core value: the interests of the state.\footnote{See \cite[908-909]{robilant13} (``Fascist property scholars had also appropriated the social function formula. For the Fascists, the social function of property meant the superior interest of the Fascist state.'').} This stance was as effective as it was oversimplified. As di Robilant notes, ``earlier writers had been hopelessly evasive about the meaning and content of the social element of property''.\footcite[909]{robilant13} Hence, the fascist approach filled a need for clarity about the implications of the main idea, which was by now attracting increasing support both from the public and the academic community. Established property doctrine, it was widely felt, was both ineffective and unfair to ordinary people. Rather than securing productivity and a livelihood for all, property was used mainly as an instrument for maintaining the privileged position of the elites. By promising to change this state of affairs, the fascists attracted many to their cause.

As di Robilant notes, supporters of the fascist idea of property made clear that ``social function meant the productive needs of the Fascist nation''.\footcite[909]{robilant13} But at the same time, they cleverly denied that there was a ``contradiction between subordinating individual property rights to the larger interest of the Fascist state and the liberal language of autonomy, personhood, and labor''.\footcite[900]{robilant13} In this way, fascist scholars could claim that fascist liberalism was true liberalism, thereby subverting the conceptual basis for the traditional idea of liberal justice.\footcite[900]{robilant13} In this situation, there was reason to suspect that clinging to liberal dogma would be a largely ineffective response. Moreover, it seemed undeniable that fascism's appeal was rooted in real concerns about the fairness and effectiveness of the liberal legal order. 

Hence, many non-fascists shunned away from uncritical defense of traditional liberalism. Instead, they agreed that property's social function should come into focus, but emphasized the plurality of values that could potentially inform this function, not the interests of the state. In addition, they also noted that property rights were invariably associated with {\it control} over resources, and that the social functions of property depended on the resources in question. To own property, they argued, provides individuals with a source of privacy, power and freedom that is, as a matter of fact, highly valued. It is valued, moreover, for its implications in a social context. To capture these insights, Italian scholars adopted the metaphor of a ``tree'', by describing the core social function of property as the trunk, while referring to the various resource-specific values attached to property as branches.\footcite[894-916]{robilant13} As di Robilant notes regarding these theorists:

\begin{quote}
The rise of Fascism, they realized, was the
consequence of the crisis of liberalism. It was the consequence of liberals' insensibility to new ideas about the proper balance between individual rights and the interest of the collectivity.\footcite[907]{robilant13}
\end{quote}

In light of this, the tree-theorists concluded that continued insistence on the protection of the autonomy of owners was not a viable response. Instead, they adopted a theory that ``acknowledges and foregrounds the social dimension of property'', but without committing themselves to fascist ideas about the supreme moral authority of the state.\footcite[907]{robilant13} The value of autonomy was in turn recast in terms of property's social function. Arguably, this served to make the case far more compelling. Protecting autonomy could be seen as an aspect of protecting property's freedom-enhancing function, both at the individual level and as a way of ensuring a right to self-governance and sustenance for families and local communities. This, moreover, could not easily be derided as tantamount to protecting unfair privilege and entitlement. In fact, the suggestion was made in an effort to protect democracy itself.

I believe the story of fascist appropriation of the social function theory provides further weight to my claim that it is sensible to  maintain a descriptive perspective on its core features. Indeed, the readiness with which the fascists embraced social function theorizing serves as a reminder that we cannot easily predict what normative values may come to be promoted on its basis. Hence, it is also call for continuous vigilance when it comes to normative assessment and debate. At the same time, we are reminded of the danger of attaching too much normative prestige to a theory that is abstract and open to various interpretations.

In particular, it seems to me that failure to recognize the descriptive nature of the core idea can lead to unrealistic expectations of what the social function theory actually provides. In addition, it will make it harder for the theory to gain acceptance as a conceptual common ground from which to depart when engaging in debate. Indeed, if no division is recognized between normative and descriptive aspects, the historical record would allow detractors to make a {\it prima facie} plausible attack on the social function theory by arguing that it is fascism in disguise, or that fascism, rather than liberal justice, is where we end up in practice should we chose to adopt it.\footnote{This would echo the claim already made by Claeys, that the theory (when coupled with virtue ethics) might become a slippery slope towards the kind of extremism and revolt against oppression that gave rise to the Rwanda genocide in the early 1990s \cite[926-927]{claeys09}.}

In response, one might retort that this is cherry picking the historical facts, or that the fascists misunderstood or perverted the theory. That is certainly plausible, but the point I am trying to make here is that this kind of debate is in itself unhelpful. Unless the social function theory is rendered neutral enough to be acceptable as the conceptual premise of debate, it is likely going to fail -- in a purely epistemic sense -- as a template for negotiating conflicts about property. Those who oppose the norms associated with the theory will oppose also the core descriptive content, if they feel that the latter commits them to the former. I believe that this, in turn, suggests that those advocating on behalf of the social function theory should take care to avoid rhetorical hubris. The main point to convey, I believe, is that the theory is in fact more accurate, in a purely epistemic sense, than other conceptualizations of property.

The story of the fascist appropriation of the social function theory also points to the danger that often attach to abstract theories with normative implications: That they allow us to opportunistically recast whatever values we wish to promote, by providing qualifications for them in abstract terms that are hard to refute. The fascists did this, and the non-fascists responded. Hence, in the end one could do little more than hope that the fascists' vision of their state as an ``ethical state'' that ``every man holds in his heart'' would eventually prove less attractive then the promise of self-governing communities bustling with diversity in life and character.

\subsection{Towards a normative stance}

The social function theory can facilitate a new kind of normative reasoning, arising from how the theory allows us to recognize more subtle distinctions between different kinds of property and different kinds of circumstances. For instance, staunch entitlements-based approach to autonomy will leave us with little room to differentiate between the protection of investment property and the protection of a home, unless such a distinction is explicitly provided for in the law. But a social function approach compels us to notice the difference and to acknowledge that it might be legally, as well as ethically, relevant. Hence, if we seek to argue for protection of investment property, we must in principle be prepared to face counter-arguments that revolve around particulars of the investor's role in society and his relationship to the community of people that are affected by how he manages his property. Similarly, if someone argues against protecting home ownership, we can respond by drawing on additional arguments based on the importance of the home both to the owner, her family and friends, and her community. Under the social function theory, it becomes generally relevant to address how a home creates a sense of belonging and provides a basis for membership in social structures.

I believe normative assessments should aim to be as concrete as possible. That said, I still think it is worthwhile to provide more abstract forms of expression for core values, to clarify the ethical premises that provide the basis for concrete value-based conclusions. To me, therefore, normative theories should aim to be meta-ethical, not just ethical. They should provide a vocabulary and a conceptual framework tailored to advancing one's values. However, they should recognize that the ultimate expression of those values is given in relation to concrete facts. This, I believe, is prudent in light of how abstract ethical assertions are necessarily imprecise, and run the risk of being distorted or exaggerated, particularly as they gain influence.

Invariably, the most accurate information regarding the values I rely on when assessing cases will be conveyed by my assessment of the cases, not by my theorizing. On a deeper level, I am inclined to believe that value-systems are more or less unique to individuals, so that ethical theories are helpful primarily in that they provide an introduction to keywords and important lines of argument that will recur in different forms. As such, they enhance understanding, making it easier to communicate ideas and opinions in such a way that potential respondents are likely to enjoy a somewhat less inaccurate impression of what they are responding to. 

In short, I believe that ethics make moral judgments communicable, allowing new ideas to be created in the minds of individuals. It should come as no surprise to the reader, therefore, that I believe in ethical men and women, but not in ``ethical Man'' or -- God forbid --  the ``ethical State''. Luckily, I find some support for this world view in recent theories that have been proposed as normative extensions of the social function theory of property. These are the subject of my next section.

\section{Human flourishing}\label{sec:hf}

Taking the social function theory seriously forces us to recognize that a person's relation to property can be partly constitutive of that person's social and personal identity, including both its political and economic components. Hence, property influences people's preferences, as well as what paths lie open to them when they consider their life choices.\footnote{See generally \cite{alexander09}.} This effect is not limited to the owner, it comes into play for anyone who is socially or economically connected to property in some way. The life-significance of property might be clearly felt by a potentially large group of non-owners as well.\footcite[128-129]{alexander09d} The importance of property is obviously reduced if we move away from it in terms of social or economic distance. Hence, in many cases, property will be most important to its owner, simply because she is closest to it. This is not always the case, however, especially not if property rights are unevenly distributed, or in the possession of disinterested or negligent owners. Moreover, as mathematically oriented sociologists take pride in pointing out, social connections are ubiquitous  and the world is often smaller than it seems.\footnote{See generally \cite{schnettler09}.}

Hence, there is certainly potential for making wide-reaching socio-normative claims on the basis of this perspective on the meaning and content of property. But which such claims {\it should} we be making? According to some, we should adjust our moral compass by looking to the overriding norm of {\it human flourishing} as a guiding principle of property law. Colin Crawford, for example, explicitly argues that the social function theory of property should ``secure the goal of human flourishing for all citizens within any state''.\footcite[1089]{crawford11} In a recent article, Alexander goes even further, by declaring that human flourishing is the ``moral foundation of private property''.\footcite[1261]{alexander14} 

As I have already explained, I believe -- in contrast to both Crawford and Alexander -- that it is useful to decouple such normative claims from the descriptive core of the social function theory.\footnote{Crawford comments that the social function theory on its own  ``is not self-defining and invites many interpretations'', see \cite[1089]{crawford11}. The normative theory he proposes is clearly aimed at filing this perceived gap, by pinning down normative commitments that Crawford believes are intrinsic to the theory. However, as I have already argued, I reject this approach, since it unwisely downplays the fact that the social function theory can serve as a common ground among commentators with widely divergent normative views. Indeed, Crawford himself refers unfavorably to a writer who addresses the social function theory, but who, according to Crawford, proposes that ``property's social function is best served by focusing on overall economic production and efficiency in a given society, allowing the market's invisible hand to work its magic'', \cite[see][1089]{crawford11}. Against Crawford, I would argue that it is better to counter such a claim by arguing why it is normatively wrong than by suggesting that people with such values should be discouraged from attempting to argue for them on the basis of a social function understanding of property. Rather, by encouraging such an argument it should become easier to make the case why the values promoted are ultimately undesirable. This, at least, should follow if Crawford is otherwise largely correct (as I think he might be).} I therefore refer to the more distinctly normative aspects of their work as human flourishing theorizing. 

Human flourishing has a good ring to it, but what does it mean? According to Alexander, several values are implicated, both public and private.\footnote{See generally \cite{alexander14,alexander11}.} Importantly, Alexander stresses that human flourishing is {\it value pluralistic}.\footnote{\cite[750-751]{alexander09}.} There is not one core value that always guarantees a rewarding life. To flourish means to negotiate a range of different impulses, both internal and external. Importantly, these all act in a social context which influences their meaning and impact.\footcite[1035-1052]{alexander11}

For the family of a homeowner, the value of the ownership tends to be great; a home is a home for any non-owner living there, just as much as it is a home for the owner. This, in turn, creates both commitments and opportunities for the owner, which may or may not find recognition in the law and our legal reasoning. Regardless of this, it certainly carries significant importance both to her life and the lives of those that depend on her. If property is rented out as a home to someone else, the importance of ownership may be {\it greater} to a non-owner. Indeed, assuming a society where tenancy is a well-functioning social institution, the continuation of the established property pattern might well be of greater importance to the tenant than it is to the owner.

The effect on non-owners can also be restrictive in socially desirable ways. If an apartment has an owner, it discourages squatters, for instance. Moreover, this effect clearly depends also on {\it who} the owner is and the choices she makes in managing her property. If the owner lives in the apartment, squatting is hopefully not going to be an issue. But even the owner of an unoccupied apartment can discourage squatting by managing her property well. However, if owners mismanage their apartments, for instance because they seek to obtain demolition licenses, squatters can take opportunity of this. The risk, of course, increases if housing cannot be afforded by a large number of society's members. In this case, it is natural to argue that something is amiss with the prevailing property structures.

Now, the social function theory of property can also come into play, since it allows us to attach significance to this also when discussing the property rights of individual owners.  In particular, we are not compelled to pretend as though possible failures of property as a social institution are irrelevant when considering rights and responsibilities attached to it. As a matter of fact, they are not; actual squatters clearly affect the owner, influencing both the meaning and the value of her property, both to her, potential buyers, the local government, the state, and other interested parties. Even the mere {\it risk} of squatting can play such a role. But a property theory which does not recognize the social function of property might not allow us to take this into due regard. As long as the standard expectation of an owner is to be able to enjoy her apartment free of squatters, an entitlements-based view on property could easily force us into denial regarding actual (risk of) squatters.

In particular, we would be led to consider squatting as an interference with the owner's rights which the state can not, on pain of disrespecting property, recognize as a legitimate response to mismanagement and imbalances in the property structure. The normative significance of real life -- where squatting often happens due to badly managed property -- is discounted  because our conceptual glasses block it out. Then, the almost unavoidable consequence is that the state also recognizes a {\it positive} obligation to forbid squatting, and to forcibly remove squatters on behalf of owners. Under a narrow entitlements-based conception, this is the natural outcome, and must be classified as an act of protecting private property. Hence, under classical liberal values, it also becomes {\it good}. Here, however, the social function theory permits us to take a highly divergent view, to carry forward different value-judgments.

In particular, if squatting is recognized as creating new interests and obligations attaching to the property, it may now be argued that  it is the use of state power to evict that is the most severe act of interference. Not only interference in whatever housing rights the squatters may have, but in fact also as an interference in {\it property}. Hence, such state action might itself be morally suspect and held to be in need of further justification. In the Netherlands, the Supreme Court adopted a line of reasoning reflecting these insights, when it held that the right not to be disturbed in one's home life also applied to squatters. Hence, the property owner could not forcibly evict people who had taken up residence in her property.\footnote{See NJ 1971/38. The court held that the lower court had erred in taking it proven that the ``house in the original charge was ``in use'' by the owner of this house'', as required by the statute under which the squatters were tried. Instead, the Supreme Court held that ``art.138, in so far as it mentions houses, is specifically aimed at protecting home rights, in connection with which the words ``in use'' (differently than the court judged) can only be understood as ``actually in use as a house'' , as in accordance with ordinary use of language''. The upshot was that it was the squatters, not the owner, who enjoyed protection under the statute. In terms of the bundle theory, a right thought to be in the owner's bundle was deemed to actually belong to the bundle of the squatters, as this corresponded better to the circumstances of the case and the purposes meant to be served by the statute in question.}

In South Africa, a somewhat similar line of reasoning was adopted in the recent case of {\it Modderklip East Squatters v. Modderklip Boerdery (Pty) Ltd}, analysed in depth by Alexander and Pen\~{n}alver.\footcite[154-160]{alexander11} The case dealt with squatting on a massive scale: Some 400 people had taken up residence on land owned by Modderklip Farm, apparently under the belief that it belonged to the city of Johannesburg. The owner attempted to have them evicted and obtained an eviction order, but the local authorities refused to implement it. Eventually, the settlement grew to 40 000 people and Modderklip Farm complained that its constitutional property rights had not been respected.

The Supreme Court of Appeal concluded that Modderklip's property rights had indeed been violated, but noted that so had the rights of the squatters, since the state had failed to provide them with adequate housing.\footnote{See \cite{modderklip04}.} However, they upheld the eviction order and granted Modderklip Farm compensation for the state's failure to implement it. The Constitutional Court, on the other hand, while agreeing that the eviction order was valid, concluded that as long as the state failed in its obligations towards the squatters, the order should not be implemented.\footcite{modderklip05} The eviction of the squatters, in particular, was made contingent upon an adequate plan for relocation. In the meantime, Modderklip would receive monetary compensation, from the state rather than the squatters. In this way, the Court recognized the social function of property; they refused to give full effect to Modderklip's property rights as long as that meant putting other rights in jeopardy. The fact that the squatters had no place to go, in particular, was allowed to influence the content of Modderklip's right, making it impermissible to implement a standing eviction order.

It is possible to cast this outcome as an interference in property rights that was regarded as acceptable in the public interest. However, the reconceptualization in terms of property itself having a social function appears highly attractive. Moreover, it is also consistent with the South African constitution, which also focuses on property's social dimension.\footnote{See section 25 of the Constitution of the Republic of South Africa, Act 108 of 1996.} Thinking about cases like {\it Modderklip} in terms of property's social function allows us to remove the state as an intermediary between the owner and the other interested parties, in this case the squatters. As argued by Alexander and Pe\~{n}alver, it becomes possible to think of the Court as adjudging based on Modderklip's own responsibility, as an owner, towards other members of the community that have an interest in the property.\footnote{\cite[157]{alexander11} (``The courts' unwillingness to ratify Modderklip's desire to remove the squatters from its land illustrates the courts' willingness to take seriously the obligations of owners, not only as they concern owners' direct relationship with the state but also in relation to the needs of other citizens'').}

On this basis, it becomes easier to conclude that it is permissible for courts to take the social context into account even in the absence of any specific state action or legislation to indicate that this should be done, or that the public interest is at stake. Indeed, one of the problems in {\it Modderklip} was that the state had failed also in its responsibility towards the squatters. Moreover, while the local sheriff had refused to implement it, an eviction order had in fact been granted. Hence, thinking of the case as interference in the public interest becomes difficult.

More importantly, by taking into account the social function of property, it becomes possible to argue for the outcome in Modderklip positively on the basis of property values. In this way, property is no longer seen to stand in the way of justice in cases such as this. We need not ``interfere'' with rights to secure an appropriate outcome, we only need to apply property law. As Alexander puts it in another recent article: 

\begin{quote} The values that are
part of property's public dimension in many instances are necessary
to support, facilitate, and enable property's private ends.
Hence, any account of public and private values that depicts them as categorically
separate is grossly misleading. One important consequence of this
insight is that many legal disputes that appear to pose a conflict between
the private and public spheres or that seemingly
require the involvement of public law can and
should, in fact, be resolved on the basis of private law -- the law
of property alone.\footcite[1295-1296]{alexander14} \end{quote}

Protection of property, when property is understood in this way, becomes a potential source of justice, also for squatters. The basic values attached to property -- freedom, liberty, autonomy -- have not really changed, but are applied in a new way. In particular, they no longer only apply to the owner's interest in property, but also to that of other individuals closely connected to it. This normative turn, I argue, will potentially strengthen the institution of property itself, while also serving to loosen the compulsiveness of the idea that the ultimate expression of the public interest is found in the actions taken by the state. It suggests rather the view that the public interest manifests wherever the public may reside, including in property. This conclusion requires taking a normative stance, but a minimal one; we merely extend the scope of values traditionally attached to property.\footnote{Arguably, cases such as {\it Modderklip} might be taken to suggest that the social function theory, as soon as it is applied for the purposes of normative assessment, will systematically guide us to conclude that owners are not entitled to as many benefits as would otherwise follow from their property rights. It is fortunate, therefore, that the entire remainder of the thesis will focus on economic development takings, where it will typically appear more natural to conclude the opposite. In these cases, on a common- sense understanding of justice, applying the social function theory will allow us to recognize a sense in which owners should receive {\it increased} protection and more benefits, as a consequence of how such interferences can prove particularly damaging, both to the owner and to the social fabric of democracy.} 

That said, in the case of {\it Modderklip} the court was clearly faced with a value conflict that it is hard to resolve by looking to traditional liberal values. If these apply equally to the squatters, we are left with deadlock rather than resolution. Indeed, this was also reflected in the outcome of the case, which did not resolve the matter, but merely concluded that the state had failed in its obligations towards both of the parties. What should the solution be in the end? Should the squatters be allowed to stay, following condemnation of Modderklip's land, or should alternative housing be provided so that the eviction order can be carried out? The answer requires us to resolve a normative conflict, and how to do so might not be obvious. Moreover, value pluralism suggests that we must be prepared to engage with multiple ways of looking at the matter. In the interest of stability of property as an institution, allowing the squatters to succeed in establishing lasting title to the land might be considered unwise. Against this pragmatic and largely technical value, one would have to consider the values of community and belonging that now attach the squatters to their new homes. These two values are largely incommensurable, and it is not clear how to choose between them.

Still, Alexander maintains that human flourishing provides an ``objective'' standard on which to approach dilemmas such as these. Moreover, he ``rejects the view that what is good or valuable for a person is determined entirely by that person's own evaluation of the matter''.\footcite[1263]{alexander14} Some things are good for people, Alexander argues, irrespectively of whether or not people know so themselves. Hence, it may perhaps be argued that what is truly good for Modderklip is to come to an arrangement with the squatters and the state, to resolve the problem amicably. Moreover, failure to do so might entitle the state to take action that would otherwise seem to undermine the stability of property. This, then, would be partly due to this being conducive also to the flourishing of the people behind Modderklip, not only the squatters.

That, clearly, might be derided as an overly intrusive and moralistic way to approach property law. More generally, as Alexander notes, the exact content of goodness is ``necessarily contestable''. It consists of a list of different values which are all open to dispute, both as to their relevance and their precise meaning.\footcite[1263]{alexander14} Alexander goes on to list some key values that he believes are central, but the list is not meant to be exhaustive.\footcite[1764-1776]{alexander14}

Among the key values that Alexander discusses, we find many core private values that are commonly seen as important goals for the institution of property. This includes values such as autonomy and self-determination, both of which will feature heavily later in this thesis. However, Alexander also considers several public values, such as equality, inclusiveness and community. These too will be important later, as I will draw on them in my own normative analysis of economic development takings. I will be particularly concerned with the value of {\it participation}, understood, following Alexander, in terms of its broad social function.\footcite[1275-1276]{alexander14}

In my view, this value is closely related to the value of democracy. Participation in local decision-making processes is the root which enables democracy to come to fruition at the regional and national level. Moreover, participation is a value that will give me occasion to make particular policy suggestions regarding the correct way to approach economic development takings. Devoting some time to discussing this value in the abstract will therefore be helpful.

Alexander traces the value of participation back to Aristotle and the republican tradition. He notes, however, that this tradition involves a notion of participation that is somewhat narrowly drawn. For thinkers in the republican tradition, participation tends to mean public participation, meaning people's engagement with the formal affairs of the polity.\footcite[1275]{alexander14} For Alexander, participation has a broader meaning, involving also the value of being included in a community. He writes:

\begin{quote}
We can understand participation more broadly as an aspect of inclusion. In this sense participation means belonging or membership, in a robust respect. Whether or not one actively participates in the formal affairs of the polity, one nevertheless participates in the life of the community if one experiences a sense of belonging as a member of that community.\footcite[1275]{alexander14}
\end{quote}

Importantly, participation in a community can have a crucial influence also on people's preferences and desires. In this way, it is also invariably relevant -- behind the scenes -- to any assessment of property that focuses on welfare, utility or public participation in the classical sense. As Alexander and Pe\~{n}alver put it, drawing on the work of Amartya Sen and Martha Nussbaum:\footcite{sen84,sen85,sen99,nussbaum00,nussbaum02}
\begin{quote}
The communities in which we find ourselves play crucial roles in the formation of our preferences, the extent of our expectations and the scope of our aspirations.\footcite[140]{alexander09}
\end{quote}
Therefore, for anyone adhering to welfarism, rational choice theory, utilitarianism or the like, neglecting the importance of community is not only normatively undesirable, it is also unjustified in an epistemic sense. In particular, it should be recognized as a descriptive fact that community is highly relevant to {\it any} normative theory that attempts to take into account the preferences and desires of individuals.\footnote{Again, I think Alexander and other theorists attempting to incorporate such ideas in property law could benefit from making this descriptive point separately, so as to enable it to be considered in isolation from the more contentious normative arguments they construct on the basis of it.} But Alexander and Pe\~{n}alver go further, by arguing that participation in a community should also be seen as an independent, irreducibly social, value, not merely as a determinant of individual preferences and a precondition for rational choice. They write:

\begin{quote}
Beyond nurturing the individual capabilities necessary for flourishing, communities of all varieties serve another, equally important function. Community is necessary to create and foster a certain sort of society, one that is characterized above all by just social relations within it. By ``just social relations'', we mean a society in which individuals can interact with each other in a manner consistent with norms of equality, dignity, respect, and justice as well as freedom and autonomy. Communities foster just relations with societies by shaping social norms, not simply individual interests.\footcite[140]{alexander09}
\end{quote}

This, I think, is a crucial aspect of participation. Moreover, it is one that it is hard, if at all possible, to incorporate in theories that take  preferences and other attributes of individuals as the basis upon which to reason about property. For instance, if people in a community come under pressure to sell their homes to a large commercial company that wishes to raze them in order to construct a shopping mall, it may be appropriate to consider this as an unjustifiable attack on their property rights. Importantly, this may be so {\it irrespectively} of what the individual owners themselves think they should do. If they are offered generous financial compensations for their homes, or are threatened by eminent domain, economic incentives might trump the value of social inclusion and participation for all or a majority of these owners. As a consequence, the community might decide to sell.  

Even so, in light of the value of community, it would be in order for planning authorities, maybe even the judiciary, to view such an  agreement as an {\it attack on their property}. It is clear, in particular, that by the sale of the land, the ``just social relations'' inhering in the community will be destroyed. The members of the community -- including all the non-owners -- will lose their ability to participate in those relations. More concretely, the nature of the property rights that once contributed to sustaining ``just relations'' will now be transformed into property rights that serve different purposes. This includes aiding the concentration of power and wealth in the hands of commercially powerful actors. Such a change in the social function of property might have to be regarded -- objectively speaking -- as a threat to participation, community and democracy. Hence, on the human flourishing theory, it is also a threat to property. Our property institutions, therefore, should protect against it.

To demonstrate the general significance of such a line of normative reasoning, it is illustrative to mention a scenario -- not directly implicating property -- that is currently beginning to attract much attention in legal scholarship. This scenario arises in relation to the right to {\it privacy}. This right, of course, is increasingly perceived to be coming under threat in the information age. Crucially, it is beginning to become clear to legal theorists that viewing privacy merely as a private right is not going to provide a sustainable template for dealing with this challenge.\footnote{See generally \cite{schafer14}.} It seems, in particular, that people are simply too willing to give it up. This, in turn, contributes to the formation of potentially harmful social structures on the web. In particular, the lack of privacy becomes an impediment to dignity, freedom and respect in web societies. In this way, both individuals and society as a whole will eventually suffer, although this truth is not reflected in our individual preferences. Hence, it has been proposed that privacy should be considered also as a {\it common good}, so that protecting the privacy of individuals, in some cases, is an imperative irrespectively of what these individuals themselves desire and prefer. Privacy, in this way, becomes also an obligation, mirroring the similar phenomenon that we have observed with respect to the right to property.

There is a subtle issue that arises on the basis of this kind of normative reasoning about individual rights. Is it appropriate, in particular, to still think of such reasoning -- and the obligations it gives rise to -- as an aspect of protecting individuals? Is it not more accurate to say that this is an {\it interference} with individual rights, undertaken to further the public interest? Indeed, when the individual himself does not want his property or privacy to be ``protected'', is it not somewhat perverse to insists that this is what is happening? 

I am inclined to answer in the negative. In my opinion, we are still talking about protecting individual rights, even when this means imposing protections on people that they themselves do not want. Undoubtedly, this is {\it also} an interference in their rights, but just as different rights of different people can sometimes come into conflict, I am inclined to think that the same right, for the same person, can sometimes come into conflict with itself. This happens, in particular, when it is not possible to simultaneously protect all those functions that this right seeks to promote. 

For instance, if someone protests a taking on environmental grounds and also rejects financial compensation as immoral, the courts should still award just compensation for the land, if they find that the taking is valid. If the owner wishes, he can purge himself by making a donation to charity. Similarly, if someone attempts to commit suicide, the health services are still obliged to help, even against the patients wishes. This remains the case, moreover, even though suicide is no longer considered a criminal offense in the public interest. 

Protecting individuals against their will is condescending, no doubt, but it is still different, and often preferable, from subordinating their interests to that of the general public. If the justification for an act of interference is a vague proclamation of the ``public interest'', the individual is marginalized from the very start. A balancing act might be required, but this renders the individual relevant only to one side of the equation. On the other hand, if the act of interference is simultaneously rendered as protection, enforcement of an obligation, or a measure to enable participation, the individual occupies center stage. In so far as the public interests triumph, it is not because the individual loses, but because the public is deemed to know best how to secure the goal of human flourishing, both for the individual herself and other members of the social structures that surrounds her.

For instance, external interests of both a private and a public nature can dictate that owners should avoid becoming a nuisance to their neighbors. But under a human flourishing theory, we are also able to portray this as a case of protecting the individual's membership in the community. The public does not ``side with the neighbors'', but undertakes measures to protect the relationship between the owner and his fellows. In my opinion, a conceptual approach to property law that makes this portrayal plausible is highly desirable. 

For a second example, consider situations when environmental concerns suggest imposing restrictions on what an owner is permitted to do with his land. This too can be rendered as an act of protecting property. But doing so requires the regulatory body to relate the interference positively to the individual's interests and obligations, to ensure that they avoid adopting a narrative where the regulation is rendered as an act of enforcing the will of unnamed others against the will of specific owners. In this way, public values and the public interest can be given considerable weight, but will have to be rendered less abstract. In particular, these interests must be related concretely to the social functions of the rights protecting the individuals interfered with. The baseline for assessment remains actual persons and their well-being, not some abstract ideal of ``goodness''. Moreover, implementation of the collective will becomes a guide towards human flourishing for a society of individuals, not a goal in itself.

An individual might well be offended if the state adopts this narrative and implements behavioral restrictions by declaring ``it's for your own good''. But, I would argue, that is exactly as it should be. Any restriction of individual freedom is an offense, but one that is sometimes appropriate. If this is conveyed to people with a marginalizing ``your interests are not as important as ours'', the response might well be silence. But beneath the silence we may find disinterested apathy, or worse: contempt and despair. The interference is no longer an insult, but this is not because it is any more convincing to proclaim that interference is ``necessary for the greater good''. Rather, the interference is no longer an insult because it fails to properly engage the individual at all. The role of the person interfered with becomes passive -- she becomes an obstacle that needs to be removed. If such a dynamic of governance develops, the individual might take from this the lesson that she is unimportant in the greater scheme of things, that her interests are subordinate to those of ``the others'', and that her voice is not meant to be heard.

This is normatively undesirable. It represents a situation when the social effect of interference might become detrimental to society itself, particularly to the institution of democracy. It damages its roots, namely the ``just social structures'' that Alexander identifies as being at the core of the human flourishing theory. A better alternative, then, is to interfere in a way that constructively targets the individual, aiming to protect her by enabling her -- and compelling her -- to protect others and partake in social and political life. This can then become interference aimed at bringing the individual into the fold, making her play her part, by raising her to fruitful citizenship. Such a paternal (or maternal) state is one that cares, but one that may also be overprotective, unfair, or plain stupid. Hence, it becomes natural to resist and to revolt, but not without also carrying forward care and love for the social, political and legal structures within which this agency is (hopefully) permitted to take place.

The upshot, I believe, is that condescension in property law can be a good thing. To conceptualize an act of restriction as a means to empower the persons restricted is something they might well find offensive, but it also renders interference more meaningful to them. It provides both a reason to take a more active role in relation to the interfering power, and a possible cause for constructive resistance. Importantly, it does not force the conclusion that the public resides behind closed doors, disinterested in what the affected individual have to offer. Instead, it is an approach that encourages a response, by focusing always on the persons interfered with, whenever interference is deemed necessary. This is the vision of a bottom-up, rather than a top-down, approach to imposing the collective will on individuals. I believe it has merit. 

It will remain in the background as I now move on to apply the theories discussed in this and preceding sections. In the next section,  I return to the issue that will remain in focus for the remainder of this thesis. First, I will introduce economic development takings by considering the seminal case of {\it Kelo v City of New London}\footcite{kelo05}, which brought this category to prominence in the US discourse on property law. Then I will assess the unique aspects of such takings against the social function theory, to provide an argument that the category has significance for legal reasoning in takings law, as well as with respect to property as a constitutionally protected human right. Finally, I will provide an abstract presentation of the values that I believe should be considered important when normatively assessing the law in this area. In doing so, I will draw on the human flourishing theory, setting out the main values that will inform the concrete policy assessments I provide later. 

\section{Economic development takings}\label{sec:edt}

Constitutional property rules in many jurisdictions indicate, with varying degrees of clarity, that eminent domain should only be used to take property either for ``public use', in the ``public interest'', or for a ``public purpose''. Such a restriction can be regarded as an unwritten rule of constitutional law, as in the UK, or it can be explicitly stated, as in the basic law of Germany.\footnote{See Chapter \ref{chap:2}. Section \ref{sec:contrast} below.} In some jurisdictions, for instance in the US and in Norway, explicit property clauses exist, but are not formulated clearly.\footnote{See Chapter \ref{chap:2}, Section \ref{sec:us} and Chapter 3, Section \ref{sec:norexp} below.}

Both the Norwegian and the US property clauses appear to refer to public use only as a precondition for the duty to pay compensation. However, they are also universally read as expressing the {\it presupposition} that the power of eminent domain is only to be used in the public interest.\footnote{In the literature, it is rare to even note that a different interpretation is linguistically possible. But see \cite[205]{berger78}.} Indeed, in cases when one might say that private property is ``taken'' for a non-public use without compensation, for instance in a divorce settlement, it is not commonly regarded as an exercise of eminent domain. Rather, it is justified by making reference to a different category of rules, meant to ensure enforcement of obligations that arise between private parties independently of the state's power to single out and compulsorily acquire specific properties.

The exact boundary between eminent domain and other forms of state interference in property may not always be clear, but I will not worry too much about it in this thesis. I note, moreover, that most, if not all, legal scholars seem to agree that the power of eminent domain is meant to be exercised in the public interest. However, differences of opinion emerge when we turn to the question of whether the presupposed public use or public interest in the property taken serves also to restrict the power to take. In the US, most scholars agree that some restriction is intended, but there is great disagreement about its extent.\footcite[205]{berger78} In Norway, on the other hand, a consensus has developed that the public use limitation is so wide that it hardly amounts to a restriction at all.\footnote{See, e.g., \cite[368]{aall10}.} Moreover, the courts defer almost completely to the assessments made by the executive branch regarding the purposes that may be used to justify a taking.\footcite[368]{aall10}

Some US scholars adopt a similar stance, but others argue that the public use presupposition should be read as a strict requirement, forbidding the use of eminent domain unless the public will make actual use of the property that is taken.\footnote{Compare \cite{bell06,bell09,claeys04,sandefur06}.} Most scholars fall in between these two extremes. They regard the public use restriction as an important, practically relevant, limitation, but they also emphasize that courts should normally defer to the legislature's assessment of what counts as a public use.\footnote{See, e.g., \cite{merrill86,alexander05}. The fact that US jurists usually stress deference to the legislature, not the executive branch, should be noted as a further contrast with Norway.}

As I discuss in more depth in Chapter \ref{chap:2}, Section \ref{sec:hop}, the debate in the US has its roots in case law developed by state courts -- the federal property clause was for a long time not enforced against states. This has changed, however, and today the Supreme Court has a leading role also in this area of US law. It has developed a largely deferential doctrine, resembling the understanding of the public use limitation under Norwegian law.\footnote{See \cite{berman54,midkiff84,kelo05}.} The difference is that in the US, cases raising the issue  still regularly arise, and still prove controversial. The most important such case in recent times was {\it Kelo}, decided by the Supreme Court in 2005.\footnote{kelo05} This case saw the public use question reach new heights of controversy in the US.\footnote{See, e.g., \cite{somin09}.}

{\it Kelo} centered around the legitimacy of taking property to implement a redevelopment plan that involved construction of research facilities for the drug company Pfizer. The home of Suzanne Kelo stood in the way of this plan, and the city decided to use the power of eminent domain to condemn it. Kelo protested, arguing that making room for a private research facility was not a permissible  `public use''. She was represented by the libertarian legal firm {\it Institute for Justice}, which had previously succeeded in overturning similar instances of eminent domain at the state level.\footnote{See \url{https://www.ij.org/cases/privateproperty}.} Kelo lost the case before the state courts, but the Supreme Court decided to take it on, and they looked at it in great detail.

The precedent set by earlier federal cases was clear: As long as the decision to condemn was ``rationally related to a conceivable public purpose'', it was to be regarded as consistent with the public use restriction.\footcite[241]{midkiff84} Moreover, the role of the judiciary in determining whether a taking was for a public purpose was regarded as ``extremely narrow''.\footcite[32]{berman54} It had even been held that deference to the legislature's public use determination was required ``unless the use be palpably without reasonable foundation'' or involved an ``impossibility''.\footnote{See \cite[66]{dominion25}; \cite[680]{gettysburg96}.}

This understanding had also been reflected in the outcome of concrete cases resembling the situation in {\it Kelo}: In {\it Hawaii}, the Supreme Court had upheld a taking that would benefit private parties, with no direct benefit to the public.\footnote{\cite{midkiff84}. For a more detailed discussion, see Chapter \ref{chap:2}, Section \ref{sec:hop} below.} In {\it Berman}, it had upheld a taking for economic redevelopment of a blighted area, even though the property taken was not itself blighted.\footnote{\cite{berman54}. For a more detailed discussion, see Chapter \ref{chap:2}, Section \ref{sec:hop}.} But in the case of {\it Kelo}, the court hesitated.

Part of the reason was no doubt that takings similar to {\it Kelo} had been heavily criticized at state level, with an impression taking hold across the US that eminent domain ``abuse'' was becoming a real problem.\footnote{See, e.g., \cite[667-669]{sandefur05}.} A symbolic case that had contributed to this worry was the infamous \textcite{poletown81}. In this case, General Motors had been allowed to raze a town to build a car factory, a decision that provoked outrage across the political spectrum.\footnote{See generally \cite{sandefur05}.} The case was similar to {\it Kelo} in that the taker was a powerful commercial actor who wanted to take homes. This, in particular, served to set the case apart from  {\it Hawaii}, which involved a taking in favor of tenants, and to some extent also {\it Berman}, which involved a taking of businesses (and homes) in the interest of combating blight. Moreover, the Michigan Supreme Court had recently decided to overturn {\it Poletown} in the case of \textcite{wayne04}. Hence, it seemed that the time had come for the Supreme Court to reexamine the public use questions.\footnote{See, e.g., \cite{sandefur05,claeys04}.}

Eventually, in a 5-4 vote, the court decided to apply existing precedent and held against Suzanne Kelo. The majority also made clear that economic development takings were indeed permitted under the public use restriction, also when the public benefit was indirect and a private company would benefit commercially.\footcite[469-470]{kelo05} The backlash of this decision was severe. According to Ilya Somin, the case ranks among the most disliked decision that the Court has ever made.\footcite[2]{somin11} Some 80 - 90 \% of the US public expressed great disapproval, with critical voices coming from across the political spectrum\footcite[2108-2110]{somin09} Why did the case prove so controversial? No doubt, the discontent with the decision was fueled in large part by the fact that it was seen as a case of the government siding with the rich and powerful, against ordinary people.\footnote{\cite[630-634]{baron07}} Indeed, the party that appeared to benefit the most from the taking was Pfizer -- a multi-billion dollar company -- while Suzanne Kelo, who stood to lose, was a middle class homeowner. In this context, the taking of Kelo's home seemed morally suspect, an act of favoritism showing disregard for less influential members of society.\footnote{See, e.g., \cite{underkuffler06}.}

In addition, it is worth noting that many commentators conceptualized the {\it Kelo} case by thinking of it as belonging to a special category, by describing it as an economic development taking, a {\it taking for profit}, or, more bluntly, a case of {\it Robin Hood in reverse}.\footcite{somin05} Categories such as these had no clear basis in the property discourse before {\it Kelo}. Indeed, in terms of established legal doctrine, it would be more appropriate to say that the case revolved entirely around the notion of ``public use''. 

However, when we consider the most common reasons given for condemning the outcome in {\it Kelo}, we readily grasp why critics felt it was natural to classify the case along an additional dimension. A survey of the literature shows that many critical voices made use of a combination of substantive and procedural arguments to  paint a bleak picture of the {\it context} surrounding the decision to take Kelo's home. Important concrete factors that critics tend to stress include the imbalance of power between the commercial company and the owner, the incommensurable nature of the opposing interests, the lack of regard for the owner displayed by the decision makers, the close relationship between the company and the government, and the feeling that the public benefit -- while perhaps not insignificant -- was made conditional on, and rendered subservient to, the commercial benefit that would be bestowed on the commercial beneficiary.\footnote{See, for instance, \cite{underkuffler06,somin07,sandefur06,cohen06,hafetz09,hudson10}.}  This dynamic, in which public bodies no longer seem to be leading and pushing the process forward, but are also -- to quite some extent -- being led and being pushed, is regarded as particularly suspicious. This, in turn, is derided as a perversion of legitimate decision-making, used to argue more broadly that economic development takings such as {\it Kelo} suffer from what I will refer to here as a {\it democratic deficit}.

From a theoretical point of view, I take all of this to suggest that many critics of {\it Kelo} effectively adopted a social function view on property, by paying close attention to the wider social and political context of the taking.\footnote{For a particularly clear example of this, see \cite{underkuffler06}.} Importantly, if we now turn to the social function theory of property, we are placed in a position to engage more actively with this form of reasoning, as an integrated part of our assessment of the law. This may then in turn give us cues as to how we should reason -- within the law -- to justify a departure from the course laid down by previous cases on the ``public use'' requirement, where such a perspective was not adopted. Indeed, it seems to me that this is exactly what the minority of the Supreme Court did, particularly Justice O'Connor, who formulated a strongly worded dissent.\footnote{\cite[494-505]{kelo05}. Justice O'Connor was joined by the four other dissenters, but Justice Thomas also formulated his own dissent, taking a more narrow view and arguing for the revival of a strict reading of the public use requirement, see \cite[505-523]{kelo05}.} She writes as follows:

\begin{quote}
Any property may now be taken for the benefit of another private party, but the fallout from this decision will not be random. The beneficiaries are likely to be those citizens with disproportionate influence and power in the political process, including large corporations and development firms. As for the victims, the government now has license to transfer property from those with fewer resources to those with more. The Founders cannot have intended this perverse result.\footcite[505]{kelo05}
\end{quote}

It seems to me that the values Justice O'Connor rely on in her assessment are closely related to the idea of human flourishing presented by Alexander and others, particularly those pertaining to the political function of property as an anchor for community and democracy. Indeed, the danger of powerful groups gaining control of the power of eminent domain does not only affect the individual entitlements of owners. It also affects society, as the economic rationality used to justify interference comes to result in an implicit political statement to the effect that the property of the rich and powerful is better protected, and valued higher by the state, than property owned by regular citizens, who reside in ordinary communities.

The effect of a traditional economic development taking is that property rights are transferred from the many to the few, taken from ordinary people and given to the powerful. Hence, these cases represent a possibly pernicious redistribution of property, not necessarily in financial terms -- depending on the level of compensation -- but surely in terms of property's social function. The structural imbalances of the condemnation process itself find permanent expression in the new distribution of property. The social structures of a living community are dismantled in favor of a social structure that revolves around the commercial interest of a company. The political and social power of the community is diminished, perhaps lost in its entirety, while the political and social power of the company increases.

It seems clear that to Justice O'Connor, this too is a negative consequence of the taking. Again, we notice that recognizing this effect requires a social function approach to property. There is no clearly quantifiable individual loss -- no one particular ``stick'' in the property bundle that is not compensated. Rather, it is the community itself that is lost, a community that was not directly implicated in any ``entitlement'', but which played a crucial role in providing meaning to the totality of the bundle enjoyed by the owner. Even if we extend our perspective to account for indirect individual losses, we are not doing justice to the loss in this regard. The owner might relocate, acquire new property with a similar meaning in a new community somewhere else. But that does not make up for the fact that {\it this} community is lost forever, as {\it this} property takes on new meanings and functions. The loss to Suzanne Kelo, therefore, might  even be a significant loss to the City of New London.

Of course, the economic and social gains of development might outweigh such negative effects on community. But, arguably, the balancing of interests required in this regard can only be carried out by an institution that sufficiently recognizes the owners' and their community's right to participation and self-governance. The presence of a highly active commercial third party, in particular, means that public participation in the standard sense might be insufficient. In economic development takings, the commercial company typically appears alongside the government, as a more or less integrated part of the institutional structure making the decision to condemn. The owners, however, do not enjoy a corresponding level of participation.

In particular, their interests are only negatively defined. They are adversely effected and may object, but under standard administrative regimes they play no constructive role in the process. For instance, they are not called on to take part in the development itself, or to assess its merits more broadly than by being asked to respond based on their own individual entitlements. In fact, I think this is one of the main problems with economic development takings. I will argue for this in more depth later, but I remark here that an important reason to focus on this aspect is that it involves precisely those values that economic development takings are most likely to offend against. In particular, if the loss of community outweighs the positive effect of economic development, this is unlikely to be recognized by a process that relies mainly on the positive contribution of the developer and the expert planners.\footnote{A similar point is made in \cite{underkuffler06}.} 

The objections made by owners, moreover, may not only be given too little weight given the imbalance of power between owners and developers. As long as owners themselves focus only on the individual loss, they may not get to those issues that are the most important for property's social function. However, I do not think it is sufficient to theoretically proclaim that these aspects need to be considered. To address the democratic deficit of economic development takings, it seems likely that institutional changes will have to be made, to give those functions a voice in the decision-making process. This should ensure greater involvement by the local community (including, perhaps, even non-owners) in the decision-making process relating to development. Not only should they be asked if they have objections. They should be be included in a constructive way, perhaps even be compelled to assume an active role in relation to the proposed project.

This is a proposal that envisages owners engaging directly with both government and potential developers, consider alternative schemes, and make their own proposals. In short, this asks for a system where owners participate as a community. According to the human flourishing theory as I understand it, this is not only a right, but also an obligation. It gives a plausible basis on which to strike down economic development takings, and to do so without giving up the value of judicial deference. In addition, it is a call for institutional reform, to search for new governance frameworks that will empower owners and their communities.

It seems to me that Justice O'Connor's argument reflects some of these ideas. Indeed, she seems to believe strongly that the taking of Kelo's home would be a particularly harmful interference in the ``just social structures'' surrounding it. Importantly, a piece-by-piece entitlement-based approach to {\it Kelo} could hardly justify the degree of disapproval seen in Justice O'Connor's opinion. After all, Kelo had been offered generous compensation, there had been no clear breach of concrete procedural rules, and the claim that the taking was {\it only} a pretext to bestow a benefit on Pfizer did not seem supported by the facts.\footnote{See \cite{bell06}.} Rather, it was the overall character of the taking that could be used to argue that it was illegitimate. In this picture, moreover, the perceived lack of a clearly identifiable and direct public benefit becomes only one of several factors.

In addition, the institutional, social and political aspects of the case come into focus. The economic implications are less important to Justice O'Connor. Even the importance of homeownership to personhood does not receive the same attention as structural aspects. The problem which overshadows everything else is the concern that economic development takings represent a form of governmental interference in property that might come to systematically favor the rich and powerful to the detriment of the less resourceful. Hence, such takings may help establish and sustain patterns of inequality. Hardly anyone would openly regard this as desirable; it is not hard to agree that if Justice O'Connor's predictions about the fallout of {\it Kelo} are correct, then this is indeed be ``perverse''. 

The question, of course, is whether her predictions are warranted. This is a call for empirical and contextual assessment of economic development takings, to help us gain a better understanding of how they actual affect political, social and bureaucratic processes. In addition, it raises the question of how to {\it avoid} negative effects, that is, how to design rules and procedures that can reduce the democratic deficit of economic development takings. As I now move away from theory towards concrete assessment of economic development takings, both these questions will be in focus.

\section{Conclusion}

In this chapter, I have presented the core notion of my thesis, that of an economic development taking. I started by noting that while the notion is straightforward enough to define factually, it is far from obvious what implications it has for legal reasoning. I illustrated the subtleties involved by considering a concrete example of a commercial scheme that looked like it might well result in compulsory acquisition of land, namely Donald Trump's controversial plans to develop a golf course on a site of special scientific interest close to Aberdeen, Scotland. In the end, the plans did {\it not} require takings, as Trump was able to make creative use of property rights he acquired voluntarily, against the complaints of recalcitrant neighbors.

This turn of events made the example even more relevant to the points I have been trying to make in this chapter. It served to highlight, in particular, that the question studied in this thesis is not a black-and-white issue that sees privileged property rights enthusiasts on one side of the equation balanced against the good will of the regulatory state on the other. Rather, the example of Trump's golf course allowed me to emphasize the importance of context when assessing both the nature of property rights and the meaning of protecting them. In particular, to protect the property rights of those opposing Trump's golf course was not about protecting just any property, it was about protecting the property of members in a local community that felt it would be detrimental to this community, and to their lives, if Trump was allowed to redefine it. In particular, after Trump decided not to pursue compulsory purchase, protecting the property of these members of the community became a question of {\it restricting} the degree of dominion that Trump could exercise over his own property. Hence, under a conventional and overly simplistic way of looking at these matters, protecting property then became tantamount to restricting its use, a seeming paradox.

To resolve this paradox, and to arrive at a better conceptual understanding of economic development takings, I looked to various theories of property. I noted that there are differences between civil law and common law theorizing about property, but I concluded that these differences are not particularly relevant to the questions studied in this thesis. In particular, I observed that neither the bundle theory, dominant in the common law world, nor the dominion theory, used by civil law jurists, helped me clarify economic development takings as a category of legal thought.

I then went on to consider more sophisticated accounts of property, noting that a range of different {\it normative} theories have been proposed. These differ with respect to the values that they think the institution of property should promote, and as such they were also relevant to the question of assessing economic development takings. However, they do not allow us to zoom in on such takings in a more value-neutral way, to argue that regardless of one's normative persuasions, one should acknowledge that they deserve special attention.

I argued that in order to make this point successfully, the traditional entitlements-based perspective on property had to be abandoned. Instead, I looked to the social function theory of property, which encourages us to take a more contextual perspective on rights and obligations inherent in property. In particular, I noted that the social function theory compels us to recognize the importance of property in regulating social and political relations. Hence, economic development takings are special because they redefine the meaning of the property that is taken and cause a lasting disturbance to the established economic, social and political relationships that exist between owners, communities, state bodies, and commercial actors. The social function theory asks us to acknowledge that property rules are hardly ever neutral with regards to such effects. I identified this as the key observation that allowed me to make sense of economic development takings as category of legal reasoning.

After concluding that the social function theory allowed me to formulate a coherent conceptual basis for studying such takings, I went on to argue that in the first instance, the theory should be understood as giving us purely {\it descriptive} insights into the workings of property and its role in the legal order. In this, I advanced a different stance than many property scholars, by arguing that it would be better to decouple the more normative aspects of the theory, to allow the social function theory to serve as a common ground for further value-based debate.

I then went on to clarify my own starting point for engaging in such debate, by expressing support for the human flourishing theory proposed by Alexander and Pe\~{n}alver. I noted that this theory focuses on how property enables communities and individuals to  participate in social and political processes. I argued that protecting this function of property was good, and that this value should be considered fundamental in property law. Moreover, I noted that the human flourishing theory also contains a further important insight, concerning the scope of the state's power to protect. In particular, the theory asks us to recognize that protecting property against interference that is harmful to human flourishing is a responsibility that the state has even in cases when the individual owners themselves neglect to defend their property, for instance because of financial incentives to remain idle. In other words, some functions of property are such that owners have an obligation to preserve them, while the state has a duty to protect them, potentially even against the will of the owners.

After this, I went on to provide some introductory remarks on economic development takings, drawing on the theoretical insights collected from preceding sections. To make the discussion concrete, I considered the case of {\it Kelo}, which propelled the notion of an economic development taking to the front of the takings debate in the US. I focused particularly on the dissenting opinion of Justice O'Connor, and I argued that she approached the issue in a way that is consistent with the theoretical basis proposed in this chapter.

I will now go on to make my analysis of economic development takings more concrete, by considering how such takings are dealt with in Europe and the US respectively. I note that the category has yet to receive much attention in Europe, so the discussion focuses on the US. Here, the attention this issues has received after {\it Kelo} has been staggering. To get a broader basis upon which to asses all the various arguments that have been presented, I consider the historical background to the issue as it is discussed in the US. This involves giving a detailed presentation of the public use restriction, as it was developed in case law from the states during in the 19th and early 20th century. I then connect this discussion with recent proposals to deal with economic development takings, responding to the backlash of {\it Kelo}, by aiming to address the democratic deficit of such takings.

Later, when I begin to consider the law relating to Norwegian hydropower, I will look back at the theoretical basis provided in the present chapter to guide the analysis. In particular, I focus on certain decision-making mechanisms that have developed on the ground in Norway, as a practical response to the increased tendency for local owners to engage in hydropower development. I will argue that this shows the conceptual strength of the idea that property is irreducibly embedded in community, and that its meaning and function is not -- and should not -- be ordained from above, but should be allowed to arise from its grassroots through continuously evolving institutions of participatory democracy.

%If property rights, particularly rights to land, are distributed fairly in a local community, property is not a privilege. Even if most people do not hold land rights, as long as no one holds excessive amounts, there is no reason why owners and non-owners should not be on equal footing in the local community. They are mutually dependent on one another; non-owners need access to natural resources, while owners need access to services. Moreover, the bonds of community will tend to ensure that owners are deterred from engaging in exploitative practices towards non-owners in much the same way as non-owners are deterred from undermining property rights. 
%Tensions reached new heights following the Supreme Court case of {\it Kelo v City of New London}.\footcite{kelo05}  The company Pfizer was allowed to expropriate homes for the construction of new research facilities, and the questions arose as to whether or not this constituted "public use" in the sense of the property clause in the Fifth Amendment to the US Constitution.  The majority 5-4 found that the expropriation was constitutional, but the decision was controversial. Arguably, the attitude following {\it Kelo} has shifted towards a greater feeling of unease regarding economic takings, both among legal scholars and members of the general public.

The case has also had a great impact on academic writing on takings law in the US, where economic development cases are now usually viewed as a distinct sub-class of takings which merit particular attention. Moreover, a consensus seems to be emerging that there is a need for novel approaches to deal with such takings, possibly even new legal frameworks to resolve the tensions that typically arise. Some authors argue for a simple solution: an outright ban on economic development takings. However, the majority of scholars take a more measured approach, recognizing the need for legal frameworks that can be used to promote development projects without making illegitimate use of compulsion against land owners.


\chapter{Taking Property for Profit}\label{chap:2}

In the previous chapter, I argued that economic development takings should be considered a separate category of interference in private property. I also placed it in the theoretical landscape, by relating it to the social function theory of property. Economic development takings, I argued, raise questions about the overall effect of interference, questions that require contextual assessment. In particular, I argued that they require us to depart from the individualistic, entitlements-based approach that otherwise dominates in property law.

The significance of a new conceptual category should not be overstated. While I think the economic development label is a very helpful tool when thinking about certain takings cases, I am not suggesting that these cases should be approached uniformly and on the basis of mechanical legal assessment. Rather, the importance of context indicates that a concrete approach is in order. Moreover, while I have argued for a certain way of reasoning about economic development takings, I have so far said little about what the law has to say about them. In this chapter, I consider this question, by giving an overview of how economic development cases are dealt with in some more or less representative jurisdictions.

First, I will comment briefly on the importance of economic development takings on the global stage.

\section{Introduction}

Public-private partnerships are becoming increasingly important to the world economic order.\footnote{See generally \cite{saussier13}.} To some, they are the illegitimate children of privatization and deregulation, while others see them as efforts to make the public sector more efficient and accountable. Either way, their numbers are growing, and they appear to be here to stay.\footnote{Although their potentially pernicious effects on stability and accountability has also been noted. See, e.g., \cite{baker03} (arguing that ``the Enron scandal can be better understood as an American form of public private partnership rather than just another example of capitalism run amok'').} In this situation, it is inevitable that when eminent domain is used to acquire property for economic development, those who directly benefit will often be commercial companies rather than public bodies. In the previous chapter, I pointed out how indirect public benefits are typically used to justify such takings. Standard legitimizing reasons include the prospect of new jobs, increased tax revenues, and various other economic and social ripple effects. However, as I have indicated, economic development takings have a tendency to result in controversy.

In the US after {\it Kelo}, they have also been at the forefront of the constitutional property debate. In the rest of the world, a similar shift in academic outlook has yet to take place, but expropriation-for-profit situations are increasingly coming into focus here as well.\footnote{See, e.g., \cite{gray11,waring13,verstappen14}.} If we lift our perspective slightly, to consider commercially motivated interference more generally, it even seems appropriate to speak of a crisis of confidence in property law, particularly in relation to land rights. This is most clearly felt in the developing world, where egalitarian systems of property use and ownership are coming under increasing pressure. It has been noted, in particular, that large-scale commercial actors are assuming control over an increasing share of the world's land rights, a phenomenon known as {\it land grabbing}.\footnote{See generally \cite{borras11}.} 

So far, most research on land grabbing has looked at how commercial interests, often cooperating with nation states, exploit weaknesses of local property institutions, to acquire land voluntarily, or from those who lack formal title. However, the danger of {\it Kelo}-type reasoning has also been recognized. In particular, it has been noted how the purported public interest in economic development can be used to justify land grabs that would otherwise appear unjustifiable. In a recent article, Smita Narula cites {\it Kelo} directly and warns that procedural safeguards alone might not provide sufficient protection against abuse. She writes:
\begin{quote}
Procedural safeguards, however, can all too easily be co-opted by a state because its claims about what constitutes a public purpose may not be easy to contest. Particularly within the context of land investments, states could use the very general and under-scrutinized language of ``economic development'' to justify takings in the public interest.\footcite[157]{narula13}
\end{quote}

This underscores the broader relevance of the study of economic development takings. In addition, it reminds us that the question of what can be justified in the name of ``economic development'' is a general one, not confined to particular systems for organizing property rights. To address this, and to restore confidence in the institution of property more generally, many turn towards {\it human rights}. These scholars argue that a human right to land should be recognized on the international stage, a right that would apply even when those most affected by a land grab lack formal title.\footnote{See generally \cite{schutter10,schutter11,kunnerman13}.} If successful, this approach promises to deliver basic protection against interference in established patterns of property use independently of how particular jurisdictions approach property.

In Europe, a human rights perspective is already of great practical significance due to the European Convention of Human Rights (ECHR) and the court in Strasbourg (ECtHR). But, of course, in the context of land grabbing, protecting land rights is not primarily a question of protecting the civil law ideal of individual dominion. Rather, it is a question of providing protection against large-scale transactions that destabilize or destroy established patterns of land use, to the detriment of local communities. Nevertheless, the questions raised by the public interest  narrative -- and the notion of ``economic development'' in particular -- must be expected to arise in much the same way as in cases when formal title is acquired following a state-authorized taking.

Hence, it is somewhat surprising that the special category of for-profit takings has not received more attention from the point of view of human rights law. In human rights discourse, the focus tends to be rather on fairness and proportionality as broad benchmarks, in addition to specific values related to food security and protection of livelihoods that arise with particular urgency in the context of third-world land grabs. But how to achieve effective protection depends as much on the development of firm categories and enforcible legal principles as it does on broad benchmarks and good intentions. In this regard, I think Narula is right to stress that the lack of a convincing approach to the notion of ``economic development'' is a crucial challenge.

On the one hand, economic development is no doubt a sound overreaching goal, particularly for poor nations. But at the same time, the risk of abuse is obvious when such a vague term is used to justify dramatic interferences in property. After all, interferences in property can cause severe disturbances in people's life. This, moreover, is true for middle-class US homeowner in much the same way as it is true for members of self-sustaining agrarian communities in Africa, although the stakes might be much higher for the latter.

As illustrated by {\it Kelo}, deep conflicts can arise in this regard also in developed democracies with long established systems of formal title. In the following, I will attempt to shed further light on the issue as it arises in such legal systems, without considering the additional complications that arise when property itself is -- formally speaking -- a more fluid concept. I note, however, that according to the social function view of property, there is no need to view formally recognized property rights as completely distinct from rights arising from property use. The two are intertwined and the difference between them is at most a matter of degree.\footnote{Moreover, if the human flourishing account of property values is successfully developed, there should even be hope that a unified normative treatment can be given at some point.}

However, my case study will look to Norwegian law, a prosperous European country with a long tradition of formal title to land. Hence, it is prudent to narrow down the discussion here by focusing on similar jurisdictions.\footnote{The relation with third-world land grabbing is a highly interesting question for future work.} I will do so now, beginning with a brief look at English and German law, to illustrate that there are great differences in how different European jurisdictions think about property in general, and takings in particular. Then I turn my attention to the ECHR and I focus on presenting the proportionality test that is now at the core of property adjudication at the ECtHR.

Following this, I move on to consider the US in greater depth, both the historical debate that led to {\it Kelo} and the suggestions for reform that have emerged following its backlash. A closer look is necessary because of the sheer magnitude of writing on this issue in the US. Moreover, while much of it is repetitive and coloured by the tense political climate, I believe some historical points, as well as some recent suggestions for reform, are highly relevant also to the international setting. To single out and analyse those aspects is the main aim of this part of the chapter. Indeed, the current debating climate in the US might be an indication of what is to come also in Europe, if concerns about the legitimacy of economic development takings are not taken seriously.

%I also highlight what I believe to be a connection between the situation in the US leading up to {\it Kelo} and the present situation in Europe, illustrated by the fact that the European Court of Human Rights is now explicitly endorsing ``stronger protection'' of property rights.  I attempt to identify the reasons behind calls for a stricter approach, arguing that it is connected to the fact that interferences in property under modern regulatory regimes is sanctioned in wide a range of different circumstances, serving to undermine their status as a necessary burden imposed on owner's according to the will of the greater public. In some cases, rather, takings appear to both owners and the public as improperly motivated and socially and politically unfair. I note that this happens particularly often in economic development cases, when commercial actors benefit to the detriment of local communities. I go on to list some concrete issues that arise with respect to such takings and that have been flagged as problematic in the literature.
%
%Following up on this, I consider various proposals that have been made to resolve tensions and limit the possibility of abuse in economic development cases. The differences of opinion that have been expressed in this regard have been quite substantial, and proposals have ranged from suggesting an outright ban on economic development takings  (Somin 2007; Cohen 2006) to suggesting that the best way forward is to reassess principles for awarding compensation in such cases (Householder 2007; Lehavi and Licht 2007).

%Much of the current theory focus on assessing traditional judicial safeguards that courts can rely on to prevent abuses, pertaining primarily to the material assessment of proportionality, public purpose, and compensation. 

%In the last part of the chapter, I will focus on a very interesting strand of recent work in the US, which shifts attention towards procedural rules that can help address the worry that economic development takings tend to suffer from a democratic deficit. The core concern is that the manner in which eminent domain decisions are typically made, and the way in which owners are compensated, might be unsuitable for economic development cases. Importantly, the need for special procedures has been noted, to restore legitimacy.\footnote{See generally \cite{lehavi07,heller08}.} This ties the US debate even closer to the European context, where proportionality, not public use, has become the key notion in property protection. Several recent suggestions from the US can be conceptualized as suggestions that aim to secure fairness and proportionality, while paying less attention to the formalistic question of what constitutes a ``public use''.
%
%%Also, it allows us to be very clear about a special concern that arises for economic takings cases: under current regulatory regimes, the government and the developer together often dominate the decision-making process completely, leaving the property owners marginalized. Hence, there is often a {\it democratic deficit} in such cases, resulting in discontent and a feeling that the taking is not in the public interest at all. Importantly, some recent writers hypothesize that if the proper balance can be restored in the decision-making process, so will the decision reached appear more legitimate, also with respect to the public use clause. In my opinion, this idea is crucial, and together with the question of compensation, which raises a similar structural problem, it will guide the rest of the work done in this thesis. 
%

In response to that worry, this chapter aims to  bring into focus the following key question: What principles can be used to ensure meaningful participation and just compensation in economic takings cases, without hindering socially and economically desirable development projects? The tentative answers provided in Section \ref{sec:ir} will set the stage for the remainder of the thesis, where they will be assessed in depth against the case study of Norwegian hydropower.

%In particular, I will consider two special semi-judicial procedural systems used in such cases in Norway, one targeting compensation following expropriation, and another used as an alternative to expropriation, particularly in cases when development requires cooperation among many owners.

%I conclude by arguing that approaches along procedural lines represent the best way forward in relation to addressing issues associated with economic development takings. This raises the following problem, however: what procedural principles can be used to ensure meaningful participation, without hindering socially and economically desirable development projects? This question sets the stage for the remainder of my thesis, where I conduct a case study of expropriation for the development of hydro-power in Norway. In particular, I will consider two special semi-judicial procedural systems used in such cases in Norway, one targeting compensation following expropriation, and another used as an alternative to expropriation, particularly in cases when development requires cooperation among many owners.

\section{A European contrast}\label{sec:contrast}

Economic development takings have not become as controversial in Europe as they are in the US, but there have been cases where the issue has come up, in several different jurisdictions.\footnote{For instance, in the UK, Ireland and Germany, as well as in Norway and Sweden. See \cite[466-483]{walt11}; \cite{stenseth10}.} The European Convention of Human Rights (ECHR) contains a property clause in Article 1 of Protocol No 1 (P1(1)), but the legitimacy of economic development takings has not yet been discussed in case law from the European Court of Human Rights (ECtHR). However, it is interesting to analyse cases like {\it Kelo} against P1(1), particularly since the ECtHR has developed a doctrine that focuses on ``proportionality'' and ``fairness'' rather than the purpose of interference.\footnote{See generally, \cite[Chapter 5]{allen05}. This approach may become even more significant as a source of property protection in the future, as the ECtHR have indicated that there are ``jurisprudential developments in the direction of a stronger protection under Article 1 of Protocol No. 1'', see \cite[135]{lindheim12}.}

The fundamental question raised by economic development takings can be formulated independently of specific property clauses as follows: When, if ever, is it permissible for governments to order compulsory transfer of property rights from citizens to for-profit legal persons in order to facilitate economic development?

In this section, I address economic development takings from the point of view of European sources. I first contrast English and German law, to show that there are significant differences between European jurisdictions in this regard. I then go on to give a more detailed presentation of the unifying property clause in P1(1) of the ECHR. The case law from the ECtHR is presented and analysed in some depth, in an effort to assess how the ECtHR would be likely to approach an economic development case such as {\it Kelo}. In particular, I argue that the ``proportionality'' doctrine offers an interesting approach to economic development cases. This doctrine stipulates that a ``fair balance'' must be struck  between the interests of the property owner and the public.\footcite[Chapter 5]{allen05} I argue that such a perspective could make it easier to get to the heart of why economic development takings are often seen as problematic, without getting lost in theoretical discussions about the meaning of  terms like ``public use'' or ``public purpose''. However, I also raise the concern that the ECtHR is not the appropriate institution for applying the proportionality test concretely. Its remoteness suggests that we should also look for more locally grounded legitimacy-enhancing institutions. Such institutions will likely be better able to assess the fairness of interference in context.

I go on to discuss whether existing government institutions can serve this purpose, arguing that local courts may well be the best candidates. However, I argue that active application of the ``proportionality''-doctrine in property cases has not yet developed fully at the local level. I also discuss possible shortcomings of local courts; as judicial bodies they are not intrinsically well-suited to carry out the kind of assessment that is required. Hence, I suggest that entirely new institutional proposals might be in order. I conclude by arguing that once the need for local grounding is recognized and met, the ECtHR has the potential to play an important and constrictive role in providing oversight and developing basic principles.

\subsection{England}\label{sec:england}

In England, the principle of parliamentary supremacy and the lack of a written constitutional property clause has led to expropriation being discussed mostly as a matter of administration and property law, not as a constitutional issue.\footcite{taggart98} Moreover, the use of compulsory purchase -- the term most often used to denote takings in the UK -- has not been restricted to particular purposes as a matter of principle. The uses that can warrant compulsory alienation of property are those that parliament regard as worthy of such consideration. However, as private property itself has long been recognized as a fundamental right, the power of compulsory purchase has typically been exercised with great caution. 

In his {\it Commentaries on English Law}, William Blackstone famously described property as the ``third absolute right'' that was ``inherent in every Englishman''.\footcite[134-135]{blackstone79}  Moreover, Blackstone expressed a very restrictive view on the possibility of expropriation, arguing that it was only for the legislature to interfere with property rights. He warned against the dangers of allowing private individuals, or even public tribunals, to be the judge of whether or not the ``common good'' could justify it. Blackstone went as far as to say that the public good was ``in nothing more invested'' than the protection of private property.\footcite[134-135]{blackstone79}

Historically, Blackstone's description conveys a largely accurate impression of takings practice in England. Indeed, Parliament itself would usually be the granting authority in expropriation cases, through so-called {\it private Acts}. Hence, compulsory purchase would not take place unless it had been discussed at the highest level of government. Moreover, the procedure followed by parliament in such cases strongly resembled a judicial procedure; the interested parties were given an opportunity to present their case to parliament committees that would then decide whether or not compulsion was warranted.\footnote{See \cite[13-16]{allen00}. While this procedure clearly reflected a protective attitude towards private property, recent scholarship has pointed out that expropriation was actually used more actively in Britain following the glorious revolution, see \cite{hoppit11}.} 

On the one hand, the direct involvement of parliament in the decision-making process reflected the fundamental respect for property rights that permeated the system. But at the same time, parliamentary supremacy also meant that the question of legitimacy was rendered mute as soon as compulsory purchase powers had been granted. The courts were not in a position to scrutinize takings at all, much less second-guess parliament as to whether or not the taking was for a legitimate purpose.

Eventually, an overworked parliament developed procedures for dealing more expeditiously with takings cases, and during the 19th Century, as an industrial economy developed, private Acts granting commercial companies the power to take land grew massively in scope and importance.\footnote{See \cite[204]{allen00}.} Private railway companies, in particular, regularly benefited from such Acts.\footnote{\cite[204]{allen00}. See generally \cite{kostal97}.} During this time, the expanding scope of private-to-private transfers for economic development lead to quite a bit of political debate and controversy. Usually, it would attract particular opposition from the House of Lords. Interestingly, this opposition was not only based on a desire to protect individual property owners. It also often reflected concerns about the cultural and social consequences of changed patterns of land use.\footcite[204]{allen00} 

Hence, the early debate on economic development takings in the UK shows some reflection of a contextual approach to property protection. However, as society itself changed dramatically following increasing industrialization, an expansive approach to compulsory purchase eventually emerged triumphant. At the same time, the idea that economic development could justify takings gradually became less controversial. 

Today, the law on compulsory purchase in England is regulated in statute and the role of courts is to a large extent limited to the application and interpretation of statutory rules. Some common law rules still play an important role, such as the {\it Pointe Gourde} rule discussed in more depth in Chapter \ref{chap:5}. With respect to the question of legitimacy, however, the starting point for English courts is that this is a matter of ordinary administrative law. 

More recently, the \cite{hra98} adds to this picture, since it incorporates the property clause in P1(1) into English law. But even so, the usual approach for English courts is to judge objections against compulsory purchase orders on the basis of the statutes that warrant them, rather than constitutional or human rights principles that protect property.\footnote{The important statutes are the \cite{ala81}, the \cite{lca61}, the \cite{tcpa90} and the \cite{pcpa04}. Acquisition of Land Act 1981, the Land Compensation Act 1961, the Town and Country Planning Act 1990 and the Planning and Compulsory Purchase Act 2004.} It is typical for statutory authorities to include standard reservations to the effect that some public benefit must be identified in order to justify a CPO, but the scope of what constitutes a legitimate purpose can be very wide. For instance, to warrant a taking under the \cite{tcpa90}, it is enough that it ``facilitates the carrying out of development, redevelopment and improvement on or in relation to the land''.\footcite[226]{tcpa90} 

While various governmental bodies are authorised to issue compulsory purchase orders (CPOs), a CPO typically has to be confirmed by a government minister. The affected owners are given a chance to comment and if there are objections, a public inquiry is typically held. The inspector responsible for the inquiry then reports to the relevant government minister, who makes the final decision about whether or not it should be granted, and on what terms. The CPO may then be challenged in court, but will usually only be scrutinized on the basis of whether or not it lies within the scope of the statute authorizing it. Hence, the discussion and evaluation at court is firmly grounded in statutory rules rather than constitutional principles.

That said, the idea that property may only be compulsorily acquired when the public stands to benefit permeates the system. Indeed, this has also been regarded as a constitutional principle, for instance by Lord Denning in {\it Prest v Secretary of State for Wales}.\footcite{prest82} He said:

\begin{quote}
It is clear that no minister or public authority can acquire any land compulsorily except the power to do so be given by Parliament: and Parliament only grants it, or should only grant it, when it is necessary in the public interest. In any case, therefore, where the scales are evenly balanced – for or against compulsory acquisition – the decision – by whomsoever it is made – should come down against compulsory acquisition. I regard it as a principle of our constitutional law that no citizen is to be deprived of his land by any public authority against his will, unless it is expressly authorised by Parliament and the public interest decisively so demands. If there is any reasonable doubt on the matter, the balance must be resolved in favour of the citizen.\footcite[198]{prest82}
\end{quote}

Lord Denning also supported the doctrine of necessity, as expressed by Forbes J in {\it Brown v Secretary for the Environment}:\footcite{brown78}

\begin{quote}It seems to me that there is a very long and respectable tradition for the view that an authority that seeks to dispossess a citizen of his land must do so by showing that it is necessary, in order to exercise the powers for the purposes of the Act under which the compulsory purchase order is made, that the acquiring authority should have authorisation to acquire the land in question.\footcite[291]{brown78}
\end{quote}

In practice, these principles are mostly implicit in legal reasoning, as a factor that influences the courts when they interpret statutory rules and carry out judicial review of administrative decisions. As Watkins LJ stated in {\it Prest}:

\begin{quote}
The taking of a person's land against his will is a serious invasion of his proprietary rights. The use of statutory authority for the destruction of those rights requires to be most carefully scrutinised. The courts must be vigilant to see to it that that authority is not abused. It must not be used unless it is clear that the Secretary of State has allowed those rights to be violated by a decision based upon the right legal principles, adequate evidence and proper consideration of the factor which sways his mind into confirmation of the order sought.\footcite[211-212]{prest82}
\end{quote}

In {\it R v Secretary of State for Transport, ex p de Rothschild}, Slade LJ referred to the judgment and made clear that he did not regard it as expressing a rule concerning the burden of proof in compulsory purchase cases, but rather as more general expressions about the severity of the interference and the importance of vigilance in such cases.\footnote{rothschild89} He said that they provided ``a warning that, in cases where a compulsory purchase order is under challenge, the draconian nature of the order will itself render it more vulnerable to successful challenge''.\footcite[938]{rothschild89}

A nice example of how these sentiments influence the assessment of legitimacy of takings, showing how it is applied in economic development cases, can be found in the recent case of {\it Regina (Sainsbury’s Supermarkets Ltd) v Wolverhampton City Council}.\footcite{sainsbury10} Here a CPO was granted to allow the company Tesco to acquire land from its competitor Sainsbury, in a situation when they were both competing for licenses to undertake commercial development on this land. The decisive factor that had led the local authorities to grant the CPO was that Tesco had offered to develop a different property in the same local area, which was currently in need of regeneration. 

Sainsbury protested, arguing that the local council could not strike such a deal on the use of its compulsory purchase power. It was argued, moreover, that taking the land for incidental benefits resulting from development in a different part of town was not legitimate under the Town and Country Planning Act 1990. The UK Supreme Court agreed 4-3, with Lord Walker in particular emphasizing the need for heightened judicial scrutiny in cases of private-to-private cases for economic development.\footcite[80-84]{sainsbury10} Lord Walker even cited {\it Kelo}, to further substantiate the need for a stricter standard in such cases.\footcite[81]{sainsbury10} 

However, the main line of reasoning adopted by the majority was based on an interpretation of the Town and Country Planning Act itself. In particular, the majority held that it was improper for the local council to take into consideration the development that Tesco had committed itself to carry out on a different site.\footcite[73-79]{sainsbury10} This, in particular, was not ``improvement on or in relation to the land'', as required by the Act.\footcite[336]{tcpa90} In addition, Lord Collins, who led the majority, said that ``the question of what is a material (or relevant) consideration is a question of law, but the weight to be given to it is a matter for the decision maker''.\footcite[70]{sainsbury10} Hence, the general importance of the decision for economic development cases is unclear.

Still, it is interesting to see how the purpose of the interference featured in the Supreme Court's interpretation and application of the statutory rules. The opinion of Lord Walker is particularly interesting, since he stresses that ``The land is to end up, not in public ownership and used for public purposes, but in private ownership and used for a variety of purposes, mainly retail and residential.''\footcite[81]{sainsbury10} He goes on to state that ``economic regeneration brought about by urban redevelopment is no doubt a public good, but ``private to private'' acquisitions by compulsory purchase may also produce large profits for powerful business interests, and courts rightly regard them as particularly sensitive.``\footcite[81]{sainsbury10}

Lord Walker then makes clear that he does not think it is impermissible, as such, for the local council to take into consideration positive effects on the local area, even when these do not directly result from the planned use of the land that is being acquired. Instead, he relies explicitly on the for-profit character of the taking, by arguing that ``the exercise of powers of compulsory acquisition, especially in a ``private to private'' acquisition, amounts to a serious invasion of the current owner's proprietary rights. The local authority has a direct financial interest in the matter, and not merely a general interest (as local planning authority) in the betterment and well-being of its area. A stricter approach is therefore called for.''\footcite[84]{sainsbury10} 

Lord Walker's opinion might indicate that the narrative of economic development takings is about to find its way into English case law. Moreover, a more critical approach might be adopted in the future, when compulsory purchase powers are made available to commercial companies wishing to undertake for-profit schemes. However, for schemes where the commercial aspect appears less dominant, English courts still appear very reluctant to quash CPOs, also when the purpose is economic development. This is so even in situations when the owners have requested a stricter standard of review on the basis of human rights law. 

For instance, in the case of {\it Smith \& Others v Secretary of State for Trade and Industry}, a caravan site was compulsorily acquired for development in connection with the London Olympic Games.\footcite{smith08} Some of the owners protested, including Romani Gypsies who used the caravans as their primary residence. A public inquiry was held, after which the inspector recommended that the CPO should not be confirmed until adequate relocation sites had been identified. However, due to the ``urgency, timing and importance '' of the project, the Secretary of State decided to go ahead before a relocation scheme was put in place (although he expressed commitment to ensuring satisfactory relocation).\footcite[10]{smith08} The owners argued that without satisfactory relocation plans, the interference in the property rights was not proportional and had to be struck down on the basis of human rights law, in particular Article 8 in the ECHR regarding respect for the home and private life.\footcite[27-51]{smith08}

The Court of Appeal considered the matter in great depth, applying the doctrine of proportionality developed at the ECtHR, which goes beyond the standard room for judicial review of administrative decisions under English law. However, the Court still concluded that the taking was proportional. This was largely based on the finding that ``the issue of proportionality has to be judged against the background that everyone accepts that an overwhelming case has been made out for compulsory acquisition of the sites for the stated objectives and that compulsory purchase is justified.''\footcite[42]{smith08} 

Justice Williams arrived at this conclusion after noting that the owners' {\it only} substantial objection against the CPO was that it was confirmed before adequate relocation measures had been agree on.\footcite[42]{smith08} Hence, the question as he saw did not concern the validity of using compulsory purchase powers, but merely the timing with which it had been ordered. On this basis, he framed the question of legitimacy as one relating to the ``necessity'' standard, according to which an infringement on Convention rights is only permissible when the public interest cannot be served in some other way.\footcite[43]{smith08} A strict reading of this standard holds that an interference must be the {\it least intrusive means} of achieving the stated aim.\footnote{Such a standard has been adopted in some Convention cases, for instance in \cite{samaroo01}.}

Justice Williams argued against such a strict reading, subscribing instead to a view expressed as an {\it obiter} in the case of {\it Pascoe v The First Secretary of State}. According to this view, an interference need not be the least intrusive means of achieving the public purpose, it is sufficient that the measure is ``reasonably necessary'' to achieve that aim.\footnote{See \cite[74-75]{pascoe06} (quoting \cite[25]{clay04}).} However, while noting his agreement with this approach, Justice Williams went on to apply the stronger necessity test, and found that even if this was applied the CPO in question would still be a proportional interference.\footcite[41-50]{smith08}

It seems clear that while the taking in question was for economic and recreational development purposes, the case was marked by the finding that the legitimacy of the aim of interference -- to facilitate the London Olympics -- was beyond reproach. Hence, there was no need for, or even room for, more detailed purposive reasoning of the kind that would later be applied by Lord Walker in {\it Sainsbury}. The fact that the taking was for economic development and recreation, not for a pressing public need, was not considered relevant, and was not held against the effects on the owners. This, in particular, was not how the issue of proportionality was conceptualized. Indeed, since the case was construed to be solely about the extent to which the CPO was ``necessary'' to further its stated aim, the proportionality test that was carried out, despite being detailed, was very narrow in scope. It concerned only proportionality of the means, not of the aim itself. The question of how to weigh the public interest in a multi-billion dollar sporting event against the security of someone's home was not considered.

In later cases, a dismissive attitude towards substantive review has been adopted also in situations when the owners have argued against takings by explicitly questioning the proportionality of the inference against the importance of the aim. In the case of {\it Alliance Spring Co Ltd v The First Secretary of State}, a large number of properties were expropriated to build a new football stadium for the football club Arsenal.\footcite{alliance06} Some owners who stood to lose their business premises as a result of the scheme protested the legitimacy of the order, pointing to the fact that the inspector in charge of the public inquiry had recommended against the takings.\footcite[6-7]{alliance06} As noted by Justice Collins, the main line of legal argument presented against the taking was that it did not serve a ``proper purpose''.\footcite[19]{alliance06} It is of note that in his evaluation of this argument, Justice Collins largely focuses on presenting the assessments carried out by the inspector and the Secretary of State, who went against the recommendation and confirmed the CPO. Finding that these assessments took all relevant matters into account and where not clearly unreasonable, Justice Collins goes on to conclude as follows: 

\begin{quote}
There is nothing in the material put before and accepted by the Inspector which persuades me that that decision was ill founded or was one which the Secretary of State was not entitled to reach. Developments which result in regeneration of an area are often led by private enterprise. Mr Horton perforce accepts that that is so, but submits that this is not the sort of situation where, for example, a private development is the anchor for a particular scheme. I disagree.\footcite[19]{alliance06}
\end{quote}

Hence, unlike the case of {\it Smith}, where the Court did in fact carry out its own assessment of proportionality, albeit only in relation to the question of necessity, the {\it Alliance} Court was content with deferring to the assessment carried out by the executive branch.\footnote{This has been criticized, e.g., by Kevin Grey who describes the reference to Convention Rights in Alliance as ``worryingly brief''. See \cite{gray11}.} As such, the case largely follows the set pattern of judicial review of CPOs from before the passing of the Human Rights Act 1998. This mean that it also stands in contrast to how English courts have approach the Convention in other kinds of cases, involving other rights, such as Article 8 in {\it Smith}. 

Whether the approach taken in {\it Alliance} is good law after {\it Sainsbury} is unclear; from Lord Walker's opinion, it seems that a more substantive assessment can be demanded for similar cases in the future. While this might not imply a different outcome for a case like {\it Alliance}, it would mean that courts would have to engage in independent review of the purpose and merits of contested CPOs that benefit commercial actors. In particular, English courts would have to change the way they approach such cases, by being more prepared to assess for themselves whether a fair balance is struck between the interests of the developer and the property owners. Hence, it is not unlikely that the category of economic development takings will become an important point of reference in the future, both for the law and those who study it.

\subsection{Germany}\label{sec:germany}

In German law we find an explicit constitutional property clause. In particular, Article 14 of the Basic Law ({\it Grundgesetz}) reads as follows:

\begin{quote}
(1) Property and the right of inheritance shall be guaranteed. Their content and limits shall be defined by the laws. \\
(2) Property entails obligations. Its use shall also serve the public good. \\
(3) Expropriation shall only be permissible for the public good. It may only be ordered by or pursuant to a law that determines the nature and extent of compensation. Such compensation shall be determined by establishing an equitable balance between the public interest and the interests of those affected. In case of dispute concerning the amount of compensation, recourse may be had to the ordinary courts.\footcite[14]{basic49}
\end{quote}

Apart from the fact that the property clause is explicit, I note two further characteristic features of the protection of property in Germany. First, the constitution explicitly stresses that property comes with social obligations as well as rights. The use of property should ``serve the public good''. On the other hand, it is also made clear that expropriation is only permissible when it is ``for the public good''. Hence, it follows immediately that the purpose of expropriation is a relevant factor when determining the legitimacy of a taking, irrespectively of the specific statute used to authorise it. Importantly, it is clear already from the outset that the question of legitimacy is a \emph{judicial} question, one which the courts can only answer if they form an opinion about that constitutes the ``public good''. 

This means that it is quite natural to approach the question of economic development takings from the point of view of constitutional law. Unlike in England, disputes over the legitimacy of such takings can be comfortably adjudicated directly against a ``public good'' restriction. While this sets Germany apart on the theoretical level, it is unclear how much of an effect it has had in practice. To shed some light on this question, we can look to the two major authorities on the legitimacy of economic development takings, the cases of {\it D\"{u}rkheimer Gondelbahn} and {\it Boxberg}.\footcite{durkheimer81,boxberg86} 

In both cases, the German Constitutional court found that expropriation to the benefit of commercial interests was illegitimate. However, the Court argued for this result on the basis that there was insufficient statutory authority for such takings in the concrete circumstances complained of. That is, the Court did not directly address the question of whether the relevant statutes were compliant with Article 14 of the basic law. Instead, they interpreted statutory authorities on the assumption that they had to be, following a pattern of reasoning that appears to be rather close to the approach followed by English courts in similar cases.\footnote{Although in {\it Dürkheimer Gondelbahn}, Böhmer J gave a separate concurring judgment where he argued for this result on the basis of the public good requirement of the basic law.} It seems, in particular, that even in Germany, the public purpose restriction is primarily relevant as a factor guiding the interpretation of statutory authorities.

That said, the cases of {\it D{\"u}rkheimer Gondelbahn} and {\it Boxberg} show that in situations when the public purpose of a taking is unclear, German courts seem inclined to favor a narrow interpretation of the relevant statute. In {\it Bloxberg}, several properties were expropriated in favor in favor of the car company Daimler Benz AG, for commercial purposes. The affected local communities suffered from high unemployment rates and a slow economy, so a {\it prima facie} reasonable cases could be made that allowing Daimler to acquire the land was in the public interest, as it would facilitate economic growth. However, the Federal Constitutional Court agreed with the owners that the expropriation was invalid. This, it held, was because the taking was outside the scope of the relevant statute, which authorised expropriations for ``planning purposes''. The owners had argued extensively using Article 14 of the Basic Law and the constitutional ``public good'' restriction clearly did play a role in the Court's reasoning. But at the same time, the Court stressed that private-to-private transfers that bestow financial benefit on the acquiring party may well satisfy the ``public good'' requirement. The important issue was whether a sufficiently strong public interest could be identified, irrespectively of any windfall benefits that might fall on private parties.

In light of this, I think it is wrong to exaggerate the importance of the explicit formulation of the public use test offered in the German constitution. Its importance seems to rest mainly in the fact that it provides a particularly authoritative expression guiding the national courts' application of statutory provisions regarding expropriation of property. But developments in common law, where the public use requirement is stressed as a guiding constitutional principle, might well point in the same direction. In principle, both German and English Courts are in a good position to respond to increased tension regarding economic development takings by developing a stricter standard of judicial review in such cases.

A different aspect of German law deserves special attention, however, since it does not appear to have any clear counterpart in the common law tradition. This is the  ``social-obligation'' norm in Article 14 (2), which points to a different conceptualization of property rights as such. As argued by Alexander, the distinguishing feature of the property clause in the German Constitution is that the value of property is thought to relate more strongly to its importance for human dignity and flourishing in a social context, rather than the protection of individual financial entitlements. As Alexander notes regarding the Germans' own conceptualization of their property clause:

\begin{quote}
This theory holds that the core purpose of property is not wealth maximization or the satisfaction of individual preferences, as the American economic theory of property holds, but self-realization, or self-development, in an objective, distinctly moral and civic sense. That is, property is fundamental insofar as it is necessary for individuals to develop fully both
as moral agents and participating members of the broader community.\footcite[745]{alexander03}
\end{quote}

With such a starting point, it is not surprising that in cases such as {\it Boxberg}, resembling {\it Kelo}, German Courts will tend to adopt a strict view on legitimacy. These are cases when the property rights infringed on serve a fundamentally different function for the two opposing private parties. To the owner, the property is a home, an important source of self-identity, autonomy, security and membership in a community. To the taker, it represents an obstacle to commercial development which needs to be removed. In such a situation, it is in keeping with the spirit of the social-obligation norm of property to offer enhanced protection to the homeowner. To this owner, the property serves a purpose which is fundamentally different, and arguably more worthy of protection, then the property's purpose for the developer. A taking in this situation might therefore, because of Article 14, require a particularly clear and strong public interest.

But unless there is an asymmetry between owner and taker, heightened scrutiny does not necessarily follow. Hence, it is interesting to speculate what German courts would have made of a case such as {\it Regina (Sainsbury’s Supermarkets Ltd) v Wolverhampton City Council}. Here, the interests of owner and taker were strictly commercial nature. Both owned part of the contested land and neither one could develop the land according to their plans without buying out the other. The enhanced protection of property offered under German law would probably not have much significance in such a case. 

In fact, it might well be that German courts would be {\it more} likely to accept such a taking. First, their conceptualization of property rights appears to allow greater flexibility to adapt the level of protection to the circumstances and the purposes of the property in question. So even if is correct that private-to-private transfers for commercial projects require a ``stricter approach'' in general, as argued by Lord Walker in \textcite{sainsbury10}, the fact that the interests of the owner were also purely commercial  might make this less relevant. Second, German courts might be more inclined to have regard to socially beneficial additional commitments entered into by the applicant, even if they do not concern the property that is taken. As a tie-breaker, looking to such commitments might be as good an approach as any other.\footnote{This was the view taken by the dissenting minority in \textcite{sainsbury10}.}

Of course, objections could still be raised on the basis of general administrative law. Indeed, some might see the case as an example of government ``auctioning'' off licenses to the highest bidder. This might well be regarded as an affront to good governance. I will not delve into German law to assess the case from this perspective. My point is simply that because of the purposive and contextual nature of Article 14, it seems unlikely that a case like \textcite{sainsbury10} would turn on constitutional property law.

To sum up, German constitutional law serves to create an interesting contrast with English law regarding the question of economic development takings. On the one hand, property appears to be better protected against such takings in Germany, but on the other hand, the extent to which increased protection is offered depends more closely on the social values involved. The German system appears to look more actively at the social function of property for guidance when resolving property disputes, thereby echoing some of the ideas discussed in Chapter \ref{chap:1}. 

In the next section, I will discuss the property clause in the ECHR, which explicitly serves to set up a minimum level of property protection that provides a common standard for all member states, including Germany and the UK.

\section{The Property Clause in the European Convention of Human Rights}

The starting point for property adjudication at the ECtHR is that States have a ``wide margin of appreciation'' with regards to the question of whether or not an interference in property rights is to be considered legitimate in pursuance of the public interest.\footcite[See][54]{james86} This question is thought to depend on democratically determined policies to such an extent that it is rarely appropriate for the Court to censor the assessments made by member states. At the same time, however, the Court has gradually come to take a more active role in assessing whether or not particular instances of interference are ``proportional'' and able to strike a ``fair balance'' between the interests of the public and the interests of the individual property owner.\footnote{See \cite[69]{sporrong82} and \cite[120]{james86}. The standard account of the protection against interference inherent in P1(1) describes it as consisting of three rules. First, there is the rule of {\it legality}, asserting that an interference needs to be authorized by statute. Second, there is the rule of {\it legitimacy}, making clear that interference should only take place in pursuance of a legitimate public purpose. Both of these rules are of little practical significance, however, as the margin of appreciation has been regarded as very wide in regards to both. The third rule is the ``fair balance'' principle, which is applied by the ECtHR in almost all cases when it finds that there has been a violation of P1(1). In the following, I focus only on this rule and on those aspects of it that I think are most relevant to the question of economic development takings. For a more detailed description of P1(1) generally, I refer to \cite{allen05}.} As argued by Tom Allen, this has caused P1(1) to attain a wider scope than what was originally intended by the signatories.\footcite[1055]{allen10}.

In the case law behind this development, the focus has predominantly been on the issue of compensation, with the Court gradually developing the principle that while P1(1) does not entitle owners to full compensation in all cases of interference, the fair balance will likely be upset unless at least some compensation is paid, based on the market value of the property in question.\footnote{See \cite[103]{scordino06}. The case also illustrates that the Court has come to adopt a fairly strict approach to the question of when it is legitimate to award less than full market value.} This focus on compensation has also been reflected in academic work on P1(1), which tends to address proportionality from an economic perspective, by investigating to what extent owners are entitled to compensation based on the market value of their property. Indeed, when considering the best known case law and literature on the subject, one is left with the impression that ``fair balance'' with regards to P1(1) is crucially linked to financial entitlements, primarily used as a standard that can justify a right to compensation that goes beyond what the wording of P1(1) might initially suggest.

In recent case law, however, it has become clear that the fair balance test encompass more than this, since it also gives the Court in Strasbourg occasion to reflect on the social context and purpose of interference, in a manner largely consistent with the social function approach to property. In {\it Chassagnou and others v France} the situation was that property owners were compelled to permit hunting on their land, following compulsory membership in a hunting association which was set up to manage hunting in the local area.\footcite{chassagnou99} They protested this on the grounds that they were ethically opposed to hunting, and the Court agreed that there had been a breach of P1(1). 

In the later case of {\it Hermann v Germany} the circumstances were similar, and the Court followed the precedent set in {\it Chassagnou}, commenting also that they had ``misgivings of principle'' about the argument that financial compensation could provide adequate protection in such a case.\footcite[See][91]{hermann12}  In this way, the hunting cases illustrate that to the ECtHR, the right to property is not seen as a mere financial entitlement. Moreover, the fair balance that must be struck could well pertain to other aspects, such as the owner's right to make use of his property in accordance with his convictions and to take part in decision-making processes regarding how it should be managed.\footnote{The assessment of proportionality should be concrete and contextual, and it is not based on a narrow or formalistic concept of property as dominion. This is demonstrated, for instance, by \cite{chabauty12}. Here the Court found no violation of P1(1) although the facts seemed close to those of {\it Chassagnou}. The case differed, however, in that the owner himself was not opposed to hunting, but wanted to withdraw his land from the hunters' association to enjoy exclusive hunting rights.}

In a different, but related, development, the Court has also adopted a contextual approach in recent cases involving rent control 
schemes and housing regulation. While there are obvious financial interests at stake in such cases, for both landlords and tenants, the Court has looked to the fairness of the underlying regulation more generally, by taking into account the local social, economic and political conditions. Moreover, the Court has not shunned away from using concrete cases as a starting point for providing an assessment of the sustainability of national law as such. In {\it Hutten-Czapska v Poland}, for instance, the Court concluded that the case demonstrated ``systemic violation of the right of property''.\footcite[239]{hutten06}

The case concerned a house that had been confiscated during WW2. After the war, the property was transferred back to the owners, but in the meantime, the ground floor had been assigned to an employee of the local city council. Moreover, the state implemented strict housing regulations during this time, which eventually meant that the applicant's house was placed under direct state management.\footcite[20-31]{hutten06} Following the end of communist rule in 1990, the owners were given back the right to manage their property, but it was still subject to strict regulation that protected the rights of the tenants.\footcite[31-53]{hutten06} In addition to rent control, rules were in place that made it hard to terminate the rental contracts. Hence, it became impossible for the owners to make use of the house themselves, as they wished to do.\footcite[20-53]{hutten06} 

After an in-depth assessment of the relevant parts of Polish law and administrative practice, the Grand Chamber of the ECtHR concluded that there had been a violation of P1(1). Importantly, they did not reach this conclusion by focusing on the house as a source of financial entitlements for the owners. Rather, they focused on the overall character of the Polish system for rent control and housing regulation, as it manifested in the concrete circumstances of the applicant's case. The financial consequences for the owners were considered, as was the financial situation of the tenants.\footcite[60-61]{hutten06} The Court was particularly concerned with the fact that the total rent that could be charged for the house was not sufficient to cover the running maintenance costs.\footcite[224]{hutten06} In particular, it was noted that the consequence of this would be ``inevitable deterioration of the property for lack of adequate investment and modernisation''.\footnote{\cite[224]{hutten06}.}

In the end, the Court concluded that the combination of a rigid rent control system, rules that made it hard for owners to terminate tenancy agreements, and the fact that the State itself had set up these agreements during the days of direct state management, meant that a fair balance had not been struck.\footcite[224-225]{hutten06} The contextual nature of the Court's reasoning is evidenced not only by the extent to which the concrete circumstances are assessed against the goal of fairness, but also by how the Court explicitly places the ``social rights'' of the tenants on equal footing with the property rights of the owners.\footcite[225]{hutten06} 

It is also of interest to note how the Court reasons towards the conclusion that the Polish legal order as such is at fault. In this regard, great weight is placed on the observation that the system suffers from a lack of adequate safeguards to protect owners against imbalances such as those identified in the present case. In particular, the Court comments on ``the absence of any legal ways and means making it possible for them either to offset or mitigate the losses incurred in connection with the maintenance of property or to have the necessary repairs subsidised by the State in justified cases''. Hence, the rent control scheme alone was not the whole problem, the Court also criticized what it saw as a defective way of implementing it.\footcite[224]{hutten06} Moreover, the Court did not censor the political reasoning that motivated Polish housing legislation, but concluded instead that the ``burden cannot, as in the present case, be placed on one particular social group, however important the interests of the other group or the community as a whole''. 

I think this is the most important aspect of the case, pointing to the core function that the ECtHR should embrace more generally. It seems to me, in particular, that objections can be raised against the appropriateness of having the Court in Strasbourg assess concretely what is fair regarding the relationship between owners and tenants in a specific house in Gdynia. Its remoteness to the local conditions, as well as its lack of sensitivity and accountability to locally grounded political processes, suggest that the Court is not ideally placed to carry out the kind of contextual assessment that it prescribes for such cases. In addition, the amount of resources and time needed to independently scrutinize these aspects convincingly risks undermining its ability to deal expediently with its case load. The ECtHR will hardly be able to protect human rights in Europe on a case-by-case basis.

Instead, the aim should always be to get at the systemic features that cause perceived imbalances. As in \textcite{hutten06}, the Court serves its function best when it is able to identify a sense in which the domestic legal order needs to be improved to better comply with human rights standards. This is particularly true when, as in that case, the Court notes that the applicants have insufficient options available for achieving a fair balance by appealing to institutions within the domestic legal order. By demanding {\it institutional} changes, in particular, the Court effectively delegates responsibility for ensuring the kind of fair balance that is required under the ECHR. Moreover, by scrutinizing the procedures and principles that the states apply when fulfilling this duty, it is likely that the Court will still be able to steer and unify the development of the case law. Importantly, they would then be able to do so without having to engage extensively in concrete assessments of fairness. 

Against this, one may argue that the judicial or administrative bodies of the signatory states can easily circumvent their obligations by giving a superficial or biased assessment of the facts in human rights cases, to avoid embarrassment for the state's political or bureaucratic elite. However, this might then be raised as a procedural complaint before the ECtHR, resulting in cases revolving around Articles 6 (fair trial) and 13 (effective remedy).\footnote{I note that this also fits with recent developments at the ECtHR, toward somewhat broader scrutiny under Article 6, see \cite{khamidov07}.}  In this way, the Court can streamline its functions, by always aiming to direct attention at issues that arise at a higher level of abstraction. This, in my view, is desirable. The ECtHR should not aim to micromanage the signatory states, particularly not in relation to a norm such a P1(1), which the Court itself regards as highly dependent on context.

However, the question arises as to what kind of institutions the Court should focus on in its effort to ensure fairness in relation to Convention rights such as property. It is not given, in particular, that directing attention towards domestic judicial bodies is the most appropriate approach. Rather, it seems logical to assume that those institutions most in need of reform will be those that are actually responsible for violations. A possible lack of an effective complaints procedure would be worrying, but hardly as problematic as possible systemic weaknesses that give rise to complaints in the first place. With this change in perspective, the Court can avoid getting stuck in deference to domestic judicial bodies, but still shift attention away from concrete assessment of alleged violations. They can do so, in particular, by concretely and critically assessing those rules and procedures that are identified as causally significant to individual complaints. \footnote{In the future, one might even encounter cases when the Court prefers to remain agnostic about whether a substantive violation occurred, focusing instead on the possible violation inherent in excessive systemic risks and a shortage of adequate safeguards.}

Indeed, I think the case of \textcite{hutten06} is suggestive of a move towards such a perspective. While the Court went into great detail about the facts of the case, it {\it also} looked at the case from an alternative perspective, more in line with the suggestion sketched above. In fact, I think it is likely that the Court will eventually veer even more towards such an approach, while deferring to national judicial bodies when it comes to concrete factual assessments. If not as a result of policy, I imagine this will happen from necessity, at least in relation to rights such as property, which now seem to flood the Court.

One might ask where this would leave the proportionality doctrine. In fact, I think this doctrine still makes good sense when framed in more abstract terms as the question of what kinds of rules, and what kinds of institutions, member states need to put in place to ensure fairness. In \textcite{hutten06}, the Court moved in this direction, when it explained the basic principle as follows:

\begin{quote}
In assessing compliance with Article 1 of Protocol No. 1, the Court must make an overall examination of the various interests in issue, bearing in mind that the Convention is intended to safeguard rights that are “practical and effective”. It must look behind appearances and investigate the realities of the situation complained of. In cases concerning the operation of wide-ranging housing legislation, that assessment may involve not only the conditions for reducing the rent received by individual landlords and the extent of the State’s interference with freedom of contract and contractual relations in the lease market, but also the existence of procedural and other safeguards ensuring that the operation of the system and its impact on a landlord’s property rights are neither arbitrary nor unforeseeable. Uncertainty – be it legislative, administrative or arising from practices applied by the authorities – is a factor to be taken into account in assessing the State’s conduct. Indeed, where an issue in the general interest is at stake, it is incumbent on the public authorities to act in good time, in an appropriate and consistent manner.\footcite[151]{hutten06} 
\end{quote}

I note how the Court builds on the earlier precedent set by cases such as \textcite{sporrong82} and \textcite{james86}. The first half of the quote, therefore, stresses that the Court itself must ``look to the realities of the situation''. However, in clarifying what is meant by this, the Court goes on to emphasize procedural aspects. In particular, it is made clear that the Court regards such aspects as an integral part of those ``realities'' that need to be assessed. Indeed, the Court even makes specific reference to the importance of several values that arise in the context of administrative law, such as predictability and effectiveness.

The passage above was subsequently quoted in {\it Lindheim and others v Norway}. In this case, the applicants complained that their rights had been violated by a recent Norwegian act that gave lessees the right to demand indefinite extensions of ground leases on pre-existing conditions.\footcite[119]{lindheim12}  In the end, the Court concluded that there had indeed been a breach of P1(1). Interestingly, they engaged in the same form of assessment that they had adopted in \textcite{hutten06}. They held, in particular, that it was the Act itself which was the underlying source of the violation, not merely its concrete application against the applicants. Hence, the Court did not only award compensation, it also ordered that general measures had to be taken by the Norwegian State to address the structural shortcomings that had been identified.

In this case, the Court also commented that its decision should be regarded in light of ``jurisprudential developments in the direction of a stronger protection under Article 1 of Protocol No. 1''.\footcite[135]{lindheim12} However, in light of the change in perspective that accompanies this development, it is interesting to ask in what sense exactly the protection is stronger. It is not {\it prima facie} clear, in particular, that the Court's remark should be read as a statement expressing a change in its understanding of the content of individual rights under P1(1). Rather, I am inclined to read it as a statement to the effect that the Court now assumes it has greater authority to address structural problems under that provision. This authority, in particular, is now seen to extend also to the fair balance requirement, not only the (much more narrowly drawn) legality and legitimacy rules. In effect, this also leads to stronger protection for individuals, since it allows the Court to conclude that a violation has occurred due to ``structural unfairness'', even when it is not possible to trace this back to any ``flawed'' decision that directly targets the applicants.

What is the relevance of all this to the issue of economic development takings? A great deal, I think. Indeed, I am struck by how the reasoning of the ECtHR in recent cases on hunting and rent control mirrors the kind of reasoning that Justice O'Connor engaged in when she considered {\it Kelo}. The emphasis is on structural aspects and fairness, but the considerations made in this regard are grounded on what the facts of the concrete case reveal about the rules and procedures involved. In this way, the contextual approach to property gains focus without losing its bite. Individual entitlements play a relatively minor role, the assessment is highly relational, focusing on where the system involved places different legal persons in relation to one another. The crux of arguments used to conclude violation is the observation that the system offends against the role that owners {\it should} occupy in order to be able to meet those obligations and exercise those freedoms that society normally regards as inherent to the form of property that they possess.

In this case, interference is not only unfair, it is also a failure of governance, and a structural inconsistency. In the case of \textcite{hutten06}, this boiled down to the observation that the system which had led to the complaint sought to resolve problems in the Polish housing sector in a manner that placed the burden ``on one particular social group'', namely the owners.\footcite[225]{hutten06} This conclusion was backed up by the concrete observation that the rules and procedures in place meant that owners who were expected to maintain their properties in good condition for their tenants were in fact prevented from doing so because they were not permitted to charge rents that would cover the costs.

In the case of {\it Kelo}, Justice O'Connor argued in  a similar fashion when she concluded that the system which had led to the decision to condemn Suzanne Kelo's house was likely to function so as to systematically ``transfer property from those with fewer resources to those with more''. This conclusion was backed up by the observation that the beneficiary in {\it Kelo} was a multi-billion dollar commercial company that had been allowed to take Kelo's home because this would lead to ``economic development''. To Justice O'Connor, there was little doubt that this could become a general pattern, if safeguards were not in place. Indeed, it must be presumed that a multi-million dollar company is always in a better position than a homeowner when it comes to arguing that  ``economic development'' will result from their ownership. More subtly, her opinion also hinted at the inconsistency involved in asserting abstractly that economic development would benefit the community indirectly, all the while the development would if fact require razing it.

To conclude, I think the ECtHR is more likely to approach a case like {\it Kelo} in the manner Justice O'Connor did. Whether they would reach the same result seems more uncertain, particularly since confidence in states' ability and willingness to regulate private-public partnerships might be higher in Europe. However, it seems unlikely that the ECtHR would follow the majority in {\it Kelo}, by simply deferring to the determinations made by the granting authority. Moreover, with the recent change in perspective towards a more structural assessment of property institutions at the ECtHR, it seems that Justice O'Connor's predictions about the ``fallout'' of the {\it Kelo} decision would likely strike a cord with the justices at Strasbourg. 

\section{The US perspective on economic takings}\label{sec:us}

I now consider US law in more depth. First, I track the development of the case law on the public use restriction in the Fifth Amendment and in various state constitutions, from the early 19th Century up to the present day.\footnote{The public use clause in the US constitution was not held to apply to state takings until the late 19th Century, see \cite{chicago97}.} Many writers assert that the 19th and early 20th Century was characterized by a ``narrow'' approach to public use which eventually gave way to a broader conception.\footnote{See, e.g., \cite[483]{walt11}; \cite[203-204]{allen00}. For a more in-depth argument asserting the same, see \cite{nichols40}.} Against this, I argue that it is more appropriate to think of this period as one when courts adopted a {\it broad} approach to judicial scrutiny of the takings purpose at state level. Importantly, I also argue that while different state courts expressed different theoretical views on the meaning of ``public use'', there was a growing consensus that the approach to judicial scrutiny should be contextual, focused on weighing the rationale of the taking against the concrete social, political and economic circumstances of the local area.\footnote{A summary of state case law that supports this view is given in the little discussed Supreme Court case of \cite{hairston08}.}  In particular, I argue that early state courts did not focus as much on the exact wording of the constitutional property clause as many later commentators have suggested.

I go on to show that the doctrine of deference that was developed by the Supreme Court early in the 20th Century was directed primarily at state courts, not state legislatures and administrative bodies.\footnote{See \cite{vester30} (echoing and citing \cite{hairston08}).} I then present the case of {\it Berman}, arguing that it was a significant departure from previous case law.\footcite{berman54} After {\it Berman}, deference was suddenly taken to mean deference to the (state) legislature, so there would be little or no room for judicial review of the takings purpose. I go on to present the subsequent developments at state level, characterized by increasing worry that the eminent domain power could be abused by powerful commercial actors. I discuss the case of {\it Poletown}, where a neighborhood of about 1000 homes was razed to provide General Motors with land to assemble a car factory.\footcite{poletown81} I link this to the subsequent controversy that arose over {\it Kelo}, suggesting that it should be seen as the eventual backlash of {\it Berman}, a consequence of abandoning the contextual approach to public use in favor of an almost absolute rule of deference.

After the historical overview, I go on to briefly present the vast amount of research that has targeted economic takings in the US after {\it Kelo}. I give special attention to writers that propose new legitimacy-enhancing institutions for facilitating economic development of jointly owned land. I focus on two proposals in particular, targeting compensation and participation respectively.\footcite{lehavi07,heller08} These proposals will serve as important reference points later on, when I consider the Norwegian appraisal  and land consolidation courts in Chapters 4 and 5.

\section{The history of the public use restriction}\label{sec:hop}

Going back to the time when the Fifth Amendment was introduced, there is not much historical evidence explaining why the takings clause was included in the bill of rights, and little in the way of guidance as to how it was originally understood. James Madison, who drafted it, commented that his proposals for constitutional amendments were intended to be uncontroversial to Congress.\footnote{See letters from Madison to Edmund Randolph dated 15 June 1789 and from Madison to Thomas Jefferson dated 20 June 1789, both included in \cite{madison79}.}  Hence, it is natural to regard it as a codification of an existing principle, rather than a novel proposal. Indeed, several State constitutions pre-dating the Bill of Rights also included takings clauses, and they all seem to be largely based on a codification of principles from English Common law.\footcite[See][299]{johnson11}

As we discussed in subsection \ref{sec:england} above, the typical English attitude from this time, which was also reflected in the law, held private property in very high regard. On this background it is not surprising that Madison regarded the property clause as an uncontroversial amendment.\footnote{Indeed, early American scholars also emphasized the importance of private property. For instance, in his famous {\it Commentaries}, James Kent described the sense of property as ``graciously implanted in the human breast'' and declared that the right of acquisition ``ought to be sacredly protected'', \cite[see][257]{kent27}.} Its importance may in fact have been greater as a legitimizing force, increasing confidence in the regulatory power of the newly established state by setting up clear parameters for the exercise of that power.  However, while the principle itself was regarded as self-evident, it was never clear what it would mean in practice, particularly in cases when takings where challenged on the basis that they were not for a ``public use'''.\footcite[See][317]{johnson11} 

There are two points that I would like to record about the early common law in the US  in this regard. First, the distinction between public use and public purpose does not appear to have been considered sharp. In his {\it Commentaries}, James Kent first makes clear that the power of eminent domain is for ``public use, and public use only", but then goes on to qualify this by stating that a taking which served a ``purpose not of a public nature'' would be unconstitutional.\footcite[See][275-276]{kent27}  He does not address this limitation in any detail, however, suggesting that it was not the subject of much debate at this time. To the founders, it seems that the right to compensation was considered the more important principle, something that is also reflected in the {\it Commentaries}.\footnote{James Kent held it to be  ``founded in natural equity'' and described it as an ``acknowledged principle of universal law'', \cite[see][276]{kent27}.} The public use limitation was probably taken for granted as a matter of principle, while it had not yet proved problematic as a matter of practical adjudication. Moreover, it appears to have been accepted that takings which clearly benefited the public would be legitimate regardless of whether or not the property was physically put to use by the public.\footcite{johnson11}

An interesting early illustration of how courts approached takings controversies at this time can be found in {\it Stowell v Flagg}, a Massachusetts case from 1814. In this case, a landowner complained that his land had been flooded by a mill and sought a remedy in common law. The mill owner protested, however, since he was entitled to flood the land according to a special mill act, which allowed him to exercise the power of eminent domain to gain the right to flood his neighbor (provided statutory compensation was paid). The focus in the case was on whether a common law claim for damages could still be made, irrespectively of the act's clear intention to deprive the affected neighbors of this opportunity. Hence, the court implicitly dealt with the legitimacy of the mill act itself, and they actively engaged with the public use requirement in the state constitution when making their assessment.\footcite{stowell14} In the end, they found that the act was legitimate, and they highlighted the purpose of the interference, commenting that ``these mills, early in the settlement of this country, were of great public necessity and utility''.\footcite[366]{stowell14} 

At the same time, however, the court had misgivings about how the act had come to be applied and expressed concern that ``the legislature, as well as the courts of law in this state, seem to have been disposed rather to enlarge, than to curtail, the power of mill owners''.\footcite[366]{stowell14} Still, after noting that affected land owners were entitled to compensation under the act, the court concluded that the act had to be observed and that it precluded any claims for damages under common law. Hence, the case is an early example of judicial deference to the legislature in takings cases, while also illustrating that the public use requirement was beginning to emerge as a potentially problematic issue in its own right. The presiding judge stated that he could not help thinking that the statute was ``incautiously copied from the ancient colonial and provincial acts'', but still held in favor of the mill owner,  concluding that ``as the law is, so must we declare it''.\footcite[368]{stowell14}

While judicial deference was recognized as a guiding principle early on in US takings law, it is important to note in this regard that eminent domain was seldom used in a way that would raise serious controversy. English common law, while lacking clearly defined constitutional safeguards, was, as we have already mentioned, based on a fundamentally cautious attitude, ensuring that the power would typically only be used as a last resort. As Professor Meidinger notes, the British were never really charged with abuse of eminent domain, and private property tended to be respected, also in the colonies.\footcite[17]{meidinger80} This undoubtedly influenced early US law. Indeed, the importance of constitutional limits on the taking power was made clear by the Supreme Court early on, as a matter of principle.\footnote{As reflected in {\it de dicta} comments from {\it Calder v Bull} and {\it Vanhorne’s Lessee v Dorrance}, see \cite[388]{calder98}; \cite[310]{vanhorne95}.} Hence, the relative lack of judicial interest in the question of legitimacy does not appear to have been due to a broad view on the scope of eminent domain, but an established practice of narrow use of that power, inherited from the English.

%The Legislature declare and enact, that such are the public exigencies, or necessities of the State, as to authorise them to take the land of A. and give it to B.; the dictates of reason and the eternal principles of justice, as well as the sacred principles of the social contract, and the Constitution, direct, and they accordingly declare and ordain, that A. shall receive compensation for the land. But here the Legislature must stop; they have run the full length of their authority, and can go no further: they cannot constitutionally determine upon the amount of the compensation, or value of the land. Public exigencies do not require, necessity does not demand, that the Legislature should, of themselves, without the participation of the proprietor, or intervention of a jury, assess the value of the thing, or ascertain the amount of the compensation to be paid for it. This can constitutionally be effected only in three ways.
%1. By the parties that is, by stipulation between the Legislature and proprietor of the land.
%2. By commissioners mutually elected by the parties.
%3. By the intervention of a Jury.

The traditional attitude to eminent domain would eventually give way to a more expansive approach, however. This development became particularly marked during the period of great economic expansion and industrialization in the mid to late 19th century, when eminent domain was increasingly used to benefit (privately operated) railroads, hydroelectric projects, and the mining industry.\footcite[23-33]{meidinger80} During this time, it also became increasingly common for landowners to challenge the legitimacy of takings in court, undoubtedly a consequence of the fact that eminent domain was now used more widely, for new kinds of projects.\footcite[24]{meidinger80} Controversy arose particularly often with respect to mill acts.\footnote{\cite[24]{meidinger80}. See also \cite[306-313]{johnson11} and \cite[251-252]{horwitz73}.} Such acts were found throughout the US, and many of them dated from pre-industrial times when mills were primarily used to serve the needs of self-sufficient agrarian communities.\footnote{A total of 29 states had passed mill acts, with 27 still in force, when a list of such acts was compiled in \cite[17]{head85}. According to Justice Gray, at pages 18-19 in the same, the ``principal objects'' for early mill acts had been grist mills typically serving local agrarian needs at tolls fixed by law, a purpose which was generally accepted to ensure that they were for public use.}  However, following economic and technological advances, acts that were once used to facilitate the construction of grist mills would increasingly also be relied on by developers wishing to harness hydropower for manufacturing, and eventually, for hydroelectric projects.\footnote{See, e.g., \cite[18-21]{head85} and \cite[449-452]{minn06}.}

The mill acts typically contained provisions that enabled the mill developer to condemn both property needed for the construction itself as well as the right to damage surrounding land by flooding or deprivation of water. Such takings became increasingly controversial, however, and many legitimacy cases came before state courts in the late 19th and early 20th century. In the next subsection I present some of these cases, to shed light on how states courts developed their own approach to the question of legitimacy of takings.

\subsection{Legitimacy in state courts}\label{subsec:state}

In the mil cases, we find the first clear evidence of how the public use requirement was put to use to enable state courts to scrutinize the legitimacy of takings. Generally speaking, when a court upheld an interference in private property, it would place decisive weight on the broader purpose of interference, typically by arguing that economic ripple effects ensured that the mill was in the public interest even if the public would not literally make use of it.\footnote{See, e.g., \cite{hazen53,scudder32,boston32}. A more comprehensive list of cases adopting a broad view can be found in \cite[617]{nichols40}.} By contrast, when a court decided that an interference was unconstitutional (with respect to state constitutions), it would often focus on the use made of the mill, arguing that it did not directly benefit the public in the sense required by the public use restriction.\footnote{See, e.g., \cite{sadler59,ryerson77,gaylord03,minn06}. A more comprehensive list can be found in {\it Public benefit or convenience as distinguished from use by the public as ground for the exercise of the power of eminent domain} 54 ALR 7 (American Law Reports, 1928).} For a time, a doctrine which sought to distinguish between takings for public use and takings for a public purpose, played quite a significant role in many states. Under this doctrine, only those takings that were deemed to qualify as public use takings under a narrow view of that term would be upheld.\footnote{Professor Nichols goes as far as to conclude that this emerged as the ``majority'' opinion on public use, see \footcite[617-618]{nichols40}. But contrast this with \cite{berger78} and \cite[24]{meidinger80}, who argue that the narrow view was only dominant in a handful of states, led by New York.}

%For instance, in the case of {\it Gaylord v. Sanitary Dist. of Chicago}, the Supreme Court of Illinois held the state Mill Act to be unconstitutional, as it was not limited to traditional flour mills. In doing so, the court observed that public use was ``something more than a mere benefit to the public''.\footcite[524]{gaylord03} Similar sentiments were expressed in other decisions striking down uses of eminent domain for mill construction, for instance in Vermont, Michigan and New York.\footnote{References.}

It is tempting to associate the narrow view on public use with a more restrictive attitude towards the use of eminent domain. Similarly, it is natural to assume that a broad view on public use suggests a more relaxed attitude. To some extent, the primary sources warrant this; unsurprisingly, those who endorsed a broad view on the public use question also often spoke in favor of judicial deference in legitimacy cases, while those endorsing a narrow view tended to emphasize the importance of constitutional safeguards against abuse of eminent domain. However, it seems that both groups were quite heterogeneous and that differences of opinion about the public use requirement did not necessarily reflect any deep ideological divisions.

It is clear, for instance, that many of the courts which favored a broad interpretation of public use still viewed the constitutional limitation on the takings power as an important safeguard, not only as a guarantee for compensation but also as a restriction on the purpose of takings. Indeed, it seems that most late 19th Century Courts, including those that upheld economic takings, were influenced by the growing body of case law across the US that actively scrutinized takings, sometimes striking them down. In particular, it seems that the strict deferential view was largely abandoned in economic takings cases during this period. Deference to the legislature still played an important role and was typically called on as an important argument in takings cases. However, it became much more common to discuss legitimacy also in terms of substantive arguments, by directly addressing the context and circumstances of the taking complained of. I believe this is an important insight to record about the case law from this period; despite differences of opinion about the meaning of public use, a consensus appears to have emerged that judicial review of legitimacy was appropriate and important in economic takings cases.

A good example is the case of {\it Dayton Gold \& Silver Mining Co. v. Seawell}, concerning a Nevada Act which stipulated that mining was a public use for which the power of eminent domain could be exercised to acquire additional rights needed to facilitate extraction.\footcite{seawell76} The Supreme Court of Nevada decided that the Act was constitutional and adopted a broad understanding of the property clause in the Nevada constitution.\footnote{Nev Const Art 8 § 1.} Interestingly, it argued for this interpretation partly on the basis that it would provide {\it better} protection for landowners:

\begin{quote}
If public occupation and enjoyment of the object for which land is to be condemned furnishes the only and true test for the right of eminent domain, then the legislature would certainly have the constitutional authority to condemn the lands of any private citizen for the purpose of building hotels and theaters. [...] Stage coaches and city hacks would also be proper objects for the legislature to make provision for, for these vehicles can, at any time, be used by the public upon paying a stipulated compensation. It is certain that this view, if literally carried out to the utmost extent, would lead to very absurd results, if it did not entirely destroy the security of the private rights of individuals. Now while it may be admitted that hotels, theaters, stage coaches, and city hacks, are a benefit to the public, it does not, by any means, necessarily follow that the right of eminent domain can be exercised in their favor.\footcite[410-411]{seawell76}
\end{quote}

The quote shows that a broad understanding of ``public use'' need not be synonymous with a less cautious attitude to abuse of the takings power. Indeed, while the Court decided to uphold the Act, it did so only after a very careful assessment of both legal arguments and factual circumstances. In particular, the Court considered the importance of mining, concluding that it was the ``greatest of the industrial pursuits'' in the state, and that all other interests were ``subservient'' to it.\footcite[409]{seawell76} Moreover, the Court commented that the benefits of the mining industry was ``distributed as much, and sometimes more, among the laboring classes than with the owners of the mines and mills''.\footcite[409]{seawell76}

This shows that the Court actively engaged with the purpose of the Act, thoughtfully assessing it against the constitution. Importantly, it did not do so in isolation, as a linguistic exercise or by attempting to recreate its ``original intent''. Rather, the court approached the constitutional safeguard by making detailed references to the prevailing social and economic conditions in the state of Nevada. The Court noted the importance of deference to the legislature on matters of policy, but it did so only after it had satisfied itself that the Act could be ``enforced by the courts so as to prevent its being used as an instrument of oppression to any one''.\footcite[412]{seawell76} More generally, the court commented as follows on the public purpose test that had to be performed in takings cases, elucidating on the principles on which it should be founded:

\begin{quote}
 Each case when presented must stand or fall upon its own merits, or want of merits. But the danger of an improper invasion of private rights is not, in my judgment, as great by following the construction we have given to the constitution as by a strict adherence to the principles contended for by respondent.\footcite[398]{seawell76}
\end{quote}

In light of this, {\it Dayton Gold \& Silver Mining Co. v. Seawell} must be regarded as an early example of a {\it contextual} approach to legitimacy, characterized by the willingness of the Court to engage in a fairly detailed analysis of the concrete circumstances and consequences of takings. A formalistic approach based on the phrase ``public use'' was abandoned, but not in favor of general deference. Rather, a more nuanced view was adopted, to respect the idea that the legislature should have the final say on policy while also recognizing that courts should play a crucial role in protecting citizens from abuse of the takings power. 

The case is not unique, but rather exemplifies the type of reasoning that was used in economic takings cases at this time. Interestingly, many common elements exist between courts that upheld and struck down such takings, irrespectively of whether or not they subscribed to a narrow or broad view on the public use test. One example is {\it Ryerson v. Brown}, a case often cited as an authority in favor of a narrow view.\footcite{ryerson77} Here the Supreme Court of Michigan explicitly qualifies its decision by stating that it is ``not disposed to say that incidental benefit to the public could not under any circumstances justify an exercise of the right of eminent domain'', hardly a clear endorsement of the narrow rule. The case concerned the constitutionality of a mill act, and while the court argues that public use should be taken to mean ``use in fact'', it is clear that ``use'' is understood rather loosely, not literally as physical use of the property that is taken.\footnote{The court explains its stance on the public use restriction by stating (emphasis added) ``it would be essential that the statute should require the use to be public in fact; in other words, that it should contain provisions entitling the public to {\it accommodations}.'' The court continues with an illustrative example: ``A flouring mill in this state may grind exclusively the wheat of Wisconsin, and sell the product exclusively in Europe; and it is manifest that in such a case the proprietor can have no valid claim to the interposition of the law to compel his neighbor to sell a business site to him, any more than could the manufacturer of shoes or the retailer of groceries. Indeed the two last named would have far higher claims, for they would subserve actual needs, while the former would at most only incidentally benefit the locality by furnishing employment and adding to the local trade''. See \cite[336]{ryerson77}.} Moreover, when clarifying its starting point for judicial scrutiny of mill acts, the court explains that ``in considering whether any public policy is to be subserved by such statutes, it is important to consider the subject from the standpoint of each of the parties''. Following up on this with regards to the act in question, the court finds that `` the power to make compulsory appropriation, if admitted, might be exercised under circumstances when the general voice of the people immediately concerned would condemn it''. After considering this and other possible consequences of mill development under the act, the court eventually declares it to be unconstitutional, summing up its assessment as follows: ``What seems conclusive to our minds is the fact that the questions involved are questions not of necessity, but of profit and relative convenience''.\footcite[336]{ryerson77}

Hence, far from nitpicking on the basis of the public use phrase, the court adopts a contextual approach to takings that is in fact rather similar to the approach of {\it Dayton Gold \& Silver Mining Co. v. Seawell}. The outcome it different, but it is also based on a different assessment of the context and the consequences of the takings complained of. Importantly, the case does not rest on any {\it a priori} assumption that economic takings of the kind in question could not meet a public use test -- no general rule is relied on at all. Hence, it is somewhat strange that later commentators have focused on the case for its comments on public use rather than its broad, albeit perhaps somewhat conservative, assessment of legitimacy. 

Many of the important cases from the late 19th Century, on both sides of the public use debate, shares many crucial features with the two cases discussed above.\footnote{See, e.g., \cite{scudder32} (Eminent domain power upheld, but said: ``The great principle remains that there must be a public use or benefit. That is indispensable. But what that shall consist of, or how extensive it shall be to authorize an appropriation of private property, is not easily reducible to a general rule. What may be considered a public use may depend somewhat on the situation and wants of the community for the time being.''), \cite{fallsburg03} (Eminent domain struck down, on holding that ``the private benefit too clearly dominates the public interest to find constitutional authority for the exercise of the power of eminent domain''), \cite[538]{board91} (Eminent domain struck down, qualified by ``not only must the purpose be one in which the public has an interest, but the state must have a voice in the manner in which the public may avail itself of that use'').} In my opinion, this points to an interesting alternative perspective on legitimacy adjudication from this time. Some commentators describe the case law as chaotic, with competing conceptions of constitutional limits competing for dominance.\footcite{berger78,meidinger80}. I think this is more accurate than saying that a narrow interpretation of public use developed as a general rule. However, I also find evidence that there was in fact a broad consensus in this period regarding the need for special judicial scrutiny of economic development cases. State courts widely engaged in contextual assessment of legitimacy, and they were conscious of the special challenges that arose in a time when eminent domain was being used to facilitate economic expansion that would benefit specific commercial actors. Differences of opinion about public use terminology was an important aspect of this, but it was rarely considered in isolation from other aspects. On a deeper lever, the fact that the public use debate was regarded as important in the first place clearly suggests that deference to the legislature was not held to be an exhaustive answer to the question of legitimacy. This, in my opinion, is an important observation which appears to have been somewhat overlooked in the literature. 

It is an observation that I think is relevant not only in relation to state law, but also when considering the takings doctrine that was later developed by the Supreme Court. While the narrow view of public use was indeed losing ground at the beginning of the 20th Century, the doctrine of extreme deference that was about to be adopted at the federal level represents a largely new development. The new deference was not originally directed at the legislature, in particular, but primarily towards the judiciary at the state level. Hence, it represent a development that is in some sense incomparable to the earlier case law from the states. The balance of power between states and the federal government also played an important role, which should not be overlooked.

\subsection{Legitimacy as discussed in the Supreme Court}\label{subsec:US}

Initially, the Supreme Court held that the takings clause in the US Constitution did not apply to state takings at all.\footcite{barron33} Federal takings, on the other hand, were of limited practical significance since the common practice was that the federal government would rely on the states to condemn property on their behalf.\footcite[30]{meidinger80}. This changed towards the end of the 19th Century, particularly following the decision in {\it Trombley v. Humphrey}, where the Supreme Court of Michigan struck down a taking that would benefit the federal government.\cite{trombley71} Not long after, in 1875, the first Supreme Court adjudication of a federal taking case occurred, marking the start of the development of the Supreme Court's own doctrine on public use and legitimacy.\footcite{kohl75} Eventually, in 1897, the Court would also hold that state takings could be scrutinized under the takings clause of the constitution.\footcite{chicago97} This was a development that can be traced to the passage of the Fourteenth Amendment to the Constitution after the civil war, concerning due process.\footcite{johnson11}. Indeed, some early Supreme Court cases dealing with state takings were adjudicated against the due process clause directly.\footnote{See, e.g., \cite{head85}.}

After the Supreme Court started developing its own case law on the legitimacy issue, the deferential stance soon became entrenched. As argued by Professor Horwitz, the mid to late 19th Century was the period in US history when control over property was transferred on a massive scale from agrarian communities to various agents of industrial expansion.\footcite{horwitz73} Moreover, it was a period of great optimism about the ability of {\it laissez faire} capitalism to ensure progress and economic growth. This was also reflected in the case law on eminent domain, particularly as developed by the Supreme Court. A particularly clear expression of this can be found in {\it Mt. Vernon-Woodberry Cotton Duck Co v Alabama Interstate Power Co}.\footcite{vernon16}  This case dealt with the legitimacy of a condemnation arising from the construction of a hydropower plant, which the Alabama Supreme Court had upheld against claims that it was unconstitutional under the constitution of Alabama. The presiding judge held that it was valid using quite brisk language:

\begin{quote}The principal argument presented that is open here, is that the purpose of the condemnation is not a public one. The purpose of the Power Company's incorporation, and that for which it seeks to condemn property of the plaintiff in error, is to manufacture, supply, and sell to the public, power produced by water as a motive force. In the organic relations of modern society it may sometimes be hard to draw the line that is supposed to limit the authority of the legislature to exercise or delegate the power of eminent domain. But to gather the streams from waste and to draw from them energy, labor without brains, and so to save mankind from toil that it can be spared, is to supply what, next to intellect, is the very foundation of all our achievements and all our welfare. If that purpose is not public, we should be at a loss to say what is. The inadequacy of use by the general public as a universal test is established. The respect due to the judgment of the state would have great weight if there were a doubt. But there is none.\footcite[]{vernon16}
\end{quote}

The quote serves as an indication of how deference was fast gaining ground, without yet being established doctrine. On the one hand, the Court stresses that deference to the {\it state} judgment (rather than the judgment of the legislature) should be given great weight in legitimacy cases. On the other hand, it prefers to conclude on the basis of its own assessment of the purpose of the taking. This assessment, however, is not particularly grounded in the circumstances on the ground in Alabama, being based rather on sweeping assertions about the ``organic relations of modern society'' and the desire to ``save mankind from toil that it can be spared''. 

This judgment, from 1916, was given during the so-called {\it Lochner} era of jurisprudence in the US, when the Supreme Court  would famously engage in active censorship of regulation that was meant to promote greater social and economic equality.\footcite{cohen08} In particular, much case law from this period witnesses to a general lack of deference. Hence, it is not unexpected to find that public use cases decided on the basis of substantive arguments. However, it is rather more surprising to find that deference actually played an increasingly important role in takings cases.\footnote{The {\it Lochner} era in general was characterized by courts engaging in censorship of state regulation, but this general tendency is not well reflected in how eminent domain law developed over the same period. This is interesting, as it points to the shortcoming of another commonly held view on property protection, namely that it largely serves the interests of property-owning elites, to the detriment of regulatory efforts to promote social equality. The cases through which {\it Lochner} era courts developed the deferential stance suggest a different interpretation; those who benefited most directly from takings in these cases were commercial interests, not vulnerable groups of society. Moreover, they benefited from acquiring land rights from members of agrarian communities, not from the elites. Hence allowing such takings to go ahead was no affront to the ideology of progress through {\it laissez faire} capitalism, quite the contrary. In particular, if it is true as many have argued, that the {\it Lochner} courts were ideologically committed to the promotion of unrestrained capitalism, there was little reason for them to oppose expansion of eminent domain into the commercial arena: those who would be likely to benefit were market actors who were proposing large scale commercial development projects. Indeed, the case law from this period makes it natural to argue that the deferential stance developed primarily to cater to the needs of the capitalists, under the perceived view that they represented the class which would bring progress and prosperity to the nation as a whole.} As early as { \it United States v. Gettysburg Electric Railway Co.}, a case from 1896, deference was described as a fundamental guiding principle, which should be adhered to except in very special circumstances.\footcite{gettysburg96} In particular, Justice Peckham lended his support to the following deferential stance on the public use test:

\begin{quote}
It is stated in the second volume of Judge Dillon's work on Municipal Corporations (4th Ed. § 600) that, when the legislature has declared the use or purpose to be a public one, its judgment will be respected by the courts, unless the use be palpably without reasonable foundation. Many authorities are cited in the note, and, indeed, the rule commends itself as a rational and proper one.\footcite[680]{gettysburg96}
\end{quote}

The case did not turn on the public use issue, however, as the condemned land would be used for battlefield memorials at Gettysburg, Pennsylvania, clearly a public use. In addition, the case concerned a federal takings, authorized by Congress. In later cases, the deferential stance was not adopted in cases originating from the states. As late as in 1930, in {\it Cincinatti v Vester}, the Supreme Court commented that the ``‘It is well established that, in considering the application of the Fourteenth Amendment to cases of expropriation of private property, the question what is a public use is a judicial one".\footcite[447]{vester30} In this judgment, Chief Justice Hughes also describes in more depth how the judicial assessment of the public use question should be carried out, echoing the contextual approach that had been developed in case law from the states.

\begin{quote}
In deciding such a question, the Court has appropriate regard to the diversity of local conditions and considers with great respect legislative declarations and in particular the judgments of state courts as to the uses considered to be public in the light of local exigencies. But the question remains a judicial one which this Court must decide in performing its duty of enforcing the provisions of the Federal Constitution.\footcite[447]{vester30}
\end{quote}

In {\it Hairston v. Danville \& W. R. Co.}, the same idea was expressed even more clearly by Justice Moody, who surveyed the state case law and declared that ``The one and only principle in which all courts seem to agree is that the nature of the uses, whether public or private, is ultimately a judicial question.''\footcite[606]{hairston08} He continued by describing in more depth the typical approach of the state courts in determining public use cases:

\begin{quote}
The determination of this question by the courts has been influenced in the different states by considerations touching the resources, the capacity of the soil, the relative importance of industries to the general public welfare, and the long-established methods and habits of the people. In all these respects conditions vary so much in the states and territories of the Union that different results might well be expected.\footcite[606]{hairston08}
\end{quote}

Justice Moody goes on to give a long list of cases illustrating this aspect of state case law, showing how assessments of the public use issue is inherently contextual and varies from state to state.\footcite[607]{hairston08} He then cites three further Supreme Court cases, pointing out that all of them express similar sentiments of support for state case law on this issue.\footnote{{\it Falbrook, Clark} and {\it Strickley}} Following up on this, he points out that ``no case is recalled'' in which the Supreme Court overturned ``a taking upheld by the state {\it court} as a taking for public uses in conformity with its laws'' (my emphasis). After making clear that situations might still arise where the Supreme Court would not follow state courts on the public use issue, Justice Moody goes on to conclude that the cases cited `` show how greatly we have deferred to the opinions of the state courts on this subject, which so closely concerns the welfare of their people''.\footcite[606]{hairston08}

I believe {\it Hairston} is an important case for two reasons. First, it makes clear that initially, the deferential stance in cases dealing with state takings was largely directed at the state courts rather than the state legislature. Second, it demonstrates federal recognition of the fact that a consensus had emerged in the states, whereby scrutiny of the public use determination was consistently regarded as a judicial task.\footnote{Indeed, {\it Hariston} provides the authority for {\it Vester} on this point. See \cite[606]{vester30}.} Moreover, the Court clearly looked favorably on the contextual approach adopted in such cases, whereby state courts would look to the concrete circumstances of the individual takings and acts complained of. The Court's approval of this tradition, in particular, is explicitly given as the reason for adopting a deferential stance. Put simply, the judicial test provided at state level was held to be of such high quality that there was little use for further scrutiny; a deferential stance was assumed, but made contingent on the fact that state courts would provide the required judicial scrutiny.

Despite this, {\it Hairston} would later be cited as an early authority in favor of almost unconditional deference in {\it US ex rel Tenn Valley Authority v Welch}.\footcite[552]{welch46} This case concerned a federal taking and it cited {\it US v Gettysburg Electric R Co} as an authority in favor of strong deference with regards to the public use limitation.\footcite{gettysburg96} However, the Court also paused to note that the later case of {\it City of Cincinnati v Vester} expressed the opposite view, that the public use test was a judicial responsibility.\footcite{vester30} In a very selective citation, the Court then purports to resolve this tension by quoting {\it Hairston} and the observation made there that the Supreme Court had never overruled the state courts in takings cases. Effectively, the importance of judicial scrutiny is thereby downplayed, although as we saw, the rationale behind {\it Hairston} was that state courts already offered high-quality judicial scrutiny of the public purpose.

{\it Welch} is particularly important because it is used as an authority in the later case of {\it Berman v Parker}, which endorses almost complete deference to the legislature regarding the public use issue.\footcite[32]{berman54} This case concerned condemnation for redevelopment of a partly blighted residential area in the District of Colombia, which would also condemn a non-blighted department store. In a key passage, the Court states that the role of the judiciary in scrutinizing the public purpose of a taking is ``extremely narrow''.\footcite[32]{berman54} The Court provides only two citations for this claim, one of them being {\it Welch}. The other case, {\it Old Dominion Land Co v US}, concerned a federal taking of land on which the military had already invested large sums in buildings.\footnote{The Court commented on the public use test by saying that ``there is nothing shown in the intentions or transactions of subordinates that is sufficient to overcome the declaration by Congress of what it had in mind. Its decision is entitled to deference until it is shown to involve an impossibility. But the military purposes mentioned at least may have been entertained and they clearly were for a public use''. See \cite[66]{dominion25} Hence, the Court took the view that courts should be cautious in second-guessing the intentions of Congress on the basis of what its subordinates had subsequently done and said. This is far from a general deferential stance on public use, and no cases are cited at all, suggesting further that the Court did not think its remarks would be of general significance. Still, a partial quote, used to substantiate  broad deference to the legislature (not only Congress, but also the states) except when it involves an ``impossibility'', has become commonplace. In particular, such a quote was used in the much discussed \cite[240]{midkiff84}.}
In my view, both cases are weak authorities for prescribing general deference regarding public use. Moreover, both cases are concerned with federal takings only, while in {\it Berman} the Court explicitly says that deference is due in equal measure to the state legislature.\footcite[32]{berman54} It is possible to see this as a {\it dictum}, since the District of Columbia is governed directly by Congress, but it is a passage that has had a great impact on future cases. In effect, {\it Berman} caused departure from a significant and consistent body of case law which recognized the important role of the judiciary, at state level, in assessing the purported public purpose of takings. It did so, moreover, without engaging with any of these cases at all.

In {\it Hawaii Housing Authority v Midkiff}, the Supreme Court further entrenched the principles of {\it Berman}, in a case where the state of Hawaii had made used of the takings power to break up an oligopoly in the housing sector.\footcite{midkiff84}  However, the fact that the case made it to the Supreme Court is perhaps suggestive of an increase in the level of worry and tension associated with eminent domain in the 1980s. Indeed, Justice Sandra Day O'Connor, joined by a unanimous Supreme Court, expressed general disapproval of private takings and she appears to have felt the need to provide further qualification for the deferential view, which she did in part by observing that ``judicial deference is required because, in our system of government, legislatures are better able to assess what public purposes should be advanced by an exercise of eminent domain''. Hence, judicial deference was not regarded as an absolute and systemic imperative, as in Berman, but made contingent on the fact that legislatures are ``better able'' than courts at conducting public purpose tests. Hence, some of the contextual ideas from earlier case law is echoed in the decision, but now with respect to the legislature. It should be noted that {\it Midkiff} follows {\it Berman} also in the authorities consulted, and does not consider the cases which had focused on the importance of judicial scrutiny at state level.

The purpose of interference in {\it Midkiff} was to break up an oligopoly to the benefit of tenants, not to further economic development by allowing commercial interests to take land. Hence, the rationale behind the interference is likely to have struck the Supreme Court as sound and just. Moreover, it seems that such an interference would be easy to uphold also under the doctrine of contextual judicial scrutiny of the public use determination. Indeed, Justice O'Connor partly relies on an assessment of the merits of the taking, pointing out that  ``regulating oligopoly and the evils associated with it is a classic exercise of a State's police powers''. In conclusion, the ``extremely narrow'' room for judicial review set up by {\it Berman} seems to have been replaced by a slightly more nuanced formulation, which nevertheless made clear that a legal precedent of deference had now become entrenched. Fine readings aside, {\it Midkiff} reaffirms the main principle:  the meaning of public use can be broad, and the room for judicial review of governmental assessments in this regard is narrow.

So far we have only commented on how the Supreme Court developed its own doctrine on the public use restriction in the early 20th Century. Given that its role in takings jurisprudence was limited up to this point, it is important to consider also the effect on state case law. In particular, what was the fallout of {\it Berman}, which failed to recognize the importance of the tradition for judicial scrutiny that had developed at the state level? A detailed assessment of this against primary sources will have to be left for future work. However, it seems clear that {\it Berman} had a significant effect, both conceptually and in practice. A clear indication of this can be found in the secondary literature. Indeed, most academics following WW2 seemed to converge towards the view that the public use requirement was of little or no judicial importance. Professor Merrill, in an influential paper from 1986, goes as far as to describe it as a ``dead letter''.\footcite{merrill86}. At the same time, eminent domain became more controversial in this period, as it was also put to use more aggressively by some states.

 Some concrete cases proved particularly controversial, and they were taken to illustrate the dangers of eminent domain, particularly in relation to economic development projects. While the takings power had traditionally been used mostly to condemn agrarian land rights, it was now regularly used to condemn middle class homes. The controversy surrounding the case of Poletown Neighborhood Council v. City of Detroit  illustrates this, and the case marks a watershed moment in the history of  economic development takings in the US.\footcite[See][380-381]{sandefur05} In {\it Poletown}, the Michigan Supreme Court held that it was not in violation of the public use requirement to allow General Motors to displace some 3500 people for the construction of a car assembly factory. The majority 5-2 cites {\it Berman}, commenting that its own room for review of the public use requirement is limited.\footcite[632-633]{poletown81}

The {\it Poletown} decision was controversial, and the minority, especially Justice Ryan, was highly critical of it. He objects both to the deferential stance in general and to the majority reading of {\it Berman} in particular, pointing out that the Supreme Court's doctrine of deference was in large part directed at the state courts.\footcite[668]{poletown81} Hence, he concludes, the majority's reliance on {\it Berman} is ``particularly disingenuous''.\footcite[668]{poletown81} 

Justice Ryan was not alone in his disapproval of {\it Poletown} and the case is widely regarded as the prelude to an era of increased tensions over economic development takings in the US. This would culminate with {\it Kelo} which, despite upholding an economic development taking, also signaled a move towards more active judicial review of the public use requirement. This effect of {\it Kelo} has become more clear over time, primarily due to state responses caused by widespread disapproval with the outcome. However, it has also been remarked that both the majority and minority opinions in {\it Kelo} indicate that the Supreme Court itself may not be entirely at ease with the doctrine of strict deference that developed after {\it Berman}. In the next subsection, I will give an overview of recent developments, particularly from the secondary literature.

\section{Economic development takings after Kelo}

The fact that {\it Kelo} was decided against the homeowner met with wide disapproval by the US public. In addition, many scholars expressed concern at what they saw as an ill advised ``abdication'' of the judiciary in takings cases. The minority opinions given in {\it Kelo}, particularly the opinion of Justice O'Connor, also proved influential, causing further attention to be directed at the perceived dangers of eminent domain abuse. A massive amount of literature has since appeared devoted to studying the ``problem'' of economic takings. Moreover,  many states have seen reforms aimed to curb the use of eminent domain for economic development.\footnote{For an overview and critical examination of the myriad of state reforms that have followed {\it Kelo}, I point to \cite{eagle08}. See also \cite{somin09}.} 

As of 2014, 44 states have passed post-{\it Kelo} legislation to curb the use of eminent domain for economic development.\footnote{According to the Castle Coalition, a property activist project associated with the Institute of Justice. See \url{http://www.castlecoalition.org/} for an up-to-date survey of state legislation on eminent domain.} Various legislative techniques have been adopted by the states to achieve this. Some states, including Alabama, Colorado, Michigan, enacted explicit bans on economic development takings and takings that would benefit private parties.\footcite[See][107-108]{eagle08} In South Dakota, the legislature went even further, banning the use of eminent domain  ``(1) For transfer to any private person, nongovernmental entity, or other public-private business entity; or (2) Primarily for enhancement of tax revenue''.\footnote{South Dakota Codified Laws § 11-7-22-1, amended by House Bill 1080, 2006 Leg, Reg Ses (2006).}

In other states, more indirect measures were also taken, such as in Florida, where the legislature enacted a rule whereby property taken by the government could not be transferred to a private party until 10 years after the date it was condemned.\footcite[809]{eagle08} Many states also offer inclusive, often lengthy, lists of uses that should count as public, allowing the states to restrict the eminent domain power while also allowing condemnations that are regarded as particularly important to the state.\footcite[804]{eagle08}
However, as argued by Somin, many of these legislative reforms are largely ineffective in preventing economic development takings.\footcite[2120]{somin09} Somin also points to another interesting trend, namely that state reforms enacted by the public through referendums tend to be far more restrictive and effective in preventing economic and private-to-private takings than reforms passed through the state legislature.\footcite[2143]{somin09} 

This is a further reflection of the extent to which the US public opposed the decision in {\it Kelo}. Surveys show that as many as 80-90 \% believe that it was wrongly decided, an opinion widely shared also among the political elite.\footcite[2109]{somin09} Indeed, {\it Kelo} has had a great effect on the discourse of eminent domain in the US, and this effect is perhaps of greater importance than the various state reforms that have been enacted. According to Somin, most of the reforms have in fact been ineffective, despite the overwhelming popular and political opposition against economic development takings.\footcite[2170-2171]{somin09} 

Somin is not alone in feeling that eminent domain reform has offered more than it could deliver, this is a sentiment that is expressed both by supporters and critics of {\it Kelo}. On the other hand, while practitioners have noted that it is largely business-as-usual in eminent domain law, they also report a greater feeling of unease regarding the public use requirement, expressing hope that the Supreme Court will soon revisit the issue.\footnote{See \cite{murakami13} (``Until the Supreme Court revisits the issue, we predict that this question will continue to plague the lower courts, property owners, and condemning authorities'').} In this way, the public backlash against {\it Kelo} has served as an influential reminder that the rationale behind eminent domain for economic development is largely out of sync with the sense of fairness and justice endorsed by most non-experts. 

The underlying cause of this, according to Somin, can be traced to the fact that people are ``rationally ignorant'' about the economic takings issue. For most people, it is unlikely that eminent domain will come to concern them personally or that they will be able to influence policy in this area. Hence, it makes little sense for them to devote much time to learn more about it. This, in turn, helps create a situation where experts can develop and sustain a system based on principles that, in fact, are opposed by a large majority of citizens.\footcite[2163-2171]{somin09} Indeed, Somin argues that surveys show how people tend to overestimate the effectiveness of eminent domain reform, possibly due to the fact that symbolic legislative measures are mistaken for materially significant changes in the law.\footcite{somin09}

I think Somin's analysis is on an interesting track, although it seems wrong to assume {\it a priori} that people's critical stance on economic development takings would necessarily remain in place if they educated themselves more on the issue. Rational ignorance, in particular, should be seen as a double-edged sword in disputes of this kind. But this does nothing to detract from the main message, which is that the {\it Kelo} backlash seems to have caused greater insecurity about what the law is, without being able to significantly curb those uses of eminent domain that have been deemed problematic. In my opinion, this shows that the static legislative approach to eminent domain reform, which has dominated the scene in the US so far, needs to be supplemented by more dynamic proposals. In particular, it seems important to target the decision-making processes surrounding planning and eminent domain, to look for principles by which this process can be imbued with legitimacy. 

In a country where the population expresses antagonism towards eminent domain for economic development, a more inclusive process will likely cause such takings to become more uncommon. On the other hand, if principles of good governance are put in place, it might also restore confidence in eminent domain as a procedure by which to implement democratically legitimate decisions about how to weigh the interests of landowners against the interests of the public. In the next subsection, I will consider two proposals for principles of this kind. The first targets specifically the question of how compensation is determined in economic development cases, a crucial aspect of legitimacy. The second proposal targets the decision-making process more broadly, by proposing a framework for land assembly that is meant to replace the use of eminent domain in certain circumstances.

%\noo{But it is not the general public that are the major stakeholders in such disputes, but rather the communities that are directly affected, including both the private property owners who will be burdened and those community members who stand to benefit. A good framework for balancing their interests relies on finding appropriate principles of good governance, so that governments can play an empowering role when such decisions are made. This is crucial for legitimacy of land use planning generally, but especially for eminent domain, where the gravity of the interference means that legitimacy is unlikely to arise unless the decision to condemn is firmly rooted in the interests of the main stakeholders. To the greatest possible extent, it also seems crucial to emphasize local conditions and ensure that the decision enjoys broad local support. 
%
%Shortly after {\it Poletown} was overturned, the case of Kelo saw the legitimacy of economic takings brought before the Supreme Court once again. This time there was real doubt and disagreement among the justices regarding the scope of the public use limitation. The case revolved around the legitimacy of condemning a home in favour of a research facility for the drug company Pfizer, which was part of a development plan for the City of New London.  The owner, Suzanne Kelo, argued that the condemnation of her home was in breach of the constitution, since it was a private-to-private taking ostensibly to the benefit of Pfizer rather than any clearly defined public use or interest.
%
%In Kelo, Justice Thomas adopted the strictest view on the public use test. He entirely disregarded  the precedent set by Berman and Midkiff in favour of constitutional originalism, the doctrine which asserts that direct assessment of the wording in the Constitution, and the intentions of the founding fathers, is the approach that should be used to decide constitutional cases. Following up on this he held that actual right of use for the public was the test that had to be applied in takings cases. The hundred years of precedent preceding Kelo was described as “wholly divorced from the text, history, and structure of our founding document", and thus Justice Thomas concluded that it had to be abandoned. 
%
%Justice O'Connor, in an expression of dissent joined by Chief Justice Rehnquist and Justices Scalia
%and Thomas, argued against legitimacy on less theoretical grounds, based on the facts of the case and the precedent that would be set for similar cases in the future. Her main legal argument was that while public use should be interpreted broadly, the possibility of positive ripple effects was not enough to justify private-to-private takings. In particular, Justice O'Connor took a very bleak view on the practical consequences that would arise from allowing economic takings that could be justified only by pointing only to indirect positive consequences for the public. She commented on the majority decision to uphold the taking as follows: 
%
%Any property may now be taken for the benefit of another private party, but the fallout from this decision will not be random. The beneficiaries are likely to be those citizens with disproportionate influence and power in the political process, including large corporations and development firms. As for the victims, the government now has license to transfer property from those with fewer resources to those with more. The Founders cannot have intended this perverse result.
%
%It seems that a major point of contention among the judges in the Supreme Court was whether or not these grim predictions was a realistic assessment of what the consequences of the decision would be. Surely, anyone who agrees with Justice O'Connor in her prediction of the fallout would also agree with here conclusion that it is perverse. But the majority in Kelo, in an opinion written by Justice Stevens, disagreed with her assessment, observing instead that a more restrictive view on economic takings would make it more difficult to cater to the "diverse and always evolving needs of society". 
%
%But the majority opinion also stressed that purely private takings where not permissible, and they attached great significance to the substantive assessment that the actual taking of Suzanne Kelo's home formed part of a comprehensive development plan that would not bestow special benefit on any particular group of individuals. Moreover, Justice Kennedy, in his concurring opinion, emphasised that states should not use public purpose as a pretext for interfering in property rights to the benefit of commercial actors.
%Hence the overall impression one is left with when considering Kelo in its historical and legal context is that it reflects an increasingly cautious attitude to economic takings. The precedent of virtually unlimited deference that was set in case law from the mid-to-late 19th Century was eschewed in favour of a more contextual approach where the merits and deeper purpose of the plans underlying a taking is not axiomatically beyond the scrutiny of the courts.
%
%From considering the reception of the case by the general public, we see even more clearly how Kelo in effect marks a change in the US towards greater scrutiny. 
%
%Indeed, the voices that have dominated in the aftermath of Kelo were critical of the decision and criticized the court for not offering better protection to property owners. The case also led to an a surge of academic interest in the pubic use restriction, with many arguing for further restrictions on the scope of the takings power. 
%Hence it seems that Justice O'Connor's opinion largely reflects contemporary worries about takings in the US, worries that are now also becoming increasingly relevant to how the law develops and is understood. Many states have changed their own eminent domain codes  following Kelo, to make it harder to undertake economic takings. Moreover, the federal government also banned such takings from taking place on the basis of federal takings powers.
%It will lead us astray to delve deeply into the question of what caused this change in perspective on economic takings in the US, but we can offer a few hypothesis. First, it seems that cases such as Poletown illustrates the potential danger inherent in making the power of eminent domain available to market players. In particular, the main worry that has been raised is that the pretext of public purpose may be in the process of becoming a powerful instrument for influential market actors to gain access to regulatory powers of government. As these powers has massively expanded in the post-WW2 period, so has the potential for abuse. In addition, it seems that while those who were adversely affected by eminent domain tended to be less privileged and resourceful groups of society, the takings power is now increasingly brought to bear also against members of the middle class, who are in a better position to fight it, both legally and on the political scene.
%
%While opinions differ greatly both regarding the extent of the problem and the causes of recent controversy, there is something near consensus in the US after Kelo that economic development takings raise special problems under the current system of eminent domain, and that these need to be addressed with a view to reducing tensions and restoring faith in the system. Indeed, even the majority in Kelo hint strongly at this when they say that  
%Some have argued forcefully that a strict reading of the public use requirement is the way forward, if not by strict interpretation then by an explicit ban on economic development takings.  However, it is tempting here to echo the worries expressed in Seawell, that a strict formalistic approach to legitimacy runs the risk not only of being inflexible, but also, eventually, of offering less  protection to property owners. How, then, should we reduce the risk of abuses?
%While many have focused on the question of banning economic taking, or reconsidering the public use clause, some have addressed this question from such a broader angle. In my opinion, this is the way forward. It seems, in particular, that a complete ban on economic development takings will leave a vacuum in the current economic system, which presupposes a great deal of cooperation between commercial and public interest. Particularly when it comes to economic development, the private-public partnership model has gained influence to the point that a ban on economic development takings would likely prove impossible to implement in a satisfactory manner. 
%More generally, it seems hard to address the problem of economic takings without considering the role they play in the larger economic context within which current rules and practices have developed. Based on such considerations, I believe the procedural approach to economic takings is the appropriate one. This perspective asks us to take a closer look at judicial safeguards for protecting the role of property owners in the decision-making processes that lead up to the use of eminent domain. To some extent one might approach this on the basis of existing legal principles, asking for better scrutiny of procedural aspects, or by making it easier to bring pretext claims before the courts. However, it might also require new ideas, and, in particular, the introduction of new institutions for decision-making and administration of the eminent domain process.
%
%In the next section, I will look at two concrete proposals in more detail, one concerning the decision-making step and the other concerning the calculation of compensation. 
%They will be important because they serve as starting points for the case study that is to follow, addressing mechanisms that we will return to in Chapters x and y when we look more closely at two Norwegian legal institutions that share many features with the theoretical roposals discussed in the next section.
%}

\section{Institutional proposals for increased legitimacy}\label{sec:ir}

In this subsection, I first present the Special Purpose Development Companies proposed by Lehavi and Licht.\footcite{lehavi07} I relate this proposals to theoretical approaches to the issue of compensation, before I go on to note some shortcomings and open questions that I will later address in my case study. I then go on to consider the Land Assembly Districts proposed by Heller and Hills.\footcite{heller08} I consider this proposal in light of the stated motivation, which is to design an effective mechanism of self-governance that can replace eminent domain in economic development cases. I present some unresolved questions and argue that there is a tension in the proposal between its narrow scope, imposed to prevent majority tyranny and other forms of abuse, and its broad goal of empowering local communities. 

\subsection{Special Purpose Development Companies}

The primary distinguishing feature of economic development takings is that they give the taker an opportunity to profit commercially from the development. This may even be the primary aim of the project, with the public benefiting only indirectly through potential economic and social ripple effects. Property owners facing condemnation in such circumstances might expect to take a share in the profit resulting from the use of their land. However, in many jurisdictions, including the US, the rules used to calculate compensation prevents owners from getting any share in the commercial surplus resulting from development.\footnote{See, e.g., \cite[965-966]{fennell04}.} In particular, various {\it elimination rules} are typically in place to ensure that compensation is based entirely on the pre-project value of the land that is being taken.\footcite[See][81]{ackerman06} The policy reasons for such rules is that they ensure that the public does not have to pay extra due to its own special want of the property. After all, this is one of the main purposes of using eminent domain in the first place; to ensure that the public does not have to pay extortionate prices for land needed for important projects. However, when the purpose of the project is itself commercial in nature, there appears to be a shortage of good policy reasons for excluding this value from consideration when compensation is calculated. This is especially true when, as in the US, compensation tends to be based on the market value of the land taken. Why should a commercial condemner's prospect of carrying out economic development with a profit be disregarded from the assessment of market value? In any fair and friendly transaction among rational agents, one would expect benefit sharing in a case like this. Yet for economic development backed up by eminent domain, the application of elimination rules ensures that all the profit goes to the developer. 

Some authors have argued that failures of compensation is at the heart of the economic takings issue and that worry over the public use restriction is in large part only a response to deeper concerns about the ``uncompensated increment'' of such takings.\footcite[See][962]{fennell04} In addition to the lack of benefit sharing, previous work has identified two further problems of compensation that also tend to become exasperated in economic development cases. First, the problem of ``subjective premium'' has been raised, pointing to the fact that property owners often value their own land higher than the market value, for personal reasons.\footcite[963]{fennell04} For instance, if a home is condemned, the homeowner will typically suffer costs not covered by market value, such as the cost of moving, including both the immediate ``objective'' logistic costs as well as more subtle costs, such as having to familiarize oneself with a new local community. Second, the problem of ``autonomy'' has been discussed, arising from the fact that an exercise of eminent domain deprives the landowner of her right to decide how to manage her property.\footnote{Discussed in \cite[966-967]{fennell04}. For a general personhood building theory of property law, see \cite{radin93}. For a general economic theory of the subjective value of independence, see \cite{benz08}.}

In \footcite{lehavi07}, the authors propose a novel approach for addressing the ``uncompensated increment'' in economic takings cases. Their proposal is based on a new kind of structure that they dub a {\it Special Purpose Development Corporation} (SPDC). The idea is that owners affected by eminent domain will be given a choice between standard pre-project market value and shares in a special company. This company will exist only to implement a specific step in the implementation of the development project: the transaction of the land-rights. The SPDC may choose either to offer their rights on an auction or else negotiate a deal with a designated developer.\footcite[1735]{lehavi07} Hence, the idea is to ensure that the owners are paid a value that reflects the post-project value of the land, but in such a way that the holdout problem is avoided. In particular, the SPDC will have a single task: to sell the land for the highest possible price within a given time frame.\footcite[1741]{lehavi07} After the sale is completed, the SPDC will divide the proceeds as dividends and be wound up.\footcite[1741]{lehavi07}

Other suggestions have taken a more static approach to compensation reform, such as proposing to give owners a fixed premium in cases of economic development, or developing mechanisms of self-assessment to ensure that compensation is based on the true value the owner attributes to his own land.\footnote{A range of static proposals have been proposed in the literature: Merrill proposes 150 \% of market value for takings that are deemed to be ``suspect'', including takings for which the nature of the public use is unclear, see \cite[90-93]{merrill86}. Krier and Serkin propose a system that provide compensation for a property's special suitability to its owner, or a system where compensation is based on the court's assessment of post-project value, see \cite[865-873]{krier04}. Fennell proposes a system of self-evaluation of property for takings purposes with tax-breaks given to those who value their property close to market value (to avoid overestimation), see \cite[995-996]{fennell04}. Bell and Parchomovsky also propose self-evaluation, but rely on a different mechanism to prevent overestimation; tax liability is based on the self-reported value and no property can be sold by its owner for less than his reported value, see \cite[890-900]{bell07}.} Compared to such proposals, the idea of SPDCs is more sophisticated and should be looked at in more depth. 

The conceptual premise for the proposal is that takings for economic development can be seen as compulsory incorporation, a pooling of resources useful in overcoming market failures.\footcite[1732-1733]{lehavi07} Just as the corporation is formed to consolidate assets in order to facilitate effective management, so is eminent domain used to assemble property rights in order to facilitate efficient organization of development. According to Lehavi and Licht, this also provides a viable approach to problems of ``opportunistic behavior''; hierarchical governance after assembly ensures that order and unity can be regained even if interests in the land are distributed among a large and heterogeneous group of potentially mischievous shareholders.\footcite[1733]{lehavi07} In the words of Lehavi and Licht:

\begin{quote}
The exercise of eminent domain powers thus resembles an incorporation by the government of all landowners with a view to brining all the critical assets under hierarchical governance. Establishing a corporation for this purpose and transferring land parcels to it thus would be merely a procedural manifestation of the substantive economic reality that already takes place in eminent domain cases.
\end{quote}

As soon as we look at the rationale behind economic development takings in this way, any remnant of good policy reasons for ensuring that the developer gets all the profit seems to disappear. Rather, we are led to consider compensation as an issue entirely separate from the exercise of the takings power. After the land has been reorganized by eminent domain and an SPDC has been formed, the land rights might as well be sold {\it freely} to a developer. In this way, the land will be sold for a price that is closer to an actual market value, on the market where the land is destined for development.\footcite[1735-1736]{lehavi07} More generally, the SPDC becomes an aid that the government can use to create more favorable market conditions for transferring land that has commercial potential in its public use. Due to the compulsory pooling of resources, no owner can exercise monopoly power by holding out, but due to decoupling of compensation from assembly, the owners can now negotiate with potential developers for a share of the resulting profit. Moreover, the fact that the SPDC offers its rights on an actual market can also help ensure that more information become available regarding the true economic value of the development, something that may in turn help ensure that only the good projects will be successful in acquiring land. Hence, according to Lehavi and Licht, an additional positive effect of SPDCs is that developers and governments will shun away from using the eminent domain power to benefit projects that are not truly welfare-enhancing.\footcite[1735-1736]{lehavi07}

In addition to these substantive consequences, the SPDC-proposal also stands out because it has a significant institutional component, pointing to its potential for restoring procedural legitimacy as well as substantive fairness. Lehavi and Licht discuss corporate governance issues at some length, but without committing themselves to definite answers about how the operations of the SPDC should be organized.\footcite[1040-1048]{lehavi07} Indeed, while their proposal is perhaps most interesting because of its procedural aspects, it also appears to be rather preliminary in this regard. The main idea is to let the SPDC structure piggyback on existing corporative structures, particularly those developed for securitization of assets.\footnote{See generally \cite{schwarcz94}. For an up-to-date overview, targeting special challenges that became apparent during the 2008 financial crisis, see \cite{schwarcz13}.} The basic idea is that the corporate structure should be insulated from the original landowners to the greatest possible extent; it should have a narrow scope, it should be managed by neutral administrators, and it should entrust a third party with its voting rights.\footcite[1742]{lehavi07} This is meant to prevent failures of governance within the SPDC itself, making it harder for majority shareholders and self-interested managers to co-opt the process. For instance, if a possible developer already holds a majority of the shares in an SPDC, this structure would prevent him from using this position to acquire the remaining land on favorable terms. 

Lehavi and Licht observe that under US law, the government would often be required to make shares in an SPDC available to the landowners as a public offering.\footcite[1745]{lehavi07} Lehavi and Licht deem this to be desirable, arguing that full disclosure will provide owners with a better basis on which to decide whether or not to accept SPDC shares in place of pre-project market value. It will also facilitate trading in such shares, so that they will become more liquid and therefore, presumably, more valuable.\footcite[1746]{lehavi07} 

Lehavi and Licht's proposal is interesting, but I think a fundamental objection can be raised against it. In particular, it seems that their governance model more or less completely alienate property owners from the decision-making process after SPDC formation. Limiting the participation of owners is to a large extent an explicit aim, since governance by experts is held to increase the chances of ensuring good governance. But is expert rule really the answer?

It seems that from the owners' point of view, Lehavi and Licht's proposals for governance reduces the SPDC to a mechanism whereby they can acquire certain financial entitlements. These may exceed those that would follow from standard compensation rules, but they do not directly empower owners vis-{\'a}-vis developers and the government. Instead, a largely independent structure will be introduced. It is this new organizational structure, rather than the owners, that will now become an important actor in the eminent domain process. In principle, it is meant to represent owners, but to what extent can it do so effectively? After all, it is specifically intended to operate as neutral player, charged with maximizing the price, nothing more. Hence, it appears that the SPDC will not be able to give owners an arena to negotiate on the basis of the personal and social importance they attribute to their land rights. How the problem of ``autonomy'' is addressed by the proposal is therefore hard to see and the ``subjective premium'' also appears to be in danger, unless it can be objectively quantified and covered by the surplus from a voluntary sale. But if such quantification is possible, then why not simply tell the appraiser to award some premium under standard compensation rules?

More generally, it seems to me that while all three categories of ``uncompensated increments'' are interesting to study from a financial viewpoint, severe doubts can be raised regarding the feasibility of addressing the subjective aspects of this as a question of compensation. It may be that issues related to ``subjective premium'' and ``autonomy'' are seen as public use issues for good reason; they are hard to quantify otherwise. Moreover, attempting to do so might do more harm than good. On the one hand, it might skew the political process, since owners that have been ``bought off'' don't object to ill-advised development projects, as long as they generate financial revenue. But what about projects that are undesirable for other reasons, for instance because they completely change the character of a neighborhood, or because they are harmful to the environment? On the other hand, the very idea that money can compensate for the subjective importance of property and autonomy can itself prove offensive. At least it seems likely that it would often come to be seen as inadequate and inefficient.\footnote{For more detailed criticism of the compensation approach to the public use issue, see \cite{garnett06}.} Moreover, an owner that is compelled to give up his home after an inclusive process where the public interest has been debated and clearly communicated is likely to feel like he incurs less costs related both to his subjective premium and his autonomy. Hence, the lack of participation in the decision-making process can in itself increase the uncompensated loss. Clearly, no externally managed ``bargain-oriented'' SPDC will be able to resolve this problem. Of course, some ``objective'' elements of, such as relocation costs or cost for juridical assistance, can still be addressed under the banner of compensation. But in most jurisdictions, they already are.\footnote{See, e.g., \cite[121-126]{garnett06}.} For more subtle aspects, the aftermath of {\it Kelo} itself can serve as an illustration of how a compensatory approach is unsatisfactory:

After the case, Suzanne Kelo remained defiant, until she eventually decided to settle in 2006, for an offer of \$ 442 155, more than \$ 319 000 above the appraised value.\footcite[1709]{lehavi07} Apparently, the other owners affected by the same taking were not particularly pleased, arguing that recalcitrant owners were actually rewarded for holding out.\footcite[1709]{lehavi07} On the other hand, there is no indication that Suzanne Kelo was not genuine in her opposition to the taking. Indeed, after the long struggle she had taken part in, it is easy to imagine that financial compensation, if it was to be an effective remedy at all, would have to be very high. Even after she had settled, Kelo apparently toured the country speaking out against economic takings. This, too, is a statement to the inadequacy of a purely financial approach to legitimacy. 

I conclude that SPDCs have serious shortcoming with regards to the subjective aspects of undercompensation, aspects that can only be addressed if the focus turns towards participation. However, SPDCs do seem promising when it comes to profit-sharing. This, after all, is what the structure is specifically aiming to achieve. In addition, I agree that SPDCs will likely have a positive effect on the other actors in the eminent domain process. In particular, I agree with Lehavi and Licht that greater openness is likely to result, revealing the true merits of development projects, at least in so far as these are translatable into financial terms. The fact that developers must negotiate with an SPDC who can threaten to make the land available an an open auction will likely deter developers and government from pursuing fiscally inefficient projects. Hence, the risk that governments will subsidized such projects by giving them cheap access to land will also be reduced. In addition, the presence of a third voice, speaking on behalf of owners, is likely to help achieve a better balance of power in development takings. 

Even if the individual landowners do not have a voice in this process, the fact that the landowners are better represented as a group is then still likely to have a positive effect on legitimacy. On the other hand, as long as the power of the SPDC is limited to choosing the best offer and negotiating over price, it seems that SPDCs will easily end up being dominated by developers and government. This is a particular concern in cases when competition fails to arise after SPDC formation. To ensure that there are other interested parties, in particular, sems like an important precondition for the proposal to work in practice. In this regard, it is important to realize that a lack of interest from other developers may not be due to the superiority of the original developer's plans. It might rather be due to the fact that the scope of the assembly giving rise to the SPDC is so defined as to make alternatives unfeasible. The danger of abuse in this regard seems significant, particularly when developers themselves participate in coming up with the plans that give rise to SPDC formation. 

Moreover, as long as owners remain marginalized in the planning phase, it is easy to imagine situations where the plan itself will be formulated in such a way that only one developer is in a position to successfully implement it. A simple example would be if a prospective developer already owns some of the land that is critical to the plan, and is able to ensure that this land is kept out of the scope of the SPDC. Clearly, if SPDCs are to operate effectively, such instances of manipulation need to be avoided, suggesting that the proposal as it stands needs to be fleshed out in greater detail.

The problems addressed here both seem to point to the fact that the SPDCs, while more flexible than other suggestions, are still too static to achieve many of their objectives. In particular, to arrive at genuine market conditions for assessing post-project value, there is still a need for changes in the dynamics of the planning process underlying the taking. Moreover, to ensure legitimacy, there is a need for a mechanism that goes beyond expert bargaining and provides owners with better access to the decisionmaking process. In the next subsection, I will consider a proposal that aims to address this, by proposing a framework for self-governance. 

\subsection{Land assembly districts}

In a recent article, Heller and Hills propose a new institutional framework for carrying out land assembly for economic development. Interestingly, it is meant to replace eminent domain altogether. The goal is to ensure democratic legitimacy while also creating a template for collective decision-making that will prevent inefficient gridlock and holdouts. The core idea is to introduce {\it Land Assembly Districts} (LADs), institutions that will enable property owners in a specific area to make a collective decision about whether or not to sell the land to a developer or a municipality.\footcite[1469-1470]{heller08} Anyone can propose and promote the formation of a LAD, but both the official planning authorities and the owners themselves must consent before it is formed.\footcite[1488-1489]{heller08} Clearly, some form of collective action mechanism is required to allow the owners to make such a decision. Hiller and Hill suggest that voting under the majority rule will be adequate in this regard, at least in most cases.\footnote{See \cite[1496]{heller08}. However, when many of the owners are non-residents who only see their land as an investment, Heller and Hills note that it might be necessary to consider more complicated voting procedure, for instance by requiring separate majorities from different groups of owners. See \cite[1523-1524]{heller08}.} How to allocate voting rights in the LAD is another issue that require careful consideration, but Heller and Hills land on the proposal that they should in principle be given to owners in proportion to their share in the land belonging to the LAD.\footnote{See \cite[1492]{heller08}. For a discussion of the constitutional one-person-one-vote principle and a more detailed argument in favor of the property-based proposal, see \cite[1503-1507]{heller08}.} Owners can opt out of the LAD, but in this case eminent domain can be used to transfer the land to the LAD using a conventional eminent domain procedure.\footcite[1496]{heller08}

Heller and Hills envision an important role for governmental planning agencies in approving, overseeing and facilitating the LAD process. Their role will be most important early on, in approving and spelling out the parameters within which the LAD is called to function.\footcite[1489-1491]{heller08} Hence, it appears to be assumed that the planning authorities will define the scope of the LAD by specifying the nature of the development it can pursue. A possible challenge that arises, and which Heller and Hills do not address at any length, is that the scope of the LAD needs to be broad enough to allow for meaningful competition and negotiation after LAD formation. At the same time, however, there will probably be a push, both by governments and initiating developers, to ensure that the scope is defined narrowly enough to give confidence that rezoning permissions will not be denied at a later stage. Another potential challenge is that the planning authorities might have an incentive to refuse granting approval for LAD formation, since it effectively entails that they give up the power of eminent domain for the land in question. For this reason, Heller and Hills propose that a procedure of judicial review should exist whereby a decision to deny approval for LAD formation can be scrutinized.\footcite[1490]{heller08} 

After the formation of the LAD, the government will have a less important position, but the planning authorities will still occupy an important facilitating role. Heller and Hills envision a system of public hearings, possibly organized by the planning authorities, where potential developers meet with owners and other interested parties to discuss  plans for development.\footnote{See \cite[1490-1491]{heller08}.} In this process, it is assumed that also other voices will be represented, such as owners of adjoining land, who can use this opportunity to express objections against the project. Their role in the process is not clarified, but presumably the planning authorities would be able to offer this group some protection, if not in relation to the LADs own operations, then later in relation to the decision whether or not to grant the licenses needed for the development project.

Importantly, if the owners do not agree to forming a LAD, or if they refuse to sell to any developer, the government will be precluded from using eminent domain against them to assemble the land.\footcite[1491]{heller08} This is the crucial novel idea that sets the suggestion apart from other proposals for institutional reform that have appeared after {\it Kelo}. LADs will not only ensure that the owners get to bargain with the developers over compensation, it will also give them an opportunity to refuse any development to go ahead, if they should so decide. Hence, the proposal shifts the balance of power in economic development cases, giving owners a greater role also in preparing the decision whether or not to develop, and on what terms. In my opinion, this makes the proposal stand out as particularly interesting in the recent literature on economic takings. It is the first concrete suggestion that addresses the democratic deficit in a dynamic, procedural manner, without failing to recognize that the danger of holdouts is real and that institutions are needed to avoid it, also in economic development cases.

There are some problems with the model, however. Kelly points out that the basic mechanism of majority voting is itself imperfect, and can lead both to overassembly and underassembly, depending on the circumstances.\footcite{kelly09} He points out, in particular, that if different owners value their property differently, majority voting will tend to disfavor those with the most extreme viewpoints, either in favor of, or against, assembly. If these viewpoints are assumed to be non-strategic and genuine reflections of the welfare associated with the land, the result can be inefficiency. In a nutshell, the problem is that a majority can often be found that does not take due account of minority interests. For instance, if some owners are planning alternative development, leading them to attribute a high {\it hope}-value to their land, they can safely be ignored as long as the majority have no such plans. This could become particularly bad in cases when the alternative development itself is more socially desirable than the development that will benefit from assembly. The role of the LAD in such cases will not improve the quality of the decision to develop, since it pushes the decision-making process into a track where those interests that {\it should} prevail are voiced only by a marginalized minority inside the new institution.\footnote{Of course, one might imaging these landowners opting out of the LAD, or pursuing their own interests independently of it. However, they are then unlikely to be better off than they would be in a no-LAD regime. In fact, it is easy to imagine that they could come to be further marginalized, since the existence of the LAD, acting ``on behalf of the owners'', might detract from any dissenting voices on the owner-side.}

More generally, the lack of clarity regarding the role of LADs in the planning process is a problem. As it stands, the proposal leaves it uncertain how LADs will affect the decision-making process regarding development. But the ideal is clearly stated: LADs should help to establish self-governance in land assembly cases. In particular, Heller and Hills argue that LADs should have ``broad discretion to choose any proposal to redevelop the neighborhood -- or reject all such proposals''.\footcite[See][1496]{heller08} As they put it, two of the main goals of LAD formation is to ensure `` preservation of the sense of individual autonomy implicit in the right of private property and preservation of the larger community's right to self-government''.\footcite[See][1498]{heller08} Unfortunately, these ideals are somewhat at odds with the concrete rules that Heller and Hills propose, particularly those aiming to ensure good governance of the LAD itself. 

In relation to the governance issue, Heller and Hills echo many of the ``corporate governance''-ideas that also feature heavily in Lehavi and Licht's proposal. Indeed, in direct contrast to their comments about ``broad discretion'' and ``self-governance'', Hiller and Hills also state that ``LADs exist for a single narrow purpose -- to consider whether to sell a neighborhood''.\footcite[See][1500]{heller08} This is a good thing, according to Heller and Hills, since it provides a safe-guard against mismanagement, serving to prevent LADs from becoming battle grounds where different groups attempt to co-opt the community voice to further their own interests. As Heller and Hills puts it, the narrow scope of LADs will ensure that ``all differences of interest based on the constituents' different activities and investments, therefore, merge into the single question: is the price offered by the assembler sufficient to induce the constituents to sell?''.\footcite[1500]{heller08}

But this means that there is an internal tension in the LAD proposal, between the broad goal of self-governance on the one hand and the fear of neighborhood bickering and majority tyranny on the other. It is hard to see, in particular, how the idea of LADs with a ``narrow purpose'' is compatible with a scenario where the LAD has ``broad discretion'' to choose between competing proposals for development. If such discretion may indeed be exercised, what is to prevent special interest groups among the landowners from promoting development projects that will be particularly favorable to them, rather than to the landowners as a group? And what is to prevent landowners from making behind-the-scene deals with favored developers at the expense of their neighbors? It seems like a great challenge to come up with rules that prevent mechanisms of this kind, without also constraining the landowners so much that meaningful ``self-governance' becomes an impossibility. If a LAD is obliged to only look at the price, this might prevent abuse. But it will not give owners broad discretion to choose among development proposal. Effectively, it will render LADs as little more than a variant of SPDCs, where the owners are awarded an extra bargaining-chip: the option to refuse all offers. 

In my view, such a restriction on the operations of LADs is not desirable. It is easy to imagine cases where competing proposals, perhaps emerging from within the community of owners themselves, will emerge in response to the formation of a LAD. Such proposals may involve novel solutions that are superior to the original development plans, in which case it is hard to see any good reason why they should not be taken into account, even if they are less commercially attractive. In particular, the formation of a LAD and the competition for development that ensues creates an opportunity for tapping into a greater pool of ideas for redevelopment, ideas which may then also be rooted more firmly in the local community. Surely, getting such proposals to the table would be desirable and it would take us to the heart of self-governance. At the same time, it is easy to acknowledge that problematic situations may arise, for instance if a majority forms in favor of a scheme that involves razing only the homes of the minority, maybe on the rationale that these are ``more blighted''. That would likely give rise to accusations of unfair play, which may or may not be warranted. But irrespectively of this, an alternative project of this kind might well be a better use of the land in question, also from the point of view of the public. Hence, it would seem that the planning authorities would be obliged to give it some serious consideration. Then, however, the LAD has truly become an arena for a new kind of power play among different interest, and a potential vehicle of force for whomever secures support from a majority of owners within the district.

In their proposal, Heller and Hills are aware of this potential problem, which they propose to resolve by strict regulation. In particular, they argue that ``LAD-enabling legislation should require especially stringent disclosure requirements and bar any landowner from voting in a LAD if that landowner has any affiliation with the assembler''.\footcite{heller08} But this raises further questions. For one, what is meant by ``affiliation'' here? Say that a land owner happens to own shares in some of the companies proposing development. Should he then be barred from voting? If so, should he be barred from voting on all proposals, or just those involving companies in which he is a shareholder? If the answer is yes, how would this be justified? Would it not be easy to construe such a rule as discrimination against landowners who happen to own shares in development companies? On the other hand, if the landowner in question is allowed to vote on all other proposals, would it not be natural to suspect that his vote is biased against assembly that would benefit a competing company? Or what about the case when some of the land owners are employed by some of the development companies? Should such owners be barred from voting on proposals that could benefit their employers? This seems quite unfair as a general rule, especially if a low-level employment relationship has such a dramatic effect. But in some cases even low-level ties could play a decisive factor. This might happen, for instance, if an important local employer proposes development in a neighborhood where it has a large number of employees.

Of course, the most pressing issue that arises is the following: who exactly should be empowered to make the determination of when an affiliation is such that an owner should be deprived of his voting rights? Heller and Hills give no answer, but it is easy to imagine that whoever is given this task in the first instance, the courts would soon enough be asked to consider the question. At this point, the circle has in some sense closed in on the proposal. In particular, one might ask: why is it easier to determine if someone can be deprived of his voting rights due to an ``affiliation'', than it is to determine if someone can be deprived of his land due to some planned ``public use''?

In any event, to come up with a set of rules ensuring that LADs can deliver both self-governance and good governance largely remains an open problem. This is acknowledged by Hiller and Hills themselves, who point out that further work is needed and that only a limited assessment of their proposal can be made in the absence of empirical data. Later in the thesis, I will shed light on this challenge when I consider the Norwegian rules relating to land consolidation, showing how these can be looked at as a highly developed institutional embedding of many of the central ideas of LADs. The assessment of how they function in cases of economic development, and how they are increasingly used as an alternative to expropriation in cases of hydro-power development, will allow me to shed further light on the issues that are left open by Heller and Hills' important article.

\section{Conclusion}

In this chapter, I have given a more in-depth presentation of economic development takings in law. I began by noting that the issue is particularly pressing for land users that are not regarded as bringing about economic growth. Hence, I argued that the issue is closely related to that of land grabbing, which is currently receiving much attention, both academic and political. Under the social function understanding of property there is in principle no difference between protecting property rights arising from formal title and property rights arising from use. That said, special issues arise in the latter case, not least because it is unclear how the law should deal with rights resulting from cultural practices that western property regimes are not designed to handle. In addition, I noted that special issues related to food security and poverty arise with particular urgency in relation to land grabbing.

Moreover, the nature of my case study makes it more natural for me to focus on traditional western systems of property law. Hence, I went on to shed more light on discuss how economic development takings are dealt with in such legal systems, focusing on Europe and the US respectively. For the case of Europe, this assessment was made more difficult by the fact that the category is not an established part of legal discourse. However, by looking to England and Germany as concrete examples, I noted that such cases do arise and that they are increasingly seen as controversial. I also noted that there is a contrast between how England and Germany approach such cases, as well as how they approach property more generally. Germany, in particular, goes further in explicitly recognizing the social functions of property, by actively looking to social and political values when assessing whether interferences are legitimate. In England, similar reasoning is at most applied indirectly, as takings are approached almost entirely as an issue of administrative law. 

I then went on to consider the property protection offered by P1(1) of the ECHR, and how it is applied by the Court in Strasbourg. I zoomed in on those aspect that I believe to be the most relevant for economic development takings. While I noted that this category has yet to be discussed by the ECtHR, I argued that a recent shift in the Court's property adjudication is suggestive of the fact that it would likely approach such cases similarly to how Justice O'Connor approached {\it Kelo}. In particular, I noted how the Court has recently adopted a stricter standard of assessment. This standard, I argued, is characterized primarily by increased sensitivity to systemic imbalances causing alleged P1(1) violations. Hence, to regard economic development takings as a special category appears to fit well with recent jurisprudential developments at the Court in Strasbourg.

I went on to consider US sources on economic development takings, noting that the issue has receive an extraordinary amount of attention in recent years. I adopted an historical approach to the material, by tracing the case law surrounding the public use restriction in the fifth amendment to the US constitution, which was much debated even before the specific issue of economic development takings rose to prominence. I focused particularly on case law developed by state courts, and I argued that it shows great sensitivity to the need for contextual assessment. Indeed, I argued that originally, many state courts implicitly adopted a social function view on property when they assessed such cases.

I then looked at the history of Supreme Court adjudication of public use cases. I noted that the doctrine of deference was developed early on, but that it was initially directed mainly at state courts. In fact, I showed that the Supreme Court itself noted with approval the contextual and in-depth approach these courts would rely on when dealing with this issue.

The shift, I argued, came with \textcite{berman54}, in which the Supreme Court adopted a deferential doctrine that was directed specifically at the {\it state legislature}. This, I argued, was quite a dramatic departure from the Court's previous attitude towards {\it state} takings. In fact, it was almost entirely backed up by precedent set in cases when {\it federal} takings had been ordered by Congress. I went on to consider the fallout of \textcite{berman84} at state level, which culminated with the infamous {\it Poletown} case. This case prompted wide-spread accusations of eminent domain abuse and thus it set the stage for {\it Kelo}.

After completing the historical overview, I went on to consider the literature after {\it Kelo}. I expressed particular support for those responses that focus on the need for {\it institutional} reform, to address precisely those dangers that Justice O'Connor pointed to in her minority opinion. As a shorthand, I proposed referring to the mechanisms she identified as the {\it democratic deficit} of economic development takings. 
% I zoomed in on two of those in particular, the Special Purpose Development Companies proposed by Lehavi and Licht, and the Land Assembly Districts suggested by Heller and Hills. I gave an in-depth presentation of these two proposals, pointing out strengths and weaknesses. 
%%In coming chapters, I will refer back to this as I consider similar institutions and mechanisms that are currently operating in Norwegian law relating to hydropower development.

I then gave a thorough presentation of two recent reform suggestions that might help address this deficit. Both are institutional in nature, based on setting up formally recognized coalitions of land owners that can act as a counterweight to the disproportional power of commercial beneficiaries. The first suggestion, by Lehavi and Licht, limited its attention to compensation, recognizing the need for a system whereby the land owners are compensated based on post-project value.  This idea in itself represents a fairly dramatic break with the currently dominant doctrine in takings law, where compensation is almost always, and in almost all jurisdictions, based on the pre-project value of the land, or the {\it value to the owner}.

In Chapter \ref{chap:5}, I will return to this principle in more depth, looking at how it developed internationally, and in Norway. I will also look at how it has now been abandoned in Norwegian law, for some case types involving hydro-power development. I relate this to the special role played by the appraisal courts in Norway. The local grounding of these courts, involving lay people sitting as court appointed appraisers, allows the law to be applied in a way that adapts to the concrete circumstance in a way that may enhance the perceived fairness and legitimacy of the taking. At the same time, however, the judicial procedure, with a (limited) possibility for appeal, puts in place safeguards against abuse.

The second suggestion I looked at in depth, proposed by Heller and Hills, focused not on compensation, but on the decision-making process leading to an economic development project being implemented. This proposal is based on the idea that local communities should be entitled to greater self-governance in such cases. At the same time, it recognizes the need for a mechanism to avoid inefficient and socially harmful gridlock due to holdouts among unwilling owners. Instead of eminent domain, however, a different mechanism is proposed: a land assembly district (LAD). 

This is also a new class of institutions, and I pointed out some problems and seeming inconsistencies in the proposal, regarding the exact role they will have in the planning process. I argued that while the risk of abuse and failure increases with the level of participation, so does the positive effect on legitimacy. I concluded that to reduce the democratic deficit in economic development cases, a wide power of participation must be granted to the land owners and their (immediate) communities. This is needed, in particular, to restore balance in the relationship between owners and others directly connected with the land, the planning authorities, and the commercial actors interested in development for profit. The question that is as of yet unresolved is how to organize such participation in a way that avoids obvious pitfalls, such as administrative inefficiency and tyranny by majorities or elites that gain control of the local agenda.

In Chapter \ref{chap:6}, I will shed light on this question by considering the Norwegian institution of land consolidation, which has very long traditions. It is a flexible frameworks which includes, among other things, a a template for establishing institutions that can function as a LAD. I will focus on how land consolidation functions in cases of economic development that would otherwise likely be pursued by eminent domain. Norwegian hydro-power development will again be in focus, but I will also discuss planning law and development more generally, as the Norwegian government is now considering making consolidation, traditionally a rural institution, a primary mechanism for land development also in urban areas.

Before I delve into this, I will use the next chapter to present an overview of Norwegian hydropower and the role of waterfalls as private property.
\chapter{Introduction and Summary of Main Themes}\label{chap:intro}

\begin{quote}
Thieves respect property; they merely wish the property to become their property that they may more perfectly respect it.\footnote{G.K. Chesterton, {\it The man who was Thursday: A nightmare}.}
\end{quote}
%
%A human being needs only a small plot of ground on which to be happy, and even less to lie beneath. %\footnote{Johan Wolfgang von Goethe, {\it The sorrows of young Werther and selected writings}.}
%\end{quote}
%“That's what makes it ours - being born on it, working on it, dying on it. That makes ownership, not a %paper with numbers on it.”
%― John Steinbeck, The Grapes of Wrath 
%
%
%s
\section{Property Lost; Takings and Legitimacy}

%Judging from academic literature on the subject, property is a difficult and often paradoxical concept,
As a concept, property has been something of a problem child for western philosophy. Time and again it has demonstrated its recalcitrant nature. It has proven, in particular, that it has an inherent tendency to stir up divergences that break out of the realm of conceptual analysis to enter the realm of politics, often in a way that makes it difficult to continue a meaningful and inclusive academic debate.

From the extreme antagonism directed at it by radical Marxists to the evangelical praise bestowed on it by libertarians, there is no shortage of politically charged accounts of what actually property is, not to mention what it {\it should} be, assuming, of course, that it is entitled to exist at all. 
Moreover, there appears to be little room for rapprochement between many leading strands of thought. Indeed, one is often left with the impression that different philosophical theories of property tend to diverge largely due to personal conviction, rather than differences based on reasoned argument. %Indeed, one may argue that different ideas of property, practical and theoretical, are behind most, if not all, the major conflicts and confrontations that have shaped the society in which we live.
%Responding to this, some prominent philosophers have taken the view that property is not a concept suitable for philosophical study at all.

Indeed, some philosopher have argued that property is a lost cause, not suitable for conceptual analysis at all. Rather, it has been suggested that property is best taken as a pragmatic and contingent derivative of other notions, such as the social order, or, on a normative account, {\it justice}. 

Lawyers might object to such a claim at first sight, but actually, they rarely need to consider the philosophical underpinnings of the notion of property that they find entrenched in the law. Indeed, when grappling with hard cases there is little doubt that legal professionals are soon forced to abandon conceptual clarity in favour of considerations based on value judgements, politics and expediency.

Legal scholars, for their part, are usually content with theories of property that remain largely descriptive, settling for the more modest aim of exploring how best to think of property given the prevailing legal order, rather than trying to come up with theories to explicate its nature as a pre-legal concept.

Hence, even the legal philosopher may well end up loosing sight of property as a concept, coming to doubt that there even is such a thing. As one prominent UK scholar put it, property sometimes appears to be hanging in ``thin air''.

Still, in certain situations, we can record what seems to be empirical evidence to support the claim that humans reflect a working {\it primitive} notion of property, one which arguably pre-exists any particular social arrangements used to mould property as a socio-political category of law. Most notably, humans, as well as many other species of animal, appear to have an innate ability to recognise {\it thievery}, the taking of property by someone who is not entitled to do so.

Indeed, the famous dictum ``property is theft'', may be more than a flippant and seemingly self-contradictory comment on the origins of inequality. In fact, it might point to a possible alternative direction for investigating the nature of property as such, as a concept that emerges from a more basic distinction between legitimate and illegitimate acts of taking, broadly construed. It seems quite tempting, after all, to describe a person's property as that which they have taken by legitimate means, which may not be taken from them without due process.

Interestingly, while the abstract notion of property has arguably received more than its fair share of attention from disciplines other than law, the common-sense notion of a taking has not received much academic attention outside of the legal community. No great philosophical debates have revolved around this notion, and no chasms has opened as to the correct way to understand it. Moreover, legal scholars rarely attempt to define takings as an abstract notion, preferring instead to regard it as a derivative of the legal order surrounding property. Here, the pragmatic and often jurisdiction-bound perspective of the lawyer appears to reign supreme.

This is problematic, since the notion of property itself is so contested that it might not provide a secure foundation for thinking about takings. But it also suggests an interesting possibility; perhaps studying takings is a path towards a better understanding of property as well? After all, this is where vastly different accounts of property do seem to share at least an important common point of reference.
It is my hope that this thesis will shed some modest light on this idea.

However, I should make clear at the very outset that I will limit myself to the study a certain kind of taking, namely that which is implemented, or at least formally sanctioned, by a government. In legal language, especially as developed in the US, such acts of government are referred to as takings {\it simpliciter}, while talk of other kinds of ``takings'' require further qualification, e.g., in case of contract, theft, tax or occupation. This in itself might be cause for reflection as to the ideological commitments inherent in legal language. Moreover, it brings the issue of legitimacy to the forefront in an instructive way.

We are reminded, in particular, that under the rule of law, taking is not the same as theft. Rather, the default assumption is that the takings that take place are legitimate. If they are not, we may call them by a different name, but not before. At the same time, it falls to the legal order to spell out in further detail what restrictions may be placed on the power to take. Restrictions, in particular, appear implicit in the very notion of taking something, no less so when the taker is the government or someone that the government endorses. Indeed, if the power to take was unrestricted, how could one distinguish the act of taking something from the act of putting something to use, for a while, waiting for the next taker to come along? In particular, the idea that someone might have occasion to resist an act of taking, and may or may not have good grounds for doing so, appears fundamental to our intuitions concerning the notion itself.

But how should we approach the question of legitimacy of takings, and what conceptual categories can we benefit from when doing so? In this thesis, I aim to make a contribution to this question. I will focus on a special case, namely the so-called economic development takings, when government sanctions the taking of property in order to further economic development.

My primary interest lies in the legal questions that arise, not the overarching philosophical reflections that these might give rise to. However, I believe these introductory remarks fit my chosen topic well. I think it is striking, in particular, how recent case law from the US shows that vastly different perspectives on property may indeed come together when the focus is shifted towards the legitimacy of takings.

The best example is the case of {\it Kelo}, which propelled the category of economic development takings to the political scene, and subsequently led to a surge of academic work by US scholars. 
This case concerned a house that was taken by the government in order to accommodate private enterprise, more specifically the construction of new research facilities for Pfizer, the multi-national pharmaceutical company.

The homeowner protested the taking on the basis that it served no public use and was therefore illegitimate. In keeping with long-standing precedent in the US, she duly lost the case in the Supreme Court, but this created a backlash that is unique in the history of US jurisprudence. 
In relation to {\it Kelo}, in particular, commentators from very different ideological backgrounds came together in a shared scepticism towards the legitimacy of economic development takings.

This is particularly noteworthy in light of the fact that their scepticism had a very limited basis in US law, as the {\it Kelo} decision itself did not appear particularly controversial to property lawyers. Hence, when the response was overwhelmingly negative, from both sides of the political spectrum, it seems that people were responding to a deeper notion of what counts as a legitimate act of taking. 

The critical response to {\it Kelo}, specifically, did not appear to have been primed by the prevailing legal order. It may have been a reflection of political sentiment, but as such it arguably also involved pre-legal notions pertaining to the legitimacy of takings. Simply stated, people from across the political spectrum simply found the outcome to be {\it unfair}.

Notions of fairness such as these are surely worthy of consideration, also from legal scholars.  
At least, the point to the fact that lawyers and law makers had better recognise that cases such as {\it Kelo} are worthy of special consideration. Indeed, after {\it Kelo}, most US states have passed some sort of legislation to limit economic development takings, in a direct response to the controversy following the {\it Kelo} case. 

I believe the fact that this was largely a popular movement strongly suggests the possible relevance of economic development taking as a legal category outside the context of US law. There are certainly significant differences between takings law and practice in the US compared to many other jurisdictions, e.g., in Europe. However, the backlash of {\it Kelo}, particularly the manner in which public opinion diverged so dramatically from the outcome dictated by established case law, suggests the transformational potential inherent in the category of economic development takings as such.

As soon as critical attention is directed at the special issues that arise in cases such as {\it Kelo}, it might well be that people will have a tendency to judge the issue of fairness similarly, irrespectively of divergences in the surrounding legal framework used to deal with such cases. At any rate, it seems that the choice of thinking about certain kinds of takings as takings that primarily seeks to bestow profit on a commercial entity is a choice that is likely to carry with it also a changed perception of legitimacy.

The question, then, becomes to what extent one may appropriately speak of economic development takings in this way. Here, I believe the first important step is to acknowledge at least the {\it potential} for characterising cases of taking for economic development in this fashion. The possibly problematic nature of such takings, pertaining to the commercial interests on the side of the taker, should at least be acknowledge as a relevant dimension along which to asses cases.

This claim is by no means self-evident. For instance, it seems that many European jurisdictions implicitly reject such a perspective, already by failing to recognise that the category of economic development takings can be a useful anchor for reasoning about legitimacy. This brings me to the first key focus point that I will explore in this thesis.

\section{Economic Development Takings as a Conceptual Category}

%This thesis investigates the category of property interferences that are known as {\it economic development takings} in the US. This category came to prominence only quite recently, following the influential {\it Kelo} case.\footnote{See \cite{kelo05}.} 
While the category of economic development takings does not appear to be well established outside of the US, the influence of the US debate is now beginning to show elsewhere, including in Europe.\footnote{See, e.g., \cite{verstappen14}.} It is a problem, however, that the exact meaning of the category may differ depending on who you ask. It is quite common, for instance, to speak of ``private'' takings, seemingly more or less as a synonym to economic development takings. This, however, is clearly not appropriate if one aims for conceptual precision.

Moreover, a clear definition of economic development takings is largely missing in the literature. Rather, scholarship on these kinds of takings rests on an intuitive understanding of the term, firmly based on the US jurisprudence from which it first arose. At its core, the economic development taking is characterised by a commercial purpose, meaning that a commercial interested party, often private, stands to gain a significant financial benefit from compulsorily acquiring private property. This financial motivation for the taker contrasts with the public spirited motivation of the executive or legislative body that grants permission to use compulsion; the (stated) intention of economic development takings is to promote public interests, not to bestow commercial benefits on particular parties.

In my opinion, the tension between public interest and commercial gain in property interference is of general interest in any system of government that combines a market-based economy with wide state powers over the use and distribution of property. Hence, I believe the notion of an economic development taking makes sense to study also outside of the US context wherein it arose. In this thesis, I set out to argue for this claim in more depth. 

In the first part of my thesis, I do so from a theoretical point of view, by arguing that the category arises naturally already at the theoretical level, provided one chooses a suitable theoretical framework for reasoning about takings and property. Moreover, I set out to distil some lessons from the US debate and its history. In addition, I briefly assess the status of economic development takings in Europe, where takings that benefit commercial interests are often not recognised as raising any special questions of legal relevance. 

I argue that this is a shortcoming of the narrative of property protection in Europe, and I also suggest that the concept of an economic development taking would in fact fit well with jurisprudential developments at the ECtHR, which stresses both the need for contextual assessment and attention to possible systemic imbalances in the expropriation practices of member states.

At the same time, I note that in the US, most work on economic development takings has been anchored in the so-called ``public use'' requirement of the Fifth Amendment. Indeed, some authors argue that economic development takings are impermissible already because taking property for development cannot ever be said to constitute a ``public use'' of the property. Moreover, even scholars who reject this view tend to agree that the public use of a taking is less obvious, and should be subjected to more intense judicial scrutiny, in economic development cases.


Interestingly, requirements similar to the public use test are found in many jurisdiction, in various guises, e.g., in rules referring to the need for a {\it public interest} or a {\it public purpose} for takings. On this basis, interesting comparative work has been carried out on the basis of the idea that such a requirement is at the core of the legitimacy issue that arises for economic development takings.

In this thesis, I challenge this perspective. I do so by first reconsidering the history of the public use debate itself, as documented by case law in the US. I argue, in particular, that more attention should be paid to the fact that the state courts that originally set out to develop public use tests in the 19th century adopted a highly contextualised approach. Importantly, these courts where largely not bothered by the fact that they could not pin down any definite and consistent meaning of ``public use'' as a general concept. 

Rather, the public use test was simply used as an expedient way of subjecting various acts of taking to a concrete fairness assessment, in the hope that local courts might help deliver corrective justice in cases when the takings power appeared to have been used in an objectionable manner. In this way, the original purpose of the public use test was tailored towards setting up a framework for judicial review that appears quite similar to how the European Court of Human Rights (ECtHR) currently choose to approach cases dealing with property.

The jurisprudence at the ECtHR typically directs focus away from the question of whether the aim of a taking is legitimate in itself towards the more contextualised question of whether or not the interference is {\it proportional} given the circumstances. This, I argue, is also how the public use test was also originally used by state courts in the US, before the issue of legitimacy turned federal and became subject to a more abstract form of assessment, leading eventually to a tradition for passive deference that gave rise to {\it Kelo}.

In fact, as soon as the issue of proportionality has been flagged as the primary question, it is not clear that the words ``public use'' are of much interest at all. Hence, my conceptual assessment can be summarised by the following two propositions. First, that the notion of an economic development taking, as developed in the US, is a useful addition for thinking about the legitimacy of takings, in any jurisdiction that aims to place meaningful restrictions on the takings power. Second, that the current focus on the notion of a ``public use'', which is supposed to provide the desired protection against transgressions, is largely misguided. At the very least, I believe alternatives should also be considered. This brings me to the second focus point of my thesis.

\noo{

So far, the study of such takings has mainly been carried out by US lawyers, who asses it against the Fifth Amendment of the US constitution. This work has attracted some attention elsewhere, but the category of an economic development taking is by no means a universal category of legal analysis.

Perhaps it should be? This is the first main question that I address in this thesis. But has so far not made much of an impact in other jurisdictions, 

Hence, the somewhat counter-intuitive  Hence, one of the ultimate 
not, in particular, a straightforward 

Clearly, the most crucial question that faces any act of taking, government sanctioned or otherwise, is how we should approach objections against it.


To bring about economic development. 
Indeed, the notion of an illegitimate taking, regardless of whether it is seen as an affront to property or at it's core, certainly seems to have stirred the imagination of most property theorists.

More concretely, 


Perhaps a possible route to a better understanding of property goes by way of the study of takings, and th

re, incidentally, is also where we find an apparent sense of commonality between radically different accounts of property and its nature. It is worth noting, in particular, that 

More generally, it seems that people regularly engage in reasoning that seeks to determine what counts as legitimate and illegitimate ways of acting on objects in the material world. Perhaps it is in the midst of such judgements, is also where property as a concept plays a world.

Hence, to understand property as a concept, perhaps a viable route towards progress is to examine its negation, aiming, if nothing else, for a negative definition in terms of legitimate and illegitimate acts involving objects in the material world. In fact, such a more modest approach forces us immediately to recognise certain 

Luckily, as lawyers, we rarely have to worry about the {\it nature} of property, at least not in the philosophical sense of the word. Instead, we can focus on its {\it function}, in the legal system within which we operate. Still, the ``moderate'' and pragmatic view of property that lawyers tend to adhere to might not be very satisfying, particularly not when we are moving to the margins of the legal order, by considering hard cases that raise questions of policy and require novel normative assessments. In such circumstances, the pragmatic stance on property - as taught in law school and applied in courts - can sometimes appear bland, even vacuous, unable to accommodate solutions to genuinely difficult legal problems. These are the cases when received wisdom breaks down, the cases when logic -- or, more generally, {\it reason} -- must be called on to fill the gap left by experience, to paraphrase the famous words of Holmes.\footnote{Holmes quote and critique.}

Of course property is a social construction, a construction of law, but even so, we continue to question its {\it nature}, based on the implicit understanding that there is something more to property than the legal fictions that are used to package it in our legal order.
}

\section{The Democratic Deficit of Takings Law}

%which link the question of their legitimacy to the public use test prescribed in the Fifth Amendment of the US Constitution.
I am not the first to challenge the traditional narrative that surrounds economic development takings. Indeed, some US scholars have now begun to argue forcefully that increased judicial scrutiny of the public use requirement is neither a necessary nor a sufficient response to concerns about the legitimacy of commercially motivated takings. Instead, these authors argue that the takings procedure as such is not appropriately designed to deal with commercial incentives on the taker side.

This claim has in turn led to procedural proposals for takings law reform, most notably Professors Heller and Hills' article on Land Assembly Districts and Professor Hellavi and Lehvi's article on Special Purpose Development Companies. Both of these works propose novel institutions for collective action and self-governance, to replace (parts of) the traditional takings procedure, especially in cases where the taker has commercial incentives. 

After examining these proposals in some depth, I arrive at several objections against the details of the particular institutional arrangements proposed, particularly with regards to their likely effectiveness. It seems, in particular, that both proposals fail to recognise the full extent to which prevailing regulatory frameworks concerning land use and planning would have to be reformed in order to make their proposals work.

At the same time, I argue that these novel institutional proposals are extremely useful in that they point towards a novel way to frame the issue of legitimacy in takings law. In particular, I explore the hypothesis that traditional procedural arrangements surrounding takings suffer from a democratic deficit, a particularly powerful source of discontent in economic development cases.

This idea is the second key focus point of my thesis. In the first part of the thesis, I approach it from a theoretical point of view, by exploring the notion of {\it participation} and its importance to the issue of legitimacy, particularly in the context of economic development. It seems, in particular, that {\it exclusion} could be a particular worrying consequence of certain kinds of economic development takings, namely those that lack democratic legitimacy in the local community where the direct effects of the taking are most clearly felt.

I believe this to be a promising hypothesis, and I back it up by considering the social function theory of property and the notion of human flourishing which has recently been proposed as a normative guide for reasoning about property in general. However, I also acknowledge that the purported democratic deficit of takings law cannot be satisfactorily analysed by theoretical arguments alone. Hence, to explore it in more depth, I go on to study it from within the context of a specific jurisdiction, by offering a detailed case study of takings of Norwegian waterfalls for the purpose of hydropower development. This case study, in turn, will allow me to cast light on two further key themes. %This brings me to the second part of my thesis, which in turn consists of two main themes, where the latter aims to bring me back towards a more general setting, by delivering some recommendations for how best to deal with economic development takings.

\noo{
analysis, which must by necessity 

link the idea of the democratic deficit with theoretical work. on the category of economic development takings. In particular, I explore the notion of participation, to  and arguing that it is key to understanding common discontents that arise in for-profit taking situations. 

, by arguing that the recognition of this as a special category is closely related to a shift of focus towards procedural legitimacy. 

Building on this perspective, 

  that legal scholars and policy makers should address more actively.




I believe this suggestion is 


I go on to consider the hypothesis that economic development takings demonstrate that takings law suffer from a {\it democratic deficit}.}


\section{Failures of the Traditional Narrative}

In Norway, the traditional way of thinking about takings law revolves around the issue of compensation. The issue of legitimacy is more or less entirely covered by rules relating to the owners' right to be paid in ``full'' for the loss they suffer as a result of having to give up their property. Indeed, the right to compensation under Norwegian constitutional law is typically regarded as stronger than that which follows from the ECHR.

In so far as an owner has grievances that are directed at the act of taking as such, not the level of compensation to be paid, takings law has very little to offer. The owner is left with the possibility of arguing his point on the basis of general administrative law, which gives little or no special consideration to property and acts of taking. 

Through my case study, I present a detailed analysis of how this traditional narrative functions in relation to takings for hydropower development, when energy companies seeks to expropriation water rights from local farmers in order to profit commercially. My assessment focuses on cases when the farmers who lose their water rights wish to {\it participate} in economic development, by carrying out their own hydropower projects. This type of farmer-led hydropower development is becoming increasingly common in Norway.

Interestingly, the possibility for this kind of development results from the same legal framework that has now transformed Norwegian energy companies into commercial entities. Both changes are due to liberalisation of the electricity sector in the early 1990s. Hence, while liberalisation empowers local owners to develop hydropower, it also renders taking for hydropower as takings for profit. Unsurprisingly, this has led to tensions that Norwegian courts have had to grapple with in an increasing number of cases.

In their approach to these cases, the courts rely heavily on the traditional narrative, by reconsidering how compensation is calculated when water rights are taken for hydropower. Compensation practices have already changed dramatically. However, there has also been cases when the local owners of these rights have protested the taking as such, claiming that they should be given the opportunity to develop their own resources. These protests have been entirely unsuccessful, as the courts in Norway adopts a stance on legitimacy that is extremely deferential to the executive, provided adequate compensation is paid.

By giving a detailed assessment of a few select cases, I explore the practical consequences of this, while also aiming to bring out how decision-making process surrounding hydropower actually work in Norway. I show, in particular, that local owners risk being completely marginalised, and that new compensation practices have proven inadequate as a means of redressing concerns that arise in this regard. My conclusion is that the case study of Norwegian waterfalls demonstrate concretely the shortcomings of the traditional narrative of legitimacy of takings.

However, I also believe that Norwegian law may offer a possible path towards a solution to this problem, one that has also been put to active use in recent years, particularly in cases when farmers themselves aim to undertake hydropower development, but wish to do so against the will of other members of the local community. This brings me to the second key theme of my case study.

\section{A Judicial Framework for Compulsory Participation}

In Norway, the distribution of property rights across the rural population is traditionally highly egalitarian. This has had many consequences for Norwegian society. For one, it meant that the farmers in Norway soon became an active political force, particularly as representative democracy started to gain ground as a form of government in the 19th century. As early as in 1837, the Norwegian parliament was so dominated by farmers that it came to be described as the ``farmer's parliament''.

The Norwegian farmers were often little more than small-holders, and had few priveligues to protect. Hence, they became liberals of sorts, responsible for pushing through important early reforms, such as the abolishment of noble titles and the establishment of semi-autonomous, elected, municipality governments. 

However, the municipality governments were not the first example of local, participatory, decision-making institutions in Norway. Indeed, the highly fragmented ownership of land meant that institutions for land management are among the oldest known in Norway. One of the most important ones exists to this day, namely the {\it land consolidation court}. The final focus point of my thesis consists in an assessment of this institution and its potential as a possible procedural alternative to takings in economic development cases.

Importantly, the land consolidation procedure in Norway is a semi-judicial process that warrants the imposition of {\it compulsory participation} by primary stakeholders in decision-making processes to which they are deemed to owe a contribution. One typical situation when the institution will be invoked involves the management of jointly owned land, where the land consolidation procedure is used to ensure that local owners may reach a joint decision on how to regulate the use of their land, if necessary one that is imposed on them by the land consolidation judges.

However, the judges' power is limited in that they may only impose a measure if the gains are deemed to outweigh the loss for all stakeholders involved. In practice, land consolidation judges often act as mediators, to facilitate a collective decision. Moreover, one of the most common acts of a land consolidation judge is to set up owner's associations, in a manner that institutionally regulates the continued interaction and decision-making among the stakeholders even after the formal consolidation process has concluded.

In Chapter x, I explore this framework in some depth, focusing on its potential as an alternative to exporpriation. This is especially interesting since land consolidation is presently being put to use in order to organise hydropower development. Hence, my case study provides an excellent opportunity for comparing the land consolidation and the takings process, with respect to the overall aim of ensuring development of hydropower on equitable terms. 

Here, I argue, the land consolidation route may be preferable, as it ensured legitimacy through participation. At the same time, the procedure remains effective, since participation is in fact compulsory. I discuss possible objections to the procedure in some depth, but conclude that the continued development of the land consolidation institution provides the best way forward for addressing economic development takings in Norway.

Finally, I compare the institution of land consolidation with the institutional proposals that have been made specifically in the context of the debate on economic development takings. I argue that it compares favourably, both because it comes equipped with in-built judicial safeguards, but also because it has a broader scope. I note, however, that its use as a better alternative to economic development takings is dependent on both political will and an ability to retain key feature even in the presence of new and powerful stakeholders in the consolidation process itself.


\noo{ In the second part of the thesis, I put the theoretical framework to the test by applying it to a concrete case study, namely that of Norwegian hydropower. Following liberalisation of the energy sector in the early 1990s, hydropower is now a commercial pursuit in Norway. Moreover, there is a long tradition for granting energy producers the power to acquire property compulsorily, including the necessary rights to exploit the energy of water, rights that are subject to private property under Norwegian law. This has resulted in tension and controversy, however, as the original owners of these rights, typically local farmers and small-holders, see the commercial potential of hydropower being transferred to other commercial interests, to the detriment of their own, and their communities', interest in self-governance and economic benefit.}

\section{Structure of the Thesis}

My thesis is divided into two parts. In the first, I collect my theoretical analysis of economic development takings as a separate category of interference in property rights. My aims in this part are threefold. First, I argue that the category makes sense outside the context of US law and that it is worthy of comparative study, even if it has not yet come to prominence outside the US. I develop this argument theoretically by anchoring it in the social function theory of property. 

This theory, which I argue for as a general template for thinking about takings and property, makes it very natural to operate with a special category for cases when there are significant commercial interests on the taker side. Moreover, I believe it highlights the importance of considering the matter of legitimacy contextually and, at it's ore, as an issue of fairness and proportionality. This appears to be at odds with the standard narrative surrounding economic development takings in the US, which tends to focus on the public use requirement.

The second aim of the first part of my thesis is to explore this tension. I do so by exploring the history of the public use debate in the US, by arguing that contextual assessment was originally at the core of the notion, when legitimacy was adjudicated by state courts. Subsequent developments at the federal level, I argue, has served to change the impression of the test, creating the erroneous impression that the key question is whether or not a taking is for a ``public use'' in some abstract sense that should be pegged down by the law. Rather, the basic question is, and remains, as it has always been, whether or not the action of government is able to strike a fair balance among the interests of the affected shareholders.

The third aim of the first part of my thesis is to propose an alternative way of thinking about legitimacy, that brings out the key issue without getting sidetracked by the notion of a public use. Here, I rely on recent institutional proposals for reform of the takings procedure itself, that are meant to empower owners and local communities by providing a novel template for collective action. I analyse these proposals in some depth, raising some objections and proposing some research questions.

This then brings me to the second part of my thesis, where I study the questions that I distilled in the first part by looking at the case of Norwegian hydropower. Here, my aims are again two-fold. First, I present an analysis of the framework of takings law in Norway, and the narrative of judicial reasoning that surrounds it, exemplified by the case of takings of water rights for hydropower development. I argue that tht the traditional narrative does indeed fail, in much the same way as predicted by the theoretical considerations of the first part of the thesis. 

My second aim is to present first steps towards a possible solution, again based on a concrete assessment of Norwegian law. Here, I present the institution of land consolidation courts, as a possible framework for compulsory participation in land development that can at once provide the participation and the compulsion that appears to be necessary to move towards the goal of economic development without offending against the rights of owners and local communities.

Finally, I offer a conclusion that aims to present in short form the main problems investigated, and the solution concepts offered, through the course of this thesis.


\newcommand{\isr}[1]{{#1}}

\part{Towards a Theory of Economic Development Takings}

\chapter{Property, Protection and Privilege}\label{chap:1}

\begin{quote}
It's nice to own land.\footnote{Donald Trump, as quoted in \cite{booth12}.}
\end{quote}

\begin{quote}
A human being needs only a small plot of ground on which to be happy, and even less to lie beneath.\footnote{Johan Wolfgang von Goethe, {\it The sorrows of young Werther and selected writings}.}
\end{quote}

\section{Introduction}

In this chapter, I will propose a template for analysing economic development takings, based on legal theory.\footnote{I will not provide an extensive presentation of concepts or theoretical approaches developed in other fields, such as political science, sociology, economy, or psychology. However, all these fields engage in interesting ways with the notion of takings and property. Hence, while I focus on legal and --  to some extent -- philosophical theories, I will make a special note of relevant research questions that are also analysed in other academic disciplines. For some examples of relevant work from economics, psychology and political science respectively, see, generally, \cite{miceli11,nadler08,katz97,carruthers04}.} I argue that the category of economic development takings is relevant to legal reasoning about certain situations when private property is taken by the state. This is not {\it prima facie} clear. I am prepared to face critics who will point out that the category has no legal relevance in their jurisdictions. Fortunately, the category makes intuitive sense; it targets situations when property is, quite literally, taken for economic development. In most cases I will consider, this is even the explicitly stated aim used to justify eminent domain. Hence, the factual basis for the categorization is beyond doubt.

The juridical basis, on the other hand, cannot be taken for granted. Indeed, a superficial look at dominant legal approaches to property would seem to indicate that in many property regimes, the nature of the project benefiting from a taking is not a major issue when assessing the legitimacy of interference.\footnote{For instance, in Europe, the property jurisprudence at the ECtHR deals almost exclusively with other aspects of legitimacy. The Court typically stresses that interference must be in the public interest, but then leave this aspect of legitimacy behind after making clear that the member states enjoy a wide margin of appreciation in relation to the public interest requirement. See, e.g., \cite{james86,lindheim12}. Similarly, in the US in the 1980s, Merrill claimed that most observers thought of the public use clause in the fifth amendment of the US constitution as nothing more than a ``dead letter'', see \cite[61]{merrill86}.} 

This chapter aims to clarify why the purpose and context of a taking matters, not only as a question of public policy but also with respect to property protection and the rights of owners and their communities. I believe it is important to do so thoroughly, to establish a secure conceptual basis for the rest of the thesis. From the point of view of US law, this is not strictly necessary, since economic development takings have already gained recognition as an important category of legal reasoning.\footnote{See generally \cite{cohen06,somin07,malloy08}.} In Europe, however, this has not yet happened, at least not to the same extent.

The reason for this difference is not that US law contains special rules that directly point to distinguishing features of economic development takings.\footnote{In fact, many state laws now {\it do} contain such rules, following the backlash of the controversial decision in \cite{kelo05}. However, such rules were introduced only after the category of economic development takings first came to prominence in legal discourse. See generally \cite{eagle08,somin09,jacobs11}.} Rather, the difference is largely due to the fact that economic development takings have resulted in political controversy in the US, a controversy that has influenced both the law and legal scholars.\footnote{See, e.g., \cite[1190-1192]{somin08}.} Hence, in the absence of a similar political climate in Europe, a conceptual investigation into the very idea of an economic development taking is warranted.

This chapter argues that in order to make progress in this regard, we must broaden our theoretical outlook compared to traditional forms of legal reasoning about property. Interestingly, a suitable conceptual reconfiguration appears to be implicit in recent strands of property theory, particularly those that focus on the {\it social function} of property.\footnote{See generally \cite{alexander09a,foster11,singer00,underkuffler03,alexander06,alexander10,dagan11}.} Indeed, the crux of the main argument presented in this chapter is that the social function view compels us to pay attention to the special dynamics of power that tend to manifest in cases when private property is taken by the state for a commercial purpose.

To make clear why such takings are special, the traditional entitlements-based perspective on property has to be abandoned in favour of a perspective that emphasises the function of property as a building block of democracy and participatory decision-making, particularly at the local level. This is what the social function theory achieves, by compelling us to recognise the importance of property in regulating social and political relations. Moreover, the social function theory emphasises the social {\it obligations} attached to property, particularly with respect to communities of property dependants. Hence, property not only gives owners a right to participate in decision-making processes, it also gives them a duty to do so, not only on their own behalf, but also on behalf of local community interests.

In the context of economic development, this highlights that property rights can empower local communities in their interactions with powerful commercial and central government interests. Importantly, the use of eminent domain can undermine this crucial function of property, thereby threatening the democratic legitimacy of the decision-making process, by depriving local communities of a potentially robust source of participatory competence. Moreover, when property interests are transferred away from the local community on a permanent basis, this threatens to leave a lasting democratic deficit in the wake of economic development. This worry, I argue, is the key reason why we need to recognise economic development takings as a separate conceptual category.

To motivate the theoretical work, I will begin in Section \ref{sec:dts} by considering the Balmedie controversy, pertaining to Donald Trump's plans for a golf resort in Balmedie, a village on the east coast of Scotland. I use this concrete example to highlight tensions between property's different functions in the context of economic development. Then, in Section \ref{sec:top}, I go on to discuss theories of property, to locate a suitable starting point for further analysis. I argue that neither of the two dominant property theories of the last century, the bundle theory and the dominion theory respectively, provide such a starting point. In Section \ref{sec:socfunc}, I move on to consider the social function theory in more depth, to arrive at a more useful theoretical template. Moreover, I argue that the descriptive part of this theory can provide a valuable conceptual tool even if one does not agree with the normative assertions that are typically associated with it. In particular, I argue that normative considerations should be addressed separately.

I do so in Section \ref{sec:hf}, by building on the human flourishing account of the purpose of property. I argue that the human flourishing theory provides us with a possible path towards answers to the normative questions that arise from the social function perspective on property. In Section \ref{sec:edt}, I make the discussion more concrete by applying the social function theory to a preliminary investigation of economic development takings. The human flourishing theory is then used to formulate some overriding normative constraints that will rely on for the concrete policy assessments I offer in this thesis. In Section \ref{sec:conc1}, I offer a conclusion.

\section{Donald Trump in Scotland}\label{sec:dts}

On the 10th of July 2010, the property magnate Donald Trump opened his first golf-course in Scotland, proudly announcing that it would be the ``best golf-course in the world''.\footnote{See \cite{passow12}.} Impressed with the unspoilt and dramatic seaside landscape of Scotland's east coast, the New Yorker, who made his fortune as a real estate entrepreneur, had decided he wanted to develop a golf course in the village of Balmedie, close to Aberdeen.

To realise his plans, Trump purchased the Menie estate in 2006, with the intention of turning it into a large resort with a five-star hotel, 950 timeshare flats, and two 18-hole golf-courses. The local authorities were divided on the issue of whether to grant planning permission, which was first denied by Aberdeenshire Council.\footnote{See, e.g., \cite{bbc07}.} Critical attention was directed at the fact that the proposed site for the development had previously been declared to be of special scientific interest under conservation legislation.\footnote{See \cite{bbc07b}.} The frailty and richness of the sand dune ecosystem, many argued, suggested that the land should be left unspoilt for future generations. 

Trump was not deterred, and in the end he was able to convince Scottish ministers that he should be given the go-ahead on the prospect of boosting the economy by creating some 6000 new jobs.\footnote{See \cite{carrell08}. Trump's plans attracted significant public attention, and his interaction with Scottish decision-makers came under critical scrutiny by commentators, see, e.g., \cite{jenkins08}. For a more general assessment from the point of view of conservation interests in the UK, see \cite{koen13}.} Activists continued to fight the development, launching the ``Tripping up Trump'' campaign to back up local residents who refused to sell their properties.\footnote{See \cite{tripping15}.} One of these, the farmer and quarry worker Michael Forbes, expressed his opposition in particularly clear terms, declaring at one point that Trump could ``shove his money up his arse''.\footnote{See \cite{scotsman10}.} Trump, on his part, had described Forbes as a ``village idiot'' that lived in a ``slum''.\footnote{See \cite{bbc10}.} Moreover, he had suggested that Forbes was keeping his property in a state of disrepair on purpose, to coerce Trump to pay more for the land, to remove the blight.\footnote{See \cite{cnn07}.} Forbes was offended. He proudly declared that he would never consider selling, as the issue had become personal.\footnote{See \cite{ferguson12}.}

At the height of the tensions, Trump asked the local council to consider issuing compulsory purchase orders (CPOs) that would allow him to take property from Forbes and other recalcitrant locals against their will.\footnote{See \cite{macaskill09}. It would not have been the first time Donald Trump benefited from eminent domain. In the 1990s, he famously succeeded in convincing Atlantic City to allow him to take the home of Vera Coking, to facilitate further development of his casino facilities. But in this instance, Trump has been unsuccessful. Indeed, the taking of Vera's home was eventually struck down by the New Jersey Superior Court, an influential result that was hailed as a milestone in the fight against ``eminent domain abuse'' in the US. See \cite[297-301]{jones00}. See also \cite{gillespie08}. For the decision itself, consult \cite{banin98}.} These plans were met with widespread outrage. The media coverage was wide, mostly negative, and an award-winning documentary was made which painted Trump's activities in Balmedie in a highly negative light.\footnote{See \cite{baxter11}.} The controversy also found its way into UK property scholarship. Kevin Gray, in particular, a leading expert in property law, expressed his opposition by making clear that he thought the proposed taking would be an act of ``predation''.\footcite{gray11}

In fact, the case prompted Gray to formulate a number of key features that could be used to identify situations where compulsory purchase would be more likely to represent an abuse of power. He noted, moreover, that Trump's proposed takings would fall in line with a general tendency in the UK towards using compulsory purchase to benefit private enterprise, even in the absence of a clear and direct benefit to the public. Hence, it seemed realistic that CPOs might be used in Balmedie.\footnote{Moreover, a statutory authority is found in section 189 of the \cite{tcpsa97}, stating that local authorities have a general power to acquire land compulsorily in order to ``secure the carrying out of development, redevelopment or improvement''.} It would not be hard to argue that the public would benefit indirectly in terms of job-creation and increased tax revenues. Moreover, Scottish ministers had already gone far in expressing their support for the plans.

But then, in a surprise move, Trump announced he would not seek CPOs, claiming also, to the consternation of local residents, that it had never been his intention to do so.\footnote{See \cite{scotsman11}.} Instead, Trump decided to pursue a different strategy, namely that of containment. He erected large fences, planted trees and created artificial sand dunes, all serving to prevent the properties he did not control from becoming a nuisance to his golfing guests. One local owner, Susan Monroe, was fenced in by a wall of sand some 8 meters high. ``I used to be able to see all the way to the other side of Aberdeen'', she said, ``but now I just look into that mound of sand''.\footnote{See \cite{booth12}.} She also lamented the lack of support from the Scottish government, expressing surprise that nothing could be done to stop Trump.

There was little left to do. As soon as the decision was made to build around them, the neighbouring property owners found themselves marginalized. Trump, on his part, was declared a valuable job-creator whose activities would boost the economy in the region. He even received an honorary doctorate at Robert Gordon University, a move that prompted the previous vice-chancellor, Dr David Kennedy, to hand his own honorific back in protest.\footnote{See \cite{bbc10b}.}

In the end, then, it was not by taking the land of others that Trump triumphed in Scotland. Rather, he succeeded by exercising ``despotic dominion'' over his own.\footnote{To quote William Blackstone, \cite[2]{blackstone79b}.} This proved highly effective. After he fenced them in, his neighbours were hard to see and hard to hear. The Balmedie controversy went quiet, the golfers came, Trump got his way. As he declared during the grand opening: ``Nothing will ever be built around this course because I own all the land around it. [...] It's nice to own land.''\footnote{See \cite{booth12}.}

\subsubsection*{\ldots}

The tale of Trump coming to Scotland serves to illustrate the kind of scenario that I will be looking at in this thesis. In addition, it puts my work into perspective. For a while, it looked like Balmedie was about to become a canonical case of an economic development taking. But in the end, it became an illustration of something more subtle, namely that what it means to protect property depends on value judgements regarding opposing property interests. In particular, while Trump achieved his ends in Scotland by relying on his own property rights, he did so by undermining the property rights of others, even if he did not formally condemn those rights.

This was made possible by an exercise of regulatory and financial power. Hence, we are reminded that the function of property as such is deeply shaped by social, political and economic structures. For the powerful owner, property can be used offensively to oppress weaker parties. For the marginalised, it might well be the last line of defence against oppression. Indeed, Donald Trump's ownership of the Menie estate has a vastly different meaning than does Michael Forbes' ownership of his small farm. To many observers, the former kind of ownership will represent some combination of power, privilege and profit, while the latter will be regarded as imbued with a mix of defiance, community and sustenance. Very different values are inherent in these two forms of ownership, and after Trump came to Balmedie, they clashed in a way that required the legal order to prioritise between them.

In Trump's narrative, upholding the sanctity of property in Balmedie entails allowing him to protect his golf resort plans from what he regards as backwards locals who attempt to fight progress. If this is one's starting point, property protection might even come to involve the use of compulsory purchase of rights that are seen as a hindrance to the full enjoyment of property by a more resourceful owner. 

For Michael Forbes and the other local owners, protecting property has a completely different meaning. To them, it was paramount to protect the local community against what they saw as a disruptive and damaging plan, one that threatened to turn them and their properties into mere golfing props. Again, adequate protection might require an interference in property, to prevent Trump from using his land according to his own wishes, because this causes damage to his neighbours. 

Regardless of who we support, in the case of Balmedie, we are forced to recognise that protection implies interference and vice versa. This shows the conceptual inadequacy of the idea that property protection is all about weighing private and public interests against each other, to strike a balance between the state's power to do good and owners' right to do as they please. In reality, matters can be more subtle, involving a number of additional dimensions. Importantly, how we assess concrete situations where property is under threat depends crucially on what we perceive as the ``normal'' state of property, the alignment of rights and responsibilities that we deem  worthy of protection. Our stance in this regard clearly depends on our values. But values themselves are in turn influenced by the context of assessment within which they arise. An additional challenge is that our assessments are often influenced by our \emph{perception} of the relevant context, rather than by facts.

For example, property activists in the US tend to regard the value of autonomy as a fundamental aspect of property. But this must be understood in light of the idea that US society is founded on an egalitarian distribution of property, where ownership is meant to empower ordinary people by facilitating self-sufficiency and self-governance.\footnote{See, e.g., \cite[173]{ely07}.} Hence, the autonomy inherent in property ownership is not thought of as being bestowed on the few, but on the many. Protecting autonomy of owners against state interference is not about protecting the privileges of the rich and powerful, but is embraced as a way to protect {\it against} abuse by the privileged classes.\footnote{This narrative is enthusiastically embraced by US activists who fight economic development takings, see, e.g., \cite{castle15}.} 

This, however, is only an {\it idea} of property protection. It might not correspond to the reality surrounding the rules that have been \isr{moulded} in its image. Indeed, it has been noted that despite the great pathos of the egalitarian property idea, egalitarianism has actually played a marginal role to the development of US property law.\footnote{\cite[361]{williams98} (``Why does the egalitarian strain of republicanism have such a substantial presence in American property rhetoric outside the law but so little influence within it?'')} More worryingly still, research indicates that land ownership in the US, which are hard to track due to the idiosyncrasies of the land registration system, is not actually all that egalitarian.\footcite[246-247]{jacobs98} In this way, we are confronted with the danger of a dissociation of values, reality and the law.

In Scotland, a similar story unfolds. Here, the traditional concern is that land rights are mainly held by the elites.\footnote{See generally \cite{wightman96,wightman13}.} As a result, Scottish property activists tend to focus on values such as equality and fairness, calling also on the state to regulate and implement measures to achieve more egalitarian control over the land. Indeed, reforms have been passed that sanction interference in established property rights on behalf of local communities.\footnote{See generally \cite{lovett11,hoffman13}.} At the same time, cases like Balmedie illustrate that the Scottish government, now with enhanced powers of land administration, may well choose to align themselves with the large landowners. Moreover, research indicates that recent reforms in Scottish planning law, which serve to enhance the power of the central government, have the effect of undermining local communities and their capacity for self-governance.\footnote{See generally \cite{pacione13,pacione14}.} Again, the danger of a disconnect between influential property narratives and reality is brought into focus.

On the other hand, it seems that \isr{grass roots} property activists in the US and Scotland may well be closer in spirit than they seem. Although their perception of the role of the state is very different, they appear to share many of the same concerns and aspirations. Arguably, differences arise mainly from the fact that they operate in different contexts and engage with different discourses of property. The challenge is to find categories of understanding that allow us to make sense of both their commonalities and their differences.

I think the example of Balmedie suggests a possible first step. It illustrates, in particular, the need for a framework that will allow us to recognise that opposing the use of compulsory purchase for economic development is perfectly consistent with supporting strict property regulation to prevent the establishment of golf resorts in fragile coastal communities. Both of these positions, moreover, should be viewed as efforts to protect property. To the classical debate about the limits of the state's authority over property, such a dual position can be hard to make sense of. But in my opinion, this only points to the vacuity of the conventional narrative.

In general, I think it is hard to make sense of many contemporary disputes over property if we do not have the conceptual acumen to distinguish between (1) egalitarian property held under a stewardship obligation by members of a local community, and (2) ``feudal'' property held by large enterprises for investment. Moreover, there is no contradiction between promoting the value of autonomy for one of these, while \isr{emphasising} the need for state control and redistribution when it comes to the other. The broader theme is the contextual nature of property and its implications for protection of property rights. In the coming sections, I will propose a theoretical basis that integrates this viewpoint into legal reasoning about interference in property rights.

\section{Theories of Property}\label{sec:top}

What is property? In common law jurisdictions, the standard answer is that property is a collection of individual rights, or more abstractly, {\it entitlements}.\footnote{The term ``entitlement'' was used to great effect in the seminal article \cite{calabresi72}.} Being an owner, it is often said, amounts to being entitled to one or more among a bundle of ``sticks'', streams of protected benefits associated with, and thereby serving to legally define, the property in question.\footnote{See \cite[357-358]{merrill01}. The ``classical'' references on the bundle of rights theory in the US and the UK respectively are \cite{hohfeld17,honore61}.} This point of view was first developed by legal realists in response to the natural law tradition, which \isr{conceptualised} property in terms of the owner's dominion over the owned thing, particularly his right to exclude others from accessing it.\footcite[193-195]{klein11} In civil law jurisdictions, rooted in Roman law, a dominion perspective is still often taken as the theoretical foundation of property, although it is of course \isr{recognised} that the owner's dominion is never absolute in practice.\footnote{For a comparison between civil and common law understanding of property, see generally \cite{chang12}.}

In modern society, the extent to which an owner may freely enjoy his property is highly sensitive to government's willingness to protect, as well as its desire to regulate. To dominion theorists, this sensitivity is typically thought of as giving rise to various restrictions on property, but for bundle theorists it is rather thought of as {\it constitutive} of property itself.\footcite[7]{chang12} 

The bundle of rights theory has long historical roots in common law. Arguably, it was distilled from the traditional estates system for real property, which was turned into a theoretical foundation for thinking about property in the abstract.\footnote{See \cite[7]{chang12}   
(``The ``bundle of rights'' is in a sense the theory implicit in the common law system taken to its extreme, with its inherently analytical tendency, in contrast to the dogged holism of the civil law.'').}

However, during the 18th and 19th century, natural law and dominion theorising was also influential in common law. This is evidenced, for instance, by the works of William Blackstone and James Kent.\footnote{See generally \cite{blackstone79b,kent27}.} Towards the end of the 19th century, it became increasingly hard to reconcile such an approach to property with the reality of increasing state regulation. Hence, the bundle metaphor that gained prominence in the early 1900s can be seen as a return to a more modest perspective.\footnote{See \cite[195]{klein11}.}

On the bundle account, property rights are thought to be directed primarily towards other people, not things.\footnote{See \cite[357-358]{merrill01} (``By and large, this view has become conventional wisdom among legal scholars: Property is a composite of legal relations that holds between persons and only secondarily or incidentally involves a ``thing''.'').} This underscores an important point about property in the real world, namely that the content of rights in property are necessarily relative to a social context as well as the totality of the legal order. For instance, when relying on a bundle metaphor it becomes easy to explain that a farmer's property rights protects him against trespassing tourists, but not against the \isr{neighbour} who has an established right of way.\footnote{It has been argued that this way of thinking about property, as a web of (legal and social) normative relations between persons, does not entail the bundle of sticks idea, see \cite[23-25]{dorfman10}. I agree, and I also believe that endorsing the property-as-relations perspective is largely appropriate, even if one does not otherwise agree with the bundle perspective. Historically, however, the two ideas have in fact been closely associated with one another, so presenting them together seems appropriate. Moreover, I will not actively enter into the theoretical debate on this point, since I believe that the {\it social function} account of property, discussed in more detail in Section \ref{sec:socfunc}, takes us further than both bundle and dominion perspectives. However, as will hopefully become clear, the social function theory itself may be seen as a continuation of the property-as-relations idea, catering also to a more holistic perspective on social structures (although it otherwise manages to remain largely neutral on the bundle v dominion issue).}

By contrast, the dominion theory suggests viewing such situations as exceptions to the general rule of ownership, which implies a right to exclusion at its core. In the case of property, exceptions no doubt make up the norm. But in civil law jurisdictions one lives happily with this. It takes the grandeur away from the dominion concept, but it retains a nice and simple structure to property law. In the civil law world, it is common to say that what the owner holds is the {\it remainder}, namely what is left after deducting all positive rights that restrict his dominion.\footcite[25]{chang12} Moreover, while there may be many limitations and additional benefits attached to property, they are all in principle carved out of one initial right, namely that of the owner. In this way, the system becomes more easily navigable.

An interested party may ask, ``who owns this property?'' Then, under the dominion theory, a clear answer is expected and will usually be adequate, even if it does not give a complete picture of all relevant property rights. Under the bundle theory, on the other hand, one might be inclined to respond, ``to which stick are you referring?'' Clearly, this narrative is more complex, perhaps unduly so. 

Some common law scholars have recently elaborated on this to develop a critique of the bundle theory, by suggesting that it should at least be complemented by a firm theory of {\it in rem} rights in property. This, they argue, would allow the law to operate more effectively, by relying on a simple and clear rule that, although defeasible, would generally suffice to inform people about their relevant rights and duties in relation to property.\footnote{\cite[793]{merrill01b} (``The unique advantage of in rem rights -- the strategy of exclusion -- is that they conserve on information costs relative to in personam rights in situations where the number of potential claimants to resources is large, and the resource in question can be defined at relatively low cost.''); \cite[389]{merrill01} (``The right to exclude allows the owner to control, plan, and invest, and permits this to happen with a minimum of information costs to others.''). See also \cite{ellickson11} (arguing that Merrill and Smith's analysis nicely complements and improves upon the bundle theory).} 

In addition, some scholars point out that the bundle theory does not adequately reflect the sense in which property is a right to a {\it thing}, serving to create an attachment that is not easily reducible to a set of interpersonal legal relationships.\footnote{\cite[1862]{merrill07}. For a slightly different take on attachment, highlighting how the `thingness' of property marks its conditional nature and transferability, see \cite[799-818]{penner96}.} In the US, where the bundle theory has traditionally been dominant, critique like this seems to be gaining ground.\footnote{See generally \cite{foster10}.}

In this thesis, the efficiency and clarity of different property concepts will not be a primary concern, nor will personal attachments to things in themselves play a particularly important role.\footnote{I mention, however, that the \isr{personhood aspects} of property that are sometimes highlighted in this regard will also be relevant to my analysis of economic development takings. However, this is not something that I think warrants extensive engagement with the bundle v dominion debate. I note, for instance, that in the work of Margaret Jane Radin, one of the main proponents of \isr{personhood} accounts, the bundle theory is not challenged as much as it is readjusted, although in places it also seems to be the object of some implicit criticism, see, e.g., \cite[127-130]{radin93}.}
Hence, I will remain largely agnostic about this aspect of the debate between dominion and bundle theorists. In particular, the differences between civil and common law traditions in this regard do not cause special problems for my analysis of economic development takings. In this regard, it is more important how different ways of looking at property can influence how we assess when interference is legitimate under constitutional and human rights law. Hence, I now turn to the question of whether or not there are any significant differences between dominion and bundle theories in this regard.

\subsection{Takings under Bundle and Dominion Accounts of Property}

Bundle theorists might be expected to have a relaxed attitude towards state interference in property rights. Indeed, thinking about property as sticks in a bundle may lead one to think that property rights are intrinsically limited, so that subsequent changes to their content, made by a competent body, are reflections of their nature, not a cause for complaint. In particular, the theory conveys the impression that property is highly malleable. 

For the theorists that developed the bundle of sticks metaphor in the late 19th and early 20th century, this aspect was undoubtedly very important. By providing a highly flexible concept of property, they helped the state gain conceptual authority to control and regulate.\footcite[195]{klein11} The early bundle theorists not only developed a theory to fit the law as they saw it, they also contributed to change.

In takings law, the bundle theory actively contributed to legal developments, especially in relation to the contentious issue of so-called regulatory takings. Such takings occur when governmental control over the use of property becomes so severe that it must be classified as a taking in relation to the law of eminent domain. In the US, the question of when regulation amounts to a regulatory taking is highly controversial. The stakes are high because takings have to be compensated in accordance with the Fifth Amendment of the US constitution. At the same time, the law is unclear; the lack of statutory rules means that regulatory takings cases are often adjudicated directly against constitutional property clauses (often the relevant state constitution, in the first instance).

If property is thought of as a malleable bundle of entitlements that exists only because it is recognised by the law, it becomes natural to argue that when government regulates the use of property, it does not deprive anyone of property rights. It merely restructures the bundle. In the case of {\it Andrus v Allard}, the Supreme Court adopted such an argument when it declared that ``where an owner possesses a full ``bundle'' of property rights, the destruction of one ``strand'' of the bundle is not a taking, because the aggregate must be viewed in its entirety''.\footcite[65--66]{andrus79}

Hence, with regards to the issue of regulatory takings, the bundle theory was actively used by those who favour a less restrictive approach to interference with private property rights. However, it is wrong to conclude that the bundle theory {\it necessarily} implies a less restrictive stance on takings. Epstein, for instance, argues that as every stick in the property bundle represents a property right, government should not be permitted to remove any of them without paying compensation.\footcite[232-233]{epstein11} 

More generally, Epstein does not believe that the bundle theory is responsible for what he regards as a weakening of property rights in the US during the 20th century. Instead, he thinks this weakening resulted from a tendency among modern property scholars to adopt a ``top-down'' approach to property. According to Epstein, too many scholars view property rights as vested in, and arising from, the power of the state, not the possessions of individuals.\footnote{\cite[227-228]{epstein11} (``In my view, the nub of the difficulty with modern property law does not stem from the bundle-of-rights conception, but from the top-down view of property that treats all property as being granted by the state and therefore subject to whatever terms and conditions the state wishes to impose on its grantees'').} 

Epstein successfully shows that as a rhetorical device, the bundle of rights theory may be turned on its head compared to how it was used in {\it Andrus v Allard}. Moreover, I believe his arguments demonstrate that the bundle theory itself does not dictate any particular position on the degree of protection that private property should enjoy against state interference.\footnote{To further underscore this point, it may be mentioned that while US courts do in fact \isr{recognise} that a regulation can amount to a taking, this is practically unheard of in several other common law jurisdictions, including England and Australia. This is despite the fact that these countries all paint property in a similar conceptual light. Moreover, while the issue of regulatory takings is considered central to constitutional property law in the US, it is considered a fairly marginal issue in England, see \cite{purdue10}.}

In the civil law world, the relationship between property theorising and property values is similarly hard to pin down at the conceptual level. Again, the issue of regulatory takings illustrates this. In some civil law countries, like Germany and the Netherlands, the right to compensation is strong, while in other civil law countries, such as France and Greece, it is very weak.\footnote{See generally \cite{alterman10}.} In particular, the exclusive dominion understanding of property does not appear to commit one to any particular kind of policy on this point. 

On the one hand, it cannot be denied that property rights are enforced, and limited, by the power of government. Hanging on to the idea of dominion, then, necessarily forces us to embrace also the idea that dominion is never absolute. In this way, the theory may serve as a conceptual basis for arguing in favour of a relaxed approach to state interference. If property rights are not absolute to start with, why worry about interfering in them for the common good? But, of course, this story too may be turned on its head. Indeed, a libertarian can use the image of limited dominion to argue that property is being ripped apart at its seams. If we want to maintain our grasp of what property is, such a person might argue, we had better enhance the level of protection offered to property owners, to restore true dominion.

To me, the upshot is that the differences between common law and civil law \isr{theorising} about property are not very relevant to the question of legitimacy in the context of state interference. As such, these differences also seem comparatively unimportant to my thesis. In particular, the differences between the bundle theory and the dominion idea do not appear to speak decisively in \isr{favour} of any particular approach to economic development takings.

In terms of descriptive content, both theories appear oversimplified. They provide a manner of speech, but they do not really get us very far towards uncovering the reality of property rights in modern society. In particular, they do not provide a functional account of what role property plays in relation to the social, economic and political structures within which it resides. 

In terms of normative content, on the other hand, both the bundle theory and the dominion theory appear rather bland. They simply do not offer much clear guidance as to what norms and values the institution of property is meant to promote. They give neat ways of presenting what property looks like, but do not tell us {\it why} it should be protected. 

\subsection{Broader Theories}

Based on the discussion so far, it seems that in order to make progress towards a theory of economic development takings we need to start from a property theory with a wider scope than both the bundle account and the dominion theory. There are many candidates that could be considered. In a recent monograph on property, Alexander and Pe\~{n}alver present five key theoretical branches:
\begin{itemize}
\item {\it Utilitarian} theories, focusing on property's role in helping to maximize utility or welfare with respect to individual preferences and desires.\footnote{\cite[Chapter 1]{alexander10}.} 
\item {\it Libertarian} theories, focusing on property's role in furthering individual autonomy and liberty, as well as the importance of protecting property against state interference, particularly attempts at redistribution.\footnote{\cite[Chapter 2]{alexander10}.} 
\item {\it Hegelian} theories, focusing on the importance of property to the development of personhood and \isr{self-realisation}, particularly the expression and embodiment of free will through control and attachment to one's possessions.\footnote{\cite[Chapter 3]{alexander10}.}
\item {\it Kantian} theories, focusing on how property arises to protect freedom and autonomy in a coordinated fashion so that {\it everyone} may potentially enjoy it, through the development of the state.\footnote{\cite[Chapter 4]{alexander10}.}
\item {\it  Human flourishing} theories, focusing on property's role in facilitating participation in a community, particularly as a template allowing the individual to develop as a moral agent in a world of normative plurality.\footnote{\cite[Chapter 5]{alexander10}.}
\end{itemize}

It it beyond the scope of this thesis to give a detailed presentation and assessment of all these theoretical branches. Suffice it to say that the utilitarian approach has been by far the most influential.\footnote{See \cite[11]{alexander12} (noting also that there are many varieties of utilitarianism, including theories that might dispute the label).} The basic tenet of this paradigm is that means-end analysis on the basis of exogenous preferences and utility measures provide a sound foundation on which to reason about law and policy.

In this thesis, I will depart from this form of analysis, by regarding property instead as an integral part of social structures. On this view,  property can no longer be seen neither as an end in itself nor as a means to maximise some utility measure. Instead, property is understood in light of how it functionally relates to other building blocks of life, such as sustenance, economic activity, social interaction, interpersonal responsibility, preference change, deliberation,  and democratic decision-making.

With such a starting point, I believe the human flourishing theory has more to offer than any of the other theoretical branches mentioned above. In Section \ref{sec:hf} below, I will emphasise how this theory suggests making a range of new policy recommendations regarding how the law {\it should} approach the question of economic development takings.

First, I note that a possible objection against all the theories summarised above is that they are highly normative. They are used to argue for particular values associated with property, not to clarify the descriptive core of the notion. This is a challenge, since one of my main aims in this thesis is to argue for a descriptive proposition, namely that economic development takings make sense as a conceptual category for legal reasoning. Hence, before I move on to consider normative aspects, I first need a theoretical framework that allows me to pinpoint what makes economic development takings unique. I would like to do so, moreover, without thereby committing myself to any particular stance on how to normatively assess such takings.

To arrive at a suitable foundation in this regard, I will rely on the so-called {\it social function theory} of property.\footnote{See generally \cite{foster11,mirow10,alexander09a}. Be aware that some authors, particularly in the US, also speak of the {\it social obligation} theory, using it more or less as a synonym for the social function theory.} This theory is often thought of as a normative theory as well, in some sense a precursor to more overtly normative theories such as the human flourishing theory. However, I will argue that the social function theory has a descriptive core that can serve as a common ground for debate among scholars that do not necessarily share the same normative outlook. Crucially, the descriptive core of the social function theory also point towards a normatively neutral argument in favour of studying economic development takings.

Before making my specific point about takings, I will present the social function theory of property in some further detail. I will focus on showing that it captures aspects that are already highly relevant -- behind the scenes -- to how property rules are understood and applied in concrete situations.

\section{The Social Function of Property}\label{sec:socfunc}

There is a growing feeling among property scholars that the notion of property has been drawn too narrowly by many of the traditionally dominant theories of property. \noo{ Moreover, it has been noted that what counts as property in a given legal system, and what property entails in that system, depends largely on its social and political context, tradition, and sometimes even chance.\footnote{For a clear exposition of property's elusive nature, see \cite{gray91}.} In the US, a utilitarian law-and-economics approach, which tends to take the social and political underpinnings of property for granted, has long been regarded as standard, but the tide is turning.} Moreover, an increasing number of scholars are turning away from assessing property rules against their effectiveness in maximising utility and social welfare.\footnote{For a nice early commentary on the limits of the utility-maximizing perspective in property law, focusing on the importance of changes, choices, and narratives, see \cite{rose90}.} Instead, some scholars adopt a holistic approach, allowing property's social function to come into focus. One of the main proponents of this conceptual shift is Gregory S. Alexander, professor at Cornell University. In a recent article, he writes:

\begin{quote} Welfarism is no longer the only game in the town of property theory. In the last several years a number of property scholars have begun developing various versions of a general vision of property and ownership that, although consistent with welfarism in some respects, purports to provide an alternative to the still-dominant welfarist account.[...] These scholars emphasize the social obligations that are inherent in ownership, and they seek to develop a non-welfarist theory grounding those inherent social obligations.\footcite[1017]{alexander11}
\end{quote}

\noo{ To scholars coming from political science, sociology or human geography, this trend will not raise many eyebrows, except perhaps for the fact that it is a recent one. After all, in fields such as these, property has never been understood merely as a set of individual entitlements that are meant to result in increased welfare. Rather, property is seen as a crucial part of the fabric of society, one that entrenches privileges and bestows power.\footnote{See generally \cite{carruthers04}.} 
Even scholars who believe that the institution of property is a force for good recognise that being an owner carries with it political capital, social responsibility, and membership in a community. Those aspects, moreover, are often regarded as more important than entitlements explicitly \isr{recognised} in positive legal terms. Crucially, they are important not only to the individual owners but also to society as a whole. How property rights are distributed among the population, for instance, has obvious political and economic implications, serving as a source of power and prosperity to some groups, while \isr{marginalising} others.\footnote{See, e.g., \cite[23]{carruthers04}. (``The right to control, govern, and exploit things entails the power to influence, govern, and exploit people'').}

But what is the relevance of this to property law? Usually, jurists approach property in isolation from such concerns, and often they do so because of practical necessity. The political question of what the law should be depends on assessments of the purpose and social context of property, but in the day-to-day workings of the law, the story goes, such considerations play a lesser role, with the importance of clear and simple rules outweighing the possible benefit that would result from contextual and holistic assessment. Classical theories of property can be accused of taking such a pragmatic view too far, by failing to \isr{recognise} that many social functions are {\it intrinsic} to property, so that they may become directly relevant when the law is applied to resolve concrete disputes.

The same accusation can be raised against both bundle and dominion theorists. They both tend to leave little room for considering property as a social phenomena. It is recognised, of course, that rights in property -- bundled or otherwise -- serve to regulate social relations. But this effect is typically regarded as belonging to the periphery of property as a legal category, more relevant to sociologists than to property scholars. In addition, it is uncommon to observe that the causal relation between property rights and society is bidirectional, since the meaning and content of property itself is partly determined by the very same social structures that property helps establish and sustain. When this aspect of property is not recognised, the risk is that subtle dependencies between property and the political order are not brought into focus, even when they play an important role in practice.

This is particularly clear when it comes to socially defined obligations attached to property. Hardly anyone would protest that in practical life, what an owner will do with his property is as much constrained by the expectations of others as it is by law. But in addition to influencing the owner subjectively, expectations can take on an objective character by being embedded strongly in the social fabric. This, in turn, can give rise to a norm, or even a custom, which may be legally relevant, either because the law gives direct effect to it, or because it influences how we interpret rules relating to the use of property.\footnote{See generally \cite{penalver09,alexander09}.}
}

As an empirical observation, the fact that property rights tend to come with social obligations is beyond doubt. Hardly anyone would protest that in practical life, what an owner will do with his property is as much constrained by the expectations of others as it is by law. Moreover, examples from the law of nuisance or adverse possession serve as simple examples of how social obligations also feature in the the law of property. \noo{Crucially, the social function theory asks us to focus esHowever, in the space between informal social obligations and formal legal duties, there is an interesting space of interaction that the social function theory asks us to address. 

Indeed, informal expectations can sometimes take on an objective character merely by being embedded strongly in the social fabric. This, in turn, can give rise to a norm, or even a custom, which may be legally relevant, either because the law gives direct effect to it, or because it influences how we interpret rules relating to the use of property.\footnote{See generally \cite{penalver09,alexander09}.}}

Still, traditional property scholars have surprisingly little regard for social functions when they theorise about property. According to Alexander, the classical theories of property convey the impression that ``property owners are rights-holders first and foremost; obligations are, with some few exceptions, assigned to non-owners''.\footcite[1023]{alexander11} The social function theorists attempt to redress this conceptual imbalance. As Alexander explains, ``social obligation theorists do not reverse this equation so much as they balance it. Of course property owners are rights-holders, but they are also duty-holders, and often more than minimally so''.\footcite[1023]{alexander11}

As I discuss in the next subsection, this idea is not new. Moreover, it seems to play an important implicit role in shaping how property is understood, particularly in Europe.

\subsection{Historical Roots and European Influence}

The first expression of the social function theory has been attributed to Le{\'o}n Duguit, a French jurist active early in the 20th century. In a series of lectures he gave in Buenos Aires in 1911, Duguit challenged the classic liberal idea of property rights by pointing to their context dependence, adopting a line of argument strikingly similar to how recent scholars have criticized utilitarian discourses about property.\footnote{See \cite[1004-1008]{foster11}. For more details about Duguit's work and the contemporaries that inspired him, see generally \cite{mirow10}.} In particular, Duguit also pointed to the notion of obligation, stressing the fact that individual autonomy only makes sense in a social context where people are dependent on each other as members of  communities. Hence, depending on the social circumstances of the owner, their property could entail as many obligations as entitlements. This, according to Duguit, was not only the inescapable reality of property ownership, it was also a normatively sound arrangement that should inspire the law, more so than individualistic visions of property as a liberal protection of entitlement.\footnote{See \cite[1005]{foster11} (``The idea of the social function of property is based on a description of social reality that recognizes solidarity as one of its primary foundations'', discussing Duguit's work). It should also be noted that Duguit was particularly concerned with owners' obligations to make productive use of their property, to benefit society as a whole. This raises the question of who exactly should be granted the power to determine what counts as ``productive use''. In this way, Duguit's work also serves to underscore one of the main challenges of regulatory frameworks that seek to incorporate and draw on property's social dimension. How should decisions be made in such regimes?}

Similar thoughts have been influential in Europe, particularly during the rebuilding period after the Second World War. For instance, the constitution of Germany -- her {\it Basic Law} -- contains a property clause stating explicitly that property entails obligations as well as rights. As argued by Alexander, this has had a significant effect on German property jurisprudence, creating a clear and interesting contrast with US law.\footnote{See \cite[338]{alexander03} (``The German Constitutional Court has adopted an approach that is both purposive and contextual, while the U.S. Supreme Court has not'').}

A social perspective on property was also influential during the debate among the European states that first drafted the property clause in Article 1 of the First Protocol to the European Convention of Human Rights (P1(1) of the ECHR).\footnote{See \cite[1063-1065]{allen10}. The liberal conception of property has since gained ground in Europe, causing jurisprudential developments that have been particularly clear in the case law from the European Court of Human Rights (ECtHR). See generally \cite{allen10}.} The article was eventually formulated as follows:

\begin{quote} Every natural or legal person is entitled to the peaceful enjoyment of his possessions. No one shall be deprived of his possessions except in the public interest and subject to the conditions provided for by law and by the general principles of international law.
The preceding provisions shall not, however, in any way impair the right of a state to enforce such laws as it deems necessary to control the use of property in accordance with the general interest or to secure the payment of taxes or other contributions or penalties.
\end{quote}

I will return to this clause in more depth in Section \ref{sec:eu} of Chapter \ref{chap:2}. Here I note how it emphasises both the private right to peaceful enjoyment of possessions and the state's right to interfere with property in the general/public interest. Moreover, it does not explicitly introduce an absolute compensation requirement in case of expropriation by the state, setting it apart from many other property clauses, including that contained in the Fifth Amendment of the US constitution. This arguably reflects a recognition of the social aspects of property.\footnote{See generally \cite{allen10}.} 

However, this particular aspect of social function reasoning also fits within a traditional narrative of private property rights as narrow entitlements, leaving aside the social responsibilities attaching to property as objectives to be pursued by the state. Indeed, the chosen formulation in P1(1) appears to suggest that social aspects are external to private property, vested in the regulatory power of the state. 

This marks a possible tension with the social function theory, which asks us to recognise that social obligations are inherent in private property, attaching to owners as much as to states. The importance of this in the present context is that a social function perspective can occasionally suggest stricter limits on state interference, not out of greater concern for individual entitlements, but out of concern for property's proper functions as a building block of social and political life.

Despite the formulation in P1(1), it seems that such a perspective does in fact play a role to the European Court of Human Rights (ECtHR). \noo{At least, the case law from the Court shows that social and political considerations are not only invoked in the context of adjudicating tensions between private entitlements and the perceived necessity of state interference in the public interest.

The ECtHR emphasises {\it proportionality} and {\it fairness} when adjudicating cases involving interference in property.\footnote{See generally \cite[Chapter 5]{allen05}.} Importantly, these broad notions are assessed concretely against the context of interference, also to give appropriate weight to the social and political function of the property interfered with. As a result, specific social functions of property can justify greater protection against interference.} A series of cases involving hunting rights provide a clear example of this.\footnote{See \cite{chassagnou99,hermann12,chabauty12}.} In these cases, the Court in Strasbourg has explicitly granted stronger property protection to owners who oppose hunting on ethical grounds, compared to owners who want to retain exclusive hunting rights for themselves.

For the former group of owners, it has been held that the state may not compulsorily transfer hunting rights to hunting associations for collective management.\footnote{See \cite{chassagnou99, hermann12}.} For the latter group of owners, by contrast, the Court held in {\it Chabauty v France} that such transfers must be tolerated.\footnote{See \cite{chabauty12}.}

For owners opposing hunting on ethical grounds, an interference with their hunting right is an interference with their moral duty to act in accordance with their beliefs. Given the belief that hunting is unethical, the owner has an obligation to prevent their hunting rights from being used. If owners are deprived of their opportunity to fulfil this obligation, it changes the social function of their property in a way that publicly negates their strongly held beliefs about how it should be used.

In {\it Chassagnou and others v France}, the Court regarded this as a particularly severe interference in property, which could not be upheld despite the fact that it had been carried out in the public interest to secure sustainable management of hunting rights. The Court concluded that ``compelling small landowners to transfer hunting rights over their land so that others can make use of them in a way which is totally incompatible with their beliefs imposes a disproportionate burden which is not justified under the second paragraph of Article 1 of Protocol No. 1''.\footnote{See \cite[85]{chassagnou99}.}

Clearly, the Court is not expressing an opinion on the ethical status of hunting. Rather, the Court emphasises that the owners are entitled to have unconventional personal convictions in this regard. Indeed, managing one's property in accordance with one's convictions is part of what it means to be an owner. Moreover, as demonstrated by the ruling in {\it Chabauty}, protecting this aspect of ownership is more important to the Court than protecting the right of owners to keep the fruits of the land to themselves.

This can only be because the right to manage one's property in accordance with one's beliefs is itself regarded as a socially desirable aspect of property ownership. As such, it is entitled to increased protection under the ECHR.

\noo{In the jurisprudence of the ECtHR, this form of interference, targeting a social obligation arising from the conscience of owners, is {\it more severe} than one which directly targets  hunting entitlements.}

\noo{I will return to possible normative implications of the social function theory later. Here I would like to stress that in the first instance it merely asks us to recognise an empirical truth: Property does not arise in a vacuum, but from within a society. As a philosophical proposition, this is obvious and hardly anyone denies it. But the social function theory asks us to consider something more, namely that property {\it law} continues to influence, and be influenced by, surrounding social and political structures.}

The hunting cases show that even when the legal system does not explicitly recognise the value of a social function inherent in property, such a function can still come to play a role when assessing the legitimacy of interference against P1(1). In the next section, I make a more general claim, arguing that the law invariably prioritises between different social functions of property, even when this is not explicitly acknowledged.

\subsection{The Impossibility of a Socially Neutral Property Regime}

Property both reflects and shapes relations of power among members of a society.\footnote{This aspect of property's social function was stressed in a recent ``statement of purpose'' made by leading property scholars in support of the social function theory, see \cite{alexander09a}.} Moreover, it does not act uniformly in this way -- the effect depends on the circumstances. An indebted farmer who is prevented by state regulation from making profitable use of their land might come to find that their property has become a burden rather than a privilege. As a consequence, someone who has already amassed power and wealth elsewhere might be able to purchase the land from the farmer cheaply. By acquiring a farm and transforming it to recreational property, the outsider will symbolically and practically assert their dominance and power, while also reaping a potential financial benefit resulting from their investments in a more modern function of property. 

In some cases, this dynamic can become endemic in an area, resulting in a complete reshaping of the social fabric surrounding property. The story might go like this: first, impoverished farmers and other locals sell homes to holiday dwellers, causing house prices to soar. As a result, local people with agrarian-related incomes \isr{cannot} afford local homes, causing even more people to sell their land to the urban middle class. In this way, a causal cycle is established, the social consequences of which can be vicious, particularly to the low-income people who are displaced.\footnote{The general mechanism is well-documented and known as {\it gentrification} in human geography (often qualified as rural gentrification when it happens outside urban areas). See generally \cite{weesep94,phillips93,slater06}. For a case study demonstrating the role that state regulation can play (perhaps inadvertently) in causing rural gentrification, see \cite[1027-1030]{darling05}.} My theoretical contention is the following: setting out to regulate property in a situation like this -- when property rights pull in different directions depending on your vantage point -- requires taking some stance on whose property, and which of property's functions, one is aiming to prioritise. 
Should the law emphasise the property rights of local people who face displacement, or should it protect the property rights of outsiders wishing to invest in holiday homes? 

Some may \isr{shy} away from this way of posing the question, by arguing that it would be better to rely on neutral rules that treat all owners the same way. In the gentrification scenario, such an appeal to neutrality could be the first step in an argument against regulating the property market to prevent the displacement of local people. But would that truly be a socially neutral approach to property? Presumably, it would be in the interest of original property owners to introduce such regulation. Hence, if {\it their} property rights are to be taken seriously, should not such regulation be put in place?

Importantly, both sides of a conflict like this are in a position to adopt a property narrative to argue for their interests. If escalation occurs, it will become practically impossible to insist that  property rules are neural on the issue of property's social functions. Here it is illustrative to briefly revisit the conflict between Donald Trump and Balmedie locals, discussed in Section \ref{sec:dts}.

As long as Trump threatened to use compulsory purchase, the local people could adopt a traditional ``pro-property'' stance against Trump. But as soon as Trump decided to fence them in by relying on his own property rights, they had to adopt a seemingly contradictory view on property, whereby Trump's property rights should be limited out of concern for the community. So how do we classify the anti-Trump stance with regard to property?

The answer is unclear under classical theories, on the basis of which the locals could even be accused of having an unprincipled attitude towards private property. But under the social function stance, a completely different picture suggests itself. The locals sought to protect property, but not just any property. The property they wanted to protect was the property which served the social function of sustaining the existing community. The property they wanted to protect was the property that {\it meant} something to them.\footnote{This is more than merely observing that they wanted to protect {\it their} property. In their desire to regulate the use of Trump's property, it would seem that the locals wanted more than merely to protect their own properties. They also wanted to protect certain social functions inherent in Trump's property, against the harmful dispositions of Trump himself. I return to this subtle perspective in more depth in Section \ref{sec:x} below.}

Undoubtedly, Trump and his supporters had a similar feeling about their property rights, and the development they wanted to carry out. Hence, in conflicts such as these the law will invariably have to take a stand regarding which social functions it wishes to promote. The social function theory asks us to be upfront about this, so that property adjudication in hard cases can proceed on the basis of substantive arguments about how to balance between different social functions.

\noo{
In all likelihood, such a stand must also sometimes be taken by whoever {\it interprets} the law, since it is exceedingly unlikely that the legislature will ever be able to provide deterministic rules for resolving all conflicts of this kind. Lastly and most controversially, the courts may find occasion to curtail the power of government -- perhaps even the legislature -- if such power is usurped by powerful actors wishing to undermine property's proper functions to further their own interests. 
This, in particular, raises the question of constitutional and human rights limits to interference in property, relative to those functions that are to be protected.}

The social functions of property sometimes force the law to prioritise, when various functions come into conflict with one another. However, social functions can also work together in a way that promotes certain property uses and decision-making structures for property management. In this way, social functions can come to alleviate the law from pressures to regulate, as discussed in the next subsection.

\subsection{The Regulating Effect of Property}

Property shapes and reflects societies, but it also shapes and reflects social commitments and dependencies within those societies.\footnote{See generally \cite{alexander09}.} Again, this function of property is highly dependent on context. A small business owner, by virtue of being a member of the local community, is discouraged from becoming a nuisance to their \isr{neighbours}, regardless of formal rules found in the law of nuisance.

Everything from erecting bright neon signs to proposing condemnation of \isr{neighbouring} properties are actions that an owner will be socially deterred from taking. If the local shop owner does not conform to social expectations, he will pay a social price. Indeed, most likely even an economic price, especially if his customer-base is local. At the same time, the local connection would serve to make the business owner positively invested in the well-being of the community. This would encourage everything from sponsoring local events to hiring local youths as part-time helpers.

But at the same time, the local business owner might be discouraged from changing his business model to become more competitive, if this is perceived as a threat to other members of the community. Economic rationality might suggest that he should expand, say, by physically acquiring more space and targeting new groups of customers, but social rationality might make this an untenable proposal. This, however, might render the business economically unsustainable, particularly if it is facing fierce competition from businesses that are not similarly constrained by community ties. Moreover, even if the business is in fact viable as long as the community remains in place to support it, the perception that it is not fulfilling its commercial potential can increase external pressures both on the business and the community. Importantly, in the age of regulation for commercial facilitation, the state itself might exert pressure of this kind, by acting in a way that makes it hard to sustain local businesses that are regarded as failing commercially.

Then, if our local shop owner goes out of business, for whatever reason, the new owner might not become integrated in the community in the same way, with obvious consequences for the property's function in that community. Indeed, if we imagine that the new owner is a large commercial actor who is hoping to raze the community in order to build a new shopping center, we are at once reminded of the stark contrasts that can arise between various social functions of property. The property rights of a shop owner can be the lifeblood of a community, while the exact same rights in the hands of someone else can spell destruction. 

\noo{While this is an undeniable empirical fact of property ownership, it is far from clear what its legal ramifications are. Here it is tempting to embrace a normative stance, and argue for particular social values that the law {\it should} promote. However, I will hold on to the descriptive mode of analysis a little further. 

For it is clear that regardless of whose interests win out in the end, the changed social function of property can in turn cause further changes to occur in the law of property.} This has legal ramifications, not only for property, but also for the regulatory regime surrounding its use. For instance, it seems clear that if a non-local, for-profit, owner is to be deterred from becoming a nuisance to neighbours, new and much stricter nuisance laws might have to be put in place. The social responsibility that was previously anchored in the community must now be protected more forcefully by the state. In general, this can cause the institution of property to weaken further, as the state assumes greater powers of interference. A feedback effect might result, as increased state regulation in turn threatens to make property ownership too burdensome for average community members. Hence, the most resourceful actors, those who are able to meet the state's demands and/or protect themselves against interference, gain more and more property, while the state gains more and more regulatory power.

The descriptive lesson to take from the social function theory is that mechanisms such as this should be taken seriously as potential consequences of changes in property structures. Moreover, by prioritising between social functions of property, the law indirectly serves a regulatory function, since different property functions manifest in different kinds of concrete property uses. On the one hand, direct regulation of land use can potentially be replaced by a more nuanced approach to property law, e.g., by promoting property ownership for marginalised groups. On the other hand, attempting to serve the public interest through direct state interference can have indirect effects on private property, e.g., by leading to concentration of property rights in the hands of the most resourceful, creating a need for yet more forceful mechanisms of state control.

The broader point at stake here can also be brought out in relation to the famous ``tragedy of the commons''.\footnote{hardin68} In his seminal article, Hardin describes how individually rational users of a common resource can eventually cause the depletion of that resource. The problem arises, according to Hardin, because individuals have no proper incentive to refrain from over-exploitation; the damage will be distributed among all resource users, so it will not outweigh the benefit of individual over-use in the short term.

In response, it has been typical to regard either state management or individual private ownership as the answer.\footnote{See \cite[8-13]{ostrom90}.} In the first case, the incentive not to over-exploit is supposed to be provided by the state, while in the second it is meant to be provided by the fact that the cost of over-exploitation cannot as easily be shifted from individual to the community.

However, as Elinor Ostrom and others have shown, the traditional narrative overlooks the fact that commons tend to come with community structures that provide appropriate incentives through locally grounded institutions or social arrangements.\footnote{See generally \cite{ostrom90}.} Moreover, as long as external forces do not threaten them, such arrangements can be more robust than either individual ownership or state control.

\noo{The former can be illusory, since damaging effects may not actually be limited to one's own land, e.g., in case of environmental harms. The second, on the other hand, can be ineffective due to the remoteness of the decision-makers to the effects of their decisions (causing a higher-order lack of appropriate incentives). In addition, state-led management can also increase the risk that the most important decision-making processes are captured by powerful interests, e.g., large commercial companies.}

The ideas of Ostrom on common pool management focus on local institutions for collective decision-making, not property rights. Moreover, the idea of private property is not identified as an important anchor for such institutions, which are rather thought to operate independently of the property regime.

By contrast, legal scholars discussing Ostrom's work have sometimes emphasised how common pool governance and private rights can be viewed as two sides of the property coin. This, specifically, is a key insight behind Smith's notion of a ``semicommons'', referring to property arrangements based on a combination of individual property rights and locally grounded institutions for collective action.\footnote{See generally \cite{smith00,smith02}.}

At the same time, some authors have pointed out weaknesses with the typical local institutions that tend to emerge for managing the commons. In particular, Heller and Dagan argue that such institutions often leave inadequate room for ``exit'' -- the possibility of alienating one's rights and obligations -- potentially causing local institutions to become oppressive towards individual members.\footnote{See \cite{heller00}.}

Hence, the idea that local institutions for resource management can be anchored in private property represents a potentially attractive way of thinking. Importantly, the social function theory of property already suggests pursuing this idea. Based on the social function approach, the descriptive fact that property structures shape decision-making processes at the local level is enough to conclude that local institutions for resource management should not be looked at in isolation from the law of property. I will build on this perspective in Chapter \ref{chap:3}, when I consider the institution of land consolidation in Norway and the management tools it offers as a possible alternative to expropriation in economic development cases.

\noo{Importantly, it suggests the insight that local resource management institutions can operate on the basis of private property, and that informal management practice are in many cases anchored in such rights. 

Recognising this entails the further realisation that it would be plainly inaccurate to proceed to interfere with property or introduce new property rules on the assumption that the effects pertain only to individual entitlements. In addition, one should recognise institutional and regulatory effects.}

It also bears noting that property rights can entail a wider set of values than those associated with specific institutions for local management of resources. For instance, making property a conceptual starting point makes it natural to also recognise that private property can function to protect individuals against abuse by local elites, e.g., by offering opportunities for exit. 

In addition, it becomes possible to recognise that social obligations may inhere in private property externally to whatever institutional frameworks happen to be in place at the local level. If such frameworks are marred by corruption and malpractice, a social function theorist can take the normative stance that property ownership still carries with it duties to care for other property dependants in the community. This duty, moreover, would exist independently of the extent to which it is presently fulfilled through local practices and institutional arrangements. 

I will return to this point later, when I discuss the human flourishing theory of property and its promise of internalising economic and social rights for non-owners into the structure of property itself. First, I will argue that the it is useful to distil a descriptive core from the social function theory, so that it may serve as a common ground for debate, allowing the interchange of ideas between different normative perspectives. \noo{Differences of opinion about what the law of property should be like should not detract from the insight that when discussing the law of property, one  that how one decides in this regard has important consequences for society as a whole.

In the next section, I will first briefly consider the much discussed US case of {\it State v Shack}.\footcite{shack71} The reason for doing this is two/fold. First, the case serves as the standard example of how social function reasoning can come to influence the application of rules that seem to be neutral on property's social functions. Second, the case was used by Eric Claeys to launch an attack on social function theorising. I consider his argument, concluding that while it sheds light on the outcome of {\it State v Shack}, it does nothing to detract from the descriptive core of the social function theory, quite the opposite.}

\subsection{The Descriptive Core}

The case of {\it State v Shack} is a standard US example of how social function reasoning can come to influence the application of rules that seem to be neutral on property's social functions.\footcite{shack71} The case concerned the right of a farmer to deny others access to his land, a basic exercise of the right to exclusion. The controversy arose after the two defendants, who worked for organizations that provided health-care and legal services to migrant farmworkers, entered the land of a farmer without permission. They were there to provide services to the farmer's employees, and when the farmer asked them to leave, they refused.

In the first instance, they were convicted of trespassing in keeping with New Jersey state law. However, the Supreme Court of New Jersey overturned the verdict on appeal. The Court held that as long as the defendants were there at the request of the workers, the owner's right to exclude them was more limited. Importantly, the court argued for this result -- which was not based on a natural reading of the New Jersey trespass statute -- by pointing also to the fact that the community of migrant workers was particularly fragile and in need of protection. Their right to receive visitors on the land where they worked and lived, therefore, had to be \isr{recognised}, also in a situation when this would involve a limitation to the farmer's right to exclude.

In so far as the property rules we rely on explicitly directs us to take the social aspect of property into account when applying the law, it might be permissible for the practically minded jurist to conclude that there is little need for general \isr{theorising} about property's social dimension. This dimension, in so far as it is relevant, is primarily a matter for the legislature, not theories that seek to explain property law. However, cases like {\it State v Shack} show that the social dimension can be relevant even when it is not mentioned in any authority, even in relation to clear rules that would otherwise appear to leave little room for statutory interpretation. It arises as relevant, in such cases, because the social dimension is intrinsic to property itself.

This might be a radical claim, but it is primarily a descriptive one. Indeed, even if the case of {\it State v Shack} had gone the other way, the same conceptual conclusion might well have been appropriate. If the right to exclusion had received priority over the workers' right to receive guests and the owner's obligation to respect this, that too would be an outcome underscoring the social function of property. 

A nice demonstration of how neutrality is elusive in this regard can be found in an article by Eric Claeys, where he is critical both of the social function theory in general and {\it State v Shack} in particular.\footcite{claeys09} Importantly, despite his intention to criticise the social function theory, Claeys is led to argue against the ruling of {\it State v Shack} by pointing to those aspects of the social context that spoke in favour of the farmer.\footnote{\cite[941-942]{claeys09}.} Essentially, his argument is that by considering the social circumstances in {\it more} depth, a different outcome suggests itself.\footnote{\cite[941]{claeys09} (``there are good reasons for suspecting that there was more blame to go around in Shack than comes across in the case's statement of facts'').} 

But if this is true, it is no argument against the descriptive content of the social function theory. Rather, it becomes a further affirmation of the descriptive adequacy of such an account of property. At the same time, it becomes an argument against those who think that the social function idea dictates the ``correct'' outcome in cases such as {\it State v Shack}. As Claey's advocacy on behalf of the farmer shows, the descriptive part of the social function theory hardly entails specific normative commitments.

Claeys argues forcefully against normative fundamentalism, and he might have a point in criticising some social function theorists for normative naivety.\footnote{\cite[945]{claeys09} (``Judges might think they are doing what is equitable and prudent. In reality, however, maybe they are appealing to a perfectionist theory of politics to restructure the law, to redistribute property, and ultimately to dispense justice in a manner encouraging all parties to become dependent on them.'')} However, I do not follow Claeys when he takes this to be an argument against the form of legal reasoning that social function theories promote and which he himself skilfully engages in.\footnote{In particular, I do not follow the leap Claeys makes when he suggests that it is beneficial to keep ``discretely submerged'' what he describes as ``culture war overtones'' in legal reasoning.\cite[947]{claeys09}.} 

In {\it State v Shack}, such reasoning was clearly in order. To engage in it was far less naive than to dismiss it on the basis that it would be irrelevant to the case. Indeed, if the social function view had been dismissed, the narrow exclusion-based idea of property would in effect do {\it unacknowledged} normative work, by pushing social aspects out of sight and out of mind.

By contrast, the social function narrative pushes us towards a more complete picture of the relevant facts. This is its primary importance, in my opinion. However, many of its supporters appear to argue that the main significance of the theory is that it delivers an ethically superior approach to property law.\footnote{See, e.g., \cite{penalver09}.} Unsurprisingly, critics such as Claeys use this to launch attacks on the social function theory, by suggesting that it represents a way of thinking that will invariably lead to lessened constitutional property protection and greater risk of abusive state interference.\footnote{See \cite{claeys09} (``The more ``virtue'' is a dominant theme in property regulation, the less effective ``property'' is in politics, as a liberal metaphor steering religious, ethnic, or ideological extremism out of the public square'').} %Indeed, increasing the room for state interference is often seen as the aim of conceptual reconfiguration; the social function view of property tends to be associated with social democratic and/or redistributive political projects, by which the notion of property is recast to justify greater interference in established rights.\footnote{Despite his commitment to ``value-pluralism'', this motivation is also clearly felt in the work of Gregory Alexander. He argues, for instance, that the social obligations inherent in property imply that the ``state should be empowered and may even be obligated to compel the wealthy to share their surplus with the poor'', see \cite[746]{alexander09}. For an assessment linking similar views on property in Europe to the increasing influence of social democratic thought after the Second World War, see \cite{allen10}.}

%It is important to note, however, that while social democratic policies may be easier to justify by \isr{emphasising} the social function of property, the mere recognition that property has an important social dimension does not in itself justify such policies. Moreover, it seems that the most crucial premise used in arguments for greater state control and state-led redistribution projects concern the nature of the state, not the functions of property.

But why should it follow from property's social function that the state is the ultimate social institution to which property {\it should} answer? Why not take the view that property should answer to informal social structures, such as those that it is embedded in by virtue of owners' membership in local communities?

If so, one might as well want to limit the state's role to that of ensuring fair play among individuals and communities. Indeed, from the point of view of social function theories, the appropriateness of direct state control seems to depend on evidence that property-based social structures fail to function properly and, crucially, that state control is a {\it better} alternative, according to some normative measuring stick.

%%This requires arguments anchored in specific social and political property contexts. Hence, to move uncritically between talk of the ``community'' and talk of the state, as writers like Pe\~{n}alver and Alexander sometimes appear to do, is inappropriate.

%In my opinion, the social function view of property tells us little about how widely the state should intervene in property in a given society. It allows us to recognize the {\it possibility} that the state may have to intervene on behalf of certain property values, say those that aim to protect communities. 
%But this is no argument in favour of any general position on state interference. 

The social function theory provides us with a conceptual tool for reasoning more clearly about {\it when} it is appropriate for the state to intervene. But the Humean position, namely that the existing property structure represents a socially emergent equilibrium, remains plausible. Moreover, the normative stance that this equilibrium is a {\it good} one (or at least as good as it gets) remains as contentious as ever.

It bears emphasising that by arguing in this way, I depart from the stance taken by many contemporary scholars who advocate on behalf of social function theories. Hanoch Dagan, for instance, is a self-confessed liberal who argues for a social function understanding on the basis that it is morally superior. ``A theory of property that excludes social responsibility is unjust'', he writes, and goes on to argue that ``erasing the social responsibility of ownership would undermine both the freedom-enhancing pluralism and the individuality-enhancing multiplicity that is crucial to the liberal ideal of justice''.\footcite[1259]{dagan07}

If this is true, then it is certainly a persuasive argument for those who believe in a ``liberal idea of justice''. But for those who do not, or believe that property law is -- or should be -- largely neutral on this point, a normative argument along these lines can only discourage them from adopting a social function approach. Such a reader would be understandably suspicious that the {\it content} of the social function theory -- as Dagan understands it -- is biased towards a liberal world view. Such a reader might agree that property continuously interacts with social structures, but reject the theory on the basis that it seems to carry with it a normative commitment to promote liberalism.

Dagan is not alone in proposing highly normative social function theories. Indeed, most contemporary scholars endorsing a social function view on property base themselves on highly value-laden assessments of property institutions.\footnote{See, e.g. \cite{alexander09,crawford11,davidson11,singer09,penalver09}.} These scholars provide interesting insights into the nature of property, but they might overstate the desirable normative implications of adopting a social function view. In addition, they appear to believe we should embrace certain values and reject others. Hence, one is sometimes left with the impression that the social function theory has little to offer beyond the values with which it is imbued, which can in turn push the law in the direction that these writers deem desirable. 

For instance, it is Dagan's stated aim to propose a theory that promotes specific liberal values. ``There is room to allow for the virtue of social responsibility and solidarity'', he writes, continuing by suggesting that ``those who endorse these values should seek to incorporate them -- alongside and in perpetual tension with the value of individual liberty -- into our conception of private property''.\footcite[802]{dagan99} This view is reflected further in the concrete policy recommendations he makes, for instance in relation to the question of when it is appropriate to award less than ``full'' (market value) compensation for property following a taking.\footnote{See generally \cite{dagan14b}.}

Normative assertions like these are not necessarily wrong, but they need not be accepted in order to conclude that the social function of property should be given a more prominent place in property theory. Importantly, I think the focus on normative reasons threatens to overshadow the most straightforward reason for looking to social structures, namely that they are almost always crucially important behind the scenes, even if they go unacknowledged. 

The social function theory, rather than being ``good, period'', as Dagan suggests, is simply more  accurate, irrespective of one's ethical or political inclinations. As such, it provides the foundation for a debate where different values and norms can be presented in a way that is conducive to meaningful debate, on the basis of a minimal number of hidden assumptions and implied commitments. Thus, the first reason to accept the social function theory is epistemic, not deontic.

\noo{ That is not to say that normative theories should not be formulated on the basis of the social function theory. I maintain only that  it is useful to maintain a distinction  between the descriptive and normative aspects of such theorising. I return to normative aspects in the next section, arguing that the commitment to ``human flourishing'' endorsed by Professor Alexander is a particularly well-argued norm that arises from value-based assessment of the social function of property. This, I believe, is in large part also due to the value-pluralism inherent in this idea, suggesting as a positive normative claim that our notions of property {\it should} allow for a divergence of opinions and values to influence the law and its application in this area.}

Theories can hardly be entirely value-neutral, nor is this a goal in itself. Still, a good theory is one that can at least serve as a common ground for further discussion based on disagreement about values and priorities. \noo{According to Kevin Gray, ``the stuff of modern property theory involves a consonance of entitlement, obligation and mutual respect''.\footcite[37]{gray11} This is a rather loose way of putting it, but I believe it also points to a measured perspective that is ultimately highly appropriate.} Making room for normative divergences, moreover, can hopefully diminish the worry that a broader theoretical outlook is the first step towards unchecked state power and rule by ``judicial philosopher-kings'', as Claeys puts it.\footcite[944]{claeys09}

In the next subsection, I will argue in some more detail why such a cautious perspective is warranted, by considering how the Italian fascists appropriated the social function theory in 1930s. Building on the work of di Robilant, I will also briefly track how non-fascist property scholars opposed this development by focusing on value-pluralism, local self-governance and freedom.\footcite{robilant13} Importantly, these scholars embraced the social function theory as a common ground from which to launch a meaningful attack on more radical ideas, without alienating those with divergent views. Instead of clinging to the old-style liberal discourse that the fascists had either rejected or subverted, many Italian non-fascists were willing to engage in a discourse revolving around property's social function, by spelling out a more measured set of ideas based on this premise.

Crucially, this set the stage for a form of intellectual resistance that did not reject those aspects of fascism that had appeal to the public and which arguably reflected true insight into the unfairness and lack of sustainability of the established legal order.

\subsection{Rooting out Fascism}

\noo{ While the social function theory makes intuitive sense, it is also highly abstract. Therefore, its exact content has been notoriously hard to pin down. This is \isr{recognised} by contemporary scholars endorsing a normative view, who attempt to address this by proposing lists of values that should be taken into account while giving examples of how they should be used to inform the law in concrete areas or cases.\footnote{See, e.g., \cite{alexander14,alexander11,dagan07}.} Unsurprisingly, however, views soon diverge regarding the concrete import of a social function view on legal reasoning. Even so, the contemporary debate appears to be based on a common ground that is quite stable, also with respect to the overall notion of what good the theory can do. However, as history shows, this state of affairs is by no means guaranteed.

In a recent article, Anna di Robilant illustrates this point by tracking the history of social function theorising in Italy during the fascist era.}

According to di Robilant, the fascists were happy to embrace the social function theory of property. To 
the fascists, the social function theory provided a conceptual starting point from which to develop their idea that rights and obligations in property should be made to answer to one core value: the interests of the state.\footnote{See \cite[908-909]{robilant13} (``Fascist property scholars had also appropriated the social function formula. For the Fascists, the social function of property meant the superior interest of the Fascist state.'').} This stance was as effective as it was oversimplified. As di Robilant notes, ``earlier writers had been hopelessly evasive about the meaning and content of the social element of property''.\footcite[909]{robilant13} Hence, the fascist approach filled a need for clarity about the implications of the main idea, which was by now attracting increasing support both from the public and the academic community. Established property doctrine, it was widely felt, was both ineffective and unfair to ordinary people. Rather than securing productivity and a livelihood for all, property was used mainly as an instrument for maintaining the privileged position of the elites. By promising to change this state of affairs, the fascists attracted many to their cause.

As di Robilant notes, supporters of the fascist idea of property made clear that ``social function meant the productive needs of the Fascist nation''.\footcite[909]{robilant13} But at the same time, they cleverly denied that there was a ``contradiction between subordinating individual property rights to the larger interest of the Fascist state and the liberal language of autonomy, personhood, and labor''.\footcite[900]{robilant13} In this way, fascist scholars could claim that fascist liberalism was true liberalism, thereby subverting the conceptual basis for the traditional idea of liberal justice.\footcite[900]{robilant13} In this situation, there was reason to suspect that clinging to liberal dogma would be a largely ineffective response. Moreover, it seemed undeniable that fascism's appeal was rooted in real concerns about the fairness and effectiveness of the liberal legal order. 

Hence, many non-fascists shunned away from uncritical defence of traditional liberalism. Instead, they agreed that property's social function should come into focus, but \isr{emphasised} the plurality of values that could potentially inform this function, which could be different from the interests of the state. In addition, they also noted that property rights were invariably associated with {\it control} over resources, and that the social functions of property depended on the resources in question. To own property, they argued, provides individuals with a highly valued source of privacy, power and freedom that is worthy of protection. 

To summarise their insights, Italian scholars adopted the metaphor of a ``tree'', by describing the core social function of property as the trunk, while referring to the various resource-specific values attached to property as branches.\footcite[894-916]{robilant13} As di Robilant notes regarding these theorists:

\begin{quote}
The rise of Fascism, they realized, was the
consequence of the crisis of liberalism. It was the consequence of liberals' insensibility to new ideas about the proper balance between individual rights and the interest of the collectivity.\footcite[907]{robilant13}
\end{quote}

In light of this, the tree-theorists concluded that continued insistence on the protection of the autonomy of owners was not a viable response. Instead, they adopted a theory that ``acknowledges and foregrounds the social dimension of property'', but without committing themselves to fascist ideas about the supreme moral authority of the state.\footcite[907]{robilant13} The value of autonomy was in turn recast in terms of property's social function. Arguably, this served to make the case far more compelling. Protecting autonomy could be seen as an aspect of protecting property's freedom-enhancing function, both at the individual level and as a way of ensuring a right to self-governance and sustenance for families and local communities. This, moreover, could not easily be derided as tantamount to protecting unfair privilege and entitlement. Rather, property became elevated from an individual liberal right to a crucial building block of participatory democracy.

The story of fascist appropriation of the social function theory demonstrates why it is sensible to maintain a descriptive perspective on its core features. Indeed, the readiness with which the fascists embraced social function \isr{theorising} serves as a reminder that we cannot easily predict what normative values may come to be promoted on its basis. At the same time, we are reminded of the danger of attaching too much normative prestige to a theory that is abstract and open to various interpretations.

In particular, it seems that a failure to \isr{recognise} the descriptive nature of the core idea can lead to unrealistic expectations of what the social function theory provides. In addition, it will make it harder for the theory to gain acceptance as a conceptual common ground from which to depart when engaging in debate. Indeed, if no division is \isr{recognised} between normative and descriptive aspects, the historical record would allow detractors to make a {\it prima facie} plausible attack on the social function theory by arguing that it is fascism in disguise, or that fascism, rather than liberal justice, is where we end up in practice should we \isr{choose} to adopt it.\footnote{This would echo the claim already made by Claeys, that the theory (when coupled with virtue ethics) might become a slippery slope towards the kind of extremism and revolt against oppression that gave rise to the Rwanda genocide in the early 1990s \cite[926-927]{claeys09}.}

In response, one might retort that this is cherry picking the historical facts, or that the fascists misunderstood or perverted the theory. That is certainly plausible, but this kind of debate is in itself rather unhelpful. Unless the social function theory is rendered neutral enough to be acceptable as the conceptual premise of debate, it is likely going to fail as a template for negotiating conflicts about property. Those who oppose the norms associated with the theory will oppose also the core descriptive content, if they feel that the latter commits them to the former. This, in turn, suggests that those advocating on behalf of the social function theory should take care to avoid rhetorical hubris. The main point to convey is that the theory is more accurate, in a purely epistemic sense, than other conceptualisations of property.

That said, the new dimensions uncovered by the social function view can also inspire novel normative theories. I explore this further in the next section.

\noo{
The story of the fascist appropriation of the social function theory also points to a danger often attached to abstract theories, namely that they allow us to opportunistically recast whatever values we wish to promote by providing qualifications for them in abstract terms that are hard to refute. The fascists did this, and the non-fascists responded. Hence, in the end one could do little more than hope that the fascists' vision of their state as an ``ethical state'' that ``every man holds in his heart'' would eventually prove less attractive then the promise of self-governing communities characterised by diversity in life and character.


\subsection{Towards a Normative Stance}

The social function theory can facilitate a new kind of normative reasoning, arising from how the theory allows us to \isr{recognise} more subtle distinctions between different kinds of property and different kinds of circumstances. For instance, a staunch entitlements-based approach to autonomy will leave us with little room to differentiate between the protection of investment property and the protection of a home, unless such a distinction is explicitly provided for in the law. But a social function approach compels us to notice the difference and to acknowledge that it might be legally, as well as ethically, relevant. Hence, if we seek to argue for protection of investment property, we must in principle be prepared to face counter-arguments that revolve around particulars of the investor's role in society and their relationship to the community of people that are affected by how they manage their property. Similarly, if someone argues against protecting home ownership, we can respond by drawing on additional arguments based on the importance of the home both to the owner, their family and friends, and their community. Under the social function theory, it becomes relevant to address how a home creates a sense of belonging and provides a basis for membership in social structures.

I believe normative assessments should aim to be as concrete as possible. That said, I think it is worthwhile to provide more abstract forms of expression for core values, to clarify the ethical premises that provide the basis for concrete value-based conclusions. Therefore, normative theories should aim to be meta-ethical, not just ethical. They should provide a vocabulary and a conceptual framework tailored to advancing one's values. However, they should recognise that the ultimate expression of those values is provided only in relation to concrete facts. This, I believe, is prudent in light of how abstract ethical assertions are necessarily imprecise and run the risk of being distorted or exaggerated.

Invariably, the most accurate information regarding the values I rely on when assessing cases will be conveyed by my assessment of the cases, not by my theorising. On a deeper level, I am inclined to believe that value-systems are more or less unique to individuals, so that ethical theories are helpful primarily in that they provide an introduction to keywords and important lines of argument that will recur in different forms. As such, they enhance understanding, making it easier to communicate ideas and opinions in such a way that potential respondents are likely to enjoy a more accurate impression of what they are responding to. 

In short, I believe that ethics make moral judgements communicable, allowing new ideas to be created in the minds of individuals. I believe in ethical men and women, but not in ``ethical Man'' or -- God forbid -- the ``ethical State''. Luckily, I find some support for this view in recent theories that have been proposed as normative extensions of the social function theory of property. These are the subject of my next section.
}

\section{Human Flourishing}\label{sec:hf}

Taking the social function theory seriously forces us to \isr{recognise} that a person's relation to property can be partly constitutive of that person's social and personal identity, including both its political and economic components. Hence, property influences people's preferences, as well as what paths lie open to them when they consider their life choices.\footnote{See generally \cite{alexander09}.} This effect is not limited to the owner, it comes into play for anyone who is socially or economically connected to property in some way. The life-significance of property is often clearly felt also by a large group of non-owners.\footcite[128-129]{alexander09d} The importance of property is obviously reduced if we move away from it in terms of social or economic distance. Hence, in many cases, property will be most important to its owner, simply because they are closest to it. This is not always the case, however, especially not if property rights are unevenly distributed, or in the possession of disinterested or negligent owners. Moreover, as mathematically oriented sociologists like pointing out, social connections are ubiquitous and the world is often smaller than it seems.\footnote{See generally \cite{schnettler09}.}

Hence, there is great potential for making wide-reaching socio-normative claims on the basis of the social function perspective on the meaning and content of property. But which such claims {\it should} we be making? According to some, we should adjust our moral compass by looking to the overriding norm of {\it human flourishing} as a guiding principle of property law. In a recent article, Alexander goes as far as to declare that human flourishing is the ``moral foundation of private property''.\footcite[1261]{alexander14} 

Human flourishing has a good ring to it, but what does it mean? According to Alexander, several values are implicated, both public and private.\footnote{See generally \cite{alexander14,alexander11}.} Importantly, Alexander stresses that human flourishing is {\it value pluralistic}.\footnote{\cite[750-751]{alexander09}.} There is not one core value that always guarantees a rewarding life. To flourish means to negotiate a range of different impulses, both internal and external. Importantly, these act in a social context which influences their meaning and impact\footcite[1035-1052]{alexander11}

In the following, I consider some values that I regard as particularly important for the study of economic development takings. I start by the values enshrined in economic and social rights, which should arguably also inform how the law should approach property's social functions.

\subsection{Property as an Anchor for Economic and Social Rights}

The so-called ``second generation'' of human rights consists of basic economic and social rights that complement traditional political rights. This includes rights such as the right to housing, the right to food, and the right to work. Economic and social rights of this kind often involve property. Specifically, they often involve interests in property that are not recognised as ownership, e.g., housing rights for squatters or rights to food and work for landless rural people. 

If the notion of property is conceptualised in the traditional way, as an arrangement to protect individual entitlements, the relationship between property and economic and social rights appears to be one filled with tension. In particular, if economic and social rights require owners to give up some property entitlements, it becomes natural to portray property as an institution standing in the way of justice. 

However, the human flourishing theory can be used to tell a very different story, namely one where economic and social rights are anchored in the notion of property itself. \noo{As I have already explained, I believe -- in contrast to both Crawford and Alexander -- that it is useful to decouple normative claims from the descriptive core of the social function theory.\footnote{Crawford comments that the social function theory on its own  ``is not self-defining and invites many interpretations'', see \cite[1089]{crawford11}. The normative theory he proposes is clearly aimed at \isr{filling} this perceived gap, by pinning down normative commitments that Crawford believes are intrinsic to the theory. However, as I have already argued, I reject this approach, since it unwisely downplays the fact that the social function theory can serve as a common ground among commentators with widely divergent normative views. Indeed, Crawford himself refers \isr{unfavourably} to a writer who addresses the social function theory, but who, according to Crawford, proposes that ``property's social function is best served by focusing on overall economic production and efficiency in a given society, allowing the market's invisible hand to work its magic'', \cite[see][1089]{crawford11}. Against Crawford, I would argue that it is better to counter such a claim by arguing why it is normatively wrong rather than by suggesting that people with such values should be 
discouraged from attempting to argue for them on the basis of a social function understanding of property. Rather, by encouraging such an argument it should become easier to make the case why the values promoted are ultimately undesirable. This, at least, should follow if Crawford is otherwise largely correct (as I think he might be).} I therefore refer to the more distinctly normative aspects of their work as human flourishing theorising.}

%Indeed, if securing human flourishing is the primary purpose of property, our property rules should protect the economic and social rights of the community of property dependants as a whole, not just the owner.\footcite[1035-1052]{alexander11} Hence, protecting such economic and social rights would not necessarily entail interference with private property. In many situations, rather, the human flourishing view would allow us to add the interests of non-owners to the list of reasons why property interference is best to avoid.
Importantly, the human flourishing theory compels us to take into account the interests and needs of property dependants other than the owner. As Colin Crawford puts it, the purpose of property should be to ``secure the goal of human flourishing for all citizens within any state''.\footcite[1089]{crawford11} Consider, for instance, the right to housing. If the interests of a property owner come into conflict with the housing rights of a property dependant, the human flourishing theory encourages us to approach this as a tension {\it within} property, between different property functions. 

This way of understanding economic and social rights for non-owners can inform new perspectives on how to approach such rights when they involve property. \noo{

Indeed, if the owner of a house rent out their property as a home to someone else, the importance of the property might be {\it greater} to a non-owner. Moreover, assuming a society where tenancy is a well-functioning social institution, the continuation of the established property pattern might well be of greater importance to the tenant than it is to the owner. 

State interference in these private property rights might offend against human flourishing not only because it interferes with the landlord's autonomy and financial entitlements, but also because it interferes with the housing interests of the tenant. Both are aspects of property's social function, which the law should protect, also against state interference.

If the interests of the owner come into conflict with the housing rights of a property dependant, the resulting tension can be seen as a tension {\it within} property, between different property functions. For instance, if a piece of land has an owner, it discourages squatting. This is perhaps a good thing, but can also make it harder for the marginalised to find adequate housing.} Specifically, we are led to consider that the appropriate way to approach the rights of non-owners in relation to property also depends on who the owner is and the choices they make in managing their property.

For instance, if owners live on their land and don't own much more than they need themselves,  squatting is effectively discouraged. At the same time it becomes much harder to maintain the criticism that this is somehow an affront to the housing rights of the landless. But even the owner of an unoccupied building can discourage squatting by managing \isr{their} property well. Moreover, this too can undercut potential criticism on the basis of housing rights, for instance if the owner uses the building to engage in commercial activity that contributes to sustaining the local community.

On the other hand, if owners mismanage their properties, for instance because they seek to obtain demolition licenses or engage in other forms of speculation, squatters might take opportunity of this and feel encouraged to occupy the property. This risk clearly increases if housing cannot be afforded by a large number of a society's members. 

\noo{ Importantly, the human flourishing account allows us to argue that something is amiss with prevailing property structures in such cases. In particular, the theory allows us to recognise that possible failures of property as a social institution might be relevant when considering rights and responsibilities of private ownership. Moreover, it becomes possible to see the fulfilment of social and economic rights as aspects of property's proper function, which property law should promote.

\noo{ Squatting, for instance, clearly affect the owner, influencing both the meaning and the value of \isr{their} property, both to \isr{them}, potential buyers, the local government, the state, and other interested parties. Even the mere {\it risk} of squatting can play such a role. But a property theory which does not \isr{recognise} the social function of property might not allow us to take this into due regard. As long as the standard expectation of an owner is to be able to enjoy their apartment free of squatters, an entitlements-based view on property could easily force us into denial regarding actual (risk of) squatters.}}

If private property is thought of merely as entitlement-protection, the state might stake from this the lesson that interference in property is required to secure housing rights, even though the real problem is that property itself does not function as it should within society. Hence, the result can be that property structures are damaged further, as the state pursues policies of interference and centralised management, without addressing how private property as such can promote human flourishing.

\noo{  The normative significance of real life -- where squatting often happens due to badly managed property -- is discounted because our conceptual glasses block it out. Then, the unavoidable consequence is that the state also \isr{recognises} a {\it positive} obligation to forbid squatting, and to forcibly remove squatters on behalf of owners. Under a narrow entitlements-based conception, this is the natural outcome, which must be classified as an act of protecting private property. Hence, under classical liberal values, it also becomes {\it good}. 

Here, however, the social function theory permits us to take a highly divergent view, to carry forward different value-judgements. If squatting is \isr{recognised} as creating new interests and obligations attaching to the property, it may now be argued that it is the use of state power to evict that is the most severe act of interference. Moreover, this might be more than merely an interference in whatever housing rights the squatters might have, if any. It can also be seen as an interference in {\it property}, in need of further justification. 

In the Netherlands, the Supreme Court adopted a line of reasoning reflecting a similar insight when it held that the right not to be disturbed in one's home life also applied to squatters. Hence, the property owner could not forcibly evict people who had taken up residence in their property.\footnote{See NJ 1971/38. The court held that the lower court had erred in taking it proven that the ``house in the original charge was \isr{`in use'} by the owner of this house'', as required by the statute under which the squatters were tried. Instead, the Supreme Court held that ``art.138, in so far as it mentions houses, is specifically aimed at protecting home rights, in connection with which the words \isr{`in use'} (differently than the court judged) can only be understood as \isr{`actually in use as a house'}, as in accordance with ordinary use of language''. The upshot was that it was the squatters, not the owner, who enjoyed protection under the statute. In terms of the bundle theory, a right thought to be in the owner's bundle was deemed to actually belong to the bundle of the squatters, as this corresponded better to the circumstances of the case and the purposes meant to be served by the statute in question.}}

By contrast, the human flourishing narrative suggests a perspective whereby both owners and non-owners are recognised as victims of a failure of the state to protect property's proper functions. For a concrete example, I mention the case of {\it Modderklip East Squatters v. Modderklip Boerdery (Pty) Ltd}, analysed in depth by Alexander and Pen\~{n}alver.\footcite[154-160]{alexander11} 

The case dealt with squatting on a massive scale: some 400 people had initially taken up residence on land owned by Modderklip Farm, apparently under the belief that it belonged to the city of Johannesburg. The owner attempted to have them evicted and obtained an eviction order, but the local authorities refused to implement it. Eventually, the settlement grew to 40 000 people and Modderklip Farm complained that its constitutional property rights had not been respected.

The Supreme Court of Appeal concluded that Modderklip's property rights had indeed been violated, but noted that so had the rights of the squatters, since the state had failed to provide them with adequate housing.\footnote{See \cite{modderklip04}.} However, the Court upheld the eviction order and granted Modderklip Farm compensation for the state's failure to implement it. Hence, while the Court recognised that both housing rights and property rights were at stake, it pursued a traditional balancing approach, finding in favour of property.

The Constitutional Court, on the other hand, adopted an agnostic view on the relationship between housing rights and property rights. The Court agreed that the eviction order was valid, but concluded that as long as the state failed in its obligations towards the squatters, the order should not be implemented.\footcite{modderklip05} Hence, the owner's right to have the squatters evicted was made contingent upon an adequate plan for relocation. 

However, the Court also ordered that Modderklip should receive monetary compensation from the state. In this way, the Court implicitly \isr{recognised} the social function of property; they refused to give full effect to Modderklip's property rights as long as that meant putting other rights in jeopardy. The fact that the squatters had no place to go influenced the content of Modderklip's right, making it impermissible to implement a standing eviction order. Importantly, however, this was a failure of the state to protect property, on the basis of which compensation should be paid.

The failure to protect property that the Court recognised in {\it Modderklip} can be understood broadly, as encompassing also the failure to provide adequate housing to a large group of property dependants. Indeed, this perspective is suggested by how the housing rights of the squatters seemed to influence the content of the property rights and obligations of Modderklip. Taking this one step further means recognising that protection of property can be a potential source of justice for anyone, including squatters.

In a detailed analysis of {\it Modderklip}, Alexander and Pe\~{n}alver go quite far in this direction. They argue that the case highlights how property owners themselves can have responsibilities towards property dependants, obligations that endure as long as private property is protected.\footnote{\cite[157]{alexander11} (``The courts' unwillingness to ratify Modderklip's desire to remove the squatters from its land illustrates the courts' willingness to take seriously the obligations of owners, not only as they concern owners' direct relationship with the state but also in relation to the needs of other citizens'').} This normative turn makes property owners addressees of obligations arising from the economic and social rights of non-owners. In this way, it strengthens such rights. 

However, it also strengthens the institution of property, highlighting why it might be appropriate to grant it strong protection against interference by external forces. In particular, a human flourishing approach might serve as a bulwark against the idea that the ultimate expression of the public interest can be found in the actions taken by the state. As a replacement, the human flourishing account suggests the view that public interests manifest immediately at their place of origin, including in relation to private property.

As Alexander puts it in a recent article:

\begin{quote} The values that are
part of property's public dimension in many instances are necessary
to support, facilitate, and enable property's private ends.
Hence, any account of public and private values that depicts them as categorically
separate is grossly misleading. One important consequence of this
insight is that many legal disputes that appear to pose a conflict between
the private and public spheres or that seemingly
require the involvement of public law can and
should, in fact, be resolved on the basis of private law -- the law
of property alone.\footcite[1295-1296]{alexander14} \end{quote}

Perhaps the most important structural aspect of this insight concerns the mechanisms used to resolve tensions between different property values. Importantly, it might not be necessary to introduce intermediaries between owners and other primary stakeholders. To introduce such intermediaries, whether they are state bodies, international institutions, NGOs, or commercial enterprises, carries with it the risk that the decision-making process can be captured by external forces. It might be better, therefore, if the necessary balancing of interests occurs at the level of property law.\footnote{This can also involve institutions as long as they are directly based on property, set up to facilitate participatory decision-making about property among the class of property dependants most directly affected. For private owners of shares in jointly owner property, the land consolidation courts discussed in Chapter \ref{chap:4} are an example of such a mechanism. However, as I discuss more in that chapter, exporting this institution to a setting of non-egalitarian property ownership likely requires giving legal standing to a larger group of property dependants.}

This would also allow economic and social rights to be promoted to the private law sphere, even in the absence of specific state action or legislation. Effectively, it would become necessary for private parties to recognise that human rights give rise to obligations that apply to them directly, as property owners. If successful, this idea can obviate the need for direct state interference to secure desirable social and economic objectives. 

At the same time, a human flourishing account could be used to argue against the legitimacy of interference on the basis of broader concerns rooted in communitarian economic and social values. If such values inform property's proper function, it underscores that the state has an obligation to not undermine property. Instead, the state's duty becomes that of continuously facilitating improved legal frameworks for private property.

A clear commitment to property as an institution is needed in order for this aspect of the human flourishing idea to come to fruition. Moreover, such a commitment can not be taken for granted. A good example is again found by looking to South Africa and the fallout of the Mineral and Petroleum Resources Development Act 2002.\footnote{See \cite{mrdpa02}.} This Act introduced state ``custodianship'' over previously private mineral and petroleum rights.\footnote{See \cite[3]{mrdpa02}.} However, the Constitutional Court ruled in {\it Agri South Africa v Minister for Minerals and Energy} that this did not amount to expropriation, only deprivation.\footnote{See \cite{agri13}.} 

The property clause in section 25 of the South African Constitution gives effect to an important distinction between expropriation and deprivation, in that it only demands compensation in case of expropriation.\footnote{See \cite[18-19]{walt11}.} Hence, the decision in {\it Agri} implies that privately held mineral and petroleum rights in South Africa can be brought under state custodianship without any payment of compensation.\footnote{For a commentary on the decision, see \cite{marais15a,marais15b}.}

The state custodianship introduced by the act in question would undoubtedly deprive the current owner of their mineral rights. Moreover, the state would subsequently be empowered to grant the minerals to a third party, e.g., a competing commercial company. In my terminology, therefore, the introduction of state custodianship amounted to a sweeping authorisation for economic development takings affecting all mineral and petroleum resources found in South Africa.

The crucial finding of the majority in {\it Agri} was that this did not involve any transfer of mineral rights to the state. On plain reading, this seems quite absurd. But the majority did not engage in any plain reading. Rather, its reading was motivated by what it thought would be appropriate given the history of South Africa and the prevailing social context of property, shaped by a past of racial discrimination.

Importantly, the majority did not use this past as an argument to assess the scope of compensation rights. Instead, it relied on it only implicitly when interpreting the word ``expropriation'' in full generality. This may have been a grave mistake. As the minority points out, the understanding of expropriation established by the majority ``in effect immunises, by definition, any legislative transfer of property from existing property holders to others if it is done by the state as custodian of the country's resources, from being recognised as expropriation''.

This is not a precedent that strengthens the institution of property. Moreover, it reverts back to the traditional narrative that curing social ills requires negating private property or at least placing it under direct state management. It even seems to indirectly diminish the social obligations of property owners. Indeed, to the majority, the uncompensated taking of mineral rights was justified under reference to the ``obligation imposed by section 25 not to over-emphasise private property rights at the expense of the state’s social responsibilities''.

Hence, we see that social obligations are talked about as though they only target states. On this narrative, honouring ``social responsibilities'' can only mean enhancing state power at the expense of private property. But then the implication also seems to be that in the absence of active state interference, private property is a privilege unbridled by social obligations. In the future of mineral and petroleum exploitation in South Africa, it will be a privilege only enjoyed by those chosen by the state.

The principle established in {\it Agri} could prove very important to the future of property in South Africa. Indeed, since {\it Agri} was decided on the basis of an interpretation of the word ``expropriation'' as such, there is no need in the future to continue pointing to the social context of past injustice when arguing that the state should be allowed to carry out economic development takings  -- as a custodian -- without paying any compensation at all to the original owner. This opens the floodgates to predatory practices, creating very strong incentives for the government to abuse its power.

It is interesting to note that on this point of principle, {\it Agri} could easily have gone differently. This would have been achieved by a ruling consistent with the minority opinion, which held (1) that compensation should be paid and (2) that sufficient compensation {\it had} in fact been paid, as the mineral and petroleum act provided for a limited form of compensation in kind whereby the previous mineral owners were given a chance of maintaining substantially the same mineral interests provided they could meet certain conditions within a deadline (the mineral owner in {\it Agri} had been unable to do so due to insolvency).

This did not amount to anything resembling market value compensation, but the minority invoked the social context concretely to argue why it was nevertheless sufficient. Hence, by following the opinion of the minority, the outcome would have been the same, but the consequences for the institution of property could have been very different.

This shows the importance of one's conceptual understanding, since the main point of difference between the minority and the majority seems to have been with regards to the role that the social context of property should play in legal reasoning. The minority invoked such concerns concretely, to determine the extent of a concrete right to compensation. By contrast, the majority invoked this reasoning at the highest possible level of abstraction, by holding that owners generally have no compensation rights following property deprivation in favour of state custodianship (even when this entails a right for the state to give the property to third parties).

The human flourishing theory clearly speaks in favour of the minority view. However, it also asks us to consider the possibility of going a step further, to challenge the very idea that state custodianship is the best way to address past injustices. Indeed, it would perhaps have been more appropriate to look for a solution consistent with the protection of private property, for instance by building on {\it Modderklip} to improve the legal position of non-owning property dependants and communities.

Ideally, it should be possible to pursue key economic and social values without massively increasing the power of the state and weakening the institution of property. At the same time, it should be possible to more effectively enforce social obligations on private property owners.\footnote{Indeed, enforcing such obligations will become much harder after a new class of owners are in place, chosen and approved of by an increasingly powerful state. }

Achieving this in practice requires mechanisms that enable negotiations between competing private property interests, to facilitate a balancing of those interests through participatory decision-making rather than state management. This highlight the importance of another property value emphasises by the human flourishing theory, namely that of participation, discussed in the following section.

\noo{ he Constitutional Court of South Africa essentially held that economic development takings are not takings at all, so that no compensation needs to be paid. 


It is interesting to note that this argument is based on a rather narrow understanding of what property is, contrasting with the broader view of property implicit in the South African constitution. 


The enemies of property, it seems, are first in line when it comes to adopting a traditional narrative that only recognises a narrow role for property institutions. In this, they are closely followed by neo-liberals who are similarly interested in a narrow notion of property that frees them from obligations and allows them to engage in exploitative practices. It is not a suprise therefore that the phenomenon of land grabbing is a thouroughly collaborative effort between statist national governments in Africa and 


 narrowing the understanding of property in South African law. 


suggestion that property should be strengthened to facilitate a reduction in the level of state interference might be hard to sell in practice. 


is too narrowly understood can be should be looked at more broadly can fuel those who wish to undermine property as an institution by stripping it of content.




can be weakened 



recast in terms that suit the powers that be, no


y in which property

 emphasised by the human flourishing idea, but it is also a potential weakness. 

There is a danger that this particular aspect of the human flourishing approach will not be successful, as property is 

 in practice, as it requires powerful state actors to g

given that the Constitutional Court upheld both the property rights of {\it Modderklip} and the housing rights of the squatters. Under a traditional property narrative, by contrast, one would expect the Court to perform a balancing test to give priority to one over the other. But the Court did not do so in {\it Modderklip}, preferring instead to focus on the fact that the state had provided inadequate access to an effective remedy for both the owners and the squatters. 

However, in the more recent case of {\it Agri South Africa v Minister for Minerals and Energy}, the Constitutional Court explicitly refused to regard ``state custodianship'' over mineral rights as expropriation. 

 a property limiting perspective -- whereby private property ireplaced by state control to solve social and economic problems -- appears to be favoured by the political leadership in South Africa. Since 2002, mineral and petroleum rights in South Africa have been put under ``state custodianship'', effectively authorising uncompensated state takings of mineral rights. 


However, the reconceptualisation in terms of property itself having a social function appears highly attractive. Moreover, it is also consistent with the South African constitution, which also focuses on property's social dimension.\footnote{See section 25 of the Constitution of the Republic of South Africa, Act 108 of 1996.}



This conclusion requires taking a normative stance, but a minimal one; we merely extend the scope of values traditionally attached to property.\footnote{Arguably, cases such as {\it Modderklip} might be taken to suggest that the social function theory, as soon as it is applied for the purposes of normative assessment, will systematically guide us to conclude that owners are not entitled to as many benefits as would otherwise follow from their property rights. It is fortunate, therefore, that the entire remainder of the thesis will focus on economic development takings, where it will typically appear more natural to conclude the opposite. In these cases, on a common- sense understanding of justice, applying the social function theory will allow us to \isr{recognise} a sense in which owners should receive {\it increased} protection and more benefits, as a consequence of how such interferences can prove particularly damaging, both to the owner and to the social fabric of democracy.} 

That said, in the case of {\it Modderklip} the court was clearly faced with a value conflict that it is hard to resolve by looking to traditional liberal values. If these apply equally to the squatters, we are left with deadlock rather than resolution. Indeed, this was also reflected in the outcome of the case, which did not resolve the matter, but merely concluded that the state had failed in its obligations towards both of the parties. What should the solution be in the end? Should the squatters be allowed to stay, following condemnation of Modderklip's land, or should alternative housing be provided so that the eviction order can be carried out? The answer depends on how we resolve a normative conflict, and how to do so might not be obvious. Moreover, value pluralism suggests that we must be prepared to engage with multiple ways of looking at the matter. In the interest of stability of property as an institution, allowing the squatters to succeed in establishing lasting title to the land might be considered unwise. Against this pragmatic and largely technical value, one would have to consider the values of community and belonging that now attach the squatters to their new homes. These two values are largely incommensurable, and it is not clear how to choose between them.

Still, Alexander maintains that human flourishing provides an ``objective'' standard on which to approach dilemmas such as these. Moreover, he ``rejects the view that what is good or valuable for a person is determined entirely by that person's own evaluation of the matter''.\footcite[1263]{alexander14} Some things are good for people, Alexander argues, \isr{irrespective} of whether or not people know so themselves. Hence, it may perhaps be argued that what is truly good for Modderklip is to come to an arrangement with the squatters and the state, to resolve the problem amicably. Moreover, failure to do so might entitle the state to take action that would otherwise seem to undermine the stability of property. This, then, would be partly due to this being conducive also to the flourishing of the people behind Modderklip, not only the squatters.

This might be derided as an overly intrusive and moralistic way to approach property law. More generally, as Alexander notes, the exact content of goodness is ``necessarily contestable''. It consists of a list of different values which are all open to dispute, both as to their relevance and their precise meaning.\footcite[1263]{alexander14} Alexander goes on to list some key values that he believes are central, but the list is not meant to be exhaustive.\footcite[1764-1776]{alexander14}




Among the key values that Alexander discusses, we find many core private values that are commonly seen as important goals for the institution of property. This includes values such as autonomy and self-determination, both of which will feature heavily later in this thesis. However, Alexander also considers several public values, such as equality, inclusiveness and community. These too will be important later, as I will draw on them in my own normative analysis of economic development takings. I will be particularly concerned with the value of {\it participation}, understood, following Alexander, in terms of its broad social function.\footcite[1275-1276]{alexander14}
}

\subsection{Property as an Anchor for Democracy}

The value of participation is closely related to the value of democracy. Participation in local decision-making processes is the root which enables democracy to come to fruition at the regional and national level. Of course, the role that property plays in this regard, as it empowers and encourages owners to take active part in the political process, has been noted before.\footnote{For a thorough assessment of the idea that property, for this reason, is the most fundamental right, I refer to \cite{rose96}.} Indeed, it has been important to property theorising ever since the early days of democracy. Alexander himself traces the emphasis on participation in politics back to Aristotle and the republican tradition. He notes, however, that this tradition involves a notion of participation that is rather narrowly drawn. For thinkers in the republican tradition, participation tends to mean public participation, meaning people's engagement with the formal affairs of the polity.\footcite[1275]{alexander14} For Alexander, participation has a broader meaning, involving also the value of being included in a community. He writes:

\begin{quote}
We can understand participation more broadly as an aspect of inclusion. In this sense participation means belonging or membership, in a robust respect. Whether or not one actively participates in the formal affairs of the polity, one nevertheless participates in the life of the community if one experiences a sense of belonging as a member of that community.\footcite[1275]{alexander14}
\end{quote}

Importantly, participation in a community can have a crucial influence also on people's preferences and desires. In this way, it is also invariably relevant -- behind the scenes -- to any assessment of property that focuses on welfare, utility or public participation in the classical sense. As Alexander and Pe\~{n}alver put it, drawing on the work of Amartya Sen and Martha Nussbaum:\footcite{sen84,sen85,sen99,nussbaum00,nussbaum02}
\begin{quote}
The communities in which we find ourselves play crucial roles in the formation of our preferences, the extent of our expectations and the scope of our aspirations.\footcite[140]{alexander09}
\end{quote}
Therefore, for anyone adhering to welfarism, rational choice theory, utilitarianism or the like, neglecting the importance of community is not only normatively undesirable, it is also unjustified in an epistemic sense. In particular, it should be \isr{recognised} as a descriptive fact that community is highly relevant to {\it any} normative theory that attempts to take into account the preferences and desires of individuals.\footnote{Again, I think Alexander and other theorists attempting to incorporate such ideas in property law could benefit from making this descriptive point separately, so as to enable it to be considered in isolation from the more contentious normative arguments they construct on its basis.} But Alexander and Pe\~{n}alver go further, by arguing that participation in a community should also be seen as an independent, irreducibly social, value, not merely as a determinant of individual preferences and a precondition for rational choice. They write:

\begin{quote}
Beyond nurturing the individual capabilities necessary for flourishing, communities of all varieties serve another, equally important function. Community is necessary to create and foster a certain sort of society, one that is characterized above all by just social relations within it. By ``just social relations'', we mean a society in which individuals can interact with each other in a manner consistent with norms of equality, dignity, respect, and justice as well as freedom and autonomy. Communities foster just relations with societies by shaping social norms, not simply individual interests.\footcite[140]{alexander09}
\end{quote}

This, I believe, is a crucial aspect of participation. Moreover, it is a notion that invariably leads us to recognise that other property dependants should also have a voice, as they form part of the ``just social relations'' within the community to which the owners belong. In addition, this is a notion of participation that it is hard, if at all possible, to incorporate in theories that take preferences and other attributes of individuals as the basis upon which to reason about their legal status. Instead, the human flourishing perspective asks us to consider how property serves as an anchor for participation that shapes and influences preferences and the norms that guide them, possibly in ways that should be protected even if this seems to contradict the stated interests of owners.

For instance, if people in a community come under pressure to sell their homes to a large commercial company that wishes to construct a shopping mall, it may be appropriate to consider this as an unjustifiable attack on their property rights. Importantly, this may be so \isr{{\it irrespective}} of what the individual owners themselves think they should do. If they are offered generous financial compensations for their homes, or if they are threatened by eminent domain, economic incentives might trump the value of social inclusion and participation for all or a majority of these owners. As a consequence, the community might decide to sell.  

Even so, in light of the value of community, it would be in order for planning authorities, maybe even the judiciary, to view such an agreement as an {\it attack on their property}. It is clear that by the sale of the land, the ``just social relations'' inhering in the community will be destroyed. The members of the community -- including all the non-owners -- will lose their ability to participate in those relations. The property rights that once contributed to sustaining just relations will now be transformed into property rights that serve different purposes. This includes aiding the concentration of power and wealth in the hands of commercially powerful actors. Such a change in the social function of property might have to be regarded -- objectively speaking -- as a threat to participation, community and democracy. Hence, on the human flourishing theory, it is also a threat to property. Property institutions, therefore, should protect against it.

In Norway, a range of such rules are in place to protect agricultural property, by limiting the owners' right to sell parcels of their land without local government consent, as well as by compelling them to reside on their property and to make use of it for agricultural production.\footnote{See \cite[8|12]{la95} and \cite[4|5]{lca03}.} 

When the law actively promotes egalitarian property in this way, the natural counterpart is to limit direct state interference. The danger otherwise is that the limited economic strength of each individual property owner -- appropriate in a democracy of property owners -- is exploited by the state and others to bestow special benefits on select groups.

The broader issue at stake here is again highlighted by recent developments in South Africa, where  rules closely resembling those found for agricultural property in Norway have been proposed in a recent act on land reform. In South Africa, however, these rules have been proposed alongside a new framework of ``state custodianship'' of agricultural land, corresponding to the formulation used in the mineral and petroleum legislation, as discussed in the previous section.\footnote{See \cite{steyn15}.}

If the proposal passes, the proper functioning of agricultural property in South Africa will seem to depend very strongly on the benevolence of the state, which will greatly increase its own power to interfere with private property. This, one worries, contradicts the aim of creating a property regime that is truly democracy-enhancing.

The human flourishing perspective suggests that even when provisions to promote egalitarian ownership and community commitment are appropriate, provisions that inflate the state's authority might not be. As I believe the history of democracy in Norway shows, strict property rules to protect and promote self-governing agrarian communities can work well, as long as they are applied consistently and coupled with strong institutions of local democracy and strict limits on state power.\footnote{I discuss the role of agrarian property to the development of Norwegian democracy in more depth in Chapter \ref{ref:chap3}.}

\noo{ \subsection{The Duty to Contribute}

There is a subtle issue that arises on the basis of normative reasoning about individual rights as a source of human flourishing. Is it appropriate to think of such reasoning -- and the obligations it gives rise to -- as an aspect of protecting individual rights? Is it not more accurate to say that this kind of reasoning asks for interference with individual rights, undertaken to make sure that those rights contribute towards fulfilment of public and community interests? 

Indeed, it might seem peculiar to insist that enforcing stricter social obligations for owners is an aspect of protecting property rights. However, we might still be talking about protecting individual rights, even when this means imposing protections on people that they themselves do not want. Undoubtedly, this is {\it also} an interference in their rights, but just as different rights of different people can sometimes come into conflict, it seems that aspects of the same right, for the same person, can sometimes come into conflict with itself. This happens, in particular, when it is not possible to simultaneously protect all those functions that this right seeks to promote. 

For instance, if someone protests a taking on environmental grounds and also rejects financial compensation as immoral, the courts should still award just compensation for the land, if they find that the taking is valid. If the owner wishes, they can purge themselves by making a donation to charity. For an example from a completely different area of the law, if someone attempts to commit suicide, the health services are still obliged to help, even against the \isr{patient's} wishes. This remains the case, moreover, even though suicide is no longer considered a criminal offence. Moreover, no one would hesitate to consider this as a service to the person who is thereby prevented from taking their own life.

A similar kind of reasoning can be applied to rules stressing the social responsibilities and participatory duties of property owners. This serves owners not only by providing a bulwark against the pernicious tendency to conflate social responsibility with state interference. It also creates a narrative that can empower owners more generally, not as holders of entitlements, but as members of a society.

A proclamation to the effect that the ``public interest'' necessitates negating property rights is clearly very \isr{marginalising} to owners. They are then an obstacle to progress, not a resource that can be used to bring it about. A balancing act might be required to determine if interference is truly justified, but the individual is now relevant only to one side of the equation, the side of private harms. If, on the other hand, the act of interference itself can be rendered as a form of protection, enforcement of an obligation, or a measure to enable participation, the individual occupies center stage. The owner is now also a part of the public good that is to be achieved through measures involving their property.

On this narrative, if public interests lead to an imposition, it is not because individual rights are negated, but because the public is deemed to know best how to secure the goal of human flourishing, for owners and their society. To take an obvious example: both private and a public interests suggest that the law should discourage owners from becoming a nuisance to their \isr{neighbours}. Indeed, most jurisdictions have rules to this effect. Now, under a human flourishing theory, we are entitled to highlight how such rules protect the institution of property, also by protecting the owners' membership in their community. The public does not ``side with the \isr{neighbours}'' when it compels someone to behave with more care. Rather, the public undertakes measures to protect ``just social relations'' among owners and other community members.

For a second example, consider situations when environmental concerns suggest imposing restrictions on what an owner is permitted to do with their land. This too can be rendered as an act of protecting property. But doing so requires the regulatory body to relate the interference positively to the owner's interests and obligations, to emphasise that this in effect extracts from the owner a positive contribution to society.

Such a narrative does not prevent public values and the public interest from being given considerable weight. However, it compels us to be more concrete about how exactly state interference serves to enhance the social functions of the rights interfered with. The baseline for assessment, therefore, become actual persons and their well-being, not some abstract ideal of ``goodness''. Moreover, imposing the collective will on individuals becomes a means to ensure human flourishing for all, not an abstract goal in itself.

The upshot is that on the human flourishing account, if property interference is unavoidable, it is still important to interfere in a way that constructively targets the individual, aiming to protect them by enabling them -- and compelling them -- to protect others and partake in social and political life. This can then become interference aimed at bringing the individual into the fold, \isr{making them play their part, by raising them to fruitful citizenship.} This might be the vision of a paternal state, one that runs the risk of becoming overprotective, unfair, or just plain stupid. However, it is 
not the vision of a cold and alienating state of experts and elites that govern form afar.

Hence, even if the paternal state engenders resistance against interference, those who resist are also likely to carry forward care and affection for the social, political and legal structures within which this agency is (hopefully) permitted to take place. To \isr{conceptualise} an act of restriction as a means to empower the persons restricted is something they might find offensive, but it also renders interference more meaningful. It provides both a reason to take a more active role in relation to the interfering power, and a possible cause for constructive resistance. 

Importantly, it does not force the conclusion that the public resides behind closed doors, disinterested in what the affected individual have to offer. Instead, it is an approach that encourages a response, by focusing always on the persons interfered with, whenever interference is deemed necessary. This is the vision of a bottom-up, rather than a top-down, approach to imposing the collective will on individuals. I believe it has merit, and 
}

This raises the question of what kind of institutions we need to enable local communities and owners to  flourish and make informed decisions about how to use their properties. In the final chapter of the thesis, I discuss this concretely in the context of economic development situations, by looking to the Norwegian institution of land consolidation.

In the next section, I zoom in on economic development takings. First, I introduce such takings by considering the seminal case of {\it Kelo v City of New London}\footcite{kelo05}, which brought this category to prominence in the US discourse on property law. Then I will assess the unique aspects of such takings against the social function theory, to provide an argument that the category has significance for legal reasoning in takings law, as well as with respect to property as a constitutionally protected human right. Finally, I will provide an abstract presentation of the values that I believe are important when normatively assessing the law in this area. In doing so, I will draw on the human flourishing theory, setting out the main values that will inform the concrete policy assessments I provide later. 

\section{Economic Development Takings}\label{sec:edt}

Constitutional property rules in many jurisdictions indicate, with varying degrees of clarity, that eminent domain should only be used to take property either for ``public use'', in the ``public interest'', or for a ``public purpose''. Such a restriction can be regarded as an unwritten rule of constitutional law, as in the UK, or it can be explicitly stated, as in the basic law of Germany.\footnote{See Chapter \ref{chap:2}. Section \ref{sec:contrast} below.} In some jurisdictions, for instance in the US and in Norway, explicit property clauses exist, but do not provide much information about the intended scope of protection.\footnote{See Chapter \ref{chap:2}, Section \ref{sec:us} and Chapter \ref{chap:4}, Section \ref{sec:explaw}.}

Both the Norwegian and the US property clauses appear to refer to public use only as a precondition for the duty to pay compensation. However, this is universally understood as expressing the {\it presupposition} that the power of eminent domain is only to be used in the public interest.\footnote{In the literature, it is rare to even note that a different interpretation is linguistically possible. But see \cite[205]{berger78}.} Indeed, in cases when one might say that private property is ``taken'' for a non-public use without compensation, for instance in a divorce settlement, it is not commonly regarded as an exercise of eminent domain. Rather, it is justified by reference to a different category of rules, meant to ensure enforcement of obligations that arise between private parties independently of the state's power to single out and compulsorily acquire specific properties.

The exact boundary between eminent domain and other forms of state interference in property is not always clear, but I will not worry too much about it in this thesis. I note, moreover, that most legal scholars seem to agree that the power of eminent domain is meant to be exercised in the public interest. However, differences of opinion emerge when we turn to the question of whether the presupposed public use or public interest serves also to restrict the power to take. In the US, most scholars agree that some restriction is intended, but there is great disagreement about its extent.\footcite[205]{berger78} In Norway, on the other hand, a consensus has developed whereby the notion of public use is interpreted so widely that it hardly amounts to a restriction at all.\footnote{See, e.g., \cite[368]{aall10}.} Moreover, the courts defer almost completely to the assessments made by the executive branch regarding the purposes that may be used to justify a taking.\footcite[368]{aall10}

Some US scholars adopt a similar stance, but others argue that the public use presupposition should be read as a strict requirement, forbidding the use of eminent domain unless the public will make actual use of the property that is taken.\footnote{Compare \cite{bell06,bell09,claeys04,sandefur06}.} Most scholars fall in between these two extremes. They regard the public use restriction as an important limitation, but they also \isr{emphasise} that courts should normally defer to the legislature's assessment of what counts as a public use.\footnote{See, e.g., \cite{merrill86,alexander05}. The fact that US jurists usually stress deference to the legislature, not the executive branch, should be noted as a further contrast with Norway.}

As I discuss in more depth in Chapter \ref{chap:2}, Section \ref{sec:hop}, the debate in the US has its roots in case law developed by state courts -- the federal property clause was for a long time not applied to state takings. This has changed, and today the Supreme Court has a leading role in this area of US law. It has developed a largely deferential doctrine, resembling the understanding of the public use limitation under Norwegian law.\footnote{See \cite{berman54,midkiff84,kelo05}.} The difference is that in the US, cases raising the issue still regularly arise and prove controversial. The most important such case in recent times was {\it Kelo}, decided by the Supreme Court in 2005.\footcite{kelo05} This case saw the public use question reach new heights of controversy in the US.\footnote{See, e.g., \cite{somin09}.}

\subsection{{\it Kelo}}

{\it Kelo} centred on the legitimacy of taking property to implement a redevelopment plan that involved construction of research facilities for the drug company Pfizer. The home of Suzanne Kelo stood in the way of this plan and the city decided to use the power of eminent domain to condemn it. Kelo protested, arguing that making room for a private research facility was not a permissible ``public use''. She was represented by the libertarian legal firm {\it Institute for Justice}, which had previously succeeded in overturning similar instances of eminent domain at the state level.\footnote{See \url{https://www.ij.org/cases/privateproperty}.} Kelo lost the case before the state courts, but the Supreme Court decided to hear it and assessed its merits in great detail.

The precedent set by earlier federal cases was clear: As long as the decision to condemn was ``rationally related to a conceivable public purpose'', it was to be regarded as consistent with the public use restriction.\footcite[241]{midkiff84} Moreover, the role of the judiciary in determining whether a taking was for a public purpose was regarded as ``extremely narrow''.\footcite[32]{berman54} It had even been held that deference to the legislature's public use determination was required ``unless the use be palpably without reasonable foundation'' or involved an ``impossibility''.\footnote{See \cite[66]{dominion25}; \cite[680]{gettysburg96}.}

This understanding was also reflected in the outcome of related cases. In {\it Hawaii}, the Supreme Court had upheld a taking that would benefit private parties, with no direct benefit to the public.\footnote{\cite{midkiff84}. For a more detailed discussion, see Chapter \ref{chap:2}, Section \ref{sec:hop} below.} In {\it Berman}, it had upheld a taking for economic redevelopment of a blighted area, even though the property taken was not itself blighted.\footnote{\cite{berman54}. For a more detailed discussion, see Chapter \ref{chap:2}, Section \ref{sec:hop}.} But in the case of {\it Kelo}, the court hesitated.

Part of the reason was no doubt that takings similar to {\it Kelo} had been heavily criticised at state level, with an impression taking hold across the US that eminent domain ``abuse'' was becoming a real problem.\footnote{See, e.g., \cite[667-669]{sandefur05}.} A symbolic case that had contributed to this worry was the infamous \textcite{poletown81}. In this case, General Motors had been allowed to raze a town to build a car factory, a decision that provoked outrage across the political spectrum.\footnote{See generally \cite{sandefur05}.} The case was similar to {\it Kelo} in that the taker was a powerful commercial actor who wanted to take homes. This, in particular, served to set the case apart from {\it Hawaii}, which involved a taking in \isr{favour} of tenants, and to some extent also {\it Berman}, which involved a taking of businesses (and homes) in the interest of removing blight. Moreover, the Michigan Supreme Court had recently decided to overturn {\it Poletown} in the case of \textcite{wayne04}. Hence, it seemed that the time had come for the Supreme Court to reexamine the public use questions.\footnote{See, e.g., \cite{sandefur05,claeys04}.}

Eventually, in a 5-4 vote, the court decided to apply existing precedent and held against Suzanne Kelo. The majority also made clear that economic development takings were indeed permitted under the public use restriction, also when the public benefit was indirect and a private company would benefit commercially.\footcite[469-470]{kelo05} The backlash of this decision was severe. According to Ilya Somin, the case ranks among the most disliked decision that the Court has ever made.\footcite[2]{somin11} Some 80 - 90 \% of the US public expressed great disapproval, with critical voices coming from across the political spectrum.\footcite[2108-2110]{somin09} Why did the case prove so controversial? No doubt, the discontent with the decision was \isr{fuelled} in large part by the fact that it was seen as a case of the government siding with the rich and powerful, against ordinary people.\footnote{\cite[630-634]{baron07}} Indeed, the party that appeared to benefit the most from the taking was Pfizer -- a multi-billion dollar company -- while Suzanne Kelo, who stood to lose, was a middle class homeowner. In this context, the taking of Kelo's home seemed morally suspect, an act of \isr{favouritism} showing disregard for less influential members of society.\footnote{See, e.g., \cite{underkuffler06}.}

In addition, it is worth noting that many commentators conceptualised the {\it Kelo} case by thinking of it as belonging to a special category, by describing it as an economic development taking, a {\it taking for profit}, or, more bluntly, a case of {\it Robin Hood in reverse}.\footcite{somin05} Categories such as these had no clear basis in the property discourse before {\it Kelo}. Indeed, in terms of established legal doctrine, it would be more appropriate to say that the case revolved entirely around the notion of ``public use''. 

However, when considering the most common reasons given for condemning the outcome in {\it Kelo}, it becomes clear that many critics felt it was natural to classify the case along additional dimensions. A survey of the literature shows that many made use of a combination of substantive and procedural arguments to paint a bleak picture of the {\it context} surrounding the decision to take Kelo's home. Important aspects of this include the imbalance of power between the commercial company and the owner, the incommensurable nature of the opposing interests, the lack of regard for the owner displayed by the decision makers, the close relationship between the company and the government, and the feeling that the public benefit -- while perhaps not insignificant -- was made conditional on, and rendered subservient to, the commercial benefit that would be bestowed on a commercial beneficiary.\footnote{See, for instance, \cite{underkuffler06,somin07,sandefur06,cohen06,hafetz09,hudson10}.}  This dynamic, in which public bodies no longer seem to be leading and pushing the process forward, but are rather being led and being pushed, is regarded as particularly suspicious. This, in turn, is typically derided as a perversion of legitimate decision-making, used to argue that economic development takings such as {\it Kelo} suffer from what I will refer to here as a {\it democratic deficit}.

\noo{ It is noteworthy that in economic development situations, this broader worry comes to occupy center stage in a dispute over the legitimacy of interfering in individual property rights. On the traditional entitlements-based account of property, this is a turn of attention that it can be hard to make sense of. However, it fits nicely with the social function theory of property and the emphasis on participation in decision-making espoused by the human flourishing account of property's ends.}

From a theoretical point of view, I take all of this to suggest that many critics of {\it Kelo} effectively adopted a social function view on property, by paying close attention to the wider social and political context of the taking.\footnote{For a particularly clear example of this, see \cite{underkuffler06}.} Importantly, if we now turn to the social function theory of property, we are placed in a position to engage more actively with this form of reasoning, as an integrated part of our assessment of the law. This may then in turn give us cues as to how we should reason to justify a departure from the course laid down by previous cases on the ``public use'' requirement, where such a perspective was not adopted. Indeed, it seems to me that this is exactly what the minority of the Supreme Court did, particularly Justice O'Connor, who formulated a strongly worded dissent.\footnote{\cite[494-505]{kelo05}.}

\subsection{Justice O'Connor as a Social Function Theorist}

Justice O'Connor was joined by all the four other dissenters, but Justice Thomas also formulated his own dissent, taking a more narrow view and arguing for the revival of a strict reading of the public use requirement.\footnote{\cite[505-523]{kelo05}.} As such, Justice Thomas' dissent fits better with a traditional property narrative, while also making it less relevant outside the context of US law. Justice O'Connor, by contrast, made a broad assessment of the social and political consequences of allowing takings in cases like {\it Kelo}, an assessment that seems to be of general relevance to any jurisdiction where commercial interests benefit from the power of eminent domain. Her analysis culminates in the following warning:

\begin{quote}
Any property may now be taken for the benefit of another private party, but the fallout from this decision will not be random. The beneficiaries are likely to be those citizens with disproportionate influence and power in the political process, including large corporations and development firms. As for the victims, the government now has license to transfer property from those with fewer resources to those with more. The Founders cannot have intended this perverse result.\footcite[505]{kelo05}
\end{quote}

The values Justice O'Connor relies on here are closely related to the values associated with the notion of human flourishing, particularly those relating to the political function of property as an anchor for community and democracy. Importantly, cases like {\it Kelo} not only appear to threaten individual entitlements of owners. They also appear to threaten equality in civic society, as the economic rationality used to justify interference results in an implicit political statement to the effect that the property of the rich and powerful is better protected, and valued higher by the state, than property owned by regular citizens, who reside in ordinary communities.

The effect of a traditional economic development taking is that property rights are transferred from the many to the few, taken from ordinary people and given to the powerful. Hence, these cases represent a possibly pernicious redistribution of property, particularly in terms of property's social function. The structural imbalances of the condemnation process itself find permanent expression in the new distribution of property. The social structures of a living community are dismantled in favour of a social structure that revolves around the commercial interest of designated companies that enjoy the support of government. The political and social power of the community is diminished, perhaps lost in its entirety, while the political and social power of designated companies increase.

It seems clear that to Justice O'Connor, this too had to be recognised as a negative consequence of the taking in {\it Kelo}. Again, I stress that recognising such effects appear to require a social function approach to property. There is no clearly quantifiable individual loss -- no  particular ``stick'' in the property bundle that is not compensated. Rather, it is the community itself that is lost, a community that was not directly implicated in any formal ``entitlement'', but which played a crucial role in providing meaning to the totality of the bundle of rights and obligations enjoyed by the owners. 

Even if we extend our perspective to account for indirect individual losses, we are not doing justice to such losses. The owner might relocate, acquire new property with a similar meaning in a new community somewhere else. But that does not make up for the fact that {\it this} community is lost forever, as {\it this} property takes on new meanings and functions. The loss to Suzanne Kelo, therefore, might even be a significant loss to the City of New London, whose democracy suffers as a result of the taking.

Of course, the economic and social gains of development might outweigh such negative effects. In any event, it seems that the balancing of interests required in this regard should be carried out by an institution that sufficiently \isr{recognises} the owners' right to participation and self-governance. The presence of a highly active commercial third party, in particular, means that public participation in the standard sense might be insufficient. In economic development takings, the commercial company typically appears alongside the government, as a more or less integrated part of the institutional structure making the decision to condemn. The owners, however, do not enjoy a corresponding level of participation.

Specifically, their interests are only negatively defined. They are adversely effected and may object, but under standard administrative regimes they play no constructive role in the process. For instance, they are not called on to take part in the development itself, or to assess its merits more broadly than by being asked to respond based on their own individual entitlements. This might be one of the main problems with economic development takings, resulting in a democratic deficit. I will argue for this in more depth later, but I remark here that an important reason to focus on this aspect is that it involves precisely those values that economic development takings are most likely to threaten. Moreover, if the loss of community outweighs the positive effect of economic development, this is unlikely to be recognised following a process that relies primarily on the contribution of the developer and the expert planners.\footnote{A similar point is made in \cite{underkuffler06}.} 

The objections made by owners may not only be given too little weight. It is also possible that owners themselves unduly focus only on the individual loss. They might not even consider those issues that are most important for property's social function. To address this concern, I do not think it is sufficient to theoretically proclaim that social function aspects need to be considered. Such aspects are likely to already form part of the package of interests that expert planners are supposed to take into account. However, to address the democratic deficit of economic development takings, it seems that institutional reforms might be in order, to give owners and their community a more significant voice in the decision-making process. This is a call for greater involvement by the local community (including, perhaps, even non-owners) in the decision-making process relating to development. It is not sufficient to merely ``consult'' local communities by asking if they have objections. It is also necessary to include communities in a constructive way, perhaps even be compelled to assume an active role in relation to the proposed project.

This is a proposal that envisages owners engaging directly with both government and potential developers, by considering alternative schemes, and making their own proposals. In short, this asks for a system where owners and communities are co-authors of the government's plans for their land.  According to the human flourishing theory as I understand it, acting as such a co-author is not only an owners' right, but also their obligation. 

This perspective gives a plausible basis on which to strike down certain kinds of economic development takings. Moreover, it allows us to do so without giving up the value of judicial deference, since it focuses on the democratic deficit rather than the exact meaning given to the notion of ``public use''. In addition, it is a call for institutional reform, to search for new governance frameworks that will empower owners and their communities.

It seems to me that Justice O'Connor's argument reflects some of the ideas I have sketched here. Indeed, O'Connor seems to believe that the taking of Kelo's home would be a particularly harmful interference in ``just social structures'', to quote Alexander. Importantly, a piece-by-piece entitlement-based approach to {\it Kelo} could hardly justify the degree of disapproval seen in Justice O'Connor's opinion. After all, Kelo had been offered generous compensation, there had been no clear breach of concrete procedural rules, and the claim that the taking was {\it only} a pretext to bestow a benefit on Pfizer did not seem supported by the facts.\footnote{See \cite{bell06}.} Rather, it was the overall character of the taking that could be used to argue that it was illegitimate. In this argument, moreover, the perceived lack of a clearly identifiable and direct public benefit was only one of several key points.

In addition, the institutional, social and political aspects of the case was considered in depth. By contrast, the economic implications appear to have been less important to Justice O'Connor. Even the importance of home ownership to personhood does not receive the same attention as structural aspects pertaining to good governance. The problem which overshadows everything else is the concern that economic development takings represent a form of governmental interference in property that might systematically \isr{favour} the rich and powerful to the detriment of the less resourceful. Hence, such takings may help establish and sustain patterns of inequality. Hardly anyone would openly regard this as desirable. It is not hard to agree that if Justice O'Connor's predictions about the fallout of {\it Kelo} are correct, then it is indeed ``perverse''. 

The question, of course, is whether her predictions are warranted. This is a call for empirical and contextual assessment of economic development takings, to help us gain a better understanding of how they actual affect political, social and bureaucratic processes. In addition, it raises the question of how to {\it avoid} negative effects, that is, how to design rules and procedures that can reduce the democratic deficit of economic development takings. As I now move away from theory towards concrete assessment of economic development takings, both these questions will be in focus.

\section{Conclusion}\label{sec:conc1}

In this chapter, I have presented the core notion of my thesis, namely that of an economic development taking. I started by noting that while the notion is straightforward to define factually, it is far from obvious what implications it has for legal reasoning. I illustrated the subtleties involved by considering a concrete example of a commercial scheme that looked like it might well result in compulsory acquisition of land, namely Donald Trump's controversial plans to develop a golf course on a site of special scientific interest close to Aberdeen, Scotland. In the end, the plans did {\it not} require takings, as Trump was able to make creative use of property rights he acquired voluntarily, against the complaints of recalcitrant \isr{neighbours}.

This turn of events made the example even more relevant to the points I have been trying to make in this chapter. It served to highlight, in particular, that the question studied in this thesis is not a black-and-white balancing act between property privileges on one side and the good will of the regulatory state on the other. Rather, the example of Trump's golf course allowed me to \isr{emphasise} the importance of context when assessing both the nature of property rights and the meaning of protecting them. In particular, to protect the property rights of those opposing Trump's golf course was not about protecting just any property, it was about protecting the property of members in a local community that felt it would be detrimental to this community, and to their lives, if Trump was allowed to redefine it. In particular, after Trump decided not to pursue compulsory purchase, protecting the property of these members of the community became a question of {\it restricting} the degree of dominion that Trump could exercise over his own property. Hence, under a conventional and overly simplistic way of looking at these matters, protecting property then became tantamount to restricting its use, a seeming paradox.

To resolve this paradox, and to arrive at a better conceptual understanding of economic development takings, I looked to various theories of property. I noted that there are differences between civil law and common law theorising about property, but I concluded that these differences are not particularly relevant to the questions studied in this thesis. In particular, I observed that neither the bundle theory, dominant in the common law world, nor the dominion theory, taught to many civil law jurists, helped me clarify economic development takings as a category of legal thought.

I then went on to consider more sophisticated accounts of property, focusing on the social function theory, which emphasises how property structures, and is structured by, social and political relations within a society. I went on to argue that in the first instance, the social function theory should be understood as giving us {\it descriptive} insights into the workings of property and its role in the legal order. In this regard, I advanced a different stance than many property scholars, by arguing that we should aim to decouple descriptive insights from normative aspects of the theory, to allow the social function theory to serve as a common ground for further value-driven debate.

I then went on to clarify my own starting point for engaging in such debate, by expressing support for the human flourishing theory proposed by Alexander and Pe\~{n}alver. This theory is based on the premise that property {\it should} enable -- and even compel -- individuals and their communities to  participate in social and political processes. I argued that property's purpose in this regard is  fundamental to its proper role in a democratic society, as an anchor for participatory decision-making.  

Moreover, I noted that the human flourishing theory contains a further important insight, concerning the scope of the state's power to protect. In particular, the theory asks us to recognise that protecting property against interference that is harmful to human flourishing is a responsibility that the state has even in cases when the individual owners themselves neglect to defend their property, for instance as a result of financial incentives to remain idle. In other words, some functions of property are such that owners have an obligation to preserve them, while the state has a duty to protect them, potentially even against the will of the owners.

After this, I went on to consider economic development takings specifically, by drawing on the theoretical insights collected from preceding sections. To make the discussion concrete, I considered the case of {\it Kelo}, which propelled the notion of an economic development taking to the front of the takings debate in the US. I focused particularly on the dissenting opinion of Justice O'Connor, and I argued that she approached the issue in a way that is consistent with the theoretical basis proposed in this chapter.

I will now go on to make my analysis of economic development takings more concrete, by considering how such takings are dealt with in Europe and the US respectively. I note that the category has yet to receive much attention in Europe, so the discussion focuses on the US. Here this issue has received a staggering level of attention after {\it Kelo}. To get a broader basis upon which to \isr{assess} all the various arguments that have been presented, I consider the historical background to the issue as it is discussed in the US. This involves giving a detailed presentation of the public use restriction, as it was developed in case law from the states during in the 19th and early 20th century. I then connect this discussion with recent proposals to deal with economic development takings, responding to the backlash of {\it Kelo} by aiming to address the democratic deficit of such takings.

Later, when I begin to consider the law relating to Norwegian hydropower, I will look back at the theoretical basis provided in the present chapter to guide the analysis. In particular, I focus on certain decision-making mechanisms that have developed on the ground in Norway, as a practical response to the increased tendency for local owners to engage in hydropower development. I will argue that this shows the conceptual strength of the idea that property is irreducibly embedded in community, continuously evolving alongside institutions of participatory decision-making.
%\newcommand{\isr}[1]{{#1}}

\chapter{Taking Property for Profit}\label{chap:2}

\section{Introduction}\label{sec:intro}

In the previous chapter, I argued that economic development takings are a separate category of interference with private property. I also placed such takings in the theoretical landscape, by relating them to the social function theory of property. In particular, I argued that economic development takings raise questions that require us to depart from the individualistic, entitlements-based narrative that otherwise dominates in property theory.

This chapter develops this idea further, by considering how economic development cases are dealt with in England, where such takings have yet to be widely recognised as a separate category, and the US, where they first began to attract special attention. In addition, the chapter considers case law from the ECtHR and asks what it tells us about how to approach economic development takings under European human rights law.\footnote{So far, the issue of economic development takings have been brought into focus at the Court in Strasbourg.}
Finally, the chapter considers recent proposals for reform that focus on how to increase legitimacy by developing new institutions for self-governance to replace the traditional takings procedure in economic development cases.

I begin in Section \ref{sec:lgppp} by commenting briefly on the importance of economic development takings on the global stage. Specifically, I note that the core issues raised by such takings appear relevant also in the context of developing economies, even when property rights as such are an unstable basis on which to reason about the rights and obligations of individuals and communities. Specifically, I propose that the social function theory might offer a conceptual bridge between the study of economic development takings and the study of {\it land grabbings}, large-scale land acquisitions in the developing world. In both cases, the worry is often that local communities, who might lack formal title to the land, suffer as a result of a dramatic change in property's social function.

In Section \ref{sec:contrast}, I move on to consider the status of economic development takings in  English law. This also serves to introduce the topic of my thesis from the point of view of an important jurisdiction in Europe, where the issue of economic development takings has attracted far less attention than in the US. It appears to be gaining importance, however, as public-private partnerships and a market-oriented approach to public services has become influential in many jurisdictions, including in England.

In Section \ref{sec:echr}, I elaborate on a practically significant pan-European property clause, namely Article 1 of Protocol 1 (P1(1)) of the European Convention of Human Rights (ECHR). I argue that this clause provides an interesting perspective on the legitimacy issue, asking us to focus on the proportionality of the interference, judged relatively to its social and political context. I also consider some possible objections against the human rights approach, including the worry that the court in Strasbourg is not well-placed to be the arbiter of social and individual justice throughout Europe. At the same time, I point to some recent decisions at the Court that I believe signal hope that the case law on property is moving away from ill-conceived ``micro-management'', towards a more open-ended jurisprudence that seeks to force member states to address systemic problems that they might otherwise be reluctant or incapable of raising to the national agenda. Here the involvement of a (hopefully) politically neutral institution like the ECtHR can serve an important purpose, particularly if it manages to tailor its own case law in such a way as to leave room for local institutions of the member states to work out for themselves how to concretely resolve human rights issues flagged by the Court in Strasbourg.

In Section \ref{sec:us}, I return to the US setting, by presenting in detail how the perspective on economic development takings, mediated through case law on the public use restriction, has evolved since the 19th century until today. I structure the presentation as a story in two parts, describing the situation before and after the {\it Kelo} case. For the pre-{\it Kelo} presentation, I begin by pointing out that the case law on the public use restriction was initially developed by state courts, who would adjudicate legitimacy cases against the respective state constitutions (which typically also contain some sort of public use restriction on the takings power). 

The Supreme Court adopted deference to state {\it courts} initially, before changing their perspective by adopting a policy of deference directed rather at the state {\it legislature} (in practice also the administrative branch). I argue that this shift in Supreme Court jurisprudence can be pin-pointed to the case of {\it Berman}.\footnote{See \cite{berman54}.}

I go on to argue that this shift in case law at the federal level had the effect of destabilizing the established state approach to economic development takings, resulting in increased tension and controversy, paving the way to {\it Kelo}. In essence, my argument is that the Supreme Court was right in taking a deferential stance with respect to local institutions, but wrong in stripping the public use restriction of content, a move that threatened to undermine the authority of state courts. In effect, the federal takings jurisprudence threatened to weaken a very sensible {\it local} judicial constraint on executive power, a constraint that was also important to the proper division of power at the state level.

In Section \ref{sec:postkelo}, I follow this up by a discussion of developments after {\it Kelo}, which has seen a resurgence in state court scrutiny of the public use requirement, often backed up by state legislation that explicitly seeks to limit the scope of takings for economic development. According to some, such state reforms have been largely ineffective. %Ilya Somin, one of the most prolific writers on economic development takings in the US, has argued that this is partly due to so-called ``rational ignorance'' of political decision-makers and voters regarding the subtleties of the public use issue. The idea is that the distance between policy makers and communities affected by economic development takings is too great, so that policy makers have no incentive to consider the finer details of the takings equation. 
In principle, the US public is almost unanimously on the side of the local communities in cases like {\it Kelo}, but in practice, the great distance between political cause and effect makes effective reform policies hard to formulate. The danger is that reform proposals come to rely on oversimplified narratives tailored to centralised processes of decision-making.

In Section \ref{sec:ir}, I consider two recent suggestions that I regard as possible answers to this concern.\footnote{See \cite{lehavi07,heller08}.} These suggestions focus on the need for new frameworks for collective action, institutions that can replace the top-down dynamics of eminent domain in cases of economic development. The goal is to ensure a greater level of self-governance for the communities directly affected by the development, the individual members of which have a rational incentive to invest time and effort in reaching sophisticated compromises that can replace the use of black-white solutions (be it in the form of an economic development taking or a politically sanctioned top-down {\it ban} on such takings).

I argue that this idea embodies both a natural and necessary counterpart to increased judicial scrutiny of the public use restriction. In particular, I argue that the two ideas are mutually conducive to each other, when properly conceived. This argument will set the stage for the case study in the second part of the thesis, where I explore the tension between takings and self-governance in the context of hydropower development in Norway.

\section{The ``Underscrutinised'' Language of Economic Development}\label{sec:lgppp}

Economic development takings can be seen as a form of public-private partnership, whereby the state seeks to rely on for-profit takers and the market to fulfil some public purpose. Public-private partnerships are becoming increasingly important to the world economic order.\footnote{See generally \cite{saussier13}.} To some, they are the illegitimate children of privatisation and deregulation, while others see them as efforts to make the public sector more efficient and accountable. Either way, public-private partnerships are becoming more important, and they appear to be here to stay.\footnote{Although their potentially pernicious effects on stability and accountability has also been noted. See, e.g., \cite{baker03} (arguing that ``the Enron scandal can be better understood as an American form of public private partnership rather than just another example of capitalism run amok'').} In this situation, it is inevitable that when eminent domain is used to acquire property for economic development, those who directly benefit will often be commercial companies rather than public bodies. In the previous chapter, I pointed out how indirect public benefits are typically used to justify such takings. Standard legitimizing reasons include the prospect of new jobs, increased tax revenues, and various other economic and social ripple effects. 

Despite more or less convincing evidence of such benefits, economic development takings have a tendency to result in controversy. After {\it Kelo}, economic development takings have also been at the forefront of the constitutional property debate in the US. In the rest of the world, a similar shift in academic outlook has yet to take place, but expropriation-for-profit situations are increasingly coming into focus also on the global stage.\footnote{See, e.g., \cite{gray11,waring13,verstappen14}.} If we broaden our perspective slightly, to consider commercially motivated interference in property more generally, it even seems appropriate to speak of a crisis of confidence in property law, particularly in relation to land rights. This is most clearly felt in the developing world, where egalitarian systems of property use and ownership are coming under increasing pressure. It has been noted, in particular, that large-scale commercial actors are assuming control over an increasing share of the world's land rights, a phenomenon known as {\it land grabbing}.\footnote{See generally \cite{borras11}.} 

So far, most research on land grabbing has looked at how commercial interests, often cooperating with nation states, exploit weaknesses of local property institutions, to acquire land voluntarily, or from those who lack formal title. However, the similarity between economic development takings and state-aided land grabbings in favour of large commercial companies is striking. In particular, it has been noted how the purported public interest in economic development can be used to justify land grabs that would otherwise appear unjustifiable. In a recent article, Smita Narula cites {\it Kelo} directly and warns that procedural safeguards alone might not provide sufficient protection against abuse. She writes:

\begin{quote}
Procedural safeguards, however, can all too easily be co-opted by a state because its claims about what constitutes a public purpose may not be easy to contest. Particularly within the context of land investments, states could use the very general and under-scrutinized language of ``economic development'' to justify takings in the public interest.\footcite[157]{narula13}
\end{quote}

This quote underscores the broader relevance of the study of economic development takings. In addition, it reminds us that the question of what can be justified in the name of ``economic development'' is a general one, not confined to particular systems for organizing property rights. To address this, and to restore confidence in the institution of property more generally, many academics and policy makers turn towards {\it human rights}.\footnote{See generally \cite{schutter10,schutter11,kunnerman13}.} It has been argued, in particular, that a human right to land should be \isr{recognised} on the international stage, a right that would apply even when those affected by a land grab lack formal title. If successful, this approach promises to deliver basic protection against interference in established patterns of property use independently of how particular jurisdictions approach property.

In Europe, the human right to property is still usually understood in more conventional terms, as pertaining primarily to the rights of formally titled owners. However, a broad, social-function perspective on this right is influential due to the ECHR and the court in Strasbourg.\footnote{As discussed in Chapter \ref{chap:1}, Section \ref{sec:3}.} The issue of land grabbing highlights the importance of maintaining such a perspective, particularly when attempting to use western legal categories when analysing the developing world. In the context of land grabbing, protecting land rights is not primarily a question of protecting the civil law ideal of individual dominion. Rather, it is a question of providing protection against large-scale transactions that \isr{destabilise} or destroy established patterns of land use, to the detriment of local communities. 

In human rights discourse, particularly relating to the developing world, the focus is often on pressing problems related to food and water security as well as the protection of basic livelihoods, issues that can arise with urgency in the context of land grabbing. However, to achieve effective protection we need firm categories and enforcible legal principles to back up our benchmarks and our good intentions. In this regard, I think Narula is right to stress that the lack of a convincing approach to the notion of ``economic development'' is a crucial challenge.

As an overarching goal, economic development is no doubt sound, particularly for poor nations. The problem is that the risk of abuse is great when such a vague term is used to justify dramatic interferences in property. Such interferences typically cause severe disturbances in people's lives. This, moreover, is true for a middle-class US homeowner in much the same way as it is true for members of self-sustaining agrarian communities in Africa, although the stakes might be very different. Hence, there seems to be great potential for exchange of ideas and insight between those working on economic development takings and those studying land grabs in the developing world.

\noo{ As illustrated by {\it Kelo}, deep conflicts can arise in this regard also in developed democracies with long established and relatively stable systems of private property. In the following, I will attempt to shed further light on the issue as it arises in such legal systems, without considering the additional complications that arise when property itself is a more fragile concept. I note, however, that according to the social function view of property, there is no need to view formally \isr{recognised} property rights as completely distinct from rights arising from property use that is not based on formal title. The two are intertwined and the difference between them is at most a matter of degree.\footnote{Moreover, if the human flourishing account of property values is successfully developed, there should even be hope that a unified normative treatment can be given at some point.}}

In this thesis, I focus on legal systems where private property is well-established and relatively stable as a legal category. Moreover, my case study will look to Norway, a prosperous European country with a long tradition of an egalitarian distribution of land rights among the rural population. Hence, I will focus on situations when those affected by takings of land for economic development have a {\it prima facie} cause for objecting on the basis of recognised property rights. Therefore, the complications that occur when those most severely affected do not have formally recognised property rights will not be considered in any depth. However, I believe this a very interesting avenue for future work.

In the following, I will present a comparative background for my case study. I will begin by considering English law, where courts have generally been reluctant to broadly scrutinize the use of economic development as a justification for state interference in property. After this, I turn to the ECHR and the proportionality test that is now at the core of property adjudication at the ECtHR. I note that while states are considered to have a wide margin of appreciation with regards to the legitimacy of the purpose underlying interference, the balancing required under the proportionality test can still become a powerful basis on which to scrutinize the broader negative effects of economic development takings.

Following this, I move on to consider the US in greater depth, both the historical debate that led to {\it Kelo} and the suggestions for reform that have emerged following its backlash. There has been much written about this issue in the US. Moreover, while much of it is repetitive and coloured by the tense political climate, I believe some historical points, as well as some recent suggestions for reform, are highly relevant also to the international setting. To single out and analyse those aspects is the main aim of this part of the chapter. Indeed, the current debating climate in the US might be an indication of what is to come also in Europe, if concerns about the legitimacy of economic development takings are not taken seriously.

%I also highlight what I believe to be a connection between the situation in the US leading up to {\it Kelo} and the present situation in Europe, illustrated by the fact that the European Court of Human Rights is now explicitly endorsing ``stronger protection'' of property rights.  I attempt to identify the reasons behind calls for a stricter approach, arguing that it is connected to the fact that interferences in property under modern regulatory regimes is sanctioned in wide a range of different circumstances, serving to undermine their status as a necessary burden imposed on owner's according to the will of the greater public. In some cases, rather, takings appear to both owners and the public as improperly motivated and socially and politically unfair. I note that this happens particularly often in economic development cases, when commercial actors benefit to the detriment of local communities. I go on to list some concrete issues that arise with respect to such takings and that have been flagged as problematic in the literature.
%
%Following up on this, I consider various proposals that have been made to resolve tensions and limit the possibility of abuse in economic development cases. The differences of opinion that have been expressed in this regard have been quite substantial, and proposals have ranged from suggesting an outright ban on economic development takings  (Somin 2007; Cohen 2006) to suggesting that the best way forward is to reassess principles for awarding compensation in such cases (Householder 2007; Lehavi and Licht 2007).

%Much of the current theory focus on assessing traditional judicial safeguards that courts can rely on to prevent abuses, pertaining primarily to the material assessment of proportionality, public purpose, and compensation. 

%In the last part of the chapter, I will focus on a very interesting strand of recent work in the US, which shifts attention towards procedural rules that can help address the worry that economic development takings tend to suffer from a democratic deficit. The core concern is that the manner in which eminent domain decisions are typically made, and the way in which owners are compensated, might be unsuitable for economic development cases. Importantly, the need for special procedures has been noted, to restore legitimacy.\footnote{See generally \cite{lehavi07,heller08}.} This ties the US debate even closer to the European context, where proportionality, not public use, has become the key notion in property protection. Several recent suggestions from the US can be conceptualized as suggestions that aim to secure fairness and proportionality, while paying less attention to the formalistic question of what constitutes a ``public use''.
%
%%Also, it allows us to be very clear about a special concern that arises for economic takings cases: under current regulatory regimes, the government and the developer together often dominate the decision-making process completely, leaving the property owners marginalized. Hence, there is often a {\it democratic deficit} in such cases, resulting in discontent and a feeling that the taking is not in the public interest at all. Importantly, some recent writers hypothesize that if the proper balance can be restored in the decision-making process, so will the decision reached appear more legitimate, also with respect to the public use clause. In my opinion, this idea is crucial, and together with the question of compensation, which raises a similar structural problem, it will guide the rest of the work done in this thesis. 
%

In response to that worry, this chapter aims to bring into focus the key question of how to ensure meaningful participation for owners and their local communities in decision-making pertaining to economic development on their land. The tentative answers provided in Section \ref{sec:ir} will set the stage for the remainder of the thesis, where these answers will be assessed in depth against the case study of Norwegian hydropower.

%In particular, I will consider two special semi-judicial procedural systems used in such cases in Norway, one targeting compensation following expropriation, and another used as an alternative to expropriation, particularly in cases when development requires cooperation among many owners.

%I conclude by arguing that approaches along procedural lines represent the best way forward in relation to addressing issues associated with economic development takings. This raises the following problem, however: what procedural principles can be used to ensure meaningful participation, without hindering socially and economically desirable development projects? This question sets the stage for the remainder of my thesis, where I conduct a case study of expropriation for the development of hydro-power in Norway. In particular, I will consider two special semi-judicial procedural systems used in such cases in Norway, one targeting compensation following expropriation, and another used as an alternative to expropriation, particularly in cases when development requires cooperation among many owners.

\section{Economic Development Takings in England}\label{sec:contrast}

Economic development takings have not become as controversial in Europe as they are in the US, but there have been cases where the issue has come up, in several different jurisdictions.\footnote{For instance, in the UK, Ireland and Germany, as well as in Norway and Sweden. See \cite[466-483]{walt11}; \cite{stenseth10}.} \noo{ The P1(1) of the ECHR protects property, but the legitimacy of economic development takings has not yet been discussed in case law from the European Court of Human Rights (ECtHR). However, it is interesting to analyse cases like {\it Kelo} against P1(1), particularly since the ECtHR has developed a doctrine that focuses on ``proportionality'' and ``fairness'' rather than the purpose of interference.\footnote{See generally, \cite[Chapter 5]{allen05}. This approach may become even more significant as a source of property protection in the future, as the ECtHR have indicated that there are ``jurisprudential developments in the direction of a stronger protection under Article 1 of Protocol No. 1'', see \cite[135]{lindheim12}.}}

In this section, I address economic development takings from the point of view of English law.\noo{ I then go on to give a more detailed presentation of the unifying property clause in P1(1). The case law from the ECtHR is presented and analysed in some depth, in an effort to assess how the ECtHR would be likely to approach an economic development case such as {\it Kelo}. In particular, I argue that the proportionality doctrine offers an interesting approach to such cases. Importantly, the doctrine stipulates that a ``fair balance'' must be struck  between the interests of the property owner and the public.\footcite[Chapter 5]{allen05} I argue that such a perspective could make it easier to get to the heart of why economic development takings are often seen as problematic, without getting lost in theoretical discussions about the meaning of  terms like ``public use'' or ``public purpose''. However, I also raise the concern that the ECtHR is not the appropriate institution for applying the proportionality test. Indeed, its remoteness to most of Europe suggests that we should look for more locally grounded legitimacy-enhancing institutions. Such institutions will likely be better able to assess the fairness of interference in context.

I go on to discuss whether existing government institutions can serve this purpose, arguing that local courts may well be the best candidates. However, I also note that active application of the proportionality test in property cases might not be found at the local level. In this regard, the ECtHR could play a crucial role, by focusing on the systemic question of what issues local courts need to consider when assessing legitimacy of property interference. 

However, quite apart from this, there is reason to worry that judicial bodies are not ideally suited to carry out the kind of assessment that is required. Hence, new institutional proposals might be in order. I conclude by arguing that once the need for local grounding is recognised and met, the ECtHR has the potential to play an important and constructive role in providing oversight and developing basic principles, also with respect to new institutions that aim to deliver increased legitimacy at the local level.

\subsection{England}\label{sec:england}
}
In England, the principle of parliamentary supremacy and the lack of a written constitutional property clause has led to expropriation being discussed mostly as a matter of administrative law and property law, not as a constitutional issue.\footcite{taggart98} Moreover, the use of compulsory purchase -- the term most often used to denote takings in the UK -- has not been restricted to particular purposes as a matter of principle. The uses that can warrant compulsory alienation of property are those that parliament regard as worthy of such consideration. However, as private property itself has long been recognised as a fundamental right, the power of compulsory purchase has typically been exercised with caution. 

In his {\it Commentaries on English Law}, William Blackstone famously described property as the ``third absolute right'' that was ``inherent in every Englishman''.\footnote{See \cite[134-135]{blackstone79}. The first right is security while the second is liberty.}  Moreover, Blackstone expressed a very restrictive view on the possibility of expropriation, arguing that it was only the legislature that could legitimately interfere with property rights. He warned against the dangers of allowing private individuals, or even public tribunals, to be the judge of whether or not the ``common good'' could justify takings. Blackstone went as far as to say that the public good was ``in nothing more invested'' than the protection of private property.\footcite[134-135]{blackstone79}

Historically, Blackstone's description conveys a largely accurate impression of takings practice in England. Indeed, Parliament itself would usually be the granting authority in expropriation cases, through so-called {\it private Acts}. Hence, compulsory purchase would not take place unless it had been discussed at the highest level of government. Moreover, the procedure followed by parliament in such cases strongly resembled a judicial procedure; the interested parties were given an opportunity to present their case to parliament committees that would then decide whether or not compulsion was warranted.\footnote{See \cite[13-16]{allen00}. While this procedure reflected a protective attitude towards private property, recent scholarship has also pointed out that expropriation was in fact used very actively in Britain, particularly following the glorious revolution, see \cite{hoppit11}.}

On the one hand, the direct involvement of parliament in the decision-making process reflected a fundamental respect for property rights. But at the same time, parliamentary supremacy also meant that the question of legitimacy was rendered mute as soon as compulsory purchase powers had been granted. The courts were not in a position to scrutinize takings at all, much less second-guess parliament as to whether or not a taking was for a legitimate purpose.

During the 19th Century, as an industrial economy developed, private acts granting compulsory purchase powers to commercial companies grew massively in scope and importance.\footnote{See \cite[204]{allen00}.} Private railway companies, in particular, regularly benefited from such acts.\footnote{\cite[204]{allen00}. See generally \cite{kostal97}.} During this time, the expanding scope of private-to-private transfers for economic development led to high-level political debate and controversy. Usually, it would attract particular opposition from the House of Lords. Interestingly, this opposition was not only based on a desire to protect individual property owners. It also often reflected concerns about the cultural and social consequences of changed patterns of land use.\footcite[204]{allen00} 

Hence, the early {\it political} debate on economic development takings in the UK shows some reflection of a social function approach to property protection. At the same time, as society changed following increasing industrialisation, an expansive approach to compulsory purchase would eventually emerge as the norm.\footnote{Arguably, the social function perspective is the key to understanding why this happened. Indeed, the expanded use of private takings in England during the 19th century, particularly in connection with the railways, might have served a more easily justifiable social function than that commonly associated with economic development takings today. Waring, in particular, notes how railway takings tended to affect aristocratic landowners rather than marginalised groups (``unlike private takings today, the railway legislation was most likely to affect those who could best defend their property rights from attack''), see \cite[111]{waring09}.} The idea that economic development could justify takings gradually became less controversial.

Today, the law on compulsory purchase in England is regulated in statute and the role of courts is to a large extent limited to the application and interpretation of statutory rules. Some common law rules still play an important role, such as the {\it Pointe Gourde} rule, which stipulates that changes in value due to the compensation scheme itself should be disregarded when calculating compensation to the owner.\footnote{The rule takes it name the case of \cite{gourde47}. The underlying principle, including also statutory regulations with a similar effect, is referred to as the ``no scheme'' principle, see \cite{lawcom01}. The principle is found in many jurisdictions, see \cite{sluysmans14}. The principle is often quite contentious, and notoriously hard to apply in practice. For a recent attempt at clarifying the principle, see \cite{waters04}. I note that a strict interpretation of the no-scheme principle effectively precludes benefit sharing between takers and owners, a phenomenon that is of particularly relevance in the context of economic development takings. I will not address this particular issue in any depth here -- I choose instead to focus on legitimacy of takings in a broader, non-compensatory sense. However, the compensation aspect of economic development takings is also very interesting (and challenging). For further details, I refer to \cite{dyrkolbotn15}.} With respect to the question of legitimacy, however, the starting point for English courts is that this is a matter of ordinary administrative law.\footnote{See \cite{taggart98}.}

More recently, the \cite{hra98} adds to this picture, since it incorporates the property clause in P1(1) into English law. Even so, the usual approach in England is to judge objections against compulsory purchase orders on the basis of the statutes that warrant them, rather than constitutional principles or human rights provisions that protect property.\footnote{See \cite[121-132]{waring09}. The important statutes are the \cite{ala81}, the \cite{lca61}, the \cite{cpa65}, the \cite{tcpa90} and the \cite{pcpa04}.} It is typical for statutory authorities to include standard reservations to the effect that some public benefit must be identified in order to justify a compulsory purchase order, but the scope of what constitutes a legitimate purpose can be very wide. For instance, to warrant a taking under the \cite{tcpa90}, it is enough that it will ``facilitate the carrying out of development, redevelopment or improvement on or in relation to the land''.\footcite[226]{tcpa90} 

While various governmental bodies are authorised to issue compulsory purchase orders (CPOs), a CPO typically has to be confirmed by a government minister.\footnote{See \cite[48]{waring09}.} The affected owners are given a chance to comment, and if there are objections, a public inquiry is typically held. The inspector responsible for the inquiry then reports to the relevant government minister, who makes the final decision about whether or not it should be granted, and on what terms. The CPO may then be challenged in court, but will usually only be scrutinized on the basis of whether or not it lies within the scope of the statute authorising it, not on the basis of whether or not the purpose of the taking appears to be legitimate as such.\footnote{See, e.g., \cite[48-49]{waring09}.}

That said, the idea that property may only be compulsorily acquired when the public stands to benefit permeates the system. Indeed, this has also been regarded as a constitutional principle, for instance by Lord Denning in {\it Prest v Secretary of State for Wales}.\footcite{prest82} He said:

\begin{quote}
It is clear that no minister or public authority can acquire any land compulsorily except the power to do so be given by Parliament: and Parliament only grants it, or should only grant it, when it is necessary in the public interest. In any case, therefore, where the scales are evenly balanced – for or against compulsory acquisition – the decision – by whomsoever it is made – should come down against compulsory acquisition. I regard it as a principle of our constitutional law that no citizen is to be deprived of his land by any public authority against his will, unless it is expressly authorised by Parliament and the public interest decisively so demands. If there is any reasonable doubt on the matter, the balance must be resolved in favour of the citizen.\footcite[198]{prest82}
\end{quote}

Lord Denning also supported the doctrine of necessity, as expressed by Forbes J in {\it Brown v Secretary for the Environment}:\footcite{brown78}

\begin{quote}It seems to me that there is a very long and respectable tradition for the view that an authority that seeks to dispossess a citizen of his land must do so by showing that it is necessary, in order to exercise the powers for the purposes of the Act under which the compulsory purchase order is made, that the acquiring authority should have authorisation to acquire the land in question.\footcite[291]{brown78}
\end{quote}

In practice, these principles are mostly implicit in legal reasoning, as a factor that influences the courts when they interpret statutory rules and carry out judicial review of administrative decisions. As Watkins LJ stated in {\it Prest}:

\begin{quote}
The taking of a person's land against his will is a serious invasion of his proprietary rights. The use of statutory authority for the destruction of those rights requires to be most carefully scrutinised. The courts must be vigilant to see to it that that authority is not abused. It must not be used unless it is clear that the Secretary of State has allowed those rights to be violated by a decision based upon the right legal principles, adequate evidence and proper consideration of the factor which sways his mind into confirmation of the order sought.\footcite[211-212]{prest82}
\end{quote}

In {\it R v Secretary of State for Transport, ex p de Rothschild}, Slade LJ referred to {\it Prest} and made clear that he did not regard it as expressing a rule concerning the burden of proof in compulsory purchase cases. Rather, he took it as more general observation on the severity of property interference and the importance of vigilance in such cases.\footcite{rothschild89} He pointed to ``a warning that, in cases where a compulsory purchase order is under challenge, the draconian nature of the order will itself render it more vulnerable to successful challenge''.\footcite[938]{rothschild89}

\subsection{{\it Sainsbury's Supermarkets Ltd v Wolverhampton City Council}}

An illustration of how English courts approach objections to the legitimacy of takings is found in the recent case of {\it Regina (Sainsbury’s Supermarkets Ltd) v Wolverhampton City Council}.\footcite{sainsbury10} Here a CPO was granted to allow the company Tesco to acquire land from its competitor Sainsbury, in a situation when they were both competing for licenses to undertake commercial development on the same land, owned partly by both. 
The decisive factor that had led the local authorities to grant the CPO was that Tesco had offered to develop a different property in the same local area, which was currently in need of regeneration.

Sainsbury protested, arguing that the local council could not strike such a deal on the use of its compulsory purchase power. It was argued, moreover, that taking the land for incidental benefits resulting from development in a different part of town was not legitimate under the Town and Country Planning Act 1990. The UK Supreme Court agreed 4-3, with Lord Walker in particular emphasising the need for heightened judicial scrutiny in cases of private-to-private takings for economic development.\footcite[80-84]{sainsbury10} Lord Walker even cited {\it Kelo}, to further substantiate the need for a stricter standard in such cases.\footcite[81]{sainsbury10} 

However, the main line of reasoning adopted by the majority was based on an interpretation of the Town and Country Planning Act itself. In particular, the majority held that it was improper for the local council to take into consideration the development that Tesco had committed itself to carry out on a different site.\footcite[73-79]{sainsbury10} This, in particular, was not ``improvement on or in relation to the land'', as required by the Act.\footcite[336]{tcpa90} In addition, Lord Collins, who led the majority, said that ``the question of what is a material (or relevant) consideration is a question of law, but the weight to be given to it is a matter for the decision maker''.\footcite[70]{sainsbury10} These comments reflect the traditional approach to judicial review of CPOs under English law, demonstrating how the underlying statutory authority tends to be at the center of attention.

However, it is interesting to see how the purpose of the interference featured in the background of the Supreme Court's interpretation and application of the statutory rule. The opinion of Lord Walker is particularly interesting, since he stresses that ``the land is to end up, not in public ownership and used for public purposes, but in private ownership and used for a variety of purposes, mainly retail and residential.''\footcite[81]{sainsbury10} He goes on to state that ``economic regeneration brought about by urban redevelopment is no doubt a public good, but ``private to private'' acquisitions by compulsory purchase may also produce large profits for powerful business interests, and courts rightly regard them as particularly sensitive.``\footcite[81]{sainsbury10}

Lord Walker then makes clear that he does not think it is impermissible, as such, for the local council to take into account positive effects on the local area, even when these do not directly result from the planned use of the land that is being acquired. Instead, he relies explicitly on the for-profit character of the taking, by arguing that ``the exercise of powers of compulsory acquisition, especially in a ``private to private'' acquisition, amounts to a serious invasion of the current owner's proprietary rights. The local authority has a direct financial interest in the matter, and not merely a general interest (as local planning authority) in the betterment and well-being of its area. A stricter approach is therefore called for.''\footcite[84]{sainsbury10} 

Lord Walker's opinion might indicate that the narrative of economic development takings is about to find its way into English case law. Moreover, a more critical approach might be adopted in the future, when compulsory purchase powers are made available to commercial companies wishing to undertake for-profit schemes. However, for schemes where the commercial aspect appears less dominant, English courts still appear very reluctant to quash CPOs, also when the purpose is economic development. This is so even in situations when the owners have requested a stricter standard of review on the basis of human rights law. 

\subsection{{\it Smith \& Others v Secretary of Stare for Trade and Industry}}

In the case of {\it Smith \& Others v Secretary of State for Trade and Industry}, a caravan site was compulsorily acquired for development in connection with the London Olympic Games.\footcite{smith08} Some of the owners protested, including Romany Gypsies who used the caravans as their primary residence. A public inquiry was held, after which the inspector recommended that the CPO should not be confirmed until adequate relocation sites had been identified. However, due to the ``urgency, timing and importance'' of the project, the Secretary of State decided to go ahead before a relocation scheme was put in place (although he expressed commitment to ensuring satisfactory relocation).\footcite[10]{smith08} The owners argued that without satisfactory relocation plans, the interference in the property rights was not proportional and had to be struck down on the basis of human rights law, in particular Article 8 in the ECHR regarding respect for the home and private life.\footcite[27-51]{smith08}

The Court of Appeal considered the matter in great depth, applying the doctrine of proportionality developed at the ECtHR. Importantly, this doctrine was understood to go beyond the standard form of judicial review required under English law. However, the Court still concluded that the taking was proportional. This was largely based on the finding that ``the issue of proportionality has to be judged against the background that everyone accepts that an overwhelming case has been made out for compulsory acquisition of the sites for the stated objectives and that compulsory purchase is justified.''\footcite[42]{smith08} 

Justice Williams arrived at this conclusion after noting that the owners' {\it only} substantial objection against the CPO was that it was confirmed before adequate relocation measures had been agreed on.\footcite[42]{smith08} Hence, the question, as he saw it, did not concern the validity of using compulsory purchase powers, but merely the timing with which it had been ordered. On this basis, he framed the question of legitimacy as one relating to the ``necessity'' standard, according to which an infringement of Convention rights is only permissible when the public interest cannot be served in some other way.\footcite[43]{smith08} A strict reading of this standard holds that an interference must be the {\it least intrusive means} of achieving the stated aim.\footnote{Such a standard has been adopted in some Convention cases, for instance in \cite{samaroo01}.}

Justice Williams argued against such a strict reading, subscribing instead to a view expressed as an {\it obiter} in the case of {\it Pascoe v The First Secretary of State}. According to this view, an interference need not be the least intrusive means. Rather, it is sufficient that the measure is ``reasonably necessary'' to achieve that aim.\footnote{See \cite[74-75]{pascoe06} (quoting \cite[25]{clay04}).} However, while noting his agreement with this approach, Justice Williams went on to also apply the stronger necessity test, and found that even if this was applied the CPO in question would still be a proportional interference.\footcite[41-50]{smith08}

It seems clear that while the taking in question was for economic and recreational development purposes, the case was marked by a preliminary finding to the effect that the legitimacy of the aim of interference -- to facilitate the London Olympics -- was beyond reproach. Hence, there was no need for, or even room for, more detailed purposive reasoning of the kind that would later be applied by Lord Walker in {\it Sainsbury}. The fact that the taking was for economic development and recreation, not for a pressing public need, was not considered relevant. Moreover, since the case was construed to be solely about the extent to which the CPO was ``necessary'' to further its stated aim, the proportionality test that was carried out, despite being detailed, was very narrow in scope. It concerned only proportionality of the means, not of the aim itself. The question of how to weigh the public interest in a multi-billion dollar sporting event against the security of someone's home was not considered.

In later cases, a dismissive attitude towards substantive review has been adopted even in situations when the owners have argued against takings by explicitly questioning the proportionality of the interference against the importance of the aim. 

\subsection{{\it Alliance Spring Co Ltd v The First Secretary of State}}

In the case of {\it Alliance Spring Co Ltd v The First Secretary of State}, a large number of properties were expropriated to build a new football stadium for the football club Arsenal.\footcite{alliance06} Some owners who stood to lose their business premises protested, pointing to the fact that the inspector in charge of the public inquiry had recommended against the takings.\footcite[6-7]{alliance06} According to Justice Collins, the main argument that the owners relied on when protesting the taking was that it did not serve a ``proper purpose''.\footcite[19]{alliance06} This argument was not held to be valid, however, with Justice Collins concluding as follows: 

\begin{quote}
There is nothing in the material put before and accepted by the Inspector which persuades me that that decision was ill founded or was one which the Secretary of State was not entitled to reach. Developments which result in regeneration of an area are often led by private enterprise. Mr Horton perforce accepts that that is so, but submits that this is not the sort of situation where, for example, a private development is the anchor for a particular scheme. I disagree.\footcite[19]{alliance06}
\end{quote}

Hence, unlike the case of {\it Smith}, where the Court did in fact carry out its own assessment of proportionality, the {\it Alliance} Court was content with deferring to the assessment carried out by the executive branch.\footnote{This has been criticized, e.g., by Kevin Grey who describes the reference to Convention Rights in Alliance as ``worryingly brief''. See \cite{gray11}.} As such, the case appears to follow the pattern of judicial review of CPOs established before the Human Rights Act 1998. This means that the decision also contrasts with how English courts have approach the Convention in relation to other  rights, such as those of Article 8 addressed in {\it Smith}.

Whether the approach taken in {\it Alliance} is good law after {\it Sainsbury} is unclear; judging from Lord Walker's opinion, it seems that a more substantive assessment might be required for similar cases in the future. While this might not imply a different outcome for a case like {\it Alliance}, it would mean that courts would have to engage in independent review of the purpose and merits of contested CPOs that benefit commercial actors. In particular, English courts would have to change the way they approach such cases, by being better prepared to assess for themselves whether a fair balance is struck between the interests of the developer and the property owners. Hence, it is not unlikely that the category of economic development takings will become an important point of reference in the future, both for the law and those who study it.

\noo{ \subsection{Germany}\label{sec:germany}

In German law we find an explicit constitutional property clause. In particular, Article 14 of the Basic Law ({\it Grundgesetz}) reads as follows:

\begin{quote}
(1) Property and the right of inheritance shall be guaranteed. Their content and limits shall be defined by the laws. \\
(2) Property entails obligations. Its use shall also serve the public good. \\
(3) Expropriation shall only be permissible for the public good. It may only be ordered by or pursuant to a law that determines the nature and extent of compensation. Such compensation shall be determined by establishing an equitable balance between the public interest and the interests of those affected. In case of dispute concerning the amount of compensation, recourse may be had to the ordinary courts.\footcite[14]{basic49}
\end{quote}

Apart from the fact that the property clause is explicit, I note two further characteristic features of the protection of property in Germany. First, the constitution explicitly stresses that property comes with social obligations as well as rights. The use of property should ``serve the public good''. On the other hand, it is also made clear that expropriation is only permissible when it is ``for the public good''. Hence, it follows immediately that the purpose of expropriation is a relevant factor when determining the legitimacy of a taking, \isr{irrespective} of the specific statute used to authorise it. Importantly, it is clear already from the outset that the question of legitimacy is a \emph{judicial} question, one which the courts can only answer if they form an opinion about that constitutes the ``public good''. 

This means that it is quite natural to approach the question of economic development takings from the point of view of constitutional law. Unlike in England, disputes over the legitimacy of such takings can be comfortably adjudicated directly against a ``public good'' restriction. While this sets Germany apart on the theoretical level, it is unclear how much of an effect it has had in practice. To shed some light on this question, we can look to the two major authorities on the legitimacy of economic development takings, the cases of {\it D\"{u}rkheimer Gondelbahn} and {\it Boxberg}.\footcite{durkheimer81,boxberg86} 

In both cases, the German Constitutional court found that expropriation to the benefit of commercial interests was illegitimate. However, the Court argued for this result on the basis that there was insufficient statutory authority for such takings in the concrete circumstances complained of. That is, the Court did not directly address the question of whether the relevant statutes were compliant with Article 14 of the basic law. Instead, they interpreted statutory authorities on the assumption that they had to be, following a pattern of reasoning that appears to be rather close to the approach followed by English courts in similar cases.\footnote{Although in {\it Dürkheimer Gondelbahn}, Böhmer J gave a separate concurring judgment where he argued for this result on the basis of the public good requirement of the basic law.} It seems, in particular, that even in Germany, the public purpose restriction is primarily relevant as a factor guiding the interpretation of statutory authorities.

That said, the cases of {\it D{\"u}rkheimer Gondelbahn} and {\it Boxberg} show that in situations when the public purpose of a taking is unclear, German courts seem inclined to \isr{favour} a narrow interpretation of the relevant statute. In {\it Bloxberg}, several properties were expropriated \isr{in favour} of the car company Daimler Benz AG, for commercial purposes. The affected local communities suffered from high unemployment rates and a slow economy, so a {\it prima facie} reasonable \isr{case} could be made that allowing Daimler to acquire the land was in the public interest, as it would facilitate economic growth. However, the Federal Constitutional Court agreed with the owners that the expropriation was invalid. This, it held, was because the taking was outside the scope of the relevant statute, which authorised expropriations for ``planning purposes''. The owners had argued extensively using Article 14 of the Basic Law and the constitutional ``public good'' restriction clearly did play a role in the Court's reasoning. But at the same time, the Court stressed that private-to-private transfers that bestow financial benefit on the acquiring party may well satisfy the ``public good'' requirement. The important issue was whether a sufficiently strong public interest could be identified, \isr{irrespective} of any windfall benefits that might fall on private parties.

In light of this, I think it is wrong to exaggerate the importance of the explicit formulation of the public use test offered in the German constitution. Its importance seems to rest mainly in the fact that it provides a particularly authoritative expression guiding the national courts' application of statutory provisions regarding expropriation of property. But developments in common law, where the public use requirement is stressed as a guiding constitutional principle, might well point in the same direction. In principle, both German and English Courts are in a good position to respond to increased tension regarding economic development takings by developing a stricter standard of judicial review in such cases.

A different aspect of German law deserves special attention, however, since it does not appear to have any clear counterpart in the common law tradition. This is the  ``social-obligation'' norm in Article 14 (2), which points to a different \isr{conceptualisation} of property rights as such. As argued by Alexander, the distinguishing feature of the property clause in the German Constitution is that the value of property is thought to relate more strongly to its importance for human dignity and flourishing in a social context, rather than the protection of individual financial entitlements. As Alexander notes regarding the Germans' own \isr{conceptualisation} of their property clause:

\begin{quote}
This theory holds that the core purpose of property is not wealth maximization or the satisfaction of individual preferences, as the American economic theory of property holds, but self-realization, or self-development, in an objective, distinctly moral and civic sense. That is, property is fundamental insofar as it is necessary for individuals to develop fully both
as moral agents and participating members of the broader community.\footcite[745]{alexander03}
\end{quote}

With such a starting point, it is not surprising that in cases such as {\it Boxberg}, resembling {\it Kelo}, German Courts will tend to adopt a strict view on legitimacy. These are cases when the property rights infringed on serve a fundamentally different function for the two opposing private parties. To the owner, the property is a home, an important source of self-identity, autonomy, security and membership in a community. To the taker, it represents an obstacle to commercial development which needs to be removed. In such a situation, it is in keeping with the spirit of the social-obligation norm of property to offer enhanced protection to the homeowner. To this owner, the property serves a purpose which is fundamentally different, and arguably more worthy of protection, then the property's purpose for the developer. A taking in this situation might therefore, because of Article 14, require a particularly clear and strong public interest.

But unless there is an asymmetry between owner and taker, heightened scrutiny does not necessarily follow. Hence, it is interesting to speculate what German courts would have made of a case such as {\it Regina (Sainsbury’s Supermarkets Ltd) v Wolverhampton City Council}. Here, the interests of owner and taker were strictly commercial nature. Both owned part of the contested land and neither one could develop the land according to their plans without buying out the other. The enhanced protection of property offered under German law would probably not have much significance in such a case. 

In fact, it might well be that German courts would be {\it more} likely to accept such a taking. First, their \isr{conceptualisation} of property rights appears to allow greater flexibility to adapt the level of protection to the circumstances and the purposes of the property in question. So even if is correct that private-to-private transfers for commercial projects require a ``stricter approach'' in general, as argued by Lord Walker in \textcite{sainsbury10}, the fact that the interests of the owner were also purely commercial  might make this less relevant. Second, German courts might be more inclined to have regard to socially beneficial additional commitments entered into by the applicant, even if they do not concern the property that is taken. As a tie-breaker, looking to such commitments might be as good an approach as any other.\footnote{This was the view taken by the dissenting minority in \textcite{sainsbury10}.}

Of course, objections could still be raised on the basis of general administrative law. Indeed, some might see the case as an example of government ``auctioning'' off licenses to the highest bidder. This might well be regarded as an affront to good governance. I will not delve into German law to assess the case from this perspective. My point is simply that because of the purposive and contextual nature of Article 14, it seems unlikely that a case like \textcite{sainsbury10} would turn on constitutional property law.

To sum up, German constitutional law serves to create an interesting contrast with English law regarding the question of economic development takings. On the one hand, property appears to be better protected against such takings in Germany, but on the other hand, the extent to which increased protection is offered depends more closely on the social values involved. The German system appears to look more actively at the social function of property for guidance when resolving property disputes, thereby echoing some of the ideas discussed in Chapter \ref{chap:1}. 

In the next section, I will discuss the property clause in the ECHR, which explicitly serves to set up a minimum level of property protection that provides a common standard for all member states, including Germany and the UK.
}

\section{The Property Clause in the European Convention of Human Rights}\label{sec:echr}

The standard account of the protection against interference inherent in P1(1) describes it as consisting of three rules.\footnote{For a more detailed description of P1(1) generally, I refer to \cite{allen05}.} First, there is the rule of {\it legality}, asserting that an interference needs to be authorized by statute. Second, there is the rule of {\it legitimacy}, making clear that interference should only take place in pursuance of a legitimate public purpose. The third rule is the ``fair balance'' principle, requiring proportionality between the means and the aims in cases involving property interference.\footnote{See \cite[69]{sporrong82} and \cite[120]{james86}.} %which is applied by the ECtHR in almost all cases when it finds that there has been a violation of P1(1).

The starting point for property adjudication at the ECtHR is that States have a ``wide margin of appreciation'' with regard to the legitimacy question.\footcite[See][54]{james86} This question is thought to depend on democratically determined policies to such an extent that it is rarely appropriate for the Court to censor the assessments made by member states. At the same time, the Court has gradually adopted a more active role in assessing whether or not particular instances of interference are proportional and able to strike a fair balance between the interests of the public and the property owners. As argued by Allen, this has caused P1(1) to attain a wider scope than what was originally intended by the signatories.\footcite[1055]{allen10}

In the early case law behind this development, the focus was predominantly on the issue of compensation, with the Court gradually developing the principle that while P1(1) does not entitle owners to full compensation in all cases of interference, the fair balance will likely be upset unless at least some compensation is paid, based on the market value of the property in question.\footnote{See \cite[103]{scordino06}. The case also illustrates that the Court has adopted a fairly strict approach to the question of when it is legitimate to award less than full market value.} %The focus on compensation has also been reflected in academic work on P1(1), which tends to address proportionality from a financial perspective, by investigating to what extent owners are entitled to compensation based on the market value of their property. Indeed, when considering case law and literature on the subject, one is left with the impression that ``fair balance'' with regards to P1(1) is crucially linked to financial entitlements. It seems that d as a standard that can justify a right to compensation that goes beyond what the wording of P1(1) might initially suggest.

As mentioned in Section \ref{sec:x} of Chapter \ref{chap:1}, it has now become clear that the fair balance test encompasses more than this. In particular, the hunting cases show that the Court in Strasbourg is willing to reflect broadly on the context and purpose of interference, to critically assess the social function of the taking.

\noo{ In {\it Chassagnou and others v France} the situation was that landowners were compelled to permit hunting on their land, following compulsory membership in a hunting association which was set up to manage hunting in the local area.\footcite{chassagnou99} The owners protested this on the grounds that they were ethically opposed to hunting. The Court agreed that there had been a breach of P1(1). 

In the later case of {\it Hermann v Germany}, the circumstances were similar and the Court followed the precedent set in {\it Chassagnou}. In addition, the Court commented that they had ``misgivings of principle'' about the argument that financial compensation could provide adequate protection in such a case.\footcite[See][91]{hermann12}  In this way, the hunting cases illustrate that to the ECtHR, the right to property is more than a  financial entitlement. The fair balance that must be struck could pertain to other aspects, such as the owner's right to make use of his property in accordance with his convictions and to take part in decision-making processes regarding how it should be managed.\footnote{The assessment of proportionality should be concrete and contextual, and it is not based on a narrow or formalistic concept of property as dominion. This is demonstrated, for instance, by \cite{chabauty12}. Here the Court found no violation of P1(1) although the facts seemed close to those of {\it Chassagnou}. The case differed, however, in that the owner himself was not opposed to hunting, but wanted to withdraw his land from the hunters' association to enjoy exclusive hunting rights.}
}
Less obviously, a similar sentiment appears to be behind the Court's reasoning in recent cases involving rent control schemes and housing regulation.\footnote{See \cite{hutten06,lindheim12}.} There are obvious financial interests at stake in such cases, for both landlords and tenants. However, the Court has addressed these cases by looking to the fairness of the underlying regulation more generally, by critically evaluating the social, economic and political context. Moreover, the Court has not shied away from using concrete cases as a starting point for providing an assessment of the sustainability of national provisions as such.

\subsection{{\it Hutten-Czapska v Poland}}

The striking conclusion in {\it Hutten-Czapska v Poland}, which makes it interesting for the questions studied in this thesis, was that it demonstrated ``systemic violation of the right of property''.\footcite[239]{hutten06} The case concerned a house that had been confiscated during the Second World War. After the war, the property was transferred back to the owners, but in the meantime, the ground floor had been assigned to an employee of the local city council. The state implemented strict housing regulations during this time, which eventually led to the applicant's house being placed under direct state management.\footcite[20-31]{hutten06} Following the end of communist rule in 1990, the owners were given back the right to manage their property, but it was still subject to strict regulation that protected the rights of the tenants.\footcite[31-53]{hutten06} In addition to rent control, rules were in place that made it hard to terminate the rental contracts. Hence, it became impossible for the owners to make use of the house themselves.\footcite[20-53]{hutten06} 

After an in-depth assessment of the relevant parts of Polish law and administrative practice, the Grand Chamber of the ECtHR concluded that there had been a violation of P1(1). Importantly, they did not reach this conclusion by focusing on the owners and the interference that had taken place with respect to their individual entitlements. Rather, they focused on the overall character of the Polish system for rent control and housing regulation, as it manifested in the concrete circumstances of the applicant's case.

The financial consequences for the owners were considered to shed light on a broader question of sustainability, as was the financial situation of the tenants.\footcite[60-61]{hutten06} The Court was particularly concerned with the fact that the total rent that could be charged for the house in question was not sufficient to cover the running maintenance costs.\footcite[224]{hutten06} In particular, it was noted that the consequence of this would be ``inevitable deterioration of the property for lack of adequate investment and modernisation''.\footnote{\cite[224]{hutten06}.}

In the end, the Court highlighted how three factors combined to bring both owners and their properties  to a precarious position. First, the rigid rent control system made it hard to sustainably manage rental property. Second, tenancy regulation made it hard for owners to terminate tenancy agreements. Third, the Court noted that the state itself had set up these tenancy agreements during the days of direct state management, shedding doubt on the legitimacy of the commitments that these contracts imposed on owners. In combination, these factors led the Court to conclude that  a fair balance had not been struck.\footcite[224-225]{hutten06} 

The contextual nature of the Court's reasoning in {\it Hutten-Czapska} is evidenced not only by the extent to which the concrete circumstances were assessed against the goal of fairness. It is also illustrated by how the Court explicitly places the ``social rights'' of the tenants on equal footing with the property rights of the owners.\footcite[225]{hutten06} The result, therefore, was not premised on a narrow understanding of property protection as an individual entitlement, but on a broader vision of property as a social institution.

It is also of interest to note how the Court concludes that the root of the problem is with the Polish legal order as such. In this regard, great weight is placed on the observation that the regulatory system suffers from a lack of adequate safeguards to protect owners against imbalances such as those identified in {\it Hutten-Czapska}. In particular, the Court reflects on the position of owners and comments on ``the absence of any legal ways and means making it possible for them either to offset or mitigate the losses incurred in connection with the maintenance of property or to have the necessary repairs subsidised by the State in justified cases''. Hence, the rent control scheme alone was not the whole problem, the Court also criticised what it saw as a defective way of implementing it.\footcite[224]{hutten06} Moreover, the Court did not censor the political reasoning that motivated Polish housing legislation, but concluded instead that the ``burden cannot, as in the present case, be placed on one particular social group, however important the interests of the other group or the community as a whole''. 

I think the structural argument at work here is key to understanding the case, pointing also to the core function that the ECtHR should embrace more generally. It seems to me, in particular, that objections may well be raised against the appropriateness of having the Court in Strasbourg assess concretely what is fair regarding the relationship between owners and tenants in a specific house in Gdynia. The Court's remoteness to the local conditions, as well as its lack of accountability to local democratic institutions suggests that the Court is not ideally placed to carry out the kind of contextual assessment that it itself prescribes for such cases. In addition, the amount of resources and time needed to independently scrutinize these aspects concretely risks undermining the Court's ability to deal expediently with its case load. The ECtHR will hardly be able to protect human rights in Europe on a case-by-case basis.

Instead, the aim should always be to get at the systemic features that cause perceived imbalances. As in \textcite{hutten06}, the Court serves its function best when it is able to use concrete information about a suspect case to identify a sense in which the domestic legal order needs to be improved to better comply with human rights standards. This is particularly true when, as in that case, the Court notes that the applicants have insufficient options available for achieving a fair balance by appealing to institutions within the domestic legal order. By demanding {\it institutional} changes, the Court effectively delegates responsibility for ensuring the kind of fair balance that is required under the ECHR. Moreover, by scrutinizing the procedures and principles that the states apply when fulfilling this duty, it is likely that the Court will still be able to steer and unify the development of the case law. 

Importantly, they would then be able to do so without having to engage extensively in concrete assessments of fairness. Against this, one may argue that the judicial or administrative bodies of the signatory states can easily circumvent their obligations by giving a superficial or biased assessment of the facts in human rights cases, to avoid embarrassment for the state's political or bureaucratic elite. However, this might then be raised as a procedural complaint before the ECtHR, resulting in cases revolving around Articles 6 (fair trial) and 13 (effective remedy).\footnote{I note that this also fits with recent developments at the ECtHR, toward somewhat broader scrutiny under Article 6, see \cite{khamidov07}.}  In this way, the Court can streamline its functions, by always aiming to direct attention at issues that arise at a higher level of abstraction. This, in my view, is desirable. The ECtHR should not aim to micromanage the signatory states, particularly not in relation to a norm such a P1(1), which the Court itself regards as highly dependent on context.

However, the question arises as to what kind of institutions the Court should focus on in its effort to ensure fairness in relation to Convention rights such as property. It is not given, in particular, that directing attention towards domestic judicial bodies is the most appropriate approach. Rather, it is logical to assume that those institutions most in need of reform will be exactly those that are most often responsible for violations. A possible lack of an effective complaints procedure would be worrying, but not as problematic as systemic weaknesses of those institutions that act in ways that give rise to complaints in the first place. 

By shifting attention towards the institutional context of the primary decision-maker, the Court can also avoid getting stuck in deference to domestic judicial bodies. This can then be accomplished alongside a shift of attention away from concrete assessment of alleged violations. The Court can achieve this by concretely and critically assessing those rules and procedures that are identified as causally significant to individual complaints, at the administrative rather than the judicial level.\footnote{In the future, one might even encounter cases when the Court prefers to remain agnostic about whether a substantive violation occurred, focusing instead on the possible violation inherent in excessive systemic risks and a shortage of adequate safeguards.}

Indeed, the case of {\it Hutten-Czapska} appears to be suggestive of a move towards such a perspective. While the Court went into great detail about the facts of the case, it {\it also} looked at the case from an alternative perspective, more in line with the suggestion sketched above. In fact, I think it is likely that the Court will eventually veer even more towards such an approach, while deferring to national judicial bodies when it comes to concrete factual assessments. If not as a result of policy, I imagine this will happen from necessity, due to the limited capacity of the Court to hear the merits of individual cases.

The proportionality doctrine could still be applied, but approached in more abstract terms as the question of what kinds of rules, and what kinds of institutions, member states need to put in place to ensure fairness. \noo{ In \textcite{hutten06}, the Court moved in this direction, especially when it explained the basic principle as follows:

\begin{quote}
In assessing compliance with Article 1 of Protocol No. 1, the Court must make an overall examination of the various interests in issue, bearing in mind that the Convention is intended to safeguard rights that are “practical and effective”. It must look behind appearances and investigate the realities of the situation complained of. In cases concerning the operation of wide-ranging housing legislation, that assessment may involve not only the conditions for reducing the rent received by individual landlords and the extent of the State’s interference with freedom of contract and contractual relations in the lease market, but also the existence of procedural and other safeguards ensuring that the operation of the system and its impact on a landlord’s property rights are neither arbitrary nor unforeseeable. Uncertainty – be it legislative, administrative or arising from practices applied by the authorities – is a factor to be taken into account in assessing the State’s conduct. Indeed, where an issue in the general interest is at stake, it is incumbent on the public authorities to act in good time, in an appropriate and consistent manner.\footcite[151]{hutten06} 
\end{quote}

I note how the Court builds on the earlier precedent set by cases such as \textcite{sporrong82} and \textcite{james86}. The first half of the quote, therefore, stresses that the Court itself must ``look to the realities of the situation''. However, in clarifying what is meant by this, the Court goes on to emphasise procedural aspects. In particular, it is made clear that the Court regards such aspects as an integral part of those ``realities'' that need to be assessed. Indeed, the Court even makes specific reference to the importance of several values that arise in the context of administrative law, such as predictability and effectiveness.
}
This perspective appears to have been adopted in the case of {\it Lindheim and others v Norway}. Here the applicants complained that their rights had been violated by a recent Norwegian act that gave lessees the right to demand indefinite extensions of ground leases on pre-existing conditions.\footcite[119]{lindheim12} In the end, the Court concluded that there had indeed been a breach of P1(1). They engaged in the same form of assessment that they had adopted in \textcite{hutten06}. Moreover, they concluded that the Ground Lease Act 1996 as such  was the underlying source of the violation -- the problem was not merely that this act had been applied in a way that offended the rights of the applicants. In light of this, the Court did not only award compensation, it also ordered that general measures had to be taken by the Norwegian state to address the structural shortcomings that had been identified.

The Court also commented that its decision should be regarded in light of ``jurisprudential developments in the direction of a stronger protection under Article 1 of Protocol No. 1''.\footcite[135]{lindheim12} However, in light of the change in perspective that accompanies this development, it is interesting to ask in what sense the protection is stronger. In particular, it is not {\it prima facie} clear that the Court's remark should be read as a statement expressing a change in its understanding of the content of individual rights under P1(1). Rather, it may be read  as a statement to the effect that the Court now assumes it has greater authority to address structural problems under that provision. In effect, this allows the Court to conclude that a violation has occurred due to structural unfairness, even when it is not possible to trace this back to any flawed decision that specifically targets the applicants.

\subsection{How Would the ECtHR Approach an Economic Development Taking?}

Is the jurisprudential developments illustrated by the rent control cases relevant to the issue of economic development takings? I believe so. Indeed, I am struck by how the reasoning of the ECtHR in recent cases on hunting and rent control mirrors the kind of reasoning that Justice O'Connor engaged in when considering {\it Kelo}.\footnote{See \cite{kelo05}.} The emphasis is on structural aspects and fairness, grounded on the facts of the concrete case, but mainly interested in what these facts reveal about the rules and procedures involved. 

This is a contextual approach that can maintain a broad focus without loosing its bite. The crux of arguments used to conclude violation is the observation that the system currently in place can offend against the role that owners {\it should} occupy in order to be able to meet those obligations and exercise those freedoms that are attached to the properties they posses.

On this narrative, interference becomes illegitimate when it demonstrates a failure of governance. In the case of \textcite{hutten06}, this boiled down to the observation that it was illegitimate to address problems in the Polish housing sector by placing the burden ``on one particular social group'', namely the owners.\footcite[225]{hutten06} This conclusion was backed up by the concrete observation that the rules and procedures in place meant that owners who were obliged to maintain their properties in good condition for their tenants were in fact prevented from doing so because they were not permitted to charge rents that would cover the costs.

In the case of {\it Kelo}, Justice O'Connor argued in a similar fashion when she concluded that the system which had led to the decision to condemn Suzanne Kelo's house was likely to function so as to systematically ``transfer property from those with fewer resources to those with more''. To Justice O'Connor, there was little doubt that this could become a general pattern, if safeguards were not put in place. %Indeed, it would have to be assumed that a multi-million dollar company would always be in a better position than a homeowner when arguing that  ``economic development'' would result from their ownership. \noo{More subtly, her opinion also hinted at the inconsistency involved in asserting abstractly that economic development would benefit the community indirectly, all the while the development would \isr{in} fact require razing it.}

To conclude, I think the ECtHR would have been likely to approach a case like {\it Kelo} in a manner consistent with Justice O'Connor's approach. Whether they would reach the same conclusion seems more uncertain, particularly since confidence in the nation states' ability and willingness to regulate private-public partnerships might be higher in Europe.\footnote{For a discussion from the point of view of English law, arguing that the prevailing regulatory regime limits the risk of eminent domain abuse largely through regulation of the takings power rather than strict property protection, see \cite{allen08}.} However, it seems unlikely that the ECtHR would follow the majority in {\it Kelo}, by simply deferring to the determinations made by the granting authority. Moreover, with the recent change in perspective towards structural assessment of property institutions, Justice O'Connor's predictions about the ``fallout'' of the {\it Kelo} decision would likely have been of significant interest to the justices at the Court in Strasbourg.

\section{The US Perspective on Economic Development Takings}\label{sec:us}

In this section, I consider US law in more depth. First, I track the development of the case law on the public use restriction found in the Fifth Amendment and in various state constitutions. I consider the jurisprudential development from the early 19th century up to the present day.\footnote{The public use clause in the US constitution was not held to apply to state takings until the late 19th century, see \cite{chicago97}.} Many writers assert that case law from the 19th and early 20th century was \isr{characterised} by a tension between `narrow' and `broad' readings of the notion of public use.\footnote{See, e.g., \cite[483]{walt11}; \cite[203-204]{allen00}. For a more in-depth argument asserting the same, see \cite{nichols40}.} Adding to this, I argue that while different state courts expressed different theoretical views on the meaning of ``public use'', there was a growing consensus that the approach to judicial scrutiny should be contextual, focused on weighing the rationale of the taking against the social, political and economic circumstances.\footnote{A summary of state case law that supports this view is given in the little discussed Supreme Court case of \cite{hairston08}.}  In particular, early state courts did not focus unduly on the exact wording of constitutional property clauses.

Following up on this, I argue that the doctrine of deference that was developed by the Supreme Court early in the 20th century was directed primarily at state courts, not state legislatures and administrative bodies.\footnote{See \cite{vester30} (echoing and citing \cite{hairston08}).} I then present the case of {\it Berman}, arguing that it was a significant departure from previous case law.\footcite{berman54} After {\it Berman}, deference was now taken to mean deference to the (state) legislature, meaning that there would be little or no room for judicial review of the takings purpose. 

This paved the way for the infamous case of {\it Poletown}, where a \isr{neighbourhood} of about 1000 homes was razed in order to provide General Motors with land to build a car factory.\footnote{See \cite{poletown81}.} I note how {\it Berman} provided a key authority used by the state court to uphold this taking. {\it Poletown} in turn links up with the even greater controversy surrounding {\it Kelo},  the eventual backlash of the deferential stance introduced in {\it Berman}.

After the historical overview, I go on to briefly present the vast amount of research that has targeted economic takings in the US after {\it Kelo}. I devote special attention to proposals for new legitimacy-enhancing institutions for facilitating economic development of jointly owned land. I focus on two suggestions in particular: targeting compensation and participation.\footcite{lehavi07,heller08} These proposals will serve as important reference points later on, when I consider the Norwegian land consolidation courts in Chapters 6.

\subsection{The History of the Public Use Restriction}\label{sec:hop}

Going back to the time when the Fifth Amendment was introduced, there is not much historical evidence explaining why the takings clause was included in the Bill of Rights.\footnote{See \cite{fifth}.} Moreover, there is little in the way of guidance as to how the takings clause was originally understood. James Madison, who drafted it, commented that his proposals for constitutional amendments were intended to be uncontroversial.\footnote{See letters from Madison to Edmund Randolph dated 15 June 1789 and from Madison to Thomas Jefferson dated 20 June 1789, both included in \cite{madison79}.} Hence, it is natural to regard the property clause as a codification of an existing principle, not a novel proposal. Indeed, several state constitutions pre-dating the Bill of Rights also included takings clauses, seemingly based on codifying principles from English common law.\footcite[See][299]{johnson11} As Meidinger notes, the Americans had never really charged the British with abuse of eminent domain, and private property had tended to be respected, also in the colonies.\footcite[17]{meidinger80} This undoubtedly influenced early US law.

Just like English scholars at the time, early American scholars emphasised the importance of private property. For instance, in his famous {\it Commentaries}, James Kent described the sense of property as ``graciously implanted in the human breast'' and declared that the right of acquisition ``ought to be sacredly protected''.\footnote{See \cite[see][257]{kent27}.} Indeed, the Supreme Court itself expressed similar sentiments early on, when it spoke of the impossibility of passing a law that ``takes property from A and gives it to B''.\footnote{This was a {\it de dicta} in \cite[388]{calder98}. See also \cite[310]{vanhorne95}.}

However, just as would happen in England, this early US attitude would soon change in response to industrial advances and a desire for economic development. As the 19th century progressed, eminent domain was used more frequently, now also to benefit (privately operated) railroad operations, hydroelectric projects, and the mining industry.\footcite[23-33]{meidinger80} During this time, it also became increasingly common for landowners to challenge the legitimacy of takings in court, undoubtedly a consequence of the fact that eminent domain was used more widely, for new kinds of projects.\footcite[24]{meidinger80} Controversy arose particularly often with respect to the so-called mill acts.\footnote{\cite[24]{meidinger80}. See also \cite[306-313]{johnson11} and \cite[251-252]{horwitz73}.} Such acts were found throughout the US, many of them dating from pre-industrial times when mills were primarily used to serve the farming needs of agrarian communities.\footnote{A total of 29 states had passed mill acts, with 27 still in force, when a list of such acts was compiled in \cite[17]{head85}. According to Justice Gray, at pages 18-19 in the same, the ``principal objects'' for early mill acts had been grist mills typically serving local agrarian needs at tolls fixed by law, a purpose which was generally accepted to ensure that they were for public use.} Following economic and technological advances, provisions originally enacted to serve local farming purposes were now being used by developers wishing to harness hydropower for manufacturing and hydroelectric plants.\footnote{See, e.g., \cite[18-21]{head85} and \cite[449-452]{minn06}.}

It is important to note, however, that mill acts could not be used to  authorise large-scale compulsory transfer of natural resources from owners to non-owners. Rather, mill acts provided management tools that could be used to ensure that owners of water resources could make better use of their rights. This would sometimes involve allowing riparian owners to interfere with, or take a necessary part of, the property of their neighbours, e.g., by constructing dams that would flood neighbouring land.\footnote{See \cite{head85} (a mill case adjudicated by the Supreme Court, including a summary of mill acts and case law from various states). See also \cite[265]{staples03}.} However, the primary purpose of most mill acts was to facilitate rational coordination among owners, to the benefit of their community as a whole. This point was frequently made by the courts to justify upholding takings on the basis of mill acts, including takings that would benefit the manufacturing industry.\footnote{See \cite{fiske31}. See also the discussion (including references to other cases) in \cite{head85}.}

As the industrial use of mill acts increased in scope, the original aim of these acts gradually became overshadowed by the strength of the commercial interests involved, leading to public use controversy relating to provisions that had not previously raised any such doubts.\footnote{See \cite{head86}.} This mechanism, deeply dependant on the social and economic context, underscores the appropriateness of adopting a social function perspective on the relevant body of case law. More generally, it seems that most of the early case law on the public use test from US state courts is characterised by a contextual understanding of property protection. In the following, I explore this in some further detail.

\subsection{Legitimacy in State Courts}\label{subsec:state}

When considering objections to the legitimacy of takings, state courts would not look to the federal Takings Clause directly, but rather base their decisions on corresponding property clauses from their own respective state constitutions.\footnote{Not all states had such property clauses, and exact formulations varied, but a public use requirement was typically observed, see \cite[293-296]{johnson11}.} Indeed, it was not until the late 19th century that state takings came to be regularly scrutinized at the federal level.\footnote{At first, the federal scrutiny took place on the basis of the due process clause in the Fourteenth Amendment, see \cite{head75}. The federal takings clause itself was only applied to state takings after 1897, see \cite{chicago97}.}

When a state court upheld an interference that would benefit commercial interests, it would typically emphasise the broader purpose, often focusing on economic ripple effects.\footnote{See, e.g., \cite{hazen53,scudder32,boston32}. A more comprehensive list of cases adopting a broad view can be found in \cite[617]{nichols40}.} By contrast, when a court decided that an interference was unconstitutional, it would often focus on the concrete use made of the property that was taken,  pointing out that it did not directly benefit the public in the sense required by the public use restriction.\footnote{See, e.g., \cite{sadler59,ryerson77,gaylord03,minn06}. A more comprehensive list can be found in \cite{alr28}.} Sometimes, the question of legitimacy would turn on how widely the notion of `use' was understood. Should this notion be interpreted narrowly, as requiring that the property had to be literally used by the public, or could it be understood broadly, as pointing to a public purpose or benefit of some sort?\footnote{According to Nichols, the narrow view emerged as the ``majority'' opinion on public use, see \footcite[617-618]{nichols40}. But contrast this with \cite{berger78} and \cite[24]{meidinger80}, who argue that the narrow view was only dominant in a handful of states, led by New York.}

This tension between broad and narrow readings of the public use clause have received much attention from legal scholarship.\footnote{See \cite{nichols40,berger78,meidinger80,johnson11}.} However, when studying the case law in more depth, a complementary picture emerges, testifying to some cohesion in the states'
jurisprudence. Regardless of their reading of the public use requirement, state courts seem to have agreed that the question of what counted as a public use was a judicial question that should be assessed concretely, not abstractly.

%For instance, in the case of {\it Gaylord v. Sanitary Dist. of Chicago}, the Supreme Court of Illinois held the state Mill Act to be unconstitutional, as it was not limited to traditional flour mills. In doing so, the court observed that public use was ``something more than a mere benefit to the public''.\footcite[524]{gaylord03} Similar sentiments were expressed in other decisions striking down uses of eminent domain for mill construction, for instance in Vermont, Michigan and New York.\footnote{References.}

A good example is the case of {\it Dayton Gold \& Silver Mining Co v Seawell}, concerning an act that gave mineral owners a right to acquire additional rights needed to facilitate extraction.\footcite{seawell76} The Supreme Court of Nevada decided that the act was constitutional on the basis of a highly contextual reading of the public use requirement in the property clause of the Constitution of Nevada. Interestingly, the Court argued against a literal (narrow) reading on the basis that such a reading would ultimately provide {\it weaker} protection of property:

\begin{quote}
If public occupation and enjoyment of the object for which land is to be condemned furnishes the only and true test for the right of eminent domain, then the legislature would certainly have the constitutional authority to condemn the lands of any private citizen for the purpose of building hotels and theaters. [...] Stage coaches and city hacks would also be proper objects for the legislature to make provision for, for these vehicles can, at any time, be used by the public upon paying a stipulated compensation. It is certain that this view, if literally carried out to the utmost extent, would lead to very absurd results, if it did not entirely destroy the security of the private rights of individuals. Now while it may be admitted that hotels, theaters, stage coaches, and city hacks, are a benefit to the public, it does not, by any means, necessarily follow that the right of eminent domain can be exercised in their favor.\footcite[410-411]{seawell76}
\end{quote}

The quote presents an argument in favour of a broad understanding of the public use requirement. However, it also prescribes broad judicial review of takings purposes, including purposes that would appear to pass a `narrow' public use test. In this way, it asks us to  resist the temptation to think that a broad understanding of public use necessarily entails a public use test that can be passed more easily.

The Court follows up on its reading of the public use requirement by giving a highly contextual assessment of the takings purpose. Specifically, it considers the social and economic importance of mining, concluding that it is the ``greatest of the industrial pursuits'' and that all other interests are ``subservient'' to it.\footcite[409]{seawell76} Indeed, the Court goes as far as to conclude that the benefits of the mining industry are ``distributed as much, and sometimes more, among the laboring classes than with the owners of the mines and mills''.\footcite[409]{seawell76} On this basis, the Court upholds the taking.

I am agnostic as to whether or not this decision was based on an accurate description of the mining industry in Nevada in the late 19th century. The importance of the decision and the remarks above does not turn on this factual question. Rather, the importance arises from the fact that the Court felt the need to scrutinize the takings purpose very broadly. The issue of legitimacy was not approached as a linguistic exercise or an attempt at recreating the original intent of the relevant property clause. Instead, the court proceeded on the basis of their assessment of the prevailing social and economic conditions in the state of Nevada. 

The Court noted the importance of deference to the legislature on matters of policy, but qualified this by remarking that any authority to take property had to be ``enforced by the courts so as to prevent its being used as an instrument of oppression to any one''.\footcite[412]{seawell76} 
Furthermore, the Court was convinced that its contextual approach in this regard would generally offer {\it increased} protection of private property compared to more formalistic approaches. The Court summarised its view on this as follows:

\begin{quote}
Each case when presented must stand or fall upon its own merits, or want of merits. But the danger of an improper invasion of private rights is not, in my judgment, as great by following the construction we have given to the constitution as by a strict adherence to the principles contended for by respondent.\footcite[398]{seawell76}
\end{quote}

The {\it Seawell} case is not unique. For another example, I mention {\it Ryerson v Brown}, a case often cited as an authority for a narrow view of public use.\footcite{ryerson77} Here the taking in question was held to be unconstitutional. However, the Supreme Court of Michigan qualified also made clear that it was ``not disposed to say that incidental benefit to the public could not under any circumstances justify an exercise of the right of eminent domain''.\footcite[337]{ryerson77}

The case concerned the constitutionality of a taking under a mill act, and while the court argues that public use should be taken to mean ``use in fact'', it is clear that ``use'' is understood rather loosely, not literally as physical use of the property that is taken.\footnote{The court explains its stance on the public use restriction by stating (emphasis added) ``it would be essential that the statute should require the use to be public in fact; in other words, that it should contain provisions entitling the public to {\it accommodations}.'' The court continues with an illustrative example: ``A flouring mill in this state may grind exclusively the wheat of Wisconsin, and sell the product exclusively in Europe; and it is manifest that in such a case the proprietor can have no valid claim to the interposition of the law to compel his \isr{neighbour} to sell a business site to him, any more than could the manufacturer of shoes or the retailer of groceries. Indeed the two last named would have far higher claims, for they would subserve actual needs, while the former would at most only incidentally benefit the locality by furnishing employment and adding to the local trade''. See \cite[336]{ryerson77}.} Moreover, when clarifying its starting point for judicial scrutiny, the court explains that ``in considering whether any public policy is to be subserved by such statutes, it is important to consider the subject from the standpoint of each of the parties''. Following up on this, the court finds, with respect to the case in question, that ``the power to make compulsory appropriation, if admitted, might be exercised under circumstances when the general voice of the people immediately concerned would condemn it''. On thi basis, the Court strikes down the taking, summing up its factual assessment as follows: ``what seems conclusive to our minds is the fact that the questions involved are questions not of necessity, but of profit and relative convenience''.\footcite[336]{ryerson77}

Hence, far from nitpicking on the basis of the public use phrase, the court adopts a contextual approach to takings that is rather similar to the approach of {\it Dayton Gold \& Silver Mining Co. v. Seawell}. The outcome \isr{is} different, but it is also based on a different assessment of the context and the consequences of the takings complained about. Importantly, the case does not rest on any {\it a priori} assumption that economic development takings of the kind in question could not meet a public use test -- no general rule is relied on at all. Hence, it is somewhat strange that later commentators have focused on the case for its ambiguous comments on public use as ``public in fact'' rather than its broad and well-reasoned assessment of actual legitimacy.\footnote{See, for instance, Justice Thomas' dissent in {\it Kelo}, \cite[513]{kelo05} (using {\it Ryerson} as a reference to support an `actual use' interpretation of the public use requirement in the fifth amendment).}

Many of the important cases from the late 19th century, on both sides of the public use debate, share crucial features with the two cases discussed above.\footnote{See, e.g., \cite{scudder32} (Eminent domain power upheld, but said: ``The great principle remains that there must be a public use or benefit. That is indispensable. But what that shall consist of, or how extensive it shall be to authorize an appropriation of private property, is not easily reducible to a general rule. What may be considered a public use may depend somewhat on the situation and wants of the community for the time being.''), \cite{fallsburg03} (Eminent domain struck down, on holding that ``the private benefit too clearly dominates the public interest to find constitutional authority for the exercise of the power of eminent domain''), \cite[538]{board91} (Eminent domain struck down, qualified by ``not only must the purpose be one in which the public has an interest, but the state must have a voice in the manner in which the public may avail itself of that use'').} Hence, a shared trait appears to have emerged among state courts during this period, namely a willingness to engage in broad judicial scrutiny of the legitimacy of economic development takings. Indeed, state courts appear to have been conscious of the special legitimacy questions that arise when eminent domain is used to facilitate economic development through commercial enterprise. The question of how to understand public use terminology was an important part of this, but it was not considered in isolation from other aspects.

This observation is relevant when considering the takings doctrine that later developed at the federal level. In particular, the broad scrutiny offered by state courts suggests that the doctrine of extreme deference that was about to be adopted by the Supreme Court resulted from a completely new development, not a continuous broadening of the public use requirement.\footnote{This contrasts with the argument given by the majority in {\it Kelo}, see \cite[479-480]{kelo05} (placing the doctrine of deference in a tradition emerging from how the narrow view of some early state courts ``steadily eroded'' because of the ``diverse and always evolving needs of society'').}

\subsection{Legitimacy as Discussed by the Supreme Court}\label{subsec:US}

Initially, the Supreme Court held that the takings clause in the US Constitution did not apply to state takings at all.\footcite{barron33} Federal takings, on the other hand, were of limited practical significance since the common practice was that the federal government would rely on the states to condemn property on its behalf.\footcite[30]{meidinger80}

This changed towards the end of the 19th century, particularly following the decision in {\it Trombley v Humphrey}, where the Supreme Court of Michigan struck down a taking that would benefit the federal government.\footcite{trombley71} Not long after, in 1875, the first Supreme Court adjudication of a federal taking occurred, marking the start of the development of the federal doctrine on public use and legitimacy.\footcite{kohl75} 

At the same time, the Supreme Court began to hear takings cases originating from the states, first on the basis of the due process clause of the fourteenth amendment, introduced after the civil war.\footnote{See, e.g, \cite{head85}.} Later, in 1897, the Supreme Court held that state takings could be scrutinized also against the takings clause of the fifth amendment.\footnote{See \cite{chicago97}.}

The early 20th century was a period of great optimism about the ability of {\it laissez faire} capitalism to ensure progress and economic growth, a sentiment that was reflected in the federal case law on eminent domain. A particularly clear expression of this can be found in {\it Mt Vernon-Woodberry Cotton Duck Co v Alabama Interstate Power Co}.\footcite{vernon16}  This case dealt with the legitimacy of condemnation arising from the construction of a hydropower plant. The Supreme Court held that it was legitimate, with the presiding judge arguing briskly as follows:

\begin{quote}The principal argument presented that is open here, is that the purpose of the condemnation is not a public one. The purpose of the Power Company's incorporation, and that for which it seeks to condemn property of the plaintiff in error, is to manufacture, supply, and sell to the public, power produced by water as a motive force. In the organic relations of modern society it may sometimes be hard to draw the line that is supposed to limit the authority of the legislature to exercise or delegate the power of eminent domain. But to gather the streams from waste and to draw from them energy, labor without brains, and so to save mankind from toil that it can be spared, is to supply what, next to intellect, is the very foundation of all our achievements and all our welfare. If that purpose is not public, we should be at a loss to say what is. The inadequacy of use by the general public as a universal test is established. The respect due to the judgment of the state would have great weight if there were a doubt. But there is none.\footcite[32]{vernon16}
\end{quote}

On the one hand, the Court notes the importance of deference to the {\it state} judgement (not specifically the judgement of the state legislature). On the other hand, it prefers to conclude on the basis of its own assessment of the purpose of the taking. This assessment, however, is not grounded in the facts of the case or the circumstances in Alabama. Rather, it is based on sweeping assertions about ``all our welfare'' and the desire to ``save mankind from toil that it can be spared''. This marks a contrast with the approach of state courts, as discussed in the previous subsection.

The contrast was even greater in cases when the takings in question had been authorised by the federal government itself. In such cases, the Supreme Court showed little willingness to subject takings purposes to public use scrutiny. In {\it United States v Gettysburg Electric Railway Co}, a case from 1896, deference to the legislature in federal takings cases was referred to as a principle that should be observed unless the judgement of the legislature was ``palpably without reasonable foundation''.\footcite[680]{gettysburg96} 

However, such a deferential stance was not adopted in cases originating from the states. In {\it Cincinatti v Vester}, a case from 1930, the Supreme Court commented that ``it is well established that, in considering the application of the Fourteenth Amendment to cases of expropriation of private property, the question what is a public use is a judicial one''.\footcite[447]{vester30} In this judgement, Chief Justice Hughes also describes how the judicial assessment of the public use question should be carried out:

\begin{quote}
In deciding such a question, the Court has appropriate regard to the diversity of local conditions and considers with great respect legislative declarations and in particular the judgments of state courts as to the uses considered to be public in the light of local exigencies. But the question remains a judicial one which this Court must decide in performing its duty of enforcing the provisions of the Federal Constitution.\footcite[447]{vester30}
\end{quote}

Notice how this echoes the contextual approach developed at the state level, while explicitly prescribing deference to state {\it courts}. In the earlier case of {\it Hairston v Danville \& W R Co}, from 1908, the same idea was expressed by Justice Moody, who surveyed the state case law and declared that ``the one and only principle in which all courts seem to agree is that the nature of the uses, whether public or private, is ultimately a judicial question.''\footcite[606]{hairston08} Justice Moody continued by describing in more depth the typical approach of the state courts in determining public use cases:

\begin{quote}
The determination of this question by the courts has been influenced in the different states by considerations touching the resources, the capacity of the soil, the relative importance of industries to the general public welfare, and the long-established methods and habits of the people. In all these respects conditions vary so much in the states and territories of the Union that different results might well be expected.\footcite[606]{hairston08}
\end{quote}

Justice Moody goes on to give a long list of cases illustrating this aspect of state case law, showing how assessments of the public use issue is inherently contextual.\footcite[607]{hairston08} %He then cites three further Supreme Court cases, pointing out that all of them express support for state case law on this issue.\footnote{{\it Falbrook, Clark} and {\it Strickley}.} 
Following up on this, he points out that ``no case is recalled'' in which the Supreme Court overturned ``a taking upheld by the state {\it court} as a taking for public uses in conformity with its laws'' (my emphasis). After making clear that situations might still arise where the Supreme Court would not follow state courts on the public use issue, Justice Moody goes on to conclude that the cases cited ``show how greatly we have deferred to the opinions of the state courts on this subject, which so closely concerns the welfare of their people''.\footcite[606]{hairston08}

{\it Hairston} is important for three reasons. First, it makes clear that initially, the deferential stance in cases dealing with state takings was primarily directed at state courts rather than legislatures and administrative bodies. Second, it demonstrates federal recognition of the fact that a consensus had emerged in the states, whereby scrutiny of the public use determination was consistently regarded as a judicial task.\footnote{Indeed, {\it Hariston} provides the authority for {\it Vester} on this point. See \cite[606]{vester30}.} Thirdly, it provides a valuable summary of the contextual approach to the public use test that had developed at the state level. 

The {\it Hairston} Court clearly looked favourably on the case law from state courts. Indeed, the judicial scrutiny provided by state courts was held to be of such high quality that there was in general little need for federal intervention. Hence, when a deferential stance was adopted in {\it Hairston}, this was contingent on the fact that state courts would continue to administer the required public use test.

Despite this, {\it Hairston} would later be cited as an early authority in favour of almost unconditional deference.\footnote{In fact, it was cited in this way also by the majority in {\it Kelo}, see \cite[482-483]{kelo05}.} This happened in {\it US ex rel Tenn Valley Authority v Welch}, concerning a federal taking.\footcite[552]{welch46} The Court first cited {\it US v Gettysburg Electric R Co} as an authority in favour of deference with regards to the public use limitation.\footcite{gettysburg96} The Court then paused to note that {\it Vester} later relied on the opposite view, namely that the public use test was a judicial responsibility.\footcite{vester30} The Court then attempts to undercut this by setting up a contrast between {\it Vester} and {\it Hairston}, by selectively quoting the observation made in the latter case that the Supreme Court had never overruled the state courts on the public use issue.\footnote{See \cite[552]{welch46}.} Hence, {\it Hairston} is effectively used to argue against judicial scrutiny, in a manner that is quite incommensurate with the full rationale behind the Court's decision in that case.

Later, {\it Welch} was used as an authority in the case of {\it Berman v Parker}.\footcite{berman54} This case concerned condemnation for redevelopment of a partly blighted residential area in the District of Colombia, which would also condemn a non-blighted department store. In a key passage, the Court states that the role of the judiciary in scrutinizing the public purpose of a taking is ``extremely narrow''.\footcite[32]{berman54} The Court provides only two references to previous cases to back up this claim, one of them being {\it Welch}.\footnote{The other case, {\it Old Dominion Land Co v US}, concerned a federal taking of land on which the military had already invested large sums in buildings. The Court commented on the public use test by saying that ``there is nothing shown in the intentions or transactions of subordinates that is sufficient to overcome the declaration by Congress of what it had in mind. Its decision is entitled to deference until it is shown to involve an impossibility. But the military purposes mentioned at least may have been entertained and they clearly were for a public use''. See \cite[66]{dominion25} A misleading and partial quote, to the effect that deference to the legislature is in order except when it involves an ``impossibility'', has since become commonplace. In particular, such a quote was repeated by the Supreme Court itself in the later case of \cite[240]{midkiff84}.}

Moreover, both of the cases cited were concerned with federal takings, while in {\it Berman} the Court explicitly says that deference is due in equal measure to the state legislature.\footcite[32]{berman54} It is possible to regard this merely as a {\it dictum}, since the District of Columbia is governed directly by Congress. However, {\it Berman} was to have a great impact on future cases. In effect, it undermined a large body of case law on judicial scrutiny of taking purposes without engaging with it at all.

In {\it Hawaii Housing Authority v Midkiff}, the Supreme Court further entrenched the principle expressed in {\it Berman}.\footcite{midkiff84} Here the state of Hawaii had made use of eminent domain  to break up an oligopoly in the housing sector. Given the circumstances of the case, it would have been natural to argue in favour of this taking on the basis that it served a proper public purpose.

However, the Court instead decided to rely on the doctrine of deference, shunning away from scrutinizing the takings purpose. Justice O'Connor, in particular, observed that ``judicial deference is required because, in our system of government, legislatures are better able to assess what public purposes should be advanced by an exercise of eminent domain''.\footcite[244]{hawaii84}

The formulation here is slightly less absolute than that given in {\it Berman}. In particular, the deferential stance is not presented as a system imperative, but rather made contingent on the fact that legislatures are ``better able'' to assess what counts as a public purpose. Moreover, Justice O'Connor also actively refers to the merits of the taking, especially when she points out that  ``regulating oligopoly and the evils associated with it is a classic exercise of a State's police powers''.\footcite[242]{hawaii84}

Despite these nuances, {\it Midkiff} reaffirmed the main principle expressed in {\it Berman}, namely that the meaning of public use is a matter for legislatures and that the room for judicial review is narrow. In light of this, it is easy to understand why {\it Kelo} was decided in favour of the taker. It would have been a clear break with earlier precedent on the public use restriction if the Supreme Court had chosen to decide otherwise. 

Formally, the case law on the federal takings clause is not binding on state courts when they assess cases against their own constitutions.\footnote{See \cite[95]{merrill86}.} Moreover, as Merrill notes, state courts have not uniformly responded by embracing deference towards their own legislatures.\footcite[65]{merrill86} Rather, many state courts continued to offer scrutiny of taking purposes, despite the signals coming from the federal level.\footcite[65]{merrill86}

It should be noted, however, that the time after {\it Berman} was also a time when many government bodies throughout the US would actively seek to condemn homes for redevelopment projects, to combat ``blight'', but often also to the benefit of commercial enterprises.\footnote{See generally \cite{pritchett03}.} Hence, continued public use scrutiny at state courts might also reflect an increased threat of eminent domain abuse. Sometimes, moreover, state courts seems to have failed in their duty to offer appropriate protection.

The case of {\it Poletown Neighborhood Council v City of Detroit} is a classic example.\footcite{poletown81} In this case, the Michigan Supreme Court held that it was not in violation of the public use requirement in the Michigan Constitution to allow General Motors to displace some 3500 people for the construction of a car assembly factory. The majority 5-2 cites {\it Berman}, commenting that the state court's room for review of the public use requirement is similarly limited.\footcite[632-633]{poletown81}

The {\it Poletown} decision was controversial, and the minority, especially Justice Ryan, was highly critical of it. He objects both to the deferential stance in general and to the majority reading of {\it Berman} in particular, pointing out that the Supreme Court's doctrine of deference outside the context of federal takings was  directed at the state courts, not state legislatures.\footcite[668]{poletown81} Hence, as he concludes, the majority's reliance on {\it Berman} was ``particularly disingenuous''.\footcite[668]{poletown81} 

Justice Ryan was not alone in his disapproval of {\it Poletown}.\footnote{Indeed, the decision would later be overturned by the Supreme Court of Michigan itself, see generally \cite{sandefur05}.} Moreover, the case is widely regarded as the prelude to an era of increased tensions over economic development takings in the US.\footnote{See \cite[664-668]{sandefur05}.} This would culminate with {\it Kelo} which, despite upholding and strengthening the deferential doctrine, also inadvertently caused a shift towards stricter public use scrutiny at the state level, as discussed in the following subsection.

\subsection{Economic Development Takings after {\it Kelo}}\label{sec:postkelo}

The fact that {\it Kelo} was decided against the homeowner met with wide disapproval among the public.\footnote{See \cite[2109]{somin09}.} In addition, many scholars expressed concern that the deferential approach had been taken too far, and that economic development takings such as {\it Kelo} were in need of more substantive public use scrutiny by courts.\footnote{For a small sample, see \cite{cohen06,underkuffler06,sandefur06,somin07,gisler10}.}
Moreover, following {\it Kelo}, much attention was directed at the perceived dangers of eminent domain abuse in the US.\footnote{See generally \cite{somin09}.} %The minority opinions given in {\it Kelo}, particularly the opinion of Justice O'Connor, proved influential, causing further attention to be directed at the perceived dangers of eminent domain abuse. A massive amount of literature has since appeared devoted to studying  economic development takings. 

Many states responded by introducing reforms aimed at limiting the use of eminent domain for economic development.\footnote{For an overview and critical examination of the myriad of state reforms that have followed {\it Kelo}, I point to \cite{eagle08}. See also \cite{somin09}.} Within two years, 44 states had passed post-{\it Kelo} legislation in an attempt to achieve this.\footnote{See \cite{castle}.} Various legislative techniques were adopted. Some states, including Alabama, Colorado and Michigan, enacted explicit bans on economic development takings and takings that would benefit private parties.\footcite[See][107-108]{eagle08} In South Dakota, the legislature went even further, banning the use of eminent domain ``(1) For transfer to any private person, nongovernmental entity, or other public-private business entity; or (2) Primarily for enhancement of tax revenue''.\footnote{South Dakota Codified Laws § 11-7-22-1, amended by House Bill 1080, 2006 Leg, Reg Ses (2006).}

In other states, more indirect measures were taken, such as in Florida, where the legislature enacted a rule whereby property taken by the government could not be transferred to a private party until 10 years after the date it was condemned.\footcite[809]{eagle08} Many states also offered lengthy lists of uses that were to count as public, designed to restrict the room for administrative discretion while allowing condemnations for purposes that were regarded as particularly important.\footcite[804]{eagle08}

Somin points to an interesting trend, namely that state reforms enacted by the public through referendums tend to be more restrictive than reforms passed through the state legislature.\footcite[2143]{somin09} Many of the more radical reform proposals, moreover, did not emerge from the state government, but were initiated by activist groups as ballot measures. In some US states, initiative processes make it possible for activist groups to put measures on the ballot without prior approval by the state legislature.\footnote{See \cite[2148]{somin09}.} As Somin observes, the reforms taking place via this route would be comparatively strict, testifying to the power of direct democracy.\footnote{See \cite[2143-2149]{somin09}.}

Indeed, the successes of popular anti-takings movements  underscores how strongly the US public opposed the decision in {\it Kelo}. Surveys show that as many as 80-90 \% believe that it was wrongly decided, an opinion widely shared also among the political elite.\footcite[2109]{somin09} 

{\it Kelo} has clearly had a great effect on the discourse of eminent domain in the US. However, the effects of the many state reforms that have been enacted are less clear. According to Somin, most of these reforms have in fact been ineffective, despite the overwhelming popular and political opposition against economic development takings.\footcite[2170-2171]{somin09} At the same time, property lawyers report a greater feeling of unease regarding the correct way to approach the public use requirement, expressing hope that the Supreme Court will soon revisit the issue.\footnote{See \cite{murakami13} (``Until the Supreme Court revisits the issue, we predict that this question will continue to plague the lower courts, property owners, and condemning authorities'').} 

Why have legislative reforms proved inadequate and ineffective? Part of the reason, according to Somin, is that people are ``rationally ignorant'' about the economic takings issue.\footnote{See \cite[2170]{somin09}.} For most people, it is unlikely that eminent domain will come to concern them personally or that they will be able to influence policy in this area. Hence, it makes little sense for them to devote much time to learn more about it. This, in turn, helps create a situation where experts can develop and sustain a system based on practices that a majority of citizens actually oppose.\footcite[2163-2171]{somin09} Indeed, Somin argues that surveys show how people tend to overestimate the effectiveness of eminent domain reform, possibly due to the fact that symbolic legislative measures are mistaken for materially significant changes in the law.\footcite[2155-2157]{somin09}

I think Somin's analysis is on an interesting track. However, it should be noted that the notion of rational ignorance is a double-edged sword with regards to his main argument. In particular, it seems possible, in theory, that the prevailing critical attitude towards economic development takings is itself an instance of such ignorance. Perhaps people would change their opinion on economic development takings if they were better educated on the issue?

However, this possibility does nothing to detract from the main message, which is that the {\it Kelo} backlash have in fact caused greater insecurity about what the law is and what it delivers, often without significantly curbing those uses of eminent domain that are regarded as most problematic. Arguably, this shows that the legislative approach so far, which has focused on introducing more elaborate and detailed versions of the public use restriction, need to be supplemented by different kinds of proposals. 

In this regard, it seems important to also target governmental decision-making processes regarding the use of private land for economic development. These processes, it seems, need to be imbued with greater legitimacy. In particular, it seems crucial that owners themselves should be granted a better chance to participate in the management of their own land, even when this involves deliberating on, and possibly taking part in, large-scale development projects. After all, it is the owners' and their communities' feeling that they are being treated unfairly that tend to lie at the root of controversies surrounding takings for economic development.\footnote{For a similar perspective, see \cite{underkuffler06}.}

If improved principles of governance are put in place, this alone might be enough to restore some confidence in eminent domain as a procedure by which to implement democratically accountable decisions about land use. However, it seems that eminent domain as such might often be an unduly blunt instrument when society desires commercial development on private land. Instead, it might be possible to devise mechanisms for collective action that replaces the use of eminent domain altogether. At least, it should be possible to devise mechanisms for benefit sharing in these cases, to make the imposition of a development project appear less unfair to local owners.

In the next subsection, I will consider two proposals for reforms of this kind, both of which rely on proposing new institutions for collective action. The first specifically targets the question of benefit sharing by proposing a special negotiation mechanism for determining the level of compensation after an economic development taking. The second proposal targets the decision-making process leading to economic development through condemnation by proposing a framework for land assembly that is meant to replace the use of eminent domain in many circumstances.

\section{Institutional Proposals for Increased Legitimacy}\label{sec:ir}

In this subsection, I first present the Special Purpose Development Companies (SPDCs) proposed by Lehavi and Licht.\footcite{lehavi07} I relate this proposals to theoretical approaches to the issue of compensation, before I go on to note some shortcomings and open questions that I will later address in my case study. I then go on to consider the Land Assembly Districts (LADs) proposed by Heller and Hills.\footcite{heller08} I consider this proposal in light of the stated motivation, which is to design an effective mechanism of self-governance that can replace eminent domain in economic development cases. I present some unresolved questions and argue that there is a tension in the proposal between its narrow scope, imposed to prevent majority tyranny and other forms of abuse, and its broad goal of empowering local communities. 

\subsection{Special Purpose Development Companies}

An important distinguishing feature of economic development takings is that they give the taker an opportunity to profit commercially from the development. This may even be the primary aim of the project, with the public benefiting only indirectly through potential economic and social ripple effects. Property owners facing condemnation in such circumstances might expect to take a share in the profit resulting from the use of their land. However, in many jurisdictions, including the US, the rules used to calculate compensation prevents owners from getting any share in the commercial surplus resulting from development.\footnote{See, e.g., \cite[965-966]{fennell04}.} In particular, various {\it elimination rules} are typically in place to ensure that compensation is based entirely on the pre-project value of the land that is being taken.\footcite[See][81]{freilich06} That is, the value of the development potential itself is not to influence the compensation payment (at least not to a greater extent than it was already reflected in the value of the property prior to the development plans). The policy reasons for such rules is that they ensure that the public does not have to pay extra due to their own special want of the property. After all, this is one of the main purposes of using eminent domain in the first place: to ensure that the public does not have to pay extortionate prices for land needed for important projects. However, when the purpose of the project is itself commercial in nature, there appears to be a shortage of good policy reasons for excluding this value from consideration when compensation is calculated. This is especially true when, as in the US, compensation tends to be based on the market value of the land taken. Why should a commercial condemner's prospect of carrying out economic development with a profit be disregarded when assessing the market value? In any fair and friendly transaction among rational agents, one would expect benefit sharing in a case like this. Yet for economic development backed up by eminent domain, the application of elimination rules ensures that all the profit goes to the developer. 

Some authors have argued that failures of compensation is at the heart of the economic takings issue and that worry over the public use restriction is in large part only a response to concerns about the ``uncompensated increment'' of such takings.\footcite[See][962]{fennell04} In addition to the lack of benefit sharing, previous work has identified two further problems of compensation that also tend to become exasperated in economic development cases. First, the problem of ``subjective premium'' has been raised, pointing to the fact that property owners often value their own land higher than the market value, for personal reasons.\footcite[963]{fennell04} For instance, if a home is condemned, the homeowner will typically suffer costs not covered by market value, such as the cost of moving, including both the immediate ``objective'' logistic costs as well as more subtle costs, such as having to familiarize oneself with a new local community. Second, the problem of ``autonomy'' has been discussed, arising from the fact that an exercise of eminent domain deprives the landowner of \isr{their right to decide how to manage their} property.\footnote{Discussed in \cite[966-967]{fennell04}. For a general personhood building theory of property law, see \cite{radin93}. For a general economic theory of the subjective value of independence, see \cite{benz08}.}

In \footcite{lehavi07}, the authors propose a novel approach for addressing the ``uncompensated increment'' in economic takings cases. Their proposal is based on a new kind of structure that they dub a {\it Special Purpose Development Corporation} (SPDC). The idea is that owners affected by eminent domain will be given a choice between standard pre-project market value and shares in a special company. This company will exist only to implement a specific step in the implementation of the development project: the transaction of the land-rights. The SPDC may choose either to offer their rights on an auction or else negotiate a deal with a designated developer.\footcite[1735]{lehavi07} Hence, the idea is to ensure that the owners are paid a value that reflects the post-project value of the land, but in such a way that the holdout problem is avoided. In particular, the SPDC will have a single task: to sell the land for the highest possible price within a given time frame.\footcite[1741]{lehavi07} After the sale is completed, the SPDC will divide the proceeds as dividends and be wound up.\footcite[1741]{lehavi07}

Other suggestions have taken a more static approach to compensation reform, such as proposing to give owners a fixed premium in cases of economic development, or developing mechanisms of self-assessment to ensure that compensation is based on the true value the owner attributes to his own land.\footnote{A range of static proposals have been proposed in the literature: Merrill proposes 150 \% of market value for takings that are deemed to be ``suspect'', including takings for which the nature of the public use is unclear, see \cite[90-93]{merrill86}. Krier and Serkin propose a system that provide compensation for a property's special suitability to its owner, or a system where compensation is based on the court's assessment of post-project value, see \cite[865-873]{krier04}. Fennell proposes a system of self-evaluation of property for taking purposes with tax-breaks given to those who value their property close to market value (to avoid overestimation), see \cite[995-996]{fennell04}. Bell and Parchomovsky also propose self-evaluation, but rely on a different mechanism to prevent overestimation; tax liability is based on the self-reported value and no property can be sold by its owner for less than his reported value, see \cite[890-900]{bell07}.} Compared to such proposals, the idea of SPDCs is more sophisticated and should be looked at in more depth. 

The conceptual premise for the proposal is that takings for economic development can be seen as compulsory incorporation, a pooling of resources useful in overcoming market failures.\footcite[1732-1733]{lehavi07} Just as the corporation is formed to consolidate assets in order to facilitate effective management, so is eminent domain used to assemble property rights in order to facilitate efficient organization of development. According to Lehavi and Licht, this also provides a viable approach to problems of ``opportunistic behavior''; hierarchical governance after assembly ensures that order and unity can be regained even if interests in the land are distributed among a large and heterogeneous group of potentially mischievous shareholders.\footcite[1733]{lehavi07} In the words of Lehavi and Licht:

\begin{quote}
The exercise of eminent domain powers thus resembles an incorporation by the government of all landowners with a view to \isr{bringing} all the critical assets under hierarchical governance. Establishing a corporation for this purpose and transferring land parcels to it thus would be merely a procedural manifestation of the substantive economic reality that already takes place in eminent domain cases.
\end{quote}

As soon as we look at the rationale behind economic development takings in this way, any remnant of good policy reasons for ensuring that the developer gets all the profit seems to disappear. Rather, we are led to consider compensation as an issue entirely separate from the exercise of the takings power. After the land has been \isr{reorganised} by eminent domain and an SPDC has been formed, the land rights might as well be sold {\it freely} to a developer. In this way, the land will be sold for a price that is closer to an actual market value, on the market where the land is destined for development.\footcite[1735-1736]{lehavi07} More generally, the SPDC becomes an aid that the government can use to create more \isr{favourable} market conditions for transferring land that has commercial potential in its public use. Due to the compulsory pooling of resources, no owner can exercise monopoly power by holding out, but due to decoupling of compensation from assembly, the owners can now negotiate with potential developers for a share of the resulting profit. Moreover, the fact that the SPDC offers its rights on an actual market can also help ensure that more information \isr{becomes} available regarding the true economic value of the development, something that may in turn help ensure that only the good projects will be successful in acquiring land. Hence, according to Lehavi and Licht, an additional positive effect of SPDCs is that developers and governments will \isr{shy }away from using the eminent domain power to benefit projects that are not truly welfare-enhancing.\footcite[1735-1736]{lehavi07}

In addition to these substantive consequences, the SPDC-proposal also stands out because it has a significant institutional component, pointing to its potential for restoring procedural legitimacy as well as substantive fairness. Lehavi and Licht discuss corporate governance issues at some length, but without committing themselves to definite answers about how the operations of the SPDC should be \isr{organised}.\footcite[1040-1048]{lehavi07} Indeed, while their proposal is perhaps most interesting because of its procedural aspects, it also appears to be rather preliminary in this regard. The main idea is to let the SPDC structure piggyback on existing corporative structures, particularly those developed for \isr{securitisation} of assets.\footnote{See generally \cite{schwarcz94}. For an up-to-date overview, targeting special challenges that became apparent during the 2008 financial crisis, see \cite{schwarcz13}.} The basic idea is that the corporate structure should be insulated from the original landowners to the greatest possible extent; it should have a narrow scope, it should be managed by neutral administrators, and it should entrust a third party with its voting rights.\footcite[1742]{lehavi07} This is meant to prevent failures of governance within the SPDC itself, making it harder for majority shareholders and self-interested managers to co-opt the process. For instance, if a possible developer already holds a majority of the shares in an SPDC, this structure would prevent him from using this position to acquire the remaining land on \isr{favourable} terms. 

Lehavi and Licht observe that under US law, the government would often be required to make shares in an SPDC available to the landowners as a public offering.\footcite[1745]{lehavi07} Lehavi and Licht deem this to be desirable, arguing that full disclosure will provide owners with a better basis on which to decide whether or not to accept SPDC shares in place of pre-project market value. It will also facilitate trading in such shares, so that they will become more liquid and therefore, presumably, more valuable.\footcite[1746]{lehavi07} 

Lehavi and Licht's proposal is interesting, but I think a fundamental objection can be raised against it. In particular, it seems that their governance model more or less completely alienate property owners from the decision-making process after SPDC formation. Limiting the participation of owners is to a large extent an explicit aim, since governance by experts is held to increase the chances of ensuring good governance. But is expert rule really the answer?

It seems that from the owners' point of view, Lehavi and Licht's proposals for governance reduces the SPDC to a mechanism whereby they can acquire certain financial entitlements. These may exceed those that would follow from standard compensation rules, but they do not directly empower owners vis-{\'a}-vis developers and the government. Instead, a largely independent structure will be introduced. It is this new \isr{organisational} structure, rather than the owners, that will now become an important actor in the eminent domain process. In principle, it is meant to represent owners, but to what extent can it do so effectively? After all, it is specifically intended to operate as neutral player, charged with \isr{maximising} the price, nothing more. Hence, it appears that the SPDC will not be able to give owners an arena to negotiate on the basis of property's social functions. Indeed, the institutional component of the SPDC proposal specifically targets a narrow, entitlements-oriented, perspective on what it means to be an owner and why property should be protected. 

\noo{
the personal and social importance they attribute to their land rights. How the problem of ``autonomy'' is addressed by the proposal is therefore hard to see and the ``subjective premium'' also appears to be in danger, unless it can be objectively quantified and covered by the surplus from a voluntary sale. But if such quantification is possible, then why not simply tell the appraiser to award some premium under standard compensation rules?

More generally, it seems to me that while all three categories of ``uncompensated increments'' are interesting to study from a financial viewpoint, severe doubts can be raised regarding the feasibility of addressing the subjective aspects of this as a question of compensation. It may be that issues related to ``subjective premium'' and ``autonomy'' are seen as public use issues for good reason; they are hard to quantify otherwise. Moreover, attempting to do so might do more harm than good. On the one hand, it might skew the political process, since owners that have been ``bought off'' don't object to ill-advised development projects, as long as they generate financial revenue. But what about projects that are undesirable for other reasons, for instance because they completely change the character of a \isr{neighbourhood}, or because they are harmful to the environment? On the other hand, }

With regards to individual aspects of property's social function, such as the personal attachments it engenders, or the sense of autonomy it provides, the idea that money can compensate for the owners' loss is not entirely implausible. Some standard examples include compensation for relocation costs and compensation for the cost of juridical assistance. And, indeed, in many jurisdictions, the law already provides for such compensation.\footnote{See, e.g., \cite[121-126]{garnett06}.}

In general, however, may would no doubt object that financial compensation is beside the point with regards to subjective values pertaining to the individual's own unique relationship with their property. The aftermath of {\it Kelo} itself can serve as an illustration of this.

After the case, Suzanne Kelo remained defiant at first, but eventually decided to settle in 2006, for an offer of USD 442 155, more than USD 319 000 above the appraised value.\footcite[1709]{lehavi07} Despite this, there is no indication that Suzanne Kelo changed her view of the taking. Indeed, after the long struggle she had taken part in, it is easy to imagine that financial compensation, if it was to be an effective remedy at all, would have to be very high. Even after she had settled, Kelo apparently toured the country speaking out against economic takings.

Hence, the significant overcompensation she received, compared to standard market value, did not restore legitimacy, not even to her personally. Moreover, to the community as a whole, it apparently did more harm than good. Indeed, the other owners affected by the same development plan were not pleased, arguing that recalcitrant owners had been unjustly rewarded for holding out.\footcite[1709]{lehavi07} 

This is indicative of the fact that when we move to consider the role of the community, including property dependants that have no formal claims as owners, the compensatory perspective falls short of providing a meaningful approach. Indeed, as laid down by Lehavi and Licht, it does not seem like SPDC will do much good for the community. It will not better enable its members to fulfil their obligations and responsibilities with regards to each other, their land, and society's desire for economic development. Rather, it will render them just as passive as under traditional eminent domain proceedings, the only difference being that they might benefit financially if the SPDC administrators do their job well.

\noo{ subjective importance of property and autonomy can itself prove offensive. At least it seems likely that it would often come to be seen as inadequate and inefficient.\footnote{For more detailed criticism of the compensation approach to the public use issue, see \cite{garnett06}.} Moreover, an owner that is compelled to give up his home after an inclusive process where the public interest has been debated and clearly communicated is likely to feel like he incurs less costs related both to his subjective premium and his autonomy. Hence, the lack of participation in the decision-making process can in itself increase the uncompensated loss. Clearly, no externally managed ``bargain-oriented'' SPDC will be able to resolve this problem.

I conclude that SPDCs have serious shortcoming with regards to the subjective aspects of undercompensation, aspects that can only be addressed if the focus turns towards participation. However, SPDCs do seem promising when it comes to profit-sharing. This, after all, is what the structure is specifically aiming to achieve. In addition, I agree that SPDCs will likely have a positive effect on the other actors in the eminent domain process. In particular, I agree with Lehavi and Licht that greater openness is likely to result, revealing the true merits of development projects, at least in so far as these are translatable into financial terms. The fact that developers must negotiate with an SPDC who can threaten to make the land available an an open auction will likely deter developers and government from pursuing fiscally inefficient projects. Hence, the risk that governments will \isr{subsidised} such projects by giving them cheap access to land will also be reduced. In addition, the presence of a third voice, speaking on behalf of owners, is likely to help achieve a better balance of power in development takings. 

This is a particular concern in cases when competition fails to arise after SPDC formation. To ensure that there are other interested parties, in particular, sems like an important precondition for the proposal to work in practice. In this regard, it is important to \isr{realise} that a lack of interest from other developers may not be due to the superiority of the original developer's plans. It might rather be due to the fact that the scope of the assembly giving rise to the SPDC is so defined as to make alternatives unfeasible. The danger of abuse in this regard seems significant, particularly when developers themselves participate in coming up with the plans that give rise to SPDC formation. 


Moreover, as long as owners remain marginalized in the planning phase, it is easy to imagine situations where the plan itself will be formulated in such a way that only one developer is in a position to successfully implement it. A simple example would be if a prospective developer already owns some of the land that is critical to the plan, and is able to ensure that this land is kept out of the scope of the SPDC. Clearly, if SPDCs are to operate effectively, such instances of manipulation need to be avoided, suggesting that the proposal as it stands needs to be fleshed out in greater detail.
}

The problems addressed here both seem to point to the fact that the SPDCs, while more flexible than other suggestions, are still too static to achieve many of their objectives. In particular, to arrive at genuine market conditions for assessing post-project value, there is still a need for changes in the dynamics of the planning process underlying the taking. Moreover,

This only adds to the number of external authorities on whose judgement and discretion the faith of the community will depend. If the SPDC proposal is thought of as representing as a layer {\it between} the owners, the government and interested developers, the result can also be a further marginalisation of individual owners, for whom it is now even harder to gain access to primary decision-makers.

To better fulfil the goal of ensuring increased legitimacy, there is a need for a mechanism that goes beyond expert bargaining and provides owners with better access to the decision-making process. In the next subsection, I will consider a proposal that aims to address this, by proposing a framework for self-governance. 

\subsection{Land Assembly Districts}

Heller and Hills propose a new institutional framework for carrying out land assembly for economic development. Interestingly, it is meant to replace eminent domain altogether. The goal is to ensure democratic legitimacy while also setting up a template for collective decision-making that will prevent inefficient gridlock and holdouts. 

The core idea is to introduce {\it Land Assembly Districts} (LADs), institutions that will enable property owners in a specific area to make a collective decision about whether or not to sell the land to a developer or a municipality.\footcite[1469-1470]{heller08} The idea is that while anyone will be able to propose and promote the formation of a LAD, the official planning authorities and the owners themselves must consent before it is formed.\footcite[1488-1489]{heller08} Clearly, some kind of collective action mechanism is required to allow the owners to make such a decision. 

Hiller and Hill suggest that voting under the majority rule will be adequate in this regard, at least in most cases.\footnote{See \cite[1496]{heller08}. However, when many of the owners are non-residents who only see their land as an investment, Heller and Hills note that it might be necessary to consider more complicated voting procedures, for instance by requiring separate majorities from different groups of owners. See \cite[1523-1524]{heller08}.} 

How to allocate voting rights in the LAD is given careful consideration, with Heller and Hills opting for the proposal that they should in principle be given to owners in proportion to their share in the land belonging to the LAD.\footnote{See \cite[1492]{heller08}. For a discussion of the constitutional one-person-one-vote principle and a more detailed argument in \isr{favour} of the property-based proposal, see \cite[1503-1507]{heller08}.} Owners can opt out of the LAD, but in this case, eminent domain can be used to transfer the land to the LAD using a conventional eminent domain procedure.\footcite[1496]{heller08}

Heller and Hills envision an important role for governmental planning agencies in approving, overseeing and facilitating the LAD process. Their role will be most important early on, in approving and spelling out the parameters within which the LAD is called to function.\footcite[1489-1491]{heller08} Hence, it appears to be assumed that the planning authorities will define the scope of the LAD by specifying the nature of the development it can pursue. 

A possible challenge that arises, not discussed by Heller and Hills at any length, is that the scope of the LAD needs to be broad enough to allow for meaningful competition and negotiation after LAD formation. To achieve this might be difficult, particularly in light of incentives to make the outcome of the LAD process more predictable. Indeed, both governments, initiating developers, and landowners eager for development might want to ensure that the scope of the LAD is defined narrowly enough to give confidence that zoning permissions will not later be denied. In addition, there is the obvious nefarious incentive that some actors might have to ensure that a specific development is chosen. In light of this, LAD regulation is needed to ensure a balanced approach to the issue of how the initial development possibility should be defined, and to what extent this definition should limit the authority of the LAD to choose alternative projects.

If the owners do not agree to forming a LAD, or if they refuse to sell to any developer, Heller and Hills suggest that the government should be precluded from using eminent domain to assemble the land.\footcite[1491]{heller08} This is a crucial aspect of their proposal that sets the suggestion apart from other proposals for institutional reform that have appeared after {\it Kelo}. A LAD will not only ensure that the owners get to bargain with the developers over compensation, it will also give them an opportunity to refuse any development to go ahead. Hence, the proposal shifts the balance of power in economic development cases, giving owners a greater role also in preparing the decision whether or not to develop, and on what terms. Hence, the LAD proposal promises to address the democratic deficit of economic development takings, without failing to \isr{recognise} that the danger of holdouts is real and that institutions are needed to avoid it.

There are some problems with the model, however. First, I observe that planning authorities might have an incentive to refuse granting approval for LAD formation. After all, doing so entails that they give up the power of eminent domain for the land in question. For this reason, Heller and Hills propose that a procedure of judicial review should exist whereby a decision to deny approval for LAD formation can be scrutinized.\footcite[1490]{heller08} However, the question then arises to what extent the courts should adopt a deferential stance in this regard, echoing the conundrum that engulfs the safeguard intended by the public use restriction. Presumably, one would want the courts to strictly scrutinise LAD rejections, to instil in governments that LADs should normally be promoted. However, would the courts be comfortable providing such scrutiny, also against a government body claiming that the ``public interest'' speaks against LAD formation? This would likely depend on the exact formulation and spirit of the LAD-enabling legislation. To work as intended, some sort of presumption in favour of LAD approval appears to be in order, but this in turn can have the effect of making it easier for powerful landowners to abuse the LAD system.

A second possible objection against the LAD proposal concerns the practicalities of the process leading up to the LAD's decision on whether or not to accept a given offer. Is it possible to organise such a process in a manner that is at once efficient, inclusive and informative, without making it too costly and time consuming? Here Heller and Hills envision a system of public hearings, possibly \isr{organised} by the planning authorities, where potential developers meet with owners and other interested parties to discuss plans for development.\footnote{See \cite[1490-1491]{heller08}. It might also be necessary for the planning authorities or other government agencies to take on some responsibilities with respect to providing guidance and assistance to less resourceful members among the owners.} The process envisioned here would resemble existing planning procedures to such an extent that additional costs could hopefully be kept at a minimum. 

The significant difference would concern the relative influence of the different actors, with the owners receiving a considerable boost as a result of the LAD. Rather than being sidelined by a narrative that sees the use of eminent domain as the culmination of planning, the owners are now likely to occupy center stage throughout, as they now will have the final say on whether or not the development will go ahead.

From this, however, arises the question of how the interests of other locals, without property rights, will be protected. Heller and Hills assumes that local non-owners will also be represented during the stages leading up to the LAD's final decision, but their role in the process is not clarified in any detail.\footcite[1490-1491]{heller08} This raises the worry that LADs might undermine local democracy by giving property owners a privileged position with respect to policy questions that should be decided jointly by all members of the local community. The risk in this regard depends heavily on the circumstances. In a context of egalitarian property ownership and sensible government regulation of land uses and LAD operations, the risk should be minimal. In principle, the local anchoring that LADs provide should also benefit non-owners, by brining the decision-making process closer and making it more easily accessible. Moreover, if some members of the local community remain marginalised, this is probably best regarded as a regulatory failure or a reflection of underlying inequality, not a shortcoming of the LAD proposal. In these cases, a reasonable approach might even be to {\it expand} the  function of LADs, by granting voting rights to a larger class of local property dependants, not only formally titled owners.\footnote{The important invariant to maintain, I believe, is that the locally anchored institution should be the active, invested, agent, while the government should be a more passive, disinterested, regulator. In the processes leading to economic development takings, this is typically reversed, with the government and the commercial interests being the active agents, while the owners are the passive agents whose property rights places some nominal limits on the authority of other parties (limits which, due to the weakness of owners as a group, might be easy to disregard).}

However, the LAD proposal raises some highly problematic issues pertaining to the proposed mechanism of collective decision-making. As Kelly points out, the basic mechanism of majority voting is deeply flawed.\footcite{kelly09} He argues, in particular, that if different owners value their property differently, majority voting will tend to \isr{disfavour} those with the most extreme viewpoints, either in \isr{favour} of, or against, assembly. If these viewpoints are assumed to be non-strategic and genuine reflections of the welfare associated with the land, the result can be inefficiency. In short, the problem is that a majority can often be found that does not take due account of minority interests. 

For instance, if a minority of owners are planning alternative development, conflicting with the LAD proposal, they might simply be ignored. Indeed, they might {\it have to be} ignored, if the development description underlying LAD formation is incompatible with the kind of development they wish to pursue. This could become particularly inefficient in cases when the alternative development is more socially desirable than the development sketched during LAD formation. In such cases, LAD formation will not improve the quality of the decision to develop, since it pushes the decision-making process into a track where those interests that {\it should} prevail are voiced only by a marginalised minority inside the new institution.\footnote{Of course, one might imagine these landowners opting out of the LAD, or pursuing their own interests independently of it. However, they are then unlikely to be better off than they would be in a no-LAD regime. In fact, it is easy to imagine that they could come to be further \isr{marginalised}, since the existence of the LAD, acting ``on behalf of the owners'', might detract from any dissenting voices on the owner-side.}

More generally, the lack of clarity regarding the role of LADs in the planning process is a problem. If LAD formation as such can be used to privilege a specific development over alternatives, developers would have a great incentive to form a LAD as early as possible. Moreover, if the presumption is in favour of allowing LAD formation, with limited room for government censorship, developers might rely on LADs to push through a {\it de facto} condemnation of property, through a procedure that might leave the minority less protected than the traditional takings process. Indeed, it would be theoretically possible for any landowner to use a LAD to condemn any neighbouring property smaller than their own. Eventually, a whole community might be taken over by one or a few powerful landowners, through a sequence of cleverly designed LAD processes and development projects.

Despite these worries, the ideal of the LAD proposal is clearly stated and highly attractive. LADs should help to establish self-governance for land assembly and economic development. In particular, Heller and Hills argue that LADs should have ``broad discretion to choose any proposal to redevelop the \isr{neighbourhood} -- or reject all such proposals''.\footcite[See][1496]{heller08} As they put it, two of the main goals of LAD formation is to ensure ``preservation of the sense of individual autonomy implicit in the right of private property and preservation of the larger community's right to self-government''.\footcite[See][1498]{heller08} Unfortunately, these ideals turn out to be at odds with some of the concrete rules that Heller and Hills propose, particularly those aiming to ensure good governance of the LAD itself.

In relation to the governance issue, Heller and Hills echo many of the ``corporate governance''-ideas that also feature heavily in Lehavi and Licht's proposal. Indeed, in direct contrast to their comments about ``broad discretion'' and ``self-governance'', Hiller and Hills also state that ``LADs exist for a single narrow purpose -- to consider whether to sell a neighborhood''.\footcite[See][1500]{heller08} This is a good thing, according to Heller and Hills, since it provides a safeguard against mismanagement, serving to prevent LADs from becoming battle grounds where different groups attempt to co-opt the community voice to further their own interests. As Heller and Hills puts it, the narrow scope of LADs will ensure that ``all differences of interest based on the constituents' different activities and investments, therefore, merge into the single question: is the price offered by the assembler sufficient to induce the constituents to sell?''.\footcite[1500]{heller08}

This means that there is a significant internal tension in the LAD proposal, between the broad goal of self-governance on the one hand and the fear of \isr{neighbourhood} bickering and majority tyranny on the other. Moreover, it is hard to see how LADs can at once have both a ``narrow purpose'' as well as enjoy ``broad discretion'' to choose between competing proposals for development. If such discretion is granted to LADs, what prevents special interest groups among the landowners from promoting development projects that will be particularly \isr{favourable} to them, rather than to the landowners as a group? What is to prevent landowners from making behind-the-scene deals with \isr{favoured} developers at the expense of their \isr{neighbours}? It might be difficult to come up with rules that prevent mechanisms of this kind, without also making meaningful ``self-governance'' an impossibility. 

If a LAD is obliged to only look at the price, this might prevent abuse. But it will not give owners broad discretion to consider the social functions of property when choosing among development \isr{proposals}. %Effectively, it will render LADs as little more than a variant of SPDCs, where the owners are awarded an extra bargaining-chip, namely the option to refuse all offers. 
In my view, it is undesirable to restrict the operations of LADs in this way. It is easy to imagine cases where competing proposals, perhaps emerging from within the community of owners themselves, will be made in response to the formation of a LAD. Such proposals may involve novel solutions that are superior to the original development plans, in which case it is hard to see any good reason why they should not be taken into account, even if they are proposed by a minority. Moreover, it is hard to see why they should be disregarded simply because they are less commercially attractive, or because the the developers of such a proposal cannot offer the highest payment to the owners.

Institutions for self-governance should be able to tap into a greater pool of ideas for redevelopment, ideas which may then also be rooted more firmly in the local community. That owners should be able to bring such proposals to the table is not only desirable, but a crucial aspect of meaningful participation. At the same time, it is easy to acknowledge that problematic situations may arise from wide LAD powers, for instance if a majority forms in \isr{favour} of a scheme that involves razing only the homes of the minority, maybe on the rationale that these are the most blighted properties.

Heller and Hills are aware of this potential problem, which they propose to resolve by strict regulation. In particular, they argue that ``LAD-enabling legislation should require especially stringent disclosure requirements and bar any landowner from voting in a LAD if that landowner has any affiliation with the assembler''.\footcite{heller08} But this raises further questions. For one, what is meant by ``affiliation'' here? Say that a landowner happens to own shares in some of the companies proposing development. Should they then be barred from voting? If so, should they be barred from voting on all proposals, or just those involving companies in which they are a shareholder? If the answer is yes, how would this be justified? Would it not be easy to construe such a rule as  discrimination against landowners who happen to own shares in development companies? On the other hand, if the landowner in question is allowed to vote on all other proposals, would it not be natural to suspect that their vote is biased against assembly that would benefit a competing company? Or what about the case when some of the landowners are employed by some of the development companies? Should such owners be barred from voting on proposals that could benefit their employers? This seems quite unfair as a general rule. But in some cases, an employment connection could play a decisive factor. This might happen, for instance, if an important local employer proposes development in a \isr{neighbourhood} where it has a large number of employees.

Of course, the most pressing issue that arises is the following: who exactly should be empowered to make the determination of when an affiliation is such that an owner should be deprived of his voting rights? Heller and Hills give no answer, but it is easy to imagine that whoever is given this task in the first instance, the courts must be prepared to deal with complaints. At this point, the circle has in some sense closed in on the proposal. In particular, one might ask: why is it easier to determine if someone can be deprived of their voting rights due to an ``affiliation'', than it is to determine if someone can be deprived of their property rights due to some planned ``public use''?

In any event, to come up with a set of rules ensuring that LADs can deliver self-governance that is also good governance largely remains an open problem. This is acknowledged by Hiller and Hills themselves, who point out that further work is needed and that only a limited assessment of their proposal can be made in the absence of empirical data. Later in the thesis, I will shed light on this challenge when I consider the Norwegian rules relating to land consolidation, showing how these can be looked at as a highly developed institutional embedding of many of the central ideas of LADs. 

The assessment of how this institution functions in cases of economic development, and how it is increasingly used as an alternative to expropriation in cases of hydropower development, will allow me to shed further light on the issues that are left open by Heller and Hills' important article.

\section{Conclusion}\label{sec:conc2}

In this chapter, I have given a more in-depth presentation of economic development takings. I began by noting that the issue is particularly pressing for land users that are not regarded as bringing about economic growth. Hence, I argued that the issue is closely related to that of land grabbing, which is currently receiving much attention, both academic and political. Under the social function understanding of property there is in principle no difference between protecting property rights arising from formal title and property rights arising from use. That said, special issues arise in the latter case, not least because it is unclear how the law should deal with rights resulting from cultural practices that western property regimes are not designed to handle. In addition, I noted that special issues related to poverty and basic necessities such as food and water arise with particular urgency in relation to land grabbing.

The nature of my case study makes it natural for me to focus on traditional western systems of property law. Hence, I went on to discuss how economic development takings are dealt with in such legal systems, focusing on Europe and the US respectively. For the case of Europe, this assessment was made more difficult by the fact that the category is not an established part of legal discourse. However, by looking to England for concrete examples, I noted that such cases do arise and that they are increasingly seen as controversial. \noo{ I also noted that there is a contrast between how England and Germany approach such cases, as well as how they approach property more generally. Germany, in particular, goes further in explicitly \isr{recognising} the social functions of property, by actively looking to social and political values when assessing whether interferences are legitimate. In England, similar reasoning is at most applied indirectly, as takings are approached almost entirely as an issue of administrative law. }

I then went on to consider the property protection offered by P1(1) of the ECHR, and how it is applied by the Court in Strasbourg. I zoomed in on those \isr{aspects} that I believe to be the most relevant for economic development takings. While I noted that this category has yet to be discussed by the ECtHR, I argued that a recent shift in the Court's property adjudication is suggestive of the fact that it would likely approach such cases similarly to how Justice O'Connor approached {\it Kelo}. In particular, I noted how the Court has recently adopted a stricter standard of assessment. This standard, I argued, is \isr{characterised} primarily by increased sensitivity to systemic imbalances causing alleged P1(1) violations. Hence, to regard economic development takings as a special category appears to fit well with recent jurisprudential developments at the Court in Strasbourg.

I went on to consider US sources on economic development takings, noting that the issue has receive an extraordinary amount of attention in recent years. I adopted an historical approach to the material, by tracing the case law surrounding the public use restriction in the fifth amendment to the US constitution, which was much debated even before the specific issue of economic development takings rose to prominence. I focused particularly on case law developed by state courts, and I argued that it shows great sensitivity to the need for contextual assessment. Indeed, it seems that many state courts originally adopted an implicit social function view of property when assessing such cases.

I then looked at the history of Supreme Court adjudication of public use cases. I noted that the doctrine of deference was developed early on, but that it was initially directed mainly at state courts. In fact, I showed that the Supreme Court itself explicitly approved the contextual and in-depth approach these courts relied on when dealing with the legitimacy issue.

The shift, I argued, came with {\it Berman}, in which the Supreme Court adopted a deferential doctrine that was directed specifically at the state legislature.\footcite{berman54} This was quite a dramatic departure from the Court's previous attitude towards state takings. Moreover, it was almost entirely backed up by precedent set in cases when {\it federal} takings had been ordered by Congress.

I went on to consider the fallout of {\it Berman} at state level, which culminated with the infamous {\it Poletown} case. This case prompted wide-spread accusations of eminent domain abuse and thus set the stage for {\it Kelo}.

After completing the historical overview, I went on to consider the literature after {\it Kelo}. I expressed particular support for those responses that focus on the need for {\it institutional} reform, to address  dangers that Justice O'Connor pointed to in her minority opinion. As a shorthand, I proposed referring to the mechanisms she identified as the {\it democratic deficit} of economic development takings. 
% I zoomed in on two of those in particular, the Special Purpose Development Companies proposed by Lehavi and Licht, and the Land Assembly Districts suggested by Heller and Hills. I gave an in-depth presentation of these two proposals, pointing out strengths and weaknesses. 
%%In coming chapters, I will refer back to this as I consider similar institutions and mechanisms that are currently operating in Norwegian law relating to hydropower development.

I then gave a thorough presentation of two recent reform suggestions that might help address this deficit. Both are institutional in nature, based on setting up formally recognised coalitions of land owners that can act as a counterweight to the disproportional power of commercial beneficiaries. The first suggestion, by Lehavi and Licht, is limited to dealing with the issue of compensation, \isr{recognising} the need for a system whereby the land owners are compensated based on post-project value. However, this idea alone represents a fairly dramatic break with the currently dominant doctrine in takings law, where compensation is almost always based on the pre-project value of the land {\it to the owner}.\footnote{This is a reflection of the no-scheme principle, mentioned briefly in Section \ref{sec:england} above. For further details, I refer to \cite{dyrkolbotn15}.}

In Chapter \ref{chap:4}, I will briefly discuss how this principle was abandoned in Norway, for some case types involving hydropower development. However, the broader point that will interest me in thesis concerns the conceptual premise of Lehavi and Licht's proposal. By suggesting that economic development takings can be viewed as a form of compulsory incorporation of private rights, they effectively undermine the justification for disallowing the original owners to take up a corresponding share in the resulting enterprise. This, in my view, is a powerful idea that has implications that go well beyond the issue of compensation. In particular, it points to the possibility of avoiding eminent domain altogether, by proposing a suitable framework for collective action regarding economic development.

%I relate this to the special role played by the appraisal courts in Norway. The local grounding of these courts, involving lay people sitting as court appointed appraisers, allows the law to be applied in a way that adapts to the concrete circumstance in a way that may enhance the perceived fairness and legitimacy of the taking. At the same time, however, the judicial procedure, with a (limited) possibility for appeal, puts in place safeguards against abuse.

The second suggestion I looked at in depth, proposed by Heller and Hills, is based on a similar idea. However, it does not go as far as to explicitly suggest that owners themselves should be granted shares in the development enterprise. Instead, the focus is on organising a process for selling the properties required, without the use of compulsion. According to this proposal, local communities should be entitled to greater self-governance in economic development scenarios. At the same time, the proposal \isr{recognises} the need for a mechanism to avoid inefficient and socially harmful gridlock due to holdouts among unwilling owners. Instead of eminent domain, however, a different mechanism is proposed, namely that of a majority decision made by a land assembly district.

This is also a new type of institution, and I pointed out some problems and seeming inconsistencies in the proposal. I highlighted the lack of clarity regarding the exact role LADs are supposed to play during the planning process. I argued that while the risk of abuse and failure increases with the level of participation, so does the overall potential for achieving a positive effect on legitimacy. I concluded that to reduce the democratic deficit in economic development cases, a wide power of participation must be granted to the land owners and their communities. This is needed, in particular, to restore balance in the relationship between owners and others directly connected with the land, the planning authorities, and the commercial actors interested in development for profit. The question that is as of yet unresolved is how to \isr{organise} such participation in a way that avoids obvious pitfalls, such as administrative inefficiency and tyranny by majorities or elites that gain control of the local agenda.

In Chapter \ref{chap:6}, I will shed light on this question by considering the Norwegian institution of land consolidation, which has a very long tradition behind it. It is a flexible \isr{framework} which includes, among other things, a template for establishing institutions that can function as a LAD. I will focus on how land consolidation functions in cases of economic development that would otherwise likely be pursued by eminent domain. The case study is based on considering hydropower development, but I will also discuss planning law and development more generally, as the Norwegian government is now considering making consolidation, traditionally a rural institution, a primary mechanism for land development even in urban areas.

Before I delve into this, I will present an overview of Norwegian hydropower and the role of waterfalls as private property. This will serve as an introduction to the second part of this thesis, to which I now turn.
\part{A Case Study of Norwegian Waterfalls}

\chapter{Norwegian Waterfalls and Hydropower}\label{chap:3}

\section{Introduction}\label{sec:into3}

Norway is country of mountains, fjords and rivers, where around 95 \% of the annual domestic electricity supply comes from hydropower.\footnote{See Statistics Norway, data from the year 2011, http://www.ssb.no/en/elektrisitetaar/.} The right to harness energy from rivers, streams and waterfalls generally belongs to local landowners under a riparian system.\footnote{This arrangement is rooted in the first known legal sources in Norway, the so-called ``Gulating'' laws, thought to have been in force well before AD 1000. See \cite[111-112,120]{robberstad81}.}  Historically, waterfalls were very important to local communities, particularly as a source of power for flour mills and saw mills.\footnote{See \cite[121]{tvedt13}.} %%Indeed, the fact that peasants in Norway controlled local water resources can help explain why they were relatively free, both economically and socially, compared to many other places in Europe.\footnote{See \cite[121]{tvedt13}.}

Following the industrial revolution, local ownership and management came under increasing pressure. At the beginning, this pressure was exerted by private commercial interests, often foreign investors, who saw the industrial potential in hydropower and started speculating in Norwegian water resources.\footnote{See \cite[30-31]{nou04}.} Later, the pressure on local self-governance was exerted mainly by the government, following the introduction of new legislation to regulate the development of hydroelectric power.\footnote{See \cite[41-57]{thue96} (describing the  regulatory system set up during this time).} This legislation set up a system that gave highly preferential treatment to public utilities over private actors, including local owners.\footnote{See \cite[46]{thue96} (describing legislation introduced to promote public utilities, including new expropriation authorities directed at local owners of waterfall).} At first, the motivation behind this reform was to facilitate a decentralised form of government control, led by public utilities controlled by the municipality governments.\footnote{See \cite[44-47]{thue96}.} However, the hydroelectric sector underwent gradual centralisation, a process that gained momentum after the Second World War when the state itself assumed a leading role.\footnote{See \cite[59-85]{thue96}. For the history of the state's involvement with hydropower generally, see \cite{thue06, skjold06,thue06b}.} At this time, local communities and local riparian owners  became increasingly marginalised. In particular, they were forced to shut down their hydroelectric plants in order to  connect the national, monopolised, electricity grid.\footnote{See \cite[p.111]{hindrum94}.}

Then, in the early 1990s, the electricity sector was reformed once again, largely inspired by the market-orientation and privatisation of the public sector in the UK under Thatcher.\footnote{See generally \cite{midttun98}.} The production sector was liberalised, while public utilities where reorganised as commercial companies.\footnote{See \cite[86]{efta07} (describing how Norwegian electricity companies, most of which are still (partly) publicly owned, now operate as for-profit, limited liability companies).} At the same time, the regulatory system was decoupled from both political and commercial decision-making processes, to become more expert-based.\footnote{[26-27]{brekke12}.} Moreover, the sector underwent additional centralisation, as a result of mergers and acquisitions among former public utilities.\footnote{See \cite[583]{bibow03}. I mention that despite significant continuous centralisation from the Second World War to this day, the Norwegian hydroelectric sector is still relatively decentralised compared to other countries, e.g., the UK, see \cite[181]\cite{midttun98}. Arguably, this is a lasting influence of a tradition based on local, egalitarian, ownership of water resources.}

Following the reform, access rights to the national grid are meant to be granted equally to all potential actors on the energy market, including private companies.\footnote{See generally \cite{hammer96}. For an interesting presentation and analysis of grid-based markets in general, see \cite{falch04}.} After the passage of the \cite{ea90}, the energy companies controlling the local grids were no longer authorised to shut out competitors.\footnote{See the \cite[3-4]{ea90}.} A side-effect of this is that it has become possible for local landowners to undertake their own hydropower projects. Local owners can now access the grid to sell the electricity they produce on Nord Pool, the largest electrical energy market in Europe.\footnote{See generally \cite{larsen06,larsen08,larsen12}.} This has led to increased tension between local interests and established hydropower companies. The following fundamental question has arisen: who is entitled to benefit from rivers and waterfalls, and who is entitled to a say in decision-making processes concerning their use?

In this chapter, I set the stage for discussing this question in more depth, by detailing how the hydropower sector is organised. I look to the law as well as to commercial and administrative practices and I focus on those aspects that have changed following the reform of the early 1990s. I pay particular attention to the growing importance and competitiveness of so-called {\it small-scale} hydropower, including development projects that are undertaken by local owners, or in cooperation with them. Several commercial actors have emerged who now specialise in such cooperation, by offering waterfall owners a significant share of the commercial value resulting from development. 

To bring out the multi-faceted character of the current debate on hydropower in Norway, I begin in Section \ref{sec:nutshell} by offering a basic overview of the Norwegian political system. I focus on the role that property has played in the history of Norwegian democracy. In Section \ref{sec:hl}, I move on to describe the law relating to hydropower development. I first identify a basic tension in the law -- some would call it an inconsistency -- between hydropower as a private right on the one hand, and a public good on the other. I then go on to explain how this tension permeates the law, by presenting the statutory regulation of hydropower development in more depth.

I follow this up by considering hydropower in practice, especially practices that relate to small-scale development in cooperation with local owners. I trace the history of the dominant model used to organise this form of development, going back to the first expression I could find of the core principles, given in the so-called {\it Nordhordlandsmodellen}, from 1996.\footnote{See \cite{dyrkolbotn96}.} This model presented a financial mechanism for benefit sharing that was later adopted by the market for small-scale hydropower development. In addition, the model expressed broader governance principles pertaining to the importance of sustainability, the involvement of non-owners, and the desirability of long-term planning based on local conditions and local participation in decision-making processes.

However, as I discuss in Section \ref{sec:future}, these aspects of the model have largely failed to make an impact. I go on to argue that a lack of social awareness  might be part of the reason why small-scale hydropower now appears to be falling out of favour. I note, moreover, how a recent change in the perception of small-scale hydropower has also lead to a resurgence of large-scale development. This is threatening to undermine the position of local communities and owners, and has led to a series of controversial cases before the courts.

This underscores the importance of maintaining a social function perspective on local ownership of riparian rights. Moreover, it provides important background and context for my study in the two chapters that follow, wherein I specifically address the use of expropriation -- and alternatives to it --  to facilitate hydropower development.

\section{Norway in a Nutshell}\label{sec:nutshell}

Norway is a constitutional monarchy, based on a representative system of government.\footnote{For Norwegian constitutional law generally, see \cite{andenes06}.} The executive branch is led by the King in Council, the Cabinet, headed by the Prime Minister. Legislative power is vested in the Storting, the Norwegian parliament, elected by popular vote in a multi-party setting.\footnote{It should be noted that the executive branch also enjoys considerable legislative power under Norwegian law. Both informally, because it prepares new legislation, and also formally, because it has wide delegated powers to issue so-called {\it directives} (forskrifter). Indeed, it is typical for acts of parliament to include a general delegation rule which permits the executive to legislate further on the matters dealt with in the act, by clarifying and filling in the gaps left open by it.} In 1884, the parliamentary system first triumphed in Norway, as the cabinet was forced to resign after it lost the confidence of parliament. The principle has since obtained the status of a constitutional custom. In particular, the cabinet can not continue to sit if parliament expresses mistrust against it. However, an express vote of confidence is not required. In practice, due to the multi-party nature of Norwegian politics, minority cabinets are quite common. These can sustain themselves by making long-term deals with supporting parties, or by looking for a majority on a case-by-case basis.

The judiciary is organised in three levels, with 70 district courts, 6 courts of appeal, and the Supreme Court. The district courts have general jurisdiction over most legal matters; there is no division between constitutional, administrative, civil, criminal courts. \footnote{However, there are distinct procedural rules for civil and criminal cases and a special court exists for {\it land consolidation}. See the \cite{lca79}. Moreover, both the district courts and the courts of appeal follow special procedural rules in {\it appraisement disputes}, for instance when compensation is awarded following expropriation. See the \cite{aa17} respectively, discussed in more detail in later chapters.} The courts of appeal have a similarly broad scope. Moreover, the right to appeal is ensured in most cases.\footnote{The right to an appeal is not absolute. In civil cases, it is generally required that the stakes are above a certain lower threshold, measured in terms of the appellants' financial interest in the outcome. See \cite[29-13]{da05}.} The Supreme Court, on the other hand, operates a very strict restriction on the appeals it will allow.\footnote{See the \cite[30-4]{da05}.} It typically only hears cases if a matter of principle is at stake, or if the law is thought to be in need of clarification.\footnote{See, generally, \cite{skoghoy08}.}

The Norwegian legal system is often said to be based on a special ``Scandinavian'' variety of civil law, which includes strong common law elements: legislation is not as detailed as elsewhere in continental Europe, some legal areas lack a firm legislative basis, it is generally accepted that courts develop the law, and the opinions of the Supreme Court are considered crucial to the legislative interpretation at the lower courts.\footnote{See, generally, \cite{bernitz07}.} At the same time, legislation remains the primary source used to resolve most legal disputes. Moreover, when applying the law, the courts tend to place great weight on preparatory documents procured by the executive branch. These documents are widely regarded as expressions of legislative intent, even though parliament is not usually actively involved in the process during the preparatory stages.

The Constitution of Norway dates back to 1814 and was heavily influenced by contemporaneous political movements, particularly in the US and France.\footnote{See generally \cite{mestad14}.} Moreover, it was influenced by a desire for self-determination, as Noway was at that time a part of Denmark-Norway, largely controlled by the Danish elite.\footnote{See generally \cite{ }.} Following the Napoleonic wars, Norwegian politicians sought to take advantage of Denmark's weak position to gain independence. In the end, Norway was instead forced to enter into a union with Sweden, but the Constitution remained in place. Moreover, after the triumph of the parliamentary system in 1884, Norway would also eventually gain independence, in 1905, following a peaceful and democratic transition process.\footnote{See generally \cite{sejersted15}.}

During the 19th century, farmers and peasants emerged as a powerful group in Norwegian politics. This, it is commonly held, was in large part due to the fact that they were also landowners, whose rights and contributions were not limited to traditional farming.\footnote{The ``classic'' presentation of the political influence of farmers in Norway is \cite{koht26}.} Importantly, Norwegian tenant farmers and small-holders had a significant degree of influence over the management of the land and its natural resources. The feudal tradition was never as strong in Norway as elsewhere in Europe.\footnote{See \cite[59-60]{pryser99}.}

\noo{ The majority of Norwegian peasants were tenants in the 17th century, but they generally enjoyed better protection against abuse. In addition, the remoteness of many rural communities and the challenging natural environment meant that large feudal estates could hardly operate effectively without granting much autonomy to local peasants. Moreover, the black death had severely taken its toll on the population in Norway, wiping out entire communities, including the feudal elites and the social and physical infrastructure that sustained them. It should be noted that this is also considered an important reason why Denmark could dominate Norway.\footnote{The origins of Denmark-Norway was the so-called Kalmar union, which also included Sweden. Initially, Norway took part on relatively equal footing with Sweden and Denmark. Later, however, after Sweden left the union, Denmark developed a much stronger position than Norway. See generally \cite{....}.}
}

The Danish-Norwegian nobility had fallen into a fiscal crisis in the 18th century, weakening their influence further. This had in turn made it possible for tenant farmers in Norway to buy land from their landowners,  including grazing grounds and non-arable land.\footnote{See \cite[59-60]{pryser99}.} As a result, the distribution of land ownership in Norway had already become highly egalitarian at the time of the Constitution. \noo{It is worth noting that farmers would typically purchase shares of larger estates. They would then acquire sole ownership over their own house and cultivated ground, while becoming co-owners of the surrounding land, alongside other local farmers.} Moreover, many resources attached to land were owned jointly by several members of the local community, as larger estates were partitioned into several individual smallholdings. As a result, Norway became a society were land ownership was not a privilege for the few, but held by the many, particularly compared to feudal Europe.\footnote{For a comparative discussion of this, focusing on how it influenced the industrialisation process in Norway, setting it apart from the industrialisation process in the UK, see \cite{brox13}.}

In 1814, the landed nobility in Norway was further marginalised. Indeed, the Constitution itself prohibited the establishment of new noble titles and estates.\footcite[23|118]{c14} Then, in 1821, all hereditary titles were abolished (although existing nobles kept their titles for their lifetimes).\footnote{See `Lov, angaaende Modificationer og nærmere Bestemmelser af den Norske Adels Rettigheder' (Act of August 1, 1821).} By the middle of the 19th century, ordinary farmers had gained even greater political influence. In fact, they emerged as the leading political class, alongside the city bureaucrats.\footnote{See generally \cite{hommerstad14}.} During this time, Norway also introduced a system of powerful local municipalities. These were organised as representative democracies, becoming miniature versions of the cherished, as of yet unfulfilled, nation state (Norway was still in a union with Sweden at this time). Even today, municipalities retain a great deal of power in Norway, particular in relation to land use planning.\footnote{They are the primary decision-makers for spatial planning, as pursuant to \cite{pb08}.} They represent a highly decentralised political structure, with a total of 428 municipalities in force as of 01 January 2013. \footnote{This is down from the all-time high of 747 in 1930. There have long been proposals to reduce the number of municipalities further, but so far the political resistance against this has prevented major reforms. See \cite{kommuner14} (report to the Ministry from an expert committee on municipality reform, 2014).}

Local control of water resources, ensured through property rights, was very important to farmers and rural communities in pre-industrial Norway. According to Terje Tvedt, 10 000 - 30 000 mills were in operation in Norway in the 1830s.\footnote{See \cite[121]{tvedt13}.} As Tvedt argues, the fact that these mills were under local control was particularly important because it helped ensure self-sufficiency. In addition, saw mills became an important source of extra income for Norwegian farming communities. \noo{While some of the larger mills were controlled and operated on behalf of non-local owners, most of them were run by the farmers themselves.} %Indeed, even during feudal times, tenant farmers often successfully argued that their tenancy entitled them to engage freely in the saw mill and timber industry, although the Danish Crown also put in place a concession system that gave them some power (until the system was abolised in 1830s).\footnote{...}

Today, the importance of water is clearly felt throughout Norwegian society. This is not because water is scarce, but rather because it is so plentiful. Not only is water power the main source of domestic energy. It also occupies a special place in Norwegian culture. It is important to the identity of many communities, particularly in the western part of the country, where majestic waterfalls are important symbols both of the hardship of the natural conditions and the sturdiness of local people. One particular aspect of this, with significant economic implications, is that waterfalls are an important asset to Norwegian tourism. The so-called ``Norway in a nutshell'' tours, for instance, have become greatly popular, based on delivering access to wild and unspoilt nature, with fjords, waterfalls, idyllic villages, and railway lines that seem to defy gravity.\footnote{See \cite{nutshell}.}

Another aspect of the same is the great tradition in Norway for local resistance against large-scale development that is considered damaging to the environment. In the 1960s and 70s, when the state embarked on their most ambitious projects, this led to a general political movement in Norway which saw progressive leftist protest groups join forces with local opposition groups in a fight against centralisation, exploitation of weaker groups, and environmental destruction.\footnote{See \cite{....}.}

This all speaks to the fact that water resources are embedded in the social fabric in Norway in such a way that an entitlements-based account of property rights to such resources would be largely inappropriate. Rather, the case of Norwegian streams and waterfalls seems to be particularly suited for an investigation based on a social function view on property. As I show in this and the following two chapters, rivers and waterfalls serve to bring out tensions between rights and obligations in property, while also shedding light on the question of how to organise decision-making processes regarding economic development.

In the next section, I argue that the present law on hydropower in Norway tends to recognise only a small part of the relevant picture. On the one hand, it recognises the financial entitlements of individual owners, which it tries to balance against the regulatory needs of the state. But it largely fails to take into account that owners have broader interests, even obligations, relating to the sustainable management of their streams and their waterfalls. Moreover, it also seems that the law is increasingly failing to take into account that commercial interests can exert a strong pull on various state bodies, particularly those that are only weakly grounded in processes of democratic decision-making.

As a result, the current narrative on water resource management in Norway appears to be based on a false dichotomy that sees the interests of ``profit-maximising'' owners, acting out of self-interest, pegged against the interests of a ``benevolent'' state, acting for the common good. In the following, I shed further light on this narrative and argue that it is deeply flawed.

\section{Hydropower in the Law}\label{sec:hl}

Under Norwegian law, rights to harness power from rivers and waterfalls are regarded as private property.\footnote{Historically, the law emphasised ownership of traditional agrarian water resources, such as fishing rights. However, new sticks were added to the waterfall bundle over the years, including the right to develop hydropower, see \cite[14-32]{vislie44}. For a detailed presentation of the history of water law in pre-industrial times, I refer to \cite{motzfeld08}.} The system is riparian, so by default, a stream belongs to the owner of the land over which the water flows.\footnote{See the \cite[13]{wra00}.} The landowners do not own the water as such -- freely running water is not subject to ownership -- and the riparian owners' right to withhold or divert water is limited.\footnote{See \cite[8|15]{wra00}.} It is common in Norway to refer to owners of hydropower rights as {\it waterfall owners} (`falleiere'), a terminology I will also adopt-\footnote{The Norwegian term `fall' has a somewhat broader meaning than its English counterpart, `waterfall'. The word `fall' is used to describe a continuous section of any stream or river, typically identified by giving the total difference in altitude over the relevant stretch of riverbed. Furthermore, the Norwegian term `falleier' refers to a legal person who possesses the rights to the hydropower over such a section. In this thesis, I will typically refer to the owners of waterfalls, streams and rivers with the intended reading being the same as the Norwegian notion of a `falleier'. If special qualification is needed, for instance to distinguish between different classes of riparian owners, I will make a note of this explicitly.}

The waterfall owners have the exclusive right to harness the potential energy in the water over the stretch of riverbed belonging to them. This right can be partitioned off from rights in the surrounding land, and large-scale hydropower schemes typically involve such a separation of water rights from land rights. In this way, the energy company acquires the right to harness the energy, while the local landowners retain ownership of the surrounding land.

Norwegian rivers, and especially rivers suitable for hydropower schemes, tend to run across grazing land and non-arable land that is owned jointly by local farmers. Hence, rights to streams and waterfalls are typically held among several members of the rural community.\footnote{Rivers tend to run through land that has not to been enclosed. Moreover, in places where there has been a land enclosure, water rights are often explicitly left out, such that they are still considered jointly owned rights belonging to the community of local farmers. For more details on (forms of) joint ownership among Norwegian farmers, see, e.g., \cite[570]{stenseth07}.} Local owners might not be willing to give up their ownership to facilitate development, especially not on terms proposed by external developers. Hence, the authority to expropriate has become an important legal instrument for Norwegian hydropower companies.

This has resulted in a tension where, on the one hand, rights to harness hydropower from streams and waterfalls are considered private property, while on the other hand, it has become common to speak of hydropower as a resource belonging to the public. Since the \cite{ica17} was amended in 2008, this ambivalence in the discourse surrounding hydropower has also been part of the statutory provisions regulating hydropower development. I quote the two relevant sections side by side below:\footnote{The first quote is taken from the general water law, with roots going back a thousand years to the so-called ``Gulating'' laws mentioned in Section \ref{sec:into3}. The second quote is taken from a law directed specifically at large-scale hydropower, introduced during the early days of the hydropower industry.}

{\begin{minipage}[t]{16em}
 \begin{aquote}{\tiny \cite[13]{wra00}} \footnotesize A river system belongs to the owner of the land it covers, unless otherwise dictated by special legal status. [...]

The owners on each side of a river system have equal rights in exploiting its hydropower.
\end{aquote}  
\end{minipage}}
{\begin{minipage}[t]{22em}
\begin{aquote}{\tiny \cite[1]{ica17} (after amendment in 2008)} \footnotesize Norwegian water resources belong to the general public and are to be managed in their interest. This is to be ensured by public ownership.
\end{aquote}
\end{minipage}} \\

The intended reading of section 1 of the \cite{ica17}, quoted on the right above, is that it expresses a ``general starting point''.\footnote{See \cite[72]{otprp61}.} According to the Ministry, it expresses no more than what has always been the purpose of the special licensing requirements for large-scale hydropower.\footnote{See \cite[72]{otprp61}.}

Despite appearances, it would be wrong to regard this as an attempt to explicitly confront the principle of private property expressed in section 13 of the \cite{wra00}, quoted on the left above. At least, such a confrontation does not appear to have been intended by the Ministry.\footnote{There are no indications in the preparatory materials that the Ministry sought to confront the principles of ownership encoded in the \cite{wra00}.} However, the Ministry's comment underscores the extent to which the government regards it as natural to interfere with private rights to waterfalls, to pursue policies that it regards to be in the public interest. Taken in this light, section 1 of the \cite{ica17} reflects the prevailing opinion that there are few, if any, recognised limits on the state's power to manage privately owned water resources.\footnote{For a reflection of the same attitude, citing the state's broad regulatory competence as the main reason not 
to nationalise Norwegian water power rights, I refer to the preparatory documents underlying the \cite{wra00}. See \cite[152-153]{nou94}.}

This aspect of the Norwegian system has become particularly significant following the liberalisation of the electricity sector in the early 1990s.\footnote{See, e.g., \cite{larsen06}.} Since then, there have been an increasing number of cases where owners who are interested in undertaking their own development schemes attempt to fend off commercial energy companies wishing to expropriate.\footnote{See, e.g., \cite{sofienlund07}.} Importantly, the state has tended to side with the commercial companies in these cases, granting them the authority to expropriate for economic development. This has resulted in several Supreme Court decisions on hydropower and expropriation in the past few years.\footnote{See \cite{uleberg08,otra10,jorpeland11,klovtveit11,otra13}.} Before discussing the case law in more detail in the next chapter, I present the most important legislation regarding hydropower development. I focus on those aspects that are particularly relevant to the position of local owners and their communities.

%Before we delve into the details, we will elaborate a bit further on the context in which the law was called upon to function in this case. Importantly, the economic, social and political context of expropriation has changed rather dramatically in recent years.

%There are two developments that have been particularly important. First, there has been a general shift from viewing electricity production as a public service to viewing it as a commercial enterprise. This has made the legitimacy of expropriation appear more controversial, and the argument is often voiced that expropriation does not happen in the interest of the public at all, but \emph{solely} in order to benefit the commercial interests of particular companies.\footnote{This has been a recurring theme in articles appearing in "Småkraftnytt", the newsletter for "Småkraftforeninga", an interest organization for owners of small-scale hydro-power, which currently have 236 associated small scale hydro-power plants, see http://kraftverk.net/ (in Norwegian). In addition to the case of Måland, the question has also been brought before the (lower) national courts in some other cases, such as \emph{Sauda}, LG-2007-176723 (Gulating Lagmannsrett, regional high court), and \emph{Durmålskraft}, see http://www.ranablad.no/nyheter/article5583405.ece (decision from the district court, as reported in a Norwegian newspaper). In both cases, the outcome was generally more favorable to the expropriating party than the local owners, and the reasoning adopted by the courts appears similar to that of \emph{Måland}.}

%In this way, expropriation of Norwegian waterfalls raises issues that have become increasingly important also in a global setting, and which seem to arise naturally in systems where economic activities are organized based on public-private partnerships. In such systems, it seems practically inevitable that cases of expropriation -- undertaken to benefit the public -- will also often come to benefit developers that are motivated by purely commercial interests. While this in itself might not be problematic, it will easily lead to the concern that the commercial interests of powerful companies is the main reason why expropriation is permitted, and that expropriation is being used as a commercial tool for powerful market forces, to the detriment of less powerful actors.

%For the case of Norwegian waterfalls, however, liberalization of the energy sector has also had a positive effect for local communities, in that it has served to make local owners more active. It has become increasingly common that they exploit their hydro-power resources themselves, often in small scale projects, and often in cooperation with companies that specialize in such development.\footnote{In 2012, the NVE granted 125 new licenses for small scale hydro-power, and at the end of the year they had 859 applications still under consideration. Source: report made by the NVE, available at http://www.nve.no/Global/Energi/Q412\_ny\_energi\_tillatelser\_og\_utbygging.pdf (in Norwegian). } This, of course, only adds to the controversy surrounding expropriation of waterfalls, especially when local owners are deprived of the opportunity for small scale development.

\subsection{The Water Resources Act}\label{sec:wra00}

The \cite{wra00} contains the basic rules regarding water management in Norway.\footnote{Act relating to river systems and groundwater of 24 November 2000 No. 82 (unofficial translation provided by the University of Oslo, \url{http://www.ub.uio.no/ujur/ulovdata/lov-20001124-082-eng.pdf}). I also mention the Water Framework Directive of the European Union, \cite{water00}. It has been implemented in Norwegian law as the Directive Regarding Frameworks for Water Management, FOR-2006-12-15-1446. It does not directly impact on the hydropower licensing procedure, however, so I will not say much about it in this thesis. However, I mention that there is some concern that the Norwegian implementation of the directive has not sufficiently recognized the need for structural reforms, preferring to rely on the established approach to water management, which is highly centralised and sector-based. See \cite{hanssen14}.} This act is not only concerned with hydropower, but regulates the use of river systems and groundwater generally.\footnote{See the \cite[1]{wra00}. A river system is defined as ``all stagnant or flowing surface water with a perennial flow, with appurtenant bottom and banks up to the highest ordinary floodwater level'', see \cite[2]{wra00}. Artificial watercourses with a perennial flow are also covered (excluding pipelines and tunnels), along with artificial reservoirs, in so far as they are directly connected to groundwater or a river system, see the \cite[2a-2b]{wra00}.} 

In section 8, the Act sets out the basic license requirement for anyone wishing to undertake measures in a river system.\footnote{Measures in a river system are defined as interventions that ``by their nature are apt to affect the rate of flow, water level, the bed of a river or direction or speed of the current or the physical or chemical water quality in a manner other than by pollution'', see the \cite[3a]{wra00}.} The main rule is that if such measures may be of ``appreciable harm or nuisance''  to public interests, then a license is required.\footnote{See the \cite[8]{wra00}. There are two exceptions, concerning measures to restore the course or depth of a river, and concerning the landowner's reasonable use of water for his permanent household or domestic animals, see the \cite[12|15]{wra00}.} The water authorities themselves decide if this condition is met.\footnote{See \cite[18]{wra00}.} Moreover, the water authorities are obliged to issue such a decision if the developer, an affected authority, or others with legal standing in the matter, request it. In relation to hydropower development, it is established practice that most hydropower projects over 1000 KW will be deemed to require a license.\footnote{See, e.g., \url{http://www.nve.no/no/Konsesjoner/Vannkraft/Konsesjonspliktvurdering/} (accessed 16 August 2014). Exceptions are possible, for instance projects that upgrade existing plants, or which utilise water flowing between artificial reservoirs.}

The basic assessment criterion is that a license ``may be granted only if the benefits of the measure outweigh the harm and nuisances to public and private interests affected in the river system or catchment area''.\footnote{See \cite[25]{wra00}.} Hence, the water authorities are empowered to decide whether a licence {\it should} be granted, if they find that the benefits outweigh the harms. The courts are very reluctant to censor the discretion of the administrative decision-makers on this point.\footnote{This is an expression of the principle of ``freedom of discretion'' for the administrative branch, a fundamental tenet of Norwegian administrative law. See generally \cite[71-74]{eckhoff14}.}

The Ministry of Petroleum and Energy maintains indirect control over the assessment process by issuing directives regarding the administrative procedure in licensing cases.\footnote{See section 65 of the \cite{wra00}.} In addition, the procedure is determined in large part by administrative practices developed by the water authorities themselves.\footnote{I return to a presentation of administrative practice in Section \ref{sec:step}.} Many of the rules in the \cite{paa67} also apply. However, these are general rules that tend to play a minor role compared to sector-specific practices.\footnote{This was a point of contention in the Supreme Court case of \cite{jorpeland11}. The case is discussed and presented in depth in Chapter \ref{chap:4}, Section \ref{sec:jorpeland}.}

A few basic procedural rules are encoded directly in the \cite{wra00}. This includes rules to ensure that the application is sufficiently documented, so that the authorities have enough information to assess its merits.\footnote{See \cite[23]{wra00}.} Moreover, a basic publication requirement is expressed, stating that applications are public documents and that the applicant is responsible for giving public notice. The intention is that interested parties should be given an opportunity to comment on the plans.\footnote{See \cite[24]{wra00}. There are some exceptions to the requirement to give public notice, however. It may be dropped in case it appears superfluous, or if the application must be rejected or postponed, see \cite[24a-24c]{wra00}.} More detailed rules for public notice of applications are given in section 27-1 of the \cite{pb08}, which also applies to licensing applications under section 8 of the \cite{wra00}.

Furthermore, an important rule of principle is given in section 22, regarding the relationship between applications for licenses and governmental ``master plans'' for the use or protection of river systems in a greater area. These plans have no clear legislative basis, but were introduced through parliamentary action in the 1980s, when the parliament decided to initiate such planning in an effort to introduce a more holistic basis for assessment of licensing applications.\footnote{Today, the planning authority is delegated to the Directorate of Natural Preservation and the NVE. See \cite{sp}.} %Moreover, the system has undergone reform as a consequence of Norway's implementation of the water directive of the European Union. See .......?!?} 
According to section 22 of the \cite{wra00}, if a river system falls within the scope of a master plan that is under preparation, an application to undertake measures in this river system may be delayed or rejected without further consideration.\footnote{See \cite[22]{wra00}, para 1.} Moreover, a license may only be granted if the measure is without appreciable importance to the plan.\footnote{See \cite[22]{wra00}, para 1.} In addition, once a plan has been completed, the processing of applications is to be based on it, meaning that an application which is at odds with some master plan may be rejected without further consideration.\footnote{See \cite[22]{wra00}, para 2.} It is still possible to obtain a license for such a project, but if it harnesses less hydropower than the project indicated by the plan, section 22 states that only the Ministry may grant it.\footnote{See \cite[22]{wra00}, para 2.}

The rules considered so far apply to any measures in river systems, not only hydropower projects. However, special procedures that apply to hydropower cases are described in other statutory provisions. The most important is the \cite{wra17}, which is specifically aimed at a certain subgroup of hydropower schemes, namely those that involve regulation of the flow of water in a river system.\footnote{See Section \ref{sec:wra17} below.} However, according to section 19 of the \cite{wra00}, many provisions from the \cite{wra17} also apply to unregulated, run-of-river, schemes, if they generate more than 40 GWh/year.\footnote{See \cite[19]{wra00}} In the next section, I will present the \cite{wra17} in more detail.

\subsection{The Watercourse Regulation Act}\label{sec:wra17}

In order to maximise the output of a hydropower scheme, the flow of water may be regulated using dams or diversions. Regulation was particularly important in the early days of hydropower, before the national electricity grid was developed.\footnote{See \cite[83]{uleberg08}.} Consumers did not want to pay for more energy than they needed when the flow was high, and they wanted to avoid power cuts in periods of drought. At this time, a few local hydropower plants were typically the only sources of electricity in any given area. This meant that regulation of the waterflow was needed to even out the level of electric output, otherwise the electricity supply would be unstable. Indeed, in the early days, it was common for electricity producers to get paid based on the stable effect they were able to deliver, rather than the total amount of energy they harnessed.\footnote{See \cite{sofienlund07}.}

This changed with the development of a wide-ranging electricity grid, which allowed electricity to be imported and exported between different geographical areas depending on the levels of output from those areas.\footnote{See \cite[83-84]{uleberg08}.} Today, producers get paid based on the total amount of electricity produced, measured in kilowatt hours (KWh). The price fluctuates over the year, and the supply-side is still influenced by instability in the waterflow in Norwegian rivers. However, the smoothing effect of the national grid means that run-of-river schemes can be carried out profitably, even if most of the electricity from the plant is produced during peak periods. %In addition, due to technological advances, the kinds of generators needed to exploit fluctuating levels of water have become much cheaper.

Despite the growing importance of run-of-river schemes, many key rules regarding hydropower development are still to be found in the \cite{wra17}.\footnote{Act relating to the regulation of watercourses of 14 December 1917 No. 17.} The Act defines regulations as ``installations or other measures for regulating a watercourse's rate of flow''. It also explicitly states that this covers installations that ``increase the rate of flow by diverting water''.\footnote{See \cite[1]{wra17}.} The core rule of the act is that watercourse regulations that affect the rate of flow of water above a certain threshold are subject to a special licensing requirement.\footnote{See \cite[2]{wra17}.}

The threshold is defined in terms of the notion of a ``natural horsepower'', such that a license is required if the regulation yields an increase of at least 400 natural horsepower in the river. Natural horsepower is a measure of the gross estimate of the power that can be harnessed from a river stably for at least 350 days a year.\footnote{See \cite[2]{wra17}.} The mathematical definition is a very simple expression, given below:

$$
nat.hp(Q,H) = 13.33 \times H \times Q
$$
This formula states that the natural horsepower of a regulation project ($nat.hp(Q,H)$) is a function of two variables, $H$ and $Q$. The constant factor $13.33$ is the force of gravity of Earth exerted on a mass of 1 kg (or, approximately, 1 litre of water). The variable $H$ is the difference in altitude (measured in metre) from the intake dam to the power generator. The variable $Q$ is the amount of water (measured in litre) stably available every second for at least 350 days per year. The result is then a gross estimate (assuming no energy loss) of the stable horsepower output of the hydroelectric plant that harnesses the power of $Q$ litres of water per second over a difference in altitude of $H$ metres. 

Section 2 of the \cite{wra17} asks us for the {\it increase} of this figure after regulation. To arrive at this number, one first uses the formula with $Q$ taken to be $Q_1$, the stable water flow prior to regulation, before calculating it with $Q$ taken to be $Q_2$, the stable water flow after regulation. The difference between the second and the first figure ($nat.hp(Q_2,H) - nat.hp(Q_1,H)$) is the increase of natural horsepower resulting from regulation.

Effectively, at a time when electricity had to be produced at a stable effect, from a stable source of power, this increase in natural horsepower was a gross estimate of the value added to the river by regulation.
%For this reasons, the courts 
%also turned to the notion of a natural horsepower to award compensation following expropriation of hydropower as such, a curious compensation practice that never had any legislative basis.\footnote{See \cite[82-83]{uleberg08}.} I return to this in more depth in the next chapter.
In the present context, suffice it to say that most watercourse regulations undertaken in the context of hydropower development will indeed yield 400 natural horsepower or more, so that they require a special license pursuant to section 2 of the \cite{wra17}. 

%In addition to these regulation projects, the \cite{wra00} stipulates that many rules in the \cite{wra17} apply to any hydropower scheme that will generate more than 40 GWh annually.\footnote{See \cite[19]{wra00}.}

The criteria for granting a regulation license are similar to those for granting a license pursuant to the \cite{wra00}. In particular, section 8 of the \cite{wra17} states that a license should ordinarily be issued only if the benefits of the regulation are deemed to outweigh the harm or inconvenience to public or private interests.\footnote{See \cite[8]{wra17}.} In addition, it is made clear that other deleterious or beneficial effects of importance to society should be taken into account.\footnote{See \cite[8]{wra17}.} Finally, if an application is rejected, the applicant can demand that the decision is submitted for review by parliament.\footnote{See \cite[8]{wra17}.}

The \cite{wra17} contains more detailed rules regarding the procedure for dealing with license applications. The most practically important is that the applicant is obliged to carry out an impact assessment pursuant to the \cite{pb08}.\footnote{Act no 71 of 27 June 2008 relating to Planning and Building Applications.} This means that the applicant must organise a hearing and submit a detailed report on positive and negative effects of the development, prior to submitting a formal application for a licence. Effectively, at least {\it two} detailed rounds of assessment are therefore required. %As I discuss in more depth in Section \ref{sec:step} below, impact assessments tend to focus on environmental issues as well as general societal consequences. The local perspective, particularly the effects of development on local owners, is usually not a primary concern. 

In addition to prescribing impact assessments, the \cite{wra17} contains more specific rules concerning the second public hearing that should take place, when the application as such is processed. First, the applicant should make sure that the application is submitted to the affected municipalities and other interested government bodies.\footcite[6]{wra17} Second, the applicant should send the application to organisations, associations and the like whose interests are ``particularly affected''.\footcite[6]{wra17} Along with the application, these interested parties should be given notification of the deadline for submitting comments, which should not be less than three months.\footnote{See \cite[6]{wra17}.} The applicant is also obliged to announce the plans, along with information about the deadline for comments, in at least one commonly read newspaper, as well as the Norwegian Official Journal.\footnote{\cite[6]{wra17}. The Norwegian Official Journal is the state's own announcement periodical.} In so far as the water authorities find it ``reasonable'', the applicant is obliged to compensate landowners and other interested parties for expenses accrued in relation to legal and expert assistance sought in relation to the application.\footcite[6]{wra17}

A license pursuant to the \cite{wra17} might be cumbersome to obtain, but a successful application also results in a significant benefit. Most importantly, the license holder then automatically has a right to expropriate the necessary rights needed to undertake the project, including the right to inconvenience other owners.\footnote{See \cite[16]{wra17}.} 

One may ask whether this approach, making expropriation a side-effect of a regulation license, is a legitimate manner in which to organise interference in property rights. It is clear that the issue of expropriation rarely receives separate treatment in regulation cases. In particular, the assessment undertaken by the water authorities is focused on the licensing issue, which does not compel them to direct any special attention towards owners' interests.\footnote{I demonstrate this, and discuss it in much more depth, in Chapter \ref{chap:4}, Section \ref{sec:jorpeland}.}

In general, the issue of who owns and controls the water resources in question receives little attention in relation to licensing applications, both pursuant to the \cite{wra17} and the \cite{wra00}. Instead, the focus is on weighing environmental interests against the interest of increasing the electricity supply and facilitating economic development. The issue of resource ownership is more prominent in relation to a third important statute, namely the \cite{ica17}.

\subsection{The Industrial Licensing Act}\label{sec:ica17}

In the early 20th century, industrial advances meant that Norwegian waterfalls became increasingly interesting as objects of foreign investment. To maintain national control of water resources, parliament passed an act in 1909 that made it impossible to purchase valuable waterfalls without a special license. The follow-up to this act is the \cite{ica17}, which is still in force.\footnote{Act relating to acquisition of waterfalls, mines, etc. of 14 December 1917 No. 16.} It applies to potential purchasers and leaseholders of rivers that may be exploited so that they yield more than 4000 natural horsepower.\footnote{Unlike section 2 of the \cite{wra17}, this asks only for the number of horsepower in the river (after regulation), not the {\it increase} of this number.}

In practice, this means that the Act does not apply to many run-of-river hydropower schemes, even large-scale projects. Even some regulation schemes fall outside the scope of the \cite{ica17}, although most large-scale regulation schemes will be covered. Originally, the main rule in the \cite{ica17} stated that all licenses granted to private parties were time-limited, and that the waterfalls would become state property without compensation when they expired, after at most 60 years.\footnote{See the old \cite[2]{ica17}, in force before the amendment on 26 September 2008.} This was known as the rule of {\it reversion} in Norwegian law.\footnote{This is a misnomer, however, in light of how most rivers and waterfalls were originally owned by local peasants, not the state.}

In a famous Supreme Court case from 1918, the rule was upheld after having been challenged by owners on constitutional grounds.\footnote{See \cite{johansen18}.} I return to this decision in Chapter \ref{chap:4}, but mention here that the main conclusion was that the rule of reversion represented a form of regulation of property, not expropriation. Hence, it could not be challenged on the basis of section 105 of the Constitution, even though the owners were not awarded compensation.

While the rule of reversion withstood internal challenges, it was eventually struck down by the EFTA Court in 2007, as a breach of the EEA agreement.\footnote{See \cite{efta07}. The EEA (European Economic Area) agreement sets up a framework for the free movement of goods, persons, services and capital between Norway, Iceland, Lichtenstein and the European Union. The EFTA (European Free Trade Association) oversees the implementation of the EEA for those members of EFTA that are also members of the EEA (all except Switzerland). For further details, see generally \cite{bull94,magnussen02,fredriksen09}.} This conclusion was fuelled by the fact that reversion only applied to privately owned companies, so that it represented a form of discrimination. After this ruling, the \cite{ica17} was amended. Today, only companies where the state controls more than 2/3 of the shares may purchase waterfalls or rivers to which the act applies.\footnote{See the \cite[2]{ica17}.}

This means that such rivers and waterfalls can only be bought, leased or expropriated by companies in which the state is a majority shareholder. In practice, however, landowners are still able to sell the land from which the right to a waterfall originates, even if this also means transferring the waterfall to a new owner. The rule is only enforced when riparian rights as such are transferred, specifically for the purpose of hydropower development. Moreover, local owners may in theory still develop hydropower in rivers and waterfalls that fall under the Act, since they already own them. But this would be difficult in practice if they are denied permission to partition the water rights off from the surrounding land, to make them available as stand-alone security for debt commitments. In effect, local owners would have a hard time acquiring financing for projects in these rivers, particularly if they do not wish to put their entire land holdings down as security.

Moreover, if they succeed in acquiring financing, a development license would likely be hard to obtain. It is quite clear, in particular, that the Norwegian government takes the view that hydropower projects in waterfalls falling under the \cite{ica17} should only be undertaken by companies in which the state has at least 2/3 of the shares. Moreover, the state does not seem willing to differentiate between development by external commercial interests and development by local owners.

For instance, in the ongoing case of {\it Sauland}, the local owners of a large waterfall wish to develop a project that would fall under the \cite{ica17}. At the same time, a large-scale development involving the same waterfall is planned by a company in which the state owns more than 2/3 of the shares. The case is still pending a final decision, but the water authorities have stated their unwillingness to assess any licensing application from the local owners, unless they ensure that the state is granted at least a 2/3 stake in the development.\footnote{Source: NVE (\url{www.nve.no}).}

In this case, the authorities use section 2 of the \cite{ica17} to deprive the owners of control over the development of their waterfalls. This, moreover, is not regarded as expropriation under Norwegian law. Hence, it would be unlikely to result in an obligation to pay compensation to original owners.\footnote{Instead, the question may be raised whether the government acted in accordance with the \cite{ica17} in this case, or extended its scope in an illegitimate way. After all, the project that the water authorities refused to consider did not in fact involve transferring control over the waterfalls to any new, non-local, owners.} The question of how to apply the \cite{ica17} in this situation has not yet been clarified by the courts. If local owners may be deprived of the development potential under the act, it also raises the further question of what the consequences will be for the level of compensation when this development potential is subsequently transferred to a company in which the state controls more than 2/3 of the shares. If the initial act of deprivation is regarded as following from regulation rather than expropriation, it would seem to follow from general expropriation law that no compensation is payable when the hydropower potential is subsequently taken by someone to whom all the necessary licenses may be granted.\footnote{I note the parallel with the {\it Agri} case in South Africa, which concerned similar mechanisms in the context of mineral rights, as discussed in Chapter \ref{chap:1}, Section \ref{sec:esr}.} In this situation, the hydropower potential might not need to be compensated, since it can no longer be said to represent a foreseeable source of income for the original owners.

The policy justification for the \cite{ica17} is based on the idea that giving preference to state-owned actors will protect the public interest in Norwegian hydropower. However, this perspective clashes with the fact that the electricity sector itself has been liberalised. The state may be a majority shareholder in the most powerful companies, but these companies are now run according to commercial principles, with little or no direct political involvement.\footnote{See \cite[86]{efta07}.}

Hence, as the EFTA court highlights in its judgement on reversion, there appears to be a lack of convincing policy reasons why state-owned companies should be given preferential treatment.\footnote{See \cite[84-87]{efta07}.} Of course, the public benefits indirectly from the fact that public bodies, as shareholders, are entitled to dividends. However, it is not clear why this benefit should be considered in a different light than other indirect financial benefits which might as well be extracted from private companies, e.g., through taxation.

The preferential treatment of publicly owned market actors also contrasts with some of the key ideas behind the basic building block of the new electricity market, namely the \cite{ea90}.

\subsection{The Energy Act}\label{sec:ea}

Before 1990, the Norwegian electricity sector was tightly regulated by the government.\footnote{See generally \cite{bye05,skjold07}.} The responsibility for the national grid was divided between various public utilities that would also typically engage in electricity production, wielding monopoly power within their districts. The most powerful utilities were controlled by the state, who also developed large-scale hydropower to supply the metallurgical industry with cheap electricity.\footnote{See \cite[67-71]{thue96}.} However, the county councils and the municipalities maintained a significant stake in the hydroelectric sector, as they often controlled the utilities responsible for the electricity supply in their own local area.\footnote{See \cite[85]{thue96}.} 
Prior to 1990, there was not competition on the electricity market, and the local monopolists could deny other energy producers access to the distribution grid.\footnote{See \cite[83-84]{uleberg08}.}

%At the same time, the system ensured that energy companies were more or less directly politically accountable. They were typically run by politically appointed boards, often organised as administrative bodies of local government.\footnote{.....} Moreover, they were not subject to commercial management principles that insulated them from the political discourse of the local area in which they operated.\footnote{.....}

This system was abandoned following the passage of the \cite{ea90}.\footnote{See generally \cite{bibow11}.} This act set up a new regulatory framework, where management of the grid was decoupled from the hydropower production sector.\footnote{See generally \cite{bye05}.} In particular, the act established a system whereby consumers could choose their electricity supplier freely. At the same time, the Act aimed to ensure that producers were granted non-discriminatory access to the electricity grid. This laid the groundwork for what has today become an international market for the sale of electricity, namely the Nord Pool.\footnote{See \url{http://www.nordpoolspot.com/About-us/}. See generally \cite{skjold07,galtung07}.}

In this way, the Energy Act served to establish a market where any actor, privately owned or otherwise, could supply electricity to the grid, and therefore profit commercially from developing hydropower. In response to this, monopoly companies were reorganised, becoming commercial companies that were meant to compete against each other, and against new actors that entered the market.\footnote{See \cite{claes11}.} %The Norwegian state retained a significant stake as shareholders in energy companies, now often alongside private investors. Moreover, as seen in the previous section, Norwegian law continue to favour companies where a majority of the shares are held by the state. To this day, the largest and most influential companies remain under majority public ownership. 
%the recent EFTA Court case, Case E-2/06, \emph{EFTA Surveillance Authority v. The Kingdom of Norway}, EFTA Court Report 2007, p.164. Here, the Court considered the old Norwegian rule of \emph{reversion}, whereby a license to undertake certain large scale hydro-power schemes (strictly speaking, a license to acquire the waterfalls needed to undertake it) came with a special clause that the private developer had to give up ownership to the State after a fixed period of time. This clause was held to be in breach of the EEA agreement since it only applied to private companies. We remark that the Norwegian government responded to this with an amendment after which reversion no longer applies, but which stated that a license to acquire waterfalls for the purpose of such large scale schemes can not be given at all to any company in which private parties own more than 1/3 of the shares.} %The aim of liberalization in Norway was not to minimize state entitlements arising from the hydropower sector. Rather, the intention was restricted to giving consumers greater freedom in choosing their energy-supplier, as well as to enhance efficiency in the sector by introducing competition.\footnote{See for instance \cite{liberal}, which offers a comparative study of the liberalization of the energy sectors in Norway and the UK.} In practice, however, this means that the state is now expected to function as a regular shareholder, which directs its companies to be run according to principles of commercial governance. Indeed, this is to some extent even a legally enforced requirement, e.g., pursuant to the free trade and competition rules of the EEA agreement. 
In addition to commercialisation, the market-orientation of the sector has also lead to centralisation, as many of the locally grounded municipality companies have disappeared as a result of mergers and acquisitions.\footnote{Today, the 15 largest companies, largely controlled by the state and some prosperous city municipalities, own roughly 80\% of Norwegian hydropower, measured in terms of annual output. See \cite[28]{otprp61}. I remark that the process of consolidation started even before the market-oriented reform of the sector. In particular, from 1960 onwards there was a significant push towards centralisation, as the state became a more dominant actor in the hydropower sector. For the state's increasing influence on the sector generally, see \cite{skjold06,thue06b}.} As a result, the local and political grounding of the electricity sector, which used to be ensured through decentralised municipal ownership, has been significantly weakened.

At the same time, the fact that any developer of hydropower is now entitled to connect to the national grid gives private actors a possibility of entering the Norwegian energy market. They may do so not merely as (minority) shareholders in former utilities, but also as {\it competitors}.\footnote{See generally \cite{larsen06,larsen08,larsen12}.}

In the next section, I give a step-by-step presentation of the licensing procedure for hydropower, which serves to put the information presented in this and preceding sections into a dynamic context.

\noo{ The \cite{ea90} contains provisions that lay down basic rules regarding various kinds of licenses needed to construct and operate electric installations, including those needed for the development of hydropower. Usually, however, these licenses are not as controversial or hard to obtain as those pertaining directly to the hydropower plant itself. In addition, they raise no particular issues relating to expropriation or control over water resources, so I will not discuss them further in this thesis. Instead, I will now give a step-by-step description of the licensing procedure for hydropower development, to summarise and elucidate on the rules presented in this and preceding sections.
}
%By far the most important aspect of the act for hydropower development, particularly in cases when different parties are interested in carrying out development, concern the right to connect to the grid. However, this right can become illusory due to the fact that the local grid company -- a monopolist -- has the right to demand contributions from hydropower developers, in so far as the grid needs to be improved in order to handle the output from their hydropower schemes. 
%
%The size of the contribution required, and the technical basis for calculating it, is largely determined by the grid company itself. In many cases, the costs can become so great as to prohibit  profitable development. This can result in conflict, particularly in cases when the grid company is affiliated with a competing hydropower developer who has a commercial interest in preventing other developers from connecting to the local grid. This issue arises in many cases involving expropriation, as the beneficiary is often also the local grid monopolist. In effect, the taker of the waterfall is tasked with calculating the cost of connecting alternative, owner-led, projects to the grid, a crucial factor in determining the  compensation payable. I return to this issue in Chapter \ref{chap:5}. 
%
%In the next section, I consider a recent piece of legislation that creates an additional financial incentive for carrying out hydropower development, by introducing a certificate system to subsidize 
%environmentally friendly energy.
%
%%The most significant step towards liberalization of the Norwegian energy sector was made in 1990 when the Energy Act was passed, an important new piece of statute reorganizing the system for the distribution of electricity.\footnote{Act nr. 50 of 29 of June 1990 relating to the generation, conversion, transmission, trading, distribution and use of electricity.} 
%
%\subsection{Electricity Certificate Act}\label{eca11}
%
%In the 1960s and 70s, hydropower projects in Norway often sparked great controversy, with environmental groups in particular protesting what they saw as unjustifiable destruction of nature in the interest of economic development. Today, however, the environmental interests in hydropower are more divided. On the one hand, many still regard hydropower skeptically as destruction of nature. On the other hand, the increased focus on global warming has led many environmentalists to embrace hydropower as a renewable energy source.
%
%In 2011, the \cite{eca11} was passed, to set up a market for trade in so-called ``green energy'', which would effectively subsidize the further development of renewable energy, including hydropower.\footnote{Act relating to electricity certificates of
%24 June 2011 No. 39.} The basic building block of the new market is the electricity certificate, which will be issued to all renewable energy projects completed before 2020. The demand for such certificates is then created artificially, as the Act stipulates that energy suppliers and certain categories of end-users are required to purchase certificates based on their electricity consumption. 
%
%The Act sets up an incentive to develop hydropower, and it contributes significantly to the profitability of hydropower development.

\subsection{The Licensing Procedure}\label{sec:step}

The water authorities in Norway are centrally organised. The most important body is the Norwegian Water Resources and Energy Directorate (NVE), based in Oslo.\footnote{See \url{www.nvn.no}.} In many cases, the NVE have been delegated authority to grant development licenses themselves, but in case of large-scale development, they only prepare the case, then hand it over to the Ministry of Petroleum and Energy.\footnote{See delegation of 19 December 2000, from the Ministry of Petroleum and Energy (FOR-2000-12-19-1705) and directive of 15 December 2000, from the King in Council (FOR-2000-12-15-1270), pursuant to \cite[64]{wra00}.} The Ministry, in turn, gives its recommendation to the King in Council, who makes the final decision.\footnote{See directive of 15 December 2000, from the King in Council (FOR-2000-12-15-1270).} Parliament must also be consulted for regulations that will yield more than 20 000 natural horsepower.\footnote{See \cite[2]{wra17}.}

%The local municipalities are becoming increasingly marginalised in relation to hydropower management. Their role is usually limited to commenting on the plans, alongside other stakeholders. %At the same time, the \cite{pb08} does play an important role, but not in so far as , which normally grants significant authority to local %municipalities does play an important role. This is because projects over a given threshold requires an impact assessment (IA) to be carried out pursuant to section ..... 

As indicated by the survey of relevant legislation given in previous sections, there are many categories of hydropower projects. Moreover, different categories call for different licenses. Hence, the first step in the application process is for the developer to determine exactly what kind of license they require. This is further complicated by the fact that some categories overlap, since they are based on different measuring sticks for assessing the scale of an hydropower project. 

One important parameter is the power of the hydropower generator, measured in MW (Megawatts). There are four categories of hydropower formulated on this basis: the micro plants (less than $0.1$ MW), the mini plants (less than $1$ MW), the small-scale plants (less than $10$ MW), and the large-scale plants (more than $10$ MW). In practice, one tends to use small-scale hydropower more loosely, to refer to all projects less than 10 MW. Still, a further qualification is sometimes required. For example, the authority to grant a license for a micro or mini plant has been delegated to the regional county councils since 2010, in an effort to reduce the queue of small-scale applications at the NVE.\footnote{See delegation letter from the Ministry of Petroleum and Energy, dated 07 December 2009, available at \url{http://www.nve.no} (accessed 24 August 2014). The county council is an elected regional government institution situated between the municipalities and the central government. There are 19 county councils in Norway as of 01 January 2015. They are comparatively less important than both the municipalities and the central government, but have several  responsibilities, particularly in relation to infrastructure, education and resource management. See generally \cite{berg15}.} The council's decision is based on a (simplified) assessment made by the regional office of the NVE. In addition, licenses for micro and mini plants may be granted even in watercourses that have protected status pursuant to environmental law.\footnote{See Decision no 240, Stortinget (2004-2005), St.prp.nr.75 (2003-2004) and Innst.S.nr.116 (2004-2005).}

For small-scale plants proper, the authority to grant a license is delegated to the NVE, with the Ministry serving as the first instance of appeal.\footnote{See delegation of 19 December 2000, from the Ministry of Petroleum and Energy (FOR-2000-12-19-1705).} For large-scale plants, the granting authority is the King in Council, based on a recommendation from the Ministry.\footnote{See directive of 15 December 2000, from the King in Council (FOR-2000-12-15-1270).} However, in practice, the decision is usually closely based on assessments and recommendations provided by the NVE.\footnote{For a detailed guide to the administrative process for large-scale applications, published by the NVE, see \cite{stokker10}.}

While the relevant licensing authority depends on the effect of the planned plant, the kind of license required depends on a different categorisation, relating to the level of planned water regulation, measured in natural horsepower. Here, there are three categories: run-of-river schemes  (less than $500$ natural horsepower), non-industrial regulations (less than $4000$ natural horsepower), and industrial regulations (more than $4000$ natural horsepower).\footnote{See \cite[2]{wra17} and \cite[1,2]{ica17}.} % 20 000 NatHp), and the very large regulations (> 20 000 NatHp). 

Almost all hydropower schemes require a license pursuant to section 8 of the \cite{wra00}.\footnote{As mentioned in Section \ref{sec:hl}, the exceptions are very small schemes (usually mini or micro) that are deemed to be relatively uncontroversial. Such schemes only require a license pursuant to the \cite{pb08}.} For run-of-river schemes, no further licenses are required for the development itself, although an operating license pursuant to the \cite{ea90} is typically required for the electrical installations.\footnote{See \cite[3-1]{ea90}.} For schemes involving a non-industrial regulation, an additional license pursuant to section 8 of the \cite{wra17} is required. Industrial regulation schemes require yet another license, pursuant to section 2 of the \cite{ica17}.

As is to be expected, the complexity of the licensing procedure tends to increase with the number of different licenses required. However, the licensing applications tend to be dealt with in parallel, so that all licenses are granted at the same time, following a unified assessment. In practice, when the \cite{wra17} applies, it structures the procedure as a whole, also those aspects that pertain to other licenses. 

In addition, yet another categorisation of hydropower schemes is used to determine the relevant application procedure. This categorisation is based on the annual production of the proposed plant, measured in GWh/year. There are three categories: simple schemes (less than $30$ GWh/year), intermediate schemes (less than $40$ GWh/year), and complicated schemes (more than $40$ GWh/year). As mentioned in Section \ref{sec:wra17}, the most important rules in the \cite{wra17} applies to complicated schemes, regardless of whether or not the scheme involves a regulation.\footnote{See \cite[19]{wra00}.} In addition, applications for such schemes must be accompanied by an impact assessment pursuant to section 14-6 of the \cite{pb08}.

This means that the applicant is required to organise a public hearing prior to submitting their formal application, to collect opinions on the project and provide an overview of benefits and negative effects of the plans, particularly as they relate to environmental concerns.\footnote{See directive of 19 December 2014 (FOR-2014-12-19-1758), pursuant to the \cite[1-2,14-6]{pb08}.} In practice, if an impact assessment is required this significantly increased the scope and complexity of the application processing.

For intermediate schemes that do not involve regulation, the rules in the \cite{wra17} do not apply. However, impact assessments {\it may} still be required.\footnote{See \cite[20]{stokker10}.} Here the threshold of 30 GWh/year has been set as an additional threshold by the NVE, who have been delegated authority to require impact assessments for hydropower projects even when these yield less than 40 GWh/year.\footnote{See directive of 19 December 2014 (FOR-2014-12-19-1758).} For the intermediate schemes, NVE decides whether an impact assessment is required on a case-by-case basis. For simple schemes, on the other hand, impact assessments will not be required. Such schemes make up the core of what is described as small-scale hydropower in daily language.

The time from application to decision can vary widely, depending on the complexity of the case, the level of controversy it raises, and the priority it receives by the licensing authority. Usually, the assessment stage itself will last 1-3 years, sometimes longer.\footnote{See \cite[84-85]{nou129}.} While large-scale schemes involve more complicated procedures, they are also typically given higher priority than small-scale schemes. In recent years, following the surge of interest of small-scale development, a processing queue has formed at the NVE.\footnote{See \cite[84]{nou129}.} This means that small-scale applications typically have to wait a long time, sometimes several years, before the NVE begins processing them.\footnote{See \cite[84]{nou129}.}

As I will discuss in more depth in the next chapter, the issue of expropriation is rarely given special attention during the application assessment. This is so even in cases when an application to expropriate waterfalls is submitted alongside the licensing applications. The issue of expropriation is rarely singled out for special treatment, at least not in cases of large-scale development. Moreover, as mentioned in Section \ref{sec:hl}, an automatic right to expropriate follows from section 16 of the \cite{wra17}.

This rule is not understood to cover the right to harness hydropower as such, but it {\it is} understood to cover the right to divert water away from river systems where the applicant has no previous riparian rights.\footnote{See \cite{jorpeland11}.} This is a {\it de facto} expropriation of a riparian right, and it is recognised as such in relation to the issue of compensation. However, it does not count as expropriation of a right to harness hydropower. Recently, the Supreme Court held that because of this,  many of the procedural rules that ordinarily apply to expropriation of riparian rights are not relevant.\footnote{See \cite{jorpeland11}.} Rather, the procedural rules and practices related to the licensing procedure are considered exhaustive.

These rules and practices pay little attention to the interests of local owners and the immediate local community. Usually, the only locally grounded actor that is recognised as playing an active role in the process is the municipality. However, even the role of the municipality is limited. Once a license is granted according to a sector-specific statute, no regular planning license needs to be obtained from the municipality government.\footnote{See \cite[12-1]{pb08}.} However, the municipalities must be notified of any application that might affect their interests, and they are expected to express their views.\footnote{See \cite[8]{wra17} and \cite[24]{wra00}.} In addition, they may protest against the plans using a form of objection that requires the NVE to enter into dialogue with them about possible changes and improvements.\footnote{See \cite[5-4,5-6]{pb08}, c.f., \cite[24]{wra00}.} If an agreement is not reached, a license can still be approved, although then always by the Ministry, not the NVE.\footnote{See \cite[5-6]{pb08}.}

The local owners are in a much weaker position. The NVE is not even obliged to notify them of licensing applications concerning their riparian rights. Rather, it is expected that the applicant notifies affected owners when submitting a license application. It is not established practice for the NVE to check that the applicant has fulfilled their obligation in this regard.\footnote{See \cite{jorpeland11}, and the discussion in Chapter \ref{chap:4}, Section \ref{sec:jorpeland}.} The applicant is also responsible for several other aspects of the assessment process, including the assessment of possible alternatives.\footnote{See \cite{stokker10}. For a concrete example of its effect in expropriation cases, I refer to \cite{jorpeland11} and the detailed assessment of this case offered in the next chapter.}

One might think that this would raise competency questions, particularly in cases involving expropriation. However, even for such cases it is established practice for the NVE to delegate to the applicant much of the responsibility for preparing the assessment material, a practice that the Supreme Court has accepted.\footnote{See \cite{jorpeland11}.}

In cases that fall under the \cite{wra17}, the NVE must send its final report and recommendation to the interested parties for comments.\footnote{See \cite[6]{wra17}.} It is established practice that local owners do {\it not} count as interested parties in this regard.\footnote{See \cite{jorpeland11}.} This includes the owners of those rivers and waterfalls that the applicant wishes to expropriate. Hence, while the municipalities and various environmental interest groups are informed of how the case progresses and asked to comment prior to the final decision, the owners must inquire on their own accord if they wish to be kept up to date on the application process.

The lack of procedural safeguards protecting the interests of local owners is reflected in the kind of assessments that tend to be carried out. It is typical for assessments to focus primarily on the benefit of increased electricity production weighed against the negative effects on the natural environment. This, indeed, is the perspective that permeates the whole system, from the rules setting out the expected content of applications, through to the procedures followed when assessing cases, towards the criteria used to determine if a license should be granted.\footnote{See \cite{stokker10}.} As a result, the opinions of environmental groups and expert agencies are typically taken seriously, while the owners typically struggle to make an impact.\footnote{See \cite{jorpeland11}.}

To sum up, hydropower cases are assessed from within an expert-based system of governance, which also relies heavily on data that is collected and presented by the applicant. Political voices tend to remain fairly distant, although special interest groups can still play an important role. Municipality companies have been replaced by commercial actors on the applicant side, while the most important administrative decision-maker, the NVE, is a centralised, expert-based, directorate.

This is the procedural context of Norwegian hydropower, entrenched in law. In the next section, I will shift attention towards practical reality, focusing on the changes that resulted from the liberalisation reform of the early 1990s. I will begin by considering the established part of the sector, by presenting the ownership and management structures surrounding large-scale plants and the management of the national grid. Then I go on to consider specifically the surge of interest in small-scale hydropower, which represents an important counterweight to the process of centralisation that has followed in the wake of the reform.

%
%
%
%
%of unit of power 
%
%
%For the local owners of waterfalls, the situation is worse, since they are not identified as stakeholders in large scale projects. They are not, in particular, mentioned in the Watercourse Regulation Act, Section 6, which regulates the steps that must be taken when preparing such cases.\footnote{Nor do the seem to be mentioned  in any of the documents setting out how the authorities deal with such cases in practice. See, for instance, the guide published by NVE \cite{rettleiar} (in Norwegian), directed at applicants, and setting out how NVE deals with cases involving large scale hydro-power.} Consequently, it is hardly surprising that in administrative practice, it has been uncommon to devote particular attention to local owners. Rather, the focus has typically been on environmental issues and the opinions of various interest groups, such as hunter or fishermen's associations.\footnote{For a more in depth account of the process, we point to the standard legal reference on Norwegian water law \cite{falk}(in Norwegian).}

\section{Hydropower in Practice}

The history of hydropower in Norway can be roughly divided into four stages. The first stage was the development that took place prior to 1909. During this time, private actors dominated, with public ownership playing a minor role.\footnote{See \cite{otprp61}.} Moreover, there were many private interests speculating in acquiring Norwegian waterfalls, anticipating the value that these would have for industrial development.\footnote{See \cite[30-31]{nou04}.}

After 1909, the introduction of licensing obligations and the rule of reversion meant that the state gained increased control over Norwegian water resources. At the same time, local municipalities began to invest in hydropower to provide electricity to its citizens, a service they were increasingly being obliged to provide.\footnote{See \cite{otprp61}.} This marked the start of the second stage of hydropower development, which saw the development of a more tightly regulated, yet still decentralised, hydropower sector.

In fact, throughout the first half of the 20th century, most hydroelectric plants were small-scale plants that supplied local communities with electricity.\footnote{See \cite[11]{utbygd46}. This is a report from the water directorate published in 1946, showing that as of 31 December 1943, $97.8 \%$ of all hydroelectric plants in Norway were small-scale plants. However, these plants contributed only $28 \%$ of the total hydroelectric power installed at that time.} Moreover, as of 31 December 1943, $89 \%$ of all hydroelectric power stations in Norway were still private, many of which were mini and micro plants that were owned and operated by the local community.\footnote{See \cite[6]{utbygd46}. See also \cite[111]{hindrum94}.} However, many bigger plants were also under private ownership, with $57 \%$ of the total hydroelectric power available at this time supplied by the private sector. Moreover, while the micro and mini plants accounted for $72.9 \%$ of the total number of plants, they only accounted for $1.6 \%$ of the total electricity supply.\footnote{See \cite[7]{utbygd46}.}

By the end of 1943, $80 \%$ of the Norwegian population had access to electricity. In scarcely populated rural areas, the corresponding figure was $70 \%$.\footnote{See \cite[7]{utbygd46}.} This shows that the decentralised approach to hydropower development, based on private ownership and local control, succeeded quite well in supplying electricity to the country's population.

However, the regulatory regime was soon to undergo a significant change, designed to facilitate industrial development and increased state control. This change came quite rapidly after the Second World War, when the central government began to invest heavily in hydropower, often to ensure economic development by subsidising the metallurgical industry.\footnote{See \cite[59-65]{thue96}.} This period saw increased marginalisation of small private electricity companies, as well as local owners.\footnote{At the same time, powerful (private) metallurgical interests benefited greatly, sometimes also at the expense of the general supply of electricity. See \cite[65-71]{tvedt96}.} Indeed, it was often demanded, as a condition for allowing local communities access to the national electricity grid, that local hydroelectric  plants had to be shut down.\footnote{See \cite[p.111]{hindrum94}.} During this time, the development of hydropower was seen as an important aspect of rebuilding the nation, a task carried out in the public interest, not primarily to supply the public with electricity, but rather to facilitate a specific kind of economic development that national politicians deemed desirable.\footnote{See \cite[59]{thue96}.}

The state-dominated system set up on this 
basis remained in place until the 1970s, when environmental concerns and discontent among the municipality governments began to emerge with greater force.\footnote{See \cite[71-75]{thue96}.} At the same time, the typical development project had grown both in scale and complexity, making environmental worries and local demands for increased benefit sharing more convincing.\footnote{See \cite[73]{thue96}.} Several actors would frequently oppose large-scale hydroelectric development, including environmental interest groups, local communities, as well as municipal and regional government institutions.\footnote{See \cite[71-72]{thue96}.} The typical response from the state was to introduce measures that sought to pacify the regional and municipal government opposition, which was structurally more serious than opposition from local people and environmental groups. The typical approach was to grant an increased share of the financial benefit to local and regional institutions of government, to instil support for state-led development plans.\footnote{See \cite[73-76]{nilsen08}.} This generally worked quite well, but also to some extent limited the centralisation process and the state's power over the hydroelectric sector.\footnote{See \cite[85]{thue96}.}

The fourth state of hydropower development began in 1990 after the passage of the \cite{ea90}. The liberalisation that followed saw the transformation of the hydropower sector into a commercial market, based on profit-maximising and competition. As a result, the structure of decentralised management withered away further, as many municipality companies were either bought up by more commercially aggressive actors or forced to merge and change their business practices in order to remain competitive.\footnote{See \cite[583]{bibow03} (commenting on the increased consolidation of power on the electricity market, following acquisitions and mergers after 1990).} At the same time, a new decentralised force emerged in the sector, in the form of local owner-led projects.\footnote{See Section \ref{sec:small} below.}

%While private, these actors are not comparable to the industrial speculators that prompted the legislative response in the early 20th century. Rather, they arose as a reflection of the egalitarian system of land ownership in Norway, carrying forward an ancient tradition of local control over water resources that had been temporarily superseded by the political legitimacy of local municipalities.  
%As mentioned, owners of waterfalls are typically groups of ordinary local residents, most often farmers. With farming on the decline in Norway, particularly in those parts rich in water power, this development represents a chance for rejuvenation for rural communities. 

The core idea behind the \cite{ea90} was that the electricity sector should be restructured in such a way that production and sale of electricity, activities deemed suitable for market regulation, would be kept organisationally separate from electricity distribution over the national grid, a natural monopoly. However, the act itself does not explain in any depth how this is to be achieved. In practice, moreover, the divide has not been strictly implemented. In particular, most of the large energy companies in Norway continue to maintain interests in both distribution, production and sale of electricity, a phenomenon known as ``vertical integration''.\footnote{See \cite[580-583]{bibow03}.} In fact, since the liberalisation reform also caused centralisation, the degree of vertical integration in the electricity sector has increased since the passage of the \cite{ea90}.\footnote{See \cite[583]{bibow03}.}

The water authorities try to respond to this, particularly by making use of their authority to give organisational directives when they grant distribution licenses.\footnote{See \cite[4-1]{ea90}, para 2, no 1.} Moreover, electricity companies are required to keep separate accounts for production, distribution and sale of electricity.\footnote{See directive of 11 March 1999 (FOR-1999-03-11-302), s 4-4 a and s 2-6, issued by the NVE pursuant to directive of 7 December 1990 (FOR-1990-12-07-959), s 9-1, cf., \cite[10-6]{ea90}.} It is also required that transactions across these functional divides are clearly marked, and that they are based on market prices.\footnote{See directive of 11 March 1999 (FOR-1999-03-11-302), s 2-8.} Moreover, the NVE serves a control function in this regard, as they review the accounts of distributors on an annual basis.\footnote{See directive of 11 March 1999 (FOR-1999-03-11-302), s 2-1.}

The water authorities have also sometimes required that a separate company is set up to manage the distribution activities.\footnote{See \cite[581-582]{bibow03}.} However, it is permitted for this reorganisation to take place through the formation of a conglomerate, under a single parent company that controls both the distribution company, the production company and the sales company. Indeed, this model has now been implemented by most of the large energy companies in Norway.\footcite[582]{bibow03}

It seems unclear whether this approach really achieves the stated objective. By adopting the conglomerate model of organisation, the major players on the market have successfully gained control over a larger share of both the production and distribution facilities for electricity. Hence, these actors effectively control the core infrastructure that makes up the backbone of the Norwegian electricity sector. The {\it intention} is that monopoly power should only be exercised with respect to the distribution grid on strictly regulated, non-discriminatory, terms. But is this realistic when the conglomerate including the grid operator has significant stakes in production and sale of electricity?

This question calls for a separate study, and I will not be able to address it in any depth here. However, I will direct attention at one aspect of this that arises with particular urgency for small-scale development of hydropower. It is quite common, in particular, that small-scale projects remain unrealised because the grid is regarded to lack sufficient capacity to accommodate new electricity.\footnote{See, e.g., \cite[84,161-162]{nou129}.} The distribution company is authorised to deny access in such cases, in keeping with their responsibility for providing an efficient and stable public service.\footnote{See \cite[3-4]{ea90}, c.f., directive of 7 December 1990 (FOR-1990-12-07-959), s 3-4.}

Often, the distribution company will be a sister company of an energy producer operating in direct competition with the company seeking access. This raises obvious questions about the impartiality of the assessments carried out by the distribution company. In expropriation cases, this tends to become a thorny issue, particularly in relation to the assessment of the cost of undertaking alternative development schemes. Riparian owners are rarely pleased when they realise that the expropriating party is part of the same conglomerate as the grid company that estimates the grid connection costs associated with owner-led development.

The water authorities themselves have recognised that access rights soon become illusory if it is too easy for the grid companies to deny access based on efficiency considerations.\footnote{See \cite{otprp62}.} At the same time, they point to the need for responsible management of the national grid, which, as they see it, requires delegation of authority to the grid companies. Hence, the authorities are left with a dilemma. So far, they have responded to this mainly by issuing more regulation, not by attempting to reduce the level of power-concentration in the electricity sector.

The Energy Act was changed in 2009, such that grid companies are now subjected to a more wide-reaching duty to allow access for producers, even when this necessitates new investments.\footnote{See Act no 105 of 19 June 2009 regarding changes in the \cite{ea90}.} But who should pay, and how much? This is often unclear, and while the water authorities have a supervisory function, it is the grid companies themselves that determine this in the first instance.\footnote{See directive of 7 December 1990 (FOR-1990-12-07-959), s 3-4.} In addition, grid companies may still deny access in cases when the needed investments are not ``socio-economically rational''.\footnote{See \cite[3-4]{ea90}. The authority to decide whether this requirement is fulfilled is vested with the Ministry.} Hence, it seems that these new rules only push the question of fairness and accountability further into the details of the decision-making process, without addressing the underlying problem of power concentration. %Unsurprisingly, controversies continue to arise, particularly when owner-led and small-scale projects remain unfulfilled due to grid constraints.

%Functional division was only one of two core intentions behind the liberalization of the 1990s. The other aim, which has been fulfilled to a greater degree, was for centralization. Indeed, since 1990 there has been an unprecedented consolidation of power compared to the earlier days when municipality-owned companies dominated the sector. While this was an aim of the reform, intended to bring about greater efficiency, enhanced expertise and increased competitiveness, it has later been criticized. It has been pointed out, in particular, that the practical consequence of liberalization has been that the local accountability of the electricity sector has been lost, both organizationally and politically.\footnote{See \cite{agnell11}. 

As I have already mentioned, the market-orientation of the electricity sector has reduced the level of political control and accountability. Today, a management model based on economic rationality and expert rule has become dominant. According to Brekke and Sataøen, this serves to set the reform that took place in Norway apart from similar energy reforms in Sweden and the UK.\footnote{See \cite{brekke12}.} Moreover, Brekke and Sataøen argue that this might be an underlying cause of some recent controversies, particularly with regards to the development of the national grid. 

The most serious case so far is that of {\it Sima - Samnanger}, a new distribution line that will cut through the area known as {\it Hardanger}, a scenic part of south-western Norway. The local population vigorously protested the plans.\footnote{See \cite[22-23]{brekke12}.} It would destroy a valuable part of Norwegian nature, they argued, without providing the people living there with anything in return.\footcite[26-27]{brekke12} Despite an extensive campaign against the plans, the government still pushed ahead, arguably demonstrating both the power of the central government and its willingness to use it in cases involving  management of the national grid.\footcite[27]{brekke12}

The growth of the small-scale hydropower sector acts as a counterweight to centralised management and expert rule, giving local communities and owners a new voice, as market participants. Since the mid- to late 1990s, the small-scale sector has grown significantly. It has been estimated that about one third of the remaining potential for hydropower in Norway, measured in annual energy output, will come from small-scale projects.\footnote{See \cite[231]{nou129}.}

Many established energy companies have entered into the small-scale market, but they are facing serious competition from new actors, several of which are  owner-controlled and locally based. This development has been a counterweight to the increasingly centralised ownership pattern in the hydroelectric sector. In many ways, owner-led and owner-cooperating companies have replaced the municipality companies as the local anchor of Norwegian hydropower.

In a recent report, the potential for profitable small-scale hydropower projects was estimated to be around 20 TWh/year.\footnote{See \cite{aanesland09}. For comparison, suggesting the scale of this potential, I mention that the total consumption of electricity in Norway in 2011 amounted to 114 TWh, see \url{http://www.ssb.no/en/energi-og-industri/statistikker/elektrisitetaar}.} On this basis, the authors of the report estimate that the total present-day value of the waterfalls suitable for small-scale hydropower is about 35 billion Norwegian kroner, i.e., about 3.5 billion pounds.\footnote{See \cite[1]{aanesland09}.} This calculation is based on a model where the waterfalls are exploited in cooperation with a commercial company, {\it Småkraft AS}. It might be an underestimate of what small scale hydropower could represent for local communities if they remain in charge of development themselves.

Small-scale hydropower has become socially and political significant in Norway. In the report mentioned above, it is estimated that the value of rivers and waterfalls amount to just under 50 \% of the total equity in Norwegian agriculture.\footcite[1]{aanesland09} Moreover, hydropower is increasingly seen as a possibility for declining regions to counter depopulation and poverty. It promises to give these communities a chance to regain some autonomy and influence with respect to how local natural resources should be managed. In some regions, small-scale hydropower is the only growth industry. Hence, it takes on great political and social importance, not just for the owners of waterfalls, but for the community as a whole.

For an example of a community where small-scale hydropower has played such a role, I point to Gloppen, a municipality in the county of Sogn og Fjordane, in the western part of Norway. 19 schemes have already been successfully carried out, all except one by local owners themselves, amounting to a total production of over 250 GWh/year. This prompted the mayor to comment that ``small scale hydro-power is in our blood''.\footnote{See \cite{starheim12}.} When interviewed, he also directed attention at the fact that hydropower had many positive ripple effects, since it significantly increased local investment in other industries, particularly agriculture, which had been severely on the decline.

To achieve such effects, it is important to organise development in an appropriate manner. Moreover, to explain how waterfalls came to be as valuable as they are today, it is crucial to direct attention to the way in which waterfall owners initially asserted themselves on the market. In the following, I do this by giving an in-depth presentation of an early model for local involvement in hydropower development, published at a seminar in 1996.\footnote{See \cite{dyrkolbotn96}.} This model contains an early expression of several ideas that would prove influential to the development of the small-scale hydropower sector.

However, certain other aspects of the model have not been widely adopted. These are aspects that pertain to the balance of power between owners and developers, as well as the relationship that should be established with larger communities of non-owners, including environmental groups and other water users. Hence, considering the model in some depth, and assessing its impact, will allow me to shed light on desirable social functions of waterfall ownership, and the extent to which such functions are fulfilled on the market today.

%In the following I will discuss this part of the sector in more depth, by tracking the development of the current practices that characterize their operations on the market. 

%In this section, I will present the typical steps involved in obtaining a license to develop hydropower. As I have already discussed, different rules apply to different kinds of projects. Here I will focus on the common elements, particularly those that relate to the role of local owners and communities. In this respect, the various procedures do not differ significantly, although the more extensive application process required for large schemes will obviously tend to result in greater attention directed at the plans, something that  might in turn result in greater local involvement in the decision-making process.\footnote{The most important procedural distinction is made between those applications that require an impact assessment (IA) and those that do not. The current rule is that projects that will produce more than 40 GWh/year require an impact assessment, while projects that will produce more than 30 GWh/year might require an impact assessment, as determined by the NVE. See {\it Directive Regarding Impact Assessment} of 26 June 2009, issued by the Ministry of the Environment pursuant to sections 4-2 and 14-6 of the \cite{pb08}.}

%Before the introduction of a centralized system of management of water resources, local farmers made extensive use of the power inherent in water. According to Terje Tvedt, there were 20 -- 30 000 watermills in Norway by the 1830s.\footnote{References} These were not used to produce electricity, but as grist mills, to create flour. This pattern of use rendered waterfalls of limited interest to outside investors or the state, so no significant pressures were placed on the local communities and their exercise of self-governance. On the other hand, Tvedt also stresses the importance of water, including its political implications, and how they gave local communities a better basis on which to claim a right to self-governance also in other matters. Indeed, the control over water, emerging as a side-effect of the control over land, played an important role in empowering Norwegian farmers and their communities in the 19th century. 
%
%Towards the end of the 19th century, local self-governance of water came under increasing pressure from outside investors. This happened in response to increased interest in hydropower as a potential source of energy for industrial exploits. Indeed, more and more waterfalls were purchased as mere objects of speculation, often by foreigners. As a result, the Norwegian government felt compelled to introduce regulation to protect national interests. This led to the passing of the predecessors of the \cite{ica17} and \cite{wra17}, referred to in Norway as the ``panic acts'' of 1909. They set up the basic license requirement for purchase of waterfalls and development of large-scale hydropower involving regulation, while also creating a space for the state as a marker player, by introducing the rule of reversion. 
%
%Still, however, development of hydropower was mostly undertaken by private companies, either large industrial companies like Norsk Hydro AS, or else community-owned companies that wanted to supply electricity locally. It was not until after WW2 that the state assumed the role as the leading hydropower developer in Norway. This was also when the national electricity grid was established and put under direct state control. Previously, the grid had been operated by a mix of state and private actors, who had organized themselves in a joint umbrella organization. Now, however, hydropower development for electricity production was linked with management of the grid, by the establishment of municipal and state-controlled electricity providers, organized as public bodies rather than commercial companies. 
%
%This area saw the extensive use of expropriation to facilitate hydropower. In1940, an act was passed which provided the necessary authority to compulsorily acquire waterfalls for hydropower. This act explicitly stipulated that expropriation should only take place in favor of projects that would serve the electricity supply in the local area. Hence, at this time, the use of expropriation to further the development of hydropower was specifically linked to the idea that the electricity supply should be organized as a non-commercial public service.
%
%During this time, many local hydropower plants that had previously supplied electricity locally were shut down. In particular, they were not allowed to connect to the national grid. Moreover, the grid authorities would often explicitly require these plants to shut down as a precondition to allow the local community to connect to the national grid. The waterfall owners were left marginalized during this time, as they had no real opportunity to develop hydropower themselves, or sell their rights to anyone else than the local monopolist. 
%
%However, as I will discuss in more detail in Chapter \ref{chap:5}, a system was developed -- not through legislation, but by the appraisal courts -- to ensure that the owners would receive at least some compensation for their waterfalls. This was despite the fact that under the prevailing regulatory regime, waterfalls were practically worthless to local owners. Hence, the compensation measures adopted reflected the tradition for local control over waterfalls. If such measures had not been introduced, in particular, local owners would hardly be entitled to any compensation at all under ordinary Norwegian expropriation law.
%
%Following liberalization of the energy sector, the basic premise for this way of organizing hydropower development was lost. In particular, hydropower development was now to be regarded as a commercial enterprise, within a market-based system. This removed the conceptual legitimacy of a system that marginalized local owners. Moreover, the most important practical impediment to owner-led development was also removed, as non-discriminatory access to the grid was provided for in statute. 
%
%As a result, there has been a surge of interest in the exploitation of waterfalls in owner-led hydropower schemes. In addition, the established energy companies have, to some extent, embraced this new situation by competing also in attempting to strike deals with the local owners. In this way, they recognize the right of self-governance and owner-management of hydropower, thereby helping to further undermine the rationale behind using expropriation. 
%
%But not all companies have responded in this way. Many of the largest and most influential companies ahve refused to alter their practices, and expect that they will still be able to enjoy the use of expropriation. These actors, which are often partly state-owned, refuse to recognize the right of waterfall owners to assume a decisive role in the mangement of their resource. Moreover, the are unwilling to enter the market for waterfalls in competition with actors that wish to collaborate with local owners. This has created considerable tension in recent years. 

%As we have mentioned, the typical owners of Norwegian waterfalls are communities of farmers and smallholders. Historically, the right to land-based resources, especially in the mountainous areas of the west and the north, where most valuable waterfalls are located, was held by local people, the same people who made use of it on a day to day basis. The main reason for this, which by European standards stands out as quite unusual, was that Norway never really had a separate class of landed nobility. Consequently, the Norwegian farmer occupied a position of relative autonomy and freedom, even to the point of exercising significant political influence, especially in the early days of Norwegian democracy.\footnote{During the 19th century the two dominant group in Norwegian politics were the farmers and the civil servants, and the former group exercised great influence in the Norwegian parliament, with the 1833 election leading to what became known as the farmers parliament. The "classic" academic treatment of farmers' influence over 19th century Norwegian politics is \cite{Koht} (in Norwegian). More on the author and a summary of his views can be found here, http://en.wikipedia.org/wiki/Halvdan\_Koht.} Following industrialization, however, their role became much more marginal, and farming has steadily become more and more unprofitable, with many farming communities having already disappeared, and many others threatened by depopulation. In light of this, the possibility of undertaking small scale hydro-power is often seen as being important to the survival of rural communities themselves, not just as a means for individual members of such communities to make a profit.

\section{{\it Nordhordlandsmodellen}}

In five brief points, the {\it Nordhordlandsmodellen} sets out a framework for cooperation between waterfall owners, professional hydroelectricity companies, local communities, and greater society.\footnote{See \cite{dyrkolbotn96}. The model was presented at a seminar in 1996, as the result of a collaboration between Otto Dyrkolbotn, a farmer and a lawyer, and Arne Steen, the director of {\it Nordhordland Kraftlag}, a municipality-owned energy company.} %The first three of these points would later provide the blueprint for many commercial actors hoping to cooperate with local owner. The last two points, which seek to address environmental concerns and the interests of the community more broadly, have largely been ignored.
The first point makes clear that the aim of cooperation should be to ensure local ownership and control: external interests should never be allowed to hold more than 50 \% of the shares in the development company. If this company is organised as a limited liability firm, then the plan stipulates that local residents -- not necessarily owners -- are to be given a right of preemption in the event that shares come up for sale. The possibility of organising the development company as a local cooperative is also mentioned.\footnote{References needed.}

The second point of the model sets out a method for valuing the riparian rights prior to development. It stipulates that the appraisement should reflect the real value of such rights, normally estimated on the basis of lease capitalisation. That is, one assumes first that the riparian owners are entitled to rent based on the level of annual production in the planned hydropower project. Then, for the purpose of appraisement, the expected rent is capitalised to find the present value of the riparian rights, relative to the development project in question.\footnote{This approach stands in stark contrast to the earlier valuation method used in the electricity sector, which relied on a purely theoretical assessment based on the aforementioned notion of a natural horsepower. See \cite{dyrkolbotn15,hellandsfoss97}.}

After such a value has been calculated, the model stipulates that owners are to be given a choice of either leasing out their water rights to receive rent, or to use the capitalised value of (part of) this rent as equity to acquire shares in the development company. The third point in the model then offers a clarification, by stating that the development company should not in any event acquire ownership of riparian rights, but only a time-limited right of use. After 25-35 years, this usufruct should fall away and full dominion over the river should revert back to the landowners, free of charge. 
This is the proposed rule even in cases when the landowners themselves initially control the majority of the shares in the development company. 

The model goes on to demonstrate the commercial viability of this organisational model, by pointing to a concrete municipality-owned energy company that stated its willingness to cooperate with owners on such terms, to help with financing and share the risk.\footnote{The company in question is Nordhordland Kraftlag, where one of the authors of the model, Arne Steen, was a director.}
The fourth and fifth points of the model describe the intended role of the local development company in society, by making clear that hydropower development should not take place in isolation from other interests and potential uses of the affected river. Rather, potential developers should take on formal obligations towards other user groups. Moreover, obligations should not only be negatively defined, as duties to minimise or avoid harms. Positive obligations should also be introduced, such as duties to improve other qualities of the river system, and to engage in active cooperation with other users.

It is explicitly stated that environmental concerns should be given due regard. As an illustration of a positive obligation arising from this, the model goes on to make clear that less invasive projects may have to be considered, even if this is not prescribed by the authorities. In some cases, environmental concerns can suggest solutions that are not economically optimal. In addition, fishing and tourism are mentioned as concrete examples of other water uses that the hydropower company should actively seek to promote.

The overall aim, it is made clear, is to ensure sustainable management of the river system as a whole. Interestingly, the model predicts that local ownership will make this easier, by making a structural  contribution to sustainability that exceeds what can be achieved through governmental regulation alone. This claim is then illustrated by a concrete example of a case in which the local owners decided to pursue a scheme that was less invasive than the project endorsed by the water authorities.\footnote{Today, this project has become Svartdalen Kraftverk, finalised in 2006. It produces 30 GWh annually, enough electricity for about 1500 households, see \url{http://no.wikipedia.org/wiki/Svartdalen_kraftverk}.}

The model goes on to emphasise the need for integrated processes of resource planning and decision-making, to ensure that hydropower development is not approached as an isolated economic and environmental concern, but looked at in a broader social and political context. To achieve this, it is argued that local communities need to play an important role in the management of water resources. Another concrete example follows, regarding the master plan for {\it Romarheimsvassdraget}, a river system in the municipality of Lindås, in the county of Hordaland.

This river system was originally intended for large-scale development undertaken by BKK AS, with no involvement of local owners.\footnote{BKK AS is one of the 15 biggest hydropower companies in Norway, and would later also purchase Nordhordland Kraftlag.} The project would involve a total of three river systems, such that the water from {\it Romarheimselva} and another river would be diverted to a neighbouring municipality for hydropower development there. After local owners got involved in the planning, they argued against these plans. Eventually, they were successful, as the NVE agreed to endorse an alternative consisting of 7 distinct run-of-river projects undertaken in cooperation with local owners.\footnote{See {\it Vassragsrapport nr. 25}, Direktoratet for Naturforvaltning, 1999.}

It is important to note that when {\it Nordhordlandsmodellen} was formulated, owner-led development of hydropower was still a recent phenomenon, driven forward by individual owners and local groups that saw the potential and had enough know-how to get organised. Later, however, commercial companies emerged that specialised in cooperating with local owners.\footnote{For a good survey of later developments, I point to \cite{larsen06,larsen08,larsen12}.} Today, many such companies operate, making it relatively easy to initiate a process of owner-led development. Moreover, owners that are not themselves aware of the potential inherent in their riparian rights may be approached by interested commercial actors. These actors will then tend to compete for the chance of striking a deal with the owners. Most of them rely on cooperation on terms that reflect the main ideas expressed in the first three points of {\it Nordhordlandsmodellen}.

However, several adjustments have become standard, adjustments which systematically benefit the external partner: the requirement that locals should at all times control a majority of the shares is dropped, the period of usufruct is typically longer than 35 years, the reversion to the landowners after this time is made conditional on payment for machines and installations, and no preemption rights are granted to local residents. Importantly, however, the core idea that riparian rights are to be valued based on a capitalisation of future rent is accepted. This means, in turn, that local owners rarely need to raise any additional capital to acquire shares in the development company. Moreover, the rent itself can become a significant source of income.

There are two main approaches to calculating this rent. The first approach, introduced already in {\it Nordhordlandsmodellen}, specifies the rent as a percentage of the gross income from sale of electricity, today often around 10-20 \%.\footnote{Source: contracts presented to the court in \cite{sauda09} (available from the author upon request). See also \cite[55-57]{hauge15}.} In this way, passive owners need not take on any risk related to the performance of the hydropower company. The second approach has been developed by the company Småkraft AS, which is now the leading market actor specialising in cooperation with local owners.\footnote{It is owned by several large-scale actors on the energy market, see \url{www.smaakraft.no}.} According to their model, riparian owners are paid a share of the annual {\it surplus} from hydropower generation.\footnote{See \cite[57-60]{hauge15} (also discussing variants of this contractual idea, based on how the surplus is actually defined in the contract).}

This share is usually higher than the rent payable based on the net income; often, the owners are entitled to $50 \%$ of the profit.\footnote{Source: contracts presented to the court in \cite{sauda09} (available from the author upon request). See also \cite[58]{hauge15}.} Hence, if the project is a success, the riparian owners might be better compensated. However, they do accept some risks as though they were shareholders, and they do so even though they might not have much of a say in how the company is run.\footnote{To limit the risk for owners, companies such as Småkraft AS also operates a system of ``guaranteed'' rent, but this rent is usually quite a lot less than what the owners could expect from an agreement based solely on rent based on gross income. Source: contracts presented to the court in \cite{sauda09} (available from the author upon request).}

To illustrate the financial scale of the rent agreements that have now become standard, let us consider a typical small-scale hydropower plant that produces 10 GWh annually. With an electricity price of 0.3 NOK/KWh, this gives the hydropower plant an annual gross income of NOK 3 million. If the rent payable is 20 \%, the waterfall owners will receive NOK 600 000 annually, approximately GBP 60 000. By contrast, if the rights were expropriated, the traditional method of calculating compensation would be unlikely to result in more than NOK 600 000 as a {\it one-time payment} for a waterfall that yields 10 GWh/year.\footnote{For further details on the compensation issue, see \cite{dyrkolbotn14,dyrkolbotn15,dyrkolbotn15a}. Sometimes, the difference in valuation would be even greater, since the natural horsepower of a development project is highly sensitive to the level of regulation of the waterfall, much more so than the value of the development. For an demonstration of how this affected compensation according to the natural horsepower method, one may consider the case \cite{hellandsfoss97}, which went to the Supreme Court. Here the owners were paid just over NOK 1 million for a waterfall that would yield 100 GWh/year.}

Hence, the financial consequences of the ideas expressed in {\it Norhordlandsmodellen} have been dramatic. At the same time, it is clear that the latter two points of the model, addressing the importance of responsible and inclusive management of river systems, have not had the same degree of influence on the market. In the next section, I address this in more depth and comment on some recent developments that threaten to undermine the status of small-scale development as a sustainable alternative to large-scale exploitation. I argue, in particular, that the future of hydropower will likely leave local owners and their communities marginalised once again, unless a social function approach to small-scale development is adopted and entrenched in the law.

\section{The Future of Hydropower}\label{sec:future}

In recent years, there has been a growing tension between the small-scale hydropower sector and environmental groups. There is talk of a brewing ``hydropower battle'', as environmentalists grow increasingly critical of what they regard as predatory practices.\footnote{See \cite{haltbrekken12}.}
Reports on small-scale producers who violate regulations help fuel the negative impression of the industry.\footnote{In 2010, the NVE conducted randomised inspections and announced that 4 out of 5 mini and micro plants operated in violation of regulations pertaining to the amount of water they may use at any given time. See \cite{ulovlig10}. In the largest newspaper in Norway, this was reported under the heading that four out of five small-scale plants break the law, see \cite{ulovlig10b}. This is misleading, since mini and micro plants are distinct from small-scale plants proper. Most importantly, the former kinds of plants do not usually require a sector-specific development license. Because of this, it also seems plausible that the reported violations might in large part be due to a lack of knowledge and professionalism, not predation. I remark that questions later emerged regarding the accuracy of the report itself. Apparently, one of the plants that was reported to have violated regulations did not even exist, see \cite{tvilsom10}.} At the same time, the price of electricity has been much lower in recent years than what had previously been forecast, causing severe financial difficulty for many small-scale developers.\footnote{See \cite{sunde14}.} This has also revealed that some of the actors on the market have engaged in speculative practices, by aggressively entering into agreements with local owners, without carrying out much actual hydropower development.\footnote{See \cite{endresen14}.}

On the regulatory side, the water authorities have announced that they will adopt stricter procedures to assess licenses for small-scale hydropower.\footnote{See \cite{lie12}.} In addition, different planning routines have been adopted to ensure that small-scale schemes are no longer considered individually, but in so-called ``packages'', collecting together applications from the same area. As a consequence of these changes, the number of rejected applications have increased dramatically in recent years.\footnote{In 2013, the number of rejections tripled compared to previous years, while the number of accepted applications remained stable. See \cite{sunde14b}.}

Many powerful market actors, who still favour a traditional mode of exploitation, have seized the opportunity to revive the idea of large-scale exploitation.\footnote{See, e.g., \cite{alexandersen14}.} This, they argue, is preferable also from an environmental point of view. It might be more damaging to the affected area, the argument goes, but at the same time, a few large-scale projects mean that many other areas can be left undisturbed with no loss of total energy output. This argument has proven influential in many quarters, particularly among state agencies, such as the NVE and the Norwegian Environmental Agency.\footnote{See \cite{nilsen11}.} It has also been claimed that this perspective is backed up by research done on environmental effects of small-scale and large-scale projects.\footnote{See generally \cite{bakken12,bakken14}.} 

The argument used to back up this conclusion is that small-scale plants indirectly affect a greater total area of land, per energy unit produced.\footcite[96-99]{bakken14} This is no doubt true, since small-scale development is a decentralised approach to hydropower. In particular, this form of  development requires plants at many different sites to match the energy produced by a single larger plant. But is the accumulative effect on the environment of small-scale development on many sites more damaging than the effect of a much more invasive project on a single site?

This seems to depend on one's starting point when it comes to qualitatively assessing the negative impacts arising from small-scale development compared to large-scale projects. The research so far has provided little or no information or discussion to shed light on this question. In particular, the parameters used to compare small-scale and large-scale developments are mostly defined quantitatively, in terms of generic buffer zones that do not take into account differences in the severity of different kinds of environmental intrusions.\footcite[95]{bakken14}

The only buffer zone that is not defined in this way is the {\it scenic} buffer, the area from which some installation can be seen. Here the model takes into account that a large installation should be assessed using a larger buffer zone than a small one, since the former is visible over a greater area. But even for this parameter, no distinction is made based on the actual visual impression; a large dam that dries up a river and makes it possible to regulate the water level in a lake by several meters counts the same as a small cabin with a generator inside, as long as both can be seen.\footnote{See \cite[95]{bakken14}.} For the other parameters, the data analysis is even more dubious, since the buffers are set uniformly according to general rules of thumb.\footnote{See \cite[95]{bakken14}.} For instance, a conflict with a threatened species is assumed to arise whenever a technical installation occurs within a certain distance from its natural habitat.\footnote{See \cite[95]{bakken14}.} Importantly, nothing is said about the severity of conflict, and no distinction is made between a minor installation and a massive disturbance.

Despite the shortcomings of the research presently available, the observation that more land is affected by small-scale hydropower, per produced unit of electricity, has struck a cord with administrative decision-makers. In particular, the idea that large-scale development is better for the environment is fast gaining ground in Norway, representing a complete reversal compared to the political narrative that has dominated for the last 15-20 years.

In his New Year's speech 01 January 2001, the Prime Minister went as far as to declare that the age of large-scale development was over.\footnote{See, e.g., \cite[34]{haltbrekken12}.} The same phrase was then repeated in the policy platforms of two successive national governments, in 2005 and 2009 respectively.\footnote{See the ``Soria Moria'' declaration from 2005, p 57, and ``Soria Moria II'', from 2009, p 52 (available at \url{www.regjeringen.no}).} But as administrative practices and case law on hydropower shows, the end of large-scale exploitation has proved impossible to implement. Despite being official policy at the highest level of government for almost 15 years, large-scale development interests continue to dominate the hydropower sector.\footnote{I believe the material presented in this thesis warrants making this claim. Moreover, it is underscored by the two recent Supreme Court decisions in \cite{jorpeland11} and \cite{otra13}.} Interestingly, the leading national politicians are now changing their position as well.\footnote{See \cite{liemin14} (reporting on recent public statements made by the Minister in support of large-scale development).} This seems to suggest that politicians have yielded to pressure exerted by expert planners and commercial interests in this matter.

%Despite this, the legal position of owners and local communities has been weakened in recent years, to make room for large-scale development interests. The best example of this is the case of {\it Otra}, concerning a large-scale development project in the southern part of Norway.

%The developer of this project was granted permission to expropriate riparian rights, resulting in a legal conflict that went to the Supreme Court twice.\footnote{See \cite{otra10,otra13}.} The waterfall owners argued that they should be compensated for the loss of a small-scale development potential, but the developer disagreed. First, the owners were successful in their main claim, but the Supreme Court sent the case back to the appraisal court of appeal on a technicality.\footnote{See \cite{otra10}.} Then, the second time the case came before this court, it was decided that compensation for the lost small-scale potential should not be awarded. The reason given was that a small-scale scheme could not expect to obtain a development license, since large-scale schemes were preferred by the licensing authorities.\footnote{See \cite{otra13}. I note that the court's reasoning on this point is at odds with what is known as the ``no-scheme'' principle in the UK, see e.g., \cite{lawcom01}. This principle states, roughly, that compensation following expropriation should not reflect changes in value that are due to the expropriation scheme. A no-scheme principle is typically observed in Norway as well, but it is a general feature of Norwegian expropriation law that this principle is applied rather narrowly in case the expropriation scheme follows from public planning (in which case applying a no-scheme rule tends to result in higher compensation). See, e.g., \cite{stordrange07}.} The consequence was that the waterfalls were compensated based on the traditional method, resulting in a fraction of the compensation originally awarded. Only a few years earlier, leading hydropower lawyers had predicted that this method was a thing of the past.\footnote{See \cite{larsen12}.} After the decision in {\it Otra II}, there is reason to believe that the future will hold the opposite scenario in store; compensation for small-scale potentials will be a thing of the past, at least in all cases when expropriation takes place to benefit large-scale development. 

%Indeed, one of the first clear renunciations of the small-scale narrative came during the opening of the {\it Otra} plant, in 2014. The Minister himself presided over the festivities, and used the occasion to explicitly reject the previous political line, by publicly declaring that he was in favour of more large-scale development.\footnote{See \url{http://www.tu.no/kraft/2014/10/28/energiministeren-etterlyser-mer-regulerbar-vannkraft}.}

This political shift is likely to result in a further weakening of property rights and the rights of local communities. For example, it provides indirect political legitimacy to the NVE, who now pursue an explicit policy of prioritising applications for large-scale projects when these come into conflict with small-scale schemes in the same rivers.\footnote{See letter from the NVE of 21 March 2012 regarding new routines for the assessment of hydropower applications.} In effect, the NVE will refuse to consider applications from owners as long as there are applications pending that might result in the expropriation of their property.
%\footnote{The NVE have tended to apply such a priority rule for many years, but the legality of such an approach has been somewhat unclear (it depends on whether or not giving priority to takers is necessary to facilitate public planning, see \cite[21]{wra00}). However, after the decision in \cite{jorpeland11}, where a waterfall taking was unsuccessfully challenged on procedural grounds, the NVE seems less hesitant in general to adopt a ``strict'' line when dealing with recalcitrant owners.}

All in all, it seems that small-scale hydropower is currently loosing both commercial force and political credibility as a sustainable alternative for development. The underlying causes of this deserve more attention than I can devote to them in this thesis. It would be particularly interesting to conduct a further examination into the effects of lobbying and the relationship between commercial interests and bureaucratic elites.

In addition, I would like to emphasise a different aspect, namely how many of the recent misfortunes for the small-scale sector seem to underscore the importance of adopting a broader, non-commercial, perspective on privately held rights to waterfalls. It seems that the small-scale sector needs to be challenged with its failure to comprehensively address social and environmental concerns. Moreover, it seems plausible to hypothesise that part of the reason why the small-scale industry has been so easily undermined has to do with the fact that the industry itself has failed to broadly mobilise property owners and local communities in decision-making processes.

In fact, the small-scale industry has on occasion actively sought to undermine property rights, possibly in an effort to mimic the successes of their large-scale competitors. The industry has argued, in particular, that expropriation should be made more easily available as a tool for small-scale developers and owners who wish to take property from reluctant neighbours.\footnote{See \cite{brekken07,brekken08}. The articles are written by a leading Norwegian energy lawyer, apparently in his capacity as legal representative of ``Småkraftforeningen'', an interest organisation for small-scale hydropower (the articles are published in the newsletter of this organisation).} The argument rests on a peculiar form of anti-discrimination reasoning; as long as large-scale developers are allowed to take property by force, small-scale developers should be allowed to do the same. In a world where takings are endemic, this might make some sense. However, it is hardly an attitude that helps the small-scale industry preserve its image as the more sustainable hydropower option.

%At the same time, one should not disregard the fact that many power companies appae who lobby may feel threatened by the surge of interest in small-scale hydropower. As I will discuss in the next chapter, recent case law on expropriation shows how the largest actors are working very hard to regain control over the market. Recently, they have been successful in court, and this too might be a reason behind the shift in political climate. Nevertheless, it strikes me as appropriate to direct a critical eye towards the small-scale industry itself. It seems, in particular, that the objective of profit-maximising has taken center stage to an extent that might harm the sector. %According to Norhordlandsmodellen, the environmentalists were natural partners of the small-scale hydropower movement. But now, they are fast becoming its sworn enemies.\footnote{See \cite{haltebrekken12}.} I think this aspect, in particular, is dangerous to the future of the industry, as it previously benefited greatly precisely because of its image as a more environmentally friendly alternative.

These critical remarks should not detract from the fact that the growth in small-scale hydropower has led to dramatically increased benefit sharing with many local owners of rivers and waterfalls. However, it is important that this effect is not looked at in isolation, as a matter completely separate from the broader societal consequences of new commercial practices. If one fails in this regard, the pernicious image of owners as socially passive ``profit-maximisers'' gains a firmer hold both on the political and legal narrative. The negative consequences of this for property as an institution appears to be apparent already in Norway, as I will discuss in the next chapter when I consider recent case law on expropriation for hydropower development.

The call for a broader understanding of the role of small-scale hydropower and owner-led development echoes the theoretical discussion presented in Chapter \ref{chap:1}. There I argued that an entitlements-based perspective on property rights fails to do justice to the issues that arise in the context of economic development. In relation to hydropower development, this insight is strongly implicit in {\it Nordhordlandsmodellen}. However, in the current debating climate in Norway, it seems to be at risk of disappearing from view.

To counter this, I believe the social function view of property must be developed further, so that concrete policy recommendations can be formulated on its basis. The aim, I believe, should be to arrive at frameworks for participatory decision-making regarding hydropower that allows local owners and communities to contribute constructively when society desires commercial development based on  their water rights.

I return to this issue in Chapter \ref{chap:6}, where I argue that the Norwegian institution of land consolidation can be used to achieve this. First, I will zoom in on the issue of expropriation, where the mechanisms identified in this section often lead to concrete legal disputes. This will bring into focus important issues surrounding the status of economic development takings under Norwegian law.

\section{Conclusion}\label{sec:conc3}

In this Chapter, I introduced my case study and provided background information that places it in a broader context with respect to Norwegian law. I presented the legal and regulatory framework surrounding hydropower development, while also tracing its history back to pre-industrial times. I noted that local rights to hydropower has a long tradition in Norway. However, I also observed that after the advent of the industrial age, and particularly following the Second World War, the state took the view that hydropower was a public good that should be exploited for industrial development in the public interest.

The tension that followed now permeates the law on hydropower, particularly following the liberalising reform of the early 1990s. This reform reorganised hydropower development as a commercial pursuit. At the same time, local owners were empowered by the reform, as they were now able to engage in commercial hydropower development themselves. This was made possible by the fact that a market for electricity was set up, founded on the idea that all actors should have access to the electricity grid on non-discriminatory terms. 

I discussed the resulting system in some depth, addressing also the question of whether or not the market functions as intended. I noted that the energy reform led to increased concentration of power in the electricity sector, where commercial companies partly owned by the state now wield more power than before. This, I argued, threatens to undermine the intentions behind the reform. I also looked at the extent to which the regulatory framework is able to accommodate new actors and true competition on non-discriminatory terms. I focused particularly on the status of locally led projects as well as the companies that specialise in cooperating with owners. I also discussed controversies that have resulted, particularly relating to the perceived discrimination of smaller actors on the market.

Then I went on to present a prototype for the model by which the smaller actors now tend to organise themselves. I observed that they too appear to have adopted a strongly commercial outlook on the meaning of local hydropower development. I discussed how this departs from earlier ideas, which were based on seeing local development as an expression of local democracy and local management of resources. This earlier vision actively sought to ensure sustainability and incorporate other water interests in the decision-making, a perspective that now seems to be largely missing.

I concluded by arguing that this might be a contributing reason why small-scale development is now falling out of favour. Today, critical voices claim that large-scale development is better, not only because it is more commercially optimal, but also because it is more environmentally friendly. Moreover, issues relating to ownership, control, benefit sharing and local participation, appear only at the fringes, both of the current debate and the current regulatory framework.

This state of affairs, I think, foreshadows many of the issues that will be brought into focus in the next chapter. There, I will look specifically at expropriation of waterfalls, by tracking the position of owners under the current regime. I will argue that the law as it stands is based on a perspective that blocks out both the significant commercial interests of the taker, as well as the significant social functions and obligations of the original owners. The issue of expropriation, in particular, will invariably raise questions that seem difficult to address without adopting a broader view, which also takes into account the owners' communities and their role within it.
\chapter{Taking Waterfalls}\label{chap:4}

\section{Introduction}\label{sec:intro4}

In this chapter, I address expropriation of waterfalls in more depth, particularly the administrative practices that have evolved in relation to such expropriation. My main aim is to shed light on how these practices impact on the position of owners and local communities.

In Norway, the water authorities tend to consider expropriation of riparian rights as a natural component of hydropower development. In particular, a license to expropriate from local owners is typically considered a more or less automatic consequence of a development license. Moreover, as discussed in the previous chapter, the administrative licensing assessment tends to focus on the environmental consequences of development, not how interference in property affects owners and local communities.

As a result, a {\it presumption} has
developed, whereby the administrative decision-makers consider a license to undertake large-scale development as an indication that an expropriation order should also be granted.\footnote{The leader of hydropower licensing division of the NVE made an explicit statement to this effect in \cite{flatby08}.} Importantly, this presumption still remains in place, even though the regulatory and economic context of riparian expropriation has changed dramatically as a result of the liberalisation of the electricity sector in the early 1990s. 

In this chapter, I give a detailed presentation of the relevant statutory rules, the history of the law in this area, and current administrative practices. I then illustrate how the water authorities and the courts perceive and apply the law in this area, by carefully presenting the recent Supreme Court case of {\it Jørpeland}.\footnote{See \cite{jorpeland11}.}

The structure of the chapter is as follows. I begin in Section \ref{sec:explaw} by giving a brief overview of expropriation law generally, as well as special statutory rules relating to hydropower. In Section \ref{sec:twp}, I present the historical context to these rules. In short, I argue that because hydropower development was carried out by public utilities, expropriation for hydropower development enjoyed a high degree of political legitimacy. In addition, the lack of an open market meant that owners could not benefit commercially from developing hydropower themselves. Hence, their financial loss following expropriation was limited. In fact, expropriation (or voluntary sale) of riparian rights was usually the best an owner could hope for in terms of benefiting financially from hydropower.

%The story begins in the late 19th century, when the first statutory authorities for such expropriation began to emerge. Initially, these authorities were very narrow, however, and they did not cover expropriation of waterfalls, only additional land and rights that waterfall owners might need to develop their resource. 
%
%Later, in the early 20th century, the public sector began to expropriate waterfalls for hydropower, but this was only authorised on narrowly defined conditions, to enable the state to provide a new public service: electricity supply. Private companies could not expropriate waterfalls unless they were already majority owners in the local area, a situation that did not change until the passage of the \cite{wra00}. 
%
%Even before this, when hydropower development was still seen as a public service, it was sometimes met with resistance by local people and environmental groups, particularly as the state begun to pursue large-scale projects after World War Two. I discuss the case law that developed in this regard, particularly in relation to procedural rules. I conclude that the public-sector characteristics of hydropower development led the courts to defer very broadly to the discretion of the executive and the legislature, also in relation to the content and scope of provisions in administrative law. 

As discussed in the previous chapter, this changed following liberalisation of the electricity sector in the early 1990s. Ten years later, a new  expropriation authority was also introduced, in the \cite{wra00}. For the first time in Norwegian history, waterfalls and rivers could now be expropriated for purely commercial gain, also by private companies. In Section \ref{sec:twpp}, I place this change in the law in an historical context, before presenting the expropriation framework currently in place. I note that apart from the increased scope of expropriation, the framework developed prior to liberalisation remains largely in place.

%I note that this change in the law was not given much attention by the executive committee that prepared the act. It was described merely as a ``simplification'' of existing rules. Moreover, the legislature did not address the change at all when the Act passed through parliament. In fact, the wording of the \cite{wra00} does not explicitly make clear that private expropriation is now permitted. However, the Act empowers the executive to decide, using directives, what class of legal persons can be given a license to expropriate waterfalls. Such a directive has been issued, with little or no debate, granting the possibility to benefit from an expropriation for hydropower development to ``anyone''.
%I present the law relating to expropriation of waterfalls in some depth, before I go on to consider more concretely how the procedure plays out in the context of for-profit takings. Here I anchor the presentation in the recent Supreme Court case of {\it Jørpeland}, where the issue of procedural legitimacy arose after a commercial company was granted permission to deprive local owners of waterfall rights that the owners wished to make use of in their own hydropower project.\footcite{jorpeland11}

In Section \ref{sec:jorpeland}, I use the case of {\it Jørpeland} to show how the expropriation framework currently in place leaves owners particularly marginalised. Moreover, I note that their standing is very weak under administrative law, as a result of how the expropriation issue is overshadowed by the licensing question.

I then argue that the Supreme Court adheres to a narrow perspective on the meaning of property protection, taking it to be an issue that begins and ends with the question of compensation. In my opinion, this fails to do justice to the most important issue that arises when waterfalls are taken for profit, namely the question of democratic legitimacy. 

I conclude the chapter by elaborating on this theme, connecting it also with the theoretical discussion in Part I of the thesis. This sets the stage for the final chapter, where I consider land consolidation as a democracy-enhancing alternative to expropriation in hydropower cases.

%As discussed in Chapter \ref{chap:4}, the hydropower sector has now become depoliticised, expert-dominated and market-oriented. In light of this, my overarching argument is that the continued annihilation of local property rights threatens to render the state's functions in this sector subservient to the interests of the most powerful market actors, not the interests of the Norwegian people.

\section{Norwegian Expropriation Law: A Brief Overview}\label{sec:explaw}

As mentioned in Chapter \ref{chap:2}, the right to property is entrenched in section 105 of the Norwegian Constitution. There it is made clear that when property is taken for public use, full compensation is to be paid to the owner. The public use requirement is understood very broadly. According to leading legal scholars in Norway, it places no practically significant limit on the state's authority to expropriate.\footnote{...}

However, it is a rule of unwritten constitutional law that administrative decisions which affect the rights of individuals can only be carried out when they are positively authorised by law.\footnote{See generally \cite{hogberg11}.} Moreover, the Constitution is not understood as providing an authority for the state to expropriate, it merely expresses the presupposition that expropriation is possible.\footnote{See, e.g., \cite[6]{fleischer86}.} Hence, when applying eminent domain, the government needs to justify this on the basis of specific authorising provisions. 

Historically, there was no general act relating to expropriation, and a range of different acts provided the necessary authority to expropriate for specific purposes such as roads, public buildings, and schools.\footnote{See \cite[11-12]{nut54}.} Today, many of these authorities have been collected, broadened, and included in the \cite{ea59}.\footnote{Act no 3 of 23 October 1959 Relating to Expropriation of Real Property.} Still, some specific authorities remain, such as section 16 of the \cite{wra17}, which authorises expropriation for watercourse regulation.

Following the introduction of the \cite{wra00}, the general authority used to expropriate waterfalls has been included in the general act on expropriation.\footcite[2 no 51]{ea59}. Here it is stated that expropriation may take place in order to facilitate ``hydropower production''. In addition, it is made clear that expropriation can only be authorised if the benefits undoubtedly outweigh the harms. 

This sets expropriation orders apart from the various hydropower licenses discussed in Sections \ref{sec:wra00}-\ref{sec:ea} of Chapter \ref{chap:3}. For an applicant to obtain development licenses, it is sufficient to show that the benefits outweigh the harms, it need not be ascertained that this is {\it undoubtedly} the case. However, the practical significance of this difference is limited. According to the Supreme Court, the additional requirement in expropriation cases means only that it should be clear that the benefit is greater, it does not imply that the benefit has to be qualitatively more significant than in the licensing cases.\footnote{See \cite{lovenskiold09}.}

The authorising authority is the King in Council. However, this authority can be delegated to ministries or other state bodies that the King in Council can instruct.\footnote{See \cite[5]{ea59}.} The compensation to the owner is determined following a judicial procedure administered by the so-called appraisement courts.\footnote{\cite[2]{ea59}.} This is the name given to the regular civil courts when they hear appraisement cases, observing the special procedure set out in the \cite{aa17}. The appraisal procedure emphasises the importance of factual assessment and lay discretion (the appraisal court typically sits with four lay judges).\footnote{See \cite[11-12]{aa17}.} In addition, there are special rules regarding costs, indicating that the expropriating party is usually required to pay for the procedure, include the owners' legal expenses.\footnote{See \cite[54]{aa17}.} In other regards, the appraisal procedure resembles a typical adversarial process before a civil court.\footnote{See generally \cite{dyrkolbotn15}.} 

The \cite{ea59} states that unless the Kind in Council decides otherwise, expropriation orders may only be granted to state or municipality bodies. This is formulated as a limiting principle, but in effect it serves as a general authorisation for the executive to decide, without parliamentary involvement, that a larger class of legal persons may be granted expropriation licenses. 

For many purposes, directives have been issued that extend the class of possible beneficiaries to any legal person, including companies operating for profit. In 2001, such a directive was issued for the authority to expropriate in favour of hydropower production.\footnote{See Directive no 391 of 06 April 2001.} 

In addition to providing a general authority for expropriation, the \cite{ea59} also contains several procedural rules. These are collected in Chapter 3 of the Act. Here the Act sets out minimal requirements for what an application for an expropriation license must include, stating that it should make clear who will be affected, how the property is to be used, and what the purpose of acquisition is.\footnote{See \cite[11]{ea59}.} In addition, the Act requires the applicant to specify exactly what property they require, and to include information about the type of property in question and the current use that is made of it.

The owners must be notified, and the starting point is that every owner should be given individual notice, although this obligation is loosened when it is ``unreasonable difficult'' to fulfil\footnote{See \cite[12]{ea59}, para 2.} In such cases, it is sufficient that the documents of the case are made available at a suitable place in the local area. In addition, a public announcement must then  be made in the official notification publication of the government, as well as in two widely read local newspapers.\footnote{See \cite[12]{ea59}.}

The licensing authority is required to ensure that the facts of the case are clarified to the ``greatest extent possible''.\footnote{The Norwegian expression is ``best råd er'', which literally means ``best possible way''. See \cite[12]{ea59}, para 2.} This formulation seems very strict, but is also highly non-specific. In practice, the level of scrutiny given to the expropriation question under Norwegian law varies greatly depending on sector-specific administrative practices.  

Established practice from several fields, including the hydropower sector, suggests that when expropriation takes place to implement a public plan or a licensed development, little attention is devoted to expropriation as a special issue.\footnote{In relation to zoning plans, this has been made clear in a series of Supreme Court decisions, see \cite{namsos98,bo99}. In relation to hydropower, see Section \ref{sec:jorpeland}.}

A decision to grant an expropriation license must be justified, and the parties should be informed of the reasons for the decision.\footnote{See \cite[12]{ea59}, para 3.} This expropriation-specific rule is largely superfluous, however, as the obligation to give reasons would in most cases follow independently from general administrative law, c.f., Section \ref{sec:paa67}.

The applicant must cover costs incurred by owners in relation to a pending application for expropriation.\footnote{See \cite[15]{ea59}.} The exact formulation is that the applicant is obliged to cover the costs that ``the rules in this chapter carry with them''. That is, the applicant is obliged to cover the costs that are related to the owners' rights pursuant to Chapter 3 of the \cite{ea59}. In practice, an owner will be denied costs if the competent authority takes the view that they are unreasonable or disproportionate to their interests in the case.\footnote{If the case progresses to an appraisement dispute, the competent authority to decide on costs is the appraisement court. Otherwise, the decision is left with the executive. See \cite[15]{ea59}.}

%Particularly problematic are cases for which there is no clear division between those aspects of the case that relate to expropriation and those that relate to other licenses or land use planning more generally. This is the situation, for instance, in relation to hydropower development. In such cases, it is unusual for local owners to get any significant coverage of costs relating to the application processing. Legal expenses, for instance, are rarely covered unless they are incurred in relation to a subsequent appraisement dispute. This can be a problem for owners that wish to resist expropriation. Obviously, it is crucial for them to voice convincing objections already at the application processing stage.

In addition to the procedural rules in the \cite{ea59}, many rules of administrative law apply in expropriation cases. In the next section, I give a brief overview of administrative law in Norway, including the most relevant rules of the \cite{paa67}.

\subsection{The Public Administration Act}\label{sec:paa67}

Starting in the late 19th century, the importance of public administration gradually increased in Norway.\footnote{See \cite[8-12]{nut58}.} This development gained momentum after the Second World War, when administrative bodies also came to be placed more directly under centralised political control. At the same time, the traditional administrative ideal based on strict adherence to the letter of the law was replaced by a form of management that actively sought to pursue political goals.\footnote{See generally \cite{gronlie00}.} The ambit of administrative decision-making power widened significantly. Many new administrative bodies were set up, while many of those already established were empowered greatly as new statutory rules were introduced that specified the competence of administrative bodies in broader and broader strokes.

As administrative bodies became increasingly powerful, concerns arose regarding the relative lack of procedural safeguards to protect the individuals affected by administrative decisions. This concern was also fuelled by the fact that as the importance of state regulation increased, so did the power of the administrative branch to make decisions that would directly affect the rights and obligations of specific individuals.\footnote{See \cite[12-16]{nut58}.}

In response to this, minimum standards of administrative due process were encoded in the \cite{paa67}.\footnote{Act no 86 of 10 February 1967 Relating to Procedure in Cases Concerning the Public Administration.} This Act sets out the fundamental procedural principles that government bodies must follow when preparing to make administrative decisions. Some rules apply to any such decision, but a particularly important class of rules apply specifically to so-called {\it individual decisions}, namely those that affect the rights and responsibilities of one or more specific legal persons.\footcite[2]{paa67} Clearly, owners of property targeted by an expropriation application fall into this category, so the \cite{paa67} applies in expropriation cases.

Many of the rules in the \cite{paa67} mirror those of the \cite{ea59}. However, the \cite{paa67} tends to include broader and more detailed formulations. For instance, the duty to give advance notice is accompanied by more information about what kind of information such a notice must contain.\footnote{See \cite[16]{paa67}. Just like the \cite{ea59}, the rule in the \cite{paa67} makes clear that individual notices are not required if the parties are difficult to reach.} In particular, it is said that ``the advance notification shall explain the nature of the case, and otherwise contain such information as is considered necessary to enable the party to protect their interests in a proper manner''.\footcite[16]{paa67} 

Hence, it is not enough simply to inform the party that a case is under way, the Act also stipulates that the notice has to meet a minimum standard of quality. In relation to expropriation of waterfalls this becomes potentially significant, especially in light of the practice I discussed in Chapter \ref{chap:3}, whereby applicants send out these notices themselves. One may ask, in particular, what owners are supposed to think when they receive a letter from a commercial company stating that unless a friendly settlement can be reached, their waterfalls and rivers will be expropriated.\footnote{Such a formulation is typical, used for instance by the expropriating party in \cite{sauda09}. In general, according to my own experience, a generic letter is sent by the developer to those private individuals who may be affected, with no individuation based on their interests in the case (e.g., based on whether they stand to loose a small-scale hydropower potential or  are affected in some (minor) ways by building works). Clearly, this approach can discourage riparian owners from engaging in the administrative process in a manner commensurate with the fact that they own the natural resource in question.}

The duty to assess cases also follows from the \cite{paa67}, mirroring the rules of the \cite{ea59}. The formulation is similarly imprecise, as it is declared that cases must be ``clarified as thoroughly as possible'' before a decision is made.\footcite[17]{paa67} Importantly, the \cite{paa67} includes specific rules that oblige the authorities to inform parties about information they retrieve during their assessment of the case, and to actively solicit further comments from the parties.\footnote{See paras 2 and 3 of \cite[17]{paa67}.}

The duty to justify and give reasons for administrative decision is also expressed in the \cite{paa67}. The duty applies to most individual decision, with some narrowly defined exceptions concerning cases when no party can be assumed to be dissatisfied, or when giving grounds would involve disclosing privileged information.\footnote{See \cite[24]{paa67}. Moreover, the King is authorised to limit the duty to give grounds when ``special circumstances so require''.} As to the content of the reasons given, the authorities should mention the relevant rules authorising the decision, outline the factual assessment, and describe the main considerations that have been decisive for the use of discretionary power.\footcite[25]{paa67} In case law, the duty to give reasons has some practical significance, since the Supreme Court has declared that insufficient reasons can be taken as an indication that the decision itself suffers from a shortcoming.\footnote{See \cite{isene81,hauge00}.} In hydropower cases, however, the duty to give reasons is understood to pertain to the licensing question as a whole, so that the authorities are not obligated to give individuated reasons to riparian owners, pertaining specifically to the expropriation question.\footnote{See \cite{sauda09,jorpeland11} (discussed in more depth in Section \ref{sec:jorpeland}).}

Sometimes, the parties to an administrative decision are ill-equipped to look after their interests, even if the safeguards mentioned above are respected. This situation often occurs in hydropower cases, as riparian owners often lack the technical, commercial, and legal knowledge necessary to understand the value of their property and their own legal position as owners. The \cite{paa67} establishes a general duty to provide guidance, to ensure that the parties are able to look after their interests in the ``best possible way''.\footcite[11]{paa67} However, it is explicitly stated that the level of guidance must be adapted to the circumstances and the capacity that the government agency has for offering such assistance. At the same time, it is made clear that the decision-making agency must assess, on their own motion, the parties' need for guidance.

To summarise, both the law of expropriation and general administrative law impose a range of procedural rules that ordinarily apply to expropriation cases. In principle, these apply also when rivers and waterfalls are expropriated. In practice, however, they are completely overshadowed by the special rules that regulate the licensing procedure in such cases. I return to this issue in more depth in Section \ref{sec:jorpeland}. First, I elaborate on statutory rules that specifically target expropriation for hydropower, within the context of the relevant licensing procedures.

\section{Taking Waterfalls by Obtaining a Regulation License}\label{sec:special}

As I mentioned in Chapter \ref{chap:3}, Section \ref{sec:wra17}, the \cite{wra17} establishes an automatic right to expropriate rights needed to undertake a watercourse regulation. This is not understood to include a right to expropriate rivers and waterfalls needed for the hydropower development. However, it includes a right to transfer water away from a river for development somewhere else. 

This is of course a {\it de facto} license to expropriate riparian rights, since the water as such is taken by the expropriating party. Moreover, it has always been treated as expropriation of riparian rights in relation to the compensation issue.\footnote{See \cite{jorpeland11}.} Formally, however, the interference is not considered a riparian expropriation, but rather seen as an expropriation of a right to deprive rivers of water, a sort of easement whereby the developer acquires the right to interfere with the rights of riparian owners in source rivers.

In theory, the rules in the \cite{ea59} and the \cite{paa67} still apply in such cases. Indeed, the rules in the \cite{paa67} express general principles of administrative law, pertaining to all kinds of individual decisions, including both expropriation and  licensing decisions. The \cite{ea59}, for its part, explicitly states that it applies to property interferences authorised under the \cite{wra17}.\footnote{See \cite[30]{ea59}.} However, it is also stated that the rules in the \cite{ea59} only apply in so far as they are ``suitable'' and do not ``contradict'' sector-specific rules.\footcite[30]{ea59} This points to the potential caveat that while a range of procedural rules apply in theory, they may be ignored in practice, in so far as they are deemed ``unsuitable'' by some competent state body.

This is practically significant in hydropower cases. In particular, the established practice among the water authorities is to regard the procedural rules in the \cite{wra17} as exhaustive.\footnote{This was made clear through the case of \cite{jorpeland11}, where this practice also got a stamp of approval from the Supreme Court.} In addition, the material assessment requirement in the \cite{ea59} is not considered to have any independent significance alongside the assessment criterion in the \cite{wra17}.\footnote{Again, see \cite{jorpeland11}.} This is so even though case law on the former assessment criterion emphasises the interests of affected property owners in a way that case law on the licensing issue does not.\footnote{In addition, the formulation in \cite[2]{ea59} contains the additional qualification that the benefit of interference must ``undoubtedly'' outweigh the harm. No corresponding formulation is included in the \cite[8]{wra17}. Instead, the formulation there is that a license should ``normally'' not be given, unless the benefits outweigh the harms.}

%However, there is no doubt that the rules of the \cite{paa67} apply to takings of water rights pursuant to the \cite{wra17}. Moreover, there is no doubt that when a separate expropriation license is sought for waterfalls, these rules, as well as the rules in \cite{ea59} both apply. In practice, they nevertheless play a minimal role when the water authorities assess cases, as the assessment is unified, and the focus remains on balancing environmental and energy interest.

As a consequence of how the law is understood on this point, it is very hard for owners to challenge the legality of a decision to allow expropriation of their riparian rights, especially when expropriation takes place pursuant to the \cite{wra17}.\footnote{It follows from the discussion in Chapter \ref{chap:4} that large-scale development projects almost always involve a license pursuant to the \cite{wra17} (or such that the rules from this Act, including s 16 on expropriation apply pursuant to the \cite{wra00}).} Moreover, even if section 16 of the \cite{wra17} does not apply, the water authorities tend to approach the affected owners in a similar way. In particular, the practices observed with regards to the issue of property interference is largely the same in all cases when the administrative branch classifies the license application as pertaining to a large-scale project.\footnote{See \cite{flatby08}.}

For such projects, the water authorities rely on a presumption that the conditions to permit expropriation are fulfilled whenever a development license may be granted.\footnote{See \cite{flatby08}.} Hence, in order to defend themselves, owners must proceed in a roundabout manner by addressing the licensing question as such. In practice, there is little or no room for arguing on the basis of rules that protect private property.    Moreover, in order to argue that the expropriation is unlawful on procedural grounds, the owners must effectively demonstrate that the water authorities dealt with the case in contravention of sector-specific rules and practices pertaining primarily to the licensing question. This is a daunting task, particularly in light of case law developed during the period of monopoly regulation. This body of case law suggests that the courts will largely defer to the administrative branch, even when it comes to interpreting the relevant procedural rules.\footnote{The deferential stance was expressed most clearly in the {\it Alta} case discussed in Section \ref{sec:twp} below.}

As a consequence, procedural objections pertaining to the administrative assessment of existing property interests are unlikely to be successful. I am not aware of any case where such an argument has succeeded. In Section \ref{sec:jorpeland}, I will further demonstrate the present situation by tracking in detail the extent to which the Supreme Court is prepared to tolerate procedural shortcomings pertaining to the expropriation issue in hydropower cases.%\footnote{The justification for this, as I will show, is that the {\it Alta} case still serves as the primary precedent when assessing procedural complaints in hydropower cases. See, e.g., \cite{sauda09,jorpeland11}. As I will show in Section \ref{sec:twp}, the {\it Alta} case had nothing to do with expropriation and property rights. Rather, it arose from environmental concerns (and concerns about property-less indigenous people). In particular, the procedural objections that were raised in {\it Alta} were specifically related to the assessment of environmental consequences, an assessment that is usually carried out quite thoroughly  in hydropower cases (even more than usual in the {\it Alta} case, which was already very controversial).} 

First, I give a chronological presentation of how the law on expropriation of waterfalls has developed as part of the legal framework for management of hydropower. I begin with the period prior to the reform implemented by the \cite{ea90}.

\section{Taking Waterfalls for Progress}\label{sec:twp}

Historically, Norwegian law did not contain a general authority for expropriation of riparian rights.\footnote{See \cite[29]{amundsen28}.} In the \cite{wra88}, a range of provisions authorised appropriation of water rights and land for specific purposes, but the criteria were narrow.\footnote{See \cite[69-85]{dahl88}. In addition, the purpose of expropriation was largely understood to be binding also on future use, so that the taker would not gain unrestricted control over the rights they acquired. Rather, they were obliged to use these rights to pursue the specific public purpose for which expropriation was authorised. See, e.g., \cite[133-140]{rygh12}.} Rivers and waterfalls as such could never be made subject to expropriation, and expropriation of other water rights could only be permitted in so far as the affected owners were not thereby deprived of any water power that they could reasonably make use of themselves.\footnote{See \cite[58,60]{dahl88}.}

Specifically, expropriation for hydropower development was not permitted, except to the benefit of riparian owners who needed to acquire surrounding land in order to exploit their existing water rights.\footnote{See the \cite[15-16]{wra88}. See also the commentary in \cite[60-65]{dahl88}.} At the same time, riparian owners could apply for licenses to engage in various industrial exploits, in some cases also when this would prove damaging to other landowners, for instance through deprivation of water or flooding.\footnote{See \cite[14]{wra88}. See also the commentary in \cite[54-60]{dahl88}.} These rules are similar to many of the rules found in contemporaneous mill acts from the US, c.f., the discussion in Chapter \ref{chap:2}, Section \ref{sec:hop}. As in the US, the kinds of takings in question here could be classified as economic development takings. However, the source of the economic development potential was never taken from the owners in these cases. Rather, the takings only targeted additional rights that were needed in order for the existing owners to realise the full potential of their own resources.

In fact, an important principle of expropriation law at this time was that no property could be taken if the taker's interest in that property was the same as that of its current owner.\footnote{See \cite[168-170]{dahl88}.} This applied regardless of whether or not the owners, subjectively speaking, were likely to pursue those interests optimally. Hence, expropriation of water power was ruled out already as a matter of principle. In particular, as the regulatory system of the day made private hydropower development possible, a private riparian owner was regarded as possessing a hydropower interest. As a result, such owners could not be deprived of their rights by a taker whose interest was also to undertake hydropower development.

Following industrial advances, the interest in hydropower exploded in the late 19th century.\footnote{See \cite[58-59]{falkanger87}. See also the discussion in Chapter \ref{chap:3} Section \ref{sec:x}.} As a result, the state increasingly came to see it as a political priority to sensibly regulate the hydropower sector. As discussed in Chapter \ref{chap:3}, the most important expressions of this came in the form of two new licensing acts, namely the \cite{wra17} (Section \ref{sec:wra17} and the \cite{ica17} (Section \ref{sec:ica17}).

%Recall that the \cite{ica17} set up a licensing framework that would make it less attractive for speculators to purchase waterfalls, while also ensuring that the waterfalls purchased came under state ownership after a concession period. The \cite{wra17}, on the other hand, established a novel principle, namely that the right to regulate the flow of water in a river system did not belong to the riparian owners, but the state. This did not imply any difference in the right to use rivers for hydropower generation. This right still belonged to the local landowners. Hence, to undertake hydropower development involving watercourse regulation, both the riparian rights of local landowners as well as the regulation right of the state would be required.

Following up on this, parliament soon passed legislation that authorised expropriation of riparian rights for the benefit of public bodies, also when the purpose was hydropower development.\footnote{Legislation that made it possible to expropriate waterfalls to the benefit of the municipalities was introduced in 1911, and a similar authority that authorised expropriation in favour of the state appeared in 1917, see \cite[29]{amundsen28}.} In 1940, these authorities were consolidated and integrated in the general water resources legislation, through the \cite{wra40}.\footnote{This act has since largely been replaced by the \cite{wra00}.} According to this act, the authority to expropriate waterfalls could be granted only to the state and the municipalities. Moreover, the municipalities could only expropriate waterfalls when the purpose was to provide electricity to the local district.\footnote{See the \cite[148]{wra40}. See also the commentary in \cite[201-210]{sorensen41}.}

Private parties could only expropriate in exceptional circumstances, when they already owned more than 50 \% of the riparian rights that sought to exploit.\footnote{See the \cite[55]{wra40}. See also the commentary in \cite[70-74]{sorensen41}. I remark that this was a novel rule in the 1940 Act, which contradicted earlier theories about the legitimacy of allowing expropriation for private benefit.} Moreover, whenever expropriation took place, it was felt that benefit sharing with local owners was required. Hence, special rules were introduced to ensure that takers would have to pay {\it more} than full compensation (typically a 25 \% premium, but in some cases the owner was also given a right to opt for compensation in the form of a proportion of the electricity output of the plant).\footnote{See \cite[70-91,184,210]{sorensen41}.}

As I showed in Chapter \ref{chap:4}, the electricity supply in Norway just after the passage of the \cite{wra40} was already well developed, with 80 \% of the population having access to electricity. Moreover, in the rural areas the supply often came from one among a vast number of small, local, power plants. In light of the progress already made and the highly decentralised structure of the hydroelectric sector at this time, one might have expected expropriation to remain a relatively rare occurrence.

However, the use of expropriation to facilitate hydropower development increased greatly after the war, as the state itself became engaged much more actively with hydropower development, also for commercially oriented industrial purposes.\footnote{See \cite[59-71]{thue96}. See also \cite{skjold06}.}

Hence, despite the spirit and wording of the \cite{wra40}, this was the time when expropriation of rivers and waterfalls became a measure to facilitate economic development. At first, this would still take place on non-commercial, politically governed, terms. But the increased prevalence of expropriation seen during this time had little to do with a pressing need to supply electricity to the people. Rather, it was a consequence of an increased political demand for industrial hydropower, combined with the fact that the hydropower sector was reorganised and brought under increasingly centralised political control.\footnote{See \cite[69-71]{thue96}.}

As I mentioned in the previous chapter, many local, privately owned, hydropower plants were shut down during this period, as a result of an explicit policy meant to create government monopolists.\footnote{See Chapter \ref{chap:3} Section \ref{sec:x}.} Moreover, as a result of centralisation, a growing share of the financial benefits from development would also accrue to urban areas, as local development companies were replaced by state companies and companies dominated by prosperous city municipalities.\footnote{In 2007, as the result of a gradual centralisation process, the 15 largest hydropower companies in Norway, which are largely controlled by the state and some city municipalities, owned roughly 80\% of Norwegian hydropower, measured in terms of annual output. In 2006, the public owners of hydropower in Norway benefited from receiving more than NOK 9 billion in dividends. See \cite[28]{otprp61}.} The interpretation of the supply requirement in the \cite{wra40} was also significantly relaxed, especially following the development of the national electricity grid. It was no longer obvious, from a technical point of view, when exactly a hydropower development could be said to qualify as making a contribution to the local electricity supply. The electricity was not necessarily used locally. Indirectly, however, one could still argue that the local supply situation would improve whenever more electricity was supplied to the national grid.

\noo{ While the public spiritedness of hydropower development was arguably reduced, the rule that private parties could not expropriate riparian rights was still respected. It remained in place until 2000, when the law was changed to make it possible for any legal person to expropriate Norwegian rivers and waterfalls in order to develop hydroopower.\footnote{This change in the law was effected by an executive directive, not an explicit act of the legislature, as discussed in Section \ref{sec:twpp}.}

In light of this, the vast majority of cases dealing with waterfall expropriation under Norwegian law can not be looked at as takings for profit, even though they increasingly became economic development takings. Certainly, the desire for economic development played a crucial part in motivating state and municipality development projects in post-war Norway. But their activities in this regard were not themselves commercial in nature. Rather, supplying electricity was regarded as a public service, one that would in turn stimulate commercial activity in other areas of the economy.\footnote{See generally \cite{thue06b,skjold06}.}
}

%In the following subsection, I present the case law that developed during the post-war period. In light of how the courts have chosen to approach recent controversies, this body of case law is still highly relevant, even though the context of interference has changed as a result of liberalisation.

\subsection{The Supreme Court on the Rule of Reversion}\label{sec:prelib}

The period before liberalisation was not free from conflict regarding the legitimacy of measures undertaken to facilitate hydropower. Already the first assertion of state control, embodied in the licensing acts of the early 20th century, resulted in significant controversy. At this time, there was a feeling of unease regarding the extent to which the state could regulate the hydropower sector without offending against the property clause in the Constitution.

This debate culminated in the conflict surrounding the rule of reversion that was introduced by the licensing acts passed between 1906 and 1917. As mentioned, the rule of reversion meant that in order to purchase riparian rights from private owners, the purchaser had to agree to a licensing condition stating that eventually, after at most 60 years, the state would acquire the waterfalls without paying compensation.

The question that arose was whether this should be regarded as a form of expropriation. If so, compensation would have to be paid pursuant to section 105 of the Constitution. This question resulted in fierce conflict, with some influential legal scholars attacking the rule as a ploy by the state to confiscate Norwegian rivers  without compensating owners.\footnote{See \cite{morgenstierne14}.} However, in a 4-3 decision, the Supreme Court held that section 105 did not apply, since reversion was a licensing condition, not an independent act of property deprivation. \footcite{johansen18} No owner was compelled to hand over their rights to the state.\footnote{See \cite[406]{johansen15}.} Moreover, no owner was compelled to sell their rights. Rather, their willingness to do so was a precondition for the rule of reversion to apply.

One of the judges summed up the majority reasoning by commenting that he would not regard it as expropriation if the state were to forbid sale of riparian rights to private parties altogether.\footnote{See \cite[407]{johansen18}.} Why then, he asked, should it be regarded as expropriation if such a sale was allowed to take place only on specific conditions? Against this, the minority argued that the licensing requirement as such was so severe that it had to be regarded as a {\it de facto} expropriation, a regulatory taking in US terminology.\footnote{See \cite[412-413]{johansen18}. For a brief discussion on regulatory takings, see Chapter \cite{chap:1} Section \ref{sec:x}.} In addition, the minority argued that  as the purpose of the reversion rule was to ensure that water rights were eventually brought under state ownership, this rule itself could not be understood merely as an act of regulation. According to the minority, the rule also invoked the power of eminent domain.\footnote{See \cite[415-416]{johansen18}.} By contrast, the majority chose to regard the eventual transfer to state ownership as a secondary purpose only, which could justifiably be pursued solely on the basis of the state's regulatory power.\footnote{See \cite[407]{johansen18}.}

After the Supreme Court upheld the rule of reversion, the legal foundation for tight state regulation of the hydropower sector solidified. 
As discussed in the previous chapter, the state pursued increasingly complex hydropower projects after the Second World War. At this time, it also became common to divert water over great distances, to collect water from several different rivers in a common reservoir for joint exploitation. Such projects became known as ``gutter'' projects. Since such projects were covered by the \cite{wra17}, the practical importance of the expropriation authority in section 16 of this act also increased dramatically.\footnote{See \cite[11]{insst59}. This was a proposition to parliament regarding an amendment of the \cite{wra17}. The amendment proposed to remove an earlier rule that applied only to diversion regulations, whereby a license to divert water from a river should {\it normally} only be granted when the riparian owners in the source river agreed to the measure. This rule made licenses harder to obtain in the diversion cases. However, following the department's recommendation, the rule was removed in 1959. The department argued that the rule had an ``unfortunate effect'' on the administrative procedure in large-scale diversion cases, noting also the vastly increasing complexity and scale of typical diversion regulations. The minority in the parliamentary committee recommended against the amendment, noting that it would ``greatly increase'' the authority to expropriate waterfalls, contrasting with the expropriation rules in the \cite{wra40}, see \cite[14]{innst59}. The majority countered this argument by maintaining that the regulatory power of the state would be used to prevent any abuse of power, and that the practical significance of the amendment would be limited to ensuring a ``more rational'' procedural approach to large-scale applications, see \cite[14]{innst59}.}

As discussed in Section \ref{sec:x} of the previous chapter, the opposition to hydropower grew proportionally to the scale and complexity of typical development projects. The critical focus was often on environmental effects, but the interests of local people also featured in these debates. 
Moreover, local interest were often aligned with the environmental interests. In a situation when local owners could not themselves benefit commercially  from hydropower, their response was often to oppose it. 

The controversies regarding large-scale development culminated in the case of {\it Alta}, still arguably the most important Supreme Court precedent in the area of hydropower law.

\subsection{The {\it Alta} Controversy}\label{sec:alta}

The {\it Alta} case went before the Supreme Court in 1982 after a long period of high-intensity conflict going back to the mid-seventies.\footnote{See \cite{alta82}. For commentaries, see \cite{eckhoff82,boe83,hagvar88}.} In {\it Alta}, the affected local population largely lacked formal title to the property they sought to defend. This was because the development in question would take place in the northernmost part of Norway, in the native land of the Sami people.\footnote{For Sami law generally, see \cite{skogvang02}.}

Norway has a history of discrimination against the Sami, and as their culture is largely nomadic, their land rights were never formalised in private law.\footnote{See \cite[149-156]{ravna12s}} As a result, land and natural resources in the county of Finnmark are largely owned by the state, at least in the sense of the state appearing as the nominal {\it in rem} owner.\footnote{In the past 30 years, partly as a response to the controversy of the {\it Alta} case, there has been a gradual change in attitude, whereby the rights of the Sami people receives greater legal recognition. In 2007, formal title to most of the land in the county of Finnark was transferred to a special state agency which is regulated by a special statute that obliges it to manage the land with due regard to customary and prescriptive rights of aboriginal groups and local people. See generally \cite{bull08}.}

Due to the sensitive context of interference, the {\it Alta} plans met with particularly strong criticism, both from environmental groups and groups fighting for aboriginal rights. A broad political movement was mobilised in opposition to the plans, eventually resulting in several serious cases of civil disobedience.\footnote{This included hunger strikes and attempts at sabotage, see \cite[80-83]{nilsen08}. For the Alta controversy generally, see \cite{altawiki;hjorthol06}.} The case also came before the courts, as the local population and environmental groups claimed, primarily on the basis of administrative law, that the development licenses that had been granted were invalid.\footnote{See \cite{eckhoff82}.}

The {\it Alta} case did not involve expropriation of the right to harness hydropower. Hence, one might think that the case has limited relevance to the issues addressed in this thesis. However, because of the priority given to the licensing procedure over specific expropriation procedures, the principles expressed in {\it Alta} also largely determine the legal position of waterfall owners whose rights to hydropower are expropriated.\footnote{See \cite{sauda09,jorpeland11}.}

{\it Alta} was admitted to the Supreme Court in plenum, directly on appeal from the district court.\footnote{This is a special arrangement available in cases that raise important questions of principle, cf., \cite[30-2]{cda05} and \cite[5]{ca15}.} The presiding judge commented that as far as he knew, it was the longest and most extensive civil case that the Court had ever heard.\footcite[254]{alta82} In an opinion totalling 138 pages, the Court considers a long range of objections against the development licenses, all of which are either rejected or held to provide insufficient reasons to declare the licenses invalid.

The opponents of the {\it Alta} development also argued on the basis of human rights and international law.\footnote{First, on the basis of articles 1 and 27 of the \cite{fnp}. Second, on the basis of \cite{ilo107} (later replaced by \cite{ilo169}). Third, on the basis of P1(1) of the \cite{echr}.} As noted by Eckhoff, these arguments raised subtle legal questions about how to apply the relevant principles of international law to a concrete dispute over hydropower development.\footnote{See \cite[351-352]{eckhoff82}. One of the most important international instruments, namely ILO Convention No 107, was not ratified by Norway at the time of {\it Alta} (Norway later ratified its replacement, ILO Convention No 169). However, it was argued that it had the status of customary international law. See generally \cite{eide80}.}
However, the Court refused to consider such  questions, finding that the negative effect of the hydroelectric plant was not so severe as to raise  human rights issues.\footnote{See \cite[299-300]{alta82}. See also \cite[351-352]{eckhoff82}.}

Instead, the Supreme Court approached the case on the basis of administrative law, focusing on the  procedural rules of the \cite{wra17}. In this regard, the opponents of the {\it Alta} development had pointed to a large number of purported shortcomings of the decision-making process. 

First, it had been argued that the original licensing application did not meet the requirements stipulated in section 5 of the \cite{wra17}. Essentially, the original application contained little more than technical details about the planned development, with hardly any identification or assessment of deleterious effects.\footnote{See \cite[264-265]{alta82}.} This shortcoming had been openly acknowledge by the water authorities themselves, who had nevertheless initiated a public hearing.\footnote{See \cite[265]{alta82}.}

The Supreme Court concluded that this was ``clearly unfortunate''.\footcite[265]{alta82} However, several reports and assessments had subsequently been provided by the water authorities, to fill the gaps left open by the initial application. For this reason, the Supreme Court held that the initial mistakes were irrelevant, since it was the licensing process as a whole that should be assessed.\footnote{See \cite[265-266]{alta82}.} Shortcomings at specific stages in the assessment would not be given weight unless they could be seen to imbue the process with a dubious character overall.\footcite[265]{alta82}

The Court then moved on to assess whether the process as a whole fulfilled procedural requirements, particularly those laid down in sections 5 and 6 of the \cite{wra17}. In addition, it had to be considered whether the assessment of the licensing criteria in section 8 of the \cite{wra17} had been sufficiently detailed.\footnote{Recall section 16 of the \cite{paa67}, requiring assessments to be as detailed as ``possible''.}

In this regard, those who objected to the licenses pointed to a range of negative effects that they believed had not been considered, or had not been considered in enough depth. In relation to nomadic reindeer interests, it was argued that the water authorities had failed to adequately consider the indirect negative consequences of hydroelectric development on reindeer farming the local area.\footnote{See \cite[176-179]{alta82}.} These effects were described as ``practically catastrophic'' by some expert witnesses.\footnote{See \cite[278]{alta82}.} By contrast, the water authorities had not devoted much attention to the possibility of indirect consequences, citing the difficulty (described as an ``impossibility'') of attempting to quantify such effects.\footnote{See \cite[277]{alta82}.}

After considering the reports and assessments in some depth, the Supreme Court did not find fault with the procedure in this regard. Importantly, the Court stressed that the water authorities had been well aware of the possibility of indirect negative consequences. The water authorities had simply chosen, as a matter of expert discretion, not to place much weight on such consequences.\footnote{See \cite[279]{alta82}.} This, according to the Supreme Court, could be regarded as an expression of disagreement with those claiming that the effects would be catastrophic.\footnote{See \cite[278]{alta82}.} As a result, the grounds for claiming procedural error disappeared, as the lack of attention directed at indirect consequences was held to reflect an (implicit) factual assessment to the effect that they where not particularly severe.

The structure of the argument used here is more interesting than the factual basis for claiming that the effects would be catastrophic. The argument structure, in particular, serves to recast a lack of assessment, a possible procedural error, as an exercise of factual discretion.

The {\it Alta} Court also made some apparent statements of principle in this regard. In particular, the Court held that since the licensing decision itself is discretionary, it is appropriate to grant the executive some margin of appreciation also with regards to the question of how to interpret vague requirements of administrative law.\footnote{See \cite[262-264]{alta82}.}

%By contrast, the appellants argued on the basis that the content and scope of procedural rules is a question for the judiciary. This line of argument was described by the {\it Alta} Court as ``very formal''.\footnote{See \cite[262]{alta82}.}

The Court made a second decision of principle when it supported the state's contention that the administrative licensing assessment did not have to be as thorough as that required in a subsequent appraisement dispute.\footnote{See \cite[279|330]{alta82}.} Hence, the meaning of the obligation to clarify cases to the best possible extent is put into perspective: assessments of deleterious effects may sometimes be omitted at the decision-maker's discretion, also in circumstances when such assessments will be needed later to clarify the owners' actual loss for the purpose of calculating compensation.

In relation to the negative effects on fishing, the Court conceded that the assessments could have been better, but went on to point out that the purpose of assessment was only to answer yes or no to development, not to give a detailed presentation of its effects.\footcite[330]{alta82} Crucially, the Court notes that if additional negative effects are uncovered after the licences have been granted, this can be addressed through compensation payments and future regulatory measures.\footcite[330]{alta82}

In effect, the risk of factual error is downplayed by making reference to the owners' compensation right and the regulatory power of the state. This echoes the dichotomy mentioned in Chapter \ref{chap:3}, whereby there is a tendency in Norwegian law to perceive the interests of local people as revolving around financial entitlements.

In {\it Alta}, the Court agreed that erroneous information had been provided in relation to some issues, particularly regarding alternative ways to meet the need for electricity in Finnmark.\footnote{See \cite[346-357]{alta82}.} However, the Supreme Court did not regard the factual errors in this regard as relevant to the licensing decision.\footnote{See \cite[346]{alta82}.}

Here a third clarification of principle took place. The Court held, in particular, that the duty to consider alternatives -- different ways in which the public purpose could be satisfied -- is very limited in hydropower cases.\footnote{See \cite[346]{alta82}.} This position of principle, in turn, was the key building block that the Court used to argue that errors and inadequacies in the information provided about alternatives were irrelevant.\footcite[346]{alta82} 

The Court's perspective in this regards appears to have been somewhat at odds with how the parliament approached the case. There was little doubt that the favourable political assessment of the {\it Alta} development depended heavily on the perceived electricity crisis in Finnmark and the supply situation in Norway generally, as well as the perceived inadequacies of alternative solutions.\footnote{See \cite[338-347]{alta82}.}

In relation to this question, the legal counsel acting for the state in {\it Alta} suggested explicitly that as these aspects came into focus only at the political stage of the decision-making, they were largely irrelevant.\footcite[341]{alta82} This line of argument is rather striking, since the decision to grant the license was in fact made by parliament, which had dealt with the case on three separate occasions.\footnote{See \cite[342]{alta82}.}

The Supreme Court did not address the state's arguments in this regard explicitly. However, it is worth noting how briefly the Court comments on the issue of factual errors concerning alternatives.\footnote{See also the surprise expressed in \cite[349-351]{eckhoff82}.} By contrast, the Court goes into painstaking detail regarding issues that seem to have been far less important to the political decision-makers.

%In relation to the duty to assess alternatives, the Court says nothing expect that the duty is very limited. For the details, which demonstrate factual inadequacies in the material given to the political decision-makers, the Court only refers briefly to the state's arguments. These arguments, based on the contention that the inadequacies were not significant, is accepted with no further discussion.\footcite[346]{alta82}

The dismissive attitude towards the duty to correctly assess alternatives is a controversial aspect of the {\it Alta}-decision.\footnote{See \cite[311]{falk}. For criticism of the Supreme Court on this point, see \cite[580-584]{backer86}.} More generally, the decision in {\it Alta} has met with criticism from commentators arguing that the decision shows the extent to which the courts tend to identify themselves with other organs of state.\footnote{See \cite[64]{graver88} (commenting also that ``government prestige'' was at stake).} Some argued that {\it Alta} would have a limited impact as a precedent.\footnote{See\cite[580-584]{backer86}.} 

However, this has not proved accurate. Indeed, {\it Alta} continues to receive regular citations by the Supreme Court. It stands as a crucial, yet controversial, precedent, not only in hydropower law, but also in administrative law more generally.\footnote{See \cite{ambassade09,jorpeland11}.}

However, it should be mentioned specifically that after the {\it Alta} decision, the legal position of the Sami people have greatly improved.\footnote{See generally \cite{gausaa07}. Gauslaa presents the emergence of {\it Sami law}, a collection of rules and principles serving to protect established land use patterns and the Sami way of life while also giving the Sami people a better opportunity to partake in decision-making processes that affect them as group.} Moreover, the controversy surrounding {\it Alta} has been regarded as a catalyst for change in this regard.\footnote{See \cite[156]{ravna12s}.} Hence, it is unlikely that the courts today would be as quick as the {\it Alta} court to dismiss arguments based on aboriginal rights.\footnote{See \cite[180]{gausaa07}.}

With regards to the position of local owners, on the other hand, the {\it Alta} decision is regarded to express key principles that still apply.\footnote{See \cite{jorpeland11}. See also \cite[312]{falk}.} That said, administrative practices have changed since {\it Alta}, arguably also in direct response to the criticism that was directed at this decision.\footnote{See \cite[122-123]{backer10}.} Today, the assessment of licensing applications typically involve a more extensive assessment of environmental effects and possible alternatives weighed against their effects in this regard.\footnote{For an overview of current procedures, see Chapter \ref{chap:3} Section \ref{sec:x}. See also \cite[625-659]{backer86}.}

However, as I will discuss in more depth below, the position of local owners during the assessment stages appears unaltered by the increased intensity of assessment regarding environmental effects. This demonstrates a failure to adapt administrative procedures to the new reality that has emerged following liberalisation of the electricity sector. Today, the taking of a waterfall affects the owner in a very different way than it did prior to liberalisation.

First, as discussed in the previous chapter, takings of waterfalls now have the effect of depriving original owners of a resource that they could potentially develop themselves. Second, as I discuss in the next section, takings of waterfalls have become pure takings for profit.

\section{Taking Waterfalls for Profit}\label{sec:twpp}
\noo{ 
As I mentioned in the previous section, private companies could not expropriate waterfalls in Norway prior to the passage of the \cite{wra00}. Moreover, the public purpose requirement was formulated quite strictly, particularly in cases when the development was undertaken by municipality companies. I also mentioned how the hydropower sector developed after the Second World War, from a sector dominated by private and municipality companies, to a sector dominated by the state. This development, in turn, was accompanied by increased conflicts and doubts regarding the legitimacy of the established licensing procedures, particularly the highly centralised nature of the decision-making process. 

Even so, the debate at this time was still very much anchored in a system that presupposed political management of the hydropower sector as a public service provider. Importantly, the conflicts rarely, if ever, involved significant commercial interests on the part of the local riparian owners. Many critics argued that the fiscal interest of the state could not be used to justify destruction of nature and local patterns of land use. But in financial terms, the value of what was destroyed was typically negligible compared to the value of the hydropower development.

As a result, controversies relating to the legitimacy of interference involved riparian rights only at their periphery. More focused conflicts involving such rights arose in relation to the question of compensation, but the issues typically discussed in this regard were also of relatively minor structural importance.

In Chapter \ref{chap:3}, I presented the reform of the energy sector of the early 1990s, after which hydropower development has been regarded as a commercial pursuit. Following the regulatory reform, a new general statute dealing with water resources was also proposed, eventually leading to the passage of the \cite{wra00}. This Act provided the first authority for the state to allow developers to take waterfalls compulsorily for profit. Moreover, it made possible the later executive directive by which waterfalls could be expropriated and handed over to {\it any} legal person, including private companies.
}

After the legal and regulatory reforms of the 1990s, takings of waterfalls for hydropower have become takings for profit. However, this change in the function of expropriation received little attention when these reforms were introduced. Moreover, when the \cite{wra00} was proposed, the new expropriation authority was not singled out for political consideration. In fact, the increased scope of expropriation was not mentioned at all when the Ministry presented their proposal to parliament. Rather, the new expropriation authority was described merely as a ``simplification'' of existing law.\footcite[223-225]{otprp39}

This was grossly inaccurate. For the first time in Norwegian history, private commercial interests would be able to expropriate waterfalls. The original proposal to this effect stemmed from the report handed to the Ministry by a commission appointed to prepare a new act relating to water resources. The commission report mentions briefly that the proposed expropriation authority would imply increased scope for expropriation. However, it does not discuss the desirability of this in any depth.\footcite[235-237]{nou94} 

The report totals almost 500 pages, but devotes only three of those pages to discussing the new expropriation authority. Here the committee notes that a range of different authorities for expropriation has long co-existed in the law, with many of them positing strict and concrete public interest requirements as a precondition for granting a license. This, the commission argues, is not a very ``pedagogical'' way of providing expropriation authorities.\footcite[235]{nou94} Moreover, the commission notes that it runs the risk of omitting important purposes for which expropriation should be possible. Hence, the commission proposes to replace all older authorities by a sweeping authority that makes expropriation possible for any project that involves ``measures in watercourses''.\footcite[235-236]{nou94}

The commission comments that their formulation might seem wide, but remark that this is not a problem since the executive can simply refuse to issue an expropriation order when they regard expropriation as undesirable.\footcite[235]{nou94} The commission does not reflect on the  constitutional consequences of such a perspective, neither in relation to property rights nor in relation to the balance of power between the legislature, the executive and the courts. Instead, the commission offers a brief presentation of the rationale behind dropping the local supply restriction for municipal expropriation. They comment that these rules complicate the law and might make desirable expropriations impossible.\footcite[235]{nou94} Nothing is said to clarify what kind of desirable expropriations the committee think might be left out. 

Importantly, the committee do not relate their proposals to the recent market-based reform of the energy sector. Hence, the obvious practical consequence of their proposal, namely that expropriation of waterfalls would be made available as a profit-making mechanism, is not discussed or critically assessed.

The issue of {\it who} should be permitted to benefit from an expropriation license is also dealt with very superficially. In this regard, the commission structure their presentation around the so-called {\it redemption} rule of the \cite{wra40}. As mentioned briefly in Section \ref{sec:twp}, this rule made it possible for the majority owners of a waterfall to compulsorily acquire minority rights, if this was necessary to facilitate hydropower development. Hence, it was a rule that provided only a limited opportunity for private takings, restricted to owners themselves or external developers that had been able to reach a deal with a locally based majority.

The main justification given by the commission for introducing a general private takings authority is that the special redemption rule had not been much used in practice.\footcite[236]{nou94} Why this is an argument in favour of opening up for private expropriation in general is not made clear. Indeed, it seems just as natural to regard it as an argument {\it against} doing so. Why extend the possibility for private expropriation if the demand for such expropriation has been limited? 

Presumably, the commission thought there would be a demand for such expropriation in the future, but this is not stated explicitly, nor is the appropriateness of it discussed. As to the requirement that private takers must already control a majority of the waterfall rights in the local area, the commission only remarks that it regards such a restriction as old-fashioned.\footcite[236]{nou94} No discussion is offered regarding the consequences for local owners.

Since the passage of the \cite{wra00}, it has become clear that the new authority for expropriation is a highly significant and controversial aspect of the act. Today, practically all cases of waterfall expropriation imply that local owners are deprived of a small-scale development potential in favour of a commercial company.

\subsection{\it Sauda}

In {\it Sauda}, a case before the court of appeal,  the riparian owners formally protested a license that granted a private company the right to expropriate their rivers and waterfalls.\footnote{See \cite{sauda09}.} The owners' principal argument was that the executive could not grant such a right to a private party, since this had not been sanctioned by parliament. 

This argument appeared weak, since the \cite{ea59} had been amended to ensure that the executive would be authorised to decide what legal persons could expropriate for hydropower purposes. However, the owners argued that the executive had not appropriately informed parliament that this would be the consequence of the amendment. In particular, the amendment itself had been passed as a mere formality following the adoption of the \cite{wra00}. 

The owners presented the written testimony of two members of the parliamentary committee that had prepared the Act. Neither of them could recollect that they had been aware that the Act would make private expropriation possible. This was not conveyed to them by the executive. Moreover, it was not explicitly stated anywhere in the Act itself. Rather, it followed implicitly from three different sections in two separate acts. In the entire collection of preparatory documents, the change was discussed only once, in the report from the committee to the Ministry. 

On this basis, the owners argued that the purported expropriation authority was not constitutionally valid, since parliament had not intended it. Unsurprisingly, this argument was rejected. According to the court, it had to be assumed that parliament understood the consequences of their own legislative actions. 

However, while the owners lost the validity dispute, the level of compensation they received was dramatically increased compared to earlier practice. Because of this, the development company appealed the decision to the Supreme Court, with the owners lodging a counter-appeal regarding the question of legitimacy. The Supreme Court decided not to hear the case. 

Indeed, it had recently addressed the compensation question in the case of {\it Uleberg}.\footnote{See \cite{uleberg08}.} Here a new principle of market-value compensation was introduced, for those cases when small-scale development by owners was deemed to have been ``foreseeable'' in the absence of expropriation.

If this requirement is met, the owners can now expect at least 10-20 times more in compensation than they would get under the traditional approach, which was based on a  theoretical estimation of the value of the riparian rights.\footnote{For a more in-depth presentation of compensation issues, I refer to \cite[71-76]{dyrkolbotn15}.}

In addition to raising the compensation issue and the issue of constitutional legitimacy, {\it Sauda} raised procedural questions. The owners argued that mistakes had been made and that the administrative expropriation decision was therefore invalid. The court did not agree. 

The procedural arguments made here foreshadow the later case of {\it Jørpeland}. Here the owners were initially successful, as the district court held that the decision to expropriation was invalid due to procedural mistakes. In particular, the district court held that the practices  developed during the monopoly era were no longer appropriate.\footnote{See the decision by Stavanger Tingrett in \cite{jorpeland09}.} 

This decision was overturned on appeal, a decision that was in turn upheld by the Supreme Court. Eventually, the case was decided by an application of the precedent set by {\it Alta}, demonstrating the continued importance of this case in the context of expropriation for commercial development.\footnote{See the decision by Gulating Lagmannsrett in \cite{jorpeland11a}. See also the Supreme Court decision in \cite{jorpeland11} ({\it Jørpeland}).} In the next section, I will consider the {\it Jørpeland} case in some depth, to bring out how administrative practices relating to expropriation for hydropower can work in practice.

\section{{\it Ola Måland v Jørpeland Kraft AS}}\label{sec:jorpeland}

The expropriating party was a public-private commercial partnership, Jørpeland Kraft AS. This limited liability company is jointly owned by Scana Steel Stavanger AS, with 1/3 of the shares, and Lyse Kraft AS, with the remaining shares. The former is a private steelworks company located in the small town of Jørpeland in Rogaland county, south-western Norway. Historically, this company was a major employer in Jørpeland, which is located by the sea, next to a mountainous area.

The main source of energy for the steel industry in Norway is hydropower and Scana Steel Stavanger AS was no exception. The company used energy harnessed from the rivers in the area, particularly the river which reaches the sea near Jørpeland. Moreover, the water from this river is supplemented by water from other rivers in the area that are diverted so that it can be exploited along with the water from the Jørpeland river.

Recently, Norwegian steel companies have become less profitable, due largely to increased foreign competition and a significant increase in costs of operation associated with this type of industry in Norway.\footnote{Salary costs, in particular, have become prohibitive. See, e.g., \emph{Information Booklet about Norwegian Trade and Industry}, published by the Ministry of Trade and Industry in 2005.} This has led to many such companies shifting their attention away from labour-intensive steel production, focusing instead on producing electricity, selling it directly on the national grid. Jørpeland Kraft AS was established as part of such a move to exploit the energy resources in Jørpeland.


The role played by the majority shareholder, Lyse Kraft AS, is important in this regard. Indeed, as I discussed in Chapter \ref{chap:3}, Norwegian law favours companies where the majority of the shares are held by the state or the municipalities. Lyse Kraft AS operates for profit, organised as a limited liability company, but it is publicly owned, with the city of Stavanger as the main shareholder. Hence, it is a very valuable partner to Scana Steel. In addition to being under public ownership, Lyse Kraft AS is responsible for the electricity grid in the region, so is well-positioned to access the electricity market.

Lyse Kraft AS was established as a merger between several former monopoly companies in the Stavanger region. These were all reorganised following liberalisation of the sector in the early 1990's. As discussed in Chapter \ref{chap:3}, old monopolists still enjoy considerable power and influence, particularly with regards to state bodies that are tasked with regulating the industry. This is another important reason why they can serve as valuable partners for private companies wishing to make a profit from Norwegian rivers and waterfalls.

As attention shifts from harnessing rivers for the purpose of industrial production to the purpose of producing electricity to sell on the national grid (and, increasingly, to export abroad), new variables determine the degree of profitability. On the cost side, what matters most is the one-time investment required to construct the hydropower plant.\noo{\footnote{For an overview of the considerations made when assessing the commercial value of hydropower, I point to \cite{jensen07}. In fact, due to the importance that small-scale hydropower has assumed in recent years, investigating models for investing in such projects has become an active field of research in Norway, see for instance \cite{investment}.}} Maintaining and operating a hydropower station tends to be comparatively inexpensive. On the income side, what matters is the price of energy on the electricity market, a market that is no longer anchored in local conditions of supply and demand.

Importantly, as long as energy production is the sole focus, the business no longer depends in any significant way on the local labour force. Hence, large-scale exploitation becomes much more profitable than the medium or small-scale power plants that would otherwise be suitable for local industrial exploits. Indeed, it was in keeping with a general trend in Norway when Jørpeland Kraft AS, following  their new commercial strategy, proposed to undertake measures to increase their energy output. This could be achieved relatively cheaply, by further constructions aimed at diverting water away from nearby rivers into dams that were already built to collect the water from the Jørpeland river.

\subsection{Facts}

One relatively small river from which Jørpeland Kraft AS suggested to extract water is owned by Ola Måland and five other local farmers. This waterfall is not located in Jørpeland kommune and it does not reach the sea at Jørpeland. Rather, it runs through the neighbouring municipality of Hjelmeland, on the other side of a mountain range, until it eventually reaches the sea at Tau, another neighbouring municipality. 

The plans to divert the water would deprive the original owners of water along some 15 km of riverbed, all the way from the mountains on the border between Hjelmeland and Jørpeland, to the sea at Tau. Not all the water would be removed, but the flow of water would be greatly reduced in the upper part of the river known as {\it Sagåna}, the rights to which is held jointly by Ola Måland and five other local farmers from Hjelmeland.

The water in question comes from a lake called \emph{Brokavatn}, located 646 meters above sea level, where altitude soon drops rapidly, making the river suitable for hydropower development. Plans were already in place for such a project, which would use the water from just below the altitude of Brokavatn, to the valley in which the original owners' farms are located, about 80 meters above sea level. 

A rough estimate of the potential of this project was made by the NVE itself, stating that the energy yield would be 7.49 GWh per annum.\footnote{See \cite{jorpeland09}.} This is about five times more energy than the water from Brokavatn would contribute to the project proposed by Jørpeland Kraft AS.\footnote{See \cite{jorpeland09}.}

Importantly, the estimate was not made in relation to the expropriation case, but as part of a national project to survey the remaining energy potential in Norwegian rivers.\footnote{The survey was carried out in 2004 and its results are summarised in \cite{jensen04}.} Ola Måland and the other owners of the river were not identified as significant stakeholders and were not notified of the assessment that had been made. Moreover, when Jørpeland Kraft AS submitted an application to the NVE for permission to divert the water, the owners were not notified by the NVE.\footnote{However, a generic orientation letter was apparently sent by Jørpeland Kraft AS, a letter that the owners themselves could not remember having received.}

Rather, the approach to the case was the traditional one, with an assessment directed at evaluating the environmental impact. Many interest groups were called on to comment on environmental consequences, and public debate arose with respect to the balancing of commercial interests and the desire to preserve wildlife and nature.

Despite not being asked to do so, one of the local owners, Arne Ritland, also commented on the proposed project. He did this in an informal letter sent directly to Scana Steel Stavanger AS. In this letter, he inquired for further information and protested the proposed diversion of water from Brokavatn. He also mentioned the possibility that an alternative hydropower project could be undertaken by original owners, but he did not go into any details, stating only that a locally owned hydropower plant had previously been in operation in the area. 

The plant he was referring to dates back to the time before there was a national grid. It ensured a local supply of electricity, but has since been shut down, in keeping with the general trend mentioned in Chapter \ref{chap:3}.\footnote{See \cite{jorpeland09}.}

Arne Ritland received a reply from Scana Steel Stavanger AS, which stated that more information on the project and its consequences would soon be provided. Ritland did not pursue the matter further at this time. Meanwhile, Scana Steel Stavanger AS submitted his letter to the NVE, who in turn presented it as a comment directed at the application. 

This prompted Jørpeland Kraft AS to undertake their own survey of alternative hydropower in Sagåna. The conclusions, but not the report itself, were sent to the water authorities. The owners were not informed that such an investigation was being conducted by the expropriating party, as a response to Ritland's letter.

Moreover, the water authorities did not take steps to investigate the commercial potential of local hydropower on their own accord. Instead, they referred to the conclusion presented by Jørpeland Kraft AS, stating that if the local owners decided to build two hydropower plants in Sagåna, then one of them, in the upper part of the river, would not be profitable, neither with nor without the contested water. The other project, in the lower part, could apparently still be carried out, even after the diversion. 

No mention was made of what the original owners stood to loose, nor was there any argument given as to why it made sense to build two separate small-scale power plants in Sagåna. Nevertheless, the NVE handed the expropriating party's findings over to the Ministry, without conducting their own assessment and without informing the original owners.

In addition to the report made by Jørpeland Kraft AS, Hjelmeland kommune, the local municipality government, also commented on the possibility of local hydropower. In their statement to the NVE, they directed attention to the data in the NVE's own national survey, which suggested that a single hydropower plant in Sagåna would be a highly beneficial undertaking. On this basis, they protested the diversion, arguing that original owners should be given the possibility of undertaking such a project. 

This statement was not communicated to the original owners, and in their final report the NVE dismissed it by stating that the most efficient use of the water would be to transfer it and harness it at Jørpeland.

In addition to the statement made by Ritland, one other property owner, Ola Måland, commented on the plans. He did so without having any knowledge of the commercial potential of the waterfall. Moreover, he had not been informed of the statement made by Hjelmeland Kommune. On this basis, he expressed his support for Jørpeland Kraft's plans, citing that the risk of flooding in Sagåna would be reduced. He also phrased his letter in such a way that it could be interpreted as a statement on behalf of the owners as a group. However, Måland was the only person who signed.

In the final report to the Ministry, the NVE refer to Måland's letter and state that the original owners are in favour of the plans. For this reason, the NVE argues, the opinion of Hjelmeland kommune should not be given any weight. The NVE neglects to mention that Arne Ritland's statement strongly opposed expropriation. Moreover, earlier in the report, where all incoming statements are reported, Ritland is referred to as a private individual, while Ola Måland is referred to as a property owner who speaks on behalf of the owners as a group.

The report made by the NVE was not communicated to the affected local owners at all, so the owners had no chance of correcting mistakes. However, the report was sent to many other stakeholders, including Hjelmeland kommune. In light of NVE's conclusions, the municipality changed their original position and informed the Ministry that they would not press for local hydropower, since this was not what the original owners wanted.

All of this happened without the owners' knowledge. However, while the case was being prepared by the water authorities, the original owners had begun to seriously consider the potential for hydropower on their own accord. In late 2006, Jørpeland Kraft's application reached the Ministry and a decision was imminent. At the same time, the owners were under the impression that they would receive further information before the case progressed to the assessment stage. 

As the owners had now come to realise the commercial value of the water from Brokavatn, they approached the NVE, inquiring about the status of the plans proposed by Jørpeland Kraft AS. They were subsequently informed that an opinion in support of the transfer had already been delivered to the Ministry. This communication took place in late November 2006, summarised in minutes from meetings between local owners, dated 21 and 29 November. On 15 December 2006, the King in Council granted a concession for Jørpeland Kraft AS to transfer the water from Brokavatn to Jørpeland.

At this point, it had become clear to the original owners that the water from Brokavatn would be crucial to the commercial potential of their own project. They also retrieved expert opinions that strongly indicated that the NVE was wrong when they concluded that diverting the water would be the most efficient use of the water. In light of this, the owners decided to question the legality of the licence (with the corresponding permission to expropriate). They argued, in particular, that the administrative decision to grant the license was invalid.

\noo{ According to the owners, the expropriation was materially unjustified. Moreover, they contended that the administrative process had not fulfilled procedural requirements. 

The district court, Stavanger tingrett, held in favour of the owners.\footnote{See \cite{jorpeland09}.} The court emphasised that the authorities were obliged to properly take into consideration the fact that the riparian rights in Sagåna could have been exploited commercially by the owners. Moreover, the court noted that in order to ensure a proper assessment, the authorities should have ensured that the owners were kept informed about the progress of the case, so that they would have an opportunity to comment in a timely fashion. The court also noted that the administrative decision was based on factual mistakes, regarding both the owners' opinion of the plans and the energy potential of Jørpeland Kraft's plans compared to an owner-led project in Sagåna. This, the court held, meant that the concession had to be declared invalid.

This view was rejected by the court of appeal, Gulating lagmannsrett, which held that sufficient steps had been taken to clarify the commercial interests of the owners.\footnote{See \cite{jorpeland11a}.} Moreover, the court held that established practices regarding the preparation and evaluation of such cases -- dating from a time when it was not feasible for original owners to undertake hydropower schemes -- still provided adequate protection. The owners appealed the decision to the Supreme Court, who decided to hear the case. However, the appeal was eventually rejected, and the Supreme Court went even further than the court of appeal when they declared that established administrative practices were beyond reproach.\footnote{See \cite{jorpeland11}.}
}

In the following section, I present the main legal arguments relied on by the parties, as well as a summary of how the three national courts judged the case.

\subsection{Legal Arguments}\label{view}

First, the owners argued that procedural mistakes had been made by the water authorities when preparing the case. This, in turn, had resulted in factual mistakes forming the basis of the decision to grant the development license. Since the outcome might have been different if these mistakes had not been made, the owners concluded that the development license could not be upheld.

Second, the owners argued that expropriation of their rights would result in the loss of an economic development potential. Moreover, they argued that the economic loss would be greater than the gain even for the public, since the owners were in a position to make more efficient use of the water rights in question. Therefore, allowing expropriation would only serve to benefit the commercial interests of Jørpeland Kraft AS, to the detriment of both local and public interests. On this basis, the owners contended that the decision to grant the licence was a manifestly ill-founded decision that could not be upheld.\footnote{...}

Third, the owners argued that the government had not fulfilled its duty to consider the case with due care. In particular, the assessment of local community interests and the interests of local owners had not been satisfactory. Particular attention was directed at the fact that local owners had not been informed about the progress of the case, and had not been told of assessments pertaining to their interests.

Fourth, the owners argued that irrespective of how the matter stood with respect to national law, the expropriation was unlawful because it would be in breach of the provisions in P1(1) of the ECHR regarding the protection of property.

Jørpeland Kraft AS protested, arguing that it was the responsibility of the owners to actively provide information about possible objections against the project and that the process had therefore been in accordance with the law. Unfortunate misunderstandings, if any, were due to the fact that the owners had neglected their responsibilities in this regard. Moreover, Jørpeland Kraft AS argued that it was not for the courts to subject the assessment of public and private interests to any further scrutiny, since this was a matter for the administrative branch.

%Indeed, according to Norwegian national law, it is traditionally held that unless the exercise of power it clearly unjustified, the courts do not have the authority to overturn decisions based on administrative discretion, unless procedural mistakes can be identified. While a judicially scrutinized standard of \emph{reasonableness} has been recognised, the admissibility of court interference in administrative discretionary decisions is still limited under Norwegian national law.\footnote{See \cite{eckhoff14}, in particular, Chapters 24 and 29.}

Finally, Jørpeland Kraft AS argued that there was no issue of human rights at stake in the case. They argued for this by pointing to the fact that the procedural rules had been followed and that the material decision was beyond reproach. Moreover, Jørpeland Kraft AS suggested that since the owners would be compensated financially by the courts for whatever loss they incurred, no human rights issues could possibly arise.

The matter went before the district court in the city of Stavanger, which decided in favour of the owners on 20 May 2009.\footnote{See \cite{jorpeland09}.} In the following, I offer a presentation of the reasons given by this court, leading to the conclusion that the expropriation was unlawful and that the diversion of Brokavatn could not be carried out.

\subsection{The District Court}

The district court of Stavanger agreed with the original owners that the decision to grant the license was based on an erroneous account of the relevant facts. Moreover, the court concluded that it was evident, from the assessment carried out by the NVE themselves (in their national survey of small-scale hydropower potential), that allowing the applicants to use the water from Brokavatn in their own hydroelectric scheme would be the most efficient way of harnessing the hydropower potential. This, the court noted, directly contradicted what the NVE had stated in their report.

In addition, the court noted that Hjelmeland kommune had in fact referred to the national survey in their objection against the application. Hence, the court found that the NVE were obliged to at least consider the assessment of hydropower potential contained therein.

The court substantiated their decision by giving several direct quotes from the report made by the NVE. For instance, from p 199, as quoted by the district court:
%\begin{quote}Hjelmeland kommune ser helst at kraftressursene i vassdraget blir utnyttet av lokale %grunneiere. 
%Dette står i kontrast til uttalelsen fra grunneierne selv som ønsker at overføring blir gjennomført, 
%slik at flom og erosjonsskader kan bli noe redusert. NVE mener at den beste utnyttelsen med tanke 
%på kraftproduksjon vil være å tillate overføringen da en slik løsning vil innebære at vannet utnittes i 
%størst fallhøyde. Når dette samtidig er grunneiernes eget ønske har vi ikke tillagt Hjelmeland 
%kommunes synspunkt på dette noen vekt
%\end{quote}
%Our own translation follows below: 
\begin{quote}
Hjelmeland kommune would like the hydroelectric potential in the waterfall to be exploited by 
local property owners. This contrasts with the statement given by the property owners 
themselves, who wish that the transfer of water takes place, so that damage due to flooding can be 
somewhat reduced. NVE thinks that the best use of the water with respect to hydroelectric 
production is to allow a transfer, since this means that the water can be exploited over the greatest
distance in elevation. When this is also the property owners' own wish, we will not attribute any 
weight to the views of Hjelmeland kommune.
\end{quote}

The district court concluded that as this was a factually erroneous account of the situation, the decision made to allow transferral of the water could not be upheld. The court offered the following overall assessment of the case:

\begin{quote}
It is the opinion of the court, having considered how the case was prepared by the authorities, that the factual basis for the decision made by the government suffers from several significant mistakes and is also incomplete.
\end{quote}

In light of this, Stavanger tingrett concluded that the decision to grant concession for diversion of water was invalid. Here the court relied on a well-established principle of administrative law: while the exercise of discretionary powers is usually not subject to review by court, a decision based on factual mistakes is nevertheless invalid if it can be shown that the mistakes in question were such that they could have affected the outcome.\footnote{This is not provided for explicitly in statue, but it is one of the core unwritten legal principles of administrative law. See \cite{eckhoff14}.}

Concerning the second requirement, that the factual mistakes could have affected the outcome, Stavanger tingrett found that it was clearly fulfilled in this case since the owner-led hydropower project was in fact a \emph{better} use of the resource, even with respect to public interests. In any event, the requirement with regards to factual and procedural mistakes is only that the mistakes \emph{could} have affected the outcome; in the presence of mistakes, the burden of proof is shifted onto the party seeking to defend the decision.

Since Stavanger tingrett held that the license to allow diversion was invalid because of factual mistakes, there was no need to consider claims regarding the legitimacy of the decision with respect to human rights law. Stavanger tingrett did conclude, however, by making a more overarching assessment of the case. Here the court states that the procedure followed in preparing the case did not have sufficient regard for owners. This, in particular, was the likely cause of the mistakes that had been made. The court also stated that the standard of assessment when considering owners' interests had to be interpreted more strictly now that hydropower development was an option available to original owners.

\noo{In this regard, t also seems that Stavanger Tingett found some additional support in its interpretation of Norwegian law that was based on human rights concerns, especially the fact that expropriation, in circumstances such as those of this case, appeared to be a major interference in the rights of owners, and that established practice developed under a different regulatory regime was therefore no longer able to provide adequate protection.}

Jøpeland Kraft AS appealed the decision and the case then went before the court of appeal, Gulating lagmannsrett, who found in favour of Jørpeland Kraft AS. 

\subsection{The Court of Appeal}

In their argument, the court of appeal do not rely on direct assessment of the report made by NVE, nor do they mention the expert statements retrieved by the opposing parties. Instead, the court of appeal base their decision on general considerations concerning the need for efficient procedures in hydropower cases. Such reasoning provides the apparent grounds for making the following crucial observation concerning the facts:

\begin{quote}... It was not a mistake to take Ola Måland's statement into consideration, as he was, and still is, a significant property owner. NVE's statement to the effect that granting the concession will facilitate a more effective use of the water seems appropriate, as it refers to a current hydroelectric plant that exploits a waterfall of 13.5 meters.
\end{quote}

The court of appeal do not mention the statement made by Hjelmeland kommune, nor do they comment on the fact that alternative hydropower, as suggested by the municipality, would amount to exploiting the water over a difference in altitude of some 550 meters. In fact, the hydroelectric plant that they do mention has nothing to do with Ola Måland and the other owners, but exploits the same water further downstream. The persons affected by this were no longer parties to the dispute (since their interests were obviously quite negligible).

This waterfall was only brought up in the testimony of a representative from the NVE. When pressed on the matter, this representative claimed that the reasonable way to interpret the assessment made by the NVE was to see it as an assessment concerning the existing hydropower plant further downstream. In light of the statement provided by Hjelmeland kommune, to which the report explicitly refers, this is an ungrounded interpretation. But the regional court adopted it, without further comment.

The court of appeal seems to have assumed, quite generally, that the traditional practices adopted by the water authorities in hydropower cases still provided adequate protection for original owners.

The court of appeal's decision was appealed by Ola Måland and the other owners. The Supreme Court decided to consider the juridical aspects of the case.\footnote{The appeal concerning the assessment of facts would not be considered. Hence, the Supreme Court would base themselves on the presentation of facts given by the court of appeal.}

\subsection{The Supreme Court}

The Supreme Court ruled in favour of Jørpeland Kraft AS.\footnote{See \cite{jorpeland11}.} The Court comments on the relevant facts on p 9 of the decision. There, they state that Jørpeland Kraft AS had considered the possibility that a hydroelectric scheme could be undertaken by local property owners. As I have already mentioned, a statement on this was provided to the NVE by Jørpeland Kraft AS themselves. This statement mentioned one possible project that was deemed not to be commercially viable. However, recall that in the same statement another project was also identified -- in the same river, using the same water -- that Jørpeland Kraft AS claimed was such a good project that it could be carried out even after the water from Brokavatn had been diverted.

The statement does not say anything about what the property owners would stand to loose when the water from Brokavatn disappeared, and the Supreme Court also remain silent on this. Nor do they mention that the statement was never handed over to the applicants, and that the details of the calculations were never handed over to, or considered by, the NVE. In fact, the full report first appeared during the hearing at Gulating lagmannsrett. This fact was not considered relevant by the Supreme Court.

Moreover, the Supreme Court remained silent on the fact that the conclusion concerning efficiency of exploitation contradicts both the NVE's own assessment, the statement made by Hjelmeland Kommune, and also all subsequent assessments made both on behalf of the applicants and on behalf of Jørpeland Kraft AS. All of the above were presented to all national courts, including the Supreme Court.

As to the legal questions raised by the case, the Supreme Court makes a more detailed argument than the regional court, culminating in the conclusion that established practices still provide adequate protection. Interestingly, the Supreme Court base their arguments in this regard on the premise that the case does \emph{not} involve expropriation of waterfalls. A similar sentiment was also expressed by Gulating lagmannsrett. However, the true force of this point of view did not become apparent until the case reached the Supreme Court.

The Court first concludes that a legal basis for the concession to transfer the water is to be found in the \cite[16]{wra17}. Moreover, they conclude that while this provision alone does not provide a right to expropriate the waterfall, it does give the applicant a right to divert the water away from it. While the Supreme Court notes that this amounts to an interference in property rights, they take it as an argument in favour of regarding the rules in the \cite{wra17} as the primary source of guidance also in relation to procedural rules. 

Hence, the provisions in the \cite{ea59} are regarded as relatively unimportant compared to the rules of the \cite{wra17} and established practices with respect to the provisions in this Act. Moreover, the main reason the Court give for this is that the diversion of water is \emph{not} to be considered as an expropriation of a waterfall. As I have mentioned earlier, there is no rule in the \cite{wra17} which states that the authorities are required to consider specifically the question of how the regulation affects the interests of property owners. Such a rule is only found in \cite[2]{ea59}. But according to the Supreme Court, this rule does not apply at all in cases where water is being diverted away from a river.\footnote{It follows, by implication, that the Court regards this rule as ``in conflict with'' or ``unsuitable'' in such cases, c.f., \cite[31]{ea59}.} This is so, according to the Supreme Court, because transferral of water is not regarded as a case of expropriation of a right to the waterfall, but merely an expropriation of a right to deprive the waterfall of water.

This is significant in two ways. First, it is important with respect to the legal status of owners who are affected by projects involving transferral of water. In Norwegian law after {\it Jørpeland}, it seems that established practice with respect to the assessment of such cases, focusing on environmental aspects and the positions taken by various interest groups, is beyond reproach already because such cases do not involve expropriation of waterfalls. 

However, the Norwegian water authorities themselves indicate that they follow similar practices for all large-scale hydropower cases, by relying on an expropriation presumption when the automatic right to expropriate does not apply directly.\footnote{See \cite{flatby08}.} Hence, it remains to be seen if cases involving explicit expropriation of the waterfalls themselves will be treated differently by the courts. If so, it leads to the peculiar situation that the level of protection for owners depends on the technicalities of how the developer proposes to gain control over the water.

The difference appears arbitrary from the point of view of owners. But of course, it will soon cease to be arbitrary for developers, who must be expected to favour gutter projects, collecting water from many small rivers and diverting it, since this mode of exploitation makes it easier to acquire necessary rights. On the other hand, if the Supreme Court is to be understood as saying that traditional practices are adequate in general, the consequences of the decision seem fairly dramatic for local owners. It would then appear to be practically impossible to solicit any kind of judicial review in hydropower cases, even in circumstances when the factual basis of the licensing decision is manifestly erroneous. Moreover, the administrative branch is exempt even from the duty of informing owners of assessments made regarding them and their property interests.

To illustrate that a lack of consultation is a general problem, not confined to the particular case of {\it Jørpeland}, I will conclude by offering a quote from Harald Solli, director of the Section for Concessions at the Ministry of Petroleum and Energy. Sollie submitted written evidence to the Supreme Court regarding the practices followed in cases involving expropriation of waterfalls. Below, I give one of several exchanges that demonstrate how current practices leave local owners in a precarious position.

\begin{quote}
Q: In cases such as this, should owners affected by a loss of small-scale hydropower potential be kept informed about the factual basis on which the authorities plan to base their decision? I am thinking especially about those cases in which the authorities make an assessment regarding the potential for small-scale hydropower on affected properties. \\
A: Affected owners must look after their own interests. The assessments made by the NVE in their report is a public document, and it can be accessed through the homepage of the NVE.
\end{quote}

By their reasoning in \emph{Jørpeland}, it appears that the Supreme Court gave this dismissive attitude towards local owners a stamp of approval. In light of this, I believe the study of the law in a socio-legal setting becomes all the more relevant. For while this attitude might be a correct interpretation of current national law, as determined in the final instance by the Supreme Court, it seems pertinent to ask if it is \emph{reasonable} law. Also, it seems that one must ask if a case can be made with respect to human rights, by arguing that the protection awarded is insufficient wih regards to P1(1). This point was raised in \emph{Jørpeland}, but did not receive any attention from the Supreme Court. In the following section, I briefly describe some further questions raised by the case.

\subsection{A Predatory Taking?}

Following \emph{Jørpeland}, I conclude that the liberalisation of the energy sector does not imply that original owners are entitled to increased protection or participation during decision-making processes regarding the use of rivers and waterfalls. This, at least, is the view held by the Norwegian judiciary. Of course, one should not overlook the possibility that the water authorities themselves will eventually adopt new practices regarding the assessment of such cases. So far, however, it seems that they stick quite closely to the established routine.

Hence, it seems reasonable to ask about the sustainability of these practices. In fact, I believe the case of \emph{Jørpeland} illustrates why the current system is inadequate, and how it can lead to decisions that appear ill-founded and leave the affected communities marginalised. The likelihood of factual mistakes, in particular, might increase greatly when the involvement of the local population in the decision-making process is not ensured.

More importantly, it seems that when decisions are made following the traditional procedure, it can often be hard or impossible to see any legitimate reason why the project proposed by the developer would be a better form of exploitation than allowing the local owners to carry out their own projects. In the case of \emph{Jørpeland}, it seemed that small-scale hydropower would be a better way of harnessing the water in question, even in the sense that it would be more efficient, providing the public with more electricity at a lower cost. More generally, unless the issue of alternative exploitation in small-scale hydropower is considered during the assessment stage, one risks making decisions that are not in the public interest at all. 

Additionally, this can send out the signal that expropriation of owners' rights is undertaken solely in order to benefit the commercial interests of the energy company applying for a development license. At this point, it seems appropriate to recall the concerns expressed by US Justice O'Connor in the case of {\it Kelo}, regarding the disproportionate influence of powerful commercial actors in takings cases.\footnote{As discussed in Chapter 1.} In relation to these concerns, a major point of contention is  whether or not Justice O'Connor's grim predictions about the fallout of the {\it Kelo} decision did indeed reflect a realistic analysis. 

Surely, anyone who agrees with Justice O'Connor that the powerful will usurp the power of eminent domain to the detriment of the poor, would also agree with her conclusion that it is perverse. However, whether her pessimism is warranted by empirical fact seems less clear. In this context, I believe the case of Norwegian hydropower serves a broader purpose, since it demonstrates that a loose interpretation of the public interest requirement can  indeed lead to transfers of property from those with fewer resources to those with more.

However, we also need to be clear about the fact that property has a social and political function that goes beyond the financial interests of individuals. For the Norwegian case at least, it seems particularly relevant to ask if local people, by virtue of their right to property and their original attachment to the land, have a legitimate expectation \emph{both} that their commercial interests should be protected, \emph{and} that they should be granted a say in decision-making processes. 

Protection of individual financial interests does not necessarily imply social protection, and the right to participate might be both more significant, and harder won, than the right to be compensated according to whatever the courts regard as the market value of the property in question.

\noo{ Another perspective, which I will pursue in more detail in the next chapter, is the question of how property rights relate to the overreaching goal of sustainable development of natural resources. Rather than seeing property rights as a means towards securing sustainable development, it is presently very common to see it as an impediment. This, indeed, has shaped much of the Norwegian discourse regarding environmental law and policy, including the law relating to hydropower.\footnote{For example, such a sceptical view of property rights appear to provide an overarching perspective on the law of sustainability in \cite{backer12} (a widely used textbook on environmental law in Norway, by one of the most influential jurists over the past 25 years).}

Moreover, a typical justification given for interference in property is that an equitable and responsible management of natural resources requires it. It seems, however, that an egalitarian system of private ownership of resources -- as we find in Norway -- could itself serve as a sustainable basis for management of these resources. In particular, private property rights seems like a potentially robust way in which local communities can be given a degree of self-determination concerning how to manage local resources. 

This is in itself considered desirable from the point of view of sustainability, but often, it appears that the favoured approach to securing it is through administrative arrangements that is supposed to empower local people. The voice that locals get, under such an arrangement, is heard at the discretion of the administrative branch. Hence, it is easily drowned in the context of large-scale commercial development, particularly when a narrative is established whereby this development is carried out for the ``common good''. I believe the case of Norwegian hydropower illustrates this point, and also shows that there is every reason to remain sceptical when commercial interests claim that they embody the public interest.

If property becomes a privilege for the few, rather than an obligation for the many, there is reason to expect that the level of vigilance will be diminished in this regard. In this way, takings for profit serve a double purpose in that they establish a culture of property which threatens to transform it from an egalitarian institution focused on virtue, to an elitist institution focused on entitlement.
The repercussions of this, I believe, extend well beyond the local communities that are adversely affected in the first instance. For example, it seems to me that an egalitarian property regime is one of the paramount guarantees we have that the state will be able to effectively and rationally exercise its regulatory powers, without bias. 

In a system where property no longer acts as an equaliser, but rather as a marker of political and commercial power, one must expect that the government itself will be more easily intimidated by commercial interests. This becomes particularly likely if such interests permeate the political system itself. The case of hydropower in Norway illustrates how this can happen, as a result of naive policies of state-ownership, aiming to protect the public interest, but effectively serving to render the public will subservient to the imperatives of the market. 

Moreover, a capitalist elite which commands significant regulatory power may not take lightly to what they perceive as undue political interference in their business practices. A company which is partly owned by the state, operates the electricity grid on behalf of the public, and is accustomed to expropriating property without meeting with much resistance, is likely to act with great confidence and boldness also when it faces political opposition. Much more so, one would presume, than a group of peasants, the original owners of waterfalls.

I think the case of \emph{Jørpeland} suggests that these perspectives are all relevant when considering takings for profits and their consequences. In the next chapter, I will turn to the question of how to ensure commercial development via compulsion by a different route, {\it without} incurring the negative effects associated with expropriation in such circumstances. Here I believe the fundamental challenge is one of setting up a decision-making structure that can ensure a balance between the various stakeholders. Importantly, the decision-making structure itself should recognise that local owners are those who have the most to loose, and gain, from undertakings involving their property. 

Indeed, I believe that the appropriate form of democratic decision-making involving property interests needs to reflect both the obligations and rights associated with those interests. The owners must not be conceptualised as an impediment to development, but as a potential driving force. Moreover, they cannot be conceived of merely as interested bystanders, but must be recognised as {\it the} primary actor in any collective action that seeks to transform their property so that it may serve a more valuable social function. No doubt, to achieve such a transformation may require compulsion. But it need not require outright expropriation, as I will attempt to demonstrate in the next chapter. The overriding concern is to develop strategies whereby the state can impose new property uses without undermining existing distributions of property rights and obligations among its citizens.
}

\section{Conclusion}\label{conc}

\noo{ In this Chapter, I have argued that the law relating to expropriation of waterfalls in Norway is based on a tradition that sees owners as profit-maximising and the state as welfare-seeking. }
In this Chapter, I have studied the law and practices relating to the taking of riparian rights under Norwegian law. I observed that the question of striking a balance between private and public interests is approached under the presumption that private property rights embody mainly private values, while public values are pursued through regulation that ensures public ownership and control. I tracked how this perspective shaped the law of expropriation of waterfalls, so that expropriation could only take place for narrowly defined public purposes and only to the benefit of public bodies.

I noted, however, how the increasing centralisation of the energy sector and the increasing scale of projects following the Second World War led to increased worry about the legitimacy of interference in property and the natural environment. I concluded that the ensuing conflicts failed to make much of an impact on the law relating to hydropower, which was still organised as a public service at this time. 
The perceived level of political control over the sector meant that courts shunned away from adopting a strict view on legitimacy. 

However, this did not mainly apply to the question of the authority to expropriate, which was hardly raised at all in the period between the reversion controversy of the early 20th century and the market-reform of the early 1990s. It applied mainly to general procedural rules. Here the Supreme Court adopted a stance whereby these rules were themselves considered largely ``discretionary'' in nature. Hence, it would fall under the authority of the executive to determine their scope and application in concrete cases.

I noted how this perspective has been maintained by the courts and the executive even after liberalisation. I argued that today, expropriation for hydropower development must be regarded as takings for profit, typical examples of economic development takings. I discussed how the law came to be changed on this point, with a dramatically widened expropriation authority introduced in conjunction with the \cite{wra00}.

I concluded with a description of the fallout from this, as expressed concretely in the case of {\it Jørpeland}. This case served to illustrate that administrative practices developed and sanctioned during the monopoly days are now applied uncritically in the context of competing commercial interests. As a result,  expropriation has become an important tool that the powerful market players can use to gain the upper hand in competition with locally based companies or smaller companies that rely on cooperation with owners. I noted that the law as it stands is unprepared for dealing with this dynamic. 

Still, in the case of {\it Jørpeland}, the Supreme Court explicitly denied that established practices were in need of revision. Moreover, the Court refused to reconsider the established interpretation of the scope of procedural rules in hydropower cases, rejecting arguments to the effect that these must now be understood to provide protection for waterfall owners that matches the protection offered to other affected parties.

\noo{ In the next Chapter, I will consider an aspect of the law were the Supreme Court {\it has} taken the view that a revision of established practices is in order, namely in relation to the question of compensation. I note, however, that the Court's emphasis on the compensation issue serves to reinforces the idea that private property rights pertain mainly to financial entitlements. As I have already argued, this perspective hardly does justice to the role of private ownership of waterfalls in Norway. }

In the next chapter, which is the last chapter of the thesis, I consider land consolidation as an alternative to expropriation. I will argue that it has great potential for successfully addressing holdouts among local owners without giving rise to many of the problems associated with using expropriation to facilitate commercial development of hydropower. However, as demonstrated in the present chapter, Norwegian courts do not seem willing to recognise the shortcomings of the current system. Until they do, or are directed to do so by political bodies or international tribunals, it is unlikely that expropriation law will evolve much from its current fixation on the compensation issue and its unshaken belief in public-private commercial partnerships as arbiters of the common good.

%\chapter{Just compensation}\label{chap:5}

\section{Introduction}\label{sec:into5}

In this Chapter, I consider the question of compensation for waterfalls in more depth. The main issue that arises is whether or not owners should be compensated for the loss of a commercial hydropower potential. If so, the compensation payments can be very large, so large that expropriation will no longer be a feasible option. Traditionally, however, no such compensation was awarded and the amounts paid to owners were negligible. In fact, owners would often have been left with nothing at all, were it not for the fact that a theoretical compensation formula was developed which avoided this outcome, by ensuring some degree of benefit sharing.

The question of whether or not to base compensation on the loss of a commercial hydropower potential is closely related to the so-called ``no scheme'' principle, according to which compensation is to be based on the value of the property such as it would have been if the expropriation scheme had not been authorised. If one takes the view that hydropower development is the prerogative of the party that obtained such an authorisation, it follows from the principle that no compensation is payable to the owners of the waterfalls, at least not for the hydropower potential. The value of hydropower, in particular, is then regarded as being due to the scheme, not due to the natural resource that the waterfall represents. This perspective was implicitly adopted in Norwegian law from the early 20th to the early 21st century, when it began to loose ground due to the liberalization of the energy sector.

The structure of this chapter is as follows: In Section \ref{sec:nsp}, I provide a comparative and theoretical context for the case study that is to follow. I do so by discussing the origin and current status of the no-scheme principle in UK law. This facilitates a broader perspective on the data presented on Norwegian law in subsequent sections.  It also allows me to make a more general point, namely that the need to distinguish between commercial/private and public values inherent in a development project arises with great force when one attempts to apply the no-scheme principle in the context of an economic development taking. I identify the lack of a well-developed framework for making such a distinction as one of the main problems associated with such takings. The main worry is that the principle, when applied to commercial values associated with a development scheme, results in {\it discrimination}. Some categories of owners are entitled to commercial benefits that other categories of owners are effectively deprived of by an application of the no-scheme principle.

In Section \ref{sec:norcom}, I go on to present Norwegian compensation law, with a focus on various manifestations of the no-scheme principle. I also present the special judicial procedure used to award compensation following expropriation. I pay particular attention to the fact that it relies on the use of lay appraisers, who also have considerable influence over the application of the law in such cases. This system, I argue, is potentially very flexible, allowing compensation awards to be based on broad and contextual fairness considerations, rather than static application of special rules. It means, in particular, that the no-scheme principle is not applied without exception, even if it does have status as a general principle. I finish this section by noting that the traditional system has been somewhat undermined since WW2, following legislation specifically aimed at reducing compensation payments and narrowing the room for lay discretion in appraisal disputes.

In Section \ref{sec:nathp}, I move on to consider compensation for waterfalls. The first thing I note is that the no-scheme principle was not traditionally applied, since it would have led to little or no compensation for owners during the monopoly era. Instead, a theoretical method was used, based on the notion of natural horsepower. This method was meant to give owners a share of the benefit in hydropower development and was developed by the appraisal courts early in the 20th century. It was modelled on the market for waterfalls that had existed prior to monopolization, but as the years went by, the method became farther and farther removed from the physical and economic realities of the hydropower industry. Increasingly, it resulted in no more than a symbolic form of benefit sharing with owners.

Following liberalization the traditional method has been abandoned for several categories of cases, a development that I discuss in Section \ref{sec:fa}. The crucial condition for applying the new method, based on market-value assessment, is that an alternative development scheme would have been ``foreseeable'' in the absence of the expropriation project. How to interpret the meaning of ``foreseeable'', and how to determine the scope of the ``expropriation project'', are crucial issues currently being worked out in Norwegian case law. I link the reform in this area with the institutional framework surrounding appraisal disputes. I note, in particular, that the natural horsepower method was first abandoned by the appraisal courts, with the lay appraisers assuming a leading role. 

Unfortunately, the market-based method also raises problems, the most severe of which are related to the scope of the no-scheme principle. The lack of clarity about its scope and implication has created a situation where it appears that if owners are lucky, or employ skilled arguers, they can collect a very substantial sum of money with little or no effort and with no social responsibilities attached. On the other hand, if they are unlucky, they are forced to give up what is often the most valuable asset of their local community for nothing but a symbolic payment. I conclude by arguing that a much better approach would be to try and get owners involved in sustainable hydropower in a way that can remove the need for expropriation altogether. 

This sets the stage for the last chapter of this thesis, where I return to the question of how to replace expropriation by mechanisms of participatory democracy, referring back also to the discussion in Chapter \ref{chap:1}.

%As development is now organized as a commercial pursuit, this should in principle be possible, since the owners {\it do} have an incentive to get involved, also in cases when the public dictate the set of possible terms through strict regulation. In practice, however, what is needed is a mechanism for organizing such owner-involvement. This mechanism will undoubtedly also need to be endowed with powers of coercion if it is to be effective.
%
%The Supreme Court struck down their judgement on a technicality, but refused to reject the principle that lay people were free to adopt a new method in cases when the traditional method would not adequately reflect the value of ``foreseeable'' use. I argue that this shows the strength of a long tradition of respecting the discretion of lay people in appraisement disputes. Many legal scholars, in particular, had previously regarded the natural horsepower method as a {\it rule of law}, set by precedent.
%
%The method has not been abandoned as a matter of principle, however. As made clear recently by the Supreme Court, it is still to be applied in cases when a calculation based on ``foreseeable use'' does not lead to higher compensation payments. The crucial question becomes what exactly is meant by this notion. I address this in some depth, by pointing to how Norwegian law in general is  marked by a tendency to disregard any use that is not sanctioned by public plans, including in cases when these plans themselves provide the rationale for expropriation. This appears to be contrary to the no-scheme principle, demonstrating more generally that only ``one half'' of the principle tend to apply in a Norwegian setting. In so far as the principle precludes giving the owner a share of the expropriation surplus, it is applied, but in so far as it entitles him to compensation based on a future use that is rendered unforeseeable by the planning underlying the expropriation license, it is not.
%
%There are some exceptions to this, however, and the Supreme Court has indicated that one of them applies to hydropower cases. At the same time, however, it has been stressed that even if the expropriation plans themselves are not binding for the compensation assessment, the ``public rationale'' underlying these plans must be taken into consideration when awarding compensation. In effect, this means that compensation is not offered for alternative uses in so far as the project proposed by the expropriating party is superior and could not be undertaken by the current owner. In effect, it seems that a partly {\it subjective} standard is introduced into compensation law, whereby local owners are denied compensation for a commercial value that is deemed to be such that it is only realizable by the expropriating party.
%
%The Supreme Court has not been entirely consistent about the scope and exact content of the ``public rationale'' principle, however,   and the issue is still very much contested in Norwegian courts. In Section \ref{sec:ko}, I illustrate the current unclear state of the law by contrasting two recent Supreme Court cases. In the first, the court embraced an objective version of the ``public rationale'' principle by holding that as the expropriating party's project resulted in more public benefits, compensation could be based on the premise that the owners' foreseeable use of the waterfalls was to cooperate with the large energy company in realizing the plans, to take their share of its commercial potential. 
%
%In {\it Otra II} on the other hand, the Court held that this should not be the conclusion in so far as cooperation was deemed to be ``impractical'', following a concrete assessment of the facts. It seems quite clear that the notion of ``impracticality'', as it was used here, serves to introduce a subjective assessment standard, contrary to what otherwise dictated by Norwegian compensation law. 
%I go on to consider the merits of {\it Otra II} against human rights law, anticipating also the outcome of the appeal currently lodged with the ECtHR in Strasbourg.
%
%Finally, I conclude that the case law on compensation demonstrates the intrinsic inadequacy of a narrow perspective on takings for profit. It seems clear, in particular, that all of the approaches currently in use to calculate compensation for waterfalls leave great room for bickering, manipulation and long-winded court battles. Moreover, the factual premise for the calculation is typically extremely uncertain, meaning that the whole procedure appears as something of a gamble, for both owners and developers. Hence, the developers favour the use of the natural horsepower method, which is completely removed from the reality of hydropower, but deliver predictably low compensation payments that will not prove too damaging to the profit-margin of the development company. On the other hand, owners have an incentive to push for compensation mechanisms that will allow them to collect the entire financial potential of hydropower development without actually investing any effort in planning or administerting such development, and without subjecting themselves to any of the risks involved. 

\section{The ``no scheme'' principle}\label{sec:nsp}

In most jurisdictions, a fundamental principle relating to compensation following expropriation is that compensation should be calculated without taking into account changes in the property's value that are due to the expropriation, or the scheme underlying it. In short, compensation should be based on the owner's loss, not the taker's gain. In a recent Law Commission consultation paper, this principle is referred to as the \emph{no-scheme} rule, a terminology I will also adopt here, noting that while the exact details of the rule might differ between jurisdictions, the underlying principle appears to play a crucial role in both civil and common law traditions for regulating compensation following expropriation.\footnote{I am not aware of a single jurisdiction that does not include some rule corresponding to (aspects of) the no-scheme principle. I mention that in addition to the jurisdictions discussed in this section, no-scheme rules are also found in pure civil law jurisdictions like Germany and the Netherlands, see \cite[5,21]{sluysmans14}.}

While the no-scheme principle is easy enough to comprehend when it is stated in general terms, it raises many difficult questions when it is to be applied in concrete cases. What the rule asks of the valuers, in particular, is quite daunting; they are forced to consider a counterfactual ``no-scheme world'', and they must calculate the value of the property based on the workings of such an imaginary world. The crucial question that arises, of course, is the question of what exactly this world should be taken to look like.

In the first instance, it might be tempting to state simply that this is a ``question of fact for the arbitrator in each case'', as expressed by the Privy Council in \emph{Fraser}, a Canadian case from 1917.\footnote{\cite[194]{fraser17}.} However, as the history of the no-scheme rule has shown, this point of view is not tenable.\footnote{For a history of the rule in UK law, clearly illustrating the difficulty in interpreting it and applying it to concrete cases, I point to Appendix D of \cite{lawcom03}. See also \cite{lawcom01}.}  The problem is that the nature of the no-scheme world cannot be determined without making a vast range of assumptions, many of which appear to depend on how one understands the law. The challenges that arise were discussed in great detail by Lord Nicholls in the recent case of \emph{Waters}. He described the task as ``daunting'', noting also that some of the more recent statutory provisions ``defy ready comprehension''.\footnote{\cite[19]{waters04}.}

\noo{
\begin{quote}
The extreme complexity of the issues that I have had to consider, the
uncertainty in the law, the obscurity of the statutory provisions, and
the difficulties of looking back over a long period of time in order to
decide what would have happened in the no-scheme world
demonstrate, in my view, that legislation is badly needed in order to
produce a simpler and clearer compensation regime. I believe that
fairness, both to claimants and to acquiring authorities, requires
this
\end{quote}
}
The Lords clearly saw \emph{Waters} as an opportunity to offer a clarification on the no-scheme rule and how to interpret it. In particular, their judgement went into more detail than what seemed necessary for the case at hand. Even if it was not needed for the result, the Lords also addressed many of the issues raised by the Law Commission in their recent report, focusing particularly on resolving the tension which was identified there between the principle relied on in the \emph{Pointe Gourde} case and the reasoning adopted in the so-called \emph{Indian} case from 1939.\footnote{\cite{indian39,gourde47}.} In the \emph{Indian} case, the scheme was given a very narrow interpretation, with Lord Romer interpreting the scope as follows.\footcite[319]{indian39}

\begin{quote}
The only difference that the scheme has made is that the acquiring
authority, who before the scheme were possible purchasers only, have
become purchasers who are under a pressing need to acquire the
land; and that is a circumstance that is never allowed to enhance the
value.
\end{quote}

Importantly, this did not entail that the purchaser's demand for the property was to be disregarded, since, as Lord Romer puts it:\footcite[316-317]{indian39}

\begin{quote}
[...] The fact is that the only possible purchaser of a potentiality is
usually quite willing to pay for it […]
\end{quote}

In \emph{Pointe Gourde}, a different stance appears to have been adopted.\footcite{gourde47} The case concerned a quarry that was expropriated for the construction of a US naval base in Trinidad. The quarry had value to the owner as a business, and the valuer had found that if the quarry had not been forcibly acquired, it could also have supplied the US navel base on a voluntary basis, thereby increasing its profits. However, the value of this potential fell to be disregarded, with Lord MacDermott describing the no-scheme rule as follows:\footcite[572]{gourde47}

\begin{quote}
It is well settled that compensation for the compulsory acquisition of
land cannot include an increase in value, which is entirely due to the
scheme underlying the acquisition
\end{quote}

Seemingly, this is at odds with the position taken by Lord Romer in the {\it Indian} case. It seems clear that in the absence of a compulsory purchase order, the US would have been ``quite willing'' to pay for the quarry's services. Still, this potential had to be disregarded. 

In \emph{Waters}, both Lord Nicholls and Lord Scott addressed the tension between the two decisions in great detail. They then offered a reconciliatory interpretation, one which seems to narrow the no-scheme rule compared to how it has most commonly been understood following \emph{Pointe Gourde}. Moreover, the House of Lords also noted the need for reform and legislation, with Lord Scott describing the current state of the law as ``highly unsatisfactory''.\footcite[164]{waters04}

To explain how a seemingly simple principle could become so troubling in practice, I think it is important to start by noting that after the introduction of extensive planning legislation in the 20th century, development of property tends to be contingent on governmental licenses and plans. Moreover, the power to expropriate is often granted as a result of comprehensive regulation of the property-use in an area, often following public plans that encompass more than the particular project that will benefit from compulsory purchase. As a result, it has become increasingly difficult to ascertain what is meant by the ``scheme'' in compensation cases. Does it include the whole planning history leading to expropriation, does it only refer to the power to expropriate, or is it something in between?

A fine balancing act must be made when attempting to answer this question. Under a wide interpretation of ``the scheme'', forcing the valuer to entertain many counterfactual assumptions, the property owner might come to feel that he is not compensated for his true loss, but rather an imaginary one. Indeed, the no-scheme world that the valuer must consider can end up being far removed from the actual one, forcing him to go back many years, perhaps decades, to establish what would have been the status of the property in question if the sequence of planning steps eventually leading to expropriation had not taken place. 

This can leave the property owner in an unpredictable and very weak position. Taken to extremes, the no-scheme principle can then also come to run amiss with respect to human rights law and constitutional provisions protecting private property. On the other hand, if the scheme is interpreted too narrowly, one runs the risk of endangering important public schemes by compelling the public to pay extortionate amounts. In many cases, it is undoubtedly true that the value of property is increased by public investments and plans for the area in which the property is found. Moreover, one may ask if it is right to pay compensation based on increases in value that result from investments and plans that would not have materialised unless the power to expropriate had been anticipated. This, it may be argued, would be a form of double payment that should be avoided.

As noted by the Law Commission, it is important to keep in mind that the no-scheme rule serves at two distinct purposes.\footcite[69-70]{lawcom03} First, the rule has an important \emph{positive} dimension, enhancing compensation payments. Property owners are not only compensated for the direct loss of their property, but also for the possible depreciation of their property's value following the decision to carry out a scheme which requires expropriation. Seemingly, this is easy to justify: It seems intuitively unreasonable if the deleterious effects of a threat of compulsion is permitted to result in reduced compensation payments.

However, under the extensive planning regimes common today, it is not clear where to draw the line. When is the regulation leading up to the scheme to be regarded as reflecting general public control over property use, and when is it to be regarded as a measure specifically aimed at compelling private owners to give up their property? As we will see when we consider the role of the no-scheme rule in Norwegian law, this question can easily become highly controversial, especially when it is linked with the more general question of whether or not the state should be liable to pay compensation for regulation that adversely affects the potential for future development. In jurisdictions that do not recognize owners' right to such compensation, like Norway and England, it is easily argued that the positive aspect of the no-scheme rule must be limited correspondingly. Why should a depreciation of value following regulation imply compensation when the property is eventually expropriated, but not otherwise?

In addition to its positive dimension, the no-scheme rule also has an important \emph{negative} dimension, expressed in {\it Pointe Gourde} as the principle that an {\it increase} in value should be disregarded when it is ``entirely due to the scheme''. The negative dimension has attracted more interest and controversy than the positive dimension, especially in the UK. This is also the aspect of the rule that was at the center of attention in {\it Waters}.

It is not surprising that the negative aspect of the no-scheme principle more often results in complaints, as property owners stand to loose whenever it is applied. However, on a traditional understanding of the public purpose of expropriation, the negative aspect of the rule is also seemingly easy to justify. In \emph{Waters}, Lord Nicholls describes the most important policy reasons as follows:\footcite[18]{waters04}

\begin{quote}
When granting a power to acquire land compulsorily for a particular purpose Parliament cannot have intended thereby to increase the value of the subject land. Parliament cannot have intended that the acquiring authority should pay as compensation a larger amount than the owner could reasonably have obtained for his land in the absence of the power. For the same reason there should also be disregarded the ``special want'' of an acquiring authority for a particular site which arises from the authority having been authorised to acquire it.
\end{quote}

This appears like a reasonable justification. Notice, however, that Lord Nicholls avoids using the word ``scheme''. In particular, he does not identify the scheme's absence as the measuring stick for ascertaining on what basis parliament intends compensation to be based. Rather, Lord Nicholls speaks of what the owner could reasonably have obtained in the \emph{absence of the power} to acquire the land compulsory. In this way, he seems to prescribe a rather narrow interpretation of the negative dimension of the no-scheme rule.\footnote{See also the commentary offered in \cite{newuk}.} It is the power to expropriate that should not give rise to an increased value, nothing at all is said at this stage about the scheme that benefits from it.

It would appear, therefore, that there is nothing in principle that prevents the property from being compensated on the basis of its value in a scheme that differs from the scheme underlying expropriation only in that it does not have such powers. Indeed, this subtle caveat appears to be rather crucial for the remainder of Lord Nicholls' arguments, when he attempts to reconcile the principle adopted in the \emph{Indian} case with the \emph{Pointe Gourde} case.

It would lead me too far astray to go into all the subtle details about the interpretation of the no-scheme rule in UK law and the possible implications of \emph{Waters}. Rather, I would like to focus on one specific aspect, namely the application of the principle when the scheme in question is a commercial enterprise. The UK Supreme Court touched on this issue in the recent case of  \emph{Bocardo}.\footnote{\cite{bocardo10}.} The case was decided under dissent, suggesting that the clarifications offered in \emph{Waters} have not been as conclusive as one might have hoped.

\emph{Bocardo} concerned a reservoir of petroleum that extended beneath the appellant's estate. The petroleum could not be extracted without carrying out works beneath their land. The first question that arose was whether or not extraction of the petroleum amounted to an infringement of property rights. This was answered in the affirmative. The second question that arose was what principle of compensation should be adopted to compensate the owner. The Supreme Court, following some deliberation, found that the general rules applied, meaning that the case should be decided on the basis of an application of the no-scheme principle.

However, opinions differed as to the correct interpretation of this principle, as well as how the facts should be held against the law. The crucial point of disagreement arose with respect to whether or not the special suitability, or \emph{key value}, of the appellant's land, \emph{pre-existed} the petroleum scheme.

In \emph{Waters}, the House of Lords had cited and expressed support for the following passage, taken from Mann LJ's judgement in \emph{Batchelor}.\footnote{\cite[361]{batchelor89}. Cited by Lord Nicholls at \cite[65]{waters04}.}

\begin{quote}
If a premium value is ``entirely due to the scheme underlying the acquisition'' then it must be disregarded. If it was pre-existent to the acquisition it must in my judgement be regarded. To ignore the pre-existent value would be to expropriate it without compensation and would be to contravene the fundamental principle of equivalence.
\end{quote}

%(see \emph{Horn v Sunderland Corporation})
Relying on this distinction between the potentialities that are ``pre-existing'' and those that are due to the scheme, the minority in \emph{Bocardo}, led by Lord Clarke, made the following observation.\footcite[42]{bocardo10}

\begin{quote}
Anyone who had obtained a licence to search, bore for and get the petroleum under Bocardo’s
land would have had precisely the same need to obtain a wayleave to obtain access
to it if it was not to commit a trespass. So it was not the respondents' scheme that
gave the relevant strata beneath Bocardo’s land its peculiar and unusual value. It
was the geographical position that its land occupies above the apex of the
reservoir, coupled with the fact that it was only by drilling through Bocardo’s land
that any licence holder could obtain access to that part of the reservoir that gives it
its key value.
\end{quote}

This view was rejected by the majority, led by Lord Brown, who interpreted the no-scheme rule quite differently:\footcite[83]{bocardo10}

\begin{quote}To my mind it is impossible to characterise the key value in the ancillary
right being granted here as ``pre-existent'' to the scheme. There is, of course,
always the chance that a statutory body with compulsory purchase powers may
need to acquire land or rights over land to accomplish a statutory purpose for
which these powers have been accorded to them. But that does not mean that upon
the materialisation of such a scheme, the ``key'' value of the land or rights which
now are required is to be regarded as “pre-existent”.
\end{quote}

While the case was resolved in keeping with this view, the dissent suggests that the clarification in \emph{Waters} has not resolved all issues. Moreover, it suggests that special questions arise when the expropriation scheme itself involves the realisation of a commercial potential inherent in the land that is taken. Is it permissible for government to grant the value of this potential to the taker -- by granting him the necessary licenses -- without subsequently recognizing the potential as having been taken from the owner? 

This issue does not \emph{not} primarily depend on the scope of the scheme as such. In {\it Bocardo}, for instance, it was obvious that the scheme was the entire project aimed at extracting petroleum from the reserve, including the necessary works beneath the appellant's estate. But even so, it was still unclear whether the special value of the appellant's land could be said to have been {\it caused} by the scheme. The issue that arises in these kinds of situations is ontological: When should we attribute a given value to an act of government, and when should we attribute it to nature, as a fruit of the land? Or in more practical legal terms: When is a given property value that is unlocked by a development scheme part of the original owner's bundle of rights?

To answer this question, it is tempting to look for a causal link between scheme and value, to substantiate the claim that the value was not in fact pre-existent. But as \emph{Bocardo} illustrates, it is not always obvious what should be taken as good evidence for such a link. It seems that one's perspective on this will tend to depend also on one's point of view on the much more general question of what values one recognize as inherent in property rights.

When Lord Clarke remarked that the state, following nationalisation in 1934, could have given the right to extract the petroleum to \emph{someone else}, he was certainly correct. Hence, I also agree with him that ``the key value was not created by the 1934 Act or the grant of the petroleum licence to Star''.\footnote{See \cite[163]{bocardo10}.} But whose value was it, and was it a commercially realisable value? Here, Lord Clarke appears to assume that the value must belong to the property owner and that this owner would also have been able to make a profit from it in the absence of the expropriation scheme. This, I believe, is a leap that requires further justification. Just because some property has key value does not mean that the owner of the property is entitled to that value, or that it can ever be translated into a financial profit.

On the one hand, it is easy to agree with Lord Clarke that compulsory acquisition of a wayleave is no precondition for an extraction scheme. The project could well have been carried out by a developer who was willing to pay the owner for the special suitability of his land. But on the other hand, it does not seem obvious that the owner is meant to be able to demand such payment under the regulatory system currently in place. Hence, even in the absence of a causal link between scheme and value, one might be entitled to conclude that the special value falls to be disregarded because it has already effectively been removed from the owner's bundle.

In the case of {\it Bocardo}, I think this perspective would have been particularly helpful to Lord Brown, who argued that the value of the strata was not pre-existent. As it stands, his argument seems rather strained. After all, it was the physical conditions that gave the land its value, not the abstract fact that a development license had been granted. However, by looking at his argument in more depth, it is tempting to rephrase his conclusion by saying that he regarded the special suitability of the strata as having no commercial value under the prevailing regulatory regime.

In the end, I am agnostic about the correct way to judge {\it Bocardo}, but I think the crucial question that it raised was the following: did parliament intend to give petroleum developers a right to extract substrata resources without sharing the profits with affected surface owners? If no clear answer is available, conflicts can result, particularly if the question itself is obfuscated, as I think it was in {\it Bocardo}. It seems to me, in particular, that the focus on causality and the notion of ``pre-existence'' was not very helpful. Rather, I think the crucial keyword should have been benefit sharing.

The first question to ask in this regard is what parliament intended when it set up the current regulatory framework. If this is unclear or the evidence suggests that benefit sharing was not intended, the question becomes whether or not benefit sharing is nevertheless required on the basis of constitutional or human rights law. In a case like {\it Bocardo}, the latter question is unlikely to arise with any great force. It seems to me, in particular, that the question of how to deal with a property's ``key value'' in relation to other property is usually a question that can be resolved merely by pointing to the legitimate public interest in avoiding unwanted holdouts.

Even so, if the courts engage with the question of benefit sharing without being explicit about it, the lack of democratic accountability can become a worry. I think it is important to emphasize the political sensitivity of the range of complex rules found in compensation law. If not, a crisp political question risks becoming obfuscated to the extent that it can only be engaged with in a meaningful way by legal professionals. This, in turn, increases the chance of abuse and undue influence of special interest groups. While most people remain ignorant of the political work done by the courts in this regard, those who stand to gain the most are free to lobby and argue on technical points to gradually shape the law of benefit sharing according to their own interests. A conceptual shift might be needed to prevent this development from becoming precarious to the legitimacy of compensation law in general, and the no-scheme rule in particular. 

In addition, the question becomes much more pressing in cases when the development potential as such is subject to expropriation. An extreme case arises when natural resources are expropriated. For an illustration which also links up to my case study, I mention particularly the cases of \emph{Cedars} (1914) and \emph{Fraser} (1917), two Canadian compensation disputes regarding expropriation for hydropower. They were cited as important authorities by both the Law Commission and the House of Lords in \emph{Waters}.\footnote{\cite{cedars14,fraser17}.} 

In \emph{Fraser}, it was the waterfalls themselves that were subject to expropriation, yet the Privy Council still found that the value of the potential for hydropower exploitation of these falls should be disregarded when compensating them. The reasoning adopted seems to follow a standard ``value to the owner'' approach. However, reflecting back on {\it Bocardo}, it is hard to see how anyone could think that the value of the waterfalls were not ``pre-existent'' to the scheme to develop them. Surely, as a natural resource, a waterfall has significant value in itself, independently of any particular ``scheme''? 

Not so, according to the Privy Council, who found that the owners of waterfalls could not themselves have developed hydropower. Here, a subjective standard was in effect employed, whereby the bundle of rights associated with a property depended not only on the property itself but also on the nature of its owner. This unequal treatment of owners is such that is could, in my opinion, now be attacked from the point of view of human rights and constitutional law.\footnote{Although such an approach might not be required to overrule them, as the Canadian cases already appear to be at odds with both {\it Waters} and {\it Bocardo}.}

However, in order to make such an attack, it is necessary to use a working distinction between commercial and non-commercial aspects of a development scheme. The pre-existence test is inadequate. For instance, there can be no doubt that the energy inherent in water pre-exists any scheme seeking to harness it. Moreover, it seems clear that energy has great value, meaning that the value of a waterfall pre-exists any scheme for hydropower exploitation. However, we must also ask: what \emph{kind} of value is it?

To illustrate why this is a relevant consideration, consider a case where the property value is enhanced for the owner because of a personal attachment. In this case, it seems fair to differentiate, so that the owner's subjective attachment to the property is taken into account, potentially leading to a higher compensation payment then any other owner would receive. It is irrelevant, moreover, whether or not the particular aspect of the property to which the owner is attached is pre-existing. The relevant consideration is simply whether or not the value in question is such that one thinks it {\it should} be compensated. The value is {\it not} commercial, however, but personal (and, in so far as it receives recognition, also public). This is {\it why} differential treatment becomes justifiable. 

Similarly, in so far as a piece of land is particularly suited for building a school, it seems unproblematic to deny benefit sharing with the owner. In this case, the suitability is pre-existent, but it reflects a value to the public, not to commerce. Hence, a disregard rule can safely be applied, even though the public would been willing to pay large amounts in friendly negotiations. But what if the land was not particularly suited for a school, but for a shopping mall? Here I believe a different standard is needed. It seems, in particular, that benefit sharing is required in this case since one would otherwise illegitimately discriminate between owners. Why should the owners of shares in a shopping mall be allowed to profit, when the owners of the suitable land are not?

As a practical test, I propose the heuristic whereby one regards the commercial value of the development as evidence that disregard rules like the no-scheme principle should not be applied. The underlying rationale behind this heuristic is based on the public interest requirement. It seems to me, in particular, that disregard rules are also in need of justification based on the needs of the public.
In my opinion, the public interest/purpose requirement extends to compensation in such a way that a value needs to be identified as a public value in order for it to be legitimate to disregard this value when compensating the owner. 

More generally, I fail to see how it could ever be legitimate to apply a no-scheme principle unless it serves the public good. If the principle is applied in a way that results in a commercial benefit to the taker and a commercial loss to the owner, I would argue that it renders the expropriation as a whole unsafe in relation to the public interest requirement. One aspect of the interference, at least, then lacks proper motivation. From this I arrive at the general conclusion that values which are recognized as commercial should never be disregarded.

The distinction between commercial and public values is obviously not written in stone, but is down to a political decision. Moreover, it can hardly be regarded as permanent. In addition, it can often be difficult to assess where the line is to be drawn, especially in cases when public-private partnerships are relied on to provide public services. Nevertheless, it seems to me that the public interest requirement in constitutional and human rights law makes it necessary to be explicit about private and public values also in relation to compensation. Moreover, it seems like doing so could be very helpful in many cases, such as {\it Bocardo} and {\it Fraser}.

For instance, even if the public value of hydropower pre-exists an hydropower scheme, this does \emph{not} necessarily mean that there is any pre-existent commercial value in hydropower. What counts as {\it commercial} value, in particular, must first be answered. This, moreover, depends entirely on whether or not the public has settled on a regulatory regime that allows commercial exploitation.
Hence, I arrive at the following suggestion for a modified version of the ``pre-existence'' test: An owner should always be compensated for the value of any pre-existent \emph{commercial} value that his property has.\footnote{Certainly, a clarification along these line would not resolve all issues. It would not, for instance, offer any conclusive guidance with respect to the specific issues related to "key value" raised in \emph{Bocardo}.} 

To answer the question of what should be regarded as a pre-existing commercial value, one must take a broad look at the prevailing regulatory regime. Moreover, one must expect that the assessment will depend on the context of regulation, in particular the extent to which the state \emph{allows} the disputed value to be commercially realized. The law relating to compensation should be such that it can tolerate significant changes in these parameters. The theoretical question that arises concerns only the conceptual foundation for the assessment. The actual lines that must be drawn are all drawn in the sand, as usual.

In the next section, I will address Norwegian compensation law to shed light on some such lines that have been drawn in relation to waterfalls, which have recently been washed away and redrawn following liberalization of the hydropower sector. This will allow me to shed light both on the no-scheme rule and alternatives to it. 

%Moreover, I note how the Norwegian system was originally based on a rejection of the idea that all disputes had to be resolved uniformly on the basis of a battery of specific rules. Instead, great emphasis was placed on the discretion of lay people. In later years, however, Norwegian compensation law has developed along a similar trajectory to that of the UK. The no-scheme principle, in particular, has now been addressed in so many different ways and by some many different sources of authority that it appears just as much in defiance of ready comprehension as in the UK before {\it Waters}. I will now try to untangle the web somewhat, while moving towards the special points I would like to make based on the idiosyncratic case of waterfalls. 
%
%But the assessment itself was above all else discretionary. Legitimacy of the process was ensured in a bottom-up fashion, by the involvement of lay people sitting as appraisers, alongside a regular judge.
%
%I note, however, that this system has largely been modified so that, today, the appraisal courts are far more constrained from the top down, by legislation and the Supreme Court. I argue that this has been a source of difficulty for the system, particularly in relation to the no-scheme principle which, as I argued above, necessitates a concrete assessment, and can not -- should not -- be resolved by all-encompassing principles. I note, in particular, how the increasingly constrained room for discretion by lay people means that the distinction between commercial and public value -- which must now be determined centrally -- becomes muddled. In many cases, the idea acting as a premise for the general rules applied simply does not correspond to reality. This often leads to unacknowledged commercial windfalls for takers, arising when owners are denied compensation for commercially valuable rights that the law presupposes to be wholly public, even though they are not. 

\section{Appraisal courts and ``foreseeable alternatives''}

The owner's right to compensation following expropriation of property is enshrined in very simple terms in Section 105 of the Norwegian Constitution.\footnote{\cite[105]{grunnloven14}.} This section states simply that \emph{full compensation} is to be paid in all cases when the public interest warrants the compulsory acquisition of property. For more than 150 years, this was the sole legislative basis for compensation rules in Norway. The methods used to calculate full compensation in different scenarios developed entirely through case law.

According to a long legal tradition in Norway, the discretionary aspects of property valuation is regulated by a special procedure, with a significant reliance on so called \emph{unwilling appraisers}. These are members of the public who have no interests in the case at hand. They may be chosen, however,  specifically for their suitability in judging the value of the contested property, either because they are resident in the local area or because they have special expertise.

The appraisal procedure has a long history, going back to customary law that pre-dates the constitution. The rules regulating it today are found in the \cite{aa17}.\footnote{Act no 1 of 1 June 1917 relating to Appraisal Disputes and Expropriation Cases.} Appraisal cases are organised similarly to civil disputes, and the procedure is administered by the district courts.\footnote{\cite[5]{aa17}.} Appraisal courts are usually composed of a panel consisting of one professional judge and four appraisers, with no special juridical qualifications. 

The standard arrangement is that appraisers are chosen from the general public in the district where the property in question is located. But the Act opens up for the possibility that appraisers may also be chosen for their special technical expertise.\footnote{See \cite[11|12]{aa17}.} Their role in the procedure is on par with the judge, and the panel decides jointly both the legal and the technical questions, usually on the basis of technical reports put forth by the parties. These reports are presented during the main hearing, and may be challenged by the parties, in more or less the same way as the district court hears evidence in a regular civil dispute.\footnote{See particularly \cite[22|27]{aa17}, with further references to the \cite{da05} (Act No 90 of 17 June 2005 relating to the Mediation and Procedure in Civil Disputes).} 

There is a possibility for appeal to the appraisal Court of Appeal, which is the regular Court of Appeal sitting as an appraisal court in accordance with the rules of the \cite{aa17}. The right to having an appeal heard is not absolute; whether the appraisal Court of Appeal will hear the case depends on its importance, according to rules that correspond to those in place for regular civil disputes.\footnote{See \cite[32]{aa17}.} The procedure closely corresponds to the procedure followed in appraisal disputes at the district level.\footnote{See \cite[38]{aa17}.} However, the decision made by the appraisal Court of Appeal is final as far the appraisal assessment is concerned. An appeal to the Supreme Court can only be accepted on legal grounds.

As a consequence of this, the appraisal courts have been very important in interpreting and developing the law relating to compensation in Norway. Their importance was particularly great all the while the meaning of ``full compensation'' was not clarified further in statute. The presence of lay people sitting as judges is consistent with how many civil disputes are resolved. But what makes the appraisal courts unique is that in these cases the lay people where traditionally allowed to engage with the issue under very few restraints, beyond procedural rules and the words of the Constitution.

At the same time, the practical viewpoint enforced by the procedural form meant that legal questions would often remain in the background in such cases. Typically, the legal issues would only come to the forefront if the Supreme Court decided to hear the case as a matter of principle. 

The primary criticism voiced against this system, particularly following the Second World War, was that it gave the appraisal courts too much discretionary power. Hence, the argument went, legislation was needed to make the outcome of appraisal cases more predictable.\footnote{See, for instance, Part 2, Chapter 1 of \cite{nut69}, handed over to the Ministry of Justice by the so called Husaas committee, appointed by the King in Council 6 Aug 1965.} However, while the law regarding compensation was not formalized in written form, there were legal scholars who developed theories and aimed to explicate its content based on the body of case law that was available.

Also, the Supreme Court did regularly hear cases concerning legal arguments regarding compensation, and they developed a consistent position on at least some of the more critical and recurring legal issues. At this time, the central source of legal reasoning regarding appraisal was still to be found in the constitution itself. As a result, theories of compensation law tended to be \emph{absolutist}, in the sense that they looked directly to wording in Section 105, also when tackling specific problems of interpretation. 

\subsection{Constitutional absolutism}

Absolutism was widely endorsed by Norwegian legal scholars as late as in the 1940s. The well-known legal scholar Magne Schjødt summed it up as follows:\footcite[177]{schjodt47}

\begin{quote}
When an owner is entitled to compensation, he is entitled to have his full economic loss covered. He should receive full compensation, see p 42 ff. This is the great principle that remains absolute and any dispute must be resolved on its basis.
\end{quote}

A typical example of the style of legal reasoning that this view gave rise to can be found in thes writings of the prominent legal scholar Frede Castberg. He specifically addressed also the no-scheme principle, by asking about the extent to which increases in value due to the scheme underlying expropriation was to be taken into account when calculating compensation. His reasoning in this regard was based directly on a reading of the Constitution. Moreover, it was based on the principle of \emph{equality}, which was at that time considered particularly crucial in understanding constitutional law. The following quote serves to sum up Castberg's position on the no-scheme principle:\footcite[268]{castberg64b}

\begin{quote}
The owner is entitled to full compensation. The expropriation should not leave him worse off economically than other owners. Hence if the public has knowledge that an industrial undertaking is being planned, that a railway will be built etc, and this affects the value of property generally in a district, then the increased value of the property that will be expropriated must be taken into account. If not, the owners of such property will be worse off than other owners from the same district. On the other hand, if the expectation of the scheme underlying expropriation leads to a general depreciation of value, then it is this new value -- not the original value -- that is relevant for calculating compensation. The crucial question is what the actual value is, when expropriation takes place.
\end{quote}

% e mention that the problem analyzed by Castberg in this passage has been considered in many jurisdiction, and is dealt with in common law by the so called \emph{no-scheme} rule. This is more a principle than a single rule, and it is typically understood as a mechanism that is meant to ensure that changes in value due to the scheme underlying expropriation are disregarded.\footnote{For an history of the rule in common law (primarily the UK), which also illustrates the difficulty in interpreting it and applying it to concrete cases, we point to Appendix D of Law Commission Report No 286, 2003} In comparative terms, Castberg appears to favor a \emph{narrow} interpretation of the principle -- a restrictive view on when additional value due to the scheme should be disregarded -- quite close in spirit to the so called \emph{Indian} case from 1939\footnote{\emph{Vyricherla Narayana Gajapatiraju v Revenue Divisional Officer, Vizagapatam} [1939] AC 302.}, which was been much discussed in common law and was dealt with extensively by the House of Lords as late as in 2004.\footnote{In the case of \emph{Waters and other v Welsh National Assembly} [2004] UKHL 19. 
%The primary precedent for a broader interpretation of the non-statutory no-scheme rule, on the other hand, is \emph{Pointe Gourde}, \emph{Pointe Gourde Quarrying and Transport Co v Sub-Intendent of Crown Lands} [1947] AC 565, PC, 572, per Lord MacDermott. This case proved highly influential for the understanding of compensation rules in the post-war period, in many common law jurisdictions, but has recently been challenged by a renewed interest in more narrow viewpoints such as that expressed in the \emph{Indian} case, see  \cite{newuk} and also the case of \emph{Star Energy Weald Basin Limited and another (Respondents) v Bocardo SA (Appellant) [2010] UKSC 35}.}In the context of Norwegian law, it is of particular interest to note how Castberg's views in this regard is arrived at through considering the constitution itself, founded on the principle of equality.\footnote{In this way, he arrives at a narrow no-scheme rule quite abstractly, and through a different route than the one adopted in the \emph{Indian} case, where the outcome appears to have turned crucially on the particular facts in the case, a close reading of precedent, as well as the perceived fairness of the result.}

As Castberg bases his analysis on the exact wording of the Constitution, he does not engage in any reasoning based on the extent to which it can be regarded as socially fair for the public to pay compensation for value that arise due to the beneficial consequences of the public project itself. Crucially, he does not address the concern that this can be seen as a form of double payment. Such pragmatic and utilitarian points were not widely used to interpret the law in the legal tradition Castberg was part of. This, in particular, is why I think it is appropriate to use the label of constitutional absolutism to describe this kind of reasoning.

However, it is not correct to think that such reasoning is necessarily ``owner-friendly''. To see this, it is enough to note that Castberg, in the quote above, explicitly states that depreciation of value due to the scheme should not be disregarded. In addition, Castberg did not intend to reject the no-scheme principle altogether. In particular, he explicitly denied that owners of expropriated property should ever be able to claim compensation based on the special want of the acquiring party:\footcite[268]{castberg64b}

\begin{quote}
The situation is different if the property has increased value due to the expectation that it will be expropriated. The owner can not demand that this increase is compensated since that would be the same as giving him a special advantage compared to those from whom no property is expropriated.
\end{quote}

Hence, Castberg accepts a narrow version of the no-scheme principle, similar in spirit to that presented by Lord Romer in the {\it Indian} case. Castberg's view appears to have been shared by many academics of his day, and it was also largely reflected in case law from the Supreme Court.\footnote{See below.} At the same time, the very nature of the system for deciding appraisal disputes gave the local appraisers great freedom in adapting the principles in a way that suited the concrete circumstances.

To some extent, this would also involve making an assessment of what was regarded as a fair and just outcome. Hence, while the theory of the time was absolutist, case law was more multi-faceted. Importantly, fairness was seen as a concrete issue that had to be addressed on a case by case basis, an approach that would not necessarily lead to general rules. The Supreme Court largely sanctioned this approach, by passively respecting the discretion of the appraisal courts, as vested in them within an absolutist theoretical framework.

But as long as they did not cross the line with regards to the constitution, the appraisal courts were largely allowed to adopt broader viewpoints as well. The point was, however, that such viewpoints were \emph{not} extensively codified in terms of special principles used to deal with special case types or issues. Rather, they arose as a logical consequence of the way in which appraisal disputes were organized. Social justice and fairness perspectives were not excluded, but could in fact play an important role in practice.. However, such perspectives arose \emph{indirectly} through a \emph{decentralized} system which gave local courts great freedom when applying the law.

The way in which the no-scheme principle was applied serves as a nice illustration of this. On the one hand, the theoretical views of Castberg were widely accepted, but at the same time they were regarded as no more than guidelines that had to be adapted to the circumstances. Moreover, it was not unheard of for the appraisers to disagree with the judge about how this should be done, and to award compensation according to a different understanding of the law than that favoured by the judge. 

This happened, for instance, in the case of \emph{Tuddal}, where land was expropriated for construction of a power grid.\footcite{tuddal56} In relation to this, the expropriating party also acquired the right to use a private road. According to the juridical judge in the appraisal court of appeal, and consistent with the teaching of Castberg, compensation should be awarded solely on the basis of what the owners stood to lose. In this case, that would mean compensation based on the increased cost in maintaining the road resulting from increased use. However, the lay appraisers found this result unreasonable and awarded compensation also for the special value the use of the road would have for the acquiring party. The Supreme Court, although they found fault with the reasons given by the law appraisers, agreed that such compensation was possible in principle. The presiding judge offered the following perspective:\footcite[111]{tuddal56}

\begin{quote}
Since they were the private owners of the road, A/S Tuddal could, before the expropriation, refuse to let the Water Authorities make use of it. Hence it might be possible for A/S Tuddal, through negotiation and voluntary agreement with the Water Authorities or others with a similar interest, to demand a reasonable fee, and in this way achieve a greater total benefit than full compensation for damages and disadvantages. Following the expropriation, it is no longer possible for A/S Tuddal, in its dealings with the Water Authorities, to economically benefit from their ownership of the road in this way. If the company suffer an economic loss as a result of this, I believe they are entitled to compensation. Whether or not such an opportunity as I have mentioned -- all things considered -- was present at the time of the expropriation, falls to the appraisal court to decide, on the basis of whether or not an economic loss is suffered beyond that which follows from damages and disadvantages. On this basis, I assume that the appraisal court of appeal's decision to awarded compensation for the value of the right of way that is acquired can not -- in and of itself -- be regarded as an erroneous application of the law.
\end{quote}

The Supreme Court's reasoning illustrates two points. First, we see how the Supreme Court adopts absolutism in its interpretation of the law. Through careful use of wording, the compensation premium is not conceptualized as compensation based on the value of the road to the acquiring authority, but rather as compensation for the loss of potential profit following from a voluntary agreement. Hence, a seeming contradiction with the no-scheme principle is avoided.\footnote{This particular interpretation of full compensation led to arguments in the post-war period, regarding whether or not owners had a right to compensation based on the loss of profit from hypothetical voluntary agreements with the acquiring party. In the end, a consensus formed that this type of compensation should not in general be awarded. See \cite{nut69},Part 2, Chapter 4, Section 2.E.}

But {\it Tuddal} also illustrates a second important point, namely that the Supreme Court was prepared to defer greatly to the judgement of the appraisal court. It is stated explicitly that it falls for this court to decide whether or not the opportunity to profit from the road by negotiating with the expropriating party was present at the time of expropriation. This is particularly noteworthy in light of the dissent of the juridical judge in the appraisal court of appeal, and in light of the dominant legal theorizing of the day, which did  not seem to support the idea that a premium should ever be paid in a situation like this. Hence, the decision tells us that the Supreme Court went far in defending the discretion of the laypeople, as a \emph{systemic} feature. They tested with great caution whether it was truly outside the permissible legal boundary, but concluded that it should simply be regarded as an exercise of the lay judgement that the system presupposed.

This impression of the case is accentuated when considering other cases dealing with similar issues. Across the board, I note a strong  tendency to defend the role of the laypeople in the appraisal process. A particularly clear expression of this can be found in \emph{Marmor}, also from 1956, where the Supreme Court overturned a decision made by the appraisal court of appeal on the grounds that the court had been {\it too} reliant on general principles.\footnote{\cite{marmor56}.} This, the Supreme Court held, offend against both the principle of full compensation and the principle of discretionary evaluation by laymen.

The case involved expropriation of a private railway track, for the construction of a public railway. It was clear that the track which was being expropriated did not have market value in general, so the expropriating party argued that the value of these tracks to the public railway should not be taken into account when calculating compensation. The appraisal court of appeal agreed, pointing to the standard teaching of the day. The Supreme Court, on the other hand, struck down the decision because they felt that a standardized approach to the case was inappropriate given the circumstances. The presiding judge argued as follows:\footcite[498-499]{marmor56}

\begin{quote}
In my opinion one can not simply assume that a property does not have market value when it has no value for anyone other than the expropriating party. The question needs to be assessed concretely. I agree with the expropriating party -- as has also been confirmed on several occasions by the Supreme Court -- that in general one should not take into consideration the special value that the purpose of expropriation gives the property. This should not lead to a spike in compensation payments. On the other hand, I can not agree that it is automatically reasonable, or in keeping with Section 105 of the constitution, if the expropriating party in cases like the present one could acquire property at a price that is below what it would be natural and commercially appropriate to pay in a voluntary purchase.
\end{quote}

Again, I note the two main building blocks used in the argument: First, the standard reference to the constitution, and secondly, a reference to the need for \emph{concrete assessment}. This further reflects the strong confidence that the Supreme Court had in the integrity and autonomy the appraisal procedure. Moreover, I notice how, in this case, absolutism regarding the constitutional protection of property is \emph{not} used to argue for specific rules or principles, but rather to back up the argument that compensation should result from real assessment, and not be overly reliant on such rules. In the case of {\it Marmor}, this was the outcome even if the rules in question had the status of valid guidelines that had also been backed up by a series of Supreme Court decisions.

In addition to making the overreaching remarks quoted above, the Supreme Court also gave pointers as to the kinds of facts that should be considered. For instance, the presiding judge paid particular attention to the wider \emph{context} of expropriation, and the manner in which expropriation was used to benefit certain interests. He also noted how expropriation had come to replace voluntary agreement as the standard means of acquisition for this type of development. Therefore, the practice of using expropriation effectively prevented a market from developing, a market that might otherwise have appeared naturally:\footcite[499]{marmor56}

\begin{quote}
I also point to the fact that the case concerns an area of activity where the expropriating party has a {\it de facto} monopoly which makes it impossible for anyone else to make use of the property for the same purpose. This in itself makes it questionable to simply assume that the lack of financial value for other purchasers provides the appropriate basis for calculating compensation. When considering this question, it is also appropriate to take into account that we have lately seen a great increase in the use of expropriation to undertake projects such as this. Compulsion is becoming the primary mode for acquisition of property -- not voluntary sale following friendly negotiations.
\end{quote} 

In my opinion, the importance of this decision, which makes it highly relevant even today, is not that it seems to favour a narrow interpretation of the no-scheme principle. In fact, I think it is erroneous to read the judgement as expressing general support for any particular interpretation. In addition, I do not think the decision can be read as supporting a general principle by which compensation can always be based on the value of hypothetical agreements that could have been made with the expropriating party. Rather, I take the judgement to be an expression of scepticism towards blind obedience to \emph{any} set of detailed rules for calculating compensation that serve to limit the room for lay discretion.

At the very least, it seems clear upon closer inspection of the argument that the main objective of the Court was not to express any particular view regarding the content of the no-scheme principle, but rather to instil to the appraisal courts that they could not use this rule as an excuse not to engage in concrete assessment to ensure a reasonable outcome in keeping with the constitution.

I believe this point is important to stress. It illustrates how absolutism need not, and did not, result in a rigid system with little room for assessment based on justice and fairness, broadly conceived. Quite the contrary, the absolutism endorsed by the Supreme Court, and inherent in the Norwegian system of appraisal courts, was not characterized by blind obedience to specific rules, like those proposed by Castberg. Rather, the system was flexible, and it was explicitly intended to function such that fairness assessments based on concrete circumstances could be accommodated.\footnote{Going back to even older legal scholarship, we see that this view on the meaning of absolutism has a long history in Norway. It is present, for instance, in the work of the famous 19th century scholar Aschehough, who stressed the link between the constitution and the appraisal procedure when he considered the (then) hypothetical situation that legislation would be introduced with the specific aim of reducing the level of compensation payments following expropriation. See \cite[48]{aschehough93} 

%\begin{quote}
%If it becomes common practice to award compensation payments that are unreasonably high, this would make important public projects more expensive and difficult to carry out, greatly to the detriment of society. In many cases it might not be possible to rely on legislation to prevent such excessive compensation payments, since this would restrict the appraisers too much. To some extent this might be possible, however, and as far as it goes, parliament must be permitted to do so. However, if enacted rules clearly lead to less than full compensation in an individual case, they will be overruled by Section 105 of the constitution, and fall to be disregarded in that particular case.
%\end{quote}
%
%This quote is important because it does not rely on any particular interpretation of the constitutional demand for full compensation, but sees this inherently as an issue that needs to be resolved by concrete assessment of individual cases. Absolutism to Aschehough implies freedom and responsibility for the appraisers; freedom to judge individual cases by its merits, and a responsibility to award full compensation, irrespectively of any specific rules that might be in place to curtail excessive payments. The important point is that Aschehough here sees absolutism as a principle that should be applied to cases, not to principles. He does not argue that rules introduced to limit compensation payments would be inadmissible merely because they might sometimes suggest less than full compensation. Rather, he takes it for granted that it falls to the appraisal courts to apply the rules in a way that would prevent such outcomes. As long as the appraisal courts remain free to apply the rules in such a way that full compensation is awarded, specific rules intending to prevent excessive payments can happily coexist with absolutism.
%
%The subtle view taken by Aschehough was largely overlooked in debates following the introduction of the Compensation Act 1973, which served to introduce radical rules of exactly the kind he had predicted and considered 80 years earlier. The consequence was, as I will discuss in more depth in the next section, that the Supreme Court was forced to actively steer the interpretation of the Act to ensure that section 105 would not be violated in concrete cases. Hence, the introduction of legislation served to destabilize the system, by narrowing the room for lay judgements and increasing the reliance on legislation and special principles developed by the Supreme Court. This development is the subject of the next subsection.

%More generally, the 60s and 70s appears to be a period when the crucial role of the appraisal procedure was to some extent forgotten, and also undermined, following a heated political and ideological debate regarding the appropriateness and admissibility of introducing rules to ensure that compensation payments were brought down to a lower level. This had deep and lasting effects on Norwegian compensation law, and it is popularly described as a period when the social democrats won recognition for the principle that social fairness suggested the introduction of compensation rules and disregards that were more extensive than what had previously been considered appropriate. 
%
%This was conceived of as a fight for social justice against outdated and conservative ideas of constitutional absolutism. But it seems to us that this view of the history of Norwegian compensation law is erroneous, and largely unhelpful. The approach taken by Aschehough, in particular, placing emphasis on the important role played by the appraisers in achieving fairness and justice in concrete cases, does not appear to contradict social democratic goals at all. In fact, it seems that his approach might be better suited to serve such goals, and to accommodate a variety of different political opinions and ideas, than an approach which is based on attempting to flesh out in painstaking detail how the appraisal courts should go about achieving the balance between social fairness and owners' rights. We will return to this point later, but first we will take a closer look at the history of the radical Compensation Act 1973 and the censorship to which it was subjected by the Supreme Court, leading to the Compensation Act 1984, currently in place.

\subsection{Pragmatism}\label{sec:prag}

Following the Second World War, the social democratic \emph{Labour Party} gained a secure grip on political power in Norway. As a result, many reforms were carried out that would reshape Norwegian society. One of the most important reforms concerned the introduction of extensive planning law to ensure that land use was put under public control.\footnote{See generally \cite{thomassen97,kleven11}.} As a result of this, the period also saw expropriation being used more extensively to further public projects, such as the large scale construction of hydropower to ensure general supply of electricity.\footnote{See generally \cite{skjold06,thue06b}.} As a result of these changes, the opinion was soon voiced that there was a need for a more uniform approach to compensation, which collected some basic principles in a common body of written law. In addition, it was an explicitly stated political goal to bring compensation payments down.

In 1965, the so called \emph{Husaas committee} was appointed by the King and charged with the task of assessing the compensation rules currently in place.\footnote{Appointed by the King in Council on 6. Aug 1965.} The committee was also ordered to make a concrete suggestion regarding the need for additional principles of compensation, and to consider if these should be given in the form of a special compensation act. Initially there was some doubt as to the extent to which is was at all permissible to give rules regulating compensation, as the constitution itself addressed the matter. 

However, the committee noted that some rules had already been introduced for specific case types, for instance in relation to expropriation for hydropower development.\footnote{As discussed in Chapter \ref{chap:..}, Section \ref{sec:...} in relation to the \cite[16]{wra17}.} In addition, legal scholars of the day were generally of the opinion that compensation rules could be given, on the understanding that the courts would deviate from them in so far as they seemed to go against the Constitution. Hence, the Constitution was not understood to stand in the way of more specific rules.\footnote{\cite[136-137]{nut69}.} According to the minority of five, no such rules were actually needed, but the majority of ten disagreed.\footcite[137]{nut69} }

When considering the question of what kind of rules should be introduced, the Committee looked to case law as well as existing literature on compensation. They were faced with highly divergent opinions on the subject. Since WW2, in particular, a pragmatic view on property rights had taken hold, whereby an absolute right to property was increasingly felt to stand in the way of efforts to rebuild the country and ensure development following the great war. The Labour party had secured a firm grip on government at this point, so there was also an ideological shift taking place that emphasised the importance of building a welfare state over protecting the entitlements of individuals.

This was by no means a consensus view among legal scholars, however, and it was particularly contentious with regards to property. 

%As a result, some disagreed strongly with the very idea of legislation regarding compensation, and tensions arose that have led to much legal controversy and are still important in the law today.

%The majority pointed out that a vague general principle such as that provided by the constitution would by necessity have to be interpreted in order to be applied to concrete cases.\footnote[137]{nut69} Hence, it was not only permissible, but also desirable, for parliament to give more detailed instructions as to how is should be applied and understood by the courts and the appraisal courts. Leaving it to the judiciary to flesh out the exact meaning of full compensation through case law, it was felt, was not appropriate in a regulatory regime where expropriation had become increasingly important as a means to ensure modernization and development of critical infrastructure.

%In addition to this, the Supreme Court itself had recently expressed its support for a new view on regulation of property use, supported by contemporary legal scholars and politicians, whereby the State was regarded as having wide discretionary powers to determine how property should be used. This right to regulate, in particular, was increasingly coming to be seen as a right that did not infringe on property rights, so that the State would not have to compensate owners if they exercised it, except in special cases.\footnote{See, in particular, Rt. 1970 p. 67.}.

This problem area was mapped out in some detail by the Husaas committee, who traced the pragmatic view on compensation, identifying it using the following quote by the leading scholar Knoph from \cite[113]{knoph39}.

\begin{quote}
Since Section 105 is a rule prescribing practical justice, directed at parliament, and not an ethical postulate of absolute validity, it must be permitted to make technical legal considerations, so that one accepts compensation rules that lead to correct and just results on average, even if it does not grant the owner full individual justice in every case.
\end{quote}

Many disagreed vehemently with this perspective, based on absolutist principles.\footnote{See, e.g., \cite[20-22]{robberstad57};\cite[44]{schjodt47}.} The prominent legal scholar Schjødt, for instance, describes Knoph's reading of the law scathingly as follows:\footcite[44]{schjodt47}

\begin{quote}Luckily it has not had any effect on judicial practice whatsoever. No court of law would accept that compensation should be set according to a norm that may be practical and just in general, but does not grant the owner full compensation in all individual cases.
\end{quote}

By the late 1960s, however, Knoph's view was beginning to find favour among legal scholars.\footnote{See \cite[17]{fleischer68};\cite[41]{opshal68}.} When assessing the writings on the subject, the Husaas committee noted this tension. In response to it, they proposed a set of general principles for compensation which are still largely with us today. They were moving in a pragmatic direction, but rather cautiously. Hence, they refrained from encoding principles that would appear too offensive to the absolutists, even if the pervading political sentiment was that compensation rights had to be limited to ensure more effective state regulation of property use.

Importantly, the Husaas committee distilled from case law the principle that owners could only demand compensation based on the value of a specific use of the property when this use was ``foreseeable''.
The committee sought to codify this idea, which they saw as expressing an interpretation of Section 105 that was already largely entrenched in case law.\footcite[134]{nut69} This led to the following conclusion:\footnote{\cite[142]{nut69}.}

\begin{quote}
It is the view of the committee that it is correct to encode in the act the principle that the owner is entitled to compensation based on the value that results from taking into account the foreseeable and natural use of the property, given its location and the surrounding conditions. The exact meaning of ``natural and foreseeable'' use must be decided after a concrete assessment in individual cases. By encoding this general principle, however, it will become clear that compensation should not be based on private or public plans unless these plans coincide with the use of the property that is natural and foreseeable, independently of the scheme underlying expropriation.
\end{quote}

Importantly, I note how the committee actually does more than just encode a foreseeability constraint. They also state outright that this constraint is taken to imply the no-scheme principle, since they stipulate that the assessment of what counts as foreseeable and natural must be made independently of the scheme underlying expropriation. Since this statement is made quite generally, it also seems that the committee expresses a broader view on the no-scheme principle than that endorsed by Castberg.\footcite[268]{castberg64b} It is no longer only the special want of the expropriating party that should not be taken into account, the entire scheme ``underlying'' expropriation should be disregarded.

But in fact, this view was not in keeping with the political motivation for an act regarding compensation. It was too owner-friendly. Hence, the Ministry of Justice deviated from it in their final proposition to parliament. Instead of encoding existing principles, they sought a more aggressively pragmatic system whereby compensation would in general be based on the value of the \emph{current use} of the property.\footnote{\cite[19-20]{otprp59}.} In this way, the argument went, the public no longer had to pay a financial premium to owners based on possible future uses that would in any event, in most cases, be reliant on public development permissions.\footnote{\cite[17-20]{otprp59}.} 

%Such permissions, it was argued, could never be foreseeable in circumstances when it was in the public interest that the property should be expropriated, and hence all future development potential should in principle fall to be disregarded.

%The Ministry commented on this as follows:\footnote{\cite[19-20]{otprp59}.}
%
%\begin{quote}
%The Ministry is of the opinion that it is particularly important to arrive at a rule that can bring the assessment of property value down to a realistic level, and believes that the natural starting point for such an assessment must be the current use of the property, especially for expropriation of real property. As mentioned, it is the opinion of the Ministry that a practice has developed that gives too much weight to more or less uncertain future possibilities for the property, something that has led to a sharp rise in compensation payments.
%\end{quote}

After intense debate in parliament, where the minority center-right parties all opposed its introduction, the current use rule was eventually encoded in section 4, no 1 of the \cite{ca73}.\footnote{Act No 4 of 26 March 1973 Regarding Compensation following Expropriation of Real Property.} This was largely seen as a social democratic victory and a clear indication that the pragmatic approach to property protection was gaining ground. When clarifying their principled starting point regarding what should count as \emph{realistic}, the Ministry made the following assertion regarding the scope of the constitutional protection offered in Section 105, showing the ideological underpinnings of the new Act:\footcite[17]{otprp70}

\begin{quote}
However, a right to complete -- or almost complete -- equality can not be derived from the constitution. It must be taken into account that we are here discussing equality with regards to increases in property value that are, in themselves, undeserved. [...]  %  The starting point must be that it is not, in and of itself, contrary to the constitution that one property owner do not benefit from the same increase in value as another, when the increase in value, for both of them, is due to public investment and does not stem from their own efforts. \\ \\
Certainly, it would be best to avoid any kind of inequality, if it was possible. But the examples we have considered illustrate that, today, inequality between property owners is tolerated with regards to public investments and regulation, and that, moreover, practical and economic considerations dictate that we \emph{should} make use of differential treatment in this regard.
\end{quote}

This echoes Knoph, but also goes much further. In particular, the Ministry explicitly states that differential treatment is appropriate in the context of expropriation, and, by implication, that this should be done precisely to avoid compensation payments that include compensation for ``undeserved'' increases in value. Also, in proposing that compensation payments should be based on current use, the scope of ``undeserved value'' is made very wide. In principle it would seem to include \emph{any} value that could be attributed to an as of yet unrealized potential that the property in question might have. The question of whether or not this value was reflected in the market value of the property, in particular, was not regarded as relevant. This was in itself radical, since market value based on the likely use of an ``average buyer'' had previously been the dominant starting point for appraissal courts when awarding compensation.\footcite[112-113]{nut69}

The conceptual significance of this change in perspective should not be underestimated. Here the Ministry stood firmly behind a pragmatic view. Perceived social fairness was the overriding constraint, also with respect to constitutional property protection. However, on taking this view to its logical conclusion, it was recognized that any general compensation rules that might be introduced should themselves be subject to a fairness test, so that, for instance, the current use principle could not itself be absolute or without exception. 

Rather, it could only be applied in so far as it served the overreaching goal of social justice and fairness which was regarded as the fundamental component of property protection that made such a rule possible. This, in particular, seems like a crucial observation, and one that has in my opinion been overlooked, with unfortunate consequence for the subsequent debate and development of the law. Indeed, it echoes the sentiment behind the age-old procedural arrangement that placed high value on the free discretion of the appraisal courts. Hence, it points to the possibility of finding some \emph{common ground} between absolutist and pragmatist views on compensation.

Sensible voices from both camps seemed to arrive at the conclusion that in the end, there was no way around a concrete and contextual assessment, where social fairness values are (hopefully) used as a guide. In an attempt to translate aspects of such a perspective into legislation, the Ministry set out two exceptions to the current use rule. The first, which received by far the most attention, was based on the notion of equality between owners in same local area.\footcite[19]{otprp70} It stipulated that the appraisal courts should be free to deviate from the current use rule in so far as it felt that it was reasonable to do so in order to ensure a reasonable degree of equality between neighbouring owners.\footnote{This principle was eventually encoded in section 5, no 1-3 of the \cite{ca73}. It would prove highly controversial, since it was only formulated as a rules that ``could'' be used to increase the compensation. In \emph{Kløfta}, the Supreme Court eventually deviated from this and overruled the Act by making clear that additional compensation was \emph{obligatory} in a range of cases when the intention had clearly been that the rule should be used sparingly. In this way, and possibly inadvertently, the Supreme Court ended up defending owners' interest by \emph{limiting} the power of the appraisal courts.}

However, the Ministry also noted the need for a second exception, which is in my opinion far more important and interesting. This exception pointed to the need to ensure equality between the taker and the owner, in so far as the taker could not be regarded as the embodiment of purely public values.

%
%\begin{quote}
%One is aware that the principle of current use compensation cannot be without exception. Even though this rule will be fair in general it can, in some cases, disproportionately disadvantage property owners. One has therefore suggested rules that modify the principle to some extent. These are given for somewhat different reasons. \\ \\
%
%One case addresses the situation when current use compensation means that a property owner will be significantly worse off that other owners of similar property in the same district, according to how these properties are normally used. In these cases, the principle of equality suggest that the owner receives some -- but not necessarily full -- compensation for the inequality that would otherwise arise from the fact that his property was made subject to expropriation. %Etter departementets oppfatning har en ekspropriat etter grunnloven ikke noe krav på å bli satt helt i samme stilling som om ekspropriasjonen ikke var skjedd, en forskjellbehandling innen rimelige grenser må grunnloven tillate når dette tilsies av tungtveiende samfunnsmessige grunner. 
%\end{quote}

Importantly, the rule sought to address precisely the situation that arises when the taker benefits commercially from the expropriation. Moreover, it addressed the question of the \emph{power balance} between the expropriating party and the owner. In the words of the Ministry:\footcite[19]{otprp70}

\begin{quote}
The second modification we make has to do with the relationship between the property owner and the expropriating party. If the use of the property that the expropriation presupposes gives the property a value that is significantly higher than the value suggested by current use, this will entail a transfer of value from the property owner to the acquiring party. In some cases this might be unreasonable. As an example of when this can become an issue, we mention an agricultural property that is expropriation for the purposes of industrial production. In such a case it might be natural that the owner receives a certain share in the increased value that the new use of the property will lead to.[...] %This would be different than, say, a situation where an agricultural property is expropriated for constructing a road or for setting up recreational outdoor grounds. In such cases, the expropriation will not lead to any such economically advantageous use of the property that will give the expropriating party an economic advantage. 

To establish a flexible system, the Ministry has concluded that it is practical that the King gives rules concerning the cases where an enhanced compensation payment, based on these principles, might be appropriate. This should not be decided by individual assessment, but governed by rules for special case types. Hence, the proposed Act states that the King can pass regulation concerning this matter.
\end{quote}

This quote goes right to the heart of one of the main problems of economic development takings, and proposes a possible remedy. However, the Ministry took the view that this remedy should {\it not} be administered by the appraisal courts, but should be left in the hands of the executive. Already here I note a reason worry whether this could then ever become an ineffective way of achieving fairness in practice. Indeed, the fact that this aspect of the 1973 Act has been largely overlooked and forgotten seems to prove my point. No rules such as those proposed by the Ministry as a possibility have in fact been introduced, the entire profit still goes to the taker in cases when commercial schemes benefit from expropriation.

%More broadly, it is hard to disagree that the context of expropriation must by necessity come to play a crucial role for any approach based on compensating the ``deserved'' value. What this value should be taken to be, in particular, can hardly be determined once and for all and in general terms, but must rather be subject to continuous revision depending on how expropriation is \emph{actually used} in society. This includes looking to the purpose it is meant to serve, the parties who stand to benefit, and the groups who tend to loose their land.

%Indeed, stipulating that compensation should be ``deserved'' appears to provide a benchmark that is just as unclear as the stipulation that compensation should be ``full''. It seems, in particular, that the inherent ambiguity of these terms allows us to draw two conclusions: first, that they might very well have the same meaning, and second, that they cannot possibly be defined once 
%and for all by any act of parliament, or by any decision in the Supreme Court.
%
%But this suggest, against the Ministry and the overall spirit of the 1973 Act, that the system of appraisal courts has an important role to play in ensuring fairness in individual cases. It is hard to see how the objective of social fairness and justice for the individual can be reached without making heavy use of appraisers with discretionary competence. 

The procedural and contextual aspect of fairness seems to have been overlooked by those pushing for the 1973 Act. Since the appraisal courts were regarded as compensating owners too generously, their freedom of discretion was seen as a problem rather than as a path towards a solution. I think this regrettable. If the new Act had been slightly more temperate in its approach, by encouraging the appraisal courts to take a broader view on fairness, rather than to force them to adopt current use value as a baseline, it might have been a success. Instead, it caused an outcry, with attention shifting away from practical matters towards doctrinal issues. The primary such issue, and the most serious one, concerned the question of whether the Act as such was in breach of the constitution. This was eventually considered by the Supreme Court in the case of \emph{Kløfta} in 1976.\footnote{\cite{klofta76}.}. 

Following this decision, the 1973 Act would be significantly reinterpreted to make it appear less offensive to the constitutional standard of full compensation. However, it seems to me that the Supreme Court largely accepted that the intention behind the Act should be respected and that appraisal practice needed to be adjusted accordingly. In this, the Supreme Court signalled loyalty to the political system and the democratic process. However, in implementing this adjustment in practice, they also, possibly inadvertently, set up a system where the role of the local appraisal courts were undermined even further.

Not only were they constrained by an Act that seemed to run counter to the Constitution, they were now also ordered from above to openly deviate from its exact wording, but only for a select group of cases meeting certain pre-defined criteria. In essence, the Supreme Court itself assumed greater control over how compensation law was to be applied, no longer merely in broad strokes, but increasingly also by developing special rules for specific case types.\footnote{The clearest indication of this shift is found in recent case law wherein the Supreme Court has provided a myriad of detailed rules and directions regarding how appraisal courts should decide on the thorny issue of whether to consider public plans binding for the compensation award or to disregard them under a no-scheme rule. See generally \cite[7-9]{nou03}.} In the following section, I describe this in more detail.

\subsection{The top-down approach}\label{sec:regab}

Following the decision in {\it Kløfta}, the \cite{84} was introduced. It reverted back to the ``foreseeability'' test proposed by the Husaas committee. In section 4, it is stated that financial compensation (as opposed to compensation in kind) is to be based on either value of use or value of sale, whichever is highest.\footcite[4]{ca84} Sections 5 and 6 describes in more detail how the calculation should be carried out.

In both regards, the principal requirement is that the value is calculated based on a use of the property that is foreseeable and natural given the surrounding conditions. In relation to the value of sale, there is an additional requirement, namely that the use must be one that an ``average'' buyer would be likely to make of the property. Hence, the value of sale should be set as a general market value, not a value arising from selling the property to a specially interested party.

The extent to which the foreseeability requirement entails that the use in question has to be in accordance with public plans currently in place has been disputed. In general, compensation is only based either on uses permitted by public plans currently in place or uses that seem likely to be permitted in the future. In Norwegian law, whether a use is foreseeable is an ``either/or'' question. 

No compensation is given to reflect the so-called ``hope'' value, namely the part of a property's value that depends on the perceived likelihood of a change in planning status and future possibilities.\footnote{By contrast, compensation tends to include such an element in the UK.} If a permission for future use is deemed likely, it is subsequently regarded as a certainty for purposes of compensation, although the present-day value of a future possibility is usually calculated in a way that takes interest and inflation into account.\footnote{These calculations tend to be notoriously schematic, however, quite far removed from the realities of the financial system.} Similarly, if a future possibility is deemed unlikely, no compensation is paid for it whatsoever.

In effect, the best an owner can hope for in Norway is that the likelihood of having received more than full compensation is greater than the likelihood of having received less.

Tensions and disputes tend to arise either directly in relation to the foreseeability test, or else in relation to one of the disregard rules that encode aspects of the no-scheme principle. The disregard rules included in statute are all formulated in relation to the value of sale, although they are also regarded as applying to value of use assessments. To some extent, they may also be redundant, in so far as they already follow from the foreseeability test.\footnote{I recall that the Husaas committee itself thought that a rather wide no-scheme principle would follow already from the foreseeability test. See above.}

The main disregard rule included explicitly in the \cite{ca84} is formulated very similarly to the no-scheme rule in the UK. It states that one should not take into account changes in value that can be attributed to the ``expropriation measure''.\footcite[5]{ca84} Interestingly, the notion of an ``expropriation measure'' is defined in the Act. In section 2, the expropriation measure is said to be the ``activity, installation, or purpose'' benefiting from expropriation. Hence, while there is a definition, it is (as expected) very vague. 

If the definition had only included the second item -- that of an ``installation'' -- it would amount to a meaningful restriction. However, as it stands, an ``expropriation measure'' seems like it could include just about anything that stands in some kind of relationship to the expropriation order.
The purpose of the expropriation is included in the list, which, if taken literally, would lead to rather absurd results. 

For instance, if houses next to a small public road are expropriated for the construction of a motorway, the wording suggests that even the presence of the public road must be disregarded when assessing their value. This would seem to follow, in particular, in so far as the pubic road was built in pursuance of the same purpose as the motorway now being constructed. Luckily, the rule is not understood in this way in practice. 

However, a second rule expressed in section 5 of the \cite{ca84} states that changes in value due to other investments that the expropriating party has carried out, or plans to carry out, also falls to be disregarded. The condition is that they have {\it either} been carried out in relation to the expropriation measure {\it or} during the last 10 years.\footnote{See the third and fourth paragraph of \cite[5]{ca84}.} Hence, the disregard rule in section 5 is stronger than most no-scheme rules, in that previous or planned investments must sometimes be disregarded even if they stand in no relation between the expropriation scheme besides being carried out by the same party. In so far as the expropriating party is a public body, even this is relaxed, since all investments carried out by {\it any} public body is then to be disregarded, limited only by the 10 years rule.\footcite[5]{ca84} 

%When the constitutionality of the Compensation Act 1973 came before the Supreme Court in \emph{Kløfta}, they chose to sit as a grand chamber and they reached a decision under dissent, being divided into two fractions, consisting of 9 and 8 supreme judges respectively. However, both fractions approached the problem of constitutionality by endorsing an interpretation of Section 5 nr. 1 in the Compensation Act 1973 that gave the exception to the current use much wider scope than what had been intended by parliament. The majority went farthest, and unlike the minority they also regarded the compensation payment in the concrete case to be insufficient. The first voter for the majority commented as follows on the constitutional aspect of the case.\footcite[7-8]{klofta76}
%
%\begin{quote}
%[...] But the main question in this case, is whether or not it is in keeping with Section 105 to generally award compensation at a level below the market value that could legally be estimated, and that the owner could actually have achieved, if expropriation had not taken place. In my view, this involves allowing expropriation to transfer a right that the owner had, with a value to which he was entitled. If he is refused compensation for this value, he would, depending on the circumstances, be left significantly worse off than others in a similar position, who owns property that is not expropriated. Such a result I cannot accept. It would be a breach of established customary law and a practice that has been established throughout the years both by the appraisal courts and the Supreme Court. I refer particularly to Rt 1951 s. 87 (particularly p. 89, Opdahl). This practice is in itself a significant contribution to interpreting Section 105 on this point.
%\end{quote}
%
%I note the emphasis placed on \emph{market value} in the majority's reasoning. This may appear to be in keeping with an absolutist doctrine, but as I have mentioned, it can have unfortunate, possibly unintended, consequences for property owners, especially when combined with a restrictive view on what counts as foreseeable future development. I note, however, a technical point that might be of some significance for the interpretation of \emph{Klofta}: Instead of stating outright that a market value rule follows from the wording of the Constitution as such, the majority takes the view that this interpretation suggests itself based on the compensation practice that had currently been established. This might limit the scope of the majority's remarks in this regard, but it also serves to give further support to the claim that the role of the appraisal courts, and their assessments, still had a strong position in Norwegian compensation law at the time of \emph{Kløfta}. 
%
%I remark that the minority disagreed on the constitutional status of the market value rule. Indeed, it was in this regard that the difference of opinion between the minority and the majority was most clearly felt. The minority, in particular, explicitly rejected the view that this rule could be derived from the constitution itself, and they also disagreed with the understanding that it would have status as a constitutional rule simply because it had been adopted in practice. This bestowed merely the status of ordinary legal precedent. As expressed by the minority:\footcite[23-24]{klofta76}
%
%\begin{quote}
%Case law from this area cannot be understood as preventing parliament from changing the rules in accordance with what they regard as necessary. That would prevent a reasonable and natural development and would not be in keeping with the consensus view that Section 105 of the constitution is a rule that must be interpreted in light of, and adapted to, how society has developed and how the law is viewed. I believe the practice that have evolved cannot be decisive if a new situation and new needs require a different solution. Whether the Compensation Act is in breach of the right to full compensation enshrined in the constitution, must depend on an interpretation of the wording in the constitution itself.[...] \\ \\
%In my opinion, neither the intentions of parliament nor the way they are sought implemented through Sections 4 and 5 are in breach of the equality principle upon which Section 105 of the constitution is based. It does not follow from the constitution that an owner is in all circumstances -- and irrespectively of the economic forces from which the market value results -- entitled to compensation that is at least as great as the greatest legal value that the property could represent on a free market. A different matter is that Section 105 of the constitution could be important to the interpretation and application of the rules.
%\end{quote} 
%
%Hence, the market value rule was explicitly renounced as a constitutional principle by the minority, who nevertheless conceded that the constitution could be used to interpret Sections 4 and 5 of the Compensation Act 1973. Both the minority and the majority agreed, however, that  it would be wrong to go on to consider Section 4 of the Compensation Act 1973 in isolation. For the majority, this would clearly have led to the Compensation Act 1973 being held to be in breach of the constitution, something that was avoided since the Supreme Court chose to consider the law as a whole, with the majority using the reasoning detailed above to argue for a new interpretation of Section 5, rather than as a means to undermine Section 4. Still, their interpretation of Section 5 went well beyond what it seemed that parliament had intended, leading some scholars to claim that \emph{Kløfta} should be read as holding that the Compensation Act 1973 was unconstitutional.\footcite[477]{andenes86} In the words of the majority:\footcite[12-13]{klofta76}
%
%\begin{quote}
%The purpose of this rule is to award compensation beyond current use in cases where valuations according to section 4 could be in breach with section 105 of the Constitution. As it stands, section 5 no 1 is not sufficiently suited for this purpose. By its wording it gives the appraisal courts an opportunity to assess whether or not it is reasonable to award additional compensation, even when the conditions for this is otherwise met, and even then with the limitation that the compensation would otherwise be significantly unreasonable. Such a free position for the individual appraisal courts -- without possibility of legal appeal -- would not be in keeping with the purpose of the rule and the demand for full compensation set out in the Constitution.
%\end{quote}
%
%On this basis, the Supreme Court chose to interpret section 5 no 1 in such a way that whenever the conditions were fulfilled, the appraisal courts were \emph{obliged} to award additional compensation, On this basis they found that the property owners in \emph{Kløfta} was entitled to have their compensation looked at again, in a new round before the appraisal courts. The minority agreed in principle, yet did not go as far as the majority, concluding that based on the particular facts at hand section 5 had been adequately considered by the appraisal court in this particular case.\footcite[22]{klofta76} In addition, the majority went quite far in suggesting that ``full compensation'' entitled the owner to {\it market value} compensation, whenever this would result in a higher award than a ``value of use'' approach.\footcite[14]{klofta76} Moreover, they adopted a more narrow interpretation of the (negative) no-scheme rule, whereby public plans not closely related to the expropriation project should not be disregarded.\footcite[15-16]{klofta76} In these matters, the minority took a different view, arguing against market value as a general benchmark and in favour of a broader no-scheme rule.\footnote{\cite[22-23|30-31]{klofta16}.}
%
%The upshot of \emph{Kløfta} was that section 5 no 1 came to be seen as an obligatory rule, leading to compensation having to be enhanced whenever the current use rule led to payments that did not reflect the market value of comparable properties. However, the conditions stated in section 5 no 2 and no 3 were still regarded as relevant, and in interpreting these conditions, a body of law developed whereby the market value rule was applied in a way that would come to involve significant reduction in compensation compared to what would result from practice as it had been prior to the Compensation Act 1973. In this way, the pragmatic approach proved triumphant, not because current use value was introduced as the general starting point, on the contrary, but because a range of new disregards were introduced to reduce the level of compensation in a range of different circumstances. After \emph{Kløfta}, in particular, the following rules were all considered legitimate ways to decrease the level of compensation.

%In section 5 no 3 and no 4, the Expropriation Compensation Act 1973 encoded the following three disregard principles that are all, to varying degrees, still important in compensation law today. 
%
%\begin{enumerate}
%\item Changes in value that are due to the expropriation scheme should be disregarded, both when these are already carried out as well as when they are planned, c.f., section 5 no 2 of the \cite{ca73}.
%\item To the extent that it is regarded reasonable, \emph{increases} in value that are due to public plans or investments should be disregarded, irrespectively of whether or not they have already been carried out, c.f., section 5 no 2 of the \cite{ca73}.
%\item An increased value falls to be disregarded if it results from considering a use of the property which is not in accordance with public plans, c.f., section 5 no 3 of the \cite{ca73}.
%\end{enumerate}
%
%While the \cite{ca73} has now been replaced by the \cite{ca84}, the formulation given in the 

These rules severely limits the level of compensation payments, and in many cases it appears to make the principle of full compensation based on market value rather illusory, even if this was the principle endorsed by the Supreme Court in {\it Kløfta}. On the one hand, the foreseeability test can serve to rule of value arising from any use of the property that is not in keeping with the current public plan. At the same time, the no-scheme rule explicitly encoded in section 5 can be used to also disregard values that are due to this plan, particularly if they are regarded as standing in some relation to the expropriation measure. 

The outcome could easily become, logically speaking, that no compensation can be awarded whatsoever. However, the system tends to revert back to the current use compensation in such cases. For instance, if agricultural land is expropriated for the purpose of a motorway, and it would otherwise appear foreseeable that it might be used for housing in the future, the compensation will usually be based on agricultural use because the value for housing is disregarded under a foreseeability test while possible increases in value due to the motorway plan itself is disregarded under the no-scheme rule.

In practice, with virtually all novel economic activity making use of land is dependent on acquiring new planning permissions, the current use rule will typically be applied as intended by the \cite{ca73}. The main difference is that the rule is not thought of, or described, as an absolute. It rather tends to arise merely as a side effect of other rules.\footnote{A similar point was made in \cite{stordrange94}.} Outcomes that are in keeping with current use thinking will typically be designated as ``full compensation based on market value'' -- the standard phrase adopted in most appraisal judgements -- notwithstanding the fact that the accuracy of such a description depends on the disregards that have been applied.

%The \cite{ca84} was eventually introduced to reflect the principles laid down in \emph{Kløfta}, but it did not in any essentially way change or influence the course of the law that had already been set. Its main purpose was to bring the wording of the legislation more into keeping with how the law was interpreted by the Supreme Court. It explicitly returned to the starting point of the Husaas committee, namely that the compensation should be based on the value of the "foreseeable use" that the owner himself, or an average buyer, might make of the property. But it maintained and endorsed disregard rules no 1-3, except for restricting disregard no 2 to public investments, such that increased value due to public plans currently in place could not be disregarded.\footnote{In this way, the paradox mentioned above, that compensation could become impossible to award because there was no possible basis upon which to calculate it, was avoided.}

The statutory rules do not provide clear guidance as to how the disregard rules should be understood or applied, nor do they consider or resolve the question of when, if ever, they would need to be applied with caution in order not to go against the constitution. However, following {\it Kløfta}, there has been a growing expectation that cases where such issues arise should be resolved by crisp rules, not by the discretion of the appraisal courts. The age when the appraisal courts were considered free to assess cases directly against the Constitution is gone. Rather, an ethos had taken hold where the need to curb the freedom of appraisers, in the interest of ensuring predictability and centralized control, is emphasized.

As a result, difficult cases now routinely end up in the Supreme Court. Here, difficult circumstances are used as the basis for formulating more and more specific rules for special case types. As an example of this mechanism, it is enlightening to consider the case law surrounding the question of whether public plans currently in place are binding when calculating compensation. This rule cannot apply without exception, as recognized already by the \cite{ca73}. But when is it permissible to deviate from it?

The question has arisen in many Supreme Court cases following {\it Kløfta}. \emph{Østensjø} concerned land that was being expropriated for housing purposes, but such that one unlucky owner would only contribute land used for infrastructure that would serve the larger housing project.\footnote{\cite{ostensjo77}.} In this case, the Supreme Court agreed that he was entitled to compensation based on value of his land for housing purposes, irrespectively of the fact that \emph{his} land could not be used in this way according to the plan. However, in many other cases, the disregard rule is upheld even when it is hard to see it as either fair or just, simply on account of it having status as a general rule.\footnote{For instance in \cite{malvik93}. In this case, owners of property used for a motorway were only entitled to compensation based on current agricultural use because the planned motorway-use was assumed binding for the compensation assessment under the market value approach.}

One example is found in \emph{Sea Farm} which dealt with the issue of whether or not the owner of a commercial property should be awarded compensation for the value of investments carried out by the previous tenant.\footcite{seafarm08} There was no doubt that the owner was entitled to these investments, but since the acquiring authority was the only purchaser who was likely to benefit commercially from them, no compensation was awarded for the loss of these investments. This, in particular, followed from a strict reading of the requirement that compensation should be based on the foreseeable use that an "average" buyer could make of the property, encoded in Section 5 of the Compensation Act 1984. Adherence to the wording used in the act seems to have taken priority over an assessment based on the facts of the case. It seems difficult to argue that it would be either unjust or unreasonable, in particular, to compensate the owner for investments that would prove commercially valuable to the acquiring party.\footnote{The decision was sharply criticized by a former supreme judge. See \cite{skoghoy08}.}

In my opinion, this example illustrates how the development of compensation law towards greater reliance on specific rules rather than concrete assessment based on general principles can be harmful. I also threatens to undermine the idea behind the special procedure used to decide appraisal disputes, which has a long history in Norwegian law.\footnote{One might ask if it has status of constitutional customary law, especially since it concerns the mechanism by which a constitutional rule is meant to be upheld.} It also seems to severely underestimate the extent to which compensation rules, when applied to concrete cases, must and should be interpreted based on the context of the case. It seems difficult, if not completely impossible, to achieve social fairness and individual justice by a set of specific rules on the basis of which all legal issues can be resolved mechanically by blind application of such rules. %Moreover, it would be wrong to think that Section ... of the Appraisal Act 1917, encoding the principle that laymen should take part in the decision-making both with regards to legal and technical matters that arose in appraisal disputes.

In the following section, I will turn to waterfalls and hydropower. Interestingly, the compensation practices developed in this regard often deviate significantly from the general approach to compensation. The special approach developed, in particular, as a result of the perceived unfairness of denying benefit sharing altogether in such cases. Hence, looking to waterfalls serves to underscore my point about the importance of a flexible system. 

%The main benefit sharing principle that was developed was known as the {\it natural horsepower method} for calculating compensation for waterfalls following expropriation. It was initially developed by the appraisal courts as an ad hoc approach to ensuring some benefit sharing in hydropower cases. Later, however, many came to regard it as a binding principle of customary law. Gradually, it came to be applied by the courts with little or no regard for how well it suited the circumstances of the case and the changing realities of the hydropower sector. As a result, the method became hopelessly outdated, leading to compensation payments that had little or nothing to do with the actual value of waterfalls for hydropower. 

%Today, while the method has been abandoned for certain case types, it is still applied as the default rule for compensation waterfalls.
%
%
% address this issue in more detail, and we will argue for a different conceptual approach to compensation law, grounded both in the procedural tradition of appraisal courts and the more subtle parts of the absolutist and pragmatic theoretical traditions. It seems to me that the most striking lesson that should be drawn from considering the history of Norwegian compensation law is that a \emph{contextual} view of compensation has been a common denominator that both the absolutist and pragmatist camps have endorsed. Unfortunately, this common element was overshadowed by political conflict regarding the weighing of different values. However, there can be little doubt that social fairness and individual justice should \emph{both} to be regarded as important objectives for compensation rules. Moreover, while they may sometimes be opposing, they need not be, and their exact relationship depends largely on the circumstances. It seems to us that it is simply inappropriate to let particular political sentiments regarding their relationship and relative importance, sentiments that are usually dependent on the particulars of the prevailing political, social and economic conditions, dictate the development of the legal framework for resolving compensation disputes.
%
%Considering current trends and recent issues in expropriation law, particularly related to commercial expropriation, further suggests that a different perspective is needed on this matter. In particular, we believe it is time to recall the idea of the independent and impartial discretion of the appraisal court, relying on the good common sense of laymen as well as the legal expertise of judges. The appraisal courts should in our opinion be set with the task of more actively evaluating how fairness and justice is best served in individual cases, at least if the overall goal is truly to arrive at a socially fair and individually just compensation system. We discuss this idea in more detail in the final section below.

\section{``Natural horsepowers''}

Following the introduction of a general expropriation authority covering waterfalls in the early 20th century, the question of how to value waterfalls came before the appraisal courts. The regulatory regime that was established made private commercial development difficult or impossible, and this in turn meant that the commercial market for waterfalls all but disappeared. Hence, a strict application of the no-scheme rule could lead to no compensation being paid at all. Arguably, a waterfall had no value to anyone except the acquiring authority, since no alternative development scheme could be regarded as foreseeable.

The appraisal courts did not follow this point of view to its logical conclusion. Instead, they introduced a theoretical formula for calculating waterfall compensation. In effect, this method served to create an artificial market for waterfalls, controlled by the appraisal courts. Initially, this artificial market was modelled on the actual market that had existed prior to the regulatory reform. Over time, however, the waterfall ``market'' would slide further and further into the legal sphere, away from the physical and commercial reality of hydropower development.

The key notion used to determine the price of a waterfall on this market was that of a {\it natural horsepower}, a gross measure of electric effect.\footnote{A horsepower, of course, is an old-fashioned unit of effect which is still sometimes used, e.g., in relation to cars. In the context of electricity, it is replaced by {\it Watts}, such that 1 horespower (hp) = 745.69 Watts.} As I mentioned in Chapter \ref{chap:4}, the lack of a national grid at this time meant that the value of a hydropower plant was largely determined by the stable effect that the plant could deliver, not the total amount of electricity that could be produced. This, in turn, was a function of the degree of water regulation implemented by the hydropower developer. 

To simplify the calculation, the natural horsepower of a waterfall was introduced as a gross estimate of the stable effect that could be ensured given a choice regarding the level of regulation of the watercourse. The value of the waterfall itself was then determined by fixing a price per natural horsepower. This price was set on the basis of prices paid for other waterfalls, with some adjustments typically carried to take into account the level of cost and benefits associated with the hydropower project in question.

As I remarked in Chapter \ref{chap:4}, the notion of a natural horsepower is used in other contexts as well, for instance to determine what kind of licenses a development project requires. The use made of it to calculate compensation for waterfalls had no legislative basis, but arose as a result of the appraisal courts' efforts to calculate market prices. After the actual market based on the natural horsepower method disappeared, the method stuck and was applied as a matter of custom.\footnote{See generally the description of the history of the method given by the Supreme Court in \cite{uleberg08}.}
%
%
%prove shockingly unfair to owners of waterfalls. Presumably, since waterfalls could not be exploited for any significant commercial gain except through hydro-power exploitation, disregarding the hydro-power scheme when calculating compensation could lead to nil or close to nil being awarded to the owner. But this was not seen as an acceptable outcome, and instead the Norwegian courts introduced a special method to compensate waterfalls that gave the owner a \emph{share in the value of the hydro-power scheme} for which expropriation was taking place.
%
%Norway did not at this time have any legislation specifically aimed at regulating compensation following expropriation, and when formulating the special rules for compensation of waterfalls, the Norwegian courts seems to have relied on an analogical application of the gross valuation techniques introduced in the Industrial Concession Act 1917 and the Watercourse Regulation Act 1917.\footnote{Act No. 17 of 14 December 1917 relating to Regulations of Watercourses and Act No. 16 of 14 December 1917 relating to Acquisition of Waterfalls, Mines and other Real Property}. Neither of these acts were aimed at compensating owners, but they relied on methods for assessing the potential and significance of hydro-power projects with respect to the question of whether or not a special concession from the State was required.\footnote{To acquire the waterfall and the right to regulate the water-flow respectively.} In effect, by relying on the methods of valuation introduced there, the compensation mechanism that was introduced deviated completely from the "value to the owner" principle. On the other hand, it also closely mimicked the manner in which owners of waterfalls would be compensated on the market in the early days, prior to the introduction of our concession laws, when speculators would pay for waterfalls on the basis of what they assumed to get out of them in subsequent hydro-power projects.

In the Supreme Court case of \emph{Hellandsfoss}, some 80 years after it was first introduced, the natural horsepower method was described and put into context as follows:\footcite[1599]{hellandsfoss97}
\begin{quote}
The principle set out in the Compensation Act, Section 5, is that compensation should be determined on the basis of an estimation of what ordinary buyers would pay for the property in a voluntary sale, taking into account such use of the property as could reasonably be anticipated. For waterfalls, however, this often offers little guidance, and the value of waterfall rights have traditionally been determined based on the number of natural horsepowers in the fall, which are then multiplied by a price per unit. The unit price is determined after an overall assessment of the waterfall, including the cost of the scheme, its location, and levels of compensation paid for similar types of waterfalls in the past. The number of natural horsepowers is calculated by the formula ``natural horsepower = $13.33 \ \times \ Qreg \ \times \ Hbr$'', where $Qreg$ is the regulated water flow and $Hbr$ is the height of the waterfall.
\end{quote}

In this formula, $Qreg$ represents a quantity of water, measured in cubic meters per second (m3/sec), while $Hbr$ represents height measured in meters. The number $13.33$ is the force of the gravitational pull on earth measured in horsepower. 

In the standard account of the natural horsepower method, it is often said that the number of natural horsepower in a waterfall is a measure of gross effect, giving us the amount of ``raw'' power in the waterfall.\footnote{See \cite{vislie02}.} This is not accurate. Indeed, from the quote given above it is clear that the natural horsepower does {\it not} depend only on the nature of the waterfall. It also depends on the specific plans for development presented by the expropriating party. In particular, the quantity $Qreg$ is entirely a function of how the developer {\it chooses} to develop the waterfall, in that it measures the ``regulated water flow''.\footnote{In addition, the quantity $Hbr$ depends on the height over which the developer plans to make use of the water. The development potential that the owner is deprived of can amount to either more or less than this, depending on the nature of alternative schemes.}

In {\it Hellandsfoss}, the Supreme Court itself glosses over this point when it speaks of the ``natural horsepower in the fall''. It would be more accurate to speak of the natural horespower of the particular development scheme benefiting from expropriation.\footnote{Regulation of a watercourse can involve building a reservoir and/or installations that transfer water from one river to another. Then, if there is excess water, for instance due to flooding, water can be stored in the dam for later use. When there is no drought, the stored water can be released. In this way, it becomes possible to even out the water-flow over the year. Today, however, many hydropower plants, particularly smaller ones, involve little or not regulation. Instead, such run-of-river scheme operate by harnessing energy from whatever water is present in the river at any given time.}

But how exactly is the regulated water flow determined for the purposes of compensation estimation? In section 2 of the \cite{ica17}, it is said that the regulated water flow is to be determined ``on the basis of the increase of the low water flow of the watercourse, which the regulation is supposed to cause beyond the water flow which is considered foreseeable for 350 days a year.'' Hence, the idea is that only the {\it increase} in water flow is to be measured. This means that if the developer proposes a run-of-river project with no regulation, then the natural horsepower of the project will automatically be $0$.\footnote{In fact, things could become even worse for the owner, since the proposed project might lead to $Qreg$ becoming a {\it negative} number. This follows from section 10 of the \cite{wra00}. Here the NVE is given the power to compel the owner of a hydropower scheme to ensure that a certain quantity of water is always allowed to pass through the intake of the plant. This flow of water is typically referred to as the {\it minimum water flow}, but is sometimes used in place of the low water-flow before regulation when calculating the natural horsepower of a project. The idea behind imposing a minimum water-flow is to reduce the negative environmental impact. For many run-of-river schemes, the minimum water flow ordered by the NVE is higher than the low water flow after regulation. Hence, if the minimum water flow is subtracted from the low water flow after regulation, the result is a negative number. That is, one might end up with a {\it negative} $Qreg$.} But this outcome was averted in practice by an {\it ad hoc} adaptation of the traditional method. In relation to compensation, it became established practice to omit the deduction of the previous water flow, so that one would use the entire low water-flow after regulation as $Qreg$. That is, the quantity used for $Qreg$ when computing the natural horsepower of a waterfall for purposes of compensation is the estimated amount of water that is present in the river for at least 350 days a year after regulation.

This means that the natural horsepower of a development scheme has little bearing on the amount of energy that will actually be harnessed from it. Today, modern electricity generators can produce electricity at varying levels of effect, depending on the water-flow of the river. But the water-flow is not a constant as assumed by the natural horsepower formula. Rather, it varies considerably over the year.

As a result, the natural horsepower of a regulation does not have much to do with the value of neither waterfalls nor hydroelectric plants.\footnote{See generally \cite{sofienlund08}.} Indeed, the annual income of a hydroelectric plant has nothing to do with natural horsepower, it is solely a function of the price paid per kilowatthour and the total number of kilowatthours harnessed over the year (kWh/year).\footcite{sofienlund08} The amount of energy generated in a power plant could be measured in other units than kWh, e.g. in terms of the amount of horsepower-hours per year. But the important point to keep in mind is that an energy producer gets paid for the amount of energy he can deliver, \emph{not} the effect he can maintain in his station over a long duration of time. %Hence, even if if we uwould se kilowatt instead of horsepower and talk of the natural kilowatt of a hydropower plant, the quantity we are discussing is the same, and still has little or no bearing on the value of the waterfall.

Talking of natural horsepower therefore serves to give a skewed picture of the potential of a waterfall, especially for run-of-river projects. It is not unusual that the low water-flow in a river amounts to only about 3-5 \% of the average water supply. In modern hydropower projects, one would expect 70-80 \% of this water-flow to be harnessed for energy production even in the absence of any regulation. Hence, in these cases, the natural horsepower method, as it was traditionally applied, would only compensates the owners for about 5 \% of the energy that would actually be harnessed from their waterfalls.\footnote{sofienlund08}

This observation, which is trivial given a rudimentary understanding of the energy business, was not made in the context of expropriation until late in the 1990s. Moreover, the point was raised against the advice of legal experts who regarded the established method as a principle of customary law.\footnote{In the aforementioned case of {\it Hellandsfoss}, for instance, a local owner raised the issue with his legal council, who advised against raising it as an issue before the appraisal courts. The owner listened to his legal council, resulting in a compensation payment that is only a small fraction of what he would be entitled to under the method used in some more recent cases, e.g., in \cite{sauda08}. Source: Private correspondence.} At the same time, both engineers and government officials were well aware of the inadequacies of the method, as illustrated for instance by the following passage from a governmental report made in 1991:\footnote{\cite[19]{otprp50}, discussing the notion of natural horsepower in connection to the uses made of that term in other parts of the law.}

\begin{quote}
The Ministry of Petroleum and Energy has considered moving a proposition for changing the hydrological definitions in the Industrial Concession Act 1917 and the Watercourse Regulation Act 1917. Today the act uses a calculation method based on an increase in regulated water-flow, i.e. that of natural horsepower.[.......] The hydrological definitions of these acts, supposed to indicate how much electricity can be generated, were made on the basis of technical and operative conditions differing very much from contemporary circumstances. In implementing the definitions referred to above one has tried to adapt to the new technological realities of the present day. Therefore, in practice, a calculation based on current production is used instead. From several quarters, particularly the Association of Waterfall Regulators, there has been raised a strong wish to authorize this practice by altering the definitions of the relevant laws. The Department of Oil and Energy agree, but have not as yet made a sufficient elucidation of the issues to be able to move a proposition of alteration of these acts.
\end{quote}

The quote shows that in administrative practice, it had become common to deviate from the definition of a natural horsepower, since it no longer reflected a relevant figure. A similar move would not be made in the context of expropriation for another 20 years.\footnote{A ``natural horsepower'' calculation modified along the lines described by the Ministry in 1991 is now sometimes used also in compensation cases, following its adoption for some of the waterfalls that were expropriated in the case of \cite{sauda08}.}

Within the ranks of the specialized water authorities, the inadequacies of the natural horsepower method had been known even longer. Here it had also been noted that the method did not give rise to realistic estimates of the value of waterfalls. The first record I can find of such an admission dates back to 1957, from an article written by the director at the NVE which was published in their internal newsletter.\footnote{See \cite{....}. The director even went as far as to illustrate a different method, which would also be outdated given today's regulatory regime, but which would reflect contemporary \emph{actual} valuations, used by the NVE itself.}

Considering the physics behind the traditional method is enough to reveal that it fails to give rise to valuations that reflect the value of waterfalls, under any reasonable set of assumptions about the correct general compensation principles one should adopt. Important in this regard is the fact that  the method relies on data that depends entirely on the expropriating party's project. The compensation to the owner depends not on their loss, but on the technical details of the project that the expropriating party proposes. This clearly deviates from even a narrow interpretation of the no-scheme principle.

However, while the idea of compensating the owner of waterfalls by a price per natural horsepower is fundamentally flawed at the theoretical level, there are even more serious concerns that arise when one begins to consider the way in which the unit price has been determined {\it in practice}. The traditional approach to this question has had a particularly dramatic effect on the level of compensation payments. 

In case law based on the traditional method, it is often said that the price set per natural horsepower is set according to ``market price'' for waterfalls. But for the most part, what this means is that the court looks to prices awarded in earlier compensation cases. This practice gave rise to a price level that was entirely artificial. It reflected, more than anything else, the power balance between buyer and seller in the courtroom. It was certainly no genuine market value, even if it was described as such. This has become very clear after the adoption of new, genuinely market-based, methods in recent years.\footnote{See generally \cite{larsen08}.}

Indeed, while the unit price for a natural horsepower did increase somewhat during the first 80 years that the traditional method was used, this increase neither reflected the value of hydropower in particularly nor the level of inflation in general.\footnote{See \cite{sofienlund08}.} Moreover, while the price-level was determined by the courts, some voluntary agreements were also made on the basis of the same method. These could then in turn be used to back up the claim that this was a genuine market-based valuation principle. In this way, it became possible to legitimize an increasing imbalance of power between owners and purchasers. In the end, this imbalance became extreme.

For instance, in 2002 a waterfall belonging to local landowners in the rural community of Måren, located in south-western Norway, was sold for the sum of kr 45 000 (roughly £ 4500), based on traditional calculations.\footnote{Source: private correspondence.} The waterfall has now been exploited in a small-scale hydro-power plant belonging to the large energy company BKK, with annual energy output of 21 GWh.\footnote{$http://www.bkk.no/om_oss/anlegg-utbygging/Kraftverk_og_vassdrag/andre-vassdrag/article29899.ece$} For comparison, I mention that in the case of \emph{Sauda}, where a more realistic market-based method was used, the owners received a compensation which totalled about 1 kr/kWh annual production.\footnote{LG-2007-176723 (I acted as council for some of the owners in this case).} Applied to the Måren case, this would take the compensation from kr 45 000 to kr 21 000 000. That is, the price would have been almost 500 times higher.\footnote{In fact, the Måren waterfalls were cheaper to exploit, so in reality, one would expect that the new method applied to Måren would yield even greater compensation per kWh. I also remark that the value awarded in \emph{Sauda} was market-value, not value of use. It was assumed, in particular, that the owners would have to cooperate with a ``professional'' energy company to develop hydropower. This, in effect, halved the compensation awarded, since the Court's decision was based on the premise that the professional company was willing to pay about 50\% of the profit as rent to the owners.}

The case of Måren illustrates an important point, namely that when the traditional method was used, and described as the ``market value'' of waterfalls by the courts, this became a self-fulfilling prophecy. The prices paid in voluntary transactions were influenced by the practice adopted by the courts far more than the other way around. This, indeed, appears to be a general danger in cases when expropriation is widely used for some particular purpose. The prices paid can easily be kept artificially low by developers making use of expropriation as soon as prices begin to rise. In that way, by relying on what is ostensibly ``market value'' compensation, an artificial price level can be established and maintained. 

I mention that in a setting where the owners are politically powerful and can exert undue influence on the compensation process, the effect can be reversed, so that the ``market based'' approach leads to inflated compensation levels, including elements of holdout value. The general point is that the market approach can be turned to the advantage of the most resourceful and powerful groups, particularly in situations when expropriation is widely used for a particular kind of development. In such cases, a market-based approach is not as politically neutral and ``objective'' as its proponents tend to argue.


The potential severity of this mechanism is nicely illustrated by the case of Norwegian waterfalls. In my opinion, preventing such a mechanism from undermining the fairness of a compensation regime is a main challenge associated with regulatory systems that presuppose extensive use of expropriation. Moreover, in case expropriation is used to further economic development by commercial actors, it is likely that the effect will be detrimental to owners, while creating increased financial incentives for developers to favour expropriation. In this way, a vicious circle is established which can make it hard to break out of the ``expropriation loop'', even though alternatives exist that fulfil the same public interests while ensuring far more equitable forms of benefit sharing and participation.

\section{{\it Kløvtveit} and {\it Otra Kraft}}

Following the liberalization of the Norwegian energy sector in the 1990s, the traditional method came under increasing pressure. It was argued to be unjust by owners and it was held to be illogical by engineers working on developing small-scale hydropower.\footnote{See generally \cite{dyrkolbotn96}.} Eventually, legal professionals followed suit and came to the realization that established compensation  rules based on market value could be applied.\footnote{See generally \cite{larsen06}.} 

Indeed, a new market for waterfalls had begun to develop at this point, following the increased interest in small-scale hydropower and the formation of new companies specializing in cooperating with local owners. For transactions of rights to waterfalls taking place in this market, the traditional method of valuation was not used. In fact, waterfalls were rarely sold at all, but rather leased to the development company for an annual fee. Typically, this fee was calculated by fixing a percentage of the energy produced during the year, and compensating the owners of the waterfall by multiplying this with the market price for electricity obtained throughout the year, possibly deducting production specific taxes, but with no deduction of other cost. In effect, owners would get a fee corresponding to a set percentage of annual gross income in the hydro-power plant.\footnote{See \cite{larsen06}.}

Usually, such a fee entitles the owners to 10-20\% of the income from sale of electricity, depending on the cost of the project. Moreover, it is common that the owners are entitled to up to 50\% of the income derived from so-called \emph{green certificates}, a support mechanism for new renewable energy projects, corresponding to the Renewables Obligation in the UK.\footnote{See http://www.ofgem.gov.uk/Sustainability/Environment/RenewablObl/ for further details.} Essentially, and somewhat simplified, the scheme allows the energy producer to collect a premium on his sale of electricity, which, owning to its ``green'' status, is valued more highly by buyers (usually electricity suppliers), who are required to ensure that a certain proportion of the energy they offer to their customers is considered green. In Norway, such a scheme has been talked about for years, but was only put in force in 2012.\footnote{http://www.regjeringen.no/en/dep/oed/Subject/energy-in-norway/electricity-certificates.html?id=517462} Currently, energy producers can claim a premium of about 2 pp per KWh per year, meaning that about a third of the annual income for new renewable energy projects comes from the sale of green certificates.\footnote{While the premium must be expected to go down somewhat as the certificate market matures and more energy producers acquire "green" status, it will certainly remain an important source of extra income for renewable energy producers also in the future.}

Since these leasehold agreements tie compensation to the fate of the hydropower project, several questions arise when attempting to estimate a present-day value of a waterfall on this market. The valuers first have to determine what the most likely project looks like. Then they have to determine what the annual production will be. After this, they must assess the cost of constructing the plant, something that will in turn make it possible to estimate the level of rent likely to be paid to the waterfall owners. Then, since this rent is set as a percentage of the income from sale of electricity and energy certificates, the need arises to stipulate future prices, usually for as long as 40 years (the usual length of a leasehold). Finally, a present-day value can be calculated based on this cash flow.

The appraisal courts began to use just such a model around 2005. The first case of this kind to reach the Supreme Court was \emph{Uleberg}. In the appraisal Court of Appeal, the lay appraisers overruled the juridical judge and awarded compensation based on the new method. The Supreme Court ordered a retrial on a technicality, but it also commented that it supported the adoption of the new method in cases when \emph{alternative} small-scale development was deemed a \emph{foreseeable} use of the waterfall in the absence of the expropriation scheme.\footnote{\cite{uleberg08}.} Since \emph{Uleberg}, the new method has continued to be used in many cases before appraisal courts.\footnote{See generally \cite{larsen06,larsen08,larsen11}, a series of Norwegian papers discussing the new method.}

It is important to note that it was the lay appraisers that pushed for a new method initially, against the judgement of the legal professionals. This shows, in my opinion, that the old system of lay judgement in appraisal disputes still plays a role in Norway. Moreover, it demonstrates that it has positive qualities that should be preserved in the future. However, the new method is certainly not without its own problems. 

Unsurprisingly, it tends to lead to a rather protracted process of valuation, mostly dominated by experts. Moreover, given all the uncertain elements of the calculation, it is typical that the opposing parties produce expert witnesses that diverge significantly in their valuations. While this can be problematic, the fundamental \emph{legal} challenge arises with respect to the no-scheme rule. In particular, what hydropower scheme should the compensation be based on? Several questions arise, as listed below.

\begin{itemize}
\item (1) Is it foreseeable that the waterfall could be used in a hydropower project in the absence of a power to expropriate?
\item (2) If the answer to question (1) is yes, what would such a scheme look like?
\item (3) Is it foreseeable that such a scheme would obtain the necessary licenses?
\item (4) Does the no-scheme rule imply that the project benefiting from expropriation cannot be regarded as a foreseeable scheme for the purpose of compensation?
\item (5) Is the fact that the scheme underlying expropriation obtained a development license to be regarded as evidence that no other scheme would be likely to obtain such a license?
\item (6) How should compensation be calculated if it is determined that no hydropower scheme would have been foreseeable in the absence of the power to expropriate? 
\end{itemize}

In some cases, for instance when the project benefiting from expropriation is not commercially viable but is carried out for public purposes with the help of special state funding, the answer to question (1) might be no. However, in most cases, the question will be answered in the affirmative, since the scheme benefiting from expropriation already serves as an indication that the waterfall can be commercially harnessed for energy. However, here the no-scheme rule comes into play and creates severe difficulty once we reach question (2). For what kind of scheme can be assumed foreseeable all the while we are obliged to disregard the scheme underlying expropriation? 

In most cases so far, the owners have claimed that compensation should be based on the value of a small-scale hydropower scheme. Since such a scheme is likely to be clearly distinct from the expropriation scheme, one might think that the no-scheme rule will not come into play. This, however, is not necessarily the case. It appears, in particular, that the answer to question (3), asking about the likelihood of obtaining licenses, will still depend on how one views the no-scheme rule. It seems, in particular, that anyone who answers question (5) in the affirmative, will be inclined to say that the alternative project could not expect to get planning permission. This is so, such a person might argue, precisely \emph{because} licenses were granted to the expropriating party. This line of reasoning has been consistently advocated by the large energy companies, ever since the new method emerged.\footnote{See, e.g., \cite{klovtveit11,otra11,otra13}. The argument is often sugar-coated by pointing to the reasons underlying the decision to grant a license -- typically energy efficiency -- rather than by focusing on the formal license itself. In this way, one arrives at an interpretation of the no-scheme rule whereby the scheme can perhaps be said to have been disregarded even though one still takes into account reasons why it should be preferred over other schemes.}

Then the question arises: Is someone who reasons like this at odds with the no-scheme rule? It would seem so, but remember the earlier discussion on the no-scheme rule in Norwegian law, where I noted that the rule has tended to be applied much more narrowly along its positive dimension. Following up on this, it can be argued that while the expropriation scheme is to be disregarded for the purpose of compensation valuation, the regulation underlying the scheme -- or at least the rationale behind this regulation -- is nevertheless to be taken into account. If this point of view is adopted, then the conclusion can easily become that alternative development is to be regarded as unforeseeable. The reason, moreover, will be precisely the fact that the expropriation scheme received a development license. 

Indeed, this line of reasoning was given a stamp of approval in the recent Supreme Court case of \emph{Otra II}.\footcite{otra13} Here, the presiding judge made the following remarks, quoting Gulating Lagmannsrett (the appraisal Court of Appeal), expressing his support, and adding a few comments of his own.\footcite[]{otra13}

\begin{quote}
"[....] The Court of Appeal finds it difficult to distinguish this case from other cases when it has been established that alternative development is not foreseeable. It does not seem relevant whether this is the case because the alternative is not commercially viable or because the alternative must yield to a different exploitation of the waterfall" 
I agree with the Court of Appeal, and I would like to add the following: As the survey of the general principles have shown, it is assumed, both in the Expropriation Act, Sections 5 and 6, and in case-law, that only the value of a foreseeable alternative should be compensated. This starting point means that it would be in breach of the general arrangement if a waterfall that can not be used in foreseeable small-scale hydro-power was to be compensated as if it could be put to such use.
\end{quote}

Having used the development license granted to the expropriating party as evidence that alternative development was unforeseeable, the Court needed to answer question (6) by coming up with some alternative way of compensating the owners.  To do so, the Court was again faced with considering the implications of the expropriation scheme. One possibility would be to ensure that the negative and positive dimensions of the no-scheme rule came to be aligned with one another. That is, as the expropriation scheme was used to rule out alternatives, one might then proceed to use it also as the basis for valuation. Indeed, this is what the Supreme Court did. But at this point, the adherence to the no-scheme rule and a market-based approach spelled doom for the waterfall owners. As the presiding judge reasoned:\footcite[]{otra13}

\begin{quote}
Based on the arguments presented to the Supreme Court, I find it safe to assume that there does not today exist any market for the sale and leasing of waterfalls for which alternative development is not foreseeable, but where the waterfalls can be used in more complex hydro-power schemes. The appellants have not been able to produce documents or prices to document the existence of such a market
\end{quote}

The implicit assumption is that in order to value the waterfall according to its potential for hydropower production, a market needs to be identified. It is \emph{not} considered sufficient that the scheme for which expropriation takes place is itself a hydropower project, on the basis of which the  waterfall value could be assessed following exactly the same steps as in the new method. I also remark that it is very hard to imagine how a market of the kind asked for here could ever develop. After all, any alternative buyers are, by the Court's reasoning a few lines earlier, effectively excluded from being taken into account. In this case, if there was to be a market, it would presumably have to be one that emerged entirely out of the benevolence of the expropriating party.

In fact, the Supreme Court's reasoning in \emph{Otra II} serve as an excellent example of the type of reasoning that makes the no-scheme rule highly problematic for cases of expropriation that benefit commercial schemes. On the one hand, the rule can be used to argue that the inherent value of the scheme itself should not provide a basis for calculating the compensation. On the other hand, it can be used to argue that alternatives must be disregarded in so far as they represent the same kind of exploitation as the expropriation scheme, because they are inferior to it according to the state.

When taken to its logical conclusion, this line of reasoning leads to an offensive result; The commercial value of the property is not to be compensated because the optimal commercial use is the use that the expropriating party aims to make of it. Note that the conclusion is not just that this optimal value, inherent in the scheme, should not be compensated. No, the conclusion in \emph{Otra II} was that \emph{no} compensation could be estimated for any use of the same \emph{kind}, since such use was not foreseeable, owing to the absence of a market.

It is certainly possible to argue that this decision represent a misguided application of the no-scheme rule. In effect, the Supreme Court allowed the licences given to the expropriating party to act as evidence that alternative development was unforeseeable, while it used the no-scheme rule to argue that the hydropower scheme for which this planning permission was given could not itself form basis for compensation payment based on market value.  On the other hand, it seems that even if we disregard the scheme completely, it is unnatural to base the compensation payment on the value of a hydropower scheme that is less beneficial, both commercially and in terms of resource efficiency, than the scheme for which expropriation takes place. Such a scheme would not, one must presume, \emph{actually} have been carried out, regardless of the questions of whether or not it would have been given licenses in the absence of a preferable scheme. 

However, it is not seem particularly difficult to determine what would have been a foreseeable use in these cases, if one assumed only that the power to expropriate had not been granted. If so, it would seem all but certain that a scheme corresponding closely to that underlying expropriation would be implemented. This scheme, however, would be carried out on the basis of sharing the commercial benefit with the owners, not on the basis of expropriation. 

But in \emph{Otra II}, this line of thought was also rejected.\footnote{Although this was in part due to the point not having been argued before the Court of Appeal.} Instead, the Court states that a return to the traditional method is in order. However, they do not apply it in the traditional way. Rather, they sanction a modified version of it that moves away from compensation based on the level of stable effect towards compensation based on average effect.\footnote{That is, they replace the low water-flow by the average water-flow in the definition of Qref, c.f., Section \ref{sec:nathp}.}In addition, they also sanctioned the use of a significantly increased unit price compared to earlier times.

What to make of this? In fact, it seems hard indeed to make sense of since, effectively, by relying on the traditional method, the Supreme Court contradicts its own conclusion that compensations should be based on market value. Instead, they rely on a method that, in effect, is based on an attempt to quantify the value of the waterfall as it is being used by the expropriating party in his project. However, by relying on a technical method that has been completely outdated, it becomes difficult to assess the outcome properly, at least for a non-expert. This is so even after the modifications have been implemented, which make the method appear somewhat less irrational from a physical point of view.

But it is still noteworthy that the Supreme Court prefers the obscurity of the traditional method, as an established custom, over the explicit conclusion that it simply is not tenable to adopt the ``value to the owner'' principle in cases like this, as least not as that principle is construed in Norwegian law.

In any event, I think there is good reason to be critical of the Supreme Court for sanctioning the view that alternative development was unforeseeable in {\it Otra II}. Still, it is not possible to escape the fact that this reflects a general tendency in Norwegian law, whereby the positive dimension of the no-scheme rule is much weaker than the negative part. Even if it appears unreasonable, it might very well be a correct application of national law. Moreover, it could very well have been that alternative development was unforeseeable for \emph{some other reason}, for instance because the only commercially viable exploitation was the scheme planned by the expropriating party. In this case, the problem of how to compensate the owners in the absence of an alternative form of exploitation would still arise. It is this question, in particular, which seems entirely unsatisfactorily resolved under an application of a ``value to the owner'' principle.

This is witnessed by \emph{Otra II}, and, in fact, it appears that the Supreme Court, in their decision  \emph{not} to follow their own reasoning to its logical consequence, makes quite a powerful statement. For all intents and purposes, the Supreme Court \emph{rejects} the "value to the owner" principle, but they obscure this by wrapping it up in the traditional method, which is deeply flawed. However, the problem it attempts to solve appears significant, and it pertains directly to the question discussed more generally in Section \ref{sec:noscheme}, namely how to compensate owners that loose their land to commercial schemes. 

It seems that even the fiercest supporters of limiting owners' right to compensation tend to find it too offensive to apply this principle when it leaves the owners with no form of compensation in cases when they are forced to give up property to purely commercial undertakings. Indeed, such a practice would surely also be in breach of the human rights law. In these cases, the subjective aspect of the ``value to the owner'' principle is impossible to maintain. If the commercial value falls to be disregarded for no other reason than the fact that the state happens to have granted planning permission to the expropriating party rather than the owner, this is not only dubious with respect to human rights protecting property, but also appears to be a case of \emph{discrimination}, e.g., as prohibited by ECHR Article 14.

The problem does not arise when the buyer sees value in the property that is of a different \emph{kind} than that realizable by \emph{any} private owner. In this case, the rule simply states that the owner should not be able to demand that ``public value'' is transformed into commercial value just for him. This appears like a reasonable principle. But when there is commercial value already present on the "public" side of the transaction, it seems completely unwarranted that the public should be allowed to transfer this value from the owner to someone else without compensation. Thus, it seems that more accurately and acceptably, the ``value to the owner'' principle should be thought of as a ``commercial value'' principle. It seems, in particular, that the principle need to be stripped of any suggestion that a preferential financial position is to be awarded to whoever benefits from expropriation.\footnote{Exceptions might be possible to imagine, but, one would think, only when they can be construed as falling under the ``public value'' banner in some way.}

It seems unfortunate that this aspect has not been made explicit, and the difficulties that arise in the absence of this nuance are nicely illustrated by the case of Norwegian waterfalls. Still, as the case of \emph{Otra II} indicates, an interpretation of the ``value to the owner'' principle along less offensive lines is in reality already in place with regards to Norwegian hydro-power. Here it seems that ``value to the owner'' has in fact \emph{never} been applied in the traditional way. Hopefully, rather than obscuring this fact by relying on an unsatisfactory and artificial method for calculating the compensation, the future will see further developments that recognize the need for new principles. 

It should be recognized, in particular, that as the law has been applied for the last 80 years, despite its grave flaws and injustices, there has always been an implicit recognition in Norwegian law that the owners of waterfalls are \emph{entitled to their share} of the commercial benefits of hydropower. 
In fact, in the recent Supreme Court case of \emph{Kløvtveit}, a novel approach along the lines I am advocating was applied in circumstances similar to that of {\it Otra II}.\footcite{klovtveit11} The conclusion here too was that alternative development was not foreseeable. However, unlike in \emph{Otra II}, the lay appraisers in the Court of Appeal had compensated the owners based on the fact that they regarded it as foreseeable that in the absence of the scheme, the waterfalls would have been exploited in exactly the same way, except that it would have happened in the form of \emph{cooperation} between the owners and the expropriating party. By this line of reasoning, the Court effectively seems to have adopted a more modern ``commercial value'' principle, to replace the traditional method. 

For commercial projects, it seems that in the absence of a power to expropriate, any rational buyer would look to cooperate with the owners. This would not necessarily be a safe assumption to make for non-commercial projects. Such projects may fail to provide the necessary incentives for cooperation, even though they should nevertheless be carried out in the public interest.

I mention that \emph{Kløvtveit} was discussed in \emph{Otra II}. But the presiding judge chose to focus on what he regarded as the ``practical problems'' associated with the prospect of cooperation and a compensation award calculated on this premise. The cooperation model was not the center of attention in the case, however, so one can only hope that \emph{Kløvtveit}, rather than \emph{Otra II}, will become the influential precedent for future cases.

\section{Conclusion}

In this chapter I have presented the current compensation regime associated with waterfalls and I have related it to the broader question of how compensation should be calculated when commercial companies benefit from the property that is taken. I focused particularly on the no-scheme principle, which plays an important role in this regard in many jurisdictions, including in Norway. But I also emphasized another aspect particular to the Norwegian system, namely its reliance on the judgement of lay appraisers. 

I noted that the appraisal courts would typically operate largely unconstrained by specific evaluation rules, as they were directly guided by the Constitution and its requirement that ``full compensation'' had to be paid. I argued that this system was both flexible and capable of facilitating broad fairness considerations. The notion that constitutional absolutism was a rigid system, I argued, is largely unfounded. The system had a procedural flexibility that should not be underestimated and which served as a counterbalance to its seeming adherence to a strict dogma.

In fact, when moving on to consider the case of waterfalls in more depth, I noted how this flexibility was used to great effect. It allowed the appraisal courts to ensure that some benefit sharing was maintained in hydropower cases even after the regulatory system was transformed so that benefit sharing following expropriation would be hard or impossible to achieve in a system based on a legal formalization of the no-scheme principle. For over 80 years, the courts happily deviated from it entirely when awarding compensation for waterfalls. Remarkably, this practice continued even after legislation was passed that provided much more specific guidelines to the appraisal courts, and which seemingly enforced a strict no-scheme principle in Norwegian law.\footnote{More generally, however, I noted how this legislation, and the Constitutional battles that followed it, has lead to a development whereby the appraisers are somewhat marginalized and the Supreme Court itself has assumed greater power in directing them, by providing their own interpretation of a body of legislation that contains many specific rules that are hard to apply to concrete cases in a uniform fashion.}

However, as the appraisal courts were marginalized by increasing levels of top-down control, first by the legislator and later by the Supreme Court, the method that was developed to compensate waterfalls would itself develop into a fixed and rigid rule. It was not adapted, in particular, to reflect technological and economic progress. Since these were particularly rapid and ground-breaking in the energy sector, the result was a very severe mismatch between the real value of waterfalls and the compensation paid following expropriation. 

In the final part of the chapter I then considered recent cases where the traditional method has been abandoned in favour of a market-based approach which is based on the general rules governing compensation today. I found that while these cases tend to result in payments that more closely reflect actual commercial values, they raise severe problems of their own. Here, in particular, the no-scheme principle re-emerges on the scene with full force, becoming a very effective tool for those who seek to argue that hydropower development is not a fruit of property but belong to those who obtain expropriation and development licenses from the state. 

If such arguments are successful, the market-value approach can lead to worse outcomes for local owners of waterfalls than what they would be entitled to under the traditional method. The deeper question that arises, of course, is the following: what is equitable benefit sharing in these cases, and how can it be ensured? A second question is whether owners can in fact {\it demand} some level of benefit sharing on the basis of human rights law. This question is now coming into focus in Norway, as the Supreme Court's decision in {\it Otra II} has been brought before the ECtHR.
 
It is my opinion that the best way to ensure benefit sharing under a compensatory approach is to revive the old system of an independent appraisal procedure relying on the discretion of lay people from the local area. The most important aspect of this, I believe, is that it enhances the democratic legitimacy of the compensatory approach. It is clear that there is a great deal of uncertainty in the kinds of calculations one must engage in to assess the commercial value of a waterfall. Therefore, the temptation to rely blindly on experts and special rules that are not properly understood becomes great. This might reduce the uncertainty involved, but only to some extent. Moreover it also highly increased the risk of unfairness and opens up the possibility that powerful interests can sereptisiously usurp the procedure for their own interests. Compared to this, a system based on direct fairness assesments carried out by noraml people, on the basis of (hopefully) neutral information provided by experts, might well be the best option.

However, I think the inherent difficulty in devising appropriate compensation mechanisms for commercial potentials suggest that the compensatory approach might be misguided altogether. In addition, as soon as one begins to look at the social function of property, and its role in human flourishing, it seems that any kind of financial compensation is going to provide an inadequate reply to deprivation of commercially interesting property. Such a system speaks volumes about {\it who} society deems capable of carrying out commercial projects. The discrimination suffered by property owners of the {\it wrong kind}, should also not be underestimated. 

In light of this, I think it is appropriate to consider alternatives to expropriation in cases when economic rationales dictate economic development. Interestingly, the Norwegian system has an entire legal framework in place that can elegantly facilitate such a shift, should enough political be mustered to compel government and developers to make use of it. In the case of hydropower development, it already being put to the test in an increasing number of cases for substantial development projects, as local developers tend to shun away from outright expropriation of property belonging to unwilling neighbours. 

The land consolidation mechanisms that can be used to facilitate compulsory development in these situations form part of an ancient semi-juridical system of land management in Norway. In my opinion, this framework also points towards the future, as it provides a highly flexible approach for dealing with property and economic development under varying degrees of compulsion. In my next and final chapter I will present it in more depth and argue that it can often provide solutions that are both more effective and more equitable than solutions arrived at in a system that relies on expropriation. The compensation issue, in particular, is resolved simply by giving unwilling owners low-risk financial instruments tied to the development that is ordered to take place on their property.


\chapter{Compulsory participation in economic development projects}\label{chap:6}

\section{Introduction}\label{sec:intro6}

In this Chapter, I will consider {\it alternatives} to expropriation in the context of economic development. This is also where I ended my theoretical discussion in Part I, by presenting and analysing the proposals of Heller and Hills in the context of the US debate on economic development takings. Here, I return to this point in the context of my case study, by exploring the Norwegian institution of {\it land consolidation}. 

In recent years, this institution has been used extensively to facilitate hydropower projects. So far, however, it is used almost exclusively in situations when owners themselves organize such projects. In these situations, expropriation is rarely sought and rarely authorized. Instead, various consolidation measures are used, including the practically important measure of a ``use directive'', to set up an organizational framework and a binding plan for development involving jointly owned property,

In addition to the practical use of consolidation in the context of hydropower development, there are recent legislative developments in Norway that sees the consolidation alternative gain importance also in relation to other forms of development. This includes urban and non-agrarian development projects, something that represents a departure from the tradition of land consolidation in Norway. Some argue that these uses of land consolidation leave the owners in a precarious position, and may weaken private property rights. In this Chapter, I argue for the opposite perspective, that the use of consolidation in these new contexts will enhance property as an institution. Moreover, I argue that it can be used to  address the democratic deficit of economic development takings in a very elegant way, provided the land consolidation process itself remains intact, as a service to owners and local communities, and is not usurped by external actors.

I begin in Section \ref{sec:lce}, by presenting the basic idea of using land consolidation as an alternative to expropriation. I discuss broad notions of land consolidation in general, relating them also to the discussing found in Heller and Hills' article. Then I point out some special features of the Norwegian system, which I believe make it particularly natural to consider in the context of economic development cases. 

Then, in Section \ref{sec:lcc}, I present the Norwegian system of land consolidation in more depth, focusing on the procedural rules and the rules in place to protect property against unwarranted interference. I focus particularly on the so-called ``no-loss'' guarantee, which states that no consolidation measure can be implemented unless the benefits make up for the harm, for all affected properties. Hence, land consolidation is quite distinct from expropriation in general. However, in situations when benefit sharing is possible or natural, it becomes possible to fulfil the no-loss criterion, and in doing so land consolidation could become a powerful alternative to expropriation for such cases. 

In Section \ref{sec:lch}, I discuss land consolidation specifically in the context of hydropower development. I consider several cases in detail, based on court documents and recent work done in a master thesis on consolidation, where the author carried out interviews with affected owners. Then, in Section \ref{sec:lca}, I offer a assessment and discuss some challenges. I argue more extensively hat the land consolidation alternative is should be seen as a possible way to strengthen property rights, not as a threat. I also pinpoint what I believe to be the main challenge, namely to ensure that the land consolidation process remains intact as a service to owners and local communities, even after powerful commercial interests enter the scene. In Section \ref{sec:conc}, I conclude.

\section{Land consolidation as an alternative to expropriation}\label{sec:lce}

The notion of land consolidation is somewhat ambiguous. At its core, it refers to  mechanisms whereby boundaries in real property are redrawn to reduce fragmentation, without affecting the relative value of the different owners' holdings. However, it is also common to use consolidation to refer to mechanisms for pooling together small parcels of land to create larger units. There is a tension between these two notions of consolidation, with some claiming that consolidation in the latter sense is sometimes used to surreptitiously bestow benefits on powerful property owners, at the expense of weaker groups.

In light of this, I should stress at the outset that I will use the term land consolidation in a very broad sense in this chapter, much wider than {\it both} of the interpretations mentioned above. Land consolidation, as I use the term, refers to any mechanism by which the state intervenes, at the request of some interested party, to reorganize property rights in a given local area. Hence, a consolidation measure might as well involve {\it increased} fragmentation of property, if this is deemed a rational form of ``consolidation'' of the property values of the affected area. Importantly, I also use land consolidation to refer to efforts directed at organizing the {\it use} of property, not just redrawing boundaries.

Some might argue that this terminology is strained, but I adopt it for a reason. It is motivated by the fact that in Norway, the institution known as ``jordskifte'', officially translated as land consolidation, has a very broad meaning. Norway is not unique in this regard. Land consolidation has a similarly broad scope in many jurisdictions of continental Europe, as well as in Japan and in parts of the developing world.\footnote{See \cite{sky07;viliainen04}.} Moreover, in Heller and Hills' work on land assembly districts, a comparison with land consolidation is presented, based on the same broad notion that I use here.\footcite{heller08} One of my main aims in this chapter is to pick up on this, by offering a more detailed comparison and assessment, specifically anchored in the Norwegian system and its application in the context of hydropower development.

As land consolidation tends to involve interference in property rights, one may ask about the legitimacy of various consolidation measures, held against rules that protect private property owners. In some cases, it can also be argued that land consolidation {\it is } a form of expropriation, even if it is not necessarily recognized as such in the jurisdiction where it takes place. Such legitimacy issues have in fact been raised before the ECtHR on a few occasions, where the Court has found that land consolidation measures have been applied in breach of P1-1. Similarly, in the US, a proposal to introduce a land consolidation regime was struck down in the 1980s, as not in keeping with the property clauses of the US constitution.

\noo{ Move: On the other hand, if land consolidation is used to facilitate or impose specific uses of property, it can also be used as an {\it alternative} to expropriation, a compulsory measure that can obviate the need for depriving owners of their property rights. I think this latter perspective on land consolidation is particularly interesting, and it is the perspective I adopt in this chapter.}

In relation to the legitimacy issue, the Norwegian system stands out in two important regards. First, the consolidation procedure is managed by judicial bodies, namely the {\it land consolidation courts}. Second, land consolidation is largely seen as a service to owners, not a tool for increased state control and top-down management. In particular, a case before the land consolidation courts is almost always initiated by (some of) the affected owners themselves, in an effort to clarify who owns what, readjust boundaries, or organize the use of property in the local area. Moreover, it is a core principle of law that no land consolidation measure may be undertaken unless the benefits make up for the harms, for all the properties involved. This is known as the ``no-loss'' criterion. It is one of the key principles of land consolidation in Norway. The combination of a judicial procedure that places great emphasis on owner-participation and a no-loss criterion means that, arguably, land consolidation in Norway {\it strengthens} property as an institution. 

However, the beneficial effects of land consolidation are not primarily supposed to target individual owners, but rather the properties as such, as productive units  of importance to the community of owners as a whole. Hence, I believe land consolidation can serve as an effective countermeasure against two of the most widely discussed challenges to any property regime. First, consolidation can serve to protect an egalitarian distribution of property rights against the deleterious effect of inefficiency and underdevelopment that might otherwise arise from fragmentation. Importantly, it can do so without disturbing the underlying property structure and without necessarily resulting in disproportionate benefits or harms to owners and other private parties. In particular, land consolidation can obviate the need for handing property rights over to powerful market actors to ensure development. Second, land consolidation can serve to ensure sustainable and rational management of jointly owned land, without necessarily forcing an enclosure process (enclosure {\it can} be the result of land consolidation, but it is only one of many measures in the consolidation toolbox). 

In short, land consolidation can be used to address both commons and anti-commons problems, in a way that protects, and possibly enhances, desirable social functions of property, through a judicial system of participatory/adversarial decision-making. Hence, land consolidation in Norway is based on a conceptual premise that -- potentially -- offers protection to owners and their properties, by recognizing them as members of a community that are mutually dependent on each other. In this way, the form of property protection offered in the context of land consolidation is distinct from the protection offered in the context of expropriation law, in a manner that is in itself interesting, particularly from the perspective of property's social functions, as discussed in Part I.

Importantly, since land consolidation can be used to facilitate or impose specific uses of property, it can also be an {\it effective} alternative to expropriation, a compulsory measure that can obviate the need for depriving owners of their property rights. I think this perspective on land consolidation is particularly interesting, and it is the point of view I adopt in this chapter. Indeed, consolidation as an alternative to expropriation is particularly natural for economic development projects. Importantly, it seems that the no-loss criterion should be possible to fulfil in these cases, through benefit sharing. 

The land consolidation courts can, moreover, {\it impose} benefit sharing on the parties. However, it  is usually {\it not} permitted to address the no-loss criterion by compensatory means, particularly not if those means are monetary. Instead, the general idea is not only that the benefits resulting from the consolidation measure must match the harms, the benefits must also be distributed fairly among the affected properties.\footnote{The latter principle is not as strictly encoded in the law, but finds formal expression in certain special provisions. Hence, the extent to which fair benefit sharing is {\it actually} achieved following consolidation to facilitate large-scale economic development projects, is an interesting question that I return to in ....} So the principle of benefit sharing at work is not one where the owner is a passive recipient of compensation, but rather an active participant in the development itself, possibly against his own will. This, too, I find highly interesting, particularly from the point of view of the human flourishing conceptions of property that I discussed in Part I.

More pragmatically, the emphasis on benefit sharing in land consolidation reveals a concrete potential advantage of land consolidation over expropriation, from the owners' point of view. In particular, while some sort of benefit sharing is typically ensured through land consolidation, it is much less commonly achieved through compensation in the context of expropriation.\footnote{This is largely due to the so-called {\it no scheme} principle, which states that compensation to the owner following expropriation should not reflect changes in value that are due to the expropriation scheme. I am not aware of a single jurisdiction that does not include a variant of this principle. For a detailed investigation into the question of whether or not it stands in the way of benefit sharing in economic development cases, I point to \cite{....}.} This means that the use of land consolidation in place of expropriation has considerable potential also in relation to the problem of the ``uncompensated increment'' in economic development takings, as discussed in Part I.

On the other hand, this also means that commercially motivated developers may have an {\it incentive} to favour expropriation. This, then, raises the question of whether or not calling for the use of land consolidation as an alternative can act as a {\it defence} against expropriation, or, if this is not possible presently, if such a defence {\it should} be open to property owners facing condemnation for commercial purposes. Secondly, the fact that consolidation can act as an alternative to expropriation  also gives developers an incentive to push for changes in land consolidation law itself, so that it will become more profitable for them to make use of it.

Hence, one question that arises in the present context is the following: will land consolidation remain a service to owners, or will it become a service to developers who seeks cheap access to property owned by others? This question is becoming increasingly relevant in Norway, as the scope of land consolidation has been broadened in recent years, so that it {\it can} in fact be applied in many expropriation contexts, also to facilitate commercial development in urban areas.

So far, the Norwegian system is moving along a trajectory where land consolidation as an alternative to expropriation is primarily seen as a service to developers and the public, not as a means for empowering owners. It is noteworthy, in particular, that owners are not normally entitled to demand land consolidation in place of expropriation. Instead, following a change in the law that takes effect in 2016, the {\it developers} will be granted a new right, namely that of bringing a case before the land consolidation courts, to seek help in implementing their project. In fact, developers might well be motivated to do so, because of the potential for reduced administrative costs, a more effective and flexible procedure, and a chance of limiting compensation claims by imposing (cheaper) compensatory consolidation measures (e.g., by providing owners with replacement property).

\noo{However, the issue of benefit sharing is bound to come up, particularly in the context of commercial development. In this regard, the risk for developers is that they will be compelled to share the benefits with the owners. However, in order for this to happen, the Land Consolidation Court must actively take steps to make it happen, by recognizing the owners' right to benefit sharing. Moreover, while benefit sharing is a fundamental principle for land consolidation among owners, it remains to be seen if this way of thinking will be preserved when new and powerful external actors enter the scene.}

However, it seems that in order to be a truly effective alternative to expropriation, not only the takers, but also the owners, should be granted the opportunity to request implementation by consolidation. In addition to the question of whether a land consolidation measure can be requested in an expropriation scenario, one must also ask how exactly it would work, and what policy aims it could help achieve. Here there is already quite some data available, arising from situations when owners themselves are behind economic development, but prefer to make use of consolidation measures, instead of expropriation, in order to deal with their neighbours.

Interestingly, in the context of hydropower development, this use of land consolidation has become very important in recent years. In 2009, the Court Administration estimated that land consolidation had helped realise small-scale hydropower projects with a total annual energy output of about x TWh/year. Moreover, in a recent Supreme Court case, the importance of land consolidation was stressed specifically, as a justification for requiring a commercial taker to pay additional compensation to the owners of waterfalls that were to be used for hydropower generation.

In the next section, I give some more details on the Norwegian system. Then, in Section \ref{sec:x}, I return to the use of land consolidation to facilitate small-scale hydropower, which I approach as a test case for the more general proposition that land consolidation can be a legitimacy-enhancing alternative to expropriation, particularly in the context of economic development.

%%%%%%%%%%%%%%%%%%%%%%%%%%%%%%%%%%%%%

\noo{
In particular, a case can now be brought before the land consolidation court by an external developer who would otherwise need to expropriate land to implement a project. However, this change in the law also contributes to a shift away from seeing land consolidation as a service to owners, towards seeing it also a service to developers who seek control of property they do not own. This shift could in turn change the dynamics of land consolidation in a way that makes it less distinct from expropriation.

Even though this definition is broad, I note that a clear distinction can be drawn between land consolidation and national or regional land {\it policies}, which do not target specific properties. The distinction between land consolidation and land-use planning can be harder to draw, but looking to the theoretical starting points of these two kinds of interventions, suffice to establish sketch.

While state planning is an expression of the state's right to regulate the use of land, a land consolidation measure is a {\it service} provided by the state, to facilitate property uses and structures that are deemed desirable from the point of view of the properties as productive units under private ownership.

The distinction between consolidation and measures of land reform may sometimes also be difficult to draw, particularly with my wide notion of consolidation. However, while land reforms tend to arise from centrally directed measures that apply generally within a jurisdiction and come about as the result of a special political initiative, consolidation usually denotes a more flexible framework where local communities are restructured in a way that aims to bring benefits to all owners and rights holders within that community. As such, consolidation rules may alleviate the need for new land reforms, and they may come to represent a ``bottom up" approach to the restructuring of real property.\footnote{The potential for this has been noted even for the traditional understanding of consolidation, as a reduction in the level of property fragmentation. See, e.g., \cite{oldenburg90}. For a different perspective, arguing that land consolidation is generally not sufficient to achieve the noble ends of land reform, see \cite{lipton74}. For a more recent, comprehensive, assessment of the relationship between land reform and consolidation (in the narrow sense), I refer to \cite[237-244]{lipton09}.}

In the following, I adopt this normative stance on the {\it purpose} of consolidation. Hence, I use land consolidation to refer to a regulated process of land reorganization that come about as a result of a concrete, often local initiative, has a limited geographical scope, relies on the involvement of the local population, and seeks to promote the best interests of all the affected land users. I remark that while Norway has a particularly broad approach, land consolidation more or less in line with my understanding here serves an important function in many jurisdictions.\footnote{For a survey of contemporary land consolidation rules in Europe, reflecting also the need for a wide understanding of the term, I point to \cite{vitikainen2004}.}

One attractive feature of consolidation is that it provides a flexible, dynamic, framework that allows for gradual adaptation of ownership structures, so that they better suit prevailing economic and social conditions. Moreover, land consolidation can become significant in relation to concrete development projects, particularly when such projects necessitate cooperation among several owners. This, in particular, is the use of consolidation that I aim to shed particular light on in this chapter, by giving a case-study of Norwegian law.

Mechanisms for facilitating and organizing concrete development schemes are now integrated into the law relating to consolidation in Norway. These rules do not form part of the historical core of consolidation rules in Norway, however, the focus on land consolidation for development is of a more recent date. It is reflected in a number of new provisions, most recently in the Land Consolidation Act 2013 which will take effect on 1 January 2016.\footnote{Act no 97 of 10 June 2013 relating to the determination and change of structures of ownership- and rights to real property etc.}

When land consolidation is used as a means to organize development projects it also becomes natural to view it as an alternative to expropriation, especially in cases when development has commercial potential and is meant to be carried out by companies operating for profit. Moreover, the controversy that often surrounds such cases further suggests that it should be explored to what extent processes of consolidation can replace expropriation as an implementation mechanism for development of this kind. As we will see, the principle of local participation and benefit sharing is more firmly entrenched in the rules and procedures that govern the consolidation process than in the processes that govern the use of expropriation. 

The contrast between expropriation and consolidation is particularly clear in Norwegian law, where a ``no loss" principle is enforced with regards to the latter, protecting all affected owners and rights holders. It states that the consolidation process should not leave any owner or rights holder worse off after consolidation. The aim of consolidation is to bestow a benefit on \emph{all} interested parties.\footnote{See Section 3 a) of the Land Consolidation Act 1979 (currently in force) and Section 3-18 of the Land Consolidation Act 2013. For a paper discussing the rule in more detail we point to \cite{rygg1998}. Rygg is also critical of what he sees as a development away from a strict interpretation of the no loss rule.} For instance, if ownership is highly fragmented, consolidation mechanisms may be used to exchange property between owners or to introduce joint ownership, but due to the no loss rule it will not be possible to use consolidation in order to deprive some owners of their property to the benefit of others.

In the following, I map the differences between consolidation and expropriation in Norwegian law, starting with an overview of the land consolidation rules, focusing on the development towards giving these rules greater application in connection with concrete development schemes. I then study some cases of locally controlled hydro-power development where land consolidation was used as a means to organize projects involving many different owners and rights holders. We argue that these cases illustrate how the consolidation rules currently in place are well suited to meet local demands for participation and benefit sharing, more so than the existing framework regulating expropriation.

The structure of the remaining part of the chapter is as follows. In Section \ref{sec:2} I briefly present the basic rules regarding land consolidation in Norwegian law, including a presentation of the special consolidation courts used to administer the process. Then in Section \ref{sec:3} I go on to consider in more depth the rules relating to so-called \emph{use directives}, permitting the court to actively pursue development projects on behalf of, and in cooperation, with local owners. Use directives represent a form of compulsory cooperation which I believe deserves further attention in the context of land development, especially as an alternative to expropriation. I follow up with a case-study of hydropower in Section \ref{sec:4}, and in Section \ref{sec:5} I contrast the use of directives with more commonly seen approaches to pooling of resources and commercial land development. In Section \ref{sec:6} I offer a conclusion.
}

\section{The system of land consolidation courts}\label{sec:lcc}

Rules regarding land consolidation have a long history in Norwegian law. The first consolidation rules were included already in King Magnus Lagabøte's \emph{landslov} (law of the land) from 1274, the first piece of written legislation known to have been introduced at the national level in Norway.\footnote{See Chapter 4, Section 2 in \emph{Jordskifterettens stilling og funksjoner}, NOU 2002 no 9, report to the Department of Agriculture from special committee appointed by the King in Council 10 October 2000.} The earliest rules targeted jointly held rights in farming land, giving any owner or tenant farmer on that land an opportunity to demand apportionment that would give him exclusive rights on a parcel of land corresponding to his share of the joint rights.\footnote{The share in joint rights belonging to each individual farm was historically determined based on the amount of rent (``skyld") that each farmer paid to the land owner. However, following the union with Dennmark and especially after the advent of enlightened absolutism, tenant farmers in Norway increasingly bought their land from increasingly marginalized Danish land owners. Indeed, tenant farming became uncommon in Norway after the 18th Century, but the notion of skyld was kept as a measure of the share each farm had in the now jointly owned larger estate, and it is still relied on in various contexts, such as for the purpose of apportionment. References needed.}

Many of the rules currently in place were developed in the 19th Century. At this time, the main use of land consolidation was still to divide jointly owned land into parcels, but the relative importance of such measures increased greatly since it was seen as a necessary adjustment in an age when industrialization introduced a range of new and more efficient farming techniques. In particular, as changes in farming methods resulted in an increased need for capital in agriculture, full ownership came to be regarded as more favourable since it meant that better security could be offered to financial institutions.\footnote{References needed.} 

However, in some cases it was noted that full division of ownership might not be required and that use directives could be employed instead, to provide the individual farmers with clear rights of use over joint land. In addition to providing clarity and security for users, the rules introduced to facilitate this also created a legal framework for implementing development without altering the underlying structure of ownership, even in cases when development would require considerable pooling of resources and decision-making power. The rules were initially targeted at more rational organization of land use in agriculture, enabling rural communities to adapt to changing economic conditions without fundamentally altering them or leading to displacement or depopulation. Hence they were mostly relied on in connection with farming, and not commonly used to facilitate different kinds of development.

In recent years this has changed. Today, use directives are increasingly applied also to organize development projects that are not associated with traditional farming. Moreover, many additional mechanisms of land consolidation have been introduced, all aiming in various ways to ensure better organization of land use and ownership. These mechanisms are administered by the \emph{consolidation court}, a special tribunal which has land consolidation as its sole task.\footnote{It appears to be a unique administrative unit in the European setting, although Austria have land tribunals that resemble it. References.} There are three main categories of consolidation tools that the court may use, and they are summarized in the following.

\begin{itemize}
\item \emph{Apportionment of land}: Rules that empower the court to dissolve systems of joint ownership by apportioning to each estate a parcel corresponding to its share, or by reallocating property through exchange of land. This is the traditional form of land consolidation in Norway, and the main legislative basis for it is provided in Section 2 a-b) of the Land Consolidation Act 1979.
\item \emph{Delimitation of boundaries:} Rules that empower the court to determine, mark and describe boundaries between properties and the content and extent of different rights of use attached to the land. The main legislative basis for this form of consolidation is found in Section 88 of the Land Consolidation Act 1979.
\item \emph{Directives for use}: Rules that empower the court to prescribe rules for the use of jointly held land, and to organize such use, including setting up organizational units for carrying out specific development projects, as described in Section 2 c) and Sections 34-35 of the Land Consolidation Act 1979. 
\end{itemize}

In all cases, the consolidation court can only employ these tools when they are called on to do so by someone who is regarded as having a valid legal interest in the matter.\footnote{See \cite[5]{lca79}.} Traditionally, this meant one of the owners of the land involved, but gradually the system have also come to recognize that others might have a legitimate interest in consolidation. This includes developers who have obtained planning permission for specific projects that require reorganization of property rights. The legal role of such actors is currently changing from passive to more active, and this development raises particular questions that we will return to below. For now, I note that the traditional situation is that a consolidation process is initiated by one of the local owners of the land. It is still true, in particular, that consolidation is primarily a mechanism by which any one among the owners can work out what their legal position is, and, if the court agrees that it would be favourable to the use of the land, can ensure that the rights in the land are restructured.

The condition that restructuring only takes place when it is regarded as favourable is an additional condition that limits the court's authority to take action that involves apportionment and directives for use. To determine whether or not it has been met, the court will look to the current economic and political climate, and so the consolidation rules are \emph{dynamic}, capable of being adapted to the circumstances. In this regard the court is also influenced by what it regards as the prevailing public interests in property use, and recent developments in consolidation law stress the importance of this link, with recent reforms seeking to strengthen it.\footnote{See for instance Prop. 101 L (2012-2013) (report from the Department of Agriculture regarding the new Consolidation Act).}

However, the contextual nature of land consolidation has always been clear. We have already mentioned how the basic building blocks of the current system can be traced back to the influence of technological advances in farming and the modernization processes that Norwegian society underwent in the 19th Century. The law responded to these changes, and consolidation became a vitally important instrument for change and development in this period. It was also at this time that it was decided to establish a tribunal system for administering the process, first in the Land Consolidation Act from 1857 and then revised and developed further in acts from 1882 and 1950.\footnote{An overview of the history of consolidation law is given in Chapter 3 of Prop 101 L (2012-2013).} The procedural rules closely mimics those that pertain to the regular civil courts. This ensures that consolidation tools are only put to use if a court orders it, and only after a public hearing where all involved parties are given an opportunity to present their case, give supporting evidence, and to contradict each others' testimony. For a more detailed description of the consolidation court, I refer the reader to Section \ref{subsec:21} below.

The current system for land consolidation is based on the Land Consolidation Act 1979, but this act will be replaced in 2016 when the new Land Consolidation Act 2013 will take effect. The new act was passed on 10 June 2013, and while it does not introduce any dramatic changes to the law, it further widens the scope of consolidation, particularly with regards to directives for joint use, as discussed in Section \ref{sec:3} below. The new act also contains an explicit description of the purpose of land consolidation, which we now quote.\footnote{Act no 97 of 10 June 2013 relating to the determination and change of structures of ownership- and rights to real property etc. (henceforth the \emph{Land Consolidation Act 2013}). It will take effect on January 1 2016.}

\begin{quote}
Section 1-1 The purpose of the Act

The purpose of the act is to facilitate efficient and rational use of real property in the best interests of the owners, rights holders and society. This objective will be pursued by the land consolidation courts which will implement remedies for unpractical structures concerning ownership and use of property, ascertain and determine property boundaries, as well as decide appraisal disputes and other cases as pursuant by this and other acts.

The act also seeks to facilitate fair, responsible, quick and effective processing of cases in independent and impartial public courts that will operate in such a way as to enhance confidence in the consolidation process.
\end{quote}

This statement of purpose highlights how the new act incorporates and extends the trend towards giving the consolidation process wider scope. I also note how it reiterates and emphasises that the process is to be tribunal in nature. In my opinion, it also suggests that land consolidation is likely to become more important in the future, increasingly also outside the traditional agricultural setting within which the ancient body of law regulating it has hitherto developed.\footnote{For instance, following a change in the Land Consolidation Act 1979 in 2006, Land Consolidation may now also be called on in order to manage restructuring of ownership in urban areas, in connection with specific development schemes. This rule has been extended further in the new Act, and it will be interesting to see how the division of labour will be in the future, between planning authorities, regular courts and the land consolidation courts.}

It is now explicitly stated that the purpose of land consolidation is to make conditions of property use more favourable for all the affected owners and rights holders. Hence the new act accentuates how consolidation represents a form of interference that is fundamentally different from expropriation. As before, the consolidation court is not empowered to take action unless it is called on by one of the stakeholders in the property, see Section 1-5 of the Land Consolidation Act 2013. However, according to the new act, a developer who has obtained permission to expropriate is to be counted as a stakeholder in that land for the purposes of consolidation.\footnote{Previously, a developer was only regarded as a stakeholder in consolidation in some cases of public projects, c.f. Sections 5, 88 and 88 a) of the \cite{lca79}.} This further reflects how it is becoming increasingly natural to see land consolidation as an alternative to expropriation. It also flags how the relationship between expropriation and consolidation is now becoming an important topic in Norwegian land law.

In 2005 the Department of Agriculture made some comments in this regard, in connection with a revision of the current 1979 act that gave consolidation greater applicability in urban areas and with respect to implementing public plans.\footnote{See in particular Section 2 h-i) of the Land Consolidation Act 1979.} Some members of the preparatory committee had raised the concern that giving consolidation extended scope in this way would be problematic since it would encroach on expropriation law and effectively render consolidation a form of expropriation. The Department disagreed, commenting as follows.\footnote{See Chapter 3.3 of Ot.prp. no 78 (2004-2005), report to parliament from the Department of Agriculture regarding changes in the Land Consolidation Act 1979.}

\begin{quote}
The Department would like to point out that one of the main preconditions for consolidation is that a net profit is created for the land in question. This profit is then divided among the parties in an orderly fashion. Individually, the law also guarantees that no one suffers a loss, see Section 3 a). [...] \\ \\ In the Departments opinion, expropriation takes place on a different factual and legal basis. In cases of expropriation the public makes decisions that deprives the parties of economic value. The purpose then becomes to compensated them in accordance with Section 105 of the Constitution, not to increase the value of their land or the annual income they may derive from it.
\end{quote}

When preparing the new act, the Department of Agriculture reiterated this position but they did not reflect further on the question of the exact relationship between consolidation and expropriation. They observed, however, that changing the law so that expropriating parties could appear in consolidation cases was \emph{reasonable} since it would then be left up to the developer whether to make use of his permission to expropriate or to rely on consolidation instead.\footnote{See page 84 of Prop.101 L (2012-2013).} Indeed, it seems that in many cases, the well-organized and tightly regulated process of consolidation might be a more practical alternative for developers than the more fragmented rules and administrative bodies that come into play following traditional expropriation. 

However, it seems clear that the choice made by the expropriating party in this regard will tend to be even more important for the affected owners and rights holders. In particular, as the Department themselves made clear in the passage quoted above, it is an absolute precondition for implementation of any consolidation measures that alter the rights structure that they must serve to make the structure of ownership and use more favourable for \emph{everyone}. This, moreover, refers explicitly to the \emph{area within which consolidation takes place}, as stated in Section 3-3 of the new act. No similar rule is in place to protect the affected local area following expropriation. Moreover, the practices that have developed for dealing with consolidation cases are centred on the interests of the local owners and their land to an extent that is quite different from any procedure that is currently in place to facilitate development by use of expropriation.

For instance, the rule regarding expropriation that corresponds most closely to the no-loss rule  
requires merely that the benefit to private and public interest exceeds the disadvantages \emph{overall}, not locally and certainly not for individual local owners.\footnote{See Section 2 of the Expropriation Act 1959.} However, I also note that the strict consolidation rules do not serve as a restriction on \emph{what} kind of development should be carried out, only on \emph{how} it should be organized. The former question is left to the planning authorities, and the consolidation courts must always base their decisions on existing public regulation of property use.\footnote{In Section 3-17 of the Land Consolidation Act 2013 it is explicitly stated that the consolidation court cannot prescribe solutions that are not in keeping with such regulation. However, it is also made clear that the consolidation court itself can apply for necessary planning permissions on behalf of the owners and the land in question.}

Hence, if the public interest suggests a particular form of land use, the fact that a planning decision detailing development of such use is implemented through consolidation does not entitle the court to review the plans themselves, going against the public interest. But it does introduce an obligation, emerging at the time of implementation, to turn specifically to the interests of original owners and rights holders and to look for solutions that minimize the burden and maximizes the benefit for all the involved parties.

The rules that give the consolidation court authority to give directives of use are particularly relevant in this regard, and we return to them in Section \ref{sec:3} below. First we will present the consolidation process itself. It seems, in particular, that the procedural guarantees resulting from the fact that this process is organized as a tribunal are in themselves an important factor to consider when looking at consolidation as an alternative to expropriation.

\subsection{A Brief Presentation of the Consolidation Process}\label{subsec:21}

A consolidation case is usually initiated by an owner or a permanent rights holder.\footnote{See Section 5, Paragraph 1 of the Land Consolidation Act 1979.} The request for consolidation measures is to be directed at the relevant district consolidation court, one of the 34 district courts for land consolidation that have been set up by the King in accordance with Section 7 of the Land Consolidation Act 1979. The request is meant to include further details about the affected properties, the owners and rights holder involved, as well as the specific issues that consolidation should address. But the requirements in this regard are not usually interpreted very strictly and the district consolidation court will often take on quite some responsibility for further clarifying what the case should encompass, more so than in civil disputes.\footnote{References needed.} Even so, the court is entitled to reject the request due to technical shortcomings, following the same rules as those which applies to civil disputes.\footnote{See Section 12, Paragraph 2 of the Land Consolidation Act 1979, which refers to Section 16-5 of the Civil Dispute Act 2005}

If the court decides that the request is well-formed and that it includes sufficient detail to permit consideration of the substance, they go on to prepare public hearings, following the rules set out in Chapter 3 of the Land Consolidation Act 1979. These rules mirror those that are in place for civil hearings in general, including the duty to inform affected parties (Section 13), the parties' right to present their claims, and their duty and right to give testimony and provide evidence supporting it (Sections 15, 17 a) and 18). As in civil cases, the decision is usually only reached after at least one hearing in which the parties are present and permitted to contradict the evidence provided and the testimony given by other parties. However, unlike in civil cases, the main hearing typically takes place on the disputed land itself, and consists in mapping and clarifying the prevailing conditions aided by visual inspection. Moreover, a consolidation case will usually not take the form of a two-party adversarial process, but rather as a multi-party discussion where the court interacts with a large number of interested parties who may have a range of common as well as conflicting interests. Usually, consolidation cases involve at least 10 or more different parties, and in some cases there can be hundreds. In addition, it is quite common that the parties are not represented by legal council, but rather take an active part in the process themselves.\footnote{References needed.}

The request for consolidation will be the court's point of departure when assessing the case, but the court is not bound by the claims put forth in it, or by the claims put forth by the other parties. This again marks a differences with most cases of civil dispute. With a few exceptions explicitly listed in statute, the consolidation court may decided to use any measure that it deems suitable to ensure a favourable structure of rights and ownership for the future. However, there is some restriction placed on the court in that the measures taken must be regarded as \emph{necessary} in light of considerations based on the original request.\footnote{See Sections 26 and 29 of the Land Consolidation Act 1979.} So while the court should remain focused on the issues raised by the parties, it should be free to address these issues using the tools they deem most suited for the job. The consolidation court, in particular, is meant to be a general ``problem solver", more so than the ordinary civil courts.

When a decision is reached, the rules in Section 17 and 22 of the 1979 act ensure that the parties are notified and that the decision is presented and argued for in keeping with the rules of the Civil Dispute Act 2005. The appropriate form of the decision will depend on its content. A regular civil ruling is the form used for decisions that only involve ascertaining the boundaries between properties, while a special ``consolidation decision" is the form relied on to implement apportionment and directives of use. The difference becomes clear as soon as we consider the appeals procedure; while civil rulings are dealt with by the regular courts of appeal, the consolidation decisions can only be appealed to one of 4 designated consolidation courts of appeal.\footnote{See Section 61 of the Land Consolidation Act 1979.} 

In the latter case, the procedural rules remain largely the same in the consolidation court of appeal, meaning also that there is a new assessment of all aspects of the case.\footnote{See Section 69 of the Land Consolidation Act 1979.} When the case is concluded here, however, it can only be appealed on the grounds that it is based on an incorrect understanding of the law, or that procedural mistakes have been made. In this case, the ordinary appeal courts have authority, with the Supreme Court being the last instance of possible appeal.\footnote{See Section 71 of the Land Consolidation Act 1979.}

From the brief overview of the process given above, we see that consolidation cases are different from other civil cases in that they have fundamentally different scope. A consolidation case is not primarily centred on deciding the merits of individual claims, but rather at introducing structures of ownership and rights that will prove favourable in general. In this respect the process has an administrative character. However, the fact that it is organized more or less like a regular civil dispute means that the protection of each affected party, and the influence of the local rights holders as a group, is much stronger than what would tend to be the case if these decisions were made by regular administrative bodies.

Given this context of arbitration, it is not surprising that the judges appointed to the consolidation courts are required to have a special skill set, different from that of regular civil law judges. In fact, consolidation judges are required to have successfully completed a special masters level degree in consolidation, which is not a law degree at all but a separate form of education.\footnote{See Section 7, Paragraph 5 of the Land Consolidation Act 1979. The degree in question is currently offered only at the Norwegian College of Life Sciences and Agriculture.} 

The consolidation court also relies on the participation of lay judges who sit alongside the specialist judge.\footnote{See Section 8 of the Land Consolidation Act 1979.} These judges are appointed by the specialist judge from a committee of laymen that are elected by the local municipalities in accordance with Section 64 of the Courts Act 1915.\footnote{See Section 8 of the Land Consolidation Act 1979.} In the district courts, the specialist judge usually sits with two appointed lay judges which he chooses himself from among members of the the relevant local committees. In the court of appeal, the specialist sits with 4 laymen, and in complicated cases 4 laymen may also be called on in the district courts, but only if one of the parties requests it.\footnote{See Section 9, Paragraph 2 of the Land Consolidation Act 1979.} To the extent possible, the appointed laymen should have special knowledge of the issues raised by the case, but they are drawn from the general population.\footnote{See Section 9, Paragraph 5.}

Summing up, we observe that the consolidation process has both administrative and adversarial characteristics. While the content and scope of the court's decision will often have an administrative flavour and is not primarily directed at settling any specific dispute, the process is judicial. Hence everyone is entitled, and to some extent even \emph{obliged}, to have his voice heard and to partake in the process. Moreover, while the process is guided and overseen by the court, it is fundamentally based on considerations arising from the interests of the parties. However, this interest is always interpreted in light of prevailing notions of what counts as favourable and rational property use. Importantly, in relation to this latter assessment, the court will look beyond the interests of the individual owners. The court will pose the question with regards to the use of the land as such, drawing on its understanding of the relevant economic, social and political conditions.\footnote{References needed.} But the decisions made are always prepared using information that is retrieved and discussed in public hearings, so the affected parties will take part in discussions that may also address more overreaching concerns about the form of land use that should be regarded as favourable for the area in question.

To flag the dual nature of the consolidation process it is tempting to designate it as a process of judicially structured \emph{deliberation}. The final decision-making authority is granted to the court, but the court is required to act on behalf of the rights holders, in the best interests of their land, and based on the information that they themselves provide. This particular form of decision-making based on multi-party deliberation is interesting in its own right, as it provides a template for management of land that seems capable of catering both to the idea of public oversight and control as well as to the idea of local participation. In addition to this, it seems to be a form of land management that might be especially suitable as a means to implement concrete projects undertaken in the public interest, particularly when these would otherwise appear to adversely affect individual land owners and local communities.

This is of particular interest in mixed economies such as seen in Norway, where decisions regarding development and use of property are typically made by the public but carried out by private property owners. In many cases, implementation of public policy requires some form of reorganization of ownership and rights structures, the most common being a pooling of resources from many different owners. Such processes have tended to be implemented rather crudely, by displacing the original owners in favour of commercial companies who serve as state agents. This relies on the use of expropriation, and it typically completely deprives the original owners of any chance to take part in the future development of the land. The land consolidation rules allows us to consider alternative means of implementation in such cases. 

They allow us to ask whether a more measured approach might be sufficient, allowing the original owners to retain their rights, but restructuring them using consolidation mechanisms. This question is particularly interesting to consider due to the high levels of tension often associated with cases of commercial expropriation, where companies operating for profit benefit from implementing public plans. Critics argue that such uses of expropriation are both unfair in themselves and also destabilizing in that they raise doubts abut the true motives behind specific acts of public planning.\footnote{References needed.} In particular, it seems that in a system of land management where development is organized in this way commercial companies will have much to gain from attempting to exert influence over the planning process, particularly if they can also succeed in being granted permission to expropriate property rights that they would otherwise have to acquire on an open market. However, to counter critics it may appear easy to argue from necessity, by pointing out that the system is the best known alternative for efficient and rational economic development in a system based on public control over planning and private rights to property.

In relation to this debate it seems that the consolidation procedure takes on particular relevance. It may point to an alternative, a system of public-private development where the original owners and local communities are better integrated into the process. Moreover, it allows us to introduce an additional conceptual layer between the planning stage and the implementation step, a layer of management devoted to translating public plans into concrete action by orderly restructuring of existing ownership patterns. This, in particular, might be a layer of administration that deserves more attention and more fine-grained tools than those currently offered in systems relying on expropriation. Clearly identifying such a consolidation layer in property management might also make for a cleaner delineation between commercial implementation on the one hand, governed by the market, and public planning on the other, governed by administrative law and political bodies. 

In the next section, I argue that Norwegian consolidation law already include tools that make it possible to view consolidation in this light. The rules that I believe warrant this conclusion are the rules relating to joint use directives, briefly mentioned above. I present them in more detail below, noting that recent changes in consolidation law give them wider applicability in relation to concrete development projects. 

\subsection{Joint use, joint action and joint investment}\label{sec:3}

In accordance with Section 2 c) of the Land Consolidation Act 1979 the consolidation court can give directives regarding the use of land which involves more then one property. Typically this will target land or land rights that are owned jointly or for which some form of shared use has already been established. However, if the court finds that there are \emph{special reasons} for giving joint use directives, it can do so even if there is no prior connection between the different rights and properties in question.\footnote{See Section 2 c), Paragraph 2 of the Land Consolidation Act 1979.} Traditional examples include directives for the shared use of a private road which crosses several different properties, or regulation of hunting that takes place across property boundaries.

The joint use rules emerged as an alternative to apportionment of jointly owned property, a more subtle and less invasive measure that could often give rise to the same positive effect as a full division of ownership, but without leading to unwanted fragmentation of control and use of property. Hence in the now repealed Land Consolidation Act 1950 it was stated that joint use directives should be the \emph{primary} mechanism of consolidation, and that apportionment should only take place if such directives were deemed insufficient to reach the goal of creating more favourable conditions for the use of the land.\footnote{References needed.} In the 1979 act the two mechanisms were formally put side by side, but in cases that are motivated by a specific planned use of the land in question, directives will still be the main tool relied on by the court.

Moreover, there has been a gradual increase in the willingness of the court to rely on use directives to facilitate \emph{new development} on the land, not just as a means to regulate an existing activity. In parallel with this development, the consolidation court has gradually come to take on cases that pertain to organization of land use that was previously thought to lie outside its area of competence. The more restricted view on use directives and on the function of consolidation in general is reflected in the way the 1979 act lists a range of different concrete circumstances in which such directives might be applied.\footnote{See Section 35 of the Land Consolidation Act 1979.} The list is not understood to be exhaustive however, and the courts have gradually come to feel less deterred by it and more willing to consider new types of cases.\footnote{References needed.}

Hence in the new act of 2013, the list is replaced by an explicit general rule which makes it clear that the  consolidation courts have the authority to give directives whenever they regard this to be favourable to the properties involved.\footnote{See Section 3-8 of the Land Consolidation Act 2013.} In addition to this, the new act also introduces a general rule which gives the court authority to \emph{set up} systems of joint ownership when a joint use directive is deemed insufficient.\footnote{See Section 3-5 of the Land Consolidation Act 2013.} Hence in the new act apportionment and pooling of property is on equal footing, although a priority rule is introduced for the latter; pooling will only be considered if directives of joint use are regarded as an insufficient means to ensure more favourable conditions. Moreover, the new act maintains the principle that directives regarding the joint use of land for which there are no existing joint rights can only be given if there are special reasons.

This requirement is not intended to be very strict, and the Ministry of Agriculture was initially inclined to remove it. However, it was eventually decided that it should be kept in order to flag that there two distinct questions that arise in such cases. The court must first consider the question of whether or not joint use is in itself desirable, before it goes on to consider how to best organize such use.\footnote{For a discussion on this see page 140-141 of Prop. 101 L (2012-2013).}

In addition to giving directives prescribing how joint use is to be organized, the consolidation court may also give rules compelling the owners to take joint action to help facilitate better realization of the potential inherent in the land. Rules to this effect were first introduced in the 1979 act, in Section 2 e) and Sections 42-44. These rules only pertain to joint action by property owners (see Section 34 a), and they have wider scope in relation to specific case types (Sections 43 and 44). Following the new act, however, the consolidation courts will have authority to prescribe joint action also for right holders, and the special rules listing concrete circumstances will be replaced by a general joint action rule.\footnote{See Section 3-9 of the Land Consolidation Act 2013.} This broadens the scope of these rules in accordance with the general spirit of the new act. Indeed, when they commented on this change in the law, the Department noted that the rules in question have been widely used following their introduction in 1979 and that applying them is now one of the core responsibilities of the consolidation court.\footnote{See page 146 of Prop. 101 L (2012-2013).}

I note that joint action directives can include prescriptions for joint investments.\footnote{See Section 3-9 of the Land Consolidation Act 2013.} On the one hand this means that such directives can be used to facilitate capital intensive new development, but it also raises the question of the extent to which it is legitimate to rely on compulsion in this regard, directed towards individual owners of property. The extent of the joint actions and investments required to undertake development projects can easily become quite burdensome for these individuals, and this is especially likely to arise as a concern in cases where the land lends itself well to large- scale commercial development.

The 1979 act attempts to resolve this in Section 34 b) and in Section 42. The former states that if joint actions or investments may come to involve``great risk", the court must set up two \emph{distinct} organizational units to undertake it. First, the rights needed to undertake the scheme will be pooled together and managed by an owners' association, and then, to undertake the scheme itself, a cooperative company structure will be set up on behalf of the owners. Hence the risk is diverted away from the individual owners onto a company controlled by them. This company will be entitled to any potential profit from the scheme, but it will also be required to pay compensation to the owner' association on terms established by the parties themselves, with the help of the court.\footnote{See Section 34 b) of the Land Consolidation Act 1979.} Moreover, the owners are entitled to shares in this company proportional to their share of the relevant rights in the land, as determined by the consolidation court. An owner is not obliged to take part in the undertaking by acquiring such shares, but he will benefit from membership in the owners' association regardless of whether or not he chooses to do so.

After this brief survey of the rules, I conclude that the land consolidation courts in Norway already have all the tools they need to organize development projects on behalf of local owners. Moreover, the process of consolidation means that they must do so in a way that enables the original owners to retain considerable decision-making power as well as the right to any commercial benefit that may result from the development. Hence, the rules currently found in consolidation law adds weight to the claim that and consolidation might point to an alternative and possibly fruitful way of implementing development projects in a system which presupposes that development takes place through commercial initiatives on the basis of public  planning and control. In particular, the system already provide the tools needed to organize large-scale development even when it requires considerable reorganization of land rights and diversification of risk. Consolidation may therefore become an alternative way of pooling together fragmented rights for the purpose of development, without displacing the original owners in favour of commercial companies who have no prior connection to the local community in which development takes place.

In addition to this, I observe that the consolidation rules also point to a form of implementation that will allow the public to exercise \emph{more} extensive oversight and control. Not only is the position of the original owners much better protected under this system, but it also greatly \emph{curbs} the power and influence of commercial forces with no prior connection to the land. Hence, it must be expected that implementation through consolidation is better suited also to serve the social and political aims which originally motivated the underlying planning decision. Indeed, commercial development through consolidation give the public a greater say in the implementation stage; after all, the development is organized as a cooperative, and the company structure is set up and regulated by the courts who is obliged to consider also the public and societal interests in land use.

In addition to this, after the new act takes affects, both planning authorities and commercial developers may take up a role as formally recognized parties in the consolidation process. This seems particularly useful in connection with large scale industrial development, as it might otherwise be hard to implement such projects successfully. In these cases, then, the consolidation system sets up an arena for interaction and deliberation between the three main groups of stakeholders: the public, the local owners and the commercially motivated developers. Such an arena is so far missing at the implementation stage of big development projects, while recent controversies regarding expropriation suggest that it might come to serve an important function.

It remains unclear to what extent the consolidation rules will actually be used in this way. But as we will see in the next section, consolidation is beginning to emerge as an important means for organizing local hydro-power development. On the theoretical side, then, it is also unclear to what extent original owners may \emph{demand} that the rules are put to use. For instance, may an owner request consolidation to prevent a permission to expropriate from being implemented? It will be interesting to see how the Norwegian legal systems will deal with this and related questions, after the new act takes effect in 2016.

I conclude this section by addressing a new special rule that has been included in the new act, and which is specifically targeted at the planning authorities, encouraging them to make use of consolidation to achieve  greater fairness in public planning. The new rules are contained in Chapter 5 of the new act and they target benefit that arises from planning in cases when the benefit appears to fall disproportionately on some owners. Such cases of ``windfall" benefit due to public plans are often flagged as problematic, and they arise with particular frequency in systems based on commercial implementation. For example, if one parcel of land is designated for housing and some neighbouring land is designated as a playground, it might easily come to be seen as unfair that a considerable financial benefit falls to the owner of the land designated for housing, while the playground owner is left with virtually nothing.

Following a change of the Consolidation Act in 2006 which has been further extended in the new act, the law now makes it possible for the planning authorities to decide that apportionment of the \emph{benefit} arising from the plan may be carried out by the consolidation courts.\footnote{See Section 3-30 of the Land Consolidation Act 2013 and Section 12-7 nr. 13 of the Planning and Buildings Act 2008.} When doing so, the consolidation court will follow the same procedure as in other cases, and it will allocate the benefit arising from the plan based on an assessment of the development potential of the different parcels. Importantly, the court will consider this question independently, and the decision will not be based on the particular manner in which the plan dictates that development is to be carried out. For instance, if the land used for the playground could just as well have been used for housing, the court may decide that the rights to housing development is to be shared equally between the two properties. On the other hand, if there is some independent reason why the playground property is not suited for housing, the court will reduce this property's share in the housing development correspondingly.

The court can implement solutions such as this more effectively and rationally by applying the other tools that it has available. For instance, if the the owner of he playground is entitled to an equal share in housing, then apportionment can be used to actually provide him with such a share, trading it for a corresponding share in the playground. However, if such material reallocation of development rights prove unfeasible, the new act also opens up for a solution where the benefit sharing is implemented using financial compensation.\footnote{See Section 3-32 of the Land Consolidation Act 2013.}

To sum up, directives of use rules are highly versatile and may be used to organize extensive projects of land development on behalf of original owners. This form of development makes it possible for original owners to maintain their interest in the land, it can prevent the need for expropriation, and it may give the public a greater opportunity to exert influence and control over how their planning decisions are implemented in practice. In the next section, I consider in depth the particular case of hydropower, where the consolidation courts have recently started to make use of a wide arsenal of its tools to ensure that development can be carried out in this way. I think this case-study sheds further light on consolidation as an alternative to expropriation, and further strengthens the argument that directives of use issued by a consolidation court can in many cases obliterate the need for depriving local people of their resources to implement public development plans.

\section{Compulsory participation in hydropower development}\label{sec:lch}

In this section, I look at four recent cases in detail, all of which involved directives of use for hydropower development by original owners. The waterfalls dealt with in these cases are all located in the county of \emph{Hordaland}, in south-western Norway. Three of the cases involved small-scale hydro-power which some of the owners wanted to develop themselves, while the fourth was a case when the owners were also considering a development plan which would involve cooperation with an external energy company. The cases are particularly interesting because we have access to data on how the process of consolidation, and the outcome, was perceived by the owners themselves. Interviews were conducted and used in a recent master thesis on land consolidation which is devoted to the study of how consolidation measures is now increasingly being used to facilitate hydropower development \cite{master}.

In the following, I first present each case separately, focusing on those issues that were raised regarding how to organize development, the solutions prescribed by the court, and the subsequent reception among the parties. I then assess this from the point of view of developing a better understanding of compulsory cooperation as an alternative to expropriation. I conclude with some unresolved questions, particularly regarding those situations when the court is called on to resolve disagreement regarding how the development itself should be organized. These are the cases when the relationship between consolidation law and other legal frameworks, such as company law, planning law, and water law, becomes pressing, and there are many unresolved questions.

\subsection{\emph{Vika}}

The case was brought before the consolidation court in 2005, by owners who all agreed that hydropower development should be pursued.\footnote{Haugalandet og Sunnhordland jordkifterett, case no 1210-2005-0014.} The owners disagreed on how to organize the owners' association, and on how the shares in this association were to be divided among the different properties involved, 15 in total. The main principle was agreed upon from the start, however, namely that the owners would rent out their waterfall to a separate development company which every owner would have a right (but not a duty) to take part in. 

The parties in \emph{Vika} were highly involved in the consolidation process, and the statutes for the owners' association were based on suggestions made by the owners themselves. The main point of disagreement concerned how the shares in this association should be allotted, a question that was made more difficult by the fact that some owners benefited from old water-mill rights in the river. In the end, the consolidation court landed on the view that these rights were tied to the form of use relevant at the time they were established, and did not regard them as having any financial value. Hence these rights were extinguished without compensation, as provided for in Sections 2, 36 and 38 in the Land Consolidation Act 1979.

There was also some disagreement about whether the number of votes in the owners' association should be tied to the number of shares belonging to each owner, or if the owners should simply be allotted one vote each, irrespectively of their share in the waterfall. The consolidation court went for the first option, but the way in which they allotted shares in the owners' association deserves special mention. In particular, the court decided to take into account that some additional water entered the waterfall from smaller rivers where only a sub-group of the owners had waterfall rights. These owners' share in the association was increased accordingly, and this is surprising in light of Norwegian water law, as water rights are otherwise not tied to where the water comes from, but arises solely from the rights one has in the waterfall itself. 

The statutes of the owners' association also contains a second interesting provision, based on a suggestion made by the owners. It is a rule to the effect that all rights in the association are to be tied to the larger agricultural properties that give rise to them, and that they can not be divided from these properties and transferred to new owners separately. In Norway, such division of agricultural land would in any event require permission from the local municipality.\footnote{See Section 12 of the Land Act 1995.} In recent years, however, this protection of farming communities has grown weaker in practice, and it was the view of the owners in \emph{Vika} that a dissociation of water rights from underlying agricultural land should be forbidden altogether.

According to \cite{master}, interviews conducted with the parties demonstrated that a general consensus had developed whereby the land consolidation procedure was seen as a success. It allowed for an orderly and fair decision-making process regarding the conflicts that had arisen, and it was based on continuous interaction between the owners and the court, where everyone felt they had been given an opportunity to have his voice heard. Initially, tensions among the owners had been high, but the consolidation process had served to alleviate them. Some owners also pointed to the fact that the main hearing had been physically conducted in the local community, in a meeting hall that was familiar to the owners. This also gave them a feeling that they were meant to actively partake in the decision-making process. 

When the interviews were conducted, some 5 years after the case was concluded, the owners also appears to agree that the association was working as intended and that the climate of cooperation among the owners was good. The hydro-power scheme itself had been completed in 2008, yielding an annual production of around 15 GWh per year, providing enough energy for around 700 households. Moreover, following the experience of land consolidation, a culture of deliberation towards consensus had developed among the owners, and great emphasis had subsequently been placed on attempting to find common ground and to reach agreement on important issues. This was reflected, for instance, in the fact that the owner who contributed the land for the power station was given a generous annual fee, in addition to his compensation as a waterfall owner. According to \cite{master}, this fee exceeds what he would likely get if this decision had been left to the discretion of the consolidation court. Hence it reflected a premium that the owners were now willing to pay to ensure agreement and a continued good climate for cooperation.

All in all, we agree with \cite{master} that the case of \emph{Vika} serves as an example of how land consolidation can empower local communities and may enable them to embark on substantial development projects.

\subsection{\emph{Oma}}

The second case we will consider is the case of \emph{Oma}, which was brought before the courts in 2006.\footnote{Nord- og Midthordaland jordskifterett, case no 1200-2006-0015.} In this case there were four involved properties. The owners of three of them, $A,B$ and $C$, wanted to develop hydro-power, while the fourth, owner $D$, was opposed to the development. Rather than attempting to expropriate the necessary rights from owner $D$, owners $A,B$ and $C$ took the case to the consolidation court. They argued that development would benefit all the properties involved, and also pointed out that a more restricted project, which would not make use of owner $D$'s rights, would be less economical. Hence in their view, the consolidation court should compel $D$ to cooperate in a joint scheme. Owner $D$ protested, arguing that the project would not economically benefit him, and that it would also be to the detriment of his plans to build cottages for holiday dwellers in the same area.

The case of \emph{Oma} differs from that of \emph{Vika} in that the question of whether it was appropriate to use compulsion was more prominent. In particular, this aspect came up already in relation to the question of whether or not hydro-power development should be pursued at all. As we discussed in Section \ref{sec:3}, the fact that some owners do not desire development does not prevent the consolidation court from putting directives in place to facilitate it, but the courts often exercise restraint in such cases. In \emph{Oma}, however, the court agreed with the majority of the owners argued that an owners' association with compulsory membership should be set up. In doing so, the court relied on Section 2 c) of the Land Consolidation Act 1979. To justify the use of compulsion against $D$, the court first observed that joint development of hydro-power would benefit all the properties in question, including $D$. Then they commented specifically on owner $D$'s plans for building of cottage homes, noting first that he was unlikely to be given planning permission, and secondly that hydro-power would not in any event adversely affect such plans in any significant way. Moreover, the court noted that while owner $D$'s rights were relatively minor, they were quite crucial for the profitability of the project, particularly because owner $D$ controlled the best location for the construction of a dam to collect the water used in the scheme. Overall, the court's conclusion was that a joint hydro-power scheme would be a better option for everyone than a project that did not include owner $D$'s property.

The question then arose as to how the shares in the owners' association, and the right to rent that would go with it, should be divided among the owners and their land. In regards to this question, the court departed significantly from one of the basic principles that have been entrenched in Norwegian water-law since the early 20th Century. The principle in question states that no right to hydro-power can be derived from being in possession of land suitable for the construction of dams or other facilities necessary to exploit the waterfalls.\footnote{The principle was is reflected in Supreme Court decisions as early as \emph{Herlandsfossen} and \emph{Drammenselven}, Rt. 1922 p. 489 and Rt. 1923 p. 185 respectively, and has been maintained consistently ever since.} But the land consolidation court broke with this principle in the case of \emph{Oma}, deciding instead to set the value of the land designated for construction of a dam and a power station to represent $6 \%$ of the total value of the rights that went into the owners' association. The proportion of financial benefit and decision-making power awarded to the unwilling owner $D$ thus increased accordingly, since these right were all held by him. In fact, his share went from $1.75 \%$ to $7.75 \%$, so the consolidation process itself led to a situation where he would have a far greater incentive for supporting the development. In many ways, the decision in \emph{Oma} was more to the benefit of owner $D$ than any other among the involved parties. If the rights in question had been expropriated, for instance, he would be given next to nothing in compensation and would lose his rights forever. Instead, the solution prescribed by the consolidation court gave him a lasting and substantial interest in local hydro-power.

According to \cite{master}, interviews with the parties shows how the process and outcome of consolidation in \emph{Oma} served to create a much better climate for further cooperation among the parties. Indeed, when the interviews where conducted, 4 years after the courts' decision, owner $D$ had changed his mind and was now in favor of the development. Moreover, he had also decided that he wanted to take part in the development company. He was not obliged to do so, but his right to take part was encoded in the deal with the development company, as detailed in the statutes of the owners' association and in keeping with Section 34 b) no 3 of the Land Consolidation Act 1979.

The owners all reported that the consolidation process had been very successful and that the court had listen to them, allowing everyone to have their voices heard. Moreover, some owners reported that the court had cleverly maintained a ``birds eye view" on the best way to develop the land in question, ensuring both long terms benefit to all involved properties as well as creating an improved climate for cooperation and mutual understanding. The consensus was that making concessions to owner $D$ was appropriate and had been in the interest of all the involved parties. In 2011 the hydro-power project in \emph{Oma} was completed and today its output is roughly 3 GWh per year.

We think the case of \emph{Oma} serves as a good illustration of how consolidation can be an effective instrument for facilitating locally controlled development, also in cases when this requires the use of compulsion against some owners. Interestingly, in this case the successful outcome appears to be partly due to the fact that the consolidation court actively used its discretionary powers when deciding how to organize joint use. This power allowed them to deviate from established rights-based legal doctrine and adopt a more context-dependent approach, pursuing solutions that suited the situation better. Interesting legal questions arise in this regard, particularly regarding the competence that the consolidation court has in such cases, and the extent to which decisions can be made subject to review by the normal courts on the basis that they do not follow established principles and practices. For instance, one may ask what would have happened if the majority owners in \emph{Oma} had appealed the decision to the regular courts on the basis that $D$ was awarded too many shares in the owners' association. Would this be regarded as a question of the court's interpretation of the law regarding the owners' \emph{rights}, or would it be regarded as a discretionary decision regarding the best way to organize development? In the first case, the decision would almost certainly have been overturned on appeal, but in the latter case it would likely be beyond reproach.

A second interesting question that arises is whether or not consolidation can work as well as it did in \emph{Oma} in cases where conflicts run more deeply, or where the parties favoring development are a minority among the owners. The next two cases we consider shed some light on this issue.

\subsection{\emph{Djønno}}

This case was brought before the courts in 2006, by a local owner $A$ who wanted to develop hydro-power in a small river crossing his land, the so called \emph{Kvernhusbekken}.\footnote{Indre Hordaland Jordskifterett, case no 1230-2006-0010.} This owner wanted the court to help him implement a hydro-power project, by compelling the other owners, $B, C$ and $D$, to rent out their share in the necessary rights on terms dictated by the court. The starting point for the other owners was that they did not want any hydro-power development at all, and they were not willing to rent out their rights to owner $A$ or any other developer. There was also a dispute regarding the ownership of the waterfall rights, with $A$ believing initially that he controlled a large majority. It soon became clear that this was not the case, and before the main hearing the parties agreed that the waterfall in question was owned jointly with shares divided according to each owners' share in unconsolidated farming land. Owner $A$'s share in these rights did not amount to more than $5 \%$, so his own financial interest in hydro-power was in fact limited compared to the owners who opposed development.

On the other hand, the rights needed for the necessary physical constructions were predominantly held by owner $A$ alone, and $A$ maintained his position that the court should use compulsion to allow him to go on with his plans. The court agreed that hydro-power would be rational use of the waterfall, and they initially assessed the case against Section 2 e) regarding compulsory joint undertakings. A decision made on the basis of this provision would allow the court to give more concrete directives regarding how the hydro-power development should be carried out, but in the end the court held that this would place too much of a burden on the owners opposing hydro-power. Hence they chose to decide the case on the basis of Section 2 c), as in the other cases we have considered. By doing so they also restricted the scope of their decision to the establishment of an owners' association that would be responsible for renting out the rights. The court would not consider the question of deciding on a concrete scheme.

The model used for the owners' association was similar to the one the court adopted in \emph{Oma}. This included an adjustment of the rights in the owners' association reflecting the special importance of land needed for physical constructions. In total, these rights were estimated at a value corresponding to $6 \%$ of the shares in the association. Since these rights were all held by owner $A$ alone, his share in the association was doubled. In addition to this, owner $A$ purchased the shares from owner $B$, so that his total share ended up amounting to $22 \%$. Still, for the majority of stakeholders, membership in the association was imposed by the consolidation court against their will.

The wording of the statutes for the association apparently attempts to take into account that it would be run by a majority of unwilling shareholders. The wording used is different from that used in the other statutes, and it is stated in very clear terms that the association is going to rent out the rights in the waterfall such that hydro-power can be developed. In \emph{Oma} and \emph{Vika}, on the other hand, the statutes merely state that this is the \emph{purpose} of the association, leaving the shareholders with greater freedom to determine whether or not to go through with development.

More generally, it seems that in the case of \emph{Djønno} the court attempted to facilitate development not so much by trying to make the owners more positive towards development, but rather by giving the proponent more power, providing him with a starting point which would make it easier to later enforce concrete hydro-power plans.

In interviews, those who were compelled to take part in the association against their will expressed dissatisfaction and surprise at the result. Moreover, while the association had ostensibly tried to be loyal to the wording of the statutes, and had looked for partners who might be interested in developing hydro-power, there had been no willingness among the majority to engage actively with this work. No deals had been made, no separate development company had been set up, and the conflict among the owners was ongoing. As of 2011, owner $A$ was still pushing for development on terms that were unacceptable to the other owners. Hence while the case of \emph{Djønno} is an example that consolidation can be used even when it involves compulsion against the majority of owners, it also serves to illustrate that the chance of a successful outcome may then be more limited.

The question arises as to why this is so, and how such cases will be dealt with by courts in the future. According to owner $A$, the problem was that the directives of use were not specific enough and that they should not have been restricted to merely setting up an owners' association for renting out the rights. In this case, more was needed. The court should actively engage also with the question of how the development company should be organized, and at least give guidance as to \emph{who} should be set with the task of carrying it out. Among the majority owners, on the other hand, the feeling appears to have been that the development in question, which they would be required to partake in against their will, was more or less doomed to fail already from the start.

This reflects two interesting viewpoints regarding such cases. Indeed, it seems reasonable to assume that unless one is prepared to see an increase in the use of compulsion, compulsory cooperation will only work when at least a basic agreement that development should take place can be established among the majority of the involved owners.

\subsection{\emph{Tokheim}}

This case was brought before the consolidation court in 2008, by the owners of \emph{Tokheimselva}.\footnote{Indre Hordaland jordskifterett, Case no 1230-2008-0020.} The five involved owners all agreed that development should take place, but they disagreed about how it should be done, and about the proportion of each owners' share in waterfall. Some owners argue that development should be organized by the owners themselves, but other owners thought it would be best to rent out the rights to an external developer. The case was further complicated by the fact that the waterfall in question was so big that it would be possible to develop hydro-power that would require transferral concession pursuant to the Industrial Concession Act 1917, a concession that can only be given when the purchaser is a company where the State controls at least $\frac{2}{3}$ of the shares. 

Like the precious cases we have considered the consolidation court eventually based its decision on Section 2 c) of the Land Consolidation Act 1979, setting up an owners' association such that each owner was allotted a share in accordance to the rights that the court found he had in the waterfall. Unlike the previous cases we have considered, there was no adjustment made for land that would be needed for physical constructions. However, the statutes state that owners will be entitled to a lump sum estimated on the basis of the damages and disadvantages that a concrete hydro-power project will bring. This also marks a departure from established practice in expropriation law, where it has been a long established principle that owners can be compensated on the basis of \emph{either} the value of their waterfalls \emph{or} the damages and disadvantages caused by the project, not both.\footnote{See for instance the case of \emph{Vikfalli}, \cite{vikfalli71}.} 

In other respects, the statutes for the owners' association follow the same model adopted in the previously considered cases. They do not, however, resolve any of the controversial questions regarding how development should be carried out, and the question of the extent to which interested owners should be given the opportunity to develop the resource themselves. This was the issue that the main conflict in the case centred on, and the consolidation court explicitly decides not to implement any solutions in this regard. In particular, the statutes of the owners' association explicitly provides separate rules that cover both the case that a group of owners undertake development themselves, and the case that development is carried out by an external company. 

In interviews, the owners expressed that they were happy with how the case was dealt with by the court. Everyone appears to have been heard, and the owners' association was set up in consultation with the parties. However, the main issues were still unresolved as of 2011, and this was felt by the owners as a major shortcoming of the outcome of consolidation. Some of the owners expressed criticism against the court for not engaging more actively with what appeared to be the most pressing issues.

The case of \emph{Tokheim} serves to illustrate that established practices of consolidation, while being well received and understood by local owners, face some new challenges in relation to hydro-power, challenges that consolidation courts might be reluctant to take on. It seems that the court in \emph{Tokheim} felt that they were not in a position to assess the question of what kind of development would be best, and it also seems that they were wary of expressing any opinion about the legal status of a project led by local owners, in relation to concession law. They did not, in particular, form an opinion about whether it would be possible for local owners to carry out their own large scale development, in a waterfall that might otherwise be subject to the provisions set out in the Industrial Concession Act 1917.

It remains to be seen whether such an agnostic attitude can be maintained by the consolidation courts as local owners increasingly turn to them for help in resolving disputes regarding hydropower. Moreover, it will be interesting to see how the new Land Consolidation Act 2013 will influence case law in this area. It seems that a case like \emph{Tokheim} could benefit from the court taking a broader view, possible even including public bodies as parties in the case, as will become possible when the new Act takes effect. In this way one could perhaps have hoped for a more conclusive outcome, a solution that gave sufficient consideration both to the public interest and the interests of local owners.

\section{Assessment and Future Challenges}\label{sec:lca}

The concrete cases that I discussed in the previous section shows, in my opinion, that the system of land consolidation is well suited as an alternative to expropriation in the context of hydropower development. At the same time, the cases suggest that the land consolidation courts may find it hard to deliver effective directives of use in situations when the different stakeholders disagree fundamentally about how the water resources should be managed. In addition, one may question the effectiveness of land consolidation courts in contexts when rules and regulations from other areas of law come into play. It seems, in particular, that the land consolidation courts may be cautious about implementing solutions that they fear will raise questions in relation to the special legal provisions that regulate the form of economic development that their directives aim to facilitate.

In so far as the sector-specific rules disadvantage owners and benefit external commercial interests, as is the case for hydropower development, one may rightly fear that the land consolidation courts will become impotent in situations when powerful market players enter the scene. It may be considerably easier to strike a fair balance between the interests of local farmers of comparable economic and political standing, then to do the same when one of the stakeholders is a partly state-owned power company that is accustomed to expropriating the property rights that it desires.

Paradoxically, the impotence of the land consolidation courts may be enhanced by the fact that they are not authorized to make use of appropriate forms of compulsion against owners, on pain of interfering too much in property as an individual right. This, in particular, threatens to undermine the effectiveness of the land consolidation court as an alternative to expropriation, making it possible to argue that the public interest in development can not be sufficiently accommodated through the use of consolidation measures. 

In fact, there is some evidence to suggest that land consolidation law might offer \emph{too} much protection to owners in order to circumvent such objections. One example is the Supreme Court case of 
{\it Holen v Holen}, concerning a quarry owned by a local farmer and landowner.\footcite{holen95} In order to continue extracting his minerals, the owner of the quarry would have to interfere with the property of a neighbouring owner, who was using his land for more traditional forms of agriculture. This owner was unwilling to reach an agreement with the quarry owner, so the latter brought a case before the land consolidation court. The court noted that it would be possible to reach an accommodation that would benefit both parties, and issued directives of use that would allow the quarry to continue its operations.

The directives involved giving the agriculturally minded farmer a replacement property, to make up for his loss of the property needed to access the minerals. However, the consolidation court also noted that the quarry would, in the future, also be likely to extract minerals that belonged to this owner. For the minerals as such, awarding replacement property made little sense, so the court decided that the minerals should still belong to the previous surface owner (who had no interest in extracting them). However, a directive of use was issued that gave the quarry owner a right to extract these minerals, provided he paid market value to the owner. 

Hence, not only was the farmer awarded replacement property for agricultural purposes, he was also granted a share of the benefits that would result from the continued operation of his neighbour's quarry. This, it seems, was clearly beneficial to his property, economically speaking. The owner himself, however, objected to the arrangement, since he was opposed to the quarry as such. The Supreme Court found in his favour. Interestingly, this was not because they sanctioned his right to oppose the continued operations of the quarry, or because they thought the replacement property or the payment model was inappropriate. Instead, the Court held that the right to extract the farmer's minerals could not be transferred to someone else, even if the farmer was ensured payment. This, the Court held, was a compulsory measure that fell outside the scope of use directives in land consolidation.

The perspective underlying this decision is interesting, because it underscores a reluctance to use land consolidation in what would otherwise be a fairly typical expropriation scenario. As such, it also raises doubts about the feasibility of proposing land consolidation as a practical alternative for such scenarios. However, {\it Holen v Holen} was decided in 1995, and as I have already mentioned, the legislature has signalled a shift in the law in recent years, by explicitly facilitating the use of land consolidation as an alternative to expropriation in certain circumstances. This has been criticized, however, by scholars arguing that private property rights receive a more adequate form of protection when normal expropriation procedures are observed. In light of earlier case law, this criticism must be taken seriously, also a possible formal objection against awarding the land consolidation courts increased powers of compulsion. Today, the exact relationship between land consolidation and expropriation law, including the constitutional property clause, appears to be an increasingly relevant open question that awaits further clarification in case law. 

I would like to stress, however, that I do not agree with those who argue that land consolidation offers less protection to owners than administrative expropriation. The property protection offered in the context of land consolidation is quite different, but not necessarily weaker. This, moreover, depends on one's vision of property, and what property values one deems to be most in need of protection. An administrative expropriation procedure might offer more {\it formal} safeguards. A range of procedural rules must be observed, pertaining to notification to the owners, impact assessments, a duty to provide guidance and reasons for the decision, and a possibility (sometimes several) for administrative appeal. Then, after an expropriation order has been granted, the owner can challenge its validity before the appraisal court (which also awards compensation), in principle at the expense of the expropriating party. 

In practice, however, the administrative expropriation procedure can easily leave the owner marginalized, as they are overshadowed by other more powerful stakeholders that are not property owners. This is particularly clear in situations when expropriation arises as a result of more comprehensive planning or licensing procedures, that do not focus on the owners' interest. As discussed in Chapter \ref{chap:x}, this as is the case, for instance, in the context of hydropower development. In addition to this, the possibility of raising validity objections before the courts is mostly a theoretical one. It is very unusual for such objections to be made successfully, as the courts typically defer to the discretion of the administrative decision-maker in expropriation cases.

In the context of land consolidation, on the other hand, the interests of the owners are meant to occupy center stage throughout the proceedings. Moreover, the owners have a formal standing in a deliberative and adversarial context, presupposing their active input to a greater extent than in the context of an administrative decision. In addition, the {\it grounds} for imposing compulsory measures that interfere with property rights need to be anchored specifically in the interests of the affected properties themselves. A measure is warranted only when it benefits the properties as such, in addition to whatever broader societal benefits that might arise. Clearly, this latter principle offers substantial protection of a kind that is completely absent in the context of administrative expropriation. In these contexts, rather, the premise is that the the affected properties and their owners will suffer disadvantages and losses that they can be compelled to bear in the public interest.

Such a narrative is often unavoidable for typical public interest takings, but seem misplaced in the context of takings for profit, since these will as a matter of fact increase the value of properties that are taken. Here, the land consolidation approach seems appropriate, also in situations when interference in established property rights appear necessary to facilitate overall benefits to the community of property owners. Of course, there are challenges that must be dealt with, particularly when some property owners appear to benefit more than others, or when the notion of benefit itself is hard to pin down because property owners disagree about the most important property values inherent in their land. However, it seems to me that a procedural framework that focuses on the community of property owners as the primary stakeholder, is well suited to dealing with these challenges. Much better suited, it would seem, than an administrative decision-making process that conflates the taking for profit scenario to a run-of-the-mill expropriation scenario or an instance of spatial or sector-based planning, with expropriation as a mere side-effect.

Hence, I conclude that principled objections against land consolidation in expropraition contexts appear largely misplaced when for the sub-group of takings that realise commercial potentials. However, a second question arises, of a more practical nature. Will the land consolidation process work in practice, if it is applied to organize commercial development. Increased powers of compulsion might be required, and in keeping with my argument above, I believe such powers may well be granted, as long as land consolidation remains directed at improving the situation for existing properties and their owners, rather than bestowing benefits on someone else. A second question, which I think is far more challenging, concerns the future development of the land consolidation procedure itself. Will it remain a service to owners, placing them at the center of attention, even if its scope is broadened and other, more powerful, stakeholders enter the stage?

It is too early to say, since there has not yet, to my knowledge, been any cases where the issue has come into focus. However, as any legal person with a right to expropriate may now act as a party to a consolidation dispute, the question is bound to arise, in various forms. What will the role of the new parties be? Is the land consolidation procedure still going to be a service to owners, providing a forum for equitable interaction with potential developers, or will it become a service to developers, providing a template for cheap and easy access to property? It will be very interesting to follow this development further, to see if the promise of using land consolidation to regain legitimacy for the use of compulsion to facilitate economic development can be fulfilled.

\section{Conclusion}\label{sec:conc}

In this Chapter, I have addressed land consolidation as an alternative to expropriation for economic development, anchored in a case study of hydropwer. I started by presenting the basic idea of using land consolidation in this way, emphasising that the notion of consolidation at work here is a broad notion that includes measures seeking to enforce particular uses of property. I briefly presented a comparative vision of this kind of land consolidation, noting that a broad notion is a work in many jurisdictions. I then focused specifically on the Norwegian context, where the judicial decision-making framework for land consolidation sets the procedure apart from that found in many other jurisdictions. I also noted how the procedure is conceputalized as a service to owners, with a no-loss guarantee in place to ensure that consolidation measures are only implemented when the benfefits make up for the harms for all the involved properties individually.

I then went on to present the Norwegian system in more detail, focusing on procedural aspects, particularly those related to so-called use directives, that empower the consolidation courts to impose and organize joint use of property rights, including economic development projects. I noted how recent changes in the law envisions an extended scope for these rules, including in the context of non-agrarian and urban development. I then went on to consider some concrete examples, from the context of hydropower development, where owner-led projects already tend to rely on land consolidation rather than expropriation, to facilitate development. I concluded that while the land consolidation alterantice works well when there is basic agreement among the owners that development is desirable, it seems somewhat less effective when there is deep disagreement about how, or whether, development should proceed. 

In these contexts, I argued, it might be necessary to enhance the power of the land consolidation court, also in the direction of increasing its power to compel land owners to take part in, or allow the implementation of, development projects that they disagree with. While this is already possible, to some extent, the power of the land consolidation court in this regard appears somewhat limited, particularly in light of earlier case law that has stressed the distinction between consolidation and expropriation. However, recent legislative developments suggest that this is perspective is now changing, with an increased emphasis on land consolidation alternatives even in cases that require quite severe interferences with the interests of individual property owners.

I argued against those that see this as a threat to property, by pointing out that the formal protection awarded to owners through administrative law is hardly as practically relevant as the fact that the land consolidation process, as traditionally administered, is continuously centred on owners and property interests, making sure that external interests, particularly private interests, can not dominate the process. This, however, might be set to change now that such actors are about to receive a new formal standing in consolidation disputes, and will be granted the opportunity to bring cases before these courts themselves, if they favour it over expropriation. On the one hand, his change will enhance the power of the land consolidation court, making it more effective in dealing with cases that involve external parties. On the other hand, there is a possibility that the presence of new and powerful stakeholders will change the nature of the land consolidation process itself, so that it becomes yet another planning instrument that favour powerful developers, not a property-enhancing institution that promotes self-governance. 

I believe, however, that the land consolidation regime in Norway functions in a way that sheds interesting light on collective-action alternatives to expropriation. It is also a dynamic and versatile framework, more so than other suggestions, such as the land assembly districts proposed by Heller and Hills. In general, it seems that decision-making in a context where local interests are largely vested in property rights require special procedures, if one is to prevent the formation of a democratic deficit. It simply appears imbalanced to make use of the standard administrative planning institutions in such cases, particularly when these institutions are dominated by external, commercial, actors. This is particularly clear when the democratic grounding of these institutions is weak, or centralized, as is the case for Norwegian hydropower. In my opinion, the institution of land consolidation can provide a useful kind of democracy-on-demand for such sectors, facilitating a better balance between the interests of local community, the interests of commercial actors, and the interests of society as a whole. 
\chapter{Conclusions}

In this final section of my thesis, I would like to take a step back to briefly follow two broader threads that I believe run through my thesis. The first concerns the many senses of taking that have been brought into focus throughout the analysis.

\section{Many Aspects of Taking}

The most obvious way to describe a taking is to say that it involves the transfer of property from one legal person to another. However, as I noted in the first Chapter of this thesis, property itself is highly multifaceted, serving a range of social and individual functions. Hence, when we begin to unpack the property bundle, we are confronted with a multitude of different senses in which a taking impacts on owners, their communities, and society as a whole.

The economic consequences of a taking might be the most easily recognisable, particularly in the economic development cases. But as my work in this thesis has shown, other consequences can be just as important, particularly those pertaining to property as an anchor for local democracy. If jointly owned property is taken from a community, with full compensation paid to all individual owners, the community suffers a distinct uncompensated loss, namely the loss of future self-governance opportunities.

It is interesting to note that in the traditional narrative on takings, social and political effects are typically only recognised on one side of the takings equation, namely that of the taker and the public interest. Such is the conventional narrative; the owner as an individual suffers an economic loss in order for the taker, society as a whole, to achieve democratically determined political goals. But in economic development cases, the picture is quite different. In these cases, it is often the case that local communities are deprived of political capital in order for specific commercial interests to make a profit. 

In such cases, it might well be that the balancing of different reasons for and against the taking has taken place prior to the decision to interfere with property. The plans for development themselves may well precede any specific property-oriented implementation steps, such as the use of eminent domain. It might even be that democratically accountable bodies responsible for land use planning have already concluded that some local community interests must give way to other interests.

In these cases, it might be tempting to argue that a narrow takings narrative is appropriate because it pertains only to the final implementation step, which is the only one that involves property rights. But this argument, I believe, rests on a flawed perception of what property is, and should be, in a democratic society. Invariably, property has to do with decision-making and power. If the decision-making process does not grant significant self-determination rights to affected property owners, a taking is already in progress. It might be justified, but it is still a taking. 

More worryingly, it is clear that this kind of taking carries with it a great potential for differential treatment, discrimination, and corruption. The traditional takings narrative does a good job of setting up a framework that makes it difficult to simply pay higher compensation to certain kinds of people, without offering any justification. But with respect to the aspects of taking not recognised, e.g., pertaining to what role the owner has during the planning stages, differences in treatment will not even be notices. But if property is owned by the right sorts of people, then invariably it {\it will} come with considerable decision-making power. 

If property is owned by the marginalised, on the other hand, the most severe act of taking will sometimes have taken place even before the land use planning begins, by the fact that the owners are placed entirely on the sidelines. This, I argued in Part II of this Chapter, is how the Norwegian system for management of hydropower approach riparian owners works. 

At the very outset of planning, often decades before any formal decision to expropriate has been made, a considerable portion of the substance of property is taken from the owners, who are completely excluded from the rest of the decision-making process. In Norway, such takings processes, that clearly transcend the traditional financial narrative, have progressed to the point that even the law today provides an ambiguous account of Norwegian water resources as private property belonging to the general public.

Perhaps, then, the nature of property itself has changed, so that there is nothing left except those financial entitlements that Norwegian expropriation law recognised. If so, the change has not come about by any legislative move, nor has it been preceded by any kind of debate. It has simply emerged, gradually and unplanned, as a result of sector-based regulation and administrative practices. The process, therefore, meets neither the requirements of land reform or expropriation. It is an unacknowledged process about which the law in Norway has had nothing much to say at all, for which silence still persists. 

This is unfortunate. Even if riparian rights should be stripped of all content except a financial entitlement, this should happen on the basis of debate and democratic decision, not because the law fails to cater to a descriptively accurate notion of property.


not how a taking should be carried out, nor does it meet the standard 




 the decision-making process {\it will} normally reflect the power of property, if other contextual factors are not 

although to different degrees depending on other contextual factors, most notably the social status of the owner. Hence, property plays a constuti


particularly 


Property is not merely a placeholder for transient entitlements. It also both help make up and is shaped by the social and political context. 


 This was the key point that I argued for in Chapter 1 of the thesis, by looking to the social function theory of property and the notion of human flourishing. 


which focuses specifically on property rights only {\it after} the public interest has been mapped out and formulated by the decision-makers? 



 individual financial entitlements 

However, the property perspective, which is imposed onl

As I have demonstrated in this thesis, the economic development takings turn this narrative completely on  xplored in depth in this thesis, the con

Rather, the typical narrative places such aspects

After all, on the taker side, many of the primary concepts used to conceptualise typical takings are neither economic nor individualistic.

 on the owner side, do not 

The consequence of this can be that former owners and their communities will be marginalised more generally, as their position within society weakens. In turn, it will become easier to take more property from them, under increasingly weaker arguments of public interest. In the end, when egalitarian property rights no longer provide a foundation for decision-making about land use, the risk is high that a corresponding inequality in decision-making power will follow quite generally. Democracy as such might be at risk.

In any event, the land-less will not have a voice unless they can find different means of asserting themselves. The possibility of achieving participatory equality without egalitarian property should not be overlooked, of course. However, it seems safe to say that the track record of alternative ideas, whereby equality is pursued through institutional arrangements alone, is unimpressive. 

In almost all countries that score well on parameters such as democracy, living standard, transparency, and the rule of law, we find private property rights as a core legal principles. Moreover, while property might be unequally or unfairly distributed among the population, property rights are typically distributed widely enough to give rise to a natural division of power and a plurality of perspectives. Indeed, even the land-less may sometimes attain a voice, albeit a very limited one, if they are still in possession of their own labour.

The negation of property rights

This, in turn, is the very foundation of both democracy and the rule of law.



 as an underlying source of division of power.


 
\section{Some Ways of Giving Back}

\subsection{Locating Primary Stakeholders; The importance of Communities}

\subsection{Making Influence Proportional to Stakes; the Closeness-to-Consequences Test}

\subsection{Robust and Flexible Institutions for Collective Action; the Possibility of a Judicial Approach}

\subsection{Beware of Big Units; the Fine Line between Representation and Usurpation}

\subsection{The Importance of Redundancy; Property Regained}

It seems that property dislikes being concentrated in the hands of the few. 
% \chapter{Sixth Chapter Title}


\section{First Section}
\subsection{First Subsection}
Here is some text. 

\subsection{Second Subsection}

\section{Conclusion}
Here is some more text. 		
% \chapter{Seventh Chapter Title}


\section{First Section}
\subsection{First Subsection}
Here is some text. 

\subsection{Second Subsection}

\section{Conclusion}
Here is some more text. 
% \chapter{Eighth Chapter Title}


\section{First Section}
\subsection{First Subsection}
Here is some text. 

\subsection{Second Subsection}

\section{Conclusion}
Here is some more text.  

%\bibliographystyle{Classes/CUEDbiblio}
%\bibliographystyle{oxford_en}
%\bibliographystyle{Classes/jmb} % bibliography style
%\renewcommand{\bibname}{References} % changes default name Bibliography to References
%\addcontentsline{toc}{chapter}{Bibliography} %adds References to contents page
%\bibliographystyle{Classes/jmb} % bibliography style


%\printindex
\nocite{*}

%If you want to input ship names, put them HERE (after \nocite, before %bibliography.tex

\chapter*{Bibliography}
\addcontentsline{toc}{chapter}{Bibliography}

% This filter is used to identify works which are either of the inbook or incollection type
\defbibfilter{inbookorincoll}{%
  \( \type{inbook} \or \type{incollection} \)}

% Define a bibheading that prints a subheading, with appropriate addition to table of contents, and sets right and left marks accordingly
\defbibheading{mysubbibintoc}{%
  \section*{#1}%
  \addcontentsline{toc}{section}{#1}%
  \markboth{BIBLIOGRAPHY -- \MakeUppercase{#1}}{BIBLIOGRAPHY -- \MakeUppercase{#1}}}

% BOOKS

\printbibliography[title={Books}, type=book, heading=mysubbibintoc, category = cited]

% WORKS IN COLLECTIONS

\printbibliography[title={Contributions to Collections}, filter=inbookorincoll, heading=mysubbibintoc, category = cited]

% ARTICLES IN JOURNALS

\printbibliography[title={Articles}, type=article, heading=mysubbibintoc, category = cited]

% ALL OTHER WORKS INCLUDING UNPUBLISHED MATERIAL

\printbibliography[title={Other Works}, nottype=book, nottype=jurisdiction, nottype=legal, nottype=legislation, nottype=article, nottype=inbook, nottype=incollection, heading=mysubbibintoc, category = cited]
. I have left one example, commented out, which should work (assuming you have the case, etc.).

\index[casesgb]{Achilleas, The@\emph{Achilleas,} The|see{Transfield Shipping Inc v Mercator Shipping Inc}}


%%bibliography.tex

\chapter*{Bibliography}
\addcontentsline{toc}{chapter}{Bibliography}

% This filter is used to identify works which are either of the inbook or incollection type
\defbibfilter{inbookorincoll}{%
  \( \type{inbook} \or \type{incollection} \)}

% Define a bibheading that prints a subheading, with appropriate addition to table of contents, and sets right and left marks accordingly
\defbibheading{mysubbibintoc}{%
  \section*{#1}%
  \addcontentsline{toc}{section}{#1}%
  \markboth{BIBLIOGRAPHY -- \MakeUppercase{#1}}{BIBLIOGRAPHY -- \MakeUppercase{#1}}}

% BOOKS

\printbibliography[title={Books}, type=book, heading=mysubbibintoc, category = cited]

% WORKS IN COLLECTIONS

\printbibliography[title={Contributions to Collections}, filter=inbookorincoll, heading=mysubbibintoc, category = cited]

% ARTICLES IN JOURNALS

\printbibliography[title={Articles}, type=article, heading=mysubbibintoc, category = cited]

% ALL OTHER WORKS INCLUDING UNPUBLISHED MATERIAL

\printbibliography[title={Other Works}, nottype=book, nottype=jurisdiction, nottype=legal, nottype=legislation, nottype=article, nottype=inbook, nottype=incollection, heading=mysubbibintoc, category = cited]



\end{document}
