\documentclass{book} % use larger type; default would be 10pt

\usepackage[utf8]{inputenc} % set input encoding (not needed with XeLaTeX)
\usepackage{url}
%%% Examples of Article customizations
% These packages are optional, depending whether you want the features they provide.
% See the LaTeX Companion or other references for full information.

%%% PAGE DIMENSIONS
\usepackage{geometry} % to change the page dimensions
\geometry{a4paper} % or letterpaper (US) or a5paper or....
% \geometry{margin=2in} % for example, change the margins to 2 inches all round
% \geometry{landscape} % set up the page for landscape
%   read geometry.pdf for detailed page layout information

\usepackage{graphicx} % support the \includegraphics command and options

% \usepackage[parfill]{parskip} % Activate to begin paragraphs with an empty line rather than an indent

%%% PACKAGES
\usepackage[style=oscola,
            backend=biber,
            babel=hyphen]{biblatex}
\usepackage{booktabs} % for much better looking tables
\usepackage{array} % for better arrays (eg matrices) in maths
\usepackage{paralist} % very flexible & customisable lists (eg. enumerate/itemize, etc.)
\usepackage{verbatim} % adds environment for commenting out blocks of text & for better verbatim
\usepackage{subfig} % make it possible to include more than one captioned figure/table in a single float
% These packages are all incorporated in the memoir class to one degree or another...

%%% HEADERS & FOOTERS
\usepackage{fancyhdr} % This should be set AFTER setting up the page geometry
\pagestyle{fancy} % options: empty , plain , fancy
\renewcommand{\headrulewidth}{0pt} % customise the layout...
\lhead{}\chead{}\rhead{}
\lfoot{}\cfoot{\thepage}\rfoot{}

%%% SECTION TITLE APPEARANCE
\usepackage{sectsty}
\allsectionsfont{\sffamily\mdseries\upshape} % (See the fntguide.pdf for font help)
% (This matches ConTeXt defaults)

%%% ToC (table of contents) APPEARANCE
\usepackage[nottoc,notlof,notlot]{tocbibind} % Put the bibliography in the ToC
\usepackage[titles,subfigure]{tocloft} % Alter the style of the Table of Contents
\renewcommand{\cftsecfont}{\rmfamily\mdseries\upshape}
\renewcommand{\cftsecpagefont}{\rmfamily\mdseries\upshape} % No bold!
\newcommand{\noo}[1]{}

\addbibresource{thesis.bib}

%%% END Article customizations

%%% The "real" document content comes below...

\begin{document}

\chapter{Introduction and Summary of Main Themes}\label{chap:intro}

\begin{quote}
Thieves respect property; they merely wish the property to become their property that they may more perfectly respect it.\footnote{G.K. Chesterton, {\it The man who was Thursday: A nightmare}.}
\end{quote}

\tableofcontents

%A human being needs only a small plot of ground on which to be happy, and even less to lie beneath. %\footnote{Johan Wolfgang von Goethe, {\it The sorrows of young Werther and selected writings}.}
%\end{quote}
%“That's what makes it ours - being born on it, working on it, dying on it. That makes ownership, not a %paper with numbers on it.”
%― John Steinbeck, The Grapes of Wrath
%
%
%
\section{Property Lost; Takings and Legitimacy}

%Judging from academic literature on the subject, property is a difficult and often paradoxical concept,
Throughout the history of philosophy, much work has been devoted to the study of private property as a concept. But as these things often go, no consensus regarding its nature has been forthcoming. Moreover, property has proven to be something of a problem child. There has been no shortage of interests in it, but time and again property has demonstrated its recalcitrant nature, most notably its inherent tendency to stir up divergences that blur the line between conceptual analysis and politics. \noo{This might not be a problem in itself, depending on one's methodological stance, but in practice it has had an unfortunate tendency to disrupt attempts at conducting a principled and inclusive academic debate.}

Indeed, from radical Marxism on the one hand to extreme libertarianism on the other, there are plenty of politically charged accounts of what private property is, not to mention what it {\it should} be, assuming, of course, that it will be allowed to exist at all.
%one may be left with the impression that %evangelical praise bestowed on it by libertarians, there is no shortage of politically charged accounts of what property is, not to mention what it {\it should} be, assuming, of course, that it is entitled to exist at all. 
Moreover, there appears to be little room for rapprochement between many leading strands of thought. %Indeed, one is sometimes left with the impression that different philosophical theories of property tend to diverge largely due to sympathies, rather than differences based on reasoned argument.

In a response to this, many legal scholars from the late 19th century onwards have chosen to focus their attention on developing more modest theories, taking property as a legal construct without worrying too much about pre-legal intuitions. This trend has been influential, and even political philosophers may now commonly be found endorsing an agnostic position on the role of property in a just society.\footnote{See, e.g., \cite[274]{rawls71}.} Rather than viewing this question as foundational, these scholars tend to regard it from a practical angle, as a question of design.

Some legal theorists have gone even further, by denying that there even is such a thing as property and by arguing that the word itself, along with ``ownership'', should be removed from legal language altogether. Indeed, while legal professionals often talk about property, they rarely need to consider the philosophical underpinnings of the notion, much less pledge any allegiances in this regard. In addition, when grappling with hard cases, lawyers will soon abandon conceptual clarity in favour of considerations based on social values, politics and expediency.

In a sense, therefore, one may wonder if property has not in fact been lost to the law. As one prominent UK property scholar put it, as soon a close scrutiny of property law gets under way, property itself ``vanishes into thin air''.\footnote{See \cite[306-307]{gray91}.}
%Indeed, one may argue that different ideas of property, practical and theoretical, are behind most, if not all, the major conflicts and confrontations that have shaped the society in which we live.
%Responding to this, some prominent philosophers have taken the view that property is not a concept suitable for philosophical study at all.
%According to some philosophers, property as a concept is a lost cause, not suitable for conceptual analysis at all. Instead, these scholars have suggested that property should be taken as a pragmatic and contingent derivative of other notions, such as the social order, or, on a normative account, by regarding it as an expedient of {\it justice}.
%Moreover, legal scholars are usually content with theories of property that remain largely descriptive, settling for the more modest aim of exploring how best to think of property given the prevailing legal order, rather than trying to come up with theories to explicate its nature as a pre-legal concept. Indeed, even legal philosophers are sometimes found doubting that there even is such a thing as property.

At the same time, there is empirical evidence to support the claim that humans do in fact possess a working {\it primitive} notion of property, one which pre-exists any particular social arrangements used to mould property as a socio-political or legal category.\footnote{See\cite{stake06}.} Perhaps most notably, humans, along with a seemingly rather select group of other animals, appear to have an innate and non-trivial ability to recognise {\it thievery}, the taking of property by someone who is not entitled to do so.\footnote{See \cite[11-13]{brosnan11}.}

Taken in this light, Proudhon's famous dictum ``property is theft'', may be more than a flippant and seemingly self-contradictory comment on the origins of inequality. It might point to a possible alternative direction for investigating the nature of ownership, as a concept that emerges from a distinction between legitimate and illegitimate acts of taking, broadly construed. It seems quite tempting, after all, to describe a person's property as that which has been taken by them (or for them) by legitimate means, and which may not be taken from them without due process.

Interestingly, while the abstract notion of property has arguably received more than its fair share of attention from disciplines other than law, the common-sense notion of a taking has not received much academic attention outside of the legal community. No great philosophical debates have revolved around this notion, and no chasms has opened as to the correct way to understand it. Moreover, legal scholars rarely attempt to define takings as an abstract notion, preferring instead to regard it as a derivative of the legal order surrounding property. In relation to the act of taking, the pragmatic and often jurisdiction-bound perspective of the lawyer appears to reign supreme.

This is problematic, since the notion of property itself is so contested that it might not provide a secure foundation for thinking about such acts. But it also suggests an interesting possibility; perhaps studying takings is a path towards a better understanding of property as well? After all, this is where vastly different accounts of property do seem to share at least an important common point of reference.

In this thesis, I will study a certain kind of taking, namely that which is implemented, or at least formally sanctioned, by a government. In legal language, especially as developed in the US, such acts of government may be referred to as takings {\it simpliciter}, while talk of other kinds of ``takings'' require further qualification, e.g., in case of contract, theft, tax or occupation. This somewhat counter-intuitive terminology might in itself be cause for reflection as to the ideological commitments inherent in legal language. More importantly to me, however, is that it brings the issue of legitimacy to the forefront in a nice way.

We are reminded, in particular, that under the rule of law, taking is not the same as theft. Rather, the default assumption is that the takings that take place under the rule of law are legitimate. If they are not, we may call them by a different name, but not before. At the same time, it falls to the legal order to spell out in further detail what restrictions may be placed on the power to take. 

Indeed, restrictions appear implicit in the very notion of taking. If the power to take was unrestricted, how could one distinguish the act of taking something from the act of putting something to use, for a while, waiting for the next user to come along? In particular, the idea that someone might have occasion to resist an act of taking, and may or may not have good grounds for doing so, appears fundamental to our pre-legal intuitions.

But how should we approach the question of legitimacy of takings from the point of view of legal reasoning, and what conceptual categories can we benefit from when doing so? In this thesis, I aim to make a contribution to this question. I will focus on a special case, namely the so-called economic development takings, when government sanctions the taking of property in order to further economic development.

My primary interest lies in the legal questions that arise, not the overarching philosophical reflections that these might give rise to. However, I believe my introductory remarks here serve to at least sketch an interesting broader perspective on my topic, one which I should like to explore in more detail in the future. I have found it striking, in particular, how recent case law from the US shows that vastly different perspectives on property may indeed come together when focus is shifted away from the thing itself towards the legitimacy of the act of taking it.

The best example is the case of {\it Kelo v City of New London}, which propelled the category of economic development takings forward, first to the political scene, then by causing a surge of academic work by US scholars.\footcite{kelo05}. The {\it Kelo} case concerned a house that was taken by the government in order to accommodate private enterprise, namely the construction of new research facilities for Pfizer, the multi-national pharmaceutical company.

The homeowner, Suzanne Kelo, protested the taking on the basis that it served no public use and was therefore illegitimate under thee Fifth Amendment of the US Constitution. The Supreme Court eventually rejected her arguments, but this decision created a backlash that appears to be unique in the history of US jurisprudence. In their mutual condemnation of the {\it Kelo} decision, commentators from very different ideological backgrounds came together in a shared scepticism towards the legitimacy of economic development takings.

Interestingly, however, their scepticism lacked a clear foundation in US law at the time, as the {\it Kelo} decision itself did not appear particularly controversial to most property lawyers. Hence, when the response was overwhelmingly negative, from both sides of the political spectrum, it seems that people may have been responding to a deeper notion of what counts as a legitimate act of taking.

The critical response to {\it Kelo}, specifically, did not appear to have been primed by the prevailing legal order. It may have been a reflection of widely shared political sentiments, but as such it arguably also involved pre-legal notions pertaining to legitimacy. Simply stated, people from across the political spectrum simply found the outcome {\it unfair}.

This is surely worthy of consideration from legal scholars, and in the US it has received plenty of it. There, it is now hard or impossible to deny that cases such as {\it Kelo} belong to a separate category of takings that raises special questions that are also legally relevant. Moreover, after {\it Kelo}, most US states have passed some sort of legislation to limit economic development takings, in a direct response to the controversy following the {\it Kelo} case. 

In my opinion, it is appropriate to dwell for a while on the fact that this upheaval of US takings law was largely the result of a popular movement. In particular, I think this suggests the possible relevance of economic development takings as a legal category quite generally, also outside of the US. There are certainly significant differences between takings law and practice in the US compared to many other jurisdictions, e.g., in Europe. However, the backlash of {\it Kelo}, particularly the clear divergence between public opinion on the one hand and established case law on the other, suggests to me the transformational potential inherent in the category of economic development takings itself.

As soon as critical attention is directed at the special issues that arise in cases such as {\it Kelo}, it might well be that people will have a tendency to judge the issue of fairness similarly, irrespectively of differences in the surrounding legal framework used to deal with such cases. After all, it seems only natural that characterising certain kinds of takings as takings for profit can lead to a changed perception of their legitimacy.

The question, then, becomes to what extent one may appropriately speak of economic development takings in this way. Here, I believe the first important step is to acknowledge at least the {\it risk} that takings for economic development can be improperly influenced by commercial interests. The risk of capture, moreover, is clearly higher in such cases than in cases when takings take place to benefit a concretely identified public interest, such as the building of a new school or a public road. By contrast, the presence of strong commercial interests on the side of the taker in economic development cases deserve to be singled out as a relevant additional dimension along which to asses such cases.

This claim is by no means self-evident. For instance, it seems that many European jurisdictions implicitly reject such a perspective, already by failing to recognise that the category of economic development takings can be a useful anchor for reasoning about legitimacy in takings law. This brings me to the first key contribution of this thesis, which is a detailed analysis of economic development takings as a conceptual category for legal reasoning.

\section{Economic Development Takings as a Conceptual Category}

%This thesis investigates the category of property interferences that are known as {\it economic development takings} in the US. This category came to prominence only quite recently, following the influential {\it Kelo} case.\footnote{See \cite{kelo05}.} 
While the category of economic development takings does not yet appear to be well established outside of the US, the influence of the US debate is beginning to show elsewhere, including in Europe.\footnote{See, e.g., \cite{verstappen14}.} It is a problem, however, that the exact meaning of the category may differ depending on who you ask. It is quite common, for instance, to speak of ``private'' takings more or less as a synonym of economic development takings. But there are some differences here that I think one should keep in mind, especially if one is aiming for conceptual precision.

First, a private taking already carries with it an implicit pointer to a possible lack of legitimacy, at least in jurisdictions that explicitly single out {\it public} interests as those that must be used to justify takings. Speaking of an economic development taking, on the other hand, does not carry with it any such implicit commitment. If economic development takings are in need of special scrutiny, the reason cannot be simply that it involves {\it prima facie} illegitimate interests. After all, many would agree that economic development itself is usually in the public interest. Hence, using that notion avoids the bias that might arise from designating a taking as ``private''.

A second difference between private takings and economic development takings is that the former notion is far easier to define. In fact, I think it is {\it too} easy. It is very tempting, in particular, to simply say that a private taking occurs whenever the legal person taking title to the property in question is a private company or individual. But this is too simple. It might well be that a private organisation, for instance a regulated charity, functionally mimics ``public'' takers. Moreover, it is not hard to imagine public bodies that are functionally equivalent to private enterprises. 

Indeed, consider a taking that benefits a publicly owned limited liability company. According to the simple definition of a private taking, this would not fall under that term, even if the interests involved are completely or predominantly of a private-law nature. After all, publicly owned commercial companies are often expressly required to pursue profit maximisation on behalf of their shareholders. It seems improper, therefore, to consider them as public service providers.

For economic development takings, we face a different definitional problem. Here, a clear definition appears to be missing in the literature. Rather, scholarship on these kinds of takings rests on an intuitive understanding of the term, firmly based on the US jurisprudence from which it first arose. 
By contrast, in this thesis I aim to offer a more explicit account of what characterises economic development takings and what makes them interesting.

I argue that the key defining feature of such takings is that they involve particularly strong financial incentives for the taker. This often results in contrasts between the apparent motivations of the taker and the public spirited motivation of the executive or legislative body that grants permission to use compulsion; the (stated) intention of economic development takings is to promote public interests, not to bestow commercial benefits on particular parties. In my opinion, the  importance of economic development as a category of takings is that it helps us flag those cases when this contrast is so strong as to suggest that we further question the legitimacy of the undertaking as a whole.

Here the vagueness of the notion of economic development is in itself an important cue. Indeed, if the decision-maker fails to identify concrete public interests, but must rely on such a vague notion, this in itself might be a good cause for increased scrutiny. This, moreover, is an observation that appears to be of generally validity, also outside the context of US law. At the very least, the tension between public interest and commercial gain in property interference is of general interest in any system of government that combines a market-based economy with wide state powers over the use and distribution of property. The question becomes how one should reason about this tension in a meaningful way, to analyse economic development takings in a manner that can yield legally relevant insights.

This is the key question that I tackle in this thesis, both theoretically and by a case study of Norwegian law. 
\noo{ Later in this thesis, I will devote much attention to this question. First I will do so from a theoretical point of view, by first arguing that the category of economic development takings arises naturally already at the theoretical level, provided one chooses a suitable theoretical framework for reasoning about takings and property. Following up on this, I set out to distil some general lessons from the US debate and its history. In addition, I briefly assess the status of economic development takings in Europe, where takings that benefit commercial interests are often allowed to pass without raising special questions, and where the legal relevance of the category of economic development takings may still be called into doubt.

In fact, I argue that this is a shortcoming of the narrative of property protection in Europe, and I also suggest that the concept of an economic development taking would in fact fit well with jurisprudential developments at the ECtHR, which stresses both the need for contextual assessment and attention to possible systemic imbalances in the expropriation practices of member states.

%Similar requirements, interestingly, are found in many other jurisdictions, and is also found in the property clause in the European Convention of Human Rights (ECHR). But in Europe, it is often understood very loosely, as a clause that places little or no practical limit on the state's taking power. The contrast with the suggestions that are now being considered by US scholars is very great.
}
In the US, most work on this has been anchored in the so-called ``public use'' requirement of the Fifth Amendment. In fact, some US scholars argue that economic development takings are impermissible already because taking property for development cannot ever be said to constitute a ``public use'' of the property. Moreover, even scholars who reject this view tend to agree that the public use of a taking is less obvious, and should be subjected to more intense judicial scrutiny, in economic development cases.

Interestingly, requirements similar to the public use test are found in many jurisdiction, in various guises, e.g., in rules referring to the need for a {\it public interest} or a {\it public purpose} for takings. On this basis, interesting comparative work has been carried out on the basis of the idea that such a requirement is at the core of the legitimacy issue that arises for economic development takings.

In this thesis, I challenge this perspective. I do so by first reconsidering the history of the public use debate itself, as documented by case law in the US. I argue, in particular, that more attention should be paid to the fact that the state courts that originally set out to develop public use tests in the 19th century adopted a highly contextualised approach. Importantly, these courts where largely not bothered by the fact that they could not pin down any definite and consistent meaning of ``public use'' as a general concept. 

\noo{ Rather, the public use test was simply used as an expedient way of subjecting various acts of taking to a concrete fairness assessment, in the hope that local courts might help deliver corrective justice in cases when the takings power appeared to have been used in an objectionable manner. In this way, the original purpose of the public use test was tailored towards setting up a framework for judicial review that appears quite similar to how the European Court of Human Rights (ECtHR) currently choose to approach cases dealing with property.}

I identify a parallel between this approach and contemporary jurisprudence at the ECtHR, which typically directs focus away from the question of purpose legitimacy towards the contextual question of whether or not interference is {\it proportional} given the circumstances. This, I argue, is also how the public use test was also originally used by state courts in the US, before the issue of legitimacy turned federal and became subject to a more abstract form of assessment, resulting in a doctrine of deference.

I conclude that the current focus on the notion of a ``public use'', which is supposed to provide the desired protection against transgressions, is largely misguided. At the very least, I believe alternatives should also be considered. This brings me to the second focus point of my thesis.

\section{A Democratic Deficit in Takings Law?}

I am not the first to challenge the traditional narrative that surrounds economic development takings. Indeed, some US scholars have now begun to argue forcefully that increased judicial scrutiny of the public use requirement is neither a necessary nor a sufficient response to concerns about the legitimacy of commercially motivated takings. Instead, these authors point out that the takings procedure as such does not seem able to appropriately deal with commercial incentives on the taker side.

This has been accompanied by procedural proposals for takings law reform, most notably Professors Heller and Hills' article on Land Assembly Districts and Professor Lehavi and Licht's article on Special Purpose Development Companies.\footnote{See \cite{heller08,lehavi07}.} Both of these works propose novel institutions for collective action and self-governance, to replace (parts of) the traditional takings procedure, especially in cases where the taker has commercial incentives.

By examining these proposals in some depth, I arrive at several objections against the details of the particular institutional arrangements proposed, particularly with regards to their likely effectiveness. It seems, in particular, that both proposals fail to recognise the full extent to which prevailing regulatory frameworks concerning land use and planning would have to be reformed in order to make their proposals work.

At the same time, I argue that these novel institutional proposals are extremely useful in that they point towards a novel way to frame the issue of legitimacy in takings law. In particular, I explore the hypothesis that traditional procedural arrangements surrounding takings suffer from a democratic deficit, a particularly powerful source of discontent in economic development cases.

This idea is the second key focus point of my thesis. First, I approach it from a theoretical point of view, by exploring the notion of {\it participation} and its importance to the issue of legitimacy, particularly in the context of economic development. It seems, in particular, that {\it exclusion} could be a particular worrying consequence of certain kinds of economic development takings, namely those that lack democratic legitimacy in the local community where the direct effects of the taking are most clearly felt.

I believe this to be a promising hypothesis, and I back it up by considering the social function theory of property and the notion of human flourishing which has recently been proposed as a normative guide for reasoning about property interests. I pay particular attention to the importance of communities that has been highlighted in recent work, as a way to bridge the gap between individualistic and collectivist ideas about fairness in relation to property.

I take this a step further, by arguing that a focus on communities naturally should bring institutions of local democracy to the forefront of our attention. The role that property plays in facilitating democracy has been emphasised before by other scholars, and I think it has considerable merit. However, I also argue that it is important to resist the temptation of viewing its role in this regard through an individualistic prism. It is especially important to take into account additional structural dimensions that may supervene on both property and democracy, such as tensions between the periphery and the centre, the privileged and the marginalised, as well as between urban and rural communities.

It is especially important, I think, to appreciate the effect takings can have on local democracy. For one, excessive taking of property from certain communities might be a symptom of failures of democracy as well as structural imbalances between different groups and interest. But even more worrying are cases when the takings themselves, brought on by a commercially motivated rationale, appears to undermine the authority of local arrangements for collective decision-making and self-governance. This dimension of legitimacy, in particular, is one that I devote special attention to throughout this thesis.

I also believe, however, that it is hard to get very far with this sub-theme through theoretical arguments alone. Hence, to explore it in more depth, I go on to assess it from an empirical angle, by offering a detailed case study of takings of Norwegian waterfalls for the purpose of hydropower development. This case study, in turn, will allow me to cast light on two further key themes, that I now introduce. %This brings me to the second part of my thesis, which in turn consists of two main themes, where the latter aims to bring me back towards a more general setting, by delivering some recommendations for how best to deal with economic development takings.

\section{Putting The Traditional Narrative to the Test}

In Norway, the traditional way of thinking about legitimacy of takings is grounded in the notion that owners are entitled to monetary compensation. The law of expropriation clearly reflects the importance attributed to this idea; the constitution itself stipulates that owners have a right to be paid in ``full'' for the loss they suffer as a result of giving up their property. Consequently, the right to compensation in Norway is generally regarded as stronger than in many other jurisdictions, including those that adhere to the minimal standard imposed by the ECHR.

On the other hand, the story of legitimacy more or less begins and ends with the issue of compensation. Hence, if an owner has grievances that are directed at the act of taking as such, not the amount of money they receive, takings law has very little to offer. In fact, it does not appear to have anything at all to offer, that does not already follow from general principles of administrative law. The owner can certainly argue that the decision to authorise the taking was in breach of procedural rules, or grossly unreasonably, but the chance of succeeding by making such arguments are slim, arguably no higher than in administrative cases that do not involve interference with property rights.

This narrative of legitimacy is not unique to Norway. It seems that in Europe, unlike in the US, the issue of legitimacy is often seen as predominantly concerned with the issue of compensation. In particular, the jurisprudence at the ECtHR is typically focused on compensatory issues. Moreover, while many constitutions of Europe, including the Norwegian, include public interest clauses, the courts make little or no use of these when adjudicating takings complaints. In the words of the ECtHR, the member states enjoy a ``wide margin of appreciation'' when it comes to determining what counts as a public interest.

Through my case study, I present a detailed analysis of how this traditional narrative actually plays out in Norway, in relation to takings for hydropower development. Such takings form an interesting sub-class because they are clearly economic development takings, in the most interesting sense of the word. Since the early 1990s, the hydropower sector in Norway has been deregulated, so the hydropower companies, to which the government may grant permission to expropriate, are now predominantly commercial entities. Moreover, the property that they seek to takes is not merely some ancillary rights that they need to develop the country's resources. In Norway, the right to harness the power of water is a private right, under a riparian system that is otherwise quite similar to that found in the UK.

Hence, the primary right that the energy companies tend to take is the right to harness the natural resource itself, a right that is typically held jointly by groups of small-holders and local farmers. In effect, the established hydropower sector in Norway is entirely dependent on taking natural resources from local communities by use of compulsion, with the help of government, in order to exist. Since deregulation, however, not only have energy companies been reorganised as limited liability commercial companies, local owners have also begun to make use of their right to harness water power by undertaking their own hydropower projects. 

As a result, local owners now regularly protest expropriation of their rights on the grounds that they wish to {\it participate} in economic development, by carrying out alternative development projects, or even by cooperating with the established energy companies who wish to take their water rights. Hence, while liberalisation empowers local owners to develop hydropower, it also renders taking for hydropower as takings for profit. Unsurprisingly, this has led to tensions that Norwegian courts have had to grapple with in an increasing number of cases.

In their approach to these cases, the courts rely heavily on the traditional narrative, by reconsidering how compensation is calculated when water rights are taken for hydropower. Compensation practices have already changed dramatically. However, there has also been cases when the local owners of these rights have protested the taking as such, claiming that they should be given the opportunity to develop their own resources. These protests have been entirely unsuccessful, as the courts in Norway adopts a stance on legitimacy that is extremely deferential to the executive, provided adequate compensation is paid.

In my case study, I start by presenting the legal framework, including a short excursion into legal history, before I give a detailed assessment of a few select cases. This concrete empirical approach will allow me to explore the practical consequences of the current takings narrative, while also aiming to bring out how decision-making process surrounding hydropower work in practice. My main finding is that local owners risk being marginalised by the current regulatory framework, and that new compensation practices have proven inadequate as a means of redressing concerns that arise in this regard. My conclusion is that the case study of Norwegian waterfalls demonstrate concretely the shortcomings of the traditional narrative of legitimacy of takings.

However, I also believe that Norwegian law may offer a possible path towards a solution to this problem, one that has also been put to active use in recent years, particularly in cases when farmers themselves aim to undertake hydropower development, but wish to do so against the will of other members of the local community. This brings me to the second key theme of my case study.

\section{A Judicial Framework for Compulsory Participation}

In Norway, the distribution of property rights across the rural population is traditionally highly egalitarian. This has had many consequences for Norwegian society. For one, it meant that the farmers in Norway soon became an active political force, particularly as representative democracy started to gain ground as a form of government in the 19th century. As early as in 1837, the Norwegian parliament was so dominated by farmers that it came to be described as the ``farmer's parliament''.

The Norwegian farmers were often little more than small-holders, and had few privileges to protect. Hence, they became liberals of sorts (although also known for their fiscal conservatism). The farmers as a class were responsible for pushing through important early reforms, such as the establishment of semi-autonomous, elected, municipality governments.\footnote{The farmers were also responsible for abolishing noble titles in Norway. Clearly, they owed no allegiance to the established aristocracy of landed nobility in Europe.}

However, the municipality governments were not the first example of local, participatory, decision-making institutions in Norway. Indeed, the highly fragmented ownership of land meant that institutions for land management are among the oldest known in Norway. One of the most important ones exists to this day, namely the {\it land consolidation court}. The final focus point of my thesis consists in an assessment of this institution and its potential as a possible procedural alternative to takings when compulsion appears to be needed in order to ensure economic development.

Importantly, the land consolidation procedure in Norway is a semi-judicial process that warrants the imposition of {\it compulsory participation} by primary stakeholders in decision-making processes to which they are deemed to owe a contribution. One typical situation when the institution will be invoked involves the management of jointly owned land, where the land consolidation procedure is used to ensure that local owners may reach a joint decision on how to regulate the use of their land, if necessary one that is imposed on them by the land consolidation judges.

However, the judges' power is limited in that they may only impose a measure if the gains are deemed to outweigh the loss for all stakeholders involved. In practice, land consolidation judges often act as mediators, to facilitate a collective decision. Moreover, one of the most common acts of a land consolidation judge is to set up owner's associations, in a manner that institutionally regulates the continued interaction and decision-making among the stakeholders even after the formal consolidation process has concluded.

I explore this framework in some depth, focusing on its potential as an alternative to exporpriation. This is especially interesting since land consolidation is presently being put to use in order to organise hydropower development. Hence, my case study provides an excellent opportunity for comparing the land consolidation and the takings process, with respect to the overall aim of ensuring development of hydropower on equitable terms. 

Here, I argue, the land consolidation route may be preferable, as it ensures legitimacy through participation. At the same time, the procedure remains effective, since participation is in fact compulsory. I discuss possible objections to the procedure in some depth, but conclude that the continued development of the land consolidation institution provides the best way forward for addressing economic development takings in Norway.

Finally, I compare the institution of land consolidation with the institutional proposals that have been made specifically in the context of the debate on economic development takings. I argue that it compares favourably, both because it comes equipped with in-built judicial safeguards, but also because it has a broader scope. I note, however, that its use as a better alternative to economic development takings is dependent on both political will and an ability to retain key feature even in the presence of new and powerful stakeholders in the consolidation process itself.

\noo{ In the second part of the thesis, I put the theoretical framework to the test by applying it to a concrete case study, namely that of Norwegian hydropower. Following liberalisation of the energy sector in the early 1990s, hydropower is now a commercial pursuit in Norway. Moreover, there is a long tradition for granting energy producers the power to acquire property compulsorily, including the necessary rights to exploit the energy of water, rights that are subject to private property under Norwegian law. This has resulted in tension and controversy, however, as the original owners of these rights, typically local farmers and small-holders, see the commercial potential of hydropower being transferred to other commercial interests, to the detriment of their own, and their communities', interest in self-governance and economic benefit.}

\section{Structure of the Thesis}

My thesis is divided into two parts. The first is devoted to setting up a conceptual framework and a knowledge base for analysing the legitimacy of economic development takings. I start in Chapter 2, by examining theories of property as a legal concept, particularly the so-called social function theory. This theory is distinguished by the fact that it highlights property as an anchor of responsibilities as well as rights. To social function theorists, property must be understood in terms of how it shapes and regulate social systems that are important to society, not just in terms of owners as individuals with special entitlements.

The proponents of the social function theory often make strong normative claims about property, but here I argue that it is fruitful to take a step back and examine the descriptive content of the theory separately. This, I believe, can help us locate a theoretical template that is less conceptually impoverished than many other descriptive theories of property as a legal concept. My hope is that this can render the theory more attractive as a common ground for meaningful discussion among theorists with very different normative ideas and commitments.

Returning to my core topic, I go on to argue that the social function theory suggests that economic development takings should, already for the sake of descriptive accuracy, be treated as a separate category when reasoning about legitimacy. This insight, I note, does not arise in the same way from the two main traditional strands of theorising about property in law, based on the {\it dominion} concept and the {\it bundle of rights} metaphor respectively. On these accounts, property is understood in largely individualistic terms that make it hard to justify why the purpose of the taking should be of any concern at all to the affected owner, as long as due process has been observed.

After establishing economic development takings as a category of descriptive analysis, I set out to provide a theoretical template from which to embark on normative assessment. Here I turn to the notion of {\it human flourishing} which has been proposed as a key concept when reasoning about the {\it purposes} and {\it values} of private property. Importantly, the notion of human flourishing is usually accompanied by theories that endorse value pluralism, while also flagging the importance of property to the well-being of communities. This latter aspect, in particular, will be important throughout the remainder of the thesis. 

In the final part of Chapter 2, I make a first pass at substantive assessment, by applying the theoretical framework I develop to provide a brief analysis of the {\it Kelo} case and related academic work in the US. In Chapter 3, I go on to present a much more detailed account of how legitimacy of economic development takings are dealt with in the US compared to Europe. I focus particularly on the history of the public use debate in the US, to argue that there are important commonalities between how the public use restriction was originally applied (at state level) and how the ECtHR current adjudicates property cases, by assessing the {\it proportionality} of the interference.

I follow this up by considering the role of courts as arbiters in relation to proportionality. I argue, in particular, that recent developments at the ECtHR might indicate not only increased level of scrutiny, but also a shift of attention towards examining systemic imbalances. Moreover, I follow those who argue that courts are not well placed to actually {\it ensure} proportionality. In particular, I argue that neither national nor international courts are well placed to actively steer the takings process through a myriad of rules that seek to explicate what counts as legitimate in any given scenario. Rather, I locate an institutional gap for hard cases, where the traditional takings procedures simply appear to be inadequate.

In the final part of Chapter 3, I build on this by considering in depth some proposals for institutional reform that have emerged in the literature from the US. Here the focus is on designing mechanisms for collective action and self-governance that can replace takings in the traditional sense, in cases when there are strong economic incentives for development. 

This promises to ensure forms of benefit sharing with owners and their communities that are unavailable when a traditional compensatory approach is adopted. In addition, proposals for institutional reform can enhance democracy and human flourishing by giving a more prominent place to owners and local communities in those critical decision-making processes that may lead to development and reconfiguration of established property patterns. I pinpoint some shortcomings of existing proposals, raise some questions, and argue that the institutional route is the best way forward for addressing the legitimacy issue in further depth.

This preliminary conclusion leads to the second part of my thesis, where I apply the insights gained from the first half to analyse and distil lessons from the Norwegian framework for expropriation in the context of hydropower development. In Chapter 4, I begin by offering a brief introduction to the Norwegian legal system, before presenting in more detail the rules regulating the right to harness the power of water. I follow up on this by presenting empirical data on the hydropower sector, aiming to shed light on how the regulatory frameworks work in practice. 

I emphasise the current tension between hydropower projects controlled by local owners and their communities and competing projects controlled by large commercial, partly state-owned, companies, that rely on expropriation. I also study the effects local hydropower initiatives can have on the communities in which development takes place, and I present an early vision of the social function of the owner-led hydropower industry in some depth. I conclude by a cautionary assessment of current developments in the owner-led industry, where commercial forces appear to be gaining ground at the expense of more rounded perspectives on the purpose of development.

In Chapter 5, I go on to specifically consider rules and practices relating to the expropriation of the right to harness water power. I give a general introduction to Norwegian expropriation law, while focusing on special rules that apply to hydropower development. I show, moreover, how the current regulatory regime is strongly influenced by the fact that the hydropower sector used to be organised as a state monopoly, under decentralised political control. This democratic and public anchor was largely removed, however, as the sector was deregulated in the early 1990s. As a result, current practices in Norway render takings for hydropower as pure takings for profit. This, I argue, has contributed to a recent surge in cases where local owners challenge the legitimacy of established practices.

I go on to study one such case in great depth, to bring out how the current framework works in practice. I find that it can leave local owners marginalised and excluded from the key decision-making processes that eventually lead to the taking of their property by commercial companies. At the same time, I note how the compensation procedure has been reformed, offering a financial windfall to some owners individually. However, I note how these reforms have been actively opposed by the hydropower industry which currently appears to be gaining ground in its efforts to reverse recent changes in compensation practice. 

I conclude by arguing that the traditional narrative of legitimacy has proven inadequate in Norway, as it unduly focuses on compensation without tackling underlying imbalances in the division of decision-making power among owners, local communities, regulators and commercial companies. In this way, my analysis of the Norwegian case culminates in the same conclusion as my theoretical work, pointing to the need for institutional reforms that can give local owners and their communities a stronger voice in decision-making processes concerning their natural resources.

In Chapter 6, I go on to analyse the Norwegian land consolidation courts as potential answers to this challenge. I begin by presenting the legal framework surrounding this institutional arrangement which has long traditions in Norway. Moreover, I note that it appears to be growing in importance, and that the land consolidation court is widely authorised to order collective action among property owners as well as to set up more permanent institutions for self-governance, with clearly defined rules and purposes. Hence, consolidation courts can in fact be used to order economic development and set up framework that compel owners to participate in it. 

After briefly presenting the procedure itself, I focus on important property-based protections against abuse, such as the no-loss guarantee which states that no consolidation measure can be ordered unless the gains match the loss for all the properties involved. I note, in particular, how the focus here is on the affected properties as such, not on the individuals who happen to have rights to them. After presenting the land consolidation courts, I go on to study how they are used in practice in cases of hydropower development. 

Interestingly, while the established commercial companies continue to rely on expropriation of water rights to facilitate development, local communities that face internal disagreements about development are far more likely to turn to the land consolidation court. Hence, there is now empirical evidence available on consolidation as an alternative to expropriation in these cases. While the development projects facilitated by consolidation are still typically small-scale compared to those carried out with the help of expropriation, I still believe this material is highly interesting. 

I analyse several concrete cases of consolidation for hydropower development to bring out how this works in practice. I conclude with an assessment of the consolidation alternative, proposing that it looks very promising and should be explored further. Moreover, I note the striking similarities between the consolidation framework and the institutional proposals that have emerged in the US, which I presented at the end of the first part of the thesis. I note, moreover, that the there are some key differences that I believe speak in favour of Norwegian consolidation courts. 

The judicial framework, the flexibility of the procedure, the possibility of using compulsion imposed by a neutral party, as well as the built-in procedural safeguards, all seem to be strengths of the Norwegian system. At the same time, however, I note some possible weaknesses, particularly the worry that the land consolidation institutions themselves risk being captured by powerful actors. In addition, worries related to the cost of the procedure, as well as its effectiveness in case of large-scale development, are addressed. Overall, however, my conclusion is that the core ideas inherent in this institutions are sound and can serve as a template for creating legitimacy enhancing institutions for compulsory economic development elsewhere. This observation concludes the material work of this thesis, and in Chapter 7, I offer my final conclusions.

\end{document}