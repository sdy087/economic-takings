\chapter{Compulsory participation in economic development projects}\label{chap:6}

\section{Introduction}\label{sec:intro6}

In this Chapter, I will consider {\it alternatives} to expropriation in the context of economic development. This is also where I ended my theoretical discussion in Part I, by presenting and analysing the proposals of Heller and Hills in the context of the US debate on economic development takings. Here, I return to this point in the context of my case study, by exploring the Norwegian institution of {\it land consolidation}. 

In recent years, this institution has been used extensively to facilitate hydropower projects. So far, however, it is used almost exclusively in situations when owners themselves organize such projects. In these situations, expropriation is rarely sought and rarely authorized. Instead, various consolidation measures are used, including the practically important measure of a ``use directive'', to set up an organizational framework and a binding plan for development involving jointly owned property,

In addition to the practical use of consolidation in the context of hydropower development, there are recent legislative developments in Norway that sees the consolidation alternative gain importance also in relation to other forms of development. This includes urban and non-agrarian development projects, something that represents a departure from the tradition of land consolidation in Norway. Some argue that these uses of land consolidation leave the owners in a precarious position, and may weaken private property rights. In this Chapter, I argue for the opposite perspective, that the use of consolidation in these new contexts will enhance property as an institution. Moreover, I argue that it can be used to  address the democratic deficit of economic development takings in a very elegant way, provided the land consolidation process itself remains intact, as a service to owners and local communities, and is not usurped by external actors.

I begin in Section \ref{sec:lce}, by presenting the basic idea of using land consolidation as an alternative to expropriation. I discuss broad notions of land consolidation in general, relating them also to the discussing found in Heller and Hills' article. Then I point out some special features of the Norwegian system, which I believe make it particularly natural to consider in the context of economic development cases. 

Then, in Section \ref{sec:lcc}, I present the Norwegian system of land consolidation in more depth, focusing on the procedural rules and the rules in place to protect property against unwarranted interference. I focus particularly on the so-called ``no-loss'' guarantee, which states that no consolidation measure can be implemented unless the benefits make up for the harm, for all affected properties. Hence, land consolidation is quite distinct from expropriation in general. However, in situations when benefit sharing is possible or natural, it becomes possible to fulfil the no-loss criterion, and in doing so land consolidation could become a powerful alternative to expropriation for such cases. 

In Section \ref{sec:lch}, I discuss land consolidation specifically in the context of hydropower development. I consider several cases in detail, based on court documents and recent work done in a master thesis on consolidation, where the author carried out interviews with affected owners. Then, in Section \ref{sec:lca}, I offer a assessment and discuss some challenges. I argue more extensively hat the land consolidation alternative is should be seen as a possible way to strengthen property rights, not as a threat. I also pinpoint what I believe to be the main challenge, namely to ensure that the land consolidation process remains intact as a service to owners and local communities, even after powerful commercial interests enter the scene. In Section \ref{sec:conc}, I conclude.

\section{Land consolidation as an alternative to expropriation}\label{sec:lce}

The notion of land consolidation is somewhat ambiguous. At its core, it refers to  mechanisms whereby boundaries in real property are redrawn to reduce fragmentation, without affecting the relative value of the different owners' holdings. However, it is also common to use consolidation to refer to mechanisms for pooling together small parcels of land to create larger units. There is a tension between these two notions of consolidation, with some claiming that consolidation in the latter sense is sometimes used to surreptitiously bestow benefits on powerful property owners, at the expense of weaker groups.

In light of this, I should stress at the outset that I will use the term land consolidation in a very broad sense in this chapter, much wider than {\it both} of the interpretations mentioned above. Land consolidation, as I use the term, refers to any mechanism by which the state intervenes, at the request of some interested party, to reorganize property rights in a given local area. Hence, a consolidation measure might as well involve {\it increased} fragmentation of property, if this is deemed a rational form of ``consolidation'' of the property values of the affected area. Importantly, I also use land consolidation to refer to efforts directed at organizing the {\it use} of property, not just redrawing boundaries.

Some might argue that this terminology is strained, but I adopt it for a reason. It is motivated by the fact that in Norway, the institution known as ``jordskifte'', officially translated as land consolidation, has a very broad meaning. Norway is not unique in this regard. Land consolidation has a similarly broad scope in many jurisdictions of continental Europe, as well as in Japan and in parts of the developing world.\footnote{See \cite{sky07,vitikainen04}.} Moreover, in Heller and Hills' work on land assembly districts, a comparison with land consolidation is presented, based on the same broad notion that I use here.\footcite{heller08} One of my main aims in this chapter is to pick up on this, by offering a more detailed comparison and assessment, specifically anchored in the Norwegian system and its application in the context of hydropower development.

As land consolidation tends to involve interference in property rights, one may ask about the legitimacy of various consolidation measures, held against rules that protect private property owners. In some cases, it can also be argued that land consolidation {\it is } a form of expropriation, even if it is not necessarily recognized as such in the jurisdiction where it takes place. Such legitimacy issues have in fact been raised before the ECtHR on a few occasions, where the Court has found that land consolidation measures have been applied in breach of P1-1. Similarly, in the US, a proposal to introduce a land consolidation regime was struck down in the 1980s, as not in keeping with the property clauses of the US constitution.

\noo{ Move: On the other hand, if land consolidation is used to facilitate or impose specific uses of property, it can also be used as an {\it alternative} to expropriation, a compulsory measure that can obviate the need for depriving owners of their property rights. I think this latter perspective on land consolidation is particularly interesting, and it is the perspective I adopt in this chapter.}

In relation to the legitimacy issue, the Norwegian system stands out in two important regards. First, the consolidation procedure is managed by judicial bodies, namely the {\it land consolidation courts}. Second, land consolidation is largely seen as a service to owners, not a tool for increased state control and top-down management. In particular, a case before the land consolidation courts is almost always initiated by (some of) the affected owners themselves, in an effort to clarify who owns what, readjust boundaries, or organize the use of property in the local area. Moreover, it is a core principle of law that no land consolidation measure may be undertaken unless the benefits make up for the harms, for all the properties involved. This is known as the ``no-loss'' criterion. It is one of the key principles of land consolidation in Norway. The combination of a judicial procedure that places great emphasis on owner-participation and a no-loss criterion means that, arguably, land consolidation in Norway {\it strengthens} property as an institution. 

However, the beneficial effects of land consolidation are not primarily supposed to target individual owners, but rather the properties as such, as productive units  of importance to the community of owners as a whole. Hence, I believe land consolidation can serve as an effective countermeasure against two of the most widely discussed challenges to any property regime. First, consolidation can serve to protect an egalitarian distribution of property rights against the deleterious effect of inefficiency and underdevelopment that might otherwise arise from fragmentation. Importantly, it can do so without disturbing the underlying property structure and without necessarily resulting in disproportionate benefits or harms to owners and other private parties. In particular, land consolidation can obviate the need for handing property rights over to powerful market actors to ensure development. Second, land consolidation can serve to ensure sustainable and rational management of jointly owned land, without necessarily forcing an enclosure process (enclosure {\it can} be the result of land consolidation, but it is only one of many measures in the consolidation toolbox). 

In short, land consolidation can be used to address both commons and anti-commons problems, in a way that protects, and possibly enhances, desirable social functions of property, through a judicial system of participatory/adversarial decision-making. Hence, land consolidation in Norway is based on a conceptual premise that -- potentially -- offers protection to owners and their properties, by recognizing them as members of a community that are mutually dependent on each other. In this way, the form of property protection offered in the context of land consolidation is distinct from the protection offered in the context of expropriation law, in a manner that is in itself interesting, particularly from the perspective of property's social functions, as discussed in Part I.

Importantly, since land consolidation can be used to facilitate or impose specific uses of property, it can also be an {\it effective} alternative to expropriation, a compulsory measure that can obviate the need for depriving owners of their property rights. I think this perspective on land consolidation is particularly interesting, and it is the point of view I adopt in this chapter. Indeed, consolidation as an alternative to expropriation is particularly natural for economic development projects. Importantly, it seems that the no-loss criterion should be possible to fulfil in these cases, through benefit sharing. 

The land consolidation courts can, moreover, {\it impose} benefit sharing on the parties. However, it  is usually {\it not} permitted to address the no-loss criterion by compensatory means, particularly not if those means are monetary. Instead, the general idea is not only that the benefits resulting from the consolidation measure must match the harms, the benefits must also be distributed fairly among the affected properties.\footnote{The latter principle is not as strictly encoded in the law, but finds formal expression in certain special provisions. Hence, the extent to which fair benefit sharing is {\it actually} achieved following consolidation to facilitate large-scale economic development projects, is an interesting question that I return to in ....} So the principle of benefit sharing at work is not one where the owner is a passive recipient of compensation, but rather an active participant in the development itself, possibly against his own will. This, too, I find highly interesting, particularly from the point of view of the human flourishing conceptions of property that I discussed in Part I.

More pragmatically, the emphasis on benefit sharing in land consolidation reveals a concrete potential advantage of land consolidation over expropriation, from the owners' point of view. In particular, while some sort of benefit sharing is typically ensured through land consolidation, it is much less commonly achieved through compensation in the context of expropriation.\footnote{This is largely due to the so-called {\it no scheme} principle, which states that compensation to the owner following expropriation should not reflect changes in value that are due to the expropriation scheme. I am not aware of a single jurisdiction that does not include a variant of this principle. For a detailed investigation into the question of whether or not it stands in the way of benefit sharing in economic development cases, I point to \cite{....}.} This means that the use of land consolidation in place of expropriation has considerable potential also in relation to the problem of the ``uncompensated increment'' in economic development takings, as discussed in Part I.

On the other hand, this also means that commercially motivated developers may have an {\it incentive} to favour expropriation. This, then, raises the question of whether or not calling for the use of land consolidation as an alternative can act as a {\it defence} against expropriation, or, if this is not possible presently, if such a defence {\it should} be open to property owners facing condemnation for commercial purposes. Secondly, the fact that consolidation can act as an alternative to expropriation  also gives developers an incentive to push for changes in land consolidation law itself, so that it will become more profitable for them to make use of it.

Hence, one question that arises in the present context is the following: will land consolidation remain a service to owners, or will it become a service to developers who seeks cheap access to property owned by others? This question is becoming increasingly relevant in Norway, as the scope of land consolidation has been broadened in recent years, so that it {\it can} in fact be applied in many expropriation contexts, also to facilitate commercial development in urban areas.

So far, the Norwegian system is moving along a trajectory where land consolidation as an alternative to expropriation is primarily seen as a service to developers and the public, not as a means for empowering owners. It is noteworthy, in particular, that owners are not normally entitled to demand land consolidation in place of expropriation. Instead, following a change in the law that takes effect in 2016, the {\it developers} will be granted a new right, namely that of bringing a case before the land consolidation courts, to seek help in implementing their project. In fact, developers might well be motivated to do so, because of the potential for reduced administrative costs, a more effective and flexible procedure, and a chance of limiting compensation claims by imposing (cheaper) compensatory consolidation measures (e.g., by providing owners with replacement property).

\noo{However, the issue of benefit sharing is bound to come up, particularly in the context of commercial development. In this regard, the risk for developers is that they will be compelled to share the benefits with the owners. However, in order for this to happen, the Land Consolidation Court must actively take steps to make it happen, by recognizing the owners' right to benefit sharing. Moreover, while benefit sharing is a fundamental principle for land consolidation among owners, it remains to be seen if this way of thinking will be preserved when new and powerful external actors enter the scene.}

However, it seems that in order to be a truly effective alternative to expropriation, not only the takers, but also the owners, should be granted the opportunity to request implementation by consolidation. In addition to the question of whether a land consolidation measure can be requested in an expropriation scenario, one must also ask how exactly it would work, and what policy aims it could help achieve. Here there is already quite some data available, arising from situations when owners themselves are behind economic development, but prefer to make use of consolidation measures, instead of expropriation, in order to deal with their neighbours.

Interestingly, in the context of hydropower development, this use of land consolidation has become very important in recent years. In 2009, the Court Administration estimated that land consolidation had helped realise small-scale hydropower projects with a total annual energy output of about x TWh/year. Moreover, in a recent Supreme Court case, the importance of land consolidation was stressed specifically, as a justification for requiring a commercial taker to pay additional compensation to the owners of waterfalls that were to be used for hydropower generation.

In the next section, I give some more details on the Norwegian system. Then, in Section \ref{sec:lch}, I return to the use of land consolidation to facilitate small-scale hydropower, which I approach as a test case for the more general proposition that land consolidation can be a legitimacy-enhancing alternative to expropriation, particularly in the context of economic development.

%%%%%%%%%%%%%%%%%%%%%%%%%%%%%%%%%%%%%

\noo{
In particular, a case can now be brought before the land consolidation court by an external developer who would otherwise need to expropriate land to implement a project. However, this change in the law also contributes to a shift away from seeing land consolidation as a service to owners, towards seeing it also a service to developers who seek control of property they do not own. This shift could in turn change the dynamics of land consolidation in a way that makes it less distinct from expropriation.

Even though this definition is broad, I note that a clear distinction can be drawn between land consolidation and national or regional land {\it policies}, which do not target specific properties. The distinction between land consolidation and land-use planning can be harder to draw, but looking to the theoretical starting points of these two kinds of interventions, suffice to establish sketch.

While state planning is an expression of the state's right to regulate the use of land, a land consolidation measure is a {\it service} provided by the state, to facilitate property uses and structures that are deemed desirable from the point of view of the properties as productive units under private ownership.

The distinction between consolidation and measures of land reform may sometimes also be difficult to draw, particularly with my wide notion of consolidation. However, while land reforms tend to arise from centrally directed measures that apply generally within a jurisdiction and come about as the result of a special political initiative, consolidation usually denotes a more flexible framework where local communities are restructured in a way that aims to bring benefits to all owners and rights holders within that community. As such, consolidation rules may alleviate the need for new land reforms, and they may come to represent a ``bottom up" approach to the restructuring of real property.\footnote{The potential for this has been noted even for the traditional understanding of consolidation, as a reduction in the level of property fragmentation. See, e.g., \cite{oldenburg90}. For a different perspective, arguing that land consolidation is generally not sufficient to achieve the noble ends of land reform, see \cite{lipton74}. For a more recent, comprehensive, assessment of the relationship between land reform and consolidation (in the narrow sense), I refer to \cite[237-244]{lipton09}.}

In the following, I adopt this normative stance on the {\it purpose} of consolidation. Hence, I use land consolidation to refer to a regulated process of land reorganization that come about as a result of a concrete, often local initiative, has a limited geographical scope, relies on the involvement of the local population, and seeks to promote the best interests of all the affected land users. I remark that while Norway has a particularly broad approach, land consolidation more or less in line with my understanding here serves an important function in many jurisdictions.\footnote{For a survey of contemporary land consolidation rules in Europe, reflecting also the need for a wide understanding of the term, I point to \cite{vitikainen2004}.}

One attractive feature of consolidation is that it provides a flexible, dynamic, framework that allows for gradual adaptation of ownership structures, so that they better suit prevailing economic and social conditions. Moreover, land consolidation can become significant in relation to concrete development projects, particularly when such projects necessitate cooperation among several owners. This, in particular, is the use of consolidation that I aim to shed particular light on in this chapter, by giving a case-study of Norwegian law.

Mechanisms for facilitating and organizing concrete development schemes are now integrated into the law relating to consolidation in Norway. These rules do not form part of the historical core of consolidation rules in Norway, however, the focus on land consolidation for development is of a more recent date. It is reflected in a number of new provisions, most recently in the \cite{lca13} which will take effect on 1 January 2016.\footnote{Act no 97 of 10 June 2013 relating to the determination and change of structures of ownership- and rights to real property etc.}

When land consolidation is used as a means to organize development projects it also becomes natural to view it as an alternative to expropriation, especially in cases when development has commercial potential and is meant to be carried out by companies operating for profit. Moreover, the controversy that often surrounds such cases further suggests that it should be explored to what extent processes of consolidation can replace expropriation as an implementation mechanism for development of this kind. As we will see, the principle of local participation and benefit sharing is more firmly entrenched in the rules and procedures that govern the consolidation process than in the processes that govern the use of expropriation. 

The contrast between expropriation and consolidation is particularly clear in Norwegian law, where a ``no-loss" principle is enforced with regards to the latter, protecting all affected owners and rights holders. It states that the consolidation process should not leave any owner or rights holder worse off after consolidation. The aim of consolidation is to bestow a benefit on \emph{all} interested parties.\footnote{See \cite[3 a)]{lca79} and \cite[3-18]{lca13} (takes effect in 2016). For a paper discussing the rule in more detail we point to \cite{rygg98}. Rygg is also critical of what he sees as a development away from a strict interpretation of the no-loss rule.} For instance, if ownership is highly fragmented, consolidation mechanisms may be used to exchange property between owners or to introduce joint ownership, but due to the no loss rule it will not be possible to use consolidation in order to deprive some owners of their property to the benefit of others.

In the following, I map the differences between consolidation and expropriation in Norwegian law, starting with an overview of the land consolidation rules, focusing on the development towards giving these rules greater application in connection with concrete development schemes. I then study some cases of locally controlled hydro-power development where land consolidation was used as a means to organize projects involving many different owners and rights holders. We argue that these cases illustrate how the consolidation rules currently in place are well suited to meet local demands for participation and benefit sharing, more so than the existing framework regulating expropriation.

The structure of the remaining part of the chapter is as follows. In Section \ref{sec:lcc} I briefly present the basic rules regarding land consolidation in Norwegian law, including a presentation of the special consolidation courts used to administer the process. Then I go on to consider in more depth the rules relating to so-called \emph{use directives}, permitting the court to actively pursue development projects on behalf of, and in cooperation, with local owners. Use directives represent a form of compulsory cooperation which I believe deserves further attention in the context of land development, especially as an alternative to expropriation. I follow up with a case-study of hydropower in Section \ref{sec:lch} and provide an assessment in Section \ref{seclca}. \noo{nd in Section \ref{sec:5} I contrast the use of directives with more commonly seen approaches to pooling of resources and commercial land development. In Section \ref{sec:6} I offer a conclusion.
}}

\section{The system of land consolidation courts}\label{sec:lcc}

Rules regarding land consolidation have a long history in Norwegian law. The first consolidation rules were included already in King Magnus Lagabøte's \emph{landslov} (law of the land) from 1274, the first piece of written legislation known to have been introduced at the national level in Norway.\footnote{See Chapter 4, Section 2 in \cite{nou02}.} The earliest rules targeted jointly held rights in farming land, giving any owner or tenant farmer on that land an opportunity to demand apportionment that would give him exclusive rights on a parcel of land corresponding to his share of the joint rights.\footnote{The share in joint rights belonging to each individual farm was historically determined based on the amount of rent (``skyld") that each farmer paid to the land owner. However, following the union with Denmark and especially after the advent of enlightened absolutism, tenant farmers in Norway increasingly bought their land from increasingly marginalized Danish land owners. Indeed, tenant farming became uncommon in Norway after the 18th Century, but the notion of ``skyld'' was kept as a measure of the share each farm had in the now jointly owned larger estates. The notion is still important, for instance in apportionment proceedings, as discussed in  \cite{ravna09a}.}

Many of the rules currently in place were developed in the 19th Century. At this time, the main use of land consolidation was still to divide jointly owned and fragmented property rights into parcels in hope that this would facilitate higher intensity farming and better management of resources.\noo{, but the relative importance of such measures increased greatly since it was seen as a necessary adjustment in an age when industrialization introduced a range of new and more efficient farming techniques. In particular, as changes in farming methods resulted in an increased need for capital in agriculture, full ownership came to be regarded as more favourable since it meant that better security could be offered to financial institutions.\footnote{References needed.} } However, it was noted that individuation of property rights was not necessarily required since collective-action mechanisms could be employed instead, to facilitate a more dynamic approach.\footnote{This idea was pursued in various ways, not only in relation to land consolidation. For instance, legislation was passed in the 19th century to set up a new management structure for the commons, to avoid overexploitation and ensure rational management, without necessitating enclosure. See generally \cite{stenseth10}.}

The land consolidation rules regarding use directives must be understood in this context. They were introduced to facilitate  a legal framework for implementing or altering specific use patterns without altering the underlying structure of ownership. Moreover, the rules were supposed to work even in cases that appeared to require quite considerable pooling of resources and decision-making power. 

The initial target uses were agricultural, meant to enable rural communities to adapt to changing economic conditions without fundamentally altering them or leading to displacement or depopulation. Hence, use directives were mostly relied on in connection with farming, and not commonly used to facilitate different kinds of development.

In recent years, this has changed. Today, use directives are increasingly applied also to organize development projects that are not associated with traditional farming. Moreover, many additional mechanisms of land consolidation have been introduced, all aiming in various ways to ensure better organization of land use and ownership. These mechanisms are administered by the \emph{consolidation court}, a special tribunal which has land consolidation as its sole task.\footnote{It appears to be a unique administrative unit in the European setting, although Austria have land tribunals that resemble it. References.} There are three main categories of consolidation tools that the court may use, and they are summarized in the following.

\begin{itemize}
\item \emph{Apportionment of land}: Rules that empower the court to dissolve systems of joint ownership by apportioning to each estate a parcel corresponding to its share, or by reallocating property through exchange of land. This is the traditional form of land consolidation in Norway, and the main legislative basis for it is provided in the \cite[2 a)-b)]{lca79}.
\item \emph{Delimitation of boundaries:} Rules that empower the court to determine, mark and describe boundaries between properties and the content and extent of different rights of use attached to the land. The main legislative basis for this form of consolidation is found in the \cite[88]{lca79}.
\item \emph{Directives for use}: Rules that empower the court to prescribe rules for the use of jointly held land, and to organize such use, including setting up organizational units for carrying out specific development projects, as described in \cite[2 c)|34-35]{lca79}.
\end{itemize}

In all cases, the consolidation court can only employ these tools when they are called on to do so by someone who is regarded as having a valid legal interest in the matter.\footnote{See \cite[5]{lca79}.} Traditionally, this meant one of the owners of the land involved, but gradually the system have also come to recognize that others might have a legitimate interest in consolidation. This includes developers who have obtained planning permission for specific projects that require reorganization of property rights. The legal role of such actors is currently changing from passive to more active, and this development raises particular questions that we will return to below. For now, I note that the traditional situation is that a consolidation process is initiated by one of the local owners of the land. It is still true, in particular, that consolidation is primarily a mechanism by which any one among the owners can work out what their legal position is, and, if the court agrees that it would be favourable to the use of the land, can ensure that the rights in the land are restructured.

The condition that restructuring only takes place when it is regarded as favourable is an additional condition that limits the court's authority to take action that involves apportionment and directives for use. To determine whether or not it has been met, the court will look to the current economic and political climate, and so the consolidation rules are \emph{dynamic}, capable of being adapted to the circumstances. In this regard the court is also influenced by what it regards as the prevailing public interests in property use, and recent developments in consolidation law stress the importance of this link, with recent reforms seeking to strengthen it.\footnote{See for instance \cite{prop12} (proposal from the Ministry of Agriculture to the parliament regarding the Consolidation Act 2013).}

However, the contextual nature of land consolidation has always been clear. We have already mentioned how the basic building blocks of the current system can be traced back to the influence of technological advances in farming and the modernization processes that Norwegian society underwent in the 19th Century. The law responded to these changes, and consolidation became a vitally important instrument for change and development in this period. It was also at this time that it was decided to establish a tribunal system for administering the process, first in the Land Consolidation Act from 1857 and then revised and developed further in acts from 1882 and 1950.\footnote{An overview of the history of consolidation law is given in Chapter 3 of \cite{prop12}.} The procedural rules closely mimics those that pertain to the regular civil courts. This ensures that consolidation tools are only put to use if a court orders it, and only after a public hearing where all involved parties are given an opportunity to present their case, give supporting evidence, and to contradict each others' testimony. For a more detailed description of the consolidation court, I refer the reader to Section \ref{subsec:lcp} below.

The current system for land consolidation is based on the \cite{lca79}, but this act will be replaced in 2016 when the new \cite{lca13} takes effect.\footnote{Act no 97 of 10 June 2013 relating to the determination and change of structures of ownership- and rights to real property etc.} The new act was passed on 10 June 2013, and while it does not introduce any dramatic changes to the law, it further widens the scope of consolidation, particularly with regards to directives for joint use. The new act also contains the following explicit description of the purpose of land consolidation:

\begin{quote}
Section 1-1 The purpose of the Act

The purpose of the act is to facilitate efficient and rational use of real property in the best interests of the owners, rights holders and society. This objective will be pursued by the land consolidation courts which will implement remedies for unpractical structures concerning ownership and use of property, ascertain and determine property boundaries, as well as decide appraisal disputes and other cases as pursuant by this and other acts.

The act also seeks to facilitate fair, responsible, quick and effective processing of cases in independent and impartial public courts that will operate in such a way as to enhance confidence in the consolidation process.\footcite[1]{lca13}
\end{quote}

This statement of purpose highlights how the new act incorporates and extends the trend towards giving the consolidation process wider scope. I also note how it reiterates and emphasises that the process is to be tribunal in nature. In my opinion, it also suggests that land consolidation is likely to become more important in the future, increasingly also outside the traditional agricultural setting within which the ancient body of law regulating it has hitherto developed.\footnote{For instance, following a change in the \cite{lca79} in 2006, land consolidation may now also be called on in order to manage restructuring of ownership in urban areas, in connection with specific development schemes. This rule has been extended further in the new Act, and it will be interesting to see how the division of labour will be in the future, between planning authorities, regular courts and the land consolidation courts.}

It is now explicitly stated that the purpose of land consolidation is to make conditions of property use more favourable for all the affected owners and rights holders. Hence the new act accentuates how consolidation represents a form of interference that is fundamentally different from expropriation. As before, the consolidation court is not empowered to take action unless it is called on by one of the stakeholders in the property.\footnote{See \cite[1-5]{lca13}.} However, according to the new act, a developer who has obtained permission to expropriate is to be counted as a stakeholder in that land for the purposes of consolidation.\footnote{Previously, a developer was only regarded as a stakeholder in consolidation in some cases of public projects, c.f., \cite[5|88|88 a)]{lca79}.} This further reflects how it is becoming increasingly natural to see land consolidation as an alternative to expropriation. It also flags how the relationship between expropriation and consolidation is now becoming an important topic in Norwegian land law.

In 2005 the Ministry of Agriculture made some comments in this regard, in connection with a revision of the \cite{lca79} that gave consolidation greater applicability in urban areas and with respect to implementing public plans.\footnote{See, in particular, \cite[2 h-i)]{lca79}.} Some members of the preparatory committee had raised the concern that giving consolidation extended scope in this way would be problematic since it would encroach on expropriation law and effectively render consolidation a form of expropriation. The Ministry disagreed, commenting as follows.

\begin{quote}
The Ministry would like to point out that one of the main preconditions for consolidation is that a net profit is created for the land in question. This profit is then divided among the parties in an orderly fashion. Individually, the law also guarantees that no one suffers a loss, see s 3 a). [...] \\ \\ In the Ministry's opinion, expropriation takes place on a different factual and legal basis. In cases of expropriation the public makes decisions that deprives the parties of economic value. The purpose then becomes to compensated them in accordance with s 105 of the Constitution, not to increase the value of their land or the annual income they may derive from it.\footnote{See Chapter 3.3 of \cite{otprp78} (report to parliament from the Ministry regarding changes in the \cite{lca79}.}
\end{quote}

When preparing the new act, the Ministry of Agriculture reiterated this position but they did not reflect further on the question of the exact relationship between consolidation and expropriation. They observed, however, that changing the law so that expropriating parties could appear in consolidation cases was \emph{reasonable} since it would then be left up to the developer whether to make use of his permission to expropriate or to rely on consolidation instead.\footnote{See \cite[84]{prop12}.} Indeed, it seems that in many cases, the well-organized and tightly regulated process of consolidation might be a more practical alternative for developers than the more fragmented rules and administrative bodies that come into play following traditional expropriation. 

However, it seems clear that the choice made by the expropriating party in this regard will tend to be even more important for the affected owners and rights holders. In particular, as the Ministry themselves made clear in the passage quoted above, it is an absolute precondition for implementation of any consolidation measures that alter the rights structure that they must serve to make the structure of ownership and use more favourable for \emph{everyone}. This, moreover, refers explicitly to the \emph{area within which consolidation takes place}, as stated in s 3-3 of the new act. No similar rule is in place to protect the affected local area following expropriation. Moreover, the practices that have developed for dealing with consolidation cases are centred on the interests of the local owners and their land to an extent that is quite different from any procedure that is currently in place to facilitate development by use of expropriation.

For instance, the rule regarding expropriation that corresponds most closely to the no-loss rule  
requires merely that the benefit to private and public interest exceeds the disadvantages \emph{overall}, not locally and certainly not for individual local owners.\footnote{See the \cite[2]{ea59}.} However, I also note that the strict consolidation rules do not serve as a restriction on \emph{what} kind of development should be carried out, only on \emph{how} it should be organized. The former question is left to the planning authorities, and the consolidation courts must always base their decisions on existing public regulation of property use.\footnote{In the \cite[3-17]{lca13} it is explicitly stated that the consolidation court cannot prescribe solutions that are not in keeping with such regulation. However, it is also made clear that the consolidation court itself can apply for necessary planning permissions on behalf of the owners and the land in question.}

Hence, if the public interest suggests a particular form of land use, the fact that a planning decision detailing development of such use is implemented through consolidation does not entitle the court to review the plans themselves, going against the public interest. But it does introduce an obligation, emerging at the time of implementation, to turn specifically to the interests of original owners and rights holders and to look for solutions that minimize the burden and maximizes the benefit for all the involved parties.

The rules that give the consolidation court authority to give directives of use are particularly relevant in this regard, and I return to these rules below, after I have presented the consolidation process itself. It seems, in particular, that the procedural guarantees resulting from the fact that this process is organized as a tribunal are in themselves an important factor to consider when looking at consolidation as an alternative to expropriation.

\subsection{A Brief Presentation of the Consolidation Process}\label{subsec:lcp}

A consolidation case is usually initiated by an owner or a permanent rights holder.\footnote{See s 5, para 1 of the \cite{lca79}.} The request for consolidation measures is to be directed at the relevant district consolidation court, one of the 34 district courts for land consolidation that have been set up by the King in accordance with s 7 of the \cite{lca79}. The request is meant to include further details about the affected properties, the owners and rights holder involved, as well as the specific issues that consolidation should address. But the requirements in this regard are not usually interpreted very strictly and the district consolidation court will often take on quite some responsibility for further clarifying what the case should encompass, more so than in civil disputes.\footnote{References needed.} Even so, the court is entitled to reject the request due to technical shortcomings, following the same rules as those which applies to civil disputes.\footnote{See s 12, para 2 of the \cite{lca79}, which refers to s 16-5 of the \cite{cda05}.}

If the court decides that the request is well-formed and that it includes sufficient detail to permit consideration of the substance, they go on to prepare public hearings, following the rules set out in Chapter 3 of the \cite{lca79}. These rules mirror those that are in place for civil hearings in general, including the duty to inform affected parties (s 13), the parties' right to present their claims, and their duty and right to give testimony and provide evidence supporting it (s 15, 17 a) and 18). As in civil cases, the decision is usually only reached after at least one hearing in which the parties are present and permitted to contradict the evidence provided and the testimony given by other parties. However, unlike in civil cases, the main hearing typically takes place on the disputed land itself, and consists in mapping and clarifying the prevailing conditions aided by visual inspection. Moreover, a consolidation case will usually not take the form of a two-party adversarial process, but rather as a multi-party discussion where the court interacts with a large number of interested parties who may have a range of common as well as conflicting interests. Usually, consolidation cases involve at least 10 or more different parties, and in some cases there can be hundreds. In addition, it is quite common that the parties are not represented by legal council, but rather take an active part in the process themselves.\footnote{References needed.}

The request for consolidation will be the court's point of departure when assessing the case, but the court is not bound by the claims put forth in it, or by the claims put forth by the other parties. This again marks a differences with most cases of civil dispute. With a few exceptions explicitly listed in statute, the consolidation court may decided to use any measure that it deems suitable to ensure a favourable structure of rights and ownership for the future. However, there is some restriction placed on the court in that the measures taken must be regarded as \emph{necessary} in light of considerations based on the original request.\footnote{See s 26 and 29 of the \cite{lca79}.} So while the court should remain focused on the issues raised by the parties, it should be free to address these issues using the tools they deem most suited for the job. The consolidation court, in particular, is meant to be a general ``problem solver", more so than the ordinary civil courts.

When a decision is reached, the rules in s 7 and 22 of the \cite{lca79} ensure that the parties are notified and that the decision is presented and argued for in keeping with the rules of the \cite{cda05}. The appropriate form of the decision will depend on its content. A regular civil ruling is the form used for decisions that only involve ascertaining the boundaries between properties, while a special ``consolidation decision" is the form relied on to implement apportionment and directives of use. The difference becomes clear as soon as we consider the appeals procedure; while civil rulings are dealt with by the regular courts of appeal, the consolidation decisions can only be appealed to one of 4 designated consolidation courts of appeal.\footnote{See  the \cite[61]{lca79}.}

In the latter case, the procedural rules remain largely the same in the consolidation court of appeal, meaning also that there is a new assessment of all aspects of the case.\footnote{See Section 69 of the Land Consolidation Act 1979.} When the case is concluded here, however, it can only be appealed on the grounds that it is based on an incorrect understanding of the law, or that procedural mistakes have been made. In this case, the ordinary appeal courts have authority, with the Supreme Court being the last instance of possible appeal.\footnote{See Section 71 of the Land Consolidation Act 1979.}

From the brief overview of the process given above, we see that consolidation cases are different from other civil cases in that they have fundamentally different scope. A consolidation case is not primarily centred on deciding the merits of individual claims, but rather at introducing structures of ownership and rights that will prove favourable in general. In this respect the process has an administrative character. However, the fact that it is organized more or less like a regular civil dispute means that the protection of each affected party, and the influence of the local rights holders as a group, is much stronger than what would tend to be the case if these decisions were made by regular administrative bodies.

Given this context of arbitration, it is not surprising that the judges appointed to the consolidation courts are required to have a special skill set, different from that of regular civil law judges. In fact, consolidation judges are required to have successfully completed a special masters level degree in consolidation, which is not a law degree at all but a separate form of education.\footnote{See s 7, para 5 of the \cite{lca79}. The degree in question is currently offered only at the Norwegian College of Life Sciences and Agriculture.} 

The consolidation court also relies on the participation of lay judges who sit alongside the specialist judge.\footnote{See s 8 of the \cite{lca79}.} These judges are appointed by the specialist judge from a committee of laymen that are elected by the local municipalities in accordance with s 64 of the \cite{ca15}.\footnote{See s 8 of the \cite{lca79}.} In the district courts, the specialist judge usually sits with two appointed lay judges which he chooses himself from among members of the the relevant local committees. In the court of appeal, the specialist sits with 4 laymen, and in complicated cases 4 laymen may also be called on in the district courts, but only if one of the parties requests it.\footnote{See s 9, para 2 of the \cite{lca79}.} To the extent possible, the appointed laymen should have special knowledge of the issues raised by the case, but they are drawn from the general population.\footnote{See s 9, para 5 of the \cite{lca79}.}

Summing up, the consolidation process has both administrative and adversarial characteristics. While the content and scope of the court's decision will often have an administrative flavour and is not primarily directed at settling any specific dispute, the process is judicial. Hence everyone is entitled, and to some extent even \emph{obliged}, to have his voice heard and to partake in the process. Moreover, while the process is guided and overseen by the court, it is fundamentally based on considerations arising from the interests of the parties. However, this interest is always interpreted in light of prevailing notions of what counts as favourable and rational property use. Importantly, in relation to this latter assessment, the court will look beyond the interests of the individual owners. The court will pose the question with regards to the use of the land as such, drawing on its understanding of the relevant economic, social and political conditions.\footnote{References needed.} But the decisions made are always prepared using information that is retrieved and discussed in public hearings, so the affected parties will take part in discussions that may also address more overreaching concerns about the form of land use that should be regarded as favourable for the area in question.

To flag the dual nature of the consolidation process it is tempting to designate it as a process of judicially structured \emph{deliberation}. The final decision-making authority is granted to the court, but the court is required to act on behalf of the rights holders, in the best interests of their land, and based on the information that they themselves provide. This particular form of decision-making based on multi-party deliberation is interesting in its own right, as it provides a template for management of land that seems capable of catering both to the idea of public oversight and control as well as to the idea of local participation. In addition to this, it seems to be a form of land management that might be especially suitable as a means to implement concrete projects undertaken in the public interest, particularly when these would otherwise appear to adversely affect individual land owners and local communities.

This is of particular interest in mixed economies such as seen in Norway, where decisions regarding development and use of property are typically made by the public but carried out by private property owners. In many cases, implementation of public policy requires some form of reorganization of ownership and rights structures, the most common being a pooling of resources from many different owners. Such processes have tended to be implemented rather crudely, by displacing the original owners in favour of commercial companies who serve as state agents. This relies on the use of expropriation, and it typically completely deprives the original owners of any chance to take part in the future development of the land. The land consolidation rules allows us to consider alternative means of implementation in such cases. 

They allow us to ask whether a more measured approach might be sufficient, allowing the original owners to retain their rights, but restructuring them using consolidation mechanisms. This question is particularly interesting to consider due to the high levels of tension often associated with cases of commercial expropriation, where companies operating for profit benefit from implementing public plans. Critics argue that such uses of expropriation are both unfair in themselves and also destabilizing in that they raise doubts abut the true motives behind specific acts of public planning.\footnote{References needed.} In particular, it seems that in a system of land management where development is organized in this way commercial companies will have much to gain from attempting to exert influence over the planning process, particularly if they can also succeed in being granted permission to expropriate property rights that they would otherwise have to acquire on an open market. However, to counter critics it may appear easy to argue from necessity, by pointing out that the system is the best known alternative for efficient and rational economic development in a system based on public control over planning and private rights to property.

In relation to this debate it seems that the consolidation procedure takes on particular relevance. It may point to an alternative, a system of public-private development where the original owners and local communities are better integrated into the process. Moreover, it allows us to introduce an additional conceptual layer between the planning stage and the implementation step, a layer of management devoted to translating public plans into concrete action by orderly restructuring of existing ownership patterns. This, in particular, might be a layer of administration that deserves more attention and more fine-grained tools than those currently offered in systems relying on expropriation. Clearly identifying such a consolidation layer in property management might also make for a cleaner delineation between commercial implementation on the one hand, governed by the market, and public planning on the other, governed by administrative law and political bodies. 

In the next section, I argue that Norwegian consolidation law already include tools that make it possible to view consolidation in this light. The rules that I believe warrant this conclusion are the rules relating to joint use directives, briefly mentioned above. I present them in more detail below, noting that recent changes in consolidation law give them wider applicability in relation to concrete development projects. 

\subsection{Joint use, joint action and joint investment}\label{sec:3}

In accordance with s 2 c) of the \cite{lca79}, the consolidation court can give directives regarding the use of land which involves more then one property. Typically this will target land or land rights that are owned jointly or for which some form of shared use has already been established. However, if the court finds that there are \emph{special reasons} for giving joint use directives, it can do so even if there is no prior connection between the different rights and properties in question.\footnote{See s 2 c), para 2 of the \cite{lca79}.} Traditional examples include directives for the shared use of a private road which crosses several different properties, or regulation of hunting that takes place across property boundaries.

The joint use rules emerged as an alternative to apportionment of jointly owned property, a more subtle and less invasive measure that could often give rise to the same positive effect as a full division of ownership, but without leading to unwanted fragmentation of control and use of property. Hence, in the now repealed Land Consolidation Act 1950 it was stated that joint use directives should be the \emph{primary} mechanism of consolidation, and that apportionment should only take place if such directives were deemed insufficient to reach the goal of creating more favourable conditions for the use of the land.\footnote{References needed.} In the 1979 Act, the two mechanisms were formally put side by side, but in cases that are motivated by a specific planned use of the land in question, directives will still be the main tool relied on by the court.

Moreover, there has been a gradual increase in the willingness of the court to rely on use directives to facilitate \emph{new development} on the land, not just as a means to regulate an existing activity. In parallel with this development, the consolidation court has gradually come to take on cases that pertain to organization of land use that was previously thought to lie outside its area of competence. The more restricted view on use directives and on the function of consolidation in general is reflected in the way the 1979 act lists a range of different concrete circumstances in which such directives might be applied.\footnote{See s 35 of the \cite{lca79}.} The list is not understood to be exhaustive however, and the courts have gradually come to feel less deterred by it and more willing to consider new types of cases.\footnote{References needed.}

Hence in the new act of 2013, the list is replaced by an explicit general rule which makes it clear that the  consolidation courts have the authority to give directives whenever they regard this to be favourable to the properties involved.\footnote{See s 3-8 of the \cite{lca13}.} In addition to this, the new act also introduces a general rule which gives the court authority to \emph{set up} systems of joint ownership when a joint use directive is deemed insufficient.\footnote{See s 3-5 of the \cite{lca13}.} Hence, in the new act apportionment and pooling of property is on equal footing, although a priority rule is introduced for the latter; pooling will only be considered if directives of joint use are regarded as an insufficient means to ensure more favourable conditions. Moreover, the new act maintains the principle that directives regarding the joint use of land for which there are no existing joint rights can only be given if there are special reasons.

This requirement is not intended to be very strict, and the Ministry of Agriculture was initially inclined to remove it. However, it was eventually decided that it should be kept in order to flag that there two distinct questions that arise in such cases. The court must first consider the question of whether or not joint use is in itself desirable, before it goes on to consider how to best organize such use.\footnote{For a discussion on this see \cite[140-141]{prop12}.}

In addition to giving directives prescribing how joint use is to be organized, the consolidation court may also give rules compelling the owners to take joint action to help facilitate better realization of the potential inherent in the land. Rules to this effect were first introduced in the 1979 Act, in s 2 e) and sections 42-44. These rules only pertain to joint action by property owners (see s 34 a), and they have wider scope in relation to specific case types (s 43 and s 44). Following the new act, however, the consolidation courts will have authority to prescribe joint action also for right holders, and the special rules listing concrete circumstances will be replaced by a general joint action rule.\footnote{See s 3-9 of the \cite{lca13}.}

This broadens the scope of these rules in accordance with the general spirit of the new act. Indeed, when they commented on this change in the law, the Ministry noted that the rules in question have been widely used following their introduction in 1979 and that applying them is now one of the core responsibilities of the consolidation courts.\footnote{See \cite[146]{prop12}.}

I note that joint action directives can include prescriptions for joint investments.\footnote{See s 3-9 of the \cite{lca13}.} On the one hand this means that such directives can be used to facilitate capital intensive new development, but it also raises the question of the extent to which it is legitimate to rely on compulsion in this regard, directed towards individual owners of property. The extent of the joint actions and investments required to undertake development projects can easily become quite burdensome for these individuals, and this is especially likely to arise as a concern in cases where the land lends itself well to large-scale commercial development.

The 1979 Act attempts to resolve this in s 34 b) and s 42. The former states that if joint actions or investments may come to involve``great risk", the court must set up two \emph{distinct} organizational units to undertake it. First, the rights needed to undertake the scheme will be pooled together and managed by an owners' association, and then, to undertake the scheme itself, a cooperative company structure will be set up on behalf of the owners. Hence the risk is diverted away from the individual owners onto a company controlled by them. This company will be entitled to any potential profit from the scheme, but it will also be required to pay compensation to the owner' association on terms established by the parties themselves, with the help of the court.\footnote{See s 34 b) of the \cite{lca79}.} Moreover, the owners are entitled to shares in this company proportional to their share of the relevant rights in the land, as determined by the consolidation court. An owner is not obliged to take part in the undertaking by acquiring such shares, but he will benefit from membership in the owners' association regardless of whether or not he chooses to do so.

After this brief survey of the rules, I conclude that the land consolidation courts in Norway already have all the tools they need to organize development projects on behalf of local owners. Moreover, the process of consolidation means that they must do so in a way that enables the original owners to retain considerable decision-making power as well as the right to any commercial benefit that may result from the development. Hence, the rules currently found in consolidation law adds weight to the claim that and consolidation might point to an alternative and possibly fruitful way of implementing development projects in a system which presupposes that development takes place through commercial initiatives on the basis of public  planning and control. In particular, the system already provide the tools needed to organize large-scale development even when it requires considerable reorganization of land rights and diversification of risk. Consolidation may therefore become an alternative way of pooling together fragmented rights for the purpose of development, without displacing the original owners in favour of commercial companies who have no prior connection to the local community in which development takes place.

In addition to this, I observe that the consolidation rules also point to a form of implementation that will allow the public to exercise \emph{more} extensive oversight and control. Not only is the position of the original owners much better protected under this system, but it also greatly \emph{curbs} the power and influence of commercial forces with no prior connection to the land. Hence, it must be expected that implementation through consolidation is better suited also to serve the social and political aims which originally motivated the underlying planning decision. Indeed, commercial development through consolidation give the public a greater say in the implementation stage; after all, the development is organized as a cooperative, and the company structure is set up and regulated by the courts who is obliged to consider also the public and societal interests in land use.

In addition to this, after the new act takes affects, both planning authorities and commercial developers may take up a role as formally recognized parties in the consolidation process. This seems particularly useful in connection with large scale industrial development, as it might otherwise be hard to implement such projects successfully. In these cases, then, the consolidation system sets up an arena for interaction and deliberation between the three main groups of stakeholders: the public, the local owners and the commercially motivated developers. Such an arena is so far missing at the implementation stage of big development projects, while recent controversies regarding expropriation suggest that it might come to serve an important function.

It remains unclear to what extent the consolidation rules will actually be used in this way. But as I discuss in the next section, consolidation is beginning to emerge as an important means for organizing local hydropower development. On the theoretical side, then, it is also unclear to what extent original owners may \emph{demand} that the rules are put to use. For instance, may an owner request consolidation to prevent a permission to expropriate from being implemented? It will be interesting to see how the Norwegian legal systems will deal with this and related questions, after the new Act takes effect in 2016.

I conclude this section by addressing a new special rule that has been included in the new act, and which is specifically targeted at the planning authorities, encouraging them to make use of consolidation to achieve  greater fairness in public planning. These rules are contained in Chapter 5 of the new act and they target benefit that arises from planning in cases when the benefit appears to fall disproportionately on some owners. Such cases of ``windfall" benefit due to public plans are often flagged as problematic, and they arise with particular frequency in systems based on commercial implementation. For example, if one parcel of land is designated for housing and some neighbouring land is designated as a playground, it might easily come to be seen as unfair that a considerable financial benefit falls to the owner of the land designated for housing, while the playground owner is left with virtually nothing.

Following a change of the Consolidation Act in 2006 which has been further extended in the new act, the law now makes it possible for the planning authorities to decide that apportionment of the \emph{benefit} arising from the plan may be carried out by the consolidation courts.\footnote{See s 3-30 of the \cite{lca13} and s 12-7 no 13 of the \cite{pb08}.} When doing so, the consolidation court will follow the same procedure as in other cases, and it will allocate the benefit arising from the plan based on an assessment of the development potential of the different parcels. Importantly, the court will consider this question independently, and the decision will not be based on the particular manner in which the plan dictates that development is to be carried out. For instance, if the land used for the playground could just as well have been used for housing, the court may decide that the rights to housing development is to be shared equally between the two properties. On the other hand, if there is some independent reason why the playground property is not suited for housing, the court will reduce this property's share in the housing development correspondingly.

The court can implement solutions such as this more effectively and rationally by applying the other tools that it has available. For instance, if the the owner of he playground is entitled to an equal share in housing, then apportionment can be used to actually provide him with such a share, trading it for a corresponding share in the playground. However, if such material reallocation of development rights does not prove feasible, the new act also opens up for a solution where the benefit sharing is implemented using financial compensation.\footnote{See s 3-32 of the \cite{lca13}.}

To sum up, directives of use rules are highly versatile and may be used to organize extensive projects of land development on behalf of original owners. This form of development makes it possible for original owners to maintain their interest in the land, it can prevent the need for expropriation, and it may give the public a greater opportunity to exert influence and control over how their planning decisions are implemented in practice. In the next section, I consider in depth the particular case of hydropower, where the consolidation courts have recently started to make use of a wide arsenal of its tools to ensure that development can be carried out in this way. I think this case-study sheds further light on consolidation as an alternative to expropriation, and further strengthens the argument that directives of use issued by a consolidation court can in many cases obliterate the need for depriving local people of their resources to implement public development plans.

\section{Compulsory participation in hydropower development}\label{sec:lch}

In this section, I look at four recent cases in detail, all of which involved directives of use for hydropower development by original owners. The waterfalls dealt with in these cases are all located in the county of \emph{Hordaland}, in south-western Norway. Three of the cases involved small-scale hydro-power which some of the owners wanted to develop themselves, while the fourth was a case when the owners were also considering a development plan which would involve cooperation with an external energy company. The cases are particularly interesting because we have access to data on how the process of consolidation, and the outcome, was perceived by the owners themselves. Interviews were conducted and used in a recent master thesis on land consolidation which is devoted to the study of how consolidation measures is now increasingly being used to facilitate hydropower development \cite{stokstad11}.

In the following, I first present each case separately, focusing on those issues that were raised regarding how to organize development, the solutions prescribed by the court, and the subsequent reception among the parties. I then assess this from the point of view of developing a better understanding of compulsory cooperation as an alternative to expropriation. I conclude with some unresolved questions, particularly regarding those situations when the court is called on to resolve disagreement regarding how the development itself should be organized. These are the cases when the relationship between consolidation law and other legal frameworks, such as company law, planning law, and water law, becomes pressing, and there are many unresolved questions.

\subsection{\emph{Vika}}

The case was brought before the consolidation court in 2005, by owners who all agreed that hydropower development should be pursued.\footcite{vika05} The owners disagreed on how to organize the owners' association, and on how the shares in this association were to be divided among the different properties involved, 15 in total. The main principle was agreed upon from the start, however, namely that the owners would rent out their waterfall to a separate development company which every owner would have a right (but not a duty) to take part in. 

The parties in \emph{Vika} were highly involved in the consolidation process, and the statutes for the owners' association were based on suggestions made by the owners themselves. The main point of disagreement concerned how the shares in this association should be allotted, a question that was made more difficult by the fact that some owners benefited from old water-mill rights in the river. In the end, the consolidation court landed on the view that these rights were tied to the form of use relevant at the time they were established, and did not regard them as having any financial value. Hence these rights were extinguished without compensation, as provided for in Sections 2, 36 and 38 in the Land Consolidation Act 1979.

There was also some disagreement about whether the number of votes in the owners' association should be tied to the number of shares belonging to each owner, or if the owners should simply be allotted one vote each, irrespectively of their share in the waterfall. The consolidation court went for the first option, but the way in which they allotted shares in the owners' association deserves special mention. In particular, the court decided to take into account that some additional water entered the waterfall from smaller rivers where only a sub-group of the owners had waterfall rights. These owners' share in the association was increased accordingly, and this is surprising in light of Norwegian water law, as water rights are otherwise not tied to where the water comes from, but arises solely from the rights one has in the waterfall itself. 

The statutes of the owners' association also contains a second interesting provision, based on a suggestion made by the owners. It is a rule to the effect that all rights in the association are to be tied to the larger agricultural properties that give rise to them, and that they can not be divided from these properties and transferred to new owners separately. In Norway, such division of agricultural land would in any event require permission from the local municipality.\footnote{See Section 12 of the Land Act 1995.} In recent years, however, this protection of farming communities has grown weaker in practice, and it was the view of the owners in \emph{Vika} that a dissociation of water rights from underlying agricultural land should be forbidden altogether.

According to \cite{stokstad11}, interviews conducted with the parties demonstrated that a general consensus had developed whereby the land consolidation procedure was seen as a success. It allowed for an orderly and fair decision-making process regarding the conflicts that had arisen, and it was based on continuous interaction between the owners and the court, where everyone felt they had been given an opportunity to have his voice heard. Initially, tensions among the owners had been high, but the consolidation process had served to alleviate them. Some owners also pointed to the fact that the main hearing had been physically conducted in the local community, in a meeting hall that was familiar to the owners. This also gave them a feeling that they were meant to actively partake in the decision-making process. 

When the interviews were conducted, some 5 years after the case was concluded, the owners also appears to agree that the association was working as intended and that the climate of cooperation among the owners was good. The hydro-power scheme itself had been completed in 2008, yielding an annual production of around 15 GWh per year, providing enough energy for around 700 households. Moreover, following the experience of land consolidation, a culture of deliberation towards consensus had developed among the owners, and great emphasis had subsequently been placed on attempting to find common ground and to reach agreement on important issues. This was reflected, for instance, in the fact that the owner who contributed the land for the power station was given a generous annual fee, in addition to his compensation as a waterfall owner. According to \cite{stokstad11}, this fee exceeds what he would likely get if this decision had been left to the discretion of the consolidation court. Hence it reflected a premium that the owners were now willing to pay to ensure agreement and a continued good climate for cooperation.

I agree with \cite{stokstad11} that the case of \emph{Vika} serves as an example of how land consolidation can empower local communities and may enable them to embark on substantial development projects.

\subsection{\emph{Oma}}

The second case we will consider is the case of \emph{Oma}, which was brought before the courts in 2006.\footcite{oma06} In this case there were four involved properties. The owners of three of them, $A,B$ and $C$, wanted to develop hydro-power, while the fourth, owner $D$, was opposed to the development. Rather than attempting to expropriate the necessary rights from owner $D$, owners $A,B$ and $C$ took the case to the consolidation court. They argued that development would benefit all the properties involved, and also pointed out that a more restricted project, which would not make use of owner $D$'s rights, would be less economical. Hence in their view, the consolidation court should compel $D$ to cooperate in a joint scheme. Owner $D$ protested, arguing that the project would not economically benefit him, and that it would also be to the detriment of his plans to build cottages for holiday dwellers in the same area.

The case of \emph{Oma} differs from that of \emph{Vika} in that the question of whether it was appropriate to use compulsion was more prominent. In particular, this aspect came up already in relation to the question of whether or not hydro-power development should be pursued at all. As we discussed in Section \ref{sec:3}, the fact that some owners do not desire development does not prevent the consolidation court from putting directives in place to facilitate it, but the courts often exercise restraint in such cases. In \emph{Oma}, however, the court agreed with the majority of the owners argued that an owners' association with compulsory membership should be set up. In doing so, the court relied on Section 2 c) of the Land Consolidation Act 1979. To justify the use of compulsion against $D$, the court first observed that joint development of hydro-power would benefit all the properties in question, including $D$. Then they commented specifically on owner $D$'s plans for building of cottage homes, noting first that he was unlikely to be given planning permission, and secondly that hydro-power would not in any event adversely affect such plans in any significant way. Moreover, the court noted that while owner $D$'s rights were relatively minor, they were quite crucial for the profitability of the project, particularly because owner $D$ controlled the best location for the construction of a dam to collect the water used in the scheme. Overall, the court's conclusion was that a joint hydro-power scheme would be a better option for everyone than a project that did not include owner $D$'s property.

The question then arose as to how the shares in the owners' association, and the right to rent that would go with it, should be divided among the owners and their land. In regards to this question, the court departed significantly from one of the basic principles that have been entrenched in Norwegian water-law since the early 20th Century. The principle in question states that no right to hydro-power can be derived from being in possession of land suitable for the construction of dams or other facilities necessary to exploit the waterfalls.\footnote{The principle was is reflected in Supreme Court decisions as early as \emph{Herlandsfossen} and \emph{Drammenselven}, Rt. 1922 p. 489 and Rt. 1923 p. 185 respectively, and has been maintained consistently ever since.} But the land consolidation court broke with this principle in the case of \emph{Oma}, deciding instead to set the value of the land designated for construction of a dam and a power station to represent $6 \%$ of the total value of the rights that went into the owners' association. The proportion of financial benefit and decision-making power awarded to the unwilling owner $D$ thus increased accordingly, since these right were all held by him. In fact, his share went from $1.75 \%$ to $7.75 \%$, so the consolidation process itself led to a situation where he would have a far greater incentive for supporting the development. In many ways, the decision in \emph{Oma} was more to the benefit of owner $D$ than any other among the involved parties. If the rights in question had been expropriated, for instance, he would be given next to nothing in compensation and would lose his rights forever. Instead, the solution prescribed by the consolidation court gave him a lasting and substantial interest in local hydro-power.

According to \cite{stokstad11}, interviews with the parties shows how the process and outcome of consolidation in \emph{Oma} served to create a much better climate for further cooperation among the parties. Indeed, when the interviews where conducted, 4 years after the courts' decision, owner $D$ had changed his mind and was now in favour of the development. Moreover, he had also decided that he wanted to take part in the development company. He was not obliged to do so, but his right to take part was encoded in the deal with the development company, as detailed in the statutes of the owners' association and in keeping with Section 34 b) no 3 of the Land Consolidation Act 1979.

The owners all reported that the consolidation process had been very successful and that the court had listen to them, allowing everyone to have their voices heard. Moreover, some owners reported that the court had cleverly maintained a ``birds eye view" on the best way to develop the land in question, ensuring both long terms benefit to all involved properties as well as creating an improved climate for cooperation and mutual understanding. The consensus was that making concessions to owner $D$ was appropriate and had been in the interest of all the involved parties. In 2011 the hydro-power project in \emph{Oma} was completed and today its output is roughly 3 GWh per year.

We think the case of \emph{Oma} serves as a good illustration of how consolidation can be an effective instrument for facilitating locally controlled development, also in cases when this requires the use of compulsion against some owners. Interestingly, in this case the successful outcome appears to be partly due to the fact that the consolidation court actively used its discretionary powers when deciding how to organize joint use. This power allowed them to deviate from established rights-based legal doctrine and adopt a more context-dependent approach, pursuing solutions that suited the situation better. Interesting legal questions arise in this regard, particularly regarding the competence that the consolidation court has in such cases, and the extent to which decisions can be made subject to review by the normal courts on the basis that they do not follow established principles and practices. For instance, one may ask what would have happened if the majority owners in \emph{Oma} had appealed the decision to the regular courts on the basis that $D$ was awarded too many shares in the owners' association. Would this be regarded as a question of the court's interpretation of the law regarding the owners' \emph{rights}, or would it be regarded as a discretionary decision regarding the best way to organize development? In the first case, the decision would almost certainly have been overturned on appeal, but in the latter case it would likely be beyond reproach.

A second interesting question that arises is whether or not consolidation can work as well as it did in \emph{Oma} in cases where conflicts run more deeply, or where the parties favoring development are a minority among the owners. The next two cases we consider shed some light on this issue.

\subsection{\emph{Djønno}}

This case was brought before the courts in 2006, by a local owner $A$ who wanted to develop hydro-power in a small river crossing his land, the so called \emph{Kvernhusbekken}.\footcite{djonno06} This owner wanted the court to help him implement a hydro-power project, by compelling the other owners, $B, C$ and $D$, to rent out their share in the necessary rights on terms dictated by the court. The starting point for the other owners was that they did not want any hydro-power development at all, and they were not willing to rent out their rights to owner $A$ or any other developer. There was also a dispute regarding the ownership of the waterfall rights, with $A$ believing initially that he controlled a large majority. It soon became clear that this was not the case, and before the main hearing the parties agreed that the waterfall in question was owned jointly with shares divided according to each owners' share in unconsolidated farming land. Owner $A$'s share in these rights did not amount to more than $5 \%$, so his own financial interest in hydro-power was in fact limited compared to the owners who opposed development.

On the other hand, the rights needed for the necessary physical constructions were predominantly held by owner $A$ alone, and $A$ maintained his position that the court should use compulsion to allow him to go on with his plans. The court agreed that hydro-power would be rational use of the waterfall, and they initially assessed the case against Section 2 e) regarding compulsory joint undertakings. A decision made on the basis of this provision would allow the court to give more concrete directives regarding how the hydro-power development should be carried out, but in the end the court held that this would place too much of a burden on the owners opposing hydro-power. Hence they chose to decide the case on the basis of Section 2 c), as in the other cases we have considered. By doing so they also restricted the scope of their decision to the establishment of an owners' association that would be responsible for renting out the rights. The court would not consider the question of deciding on a concrete scheme.

The model used for the owners' association was similar to the one the court adopted in \emph{Oma}. This included an adjustment of the rights in the owners' association reflecting the special importance of land needed for physical constructions. In total, these rights were estimated at a value corresponding to $6 \%$ of the shares in the association. Since these rights were all held by owner $A$ alone, his share in the association was doubled. In addition to this, owner $A$ purchased the shares from owner $B$, so that his total share ended up amounting to $22 \%$. Still, for the majority of stakeholders, membership in the association was imposed by the consolidation court against their will.

The wording of the statutes for the association apparently attempts to take into account that it would be run by a majority of unwilling shareholders. The wording used is different from that used in the other statutes, and it is stated in very clear terms that the association is going to rent out the rights in the waterfall such that hydro-power can be developed. In \emph{Oma} and \emph{Vika}, on the other hand, the statutes merely state that this is the \emph{purpose} of the association, leaving the shareholders with greater freedom to determine whether or not to go through with development.

More generally, it seems that in the case of \emph{Djønno} the court attempted to facilitate development not so much by trying to make the owners more positive towards development, but rather by giving the proponent more power, providing him with a starting point which would make it easier to later enforce concrete hydro-power plans.

In interviews, those who were compelled to take part in the association against their will expressed dissatisfaction and surprise at the result. Moreover, while the association had ostensibly tried to be loyal to the wording of the statutes, and had looked for partners who might be interested in developing hydro-power, there had been no willingness among the majority to engage actively with this work. No deals had been made, no separate development company had been set up, and the conflict among the owners was ongoing. As of 2011, owner $A$ was still pushing for development on terms that were unacceptable to the other owners. Hence while the case of \emph{Djønno} is an example that consolidation can be used even when it involves compulsion against the majority of owners, it also serves to illustrate that the chance of a successful outcome may then be more limited.

The question arises as to why this is so, and how such cases will be dealt with by courts in the future. According to owner $A$, the problem was that the directives of use were not specific enough and that they should not have been restricted to merely setting up an owners' association for renting out the rights. In this case, more was needed. The court should actively engage also with the question of how the development company should be organized, and at least give guidance as to \emph{who} should be set with the task of carrying it out. Among the majority owners, on the other hand, the feeling appears to have been that the development in question, which they would be required to partake in against their will, was more or less doomed to fail already from the start.

This reflects two interesting viewpoints regarding such cases. Indeed, it seems reasonable to assume that unless one is prepared to see an increase in the use of compulsion, compulsory cooperation will only work when at least a basic agreement that development should take place can be established among the majority of the involved owners.

\subsection{\emph{Tokheim}}

This case was brought before the consolidation court in 2008, by the owners of \emph{Tokheimselva}.\footcite{tokheim08} The five involved owners all agreed that development should take place, but they disagreed about how it should be done, and about the proportion of each owners' share in waterfall. Some owners argue that development should be organized by the owners themselves, but other owners thought it would be best to rent out the rights to an external developer. The case was further complicated by the fact that the waterfall in question was so big that it would be possible to develop hydro-power that would require transferral concession pursuant to the Industrial Concession Act 1917, a concession that can only be given when the purchaser is a company where the State controls at least $\frac{2}{3}$ of the shares. 

Like the precious cases we have considered the consolidation court eventually based its decision on Section 2 c) of the Land Consolidation Act 1979, setting up an owners' association such that each owner was allotted a share in accordance to the rights that the court found he had in the waterfall. Unlike the previous cases we have considered, there was no adjustment made for land that would be needed for physical constructions. However, the statutes state that owners will be entitled to a lump sum estimated on the basis of the damages and disadvantages that a concrete hydro-power project will bring. This also marks a departure from established practice in expropriation law, where it has been a long established principle that owners can be compensated on the basis of \emph{either} the value of their waterfalls \emph{or} the damages and disadvantages caused by the project, not both.\footnote{See for instance the case of \emph{Vikfalli}, \cite{vikfalli71}.} 

In other respects, the statutes for the owners' association follow the same model adopted in the previously considered cases. They do not, however, resolve any of the controversial questions regarding how development should be carried out, and the question of the extent to which interested owners should be given the opportunity to develop the resource themselves. This was the issue that the main conflict in the case centred on, and the consolidation court explicitly decides not to implement any solutions in this regard. In particular, the statutes of the owners' association explicitly provides separate rules that cover both the case that a group of owners undertake development themselves, and the case that development is carried out by an external company. 

In interviews, the owners expressed that they were happy with how the case was dealt with by the court. Everyone appears to have been heard, and the owners' association was set up in consultation with the parties. However, the main issues were still unresolved as of 2011, and this was felt by the owners as a major shortcoming of the outcome of consolidation. Some of the owners expressed criticism against the court for not engaging more actively with what appeared to be the most pressing issues.

The case of \emph{Tokheim} serves to illustrate that established practices of consolidation, while being well received and understood by local owners, face some new challenges in relation to hydro-power, challenges that consolidation courts might be reluctant to take on. It seems that the court in \emph{Tokheim} felt that they were not in a position to assess the question of what kind of development would be best, and it also seems that they were wary of expressing any opinion about the legal status of a project led by local owners, in relation to concession law. They did not, in particular, form an opinion about whether it would be possible for local owners to carry out their own large scale development, in a waterfall that might otherwise be subject to the provisions set out in the Industrial Concession Act 1917.

It remains to be seen whether such an agnostic attitude can be maintained by the consolidation courts as local owners increasingly turn to them for help in resolving disputes regarding hydropower. Moreover, it will be interesting to see how the new Land Consolidation Act 2013 will influence case law in this area. It seems that a case like \emph{Tokheim} could benefit from the court taking a broader view, possible even including public bodies as parties in the case, as will become possible when the new Act takes effect. In this way one could perhaps have hoped for a more conclusive outcome, a solution that gave sufficient consideration both to the public interest and the interests of local owners.

\section{Assessment and Future Challenges}\label{sec:lca}

The concrete cases that I discussed in the previous section shows, in my opinion, that the system of land consolidation is well suited as an alternative to expropriation in the context of hydropower development. At the same time, the cases suggest that the land consolidation courts may find it hard to deliver effective directives of use in situations when the different stakeholders disagree fundamentally about how the water resources should be managed. In addition, one may question the effectiveness of land consolidation courts in contexts when rules and regulations from other areas of law come into play. It seems, in particular, that the land consolidation courts may be cautious about implementing solutions that they fear will raise questions in relation to the special legal provisions that regulate the form of economic development that their directives aim to facilitate.

In so far as the sector-specific rules disadvantage owners and benefit external commercial interests, as is the case for hydropower development, one may rightly fear that the land consolidation courts will become impotent in situations when powerful market players enter the scene. It may be considerably easier to strike a fair balance between the interests of local farmers of comparable economic and political standing, then to do the same when one of the stakeholders is a partly state-owned power company that is accustomed to expropriating the property rights that it desires.

Paradoxically, the impotence of the land consolidation courts may be enhanced by the fact that they are not authorized to make use of appropriate forms of compulsion against owners, on pain of interfering too much in property as an individual right. This, in particular, threatens to undermine the effectiveness of the land consolidation court as an alternative to expropriation, making it possible to argue that the public interest in development can not be sufficiently accommodated through the use of consolidation measures. 

In fact, there is some evidence to suggest that land consolidation law might offer \emph{too} much protection to owners in order to circumvent such objections. One example is the Supreme Court case of 
{\it Holen v Holen}, concerning a quarry owned by a local farmer and landowner.\footcite{holen95} In order to continue extracting his minerals, the owner of the quarry would have to interfere with the property of a neighbouring owner, who was using his land for more traditional forms of agriculture. This owner was unwilling to reach an agreement with the quarry owner, so the latter brought a case before the land consolidation court. The court noted that it would be possible to reach an accommodation that would benefit both parties, and issued directives of use that would allow the quarry to continue its operations.

The directives involved giving the agriculturally minded farmer a replacement property, to make up for his loss of the property needed to access the minerals. However, the consolidation court also noted that the quarry would, in the future, also be likely to extract minerals that belonged to this owner. For the minerals as such, awarding replacement property made little sense, so the court decided that the minerals should still belong to the previous surface owner (who had no interest in extracting them). However, a directive of use was issued that gave the quarry owner a right to extract these minerals, provided he paid market value to the owner. 

Hence, not only was the farmer awarded replacement property for agricultural purposes, he was also granted a share of the benefits that would result from the continued operation of his neighbour's quarry. This, it seems, was clearly beneficial to his property, economically speaking. The owner himself, however, objected to the arrangement, since he was opposed to the quarry as such. The Supreme Court found in his favour. Interestingly, this was not because they sanctioned his right to oppose the continued operations of the quarry, or because they thought the replacement property or the payment model was inappropriate. Instead, the Court held that the right to extract the farmer's minerals could not be transferred to someone else, even if the farmer was ensured payment. This, the Court held, was a compulsory measure that fell outside the scope of use directives in land consolidation.

The perspective underlying this decision is interesting, because it underscores a reluctance to use land consolidation in what would otherwise be a fairly typical expropriation scenario. As such, it also raises doubts about the feasibility of proposing land consolidation as a practical alternative for such scenarios. However, {\it Holen v Holen} was decided in 1995, and as I have already mentioned, the legislature has signalled a shift in the law in recent years, by explicitly facilitating the use of land consolidation as an alternative to expropriation in certain circumstances. This has been criticized, however, by scholars arguing that private property rights receive a more adequate form of protection when normal expropriation procedures are observed. In light of earlier case law, this criticism must be taken seriously, also a possible formal objection against awarding the land consolidation courts increased powers of compulsion. Today, the exact relationship between land consolidation and expropriation law, including the constitutional property clause, appears to be an increasingly relevant open question that awaits further clarification in case law. 

I would like to stress, however, that I do not agree with those who argue that land consolidation offers less protection to owners than administrative expropriation. The property protection offered in the context of land consolidation is quite different, but not necessarily weaker. This, moreover, depends on one's vision of property, and what property values one deems to be most in need of protection. An administrative expropriation procedure might offer more {\it formal} safeguards. A range of procedural rules must be observed, pertaining to notification to the owners, impact assessments, a duty to provide guidance and reasons for the decision, and a possibility (sometimes several) for administrative appeal. Then, after an expropriation order has been granted, the owner can challenge its validity before the appraisal court (which also awards compensation), in principle at the expense of the expropriating party. 

In practice, however, the administrative expropriation procedure can easily leave the owner marginalized, as they are overshadowed by other more powerful stakeholders that are not property owners. This is particularly clear in situations when expropriation arises as a result of more comprehensive planning or licensing procedures, that do not focus on the owners' interest. As discussed in Chapter \ref{chap:x}, this as is the case, for instance, in the context of hydropower development. In addition to this, the possibility of raising validity objections before the courts is mostly a theoretical one. It is very unusual for such objections to be made successfully, as the courts typically defer to the discretion of the administrative decision-maker in expropriation cases.

In the context of land consolidation, on the other hand, the interests of the owners are meant to occupy center stage throughout the proceedings. Moreover, the owners have a formal standing in a deliberative and adversarial context, presupposing their active input to a greater extent than in the context of an administrative decision. In addition, the {\it grounds} for imposing compulsory measures that interfere with property rights need to be anchored specifically in the interests of the affected properties themselves. A measure is warranted only when it benefits the properties as such, in addition to whatever broader societal benefits that might arise. Clearly, this latter principle offers substantial protection of a kind that is completely absent in the context of administrative expropriation. In these contexts, rather, the premise is that the the affected properties and their owners will suffer disadvantages and losses that they can be compelled to bear in the public interest.

Such a narrative is often unavoidable for typical public interest takings, but seem misplaced in the context of takings for profit, since these will as a matter of fact increase the value of properties that are taken. Here, the land consolidation approach seems appropriate, also in situations when interference in established property rights appear necessary to facilitate overall benefits to the community of property owners. Of course, there are challenges that must be dealt with, particularly when some property owners appear to benefit more than others, or when the notion of benefit itself is hard to pin down because property owners disagree about the most important property values inherent in their land. However, it seems to me that a procedural framework that focuses on the community of property owners as the primary stakeholder, is well suited to dealing with these challenges. Much better suited, it would seem, than an administrative decision-making process that conflates the taking for profit scenario to a run-of-the-mill expropriation scenario or an instance of spatial or sector-based planning, with expropriation as a mere side-effect.

Hence, I conclude that principled objections against land consolidation in expropraition contexts appear largely misplaced when for the sub-group of takings that realise commercial potentials. However, a second question arises, of a more practical nature. Will the land consolidation process work in practice, if it is applied to organize commercial development. Increased powers of compulsion might be required, and in keeping with my argument above, I believe such powers may well be granted, as long as land consolidation remains directed at improving the situation for existing properties and their owners, rather than bestowing benefits on someone else. A second question, which I think is far more challenging, concerns the future development of the land consolidation procedure itself. Will it remain a service to owners, placing them at the center of attention, even if its scope is broadened and other, more powerful, stakeholders enter the stage?

It is too early to say, since there has not yet, to my knowledge, been any cases where the issue has come into focus. However, as any legal person with a right to expropriate may now act as a party to a consolidation dispute, the question is bound to arise, in various forms. What will the role of the new parties be? Is the land consolidation procedure still going to be a service to owners, providing a forum for equitable interaction with potential developers, or will it become a service to developers, providing a template for cheap and easy access to property? It will be very interesting to follow this development further, to see if the promise of using land consolidation to regain legitimacy for the use of compulsion to facilitate economic development can be fulfilled.

\section{Conclusion}\label{sec:conc}

In this Chapter, I have addressed land consolidation as an alternative to expropriation for economic development, anchored in a case study of hydropwer. I started by presenting the basic idea of using land consolidation in this way, emphasising that the notion of consolidation at work here is a broad notion that includes measures seeking to enforce particular uses of property. I briefly presented a comparative vision of this kind of land consolidation, noting that a broad notion is a work in many jurisdictions. I then focused specifically on the Norwegian context, where the judicial decision-making framework for land consolidation sets the procedure apart from that found in many other jurisdictions. I also noted how the procedure is conceputalized as a service to owners, with a no-loss guarantee in place to ensure that consolidation measures are only implemented when the benfefits make up for the harms for all the involved properties individually.

I then went on to present the Norwegian system in more detail, focusing on procedural aspects, particularly those related to so-called use directives, that empower the consolidation courts to impose and organize joint use of property rights, including economic development projects. I noted how recent changes in the law envisions an extended scope for these rules, including in the context of non-agrarian and urban development. I then went on to consider some concrete examples, from the context of hydropower development, where owner-led projects already tend to rely on land consolidation rather than expropriation, to facilitate development. I concluded that while the land consolidation alterantice works well when there is basic agreement among the owners that development is desirable, it seems somewhat less effective when there is deep disagreement about how, or whether, development should proceed. 

In these contexts, I argued, it might be necessary to enhance the power of the land consolidation court, also in the direction of increasing its power to compel land owners to take part in, or allow the implementation of, development projects that they disagree with. While this is already possible, to some extent, the power of the land consolidation court in this regard appears somewhat limited, particularly in light of earlier case law that has stressed the distinction between consolidation and expropriation. However, recent legislative developments suggest that this is perspective is now changing, with an increased emphasis on land consolidation alternatives even in cases that require quite severe interferences with the interests of individual property owners.

I argued against those that see this as a threat to property, by pointing out that the formal protection awarded to owners through administrative law is hardly as practically relevant as the fact that the land consolidation process, as traditionally administered, is continuously centred on owners and property interests, making sure that external interests, particularly private interests, can not dominate the process. This, however, might be set to change now that such actors are about to receive a new formal standing in consolidation disputes, and will be granted the opportunity to bring cases before these courts themselves, if they favour it over expropriation. On the one hand, his change will enhance the power of the land consolidation court, making it more effective in dealing with cases that involve external parties. On the other hand, there is a possibility that the presence of new and powerful stakeholders will change the nature of the land consolidation process itself, so that it becomes yet another planning instrument that favour powerful developers, not a property-enhancing institution that promotes self-governance. 

I believe, however, that the land consolidation regime in Norway functions in a way that sheds interesting light on collective-action alternatives to expropriation. It is also a dynamic and versatile framework, more so than other suggestions, such as the land assembly districts proposed by Heller and Hills. In general, it seems that decision-making in a context where local interests are largely vested in property rights require special procedures, if one is to prevent the formation of a democratic deficit. It simply appears imbalanced to make use of the standard administrative planning institutions in such cases, particularly when these institutions are dominated by external, commercial, actors. This is particularly clear when the democratic grounding of these institutions is weak, or centralized, as is the case for Norwegian hydropower. In my opinion, the institution of land consolidation can provide a useful kind of democracy-on-demand for such sectors, facilitating a better balance between the interests of local community, the interests of commercial actors, and the interests of society as a whole.