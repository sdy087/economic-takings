\chapter{Compulsory Participation in Hydropower Development}\label{chap:6}

\section{Introduction}\label{sec:intro6}

In this Chapter, I will consider an alternative to expropriation in the context of economic development. This picks up the thread from the final section of Part I, where I presented and analysed the land assembly proposal put forth by Heller and Hills in the US. I now return to this way of approaching the issue of economic development takings, by exploring the Norwegian institution of {\it land consolidation}.

In recent years, this institution has been used extensively to facilitate hydropower projects. So far, however, it is used almost exclusively for small-scale development projects organised by local owners. In these situations, expropriation orders are rarely sought and rarely authorised, even if some owners object to the plans. Instead, various consolidation measures are used, including the practically important ``use directives'', serving to set up organisational frameworks for compulsory implementation of a development plan, possibly against the owners' own wishes. Importantly, however, use directives presuppose the continued participation of owners in the development, making this measure clearly distinct from a traditional taking.

Essentially, a use directive can be used to take some of the holdout power away from owners, without depriving them of their property. Instead, owners are compelled to cooperate and participate in a decision-making process that has economic development as an overarching, binding, aim.

The importance of this kind of land consolidation is most clearly felt in relation to traditional agrarian pursuits and,  more recently, hydropower development. However, recent legislative developments in Norway mean that the consolidation alternative is also gaining importance in relation to other forms of development. This includes urban and non-agrarian development projects, marking a departure from the tradition of land consolidation as a purely agricultural institution. 

Some argue that novel uses of land consolidation leave the owners in a precarious position by weakening private property rights.\footnote{See, e.g., \cite{stenseth07}.} In this Chapter, I explore the opposite hypothesis, namely that the use of consolidation for economic development can strengthen property as an institution, particularly when use directives replace traditional expropriation proceedings. In fact, I will argue that the Norwegian system of land consolidation can be used to address the democratic deficit of economic development takings in an elegant way.

This vision presupposes that the land consolidation process continues to function as a service to owners and local communities, as a means of helping them to implement development projects in accordance with public interests. Arguably, this requires a clear commitment on part of the state to prevent abuse of consolidation measures by commercial interests and public-private partnerships that seek access to property rights held by weaker parties.\footnote{As I will discuss in this chapter, the main worry in this regard is that the land consolidation procedure itself might be transformed, if the consolidation alternative gains relevance in relation to large-scale economic development. To address this worry, I will consider the safeguards that are meant to protect the integrity of the traditional process, to assess whether they are strong enough to prevent the process from degenerating when new and powerful actors enter the consolidation scene.} Assuming that such a commitment is made, to protect the integrity of the established system, I will argue that the land consolidation alternative is a highly promising way to deal with many of the challenges that arise at the intersection between private property, local community, and economic development in the public interest. I also believe the Norwegian model can inspire similar solutions elsewhere, particularly in jurisdictions that are committed to an egalitarian ideal of property ownership.

The structure of the chapter is as follows. I begin in Section \ref{sec:lce}, by presenting the basic idea of using land consolidation as an alternative to expropriation. I first discuss land consolidation as a concept, relating this also to the discussing found in Heller and Hills' article. Then I point out some special features of the Norwegian system. I argue that this system has special features that make it particularly natural to consider further as an alternative to expropriation in the context of economic development.

Then, in Section \ref{sec:lcc}, I present the Norwegian system of land consolidation in more depth, focusing on the procedural rules and the rules that protect property rights against disproportionate interference. I focus particularly on the so-called ``no-loss'' guarantee, which states that a consolidation measure can only be implemented when the benefits will make up for the harms, for all affected properties individually.\footnote{In terms of economic theory, this amounts to requiring that all measures should lead to Pareto improvements, see \cite[59-61]{miceli11}. Moreover, what is required is actual improvement, not merely {\it potential} improvement (known as Kaldor-Hicks improvement, see \cite[61-63]{miceli11}). Unusually, the no-loss guarantee requires Pareto improvements in or in relation to the affected {\it properties}; no mention is made of their respective individual owners. As a result, the no-loss criterion is also averse to the monetization of benefits and harms -- it is normally not possible to fulfil the no-loss requirement by paying compensation to the owners of adversely affected properties, see \cite[394]{sky09}. Improvements must typically be rendered in kind, in a manner that offsets potential losses to the property as such, independently of the owner's own (hidden or revealed) valuations, see \cite[371-372]{sky09}. Hence, the starting point here is completely different from that normally assumed in economic analyses of takings law, where complete monetization and individuation is standard (usually starting from the notion of the owners' {\it reservation price} -- the price at which they would be willing to sell if they behaved non-strategically).}

This safeguard is itself an indication that land consolidation is quite distinct from expropriation. In particular, the consolidation narrative does not rely on a perspective whereby interference takes place because property rights must give way to public interests. Rather, consolidation relies on proof that benefits will outweigh harms at the local level, with respect to each affected property. However, this requirement targets the property as a functional unit, irrespective (in principle) of the specific interests of its current owner. Hence, depending on what functions are regarded as more important, the interests of the owner might have to give way to other, locally grounded, priorities.
%At the same time, the compensatory perspective is abandoned; the owners' financial entitlements are simply subordinated to the interests found to be inherent in their properties. No compensation is payable as long as their properties benefit. 

Hence, land consolidation in Norway relies on a very clear commitment to a social-function perspective on property: property functions, not individual entitlements, take center stage throughout the process. This might limit the power of owners, but it does not marginalise them. After all, it is hard to deny that one of the primary functions of private property is to bestow rights and obligations on its owner. Moreover, in normal circumstances, it would be safe to assume that when a property benefits, then so does whoever owns it. In addition, as I will discuss below, the land consolidation process contains special safeguards that protect owners as individuals.

Importantly, the consolidation perspective can justify imposing economic development project against the wishes of owners, without giving rise to a marginalising eminent domain narrative whereby all links are to be severed between the original owners and their property. Indeed, the no-loss criterion can typically be satisfied in cases of commercial development, through appropriate forms of benefit sharing anchored in local property units (sharing the benefit with the owner as an individual is, in principle, not sufficient). In this way, land consolidation becomes a practically feasible alternative to economic development takings, in a manner that also underscores the current owners' right to participate in, and derive a profit from, the undertaking.

In Section \ref{sec:lch}, I explore this idea concretely, in light of empirical evidence. Specifically, I discuss the use of land consolidation as an alternative to expropriation in relation to hydropower development. I consider several cases in detail, based on court documents and a recent master thesis on land consolidation, for which the author carried out interviews with affected owners.\footnote{See \cite{stokstad11}.} Then, in Section \ref{sec:lca}, I offer an assessment and discuss some future challenges. I also note some shortcomings of the current system, while arguing that these do not detract from the great potential of using consolidation to facilitate economic development, particularly in commercial settings.

\section{Land Consolidation as an Alternative to Expropriation}\label{sec:lce}

The notion of land consolidation is widely used on the international stage, but it is somewhat ambiguous. Often, it refers to mechanisms whereby boundaries in real property are redrawn to reduce fragmentation, without affecting the relative value of the different owners' holdings.\footnote{See, e.g., the entry on {\it land consolidation} in \cite{mayhew09}.} However, it is also common to use consolidation to refer to mechanisms for pooling together small parcels of land to create larger units.\footnote{See, e.g., \cite{lerman06}.} There is a tension between these two notions of consolidation, with some claiming that consolidation in the latter sense is sometimes used to surreptitiously bestow benefits on powerful property owners, at the expense of weaker groups.\footcite[237-239]{lipton09}

In light of this, I should stress at the outset that I will use the term land consolidation in a very broad sense in this chapter, much wider than {\it both} of the interpretations mentioned above. Land consolidation, as I use the term, refers to any mechanism by which the state intervenes, at the request of some interested party, to (re)organise property rights and uses in a given local area. Hence, a consolidation measure might as well involve {\it increased} fragmentation of property, if this is deemed a rational form of consolidation of the property {\it values} of the affected area. Importantly, I also use land consolidation to refer to efforts directed at {\it managing} property, not just redrawing boundaries.

Some might argue that this terminology is strained, but I adopt it for a reason. It is motivated by the fact that in Norway, the institution known as ``jordskifte'', which is officially translated as land consolidation, has exactly such a broad meaning.\footnote{See, e.g., \cite{reiten09,rognes03}.} I note that land consolidation also has a broad scope in many other jurisdictions of continental Europe, as well as in Japan and in parts of the developing world.\footnote{See \cite{sky07,vitikainen04}.} Moreover, in Heller and Hills' work on land assembly districts, a comparison with land consolidation is presented, based on broad definitions of that term.\footcite{heller08} One of my main aims in this chapter is to pick up on this, by offering a more detailed comparison and assessment, specifically anchored in the Norwegian system and its application in the context of hydropower development. The Norwegian system deserves special attention in this regard because it is particularly broad, especially in its authority to issue use directives.

As land consolidation tends to involve interference in property rights, one may ask about the legitimacy of various consolidation measures, held against rules that protect private property owners.\footnote{For an analysis of the Norwegian land consolidation process held against the provisions of the ECHR, I refer to \cite{utgard09}.} Such legitimacy issues have  been raised before the ECtHR on a few occasions, resulting in the Court finding fault with the Austrian system of land consolidation in particular.\footnote{These decisions must be understood in light of the potentially excessive duration of consolidation proceedings under Austrian law, during which restrictions are also imposed on the owners, limiting their opportunities to enjoy their properties while awaiting a final outcome. See, e.g, \cite{erkner87,poiss87}.} Moreover, one may sometimes argue that a land consolidation measure {\it is} a form of expropriation, even if it is not recognised as such by the legislature or the executive. In the US, for instance, a land consolidation provision ordering escheat (to Indian tribes) of fractional property interests in Indian reservations was struck down as an uncompensated taking by the Supreme Court.\footnote{See \cite{hodel87}.}

\noo{ Move: On the other hand, if land consolidation is used to facilitate or impose specific uses of property, it can also be used as an {\it alternative} to expropriation, a compulsory measure that can obviate the need for depriving owners of their property rights. I think this latter perspective on land consolidation is particularly interesting, and it is the perspective I adopt in this chapter.}

In relation to the legitimacy issue, the Norwegian system stands out in two important respects. First, the consolidation procedure is managed by judicial bodies, namely the {\it land consolidation courts}.\footnote{See generally \cite{langbach09}. The fact that the land consolidation process is administered by a judicial body appears to be unique to Norway, see \cite[45]{sky01}.} Second, land consolidation is largely seen as a service to owners, not a tool for increased state control and top-down management.\footnote{See generally \cite{sky09}.} In particular, a case before the land consolidation courts is almost always initiated by (some of) the affected owners themselves and the court often acts as a ``problem-solver'', aiming to facilitate dialogue and cooperation among owners.\footnote{See generally \cite{rognes98,rognes03,rognes07}.} Moreover, the no-loss requirement is a core principle of consolidation law, ensuring that no consolidation measure can take place unless the benefits make up for the harms, for all the properties involved.\footnote{See the \cite[3 a)]{lca79}.} Indeed, this remains on of the key principles of land consolidation in Norway.\footnote{See generally \cite{rygg98}.} The combination of a judicial procedure that emphasises owner-participation and a no-loss criterion that ensures local benefits means that, arguably, land consolidation in Norway {\it strengthens} property as an institution.

Moreover, land consolidation can serve as an effective countermeasure against two of the most widely discussed challenges to any property regime. First, consolidation can serve to protect an egalitarian distribution of property rights against the deleterious effect of inefficiency and underdevelopment that might otherwise arise from fragmentation. Importantly, it can do so without disturbing the underlying property structure and without bestowing disproportionate benefits or harms on certain owners or other select groups (assuming egalitarian property rights and/or legal standing for local property dependants in consolidation proceedings). In particular, land consolidation can ensure commercial development without pooling together property rights and without handing property over to powerful market actors. Second, land consolidation can serve to ensure sustainable and rational management of jointly owned land, without necessarily forcing an enclosure process (enclosure {\it can} be the result of land consolidation, but it is only one of many measures in the consolidation toolbox). 

In short, land consolidation can be used to address both commons\footcite{hardin68} and anti-commons\footcite{heller98} problems, in a way that protects, and possibly enhances, desirable social functions of property, through a judicial system that combines participatory and adversarial decision-making. Hence, land consolidation in Norway is based on a conceptual premise that -- potentially -- offers protection to owners and their properties, by recognising them as members of a community that are mutually dependent on each other. In this way, the form of property protection offered in the context of land consolidation is distinct from the protection offered in the context of expropriation. This in itself is interesting, particularly from the perspective of property's social functions.

The vision of land consolidation at work here is one that sees it as a means for setting up a mini-democracy on demand, to organise decision-making processes in a way that grants those most intimately affected -- the owners and (possibly) other property dependants -- a say that is proportional to their stake in the matter at hand. As discussed in Part I, this is something that it often seems hard or impossible to achieve through the standard administrative/political route, particularly when powerful commercial actors engage in extensive lobbying and are allowed to assume the position of primary stakeholders in projects involving the property of others.

Importantly, since land consolidation can be used to impose specific uses of property, it can also be an {\it effective} alternative to expropriation, a compulsory measure that can obviate the need for depriving owners of their property rights. Depending on the compensation regime, the costs associated with eminent domain can be higher than those associated with land consolidation.\footnote{This is the case, for instance, when consolidation is used to facilitate owner-led hydropower development, as discussed below.} Clearly, consolidation as an alternative to expropriation is particularly natural for economic development projects. The no-loss criterion will typically be possible to fulfil in these cases, through benefit sharing. Moreover, it becomes the responsibility of the land consolidation court to {\it ensure} that a sufficient degree of benefit sharing results, so that consolidation measures may be applied in accordance with the law.\footnote{For a detailed discussion of the extent of the court's duties in this regard, also discussing recent changes in the law that might indicate a weakening of the no-loss guarantee, see \cite{hauge15}.}

Interestingly, it is usually also assumed that the benefits resulting from consolidation should be distributed among the affected properties in accordance with their relative value prior to the consolidation measure\footnote{This principle is not as strictly encoded as the no-loss criterion, but is formulated as an ``ought''-rule. See the \cite[31|41]{lca79}. In my opinion, this is a weakness of the current framework. I mention that for the special case of consolidation to implement a zoning plan, the rule is absolute, see \cite[3 b)]{lca79}. See also \cite{hauge15}.} Hence, it may be argued that the principle of benefit sharing at work here is not compensatory at all, but rather one that sees the owners as active participants in the development project, even when it takes place against their will. I find this highly interesting, particularly from the point of view of the social obligation and human flourishing conceptions of property that I discussed in Part I. Under such theories, it makes sense to impose obligations on owners to participate in the fulfilment of public interests, particularly when they themselves also stand to benefit from doing so.

The emphasis on benefit sharing in land consolidation also reveals a concrete advantage of this institution compared to traditional expropriation. In particular, while benefit sharing is typically required under consolidation law, it is hardly ever achieved through compensation in the context of expropriation for economic development.\footnote{This is largely due to the so-called {\it no scheme} principle, which states that compensation to the owner following expropriation should not reflect changes in value that are due to the expropriation scheme. I am not aware of a single jurisdiction that does not include a variant of this principle. For a detailed investigation into the question of whether or not it stands in the way of benefit sharing in economic development cases, I point to \cite{dyrkolbotn15}.} This means that the use of land consolidation in place of expropriation has considerable potential also in relation to the worry that owners are undercompensated following economic development takings.\footnote{Many scholars adhering to an entitlements-based perspective on property argue that the tendency for undercompensation is in fact the core problem associated with economic development takings.\cite{fennel04,lehavi07,bell07}.}

This also means that commercially motivated developers may have an {\it incentive} to favour expropriation over consolidation. Hence, the question becomes whether or not owners should be able to use land consolidation as a {\it defence} against expropriation. If owners are granted such a right, it would become a very powerful version of what is known in some jurisdictions as the ``self-realisation'' mechanism, a rule whereby owners can sometimes preclude a proposed taking by proposing to implement the underlying project themselves.\footnote{General rules to this effect are found in several jurisdictions in continental Europe (but not in Norway). See generally \cite{sluysmans14}.}
Even in the absence of any legislative initiate in this direction, one might ask whether owners can already achieve this under Norwegian consolidation law. Can the owners preclude expropriation by asking the court to organise the desired development as a consolidation measure?

As long as the expropriation application is still pending a final decision, the owners could theoretically do this. Moreover, in many cases of large-scale hydropower development, it would no doubt be desirable from the point of view of both the owners and their properties to avoid expropriation of riparian rights. Hence, one might wonder if the land consolidation courts would not in fact be {\it obliged} to take on such a case. I am not aware on any case law that sheds light on this and I imagine that land consolidation courts would hesitate quite a bit, particularly if the proposed expropriation concerns large-scale development. Moreover, it is not clear how the expropriation authorities would react if the land consolidation courts did decide to get involved. In principle, an ongoing consolidation case, or even a formally valid use directive, would not in itself prevent expropriation from taking place.

After an expropriation order has been granted, the law as it stands leaves little or no room for a consolidation defence. Quite the contrary, the land consolidation courts would have to respect a valid expropriation order and might even be called on to implement it, by awarding replacement land or financial damages to affected owners as part of a consolidation procedure.\footnote{See \cite[6]{lca79}.}

In the future, if the consolidation alternative to expropriation is to develop successfully, I believe the the owners' right to request consolidation in place of expropriation must be strengthened. Moreover, the property narrative should emphasise the owner-empowering potential inherent in using consolidation as an alternative to expropriation. So far, there are not many signs of this happening in Norway. Rather, the Norwegian system is moving along a trajectory where land consolidation as an alternative to expropriation is increasingly seen as a service to developers and the state. It is noteworthy, in particular, that following a change in the law that takes effect in 2016, private developers without established property interests will be granted the right to bring a case before the land consolidation courts, to seek help in implementing projects that would otherwise necessitate expropriation. Developers might well be motivated to do so, since this could result in reduced administrative costs and (cheaper) consolidation measures replacing the need for paying monetary compensation (e.g., because the land consolidation court provides affected owners with replacement property).

In light of this, one must ask the following: will land consolidation remain a service to owners, or will it become a service to developers who seek cheap access to property owned by others? This question is about to become pressing in Norway, as the scope of land consolidation continually broadens, making it intersect with expropriation law to a greater extent than before. In addition to the new rules granting developers a formal standing in certain consolidation disputes, this development is also strongly felt in the move to apply land consolidation also in the context of urban development, outside the traditional scope of agricultural pursuits.\footnote{See generally \cite{stenseth07}.}

\noo{However, the issue of benefit sharing is bound to come up, particularly in the context of commercial development. In this regard, the risk for developers is that they will be compelled to share the benefits with the owners. However, in order for this to happen, the Land Consolidation Court must actively take steps to make it happen, by recognising the owners' right to benefit sharing. Moreover, while benefit sharing is a fundamental principle for land consolidation among owners, it remains to be seen if this way of thinking will be preserved when new and powerful external actors enter the scene.}

The idea that consolidation can serve as an alternative to expropriation also raises practical questions concerning how it would work. Here there is already some interesting empirical data available, arising mainly from situations when some owners wish to undertake economic development projects on jointly owned land against the will of other owners. In these situations, it is quite common for the owners who desire development to bring a case before the consolidation courts, rather than seeking permission to expropriate.

Interestingly, in the context of hydropower development, this use of land consolidation has become very important in recent years. In 2009, the Court Administration reported that land consolidation had helped realise 164 small-scale hydropower projects with a total annual energy output of about 2 TWh/year.\footnote{See \cite{gevinst09}. For the scale, I mention that 2 TWh/year is roughly what it takes to supply Bergen with electricity, the second largest city in Norway with around 250 000 inhabitants.}

%For an additional perspective, I also mention that this is only about four times the amount of energy produced by the single large-scale plant that resulted in the {\it Alta} controversy discussed in Chapter 5.} 
Moreover, in a recent Supreme Court case, the importance of land consolidation was stressed specifically, as a justification for requiring a commercial taker to pay additional compensation to the owners of riparian rights taken for hydropower development.\footnote{See \cite{klovtveit11}.}

In the next section, I give further details on the Norwegian system, focusing on the system of use directives. Then, in Section \ref{sec:lch}, I consider land consolidation to facilitate small-scale hydropower specifically. I approach this as a test case for the proposition that land consolidation can be a legitimacy-enhancing alternative to expropriation for economic development more generally.

%%%%%%%%%%%%%%%%%%%%%%%%%%%%%%%%%%%%%

\noo{
In particular, a case can now be brought before the land consolidation court by an external developer who would otherwise need to expropriate land to implement a project. However, this change in the law also contributes to a shift away from seeing land consolidation as a service to owners, towards seeing it also a service to developers who seek control of property they do not own. This shift could in turn change the dynamics of land consolidation in a way that makes it less distinct from expropriation.

Even though this definition is broad, I note that a clear distinction can be drawn between land consolidation and national or regional land {\it policies}, which do not target specific properties. The distinction between land consolidation and land-use planning can be harder to draw, but looking to the theoretical starting points of these two kinds of interventions, suffice to establish sketch.

While state planning is an expression of the state's right to regulate the use of land, a land consolidation measure is a {\it service} provided by the state, to facilitate property uses and structures that are deemed desirable from the point of view of the properties as productive units under private ownership.

The distinction between consolidation and measures of land reform may sometimes also be difficult to draw, particularly with my wide notion of consolidation. However, while land reforms tend to arise from centrally directed measures that apply generally within a jurisdiction and come about as the result of a special political initiative, consolidation usually denotes a more flexible framework where local communities are restructured in a way that aims to bring benefits to all owners and rights holders within that community. As such, consolidation rules may alleviate the need for new land reforms, and they may come to represent a ``bottom up" approach to the restructuring of real property.\footnote{The potential for this has been noted even for the traditional understanding of consolidation, as a reduction in the level of property fragmentation. See, e.g., \cite{oldenburg90}. For a different perspective, arguing that land consolidation is generally not sufficient to achieve the noble ends of land reform, see \cite{lipton74}. For a more recent, comprehensive, assessment of the relationship between land reform and consolidation (in the narrow sense), I refer to \cite[237-244]{lipton09}.}

In the following, I adopt this normative stance on the {\it purpose} of consolidation. Hence, I use land consolidation to refer to a regulated process of land reorganisation that come about as a result of a concrete, often local initiative, has a limited geographical scope, relies on the involvement of the local population, and seeks to promote the best interests of all the affected land users. I remark that while Norway has a particularly broad approach, land consolidation more or less in line with my understanding here serves an important function in many jurisdictions.\footnote{For a survey of contemporary land consolidation rules in Europe, reflecting also the need for a wide understanding of the term, I point to \cite{vitikainen2004}.}

One attractive feature of consolidation is that it provides a flexible, dynamic, framework that allows for gradual adaptation of ownership structures, so that they better suit prevailing economic and social conditions. Moreover, land consolidation can become significant in relation to concrete development projects, particularly when such projects necessitate cooperation among several owners. This, in particular, is the use of consolidation that I aim to shed particular light on in this chapter, by giving a case-study of Norwegian law.

Mechanisms for facilitating and organising concrete development schemes are now integrated into the law relating to consolidation in Norway. These rules do not form part of the historical core of consolidation rules in Norway, however, the focus on land consolidation for development is of a more recent date. It is reflected in a number of new provisions, most recently in the \cite{lca13} which will take effect on 1 January 2016.\footnote{Act no 97 of 10 June 2013 relating to the determination and change of structures of ownership- and rights to real property etc.}

When land consolidation is used as a means to organise development projects it also becomes natural to view it as an alternative to expropriation, especially in cases when development has commercial potential and is meant to be carried out by companies operating for profit. Moreover, the controversy that often surrounds such cases further suggests that it should be explored to what extent processes of consolidation can replace expropriation as an implementation mechanism for development of this kind. As we will see, the principle of local participation and benefit sharing is more firmly entrenched in the rules and procedures that govern the consolidation process than in the processes that govern the use of expropriation. 

The contrast between expropriation and consolidation is particularly clear in Norwegian law, where a ``no-loss" principle is enforced with regards to the latter, protecting all affected owners and rights holders. It states that the consolidation process should not leave any owner or rights holder worse off after consolidation. The aim of consolidation is to bestow a benefit on \emph{all} interested parties.\footnote{See \cite[3 a)]{lca79} and \cite[3-18]{lca13} (takes effect in 2016). For a paper discussing the rule in more detail we point to \cite{rygg98}. Rygg is also critical of what he sees as a development away from a strict interpretation of the no-loss rule.} For instance, if ownership is highly fragmented, consolidation mechanisms may be used to exchange property between owners or to introduce joint ownership, but due to the no loss rule it will not be possible to use consolidation in order to deprive some owners of their property to the benefit of others.

In the following, I map the differences between consolidation and expropriation in Norwegian law, starting with an overview of the land consolidation rules, focusing on the development towards giving these rules greater application in connection with concrete development schemes. I then study some cases of locally controlled hydro-power development where land consolidation was used as a means to organise projects involving many different owners and rights holders. We argue that these cases illustrate how the consolidation rules currently in place are well suited to meet local demands for participation and benefit sharing, more so than the existing framework regulating expropriation.

The structure of the remaining part of the chapter is as follows. In Section \ref{sec:lcc} I briefly present the basic rules regarding land consolidation in Norwegian law, including a presentation of the special consolidation courts used to administer the process. Then I go on to consider in more depth the rules relating to so-called \emph{use directives}, permitting the court to actively pursue development projects on behalf of, and in cooperation, with local owners. Use directives represent a form of compulsory cooperation which I believe deserves further attention in the context of land development, especially as an alternative to expropriation. I follow up with a case-study of hydropower in Section \ref{sec:lch} and provide an assessment in Section \ref{seclca}. \noo{nd in Section \ref{sec:5} I contrast the use of directives with more commonly seen approaches to pooling of resources and commercial land development. In Section \ref{sec:6} I offer a conclusion.
}}

\section{Land Consolidation in Norway}\label{sec:lcc}

Rules regarding land consolidation have a long history in Norwegian law. The first consolidation rules were included already in King Magnus Lagabøte's \emph{landslov} (law of the land) from 1274, the first piece of written legislation known to have been introduced at the national level in Norway.\footnote{See Chapter 4, Section 2 of \cite{nou02}.} The earliest rules targeted jointly held rights in farming land, giving owners and rights holders on that land an opportunity to demand apportionment that would give them exclusive rights on a parcel of land corresponding to their share of the joint rights.\footnote{The share in joint rights belonging to each individual farm was historically determined based on the amount of rent (``skyld'') that each farmer paid to the land owner (a figure that was also used to determine the level of taxation). However, following the union with Denmark and especially after the advent of enlightened absolutism, tenant farmers in Norway increasingly bought their land from their land owners.  Indeed, tenant farming became relatively uncommon in Norway after the 18th Century. But the notion of ``skyld'' was kept as a measure of the share each farm had in the now jointly owned larger estates. The notion is still important, for instance in apportionment proceedings, as discussed in \cite{ravna09a}.} The land consolidation courts still provide this function, but additional rules were introduced during the 19th century. At this time, the main use of land consolidation was to pool together fragments and divide up jointly owned land, to create larger single-owner parcels that could facilitate higher-intensity farming and, it was believed, better resource management. However, it was noted that individuation of property rights was not necessarily required or desirable. Indeed, collective-action mechanisms often seemed more appropriate, as a means to avoid disturbing the established property structure.\footnote{This idea was behind a range of provisions introduced during the 19th century, not all pertaining to land consolidation. For instance, a special management structure was set up to govern forestry on common land, to avoid overexploitation and ensure rational management without necessitating enclosure. See generally \cite{stenseth10a}.}

The rules regarding use directives emerged from this context. They were introduced to facilitate a legal framework for land management rooted in property interests. At the same time, the rules would facilitate a considerable pooling of resources and decision-making power, to set the stage for more intensive and coordinated forms of land use.

The initial objective was to enable rural communities to adapt to changing economic conditions without fundamentally altering them or leading to displacement or depopulation. Moreover, the scope of use directives was typically limited to the regulation and reorganisation of already established forms of joint use. It was relatively uncommon to employ use directives to facilitate completely new kinds of development.

Over the last few decades, this has changed. Today, use directives are increasingly applied also to organise development projects that are not agricultural in the traditional sense, even for properties that have no prior connection with one another. Moreover, many additional mechanisms of land consolidation have been introduced, all aiming in various ways to ensure better organisation of land use and ownership. It is helpful to recognise three main categories of consolidation tools, as summarised in the following table:

\begin{itemize}
\item \emph{Apportionment of land}: Rules that empower the court to dissolve systems of joint ownership by apportioning to each estate a parcel corresponding to its share, or by reallocating property through exchange of land.\footnote{This is the traditional form of land consolidation in Norway and the main legislative basis for it is provided in the \cite[2 a)-b)]{lca79}.}
\item \emph{Delimitation of boundaries:} Rules that empower the court to determine, mark and describe boundaries between properties and the content and extent of different rights of use attached to the land.\footnote{The main legislative basis for this form of consolidation is found in the \cite[88]{lca79}.}
\item \emph{Directives for use}: Rules that empower the court to prescribe rules for the use of jointly held land, and to organise such use, including setting up organisational units for carrying out specific development projects.\footnote{These rules are found in the \cite[2 c)|34-35]{lca79}.}
\end{itemize}

In all cases, the consolidation court can only employ these tools when they are called on to do so by someone with legal standing.\footnote{See \cite[5]{lca79}.} This was traditionally limited to the owners and those holding time-unlimited rights of use.\footnote{See \cite[5]{lca79}.} Today, the government also has legal standing in many kinds of consolidation cases, but most cases (about 90 \%) are still initiated by owners.\footnote{See, e.g., \cite[135]{bjerva12}.} From 2016, when the \cite{lca13} comes into force, legal standing will be granted to a larger class of non-state actors, including development companies that may be granted and expropriation licence.\footnote{See \cite[1-5(3)]{lca13}.} Moreover, legal standing will be granted to all rights- or ground lease holders.\footnote{See \cite[1-5(1)]{lca13}.}

After a case has been brought before the court, the consolidation court can implement consolidation measures in so far as they are needed to alleviate problems and difficulties preventing rational use of the affected land.\footnote{See the \cite[1]{lca79}.} To determine whether or not this requirement has been met, the court will look to the prevailing economic and social situation, as well as predictions for the future.\footnote{See generally \cite{reiten09}.} In this regard, the court is also influenced by what it regards as the prevailing public interests in property use. Today, the role of the perceived public interest is gaining importance. Moreover,  recent reforms have sought to increase the importance of public interests in consolidation disputes.\footnote{See generally \cite{prop12} (proposal from the Ministry of Agriculture to the parliament regarding the Consolidation Act 2013).}

The contextual nature of land consolidation has always been clear. Indeed, the basic building blocks of the current system can be traced back to the influence of technological advances in farming and the modernisation processes that Norwegian society underwent in the 19th century. The law responded to these changes by making consolidation an increasingly powerful instrument for change and development. It was also at this time that it was decided to establish a tribunal system for administering the process, first in the Land Consolidation Act from 1857, which was revised and developed further in 1882 and 1950.\footnote{An overview of the history of consolidation law is given in Chapter 3 of \cite{prop12}.} The procedural rules closely mimics those that pertain to the regular civil courts. This ensures that consolidation measures are only applied by the court following a public hearing where all involved parties are given an opportunity to present their case, give supporting evidence, and contradict each others' testimony. For a more detailed description of the consolidation court, I refer the reader to Section \ref{subsec:lcp} below.

The current system for land consolidation is based on the \cite{lca79}, which will be replaced in 2016 when the \cite{lca13} takes effect.\footnote{Act no 97 of 10 June 2013 relating to the determination and change of structures of ownership- and rights to real property etc.} The new act does not introduce any dramatic changes to the law, but it further widens the scope of consolidation in non-agrarian contexts. Moreover, the new act contains the following explicit description of the purpose of land consolidation, which provides a interesting bird's eye view on the legislature's perspective:

\begin{quote}
Section 1-1 The purpose of the Act

The purpose of the Act is to facilitate efficient and rational use of real property in the best interests of the owners, the rights holders and society. This objective will be pursued by the land consolidation courts which will implement remedies for impractical structures concerning ownership and use of property, ascertain and determine property boundaries, as well as decide appraisal disputes and other cases pursuant to this and other acts.

The Act also seeks to facilitate fair, responsible, quick and effective processing of cases in independent and impartial public courts that will operate in such a way as to enhance confidence in the consolidation process.\footcite[1]{lca13}
\end{quote}

This statement of purpose highlights how the new act incorporates and extends the trend towards giving the consolidation process wider scope. I also note how it reiterates and emphasises that the process is to be tribunal in nature. More generally, the quote illustrates that land consolidation is likely to become more important in the future, increasingly also outside the traditional agricultural setting within which this body of law has hitherto developed.\footnote{For instance, following a change in the \cite{lca79} in 2006, land consolidation may now also be called on in order to manage restructuring of ownership in urban areas, in connection with specific development schemes. This rule has been extended further in the new Act. It will be interesting to see how this will affect the division of power between planning authorities, regular civil courts and the (distinctly organised) land consolidation courts. See also \cite{stenseth07}.}

It is now explicitly stated that the purpose of land consolidation is to make conditions of property use more favourable for all the affected owners and rights holders. Hence, the new act underscores how consolidation represents a form of interference that is fundamentally different from expropriation. As before, the consolidation court is not empowered to take action unless it is called on to do so by one of the stakeholders in the property.\footnote{See \cite[1-5]{lca13}.} However, according to the new Act, a developer who (might?) has obtained permission to expropriate is to be counted as a stakeholder in that land for the purposes of consolidation.\footnote{Previously, a developer was only regarded as a stakeholder in consolidation in some cases of public projects, c.f., \cite[5|88|88 a)]{lca79}.} This flags how the relationship between expropriation and consolidation is now becoming an important topic in Norwegian land law.

In 2005, the Ministry of Agriculture made some comments in this regard, in connection with a revision of the \cite{lca79} that gave consolidation greater applicability in urban areas and with respect to implementing public plans.\footnote{See, in particular, \cite[2 h-i)]{lca79}.} Some members of the preparatory committee had raised the concern that giving consolidation extended scope in this way would be problematic since it would encroach on expropriation law. Also, the concern was raised that it would effectively render consolidation as a form of expropriation. The Ministry disagreed, commenting as follows.

\begin{quote}
The Ministry would like to point out that one of the main preconditions for consolidation is that a net profit is created for the land in question. This profit is then divided among the parties in an orderly fashion. Individually, the law also guarantees that no one suffers a loss, see s 3 a). [...] In the Ministry's opinion, expropriation takes place on a different factual and legal basis. In cases of expropriation the public makes decisions that deprives the parties of economic value. The purpose then becomes to compensated them in accordance with s 105 of the Constitution, not to increase the value of their land or the annual income they may derive from it.\footnote{See Chapter 3.3 of \cite{otprp78} (report to parliament from the Ministry regarding changes in the \cite{lca79}.}
\end{quote}

When preparing the new act, the Ministry of Agriculture reiterated this position, but they did not reflect further on the question of the exact relationship between consolidation and expropriation. They observed, however, that changing the law so that expropriating parties could appear in consolidation cases was \emph{reasonable}, since it would then be left up to the developer whether to make use of their permission to expropriate or to rely on consolidation.\footnote{See \cite[84]{prop12}.}

The choice made by the expropriating party in this regard will be of great importance to the affected owners and rights holders. In particular, as the Ministry themselves makes clear, it is an absolute precondition for the implementation of a rights-altering consolidation measure that it serves to make the structure of ownership and use more favourable. This requirement, moreover, refers explicitly to the \emph{area within which consolidation takes place}.\footnote{See the \cite[3-3]{lca13}.} No similar rule is in place to protect the affected local community following expropriation. Moreover, the practices that have developed for dealing with consolidation cases are centred on the interests of the local owners and their communities to a far greater extent than prevailing expropriation procedures.

For instance, the rule regarding expropriation that corresponds most closely to the no-loss rule requires merely that the benefit to private and public interest exceeds the disadvantages \emph{overall}, not locally and certainly not for each individual plot of land.\footnote{See the \cite[2]{ea59}.} At the same time, consolidation rules do not place any restrictions on the kinds of development that can be carried out. The consolidation rules pertain instead to \emph{how} it should be organised. 

Moreover, the consolidation courts must always base their decisions on existing public regulations of the property use.\footnote{In the \cite[3-17]{lca13} it is explicitly stated that the consolidation court cannot prescribe solutions that are not in keeping with such regulation. However, it is also made clear that the consolidation court itself can apply for necessary planning permissions on behalf of the owners and the land in question.}
Hence, if the public interest suggests a particular form of land use, the fact that a planning decision detailing development of such use is implemented through consolidation does not entitle the court to review the plans themselves, going against the public interest. But it does introduce an obligation, emerging at the time of implementation, to turn specifically to the interests of original owners and rights holders. Importantly, the court must look for solutions that minimise the burden and maximises the benefit for all the properties involved.

The rules that give the consolidation court authority to give directives of use are particularly relevant in this regard. Before giving further details about these rules, I briefly present the consolidation process step by step.

\subsection{The Consolidation Process}\label{subsec:lcp}

A consolidation case is usually initiated by an owner or a permanent rights holder.\footnote{See s 5, para 1 of the \cite{lca79}.} The request for consolidation measures is to be directed at the relevant district consolidation court, one of the 34 district courts for land consolidation that have been set up by the King in accordance with section 7 of the \cite{lca79}. The request is meant to include further details about the affected properties, the owners and rights holder involved, as well as the specific issues that consolidation should address.

However, this requirement is not usually interpreted very strictly, meaning that the consolidation court will often be inclined to take steps to clarify further what the case should encompass, more so than in regular civil disputes.\footcite[39]{langbach09} However, the court may still reject the consolidation request if it finds that it suffers from formal shortcomings, pursuant to the same rules as those that apply to civil disputes.\footnote{See section 12, paragraph 2 of the \cite{lca79}, which refers to section 16-5 of the \cite{cda05}.}

If the court decides that the request is well-formed and that it includes sufficient detail to permit material consideration, they go on to prepare public hearings, following the rules set out in Chapter 3 of the \cite{lca79}. These rules mirror those that are in place for civil hearings in general, including the duty to inform affected parties, the parties' right to present their claims, as well as their duty and right to give testimony and provide evidence supporting it.\footnote{See the \cite[13|15|17 a|18]{lca79}.} As in civil cases, a decision is usually made only after at least one oral hearing where the parties may present and comment on the evidence and the issues raised by the case.

Unlike in civil cases, the main hearing typically takes place on the disputed land itself and often revolves around practical rather than legal issues. Moreover, a consolidation case will usually not take the form of a two-party adversarial process, but rather as a multi-party discussion where the court interacts with a large number of interested persons who may have a range of common as well as conflicting interests. The typical case involves 5-10 people, but in some cases there can be hundreds of parties involved.\footcite[39]{langbach09} In addition, it is quite common that the parties are not represented by legal council.\footcite[109-111]{rognes00} And even if they are, the owners themselves are typically expected to take an active part in the proceedings.\footnote{See generally \cite{rognes00}.}

The request for consolidation will be the court's point of departure when assessing the case. However, the court is not bound by the claims put forth by the parties. This again marks a differences with most civil disputes. With a few exceptions explicitly listed in statute, the consolidation court may decided to use any measure that it deems suitable to ensure a favourable structure of rights and ownership for the future. However, there is some restriction placed on the court in that the measures taken must be regarded as \emph{necessary} in light of considerations based on the original request.\footnote{See s 26 and 29 of the \cite{lca79}.} In short, the court should remain focused on the issues raised by the parties, but is free to address these issues using the tools they deem most suited for the job. The consolidation court, in particular, is meant to be a ``problem solver'', more so than an ordinary civil court.\footnote{See generally \cite{rognes07}.}

When a decision is reached, the parties are notified and the decision is presented and argued for in keeping with the rules of the \cite{cda05}.\footnote{See the 
\cite[7|22]{lca79}.} The appropriate format for the decision depends on its content. A regular civil ruling is the form used for decisions that only involve ascertaining the boundaries between properties, while a special ``consolidation decision'' is used to implement apportionment and directives of use. The difference in form affects the appeals procedure; while civil rulings are dealt with by the regular courts of appeal, the consolidation decisions can only be appealed to one of 4 designated consolidation courts of appeal.\footnote{See the \cite[61]{lca79}.}

The procedural rules remain largely the same before the consolidation court of appeal, who provide an entirely new consolidation assessment.\footnote{See s 69 of the \cite{lca79}.} The decision of the consolidation court of appeal can only be appealed on the grounds that it is based on an incorrect understanding of the law, or that procedural mistakes were made. In this case, the ordinary appeal courts have authority in the first instance, while the Supreme Court is the last instance of possible appeal.\footnote{See s 71 of the \cite{lca79}.}

In general, consolidation cases are different from other civil cases mainly in that they have fundamentally different scope. A consolidation case is not primarily concerned with deciding the merits of individual claims, but focuses on introducing structures of ownership and rights that will prove favourable to the community of owners. In this respect, the process has an administrative character. However, the fact that it is organised similarly to a civil dispute means that the affected parties can expect to play a more prominent role in the decision-making process than they do when decisions are made by administrative bodies.

Given the special context of arbitration, it is not surprising that the judges appointed to the consolidation courts are required to have a special skill set, different from that of regular civil law judges. In fact, consolidation judges are required to have successfully completed a special master level degree in consolidation. This is not a law degree, but a distinct form of professional education.\footnote{See s 7, para 5 of the \cite{lca79}. The degree in question is currently offered only at the Norwegian College of Life Sciences and Agriculture.} 

The consolidation court also relies on the participation of lay people who sit alongside the specialist judge.\footnote{See section 8 of the \cite{lca79}.} These lay judges are appointed by the specialist judge from a committee of lay persons that are elected by the local municipalities.\footnote{See section 8 of the \cite{lca79} (the appointment itself is regulated in the \cite[64]{ca15}).} Ideally, the appointed laymen should have special knowledge of the issues raised by the case. However, they are drawn from the general population.\footnote{See section 9, paragraph 5 of the \cite{lca79}.}

Summing up, the consolidation process has both administrative, adversarial and participatory characteristics. While the content and scope of the court's decision will often have an administrative flavour and is not primarily directed at settling any specific dispute, the process is judicial. Hence everyone is entitled, and to some extent even \emph{obliged}, to have his voice heard and to partake in the process. Moreover, while the process is guided and overseen by the court, it is fundamentally based on considerations arising from the interests of the parties and their  expression of these interests in their own words.

However, the owners' interests are always understood also in light of prevailing notions of what counts as favourable and rational property use. Importantly, in relation to this latter assessment, the court will look beyond the expressed interests of the individual owners. The court will pose the question with regards to the use of the land as such, drawing on its understanding of the relevant economic, social and political conditions.\footnote{See generally \cite{reiten09,sky09}.} At the same time, the decisions are made on the basis of information that is presented and discussed in public hearings. Moreover, the affected parties will take part in discussions that may also address more overarching concerns about the form of land use that should be regarded as favourable for the area in question.\footnote{See also \cite{rognes07}.}

To flag the dual nature of the consolidation process it is tempting to designate it as a process of judicially structured \emph{deliberation}. The final decision-making authority is granted to the court, but the court is required to act on behalf of the rights holders, on the basis of their wishes, but always also in the best interest of their properties and their community. 

This form of decision-making based on multi-party deliberation is interesting in its own right, as it provides a template for management of land that caters to the idea of public oversight and control as well as to the idea of local participation and self-governance. It is a form of land management that seems especially suitable as a means to implement concrete projects undertaken in the public interest, particularly when these would otherwise appear to adversely affect individual land owners and local communities.

Moreover, the notion of land consolidation can serve as an additional conceptual layer between the planning stage and the implementation step of a development plan, a layer of management devoted to translating public interests and private plans into concrete action on private property. This, I believe, is a layer of administration that deserves more attention and more fine-grained tools than those currently offered in systems relying on expropriation. Clearly identifying a consolidation layer in property management for economic development can also make for a cleaner delineation between commercial implementation on the one hand, governed by the market, and public planning on the other, governed by administrative law and political bodies. In this way, one may also hope to avoid unhealthy coalescence between the two.

In the next section, I argue that Norwegian consolidation law already include tools that make it possible to rely on the consolidation courts to provide this service. The rules that I believe warrant this conclusion are the rules relating to use directives, which I now present in more detail.

\subsection{Organising the Use of Property}\label{sec:3}

Traditionally, use directives targeted property rights that were owned jointly or for which some form of shared use had already been established.\footnote{In accordance with s 2 c) of the \cite{lca79}.} However, in the 1979 Act, the power of the courts to issue use directives was extended, so that directives could also be issued when there was no prior connection between the rights and properties in question, provided \emph{special reasons} made this desirable.\footnote{See s 2 c), para 2 of the \cite{lca79}.} Traditional examples include directives for the shared use of a private road which crosses several different properties, or regulation of hunting that takes place across property boundaries.

The joint use rules emerged as an alternative to apportionment of jointly owned property, a more subtle and less invasive measure that could often give rise to the same positive effect as a full division of ownership, without leading to unwanted fragmentation. Hence, in the now repealed Land Consolidation Act 1950 it was stated that use directives should be the \emph{primary} mechanism of consolidation, such that apportionment could only take place if such directives were deemed insufficient to reach the goal of creating more favourable conditions for the use of the land.\footnote{See s 3 no 3 and 4 of the Land Consolidation Act 1950 and the discussion in \cite[30-37]{nou76}.} In the 1979 Act, the two mechanisms were formally put side by side, but the intention behind this was to ensure greater flexibility of the system, not to reduce the scope of use directives. Quite the contrary, the 1979 Act explicitly intended to promote the increased use of such directives, also in conjunction with other measures.\footnote{See the discussion in \cite[35-37]{nou76} and \cite[47-48]{otprp56}.}

Since the act was introduced, there has been a gradual increase in the willingness of the courts to rely on use directives to facilitate \emph{new development} on the land, not just as a means to regulate an existing activity.

The \cite{lca79} lists a range of different concrete circumstances in which such directives can be applied.\footnote{See s 35 of the \cite{lca79}.} Importantly, the list is not understood to be exhaustive. Hence, as the notion of agriculture has broadened to include activities such as small-scale hydropower development, the scope of use directives has followed suit.\footcite[103]{otprp57}

In the \cite{lca13}, the list of purposes is replaced by an explicit general rule which makes it clear that the consolidation courts have the authority to give directives whenever they regard this to be favourable to the properties involved.\footnote{See s 3-8 of the \cite{lca13}.} In addition to this, the new Act also introduces a general rule which gives the court authority to \emph{set up} systems of joint ownership when a joint use directive is deemed insufficient.\footnote{See s 3-5 of the \cite{lca13}.} Hence, apportionment and pooling of property is now on equal footing, although a priority rule is introduced for the latter; pooling will only be considered if directives of joint use are regarded as an insufficient means to ensure more favourable conditions. Moreover, the new Act maintains the principle that directives regarding joint use of land for which there are no existing joint rights can only be given if there are special reasons.

This requirement is not intended to be very strict and the Ministry of Agriculture was initially inclined to remove it.\footnote{For a discussion on this see \cite[140-141]{prop12}.} However, it was eventually decided that it should be kept in order to flag that two distinct questions arise in such cases. First, the court must consider the question of whether or not joint use is in fact desirable, before moving on to the question of how it should be organised.

In addition to giving directives prescribing how joint use is to be organised, the consolidation court may give rules compelling owners to take joint action to realise potentials inherent in their land. Rules to this effect were novel to the \cite{lca79}. Moreover, joint action could only be prescribed in special circumstances.\footnote{The rules are given in the \cite[2 e)|42-44]{lca79}.} Currently, joint action directives can only be directed at {\it in rem} property owners, not other parties.\footnote{See the \cite[34 a)]{lca79}.} Following the new \cite{lca13}, however, the consolidation courts will be authorised to prescribe joint action also to right holders. In addition, the existing list of circumstances that warrant joint action will be replaced by a general joint action rule.\footnote{See s 3-9 of the \cite{lca13}.}

When commenting on this change in the law, the Ministry noted that the joint action rules currently in place have been widely used. Indeed, applying them is now one of the core responsibilities of the consolidation courts.\footnote{See \cite[146]{prop12}.} Joint action directives can even include prescriptions for joint investments.\footnote{See s 3-9 of the \cite{lca13}.} On the one hand, this means that such directives can be used to facilitate capital-intensive new development, making consolidation a more effective tool to implement economic development. On the other hand, questions arise regarding the extent to which it is legitimate to rely on compulsion in this regard.

The magnitude of the joint actions and investments required to undertake projects can easily become quite burdensome for individual owners, a worry that is particularly likely to arise in case of large-scale commercial development. The \cite{lca79} attempts to resolve this by a rule stating that if joint actions or investments may come to involve ``great risk'', the court must set up two \emph{distinct} organisational units to undertake it.\footnote{See the \cite[34 b)|42]{lca79}.} First, the rights needed to undertake the scheme will be pooled together and managed by an owners' association. Then, to undertake the scheme itself, a cooperative company structure will be set up on behalf of the owners.

In this way, the risk is diverted away from the individual owners onto a company controlled by them. This company will be entitled to the profit from the scheme, but it will also be required to pay rent to the owners' association on terms established by the parties themselves, with the help of the court.\footnote{See s 34 b) of the \cite{lca79}.} Moreover, the owners are entitled to shares in this company proportional to their share of the relevant rights in the land, as determined by the consolidation court. The owners are not obliged to take part in the undertaking by acquiring such shares. Moreover, they will benefit from membership in the owners' association regardless of whether or not they also purchase shares in the development company.

This two-tier system provides a mechanism that could potentially empower owners to undertake large-scale projects that also involve external commercial actors. In particular, the owners' association could  strike a deal with such a developer, if the owners themselves do not desire to undertake the actual development. In this regard, new conflicts may arise between the owners, if some desire to undertake development themselves, while others wish to strike a deal with an external developer. Should the owners that wish to undertake their own development be prioritised by the owners' association? Should the land consolidation court be empowered to order such a prioritisation? This question has arisen in several cases concerning hydropower, as discussed in the next section.

\noo{Despite the potential for disagreement among owners, I believe it is a strength of the system that  owners retain decision-making power and a right to benefit, even when complex development schemes are to be implemented. Indeed, the rules currently found in Norwegian consolidation law adds weight to the claim that and consolidation might point to an alternative and possibly fruitful way of implementing development projects in a system which presupposes that development takes place through commercial initiatives on the basis of public  planning and control.}

For now, I conclude that the system currently in place already provides tools that allow consolidation courts to organise large-scale development on behalf of owners, even when this requires considerable property (re)organisation and diversification of risk. Importantly, I note that the consolidation rules also point to a form of implementation that is likely to allow the public to exercise more extensive oversight and control. This follows from the fact that the system clearly \emph{curbs} the power and influence of purely commercial forces by emphasising both the owners' interests and the social, economic and political aims which motivate the underlying planning decisions. Effectively, commercial development through consolidation gives the public a greater say during the implementation stage. After all, the organisational structure and the implementation plans are formulated by courts which are explicitly obliged to consider public and societal interests.

In addition to this, both planning authorities and commercial developers may take up a role as formally recognised parties in the consolidation process. This seems particularly useful in connection with large-scale industrial development, as it might otherwise be hard to implement such projects successfully. In these cases, then, the consolidation system sets up an arena for interaction and deliberation between the three main groups of stakeholders: the public, the local owners and the commercially motivated developers. Such an arena is so far missing at the implementation stage of big development projects. At this stage, owners and their communities in particular tend to become completely marginalised, particularly when expropriation is used.

It remains unclear to what extent the Norwegian consolidation rules will actually be used to give property owners a leading voice in development projects involving their properties. The tension between expropriation and consolidation has yet to arise in case law from this angle. However, consolidation is beginning to receive much attention as a practical alternative to expropriation. Hence, I believe it is only a matter of time before deeper questions of participation rights and benefit sharing will also arise.

\noo{Moreover, as I have already mentioned, consolidation has become an important means for organising owner-led hydropower development. In the following section, I describe this in more depth, and I argue that it provides a blueprint for how the consolidation system should approach large-scale commercial development in general.

On the theoretical side, it is also unclear to what extent original owners may \emph{demand} that the rules are put to use. For instance, may an owner request consolidation to prevent a permission to expropriate from being implemented? It will be interesting to see how the Norwegian legal systems will deal with this and related questions, after the new Act takes effect in 2016.

I conclude this section by addressing a new special rule that has been included in the new act, and which is specifically targeted at the planning authorities, encouraging them to make use of consolidation to achieve  greater fairness in public planning. These rules are contained in Chapter 5 of the new act and they target benefit that arises from planning in cases when the benefit appears to fall disproportionately on some owners. Such cases of ``windfall" benefit due to public plans are often flagged as problematic, and they arise with particular frequency in systems based on commercial implementation. For example, if one parcel of land is designated for housing and some neighbouring land is designated as a playground, it might easily come to be seen as unfair that a considerable financial benefit falls to the owner of the land designated for housing, while the playground owner is left with virtually nothing.

Following a change of the Consolidation Act in 2006 which has been further extended in the new act, the law now makes it possible for the planning authorities to decide that apportionment of the \emph{benefit} arising from the plan may be carried out by the consolidation courts.\footnote{See s 3-30 of the \cite{lca13} and s 12-7 no 13 of the \cite{pb08}.} When doing so, the consolidation court will follow the same procedure as in other cases, and it will allocate the benefit arising from the plan based on an assessment of the development potential of the different parcels. Importantly, the court will consider this question independently, and the decision will not be based on the particular manner in which the plan dictates that development is to be carried out. For instance, if the land used for the playground could just as well have been used for housing, the court may decide that the rights to housing development is to be shared equally between the two properties. On the other hand, if there is some independent reason why the playground property is not suited for housing, the court will reduce this property's share in the housing development correspondingly.

The court can implement solutions such as this more effectively and rationally by applying the other tools that it has available. For instance, if the the owner of he playground is entitled to an equal share in housing, then apportionment can be used to actually provide him with such a share, trading it for a corresponding share in the playground. However, if such material reallocation of development rights does not prove feasible, the new act also opens up for a solution where the benefit sharing is implemented using financial compensation.\footnote{See s 3-32 of the \cite{lca13}.}
}

To sum up, use directives are highly versatile tools that may be used to organise extensive projects of land development on behalf of local owners. This form of development organisation makes it possible for original owners to maintain their interest in the land, obviating the need for expropriation, while giving the public a greater opportunity to influence and control  how their planning decisions are implemented in practice. 

In the next section, I consider in depth the particular case of hydropower, where the consolidation courts have recently started to make use of a wide arsenal of its tools to ensure that development can be carried out in this way.

\section{Compulsory Participation in Hydropower Development}\label{sec:lch}

In this section, I look at four recent cases in detail, all of which involved directives of use for hydropower development by local owners. The waterfalls and rivers dealt with in these cases are all located in the county of \emph{Hordaland}, in south-western Norway. Three of the cases involved small-scale hydro-power which some of the owners wanted to develop themselves, while the fourth was a case when the owners were also considering a development plan which would involve cooperation with an external energy company. The cases are particularly useful because we have access to data on how the process of consolidation was perceived by the owners themselves.\footnote{This material is due to Sæmund Stokstad, who conducted interviews for his master thesis on land consolidation, devoted to the study of how consolidation measures can be used to facilitate hydropower development. See \cite{stokstad11}.}

In the following, I first present each case separately, focusing on the organisational issues, the solutions prescribed by the court, and the subsequent reception among the parties. I then assess this from the point of view of developing a better understanding of compulsory cooperation as an alternative to expropriation. I conclude with some unresolved questions, particularly regarding the relationship between consolidation law and other legal frameworks.

\subsection{\emph{Vika}}

The case was brought before the consolidation court in 2005, by riparian owners who had all agreed to pursue hydropower development.\footcite{vika05} The owners disagreed on how to organise the owners' association, and on how the shares in this association should be divided among the properties involved, 15 in total. However, a consensus had formed regarding the main organisational principle, namely that the owners would rent out their waterfall to a separate development company which every owner would have a right (but not a duty) to take part in. 

The parties in \emph{Vika} were closely involved in the consolidation process and the statutes for the owners' association were based on suggestions made by the owners themselves. The main point of disagreement concerned how the shares in this association should be allotted, a question that was made more difficult by the fact that some owners benefited from old water-mill rights in the river. 

In the end, the consolidation court held that these rights were tied to the form of use relevant at the time they were established. Hence, the rights were not regarded as having any financial value and could therefore be extinguished without compensation.\footnote{As provided for in the \cite[2|36|38]{lca79}.}

There was also some disagreement about whether the voting rights in the owners' association should be tied to the number of shares belonging to each owner, or if the owners should simply be allotted one vote each, irrespectively of their share of the relevant riparian rights. The consolidation court went for the first option. However, the way shares where allotted deserves special mention. In particular, the court decided to take into account that some additional water entered the main river from smaller rivers where only a sub-group of the owners held riparian rights. These owners' share in the association was increased accordingly. This is surprising in light of Norwegian water law, as ownership of riparian rights usually arises from ownership of land along the relevant riverbed, regardless of where the water itself comes from.\footnote{See the \cite[13]{wra00}.} Hence, this is an illustration of how the land consolidation court can opt for organisational solutions that seem rational given the concrete circumstances, even if they do not follow from any generally recognised principles of law. 

The statutes of the owners' association in {\it Vika} also contains a second interesting provision, based on a suggestion made by the owners. This provision states that all rights in the association are to be tied to the underlying agricultural properties so that they can not be sold separately. In Norway, a division of agricultural property requires permission from the local municipality.\footnote{See s 12 of the \cite{la95}.} In recent years, however, this protection of farming communities has grown weaker in practice. It is interesting, therefore, that the owners in \emph{Vika} decided that a dissociation of water rights from the underlying agricultural properties should be explicitly forbidden.

According to Stokstad, a general consensus had developed among the parties whereby the land consolidation procedure was seen as a success. It allowed for an orderly and fair decision-making process regarding the conflicts that had arisen. The resolution of the case followed continuous interaction between the owners and the court, where everyone felt they had been given an opportunity to have their voice heard. 

Initially, there were severe tensions among the owners, but the consolidation process had served to alleviate existing conflicts. Some owners also pointed to the fact that the main hearing had been physically conducted in the local community, in a meeting hall that was familiar to the owners. This also gave them a feeling that they were meant to actively partake in the decision-making process. 

When the interviews were conducted, 5 years after the case was concluded, the owners also appeared to agree that the association was working as intended and that the climate of cooperation among the owners was good. The hydropower scheme itself had been completed in 2008, yielding an annual production of around 15 GWh/year, providing enough energy for around 700 households. 

Moreover, following the experience of land consolidation, a culture of deliberation towards consensus had developed among the owners. The owners now emphasised the search for a common ground, aiming to reach agreement on important issues. This was reflected, for instance, in the fact that the owner who contributed the land for the power station was given a generous annual fee, in addition to his compensation as a riparian owner. 

According to Stokstad, this fee exceeds what he would be likely to get if this decision had been left to the discretion of the consolidation court. Hence, it reflects a premium that the owners were now willing to pay to ensure agreement and a continued good climate for cooperation.

In light of this, the case of \emph{Vika} serves as an excellent example of how land consolidation can empower local communities and enable them to embark on substantial development projects.

\subsection{\emph{Oma}}

The case of \emph{Oma} was brought before the courts in 2006.\footcite{oma06} The case involved four properties. The owners of three of them, $A,B$ and $C$, wanted to develop hydropower, while the fourth owner, $D$, was opposed to the plans. Rather than attempting to expropriate the necessary rights from owner $D$, owners $A,B$ and $C$ took the case to the consolidation court. They argued that development would benefit all the properties involved. Moreover, they  pointed out that an alternative project which would not make use of owner $D$'s rights would be less profitable. Hence, in their view, the consolidation court should compel $D$ to cooperate in a joint scheme. 

Owner $D$ protested, arguing that the project would not economically benefit him, and that it would also be to the detriment of his plans to build holiday cottages in the same area.

The case of \emph{Oma} differs from that of \emph{Vika} in that the question of whether it was appropriate to use compulsion was more prominent. In particular, this aspect came up already in relation to the question of whether or not hydropower development should be pursued at all. As I discussed in Section \ref{sec:lcc}, the fact that some owners do not desire development does not prevent the consolidation court from putting directives in place to facilitate it. 

However, the courts often exercise restraint in such cases. In \emph{Oma}, however, the court agreed with the majority that an owners' association with compulsory membership should be set up.\footnote{In doing so, the court relied on s 2 c) of the \cite{lca79}.}

To justify the use of compulsion against $D$, the court first observed that joint development of hydropower would benefit all the properties in question, including that owned by $D$. Then they commented specifically on owner $D$'s plans for building holiday homes, noting first that he was unlikely to be given planning permission, and secondly that a hydropower plant would not adversely affect such plans in any significant way. Moreover, the court noted that while owner $D$'s rights were relatively minor, they were quite crucial for the profitability of the project, particularly because owner $D$ controlled the best location for the construction of a dam to collect the water used in the scheme. Overall, the court's conclusion was that a joint hydropower scheme would be a better option for everyone than a project that did not include owner $D$'s property.

The question then arose as to how the shares in the owners' association should be divided among the owners and their land. In regard to this question, the court departed significantly from one of the basic principles of Norwegian hydropower law. This is the principle stating that no right to hydropower can be derived from being in possession of land suitable for the construction of dams or other facilities necessary to exploit other riparian rights.\footnote{The principle is well-established in expropriation law, going back to the Supreme Court decision in \cite{herlandsfossen22}.} The land consolidation court broke with this principle in the case of \emph{Oma}, deciding instead to set the value of the land designated for construction of a dam and a power station at $6 \%$ of the total value of the rights that went into the owners' association. 

The proportion of financial benefit and decision-making power awarded to the unwilling owner $D$ thus increased accordingly, since these rights were all held by him. In fact, his share went from $1.75 \%$ to $7.75 \%$, so the consolidation process itself led to a situation where he would have a far greater incentive for supporting the development. Hence, the decision in \emph{Oma} was more to the benefit of owner $D$ than any other among the involved parties. If the rights in question had been expropriated, $D$ would have been given next to nothing in compensation and would lose his rights forever. Instead, the solution prescribed by the consolidation court gave him a lasting and substantial interest in local hydro-power.

According to Stokstad, interviews conducted with the parties show how the process and outcome of consolidation served to create a much better climate for further cooperation.  Indeed, when the interviews where conducted, 4 years after the courts' decision, owner $D$ had changed his mind and was now in favour of the development. Moreover, he had also decided that he wanted to take part in the development company. He was not obliged to do so, but his right to take part was ensured by the agreement with the development company, regulated by the statutes of the owners' association.\footnote{The owners' right to take part in the development company is obligatory in some situations, pursuant to \cite[34 b) no 3]{lca79}.}

The owners all reported that the consolidation process had been very successful and that the court had listened to them, allowing everyone to have their voices heard. Moreover, some owners reported that the court had cleverly maintained a ``birds eye view'' on the best way to develop the land in question, ensuring both long term benefits to all involved properties as well as creating an improved climate for cooperation and mutual understanding. The consensus was that making concessions to owner $D$ was appropriate and had been in the interest of everyone involved. In 2011, the hydropower project was completed and today its output is roughly 3 GWh/year.

\emph{Oma} serves as a good illustration of how consolidation can be an effective instrument for facilitating locally controlled development, also in cases when this requires the use of compulsion against some owners. Interestingly, in this case the successful outcome appears to be partly due to the fact that the consolidation court actively used its discretionary powers when deciding how to organise joint use. This power allowed them to deviate from established rights-based legal doctrine and adopt a more context-dependent approach, pursuing solutions that better suited the situation. Interesting legal questions arise in this regard, particularly regarding the competence that the consolidation court has in such cases. 

For instance, one may ask what would have happened if the majority owners in \emph{Oma} had appealed the decision to the regular courts on the basis that $D$ was awarded too many shares in the owners' association. Would this be regarded as a question of the court's interpretation of the law regarding the owners' \emph{rights}, or would it be regarded as a discretionary decision regarding the best way to organise development? If a rights-based perspective was adopted, the decision would almost certainly be overturned. If not, it would seem beyond reproach, as an exercise of the consolidation courts' discretionary power.\footnote{Recall that consolidation decisions can only be appealed to the regular courts on procedural grounds or on the ground that the law has been applied incorrectly.}

A second interesting question that arises is whether or not consolidation can work as well as it did in \emph{Oma} in cases where conflicts run more deeply, or where the parties favouring development are a minority among the owners. The next two cases I consider shed some light on this issue.

\subsection{\emph{Djønno}}

This case was brought before the courts in 2006, by a local owner $A$ who wanted to develop hydropower in a small river crossing his land, the so called \emph{Kvernhusbekken}.\footcite{djonno06} $A$ wanted the court to help him implement a hydropower project, by compelling the other owners, $B, C$ and $D$, to rent out their share of the waterfall on terms dictated by the court. The starting point for the other owners was that they did not want hydropower development. Hence, they were not willing to rent out their rights to owner $A$ or any other developer. There was also a dispute regarding the ownership of the waterfall rights, with $A$ believing initially that he controlled a large majority. It soon became clear that this was not the case. As it turned out, owner $A$'s share of the riparian rights was only $5 \%$, so his financial interest in hydropower was in fact very limited compared to the owners who did not want any development.

On the other hand, the land rights needed for the necessary physical constructions were predominantly held by owner $A$ alone. For this reason, $A$ maintained that the court should use compulsion to allow him to go on with his plans. 

The court agreed that hydropower would be rational use of the waterfall, and they initially assessed the case against the rules relating to compulsory joint action.\footnote{See the  \cite[2 e)]{lca79}.} This could have resulted in concrete directives regarding how the hydropower development should be carried out, down to the level of specific investments and building steps. 

However, the court eventually held that this approach would place too much of a burden on the owners opposing hydropower. Hence, they chose to resolve the case using directives for joint use. By doing so, the court also restricted the scope of their decision to the establishment of an owners' association that would be responsible for renting out the rights. The court would not consider the question of deciding on a concrete scheme.

The model used for the owners' association was similar to the one the court adopted in \emph{Oma}. This included allocating shares in the owners' association in a way that took into account the special importance of land needed for physical constructions. In total, this land was held to correspond to $6 \%$ of the shares in the association. Since these rights were held by owner $A$ alone, his share in the association doubled. In addition to this, owner $A$ purchased the shares from owner $B$, so that his total share ended up amounting to $22 \%$. Still, for the majority, membership in the association was imposed on them against their will.

The wording of the statutes for the association took into account that it would be run by a majority of unwilling shareholders. In particular, it was stated clearly that the association was going to rent out the rights in the waterfall such that hydropower could be developed. In \emph{Oma} and \emph{Vika}, by contrast, the statutes only stated that this was the \emph{purpose} of the association, leaving the shareholders with the freedom to determine whether or not to go through with development.

In interviews, those who were compelled to take part in the association against their will expressed dissatisfaction and surprise at the result. Moreover, while the association had apparently tried to be loyal to the wording of the statutes, by looking for interested developers, there had been no willingness among the majority to engage actively with this work. No deals had been made, no separate development company had been set up, and the conflict among the owners was ongoing. 

Hence, while the case of \emph{Djønno} is an example that consolidation can be used even when it involves compulsion against the majority of owners, it also serves to illustrate that the chance of a successful outcome may be more limited.

The question arises as to how such cases should be dealt with by courts in the future. According to owner $A$, the problem was that the directives of use were not specific enough. In his opinion, the directives should not have been restricted to merely setting up an owners' association for renting out the rights. In addition, the court should have actively engaged also with the question of how the development company should be organised. Moreover, the court should have provided concrete directives as to \emph{who} should carry it out. Among the majority owners, on the other hand, the prevailing feeling was that the development in question, which they would be required to partake in against their will, was more or less doomed to fail from the start.

Hence, the case of {\it Djønno} illustrates that unless one is prepared to increase the level of direct external management, compulsory cooperation might require agreement among a majority that development should indeed take place.

\subsection{\emph{Tokheim}}

This case was brought before the consolidation court in 2008, by the owners of \emph{Tokheimselva}.\footcite{tokheim08} The five involved owners all agreed that development should take place, but they disagreed about how it should be done and about the proportion of each owners' share of the riparian rights. Some owners argued that development should be organised by the owners themselves, while other owners thought it would be best to rent out the rights to an external developer. The case was further complicated by the fact that the proposed development was so substantial that it might require a transferral concession pursuant to the \cite{ica17}. As discussed in Section \ref{sec:x} of Chapter \ref{chap:y}, such a concession can only be given to a company in which the state controls at least $\frac{2}{3}$ of the shares.

The consolidation court eventually decided to set up an owners' association. However, unlike in the previous cases I have considered, there was no adjustment made for land that would be needed for physical constructions. Instead, the statutes state that owners will be entitled to a lump sum estimated on the basis of the damages and disadvantages that a concrete hydropower project will bring. This marks a different kind of departure from established practice in expropriation law, where it has been a long established principle that owners can be compensated on the basis of \emph{either} the value of their waterfalls \emph{or} the damages and disadvantages caused by the project, not both.\footnote{See for instance the case of \emph{Vikfalli}, \cite{vikfalli71}.} 

In other respects, the statutes for the owners' association follow the same model adopted in the previously considered cases. They do not, however, resolve any of the controversial questions regarding how development should be carried out. Moreover, they fail to address the question of the extent to which interested owners should be given the opportunity to develop the water resource themselves. This was the main issue raised by the case, but the consolidation court explicitly decided not to address it. In particular, the statutes of the owners' association explicitly provides separate rules depending on how the development is to be carried out. 

In interviews, the owners expressed that they were happy with how the case was dealt with by the court. Everyone was heard and the owners' association was set up in consultation with the parties. However, the main issues were still unresolved after the case concluded. Some of the owners expressed criticism against the court for not engaging more actively with the most pressing issue.

The case of \emph{Tokheim} serves to illustrate that established practices of consolidation, while being well received and understood by local owners, face some new challenges in relation to hydropower, challenges that consolidation courts might be reluctant to take on. It seems that the court in \emph{Tokheim} felt that they were not in a position to assess the question of what kind of development would be best. Moreover, it also seems that they were particularly cautious about expressing an opinion about the legal status of a project led by local owners, in relation to concession law. The court did not, in particular, form an opinion about whether it would be possible for local owners to carry out their own large-scale development in a waterfall that might otherwise be subject to the provisions set out in the \cite{ica17}.

It remains to be seen whether such an agnostic attitude can be maintained by the consolidation courts, as local owners increasingly turn to them for help in resolving disputes regarding hydropower. Moreover, it will be interesting to see how the new \cite{lca13} will influence case law in this area. It seems that a case like \emph{Tokheim} could benefit from the court taking a broader view, possible even by including public bodies as parties in the case, as will be possible after the new Act takes effect.

\section{Assessment and Future Challenges}\label{sec:lca}

The concrete cases that I discussed in the previous section shows, in my opinion, that the system of land consolidation is well suited as an alternative to expropriation in the context of hydropower development. At the same time, the cases suggest that the land consolidation courts may find it hard to deliver effective directives if owners disagree fundamentally about how their water resources should be managed. In addition, one may question the effectiveness of land consolidation courts in contexts when rules and regulations from other areas of law come into play. It seems, in particular, that the land consolidation courts might be overly cautious about implementing solutions that they fear will contradict sector-specific provisions. In so far as sector-specific rules disadvantage owners and benefit external commercial interests, as in the case of hydropower, the worry is that land consolidation courts will become impotent as soon as a potential conflict with large-scale interests emerges.\footnote{As far as I am aware, there is not yet any case law on this issue.}

Paradoxically, the potential weaknesses of the land consolidation courts in this regard may be enhanced by the fact that they are not authorised to make use of appropriate forms of compulsion against owners, on pain of interfering too much in property as an individual right. A lack of power to compel threatens to undermine the effectiveness of the land consolidation court, thereby making it possible to argue that the public interest in development can not be sufficiently accommodated through the use of consolidation measures. Hence, one may decide to fall back on expropriation, to the detriment of owners, particularly those that oppose development.

In fact, there is some evidence to suggest that land consolidation law is currently quite vulnerable to this mechanism. One indication of this is the Supreme Court case of {\it Holen v Holen}, concerning a quarry owned and operated by a local landowner.\footcite{holen95} In order to continue extracting his minerals, the owner of the quarry would have to interfere with the property of a neighbouring owner, who was using his land for more traditional forms of agriculture. The farmer was unwilling to reach an agreement with the quarry owner, so the latter brought a case before the land consolidation court. The court noted that it would be possible to reach an accommodation that would benefit both parties and issued directives of use that would allow the quarry to continue its operations.

The directives gave the quarry owner access to the farmer's land, who was in turn granted replacement property from the quarry owner. The consolidation court also noted that the quarry would, in the future, be likely to extract minerals that belonged to the farmer. Hence, a directive of use was issued that gave the quarry owner a right to extract these minerals, provided he paid market value for them. 

Hence, not only was the farmer awarded replacement property for agricultural purposes, he was also granted a share of the benefits that would result from the continued operation of his neighbour's quarry. This was clearly beneficial to his property, economically speaking. The owner himself, however, objected to the arrangement. The Supreme Court found in his favour. This was not because they sanctioned his right to block the continued operations of the quarry, or because they thought the replacement property or the payment model was inappropriate. Instead, the Court held that the right to extract the farmer's minerals could not be transferred to someone else by a consolidation measure, even if the farmer was ensured payment. This, the Court held, was a form of compulsion that fell outside the scope of use directives in land consolidation.

The perspective underlying this decision is interesting, because it underscores a reluctance to use land consolidation in what would otherwise be a fairly typical economic development scenario. However, {\it Holen v Holen} was decided in 1995, and as I have already mentioned, the law has developed in recent years in the direction of increased use of land consolidation as an alternative to expropriation. However, I think {\it Holen v Holen} reminds us that critics might still be able to raise convincing formal objections against compulsion in land consolidation, on the basis of earlier case law. More generally, the exact relationship between land consolidation and expropriation law, including the constitutional property clause, appears to be an increasingly relevant open question that awaits further clarification.

Some might be inclined to argue that land consolidation offers less protection to owners than administrative expropriation. Admittedly, the property protection offered in the context of land consolidation has a different flavour. But that is not to say that it is weaker. However, how one judges this might depend on one's vision of property. In particular, it seems to depend on which one of property's functions one finds most worthy of protection.

Granted, an administrative expropriation procedure can offer more extensive {\it formal} safeguards. A range of procedural rules must be observed, pertaining to notification to the owners, impact assessments, a duty to provide guidance and reasons for the decision, and a possibility (sometimes several) for administrative appeal. Then, after an expropriation order has been granted, the owner can still challenge it before the appraisal courts, in principle at the expropriating party's expense. 

In practice, however, the administrative expropriation procedure often leaves the owners marginalised, as they are overshadowed by more powerful stakeholders. This is particularly clear in situations when expropriation arises as a result of more comprehensive planning or licensing procedures. As discussed in Chapter 4, this is particularly clear in the context of hydropower development. 

In addition to this, the possibility of raising validity objections before the courts in expropriation cases is mostly a theoretical one in Norway.\footnote{For a discussion on this with further references, see \cite[x]{dyrkolbotn15}.} It is very unusual for such objections to be made successfully, as the courts typically defer to the discretion of the administrative decision-maker in expropriation cases.

More generally, the narrative of expropriation is one where the owners may have to endure a loss in the public interest, for which they must be compensated as individuals. By contrast, the narrative of consolidation is one where the owners themselves are tasked with making a {\it contribution} to the development project, in the best interest of both the local community and society in general. In particular, the owner's role is no longer than of a passive {\it obstacle} to development, but is transformed to that of an active {\it participant}, one who might have to be nudged to fulfil their potential. In addition, the properties as such receive recognition as important units of assessment, independently of the interests of their current owners. Moreover, the owners as a {\it group} come into focus, as the process is meant to facilitate rational {\it collective} action.

This is achieved by placing owners in a partly deliberative, partly adversarial, context, which not only tolerates but also presupposes and critically depends on their active input to the decision-making process. In addition, the {\it grounds} for imposing compulsory measures that interfere with property rights need to be anchored explicitly in the social functions of the affected properties. A measure is warranted only when it enhances property values, possibly also in the sense of improving conditions for the communities that takes their livelihoods from the affected properties. Clearly, this broader sense in which consolidation serves to protect property is not matched by any administrative safeguards in expropriation law.

Hence, I conclude that land consolidation is highly attractive, even when it involves compulsion directed against owners. In a system based on private property rights, it seems only reasonable that owners and their communities retain their position as primary stakeholders, even if the development is large-scale, takes places in the public interest, and is imposed by compulsory means. This is not achieved through the use of expropriation. However, the institution of land consolidation demonstrates that a better option might be available, provided there is political will to prescribe it. 

\noo{ principled objections against land consolidation in expropriation contexts appear largely misplaced when for the sub-group of takings that realise commercial potentials. However, a second question arises, of a more practical nature. Will the land consolidation process work in practice, if it is applied to organise commercial development. Increased powers of compulsion might be required, and in keeping with my argument above, I believe such powers may well be granted, as long as land consolidation remains directed at improving the situation for existing properties and their owners, rather than bestowing benefits on someone else.}
 
It bears noting, however, that this positive assessment is premised on the fact that property is distributed in an egalitarian manner among the members of local populations in Norway, particularly in rural communities. In so far as land consolidation is used outside of this context, even in urban Norway, the question arises as to whether or not the processes truly empowers the community as opposed to merely the landowners.

As long as consolidation is used to complement other organs for land use planning and local decision-making, without coming into direct conflict with such organs, it is tempting to see this mainly as a worry that land consolidation is an incomplete solution. It might give owners enhanced protection, specifically by facilitating a practical alternative to expropriation, but it does so in a way that neither weakens nor strengthens the position of non-owners. 

To some extent, this might be so, but it is also a way of thinking that appears oversimplified. For one, a strong land consolidation system might have indirect effects, for instance because it makes implementation of public policies more difficult in practice. Or it could undermine local democracy by giving the owners and incentive to use land consolidation as a means of removing certain property issues from the broader political agenda. However, there is another side to this that should not be overlooked. Specifically, it seems wrong to assume that increased protection of owners cannot possibly benefit non-owners. 

At least as long as the owners are themselves {\it members} of the local community, the fact that they are offered increased protection through land consolidation should positively affect the community as a whole. As I discussed at length in Chapter 1, the social function theory of property asks us to recognise that property is a building block of communities. As such, it gives rise to entitlements and duties for owners and non-owners alike, rooted in the mutual dependencies and interactions between them. Hence, a local community represented by a handful of its own property owners might be in a much better position to participate meaningfully in decision-making processes concerning its future than a local community represented by non-local politicians, expert planners, or judges. In addition to this comes the fact that the land consolidation process {\it does} in fact include at least one participant that has to take non-owners into account, namely the judge. As discussed at length above, the judge is consistently asked to consider what is best for the properties, not the owners. This not only emphasises that broader community interests are relevant, it also reinforces the idea that the owners participating in the process are in fact representatives of something more than their own self-interest.

That being said, it seems that in situations where either the scope of consolidation is broad, or else the community representation ensured through property's social function is narrow, wider representation might be in order. In these cases, it might be considered whether the class of persons with legal standing should be extended to cover non-owners without formally recognised property rights (e.g., neighbours, tenants or employees). However, there are reasons for caution in this regard. First, there are obvious pragmatic concerns related to increased costs and complexity, possibly causing inefficiency of the process. From the Norwegian experience, it seems that a few hundred parties, while complex, is still manageable without undermining the process. More than that, and it seems that new ideas would be required. 

However, there are more serious concerns arising with respect to the core function of consolidation as a legitimacy-enhancing alternative to expropriation. Indeed, it seems that the presence of new actors, even local ones, might make consolidation a less effective instrument for protecting those that are most intimately affected and most in risk of marginalisation. It is clear, for instance, that if a large-scale development involves razing an impoverished part of town, it will not be a good idea to give legal standing in consolidation to the employees of the development company that stands to benefit. This would be so even if the employees could be classified as ``locals''. Moreover, it would arguably hold true even if the employees largely outnumbered the owners that would lose their homes. 
 
%In the standard narrative about the dangers of majoritarian democracy, this problem needs to be dealt with by the central government or by the law, not by the locals themselves. This is where the social function notion of property, as exemplified by the land consolidation system, offers an alternative. By placing functions of property center stage when making decisions, the majority rule can be replaced by a property-based form of {\it unanimity}. This, in effect, is what the no-loss principle does, on a high level of abstraction, when it declares that no property, no matter how small or insignificant, must be worse off after consolidation.\footnote{This, indeed, is how Pareto improvements among a group of individuals are interpreted theoretically: they are improvements that all owners will consent to.} 
Of course, the meaning of this must still be debated and defined by persons, but by emphasising that the improvement of property, not personal gain, is the purpose, the decision-making process now takes on some of those characteristics that make some argue that disinterested experts should be granted the power to decide about land uses.

More specifically, the question becomes how to ensure that land consolidation used to facilitate economic development will remain an enabler of property's proper functions. Will it continue to be respectful towards owners and their communities, or will it gradually be transformed into a service for developers who seek cheap access to property owned by others? This is an empirical question concerning the robustness of specific land consolidation procedures. 

In Norway, any legal person with a right to expropriate may now act as a party to a consolidation dispute, so the question is bound to arise. What will the role of the new parties be? Will they be embedded in the legal framework in such a way that they become potential partners that owners can rely on to implement projects in the public interest, or will they be regarded as the main stakeholders, whom the land consolidation courts should assist so that they may successfully impose their will on recalcitrant owners? It will be very interesting to follow this development further, to see if the promise of using land consolidation to regain legitimacy for economic development under compulsion can be fulfilled.

\section{Conclusion}\label{sec:conc}

In this Chapter, I have addressed land consolidation as an alternative to expropriation for economic development, anchored in a case study of hydropwer. I started by presenting the basic idea of using land consolidation to organise commercial projects and I emphasised that the notion of consolidation at work here is a broad notion that includes measures seeking to compel owners to use their property in the public interest. I briefly presented an overarching vision of this kind of land consolidation.

I then presented the Norwegian system, which is based on a judicial decision-making framework that sets the land consolidation procedure apart from that found in many other jurisdictions. I also noted how the procedure is conceptualised as a service to owners, with a no-loss guarantee in place to ensure that consolidation measures are only implemented when the benefits make up for the harms for all the involved properties individually.

I then went on to present rules pertaining to compulsory use of property, rules that empower the consolidation courts to organise specific development projects on behalf of owners, also against their will. I noted how recent changes in the law envisions an extended scope for such rules, including in the context of non-agrarian and urban development. I then went on to consider some concrete examples of hydropower development, where owner-led projects already tend to rely heavily on land consolidation rather than expropriation.

From this case study, I concluded that the land consolidation alternative works well when there is basic agreement among the owners that development is desirable, but can be less effective when there is deep disagreement about whether or not development should proceed at all, or if some owners object to the idea of owner-led development because they prefer to cooperate with an external developer. In these contexts, I argued, it might be necessary to enhance the power of the land consolidation court, also in the direction of extending its authority to compel owners to engage with development projects that they fundamentally disagree with. This is already possible, but only to a limited extent. Moreover, the exact boundaries are an open legal question, as recent legislative changes might suggest that case law on this issue is ripe for review.

I then argued against critics that see increased consolidation powers as a threat to property rights. In particular, I suggested that the procedural protection offered by administrative law is far less effective at protecting the interests of owners and their communities than the safeguards inherent in the consolidation process. Importantly, the land consolidation process seeks to empower owners so that they may participate, while the administrative expropriation process tends to see them as obstacles who need to be removed, but who may be entitled to protection against excessive interference.

I went on to note that recent changes in the law grant developers the right to act as parties in consolidation cases and to bring cases before the courts themselves, if they favour it over expropriation. On the one hand, such a change will enhance the power of the land consolidation court, making it more effective in dealing with cases that involve external parties. On the other hand, there is a possibility that the presence of new and powerful stakeholders will change the nature of the land consolidation process itself, so that it transforms from a property-enhancing institution for self-governance into a planning and implementation instrument for developers.

In general, I believe this Chapter has shown that the land consolidation regime in Norway functions in a way that sheds interesting light on collective-action alternatives to expropriation. Land consolidation is also a dynamic and versatile framework, more so than other suggestions, such as the land assembly districts proposed by Heller and Hills.\footcite{heller08} Hence, the institution of land consolidation can provide a useful kind of democracy-on-demand for compulsory economic development, facilitating a better balance between the interests of owners, local communities, and society as a whole. In this vision, external commercial actors are at worst going to be partners in crime, at best partners in progress. They will not, however, be allowed to dictate the terms on which development takes place. 