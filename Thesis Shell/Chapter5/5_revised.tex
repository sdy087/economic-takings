\chapter{Enabling Participation to Replace Eminent Domain}\label{chap:6}

\section{Introduction}\label{sec:6:1}

%Traditional narratives of property tend to direct attention at owners and their choices, while traditional narratives of sustainable development tend to direct attention at resources and their uses. In light of the discussion in this thesis so far, it seems that it might be a good idea to allow these two perspectives to meet up, before expropriation becomes necessary or -- as in the case of large-scale hydropower in Norway -- automatic.

In recent years, land consolidation has been used to facilitate hydropower projects in Norway, sometimes also by imposing development against the will of private owners. So far land consolidation has been used almost exclusively for small-scale projects organised by local owners themselves. In these situations, expropriation orders are rarely sought and rarely authorised, even if some owners object to the development plans.\footnote{See \cite{brekken08}.} Instead, various consolidation measures are used, including the practically important ``use directives'', serving to set up organisational frameworks for compulsory implementation of a development plan, possibly against some of the owners' wishes.

Essentially, a use directive can be used to take some of the holdout power away from the owners, without depriving them of their property. Instead, owners are encouraged to cooperate and participate in a decision-making process that has economic development as an overarching aim. Some argue that because of the compulsion involved, land consolidation can leave owners in a precarious position by weakening their property rights.\footnote{See, e.g., \cite{stenseth07}.} By contrast, this chapter sets out to make the opposite case, namely that the use of consolidation for economic development can be used to strengthen property as an institution, particularly when use directives replace traditional expropriation proceedings.
\noo{
In consolidation cases, interference in property does not take place because property rights must give way to public interests. Rather, consolidation relies on proof that benefits will outweigh harms at the local level, with respect to each affected property. However, this requirement targets the property as a functional unit, irrespective (in principle) of the specific interests of its current owner. Hence, depending on what functions are regarded as more important, the stated desires of the owner might have to give way to other priorities.

Land consolidation therefore relies on what appears to be a highly functional perspective on property: beneficial resource uses, not individual entitlements, take center stage throughout the process. This might limit the power of the owners to do as they please, but it does not marginalise them. After all, it is hard to deny that one of the primary functions of private property is to bestow rights and obligations on its owners. Moreover, in normal circumstances, it would be safe to assume that when a property benefits, then so does whoever owns it.

For this reason, it also seems that consolidation can be used to address the democratic deficit of economic development takings in an elegant way. This chapter addresses this possibility in more depth. Moreover, it sets out to show that the land consolidation court can be seen as a framework for self-governance that is conducive to sustainable management of property as a common pool resource. 

Specifically, this chapter argues that land consolidation can be used to deal with many of the challenges that arise at the intersection between private property, local community, and economic development in the interest of the public. The Norwegian model might also inspire similar solutions elsewhere, particularly in jurisdictions that are committed to an egalitarian ideal of property ownership. 
}
The chapter starts by introducing this idea in some more depth, before clarifying the consolidation alternative and its building blocks. It then goes on to demonstrate how it works in practice, in the context of hydropower development. It concludes with a discussion of possible objections to the legitimacy of consolidation and a brief assessment of the possibility of exporting the Norwegian consolidation model to other jurisdictions.

\section{Land Consolidation as an Alternative to Expropriation}\label{sec:6:2}

The notion of land consolidation is widely used on the international stage, but it is somewhat ambiguous. Often, it refers to mechanisms whereby boundaries in real property are redrawn to reduce fragmentation, without affecting the relative value of the different owners' holdings.\footnote{See, e.g., the entry on {\it land consolidation} in \cite{mayhew09}.} However, it is also common to use consolidation to refer to mechanisms for pooling together small parcels of land to create larger units.\footnote{See, e.g., \cite{lerman06}.} There is a tension between these two notions of consolidation, with some claiming that consolidation in the latter sense is sometimes used to surreptitiously bestow benefits on powerful property owners, at the expense of weaker groups.\footcite[237-239]{lipton09}

In light of this, I should stress that I will use the term land consolidation in a very broad sense in this chapter, much wider than {\it both} of the interpretations mentioned above. Land consolidation, as I use the term, refers to any mechanism by which the state intervenes, at the request of some interested party, to (re)organise property rights and land uses in a local area. Hence, a consolidation measure might as well involve {\it increased} fragmentation of property, if this is deemed a rational form of consolidation of the property {\it values} involved. Importantly, I also use land consolidation to refer to efforts directed at {\it managing} property, not just redrawing boundaries.

%Moreover, in Heller and Hills' work on land assembly districts, a comparison with land consolidation is presented, based on a broad understanding of that term (although not as broad as that found in Norway).\footcite{heller08} One of my main aims in this chapter is to pick up on this, by offering a more detailed comparison with the Norwegian system and its application in the context of hydropower development. The Norwegian system deserves special attention in this regard because it is particularly broad, especially in its authority to issue use directives.

Some might argue that this terminology is strained, but I adopt it for a reason. It is motivated by the fact that in Norway, the institution known as ``jordskifte'', which is officially translated as land consolidation, has exactly such a broad scope.\footnote{See, e.g., \cite{reiten09,rognes03}. The notion of consolidation at work in Norway appears to be quite unique, but I note that land consolidation also has a relatively broad scope in many other jurisdictions of continental Europe, as well as in Japan and in parts of the developing world. See generally \cite{sky07,vitikainen04}.} Since land consolidation measures in Norway can be used to interfere with property rights quite extensively, one may ask about the legitimacy of consolidation, held against rules that protect private property owners.\footnote{For an analysis of the Norwegian land consolidation process held against the provisions of the ECHR, I refer to \cite{utgard09}.} 

There is a shortage of case law on this regarding the Norwegian system, but legitimacy issues have been raised before the ECtHR regarding the Austrian system of land consolidation, which is also equipped with broad powers to interfere with private rights. Specifically, the Austrian system has been found to offend against the property norm in P1(1) of the ECHR in cases when the consolidation procedure has dragged out in time, while severely restricting the owners' use of their property.\footnote{See \cite{erkner87,poiss87}.} 

In some situations, it may be argued that a land consolidation measure {\it is} a form of expropriation, even if it is not recognised as such by the legislature or the executive. In the US, for instance, a land consolidation provision ordering escheat (to Native American tribes) of fractional property interests in Native American reservations was struck down as an uncompensated taking by the Supreme Court.\footnote{See \cite{hodel87}.}

In relation to the legitimacy issue, the Norwegian system stands out in two important respects. First, the consolidation procedure is managed by judicial bodies, namely the land consolidation courts.\footnote{See generally \cite{langbach09}. The fact that the land consolidation process is administered by a judicial body appears to be unique to Norway, see \cite[45]{sky01}.} Second, land consolidation is largely seen as a service to owners, not a tool for increased state control and top-down management.\footnote{See generally \cite{sky09}.} In particular, a case before the land consolidation courts is almost always initiated by (some of) the affected owners themselves and the court often acts as a ``problem-solver'', aiming to facilitate dialogue and cooperation among owners.\footnote{See generally \cite{rognes98,rognes03,rognes07}.} Finally, the so-called no-loss requirement is a core principle of consolidation law, stating that no consolidation measure can take place unless the benefits make up for the harms, for all the properties involved.\footnote{See the \indexonly{lca79}\dni\cite[3 a)]{lca79}. In terms of economic theory, this amounts to requiring that all measures should lead to Pareto improvements, see \cite[59-61]{miceli11}. Moreover, what is required is actual improvement, not merely {\it potential} improvement (known as Kaldor-Hicks improvement, see \cite[61-63]{miceli11}). Importantly, the no-loss guarantee requires Pareto improvements in or in relation to the affected {\it properties}; no mention is made of their respective individual owners. As a result, the no-loss criterion is averse to the monetization of benefits and harms -- it is normally not possible to fulfil the no-loss requirement by paying compensation to the owners of adversely affected properties, see \cite[394]{sky09}. Improvements must typically be rendered in kind, in a manner that offsets potential losses to the property as such, independently of the owner's own (hidden or revealed) valuations, see \cite[371-372]{sky09}. Hence, the starting point here is completely different from that normally assumed in economic analyses of takings law, where complete monetization and individuation is standard (usually starting from the notion of the owners' {\it reservation price} -- the price at which they would be willing to sell if they behaved non-strategically).} Indeed, this remains one of the key principles of land consolidation in Norway.\footnote{See generally \cite{rygg98}.}

The combination of a judicial procedure that emphasises owner-participation and a no-loss criterion that ensures local benefits means that, arguably, land consolidation in Norway {\it strengthens} property as an institution. Moreover, land consolidation courts can serve as an effective countermeasure against two of the most widely discussed challenges to any property regime. 

First, consolidation can serve to protect an egalitarian distribution of property against the threat of inefficiency and underdevelopment that is otherwise associated with fragmentation.\footnote{See generally \cite{heller98}.} Importantly, it can do so without disturbing the underlying property structure and without bestowing disproportionate benefits or harms on certain owners or other select groups. In particular, land consolidation can ensure commercial development without pooling together property rights and without handing property over to powerful market actors. 

Second, land consolidation can be used to manage privately owned common pool resources to tackle problems of over-exploitation and under-investment that can arise when harms and benefits are inefficiently distributed across a potentially large group of resource users.\footnote{See generally \cite{hardin68,demsetz67}.} In particular, land consolidation can ensure sustainable management of collectively owned resources without necessarily forcing an enclosure process (enclosure {\it can} be the result of land consolidation, but it is only one of many measures in the consolidation toolbox).

In short, land consolidation can be used to address both anti-commons and commons problems, in a way that protects, and possibly enhances, desirable social functions of property, through a judicial system that combines participatory and adversarial decision-making. Furthermore, land consolidation is based on a conceptual premise that -- potentially -- offers increased protection to owners and their properties, by recognising them as members of a community that are mutually dependent on each other. In this way, the form of property protection offered in the context of land consolidation is distinct from the protection offered in the context of expropriation. But it is not necessarily weaker.

%The vision of land consolidation at work here is one that sees it as a means for setting up a mini-democracy on demand, to organise decision-making processes in a way that grants those most intimately affected -- the owners and (possibly) other property dependants -- a say that is proportional to their stake in the matter at hand. Importantly, since land consolidation can be used to impose specific uses of property, it can also be an {\it effective} alternative to expropriation, a compulsory measure that can obviate the need for depriving owners of their property rights. 

%Depending on the compensation regime, the costs associated with the use of eminent domain can be much higher than those associated with land consolidation. Hence, consolidation might be a better approach also from a purely financial perspective. 

In the context of economic development, the no-loss criterion will generally be possible to fulfil  through benefit sharing. Indeed, it becomes the responsibility of the land consolidation court to {\it ensure} that a sufficient degree of benefit sharing results, so that consolidation measures may be applied in accordance with the law.\footnote{For a detailed discussion of the extent of the court's duties in this regard, also discussing recent changes in the law that might indicate a weakening of the no-loss guarantee, see \cite{hauge15}.} Moreover, ensuring a fair distribution of benefits is usually regarded as one of the key goals of consolidation, independently of the no-loss criterion. Often, this is taken to mean that the benefits should be distributed among the affected properties in accordance with their relative value prior to the consolidation measure.\footnote{This principle is not as strictly encoded as the no-loss criterion, but is formulated as an ``ought''-rule. See the \indexonly{lca79}\dni\cite[31|41]{lca79}. In my opinion, this is a weakness of the current framework. I mention that for the special case of consolidation to implement a zoning plan, the rule is absolute, see the \indexonly{lca79}\dni\cite[3 b)]{lca79}.}

This way of thinking can clearly be applied to address the compensation issue that arises following an  economic development taking.\footnote{For the compensation issue generally, see \cite{fennell04,bell07}. The land consolidation approach to benefit sharing also parallels key insights contained in the proposal for compensation reform made in \cite{lehavi07} (proposing that a special institution should be set up to allow owners to bargain for higher compensation in for-profit situations).} However, the principle of benefit sharing at work in consolidation is usually not compensatory, but rather one that sees the owners as active participants in the development project, also when it takes place against their will. This is a highly interesting shift of attention, particularly from the point of view of human flourishing conceptions of property. On such accounts, it can make good sense to impose obligations on owners to participate in the fulfilment of public priorities, particularly when they and their properties also stand to benefit from doing so.\footnote{See the discussion on the social function theory and human flourishing in chapter \ref{chap:2}, sections \ref{sec:2:4} and \ref{sec:2:5}.}

%The emphasis on benefit sharing in land consolidation also reveals a concrete advantage of this institution compared to traditional expropriation. In particular, while benefit sharing is typically required under consolidation law, it is hardly ever achieved through compensation in the context of expropriation for economic development.\footnote{This is largely due to the so-called {\it no scheme} principle, which states that compensation to the owner following expropriation should not reflect changes in value that are due to the expropriation scheme. I am not aware of a single jurisdiction that does not include a variant of this principle. For a detailed investigation into the question of whether or not it stands in the way of benefit sharing in economic development cases, I point to \cite{dyrkolbotn15}.} This means that the use of land consolidation in place of expropriation has considerable potential also in relation to the worry that owners are undercompensated following economic development takings.\footnote{Many scholars adhering to an entitlements-based perspective on property argue that the tendency for undercompensation is in fact the core problem associated with economic development takings.\cite{fennel04,lehavi07,bell07}.}

The fact that consolidation implies benefit sharing and owner participation means that commercially motivated developers may have an {\it incentive} to favour eminent domain over consolidation. Hence, the question becomes whether or not owners should be able to use land consolidation as a {\it defence} against expropriation. If owners are granted such a right, it would become a very powerful version of what is known in some jurisdictions as the ``self-realisation'' mechanism, a rule whereby owners can sometimes preclude a proposed taking by proposing to implement the required development themselves.\footnote{Rules to this effect are found in several jurisdictions in continental Europe (including a very limited rule to this effect in Norway, pertaining to housing projects), see \cite[13-14]{sluysmans15}.} Even in the absence of any legislation explicitly granting owners the right to rely on consolidation as a self-realisation argument, one might ask whether owners can already achieve the desired effect in practice. Can owners preclude expropriation by asking the court to organise the desired development as a consolidation measure?

As long as the expropriation application is still pending a final decision, the owners could theoretically hope to achieve this. Moreover, consolidation measures to implement economic development would generally fulfil the no-loss criterion. Hence, the land consolidation courts should in fact be {\it obliged} to take on such a case, even if there were also plans for expropriation. However, I am not aware of any case where the consolidation courts have actually intervened in this way. They might hesitate to get involved, particularly if the proposed development is large-scale. Moreover, even if they did decide to get involved, it is not clear how the expropriation authorities would react. In principle, an ongoing consolidation case, or even a formally valid use directive, would not in itself prevent expropriation from taking place. However, it might then become harder to justify that an economic development taking would be an appropriate measure.

After an expropriation order has been granted, things are very different. The law as it stands leaves no room for a consolidation defence in these cases. Quite the contrary, the land consolidation courts would have to respect a valid expropriation order and might even be called on to implement it, by awarding replacement land or financial damages to affected owners.\footnote{See the \indexonly{lca79}\dni\cite[6]{lca79}.}

In the future, if the consolidation alternative to expropriation is to develop successfully, it seems that the owners' right to request consolidation in place of expropriation must be strengthened. So far, there are no signs of this happening in Norway. However, the use of consolidation as an alternative to expropriation has received attention from a different angle, as a potentially valuable service to developers who seek a more efficient way of acquiring property.\footnote{See \cite[84]{prop12}.}

Following a change in the law that takes effect in 2016, private developers without existing property interests in the target area will be granted the right to bring a case before the land consolidation courts, to seek help in implementing projects that would otherwise necessitate expropriation.\footnote{See the \indexonly{lca13}\dni\cite[1-5(3)]{lca13}.} Developers might well be motivated to do so, since this could result in reduced administrative costs and (cheaper) compensation arrangements, e.g., compensation in kind through land readjustment.

In light of this, one must ask the following: will land consolidation remain a service to owners, or will it become a service to developers who seek cheap access to property owned by others? This question is about to become pressing in Norway, as the scope of land consolidation continually broadens, making it interact with expropriation law to a greater extent than before.\footnote{In addition to the new rules granting developers a formal standing in certain consolidation disputes, this development is also strongly felt in the move to apply land consolidation in the context of urban development, outside the traditional scope of agricultural pursuits. See generally \cite{stenseth07}.}

\noo{However, the issue of benefit sharing is bound to come up, particularly in the context of commercial development. In this regard, the risk for developers is that they will be compelled to share the benefits with the owners. However, in order for this to happen, the Land Consolidation Court must actively take steps to make it happen, by recognising the owners' right to benefit sharing. Moreover, while benefit sharing is a fundamental principle for land consolidation among owners, it remains to be seen if this way of thinking will be preserved when new and powerful external actors enter the scene.}

The idea that consolidation can serve as an alternative to expropriation also raises practical questions. Specifically, it seems pertinent to ask how well such arrangements could be expected to work in practice. Here there is already some interesting empirical data available, arising mainly from situations when some owners wish to undertake economic development projects on collectively owned land against the will of other owners, possibly also in cooperation with external developers. In the context of hydropower development, such uses of land consolidation have become very important in recent years. In 2009, the Court Administration reported that land consolidation had helped realise 164 small-scale hydropower projects with a total annual energy output of about 2 TWh per year.\footnote{See \cite{gevinst09}. For the scale, I mention that 2 TWh per year is roughly what it takes to supply Bergen with electricity, the second largest city in Norway with around 250 000 inhabitants.} Moreover, in the Supreme Court case of {\it Kløvtveit}, discussed briefly in the previous chapter, the importance of land consolidation was recognised also in the context of expropriation.\footnote{See chapter \ref{chap:5}, section \ref{sec:5:5:3}.} Specifically, the presiding judge pointed to the prevalence of consolidation in the context of hydropower as a justification for requiring a commercial taker to pay additional compensation to the owners. According to the Court, it would have been possible for the taker to cooperate with the owners rather than expropriate from them. Increased compensation was then required because the taker should not be allowed to benefit financially from choosing not to cooperate.\footnote{See \cite{klovtveit11}.}

To set the stage for a more in-depth presentation of consolidation for hydropower development, I will now give some further details about the Norwegian system, focusing on the rules and practices relating to use directives. %Then, in Section \ref{sec:lch}, I consider land consolidation to facilitate small-scale hydropower specifically. I approach this as a test case for the proposition that land consolidation can be a legitimacy-enhancing alternative to expropriation for economic development more generally.

%%%%%%%%%%%%%%%%%%%%%%%%%%%%%%%%%%%%%

\section{Land Consolidation in Norway}\label{sec:6:3}

Rules regarding land consolidation have a long history in Norwegian law. The first known consolidation rules were included already in King Magnus Lagabøte's \emph{landslov} (law of the land) from 1274, the first piece of written legislation known to have been introduced at the national level in Norway.\footnote{See chapter 4, section 2 of \cite{nou02}.} The earliest rules targeted common rights in farming land, giving owners and rights holders on that land an opportunity to demand apportionment that would give them exclusive rights on a single parcel.\footnote{See also the discussion of property regimes in Norway in chapter \ref{chap:4}.} The land consolidation courts still provide this function, but additional rules were introduced during the 19th century. At this time, the main use of land consolidation was to pool together fragments and divide up jointly owned land, to create larger single-owned parcels that could facilitate higher-intensity farming.\footnote{The fragmented system of land ownership that was consolidated at this time served an interesting function in the earlier agrarian economy, to promote governance through a combination of scattered individual rights and property held in common. See generally \cite{smith00,smith02}.} However, it was noted that complete individuation of property rights was not necessarily required or desirable. As an alternative, collective-action mechanisms were introduced, to facilitate economic development without disturbing the established governance structures associated with agrarian property rights.\footnote{This idea was behind a range of provisions introduced during the 19th century, not all pertaining to land consolidation. For instance, a special management structure was set up to govern forestry on common land, to avoid overexploitation and ensure rational management without necessitating enclosure. See generally \cite{stenseth10a}.}

\noo{ The rules regarding use directives emerged from this context. The initial objective was to enable rural communities to adapt to changing economic conditions without fundamentally altering them or leading to displacement or depopulation. Moreover, the scope of use directives was typically limited to the regulation and reorganisation of already established forms of joint use.\footnote{See the discussion in \cite[35-37]{nou76} and \cite[47-48]{otprp56}.} It was relatively uncommon to employ use directives to facilitate completely new kinds of development. Over the last few decades, this has changed. Today, use directives are increasingly applied also to organise development projects that are not agricultural in the traditional sense, even for properties that have no prior connection with one another. 

Before discussing this in more detail, it will be helpful to recognise three main categories of consolidation tools, as summarised in the following table:}

Today, consolidation measures can be roughly grouped into the following three categories:\footnote{I consciously omit the compensatory function that a consolidation court can serve by acting as an appraisal court, e.g., in expropriation cases, see the \indexonly{lca13}\dni\cite[1-4(d)]{lca13}. This is arguably not a consolidation power at all, but rather an additional function that distracts from the uniqueness of consolidation.}

\begin{itemize}
\item \emph{Apportionment of land}: Rules that empower the court to dissolve systems of joint ownership by apportioning to each estate a parcel corresponding to its share, or by reallocating property through exchange of land.\footnote{This is the traditional form of land consolidation in Norway and the main legislative basis for it is provided in the \indexonly{lca79}\dni\cite[2]{lca79}.}
\item \emph{Delimitation of boundaries:} Rules that empower the court to determine, mark and describe boundaries between properties and the content and extent of different rights of use attached to the land.\footnote{The main legislative basis for this form of consolidation is found in the \indexonly{lca79}\dni\cite[88]{lca79}.}
\item \emph{Directives for use}: Rules that empower the court to prescribe rules for the use of land that can benefit from joint management, including setting up organisational units for carrying out specific development projects.\footnote{These rules are found in the \indexonly{lca79}\dni\cite[2 c), 34, 35]{lca79}.}
\end{itemize}

In all cases, the consolidation court can only employ these tools when they are called on to do so by someone with legal standing.\footnote{See the \indexonly{lca79}\dni\cite[5]{lca79}.} This was traditionally limited to the owners and those holding perpetual rights of use.\footnote{See the \indexonly{lca79}\dni\cite[5]{lca79}.} Today, the government also has legal standing in many kinds of consolidation cases, but most cases (about 90 \%) are still initiated by owners.\footnote{See, e.g., \cite[135]{bjerva12}.} From 2016, when the \cite{lca13} comes into force, legal standing will be granted to a larger class of actors, including development companies that could otherwise obtain an expropriation licence.\footnote{See the \indexonly{lca13}\dni\cite[1-5(3)]{lca13}.} Moreover, legal standing will be granted to all rights- and ground leaseholders.\footnote{See the \indexonly{lca13}\dni\cite[1-5(1)]{lca13}.}

After a case has been brought before the court, the consolidation court can implement consolidation measures in so far as they are needed to alleviate problems and difficulties preventing rational use of the affected land.\footnote{See the \indexonly{lca79}\dni\cite[1]{lca79}.} To determine whether or not this requirement has been met, the court will look to the prevailing economic and social situation, as well as predictions for the future.\footnote{See generally \cite{reiten09}.} %In this regard, the court is also influenced by what it regards as the prevailing public interests in property use. The role of the perceived public interest is gaining importance; recent reforms have underscored that considerations based on the common good should inform the decisions made by the consolidation courts.\footnote{See generally \cite{prop12} (proposal from the Ministry of Agriculture to the parliament regarding the Consolidation Act 2013).}
%The contextual nature of land consolidation has always been clear. Indeed, the basic building blocks of the current system can be traced back to the influence of technological advances in farming and the modernisation processes that Norwegian society underwent in the 19th century. The law responded to these changes by making consolidation an increasingly powerful instrument for change and development. It was also at this time that it was decided to establish a tribunal system for administering the process, first in the Land Consolidation Act from 1857, which was revised and developed further in 1882 and 1950.\footnote{An overview of the history of consolidation law is given in Chapter 3 of \cite{prop12}.} 

The procedural rules of consolidation closely mimics those that pertain to regular civil courts. This ensures that consolidation measures are only applied by the court following a public hearing where all involved parties are given an opportunity to present their case, give supporting evidence, and contradict each others' testimony. 

%In the following section, I briefly elaborate on the consolidation process step by step.

\noo{

As mentioned in the previous section, the exact relationship with expropriation looks set to become a more pressing issue in the future. In 2005, the Ministry of Agriculture made some comments in this regard, in connection with a revision of the \cite{lca79} that gave consolidation greater applicability in urban areas and with respect to implementing public plans.\footnote{See, in particular, the \indexonly{lca79}\dni\cite[2 h), 2 i)]{lca79}.} Some members of the preparatory committee had raised the concern that giving consolidation extended scope in this way would be problematic since it would encroach on expropriation law. Also, the concern was raised that it would effectively render consolidation as a form of expropriation. The Ministry disagreed, commenting as follows.

\begin{quote}
The Ministry would like to point out that one of the main preconditions for consolidation is that a net profit is created for the land in question. This profit is then divided among the parties in an orderly fashion. Individually, the law also guarantees that no one suffers a loss, see s 3 a). [...] In the Ministry's opinion, expropriation takes place on a different factual and legal basis. In cases of expropriation the public makes decisions that deprives the parties of economic values. The purpose then becomes to compensate them in accordance with s 105 of the Constitution, not to increase the value of their land or the annual income they may derive from it.\footnote{See chapter 3.3 of \cite{otprp78} (report to parliament from the Ministry regarding changes in the \indexonly{lca79}\dni\cite{lca79}.}
\end{quote}

When preparing the new Act, the Ministry of Agriculture reiterated this position, but did not reflect further on the question of the exact relationship between consolidation and expropriation. The Ministry observed, however, that changing the law so that expropriating parties could appear in consolidation cases was \emph{reasonable}, since it would then be left up to the developer whether to make use of their permission to expropriate or to rely on consolidation.\footnote{See \cite[84]{prop12}.}

The choice made by the expropriating party in this regard will be of great importance to the affected owners and rights holders. In particular, as the Ministry makes clear, it is an absolute precondition for the implementation of a rights-altering consolidation measure that it serves to make the structure of ownership and use more favourable. This requirement, moreover, refers explicitly to the \emph{area within which consolidation takes place}.\footnote{See the \indexonly{lca13}\dni\cite[3-3]{lca13}.} No similar rule is in place to protect the affected local community following expropriation. Moreover, the practices that have developed for dealing with consolidation cases are centred on the interests of the local owners and their communities to a far greater extent than prevailing expropriation procedures.

For instance, the rule regarding expropriation that corresponds most closely to the no-loss rule requires merely that the benefit to private and public interest exceeds the disadvantages \emph{overall}, not locally and certainly not for each individual plot of land.\footnote{See the \indexonly{ea59}\dni\cite[2]{ea59}.} At the same time, consolidation rules do not place any restrictions on the kinds of development that can be carried out. The consolidation rules pertain instead to \emph{how} it should be organised. 

Moreover, the consolidation courts must always base their decisions on existing public regulations of the property use.\footnote{In the \indexonly{lca13}\dni\cite[3-17]{lca13} it is explicitly stated that the consolidation court cannot prescribe solutions that are not in keeping with such regulation. However, it is also made clear that the consolidation court itself can apply for necessary planning permissions on behalf of the owners and the land in question.} Hence, if the public interest suggests a particular form of land use, the fact that a planning decision detailing development of such use is implemented through consolidation does not entitle the court to review the plans themselves, going against the public interest. But it does introduce an obligation, emerging at the time of implementation, to turn specifically to the interests of original owners and rights holders. Importantly, the court must look for solutions that minimise the burden and maximises the benefit for all the properties involved.
}
\noo{
\subsection{The Consolidation Process}\label{sec:6:3:1}

A consolidation case is usually initiated by an owner or someone holding use rights.\footnote{See section 5, paragraph 1 of the \indexonly{lca79}\dni\cite{lca79}.} The request for consolidation measures is to be directed at the relevant district consolidation court, one of the 34 district courts for land consolidation that have been set up by the King in accordance with section 7 of the \cite{lca79}. The request is meant to include further details about the affected properties, the owners and rights holder involved, as well as the specific issues that consolidation should address. This requirement is not usually interpreted very strictly, meaning that the consolidation court will often be inclined to take steps to clarify further what the case should encompass, more so than in regular civil disputes.\footcite[39]{langbach09} However, the court may reject the consolidation request if it finds that it suffers from formal shortcomings, pursuant to the same rules as those that apply to civil disputes.\footnote{See section 12, paragraph 2 of the \cite{lca79}, which refers to section 16-5 of the \cite{cda05}.}

If the court decides that the request is well-formed and that it includes sufficient detail to permit material consideration, it goes on to prepare public hearings, following the rules set out in chapter 3 of the \cite{lca79}. These rules mirror those that are in place for civil hearings, including the duty to inform affected parties, the parties' right to present their claims, as well as their duty and right to give testimony and provide evidence supporting it.\footnote{See the \indexonly{lca79}\dni\cite[13|15|17 a)|18]{lca79}.} As in civil cases, a decision is usually made only after at least one oral hearing where the parties may present and comment on the evidence and the issues raised by the case.

Unlike in civil cases, the main hearing typically takes place on the disputed land itself and often revolves around practical rather than legal issues. Moreover, a consolidation case will usually not take the form of a two-party adversarial process, but rather present as a multi-party discussion where the court interacts with a large number of persons who may have both common and conflicting interests in the outcome. The typical case involves 5-10 people, but in some cases there can be hundreds of parties involved.\footcite[39]{langbach09} In addition, it is quite common that the parties are not represented by legal counsel.\footcite[109-111]{rognes00} And even if they are, the owners themselves are typically expected to take an active part in the proceedings.\footnote{See generally \cite{rognes00}.}

The request for consolidation will be the court's point of departure when assessing the case. However, the court is not bound by the claims put forth by the parties. This again marks a differences with most civil disputes. With a few exceptions explicitly listed in statute, the consolidation court may decide to use any measure that it deems suitable to ensure a favourable structure of rights and ownership for the future. However, there is some restriction placed on the court in that the measures taken must be regarded as \emph{necessary} in light of considerations based on the original request.\footnote{See sections 26 and 29 of the \cite{lca79}.} In short, the court should remain focused on the issues raised by the parties, but is free to address these issues using the tools they deem most suited for the job. The consolidation court, in particular, is meant to be a `problem solver', more so than an ordinary civil court.\footnote{See generally \cite{rognes07}.}

When a decision is reached, the parties are notified and the decision is presented and argued for in keeping with the rules of the \cite{cda05}.\footnote{See the \indexonly{lca79}\dni\cite[7|22]{lca79}.} The appropriate format for the decision depends on its content. A regular civil ruling is the form used for decisions that only involve ascertaining the boundaries between properties, while a special ``consolidation decision'' is used to implement apportionment and directives of use. The difference in form affects the appeals procedure; while civil rulings are dealt with by the regular courts of appeal, the consolidation decisions can only be appealed to one of 4 designated consolidation courts of appeal.\footnote{See the \indexonly{lca79}\dni\cite[61]{lca79}.}

The procedural rules remain largely the same at the consolidation court of appeal, which provides an entirely new consolidation assessment.\footnote{See section 69 of the \cite{lca79}.} The decision of the consolidation court of appeal can only be appealed on the grounds that it is based on an incorrect understanding of the law, or that procedural mistakes were made. The ordinary appeal courts hear the case in the first instance, while the Supreme Court is the last instance of possible appeal.\footnote{See section 71 of the \cite{lca79}.}

In general, consolidation cases are different from other civil cases mainly in that they have a fundamentally different scope. A consolidation case is not primarily concerned with deciding the merits of individual claims, but focuses on introducing structures of ownership and rights that will prove favourable to the community of owners. In this respect, the process has an administrative character. However, the fact that it is organised similarly to a civil dispute means that the affected parties can arguably expect to contribute more to the decision-making process than they do when decisions are made by administrative bodies.

Given the special context of arbitration, it is not surprising that the judges appointed to the consolidation courts are required to have a special skill set, different from that of regular civil law judges. In fact, consolidation judges are required to have successfully completed a special master level degree in consolidation. This is not a law degree, but a distinct form of professional education.\footnote{See section 7, paragraph 5 of the \cite{lca79}. The degree in question is currently offered only at the Norwegian College of Life Sciences and Agriculture.}

The consolidation court also relies on the participation of lay people who sit alongside the specialist judge.\footnote{See section 8 of the \cite{lca79}.} These lay judges are appointed by the specialist judge from a committee of lay persons that are elected by the local municipalities.\footnote{See section 8 of the \cite{lca79} (the appointment itself is regulated in the \indexonly{ca15}\dni\cite[64]{ca15}).} Ideally, the appointed laypeople should have special knowledge of the issues raised by the case. However, they are drawn from the general population.\footnote{See section 9, paragraph 5 of the \cite{lca79}.}
}

The consolidation process has both administrative, adversarial and participatory characteristics. While the content and scope of the court's decision will often have an administrative flavour and is not primarily directed at settling any specific dispute, the process is judicial. Hence everyone is entitled, and to some extent even \emph{obliged}, to have their voice heard and to partake in the process. Moreover, while the process is guided and overseen by the court, the decisions made will be based on considerations arising from the interests of the properties involved, usually as expressed by the parties in their own words.\footnote{See generally \cite{rognes07}.} 

More generally, the court is tasked with determining what is best for the land as a productive unit in the local community, in light of all relevant economic, social and political facts, including the fact that the current owners will remain in charge after the consolidation procedure ends.\footnote{See generally \cite{reiten09,sky09}.} To flag the dual nature of the consolidation process, it is tempting to designate it as a process of judicially structured \emph{deliberation}. The final decision-making authority rests with the court, but the court is required to act on behalf of the rights holders, on the basis of their wishes, but always also in the best interest of their properties and their community.

%This form of decision-making based on multi-party deliberation is interesting in its own right, as it provides a template for management of land that caters to the idea of public oversight and control as well as to the idea of local participation and self-governance. It is a form of land management that seems especially suitable as a means to implement concrete projects undertaken in the public interest, particularly when these would otherwise appear to adversely affect individual land owners and local communities.

For this reason, land consolidation is perfectly situated for providing an additional institutional layer in situations when the public wishes to facilitate or even compel economic development involving privately owned property. In the next section, I present the rules pertaining to use directives in more detail, to elaborate on how consolidation can be used to replace expropriation.

\subsection{Organising the Use of Property}\label{sec:6:3:2}

Traditionally, use directives targeted property rights that were owned jointly or for which some form of shared use had already been established.\footnote{In accordance with s 2 c) of the \cite{lca79}.} However, in the 1979 Act, the power of the courts to issue use directives was extended, so that directives could also be issued when there was no prior connection between the rights and properties in question. The requirement was that \emph{special reasons} made this desirable.\footnote{See s 2 c), para 2 of the \cite{lca79}.} Traditional examples include directives for the shared use of a private road which crosses several different properties, or regulation of hunting that takes place across property boundaries.

The rules pertaining to use directives emerged as an alternative to apportionment of jointly owned property, a more subtle and less invasive measure that could often give rise to the same positive effect as a full division of ownership, without leading to unwanted fragmentation or excessive pooling of resources. Hence, in the now repealed Land Consolidation Act 1950 it was stated that use directives should be the \emph{primary} mechanism of consolidation, such that apportionment could only take place if such directives were deemed insufficient to reach the goal of creating more favourable conditions for the use of the land.\footnote{See section 3 no 3 and 4 of the Land Consolidation Act 1950 and the discussion in \cite[30-37]{nou76}.} In the \cite{lca79}, the two mechanisms were formally put side by side, but the intention behind this was to ensure greater flexibility of the system, not to reduce the scope of use directives. Quite the contrary, the 1979 Act explicitly intended to promote the increased use of such directives, also in conjunction with other measures.\footnote{See the discussion in \cite[35-37]{nou76} and \cite[47-48]{otprp56}.}

Since the Act was introduced, there has been a gradual increase in the willingness of the courts to rely on use directives to facilitate \emph{new development} on the land, not just as a means to regulate an existing activity.\footcite[103]{otprp57} The \cite{lca79} lists a range of different circumstances in which such directives can be applied.\footnote{See section 35 of the \cite{lca79}.} But the list is not understood to be exhaustive. Hence, as the notion of agriculture has broadened to include activities such as small-scale hydropower development, the scope of use directives has followed suit.

In the \cite{lca13}, the list has been replaced altogether by a general rule which makes it clear that the consolidation courts have the authority to give directives whenever they regard this to be favourable to the properties involved.\footnote{See section 3-8 of the \cite{lca13}.} \noo{In addition to this, the new Act also introduces a general rule which gives the court authority to set up joint ownership when a joint use directive is deemed insufficient to achieve the purpose.\footnote{See section 3-5 of the \cite{lca13}.} Hence, apportionment and pooling of property is now on equal footing, although a priority rule is introduced for the latter; pooling will only be considered if directives of joint use are regarded as an insufficient means to ensure more favourable conditions.} The new Act maintains the principle that directives regarding joint use of properties with no prior connection can only be given if there are special reasons for it. However, this requirement is not intended to be very strict and the Ministry of Agriculture was initially inclined to remove it.\footnote{For a discussion on this see \cite[140-141]{prop12}.} However, it was eventually decided that it should be kept, in order to flag that two distinct questions arise in such cases. First, the court must consider whether or not joint use is in fact desirable, before moving on to the question of how it should be organised.

In addition to giving directives prescribing how joint use is to be organised, the consolidation court can give rules compelling owners to take joint action to realise potentials inherent in their land. Rules to this effect were novel to the \cite{lca79}. According to this Act, joint action can only be prescribed in circumstances covered by one of the points in a concrete list of conditions.\footnote{The rules are given in the \cite{lca79} ss 2 e), 42-44.} Moreover, joint action directives can only be directed at {\it in rem} property owners, not other parties.\footnote{See the \indexonly{lca79}\dni\cite[34 a)]{lca79}.} Following the new \cite{lca13}, however, the consolidation courts will be authorised to prescribe joint action also to groups of use right holders. In addition, the existing list of circumstances that warrant joint action will be replaced by a general joint action rule, potentially increasing the scope of such directives.\footnote{See section 3-9 of the \cite{lca13}.}

When commenting on this change in the law, the Ministry noted that the joint action rules currently in place have been widely used. Indeed, applying them is now one of the core responsibilities of the consolidation courts.\footnote{See \cite[146]{prop12}.} Joint action directives can even include prescriptions for joint investments.\footnote{See section 3-9 of the \cite{lca13}.} On the one hand, this means that such directives can be used to facilitate capital-intensive new development, making consolidation a more effective tool to implement economic development. On the other hand, questions arise regarding the extent to which it is legitimate to rely on compulsion in this regard, when the owners are required to contribute financially or put themselves at financial risk.

The magnitude of investments required to undertake complex projects can soon become quite burdensome for individual owners. The \cite{lca79} attempts to resolve this by a rule stating that if a development project will involve ``great risk'', the court must set up two \emph{distinct} organisational units to undertake it.\footnote{See the \indexonly{lca79}\dni\cite[34 b)|42]{lca79}.} First, the rights needed to undertake the scheme will be pooled together and managed by an owners' association. Then, to undertake the scheme itself, a separate development company will be set up on behalf of the owners.

In this way, the risk is diverted away from the individual owners onto a company controlled by them. This company will be entitled to the profit from the scheme, but it will also be required to pay rent to the owners' association on terms agreed on by the parties with the help of the court.\footnote{See s 34 b) of the \cite{lca79}.} The owners are entitled to shares in the development company proportional to their share of the relevant rights in the land, as determined by the consolidation court. However, they are not obliged to acquire any such shares if they do not wish to do so. If they do not, they will still benefit from membership in the owners' association.

This two-tier system provides a mechanism that can also empower owners to undertake large-scale projects, possibly by setting up partnerships with external commercial actors. Moreover, the owners' association is not always obliged to lease out the development rights to a specific owner-controlled development company. The exact rules depend on the statutes of the association, as determined by the consolidation court, but typically it will be possible for a majority of owners to lease out the development rights to an external developer, should they choose to do so. In this regard, conflicts may arise, if some of the owners wish to undertake development themselves, while others wish to strike a deal with an external company. The challenge for the consolidation court, illustrated concretely in the next section, is to organise the owners' association in such a way that the chance of later conflicts is minimised.

After the new consolidation Act takes effect in 2016, both planning authorities and commercial developers may be granted legal standing in the consolidation process. This might prove particularly useful in connection with large-scale industrial development, as it might otherwise be hard to implement such projects successfully. In these cases, the consolidation courts can now function as an arena for interaction and deliberation between the three main groups of stakeholders: the public, the local owners, and the commercially motivated developers.

\noo{Despite the potential for disagreement among owners, I believe it is a strength of the system that  owners retain decision-making power and a right to benefit, even when complex development schemes are to be implemented. Indeed, the rules currently found in Norwegian consolidation law adds weight to the claim that and consolidation might point to an alternative and possibly fruitful way of implementing development projects in a system which presupposes that development takes place through commercial initiatives on the basis of public  planning and control.}

%For now, I conclude that the system currently in place already provides tools that allow consolidation courts to organise large-scale development on behalf of owners, even when this requires considerable property (re)organisation and diversification of risk. Importantly, I note that the consolidation rules also point to a form of implementation that is likely to allow the public to exercise more extensive oversight and control. This follows from the fact that the system clearly \emph{curbs} the power and influence of purely commercial forces by emphasising both the owners' interests and the social, economic and political aims which motivate the underlying planning decisions. Effectively, commercial development through consolidation gives the public a greater say during the implementation stage. After all, the organisational structure and the implementation plans are formulated by courts which are explicitly obliged to consider public and societal interests.

%Such an arena is so far missing at the implementation stage of big development projects. At this stage, owners and their communities in particular tend to become completely marginalised, particularly when expropriation is used.

%It remains unclear to what extent the Norwegian consolidation rules will actually be used to give property owners a leading voice in development projects involving their properties. The tension between expropriation and consolidation has yet to arise in case law from this angle. However, consolidation is beginning to receive much attention as a practical alternative to expropriation. Hence, I believe it is only a matter of time before deeper questions of participation rights and benefit sharing will also arise.

To sum up, use directives are highly versatile tools that may be used to organise extensive projects of land development on behalf of local owners. This form of development organisation makes it possible for original owners to maintain their interest in the land, obviating the need for expropriation, while giving the public a greater opportunity to influence and control how their planning decisions are implemented in practice.

In the next section, I consider in depth the particular case of hydropower, where the consolidation courts have recently started to make use of a wide arsenal of its tools to facilitate owner-led development.

\section{Enabling Participation in Hydropower Development}\label{sec:6:4}

In this section, I look at four recent cases in detail, all of which involved directives of use for hydropower development by local owners. The waterfalls and rivers dealt with in these cases are all located in the county of \emph{Hordaland}, in south-western Norway. Three of the cases revolved around proposals where the owners would develop hydropower themselves,  while the fourth also involved an alternative where the owners would cooperate with an external energy company. The cases are particularly useful because we have access to data on how the process of consolidation was perceived by the owners themselves.\footnote{This material is due to Sæmund Stokstad, who conducted interviews for his master thesis on land consolidation, devoted to the study of how consolidation measures can be used to facilitate hydropower development. See \cite{stokstad11}.}

In the following, I present each case separately, focusing on the organisational issues, the solutions prescribed by the court, and the reception among the parties.

\subsection{\emph{Vika}}\label{sec:6:4:1}

The case was brought before the consolidation court in 2005, by riparian owners who had all agreed to pursue hydropower development.\footcite{vika05} The owners disagreed on how to organise the owners' association and on how the shares in this association should be divided among the properties involved, 15 in total.\footnote{See \cite[25-28]{stokstad11}.} However, a consensus had formed regarding the main organisational principle, namely that the owners would rent out their waterfall to a separate development company which every owner would have a right (but not a duty) to take part in. 

The parties in \emph{Vika} were closely involved in the consolidation process and the statutes for the owners' association were based on suggestions made by the owners themselves. The main point of disagreement concerned how the shares in this association should be allotted, a question that was made more difficult by the fact that some owners benefited from old water-mill rights in the river.\footnote{See \cite[26]{stokstad11}. In the end, the consolidation court held that these rights were tied to the form of use relevant at the time they were established. Hence, the rights were not regarded as having any financial value and could therefore be extinguished without compensation, as provided for in the \indexonly{lca79}\dni\cite[2|36|38]{lca79}.}

There was also some disagreement about whether the voting rights in the owners' association should be tied to the number of shares belonging to each owner, or if the owners should simply be allotted one vote each, irrespectively of their share of the relevant riparian rights. The consolidation court went for the first option.\footnote{See \cite[26]{stokstad11}.} However, the way shares where allotted deserves special mention. In particular, the court decided to take into account that some additional water entered the main river from smaller rivers where only a sub-group of the owners held riparian rights.\footnote{See \cite[26]{stokstad11}.} These owners' share in the association was increased accordingly. This is surprising in light of Norwegian water law, as ownership of riparian rights usually arises from ownership of land along the relevant riverbed, regardless of where the water itself comes from.\footnote{See the \indexonly{wra00}\dni\cite[13]{wra00}.} Hence, this is an illustration of how the land consolidation court can opt for organisational solutions that seem rational given the concrete circumstances, even if they do not follow from any generally recognised principles of law. 

The statutes of the owners' association in {\it Vika} also contains a second interesting provision, based on a suggestion made by the owners.\footnote{See \cite[26]{stokstad11}.} This provision states that all rights in the association are to be tied to the underlying agricultural properties so that they can not be sold separately. In Norway, a division of agricultural property requires permission from the local municipality.\footnote{See section 12 of the \cite{la95}.} In recent years, however, this protection of farming communities has grown weaker in practice. It is interesting, therefore, that the owners in \emph{Vika} decided that a dissociation of water rights from the underlying agricultural properties should be expressly forbidden.

According to Stokstad, a general consensus had developed among the parties whereby the land consolidation procedure was seen as a great success.\footnote{See \cite[39-41]{stokstad11}.} It allowed for an orderly and fair decision-making process regarding the conflicts that had arisen. The resolution of the case followed from continuous interaction between the owners and the court, where everyone felt they had been given an opportunity to have their voices heard. 

Initially, the situation had been tense, but the consolidation process had resolved all conflicts. Some owners also pointed to the fact that the main hearing had been physically conducted in the local community, in a meeting hall that was neutral yet familiar to the owners. This also gave them a feeling that they were meant to actively partake in the decision-making process. 

When the interviews were conducted, 5 years after the case was concluded, the owners also appeared to agree that the association was working as intended and that the climate of cooperation among the owners was good. The hydropower scheme itself had been completed in 2008, yielding an annual production of around 15 GWh per year, providing enough energy for around 700 households.\footnote{See \cite[41]{stokstad11}.}

\noo{Moreover, following the experience of land consolidation, a culture of deliberation towards consensus had developed among the owners. The owners now emphasised the search for a common ground, aiming to reach agreement on important issues. This was reflected, for instance, in the fact that the owner who contributed the land for the power station was given a generous annual fee, in addition to his compensation as a riparian owner.

According to Stokstad, this fee exceeds what he might have gotten if this decision had been left to the discretion of the consolidation court.\footnote{See \cite[40]{stokstad11}.} Hence, it reflects a premium that the owners were now willing to pay to ensure agreement and a continued good climate for cooperation.

In light of this, the case of \emph{Vika} serves as an excellent example of how land consolidation can empower local communities and enable them to embark on substantial development projects.
}

\subsection{\emph{Oma}}\label{sec:6:4:2}

The case of \emph{Oma} was brought before the courts in 2006.\footcite{oma06} The case involved four properties. The owners of three of them, $A,B$ and $C$, wanted to develop hydropower, while the fourth owner, $D$, was opposed to the plans.\footnote{See \cite[36-39]{stokstad11}.} Rather than attempting to expropriate the necessary rights from owner $D$, owners $A,B$ and $C$ took the case to the consolidation court. They argued that development would benefit all the properties involved. Moreover, they pointed out that an alternative project which would not make use of owner $D$'s rights would be less profitable. Hence, in their view, the consolidation court should compel $D$ to cooperate in a joint scheme. Owner $D$ protested, arguing that the project would not economically benefit him, and that it would also be to the detriment of his plans to build holiday cottages in the same area. 

The case of \emph{Oma} differs from that of \emph{Vika} since the question of whether it was appropriate to use compulsion was more prominent. In the end, the court agreed with the majority that an owners' association with compulsory membership should be set up.\footnote{In doing so, the court relied on s 2 c) of the \cite{lca79}.} To justify the use of compulsion against $D$, the court commented specifically on owner $D$'s plans for building holiday homes, noting first that he was unlikely to be given planning permission, and secondly that a hydropower plant would not adversely affect such plans in any significant way.\footnote{See \cite[36-37]{stokstad11}.} Moreover, the court noted that while owner $D$'s rights were relatively minor, they were quite crucial for the profitability of the project, particularly because owner $D$ controlled the best location for the construction of a dam to collect the water used in the scheme. Overall, the court's conclusion was that a joint hydropower scheme would be a better option for everyone than a project that did not include owner $D$'s property.

The question then arose as to how the shares in the owners' association should be divided. With regard to this question, the court departed significantly from one of the basic principles of Norwegian hydropower law. This is the principle stating that no right to hydropower can be derived from being in possession of land suitable for the construction of dams or other facilities necessary to exploit riparian rights.\footnote{The principle is well-established in expropriation law, going back to the Supreme Court decision in \cite{herlandsfossen22}. The principle was challenged unsuccessfully following the increased scale of development after the Second World War, as discussed in chapter \ref{chap:5}, section \ref{sec:5:4:2}.} The land consolidation court broke with this principle in the case of \emph{Oma}, deciding instead to set the value of the land designated for construction of a dam and a power station at $6 \%$ of the total value of the rights that went into the owners' association.\footnote{See \cite[36]{stokstad11}.}

The proportion of financial benefit and decision-making power awarded to the unwilling owner $D$ thus increased accordingly, since these rights were all held by him. In fact, his share went from $1.75 \%$ to $7.75 \%$, so the consolidation process itself led to a situation where he would have a far greater incentive for supporting the development. Hence, the decision in \emph{Oma} was more to the benefit of owner $D$ than any other among the involved parties. If the rights in question had been expropriated, $D$ would have been given next to nothing in compensation and would have lost his rights forever. Instead, the solution prescribed by the consolidation court gave him a lasting and substantial interest in local hydropower.

According to Stokstad, interviews conducted with the parties show how the process and outcome of consolidation served to create a much better climate for further cooperation.\footnote{See \cite[44-45]{stokstad11}.} Indeed, when the interviews where conducted, 4 years after the court's decision, owner $D$ had changed his mind and was now in favour of the development. Moreover, he had also decided that he wanted to take part in the development company. He was not obliged to do so, but his right to take part was ensured by the agreement made with the development company, as required by the statutes of the owners' association.\footnote{The owners' right to take part in the development company is obligatory in some situations, pursuant to the \dni\cite[34 b) no 3]{lca79}\indexonly{lca79}.}

The owners all reported that the consolidation process had been very successful and that the court had listened to them, allowing everyone to have their voices heard. Moreover, some owners reported that the court had cleverly maintained a bird's eye view on the best way to develop the land in question, ensuring both long term benefits to all involved properties as well as creating an improved climate for cooperation and mutual understanding. The consensus was that making concessions to owner $D$ was appropriate and had been in the interest of everyone involved. In 2011, the hydropower project was completed and today its output is roughly 5 GWh per year.\footnote{See \cite[45]{stokstad11}.}

\emph{Oma} serves as a good illustration of how consolidation can be an effective instrument for facilitating locally controlled development, also in cases when this requires the use of compulsion against some owners. Interestingly, the successful outcome appears to be partly due to the fact that the consolidation court actively used its discretionary powers when deciding how to organise joint use. This power allowed them to deviate from established rights-based legal doctrine and adopt a more context-dependent approach, pursuing solutions that better suited the situation. Interesting legal questions arise in this regard, particularly regarding the extent to which the consolidation court can deviate from sector-based doctrines when organising development.

For instance, one may ask what would have happened if the majority owners in \emph{Oma} had appealed the decision to the regular courts on the basis that $D$ was awarded too many shares in the owners' association. Would this be regarded as a question of the court's interpretation of the law regarding the owners' \emph{rights}, or would it be regarded as a discretionary decision regarding the best way to organise development? If a rights-based perspective was adopted, the decision would almost certainly be overturned. If not, it would seem beyond reproach, as an exercise of the consolidation courts' discretionary power.

A second interesting question that arises is whether or not consolidation can work as well as it did in \emph{Oma} in cases where conflicts run deeper, or where the parties favouring development are a minority among the owners. The next two cases I consider shed some light on this issue.

\subsection{\emph{Djønno}}\label{sec:6:4:3}

This case was brought before the courts in 2006, by a local owner $A$ who wanted to develop hydropower in a small river crossing his land, the so called \emph{Kvernhusbekken}.\footcite{djonno06} $A$ wanted the court to help him implement a hydropower project, by compelling the other owners, $B, C$ and $D$, to rent out their share of the waterfall on terms dictated by the court.\footnote{See \cite[28-31]{stokstad11}.} The starting point for the other owners was that they did not want hydropower development. Hence, they were not willing to rent out their rights to owner $A$ or any other developer. There was also a dispute regarding the ownership of the waterfall rights, with $A$ believing initially that he controlled a large majority. It soon became clear that this was not the case. As it turned out, owner $A$'s share of the riparian rights was only $5 \%$, so his financial interest in hydropower was in fact very limited compared to the owners who did not want any development.

On the other hand, the land rights needed for the necessary physical constructions were predominantly held by owner $A$ alone. For this reason, $A$ maintained that the court should compel the other owners to allow him to go ahead with his development plans. The court agreed that hydropower would be a rational use of the waterfall, and initially assessed the case against the rules relating to compulsory joint action.\footnote{See the \indexonly{lca79}\dni\cite[2 e)]{lca79}.} This could have resulted in concrete directives regarding how the hydropower development should be carried out, including at the level of specific investments and building steps.

However, the court eventually held that this approach would place too much of a burden on the owners opposing hydropower. Hence, it chose to resolve the case using directives for joint use. By doing so, the court also restricted the scope of their decision to the establishment of an owners' association that would be responsible for renting out the rights. 

The model used for the owners' association was similar to the one the court adopted in \emph{Oma}. This included allocating shares in the owners' association in a way that took into account the special importance of land needed for physical constructions. In total, this land was held to correspond to $6 \%$ of the shares in the association. Since these rights were held by owner $A$ alone, his share in the association doubled. In addition to this, owner $A$ purchased the shares from owner $B$, so that his total share ended up amounting to $22 \%$. Still, for the majority, membership in the association was imposed on them against their will.

The wording of the statutes for the association took into account that it would be run by a majority of unwilling shareholders. In particular, it was stated clearly that the association was going to rent out the rights in the waterfall such that hydropower could be developed. In \emph{Oma} and \emph{Vika}, by contrast, the statutes only stated that this was the \emph{purpose} of the association, leaving the shareholders with the freedom to determine whether or not to go through with development.

In interviews, those who were compelled to take part in the association against their will expressed dissatisfaction and surprise at the result. Moreover, while the association had apparently tried to be loyal to the wording of the statutes, by looking for interested developers, there had been no willingness among the majority to engage actively with this work. No deals had been made, no separate development company had been set up, and the conflict among the owners was ongoing. Hence, while the case of \emph{Djønno} is an example that consolidation can be used even when it involves compulsion against the majority of owners, it also serves to illustrate that the chance of a successful outcome may be more limited. 

The question arises as to how such cases should be dealt with by courts in the future. According to owner $A$, the problem was that the directives of use were not specific enough. In his opinion, the directives should not have been restricted to merely setting up an owners' association for renting out the rights. In addition, the court should have actively addressed the question of how the development company should be organised. Among the majority owners, on the other hand, the prevailing feeling was that the development in question was more or less doomed to fail from the start, since it was unwanted.

Hence, the case of {\it Djønno} illustrates that when the courts are not prepared to actively organise the development company, compulsory participation might fail in practice unless a majority agrees that development should take place.

\subsection{\emph{Tokheim}}\label{sec:6:4:4}

This case was brought before the consolidation court in 2008, by the owners of \emph{Tokheimselva}.\footcite{tokheim08} The five owners all agreed that development should take place, but they disagreed about how it should be done and about the proportion of each owners' share of the riparian rights.\footnote{See \cite[34-36]{stokstad11}.} Some owners argued that development should be organised by the owner community, while other owners thought it would be best to rent out the rights to an external developer. The case was further complicated by the fact that the proposed development was so substantial that it could require a waterfall transferral license pursuant to the \cite{ica17}. As discussed in chapter \ref{chap:4}, such a license can only be given to a company in which the state controls at least $\frac{2}{3}$ of the shares.\footnote{See the discussion in chapter \ref{chap:4}, section \ref{sec:4:3}.}

The consolidation court eventually decided to set up an owners association. However, there was no adjustment made for land that would be needed for physical constructions. Instead, the statutes state that owners will be entitled to a lump sum estimated on the basis of the damages and disadvantages that a concrete hydropower project will bring. This marks a different kind of departure from established practice in expropriation law; specifically, it rejects the established principle that owners can be compensated on the basis of \emph{either} the value of their waterfalls \emph{or} the damages and disadvantages caused by the project, not both.\footnote{See for instance the case of \cite{vikfalli71}. See also the discussion in chapter \ref{chap:5}, section \ref{sec:5:4:2}.}

In other respects, the statutes for the owners' association are similar to those adopted in the previously considered cases. Specifically, the statutes do not resolve the controversial question of how to carry out development. Moreover, nothing is said about the extent to which interested owners should be given a right of first refusal with respect to the development rights held by the owners' association. This was an important issue raised by the case, but the consolidation court explicitly decided not to address it. %In particular, the statutes of the owners' association explicitly provides separate rules depending on how the development is to be carried out. 

In interviews, the owners expressed that they were happy with how the case was dealt with by the court.\footnote{See \cite[43-44]{stokstad11}.} Everyone was heard and the owners' association was set up in consultation with the parties. However, the main issue, concerning {\it who} should develop the waterfall, was still unresolved after the case concluded. Some of the owners expressed criticism against the court on this basis.

The case of \emph{Tokheim} serves to illustrate that established practices of consolidation, while being well received and understood by local owners, face some new challenges in relation to hydropower, challenges that consolidation courts might be reluctant to take on. It seems that the court in \emph{Tokheim} felt that it was not in a position to assess the question of what kind of development would be best. The court was particularly cautious about expressing an opinion about the legal status of the project with respect to the relevant licensing legislation. %The court did not, in particular, form an opinion about how local owners should proceed to carry out their own large-scale development in a waterfall subject to the \cite{ica17}.

It remains to be seen whether such an agnostic attitude can be maintained by the consolidation courts, as local owners increasingly turn to them for help in resolving disputes regarding hydropower. Moreover, it will be interesting to see how the new \cite{lca13} will influence case law in this area. It seems that a case like \emph{Tokheim} could benefit from the court taking a broader view, possible even by including government bodies as parties in the case, to clarify the licensing status of the proposed development.

\section{Assessment and Future Challenges}\label{sec:6:5}

The cases discussed in the previous section show that the system of land consolidation can work as a practical alternative to expropriation in the context of hydropower development. At the same time, the cases suggest that the land consolidation courts may find it hard to deliver effective directives if owners disagree fundamentally about how their water resources should be managed. In addition, land consolidation courts are clearly less effective in situations when they are forced to consider rules and regulations from other areas of the law, outside their traditional area of expertise. Specifically, the land consolidation courts might be overly cautious about implementing solutions that they fear will contradict sector-specific provisions. Furthermore, the land consolidation courts might be unwilling to intervene in a potential conflict between owners and powerful commercial interests, especially if the sector-specific rules seem to speak in favour of expropriation.

Paradoxically, the potential weaknesses of the land consolidation courts in this regard may be exacerbated by the fact that these courts are not authorised to make use of sufficiently strong forms of compulsion against owners. This worry, specifically, can give rise to the argument that the public interest in development is unlikely to be realised through the use of consolidation measures alone. Hence, one may fall back on expropriation, to the detriment of all owners, including those that also oppose development by consolidation.

The consolidation alternative seems to be quite vulnerable to this mechanism. This is illustrated by the Supreme Court case of {\it Holen v Holen}, concerning a conflict between a small quarry and a neighbouring farmer.\footcite{holen95} In order to continue extracting his minerals, the owner of the quarry would have to interfere with the property of a neighbouring owner who was using his land for more traditional forms of agriculture. The farmer was unwilling to reach an agreement with the quarry owner, so the latter brought a case before the land consolidation court. The court noted that it would be possible to reach an accommodation that would benefit both parties and issued use directives that would allow the quarry to continue its operations.

The directives gave the quarry owner access to the farmer's land, who was in turn granted replacement property from the quarry owner. The consolidation court also noted that the quarry would, in the future, be likely to extract minerals that belonged to the farmer. Hence, a directive was issued that gave the quarry owner a right to extract these minerals, provided he paid market value for them. 

Hence, not only was the farmer awarded replacement property for agricultural purposes, he was also granted a share of the benefits that would result from the continued operation of his neighbour's quarry. This was clearly beneficial to his property, economically speaking. The owner himself, however, objected to the arrangement. The Supreme Court found in his favour. This was not because they sanctioned his right to block the continued operations of the quarry, or because they thought the replacement property or the payment model was inappropriate. Instead, the Court held that the farmer's  right to extract minerals could not be transferred to the quarry by a consolidation measure, even if the farmer was ensured payment for his share of the total mineral rights.\footnote{See \cite[1481]{holen95}.} Specifically, such a transfer was not held to fall within the meaning of organising joint use of their properties. 

This decision seems to suggest a reluctance to permit land consolidation being used in a way that tracks eminent domain too closely. However, {\it Holen v Holen} was decided in 1995, and as I have already mentioned, the law has developed in recent years in the direction of increased use of land consolidation as an alternative to expropriation. Moreover, it might have been possible for the land consolidation court to avoid the outcome in {\it Holen} by providing a more subtle use directive. Specifically, the court could set up an organisational arrangement that would have allowed the unwilling owner to take active part in the development company later on, if he were to his mind. If so, the arrangement as a whole would likely have fallen back inside the meaning of joint use. At least, it would then have been in keeping with current practices observed in the context of hydropower development, e.g., as seen in the {\it Oma} case discussed above. Still, {\it Holen v Holen} reminds us that critics might be able to raise convincing formal objections against compulsion in land consolidation, on the basis of earlier case law.

As mentioned earlier in this chapter, some scholars argue that land consolidation sometimes offers less protection to owners than administrative expropriation.\footnote{See \cite[318-319]{stenseth07}.} In expropriation cases, it is true that a range of procedural rules tend to apply, pertaining to notification to the owners, impact assessments, a duty to provide guidance and reasons for the decision, and a possibility (sometimes several) for administrative appeal.\footnote{See \cite[377-382]{dyrkolbotn15b}.} Moreover, after an expropriation order has been granted, the owner can still challenge it before the appraisal courts, in principle at the expropriating party's expense.\footnote{See \cite[382-384]{dyrkolbotn15b}.}

In practice, however, the administrative expropriation procedure often leaves the owners completely marginalised, as they are overshadowed by other stakeholders. This is particularly clear in situations when expropriation arises as a result of more comprehensive planning or licensing procedures, such as in the context of hydropower development.\footnote{See the discussion in chapter \ref{chap:5}. For the same point with respect to planning more generally, see \cite[376]{dyrkolbotn15b}.} In addition to this, the possibility of raising validity objections before the courts in expropriation cases is mostly a theoretical one in Norway.\footnote{For a discussion on this with further references, see \cite[384-386]{dyrkolbotn15b}.} The courts almost always defer to the discretion of the administrative decision-maker in such cases.

More generally, the narrative of expropriation is one where the owners have to endure a loss in the public interest, for which they must be compensated as individuals. By contrast, the narrative of consolidation is one where the owners themselves are tasked with making a {\it contribution} to the development project, in the best interests of both the local community and greater society. In particular, the owner's role is no longer that of a passive obstacle to development. Rather, the owner is placed in the position of active {\it participant}, one who might yet have to be nudged to fulfil their potential. In addition, the properties as such receive recognition as important resource units, independently of the interests of their current owners. Moreover, the owners as a {\it group} come into focus, as the process is meant to facilitate rational {\it collective} action.

This is achieved by placing owners in a partly deliberative, partly adversarial, context, which not only tolerates, but also presupposes, their active input to the decision-making process. In addition, the {\it grounds} for imposing compulsory measures that interfere with property rights need to be anchored explicitly in the social functions of the affected properties, not individual interests. A measure is warranted only when it enhances property values, also in the sense of improving conditions for the communities that take their livelihoods from the affected properties. Clearly, this broader sense in which consolidation serves to protect property is not matched by any administrative safeguards in expropriation law.
\noo{
Hence, I conclude that land consolidation is highly attractive, even when it involves compulsion directed against owners. In a system based on private property rights, it seems only reasonable that owners and their communities retain their position as primary stakeholders, even if the public is adamant that development needs to take place on their property.

\noo{ principled objections against land consolidation in expropriation contexts appear largely misplaced when for the sub-group of takings that realise commercial potentials. However, a second question arises, of a more practical nature. Will the land consolidation process work in practice, if it is applied to organise commercial development. Increased powers of compulsion might be required, and in keeping with my argument above, I believe such powers may well be granted, as long as land consolidation remains directed at improving the situation for existing properties and their owners, rather than bestowing benefits on someone else.}
} 
It should be noted that this positive assessment of consolidation as an alternative to expropriation is premised on the fact that property in Norway is distributed in an egalitarian manner among the members of local populations, especially in rural areas. If land consolidation is used outside of this context, even in urban Norway, one might ask whether the processes truly empowers the community, or merely the landowners.

Arguably, the existing land consolidation system is an incomplete solution to the legitimacy issues that arise in such cases. Moreover, land consolidation might even have indirect effects that make those issues harder to resolve. For instance, it might be that consolidation will undermine local democracy by allowing powerful owners to remove certain property issues from the broader political agenda. The consolidation courts could become arenas used by powerful owners to prevent marginalised group from accessing decision-making processes of societal significance.

On the other hand, it is wrong to assume that empowering owners will not also benefit non-owners. As long as the owners are themselves members of the local community, the fact that they are offered increased protection through land consolidation can positively affect the community as a whole. As I discussed at length in chapter \ref{chap:2}, the social function theory of property asks us to recognise such effects. Indeed, a community represented by local property owners might be in a much better position to participate in decision-making than a local community represented by career politicians, expert planners, or judges, who all lack local grounding.\footnote{This issue is related not only to the debate on private property rights, covered in chapter \ref{chap:2}. There is also a connection to the debate on local elites in commons and development research, c.f. also the discussion in chapter \ref{chap:3}, section \ref{sec:3:6}. While elite capture is a worry, it has been pointed out that local elites {\it can} bring significant added value to development projects, see generally \cite{amsden12}. Moreover, interventions that seek to marginalise elite members, rather than co-opting them, can turn out to do more harm than good (although the evidence on elite co-option through intervention is also non-conclusive, further suggesting the importance of the social context). See, e.g., \cite{wong13,arnall13}.}

How well a property-based form of representation will work is likely to depend closely on the distribution of property within the community, and the legal framework surrounding the property in question. If property rights are spread across many members of the community, if ownership is shared, or it obligations towards non-owners occupies a prominent place within the private law of property, it should help make decision-making through consolidation more representative of broader community interests. In addition to this comes the fact that the land consolidation process has a judicial form, meaning that a judge is already present, as an administrator and a representative of public interests. Moreover, the judge is required to consider what is best for the properties, not the owners. Hence, on a human flourishing account of which functions property should fulfil, the interests of non-owners need to be taken into account. This also entails that the owners participating in consolidation are indeed meant to be representatives of their communities, with obligations as well as rights.

It should also be emphasised that so far in Norway, the potential for elite capture in consolidation seems to be relatively limited.\footnote{Although related issues have been raised, specifically with respect to the increasing power of the consolidation court to interfere with private property, see \cite{stenseth07}.} Land consolidation is primarily used to organise decision-making among owners, who are protected individually by the no-loss guarantee. Moreover, a consolidation measure will have no direct legal consequences for non-owners or binding consequences for other branches of government. For instance, if a land consolidation court orders a community of owners to pursue hydropower development, this does not mean that the water authorities are compelled to give the necessary licenses. If environmental interests or the interests of non-owners suggest otherwise, a license will not be granted. The Norwegian system -- so far -- implements a fairly strict division of power in this regard, so that land consolidation cannot be used to capture significant power outside the context of decision-making among owners.

However, as I have already mentioned, there have been indications that the scope of consolidation will be broadened, with governmental agencies and other stakeholders more often included in consolidation proceedings that concern their areas of competence. In light of this, there is a reason to worry that the community representation might become too narrow in the future, so that consolidation itself will come to lack legitimacy. Hence, in some contexts, it might be advisable to further extend the class of persons that can be recognised as having legal standing in consolidation disputes, to include non-owners without formally recognised property rights (e.g., neighbours, tenants or employees). 

However, there are some good reasons for caution. First, there are obvious pragmatic concerns related to the increased cost and complexity of the procedure. From the Norwegian experience, it seems that a few hundred parties would be manageable, but more than that takes us to uncharted territory. Second, moving away from property as a basis for legal standing in consolidation might in fact make it {\it easier} for powerful external actors to unduly influence the process. This can be an indirect effect, arising from good intentions. For instance, if a large-scale development involves razing an impoverished part of town, it will not be a good idea to give full legal standing in consolidation to the employees of the development company that stands to benefit. This would be so even if the employees could be classified as `locals' under some imprecise standard for determining who to include in the consolidations process.\footnote{This echoes Heller and Hills' worry that local community members with an ``affiliation'' to the developer might be able to unduly influence decision-making in a Land Assembly District, as discussed in chapter \ref{chap:3}, section \ref{sec:3:6:1}.}

%In the standard narrative about the dangers of majoritarian democracy, this problem needs to be dealt with by the central government or by the law, not by the locals themselves. This is where the social function notion of property, as exemplified by the land consolidation system, offers an alternative. By placing functions of property center stage when making decisions, the majority rule can be replaced by a property-based form of {\it unanimity}. This, in effect, is what the no-loss principle does, on a high level of abstraction, when it declares that no property, no matter how small or insignificant, must be worse off after consolidation.\footnote{This, indeed, is how Pareto improvements among a group of individuals are interpreted theoretically: they are improvements that all owners will consent to.} 
%Of course, the meaning of this must still be debated and defined by persons, but by emphasising that the improvement of property, not personal gain, is the purpose, the decision-making process now takes on some of those characteristics that make some argue that disinterested experts should be granted the power to decide about land uses.

The overarching question is how to ensure that land consolidation courts remain respectful towards both owners and communities, while providing a good basis for equitable and participatory decision-making about how to manage property. In Norway after 2016, any potential taker can be granted legal standing in consolidation disputes, so this challenge is becoming pressing. What will the role of the new parties be? Will they become potential partners that owners can rely on to implement projects in the public interest, or will they be regarded as the main stakeholders, whom the land consolidation courts should assist so that they may successfully impose their will on local communities? It will be very interesting to follow this development further.

\noo{ How this story will unfold in Norway might also shed interesting light on the question of whether the Norwegian system of land consolidation for economic development could work well in other jurisdictions. In small-scale rural settings where property is distributed evenly among local people, the procedure might be expected to work well. However, in other contexts, the challenges that now face the Norwegian system might be even more acute. Even so, the underlying idea and premise of consolidation as a means to organise compulsory development seems to have great potential.
}

Clearly, the consolidation proposal is related to the theoretical argument made in chapter \ref{chap:3}, in favour of alternatives to expropriation based on local institutions for self-governance. As noted by Ostrom and others, congruence with local conditions is crucial to the success of such institutions.\footnote{See \cite[92]{ostrom90}.} Hence, no single institutional framework is likely to work in all cases. This was also the lesson drawn from the critical assessment of the Land Assembly Districts proposed by Heller and Hills.\footnote{See the discussion in chapter \ref{chap:2}, section \ref{sec:2:6}.} This proposal, aiming for self-governance while striving to limit the risk of abuse in all situations, ends up catering only to a very thin notion of participation, arguably without effectively reassuring those who worry about new ways in which governance might fail.

Setting up a Land Assembly District {\it might} still be appropriate, in which case a land consolidation court is ideally placed to order it.\footnote{Arguably, a Norwegian land consolidation court would have to set it up as negotiations for a leasehold only, as in the hydropower cases, to avoid a full transfer of property that might fall outside the meaning of ``joint use'', see \cite{holen95}.} More generally, the consolidation procedure is always temporary, but consolidation leaves a lasting effect on how decision-making takes place within the affected area. In joint use cases, a consolidation court literally sets out to design an institution for local self-governance. This institution can then subsequently be tasked with resource management after the judicial proceedings come to an end.

The underlying idea at work here is to use special tribunals to interface between governments and communities, while overseeing and gently directing decision-making at the community level. This can be a way to set up a link between theories of self-governance and legally enforcible human rights principles. Developing this link further can become a path towards sustainability that will not unduly inflate the power of states and markets, but rather allow people to flourish through participation in just social structures anchored in property. It also promises to address the crucial problem of institutional nesting in a neat way, to ensure that local decision-making takes place in an orderly and equitable fashion, in a way that can also inform and remain sensitive to decision-making processes in other institutions.\footnote{Compare the discussion in chapter \ref{chap:3} section \ref{sec:3:6}.}

This observation remains preliminary at this stage, but I think it points to a promising direction for future work. The intuitive appeal of the consolidation idea appears significant, especially because of its potential as a means to enlist the help of a judicial body to build and improve democracies from the ground up -- starting with people in their communities, and their link to the properties they rely on for their subsistence and well-being.

\section{Conclusion}\label{sec:6:6}

The Norwegian land consolidation courts are unique judicial bodies which can be tasked with enabling coordination and participation in decision-making among groups of owners, increasingly also in conjunction with commercial partners and regulators. This suggests a new take on the role that tribunals can play in relation to decision-making regarding property, economic development, and public interests. Arguably, the sensitive issues that can arise in this context, and the need for orderly interaction between owners, market actors and government bodies, is a reason for courts to get involved {\it ex ante}, by taking charge of the whole process.

The land consolidation courts in Norway can get involved in this way because they have wide powers to organise self-governance, including the possibility of using compulsion to ensure that owners participate on reasonable terms that will benefit their properties. As shown in this chapter, consolidation can therefore be a powerful alternative to takings for economic development. In practice, however, current practices of land consolidation work best when there is a basic agreement among a majority of owners that development is desirable, or at least tolerable. The current procedure is less effective when there is deep disagreement about whether or not development should proceed at all.

To make consolidation better suited for dealing with deep disagreement, it might be appropriate to enhance the power of the land consolidation court, also in the direction of extending its authority to compel owners to engage with development projects that they fundamentally disagree with. But if this is done, it is important to simultaneously ensure that land consolidation remains a service to the owners. This challenge is currently arising with greater urgency in Norway, as the legislator has recently taken steps to increase the scope of consolidation further, including granting legal standing to other stakeholders. It will be interesting to see in the years to come how this will influence the system, especially with regard to the balance of power between different stakeholders in the process.

However, as argued in this chapter, consolidation as an alternative to expropriation has already been developed far enough to shed interesting light on the legitimacy question studied in this thesis. Indeed, land consolidation has been shown to be a highly versatile framework for enabling self-governance, more so than the Land Assembly Districts proposed by Heller and Hills.\footcite{heller08} Unlike Land Assembly Districts, the alienation of property is rarely if ever the aim (or even the side-effect) of consolidation. Rather, the aim is to provide a democracy-on-demand for decision-making about economic development, in a way that should allow owners to adapt to changing property obligations while helping maintain a reasonable balance of power between them, their local communities, and society as a whole. As such, consolidation seems worth considering further also for its potential applications in different jurisdictions and property contexts.

%In this vision, external commercial actors are at worst going to be partners in crime, at best partners in prosperity. They will not, however, be allowed to dictate the terms of development.